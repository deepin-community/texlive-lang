\documentclass[11pt, a4paper, dvipdfmx]{article}

\usepackage{bxtexlogo}

\begin{filecontents}[overwrite]{library.bib}
@misc{IEEEtran,
  author = {Michael Shell},
  url = {https://www.ctan.org/tex-archive/macros/latex/contrib/IEEEtran/bibtex},
  title = {{IEEEtran}},
  month = {August},
  year = {2015}
}
@misc{jIEEEtran,
  author = {ehki},
  url = {https://github.com/ehki/jIEEEtran},
  title = {{jIEEEtran}},
  month = {March},
  year = {2020}
}
@misc{jeconbst,
  author = {ShiroTakeda},
  url = {http://shirotakeda.org/ja/tex-ja/jecon-ja.html},
  title = {{jecon.bst}},
  month = {August},
  year = {2019}
}
\end{filecontents}


\title{\texttt{jIEEEtran.bst}:\\Unofficial IEEE \BibTeX\ style for Japanese\\(ver. 0.19)}
\author{Haruki EJIRI}

\begin{document}

\maketitle

\begin{abstract}
The jIEEEtran \BibTeX\ style is an unofficial IEEE style citation format for Japanese users.
It is a customization of Michael Shell's IEEEtran.bst, and provides natural Japanese name handling.
A Python script named mixej.py enables one content with both English and Japanese format to be handled properly.
\end{abstract}


\section{Introduction}

IEEJtran.bst is an unofficial \BibTeX-style (.bst) file in IEEJ format developed for Japanese users engaging in electrical engineering.
YoshiRi implemented natural Japanese notation based on Michael Shell's IEEEtran.bst[?].
And now the author develops it in GitHub repository [?] after taking it over.
Please note jIEEEtran.bst is unofficial.
The author does not guarantee that it will work in all environments.
We cannot be held responsible for any problems that may occur.
We recommend platex/pbibtex or uplatex/upbibtex as \TeX\ engines.


\section{Configuration}

If jIEEEtran.bst has been installed by \TeX\ Live package manager, etc., and placed under PATH, write \texttt{\string\bibliographystyle\string{jIEEEtran\string}} in the preamble to complete the setup.
In the case that jIEEEtran.bst is individually downloaded and placed in the same directory as the main \TeX\ file such a reason for customization, state \texttt{\string\bibliographystyle\string{jIEEEtran\string}} in the preamble in the same way to complete the setting.
In the .bib file containing bibliographic information, Japanese authors' names should be written as:
\begin{center}
\texttt{author = \string{\string{Yamada Taro\string} and \string{Sato Jiro\string}\string}},
\end{center}
or
\begin{center}
\texttt{author = \string{Yamada, Taro and  Sato, Jiro\string}}.
\end{center}


\section{Combine English and Japanese entries}

Japanese users are sometimes required to write both English and Japanese citation formats as one bibliography entry.
A python script named \texttt{mixej.py} included in the \texttt{jieeetran} package enables combining English and Japanese entries.
To combine two entries of enkey and jpkey, users have to cite as \texttt{\string\cite\string{enkye/ej/jpkey\string}}, where enkey and jpkey are English and Japanese citation keys, respectively.
Then compile by the following seven steps:
\begin{center}
    \texttt{uplatex → python mixej.py → upbibtex → python mixej.py\\→ uplatex → uplatex → dvipdfmx}
\end{center}
instead of the typical five steps:
\begin{center}
    \texttt{uplatex → upbibtex → uplatex → uplatex → dvipdfmx}.
\end{center}
Note that the uplatex command is Unicode and Japanese compatible latex command.

The Japanese version of this document, \texttt{jieeetran.pdf}, shows examples for English and Japanese combined citations.

\end{document}
