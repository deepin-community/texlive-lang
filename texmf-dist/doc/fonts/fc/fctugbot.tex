\documentstyle[array,ltugboat]{article}
% I used ltugboat.sty v1.02 

%************************************************************************
% TUGboat editorial macros
\def\showpart#1{%
  \expandafter\def\csname **#1\expandafter\endcsname\space{}
  \expandafter\def\csname **end #1\expandafter\endcsname\space{}
  }
\def\noshowpart#1{%
  \long\expandafter\def\csname **#1\expandafter\endcsname\space##1\TUBedit##2 {}}
\def\TUBedit#1{\csname**#1\endcsname}
\noshowpart{author}
\showpart{editor}
\noshowpart{change summary}
%************************************************************************

\TUBedit{change summary}
1. Capitalization in title and heads.
2. Some spelling and grammatical changes.
3. \makesignature used.
4. Placement of Table 2 so as to be typeset on final page
\TUBedit{end change summary}
%%JK
\noshowpart{last change summary}
\TUBedit{last change summary}
1. Changed postal address because of new german ZIP codes
2. Introduced italic font to show examples 
3. Exhibited a scharfes s
\TUBedit{end last change summary}
%%/JK


\TUBedit{author}
\title{Fonts for Africa: The fc-fonts}
\TUBedit{end author}
%
\TUBedit{editor}
% remove hyphen, add cap
\title{Fonts for Africa: The fc Fonts}
\TUBedit{end editor}

\TUBedit{author}
\author{J\"org Knappen\\
        Chair of the TWG on african languages with latin writing\\
        Institut f\"ur Kernphysik\\
        Postfach 39 80\\
        D-W 6500 Mainz\\
        R. F. A.\\
        {\tt knappen@vkpmzd.kph.uni-mainz.de}}
\signature{J\"org Knappen}
\TUBedit{end author}
% 
\TUBedit{editor}
% At least the address information should go to the bottom of
% the article where the signature is placed.  Should we keep the
% chairmanship up top?
%%JK: I think so.
%%/JK
\author{J\"org Knappen\\
        Chair of the TWG on African Languages with Latin Writing}
\address{Institut f\"ur Kernphysik\\
%%JK        Postfach 39 80\\
%%JK        D-W 6500 Mainz\\
%%JK  New ZIP code, no Postfach any more:
        D 55 099 Mainz\\
%%/JK
        R. F. A.}
\netaddress{knappen@vkpmzd.kph.uni-mainz.de}
\TUBedit{end editor}
\begin{document}
\maketitle
\setlength{\extrarowheight}{2pt}
%
{% Begin a group to keep things local
\font\fc=fcr10 % You can get the fc fonts from ftp.uni-stuttgart.de
\def\FC#1{{\fc\char"#1}} % The argument is an uppercase hex number
\font\sstt=fcsstt10
%%JK  Add fc italic font:
\font\fci=fci10
%%/JK
%
\font\logo=logo10
\def\MF{{\logo META}\-{\logo FONT}}
%
\font\cyr=wncyr10
%
\makeatletter
\input fcuse.sty % Available from ftp.uni-stuttgart.de
\makeatother
%
\begin{abstract}
\TUBedit{author}
A font (fc) with 256 characters for the needs of african languages with latin 
\TUBedit{end author}
%
\TUBedit{editor}
% At one time, I recommended changing `africa/n' to `Africa/n' throughout,
% but I gather that this is consistent with `european', `german, `french',
% etc. and we should just regard this as a matter of style left to the
% author.
A font (fc) with 256 characters for the needs of african languages with latin 
\TUBedit{end editor}
writing is presented. The so-called critical languages with more than 
\TUBedit{author}
1~mio\ speakers are supported. It is implemented in \MF\ using Sauter's 
\TUBedit{end author}
%
\TUBedit{editor}
% I'm not that familiar with the term `mio', although I suppose it's a
% contraction for `million'.  Is this just American ignorance?  If so,
% we can leave it here; otherwise I would prefer to use the more standard
% `million'.  We should also be consistent with respect to the terminating
% period.  None occurs here, but there is one where `mio' is next used.
1~mio\ speakers are supported. It is implemented in \MF\ using Sauter's 
\TUBedit{end editor}
tools for the parametrisation.
\end{abstract}
%
\TUBedit{author}
\renewcommand{\abstractname}{Zusammenfassung}
\begin{abstract}
Ein Zeichensatz (fc) mit 256 Zeichen, der zum Satz afrikanischer Sprachen mit 
lateinischer Schrift entwickelt worden ist, wird vorgestellt. Er 
unterst\"utzt die sog.\ kritischen Sprachen in Afrika mit mehr als 1~Mio.\ %
Sprechern. Die Implementierung der fc-Zeichensatzfamilie geschieht in \MF\ % 
unter Ausnutzung der Sauter'schen Parametrisierungen. 
\end{abstract}
\TUBedit{end author}
%
\TUBedit{editor}
% comment out for TUGboat
%\renewcommand{\abstractname}{Zusammenfassung}
%\begin{abstract}
%Ein Zeichensatz (fc) mit 256 Zeichen, der zum Satz afrikanischer Sprachen mit 
%lateinischer Schrift entwickelt worden ist, wird vorgestellt. Er 
%unterst\"utzt die sog.\ kritischen Sprachen in Afrika mit mehr als 1~Mio.\ %
%Sprechern. Die Implementierung der fc-Zeichensatzfamilie geschieht in \MF\ % 
%unter Ausnutzung der Sauter'schen Parametrisierungen. 
%\end{abstract}
\TUBedit{end editor}
%
%
\section*{Introduction}
On the continent of africa, several hundred different languages are spoken, 
\TUBedit{author}
the numbers of speakers range from more than 100~mio.\ down to few 
\TUBedit{end author}
%
\TUBedit{editor}
% grammatical adjustment
the numbers of speakers of each ranging from more than 100~mio.\ down to few 
\TUBedit{end editor}
hundreds. Many languages do not have a written form, other language have 
\TUBedit{author}
some written form, but it is not standardised yet. Or even worse: There 
\TUBedit{end author}
%
\TUBedit{editor}
% lowercase after :
some written form, but it is not standardised yet. Or even worse: there 
\TUBedit{end editor}
exists more than one writing system for the same language (e.g.~protestant 
and catholic orthography). Clearly, it was impossible to check out all 
writing systems of all possible languages. This work confines itself to the 
so-called `critical languages' according to a definition of the US 
\TUBedit{author}
department of education in 1986. All these languages have more than 1~mio.\ %
\TUBedit{end author}
%
\TUBedit{editor}
% capitalize DoE
Department of Education in 1986. All these languages have more than 1~mio.\ %
\TUBedit{end editor}
speakers.

\TUBedit{author}
On the continent of africa, three main writing systems are in use:
\TUBedit{end author}
%
\TUBedit{editor}
% Since the previous paragraph also begins with `On the continent ...',
% let's modify a little.  Perhaps:
%%JK Good idea. Let's noindent it, as you have done it further down
\noindent
%%/JK
Three main writing systems are in use in Africa:
\TUBedit{end editor}
\begin{itemize}
\item {\it {\ae}thiopian} writing, used for several languages in 
      {\AE}thiopia and Eritrea.
\item {\it arabic} writing, used in northern africa and on the east coast 
\TUBedit{author}
      for several languages. It is now loosing ground against latin writing,
\TUBedit{end author}
%
\TUBedit{editor}
% `loosing' -> `losing' and `,' -> `;'
      for several languages. It is now losing ground against latin writing;
\TUBedit{end editor}
      languages like suaheli and hausa are now mainly written in latin.
\item {\it latin} writing is now being introduced or established in the 
      rest of africa. 
\end{itemize}
In this overview, I want to mention also some other writing systems.
The Tuareg 
living in the sahara use the over 2,000 years old {\em tifinagh} writing 
\TUBedit{author}
system, a right-to-left writing, the Kopts in {\AE}gypt still use 
\TUBedit{end author}
%
\TUBedit{editor}
% `,' -> `;'
system, a right-to-left writing; the Kopts in {\AE}gypt still use 
\TUBedit{end editor}
{\em koptic} for religious purposes. 

An interesting development was the invention of a somali alphabet by 
Osman Yusuf, which combined influences from the three major writing systems, 
but this system did not achieve official status in Somalia.

The introduction of latin writing was first done by missionaries, who 
invented orthographies quite arbitrarily. But since the first half of this 
\TUBedit{author}
century, there are attempts to standardise the newly introduced alphabets. 
\TUBedit{end author}
%
\TUBedit{editor}
% `are' -> `have been'
century, there have been
attempts to standardise the newly introduced alphabets. 
\TUBedit{end editor}

As a result of these attempts, an `african reference alphabet' emerged, 
based on the principle of {\em one sound, one sign} which is also the 
principle of the international phonetic alphabet. There are many borrowings 
from the phonetic alphabet in african writing. These efforts are now 
coordinated by the {\sc UNESCO}. 

\TUBedit{author}
\section*{The fc fonts}
\TUBedit{end author}
%
\TUBedit{editor}
% capitalize `fonts'
\section*{The fc Fonts}
\TUBedit{end editor}
The fc font encoding scheme encodes characters necessary to typeset the 
major african languages with latin writing. In selecting these languages, I 
followed the list of so-called critical languages. Unfortunately, it proved 
impossible to put really all characters into one 256-character font. 
Therefore, I applied the following selection rules:
\begin{itemize}
\item The standard \TeX\ character set is still available.
\item Characters with entirely new shapes are included with highest 
      priority. 
\item Characters which need another accent in some languages have the 
      second highest priority.
\item Characters which are difficult to handle by TeX macros are preferred 
      over those which are easier to handle (e.g. characters with diacritic 
\TUBedit{author}
      below are prefered to those with diacritics above).
\TUBedit{end author}
%
\TUBedit{editor}
% `prefered' -> `preferred'
      below are preferred to those with diacritics above).
\TUBedit{end editor}
\TUBedit{author}
\item Characters which occur only in tonal languages and are carrying a 
      tonal mark, are most likely discarded.
\TUBedit{end author}
%
\TUBedit{editor}
% `are carrying' -> `carry'
\item Characters which occur only in tonal languages and carry a 
      tonal mark, are most likely discarded.
\TUBedit{end editor}
\end{itemize}

\TUBedit{editor}
% To accentuate the regular text as opposed to the surrounding lists.
\noindent
\TUBedit{end editor}
The fc fonts have several other interesting features some of which I list 
here:
\begin{itemize}
\item The lower half of the fc encoding scheme is identical to the Cork 
      (or ec) encoding scheme for european languages.
\item If a character occurs both in the fc scheme and in the Cork scheme, 
      it has the same encoding.
\item The difference between uppercase and the corresponding 
      lowercase character is a constant.
\item It is possible to create virtual fonts to obtain all those 
      characters of tonal languages which needed to be discarded.
\item It is possible to rebuild the cm fonts and the ec fonts as virtual 
      fonts from the fc fonts. In the latter case three letters are missing
      (edh, thorn and Thorn) and cannot be done without larger loss in 
      quality. Of course, there are smaller quality losses in building 
      such a virtual font, e.g.~for \^\i, where the \^{}-accent comes out 
      too wide.
\end{itemize}

\TUBedit{editor}
% Again to accentuate the regular text.
\noindent
\TUBedit{end editor}
The following languages are supported (with the proviso about tones made 
above):   Akan, Bamileke, Basa (Kru), Bemba, Ciokwe, Dinka, Dholuo (Luo), Efik,
  Ewe-Fon, Fulani (Fulful), G\~a, Gbaya, Hausa, \d{I}gb\d{o}, Kanuri, Kikuyu, 
  Kikongo, Kpelle, Krio, Luba, Mandekan (Bambara), Mende, More, Ngala, 
  Nyanja, Oromo, Rundi, Kinya Rwanda, Sango, Serer, Shona, Somali, Songhai, 
  Sotho (two different writing systems), Suaheli, Tiv, Yao, Yoruba, Xhosa,
  and Zulu.

  I decided to support two european languages, which are not covered by the 
\TUBedit{author}
  EC-scheme,
\TUBedit{end author}
%
\TUBedit{editor}
% Elsewhere we use lowercase for ec and fc.
  ec-scheme,
\TUBedit{end editor}
namely Maltese and Sami\footnote{The Cork scheme includes the
\TUBedit{author}
letter eng ({\fc+n}) especially for Sami, but it missed the letter t with bar
\TUBedit{end author}
%
\TUBedit{editor}
% `missed' -> `is missing'
letter eng ({\fc+n}) especially for Sami, but it is missing
the letter t with bar
\TUBedit{end editor}
({\fc/t}). However, with the letter eng several african languages can be
written using the Cork scheme as well.}.
% Comment: I noticed that the examples come out too big, but regard this as 
% a feature. In fact, footnotesized, they would come out nearly illegible.

There are some writing systems which are not supported by the fc fonts. 
\TUBedit{author}

\TUBedit{end author}
%
\TUBedit{editor}
% Let's run these two paragraphs together.
\TUBedit{end editor}
First to mention is the writing of Khoi-San languages with their
characteristic click sounds. Missing is the letter $\neq$ which can be 
constructed as a macro. The obsolete orthographies of Xhosa and Zulu are not 
supported (they used a cyrillic B ({\cyr B}) as uppercase of {\fc +b}). 

Also not supported is the obsolete orthography of \d{I}gb\d{o} which 
contains an o with horizontal bar (\accent45o). A proposed alphabet for 
Tamasheq (a berber language) is not supported because there is no evidence 
that it ever caught on.

\section*{Design of the letters}
The design of the letters closely follows the {\em computer modern} fonts 
by Donald E.~Knuth. Much of the \MF\ code made its way unchanged to the fc
fonts. However, several modification were necessary in the case of italics.
Since {\fc E+ue} has both `f' and {\fc`+f'}, the latter looking like the 
italic letter f, the italic letter f was redesigned to have a straight, 
uncurved descender
%%JK Show the new italic forms
({\fci f, +f\/}).
%%/JK
\TUBedit{editor}
% It would be good to show the new italic `f' here.  Similarly for the
% other characters mentioned immediately below and the scharfes s.
\TUBedit{end editor}
Similarly, the italic letter v was redesigned to have 
sharp edge
%%JK show it
({\fci v\/}). 
%%/JK
In consequence of the last change, the italic letters w and y 
were changed, too
%%JK show them
({\fci w, y\/}).
%%/JK

\TUBedit{author}
Deliberately, I changed the appearence of the roman letter {\em 
scharfes s}. Its new shape exhibits the ligature of long and short s from 
\TUBedit{end author}
%
\TUBedit{editor}
% `appearence' -> `appearance'
Deliberately, I changed the appearance of the roman letter {\em 
scharfes s}. Its new shape exhibits the ligature of long and short s from 
\TUBedit{end editor}
which it originally derives
%%JK Show it
(\FC{FF}).
%%/JK

\section*{How to use the fc fonts}
At the present time, I support only \LaTeX\ and plain \TeX\ with the new 
font selection scheme (NFSS) by R.~Sch\"opf and F.~Mittelbach. There are 
the files {\tt fontdef.fc} and {\tt fclfont.sty} which load the fonts and 
set the standard \TeX\ commands for accents correctly. But, at the moment 
there are no standard control sequences to get the newly added characters,
except that {\verb:\|:} shall produce the universal 
%%JK tie it to the line
accent~(\FC{BF}).
%%/JK

\begin{table}[htbp]
\begin{center}
\begin{tabular}{lcc}
% Name, (Gro"s, Klein), fcuse (gro"s, Klein)
Name & Shape & Input\\ \hline
Hooktop b & {\fc +b, +B} & \verb:+b, +B: \\
Hooktop c & {\fc +c, +C} & \verb:+c, +C: \\
Hooktop d & {\fc +d, +D} & \verb:+d, +D: \\
Open e    & {\fc +e, +E} & \verb:+e, +E: \\
Long f    & {\fc +f, +F} & \verb:+f, +F: \\
Ipa gamma & {\fc +g, +G} & \verb:+g, +G: \\
Latin iota& {\fc +i, +I} & \verb:+i, +I: \\
Enj       & {\fc +j, +J} & \verb:+j, +J: \\
Hooktop k & {\fc +k, +K} & \verb:+k, +K: \\
Eng       & {\fc +n, +N} & \verb:+n, +N: \\
Open o    & {\fc +o, +O} & \verb:+o, +O: \\
Hooktop p & {\fc +p, +P} & \verb:+p, +P: \\
Esh       & {\fc +s, +S} & \verb:+s, +S: \\
Hooktop t & {\fc +t, +T} & \verb:+t, +T: \\
Variant u & {\fc +u, +U} & \verb:+u, +U: \\
Ezh       & {\fc +z, +Z} & \verb:+z, +Z: \\ \hline
Crossed d & {\fc /d, /D} & \verb:/d, /D: \\
Crossed h & {\fc /h, /H} & \verb:/h, /H: \\
Crossed t & {\fc /t, /T} & \verb:/t, /T: \\ \hline
Tailed d  & {\fc =d, =D} & \verb:=d, =D: \\
Inverted e& {\fc =e, =E} & \verb:=e, =E: \\
Long t    & {\fc =t, =T} & \verb:=t, =T: \\ \hline
\end{tabular}
\caption{Overview of the special letters in the fc fonts and their access 
via {\tt fcuse.sty}}
\label{jk:tab:fcuse}
\end{center}
\end{table}

There is another style option ({\tt fcuse.sty}) which makes the special 
characters accesible via active characters. Three characters needed to be 
activated, since the letters d and t carry the load of three new variants 
\TUBedit{author}
({\fc +d, =d, /d; +t, =t, /t}). An overview of the assignments gives
table~\ref{jk:tab:fcuse}.
\TUBedit{end author}
%
\TUBedit{editor}
% `gives' -> `is given in'; cap `table'
({\fc +d, =d, /d; +t, =t, /t}). An overview of the assignments is given in
Table~\ref{jk:tab:fcuse}.
\TUBedit{end editor}

\TUBedit{editor}
% This achieves placement of Table 2 on the last page.  Admittedly,
% this is not the way one should do it, but it appears that LaTeX can't
% adjust its textheight for the current page on the fly.
\begin{table*}[btp]
\begin{center}
\begin{tabular}{|c||c|c|c|c|c|c|c|c|c|c|c|c|c|c|c|c|}
\hline
   &x0&x1&x2&x3&x4&x5&x6&x7&x8&x9&xA&xB&xC&xD&xE&xF\\
\hline\hline
0y  
&\FC{00}&\FC{01}&\FC{02}&\FC{03}&\FC{04}&\FC{05}&\FC{06}&\FC{07}&\FC{08}
    &\FC{09}&\FC{0A}&\FC{0B}&\FC{0C}&\FC{0D}&\FC{0E}&\FC{0F}\\ \hline
1y  
&\FC{10}&\FC{11}&\FC{12}&\FC{13}&\FC{14}&\FC{15}&\FC{16}&\FC{17}&\FC{18}
    &\FC{19}&\FC{1A}&\FC{1B}&\FC{1C}&\FC{1D}&\FC{1E}&\FC{1F}\\ \hline
2y  
&\FC{20}&\FC{21}&\FC{22}&\FC{23}&\FC{24}&\FC{25}&\FC{26}&\FC{27}&\FC{28}
    &\FC{29}&\FC{2A}&\FC{2B}&\FC{2C}&\FC{2D}&\FC{2E}&\FC{2F}\\ \hline
3y  
&\FC{30}&\FC{31}&\FC{32}&\FC{33}&\FC{34}&\FC{35}&\FC{36}&\FC{37}&\FC{38}
    &\FC{39}&\FC{3A}&\FC{3B}&\FC{3C}&\FC{3D}&\FC{3E}&\FC{3F}\\ \hline
4y  
&\FC{40}&\FC{41}&\FC{42}&\FC{43}&\FC{44}&\FC{45}&\FC{46}&\FC{47}&\FC{48}
    &\FC{49}&\FC{4A}&\FC{4B}&\FC{4C}&\FC{4D}&\FC{4E}&\FC{4F}\\ \hline
5y  
&\FC{50}&\FC{51}&\FC{52}&\FC{53}&\FC{54}&\FC{55}&\FC{56}&\FC{57}&\FC{58}
    &\FC{59}&\FC{5A}&\FC{5B}&\FC{5C}&\FC{5D}&\FC{5E}&\FC{5F}\\ \hline
6y  
&\FC{60}&\FC{61}&\FC{62}&\FC{63}&\FC{64}&\FC{65}&\FC{66}&\FC{67}&\FC{68}
    &\FC{69}&\FC{6A}&\FC{6B}&\FC{6C}&\FC{6D}&\FC{6E}&\FC{6F}\\ \hline
7y  
&\FC{70}&\FC{71}&\FC{72}&\FC{73}&\FC{74}&\FC{75}&\FC{76}&\FC{77}&\FC{78}
    &\FC{79}&\FC{7A}&\FC{7B}&\FC{7C}&\FC{7D}&\FC{7E}&\FC{7F}\\ \hline
8y  
&\FC{80}&\FC{81}&\FC{82}&\FC{83}&\FC{84}&\FC{85}&\FC{86}&\FC{87}&\FC{88}
    &\FC{89}&\FC{8A}&\FC{8B}&\FC{8C}&\FC{8D}&\FC{8E}&\FC{8F}\\ \hline
9y  
&\FC{90}&\FC{91}&\FC{92}&\FC{93}&\FC{94}&\FC{95}&\FC{96}&\FC{97}&\FC{98}
    &\FC{99}&\FC{9A}&\FC{9B}&\FC{9C}&\FC{9D}&\FC{9E}&\FC{9F}\\ \hline
Ay  
&\FC{A0}&\FC{A1}&\FC{A2}&\FC{A3}&\FC{A4}&\FC{A5}&\FC{A6}&\FC{A7}&\FC{A8}
    &\FC{A9}&\FC{AA}&\FC{AB}&\FC{AC}&\FC{AD}&\FC{AE}&\FC{AF}\\ \hline
By  
&\FC{B0}&\FC{B1}&\FC{B2}&\FC{B3}&\FC{B4}&\FC{B5}&\FC{B6}&\FC{B7}&\FC{B8}
    &\FC{B9}&\FC{BA}&\FC{BB}&\FC{BC}&\FC{BD}&\FC{BE}&\FC{BF}\\ \hline
Cy  
&\FC{C0}&\FC{C1}&\FC{C2}&\FC{C3}&\FC{C4}&\FC{C5}&\FC{C6}&\FC{C7}&\FC{C8}
    &\FC{C9}&\FC{CA}&\FC{CB}&\FC{CC}&\FC{CD}&\FC{CE}&\FC{CF}\\ \hline
Dy  
&\FC{D0}&\FC{D1}&\FC{D2}&\FC{D3}&\FC{D4}&\FC{D5}&\FC{D6}&\FC{D7}&\FC{D8}
    &\FC{D9}&\FC{DA}&\FC{DB}&\FC{DC}&\FC{DD}&\FC{DE}&\FC{DF}\\ \hline
Ey  
&\FC{E0}&\FC{E1}&\FC{E2}&\FC{E3}&\FC{E4}&\FC{E5}&\FC{E6}&\FC{E7}&\FC{E8}
    &\FC{E9}&\FC{EA}&\FC{EB}&\FC{EC}&\FC{ED}&\FC{EE}&\FC{EF}\\ \hline
Fy  
&\FC{F0}&\FC{F1}&\FC{F2}&\FC{F3}&\FC{F4}&\FC{F5}&\FC{F6}&\FC{F7}&\FC{F8}
    &\FC{F9}&\FC{FA}&\FC{FB}&\FC{FC}&\FC{FD}&\FC{FE}&\FC{FF}\\ \hline
\end{tabular}
\caption{The font fcr10}\label{jk:tab:fcr10}
\end{center}
\end{table*}
\TUBedit{end editor}
\section*{Implementation}
The implementation follows closely to the one of the cm fonts. The 
distinction between parameter files, driver files and programme files is 
kept. The parameters are computed from the design size of the fonts with 
the help of the Sauter tools. This allows an easy nonlinear scaling. I 
added only one new parameter, {\tt univ\_acc\_breath}, which governs the 
shape of the universal accent. To adjust the height of accented capitals 
the parameter {\tt comma\_depth} was employed. The result is excellent with 
serif fonts, but not that good with sans serif ones. I made up one new 
parameter file, creating a {\sstt sans serif typewriter} fontshape. There 
were only minor modifications needed in the programmes of the letters i, I, 
and l to make it work.

There is not much to say about the driver files, except that there are 
additional kerns for {\fc kV, kW, mV, mW, and eV}. The ligatures from 
the Cork scheme to get german and french quotes are included, too.
\TUBedit{author}
The programme files are completely reorganised. The huge files are splitted 
\TUBedit{end author}
%
\TUBedit{editor}
% `splitted' -> `split'
The programme files are completely reorganised. The huge files are split
\TUBedit{end editor}
into smaller ones each containing only one to three letters of the 
alphabet. The shape of the letter is saved as a picture and only the 
diacritic is calculated to get an accented letter. This is done to save cpu 
time and to make the maintenance of the files easier. The accented letters 
are generated only on demand, i.e.\ if their code is known. The codes are 
given in the file {\tt coding.fc}. It is possible to use the same source 
to create other real fonts by replacing the coding file. The so-created 
fonts may not be called fc, since this abbreviation is reserved to fonts 
which are fc-encoded.

\section*{Outlook}
The fonts are done now, but there is lot of work left in creating suitable 
styles for african languages and virtual fonts, if needed. For this work, 
the speakers and users of the african languages are qualified best, and I 
hope that there will be some results of this work soon.

\TUBedit{editor}
% add signature
\makesignature
\TUBedit{end editor}

% The following dirty tricks make the table print on a separate page. In 
% principle it should fit on the bottom of the previous page, but I didn't 
% manage to get it there. It _must_ start in left column. 

\TUBedit{author}
\newpage
\null
\newpage
\begin{table}[p]
\begin{center}
\begin{tabular}{|c||c|c|c|c|c|c|c|c|c|c|c|c|c|c|c|c|}
\hline
   &x0&x1&x2&x3&x4&x5&x6&x7&x8&x9&xA&xB&xC&xD&xE&xF\\
\hline\hline
0y  
&\FC{00}&\FC{01}&\FC{02}&\FC{03}&\FC{04}&\FC{05}&\FC{06}&\FC{07}&\FC{08}
    &\FC{09}&\FC{0A}&\FC{0B}&\FC{0C}&\FC{0D}&\FC{0E}&\FC{0F}\\ \hline
1y  
&\FC{10}&\FC{11}&\FC{12}&\FC{13}&\FC{14}&\FC{15}&\FC{16}&\FC{17}&\FC{18}
    &\FC{19}&\FC{1A}&\FC{1B}&\FC{1C}&\FC{1D}&\FC{1E}&\FC{1F}\\ \hline
2y  
&\FC{20}&\FC{21}&\FC{22}&\FC{23}&\FC{24}&\FC{25}&\FC{26}&\FC{27}&\FC{28}
    &\FC{29}&\FC{2A}&\FC{2B}&\FC{2C}&\FC{2D}&\FC{2E}&\FC{2F}\\ \hline
3y  
&\FC{30}&\FC{31}&\FC{32}&\FC{33}&\FC{34}&\FC{35}&\FC{36}&\FC{37}&\FC{38}
    &\FC{39}&\FC{3A}&\FC{3B}&\FC{3C}&\FC{3D}&\FC{3E}&\FC{3F}\\ \hline
4y  
&\FC{40}&\FC{41}&\FC{42}&\FC{43}&\FC{44}&\FC{45}&\FC{46}&\FC{47}&\FC{48}
    &\FC{49}&\FC{4A}&\FC{4B}&\FC{4C}&\FC{4D}&\FC{4E}&\FC{4F}\\ \hline
5y  
&\FC{50}&\FC{51}&\FC{52}&\FC{53}&\FC{54}&\FC{55}&\FC{56}&\FC{57}&\FC{58}
    &\FC{59}&\FC{5A}&\FC{5B}&\FC{5C}&\FC{5D}&\FC{5E}&\FC{5F}\\ \hline
6y  
&\FC{60}&\FC{61}&\FC{62}&\FC{63}&\FC{64}&\FC{65}&\FC{66}&\FC{67}&\FC{68}
    &\FC{69}&\FC{6A}&\FC{6B}&\FC{6C}&\FC{6D}&\FC{6E}&\FC{6F}\\ \hline
7y  
&\FC{70}&\FC{71}&\FC{72}&\FC{73}&\FC{74}&\FC{75}&\FC{76}&\FC{77}&\FC{78}
    &\FC{79}&\FC{7A}&\FC{7B}&\FC{7C}&\FC{7D}&\FC{7E}&\FC{7F}\\ \hline
8y  
&\FC{80}&\FC{81}&\FC{82}&\FC{83}&\FC{84}&\FC{85}&\FC{86}&\FC{87}&\FC{88}
    &\FC{89}&\FC{8A}&\FC{8B}&\FC{8C}&\FC{8D}&\FC{8E}&\FC{8F}\\ \hline
9y  
&\FC{90}&\FC{91}&\FC{92}&\FC{93}&\FC{94}&\FC{95}&\FC{96}&\FC{97}&\FC{98}
    &\FC{99}&\FC{9A}&\FC{9B}&\FC{9C}&\FC{9D}&\FC{9E}&\FC{9F}\\ \hline
Ay  
&\FC{A0}&\FC{A1}&\FC{A2}&\FC{A3}&\FC{A4}&\FC{A5}&\FC{A6}&\FC{A7}&\FC{A8}
    &\FC{A9}&\FC{AA}&\FC{AB}&\FC{AC}&\FC{AD}&\FC{AE}&\FC{AF}\\ \hline
By  
&\FC{B0}&\FC{B1}&\FC{B2}&\FC{B3}&\FC{B4}&\FC{B5}&\FC{B6}&\FC{B7}&\FC{B8}
    &\FC{B9}&\FC{BA}&\FC{BB}&\FC{BC}&\FC{BD}&\FC{BE}&\FC{BF}\\ \hline
Cy  
&\FC{C0}&\FC{C1}&\FC{C2}&\FC{C3}&\FC{C4}&\FC{C5}&\FC{C6}&\FC{C7}&\FC{C8}
    &\FC{C9}&\FC{CA}&\FC{CB}&\FC{CC}&\FC{CD}&\FC{CE}&\FC{CF}\\ \hline
Dy  
&\FC{D0}&\FC{D1}&\FC{D2}&\FC{D3}&\FC{D4}&\FC{D5}&\FC{D6}&\FC{D7}&\FC{D8}
    &\FC{D9}&\FC{DA}&\FC{DB}&\FC{DC}&\FC{DD}&\FC{DE}&\FC{DF}\\ \hline
Ey  
&\FC{E0}&\FC{E1}&\FC{E2}&\FC{E3}&\FC{E4}&\FC{E5}&\FC{E6}&\FC{E7}&\FC{E8}
    &\FC{E9}&\FC{EA}&\FC{EB}&\FC{EC}&\FC{ED}&\FC{EE}&\FC{EF}\\ \hline
Fy  
&\FC{F0}&\FC{F1}&\FC{F2}&\FC{F3}&\FC{F4}&\FC{F5}&\FC{F6}&\FC{F7}&\FC{F8}
    &\FC{F9}&\FC{FA}&\FC{FB}&\FC{FC}&\FC{FD}&\FC{FE}&\FC{FF}\\ \hline
\end{tabular}
\caption{The font fcr10}\label{jk:tab:fcr10}
\end{center}
\end{table}
\TUBedit{end author}
%

}% end the group from the beginning, forget the font \fc and the 
%  definitions of fcuse.sty
\end{document}
