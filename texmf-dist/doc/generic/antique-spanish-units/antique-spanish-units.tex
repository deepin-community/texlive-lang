% TIPO DE DOCUMENTO

\documentclass[12pt,letterpaper,twoside,final]{article}

% CODIFICACIÓN DE ENTRADA

\usepackage[utf8]{inputenc}

% CODIFICACIÓN DE LA TIPOGRAFÍA

\usepackage[TS1,T1]{fontenc}

% MICROTIPOGRAFÍA

\usepackage[config=mt-MinionPro]{microtype}

% TIPOGRAFÍA

\usepackage[smallfamily,osf,footnotefigures]{MinionPro}
\renewcommand{\scdefault}{ssc}
\usepackage{MyriadPro}

% IDIOMA

\usepackage[spanish,mexico,es-noindentfirst,es-nosectiondot]{babel}

% PÁGINA

\usepackage[left=2.5cm,right=2.5cm,top=2.5cm,bottom=2.5cm,includehead=true,headsep=1.5cm]{geometry}

% ENCABEZADOS

\usepackage{fancyhdr}

% TABLAS

\usepackage{etoolbox}
\usepackage{multirow}
\usepackage{booktabs}
\usepackage[toc,enum,lineno]{tabfigures}
\usepackage{caption}

% BIBLIOGRAFÍA

\usepackage{hanging}

% MACROTIPOGRAFÍA

%\usepackage{showframe}

% ENLACES DINÁMICOS

\usepackage[hyperindex=true,final=true,bookmarks=true,bookmarksnumbered=true,
bookmarksopen=true,breaklinks=true,citecolor=black,colorlinks=true,
linkcolor=black,urlcolor=black,pdftitle={El antiguo sistema de unidades españolas},pdfsubject={Metrología},pdfauthor={Noel Merino Hernández},pdfkeywords={antiguas unidades españolas metrología},pdfproducer={pdfTeX 3.14159265-2.6-1.40.21 (TeX Live 2020/Debian)},pdfcreator={Noel Merino Hernández (muxkernel@gmail.com)}]{hyperref}

\urlstyle{rm}

\usepackage{hyperxmp}
\hypersetup{
pdfcopyright={Este documento se proporciona como es, sin ningún tipo de garantía, aunque se espera que pueda ser útil, y se distribuye bajo una licencia Creative Commons Atribución 4.0 Internacional (CC by 4.0). Por lo tanto, eres libre de compartir, copiar y redistribuir el material en cualquier medio o formato. Adaptar, remezclar, transformar y construir a partir del material para cualquier propósito, incluso comercialmente. El licenciante no puede revocar estas libertades en tanto usted siga los términos de la licencia, bajo los siguientes términos: Atribución. Usted debe dar crédito de manera adecuada, brindar un enlace a la licencia, e indicar si se han realizado cambios. Puede hacerlo en cualquier forma razonable, pero no de forma tal que sugiera que usted o su uso tienen el apoyo del licenciante. No hay restricciones adicionales. No puede aplicar términos legales ni medidas tecnológicas que restrinjan legalmente a otras a hacer cualquier uso permitido por la licencia. Avisos: Usted no tiene que cumplir con la licencia para elementos del material en el dominio público o cuando su uso esté permitido por una excepción o limitación aplicable. No se dan garantías. La licencia podría no darle todos los permisos que necesita para el uso que tenga previsto. Por ejemplo, otros derechos como publicidad, privacidad, o derechos morales pueden limitar la forma en que utilice el material.},
pdflicenseurl={https://creativecommons.org/licenses/by/4.0/deed.es}
}
\def\fileversion{1.6}
\def\filedate{(2024-01-22)}
\title{El antiguo sistema de unidades españolas\footnote{versión\space\fileversion\space\filedate}}
\author{Noel Merino Hernández\footnote{Correo electrónico: muxkernel@gmail.com}}
\spanishdatedel
% DOCUMENTO
\begin{document}
\maketitle
\parindent=5mm
\parskip=0mm
\begin{abstract}
\noindent Este documento muestra el antiguo sistema de unidades empleado por la monarquía hispánica y sus colonias americanas durante los siglos \textsc{xvi--xix} ---usado antes de la adopción del sistema métrico decimal---, y señala la importancia de desarrollar un paquete para \LaTeX\ similar a \textsf{siunitx}. El documento podría ser interesante para historiadores, economistas, metrólogos, entre otros, que deseen tener una referencia y un compendio detallado sobre este antiguo sistema de unidades.
\end{abstract}
\thispagestyle{empty}
\pagestyle{fancy}
\fancyhf{}
\fancyhead[CE,CO]{\scshape\small el antiguo sistema de unidades españolas}
\fancyhead[LE,RO]{\thepage}
\renewcommand{\headrulewidth}{0.4pt}
\microtypesetup{protrusion=false}
\renewcommand{\listtablename}{Índice de cuadros}
\listoftables
\microtypesetup{protrusion=true}
\section{Licencia del documento}
\label{sec:licencia-del-documento}
Este documento se proveé \emph{como es,} sin ningún tipo de garantía, aunque se espera que pueda ser útil, y se distribuye bajo una licencia \href{https://creativecommons.org/licenses/by/4.0/deed.es}{Creative Commons Atribución 4.0 Internacional (\textsc{cc by 4.0})}. Por lo tanto, eres libre de \emph{compartir, copiar y redistribuir} el material en cualquier medio o formato. \emph{Adaptar, remezclar, transformar y construir} a partir del material para cualquier propósito, incluso comercialmente. El licenciante no puede revocar estas libertades en tanto usted siga los términos de la licencia. Para mayores informes visite \url{https://creativecommons.org/licenses/by/4.0/deed.es}
\section{Bibliografía}
\label{sec:bibliografía}
\hangpara{1cm}{1}Héctor Vera, \emph{A peso el kilo. Historia del sistema métrico decimal en México.} Libros del escarabajo, México, 2007, 191 pp.

\hangpara{1cm}{1}José Antonio Bátiz Vázquez, «Cambios y permanencias en la moneda mexicana durante el siglo \textsc{xix}», en \emph{Memorias del Segundo Congreso de Historia Económica. La historia económica hoy, entre la economía y la historia}, Asociación Mexicana de Historia Económica \textls[30]{AC}, México \textls[30]{DF}, 2004 (27--29 de octubre). [En línea]: \url{http://www.economia.unam.mx/amhe/memoria/simposio10/Jose\%20Antonio\%20BATIZ.pdf} [Fecha de consulta: 18/09/2023].
\section{Cuadros}
\label{sec:cuadros}
% CUADRO (UNIDADES DE LONGITUD)
\renewcommand{\tablename}{Cuadro}
\begin{table}[h]\label{tab:longitud}
\centering
\caption[Unidades de longitud]{Unidades de longitud.}
\begin{tabular}{@{}lclllrrr@{}}
\toprule
\multirow{2}{*}{Unidad} & \multicolumn{7}{c}{Subunidades} \\
\cmidrule{2-8}
{} & medias & tercias & cuartas & sesmas & pulgadas & líneas & puntos \\
\midrule
Vara & \texttlf{2} & \texttlf{3} & \texttlf{4} & \texttlf{6} & \texttlf{36} & \texttlf{432} & \texttlf{5184} \\
Media & \texttlf{1} & \texttlf{1}\slantfrac{1}{2} & \texttlf{2} & \texttlf{3} & \texttlf{18} & \texttlf{216} & \texttlf{2592}  \\
Tercia o pie & {} & \texttlf{1} & \texttlf{1}\slantfrac{1}{3} & \texttlf{2} & \texttlf{12} & \texttlf{144} & \texttlf{1728} \\
Cuarta o palmo & {} & {} & \texttlf{1} & \texttlf{1}\slantfrac{1}{2} & \texttlf{9} & \texttlf{108} & \texttlf{1296} \\
Sesma & {} & {} & {} & \texttlf{1} & \texttlf{6} & \texttlf{72} & \texttlf{864} \\
Pulgada & {} & {} & {} & {} & \texttlf{1} & \texttlf{12} & \texttlf{144} \\
Línea & {} & {} & {} & {} & {} & \texttlf{1} & \texttlf{12} \\
Punto & {} & {} & {} & {} & {} & {} & \texttlf{1} \\
\bottomrule
\end{tabular}
\caption*{\textsc{Fuente:} Vera, 2007, p. 50.}
\end{table}
% CUADRO (UNIDADES AGRARIAS)
% Here siunitx package do not help because we are using common fractions. Dirty hack, use phantom.
\begin{table}[h]\label{tab:agrarias}
\centering
\caption[Unidades agrarias]{Unidades agrarias.}
\begin{tabular}{@{}lrrr@{}}
\toprule
\multirow{2}{*}{Unidad} & \multicolumn{3}{c}{Subunidades} \\
\cmidrule{2-4}
{} & largo (varas) & ancho (varas) & área (varas\textsuperscript{\figureversion{lf}2}) \\
\midrule
Sitio de ganado mayor & \texttlf{5000}\phantom{00} & \texttlf{5000}\phantom{00} & \texttlf{25000000}\phantom{00} \\
Criadero de ganado mayor & \texttlf{2500}\phantom{00} & \texttlf{2500}\phantom{00} & \texttlf{6250000}\phantom{00} \\
Sitio de ganado menor & \texttlf{3333}\slantfrac{1}{3} & \texttlf{3333}\slantfrac{1}{3} & \texttlf{11111111}\slantfrac{1}{9} \\
Criadero de ganado menor & \texttlf{1666}\slantfrac{2}{3} & \texttlf{1666}\slantfrac{2}{3} & \texttlf{2777777}\slantfrac{7}{9} \\
Caballería de tierra & \texttlf{1104}\phantom{00} & \texttlf{552}\phantom{00} & \texttlf{609408}\phantom{00} \\
Media caballería & \texttlf{552}\phantom{00} & \texttlf{552}\phantom{00} & \texttlf{304704}\phantom{00} \\
Cuarto de caballería o suerte de tierra & \texttlf{552}\phantom{00} & \texttlf{276}\phantom{00} & \texttlf{152352}\phantom{00} \\
Fanega de sembradura de maíz & \texttlf{276}\phantom{00} & \texttlf{184}\phantom{00} & \texttlf{50784}\phantom{00} \\
Solar para casa, molino o venta & \texttlf{50}\phantom{00} & \texttlf{50}\phantom{00} & \texttlf{2500}\phantom{00} \\
Fundo legal para pueblos & \texttlf{1200}\phantom{00} & \texttlf{1200}\phantom{00} & \texttlf{1440000}\phantom{00} \\
\bottomrule
\end{tabular}
\caption*{\textsc{Fuente:} Vera, 2007, p. 55.}
\end{table}
% CUADRO (UNIDADES PARA ÁRIDOS)
\begin{table}[h]\label{tab:áridos}
\centering
\caption[Unidades para áridos]{Unidades para áridos.}
\begin{tabular}{@{}lcccrrr@{}}
\toprule
\multirow{2}{*}{Unidad} & \multicolumn{5}{c}{Subunidades} \\
\cmidrule{2-6}
{} & fanegas & medias & cuartillas & almudes & cuartillos \\
\midrule
Carga & \texttlf{2} & \texttlf{4} & \texttlf{8} & \texttlf{24} & \texttlf{96} \\
Fanega & \texttlf{1} & \texttlf{2} & \texttlf{4} & \texttlf{12} & \texttlf{48} \\
Media fanega & {} & \texttlf{1} & \texttlf{2} & \texttlf{6} & \texttlf{24} \\
Cuartilla & {} & {} & \texttlf{1} & \texttlf{3} & \texttlf{12} \\
Almud & {} & {} & {} & \texttlf{1} & \texttlf{4} \\
Cuartillo & {} & {} & {} & {} & \texttlf{1} \\
\bottomrule
\end{tabular}
\caption*{\textsc{Fuente}: Vera, 2007, p. 59.}
\end{table}
% CUADRO (UNIDADES PARA LÍQUIDOS)
\begin{table}[h]\label{tab:líquidos}
\centering
\caption[Unidades para líquidos]{Unidades para líquidos.}
\begin{tabular}{@{}lcrrr@{}}
\toprule
\multirow{2}{*}{Unidad} & \multicolumn{4}{c}{Subunidades} \\
\cmidrule{2-5}
{} & barriles & jarras & cuartillos & medios \\
\midrule
Pipa & \texttlf{9} & \texttlf{82} & \texttlf{1458} & \texttlf{2916} \\
Barril & \texttlf{1} & \texttlf{9} & \texttlf{162} & \texttlf{324} \\
Jarra & {} & \texttlf{1} & \texttlf{18} & \texttlf{36} \\
Cuartillo & {} & {} & \texttlf{1} & \texttlf{2} \\
Medio & {} & {} & {} & \texttlf{1} \\
\bottomrule
\end{tabular}
\caption*{\textsc{Fuente}: Vera, 2007, p. 62.}
\end{table}
% CUADRO (UNIDADES HIDRÁULICAS)
\begin{table}[h]\label{tab:hidráulicas}
\centering
\caption[Unidades hidráulicas]{Unidades hidráulicas.}
\begin{tabular}{@{}lrrrrr@{}}
\toprule
\multirow{2}{*}{Unidad} & \multicolumn{5}{c}{Subunidades} \\
\cmidrule{2-6}
{} & surcos & naranjas & reales & dedos & pajas \\
\midrule
Buey & \texttlf{48} & \texttlf{144} & \texttlf{1152} & \texttlf{2304} & \texttlf{20736} \\
Surco & \texttlf{1} & \texttlf{3} & \texttlf{24} & \texttlf{48} & \texttlf{432} \\
Naranja & {} & \texttlf{1} & \texttlf{8} & \texttlf{16} & \texttlf{144} \\
Real o limón & {} & {} & \texttlf{1} & \texttlf{2} & \texttlf{18} \\
Dedo & {} & {} & {} & \texttlf{1} & \texttlf{9} \\
Paja & {} & {} & {} & {} & \texttlf{1} \\
\bottomrule
\end{tabular}
\caption*{\textsc{Fuente}: Vera, 2007, p. 63.}
\end{table}
% CUADRO (UNIDADES DE PESO)
\begin{table}[h]\label{tab:peso}
\centering
\caption[Unidades de peso]{Unidades de peso.}
\begin{tabular}{@{}lcrrrr@{}}
\toprule
\multirow{2}{*}{Unidad} & \multicolumn{5}{c}{Subunidades} \\
\cmidrule{2-6}
{} & arrobas & libras & onzas & adarmes & granos \\
\midrule
Quintal & \texttlf{4} & \texttlf{100} & \texttlf{1600} & \texttlf{25600} & \texttlf{921600} \\
Arroba & \texttlf{1} & \texttlf{25} & \texttlf{400} & \texttlf{6400} & \texttlf{230400} \\
Libra & {} & \texttlf{1} & \texttlf{16} & \texttlf{256} & \texttlf{9216} \\
Onza & {} & {} & \texttlf{1} & \texttlf{16} & \texttlf{576} \\
Adarme & {} & {} & {} & \texttlf{1} & \texttlf{36} \\
Grano & {} & {} & {} & {} & \texttlf{1} \\
\bottomrule
\end{tabular}
\caption*{\textsc{Fuente}: Vera, 2007, p. 66.}
\end{table}
% CUADRO (UNIDADES PARA EL ORO)
\begin{table}[h]\label{tab:oro}
\centering
\caption[Unidades para el oro (Au)]{Unidades para el oro.}
\begin{tabular}{@{}lccrcr@{}}
\toprule
\multirow{2}{*}{Unidad} & \multicolumn{5}{c}{Subunidades} \\
\cmidrule{2-6}
{} & libras & marcos & castellanos & tomines & granos \\
\midrule
Libra & \texttlf{1} & \texttlf{2} & {} & {} & {} \\
Marco & {} & \texttlf{1} & \texttlf{50} & {} & {} \\
Castellano & {} & {} & \texttlf{1} & \texttlf{8} & {} \\
Tomín & {} & {} & {} & \texttlf{1} & \texttlf{12} \\
Grano & {} & {} & {} & {} & \texttlf{1} \\
\bottomrule
\end{tabular}
\caption*{\textsc{Fuente}: Vera, 2007, p. 68.}
\end{table}
% CUADRO (UNIDADES PARA LA PLATA)
\begin{table}[h]\label{tab:plata}
\centering
\caption[Unidades para la plata (Ag)]{Unidades para la plata.}
\begin{tabular}{@{}lcccccr@{}}
\toprule
\multirow{2}{*}{Unidad} & \multicolumn{6}{c}{Subunidades} \\
\cmidrule{2-7}
{} & libras & marcos & onzas & ochavas & tomines & granos \\
\midrule
Libra & \texttlf{1} & \texttlf{2} & {} & {} & {} & {} \\
Marco & {} & \texttlf{1} & \texttlf{8} & {} & {} & {} \\
Onza & {} & {} & \texttlf{1} & \texttlf{8} & {} & {} \\
Ochava & {} & {} & {} & \texttlf{1} & \texttlf{6} & {} \\
Tomín & {} & {} & {} & {} & \texttlf{1} & \texttlf{12} \\
Grano & {} & {} & {} & {} & {} & \texttlf{1} \\
\bottomrule
\end{tabular}
\caption*{\textsc{Fuente}: Vera, 2007, p. 68.}
\end{table}
% CUADRO (UNIDADES MEDICINALES)
\begin{table}[h]\label{tab:medicinales}
\centering
\caption[Unidades medicinales]{Unidades medicinales.}
\begin{tabular}{@{}lcrccccc@{}}
\toprule
\multirow{2}{*}{Unidad} & \multicolumn{7}{c}{Subunidades} \\
\cmidrule{2-8}
{} & libras & onzas & dracmas & escrúpulos & óbolos & silicuas & granos \\
\midrule
Libra boticaria & \texttlf{1} & \texttlf{12} & {} & {} & {} & {} & {} \\
Onza & {} & \texttlf{1} & \texttlf{8} & {} & {} & {} & {} \\
Dracma & {} & {} & \texttlf{1} & \texttlf{3} & {} & {} & {} \\
Escrúpulo & {} & {} & {} & \texttlf{1} & \texttlf{2} & {} & {} \\
Óbolo & {} & {} & {} & {} & \texttlf{1} & \texttlf{3} & {} \\
Silicua & {} & {} & {} & {} & {} & \texttlf{1} & \texttlf{4} \\
Grano & {} & {} & {} & {} & {} & {} & \texttlf{1} \\
\bottomrule
\end{tabular}
\caption*{\textsc{Fuente}: Vera, 2007, p. 68.}
\end{table}
% CUADRO (UNIDADES MONETARIAS (Oro)
\begin{table}[h]\label{tab:monetarias-oro}
\centering
\caption[Unidades monetarias para el oro (Au)]{Unidades monetarias para el oro (Au).}
\begin{tabular}{@{}lccccr@{}}
\toprule
\multirow{2}{*}{Unidad} & \multicolumn{4}{c}{Subunidades} \\
\cmidrule{2-6}
{} & escudo & reales & cuartilla & tlaco & pilón \\
\midrule
Escudo & \texttlf{1} & \texttlf{16} & {} & \texttlf{128} & {} \\
\bottomrule
\end{tabular}
\caption*{\textsc{Fuente}: Bátiz, 2004, \emph{passim.}}
\end{table}
% CUADRO (UNIDADES MONETARIAS (Plata)
\begin{table}[h]\label{tab:monetarias-plata}
\centering
\caption[Unidades monetarias para la plata (Ag)]{Unidades monetarias para la plata (Ag).}
\begin{tabular}{@{}lccccr@{}}
\toprule
\multirow{2}{*}{Unidad} & \multicolumn{5}{c}{Subunidades} \\
\cmidrule{2-6}
{} & real & medio real (\slantfrac{1}{2}) & cuartilla (\slantfrac{1}{4}) & tlaco (\slantfrac{1}{8}) & pilón (\slantfrac{1}{16}) \\
\midrule
Real & \texttlf{1} & \texttlf{2} & \texttlf{4} & \texttlf{8} & \texttlf{16} \\
Medio real & {} & \texttlf{1} & \texttlf{2} & \texttlf{4} & \texttlf{8} \\
Cuartilla & {} & {} & \texttlf{1} & \texttlf{2} & \texttlf{4} \\
Tlaco & {} & {} & {} & \texttlf{1} & \texttlf{2} \\
Pilón & {} & {} & {} & {} & \texttlf{1} \\
\bottomrule
\end{tabular}
\caption*{\textsc{Fuente}: Bátiz, 2004, \emph{passim.}}
\end{table} 
\end{document}

