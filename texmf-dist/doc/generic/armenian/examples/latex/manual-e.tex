%%%%%%%%%%%%%%%%%%%%%%%%%%%%%%%%%%%%%%%%%%%%%%%%%%%%%%%%%%%%%%%%%%%%%%%%%%%%%%
%%
%% This is the `manual-e.tex' file (ArmTeX manual in English).
%%
%% This file is a part of the ArmTeX project [2022/08/14 v3.0-beta4]
%%
%% ArmTeX is a system for writing in Armenian with plain TeX and/or LaTeX(2e).
%%
%% Copyright 1997 - 2022:
%%   Serguei Dachian (Serguei.Dachian_AT_univ-lille.fr),
%%   Arnak Dalalyan  (arnak.dalalyan_AT_ensae.fr),
%%   Vardan Akopian  (vakopian_AT_yahoo.com).
%%
%% ArmTeX may be distributed and/or modified under the conditions of the LaTeX
%% Project Public License, either version 1.3 of this license or (at your
%% option) any later version.
%%
%% The latest version of this license is in
%%   http://www.latex-project.org/lppl.txt
%% and version 1.3 or later is part of all distributions of LaTeX version
%% 2005/12/01 or later.
%%
%% ArmTeX has the LPPL maintenance status `author-maintained'.
%%
%% For more details, installation instructions and the complete list of files
%% see the provided `README' file.
%%
%%%%%%%%%%%%%%%%%%%%%%%%%%%%%%%%%%%%%%%%%%%%%%%%%%%%%%%%%%%%%%%%%%%%%%%%%%%%%%


\documentclass[12pt,a4paper,draft]{article}


\expandafter\ifx\csname pdfglyphtounicode\endcsname\relax\else
\InputIfFileExists{glyphtounicode.tex}{}{}
\fi


\usepackage{mflogo}
\usepackage[latin]{armtex}

\def\mybs{\char'134}
\def\mybar{\char'174}
\def\mylbrace{\char'173}
\def\myrbrace{\char'175}
%\def\myindent{\leavevmode\hskip\parindent}
\def\myindent{\leavevmode}

\parskip=5pt


\title{\latArmTeX: a System for Writing in Armenian with \TeX\ and \LaTeX\\
{\normalsize\artm (\ArmTeX$\,$` $\,${\aroff \TeX}-um ev {\aroff \LaTeX}-um
Hayeren Lezvov Grelu Hamakarg)}}
\author{%
Sergue\"{\i} DACHIAN \thanks{{\tt Serguei.Dachian@univ-lille.fr}}\\
Arnak DALALYAN \thanks{{\tt arnak.dalalyan@ensae.fr}}\\
Vardan AKOPIAN \thanks{{\tt vakopian@yahoo.com}}
}
\date{June 1, 1999}



\begin{document}



\maketitle


\vglue -12cm
\noindent
%
\textsl{\textbf{WARNING!} This is the (almost unchanged) manual of the version
  2.0. It will be replaced by the manual of the version 3.0 before this beta
  release becomes official. A (temporary) brief description of the new
  features of \latArmTeX~3.0 can be found at the end of the ``README'' file.}
%
\vglue 10.43cm


\section{Introduction}

\myindent \latArmTeX\ is a system for writing in Armenian with \TeX\ and
\LaTeX. To use this system, you need to have a \TeX\ compiler (with plain
\TeX\ and/or \LaTeXe\ formats) as well as the \MF\ program. The system can be
used with a standard Latin keyboard (without any special support for Armenian
letters). It can also be used with any keyboard which uses an encoding having
Armenian letters in the second half (characters 128--255) of extended ASCII
table. An example of such an encoding is the ArmSCII8 Armenian standard.

\latArmTeX\ system is freeware. Feel free to give copies of it to your friends
and relatives, and be sure to include all the files. If you have any questions
and/or propositions, do not hesitate to contact us.

\latArmTeX\ installation instructions can be found in the ``README'' file.

To use the system, you need to know how to call it from your documents, how to
select different Armenian fonts, and how to enter Armenian text from the
keyboard. These operations are described in the following three sections.


\section{How to enter Armenian letters and punctuation signs}
\label{s2}

\myindent If you do not have an Armenian keyboard, Armenian letters must be
entered according to the following transliteration table.

\begin{table}[ht]
\centerline{\vbox{\offinterlineskip
\halign{\strut\vrule width1pt#&&\artm\quad #\ \hfill&\artm
#\quad\hfill&\vrule#&\quad\tt #\ \hfill&\vrule width1pt#\cr
\noalign{\hrule height 1pt depth 0pt}
&A&a&&a&&I&i&&i&&Y&y&&y&&T&t&&t&\cr
\noalign{\hrule height 1pt depth 0pt}
&B&b&&b&&L&l&&l&&N&n&&n&&R&r&&r&\cr
\noalign{\hrule height 1pt depth 0pt}
&G&g&&g&&X&x&&x&&Sh&sh&&sh &&C&c&&c&\cr
\noalign{\hrule height 1pt depth 0pt}
&D&d&&d&&C'&c'&&c'&&O&o&&o&&W&w&&w&\cr
\noalign{\hrule height 1pt depth 0pt}
&E&e&&e&&K&k&&k&&Ch&ch&&ch &&P'&p'&&p', ph&\cr
\noalign{\hrule height 1pt depth 0pt}
&Z&z&&z&&H&h&&h&&P&p&&p&&Q&q&&q&\cr
\noalign{\hrule height 1pt depth 0pt}
&E'&e'&&e'&&Dz&dz&&dz &&J&j&&j&&&ev&&ev&\cr
\noalign{\hrule height 1pt depth 0pt}
&U'&u'&&u'&&Gh&gh&&gh &&R'&r'&&r'&&O'&o'&&o'&\cr
\noalign{\hrule height 1pt depth 0pt}
&T'&t'&&t', th&&J'&j'&&j'&&S&s&&s&&F&f&&f&\cr
\noalign{\hrule height 1pt depth 0pt}
&G'&g'&&g'&&M&m&&m&&V&v&&v&&U&u&&u, ow&\cr
\noalign{\hrule height 1pt depth 0pt}
}}}
\caption{Transliteration}
\label{t1}
\end{table}

As you can see from this table, some Armenian letters have two possible
transliterations. For example, the letter {\artm <<th>>} can be obtained by
entering either [~{\tt t'}~] or [~{\tt th}~]. To obtain a capital Armenian
letter, the Latin letters in the corresponding transliteration (all, or only
the first one) must be capitalized. For example, in order to obtain the letter
{\artm <<TH>>}, you can enter [~{\tt T'}~], [~{\tt TH}~] or [~{\tt Th}~]. The
exceptions are the letter {\artm <<ev>>}, which does not have a capital, and
the letter {\artm <<u>>}, whose capital has two versions: by entering [~{\tt
    U}~] or [~{\tt OW}~] you obtain {\artm <<U>>}, while by entering [~{\tt
    Ow}~] you get {\artm <<Ow>>}.

Let us note that the transliteration scheme used in \latArmTeX\ 1.0 was
different from the one given in Table~\ref{t1}. More precisely:
%
\begin{itemize}
%
\item[--]the old transliterations [~{\tt z'}~] and [~{\tt zh}~] of the letter
  {\artm <<g'>>} are now replaced by~[~{\tt g'}~],
%
\item[--]the old transliteration [~{\tt ts}~] of the letter {\artm <<c'>>} is
  now replaced by [~{\tt c'}~],
%
\item[--]only the first of the old transliterations [~{\tt j'}~] and [~{\tt
    ch'}~] of the letter {\artm <<j'>>} is included in the new version of
  \latArmTeX.
%
\end{itemize}

Because of these modifications, a slight (re)editing of your own documents
written with \latArmTeX\ 1.0 can be necessary. We are sorry for this
inconvenience. The choice (and the modification) of the transliteration scheme
can seem surprising and/or unnecessary, but it is based on some objective
reasons which will be discussed at the end of this section.

Besides letters, the system possesses the following punctuation signs (and
miscellaneous symbols):

\begin{itemize}
\item[\artm .]\quad Armenian dot ({\artm mijaket}), entered as [~{\tt .}~]$\,$;
\item[\artm ,]\quad Armenian coma ({\artm storaket}), entered as [~{\tt ,}~]$\,$;
\item[\artm :]\quad Armenian full stop ({\artm verjaket}), entered as [~{\tt
    :}~]$\,$;
\item[\artm `]\quad Armenian separation mark ({\artm but'}), entered as [~{\tt
    `}~]$\,$;
\item[\artm |]\quad Armenian emphasis mark ({\artm shesht}), entered as
  [~{\tt\mybar}~]$\,$;
\item[\artm ?]\quad Armenian question mark ({\artm paruyg/harcakan
  nshan}),\\ \null\quad entered as [~{\tt ?}~]$\,$;
\item[\artm !]\quad Armenian exclamation mark ({\artm erkaracman/bacakanchakan
  nshan}),\\ \null\quad entered as [~{\tt !}~]$\,$;
\item[\artm -]\quad Armenian en-dash ({\artm miut'yan gc'ik}), entered as
  [~{\tt -}~]$\,$;
\item[\artm \|]\quad Armenian em-dash ({\artm anjatman gic'}), entered as
  [~{\tt \mybs textanjgic}~]\\ \null\quad or [~{\tt \mybs\mybar}~]$\,$;
\item[\artm --]\quad Armenian hyphen ({\artm ent'amna/toghadardzi nshan}),
  entered as [~{\tt -{\kern0.2em}-}~]$\,$;
\item[\artm ']\quad Armenian apostrophe ({\artm apat'arc}), entered as [~{\tt
    '}~]$\,$;
\item[\artm ...]\quad Armenian ellipsis ({\artm kaxman keter)}, entered as
  [~{\tt ...}~]$\,$;
\item[\artm ....]\quad Armenian multi-dot ({\artm bazmaket}), entered as
  [~{\tt ....}~]$\,$;
\item[\artm (]\quad left parenthesis, entered as [~{\tt (}~]$\,$;
\item[\artm )]\quad right parenthesis, entered as [~{\tt )}~]$\,$;
\item[{\artm [}]\quad left bracket, entered as [~{\tt [}~]$\,$;
\item[{\artm ]}]\quad right bracket, entered as [~{\tt ]}~]$\,$;
\item[\artm \{]\quad left brace, entered as [~{\tt \mybs textbraceleft}~] or
  [~{\tt \mybs\mylbrace}~]$\,$;
\item[\artm \}]\quad right brace, entered as [~{\tt \mybs textbraceright}~] or
  [~{\tt \mybs\myrbrace}~]$\,$;
\item[\artm \!]\quad Latin exclamation mark, entered as [~{\tt \mybs
    textexclam}~] or [~{\tt \mybs!}~]$\,$;
\item[\artm ;]\quad semicolon, entered as [~{\tt ;}~]$\,$;
\item[\artm ``]\quad English left quotation mark, entered as [~{\tt `{}`}~]$\,$;
\item[\artm '']\quad English right quotation mark, entered as [~{\tt '{}'}~]
  or [~{\tt "}~]$\,$;
\item[\artm \$]\quad dollar sign, entered as [~{\tt \mybs textdollar}~] or
  [~{\tt \mybs\$}~]$\,$;
\item[\artm \%]\quad percent sign, entered as [~{\tt \mybs textpercent}~] or
  [~{\tt \mybs\%}~]$\,$;
\item[\artm *]\quad asterisk, entered as [~{\tt *}~]$\,$;
\item[\artm +]\quad plus sign, entered as [~{\tt +}~]$\,$;
\item[\artm /]\quad slash, entered as [~{\tt /}~]$\,$;
\item[\artm <]\quad Armenian left quotation mark, entered as [~{\tt <}~] or
  [~{\tt <<}~]$\,$;
\item[\artm >]\quad Armenian right quotation mark, entered as [~{\tt >}~] or
  [~{\tt >>}~]$\,$;
\item[\artm =]\quad equality sign, entered as [~{\tt =}~]$\,$;
\item[\artm @]\quad at sign, entered as [~{\tt @}~]$\,$;
\item[\artm \?]\quad Latin question mark, entered as [~{\tt \mybs
    textquestion}~] or [~{\tt \mybs?}~]$\,$;
\item[\artm ---]\quad Latin em-dash, entered as [~{\tt
    -{\kern0.2em}-{\kern0.2em}-}~]$\,$;
\item[\artm \#]\quad hash, entered as [~{\tt \mybs texthash}~] or [~{\tt
    \mybs\#}~]$\,$;
\item[\artm \&]\quad ampersand, entered as [~{\tt \mybs textand}~] or [~{\tt
    \mybs\&}~]$\,$.
\end{itemize}

As you have probably noticed, there are two different ways (with one of them
of the form {\tt \mybs text...}) to enter some of the above symbols. The more
secure way to enter such symbols are the {\tt \mybs text...}\ commands. The
other commands producing the same symbols are, of course, more convenient, but
may conflict with other \LaTeX\ packages, or even with future releases of
\LaTeX. As we well see in Section~\ref{lload}, the user can even disable some
of the latter commands.

If you possess an Armenian keyboard --- which uses an encoding having Armenian
letters in the second half (characters 128--255) of extended ASCII table ---
you can also enter the Armenian letters and symbols present on the keyboard
directly.

Note that, unfortunately, the hyphenation in Armenian text is not performed
automatically. However, you can manually hyphenate any word, using \LaTeX's
{\tt\mybs -} and \ArmTeX's {\tt\mybs armuh} (ARMenian Unconditional
Hyphenation) commands. The last command does an unconditional hyphenation and
must be used for the words having the vowel {\artm <<u'>>} in a ``secret
syllable'' ({\artm gaghtnavank}) and when breaking the ligature {\artm
  <<ev>>}. For example, you can enter [~{\tt si\mybs-ra\mybs-marg}~], [~{\tt
    bu'\mybs armuh\ nuthyun}~] or [~{\tt Se\mybs armuh\ van}~].

Finally remark, that in some very rare cases, the obtained result may differ
from your expectations. For example, if the letter {\artm <<t>>} is
immediately followed by the letter {\artm <<h>>}, then one would naturally use
the transliteration~[~{\tt th}~].  However, the latter will be interpreted as
the letter {\artm <<th>>}. To avoid such misinterpretations, instead of [~{\tt
    th}~] one can enter [~{\tt t\mybs textbreaklig h}~] or [~{\tt t\mybs
    *h}~]. Let us note, that the second principle used for the creation and
perfection of our transliteration scheme (the first one, of course, being the
phonetic correspondence with the Eastern Armenian) is the minimization of
similar misinterpretations. For example, while typing this manual, we never
encountered such cases (except, of course, the intentionally given
examples). Below, we give the full list of such known misinterpretations.
%
\begin{enumerate}
%
\item All the words containing the pair of letters {\artm <<e\*v>>}. For
  example, the word {\artm <<tare\*verj>>} can be transliterated as [~{\tt
      tare\mybs *verj}~].
%
\item An apostrophe or an English right quotation mark following some
  letters. For example, if you enter [~{\tt ``mat''}~], you will unexpectedly
  obtain {\artm <<``mat''>>} instead of {\artm <<``mat">>}. Note that in this
  case, besides using the command {\tt \mybs *}, you have also the following
  solution: [~{\tt ``mat"}~].
%
\item The following words and the words derived from them (including compound
  words):
\begin{itemize}
\item[--]{\artm d\*zzal} [~{\tt d\mybs *zzal}~],
\item[--]{\artm t\*haj'} [~{\tt t\mybs *haj'}~],
\item[--]{\artm t\*has} [~{\tt t\mybs *has}~].
\end{itemize}
%
\end{enumerate}
%
If you encounter similar cases, do not hesitate to let us know about them, so
we can complete the above list.


\section{How to call \latArmTeX\ from your documents}
\subsection{\LaTeX\ case}
\label{lload}

\myindent In order to use \latArmTeX\ in \LaTeX, first of all you must load
the {\tt armtex} \LaTeX\ package by adding the command

{\tt \mybs usepackage\mylbrace armtex\myrbrace}

\noindent to the preamble of your document (between the commands {\tt \mybs
  documentclass} and {\tt \mybs begin\mylbrace document\myrbrace}).

The package accepts the following optional arguments: {\tt latin}, {\tt
  notstar}, {\tt notbar}, {\tt notexclam}, {\tt notdots} and {\tt safe}. These
arguments can be combined like, for example,

{\tt \mybs usepackage[latin,notbar,notexclam]\mylbrace armtex\myrbrace}\qquad.

Now, let us describe each of these arguments.

If the argument {\tt latin} is not present, \latArmTeX\ will typeset in
Armenian the main text of your document, as well as the table of contents,
chapter and section names, and so on. However, if this is not the desired
behavior, you can specify the argument {\tt latin} and use the font selection
commands described in Section~\ref{lfonts} in order to typeset Armenian text.

\latArmTeX\ redefines the standard \LaTeX\ commands {\tt \mybs*}, {\tt \mybs|}
and {\tt \mybs!}, preserving their meaning in math mode. If you run into
problems related to these redefinitions, you can disable each of them using
the optional arguments {\tt notstar}, {\tt notbar} or {\tt notexclam}
respectively.

\LaTeX\ commands {\tt \mybs vdots} and {\tt \mybs ddots} use dots from the
current font (in contrast to other commands producing dots in math
mode). \latArmTeX\ corrects this strange behavior of \LaTeX\ by redefining
these commands. If you run into problems related to these redefinitions, you
can disable them using the argument {\tt notdots}.

If you need to disable simultaneously the five above-mentioned redefinitions,
you can use the argument {\tt safe}.

Let us note that in the Armenian fonts used in \latArmTeX, the letters and
symbols are positioned according to the {\tt OT6} encoding. If you use the
{\tt armtex} \LaTeX\ package, the {\tt OT6} encoding is loaded
automatically. In this case, you do not need (and, moreover, you must not)
load the {\tt OT6} encoding ``manually'', that is, using {\tt fontenc}
standard \LaTeX\ package.

Finally, in order to use an Armenian keyboard, you must call the
\LaTeX\ standard package {\tt inputenc}, specifying the name of the encoding
produced by your keyboard as an argument (in lower case) like, for example,

{\tt \mybs usepackage[armscii8]\mylbrace inputenc\myrbrace}\qquad.

Unfortunately, only the {\tt ArmSCII8} encoding is provided with
\latArmTeX~2.0. It is defined in the ``armscii8.def'' file, which can be used
as a template for designing similar files for other encodings. In the case you
do write and test such an encoding file, do not hesitate to send it to us, so
we can include it in future releases of \latArmTeX.


\subsection{Plain \TeX\ case}

\myindent In order to use \latArmTeX\ in plain \TeX, first of all you must
load the ``arm.tex'' file by putting the following command in the beginning of
your document:

{\tt \mybs input arm}\qquad.

In order to use an Armenian keyboard, you must additionally load the
``armkb-a8.tex'' file:

{\tt \mybs input armkb-a8}\qquad.

The ``armkb-a8.tex'' file is designed for the {\tt ArmSCII8} encoding. It can
be used as a template for designing similar files for other encodings. In the
case you do write and test such an encoding file, do not hesitate to send it
to us, so we can include it in future releases of \latArmTeX.


\section{Font selection related commands}
\subsection{\LaTeX\ case}
\label{lfonts}

\subsubsection{Orthogonal commands}

\myindent \LaTeXe\ possesses a very flexible font selection system. A font is
determined by five parameters (encoding, family, series, shape and size),
which can be changed independently (in an orthogonal way). For example, the
Computer Modern Roman (cmr) family is chosen by {\tt \mybs rmfamily} command,
and italic shape by the {\tt \mybs itshape} command. So, for example, entering

{\tt \mylbrace\mybs rmfamily\mybs itshape cat\myrbrace}

\noindent will typeset the word ``cat'' in the italic shape of the cmr
family. Let us also note, that each orthogonal command has an equivalent of
the form {\tt \mybs text...}, which applies the corresponding change to the
text given in the argument. For example, an equivalent way to write the
previous example is

{\tt \mybs textrm\mylbrace \mybs textit\mylbrace cat\myrbrace\myrbrace}\qquad.

\latArmTeX\ provides two family of fonts: cmr and cmss. The first family
contains the bold and medium series of the normal, italic and slanted
shapes. The second one contains the bold and medium series of the normal and
slanted shapes. These fonts can be selected using the orthogonal commands
listed in Table~\ref{t2}.

\begin{table}[ht]
\centerline{\vbox{\offinterlineskip
\halign{\vphantom{$\Big($}\vrule width1pt#&\tt\quad\mybs #\hfill\quad&\vrule #
&\tt\quad\mybs #\hfill\quad&\vrule width1pt#\cr
\noalign{\hrule height 1pt depth 0pt}
&artmfamily&&armtm&\cr
\noalign{\hrule height 1pt depth 0pt}
&arssfamily&&armss&\cr
\noalign{\hrule height 1pt depth 0pt}
&armdseries&&armmd&\cr
\noalign{\hrule height 1pt depth 0pt}
&arbfseries&&armbf&\cr
\noalign{\hrule height 1pt depth 0pt}
&arupshape&&armup&\cr
\noalign{\hrule height 1pt depth 0pt}
&aritshape&&armit&\cr
\noalign{\hrule height 1pt depth 0pt}
&arslshape&&armsl&\cr
\noalign{\hrule height 1pt depth 0pt}
}}}
\caption{Orthogonal commands}
\label{t2}
\end{table}

The font selection commands starting with {\tt\mybs arm} correspond to
\LaTeX\ commands starting with {\tt\mybs text}. Note that the commands listed
in Table~\ref{t2} are orthogonal with each other, but are not orthogonal with
\LaTeX\ standard font selection commands. This is due to the fact that the
commands from Table~\ref{t2}, if they are called when \LaTeX\ is not already
in Armenian mode, first enter Armenian mode by performing a series of actions
which (together with the command for leaving the Armenian mode) will be
described in Section~\ref{zan}.

Finally, let us note that for historical reasons, as well as for compatibility
with \latArmTeX~1.0, the commands for selecting cmr and cmss families use {\tt
  tm} and {\tt ss} ``roots'' respectively (instead of traditional {\tt rm} and
{\tt sf}).

\subsubsection{Non-orthogonal (old style) commands}

\myindent Besides orthogonal commands described in the previous section,
\LaTeX\ also possesses old style (inherited from plain TeX) non-orthogonal
font selection commands {\tt \mybs rm}, {\tt \mybs sf}, {\tt \mybs tt}, {\tt
  \mybs bf}, {\tt \mybs it}, {\tt \mybs sl} and {\tt \mybs sc} (some of which
work also in math mode). \latArmTeX\ redefines these commands so, that if they
are called when \LaTeX\ is in Armenian mode, they first leave the Armenian
mode (see Section~\ref{zan} for more details).

\begin{table}[ht]
\centerline{\vbox{\offinterlineskip
\halign{\vphantom{$\Big($}\vrule width1pt#&\tt\quad\mybs #\hfill\quad&\vrule #
&\quad\artm #\hfill\quad&\vrule width1pt#\cr
\noalign{\hrule height 1pt depth 0pt}
&artm&&*&\cr
\noalign{\hrule height 1pt depth 0pt}
&artmit&&*&\cr
\noalign{\hrule height 1pt depth 0pt}
&artmsl&&&\cr
\noalign{\hrule height 1pt depth 0pt}
&artmbf&&*&\cr
\noalign{\hrule height 1pt depth 0pt}
&artmbfit&&*&\cr
\noalign{\hrule height 1pt depth 0pt}
&artmbfsl&&&\cr
\noalign{\hrule height 1pt depth 0pt}
&arss&&&\cr
\noalign{\hrule height 1pt depth 0pt}
&arsssl&&&\cr
\noalign{\hrule height 1pt depth 0pt}
&arssbf&&&\cr
\noalign{\hrule height 1pt depth 0pt}
&arssbfsl&&&\cr
\noalign{\hrule height 1pt depth 0pt}
}}}
\caption{Non-orthogonal commands}
\label{t3}
\end{table}

For selecting Armenian fonts, \latArmTeX\ provides its own non-orthogonal
commands which are listed in Table~\ref{t3}. The commands marked with an
asterisk work also in math mode.

\subsubsection{Math mode commands}

\myindent For selecting fonts in math mode, \LaTeXe\ provides equally (new
style) commands like {\tt \mybs mathrm}, {\tt \mybs mathbf}, and so on.  For
example, in the formula~$\mathbf P(\xi=\eta)$, the bold letter ``P'' can be
obtained by entering either [~{\tt \$\mylbrace\mybs bf~P\myrbrace(\mybs
    xi=\mybs eta)\$}~] or [~{\tt \$\mybs mathbf\mylbrace P\myrbrace(\mybs
    xi=\mybs eta)\$}~]:

\begin{table}[ht]
\centerline{\vbox{\offinterlineskip
\halign{\vphantom{$\Big($}\vrule width1pt#&\tt\quad\mybs #\hfill\quad
&\vrule width1pt#\cr
\noalign{\hrule height 1pt depth 0pt}
&mathartm&\cr
\noalign{\hrule height 1pt depth 0pt}
&mathartmit&\cr
\noalign{\hrule height 1pt depth 0pt}
&mathartmbf&\cr
\noalign{\hrule height 1pt depth 0pt}
&mathartmbfit&\cr
\noalign{\hrule height 1pt depth 0pt}
}}}
\caption{Math mode commands}
\label{t4}
\end{table}

For selecting Armenian fonts in math mode, \latArmTeX\ provides analogous
commands which are listed in Table~\ref{t4}.

Note that the commands from the last table correspond to the commands marked
with an asterisk in Table~\ref{t3}.

\subsubsection{Miscellaneous commands}
\label{zan}

\myindent As we have already mentioned in the previous sections, the commands
for selecting Armenian fonts ``if they are called when \LaTeX\ is not already
in Armenian mode, first enter Armenian mode''. Entering Armenian mode consists
in selecting the normal Armenian font ({\tt \mybs artm}) and performing the
commands {\tt \mybs armdate} and {\tt \mybs armhyph}. The last two commands
adapt the commands {\tt \mybs today} and {\tt \mybs-} to Armenian language. To
restore the original behavior of these commands, you can use the commands {\tt
  \mybs armdateoff} and {\tt \mybs armhyphoff} respectively. And the command
to leave the Armenian mode is {\tt \mybs aroff}, which selects the normal
non-Armenian font and performs the commands {\tt \mybs armdateoff} and {\tt
  \mybs armhyphoff}.

\latArmTeX\ provides also the commands {\tt \mybs armnames} and {\tt \mybs
  armnamesoff}, which respectively translate to Armenian and restore to the
original state the words like ``Chapter'', ``Part'', ``Table'', and so on. In
principle, there is no need to use these commands directly, but this can
become necessary when using, for example, the {\tt babel} package.

Finally, let us mention the commands {\tt \mybs latArmTeX} and {\tt \mybs
  ArmTeX} which typeset the logos \latArmTeX\ and \ArmTeX\ respectively. These
commands work irrespectively of whether \LaTeX\ is in Armenian mode or not,
and are (in both cases) orthogonal to the corresponding font selection
commands.


\subsection{Plain \TeX\ case}

\myindent Plain \TeX\ does not possess the flexible font selection system of
\LaTeXe. For this reason, among the above described commands, only the
commands {\tt \mybs armdate}, {\tt \mybs armhyph}, {\tt \mybs armdateoff},
{\tt \mybs armhyphoff}, {\tt \mybs aroff}, {\tt \mybs latArmTeX} and {\tt
  \mybs ArmTeX}, as well as the commands listed in Table~\ref{t3}, are
available when using \latArmTeX\ in plain \TeX.

Another difference is that the commands {\tt \mybs rm}, {\tt \mybs sf}, {\tt
  \mybs tt}, {\tt \mybs bf}, {\tt \mybs it}, {\tt \mybs sl} and {\tt \mybs sc}
do not possess the feature to leave the Armenian mode. If you have to leave
the Armenian mode, you can use the command {\tt \mybs aroff}. The latter, uses
the value of {\tt \mybs arofffont} for selecting the normal non-Armenian
font. The default value of {\tt \mybs arofffont} is {\tt\mybs rm}, but it can
be changed by entering, for example, [~{\tt \mybs let\mybs arofffont=\mybs
    bf}~].

Finally, let us note that the commands {\tt \mybs latArmTeX} and {\tt \mybs
  ArmTeX} work in a slightly different way than in \LaTeX. The first
difference is that the command {\tt \mybs ArmTeX} (resp.\ {\tt \mybs
  latArmTeX}) must be used only when \TeX\ is (resp.\ is not) in Armenian
mode. Besides, in order for the command {\tt \mybs ArmTeX} to produce the
desired result, it may be necessary to change the value of {\tt \mybs
  arofffont}. For example, in order to typeset \armbf{<<\ArmTeX>>}, you must
enter

{\tt\mylbrace \mybs artmbf \mybs let\mybs arofffont=\mybs bf \mybs
  ArmTeX\myrbrace}\qquad.




\newpage\appendix
\noindent{\LARGE\bfseries Appendix}
\section{{\tt\bfseries OT6} character tables}
\subsection{{\tt\bfseries artmr10} font}

\def\mystrut{{\vrule height 18pt depth 9pt width 0pt}}
\def\mychar#1{{\char#1}}
\def\aghyusak{%
$$
\vbox{\offinterlineskip
\halign{\vrule##&\quad\mystrut $##$\quad
&&\vrule##&\quad\hfill\mystrut{\trialfont ##}\hfill\quad\cr
\noalign{\hrule}
&&&~$0$~&&~$1$~&&~$2$~&&~$3$~&&~$4$~&&~$5$~&&~$6$~&&~$7$~&\cr
\noalign{\hrule}
%
&'00&&   &&\mychar{'001}&&\mychar{'002}&&\mychar{'003}&&\mychar
{'004}&&\mychar{'005}&&\mychar{'006}&&\mychar{'007}&\cr
\noalign{\hrule}
%
&'01&&\mychar{'010}&&\mychar{'011}&&\mychar{'012}&&\mychar{'013}&&\mychar
{'014}&&\mychar{'015}&&\mychar{'016}&&\mychar{'017}&\cr
\noalign{\hrule}
%
&'02&&   &&\mychar{'021}&&\mychar{'022}&&\mychar{'023}&&\mychar
{'024}&&\mychar{'025}&&\mychar{'026}&&\mychar{'027}&\cr
\noalign{\hrule}
%
&'03&&\mychar{'030}&&\mychar{'031}&&\mychar{'032}&&\mychar{'033}&&\mychar
{'034}&&\mychar{'035}&&\mychar{'036}&&\mychar{'037}&\cr
\noalign{\hrule}
%
&'04&&\mychar{'040}&&\mychar{'041}&&\mychar{'042}&&\mychar{'043}&&\mychar
{'044}&&\mychar{'045}&&\mychar{'046}&&\mychar{'047}&\cr
\noalign{\hrule}
%
&'05&&\mychar{'050}&&\mychar{'051}&&\mychar{'052}&&\mychar{'053}&&\mychar
{'054}&&\mychar{'055}&&\mychar{'056}&&\mychar{'057}&\cr
\noalign{\hrule}
%
&'06&&\mychar{'060}&&\mychar{'061}&&\mychar{'062}&&\mychar{'063}&&\mychar
{'064}&&\mychar{'065}&&\mychar{'066}&&\mychar{'067}&\cr
\noalign{\hrule}
%
&'07&&\mychar{'070}&&\mychar{'071}&&\mychar{'072}&&\mychar{'073}&&\mychar
{'074}&&\mychar{'075}&&\mychar{'076}&&\mychar{'077}&\cr
\noalign{\hrule}
%
&'10&&\mychar{'100}&&\mychar{'101}&&\mychar{'102}&&\mychar{'103}&&\mychar
{'104}&&\mychar{'105}&&\mychar{'106}&&\mychar{'107}&\cr
\noalign{\hrule}
%
&'11&&\mychar{'110}&&\mychar{'111}&&\mychar{'112}&&\mychar{'113}&&\mychar
{'114}&&\mychar{'115}&&\mychar{'116}&&\mychar{'117}&\cr
\noalign{\hrule}
%
&'12&&\mychar{'120}&&\mychar{'121}&&\mychar{'122}&&\mychar{'123}&&\mychar
{'124}&&\mychar{'125}&&\mychar{'126}&&\mychar{'127}&\cr
\noalign{\hrule}
%
&'13&&\mychar{'130}&&\mychar{'131}&&\mychar{'132}&&\mychar{'133}&&\mychar
{'134}&&\mychar{'135}&&\mychar{'136}&&\mychar{'137}&\cr
\noalign{\hrule}
%
&'14&&\mychar{'140}&&\mychar{'141}&&\mychar{'142}&&\mychar{'143}&&\mychar
{'144}&&\mychar{'145}&&\mychar{'146}&&\mychar{'147}&\cr
\noalign{\hrule}
%
&'15&&\mychar{'150}&&\mychar{'151}&&\mychar{'152}&&\mychar{'153}&&\mychar
{'154}&&\mychar{'155}&&\mychar{'156}&&\mychar{'157}&\cr
\noalign{\hrule}
%
&'16&&\mychar{'160}&&\mychar{'161}&&\mychar{'162}&&\mychar{'163}&&\mychar
{'164}&&\mychar{'165}&&\mychar{'166}&&\mychar{'167}&\cr
\noalign{\hrule}
%
&'17&&\mychar{'170}&&\mychar{'171}&&\mychar{'172}&&\mychar{'173}&&\mychar
{'174}&&\mychar{'175}&&\mychar{'176}&&\mychar{'177}&\cr
\noalign{\hrule}
}}
$$
}
\font\trialfont=artmr10 scaled 1600
\aghyusak


\newpage
\subsection{{\tt\bfseries arssr10} font}

\font\trialfont=arssr10 scaled 1600
\aghyusak


\newpage
\section{\latArmTeX\ version history}

\myindent \textbf{\latArmTeX\ 1.0}\ (June 25, 1997). This is the first version
of \latArmTeX. It consists essentially of Armenian fonts and provides a
minimal set of simple commands for using them in plain \TeX.

\bigskip\bigskip

\noindent \textbf{\latArmTeX\ 2.0}\ (June 1, 1999). This version of
\latArmTeX\ includes the following important changes from the previous one.
\begin{itemize}
%
\item[--]The transliteration scheme was improved (see Section~\ref{s2}).
%
\item[--]The set of \latArmTeX\ commands was essentially extended and a
  package for using the system with \LaTeXe\ was created.
%
\item[--]Some minor bugs were corrected.
%
\item[--]The Armenian manual and several example files for both \LaTeX\ and
  plain \TeX\ were written.
%
\item[--]Some unnecessary files were removed.
%
\end{itemize}



\end{document}
