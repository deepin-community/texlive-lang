%%%%%%%%%%%%%%%%%%%%%%%%%%%%%%%%%%%%%%%%%%%%%%%%%%%%%%%%%%%%%%%%%%%%%%%%%%%%%%
%%
%% This is the `manual.tex' file (ArmTeX manual in Armenian).
%%
%% This file is a part of the ArmTeX project [2022/08/14 v3.0-beta4]
%%
%% ArmTeX is a system for writing in Armenian with plain TeX and/or LaTeX(2e).
%%
%% Copyright 1997 - 2022:
%%   Serguei Dachian (Serguei.Dachian_AT_univ-lille.fr),
%%   Arnak Dalalyan  (arnak.dalalyan_AT_ensae.fr),
%%   Vardan Akopian  (vakopian_AT_yahoo.com).
%%
%% ArmTeX may be distributed and/or modified under the conditions of the LaTeX
%% Project Public License, either version 1.3 of this license or (at your
%% option) any later version.
%%
%% The latest version of this license is in
%%   http://www.latex-project.org/lppl.txt
%% and version 1.3 or later is part of all distributions of LaTeX version
%% 2005/12/01 or later.
%%
%% ArmTeX has the LPPL maintenance status `author-maintained'.
%%
%% For more details, installation instructions and the complete list of files
%% see the provided `README' file.
%%
%%%%%%%%%%%%%%%%%%%%%%%%%%%%%%%%%%%%%%%%%%%%%%%%%%%%%%%%%%%%%%%%%%%%%%%%%%%%%%


\documentclass[12pt,a4paper]{article}


\expandafter\ifx\csname pdfglyphtounicode\endcsname\relax\else
\InputIfFileExists{glyphtounicode.tex}{}{}
\fi


\usepackage{mflogo}
\usepackage{armtex}


\def\mybs{\char'134}
\def\mybar{\char'174}
\def\mylbrace{\char'173}
\def\myrbrace{\char'175}
\def\myindent{\leavevmode\hskip\parindent}

\parskip=5pt


\title{\ArmTeX$\,$` $\,${\aroff \TeX}-um ev {\aroff \LaTeX}-um Hayeren Lezvov 
Grelu Hamakarg\\
{\normalsize\aroff (\latArmTeX: a System for Writing in Armenian with \TeX\ and
\LaTeX)}}
\author{%
Sergey DASHYAN  \thanks{{\tt Serguei.Dachian@univ-lille.fr}}\\
Ar'nak DALALYAN \thanks{{\tt arnak.dalalyan@ensae.fr}}\\
Vardan HAKOBYAN \thanks{{\tt vakopian@yahoo.com}}
}
\date{1-u' hunisi 1999 th.}



\begin{document}



\maketitle


\vglue -12cm
\noindent
%
\textsl{\textbf{OWSHADROWT'YO|WN:} Sa tarberak 2.0-i (grethe anphophox)
  dzer'narkn e': Ayn kphoxarinvi tarberak 3.0-i dzer'narkov naxqan ays beta
  tho\-ghark\-man pashtonakanacowmu': \ArmTeX~3.0-i nor hnaravoruthyunneri
  (g'a\-ma\-na\-ka\-vor) hamar'ot nkaragrowmu' (angleren lezvov) karogh eq
  gu't\armuh nel~{\sl ``README''} fayli verjum:}
%
\vglue 9.93cm
%\vglue 10.43cm


\section{Nerac'uthyun}

\myindent \ArmTeX-u' {\aroff \TeX}-um ev {\aroff \LaTeX}-um hayeren lezvov
grelu hamakarg e': Ays hamakargu' o'gtagorc'elu hamar anhrag'esht e' unenal
{\aroff \TeX} c'ragiru' (og'tvac' {\aroff plain \TeX} ev/kam {\aroff \LaTeXe}
formatnerov), inchpes naev {\aroff \MF} c'ragiru': Hamakargu' karogh e'
o'gtagorc'vel inchpes latinakan steghnashar unecogh hamakargchi vra (ar'anc
oreve' hatuk ha\-ye\-re\-nac\-man), aynpes e'l kamayakan haykakan steghnashar
unecoghi vra, ayn paymanov or hayeren aybubenu' gtnvi u'ndlaynvac' {\tt ASCII}
aghyusaki erk\-rord kesum, aysinqn 128-ic 255 dirqerum: Ayd paymanin e'
bavararum, o'rinaki hamar, {\tt ArmSCII8} haykakan standartu':

\ArmTeX\ hamakargu' anvj'ar e' ({\aroff freeware}): Karogh eq ayn ar'anc
mta\-hog\-ve\-lu bag'anel dzer u'nkernerin ev barekamnerin, miayn ayn paymanov
or phoxanceq bolor fayleru': Mi tatanveq mez dimel bolor harcerov ev
ar'ajarkuthyunnerov:

\ArmTeX-i teghadrman ({\rm installation}-i) hamar anhrag'esht bolor
te\-ghe\-ku\-thyun\-ne\-ru' gtnvum en {\rm ``README''} faylum:

Hamakargu' o'gtagorc'elu hamar bavakan e' karoghanal ayn kanchel dzer
fayleric, tirapetel tar'qatesakneru' phoxogh hramannerin ev i\-ma\-nal the
inchpes petq e' nermuc'el teqstu' steghnasharic: Ays
gor\-c'o\-ghu\-thyun\-ne\-ru' nkaragrvac' en hajordogh ereq bag'innerum:


\section{Tar'eri ev ketadrakan nshanneri greladzevu'}
\label{s2}

\myindent Ethe dzer hamakargichu' o'g'tvac' che' haykakan steghnasharov, apa
ta\-r'e\-ru' petq e' nermuc'ven tar'adardzuthyan hetevyal aghyusaki hamadzayn`

\begin{table}[ht]
\centerline{\vbox{\offinterlineskip
\halign{\strut\vrule width1pt#&&\artm\quad #\ \hfill&\artm
#\quad\hfill&\vrule#&\quad\tt #\ \hfill&\vrule width1pt#\cr
\noalign{\hrule height 1pt depth 0pt}
&A&a&&a&&I&i&&i&&Y&y&&y&&T&t&&t&\cr
\noalign{\hrule height 1pt depth 0pt}
&B&b&&b&&L&l&&l&&N&n&&n&&R&r&&r&\cr
\noalign{\hrule height 1pt depth 0pt}
&G&g&&g&&X&x&&x&&Sh&sh&&sh &&C&c&&c&\cr
\noalign{\hrule height 1pt depth 0pt}
&D&d&&d&&C'&c'&&c'&&O&o&&o&&W&w&&w&\cr
\noalign{\hrule height 1pt depth 0pt}
&E&e&&e&&K&k&&k&&Ch&ch&&ch &&P'&p'&&p', ph&\cr
\noalign{\hrule height 1pt depth 0pt}
&Z&z&&z&&H&h&&h&&P&p&&p&&Q&q&&q&\cr
\noalign{\hrule height 1pt depth 0pt}
&E'&e'&&e'&&Dz&dz&&dz &&J&j&&j&&&ev&&ev&\cr
\noalign{\hrule height 1pt depth 0pt}
&U'&u'&&u'&&Gh&gh&&gh &&R'&r'&&r'&&O'&o'&&o'&\cr
\noalign{\hrule height 1pt depth 0pt}
&T'&t'&&t', th&&J'&j'&&j'&&S&s&&s&&F&f&&f&\cr
\noalign{\hrule height 1pt depth 0pt}
&G'&g'&&g'&&M&m&&m&&V&v&&v&&U&u&&u, ow&\cr
\noalign{\hrule height 1pt depth 0pt}
}}}
\caption{Tar'adardzuthyun:}
\label{t1}
\end{table}

Inchpes erevum e' ays aghyusakic, orosh tar'er unen krknaki
ta\-r'a\-dar\-dzu\-thyun. o'rinak <<th>> tar'u' stanalu hamar kareli e'
nermuc'el inchpes [~{\tt t'}~], aynpes e'l [~{\tt th}~]: Mec'atar'eru' stanalu
hamar petq e' ha\-ma\-pa\-tas\-xan tar'adardzuthyan mej latinakan
phoqratar'eru' (boloru' kam miayn ar'ajinu') phoxarinel mec'atar'erov: O'rinak
<<TH>> tar'u' sta\-na\-lu hamar kareli e' nermuc'el [~{\tt T'}~], [~{\tt TH}~]
kam [~{\tt Th}~]: Bacar'uthyunnern en <<ev>>-u', oru' mec'atar' chuni, ev
<<u>>-n, ori mec'atar'u' uni erku greladzev. nermuc'elov [~{\tt U}~] kam
[~{\tt OW}~] kstanaq <<U>>, isk nermuc'elov [~{\tt Ow}~]$\,$` <<Ow>>:

Nshenq, or \ArmTeX\ 1.0-i o'gtagorc'ac' tar'adardzuthyan hamakargu' tarbervum
e' Aghyusak~\ref{t1}-um bervac'ic: Aveli j'shgrit`
\begin{itemize}
\item[--]<<g'>> tar'i [~{\tt z'}~] ev [~{\tt zh}~] hin greladzeveru'
  phoxarinvel en [~{\tt g'}~]-ov,
\item[--]<<c'>> tar'i [~{\tt ts}~] hin greladzevu' phoxarinvel e' [~{\tt
    c'}~]-ov,
\item[--]<<j'>> tar'i [~{\tt j'}~] ev [~{\tt ch'}~] hin greladzeveric
  \ArmTeX-i nor tar\-be\-ra\-kum nergravvel e' miayn ar'ajinu':
\end{itemize}

Ays tarberuthyunneri patj'ar'ov, \ArmTeX\ 1.0-ov grvac' dzer fayleru' karogh
en phophoxuthyunneri kariq unenal: Neroghamit klineq dzez ays neghuthyunu'
patj'ar'elu hamar: Tar'adardzuthyan ayspisi u'ntruthyunu' ev phophoxuthyunu'
omanc mot karogh e' zarmanq kam dg'gohuthyun a\-r'a\-jac\-nel, sakayn ayn uni
o'byektiv patj'ar'ner, oronq kqnnarkven suyn bag'ni verjum:

Baci tar'eric hamakargu' o'g'tvac' e' naev hetevyal ketadrakan (ev och miayn)
nshannerov`

\begin{itemize}
\item[\artm .]\quad mijaket, greladzevu'` [~{\tt .}~]$\,$,
\item[\artm ,]\quad storaket, greladzevu'` [~{\tt ,}~]$\,$,
\item[\artm :]\quad verjaket, greladzevu'` [~{\tt :}~]$\,$,
\item[\artm `]\quad but', greladzevu'` [~{\tt `}~]$\,$,
\item[\artm |]\quad shesht, greladzevu'` [~{\tt\mybar}~]$\,$,
\item[\artm ?]\quad paruyg/harcakan nshan, greladzevu'` [~{\tt ?}~]$\,$,
\item[\artm !]\quad erkaracman/bacakanchakan nshan, greladzevu'` [~{\tt
    !}~]$\,$,
\item[\artm -]\quad miut'yan gc'ik, greladzevu'` [~{\tt -}~]$\,$,
\item[\artm \|]\quad anjatman gic', greladzevu'` [~{\tt \mybs textanjgic}~]
  kam [~{\tt \mybs\mybar}~]$\,$,
\item[\artm --]\quad ent'amna/toghadardzi nshan, greladzevu'` [~{\tt
    -{\kern0.2em}-}~]$\,$,
\item[\artm ']\quad apat'arc, greladzevu'` [~{\tt '}~]$\,$,
\item[\artm ...]\quad kaxman keter, greladzevu'` [~{\tt ...}~]$\,$,
\item[\artm ....]\quad bazmaket, greladzevu'` [~{\tt ....}~]$\,$,
\item[\artm (]\quad dzax kor p'akagic', greladzevu'` [~{\tt (}~]$\,$,
\item[\artm )]\quad aj kor p'akagic', greladzevu'` [~{\tt )}~]$\,$,
\item[{\artm [}]\quad dzax ughigh p'akagic', greladzevu'` [~{\tt [}~]$\,$,
\item[{\artm ]}]\quad aj ughigh p'akagic', greladzevu'` [~{\tt ]}~]$\,$,
\item[\artm \{]\quad dzax dzevavor p'akagic', greladzevu'` [~{\tt \mybs
    textbraceleft}~] kam [~{\tt \mybs\mylbrace}~]$\,$,
\item[\artm \}]\quad aj dzevavor p'akagic', greladzevu'` [~{\tt \mybs
    textbraceright}~] kam [~{\tt \mybs\myrbrace}~]$\,$,
\item[\artm \!]\quad lat. bacakanchakan nshan, greladzevu'` [~{\tt \mybs
    textexclam}~] kam [~{\tt \mybs!}~]$\,$,
\item[\artm ;]\quad ket-storaket, greladzevu'` [~{\tt ;}~]$\,$,
\item[\artm ``]\quad bacvogh angliakan chakert, greladzevu'` [~{\tt
    `{}`}~]$\,$,
\item[\artm '']\quad phakvogh angliakan chakert, greladzevu'` [~{\tt '{}'}~]
  kam [~{\tt "}~]$\,$,
\item[\artm \$]\quad dolari nshan, greladzevu'` [~{\tt \mybs textdollar}~] kam
  [~{\tt \mybs\$}~]$\,$,
\item[\artm \%]\quad tokosi nshan, greladzevu'` [~{\tt \mybs textpercent}~]
  kam [~{\tt \mybs\%}~]$\,$,
\item[\artm *]\quad astghanish, greladzevu'` [~{\tt *}~]$\,$,
\item[\artm +]\quad gumarman nshan, greladzevu'` [~{\tt +}~]$\,$,
\item[\artm /]\quad kotoraki nshan, greladzevu'` [~{\tt /}~]$\,$,
\item[\artm <]\quad bacvogh chakert, greladzevu'` [~{\tt <}~] kam [~{\tt
    <<}~]$\,$,
\item[\artm >]\quad phakvogh chakert, greladzevu'` [~{\tt >}~] kam [~{\tt
    >>}~]$\,$,
\item[\artm =]\quad havasaruthyan nshan, greladzevu'` [~{\tt =}~]$\,$,
\item[\artm @]\quad {\rm at}-i nshan, greladzevu'` [~{\tt @}~]$\,$,
\item[\artm \?]\quad lat. harcakan nshan, greladzevu'` [~{\tt \mybs
    textquestion}~] kam [~{\tt \mybs?}~]$\,$,
\item[\artm ---]\quad lat. erkar gic', greladzevu'` [~{\tt
    -{\kern0.2em}-{\kern0.2em}-}~]$\,$,
\item[\artm \#]\quad angliakan hamari nshan, greladzevu'` [~{\tt \mybs
    texthash}~] kam [~{\tt \mybs\#}~]$\,$,
\item[\artm \&]\quad angleren <<ev>>, greladzevu'` [~{\tt \mybs textand}~] kam
  [~{\tt \mybs\&}~]$\,$:
\end{itemize}

Inchpes nkateciq, orosh nshanner unen erku greladzev, oroncic meku' uni {\tt
  \mybs text...}\ tesqu': Aydpisi nshanner tpelu ar'avel vstaheli mijocu' {\tt
  \mybs text...}\ hramani o'gtagorc'umn e', minchder' nuyn nshanu' tpogh myus
hra\-ma\-nu', linelov aveli harmar o'gtagorc'elu hamar, karogh e'
an\-ha\-ma\-te\-ghe\-li linel {\aroff\LaTeX}-i orosh phathethneri ({\rm
  package}-neri), kam nuynisk {\aroff\LaTeX}-i apaga tarberakneri het: Inchpes
ktesnenq Bag'in~\ref{lload}-um, o'g\-ta\-gor\-c'o\-ghu' nuynisk hnaravoruthyun
uni <<anjatel>> ayd hramanneric omanq:

Ethe dzer hamakargichu' o'g'tvac' e' haykakan steghnasharov, apa i havelumn
veru' nkaragrvac' mijoci, steghnashari vra goyuthyun unecogh tar'eru' ev
nshanneru' karogh en nermuc'vel anmijakanoren:

Nshenq, or dg'baxtabar hayeren teqstum toghadardzumu' inq\-na\-be\-ra\-bar chi
katarvum: Ayn iragorc'elu hamar karogh eq o'gtagorc'el bun {\rm \LaTeX}-i {\tt
  \mybs-} ev \ArmTeX-i {\tt\mybs armuh} ({\aroff ARMenian Unconditional
  Hyphenation}) hramanneru': Verjin hramanu' katarum e' och paymanakan
to\-gha\-dar\-dzum ev imast uni o'gtagorc'el gaghtnavanki <<u'>> unecogh
ba\-r'e\-rum ev <<ev>> kcagiru' toghadardzov kiselis: O'rinak, karogh eq
nermuc'el [~{\tt si\mybs-ra\mybs-marg}~], [~{\tt bu'\mybs armuh\ nuthyun}~]
kam [~{\tt Se\mybs armuh\ van}~]:

Verjapes nkatenq, or shat hazvagyut depqerum, stacvac' ar\-dyun\-qu' karogh e'
tarbervel naxatesvac'ic: O'rinak, ethe <<t>> tar'in anmijapes hajordum e'
<<h>> tar'u', bnakan kliner o'gtagorc'el [~{\tt th}~]
ta\-r'a\-dar\-dzu\-thyu\-nu', sakayn verjins kmeknabanvi orpes <<th>> tar'u':
Nman thyurimacuthyunneric karogh eq xusaphel [~{\tt th}~]-i phoxaren
nermuc'elov [~{\tt t\mybs textbreaklig h}~] kam [~{\tt t\mybs *h}~]: Nshenq,
or mer tar'adardzuthyan ha\-ma\-kar\-gi steghc'man ev katarelacman erkrord
skzbunqn e' eghel (ar'ajinu' iharke linelov arevelahayereni het fonetik
hamapatasxanuthyunu') nman thyurimacuthyunneri minimizacumu': O'rinak, suyn
dzer'narku' kaz\-me\-lis, menq och mi angam aydpisi depqeri chenq handipel
(bnakanabar, baci ditavoryal bervac' o'rinakneric): Storev mejberum enq mez
hayt\-ni aydpisi thyurimacuthyunneri lriv canku'.
\begin{enumerate}
%
\item <<e\*v>> tar'axumbu' parunakogh bolor bar'eru', o'rinak` <<tare\*verj>>
  bar'u' karogh e' tar'adardzvel orpes [~{\tt tare\mybs *verj}~]:
%
\item Apatharcu' kam angliakan chakertu'` orosh tar'erin hajordelis, o'rinak
  [~{\tt ``mat''}~] nermuc'elis anspaselioren kstacvi <<``mat''>>, ayl och the
  <<``mat">>: Nshenq or ays depqum, baci {\tt \mybs *} hramani
  o'g\-ta\-gor\-c'u\-mic, ka naev hetevyal luc'umu'` [~{\tt ``mat"}~]:
%
\item Hetevyal bar'eru' ev nrancov kazmvac' barduthyunneru'`
\begin{itemize}
\item[--]d\*zzal [~{\tt d\mybs *zzal}~],
\item[--]t\*haj' [~{\tt t\mybs *haj'}~],
\item[--]t\*has [~{\tt t\mybs *has}~]:
\end{itemize}
%
\end{enumerate}
Nuynanman depqer gtnelis, xndrum enq mez imacuthyan mej dnel` veru' bervac'
canku' lrivacnelu npatakov:


\section{Inchpes kanchel \ArmTeX-u' dzer fayleric}
\subsection{{\bf \LaTeX}-i depqum}
\label{lload}

\myindent \ArmTeX-u' {\rm \LaTeX}-um o'gtagorc'elu hamar, naxevar'aj petq e'
kanchel {\rm\LaTeX}-i {\tt armtex} phathethu'` hetevyal hramanu'.

{\tt \mybs usepackage\mylbrace armtex\myrbrace}

\noindent tpelov dzer fayli skzbnamasum ({\tt \mybs documentclass} ev {\tt
  \mybs begin\mylbrace document\myrbrace} hramanneri mijev):


Ays phathethin kareli e' tal hetevyal fakultativ argumentneru'` {\tt latin},
{\tt notstar}, {\tt notbar}, {\tt notexclam}, {\tt notdots} ev {\tt safe}: Ays
argumentneru' karogh en o'gtagorc'vel miag'amanak, o'rinak`

{\tt \mybs usepackage[latin,notbar,notexclam]\mylbrace armtex\myrbrace}\qquad:

Ayg'm nkaragrenq ays argumentneri imastneru':

\ArmTeX-u'` {\tt latin} argumenti bacakayuthyan depqum, hayeren lezvov e' tpum
dzer dokumenti himnakan teqstu', inchpes naev bo\-van\-da\-ku\-thyu\-nu',
gluxneri ev bag'inneri anunneru' ev ayln: Sakayn, ethe da dzer uzac'u' che',
apa karogh eq o'gtagorc'el {\tt latin} argumentu' ev hayeren teqst sta\-na\-lu
hamar kirar'el tar'atesakneru' phoxogh hramanneru', oronq nka\-ra\-grvac' en
\ref{lfonts}~bag'num:

\ArmTeX-u' verasahmanum e' {\tt \mybs*}, {\tt \mybs|} ev {\tt \mybs!} {\rm
  \LaTeX}-i standart hra\-man\-ne\-ru'` pahpanelov nranc imastu'
mathematikakan eghanakum: Ethe dzer mot aydpisi verasahmanumneri het kapvac'
problemner ar'ajanan, apa karogh eq drancic yuraqanchyuru' <<anjatel>>`
o'gtagorc'elov ha\-ma\-pa\-tas\-xa\-na\-bar {\tt notstar}, {\tt notbar} kam
{\tt notexclam} fakultativ ar\-gu\-ment\-ne\-ru':

{\rm \LaTeX}-u' {\tt \mybs vdots} ev {\tt \mybs ddots} hramanneru'
iragorc'elis keteru' vercnum e' u'nthacik tar'atesakic (i tarberuthyun
mathematikakan eghanakum keter dnogh myus hramanneri): \ArmTeX-u' ughghum e'
{\rm \LaTeX}-i ayd tar\-o'\-ri\-nak varqu'` verohishyal hramanneru'
verasahmanelov: Ethe dzer mot aydpisi verasahmanumneri het kapvac' problemner
ar'ajanan, apa karogh eq ayn <<anjatel>> o'gtagorc'elov {\tt notdots}
argumentu':

Ethe duq npatakaharmar eq gtnum miag'amanak <<anjatel>> veru' nkaragrvac'
bolor hing verasahmanumneru', apa karogh eq o'gtagorc'el {\tt safe}
argumentu':

Nshenq, or \ArmTeX-um o'gtagorc'vogh tar'atesaknerum tar'eru' ev nu'\armuh
shanneru' dasavorvac' en {\tt OT6} dirqabashxman hamadzayn: Ethe
o'g\-ta\-gor\-c'um eq {\rm \LaTeX}-i {\tt armtex} phathethu', apa {\tt OT6}-u'
kanchvum e' inqnu'stinqyan: Ayd depqum och miayn kariq chka, ayl naev chi
kareli {\tt OT6}-u' kanchel <<dzer'qov>>, aysinqn o'gtagorc'elov {\rm
  \LaTeX}-i standart {\tt fontenc} phathethu':

Verjapes, hayeren steghnashar o'gtagorc'elu hamar petq e' kancheq
{\rm\LaTeX}-i standart {\tt inputenc} phathethu'` dzer steghnasharin
ha\-ma\-pa\-tas\-xa\-nogh dirqabashxman anunu' (phoqratar'erov) orpes argument
ta\-lov, o'rinak`

{\tt \mybs usepackage[armscii8]\mylbrace inputenc\myrbrace}\qquad:

Dg'baxtabar, \ArmTeX\ 2.0-um naxatesvac' e' miayn {\tt ArmSCII8}
dir\-qa\-bash\-xu\-mu': Verjins sahmanvac' e' {\aroff ``armscii8.def''}
faylum, oru' karogh eq orpes himq o'gtagorc'el ayl dirqabashxumner sahmanelu
hamar: Owrax klinenq stanal dzer sahmanac' ev phordzarkac' ayl
dirqabashxumneru'` \ArmTeX-i hetaga tarberaknerum u'ndgrkelu npatakov:


\subsection{{\bf Plain \TeX}-i depqum}

\myindent \ArmTeX-u' {\rm plain \TeX}-um o'gtagorc'elu hamar, naxevar'aj petq
e' kan\-chel {\rm ``arm.tex''} faylu'` hetevyal hramanu' teghadrelov dzer
fayli skzbum.

{\tt \mybs input arm}\qquad:

Hayeren steghnashar o'gtagorc'elu hamar anhrag'esht e' kanchel naev {\rm
  ``armkb-a8.tex''} faylu'`

{\tt \mybs input armkb-a8}\qquad:

Verohishyal {\rm ``armkb-a8.tex''} faylu' naxatesvac' e' {\tt ArmSCII8}
dir\-qa\-bashx\-man hamar: Ayd faylu' karogh eq himq u'ndunel ayl
dirqabashxumner sahmanelu hamar: Owrax klinenq stanal dzer sahmanac' ev
phordzarkac' ayl dirqabashxumneru'` \ArmTeX-i hetaga tarberaknerum u'ndgrkelu
npatakov:


\section{Tar'atesakneru' phoxelu het kapvac' hra\-man\-ner}
\subsection{{\bf \LaTeX}-i depqum}
\label{lfonts}

\subsubsection{O'rthogonal hramanner}

\myindent {\rm\LaTeXe}-u' uni tar'atesakner phoxelu j'kun
hamakarg. ta\-r'a\-te\-sa\-ku' oroshvum e' hing parametrerov ({\rm encoding,
  family, series, shape, size}), oronq karogh en phophoxvel iraric ankax
(o'rthogonal dzevov): O'rinak` {\rm Computer Modern Roman (cmr)} u'ntaniqu'
u'ntrvum e' {\tt \mybs rmfamily} hra\-ma\-nov, isk italik dzevu'` {\tt \mybs
  itshape} hramanov: Ayspisov, nermuc'elov o'rinak

{\tt \mylbrace\mybs rmfamily\mybs itshape cat\myrbrace}\qquad,

\noindent angleren {\rm ``cat''} bar'u' ktpvi {\rm cmr} u'ntaniqi italik
dzevov: Nshenq naev, or bolor o'rthogonal hramanneru' unen {\tt \mybs text...}
tipi hamarg'eq, oru' ha\-ma\-pa\-tas\-xan phophoxuthyunu' kirar'um e'
argumentum gtnvogh teqsti vra: O'rinak, naxord o'rinaki hamarg'eq greladzevn
e'`

{\tt \mybs textrm\mylbrace \mybs textit\mylbrace cat\myrbrace\myrbrace}\qquad:

\ArmTeX-u' parunakum e' tar'atesakneri erku u'ntaniq` {\rm cmr} ev {\rm cmss}:
Ar'ajin u'ntaniqu' uni ereq dzeveri (normal, italik ev shegh) thav ({\rm
  bold}) ev normal ({\rm medium}) tarberakneru': Erkrord u'ntaniqu' uni erku
dzeveri (normal ev shegh) thav ev normal tarberakneru': Ayd tar'atesakneru'
ka\-re\-li e' u'ntrel o'gtagorc'elov Aghyusak~\ref{t2}-owm bervac' o'rthogonal
hramanneru':

\begin{table}[ht]
\centerline{\vbox{\offinterlineskip
\halign{\vphantom{$\Big($}\vrule width1pt#&\tt\quad\mybs #\hfill\quad&\vrule #
&\tt\quad\mybs #\hfill\quad&\vrule width1pt#\cr
\noalign{\hrule height 1pt depth 0pt}
&artmfamily&&armtm&\cr
\noalign{\hrule height 1pt depth 0pt}
&arssfamily&&armss&\cr
\noalign{\hrule height 1pt depth 0pt}
&armdseries&&armmd&\cr
\noalign{\hrule height 1pt depth 0pt}
&arbfseries&&armbf&\cr
\noalign{\hrule height 1pt depth 0pt}
&arupshape&&armup&\cr
\noalign{\hrule height 1pt depth 0pt}
&aritshape&&armit&\cr
\noalign{\hrule height 1pt depth 0pt}
&arslshape&&armsl&\cr
\noalign{\hrule height 1pt depth 0pt}
}}}
\caption{O'rthogonal hramanner:}
\label{t2}
\end{table}

Tar'atesak phoxogh ayn hramanneru', oronq sksvum en {\tt\mybs arm}-ov,
ha\-ma\-pa\-tas\-xa\-num en {\rm \LaTeX}-i {\tt\mybs text}-ov sksvogh
hramannerin: Nshenq, or Aghyusak~\ref{t2}-i hramanneru' linelov phoxadardzabar
o'rthogonal, o'r\-tho\-go\-nal chen {\rm \LaTeX}-i standart hramanneri het:
Dra patj'ar'n ayn e', or haykakan eghanakum chgtnvelu depqum,
Aghyusak~\ref{t2}-i hramanneru' na\-xa\-pes mtnum en haykakan eghanak, inchu'
kayanum e' mi sharq gor\-c'o\-ghu\-thyun\-ne\-ri katarman mej, oronq, haykakan
eghanakic durs galu hramani het miasin, knkaragrven Bag'in~\ref{zan}-um:

Verjapes nshenq, or patmakan patj'ar'nerov, inchpes naev \ArmTeX\ 1.0-i het
hamategheli linelu npatakov, {\rm cmr} ev {\rm cmss} u'ntaniqneru' u'nt\-rogh
hramanneri anunneru' o'gtagorc'um en hamapatasxanabar {\tt tm} ev {\tt ss}
<<armatneru'>>` avandakan {\tt rm}-i ev {\tt sf}-i phoxaren:

\subsubsection{Och o'rthogonal (hin tipi) hramanner}

\myindent Baci naxord bag'num nkaragrvac' o'rthogonal hramanneric, {\rm
  \LaTeX}-u' uni tar'atesakner phoxelu hin tipi ({\rm plain \TeX}-ic
g'ar'angac') och o'rthogonal hramanner` {\tt \mybs rm}, {\tt \mybs sf}, {\tt
  \mybs tt}, {\tt \mybs bf}, {\tt \mybs it}, {\tt \mybs sl} ev {\tt \mybs sc}
(oroncic omanq gorc'um en naev mathematikakan eghanakum): Ays hramanneru'
\ArmTeX-u' verasahmanum e' aynpes, or haykakan eghanakum gtnvelu depqum,
naxapes durs gan haykakan eghanakic, inchu' hangamanoren nka\-ra\-grvac' e'
Bag'in~\ref{zan}-um:

\begin{table}[ht]
\centerline{\vbox{\offinterlineskip
\halign{\vphantom{$\Big($}\vrule width1pt#&\tt\quad\mybs #\hfill\quad&\vrule #
&\quad\artm #\hfill\quad&\vrule width1pt#\cr
\noalign{\hrule height 1pt depth 0pt}
&artm&&*&\cr
\noalign{\hrule height 1pt depth 0pt}
&artmit&&*&\cr
\noalign{\hrule height 1pt depth 0pt}
&artmsl&&&\cr
\noalign{\hrule height 1pt depth 0pt}
&artmbf&&*&\cr
\noalign{\hrule height 1pt depth 0pt}
&artmbfit&&*&\cr
\noalign{\hrule height 1pt depth 0pt}
&artmbfsl&&&\cr
\noalign{\hrule height 1pt depth 0pt}
&arss&&&\cr
\noalign{\hrule height 1pt depth 0pt}
&arsssl&&&\cr
\noalign{\hrule height 1pt depth 0pt}
&arssbf&&&\cr
\noalign{\hrule height 1pt depth 0pt}
&arssbfsl&&&\cr
\noalign{\hrule height 1pt depth 0pt}
}}}
\caption{Och o'rthogonal hramanner:}
\label{t3}
\end{table}

Baci ayd \ArmTeX-u' uni haykakan tar'atesakner u'ntrogh sephakan och
o'rthogonal hramanner, oronq bervac' en Aghyusak~\ref{t3}-um: Ayd
a\-ghyu\-sa\-kum astghanishov nshvac' hramanneru' gorc'um en naev
mathematikakan eghanakum:

\subsubsection{Mathematikakan eghanaki hramanner}

\myindent{\rm \LaTeXe}-u' mathematikakan eghanakum tar'atesakner phophoxelu
hamar naxatesel e' naev (nor tipi) aynpisi hramanner, inchpisiq en {\tt \mybs
  mathrm}, {\tt \mybs mathbf} ev ayln: O'rinak, $\mathbf P(\xi=\eta)$
banadzevowm, thav {\rm ``P''} tar'u' karogh eq stanal nermuc'elov inchpes naev
[~{\tt \$\mylbrace\mybs bf P\myrbrace(\mybs xi=\mybs eta)\$}~], ayn\-pes e'l
[~{\tt \$\mybs mathbf\mylbrace P\myrbrace(\mybs xi=\mybs eta)\$}~]:

\begin{table}[ht]
\centerline{\vbox{\offinterlineskip
\halign{\vphantom{$\Big($}\vrule width1pt#&\tt\quad\mybs #\hfill\quad
&\vrule width1pt#\cr
\noalign{\hrule height 1pt depth 0pt}
&mathartm&\cr
\noalign{\hrule height 1pt depth 0pt}
&mathartmit&\cr
\noalign{\hrule height 1pt depth 0pt}
&mathartmbf&\cr
\noalign{\hrule height 1pt depth 0pt}
&mathartmbfit&\cr
\noalign{\hrule height 1pt depth 0pt}
}}}
\caption{Mathematikakan eghanaki hramanner:}
\label{t4}
\end{table}

Ays hramanneri analogiayov, \ArmTeX-u' sahmanum e' ma\-the\-ma\-ti\-ka\-kan
eghanakum haykakan tar'atesakner u'ntrogh hramanner, oronq bervac' en
Aghyusak~\ref{t4}-um:

Hnaramit u'nthercoghu' knkati, or ays verjin aghyusaki hramanneru'
hamapatasxanum en Aghyusak~\ref{t3}-i astghanishov hramannerin:

\subsubsection{Zanazan hramanner}
\label{zan}

\myindent Inchpes nshel enq naxord bag'innerum, hayeren tar'atesakner
u'nt\-rogh ashxatogh hramanneru' <<haykakan eghanakum chgtnvelu depqum,
naxapes mu'tnum en haykakan eghanak>>: Haykakan eghanak mtnelu' kayanum e'
haykakan normal tar'atesaki ({\tt \mybs artm}-i) u'ntruthyan, inch\-pes naev
{\tt \mybs armdate} ev {\tt \mybs armhyph} hramanneri katarman mej: Verjin
erku hramanneru' harmarecnum en {\tt \mybs today} ev {\tt \mybs-} hramanneri
ashxateladzeveru' hayeren lezvin: Ayd hramanneri ashxateladzeveru' skzbnakanin
be\-re\-lu hamar kareli e' o'gtagorc'el hamapatasxanabar {\tt \mybs
  armdateoff} ev {\tt \mybs armhyphoff} hramanneru': Isk hayeren eghanakic
durs galu hramann e' {\tt \mybs aroff}-u', ori ashxatanqu' kayanum e' och
haykakan normal tar'atesaki u'ntruthyan, inchpes naev {\tt \mybs armdateoff}
ev {\tt \mybs armhyphoff} hramanneri ka\-tar\-man mej:
 
Goyuthyun unen naev {\tt \mybs armnames} ev {\tt \mybs armnamesoff}
hramanneru', oronq hamapatasxanabar hayeren en dardznum ev verakangnum en
skzbna\-kan vij'aki <<Glux>>, <<Mas>>, <<Aghyusak>> ev nmanatip myus bar'eru':
Skzbunqoren ays hramanneru' o'gtagorc'elu kariqu' chka, bayc karogh e'
ar'ajanal o'rinak {\tt babel} phathethu' o'gtagorc'elis:

Verjapes hishatakenq {\tt \mybs latArmTeX} ev {\tt \mybs ArmTeX} hramanneru',
oronq verartadrum en hamapatasxanabar \latArmTeX\ ev \ArmTeX\ logoneru': Ays
hramanneru' gorc'um en inchpes hayeren, aynpes e'l och hayeren
e\-gha\-nak\-ne\-rum, ev o'rthogonal en hamapatasxan eghanaki tar'atesak
phoxogh hramannerin:


\subsection{{\bf Plain \TeX}-i depqum}

\myindent {\rm Plain \TeX}-u' chuni tar'atesakneri het ashxatelu ayn j'kun
ha\-ma\-kar\-gu', orov o'g'tvac' e' {\rm \LaTeXe}-u': Ayd isk patj'ar'ov,
\ArmTeX-u' o'g\-ta\-gor\-c'e\-lis {\rm plain \TeX}-um, veru' nkaragrac'
hramanneric gorc'um en miayn {\tt \mybs armdate}, {\tt \mybs armhyph}, {\tt
  \mybs armdateoff}, {\tt \mybs armhyphoff}, {\tt \mybs aroff}, {\tt \mybs
  latArmTeX} ev {\tt \mybs ArmTeX}, inchpes naev Aghyusak~\ref{t3}-um trvac'
hramanneru':

Myus tarberuthyunn ayn e', or {\tt \mybs rm}, {\tt \mybs sf}, {\tt \mybs tt},
{\tt \mybs bf}, {\tt \mybs it}, {\tt \mybs sl} ev {\tt \mybs sc} hramanneru'
o'g'tvac' chen hayeren eghanakic inqnuruyn durs galu hat\-ku\-thyamb: Hayeren
eghanakic durs galu anhrag'eshtuthyan depqum, ka\-rogh eq o'gtagorc'el {\tt
  \mybs aroff} hramanu': Verjins och haykakan normal ta\-r'a\-te\-sa\-ki
u'ntruthyunu' iragorc'elu hamar o'gtagorc'um e' {\tt \mybs arofffont}-i
arg'equ': Ayd arg'equ' {\tt\mybs rm} e', bayc ayn kareli e' phoxel,
nermowc'elov o'rinak [~{\tt \mybs let\mybs arofffont=\mybs bf}~]:

Verjapes andradar'nanq {\tt \mybs latArmTeX} ev {\tt \mybs ArmTeX}
hramannerin: Ays hramanneru' gorc'um en grethe inchpes {\rm \LaTeX}-i depqum:
Mi tar\-be\-ru\-thyunn ayn e', or nranq petq e' o'gtagorc'ven hamapatasxan
e\-gha\-nak\-ne\-rum: Baci ayd, orpeszi {\tt \mybs ArmTeX} hramanu' hangecni
cankali ardyunqi, karogh e' anhrag'esht linel phoxel {\tt \mybs arofffont}-i
arg'equ'. o'rinak \armbf{<<\ArmTeX>>} stanalu hamar petq e' nermuc'el

{\tt\mylbrace \mybs artmbf \mybs let\mybs arofffont=\mybs bf \mybs
ArmTeX\myrbrace}\qquad:




\newpage\appendix
\noindent{\LARGE\arbfseries Havelvac'}
\section{{\tt\bfseries OT6} dirqabashxman aghyusakner}
\subsection{{\tt\bfseries artmr10} tar'atesaku'}

\def\mystrut{{\vrule height 18pt depth 9pt width 0pt}}
\def\mychar#1{{\char#1}}
\def\aghyusak{%
$$
\vbox{\offinterlineskip
\halign{\vrule##&\quad\mystrut $##$\quad
&&\vrule##&\quad\hfill\mystrut{\trialfont ##}\hfill\quad\cr
\noalign{\hrule}
&&&~$0$~&&~$1$~&&~$2$~&&~$3$~&&~$4$~&&~$5$~&&~$6$~&&~$7$~&\cr
\noalign{\hrule}
%
&'00&&   &&\mychar{'001}&&\mychar{'002}&&\mychar{'003}&&\mychar
{'004}&&\mychar{'005}&&\mychar{'006}&&\mychar{'007}&\cr
\noalign{\hrule}
%
&'01&&\mychar{'010}&&\mychar{'011}&&\mychar{'012}&&\mychar{'013}&&\mychar
{'014}&&\mychar{'015}&&\mychar{'016}&&\mychar{'017}&\cr
\noalign{\hrule}
%
&'02&&   &&\mychar{'021}&&\mychar{'022}&&\mychar{'023}&&\mychar
{'024}&&\mychar{'025}&&\mychar{'026}&&\mychar{'027}&\cr
\noalign{\hrule}
%
&'03&&\mychar{'030}&&\mychar{'031}&&\mychar{'032}&&\mychar{'033}&&\mychar
{'034}&&\mychar{'035}&&\mychar{'036}&&\mychar{'037}&\cr
\noalign{\hrule}
%
&'04&&\mychar{'040}&&\mychar{'041}&&\mychar{'042}&&\mychar{'043}&&\mychar
{'044}&&\mychar{'045}&&\mychar{'046}&&\mychar{'047}&\cr
\noalign{\hrule}
%
&'05&&\mychar{'050}&&\mychar{'051}&&\mychar{'052}&&\mychar{'053}&&\mychar
{'054}&&\mychar{'055}&&\mychar{'056}&&\mychar{'057}&\cr
\noalign{\hrule}
%
&'06&&\mychar{'060}&&\mychar{'061}&&\mychar{'062}&&\mychar{'063}&&\mychar
{'064}&&\mychar{'065}&&\mychar{'066}&&\mychar{'067}&\cr
\noalign{\hrule}
%
&'07&&\mychar{'070}&&\mychar{'071}&&\mychar{'072}&&\mychar{'073}&&\mychar
{'074}&&\mychar{'075}&&\mychar{'076}&&\mychar{'077}&\cr
\noalign{\hrule}
%
&'10&&\mychar{'100}&&\mychar{'101}&&\mychar{'102}&&\mychar{'103}&&\mychar
{'104}&&\mychar{'105}&&\mychar{'106}&&\mychar{'107}&\cr
\noalign{\hrule}
%
&'11&&\mychar{'110}&&\mychar{'111}&&\mychar{'112}&&\mychar{'113}&&\mychar
{'114}&&\mychar{'115}&&\mychar{'116}&&\mychar{'117}&\cr
\noalign{\hrule}
%
&'12&&\mychar{'120}&&\mychar{'121}&&\mychar{'122}&&\mychar{'123}&&\mychar
{'124}&&\mychar{'125}&&\mychar{'126}&&\mychar{'127}&\cr
\noalign{\hrule}
%
&'13&&\mychar{'130}&&\mychar{'131}&&\mychar{'132}&&\mychar{'133}&&\mychar
{'134}&&\mychar{'135}&&\mychar{'136}&&\mychar{'137}&\cr
\noalign{\hrule}
%
&'14&&\mychar{'140}&&\mychar{'141}&&\mychar{'142}&&\mychar{'143}&&\mychar
{'144}&&\mychar{'145}&&\mychar{'146}&&\mychar{'147}&\cr
\noalign{\hrule}
%
&'15&&\mychar{'150}&&\mychar{'151}&&\mychar{'152}&&\mychar{'153}&&\mychar
{'154}&&\mychar{'155}&&\mychar{'156}&&\mychar{'157}&\cr
\noalign{\hrule}
%
&'16&&\mychar{'160}&&\mychar{'161}&&\mychar{'162}&&\mychar{'163}&&\mychar
{'164}&&\mychar{'165}&&\mychar{'166}&&\mychar{'167}&\cr
\noalign{\hrule}
%
&'17&&\mychar{'170}&&\mychar{'171}&&\mychar{'172}&&\mychar{'173}&&\mychar
{'174}&&\mychar{'175}&&\mychar{'176}&&\mychar{'177}&\cr
\noalign{\hrule}
}}
$$
}
\font\trialfont=artmr10 scaled 1600
\aghyusak


\newpage
\subsection{{\tt\bfseries arssr10} tar'atesaku'}

\font\trialfont=arssr10 scaled 1600
\aghyusak


\newpage
\section{\ArmTeX-i phophoxman patmuthyunu'}

\myindent \armbf{\ArmTeX\ 1.0} (25-u' hunisi 1997 th.): Sa \ArmTeX-i ar'ajin
tarberakn e': Ayn himnakanum baghkacac' e' haykakan tar'atesakneric, ev
nu'\armuh ranq {\rm plain \TeX}-um o'gtagorc'elu hamar tramadrowm e' minimal
qanakov parz hramanner:

\bigskip\bigskip

\armbf{\ArmTeX\ 2.0} (1-u' hunisi 1999 th.): Naxord tarberaki nkatmamb,
\ArmTeX-i ays tarberaku' krel e' hetevyal kapital phophoxuthyunneru'`
\begin{itemize}
%
\item[--]Katarelagorc'vel e' tar'adardzuthyan hamakargu' ({\aritshape te|s}
  Bag'in~\ref{s2}):
%
\item[--]U'ndlaynvel e' \ArmTeX-i hramanneri canku' ev naxatesvel e'
  ha\-ma\-kar\-gi o'gtagorc'umu' {\rm \LaTeXe}-um:
%
\item[--]Owghghvel en orosh annshan sxalner ({\rm bug}-er):
%
\item[--]Grvel en hayeren lezvov dzer'narku' ev mi qani o'rinak fayler`
  inchpes {\rm \LaTeX}-i, aynpes e'l {\rm plain \TeX}-i hamar:
%
\item[--]Her'acvel en orosh och anhrag'esht fayler:
%
\end{itemize}



\end{document}
