%%%%%%%%%%%%%%%%%%%%%%%%%%%%%%%%%%%%%%%%%%%%%%%%%%%%%%%%%%%%%%%%%%%%%%%%%%%%%%
%%
%% This is the `raffi.tex' file (sample LaTeX text in transliteration).
%%
%% This file is a part of the ArmTeX project [2024/01/13 v3.0-beta5]
%%
%% ArmTeX is a system for writing in Armenian with plain TeX and/or LaTeX(2e).
%%
%% Copyright 1997 - 2024:
%%   Serguei Dachian (Serguei.Dachian_AT_univ-lille.fr),
%%   Arnak Dalalyan  (arnak.dalalyan_AT_ensae.fr),
%%   Vardan Akopian  (vakopian_AT_yahoo.com).
%%
%% ArmTeX may be distributed and/or modified under the conditions of the LaTeX
%% Project Public License, either version 1.3 of this license or (at your
%% option) any later version.
%%
%% The latest version of this license is in
%%   http://www.latex-project.org/lppl.txt
%% and version 1.3 or later is part of all distributions of LaTeX version
%% 2005/12/01 or later.
%%
%% ArmTeX has the LPPL maintenance status `author-maintained'.
%%
%% For more details, installation instructions and the complete list of files
%% see the provided `README' file.
%%
%%%%%%%%%%%%%%%%%%%%%%%%%%%%%%%%%%%%%%%%%%%%%%%%%%%%%%%%%%%%%%%%%%%%%%%%%%%%%%


\documentclass[12pt,a4paper]{article}


\usepackage{armtex}


\parskip=5pt

\title{ARARATYAN DASHTI AR'AVOTU'\\
       \large(hatvac' \armbf{Raffu} <<SAMVEL>> vepic)}
\author{}
\date{}



\begin{document}

\maketitle

{\leftskip=150pt\small
%
<<Isk yaysm g'amanaki (Merug'ann Arc'runi ew Vahann Mamikonean) awerecin
zqaghaqsn, ew gerecin zbnakealsn and... ew zayl geruthiwns` gawar'ac gawar'ac,
koghmanc koghmanc, phori pho\-ri, zashxarhi ashxarhi, ac'in g'oghovecin i
qa\-ghaqn Naxj'uan, zi and e'r zo'rag'oghov iwreanc zo'racn>>:
%
\par}

\bigskip

\hbox to \hsize{\hskip 150pt\hfill\small Phawstos:\hfill}

\vskip 40pt

Ar'avot e'r, Araratyan dashti lusapayc'ar' ar'avotneric meku':

Arevi ar'ajin j'ar'agaythneri nerqo` Masisi spitakaphar' gagathu' phaylum e'r
vardaguyn shoghqerov, or achq e'in shlacnum: Aragac'i psakadzev gagathu' che'r
erevum: Na der' patac' e'r dzyuni pes j'ermak mshushov, orpes mi amothxac'
harsik, or sqoghum e' yur demqu' anthaphancik shgharshov: Kanachazard
dashtavayru', coghvac' vaghordyan margaritnerov, var'vum e'r c'iac'ani
amenanurb guynerov: Phchum e'r meghm hoviku', c'aghikneru' g'ptum e'in, dalar
xotabuyseru' c'phum u c'ac'anvum e'in, ev dashti xa\-ghagh tarac'uthyunu'
o'rorvum e'r sqancheli alekoc'uthyamb:

Gegheci!k e'r ayd ar'avotu':

Thr'chunneru' urax-urax j'axrum e'in mi thuphic depi myusu': Guy\armuh
nu'zguyn thither'neru', guynzguyn c'aghikneri nman, canvac' e'in o'di mej:
Spitak aragilu', karmir otneru' horizonakan dirqum ughigh meknac', layn
thevqerov thaphaharum e'r, shtapelov depi Araqsi morutneru': Dzer'\-na\-sun
eghjeruneru', vayreni vithn u ayc'yamu', durs e'in ekel Xosrovi arqayakan
antar'neric, ev azat, hamardzak vazvzum e'in shrjaka mar\-ge\-ri vra:

Che'r erevum miayn mardu':

Amen ar'avot, arc'athya phogheri hnchyunu', orsordakan barakneri mr'nchyunu',
sigapanc' ng'uygneri xrxinju', xr'ovum e'in soramut a\-na\-sun\-ne\-ri
vaghordyan hangistu': Amehi varazu' sarsaphelov netvum e'r mthin shambutneri
mej, isk thavamaz arju' apastani tegh e'r oronum: Isk ays ar'avot chkayin
nranq,~\armemdash~chkayin naxararakan ishxanazn pa\-ta\-ni\-ne\-ru', oronc
orsordakan urax zvarj'uthyunneru' mi ar'andzin ken\-da\-nu\-thyun e'in bashxum
ereashat dashtavayrin:

Amen ar'avot thr'chunu' kardum e'r yur naxarshaluysyan meghedin, ev nra het
lseli e'r linum g'rajan mshaki ergu': Phaylum e'r mangaghu', er'um e'r
gorc'u', ev oskeghen hundzqu'` yur liar'at beghmnavoruthyamb` pargevatrum e'r
vastakabek shinakani ashxatanqu': Isk ays ar'avot chkar hndzvoru', chkar ev
herkavaru': Hasunacac' artu' mnacel e'r kisaqagh, ev an\-vas\-ta\-ke\-li
aroru' angorc' u'nkac' e'r der' chverjacrac' akosneri mot:

Amen ar'avot surb taj'ari kochnaki ar'ajin hnchman het` zarthnum e'r hovivu':
Ochxarneri anush bar'anchu', arjar'neri urax dzaynarkuthyunu' kendanacnum e'in
xotavet hovitneru' xist axorg'eli aghmukov: Isk ays ar'avot che'in erevum och
hovivu' ev och nra hoteru': Cirucan gar'\-nuk\-ne\-ru' thaphar'um e'in sar u
dzor, ev hayrakoruys orbikneri nman, karc'es, oronum e'in hovvin:

Amen ar'avot, erb c'agum e'r tvu'njyan lusatun, nra ar'ajin
j'a\-r'a\-gayth\-ne\-ru' oghjunum e'in shinakan aghjikneri ashxatanqu':
Karmir, deghin, kapuyt hagustnerov, orpes karmir, deghin, kapuyt c'aghikner,
sphr'vac' e'in linum nranq ayginerum, banjaranocnerum ev a\-ga\-rak\-ne\-rum:
Ergum e'in ev gorc'um e'in: Ev nranc uraxuthyanu' dzaynakic e'r linum ergaser
soxaku': Isk ays ar'avot che'in erevum anxonj msha\-ku\-thyun\-ne\-ri ayd
geghecik zarderu': Aygineru' mnacel e'in anxnam, a\-ga\-rak\-ne\-ru' korcrel
e'in iranc sireli banvornerin:

Arevu' bardzracav, ev orqan bardzranum e'r na, aynqan Araratyan u'ndardzak
dashtavayru', orpes mi hskayakan burvar', xnkarkum e'r yur vaghordyan
anushhotuthyunu': Amboghj hovitu' c'xum e'r, golorshianum e'r: Coghazard
busakanuthyunu' et e'r talis erknqin yur u'ndunac' mar\-gar\-tya kathilneru':

\end{document}
