% Test document for Babel-Greek
% © 2013, 2023 Günter Milde

% Configuration constants (see test-greek-8bit-luatex.tex for a use case)

\providecommand{\SETUP}{
  \usepackage[LGR,T1]{fontenc}
  \usepackage{lmodern}
  % \usepackage{kerkis}
  % \usepackage{gfsdidot}
  % \usepackage{dejavu}
  % \usepackage{textalpha} % support Greek literal chars and LICR "everywhere"
  \ifdefined \UnicodeEncodingName
    \usepackage{fontspec}
    % \setmainfont[%
    %   ItalicFont=NewCM10-Italic.otf,%
    %   BoldFont=NewCM10-Bold.otf,%
    %   BoldItalicFont=NewCM10-BoldItalic.otf,%
    %   SmallCapsFeatures={Numbers=OldStyle}]{NewCM10-Regular.otf}
    % \setsansfont[%
    %   ItalicFont=NewCMSans10-Oblique.otf,%
    %   BoldFont=NewCMSans10-Bold.otf,%
    %   BoldItalicFont=NewCMSans10-BoldOblique.otf,%
    %   SmallCapsFeatures={Numbers=OldStyle}]{NewCMSans10-Regular.otf}
    % \setmonofont[%
    %   ItalicFont=NewCMMono10-Italic.otf,%
    %   BoldFont=NewCMMono10-Bold.otf,%
    %   BoldItalicFont=NewCMMono10-BoldOblique.otf,%
    %   SmallCapsFeatures={Numbers=OldStyle}]{NewCMMono10-Regular.otf}
    %
    \setmainfont[Scale=0.92]{DejaVu Serif}
  \fi

  % Load and configure Babel:
  % Support for the pseudo-language "polutonikogreek",
  % used instead of "greek", is kept for backwards compatibility.
  \usepackage[
              greek,% comment out "greek" to test the "basic support"
              % greek.local-LGR-fixes,
              % polutonikogreek,
              english]{babel}
  % Uncomment the desired language variant
  % Default: modern monotonic Greek
  %
  % \languageattribute{greek}{polutoniko}  % "modern" polytonic Greek
  % \languageattribute{greek}{polytonic} % alias for polutoniko
  % \languageattribute{greek}{ancient}  % ancient Greek
  %
  %
  % For backwards compatibility, you can also use
  % \selectlanguage{polutonikogreek} instead of \selectlanguage{greek} etc.
  % if the "polutoniko" Greek language variant is selected.

  % To use modern and ancient Greek in one document, load the second language
  % with \babelprovide.
  % Caveat: With 8-bit Greek, the font-encoding switch to LGR fails
  % in the ToC: section titles with Greek come out wrong!
  \babelprovide[hyphenrules=ancientgreek]{ancientgreek}

  % Make ; active to prevent mapping to a middle dot:
  \languageattribute{greek}{keep-semicolon}
}

\providecommand{\SETUPDOC}{
  Test the support for the Greek language as defined in the file
  \file{greek.ldf} (source \file{greek.dtx}).
}

\documentclass[a4paper]{article}

\SETUP

\usepackage{hyperref}
\hypersetup{unicode, colorlinks=true,linkcolor=blue,urlcolor=blue}
\usepackage{bookmark}

\usepackage{parskip}


% Auxiliary commands:

% Semantic styling:
\newcommand{\file}{\texttt}
\newcommand{\cs}[1]{\texttt{\textbackslash#1}}
\newcommand{\pkgref}[1]{\emph{\href{https://ctan.org/pkg/#1}{#1}}}

% print the selected language variant
\newcommand{\GreekLanguageVariant}{%
  \ifx\captionsgreek\captionspolutonikogreek
    \ifx\captionsgreek\captionsancientgreek
      ancient%
    \else
      polutoniko%
    \fi
  \else
    monotoniko%
  \fi
}

% which engine is used?
\ifdefined \UnicodeEncodingName
  \ifdefined\luatexversion
    \newcommand*{\texengine}{LuaLaTeX}
  \else
    \newcommand*{\texengine}{XeLaTeX}
  \fi
\else
  \newcommand*{\texengine}{pdfLaTeX}
\fi

% The paragraph after \showhyphenpar is printed with all words
% shown as hy-phen-ated.
% from https://tex.stackexchange.com/questions/55282/display-hyphenation-options-for-a-specific-word/55283#55283
\def\showhyphenpar{\bgroup
   \def\par{\setparams\endgraf\composelines}
   \setbox0=\vbox\bgroup \noindent\hskip0pt\relax}
\def\setparams{\rightskip=0pt plus1fil
   \linepenalty=1000 \pretolerance=-1 \hyphenpenalty=-10000}
\def\composelines{\setbox2=\hbox{}%
   \loop \setbox0=\lastbox \unskip \unpenalty
   \ifhbox0 % je už \vbox vyprázdněn?
     \global\setbox2=\hbox{\unhbox0\unskip\unhbox2}
   \repeat
   \egroup % vyprázdněný \vbox ukončím
   \exhyphenpenalty=10000 % \box2 vyšoupnu do vnějšího seznamu
   \emergencystretch=4em \leavevmode\unhbox2 \endgraf\egroup}

\newcommand*{\TestUppercase}[1]{%
   \foreignlanguage{greek}{#1 → \MakeUppercase{#1}}%
}

% --------------------------------------------------------------------------

\begin{document}

\title{Test the Greek support for Babel}
\author{Günter Milde}
\date{\today}
\maketitle

\SETUPDOC % print additional setup information.

\makeatletter
This document is compiled with \texengine,
format version \fmtversion{} patch-level \patch@level{},
and the L3 programming layer from \ExplFileDate{}.
The Greek font encoding is \greekfontencoding{}
and the language variant is \GreekLanguageVariant.
\makeatother


\tableofcontents


\section{Language Switch}

The declaration \verb|\selectlanguage| switches between languages.

\begin{quote}
  \selectlanguage{greek}
  Τί φήις; Ἱδὼν ἐνθέδε παῖδ’ ἐλευθέραν
  τὰς πλησίον Νύμφας στεφανοῦσαν, Σώστρατε,
  ἐρῶν άπῆλθες εὐθύς;
\end{quote}

The command \verb|\foreignlanguage| sets its second argument in the language
specified as first argument. This is intended for short text parts or
single words like \foreignlanguage{greek}{Βιβλιοθήκη}.

Input may use literal Greek characters (\foreignlanguage{greek}{α \ldots Ω}),
LICR macros (\foreignlanguage{greek}{\textalpha{} \ldots{} \textOmega}),
or (if the LGR font encoding is used) the Latin transliteration
(\foreignlanguage{greek}{a \ldots{} W}).

\textbf{Warning: With 8-bit TeX, Latin letters and some symbols in the input
are mapped to Greek equivalents!}.
Without precautions, quotes copied from external sources (like this Wikpedia
entry about the question mark) may come out simply wrong:
\begin{quote}
  \selectlanguage{greek}
  Το \textbf{ερωτηματικό} (ελλ. \textbf{;} , λατ. \textbf{?}) είναι το
  σημείο στίξης το οποίο τοποθετείται στο τέλος κάθε ευθείας ερωτηματικής
  πρότασης σε πολλές γλώσσες.
\end{quote}
See section~\ref{Latin transliteration} for remedies.

There should be no inserted space before or after the language switch (this
may happen if there are unescaped linebreaks in the font or language
definitions):
\begin{quotation}
  Change script with \cs{ensuregreek}: |\ensuregreek{δοῦλος}|.

  Change language with \cs{foreignlanguage}:
                           |\foreignlanguage{greek}{δοῦλος}|.

  Change language with \cs{selectlanguage}:
                       |\selectlanguage{greek}δοῦλος\selectlanguage{english}|.
\end{quotation}


\section{Auto-strings}

\emph{Babel} defines macros for several autogenerated strings so that they
may appear in the choosen language. \emph{Babel-greek} uses LICR%
\footnote{LaTeX internal character representation} macros in
order to let the string macros work independent of the font encoding.

\selectlanguage{greek}
\begin{abstract}
  \ensureascii{Look for the abstract name. Today is} \today.
\end{abstract}
\selectlanguage{english}

Show the auto-strings for language variant ``\GreekLanguageVariant''.

\subsection{Captions}

\newcommand{\AutoStrings}{%
  \abstractname,
  \alsoname,
  \appendixname,
  \bibname,\\
  \ccname,
  \chaptername,
  \contentsname,
  \enclname,\\
  \figurename,
  \glossaryname,
  \headtoname,
  \indexname,\\
  \listfigurename ,
  \listtablename,\\
  \pagename,
  \partname,
  \prefacename,
  \proofname,\\
  \refname,
  \seename,
  \tablename
}
\selectlanguage{greek}
\AutoStrings
\selectlanguage{english}

Test correct upcasing (dropping of accents):

\selectlanguage{greek}
\MakeUppercase{\AutoStrings}
\selectlanguage{english}

\subsection{Months}


\begin{quotation}
  \selectlanguage{greek}
  \newcounter{origmonth}
  \setcounter{origmonth}{\the\month}
  \newcounter{foo}
  \stepcounter{foo} \month=\value{foo} \today \qquad \MakeUppercase{\today}

  \stepcounter{foo} \month=\value{foo} \today \qquad \MakeUppercase{\today}

  \stepcounter{foo} \month=\value{foo} \today \qquad \MakeUppercase{\today}

  \stepcounter{foo} \month=\value{foo} \today \qquad \MakeUppercase{\today}

  \stepcounter{foo} \month=\value{foo} \today \qquad \MakeUppercase{\today}

  \stepcounter{foo} \month=\value{foo} \today \qquad \MakeUppercase{\today}

  \stepcounter{foo} \month=\value{foo} \today \qquad \MakeUppercase{\today}

  \stepcounter{foo} \month=\value{foo} \today \qquad \MakeUppercase{\today}

  \stepcounter{foo} \month=\value{foo} \today \qquad \MakeUppercase{\today}

  \stepcounter{foo} \month=\value{foo} \today \qquad \MakeUppercase{\today}

  \stepcounter{foo} \month=\value{foo} \today \qquad \MakeUppercase{\today}

  \stepcounter{foo} \month=\value{foo} \today \qquad \MakeUppercase{\today}
  \month=\value{origmonth}
\end{quotation}


\section{Hyphenation}

Patterns for the Greek language variants:

\makeatletter
monotonic: \verb|\l@monogreek| = \the\l@monogreek \\
polytonic: \verb|\l@polygreek| = \the\l@polygreek \\
ancient:   \verb|\l@ancientgreek| = \the\l@ancientgreek

current: \verb|\l@greek| = \the\l@greek
\makeatletter

\newcommand*{\sample}{%
  \ensureascii{monotonic:} Ευρετήριο,  επίσης,  Απόδειξη,  Θράκη,  τραγωδία\\
  \ifdefined \UnicodeEncodingName \else
     \ensureascii{'' '' translit:} Euret'hrio, ep'ishs, Ap'odeixh, Jr'akh,
                                   tragwd'ia\\
  \fi
  \ensureascii{polytonic:} Εὑρετήριο,  ἐπίσης,  Ἀπόδειξη, Θρᾴκη, τραγῳδία\\
  \ifdefined \UnicodeEncodingName \else
     \ensureascii{'' translit:} E<uret'hrio, >ep'ishs, >Ap'odeixh, Jr'a|kh,
                                tragw|d'ia\\
  \fi
  \ensureascii{ancient:}   Εὑρετήριον, ὡσαύτως, Ἀπόδειξις, Θρᾴκη, τραγῳδία%
}

Greek paragraph:

{
 \selectlanguage{greek}
 \showhyphenpar
 \sample
 \par % (new paragraph to end \showhyphenpar before leaving group)
}

English paragraph with Greek text (\verb|\foreignlanguage{greek}|):

\showhyphenpar
\foreignlanguage{greek}{\sample}

English paragraph with Greek script (\verb|\ensuregreek|):

\showhyphenpar
\ensuregreek{\sample}

% Ancient Greek paragraph:
%
% {
%  \selectlanguage{ancientgreek}
%  \greekscript
%  \showhyphenpar
%  \sample
% }
%
% !! Warning !!
% After use of \verb|\selectlanguage{ancientgreek}| (set up with
% \cs{babelprovide} in the preamble), Greek text in section headings
% is written with the wrong font encoding in the ToC.


\section{Greek Numerals (\greeknumeral{1} \ldots{} \greeknumeral{999999})%
         \label{sec:greek-numerals}}

\emph{Babel-Greek} provides the macros \verb|\greeknumeral| and
\verb|\Greeknumeral| for the conversion of Arabic numbers from 1 to 999\,999
into their Greek counterparts (\greeknumeral{1}, \greeknumeral{2},
\greeknumeral{3}, \ldots, \greeknumeral{999999}).
See \href{babel-greek-doc.html#greek-numerals}{babel-greek-doc}
for the formation rules and configuration options
and \file{test-greeknum.pdf} for samples.

Examples:

\newcommand*{\shownumeral}[2]{\mbox{#2 = \csname #1\endcsname{#2}}}
\newcommand*{\numeralsample}[1]{%
  \shownumeral{#1}{36}\quad
  \shownumeral{#1}{94}\quad
  \shownumeral{#1}{678}\quad
  \shownumeral{#1}{2002}\quad
  \shownumeral{#1}{923090}}

\numeralsample{greeknumeral}

\numeralsample{Greeknumeral}

Users can redefine the macros \verb`\greeknumeralsix` and
\verb`\greeknumeralSix` as well as \verb`\greeknumeralninety`
\verb`\greeknumeralNinety` to configure the used symbols.

If a font misses glyphs for the Greek numeral signs, subsitute characters
may be defined with the macros
\cs{textdexiakeraia} and \cs{textaristerikeraia}.

\ifdefined \greeknumeralsix
  Example (use ``archaic kappa'', ``varstigma'' with pdftex and substitute
  chars for the numeral signs with Xe/LuaTeX):
  \renewcommand*{\greeknumeralninety}{\textqoppa}
  \renewcommand*{\greeknumeralNinety}{\textQoppa}
  \ifdefined\UnicodeEncodingName
    \renewcommand*{\textdexiakeraia}{\textquoteright}
    \renewcommand*{\textaristerikeraia}{\quotesinglbase}
  \else
    \renewcommand*{\greeknumeralSix}{\textvarstigma}
  \fi

  \numeralsample{greeknumeral}

  \numeralsample{Greeknumeral}

  The macro \verb|\Grtoday| produces the current date with the month
  and the day as greek numerals. Today is \ensuregreek{\Grtoday}.

  \subsection{Alphabetical counters}

  In line with Greek typographical tradition (and to avoid messed up
  alphabetical counters with LGR fonts), \emph{babel-greek} changes
  the internal LaTeX commands \cs{@alph} and \cs{@Alph}
  to use Greek numerals inside Greek text parts
  (see section~\ref{sec:LGR-redefinitions} for an example).

\fi

\section{Font Encoding}

TeX's standard 8-bit text fonts don't provide for Greek characters.
Every language switch to \texttt{greek} calls the \verb|\extrasgreek|
language hook which in turn calls \verb|\greekscript| to ensure a
Greek-supporting font encoding (LGR or TU).
With the current setup, this document uses
\begin{itemize}
  \item \greekfontencoding{} as \verb|\greekfontencoding|,
  \item \latinencoding{} as \verb|\latinencoding|, and
        \makeatletter \ensureascii{\cf@encoding} \makeatother
        inside \verb|\ensureascii|.
\end{itemize}

If \verb|\greekfontencoding| is LGR, \emph{babel-greek} performs additional
setup steps to fix issues with the Latin transliteration (see below).
If it is TU, \emph{babel-greek} loads Greek LICR definitions from the file
\file{tuenc-greek.def}\footnote{%
    Provided by \pkgref{greek-fontenc} since version~0.14 (2020-02-28)}.
\ifdefined\UnicodeEncodingName
  %
  % Attention: \meaning\greekfontencoding\ != \meaning\UnicodeEncodingName
  \expandafter\ifx\greekfontencoding\UnicodeEncodingName
    With this setup, it is also possible to use accent macros instead of
    pre-composed Unicode characters for letters with diacritics,
    ὃρα = \`<ορα = \accdasiavaria{ο}ρα.
  \fi
\fi

\ifdefined\greekscript
  Switching to a font encoding supporting the Greek script is possible without
  switching the Babel language using the declarations \verb|\greekscript| (no
  switch if the current encoding supports the Greek script (e.g. the Unicode
  font encodings TU and PU) or \verb|\greektext| (always switch to LGR) and the
  corresponding functions \verb|\ensuregreek| and \verb|\lgrfont|.%
    \footnote{Hyphenation patterns are not changed,
              check for wrong hyphenations.}
  These commands also work in the middle of a paragraph or word:
  \greekscript Φίλων τοῦ \textlatin{TeX} (ΕΦΤ) --
  \latintext Friends (\ensuregreek{F\'ilwn}) of TeX.
\fi


\subsection{LGR's \emph{Latin transliteration}%
            \label{Latin transliteration}}

LGR has Greek characters in the slots reserved for Latin characters and
other symbols in a TeX \emph{standard text font encoding}.
This allows the use of a \emph{Latin transliteration} for the input of Greek
characters\footnote{see \file{usage.pdf}}, however, characters that should
be printed as Latin characters must be protected from conversion by a font
encoding switch, either selecting a different language or wrapping them with
\verb|\ensureascii| (provided by the Babel core), that sets its argument
using an ASCII-compatible font encoding. The legacy declaration
\verb|\latintext| switches the font encoding to \verb|\latinencoding|.

With the Unicode font encoding TU, Latin characters can be used in Greek
text parts and the Latin transliteration does not work (but see the
last example below).

The following quote (with the Babel language set to Greek) illustrates the
problem:

Literal characters, words in the ``foreign'' script protected:
\begin{quote}
  \selectlanguage{greek}
  Φίλων τοῦ \textlatin{TeX} (ΕΦΤ) --
  \selectlanguage{english}
  Friends (\ensuregreek{Φίλων}) of TeX.
\end{quote}

Unprotected ASCII characters come out as Greek characters with LGR:
\begin{quote}
  \selectlanguage{greek}
  Φίλων τοῦ TeX (ΕΦΤ) -- Friends (Φίλων) of TeX.
\end{quote}

The Latin transliteration works in LGR but not TU:
\begin{quote}
  \selectlanguage{greek}
  F'ilwn to\~u \ensureascii{TeX} (EFT) --
  \selectlanguage{english}
  \latintext Friends (\ensuregreek{F\'ilwn}) of TeX.
\end{quote}

\ifdefined\lgrfont
  The Latin transliteration can also be used with
  Xe/LuaTeX, if the input text is wrapped in \cs{lgrfont}\footnote{
    available, if the LGR encoding is loaded
    with the \pkgref{fontenc} package}
  but may result in non-matching fonts and wrong hyphenation:
  \begin{quote}
    \selectlanguage{greek}
    \lgrfont{F'ilwn to\~u \ensureascii{TeX} (EFT)} --
    \selectlanguage{english}
    \latintext Friends (\lgrfont{F\'ilwn}) of TeX.
  \end{quote}
\fi

\subsubsection{The \texttt{keep-semicolon} attribute}

The LGR font encoding uses the Latin question mark as input for the
\emph{erotimatiko} and maps the semicolon to a middle dot (\emph{ano
teleia}). As a result, Unicode-encoded texts that use the semicolon as
\emph{erotimatiko} end up with an \emph{ano teleia} in its place:

The character 037E GREEK QUESTION MARK works with both, Xe/LuaTeX and 8-bit
TeX. However it is deprecated and Unicode normalizes it to 003B SEMICOLON.
This means that even texts wich use the GREEK QUESTION MARK may end up with
SEMICOLON after drag-and-drop or other processing and with a middle dot in
the final output.

With the \texttt{keep-semicolon} language attribute, 003B SEMICOLON is made
active and inserts an \emph{erotimatiko} also with LGR-encoded fonts, if the
text language is set to Greek
(in this document, the semicolon is
\ifnum\catcode59 = 13
  active).
\else
  not active).
\fi

\begin{tabular}{lccc}
  Input                    & \latinencoding{}
     & \greekfontencoding  & Greek language             \\
  003F QUESTION MARK       & ?
     & \ensuregreek{?}     & \foreignlanguage{greek}{?} \\
                                                        \\
  037E GREEK QUESTION MARK & {\footnotesize n/d}
     & \ensuregreek{;}     & \foreignlanguage{greek}{;} \\
  003B SEMICOLON           & ;
     & \ensuregreek{;}     & \foreignlanguage{greek}{;} \\
  00B7 MIDDLE DOT          & ·
     & \ensuregreek{·}     & \foreignlanguage{greek}{·} \\
\end{tabular}

{\footnotesize n/d}: character not defined in T1 encoding.

This attribute is ignored with Unicode fonts (where the SEMICOLON literal
always prints a semicolon character).

Test in math mode:
\newcommand{\semicolontest}{$a b; a\;b, (\mathrm{a;}\textrm{a;}2)$}
English: \semicolontest, Greek: \foreignlanguage{greek}{\semicolontest}.

\subsubsection{LGR-proofed macros}

\emph{Babel-greek} provides LGR-local variants for some
\emph{TextCommand}s that rely on a standard text encoding.\footnote{%
  These workarounds cannot be done in \file{lgrenc.def} from the
  \pkgref{greek-fontenc} package because they are not allowed in a
  ``font encoding definition file'' [\file{fntguide.pdf}].}
The fallback definitions for some \emph{textcomp} symbols compose the
symbols out of Latin letters.
The fixes must not overwrite the selection of pre-composed symbols
from \pkgref{textcomp} or TU (try copy and paste from the PDF output).

LGR fonts have a middle dot glyph at the place of the ampersand. The new
\emph{TextCommand} \verb|\textampersand| always prints an ampersand.
\makeatletter \newcommand{\AsciiSymbolTest}{(\f@encoding) ©®™
                          A\textampersand W}
\makeatother
\begin{quote}
  English: \AsciiSymbolTest\\
  English: \ensuregreek{\AsciiSymbolTest}\\
  Greek: \foreignlanguage{greek}{\AsciiSymbolTest}
\end{quote}
% English (LGR), input as macro:
% \lgrfont{|\textcopyright|\textregistered|\texttrademark|
%            A\textampersand W.}


\ifdefined\lgrfont% babel-greek and LGR font encoding loaded

  \subsection{LGR re-definitions \label{sec:LGR-redefinitions}}

  The generic macro \verb|\&| is re-defined inside Greek text parts to use
  the original definition in math mode and \cs{textampersand} in text
  mode.

  \subsubsection{Roman numerals \label{sec:roman-numerals}}

  Without fixes, Roman numerals are printed according to the Latin
  transliteration (including the conversion of ``v'' to a
  ZERO WIDTH NON-JOINER) if the font encoding is LGR:

  \makeatletter
  \newcommand{\RomanNumeralTest}{\f@encoding:
              \romannumeral 1, \romannumeral 2, \romannumeral 3,
              \romannumeral 4, \romannumeral 5, \romannumeral 6}
  \makeatother
  \begin{quote}
    \RomanNumeralTest\\
    \lgrfont{\RomanNumeralTest}
  \end{quote}

  Roman numerals are used by the default document classes, e.g., in the third
  level of enumerations or as page numbers in the frontmatter of a book.
  They may move to auto-generated document parts like the ToC,
  (hyper)references, or an index.

  As document authors cannot wrap page numbers in a ToC in \verb|\ensureascii|,
  \emph{Babel-greek} redefines the internal LaTeX commands \verb|\@roman| and
  \verb|\@Roman| to make Roman numberals LGR-proof.
  Unfortunately, this breaks Makeindex (cf. \file{test-lgr-fixes.tex}).

\fi


\subsubsection{Example}

In Greek text parts, enumerated lists use Greek numerals in the second and
fourth level and ASCII-proofed Roman numerals in the third level.

\selectlanguage{greek}
\begin{enumerate}
  \item item 1
  \begin{enumerate}
    \item item 1.1
    \begin{enumerate}
      \item item 1.1.1
       \begin{enumerate}
         \item item 1.1.1.1
         \item item 1.1.1.2
         \item item 1.1.1.3 \label{item 1.1.1.3}
       \end{enumerate}
      \item item 1.1.2
    \end{enumerate}
  \end{enumerate}
\end{enumerate}
\selectlanguage{english}

\selectlanguage{english}
Setting the language back to English should restore the alphabetic numbering:
\begin{enumerate}
  \item item 1
  \begin{enumerate}
    \item item 1.1
    \begin{enumerate}
      \item item 1.1.1
       \begin{enumerate}
         \item item 1.1.1.1
         \item item 1.1.1.2
         \item item 1.1.1.3
       \end{enumerate}
      \item item 1.1.2
    \end{enumerate}
  \end{enumerate}
\end{enumerate}

More tests of the LGR-redefinitions are in \file{test-lgr-fixes.tex}.

\ifdefined\lgrfont
\else
  \end{document}
\fi


\section{Up- and downcasing in Greek}

Capital Greek letters have diacritics (except the dialytika and sub-iota) to
the left (instead of above) and drop them in uppercase (except the
dialytika), e.g., \TestUppercase{μαΐστρος}.

Tonos and psili mark a \emph{hiatus} (break-up of a diphthong) if
placed on the first vowel of a diphthong.
A dialytika must be placed on the second vowel if they are dropped, e.g.
%
\TestUppercase{\acctonos\textalpha\textiota,
               \acctonos\textalpha\textupsilon,
               \acctonos\textepsilon\textiota,
               \'>\textalpha\textiota,
               \'>\textalpha\textupsilon,
               \accpsili\textalpha\textupsilon,
               \accpsilioxia\textepsilon\textiota}%

Some affected words:
\begin{quotation}
  % from teubner.sty:
  \TestUppercase{% άυλος → ΑΫΛΟΣ
    \acctonos\textalpha\textupsilon\textlambda\textomicron\textfinalsigma}
  \TestUppercase{% polytonic: ἄυλος → ΑΫΛΟΣ
    \'>\textalpha\textupsilon\textlambda\textomicron\textfinalsigma}

  % from http://diacritics.typo.cz/index.php?id=69
  \TestUppercase{%μάινη → ΜΑΪΝΗ
    \textmu\acctonos\textalpha\textiota\textnu\texteta}

  % from  http://de.wikipedia.org/wiki/Neugriechische_Orthographie#Das_Trema
  \TestUppercase{% κέικ → ΚΕΪΚ
    \textkappa\acctonos\textepsilon\textiota\textkappa}

  \TestUppercase{% ἀυπνία → ΑΫΠΝΙΑ
    \accpsili\textalpha\textupsilon\textpi\textnu\acctonos\textiota\textalpha}

  % from http://multilingualtypesetting.co.uk/blog/greek-typesetting-tips/
  \TestUppercase{% ρωμέικα → ΡΩΜΕΪΚΑ
    \textrho\textomega\textmu
    \acctonos\textepsilon\textiota\textkappa\textalpha}

  % from Yannis Haralambous https://hal.science/hal-02101618
  \TestUppercase{% ἀυτή → ΑΫΤΗ a cry, shout, esp. battle-shout
    \accpsili\textalpha\textupsilon\texttau\acctonos\texteta}

  % from https://r12a.github.io/scripts/grek/el.html#transforms
  % νεράιδα -> ΝΕΡΑΪΔΑ
  \TestUppercase{νεράιδα}\footnote{
    This example uses literal input.
    It fails with pre-2022 \cs{MakeUppercase}.}
\end{quotation}

The file \file{char-list.tex} in the \pkgref{greek-fontenc} package includes
a comprehensive test of case changing for all supported Greek characters and
their various input methods.


\subsection{Problems and fixes}

\subsubsection{Input variants}

Depending on the LaTeX version and input variant, there are several
limitations and problems.

With \cs{greekfontencoding} \greekfontencoding, LaTeX version \fmtversion,
and language variant ``\GreekLanguageVariant'', we get

pre-composed:
\TestUppercase{ῶ, ή, ΐ, ᾂ, άι}

transliteration:
\TestUppercase{{~w, 'h, '"i, >`a|, 'ai}}

accent macro + LICR:
\TestUppercase{\~\textomega, \'\texteta, \'"\textiota,
                        \`>\textalpha\ypogegrammeni, \'\textalpha\textiota}

accent macro + transliteration:
\TestUppercase{{\~w, \'h, \'"i, \`>a|, \'ai}}

accent macro + literal:
\ifdefined \UnicodeEncodingName
  \TestUppercase{\~ω, \'η, \'"ι, \`>ᾳ, \'αι}
\else
  → inputenc Error: Invalid UTF-8 byte sequence
\fi

\begin{itemize}
\item The implementation of \cs{MakeUppercase}
      introduced in the 2022/06 LaTeX release\footnote{cf.\
        \href{https://www.latex-project.org/news/latex2e-news/ltnews35.pdf}%
        {LaTeX News 35}}
      works (almost) fine with pre-composed literal characters but there are
      Unicode errors (unknown characters) under 8-bit TeX.
      Fixed with LaTeX 2023, babel-greek 1.13 and greek-fontenc 2.3.

\item The new \cs{MakeUppercase} did not drop
      accents input as short accent macros or with the Latin transliteration.
      Fixed with the 2023 LaTeX release, babel-greek~1.14, and
      greek-fontenc~2.4.

\item The \emph{hiatus} feature fails with the Latin transliteration.
      Use accent macros, e.g., replace \verb|k'eik| with \verb|k\'eik|.

      It also failed with pre-composed characters and LaTeX
      versions older than 2022/06.%
      \footnote{Some ``pro'' Unicode fonts provide this feature on their own,
                cf. \href{http://multilingualtypesetting.co.uk/blog/greek-typesetting-tips/}%
                {Greek typesetting without the tears}}

% \item With \emph{babel-greek} versions up to 1.11, using the tilde for the
%       \emph{perispomeni} accent inside \verb|\MakeUppercase| led to
%       \texttt{inputenc Error: Invalid UTF-8 byte "9F}.%
%       \footnote{even without loading \emph{inputenc} and with
%         \cs{UseRawInputEncoding}, cf.
%         \href{https://www.latex-project.org/news/latex2e-news/ltnews28.pdf}%
%       {LaTeX News 28}}

\end{itemize}

\subsubsection{Particuliarities of the Latin Transliteration}

To enable correct upcasing of the ``Latin transliteration'',
\emph{babel-greek} changes the uppercase
equivalent of some characters with special meaning in LGR.
To minimise side-effects, this is done:
\begin{itemize}
\item only if \cs{greekfontencoding} is a ``short macro'' expanding to LGR,
      i.e.\ not in documents using Unicode fonts
      (unless \cs{greekfontencoding} is explicitly set to \texttt{LGR}
      before loading \emph{babel-greek}),
\item only for diacritics that are actually required in the selected language
      variant (i.e.\ only for the \emph{tonos} \texttt{'},
      if the language variant is the default \texttt{monotoniko}),
\item not for the characters ``v'' (zero-width space) and ``c'' (final sigma).
      Use \verb|\textcompwordmark| instead of \texttt{v} and autosigma
      (\texttt{s}) instead of \texttt{c} in text parts that could/should
      become upcased, e.g., \lgrfont{a\textcompwordmark us $\mapsto$
      \MakeUppercase{a\textcompwordmark us}}
      not \lgrfont{avuc $\mapsto$ \MakeUppercase{avuc}}.

\end{itemize}

Since version 1.13.2, \emph{babel-greek} utilises \cs{DeclareUppercaseMapping}
(a LaTeX kernel command, new in 2023) for the required change.
For LaTeX version older than 2022/06, the \cs{uccode} is set to
the ``empty'' character 0x9f = 159.
\emph{Composite command} definitions ensure that combined accents
also work for accent characters ``upcased'' to the charcter
No 159 = 0x9f:
\TestUppercase{%
  \ifdefined \UnicodeEncodingName
    \"'υ{} \"`υ{} \`"υ{} \'"υ{} \`<α{} \'<α{}
  \else
    \"'u   \"`u   \`"u   \'"u   \`<a   \'<a   \'<ai
  \fi
  \'<\textalpha\textiota
}%
.


\newcommand{\AccentCharacters}{" | ' ` > <}
With font encoding \greekfontencoding, LaTeX version \fmtversion,
and language variant ``\GreekLanguageVariant'', we get for
\texttt{\AccentCharacters} (dialytika\footnote{%
  Unless followed by a to-be accented vowel, the quotation mark \texttt{"}
  is converted to an upper right apostrophe by LGR.},
sub-iota, tonos/oxia, varia, psili, and dasia):
\begin{quote}
  \TestUppercase{\AccentCharacters}
\end{quote}

The changed uc/lccodes have strange effects on Latin text parts
in Greek paragraphs if only the encoding is switched:
\begin{quote}
  \selectlanguage{greek}
  \foreignlanguage{english}{English:
    Let's see: \AccentCharacters{} →
    \MakeUppercase{Let's see: \AccentCharacters}}\\
  \ensureascii{\cs{ensureascii}:
    Let's see: \AccentCharacters{} →
    \MakeUppercase{Let's see: \AccentCharacters}}
\end{quote}

\subsubsection{Iota subscript vs. iota adscript}

Unicode decomposes letters with \emph{mute iota}
(GREEK CAPITAL LETTER ... WITH [... AND] PROSGEGRAMMENI)
to the base letter and a COMBINING GREEK YPOGEGRAMMENI (U+0345).%
\footnote{They are named ... WITH PROSGEGRAMMENI
  for ``historic reasons''
  (cf. \href{https://opoudjis.net/unicode/unicode_adscript.html}
  {Nick Nicholas \emph{Titlecase and Adscripts}}).}
Accordingly, the ``canonical'' LICR for all pre-composed letters with
mute iota is the base character LICR followed by \verb|\ypogegrammeni|.

The appearance in the output and upcasing results depend on the chosen font
and LaTeX version.
Compare letters followed by \verb|\ypogegrammeni| with pre-composed
characters
\begin{quotation}
  \selectlanguage{greek}
  \newcommand{\alphaypo}{\textalpha\ypogegrammeni α\ypogegrammeni{} ᾳ /
                         \textAlpha\ypogegrammeni Α\ypogegrammeni{} ᾼ /
                \accpsili\textAlpha\ypogegrammeni Ἀ\ypogegrammeni{} ᾈ}

  \phantom{MakeUppercase}     \alphaypo

  \ensureascii{MakeUppercase} \MakeUppercase{\alphaypo}

  \ensureascii{MakeLowercase} \MakeLowercase{\alphaypo}
\end{quotation}

and letters followed by \cs{prosgegrammeni}
with literal character + GREEK YPOGEGRAMMENI:
\begin{quotation}
  \selectlanguage{greek}
  \newcommand{\alphapros}{\textalpha\prosgegrammeni α\prosgegrammeni{} αι  /
                          \textAlpha\prosgegrammeni Α\prosgegrammeni{} Αι /
                 \accpsili\textAlpha\prosgegrammeni Ἀ\prosgegrammeni{} Ἀι}

  \phantom{MakeUppercase}     \alphapros

  \ensureascii{MakeUppercase} \MakeUppercase{\alphapros}

  \ensureascii{MakeLowercase} \MakeLowercase{\alphapros}
\end{quotation}

See also the \href{https://www.unicode.org/faq/greek.html#6}{Unicode FAQ}.

\end{document}
