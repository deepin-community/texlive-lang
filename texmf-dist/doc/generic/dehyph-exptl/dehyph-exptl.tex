%%% Artikelklasse mit:
%%% * Grundschriftgröße 11 Punkt,
%%% * klassischem Satzspiegel,
%%% * flachem Inhaltsverzeichnis,
%%% * Tabellenüberschriften.
\documentclass[11pt,
               DIV=9,
               toc=flat,
               captions=tableheading,
               abstract=on]{scrartcl}

\emergencystretch 1em

%%% Eingabekodierung ist UTF-8.
\usepackage[utf8]{inputenc}

%%% Schrifteinstellung:
%%% * Grundschrift Palatino,
%%% * Akzidenzschrift Bera Sans,
%%% * Schreibmaschinenschrift Latin Modern Typewriter.
\usepackage[T1]{fontenc}
\usepackage[osf]{mathpazo}
\usepackage[scaled=0.85]{berasans}
\renewcommand*{\ttdefault}{lmtt}
\usepackage{textcomp}

\linespread{1.1}
\usepackage[expansion=true,
            letterspace=80]{microtype}

%%% Lade einige Pakete.
\usepackage{ifthen}
\usepackage{calc}
\usepackage{multicol}
\usepackage{paralist}
\usepackage{fncylab}
\usepackage{tabularx}
\usepackage{booktabs}
\usepackage{ragged2e}
\usepackage{hologo}
\usepackage[defaultlines=4,
            all]{nowidow}

\newcolumntype{L}{>{\raggedright\arraybackslash}X}

\usepackage{listings}
\lstloadlanguages{[LaTeX]TeX, sh}
\lstset{
  basicstyle=\ttfamily,
  keywordstyle={},
  commentstyle={},
  columns=flexible,
  showspaces=false,
  showstringspaces=false,
  % frame=tb,
  % framesep=8pt,
  % framerule=2pt,
  xleftmargin=6pt,
  xrightmargin=6pt,
  % framexleftmargin=6pt,
  % framexrightmargin=6pt,
  inputencoding=utf8,
  extendedchars=true,
  literate={ä}{{\"a}}1 {ö}{{\"o}}1 {ü}{{\"u}}1,
}
\lstdefinestyle{LaTeX}{
  language=[LaTeX]TeX,
  basicstyle=\ttfamily,
  keywordstyle={},
  commentstyle={\itshape}
}
\lstdefinestyle{shell}{
  language=sh,
  basicstyle=\ttfamily,
  keywordstyle={},
  commentstyle={\itshape},
  upquote=true % Gravis korrekt anzeigen
}
\lstdefinestyle{Text}{
  language=,
  basicstyle=\ttfamily,
  keywordstyle={},
  commentstyle={}
}

\usepackage{needspace}

%%% Literaturverweise in runden Klammern mit Semikolon als Trenner.
% \usepackage[round,semicolon]{natbib}
% \renewcommand*{\bibnumfmt}[1]{(#1)}

%%% Literaturverzeichnis mit Sprachunterstützung.
\usepackage[fixlanguage]{babelbib}
\bibliographystyle{babalpha}
%%% Babelbib fordert trotz fixlanguage zuviele Sprachen an.
\usepackage[english,
            german,
            ngerman]{babel}
\usepackage[ngerman=ngerman-x-latest]{hyphsubst}

%%% Einstellungen für interaktive PDF-Dokumente.
\usepackage[rgb,x11names]{xcolor}
\usepackage[hyperref]{zref}
\usepackage{hyperref}
\hypersetup{
  pdftitle={dehyph-exptl},
  pdfauthor={Die deutschsprachige Trennmustermannschaft},
  pdfkeywords={TeX,
               deutsche Rechtschreibung,
               Trennmuster,
               computergestützte Worttrennung}
}
\hypersetup{
  ngerman,% Für \autoref.
  pdfstartview={XYZ null null null},% Viewer bestimmt Zoomfaktor.
  colorlinks,
  linkcolor=RoyalBlue3,
  urlcolor=Chocolate4,
  citecolor=DeepPink2
}
\newcommand*{\regelref}[1]{%
  \begingroup
  \renewcommand*{\Itemautorefname}{Regel}%
  \autoref{#1}%
  \endgroup%
}

%%% Schriftfestlegungen.
\setkomafont{title}{\normalcolor\normalfont}
\setkomafont{sectioning}{\normalcolor\normalfont}
\setkomafont{section}{\Large}
\setkomafont{subsection}{\Large\itshape}
\setkomafont{descriptionlabel}{\normalfont\itshape}

%%% Einige Makros für logische Auszeichnungen definieren.
\newcommand*{\Abk}[1]{\mbox{\textsc{\lsstyle#1}}}
\newcommand*{\Paket}[1]{\textsf{#1}}
\newcommand*{\Programm}[1]{\lstinline[style=shell]{#1}}
\newcommand*{\Datei}[1]{\texttt{#1}}

\colorlet{falschcol}{red!80!black}
\colorlet{tradcol}{green!50!black}
\colorlet{reformcol}{green!75!black}
\colorlet{unerwcol}{red!60!black}

\newcolumntype{T}{>{\color{tradcol}}l}
\newcolumntype{R}{>{\color{reformcol}}l}
\newcolumntype{U}{>{\color{unerwcol}}l}

\newcommand*{\trennung}[2]{%
  \makebox[0pt][l]{%
    \color{#1}%
    \smash{\rule[-3.5pt]{\widthof{#2}}{.7pt}}% Schriftabhängig.
  }%
  #2%
}
\newcommand*{\ftr}[1]{\trennung{falschcol}{#1}}% Falsche Trennung.

%%% Satzspiegel erneut berechnen.
\typearea{last}


%%%%%%%%%%%%%%%%%%%%%%%%%%%%%%%%%%%%%%%%%%%%%%%%%%%%%%%%%%%%%%%%%%%%%%

\begin{document}

%%% Trennausnahmen definieren.
\hyphenation{
  Back-end
  hyph-subst
  Ent-wick-ler-re-po-si-to-ri-um
  Wort-her-kunft
  Not-tren-nung
  Trenn-al-go-rith-mus
  um-bruch-in-kom-pa-tib-le
}

%%% Protokollierung der Trennungen für findhyph.
%\tracingparagraphs=1

%%% Dokumenttitel.
\author{Die deutschsprachige Trennmustermannschaft}
\title{\Paket{dehyph-exptl}\thanks{This document describes the
    \Paket{dehyph-exptl} package v0.8.}}
\subtitle{Experimentelle Trennmuster für die deutsche Sprache}
\maketitle

%%% Zweisprachige Zusammenfassung.
\selectlanguage{english}
\begin{abstract}
  This document describes the experimental hyphenation patterns for
  the German language, covering traditional and reformed orthography
  for several varieties of Standard German.  When using
  \hologo{XeTeX}/\hologo{LuaTeX}, these patterns are used
  automatically.  In other cases, they have to be loaded explicitly.
  More information can be found in the Trennmuster-Wiki%
  \footnote{\url{http://projekte.dante.de/Trennmuster}} (in German).
\end{abstract}

\selectlanguage{ngerman}
\begin{abstract}
  Dieses Dokument beschreibt die experimentellen Trennmuster für die
  deutsche Sprache, die das in Deutschland, Österreich und der Schweiz
  gebräuchliche Standarddeutsch in der traditionellen und reformierten
  Rechtschreibung abdecken.  Bei Nutzung von
  \hologo{XeTeX}/\hologo{LuaTeX} werden diese Muster automatisch
  verwendet; in anderen Fällen müssen sie i.\,d.\,R. ausdrücklich
  geladen werden.
\end{abstract}

\vfill
\begingroup
\em\large

\begin{center}
  Warnung!
\end{center}

Diese Trennmuster befinden sich im experimentellen Status.  Sie können
auf \Abk{ctan} und in den \hologo{TeX}"=Verteilungen jederzeit
durch umbruchinkompatible aktualisierte Versionen ersetzt werden.  Für
Anwendungen, die einen dauerhaft stabilen Umbruch erfordern, sind sie
nur geeignet, falls manuell eine feste Version installiert wird.
\endgroup \vfill

\clearpage
%%% Zweispaltiges Inhaltsverzeichnis.
\begin{multicols}{2}
  \RaggedRight
  \small
  \selectlanguage{ngerman}
  \tableofcontents
\end{multicols}


\section{Einleitung}
\label{sec:einleitung}

Der in \hologo{TeX} implementierte Trennalgorithmus arbeitet
musterbasiert~\cite{liang:1983}.  Prinzipiell können mit einem solchen
Algorithmus nicht alle möglichen Wörter korrekt getrennt werden.  Die
Qualität der Worttrennung einer Sprache wird jedoch maßgeblich von der
Qualität der Wortliste beeinflusst, aus der die verwendeten
Trennmuster berechnet wurden.

Obwohl die herkömmlichen Trennmuster für die deutsche Sprache bei der
Worttrennung in gewöhnlichen Texten eine akzeptable Fehlerrate
erreichen, enthalten sie doch eine Reihe von Schwächen:%
\footnote{Diese Liste bezieht sich auf die Trennmusterdateien
  \Datei{dehypht.tex}, Version~3.2a vom 3.\,3.\,1999, und
  \Datei{dehyphn.tex}, Version~31 vom 7.\,5.\,2001.}

\bigskip\smallskip
\needspace{4\baselineskip}
\noindent\textit{traditionelle und reformierte Rechtschreibung}

\begin{itemize}
\item In zusammengesetzten Wörtern treten häufig Trennfehler an
  Wortfugen auf.

\item Fremdwörter mit akzentuierten Buchstaben werden mangelhaft
  getrennt: »C\ftr{af}é«, »Ci-tr\ftr{oë}n«, »F\ftr{aç}on«,
  »vo\ftr{il}à«.

\item Die Trennmusterdateien enthalten eine Mischung aus \Abk{t1}-
  sowie unvollständigen \Abk{ot1}-kodierten Mustern.  Mit Erscheinen
  von 16-Bit"=fähigen \hologo{TeX}"=Varianten werden sauber
  \Abk{utf-8}"=kodierte Trennmuster nötig~\cite{miklavec:2008}.
\end{itemize}

\bigskip
\needspace{4\baselineskip}
\noindent\textit{traditionelle Rechtschreibung}

\begin{itemize}
\item Die herkömmlichen Trennmuster für die traditionelle deutsche
  Rechtschreibung können mit \Programm{patgen} nicht reproduziert
  werden, da die zugrundeliegende Wortliste verschollen ist.  Die
  Pflege der Trennmuster ist daher schwierig bis unmöglich.  Für freie
  Software ist dies kein zufriedenstellender Zustand.

\item Umfang und Qualität der ursprünglichen Wortliste lassen sich
  nicht mehr einschätzen.  Für die Trennmuster in traditioneller
  Rechtschreibung existiert jedoch inzwischen eine Ausnahmeliste mit
  über 3500 korrigierten Trennungen einfacher Wörter
  \cite{lemberg:2003, lemberg:2005}.%
  \footnote{\url{http://mirrors.ctan.org/language/hyphenation/dehyph/dehyphtex.tex}}

\item Wird in der traditionellen Rechtschreibung \emph{ß} durch
  \emph{ss/SS} oder \emph{sz/SZ} ersetzt, so bleibt die Trennung davon
  unberührt.  Die herkömmlichen Trennmuster berücksichtigen diese
  Regel nicht und trennen häufig den Ersatz:
  \textls{»GR\ftr{ÖS"~S}E«}, \textls{»GR\ftr{ÜS"~S}E«},
  \textls{»M\ftr{AS"~S}ES«}.%
  \footnote{Die Trennung der herkömmlichen Muster entspricht den
    Regeln der deutschen Standardsprache in der Schweiz, obwohl diese
    Sprachvarietät vom Paket \Paket{Babel} bis 2013 nicht offiziell
    unterstützt wurde.  \Paket{Babel} versucht den Mangel mit Hilfe
    des Kürzels \lstinline[style=LaTeX]+\"S+ zu kompensieren.}

\item Abweichende Schreibweisen, die in der traditionellen
  Rechtschreibung in Österreich und der Schweiz verwendet werden,
  werden mangelhaft getrennt: »Gro\ftr{s"~so}n-kel«,
  »Ku\ftr{s"~sh}and«, »Ma\ftr{ssn}ah-me«, »mi\ftr{s"~sa}ch-ten«
  (nur Schweiz) und »Ex-pre\ftr{ssz}ug«, »Fit-ne\ftr{s"~sc}en-ter«,
  »Fit-ne\ftr{sst}rai-ner«.
\end{itemize}

\bigskip
\needspace{4\baselineskip}
\noindent\textit{reformierte Rechtschreibung}

\begin{itemize}
\item Die Trennmuster für die reformierte deutsche Rechtschreibung
  wurden nicht mit \Programm{patgen} aus einer Wortliste erstellt.
  Stattdessen wurden die Trennmuster für die traditionelle
  Rechtschreibung von Hand an die reformierten Regeln
  angepasst~\cite{schmidt:1998}.  Aus diesem Grund ist die
  Worttrennung mit den Trennmustern für die reformierte
  Rechtschreibung etwas schlechter als mit den Trennmustern für die
  traditionelle Rechtschreibung.
\end{itemize}

Das Projekt \emph{Freie Wortlisten und Trennmuster für die deutsche
  Sprache} hat sich das Ziel gesetzt, neue Trennmuster hoher Qualität
für die deutsche Sprache zu erstellen, die die genannten Probleme
ausräumen.

Den experimentellen Trennmustern dieses Pakets liegt eine Wortliste
mit den etwa fünfhunderttausend häufigsten Wörtern der deutschen
Sprache zugrunde.  Vermutlich ist diese Liste erheblich umfangreicher
als die ursprüngliche Wortliste, in der Worthäufigkeiten
wahrscheinlich überhaupt nicht berücksichtigt wurden.  Die verwendete
Wortliste deckt das in Deutschland, Österreich und der Schweiz
gebräuchliche Standarddeutsch ab.

Mit den vorliegenden Trennmustern sollte für nicht"=fachsprachliche
Wörter eine sehr gute Trennqualität erreicht werden.  Insbesondere
sollte sich die Trennung häufig auftretender zusammengesetzter Wörter
verbessern.


\section{Verwenden der Trennmuster}
\label{sec:verwenden}

Die experimentellen Trennmuster werden durch zwei verschiedene
\hologo{TeX}"=Pakete zur Verfügung gestellt, die beide in einer
vollständigen Installation einer aktuellen \hologo{TeX}"=Distribution
enthalten sein sollten.

Das Paket \Paket{hyph-utf8} enthält Trennmuster für alle von
\hologo{TeX} unterstützten Sprachen.  Für die deutsche Sprache sind
die experimentellen Muster enthalten.  Diese werden standardmäßig nur
von \hologo{XeTeX}, \hologo{LuaTeX} und p\TeX\footnote{Eine in Japan
  populäre \TeX-Variante.} sowie von allen \hologo{TeX}"=Varianten für
die Schweizer Varietät der deutschen Sprache in traditioneller
Rechtschreibung\footnote{Diese Varietät wird erst seit dem Jahr 2013
  unterstützt, sodass keine Abwärtskompatibilität erforderlich ist.}
benutzt.  In den übrigen Fällen wird dagegen auf die herkömmlichen
Trennmuster zurückgegriffen.

Das Paket \Paket{dehyph-exptl} enthält die experimentellen
Trennmuster für die deutsche Sprache.  Es hat den Zweck, diese den
Nutzern der älteren \hologo{TeX}"=Varianten wie \hologo{pdfTeX}
zugänglich zu machen.  Die hierfür nötigen Vorkehrungen werden unten
beschrieben.

Siehe \autoref{sec:fragen} für Hinweise, wie die Version der in beiden
Paketen enthaltenen Muster ermittelt werden kann.

Zur manuellen Installation einer bestimmten Version der
experimentellen Trennmuster siehe \autoref{sec:installation}.

Beachte, in den folgenden Abschnitten ist \verb+<datum>+ durch das bei
der Installation angegebene Datum in \Abk{iso}"=Notation
(\verb+JJJJ-MM-TT+) oder die Zeichenkette \verb+latest+ zu ersetzen.


\subsection{Sprachvarietät und Rechtschreibung}
\label{sec:varietaeten}

Experimentelle Trennmuster stehen für die Worttrennung der deutschen
Sprache in der traditionellen und der reformierten Rechtschreibung zur
Verfügung.  Die Trennmuster unterstützen die drei in Deutschland,
Österreich und der Schweiz gebräuchlichen Hauptvarietäten der
deutschen Standardsprache.  \autoref{tab:varietaeten} zeigt die
Trennmuster, die abhängig von gewünschter Varietät und Rechtschreibung
in einem Dokument zu aktivieren sind.

\begin{table}
  \centering
  \caption{Die unterstützten Varietäten und Rechtschreibungen}
  \label{tab:varietaeten}
  \begin{tabular}{l>{\ttfamily}l}
    \normalfont Sprachvarietät
    & Trennmusterbezeichner\\
    \addlinespace\toprule\addlinespace
    \hspace*{-\tabcolsep}\normalfont\emph{traditionelle Rechtschreibung}\\
    Deutschland, Österreich
    & german-x-<datum>\\
    Schweiz
    & gswiss-x-<datum>\\\addlinespace
    \hspace*{-\tabcolsep}\normalfont\emph{reformierte Rechtschreibung}\\
    Deutschland, Österreich, Schweiz
    & ngerman-x-<datum>\\
  \end{tabular}
\end{table}

Varietäten, die sich nur in der Verwendung und Schreibung einzelner
Wörter voneinander unterscheiden, können durch gemeinsame Trennmuster
unterstützt werden.
% Die Eingabewortliste für \Programm{patgen} ist dann eine Vereinigung
% der den Varietäten entsprechenden Wortlisten.
Zum Beispiel werden Besonderheiten der österreichischen
Standardsprache in den Trennmustern für die Standardsprache
Deutschlands berücksichtigt.  Österreichische und deutsche Anwender
können daher dieselben Trennmuster verwenden (wie das auch schon bei
den herkömmlichen Trennmustern der Fall war).

Aufgrund unvereinbarer Trennregeln in der traditionellen
Rechtschreibung der Standardsprachen Deutschlands/""Österreichs und
der Schweiz werden für die letztere Varietät eigene Trennmuster
bereitgestellt.

Mit der Rechtschreibreform 1996 wurden die Trennregeln aller drei
Standardsprachen so weit angeglichen, dass für die reformierte
Rechtschreibung einheitliche Trennmuster für alle drei unterstützten
Sprachvarietäten bereitgestellt werden können.

\subsection{Aktivieren der Trennmuster}
\label{sec:aktivieren}

Im Folgenden wird dargestellt, wie die experimentellen Trennmuster bei
Verwendung unterschiedlicher \hologo{TeX}"=Varianten geladen werden
können.

Das Ergebnis für die traditionelle und reformierte Rechtschreibung mit
herkömmlichen und experimentellen Trennmustern ist in
\autoref{tab:trennvarianten} zusammengefasst.  Ob die experimentellen
Trennmuster korrekt aktiviert werden, kann mit dem folgenden kleinen
Dokument getestet werden, dessen Ausgabe in der \texttt{log}"=Datei
erscheint.

\begin{lstlisting}[style=LaTeX]
  \begin{document}
  \showhyphens{löste Fassade modernste Abendstern Mordopfer}
  \end{document}
\end{lstlisting}

%\suppressfloats[t]
\begin{table*}
  \centering
  \caption{Trennvarianten}
  \label{tab:trennvarianten}
  \begin{tabular}{llll}
    \multicolumn{2}{c}{\itshape traditionelle Rechtschreibung} &
    \multicolumn{2}{c}{\itshape reformierte Rechtschreibung}\\
    herkömmlich & experimentell & herkömmlich & experimentell\\
    \addlinespace\toprule\addlinespace
    l\ftr{ös-t}e & lö-ste & lös-te & lös-te\\
    Fas-sa-de & Fas-sa-de & Fa\ftr{ss}a-de & Fas-sa-de\\
    mo-\ftr{d-e}rn-ste & mo-dern-ste & mo-\ftr{d-e}rns-te & mo-derns-te\\
    Abend-stern & Abend-stern & Aben\ftr{ds-t}ern & Abend-stern\\
    Mo\ftr{r-do}p-fer & Mord-op-fer & Mo\ftr{r-do}p-fer & Mord-op-fer\\
  \end{tabular}
\end{table*}

\subsubsection{Plain-\TeX}
\label{subsubsec:PlainTeX}

Bei Verwendung eines \hologo{eTeX}"=basierten Compilers
(\Programm{etex}, \Programm{pdftex}, \Programm{xetex} oder
\Programm{luatex}) steht der Befehl \verb+\uselanguage+ zum Laden von
Trennmustern zur Verfügung.  Er akzeptiert neben anderen die Argumente
\texttt{german}, \texttt{ngerman} und \texttt{swissgerman} (nicht
jedoch \texttt{austrian}, \texttt{naustrian} und
\texttt{nswissgerman}).\footnote{Die zulässigen Sprachbezeichnungen
  finden sich in der Datei \Datei{language.def} der
  \hologo{TeX}"=Installation.}  Um die Silbentrennung gemäß
reformierter deutscher Rechtschreibung zu aktivieren, ist also
folgender Befehl ausreichend:

\begin{lstlisting}[style=LaTeX]
  \uselanguage{ngerman}
\end{lstlisting}

Die Compiler \Programm{xetex} und \Programm{luatex} verwenden für alle
definierten Varietäten der deutschen Sprache die experimentellen
Trennmuster, die mit dem Paket \Paket{hyph-utf8} installiert sind;
\Programm{etex} und \Programm{pdftex} tun dies nur für
\texttt{swissgerman}.

Für die Varietäten \texttt{german} und \texttt{ngerman} laden
\Programm{etex} und \Programm{pdftex} hingegen aus Gründen der
Abwärtskompatibilität die herkömmlichen Trennmuster.  Die
experimentellen Trennmuster können am einfachsten aktiviert werden,
indem etwa \lstinline[style=LaTeX]+\uselanguage{ngerman}+ durch den
folgenden Befehl ersetzt wird:

\begin{lstlisting}[style=LaTeX]
  \uselanguage{ngerman-x-latest}
\end{lstlisting}

Auf entsprechende Weise ist es mit allen Compilern möglich, eine
alternative, manuell installierte Trennmusterversion zu verwenden:

\begin{lstlisting}[style=LaTeX]
  \uselanguage{ngerman-x-<datum>}
\end{lstlisting}

Vorausgesetzt ist dabei, dass ein entsprechender Eintrag in der
Konfigurationsdatei \Datei{language.def} oder einer lokalen Version
derselben existiert.

Soll der \verb+\uselanguage+-Befehl seine gewohnte Form behalten, ist
zum Laden alternativer Trennmuster auch folgendes Vorgehen mit Hilfe
des Pakets \Paket{hyphsubst} möglich:

\begin{lstlisting}[style=LaTeX]
  \input hyphsubst.sty
  \HyphSubstLet{ngerman}{ngerman-x-<datum>}
  \uselanguage{ngerman}
\end{lstlisting}

Zu beachten ist, dass in Plain-\hologo{TeX} unabhängig vom verwendeten
Compiler Wörter mit Umlauten, \emph{ß} oder Akzentbuchstaben an vielen
Trennstellen nicht automatisiert getrennt werden können.

\subsubsection{\LaTeX{} mit dem Sprachenpaket \Paket{Babel}}
\label{subsubsec:LaTeX+Babel}

Das Paket \Paket{Babel} kennt für die traditionelle Rechtschreibung
die Sprachbezeichnungen \texttt{german}, \texttt{austrian} und
\texttt{swissgerman}, für die reformierte Rechtschreibung
\texttt{ngerman}, \texttt{naustrian} und \texttt{nswissgerman}, wobei
diese sechs Varietäten durch drei Trennmustersätze abgedeckt werden
können (\autoref{tab:varietaeten}).

Die Compiler \Programm{xelatex} und \Programm{lualatex} verwenden für
alle definierten Varietäten der deutschen Sprache die experimentellen
Trennmuster, die mit dem Paket \Paket{hyph-utf8} installiert sind;
\Programm{latex} und \Programm{pdflatex} tun dies nur für
\texttt{swissgerman}.

Für die übrigen fünf Varietäten laden \Programm{latex} und
\Programm{pdflatex} hingegen aus Gründen der Abwärtskompatibilität
standardmäßig die herkömmlichen Trennmuster.  Um stattdessen die
experimentellen Trennmuster zu aktivieren, sind für die reformierte
Rechtschreibung folgende Zeilen in die Präambel des Dokuments
aufzunehmen.

\begin{lstlisting}[style=LaTeX]
  \usepackage[T1]{fontenc}
  \usepackage[ngerman]{babel}
  \babelprovide[hyphenrules=ngerman-x-latest]{ngerman}
\end{lstlisting}

Für die österreichische Varietät:

\begin{lstlisting}[style=LaTeX]
  \usepackage[T1]{fontenc}
  \usepackage[naustrian]{babel}
  \babelprovide[hyphenrules=ngerman-x-latest]{naustrian}
\end{lstlisting}

Auf entsprechende Weise ist es mit allen \hologo{LaTeX}"=Compilern
möglich, eine alternative, manuell installierte Trennmusterversion zu
aktivieren:

\begin{lstlisting}[style=LaTeX]
  \usepackage[T1]{fontenc}
  \usepackage[ngerman]{babel}
  \babelprovide[hyphenrules=ngerman-x-<datum>]{ngerman}
\end{lstlisting}

Der Befehl \verb+\babelprovide+ erlaubt auch innerhalb eines Dokuments
das mehrfache Umschalten der Trennmuster.

\subsubsection{\hologo{XeLaTeX}/\hologo{LuaLaTeX} mit dem
  Sprachenpaket \Paket{Polyglossia}}
\label{subsubsec:LaTeX+PolyGlossia}

Für \hologo{XeLaTeX} und \hologo{LuaLaTeX} steht als Alternative zu
\Paket{Babel} das Paket
\Paket{Polyglossia}\footnote{\url{https://ctan.org/pkg/polyglossia}}
zur Verfügung.  Mit den Zeilen

\begin{lstlisting}[style=LaTeX]
  \usepackage{polyglossia}
  \setmainlanguage{german}
\end{lstlisting}

\noindent in der Präambel werden die mit dem Paket \Paket{hyph-utf8}
installierten Trennmuster für die \emph{reformierte} Rechtschreibung
geladen.  Mit den Optionen \texttt{variant=""austrian},
\texttt{variant=""swiss} und \texttt{spelling=""old} können wahlweise
die österreichische Varietät, die Schweizer Varietät sowie die
traditionelle Rechtschreibung gewählt werden, beispielsweise:

\begin{lstlisting}[style=LaTeX]
  \usepackage{polyglossia}
  \setmainlanguage[variant=swiss, spelling=old]{german}
\end{lstlisting}

Um Trennhilfen wie \verb+""+, \verb+"ff+ oder \verb+"ck+ verwenden zu
können, muss zusätzlich die Option \texttt{babelshorthands=""true}
gesetzt werden.

Mit \hologo{XeLaTeX} ist es außerdem möglich, manuell installierte
statt der voreingestellten Trennmuster zu verwenden.  Hierzu kommt das
Paket \Paket{hyphsubst} wie folgt zum Einsatz.  Für die reformierte
Rechtschreibung:

\begin{lstlisting}[style=LaTeX]
  \usepackage{polyglossia}
  \setmainlanguage{german} % ggf. variant=austrian/swiss
  \usepackage[ngerman=ngerman-x-<datum>]{hyphsubst}
\end{lstlisting}

Für die traditionelle Rechtschreibung in Deutschland und Österreich:

\begin{lstlisting}[style=LaTeX]
  \usepackage{polyglossia}
  \setmainlanguage[spelling=old]{german} % ggf. variant=austrian
  \usepackage[german=german-x-<datum>]{hyphsubst}
\end{lstlisting}

Für die traditionelle Rechtschreibung in der Schweiz:

\begin{lstlisting}[style=LaTeX]
  \usepackage{polyglossia}
  \setmainlanguage[variant=swiss, spelling=old]{german}
  \usepackage[swissgerman=gswiss-x-<datum>]{hyphsubst}
\end{lstlisting}

Mit \hologo{LuaLaTeX} in Verbindung mit \Paket{Polyglossia} ist das
Ersetzen der Trennmuster auf diese Weise nicht möglich.

\subsubsection{p\TeX/up\TeX}

Die in Japan gebräuchlichen \hologo{TeX}"=Varianten p\TeX{} und
up\TeX{} verwenden für deutschsprachige Texte die experimentellen
Trennmuster, die mit dem Paket \Paket{hyph-utf8} installiert sind.

\subsubsection{\hologo{ConTeXt}}

\hologo{ConTeXt} besitzt eigene Trennmusterdateien.  Die Trennmuster
werden aus dem Paket \Paket{hyph-utf8} übernommen; Unterstützung für
die traditionelle deutsche Rechtschreibung in der Schweiz fehlt
allerdings zur Zeit.

Siehe \autoref{sec:fragen} für Hinweise, wie die Version der
\hologo{ConTeXt}-Trennmuster bestimmt werden kann.

\section{Trennregeln und Konventionen}
\label{sec:trennregeln}

Die Trennmuster für die traditionelle Rechtschreibung in Deutschland
und Österreich orientieren sich an den verbindlichen Regeln des Dudens
in der Fassung von 1991~\cite{duden:1991}.  Dasselbe gilt für die
Trennmuster für die traditionelle Rechtschreibung in der Schweiz,
jedoch mit einer unten beschriebenen Abweichung.  Die Trennmuster für
die reformierte Rechtschreibung orientieren sich an den amtlichen
Regeln für die Rechtschreibung der deutschen Sprache in der Fassung
von 2006~\cite{amtlRegeln:2006, amtlRegeln:2006:duden}.

Die Regeln lassen gewisse Freiheiten bei der Schreibung und Trennung
von Wörtern zu.  Da sich solche Freiheiten nicht ohne weiteres auf die
maschinelle Worttrennung übertragen lassen, wurden die im folgenden
beschriebenen Konventionen getroffen.  Hauptsächlich betreffen diese
die reformierte Rechtschreibung, die zusätzliche Freiheiten eingeführt
hat.\footnote{%
  Im Ergebnis weicht in reformierter Rechtschreibung die Trennung zum
  Beispiel des Dudens (nach Sprechsilben) von der Trennung mit diesen
  Trennmustern (bevorzugt etymologisch) ab, siehe auch
  \regelref{enum:reformEtymo} und \regelref{enum:reformClusterLR}
  sowie \autoref{sec:fragen}.}  Beziehen sich die Konventionen für die
reformierte Rechtschreibung auf die traditionelle Rechtschreibung, so
werden die entsprechenden Regeln etwas ausführlicher dargestellt.  Die
folgenden Abschnitte enthalten jedoch keine vollständige Aufstellung
der Silbentrennregeln.  Diese sind den entsprechenden Regelwerken zu
entnehmen.  Es folgen zunächst einige allgemeine Hinweise:

\begin{itemize}

\item In Liangs Trennalgorithmus werden Groß- und Kleinschreibung
  nicht unterschieden~\cite{liang:1983}.  Die Schreibweisen
  \emph{Nachtritt} und \emph{nachtritt} werden aus Sicht des
  Trennalgorithmus gleich behandelt (siehe auch
  \regelref{enum:tradDoppeld} und \regelref{enum:reformDoppeld}).

\item Die von einem Programm aus diesen Mustern abgeleiteten möglichen
  Trennstellen können (u.\,a. durch Programmfehler) durchaus von denen
  der zugrundeliegenden Wortliste abweichen.  So führt zum Beispiel
  die Eingabe \lstinline[style=LaTeX]+Meta"llegierung+
  (Dreikonsonantenregel in der traditionellen Rechtschreibung) mit dem
  Paket \Paket{Babel} zu den in \autoref{tab:trennung-dreik} gezeigten
  Trennmöglichkeiten.

  \begin{table}
    \centering
    \caption{Unterschiedlich ermittelte Trennmöglichkeiten.}
    \label{tab:trennung-dreik}
    \begin{tabular}{ll}
      Quelle & Trennmöglichkeiten\\
      \addlinespace
      \toprule
      \addlinespace
      \hologo{pdfLaTeX} mit \Paket{Babel}~3.8 & Me-tall(-l)egierung\\
      \hologo{pdfLaTeX} mit \Paket{Babel}~3.9 & Me-tall(-l)e-gie-rung\\
      erwünscht \emph{(vgl. \regelref{enum:tradnstd})} & Me-tall(-l)egie-rung\\
    \end{tabular}
  \end{table}

\item Die von \hologo{TeX} gewählte Trennung kann in Einzelfällen mit
  den \hologo{TeX}- und \Paket{Babel}"=Kürzeln
  \lstinline[style=LaTeX]+\-+ und \lstinline[style=LaTeX]+"-+ geändert
  werden.  Für dokumentweite Änderungen der Trennung eignet sich das
  Kommando \lstinline[style=LaTeX]+\hyphenation+.

\item Die Datei \Datei{CHANGES} beschreibt bekannte, systematische
  Fehler der Trennmuster.

\item In den Beispielen zeigt die linke (grüne) Spalte jeweils die
  Trennung mit den experimentellen Trennmustern, die rechten (roten)
  Spalten zeigen alternative oder unerwünschte Trennungen.

\end{itemize}

\subsection{Traditionelle Rechtschreibung in Deutschland und
  Österreich}
\label{sec:tradRS}

\begin{enumerate}[\hspace{1em}\itshape{T}1]
\labelformat{enumi}{\textit{T#1}}

\item\label{enum:tradhyphenmin} Die minimale unterstützte Silbenlänge
  am Wortanfang und "~ende beträgt zwei Buchstaben
  \cite[R~178]{duden:1991}.

  Siehe \autoref{sec:fragen} für Hinweise, wie sich diese Länge
  abweichend einstellen lässt.  Wird die Mindestlänge auf weniger als
  zwei Buchstaben verringert, so können fehlerhafte Trennungen
  auftreten.

\item\label{enum:tradSinn} Sinnentstellende und irreführende
  Trennungen werden möglichst vermieden \cite[R~181]{duden:1991}
  (siehe auch \regelref{enum:tradnstd}):

  \begin{tabular}[t]{TU}
    An-alpha-bet & Anal-phabet\\
    Kaf-ka-kenner & Kafkaken-ner\\
    Tal-entwäs-se-rung & Talent-wässerung\\
  \end{tabular}

  Beachte, dass derzeit die Unterdrückung von solchen Trennstellen bis
  zu einem gewissen Grade willkürlich und subjektiv ist.  Außerdem ist
  die Erfassung weit davon entfernt, vollständig zu sein.  Um diesen
  Problemen abzuhelfen, ist für zukünftige Versionen der Trennmuster
  ein anderer, automatisierter Ansatz geplant, welcher die Anzahl
  manuell zu erfassender Fälle sehr stark reduzieren wird.

\item\label{enum:tradDoppeld} In mehrdeutigen Wörtern werden
  Trennungen nur an übereinstimmenden Trennstellen zugelassen.

  \begin{tabular}[t]{TUU}
    nachtritt & nach-tritt & Nacht-ritt\\
    Wachstu-be & Wach-stube & Wachs-tube\\
    Druckerzeug-nis & Druck-erzeugnis & Drucker-zeugnis\\
    Mu-sikerle-ben & Musik-erleben & Musi-ker-leben\\
    Fuß-balleh-re & Fußball-ehre & Fußball-lehre\\
  \end{tabular}

  Beachte, die Trennstellen »Drucker-zeugnis« und »Musiker-leben«
  sind in den Interpretationen \emph{Druck-Erzeugnis} und
  \emph{Musik-Erleben} irreführend.  Sie entfallen nach
  \regelref{enum:tradSinn} und sind nicht als übereinstimmende
  Trennstellen anzusehen.  Zur Spezialtrennung »Fußball-lehre« siehe
  auch \regelref{enum:tradnstd}.

  Für diese Regel gelten die folgenden Einschränkungen:

  \begin{itemize}

  \item Bei mehrdeutigen Wörtern endend auf \emph{"~ende},
    \emph{"~enden}, \emph{"~endes} wird stets die Trennung der
    Partizipform des Verbs verwendet.

    \begin{tabular}[t]{TUU}
      fu-ßen-de & Fuß-ende & fußende\\
      spie-len-de & Spiel-ende & spielende
    \end{tabular}

  \item Mehrdeutigkeiten, die durch die Ersatzschreibweise von Wörtern
    mit~\emph{ß} auftreten, werden nicht berücksichtigt (vergleiche
    \regelref{enum:tradEszett}).

    \begin{tabular}[t]{TUU}
      Mas-se & \textls{M\kern-.4ptA-SSE} & \textls{M\kern-.4ptA\kern-.6ptSSE}
    \end{tabular}

  \end{itemize}

\item\label{enum:tradEszett} Wird der Buchstabe~\emph{ß} durch
  \emph{ss/SS} ersetzt, so bleibt die Trennung davon unberührt
  \cite[R~179]{duden:1991}:

  \begin{tabular}[t]{T}
    \textls{GRÖSS-TE}\\
    \textls{GRÜ-SSE}\\
    \textls{M\kern-.4ptA\kern-.4pt-SSES}\\
  \end{tabular}

  Für diese Regel gilt die folgende Einschränkung:
  \begin{itemize}

  \item Wenn durch den Ersatz von~\emph{ß} an dieser Stelle keine
    eindeutige Trennung möglich ist, so wird zugunsten der Bedeutung
    des Wortes in der normalen Schreibweise getrennt (siehe auch
    \regelref{enum:tradDoppeld}).

    \begin{tabular}[t]{TTU}
      \textls{FLÖS-SE} & (wegen flös-se) & \textls{FLÖ-SSE}\\
      \textls{MAS-SE} & (wegen Mas-se) & \textls{MA-SSE}\\
    \end{tabular}

    Beachte:
    \begin{itemize}

    \item Wird \emph{ß} mit \lstinline[style=LaTeX]+\MakeUppercase+
      durch~\emph{SS} ersetzt, so bleibt \emph{SS} stets ungetrennt.
      Die Trennung richtet sich dann nach der Schreibweise
      mit~\emph{ß} im Quelldokument.

    \item Existiert ein Wort in verschiedenen Varietäten in der
      Schreibweise mit~\emph{ß} und mit~\emph{ss}, so wird aufgrund
      dieser Einschränkung \emph{s-s} stets getrennt:

      \begin{tabular}[t]{TTUU}
        Ge-scho-ße  & (AT)\\
        Ge-schos-se & (D) & \textls{GESCHO-SSE} & (AT)\\
      \end{tabular}

    \item Wenn durch den Ersatz des~\emph{ß} an entfernten Stellen
      keine eindeutige Trennung möglich wird, zum Beispiel an
      Wortfugen, so werden die betroffenen Trennungen gemäß
      \regelref{enum:tradDoppeld} unterdrückt.  In der Folge wird
      gegebenenfalls auch die Trennung von \emph{ss/SS} unterdrückt.

      \begin{tabular}[t]{TU}
        \textls{BAHN-HOFSTRASSE} & \textls{BAHNHOF-STRA-SSE}\\
                                 & \textls{BAHNHOFS-TRAS-SE}\\
      \end{tabular}

    \end{itemize}

  \end{itemize}

\item\label{enum:tradOW} In Ableitungen von Namen auf \emph{"~ow} wird
  die Nottrennung der Ableitungssilben \emph{"~er}, \emph{"~ern},
  \emph{"~ers} unterdrückt \cite[R~180]{duden:1991}:

  \begin{tabular}[t]{TU}
    Tel-tower & Teltow-er\\
    Trep-towern & Treptow-ern\\
    Pan-kowers & Pankow-ers\\
  \end{tabular}

\item\label{enum:tradnstd} Spezialtrennungen (\emph{engl.:}
  non-standard hyphenation), die nach Regeln erfolgen, die über das
  bloße Einfügen eines Trennstrichs hinausgehen, wie die \emph{ck}"~
  oder die Dreikonsonantenregel, kann \hologo{TeX} nicht automatisch
  behandeln.  Aus diesem Grund sind solche Trennstellen in diesen
  Trennmustern nicht berücksichtigt.

  \begin{tabular}[t]{lTUU}
    \lstinline[style=LaTeX]+drucken+ & drucken & druk-ken\\
    \lstinline[style=LaTeX]+Zuckerbäcker+ & Zucker-bäcker & Zuk-kerbäk-ker\\
    \lstinline[style=LaTeX]+Brennessel+ & Brennes-sel & Brenn-nessel\\
    \lstinline[style=LaTeX]+Stoffetzen+ & Stoffet-zen & Stoff-fetzen\\
  \end{tabular}

  Die Dreikonsonantenregel birgt aufgrund des ausgefallenen
  Konsonanten die Gefahr irreführender und sinnentstellender
  Trennungen (siehe auch \regelref{enum:tradSinn}).  Trennstellen, die
  in einem Abstand von zwei Lauten auf eine Wortfuge mit Anwendung der
  Dreikonsonantenregel folgen, werden daher grundsätzlich unterdrückt.

  \begin{tabular}[t]{lTUU}
    \lstinline[style=LaTeX]+Metallegierung+ & Me-tallegie-rung & Metall-legierung & Metalle-gierung\\
    \lstinline[style=LaTeX]+schnellebige+ & schnellebi-ge & schnell-lebige & schnelle-bige\\
    \lstinline[style=LaTeX]+Stilleben+ & Stilleben & Still-leben & Stille-ben\\
  \end{tabular}
  \par\nobreak
  \textit{auch:}

  \begin{tabular}[t]{lTUU}
    \lstinline[style=LaTeX]+Abfallager+ & Ab-fallager & Abfall-lager & Abfalla-ger\\
    \lstinline[style=LaTeX]+Zellstoffabrik+ & Zell-stoffabrik & Zellstoff-fabrik & Zellstoffa-brik\\
  \end{tabular}

  Das Paket \Paket{Babel} stellt verschiedene Kürzel zur Verfügung,
  u.\,a. \lstinline[style=LaTeX]+"ck+%
  \footnote{Spezialtrennungen werden in \hologo{TeX} mit Hilfe des
    Kommandos \lstinline[style=LaTeX]+\\discretionary+ kodiert.  So
    wird zum Beispiel das \Paket{Babel}-Kürzel
    \lstinline[style=LaTeX]+\"ck+ in der Eingabe während der
    Kompilation durch
    \lstinline[style=LaTeX]+\\discretionary\{k-\}\{k\}\{ck\}+ ersetzt,
    wodurch \emph{k-k}-Trennungen möglich werden.}
%
  oder \lstinline[style=LaTeX]+"ff+ \emph{etc.}, mit denen
  Spezialtrennungen im Quelldokument ausgezeichnet werden können
  (siehe auch \autoref{tab:trennung-dreik}).  Das Alternativpaket
  \Paket{Polyglossia} stellt diese Kürzel ebenfalls zur Verfügung,
  falls beim Laden der deutschen Sprache (in der Regel mit dem Befehl
  \lstinline[style=LaTeX]+\setmainlanguage+) die Option
  \lstinline[style=LaTeX]+babelshorthands=true+ angegeben wird.

  \hologo{LuaTeX}%
  \footnote{\url{http://www.luatex.org/}}
%
  soll in einer zukünftigen Version Mechanismen zur automatischen
  Behandlung von Spezialtrennungen bereitstellen.  Eine physische
  Auszeichnung im Quelltext ist dann nicht mehr erforderlich.  Die
  entsprechenden Spezialtrennmuster für die deutsche Sprache werden
  ebenfalls im Rahmen dieses Projekts erstellt.

\end{enumerate}

\subsection{Traditionelle Rechtschreibung in der Schweiz}
\label{sec:tradchRS}

Die Trennmuster für die traditionelle Rechtschreibung in der Schweiz
folgen weitgehend den Konventionen für die traditionelle
Rechtschreibung in Deutschland (siehe \autoref{sec:tradRS}).  Die
folgende Liste enthält daher nur Fälle, in denen davon abgewichen wird
oder deren Beschreibung aus anderen Gründen sinnvoll erscheint.

\begin{enumerate}[\hspace{1em}\itshape{TS}1]
\labelformat{enumi}{\textit{TS#1}}

\item\label{enum:tradchEszett} Wörter mit \emph{ß} werden gemäß den
  Regeln für die traditionelle Rechtschreibung in Deutschland
  getrennt.

\item\label{enum:tradchSS} Abweichend von \regelref{enum:tradEszett}
  wird \emph{ss/SS} immer als Doppelkonsonant behandelt und
  gegebenenfalls getrennt:

  \begin{tabular}[t]{T}
    grös-ste\\
    Grüs-se\\
    Mas-ses\\
  \end{tabular}

  Beachte, wird \emph{ß} jedoch mit
  \lstinline[style=LaTeX]+\MakeUppercase+ durch~\emph{SS} ersetzt, so
  bleibt~\emph{SS} stets ungetrennt.  Die Trennung richtet sich dann
  nach der Schreibweise mit~\emph{ß} im Quelldokument (siehe
  \regelref{enum:tradchEszett}).

\end{enumerate}

\subsection{Reformierte Rechtschreibung}
\label{sec:reformRS}

\begin{enumerate}[\hspace{1em}\itshape{R}1]
\labelformat{enumi}{\textit{R#1}}

\item\label{enum:reformhyphenmin} Die minimale unterstützte Silbenlänge
  am Wortanfang und "~ende beträgt zwei Buchstaben
  \cite[\S~107]{amtlRegeln:2006, amtlRegeln:2006:duden}.

  Siehe \autoref{sec:fragen} für Hinweise, wie sich diese Länge
  abweichend einstellen lässt.  Wird die Mindestlänge auf weniger als
  zwei Buchstaben verringert, so können fehlerhafte Trennungen
  auftreten.

\item\label{enum:reformEtymo} Falls die Trennung nach Sprechsilben und
  die etymologische (sprachgeschichtliche) Trennung kollidieren, wird
  weitgehend die etymologische Trennung gewählt
  \cite[\S~113]{amtlRegeln:2006, amtlRegeln:2006:duden}:

  \begin{tabular}[t]{RUU}
%    Heli-ko-pter & Helikop-ter\\
%    in-ter-view-en & intervie-wen\\
    in-ter-es-sant & inte-ressant\\
    Lin-ole-um & Li-noleum & Lino-leum\\
    Päd-ago-ge & Pä-dagoge & Päda-goge\\
  \end{tabular}

\item\label{enum:reformClusterLR} In Fremdwörtern bleiben die
  Buchstabengruppen \emph{bl}, \emph{pl}, \emph{fl}, \emph{gl},
  \emph{cl}, \emph{kl}, \emph{phl}; \emph{br}, \emph{pr}, \emph{dr},
  \emph{tr}, \emph{fr}, \emph{vr}, \emph{gr}, \emph{cr}, \emph{kr},
  \emph{phr}, \emph{thr}; \emph{chth}; \emph{gn}, \emph{kn} im
  allgemeinen ungetrennt, nicht jedoch \emph{str}
  \cite[\S~112]{amtlRegeln:2006, amtlRegeln:2006:duden}
  i.\,V.\,m.~\cite[R~179]{duden:1991}:

  \begin{tabular}[t]{RU}
    Ar-thri-tis & Arth-ritis\\
%    Co-gnac & Cog-nac\\
    Di-plom & Dip-lom\\
%    Fe-bru-ar & Feb-ruar\\
    igno-rie-re & ig-noriere\\
    In-te-gral & Integ-ral\\
  \end{tabular}
  \par\nobreak
  \textit{aber:}

  \begin{tabular}[t]{RUU}
    In-dus-trie & Indu-strie & Indust-rie\\
%    Ma-gis-tra-le & Magi-strale\\
    de-struk-tiv\\
    sub-lim\\
  \end{tabular}

\item\label{enum:reformSinn} Sinnentstellende und irreführende
  Trennungen werden möglichst vermieden \cite[\S~107]{amtlRegeln:2006,
    amtlRegeln:2006:duden}:

  \begin{tabular}[t]{RU}
    An-alpha-bet & Anal-phabet\\
    Kaf-ka-kenner & Kafkaken-ner\\
    Tal-entwäs-se-rung & Talent-wässerung\\
  \end{tabular}

\item\label{enum:reformDoppeld} In mehrdeutigen Wörtern werden
  Trennungen nur an übereinstimmenden Trennstellen zugelassen:

  % Das »@{}« entfernt den rechten Rand der Tabelle und verhindert
  % so eine »overfull«-Warnung.
  \begin{tabular}[t]{RUUU@{}}
    Druckerzeug-nis & Dru-ckerzeugnis & Druck-erzeugnis &
    Drucker-zeugnis\\
    Mu-sikerle-ben & Musi-kerleben & Musik-erleben & Musiker-leben\\
    nachtritt & nach-tritt & Nacht-ritt\\
    Wachstu-be & Wach-stube & Wachs-tube\\
  \end{tabular}

  Beachte, die Trennstellen »Drucker-zeugnis« und »Musiker-leben«
  sind in den Interpretationen \emph{Druck-Erzeugnis} und
  \emph{Musik-Erleben} irreführend.  Sie entfallen nach
  \regelref{enum:reformSinn} und sind nicht als übereinstimmende
  Trennstellen anzusehen.

  Für diese Regel gilt die folgende Einschränkung:

  \begin{itemize}

  \item Bei mehrdeutigen Wörtern endend auf \emph{"~ende},
    \emph{"~enden}, \emph{"~endes} wird stets die Trennung der
    Partizipform des Verbs verwendet.

    \begin{tabular}[t]{RUU}
      fu-ßen-de & Fuß-ende & fußende\\
      spie-len-de & Spiel-ende & spielende
    \end{tabular}

  \end{itemize}

\item\label{enum:reformEszett} Wird der Buchstabe~\emph{ß} durch
  \emph{ss/SS} ersetzt, so wird \emph{s"~s} getrennt \cite[\S\S~25~E3,
  110]{amtlRegeln:2006, amtlRegeln:2006:duden}:

  \begin{tabular}[t]{R}
    \textls{GRÖS-STE}\\
    \textls{GRÜS-SE}\\
    \textls{M\kern-.4ptA\kern-.6ptS-SES}\\
  \end{tabular}

  Wird~\emph{ß} mit \lstinline[style=LaTeX]+\MakeUppercase+ oder in
  Kapitälchen in~\emph{SS} gewandelt, so bleibt~\emph{SS} ungetrennt.
  Dies ist kein Fehler in den Trennmustern, sondern im
  \hologo{LaTeX}-Kern fest implementiert.

% \item\label{enum:reformOW} In Ableitungen von Namen auf \emph{"~ow}
%   bleibt \emph{"~ow} ungetrennt, wenn es den Laut [o\,:] bezeichnet.
%   Die Nottrennung der Ableitungssilben \emph{"~er}, \emph{"~ern},
%   \emph{"~ers} wird unterdrückt \cite[\S~113]{amtlRegeln:2006,
%     amtlRegeln:2006:duden}
%   i.\,V.\,m.~\cite[R~180]{duden:1991}:

%   \begin{tabular}[t]{RUU}
%     Tel-tower & Telto-wer & Teltow-er\\
%     Trep-towern & Trepto-wern & Treptow-ern\\
%     Pan-kowers & Panko-wers & Pankow-ers\\
%   \end{tabular}

\end{enumerate}


\section{Trennfehler}
\label{sec:trennfehler}

Mit den vorliegenden Trennmustern können sämtliche Wörter der
zugrundeliegenden Wortliste fehlerfrei getrennt werden.  Technisch
gesprochen endet der letzte \Programm{patgen}-Lauf mit der Meldung

\begin{lstlisting}[style=shell]
  1783028 good, 0 bad, 0 missed
  100.00 %, 0.00 %, 0.00 %
\end{lstlisting}

\noindent (der Wert vor \lstinline[style=shell]+good+ ist vom
Listenumfang abhängig).  Trotz des großen Umfangs der Wortliste lassen
sich Trennfehler in Wörtern, die nicht in der Liste enthalten sind,
nicht vermeiden.  Der Umfang der Wortliste kann allerdings nicht
beliebig erweitert werden.%
\footnote{Liangs Schema sieht nur einen begrenzten Bereich für die
  Trennstellenbewertungen vor (0--9).  Die derzeitigen Trennmuster
  vewenden Bewertungen bis zur Höhe~7.}
%
In den folgenden Fällen sollten fehlerhafte Trennungen der Trennmuster
jedoch gemeldet werden:

\begin{enumerate}[\hspace{1em}A.]

\item\label{enum:kritWLfehlerhaft} Das Wort ist bereits in der
  Wortliste enthalten.  Der Eintrag ist jedoch fehlerhaft.

\end{enumerate}

Falls das Wort nicht in der Wortliste enthalten ist, bestehen sehr
gute Chancen, dass es aufgenommen wird, wenn eines der folgenden
Kriterien erfüllt ist:

\begin{enumerate}[\hspace{1em}A.]
  \refstepcounter{enumi}% Fortsetzung der obigen Aufzählung.

\item\label{enum:kritHerkTM} Das betreffende Wort wird mit den
  \emph{herkömmlichen} Trennmustern für die traditionelle oder
  reformierte Rechtschreibung korrekt getrennt.  Korrekt bedeutet
  hier: Nicht alle möglichen Trennstellen müssen erkannt werden; es
  werden jedoch in keinem Fall falsche Trennstellen ermittelt.  Zum
  Testen kann in \TeX\ der folgende Aufruf verwendet werden (die
  Ausgabe erfolgt in der \texttt{log}-Datei):

\begin{lstlisting}[style=LaTeX]
  \showhyphens{durch Leerzeichen getrennte Wörter}
\end{lstlisting}

\item\label{enum:kritSinn} Es handelt es sich um eine orthographisch
  richtige, aber sinnentstellende oder irreführende Trennung.
  Berücksichtigt werden allerdings nur Wörter, die aus höchstens zwei
  (gegebenenfalls prä- und suffigierten) Wörtern zusammengesetzt sind,
  zum Beispiel »Talent-wässerung«.  Nicht berücksichtigt wird
  hingegen die »Talent-wässerungsanlage«.

\end{enumerate}

Einige bekannte Fehler in den Trennmustern sind in der Datei
\Datei{CHANGES} verzeichnet.  Noch nicht bekannte falsche, fehlende
und unerwünschte Worttrennungen können an die E-Mail-Adresse
\href{mailto:trennmuster@dante.de}{trennmuster@dante.de} gerichtet
werden.

Trennfehler, die in den Trennmustern nicht korrigiert werden können,
können mit Hilfe einer privaten Ausnahmeliste behandelt werden:

\begin{lstlisting}[style=LaTeX]
  \hyphenation{Tal-entwäs-se-rungs-an-la-ge Kaf-ka-kenner-klub}
\end{lstlisting}

Die aktuelle und ältere Ausgaben der Trennmuster sind im Dateibereich
des Trennmuster-Wikis erhältlich.%
\footnote{\url{http://projekte.dante.de/Trennmuster}}
%
Im Entwicklerrepositorium\footnote{%
  siehe \url{http://projekte.dante.de/Trennmuster/Entwickler}}
befindet sich ein Makefile, mit dem jederzeit neue Trennmuster erzeugt
werden können.

\nobreak
\noindent\parbox{\linewidth}{%
  \vspace*{\baselineskip}
  \raggedright
  \itshape
  Happy \TeX ing!\newline
  Die deutschsprachige Trennmustermannschaft
}


\begingroup
  \RaggedRight
  \bibliography{dehyph-exptl}
\endgroup


\appendix
\section{Dateien und Installation}
\label{sec:installation}

Die eigentlichen Trennmusterdateien liegen in \Abk{utf-8}"=Kodierung
vor (siehe \autoref{tab:dateien}, Endung \texttt{.pat}).  Sie werden
von \hologo{TeX} nicht direkt geladen, sondern durch Manteldateien,
die ebenfalls Teil des Pakets sind (Endung \texttt{.tex}).  Wird eine
8-Bit-fähige \hologo{TeX}"=Variante erkannt, übernehmen diese
Manteldateien die Konvertierung der Trennmuster in die
\Abk{t1}-Kodierung.

\begin{table}
  \centering
  \caption{Trennmuster- und Manteldateien}
  \label{tab:dateien}
  \begin{tabular}{l>{\ttfamily}l>{\ttfamily}l}
    Rechtschreibung & \normalfont Trennmusterdatei & \normalfont Manteldatei\\
    \addlinespace\toprule\addlinespace
    traditionell & dehypht-x-<datum>.pat & dehypht-x-<datum>.tex\\
    traditionell (Schweiz) & dehyphts-x-<datum>.pat & dehyphts-x-<datum>.tex\\
    reformiert & dehyphn-x-<datum>.pat & dehyphn-x-<datum>.tex\\
  \end{tabular}
\end{table}

Bei der Installation werden die Manteldateien an die in
\autoref{tab:varietaeten} gezeigten Trennmusterbezeichner gebunden.
Diese Schritte werden für verschiedene \hologo{TeX}"=Verteilungen in
der Datei \Datei{INSTALL} beschrieben.  Nach der Installation können
die experimentellen Trennmuster wie in \autoref{sec:aktivieren}
gezeigt verwendet werden.


\section{Fragen \& Antworten}
\label{sec:fragen}

\newcommand*{\fragefont}{\itshape}
\newcommand*{\themenfont}{\large\normalfont}
\newcounter{cntfrage}% Zähler für Fragen.
\newcounter{thema}% Zähler für Themenüberschriften.
\renewcommand*{\thethema}{\Roman{thema}.}
\newcounter{frage}% Zähler für Fragen.
\renewcommand*{\thefrage}{\arabic{frage}.}
\newboolean{nextfrage}

\makeatletter

%%% Neue zref-Liste frage = (type, text, anchor).
\zref@newlist{frage}
\zref@newprop{type}{f}
\zref@newprop{text}{??}
\zref@addprop{frage}{type}
\zref@addprop{frage}{text}
\zref@addprop{frage}{anchor}

%%% Fügt eine neue Themenüberschrift ein.
\newcommand*{\fragenthema}[1]{%
  \par
  \pagebreak[1]
  \vspace{1.5\baselineskip plus .6\baselineskip minus .6\baselineskip}
  \refstepcounter{cntfrage}
  \stepcounter{thema}
  \zref@setcurrent{type}{t}
  \zref@setcurrent{text}{\thethema~#1}
  \zref@labelbylist{frage:\thecntfrage}{frage}
  \noindent{\themenfont\thethema~#1\par}
}

%%% Umgebung für eine Frage mit Antwort.
\newenvironment{frageantwort}[1]{%
  \par
  \vspace{.25\baselineskip plus .1\baselineskip minus .1\baselineskip}
  \refstepcounter{cntfrage}
  \stepcounter{frage}
  % bei Aufzählungen innerhalb einer Frage römisch nummerieren
  \renewcommand{\labelenumi}{\roman{enumi}.}
  \zref@setcurrent{type}{f}
  \zref@setcurrent{text}{\thefrage~#1}
  \zref@labelbylist{frage:\thecntfrage}{frage}
  \noindent{\fragefont\thefrage~#1\par}
  \nobreak\noindent\ignorespaces
}{%
  \vspace{.25\baselineskip plus .1\baselineskip minus .1\baselineskip}
}

%%% Zeige alle Fragen in sortierter Reihenfolge.
\newcommand{\zeigefragen}{%
  \par
  \zref@refused{frage:1}
  \setcounter{cntfrage}{1}
  \setboolean{nextfrage}{true}
  \whiledo{\boolean{nextfrage}}{
    \vspace{.25\baselineskip plus .1\baselineskip minus .1\baselineskip}
    \ifthenelse{\equal{\zref@extract{frage:\thecntfrage}{type}}{f}}{% Frage
      \noindent%
      \begingroup%
        \fragefont%
        \hyperlink{\zref@extract{frage:\thecntfrage}{anchor}}{%
          \zref@extract{frage:\thecntfrage}{text}%
        }%
        \par
      \endgroup
    }{% Themenüberschrift
      \noindent%
      \begingroup%
        \themenfont%
        \zref@extract{frage:\thecntfrage}{text}%
        \par
      \endgroup
      \nobreak
    }
    \stepcounter{cntfrage}
    \zref@ifrefundefined{frage:\thecntfrage}{\setboolean{nextfrage}{false}}{}
  }
  \vspace{.5\baselineskip plus .1\baselineskip minus .1\baselineskip}
  \setcounter{cntfrage}{0}
  \setcounter{frage}{0}
}


\zeigefragen


\fragenthema{Verwenden der Trennmuster}

\begin{frageantwort}{Wie kann für die experimentellen Trennmuster aus
    dem Paket \Paket{dehyph-exptl} das Datum ermittelt werden, das
    Teil des Trennmusterbezeichners ist, wie in
    \autoref{sec:verwenden} erwähnt?}

  Die Trennmusterbezeichner werden in der Datei \Datei{language.dat}
  definiert, wo auch die Verbindung zu den Manteldateien hergestellt
  wird.  Da ein Teil der Trennmusterbezeichner bereits bekannt ist,
  \verb+german-x+, vgl. \autoref{tab:varietaeten}, kann die Datei
  \Datei{language.dat} danach durchsucht werden.  Zunächst muss der
  Ort der Datei mit Hilfe des Kommandos \Programm{kpsewhich} ermittelt
  werden.  Es folgen die vollständigen Kommandos für unixähnliche
  Shells und die Windows-Kommandozeile \Datei{cmd.exe} (einzugeben
  ohne Zeilenumbruch).  Achtung, bei den einfachen Anführungszeichen
  handelt es sich um Gravis (»Backquotes«).

\begin{lstlisting}[style=shell, caption=Shell]
  grep -i german-x `kpsewhich language.dat`
\end{lstlisting}

\begin{lstlisting}[style=shell, caption=\Datei{cmd.exe}]
  for /F "usebackq" %f in (`kpsewhich language.dat`)
    do find /i "german-x" "%f"
\end{lstlisting}

  Die Ausgabe dieser Kommandos sieht etwa wie folgt aus (das Datum
  kann abweichen):

\begin{lstlisting}
  german-x-2022-03-16 dehypht-x-2022-03-16.tex
  =german-x-latest
  ngerman-x-2022-03-16 dehyphn-x-2022-03-16.tex
  =ngerman-x-latest
\end{lstlisting}

  Die gesuchten Trennmusterbezeichner befinden sich in der ersten
  Spalte und lauten in diesem Beispiel
  \lstinline[style=LaTeX]{german-x-2022-03-16} und
  \lstinline[style=LaTeX]{ngerman-x-2022-03-16}.  In der zweiten
  Spalte kann man die Namen der Manteldateien erkennen
  (vgl. \autoref{tab:dateien}).  Die mit einem Gleichheitszeichen
  beginnenden Zeilen definieren ein Synonym für den
  Trennmusterbezeichner der unmittelbar vorangehenden Zeile in der
  Datei \Datei{language.dat}.
\end{frageantwort}


\begin{frageantwort}{Wie kann die Version der experimentellen Muster
    ermittelt werden, die im Paket \Paket{hyph-utf8} enthalten sind?}

  Das Vorgehen ähnelt dem der vorherigen Antwort.  Zunächst wird der
  Ort einer bestimmten Datei ermittelt.  Diese wird dann nach einer
  hilfreichen Zeichenkette durchsucht.

\begin{lstlisting}[style=shell, caption=Shell]
  grep dehyph `kpsewhich hyph-de-1996.tex`
\end{lstlisting}

\begin{lstlisting}[style=shell, caption=\Datei{cmd.exe}]
  for /F "usebackq" %f in (`kpsewhich hyph-de-1996.tex`)
    do find "dehyph" "%f"
\end{lstlisting}

  Die Ausgabe dieser Kommandos sieht etwa wie folgt aus (das Datum
  kann abweichen):

\begin{lstlisting}
  \message{German Hyphenation Patterns (Reformed Orthography, 2006)
    `dehyphn-x' 2022-03-16 (WL)}
\end{lstlisting}
\end{frageantwort}

\begin{frageantwort}{Wie kann die Version der experimentellen Muster
    ermittelt werden, die von \hologo{ConTeXt} verwendet werden?}

  Derzeit (Stand Februar 2021) gibt es keine Möglichkeit, die Version direkt
  zu ermitteln.  Allerdings werden die \hologo{ConTeXt}-Muster automatisch
  aus den vorhandenen \Paket{hyph-utf8}-Mustern erzeugt; die Versionen sind
  daher ident zu den in \hologo{XeTeX} verwendeten Mustern.
\end{frageantwort}

\begin{frageantwort}{Ich habe neue Trennmuster erstellt. Wie kann ich diese
installieren?}

  Die folgende Antwort richtet sich an Nutzer von \hologo{TeX} Live, die auf
  ihrem System über Administrationsrechte verfügen.

  \begin{enumerate}

  \item Überprüfen Sie, ob auf Ihrem System das Verzeichnis
  \Datei{TEXMFLOCAL\slash tex\slash generic\slash config}
  existiert.\footnote{Bei \Datei{TEXMFLOCAL} handelt es sich um den lokalen
  Baum der \hologo{TeX}-Installation, unter Unix in der Regel
  \Datei{/usr/local/texlive/texmf-local}, unter Windows \Datei{C:\textbackslash
  texlive\textbackslash texmf-local}.} Falls nicht, legen Sie es an.

  \item Überprüfen Sie, ob im Verzeichnis \Datei{TEXMFLOCAL\slash tex\slash
  generic\slash config} die Dateien \Datei{language-local.dat} und
  \Datei{language-local.def} existieren. Falls nicht, legen Sie sie an.

  \item Ergänzen Sie in der Datei \Datei{language-local.def} die folgende
  Zeile:\footnote{Als Vorbild für die erforderlichen Angaben können die
  vorhandenen Eintragungen in den Konfigurationsdateien \Datei{language.def}
  und \Datei{language.dat} der \hologo{TeX}-Installation dienen.}

\begin{lstlisting}[style=LaTeX]
  \addlanguage{<Name>}{<Datei>}{<Ausnahmedatei>}{<lmin>}{<rmin>}
\end{lstlisting}

  \noindent Dabei ist \texttt{<Name>} eine selbstgewählte Bezeichnung für die
  Trennmuster, \texttt{<Datei>} der Name der Trennmusterdatei bzw. der
  Manteldatei, wenn es eine solche gibt (vgl. \autoref{sec:installation}),
  \texttt{<Ausnahmedatei>} ggf. eine Datei mit Trennausnahmen (ansonsten leer),
  \texttt{<lmin>} die Mindestlänge einer am Wortanfang und
  \texttt{<rmin>} die Mindestlänge einer Wortende abgetrennten
  Silbe. Für deutsche Trennmuster setzt man die beiden letztgenannten Werte
  typischerweise auf \texttt{2}.

  Ergänzen Sie entsprechend in der Datei \Datei{language-local.dat} die
  folgende Zeile:

\begin{lstlisting}
  <Name> <Datei>
\end{lstlisting}

  \item Verschieben Sie Ihre Trennmusterdateien in das Verzeichnis
  \Datei{TEXMFLOCAL\slash tex\slash generic}.

  \item Führen Sie \emph{als Administrator} die Befehle

\begin{lstlisting}[style=shell]
  mktexlsr
\end{lstlisting}

  \noindent und

\begin{lstlisting}[style=shell]
  tlmgr generate --rebuild-sys language
\end{lstlisting}

  \noindent aus.

  \end{enumerate}

  Die so installierten Trennmuster können nun den Hinweisen in
  \autoref{sec:aktivieren} entsprechend aktiviert werden.
\end{frageantwort}

\begin{frageantwort}{Um in bestimmten Dokumenten einen dauerhaft stabilen
Umbruch zu erreichen, will ich verhindern, dass die zur Zeit vorhandenen
Trennmuster bei einer Aktualisierung der \hologo{TeX}-Installation verloren
gehen. Wie gehe ich vor?}

  Suchen Sie auf Ihrem System nach der benötigten Trennmusterdatei und der
  zugehörigen Manteldatei gemäß \autoref{tab:dateien} und erstellen Sie Kopien
  dieser Dateien. Folgen Sie dann den Anweisungen der Antwort auf die vorige
  Frage.

\end{frageantwort}


\fragenthema{Rechtschreibung}

\begin{frageantwort}{Verlag, Prüfer o.\,ä. bemängeln die Trennung der
    Trennmuster für die reformierte Rechtschreibung.  Zum Beispiel
    wird »In-dus-trie« getrennt, der Duden trennt jedoch
    »In-dust-rie«.}

  Die amtlichen Regeln für die Rechtschreibung der deutschen Sprache
  lassen für viele Wörter mehrere Trennvarianten zu.  Die Trennmuster
  und ebenso Wörterverzeichnisse legen sich aus praktischen Gründen
  auf eine Trennvariante fest.  Sie können daher unterschiedliche
  Trennungen verwenden, ohne dass eine von beiden falsch ist.  Aus
  diesem Grund sind Wörterverzeichnisse nicht geeignet, eine bestimmte
  Trennung auf Richtigkeit zu prüfen.  Verbindlich sind einzig die
  amtlichen Regeln für die Rechtschreibung der deutschen
  Sprache~\cite{amtlRegeln:2006}.  Häufig sind diese im Anhang eines
  Wörterbuchs abgedruckt.  Die von den Trennmustern befolgten
  Konventionen können \autoref{sec:trennregeln} entnommen werden.

  Wenn Unsicherheit darüber herrscht, wie die Rechtschreibung geprüft
  wird, sollte dies frühzeitig geklärt werden.  Nicht jedem Redakteur
  oder Prüfer ist bewusst, dass der Duden seine normative Stellung mit
  der Rechtschreibreform~1996 eingebüßt hat.
\end{frageantwort}


\begin{frageantwort}{Gibt es dudenkonforme Trennmuster für die
    reformierte Rechtschreibung?}

  Zur Zeit nicht, es ist auch nicht geplant.  Dieses Projekt ist
  jedoch offen für Vorschläge und Mitarbeit.
\end{frageantwort}


\begin{frageantwort}{Weshalb werden noch Trennmuster für die
    traditionelle Rechtschreibung bereitgestellt?}

  Die amtlichen Regeln für die Rechtschreibung der deutschen Sprache
  in der Fassung von 2006 sind nur für öffentliche Einrichtungen und
  Behörden verbindlich.  Im privaten Schriftverkehr kann man wahlweise
  die traditionelle oder die reformierte Rechtschreibung verwenden.%
  \footnote{Oder auch keine von beiden.}
%
  Daher erfreut sich die traditionelle Rechtschreibung weiterhin
  großer Beliebtheit.

  Für Texte in gebrochener Schrift ist die traditionelle
  Rechtschreibung sogar vorzuziehen.
\end{frageantwort}

\begin{frageantwort}{In der Voreinstellung für die deutsche Sprache
    beträgt die Mindestlänge abgetrennter Silben zwei Buchstaben. Wie
    kann diese Mindestlänge verändert werden?}

  Nehmen wir an, am Wortanfang sollen nicht weniger als drei und am
  Wortende nicht weniger als vier Buchstaben abgetrennt werden.  Dann
  sind folgende Befehle nötig:

  \begin{itemize}
  \item Mit Plain-\hologo{TeX}:

\begin{lstlisting}[style=LaTeX]
  \uselanguage{<sprache>}
  \lefthyphenmin=3
  \righthyphenmin=4
\end{lstlisting}

    Ersetze dabei \texttt{<sprache>} durch eine der in
    \autoref{subsubsec:PlainTeX} erwähnten Trennmusterbezeichnungen,
    beispielsweise \texttt{ngerman}.

  \item Mit \hologo{LaTeX} und \Paket{Babel}:

\begin{lstlisting}[style=LaTeX]
  \usepackage[<sprache>]{babel}
  \renewcommand*{\<sprache>hyphenmins}{34}
\end{lstlisting}

    Ersetze dabei \texttt{<sprache>} durch eine der in
    \autoref{subsubsec:LaTeX+Babel} erwähnten
    \Paket{Babel}-Sprachbezeichnungen, beispielsweise
    \texttt{ngerman}.

  \item Mit \hologo{XeLaTeX}/\hologo{LuaLaTeX} und
    \Paket{Polyglossia} (Version 1.50 oder neuer):

\begin{lstlisting}[style=LaTeX]
  \usepackage{polyglossia}
  \setmainlanguage{german}
  \setlanghyphenmins[<option>]{<sprache>}{3}{4}
\end{lstlisting}

    Ersetze dabei \texttt{<option>} und \texttt{<sprache>} durch eine der in
    \autoref{subsubsec:LaTeX+PolyGlossia} erwähnten
    \Paket{Polyglossia}-Sprachbezeichnungen, beispielsweise
    \texttt{spelling=old} und \texttt{german}.
  \end{itemize}
\end{frageantwort}

\begin{frageantwort}{Können Trennungen mit einer Silbenlänge von nur
    einem Buchstaben ermöglicht werden, zum Beispiel für den Satz in
    schmalen Kolumnen?}

  Mit diesen Trennmustern ist das nicht möglich.  Die
  Abtrennung einzelner Vokale, zum Beispiel »A-bend«, war nur
  kurzzeitig zulässig; die entsprechende Regelung von 1996 wurde mit
  der Rechtschreibreform 2006 wieder zurückgenommen.  Wird die
  Silbenmindestlänge für die Worttrennung auf eins verringert, können
  falsche Trennungen auftreten.
\end{frageantwort}


\fragenthema{Mitarbeit}

\begin{frageantwort}{Wie kann ich mich über dieses Projekt
    informieren?}

  Zentrale Anlaufstelle ist das Trennmuster-Wiki, welches sich
  momentan jedoch noch im Aufbau befindet.%
  \footnote{\url{http://projekte.dante.de/Trennmuster}}
%
  Für Fragen und Hinweise kann nach Anmeldung die Mailingliste%
%
  \footnote{\href{mailto:trennmuster@dante.de}{trennmuster@dante.de}}
%
  genutzt werden.

  \begin{itemize}

  \item Die Datei \Datei{CHANGES} enthält bekannte, systematische
    Fehler der Trennmuster.

  \item Das
    Entwicklerrepositorium%
%
    \footnote{\url{https://repo.or.cz/wortliste.git/tree/HEAD:/dokumente}}
%
    enthält im Verzeichnis \Datei{dokumente} eine Reihe von Dateien, welche
    u.\,a.  Trennregeln und Trennstile im Detail beschreibt.

  \end{itemize}
\end{frageantwort}


\begin{frageantwort}{Wie kann ich helfen?}

  Die deutschsprachige Trennmustermannschaft ist eine offene Gruppe
  und benötigt dringend weitere Mithilfe.  Interessenten sind daher
  hoch willkommen!  Zum Mitmachen gibt es mehrere Möglichkeiten:

  \begin{itemize}
  \item Die einfachste ist, die experimentellen Trennmuster ausgiebig
    zu testen und Fehler zu melden (siehe \autoref{sec:trennfehler}).

  \item Besonders hilfreich wäre Mitarbeit am Projekt.  Als Einstieg
    kann die (grobe) Aufgabenliste in der Projektbeschreibung dienen.

    Obwohl schon experimentelle Trennmuster veröffentlicht wurden,
    steht dieses Projekt noch ziemlich am Anfang.  Ziel ist, die
    verwendete Wortliste von möglichst vielen Menschen in verteilter
    Arbeit auf Richtigkeit zu prüfen.  Bis dahin bleibt jedoch noch
    viel zu tun.

  \item Mittelfristig -- nach Fertigstellung der Prüfmaske~-- kann
    auch durch systematische Durchsicht eines Teils der Wortliste
    geholfen werden.

  \item Fragen, Hinweise und Ideen sind auf der Mailingliste immer
    willkommen!
  \end{itemize}
\end{frageantwort}


\begin{frageantwort}{Sollten Trennfehler einzeln oder gesammelt
    eingereicht werden?}

  Das ist egal.  Es sollten allerdings die folgenden Hinweise beim
  Einreichen von Fehlern beachtet werden:

  \begin{description}
    \setkomafont{descriptionlabel}{\normalfont\itshape}

  \item[Aktualität] Wenn ältere Trennmusterdateien verwenden werden,
    bitte zuerst prüfen, ob der Fehler auch mit aktuellen Trennmustern
    auftritt.  Die aktuellen Trennmuster sind auf \Abk{ctan} im Paket
    \Paket{dehyph-exptl} oder im Dateibereich unter der
    Projekt-\Abk{url} erhältlich.  Im Repositorium befindet sich auch
    ein Makefile, mit dem jederzeit neue Trennmuster aus der aktuellen
    Wortliste erzeugt werden können.

  \item[Informationen] In einem Fehlerbericht sollte für das
    betreffende Wort die richtige und bei einzelnen Worteinreichungen
    auch die falsche Trennung angegeben werden.  Außerdem die genaue
    Version der verwendeten Trennmuster (traditionelle oder
    reformierte Rechtschreibung, Datum der Trennmusterdatei).

  \item[Listenformat] Es erleichtert die Korrektur, wenn
    Berichtigungsvorschläge in Form einer Liste eingereicht werden,
    die automatisch mit Skripten bearbeitet werden kann.  Die
    folgenden Konventionen sollten dabei eingehalten werden:

    \begin{itemize}
    \item Die Spalten werden mit einem Semikolon \verb+;+ getrennt.
    \item Die erste Spalte enthält das betreffende Wort in
      ungetrennter Form.
    \item Die zweite Spalte enthält das Wort in der Trennvariante nach
      traditioneller Rechtschreibung.
    \item Falls die Trennung nach reformierter Rechtschreibung davon
      abweicht, steht diese in der dritten Spalte.
    \item Ab der vierten Spalte \emph{können} weitere Trennvarianten
      folgen, etwa die falsche Trennung.  Eine kurze Erklärung sollte
      dann darauf eingehen.
    \item Unerwünschte Trennungen werden mit einem Punkt \verb+.+
      markiert.
    \item Anfang und Ende der Liste sollten klar erkennbar sein.
    \end{itemize}

    Eine Liste könnte beispielhaft so aussehen:

\begin{lstlisting}[style=Text]
  sonnendurchfluteten;son-nen-durch-flu-te-ten
  Talentwässerung;Tal-ent.wäs-se-rung
  Fensterplatz;Fen-ster-platz;Fens-ter-platz
\end{lstlisting}

    Dieses Format ist auch für einzeln eingereichte Korrekturen
    sinnvoll.

  \item[Betreff] Der Betreff einer Fehlermeldung sollte aussagekräftig
    sein.  Daher sollten mehrere Trennfehler, sofern möglich,
    thematisch zusammengefasst werden.  Außerdem kann zur einfacheren
    Zuordnung das entsprechende Kriterium aus
    \autoref{sec:trennfehler} dem Betreff der E-Mail in eckigen
    Klammern vorangestellt werden, zum Beispiel:

\begin{lstlisting}[style=Text]
  Betreff: [A] sonnendurchfluteten
\end{lstlisting}

    \noindent für einen Trennfehler im Wortes
    \emph{sonnendurchfluteten}, der durch einen Fehler in der
    Wortliste hervorgerufen wird.
  \end{description}
\end{frageantwort}

\end{document}

%%% Local Variables:
%%% mode: latex
%%% TeX-engine: default
%%% TeX-PDF-mode: t
%%% TeX-master: t
%%% coding: utf-8
%%% End:
