% !TeX encoding = utf-8
\documentclass[11pt,a4paper]{article}

\usepackage[utf8]{inputenc}
\usepackage[T1]{fontenc}
\usepackage{times}
\usepackage{hyperref}
\usepackage{french}
\author{Raymond Juillerat}
\begin{document}
\section*{eFrench Linux : mini-doc d'installation }
revue pour eFrench sous LPPL les 6 juin 2012 et 10  août 2015  par RayJ\\[3ex]

  Cette distribution de eFrench a été testée avec \TeX{}live mais elle doit 
fonctionner avec tous les autres moteurs \TeX. Cette distribution
est LPPL. 

  
\subsection*{Consignes d'installation}

 Placer le répertoire /french (et si nécessaire aussi /msg, voir plus bas) à un endroit où 
   tous les fichiers seront accessibles par{ \backslash}input.
\subsection*{Pour \TeX{}live, par exemple sous } 
   $\sim$/texmf/tex\hspace{1cm}
($\sim$ est le répertoire personnel)\\
ou

   $\sim$/texmf/tex/plain

   et annoncez à TeXLive (\$ texhash $\sim$/texmf)
   ce répertoire en tant que répertoire supplémentaire de sources .   
Cette configuration a été testée avec succès sous Ubuntu (9 à 11).

Les fichiers de /makeindex vont dans le répertoire de même nom de  \TeX{}live
ou dans le répertoire de travail selon besoins.

En tant que superviseur on peut suivre les indications de \emph{FilesInTDS.txt} .

\paragraph*{Remarque}
Si pour une raison ou une autre (ancien \TeX{}live ou autre système), french.sty annonce que msg.sty  ou msg-msg.tex manque, 
des solutions sont indiquées dans Probl\_Msg.pdf du dossier install.
\subsection*{PS}
\begin{enumerate}
\item Les messages sont codés en ANSI (latin9) 
\item    Le fichier language.dat doit définir french en tant
	 que langue, ce qui est le cas généralement
\end{enumerate}

\subsection*{Documentation }

Les manuels sont fournis sous forme de fichier en format pdf : 
efrench.pdf et frenchle.pdf, ce dernier étant aussi valable
pour efrench, les explications y sont plus détaillées 
mais concernent moins de possibilités. Et frenchle 
fait partie maintenant d'e-french.


\centering{Bonne utilisation de eFrench !}

  -- \href{mailto:raymond@juil.ch}{rayj }
pour l'équipe d'eFrench, pour toute question passer par son initiateur,
{Laurent Bloch}
\href{https://www.laurentbloch.net/MySpip3/e-french-desormais-dans-les-depots?lang=fr}{e-french désormais dans les dépôts TeXLive et MiKTeX} .

\end{document}
