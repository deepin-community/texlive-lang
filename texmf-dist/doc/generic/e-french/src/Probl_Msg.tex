% !TeX encoding = utf-8
\documentclass[11pt, a4paper]{article}

\usepackage[utf8]{inputenc}
\usepackage[T1]{fontenc}
\usepackage{times}
\usepackage{hyperref}
\usepackage{french}
\author{Raymond Juillerat}
\begin{document}
\section*{Si french annonce que msg.sty ou msg-msg.tex est absent }

Après l’installation, 
vous voulez compiler en utilisant french.sty et ce dernier s'achope à
l'absence de l'un des deux fichiers msg.sty ou msg-msg.tex. 

Avec les nouvelles version de TeXLive, il n'y a pas de problème de ce genre.
Avec MikTeX, version 2.9, seul msg-msg.tex manque.

Si msg.sty manque, alors généralement msg-msg.tex manque aussi.
C'est le cas avec d'autres environnements ou d'anciennes versions.

Une possibilité, c'est de les chercher sur CTAN.

Pour ce faire vous téléchargez de

\href{http://mirror.ctan.org/macros/latex/contrib/msg}
{http://mirror.ctan.org/macros/latex/contrib/msg/}\\
ou de \href{http://mirror.ctan.org/macros/latex/contrib/msg.zip}
{http://mirror.ctan.org/macros/latex/contrib/msg.zip}\\
au minimum les fichiers msg.ins, msg.dtx et msgfiles.dtx pour les
déposer dans un dossier de travail. Avec la commande 
\textit{latex msg.ins}, vous aurez les fichiers cherchés
 msg.sty et msg-msg.tex qui manquent  pour les mettre 
dans votre dossier french.

Une autre possibilité, c'est de les télécharger de\\
%La solution, c'est de les télécharger de\\
\href{http://svn.tuxfamily.org/viewvc.cgi/efrench\_efrenchsources/trunk/inputs/msg/}
{http://svn.tuxfamily.org/viewvc.cgi/efrench\_efrenchsources/trunk/inputs/msg/}\\


\begin{center}{Bonne utilisation de eFrench !}
\end{center}

  -- \href{mailto:raymond@juil.ch}{rayj }
pour l'équipe d'eFrench, pour toute question passer par son initiateur,
{Laurent Bloch}
\href{https://www.laurentbloch.net/MySpip3/e-french-desormais-dans-les-depots?lang=fr}{<https://www.laurentbloch.net/MySpip3/e-french-desormais-dans-les-depots?lang=fr>}

\end{document}

