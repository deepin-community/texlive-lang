% !TeX program =XeLatex
% This is frenT1lb.tex
%
%         French Torture Test with LaTeX (main style: book)
%                                    Copyright Bernard Gaulle as in french.doc
%
%                                                       last mods  2006/04/25
%
\advance\hoffset by -10pt% marge droite (chez moi) pas totalement imprimee
\documentclass[twocolumn,a4paper]{book}%  
%%%%%%%%%%%%%%%%%%%%%%%%%%%%%%%%%%%%%%%%%%%%%%%%%%%%%%%%%%%%%%%%%%%%%%%%%%%%%%
%%%% si vous n'avez pas utilise kbconfig, probablement avez-vous besoin de :
%\usepackage[T1]{fontenc}% pour utiliser les fontes a la norme Cork (DC,...)
%\usepackage[utf8]{inputenc}% pour le codage d'entree, ici utf8, XeLaTex s'en passe
\usepackage{eurosym} % pour euro
\usepackage{xltxtra}% pour le logo de XeTeX
\usepackage{fontspec}
\fontencoding{T1}
%\DeclareUnicodeCharacter{20AC}\euro de ceci aussi, XeLaTex s'en passe
%%%%%%%%%%%%%%%%%%%%%%%%%%%%%%%%%%%%%%%%%%%%%%%%%%%%%%%%%%%%%%%%%%%%%%%%%%%%%%
\usepackage{mysmall,mylist}
%\usepackage{epstopdf} %converting to PDF
%\usepackage{french}% Maintenant on peut utiliser cette syntaxe:
\usepackage[anglais,francais]{mlp}
\setlength{\textheight}{225mm}% a4paper is a little shorter...than a4.sty
\advance\hoffset by 40pt% for better placement
\listfiles

\setlength{\textwidth}{415pt}%% 
\catcode`\@=11
\let\Stop\undefined\ifx\@@end\undefined\let\@@end\end\fi
% Localisation code for numbered \typeout: "\kbAissue% localise it."
% For debugging one can remove "msg" access, just uncomment
%\let\kbAissue\relax% this line.
   \ifx\kbAissue\undefined%
% Firstly we add the material to use the "msg" package for localization.
      \def\kb@issue#1#2{\kb@issue@[#1]#2\void}% The local \issuemsg macro.
                                      % which will call the real one;
                                      % #1 is the macro message required.
                                      % #2 is the message header + msg number
                                      %    such as "^^J -234-", just message 
                                      %    number (234) is kept. 
      \let\kbAissue\kb@issue%
      \def\kb@issue@[#1]#2-#3-#4\void{\issuemsg[#1]#3(french)}%
   \ifx\issuemsg\undefined\let\kbAissue\relax\let\typeouA\typeout\fi%
   \fi%
%
\ifx\endfrench\switchtolanguage
\kbAissue% localise it.
                 \typeout{-37a- ANOMALIE : extension french active ici.}\Stop
\@@end
\fi
\def\LPLAIN{lplain}
\ifx\fmtname\LPLAIN\else\relax
                 \def\LPLAIN{LaTeX2e}
                 \ifx\fmtname\LPLAIN\else\relax
                 \typeout{**********************************************}
\kbAissue% localise it.
      \typeout{-36- ANOMALIE \string: nom de format LaTeX (\fmtname) invalide.}
      \typeouA{CORRIGER pour avoir \fmtname=\LPLAIN\space ou ancien nom.}
\Stop% Vous devez avoir dans ltvers.dtx : \def\fmtname{LaTeX2e} !
\@@end
                 \fi
\fi
\catcode`\@=12
\let\iflatex\iftrue
\def\SmsG{\typeout%
    {\string<\string<--------------------------------------------------------}}
\def\FmsG{\typeout%
    {--------------------------------------------------------\string>\string>}}
{\SmsG\obeyspaces
\typeout{(          Test de torture du style \frenchname}
\typeout{(Ce test genere deliberement quelques messages d'erreur. )}
\typeout{(Pour vous permettre de distinguer les messages d'erreur )}
\typeout{(normaux de ceux qui sont anormaux, j'ai encadre, comme  )}
\typeout{(ici, ces messages ; ne vous en preoccupez pas.  --bg    )}
\FmsG}%
\frenchtest
\ifx\endfrench\switchtolanguage\else
\kbAissue% localise it.
       \typeout{-37b- ANOMALIE \string: extension french inactive ici.}\Stop\fi
\end{document}
%%
%%      checksum        = "56133 79 312 3522"
%%
