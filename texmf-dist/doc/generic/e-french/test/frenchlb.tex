% !TeX encoding = latin9
% This is frenchlb.tex
%
%           French Torture Test with LaTeX (main style: book)
%                                    Copyright Bernard Gaulle as in french.doc
%
%                                           last mods 2006/04/25
%
\advance\hoffset by 20pt% pour centrer l'ensemble.
\documentclass[twocolumn,twoside,a4paper]{book}%  
%%%%%%%%%%%%%%%%%%%%%%%%%%%%%%%%%%%%%%%%%%%%%%%%%%%%%%%%%%%%%%%%%%%%%%%%%%%%%%
% Si vous n'avez pas utilis� kbconfig pour cr�er le format, 
% probablement avez-vous besoin de :
 \usepackage[latin9]{inputenc}% pour le codage d'entr�e, ici latin9
 \usepackage[T1]{fontenc}% pour utiliser les fontes ec (et cm en math)
% Ou alors \usepackage[decmulti]{inputenc}% pour le codage d'entr�e mais c'est
% vous qui l'aurez choisi... Moi je pr�f�re encore keyboard (cad kbconfig).
%%%%%%%%%%%%%%%%%%%%%%%%%%%%%%%%%%%%%%%%%%%%%%%%%%%%%%%%%%%%%%%%%%%%%%%%%%%%%%
\usepackage{mysmall,graphicx,mylist}
\usepackage{eurosym} %for the � 
\usepackage{epstopdf}% so graphicx accepts eps images
%\usepackage[tracefiles]{msg}% Only usefull for tracing purpose.
% Plut\^ot que \usepackage{french} on fait appel � mlp pour charger french :
\usepackage[anglais,francais]{mlp}
\setlength{\textheight}{225mm}% a4paper is a little shorter...than a4.sty
\setlength{\textwidth}{410pt}%% 
\listfiles

\catcode`\@=11
\let\Stop\undefined\ifx\@@end\undefined\let\@@end\end\fi
% Localisation code for numbered \typeout: "\kbAissue% localise it."
% For debugging one can remove "msg" access, just uncomment
%\let\kbAissue\relax% this line.
   \ifx\kbAissue\undefined%
% Firstly we add the material to use the "msg" package for localization.
      \def\kb@issue#1#2{\kb@issue@[#1]#2\void}% The local \issuemsg macro.
                                      % which will call the real one;
                                      % #1 is the macro message required.
                                      % #2 is the message header + msg number
                                      %    such as "^^J -234-", just message 
                                      %    number (234) is kept. 
      \let\kbAissue\kb@issue%
      \def\kb@issue@[#1]#2-#3-#4\void{\issuemsg[#1]#3(french)}%
   \ifx\issuemsg\undefined\let\kbAissue\relax\let\typeouA\typeout\fi%
   \fi%
%
\ifx\endfrench\switchtolanguage
\kbAissue% localise it.
                 \typeout{-37a- ANOMALIE : extension french active ici.}\Stop
\@@end
\fi
\def\LPLAIN{lplain}
\ifx\fmtname\LPLAIN\else\relax
                 \def\LPLAIN{LaTeX2e}
                 \ifx\fmtname\LPLAIN\else\relax
                 \typeout{**********************************************}
\kbAissue% localise it.
     \typeout{-36- ANOMALIE \string: nom de format LaTeX (\fmtname) invalide.}
     \typeouA{-36b- CORRIGER pour avoir \fmtname=\LPLAIN\space ou ancien nom.}
\Stop% Vous devez avoir dans ltvers.dtx : \def\fmtname{LaTeX2e} !
\@@end
                 \fi
\fi
\catcode`\@=12
\let\iflatex\iftrue
\def\SmsG{\typeout%
    {\string<\string<--------------------------------------------------------}}
\def\FmsG{\typeout{}\typeout%
    {--------------------------------------------------------\string>\string>}}
{\SmsG\obeyspaces
\typeout{(          Test de torture du style \frenchname}
\typeout{(Ce test genere deliberement quelques messages d'erreur. )}
\typeout{(Pour vous permettre de distinguer les messages d'erreur )}
\typeout{(normaux de ceux qui sont anormaux, j'ai encadre, comme  )}
\typeout{(ici, ces messages ; ne vous en preoccupez pas.  --bg    )}
\FmsG}%
\frenchtest
\ifx\endfrench\switchtolanguage\else
\kbAissue% localise it.
      \typeout{-37b- ANOMALIE \string: extension french inactive ici.}\Stop\fi%
\end{document}
%%
%%      checksum        = "10201 81 331 3632"
%%
