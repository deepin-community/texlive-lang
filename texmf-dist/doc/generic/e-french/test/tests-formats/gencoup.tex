%OZT%&frtex / This is a TeX code to generate a log file 
%         showing french hyphenated words.
% gencoup.tex                     Copyright Bernard Gaulle as in french_doc.pdf
%%
% This code was developped for MlTeX and TeX V3                   --bg 03/27/92
% with CM or EC fonts (but don't use it with TeX V2)
%                                                    last mods: --bg 2000/03/09
%
%%
%%      checksum        = "41383 90 338 4320"
%%
%
%For testing purposes ........................................  \CheckSevenBits
\ifx\STOP\undefined\def\STOP{\csname bye\endcsname}\fi%
\catcode`@=11
\def\CheckSevenBits/#1{\def\@tempa##1##2/##3{\ifx##2\empty\else%
 \message{gencoup.tex 
          -51- ERREUR : ce document n'a pas \'et\'e converti en 8-bits...}%
  \expandafter ##3\fi}\expandafter\@tempa\noexpand#1}%
\catcode`@=12
%
\ifx\french\undefined
           \ifx\l@french\undefined
             \message{Erreur : le langage french est inconnu !}
             \ifx\fhyph\undefined\else\message{Heureusement fhyph est connu !}
                                      \def\french{\fhyph}\fi
           \else\def\french{\language=\l@french}\fi% for babel
\fi
\french% set the language
%\message{Language=\the\language\space 
%         Left=\the\lefthyphenmin\space Right=\the\righthyphenmin}
\uchyph=-1% don't hyphenate upper case words
%\ifx\stophyph\undefined\else% was just for a confirmation test
%%                       \message{\message{\string\stophyph\space a\space 
%%                                         ete\space mis\space a\space 4}}
%             \stophyph=4\fi%pb with end of accent.words (MlTeX2)

%\hyphenchar\font='177 % for tests of DC v1.2, EC 
\edef\LeHC{\the\hyphenchar\font}\edef\DConePOINTtwo{127}%
% Here are Included LaTeX compatible definitions.
\catcode`@=11
\ifx\DOinput\undefined\let\DOinput\input\fi%
\gdef\newwrite{\alloc@7\write\chardef\sixt@@n}% not an outer def
\newwrite\@unused%..............................................\@unused
\message{()}
\long\gdef\@ifundefined#1#2#3{\expandafter\ifx\csname%......... \@ifundefined 
  #1\endcsname\relax#2\else#3\fi}%
\ifx\protect\undefined\let\protect\relax\fi%....................\protect
\def\IeC#1{\protect #1}% for plain .............................\IeC
\@ifundefined{typeout}{%........................................\typeout
  \global\newlinechar`\^^J%
  \gdef\typeout#1{{\let\protect\string\immediate\write\@unused{#1}}}%
                      }{}%
\typeout{Le caract\`ere de c\'esure (hyphen char) est \string'\LeHC}
% let the original showhyphen work, otherwise there is a mess.
\ifx\s@owhyphens\undefined\else\let\showhyphens\s@owhyphens\fi%
{\catcode`\]=13\def]{}\catcode`,=\active\def,{\egroup\space%no need of ^^J
                                              \showhyphens\bgroup%
                                              }%
\def\DmyDeF#1{\let\protect\relax% for 2e
              \showhyphens\bgroup#1}%
\def\Dmydef#1,#2]{\DmyDeF#1%
                  \def\temp{#2}\ifx\temp\empty\let\Dmydef\relax\fi%
                  \Dmydef#2]}%
\def\dummydef[{\Dmydef}%
\catcode`-=\active\def-{}%
% for this font:
\rm% use lgencoup.tex with LaTeX
\let\protect\noexpand% specially for NFSS
%
% Here are all the 499 french dictionnary names used by J. Desarmenien
% plus n other words that were inapropriately hyphenated in the past.
\edef\mots{\DOinput TSIv5n4  %
\def\writeEND{\message{\noexpand\endinput}}%
\typeout{^^J ************************************************************}
\typeout{^^J G\'en\'eration des cesures 8-bits (jeu test de 500 mots env.}
\typeout{^^J ************************************************************}
\dummydef[\mots\writeEND,]}}%
\typeout{^^J ************************************************************}
\typeout{^^J La generation de cesures sur un jeu test de plus de 500 mots}
\typeout{^^J \space\space\space\space\space\space\space\space\space E S T
             \space\space\space\space F A I T E}
\typeout{^^J Renommer maintenant le fichier \jobname.log en myhyph.log}
\typeout{^^J Valider l'accentuation dans ce fichier (circonflexes)}
\typeout{^^J Puis faire le test de validite (cf README.initex)}
\typeout{^^J ************************************************************}
\typeout{^^J }
\bye
\endinput
%%%%%%%%%%%%%%%%%%%%%%%%%%%%%%%%%%%%%%%%%%%%%%%%%%%%%%%%%%%%%%%%%%%%%%%%%%%%%%
