%%%%%%%%%%%%%%%%%%%%%%%%%%%%%%%%%%%%%%%%%%%%%%%%%%%%%%%%%%%%%%%%%%%%%%
% arabdoc0.tex %%%%%%%%%%%%%%%%%%%%%%%%%%%%%%%%%%%%%%%%%%%%%%%%%%%%%%%%
% \ArabTeX
% a System for Typesetting Arabic
% User Manual Version 2.05
% Report 1992/06
% User Manual Version 3.00
% Report 1993/11
% User Manual Version 4.00
% Report 1998/09
% Last changes: 03.03.2004
% Klaus Lagally
%%%%%%%%%%%%%%%%%%%%%%%%%%%%%%%%%%%%%%%%%%%%%%%%%%%%%%%%%%%%%%%%%%%%%%
%
% This file defines the structure of the paper
%
%%%%%%%%%%%%%%%%%%%%%%%%%%%%%%%%%%%%%%%%%%%%%%%%%%%%%%%%%%%%%%%%%%%%%%

\setcounter{page}{1}

%\def \abstractname {Overview}

\begin{abstract}
\ArabTeX\ is a package extending the capabilities of \TeX/\LaTeX\
to generate the Perso-Arabic writing from an ASCII transliteration
for texts in several languages using the Arabic script.
It consists of a \TeX\ macro package and an Arabic font in several sizes,
presently only available in the Naskhi style.
\ArabTeX\ will run with Plain \TeX\ and also with \LaTeXe.
It is compatible with Babel, CJK, LyX, the EDMAC package,
and Pic\TeX\ (with some restrictions);
other additions to \TeX\ may work, but have not been tried.

\ArabTeX\ is primarily intended for generating the Arabic writing,
but the standard scientific transliteration can also be easily produced.
For languages other than Arabic that are customarily written in
extensions of the Perso-Arabic script, some limited support is available.

\ArabTeX\ defines its own input notation which is both machine, and
human, readable, and suited for electronic transmission and
E-Mail communication.
However, texts in many of the Arabic standard encodings
can also be processed.
%
%Starting with Version 3.02, 
\ArabTeX\ also provides support for fully
vowelized Hebrew, both in its private ASCII input notation and in several
other popular encodings.

\ArabTeX\ is copyrighted by the author, but free. 
If you use the system for scientific work 
please give appropriate credit to the software
and the author (e.g. in the colophon of a monograph.) 
We also	appreciate a complimentary copy 
of any scientific work produced with \ArabTeX.

\ArabTeX\ may be redistributed and/or modified under the terms
of the LPPL (LaTeX Project Public License) distributed 
from CTAN archives in the directory 
\texttt{ftp://ftp.dante.de/tex-archive/macros/latex/base/lppl.txt},
either version 1 of the License, 
or (at your option) any later version.
%
There is no warranty of any kind, either expressed or implied.
The entire risk as to the quality and performance
rests with the user.
                                                  \index{copyright}

{\bf Please send error reports, suggestions and inquiries to the author:}

\begin{quote}
Prof. Klaus Lagally \\
Universit\"at Stuttgart\\
Institut f\"ur Formale Methoden der Informatik\\
Universit\"atsstra{\ss}e 38\\
70569 Stuttgart \\
GERMANY

{\tt lagally@informatik.uni-stuttgart.de}

%\vskip 1cm
%\vfill

Copyright \copyright\ 1992--2004, Klaus Lagally
\end{quote}

%\vskip 1cm
%\vfill

\emph{Note}: 
This manual describes version 4.00 of \ArabTeX.
The current version 3.11 is supposed not
to change except for error corrections.
\end{abstract}

\setcounter{page}{2}

\tableofcontents
\listoffigures
\listoftables

%%%%%%%%%%%%%%%%%%%%%%%%%%%%%%%%%%%%%%%%%%%%%%%%%%%%%%%%%%%%%%%%%%%%%%%

\input arabdoc1 % activating ArabTeX

\input arabdoc2 % input to ArabTeX
 
\input arabdoc3 % language selection

\input arabdoc4 % input coding conventions

\input arabdoc5 % transliteration

\input arabdoc6 % support for other languages

\input arabdoc7 % Hebrew mode

\input arabdoc8 % miscellaneous features

\input arabdoc9 % acknowledgments, references

%\input arabdocp % papers

%%%%%%%%%%%%%%%%%%%%%%%%%%%%%%%%%%%%%%%%%%%%%%%%%%%%%%%%%%%%%%%%%%%%%%
\appendix
%%%%%%%%%%%%%%%%%%%%%%%%%%%%%%%%%%%%%%%%%%%%%%%%%%%%%%%%%%%%%%%%%%%%%%%

\input arabdoca % obtaining ArabTeX

\input arabdocb % change history

\input arabdocc % miscellaneous utilities

\endinput
%%%%%%%%%%%%%%%%%%%%%%%%%%%%%%%%%%%%%%%%%%%%%%%%%%%%%%%%%%%%%%%%%%%%%%%
