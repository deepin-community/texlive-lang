% Preface to Asymptote overview by Jim Hefferon
\chapter*{Preface}
\Asy{} is a language for drawing mathematical figures. 
It outputs vector graphics and fits well with \TeX, \LaTeX{}, and friends.
It is great with two-dimensional graphics 
but it also sparkles with three dimensions,
including that it 
extends to 3D algorithms from \MF{} and \MP{} that are elegant and that
give beautiful curves. 

Last year I made a careful set of lecture slides for
Calculus~I and~II, and 
I drew the figures with \Asy.
It is wonderfully well-suited to the job.

However, recently I mentioned this success to someone, who later told
me that 
they had an online search for \Asy{} and found a technical reference, 
a long tutorial, and a gallery with many graphics, 
but really no quick overview.
Rather than tackle what was there, 
they stuck with what they were using.

Hence this document.
It is short, adopting a few familiar Calculus graphics.
It gives a feel for what you can do with \Asy, without being
too much 
(although I believe that the examples use every feature of 
\Asy{} that I used for my lectures).
You can work through it in an afternoon.
If you do elect to try \Asy{} then 
\href{https://asymptote.sourceforge.io/}{its web site} 
has many more advanced resources.

I hope that it is a help. 

\medskip\noindent
\textit{Remark:} This is an introduction.
Often I will do something one way, showing one option, when there
are many ways to do it.
For instance, you can include \Asy{} source in your \LaTeX{} document 
but here I have the source as standalone files.
Another example is that I will take the \Asy{} files to be in a 
\path{asy/} subdirectory,
while of course you can organize your work in many different ways.
Showing only one option is just a question of going shorter.

\medskip
\begin{flushleft}
Jim Hef{}feron  \\
University of Vermont \\
2024-Sep-29
\end{flushleft}