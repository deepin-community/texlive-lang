% !TEX TS-program = pdflatex
% !TEX encoding = UTF-8 Unicode
% arara: pdflatex: { synctex: true }
%% biblatex-cheatsheet.tex
%% Copyright 2017 Clea F. Rees
%
% This work may be distributed and/or modified under the
% conditions of the LaTeX Project Public License, either version 1.3
% of this license or (at your option) any later version.
% The latest version of this license is in
%   http://www.latex-project.org/lppl.txt
% and version 1.3 or later is part of all distributions of LaTeX
% version 2005/12/01 or later.
%
% This work has the LPPL maintenance status `maintained'.
%
% The Current Maintainer of this work is Clea F. Rees.
%
% This work consists of the file biblatex-cheatsheet.tex.
%%
\pdfminorversion=7
\documentclass[a4paper,welsh,british,landscape]{article}
\usepackage{svn-multi}
\svnid{$Id: biblatex-cheatsheet.tex 6644 2017-06-24 01:14:08Z cfrees $}
\svnRegisterAuthor{cfrees}{Clea F. Rees}
\usepackage{babel}
\usepackage[utf8]{inputenc}
\usepackage[tt=lining]{cfr-lm}
\newlength\normalparindent
\setlength\normalparindent{\parindent}
\usepackage{enumitem,geometry,url,texnames,multicol,parskip,titling,,xcolor,array,ragged2e,tabularx,verbatim}
\usepackage[flushleft]{threeparttablex}
\usepackage{csquotes}
\usepackage{microtype}
\geometry{hmargin=10mm,vmargin=10mm}
\setlength{\columnseprule}{0pt}
\setlength\parskip{.75ex plus .5ex minus .25ex}
\definecolor{blueberry}{rgb}{0.000,0.000,1.000}
\usepackage{tikz}
\usetikzlibrary{tikzmark,decorations.pathreplacing}
\hyphenation{bib-la-tex}
\usepackage[%
  pdftex,
  colorlinks=true,
  extension=pdf,
  pdfpagelabels=true,
  bookmarks=true,
  bookmarksopen=false,
  bookmarksnumbered=true,
  pdfusetitle=true,
  pdfcreator={TeX},
  pdfproducer={pdfeTeX},
  urlcolor={blueberry}]{hyperref}
\usepackage{hypdestopt}


\newcommand*{\cls}[1]{\textsf{#1}}
\newcommand*{\pkg}[1]{\textsf{#1}}
\newcommand*{\cs}[1]{\texttt{\textbackslash#1}}
\newcommand*{\env}[1]{\texttt{#1}}
\newcommand*{\marg}[1]{\texttt{\{#1\}}}
\newcommand*{\oarg}[1]{\texttt{[#1]}}
\newcommand*{\parg}[1]{\texttt{(#1)}}
\newcommand*{\meta}[1]{\ensuremath{\langle}{\normalfont\emph{#1}}\ensuremath{\rangle}}
\newcommand*{\filename}[1]{\texttt{#1}}
\newcommand*{\narg}[1][1]{\texttt{\##1}}
\newcommand*\entry[1]{\texttt{@#1}}
\newcommand*\bkey[1]{\texttt{#1}}
\let\bkeyfamily\ttfamily

\renewcommand*{\url}[1]{\href{http://#1}{\texttt{#1}}}
\newcommand*{\email}[1]{\href{mailto:#1}{\texttt{#1}}}

\setcounter{secnumdepth}{0}
\makeatletter
\setlength\droptitle{-45\p@}
\pretitle{\begingroup\centering\Large\bfseries}
\posttitle{\par\endgroup}
\predate{\relax}
\postdate{\relax}
\preauthor{\@gobble}
\postauthor{}

\def\section{\@startsection{section}{1}%
  \z@{1.5\baselineskip\@plus\fill\pagebreak[3]}{\baselineskip \nopagebreak}%
  {\normalfont\large\bfseries}}%
\def\subsection{\@startsection{subsection}{2}%
  \z@{1\baselineskip\@plus\fill\pagebreak[3]}{\baselineskip \nopagebreak}%
  {\normalfont\bfseries}}

\newcommand\mverbatim@font{% modified from verbatim.sty
  \normalfont\tmstyle\hyphenchar\font\m@ne\@noligs}

\newenvironment{mverbatim}% modified from verbatim manual, t. 2
{\verbatim\mverbatim@font}%
{\endverbatim}

\makeatother

\newcolumntype{e}[2]{@{\hskip .25em#1=\hskip .25em}>{#2}l}
\newcolumntype{T}{e{\ttfamily}{\ttfamily}}
\newcolumntype{M}{e{\ttfamily}{}}
\newcolumntype{h}{@{}>{\itshape}l@{}}

\tikzset{
  marginal/.style={midway, rotate=90, inner ysep=5pt, font=\scshape\scriptsize, align=center},
  left marginal/.style={left, anchor=south, marginal},
  right marginal/.style={right, anchor=north, marginal},
}
\newcommand*\cysylltiad[3][]{%
  \draw [decoration={brace}, decorate] ([xshift=-.25em, yshift=-.1\baselineskip]#2) -- ([xshift=-.25em, yshift=.6\baselineskip]#3) node [left marginal] {#1}}

\title{Biblatex Cheat Sheet}
\author{Clea F. Rees}
\date{}
\pagestyle{empty}

\begin{document}
\pdfinfo{%
  /Author	(Clea F. Rees)
  /Title	(Biblatex Cheat Sheet)
  /Subject	(LaTeX)
  /Keywords	(Biblatex, Biber)}%
\footnotesize
\begin{multicols}{3}%
\maketitle\thispagestyle{empty}
For further details, explanations, hints, caveats, examples and alternatives to the \bkey{backend} Biber, see \href{http://mirrors.ctan.org/macros/latex/contrib/biblatex/doc/biblatex.pdf}{the Biblatex manual}.
For a list of \emph{contributed} styles and extensions, see \url{ctan.org/topic/biblatex}.

\section{Basic Setup}\label{sec:basic}
Compilation sequence: \verb|pdflatex| $\rightarrow$ \verb|biber| $\rightarrow$ \verb|pdflatex| ($\times 2$).
\tikzmark{doc0}

\hskip.75\normalparindent\tikzmark{doc1}\hskip.25\normalparindent%
\begin{minipage}[b]{.75\linewidth}%
\begin{verbatim}
\documentclass[<language option>]{<class>}
...
\usepackage[utf8]{inputenc}
\usepackage{babel,csquotes,xpatch}% recommended
\usepackage[backend=biber,<options>]{biblatex}
\addbibresource[<options>]{<first resource>}
\addbibresource[<options>]{<second resource>}
...
\begin{document}
  ...
  \printbibliography[<options>]
  ...
  \printbibliography[<options>]
  ...
\end{document}
\end{verbatim}
\end{minipage}\hfill\mbox{}
\begin{tikzpicture}[remember picture, overlay]
  \draw ({pic cs:doc1} |- {pic cs:doc0}) +(0,-\parskip) -- ({pic cs:doc1});
\end{tikzpicture}

\section{Common Package Options}\label{sec:options}

\begin{tabular}{@{}>{\bkeyfamily}lMl@{}}
  \tikzmark{s1opt}style	&	\meta{style} & style of bibliography and citations\\
  bibstyle	&	\meta{style} & bibliography style\\
  \tikzmark{s2opt}citestyle	&	\meta{style} & citation style\\
  \tikzmark{r1opt}refsection	&	\meta{division} & new \texttt{refsection} at document \texttt{division}\\
  \tikzmark{r2opt}refsegment	&	\meta{division} & new \texttt{refsegment} at document \texttt{division}\\
  \tikzmark{c1opt}autocite	&	\meta{style} & behaviour of \cs{autocite} etc.\\
  sortcites	&	\meta{boolean}	&	whether to sort multiple citations\\
  maxnames	&	\meta{integer}	&	truncate longer name lists\\
  \tikzmark{c2opt}minnames	&	\meta{integer}	&	no.\ of names in truncated name lists\\
  \tikzmark{b1opt}backref	&	\meta{boolean}	&	whether to print ‘back references’\\
  mincrossrefs	&	\meta{integer}	&	minimum number of cross references\\
  \tikzmark{b2opt}sorting	&	\meta{sort order}	&	bibliography sort order\\
  indexing	&	\meta{boolean}	&	whether to enable indexing support\\
\end{tabular}
\begin{tikzpicture}[overlay, remember picture]
  \foreach \i in {b,c,r,s}
  \cysylltiad{{pic cs:\i2opt}}{{pic cs:\i1opt}};
\end{tikzpicture}

\section{Sources of Bibliographical Data}\label{sec:sources}

\begin{tabular}{@{}ll@{}}
	\cs{addbibresource}\oarg{\meta{options}}\marg{\meta{resource}}	&	add to default resource list\\
	\cs{addglobalbib}\oarg{\meta{options}}\marg{\meta{resource}}	&	add to global resource list\\
\end{tabular}

\begin{tabular}{>{\bkeyfamily}lTll@{}}
  \multicolumn{4}{h}{Options:}\\[.5ex]
  location	&	local & local file & (default)\\
			&	remote	& HTTP/FTP	\\
  datatype	&	bibtex & \BibTeX{}  & (default)\\
			&	ris & RIS \\
			&	zoterordfxml & Zotero RDF/XML \\
			&	endnotexml & EndNote XML \\
\end{tabular}

\begin{tabular}{ll@{}}
  \multicolumn{2}{h}{\meta{resource} must be one of:}\\[.5ex]
  \filename{\meta{filename}.bib}	&	local database\\
  \texttt{http://.../}\meta{filename}\texttt{.bib}	&	remote\\
  \texttt{ftp://.../}\meta{filename}\texttt{.bib}	&	remote\\
\end{tabular}

\cs{bibliography}\marg{\meta{filename},\meta{filename},...} adds 1+ local \BibTeX{} files.

\section{Citations}\label{sec:cite}

\begin{threeparttable}
  \begin{tabular}{@{}*{3}{l}@{}}
	\multicolumn{3}{h}{Standard commands:}\\[.5ex]
	\tikzmark{sc1}\tnote{c,m}	&	\cs{cite}\oarg{\meta{pre}}\oarg{\meta{post}}\marg{\meta{key}}	&	bare\\
	\tnote{c,m}	&	\cs{parencite}\oarg{\meta{pre}}\oarg{\meta{post}}\marg{\meta{key}}	&	parenthetical\\
	\tnote{m}	&	\cs{footcite}\oarg{\meta{pre}}\oarg{\meta{post}}\marg{\meta{key}}	&	footnote (\cs{footnote})\\
	\tikzmark{sc2}\tnote{m}	&	\cs{footcitetext}\oarg{\meta{pre}}\oarg{\meta{post}}\marg{\meta{key}}	& footnote (\cs{footnotetext})\\[.5ex]
	\multicolumn{3}{h}{Common commands:}\\[.5ex]
	\tikzmark{cc1}\tnote{c,m}	&	\cs{textcite}\oarg{\meta{pre}}\oarg{\meta{post}}\marg{\meta{key}}	&	textual\\
	\tnote{c,m}	&	\cs{smartcite}\oarg{\meta{pre}}\oarg{\meta{post}}\marg{\meta{key}}	&	context-dependent\\
	\tnote{a}	&	\cs{cite*}\oarg{\meta{pre}}\oarg{\meta{post}}\marg{\meta{key}}	&	year/title only\\
	\tnote{a}	&	\cs{parencite*}\oarg{\meta{pre}}\oarg{\meta{post}}\marg{\meta{key}}	&	year/title only\\
	\tikzmark{cc2}\tnote{m,n}	&	\cs{supercite}\oarg{\meta{pre}}\oarg{\meta{post}}\marg{\meta{key}}	&	superscript\\[.5ex]
	\multicolumn{3}{h}{Style-independent commands:}\\[.5ex]
	\tikzmark{sic1}\tnote{c,m}	&	\cs{autocite}\oarg{\meta{pre}}\oarg{\meta{post}}\marg{\meta{key}}	&	style-dependent\\
	\tikzmark{sic2}\tnote{c,m}	&	\cs{autocite*}\oarg{\meta{pre}}\oarg{\meta{post}}\marg{\meta{key}}	&	style-dependent\\[.5ex]
	\multicolumn{3}{h}{Text commands:}\\[.5ex]
	\tikzmark{tc1}\tnote{c}	&	\cs{citeauthor}\oarg{\meta{pre}}\oarg{\meta{post}}\marg{\meta{key}}	&	author list\\
	\tnote{c}	&	\cs{citeauthor*}\oarg{\meta{pre}}\oarg{\meta{post}}\marg{\meta{key}}	& compressed author list\\
	&	\cs{citetitle}\oarg{\meta{pre}}\oarg{\meta{post}}\marg{\meta{key}}	&	(short) title\\
	&	\cs{citetitle*}\oarg{\meta{pre}}\oarg{\meta{post}}\marg{\meta{key}}	&	(full) title\\
	\tnote{s}&	\cs{citeyear}\oarg{\meta{pre}}\oarg{\meta{post}}\marg{\meta{key}}	&	year\\
	\tnote{s}&	\cs{citedate}\oarg{\meta{pre}}\oarg{\meta{post}}\marg{\meta{key}}	&	date\\
	\tikzmark{tc2}&	\cs{citeurl}\oarg{\meta{pre}}\oarg{\meta{post}}\marg{\meta{key}}	&	URL\\[.5ex]
	\multicolumn{3}{h}{Multi-volume commands:}\\[.5ex]
	\tikzmark{mv1}\tnote{c,m}&	\cs{volcite}\oarg{\meta{pre}}\marg{\meta{vol}}\oarg{\meta{page}}\marg{\meta{key}}	& cite by volume + page\\
	\tnote{c,m}&	\cs{pvolcite}\oarg{\meta{pre}}\marg{\meta{vol}}\oarg{\meta{page}}\marg{\meta{key}}	& parenthetical\\
	\tnote{c,m}&	\cs{fvolcite}\oarg{\meta{pre}}\marg{\meta{vol}}\oarg{\meta{page}}\marg{\meta{key}}	& footnote (\cs{footnote})\\
	&	\cs{ftvolcite}\oarg{\meta{pre}}\marg{\meta{vol}}\oarg{\meta{page}}\marg{\meta{key}}	& footnote (\cs{footnotetext})\\
	\tnote{c,m}&	\cs{svolcite}\oarg{\meta{pre}}\marg{\meta{vol}}\oarg{\meta{page}}\marg{\meta{key}}	& context-dependent\\
	\tnote{c,m}&	\cs{tvolcite}\oarg{\meta{pre}}\marg{\meta{vol}}\oarg{\meta{page}}\marg{\meta{key}}	& textual\\
	\tikzmark{mv2}\tnote{c,m}&	\cs{avolcite}\oarg{\meta{pre}}\marg{\meta{vol}}\oarg{\meta{page}}\marg{\meta{key}}	&  style-dependent\\[.5ex]
	\multicolumn{3}{h}{Standalone citation commands:}\\[.5ex]
	\tikzmark{spc1}&	\cs{fullcite}\oarg{\meta{pre}}\oarg{\meta{post}}\marg{\meta{key}}	&	full reference\\
	\tikzmark{spc2}&	\cs{footfullcite}\oarg{\meta{pre}}\oarg{\meta{post}}\marg{\meta{key}}	&	full reference in footnote\\[.5ex]
	\multicolumn{3}{h}{Inclusion in bibliography without citation:}\\[.5ex]
	\tikzmark{inc1}&	\cs{nocite}\marg{\meta{key}}\hfill\cs{nocite}\marg{*}\hfill\mbox{}	&	inclusion only \\
	\tnote{c}	&	\cs{notecite}\oarg{\meta{pre}}\oarg{\meta{post}}\marg{\meta{key}}	& with notes \\
	\tnote{c}	&	\cs{pnotecite}\oarg{\meta{pre}}\oarg{\meta{post}}\marg{\meta{key}}	&	with parenthetical notes \\
	\tikzmark{inc2}  &	\cs{fnotecite}\oarg{\meta{pre}}\oarg{\meta{post}}\marg{\meta{key}}	&	with footnote notes \\[.5ex]
  \end{tabular}
  \begin{tablenotes}
    \item[a] Author-year and author-title styles only.
    \item[c] Capitalised command(s) also provided.
	e.g.~\cs{Textcite}, \cs{Autocites}.
	\item[m] ‘Multicite’ command(s) available.\\	e.g.~\cs{cites}\parg{\meta{multipre}}\parg{\meta{multipost}}\oarg{\meta{pre}}\oarg{\meta{post}}\marg{\meta{key}}\oarg{\meta{pre}}\\\oarg{\meta{post}}\marg{\meta{key}}\dots.
    \item[n] Numerical styles only.
	\item[s] Starred version available to include extra year information.
  \end{tablenotes}
\end{threeparttable}
\begin{tikzpicture}[overlay, remember picture]
  \foreach \i in {sc,sic,spc,tc,cc,mv,inc}
  \cysylltiad{{pic cs:\i2}}{{pic cs:\i1}};
\end{tikzpicture}

\section{\BibTeX{} Databases}\label{sec:bib}

A \BibTeX{} database file is a plain text file with extension \textt{.bib}.
It consists of entries of the following form:\tikzmark{doc7}

\hskip .5\normalparindent\tikzmark{doc5}\hskip .25\normalparindent%
\begin{minipage}[t]{.325\linewidth}

\begin{verbatim}
	@<entrytype>{<key>,
	    <field> = <value>,
	    <field> = <value>,
	    <field> = <value>,
	    ...}
\end{verbatim}
\end{minipage}%
\hfill
\begin{minipage}[t]{.05\linewidth}
  e.g.
\end{minipage}%
\hskip .5\normalparindent\tikzmark{doc8}\hskip .25\normalparindent%
\begin{minipage}[t]{.525\linewidth}
\begin{mverbatim}
	@book{tolkien-hobbit,
	  author = {Tolkien, J. R. R.},
	  title  = {The Hobbit},
	  date   = {YYYY-MM-DD},
	  ...}
\end{mverbatim}
\end{minipage}
\begin{tikzpicture}[overlay, remember picture]
  \draw
	({pic cs:doc6}) ++(0,\baselineskip) coordinate (doc6)
	({pic cs:doc7}) ++(0,\parskip-\baselineskip) coordinate (doc7)
	({pic cs:doc5} |- doc7) -- ({pic cs:doc5} |- doc6)
	({pic cs:doc8} |- doc7) -- ({pic cs:doc8} |- doc6)
	;
\end{tikzpicture}

\tikzmark{doc6}\meta{entrytype} partially determines which fields are required and which optional.
\meta{key} is a unique identifier used in citation commands to reference the entry.
\meta{field} is the name of a database field and determines the expected format of \meta{value}.
\meta{value} is the value of the relevant \meta{field} for the entry.

\subsection{\BibTeX{} Database Entry Types}\label{subsec:entries}

\begin{tabular}{@{}ll@{}}
  \multicolumn{2}{h}{Material from journals, magazines \& newspapers:}\\[.5ex]
  \tikzmark{1j}\entry{article}	&	journal, magazine or newspaper article\\
  \entry{periodical}	&	whole issue of a periodical\\
  \tikzmark{2j}\entry{suppperiodical}	&	supplemental material in periodical\\[.5ex]
  \multicolumn{2}{h}{Material from single-authored or co-authored books:}\\[.5ex]
  \tikzmark{1b}\entry{inbook}	&	book part with own title\\
  \entry{suppbook}	&	supplemental material in book\\
  \entry{bookinbook}	&  originally published as standalone book\\
  \entry{book}	&	single-volume book by author(s) of whole\\
  \tikzmark{2b}\entry{mvbook}	&	multi-volume book\\[.5ex]
  \multicolumn{2}{h}{Material from edited anthologies:}\\[.5ex]
  \tikzmark{1c}\entry{incollection}	&	contribution to anthology\\
  \entry{suppcollection}	&	supplemental material in anthology\\
  \entry{collection}	&	single-volume edited anthology\\
  \tikzmark{2c}\entry{mvcollection}	&	multi-volume collection\\[.5ex]
  \multicolumn{2}{h}{Material from conference proceedings:}\\[.5ex]
  \tikzmark{1p}\entry{inproceedings}	&	article in conference proceedings\\
  \entry{proceedings}	&	single-volume conference proceedings\\
  \tikzmark{2p}\entry{mvproceedings}	&	multi-volume conference proceedings\\[.5ex]
  \multicolumn{2}{h}{Material from works of reference:}\\[.5ex]
  \tikzmark{1r}\entry{inreference}	&	article in a reference work\\
  \entry{reference}	&	single-volume work of reference\\
  \tikzmark{2r}\entry{mvreference}	&	multi-volume reference work\\[.5ex]
  \multicolumn{2}{h}{Material from technical \& institutional publications:}\\[.5ex]
  \tikzmark{1i}\entry{manual}	&	technical or other documentation\\
  \entry{report}	&	institutional report or white paper\\
  \entry{patent}	&	patent or patent request\\
  \tikzmark{2i}\entry{thesis}	&	work completed to fulfil degree requirement\\[.5ex]
  \multicolumn{2}{h}{Material from online, informal \& other sources:}\\[.5ex]
  \tikzmark{1o}\entry{online}	&	\emph{inherently} online source\\
  \entry{booklet}	&	informally published book\\
  \entry{unpublished}	&	work not formally published\\
  \tikzmark{2o}\entry{misc}	&	 last resort (check manual first!)\\[.5ex]
  \multicolumn{2}{h}{Special entries for database management:}\\[.5ex]
  \tikzmark{1x}\entry{set}	&	(static) entry ‘set’\\
  \tikzmark{2x}\entry{xdata}	&	data-container (cannot be cited)\\
\end{tabular}
\begin{tikzpicture}[overlay, remember picture]
  \foreach \i in {j,b,c,p,r,o,x,i}
  \cysylltiad{{pic cs:2\i}}{{pic cs:1\i}};
\end{tikzpicture}

\subsection{\BibTeX{} Database Fields}\label{subsec:fields}

\begin{threeparttable}
  \begin{tabular}{@{}>{\bkeyfamily}ll@{}}
	\tikzmark{a1}author\tnote{s} &	author(s) of \bkey{title}, \bkey{authortype} specifies kind\\
	bookauthor	&	author(s) of \bkey{booktitle}\\
	editor\tnote{s}		& editor(s), \bkey{editortype} specifies role	\\
	editora/b/c	 &	secondary editor(s), \bkey{editora/b/ctype} for roles\\
	afterword		&	author(s) of afterword\\
	annotator	&	author(s) of annotations\\
	commentator	&	author(s) of commentary\\
	forward	&	author(s) of forward\\
	introduction	&	author(s) of introduction\\
	translator	&	translator(s) of \bkey{(book)title}\\
	\tikzmark{a2}holder	&	of patent\\
	\tikzmark{o1}institution	&	university or similar\\
	organization	&	manual/website publisher or event sponsor\\
	\tikzmark{o2}publisher\tnote{o}	& publisher(s)	\\
	\tikzmark{t1}title\tnote{a,o,s,u}	&	title\\
	indextitle	&	if different from \bkey{title}\\
	booktitle\tnote{a,u}	&	title of book\\
	maintitle\tnote{a,u}	&	title of multi-volume book\\
	journaltitle\tnote{u}	&	or \bkey{journal\tnote{s}}\\
	issuetitle\tnote{u}	&	title of journal special issue\\
	eventtitle\tnote{a}	&	title of conference or event\\
	reprinttitle	&	title of a reprint of the work\\
	\tikzmark{t2}series\tnote{s}	&	publication series\\
	\tikzmark{v1}volume	&	volume of journal or multi-volume book\\
	number	&	numbered issue of journal or book in series\\
	part	&	number of physical part of logical volume\\
	issue	&	non-number issue of journal\\
	volumes	&	number of volumes for multi-volume work\\
	edition	&	as \meta{integer} rather than ordinal\\
	version	&	revision number for software or manual\\
	\tikzmark{v2}pubstate	&	publication state\\
	\tikzmark{p1}pages	&	page list or range\\
	pagetotal	&	total number of pages\\
	\tikzmark{p2}(book)pagination	&	pagination format of \bkey{(book)title}\\
	\tikzmark{d1}date\tnote{o}		&	publication date as \meta{\textsc{yyyy-mm-dd}}\\
	eventdate	&	conference or event date as \meta{\textsc{yyyy-mm-dd}}\\
	\tikzmark{d2}urldate	&	access date for \bkey{url} as \meta{\textsc{yyyy-mm-dd}}\\
	\tikzmark{l1}location\tnote{o}	&	or \bkey{address}, where published\\
	\tikzmark{l2}venue	&	of event\\
	\tikzmark{e1}url		& URL\\
	doi		&	Digital Object Identifier\\
	eid		&	electronic identifier of \entry{article}\\
	eprint	&	archive-specific electronic identifier\\
	\tikzmark{e2}eprinttype	&	type of identifier, \bkey{eprintclass} for further details\\
	\tikzmark{y1}type	&	of \entry{manual}, \entry{patent}, \entry{report} or \entry{thesis}\\
	\tikzmark{y2}entrysubtype	&	for finer-grained specification of type\\
	\tikzmark{n1}addendum	&	miscellaneous data printed at end of entry\\
	note	&	miscellaneous data printed within entry\\
	\tikzmark{n2}howpublished	&	non-standard publication details\\
	language\tnote{o}	&	language of work\\
  \end{tabular}
  \begin{tablenotes}
	\item[a] An \bkey{--addon} field is available
	e.g.~\bkey{nameaddon}, \bkey{eventtitleaddon}.
	\item[o] An \bkey{orig--} field is available
	e.g.~\bkey{origdate}, \bkey{origlanguage}.
	\item[s] A \bkey{short--} field is available
	e.g.~\bkey{shortauthor}, \bkey{shortitle}.
	\item[u] A \bkey{--subtitle} field is available
	e.g.~\bkey{subtitle}, \bkey{mainsubtitle}.
  \end{tablenotes}
\end{threeparttable}

\begin{tabular}{@{}>{\bkeyfamily}ll@{}}
	\tikzmark{i1}isan	&	International Standard Audiovisual Number\\
	isbn	&	International Standard Book Number\\
	ismn	&	International Standard Music Number\\
	isrn	&	International Standard Technical Report Number\\
	issn	&	International Standard Serial Number\\
	\tikzmark{i2}iswc	&	International Standard Work Code\\
	abstract	&	record of work's abstract\\
	annotation	&	for annotated bibliographies\\
	file	&	local link\\
	library	&	library name, call number or similar\\
	\tikzmark{h1}label	&	fall-back label\\
	shorthand	&	special designator, overrides label in citations\\
	\tikzmark{h2}shorthandintro	&	override default introduction of \bkey{shorthand}\\[.5ex]
	\multicolumn{2}{h}{Special fields for non-printable data:}\\[.5ex]
	\tikzmark{s1}execute	&	arbitrary \TeX{} code\\
	keywords	&	separated list of keywords\\
	options	&	per-entry options\\
	ids	&	citation key aliases\\
	\tikzmark{r1}related	&	another entry key, \bkey{relatedoptions} for options\\
	relatedtype & relationship identifier for \bkey{related}\\
	\tikzmark{r2}relatedstring	&	override value of \bkey{relatedtype}\\
	\tikzmark{x1}entryset	&	list of entry keys in \entry{set}\\
	crossref	&	another entry key\\
	xref		&	another entry key\\
	\tikzmark{x2}xdata	&	entry key for \entry{xdata} container\\
	\tikzmark{b1}langid	&	\pkg{babel}/\pkg{polyglossia} language identifier\\
	langidopts	&	\pkg{polyglossia} options for \bkey{langid}\\
	\tikzmark{b2}gender	&	gender of \bkey{author} or \bkey{editor}\\
	\tikzmark{srt1}presort	&	modify sorting\\
	sortkey	&	sort key, overrides everything except \bkey{presort}\tikzmark{de}\\
	sortname	&	replaces \bkey{author} or \bkey{editor} when sorting\\
	sortshorthand	&	\bkey{sortkey} if entry has \bkey{shorthand}\\
	sorttitle	&	replaces \bkey{title} when sorting\\
	indexsorttitle	&	replaces \bkey{title} when sorting index\\
	\tikzmark{s2}\tikzmark{srt2}sortyear	&	replaces \bkey{year} (from \bkey{date}) when sorting\\
\end{tabular}
\begin{tikzpicture}[overlay, remember picture]
  \foreach \i/\j in {a/individuals,t/titles,x/inherit\\data,y/types,e/digital,n/misc.,o/orgs,d/dates,l/places,p/pages,v/volumes \& versions,r/related,h/labels,i/international\\standards,b/lang.,srt/sorting}
  \cysylltiad[\j]{{pic cs:\i2}}{{pic cs:\i1}};
  \draw [decoration={brace}, decorate] ([xshift=-2.5em, yshift=-.1\baselineskip]{pic cs:s2}) -- ([xshift=-2.5em, yshift=.6\baselineskip]{pic cs:s1});
\end{tikzpicture}

\section{Built-In Styles}\label{sec:styles}

\begin{threeparttable}
  \begin{tabularx}{\linewidth}{@{}>{\bkeyfamily}l>{\bkeyfamily}l>{\RaggedRight\arraybackslash}X@{}}
	\normalfont\bkeyfamily\bfseries citestyle	&	\normalfont\bkeyfamily\bfseries bibstyle & \\[.5ex]
	numeric\tnote{c,v}	& numeric	& numeric	\\
	alphabetic\tnote{v}	& alphabetic	&	alphabetic \\
	authoryear\tnote{c,ib,ic}	& authoryear	& author-year\\
	authortitle\tnote{c,ib,ic,t,tc,tic}	& authortitle	&	\\
	verbose\tnote{ib,in,n}	&	verbose	&	full reference on first citation\\
	verbose-trad1/2/3\tnote{tr}	&	&	‘traditional’ footnote citations\\
	reading\tnote{1}	&	reading	&	reading list\\
	draft	&	draft	&	show entry keys\\
	debug	&	debug	&	for debugging\\
  \end{tabularx}
  \begin{tablenotes}[para]
	\item[c] \bkey{-comp} option (compact).
	\item[ib] \bkey{-ibid} option (use \emph{ibidem}).
	\item[ic] \bkey{-icomp} option (compact \& \emph{ibidem}).
	\item[in] \bkey{-inote} option (notes \& \emph{ibidem}).
	\item[n] \bkey{-note} option (full citations as footnotes).
	\item[t] \bkey{-terse} option (omit title if unique).
	\item[tc] \bkey{-tcomp} option (compact \& terse).
	\item[tic] \bkey{-ticomp} option (compact, terse \& \emph{ibidem}).
	\item[tr] The three use different scholarly abbreviations in different ways.
	\item[v] \bkey{-verb} option (verbose).
	\item[1] Equivalent to \verb|citestyle=authortitle|.
  \end{tablenotes}
\end{threeparttable}

\section{Multiple, Divided \& Filtered Bibliographies}\label{sec:multi}

\begin{description}
  \item[Bibliography section] Document part with its own bibliography.
  \item[Bibliography segment] Document part corresponding to a sub-division of a global bibliography.
\end{description}

See package options \bkey{refsection} and \bkey{refsegment} for automated creation according to document division.
Finer-grained control is also possible:\tikzmark{doc2}

\hskip .75\normalparindent\tikzmark{doc4}\hskip .25\normalparindent
\begin{minipage}[b]{.55\linewidth}
  \begin{verbatim}
	\begin{refsection}
	    [<resource>,...]% replace default list
	    ...
	\end{refsection}
  \end{verbatim}
\end{minipage}%
\hskip .75\normalparindent\tikzmark{doc3}\hskip .25\normalparindent
\begin{minipage}[b]{.3\linewidth}
  \begin{verbatim}
	\begin{refsegment}
		...
	\end{refsegment}

  \end{verbatim}
\end{minipage}\hfill\mbox{}%
\begin{tikzpicture}[remember picture, overlay]
  \draw
	({pic cs:doc4}) ++(0,\baselineskip) coordinate (doc4)
	({pic cs:doc3}) ++(0,2\baselineskip) coordinate (doc3)
	(doc4 |- {pic cs:doc2}) +(0,-\parskip) -- (doc4)
	(doc3 |- {pic cs:doc2}) +(0,-\parskip) -- (doc3)
	;
\end{tikzpicture}


\begin{description}
  \item[Bibliography category] Topic or source type corresponding to a sub-division of a global bibliography.
\end{description}

\begin{tabular}{@{}ll@{}}
  \cs{DeclareBibliographyCategory}\marg{\meta{category}}	&	new category\\
  \cs{addtocategory}\marg{\meta{category}}\marg{\meta{key}}	&	add entry to category\\
\end{tabular}

\section{Printing Bibliographies}\label{sec:print}

\begin{tabularx}{\linewidth}{@{}l>{\RaggedRight\arraybackslash}X@{}}
  \cs{printbibliography}\oarg{\meta{options}}	&	typeset the bibliography\\
  \cs{printbiblist}\oarg{\meta{options}}\marg{\meta{name}}	&	typeset bibliography list \meta{name}\newline e.g.~\bkey{shorthand}\\
\end{tabularx}

\begin{threeparttable}
  \begin{tabularx}{\linewidth}{>{\bkeyfamily}lM>{\RaggedRight\arraybackslash}X@{}}
	\multicolumn{3}{h}{Options:}\\[.5ex]
	env	&	\meta{name}	& 	e.g.~\bkey{bibliography}\\
	heading	&	\meta{heading}	&	e.g.~\bkey{subbibliography}, \bkey{(sub)bibintoc}\\
	title		&	\meta{text}	&	\\
	prenote	&	\meta{name}\tikzmark{pn1}	&	define start/end notes with\\
	postnote	&	\meta{name}\tikzmark{pn2}	&	\cs{defbibnote}\marg{\meta{name}}\marg{\meta{text}}\\
	section	&	\meta{integer}	&	for \env{refsection} \meta{integer}\\
	segment	&	\meta{integer}	& 	for \env{refsegment} \meta{integer}\\
	category\tnote{n}	&	\meta{category}	&	only entries in \meta{category}\\
	keyword\tnote{n}	&	\meta{keyword}	&	only entries with \bkey{keyword} \meta{keyword}\\
	type\tnote{n}		&	\meta{entrytype}	&	only entries of type \meta{entrytype}\\
  \end{tabularx}
  \begin{tablenotes}
	\item[n] A negated filter is available as \bkey{not--}
	e.g.~\bkey{notcategory}\texttt{=}\meta{category}.
  \end{tablenotes}
\end{threeparttable}

\begin{tabular}{@{}>{\bkeyfamily}ll@{}}
  \cs{bibbysection}\oarg{\meta{options}}	&	all \bkey{refsection} bibliographies\\
  \cs{bibbysegment}\oarg{\meta{options}}	&	all \bkey{refsegment} bibliographies\\
  \cs{bibbycategory}\oarg{\meta{options}}	&	bibliographies for all categories\\
\end{tabular}

\begin{tikzpicture}[remember picture, overlay]
  \foreach \i in {1,2} \coordinate (pn\i) at ({pic cs:pn\i});
  \draw [decoration={brace}, decorate] ([xshift=2em, yshift=.6\baselineskip]pn1) -- ([xshift=2em, yshift=-.1\baselineskip]pn2);
\end{tikzpicture}

\section{Biber}\label{sec:biber}

\begin{minipage}{.5\linewidth}
  \begin{verbatim}
	biber [options] file[.bcf]
  \end{verbatim}
\end{minipage}%
\begin{minipage}{.5\linewidth}
  \begin{verbatim}
  	biber [options] --tool <datasource>
  \end{verbatim}
\end{minipage}

By default, Biber reads a \filename{.bcf} and produces a \filename{.bbl} which \LaTeX{} needs to produce a document's citations and bibliography.
But Biber also has a powerful ‘tool’ mode.
The manual explains the details but \verb|biber --help| is a more comprehensible starting point.

To produce a document-specific \filename{.bib}:
\begin{verbatim}
	biber --output_format=bibtex --output_resolve <filename>.bcf
\end{verbatim}
\mbox{}\smallskip

\hrule
\smallskip

{\scriptsize Copyright \copyright \svnyear{} \svnFullAuthor{\svnauthor} \email{ReesC21@cardiff.ac.uk} Rev.~\svnrev{} \svnyear--\svnmonth--\svnday{}\par}
\end{multicols}

\end{document}
