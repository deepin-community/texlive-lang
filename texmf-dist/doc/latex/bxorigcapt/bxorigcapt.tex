\documentclass[a4paper]{article}
\usepackage[scale=0.75]{geometry}
\usepackage[T1]{fontenc}
\usepackage{lmodern,textcomp}
\usepackage{color}
\definecolor{myblue}{rgb}{0,0,0.75}
\definecolor{mygreen}{rgb}{0,0.45,0}
\usepackage[colorlinks]{hyperref}
\hypersetup{linkcolor=myblue,urlcolor=mygreen,
  pdftitle={The bxorigcapt package},
  pdfauthor={Takayuki YATO}}
\usepackage{bxtexlogo}
\bxtexlogoimport{*}
\usepackage{shortvrb}
\MakeShortVerb{\|}
\newcommand{\PkgVersion}{1.0}
\newcommand{\PkgDate}{2022/04/10}
\newcommand{\Pkg}[1]{\textsf{#1}}
\newcommand{\Meta}[1]{$\langle$#1$\rangle$}
\newcommand{\Note}{\par\noindent\emph{Note.}}
\newcommand{\Means}{:\quad}
%-----------------------------------------------------------
\begin{document}
\title{The \Pkg{bxorigcapt} package}
\author{Takayuki YATO\quad (aka.~``ZR'')}
\date{v\PkgVersion\quad[\PkgDate]}
\maketitle
%\tableofcontents
\begin{abstract}
This package forces the caption names (|\chaptername|, |\today|, etc.)\ %
declared by the document class in use to be used as the caption names
for a specific language introduced by the Babel package.

Starting from version 0.3, this package also supports Polyglossia.
\end{abstract}

%===========================================================
\section{Introduction}
\label{sec:Introduction}

Suppose you have designed a document class
tailored for the Esperanto language.
The class has the following definition of caption names
and you like it:
\begin{quote}
|\newcommand\contentsname{Tabelo de Enhavo}|
\end{quote}

If a document is written solely in Esperanto,
then there is no need to employ the Babel package.
(Yes, the document class should select the hyphenation rule
for the language.)
However, when you want to create document
that contains Esperanto and German,
then you have to utilize Babel,
to have correct hyphenations for both languages.
\begin{quote}
|\usepackage[ngerman,esperanto]{babel}|
\end{quote}

But unfortunately, this changes |\contentsname|
from ``Tabelo de Enhavo'' (what you have chosen)
to ``Enhavo''
(what is declared in the language definition file of Babel),
which is unfavorable.

In fact, when using a document class for a specific language,
the most suitable caption names \emph{for that language}
should be the ones provided by the class.
The \Pkg{bxorigcapt} package realizes this natural request,
that is, it enables you
to make the caption names declared in the current document class
treated as the caption names for a specific language.


%===========================================================
\section{Package Loading}
\label{sec:Package-Loading}

\begin{quote}
|\usepackage[|\Meta{option}|,...]{bxorigcapt}|
\end{quote}

Available options are:
\begin{itemize}
\item |main| (default)\Means
  Sets the main language of Babel to the default target language.
\item \emph{a Babel language name}\Means
  Specifies the target language.
\item |warn|\Means
  Issues a warning (instead of an info)
  if Babel is never loaded in the preamble.
\item |nowarn| (default)\Means
  Negation of |warn|.
\end{itemize}


%===========================================================
\section{Usage}
\label{sec:Usage}

Once this package is loaded,
the caption names provided by the document class
(actually the ones that are effective when this package is loaded)
will be used as the caption names for the target language
(which is specified by the package option).

This package has no public commands.


%===========================================================
\end{document}
