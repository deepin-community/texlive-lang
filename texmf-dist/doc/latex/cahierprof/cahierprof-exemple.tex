\documentclass[a4paper,french,10pt]{article}
\usepackage[T1]{fontenc}
\usepackage[utf8]{inputenc}
\usepackage{babel}
\usepackage{lscape}
\usepackage{cahierprof}

%%%%%%%%%%%%%%%%%%%%%%%%%%%%%%%%%%%%%%%%%%%%%%%%%%%%%%%%%%%%%%%%%%%%%%%%%%%%%%%%
% Préambule 
%%%%%%%%%%%%%%%%%%%%%%%%%%%%%%%%%%%%%%%%%%%%%%%%%%%%%%%%%%%%%%%%%%%%%%%%%%%%%%%%

%%% Les marges de mon document.
\geometry{margin=1.2cm,head=0.6cm,headsep=10pt,foot=.6cm}

\title{
  cahierprof-exemple.tex\\
  Utilisation du cahier de textes en \LaTeX{} pour les professeurs
}
\author{
  Raphaël Giromini\\
  \texttt{raphael.giromini -- at -- gmail.com}
}
\date{Version 1.0 -- 1er septembre 2023}

% 1er lundi de l'année scolaire (pour le cahier de texte)
\setLundiRentree{4}{9}{2023}
% Début de chaque vacances scolaire pour l'année en cours (le samedi).
\setDebutToussaint{21}{10}
\setDebutNoel{23}{12}
\setDebutHiver{10}{2}
\setDebutPrintemps{6}{4}
% Les examens
% Le DNB a lieu les 24 et 25 juin.
% \setDNB{24}{25}{6}
% Les épreuves finales du bac ont lieu du 18 au 20 mars
\setBac{18}{20}{3}
% L'épreuve de philosophie a lieu le 11 juin
\setBacPhilo{11}{6}
% L'épreuve de français a lieu le 12 jun
\setBacFrancais{12}{6}
%Le Grand Oral a lieu du 17 au 28 juin
\setGO{17}{28}{6}

%%%%%%%%%%%%%%%%%%%%%%%%%%%%%%%%%%%%%%%%%%%%%%%%%%%%%%%%%%%%%%%%%%%%%%%%%%%%%%%%
% Début du document
%%%%%%%%%%%%%%%%%%%%%%%%%%%%%%%%%%%%%%%%%%%%%%%%%%%%%%%%%%%%%%%%%%%%%%%%%%%%%%%%
\begin{document}

\maketitle

\section{Calendrier des semaines de cours}

Exemple du cahier de texte à partir de l'exemple complet de
\texttt{cahierprof-doc.tex}

Le cahier de texte débute sur la page suivante au format paysage.

% Emploi du temps au format tableau (tabularx).
\begin{landscape}
  \EmploiDuTemps{
    %     & Lundi & Mardi & Merc.                  & Jeudi & Vend. \\
    8h    &       & 1G    &                        & 1G    &       \\ [1.5cm] 
    \cline{1-2}\cline{4-6}
    9h    &       &       & TSTMG                  & TSTMG &       \\ [1.5cm] \hline
    10h   & 1G    & 2nde  & 1G                     &       &       \\ [1.5cm] \hline
    11h   &       &       & \sem{2nde A.P.}{}      &       & 1G    \\ [1.5cm] 
    \hline\hline
    13h30 &       & 1G    &                        &       & 2nde  \\ [1.5cm] \cline{1-5}
    14h30 &       &       & 1G (Groupe~\sem{A}{B}) &       &       \\ [1.5cm] \hline
    15h30 &       &       & 2 euro                 &       &       \\ [1.5cm] \hline
    16h30 &       &       &                        &       &       \\ [1.5cm] \hline
  }
\end{landscape}

\section{Liste des élèves et tableau d'appel et de notes des classes}

\subsection{Liste des élèves}

Affiche la liste des élèves et la sauve dans le fichier par défaut
\texttt{ListeEleves.tex}.
\ListeEleves{}

\newpage

\subsection{Tableaux des classes}

% \Classe{Nom de la classe}{Nombre de séances dans la semaine}
% {Liste des élèves séparés par une virgule}

\Classe{Seconde}{4}
{
  Élève Seconde 1,
  Élève Seconde 2,
  Élève Seconde 3,
  Élève Seconde 4,
  Élève Seconde 5
}

\Classe{Première G}{6}
{
  Élève Première 1,
  Élève Première 2,
  Élève Première 3,
  Élève Première 4,
  Élève Première 5
}

\Classe{T STMG}{2}
{
  Élève STMG 1,
  Élève STMG 2,
  Élève STMG 3,
  Élève STMG 4,
  Élève STMG 5
}

\Classe{2 euro}{1}
{
  Élève Euro 1,
  Élève Euro 2,
  Élève Euro 3,
  Élève Euro 4,
  Élève Euro 5
}
\end{document}
