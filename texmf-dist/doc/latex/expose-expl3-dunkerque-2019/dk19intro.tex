% Time-stamp: <2020-03-21 12:02:52 administrateur>
% Création : 2019-08-24T14:27:10+0200
\section{Introduction}
\label{sec:introduction}

À l'occasion du stage de Dunkerque 2019, j'ai eu envie, pour présenter
\Expliii{}, de revenir sur un code donné par \textsc{Knuth} dans le
\TeX book et que Denis \textsc{Roegel} avait décortiqué dans
l'article \og Anatomie d’une macro\fg paru dans les \emph{Cahiers
  Gutenberg}, \no 31 (1998), p. 19-28.

Cette macro, bien évidemment écrite en \TeX{}, calcule les
\(n\)~nombres premiers pour \(n\)~entier supérieur à~\(3\).

J'ai décidé de réécrire une macro produisant le même résultat à l'aide
de \Expliii{} pour utiliser le code comme base de mon exposé. Une fois
lancé, j'ai écrit d'autres petits morceaux de code pour présenter
quelques fonctionalités du langage que le premier exemple ne m'avait
pas permis d'exposer. C'est tout cela que je vais présenter dans ce
qui suit.

%%% Local Variables:
%%% mode: latex
%%% TeX-master: "dun19expl3"
%%% End:
