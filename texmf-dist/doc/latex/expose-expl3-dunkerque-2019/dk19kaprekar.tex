% Time-stamp: <2019-09-22 16:32:18 administrateur>
% Création : 2019-08-26T12:37:33+0200

\section{Suites de \textsc{Kaprekar}}
\label{sec:suiteskaprekar}

Une suite de \textsc{Kaprekar} est définie par la donnée d'un nombre
entier, p.~ex. \(125\), d'une base \TO je prendrai systématiquement et
uniquement~\(10\) comme base de l'écriture des nombres entiers\TF et
d'un nombre~\(N\) de chiffres que j'appellerai la \emph{taille} du
nombre. On obtient un terme~\(u\sb{n+1}\) à partir du terme
précédent~\(u\sb{n}\) en appliquant l'algorithme suivant:
\begin{enumerate}\label{algokap}
\item on écrit \(u\sb{n}\) avec \(N\) chiffres dans la base choisie
  quitte à compléter avec des zéros; 
\item on range les \(N\) chiffres ainsi obtenus dans l'ordre
  décroissant pour obtenir \(v\sb{n}\); 
\item on range les \(N\) chiffres dans l'ordre croissant pour obtenir
  \(w\sb{n}\); 
\item enfin, \(u\sb{n+1} = v\sb{n} - w\sb{n}\).
\end{enumerate}

Une telle suite est évidemment finie. La commande que nous allons
détailler s'arrête sur le dernier nombre déjà obtenu.

En voici deux exemples.

On obtient \og \Kaprekar{125}\fg avec |\Kaprekar{125}| qui, par
défaut, travaille avec \(N=4\), et \og \Kaprekar[6]{125}\fg pour
lequel on a pris \(N=6\).

L'objet qui va nous servir à créer ces suites est la \gls{sequence}
\TO dont le sigle est \texttt{seq} \TF que fournit l'extension
\Pkg{l3seq}.


\section{Le code}
\label{sec:kaprekarcode}

Je rappelle que tout le code est placé entre \Macro{ExplSyntaxOn}
et \Macro{ExplSyntaxOff} pour les raisons déjà exposées.

La commande de document est \Macro{Kaprekar} dont voici le code.

\Fragment[Kaprekar, commande de
document]{73}{74}{./codes/kaprekar-def}

On y retrouve l'habituelle \Macro{NewDocumentCommand} mais un nouveau
descripteur d'argument, à savoir |O{4}|: le premier argument est
optionnel et sa valeur par défaut est~\(4\). Si on veut en fournir une
autre, on le placera entre crochet comme d'habitude.

La commande n'est, de fait et à part le truc de l'argument optionnel,
qu'un alias de \Macro{KAP_ecrire_tous_les_suivants+nn} que je vais
maintenant détailler. Mais commençons par la phase d'initialisation.

\subsection{Initialisation}
\label{sec:kaprinit}

\Fragment[Kaprekar, initialisation]{4}{6}{./codes/kaprekar-def}

Avec \Macro{seq_new+N} je crée deux \glspl{sequence}: \Macro{kaprun_seq}
et \Macro{kaprekar_seq}. La première \gls{variable} a vocation à
recevoir la suite des chiffres du nombre traité. La seconde contiendra
la suite de \textsc{Kaprekar}.

Je crée ensuite un \gls{entier} \emph{local} en suivant le modèle
de l'extension \Pkg{int}.

\subsection{De l'entier à la suite et retour}
\label{sec:entiersuiteretour}

J'ai besoin de commandes internes \TO qui pourraient être privées\TF
pour \emph{découper} un entier en la suite de ses chiffres et pour
créer un entier à partir d'une suite de chiffres. Le fait de
travailler en base fixée \TO et en base~\(10\) qui plus est\TF
facilite largement le travail. Là encore, si le cœur vous en dit,
exercice!

\Fragment[Kaprekar, de l'entier à la
suite]{9}{15}{./codes/kaprekar-def}

Le premier argument de \Macro{KAP_int_to_seq+NN} est le nombre à
découper, le second la suite qui contiendra ses chiffres. Dans les
deux cas, la commande attend une \gls{variable}.

Nous avons déjà rencontré la majorité des commandes utilisées dans la
définition. Avec \Macro{seq_put_left+NV} je place en première place
\TO à gauche \InEnglish{left}\TF la valeur contenue dans la
\gls{variable} \Macro{l_tmpb_int}. Le~|V| dans la signature assure que
la commande \Macro{l_tmpb_int} sera traitée de telle sorte que l'on
récupère bien son contenu.

Dans la boucle, on trouve \Macro{int_div_truncate+nn} qui permet de
récupérer le quotient entier de~|#1| par~|#2|.

\medbreak

\Fragment[Kaprekar, de la suite à
l'entier]{17}{22}{./codes/kaprekar-def}

Cette commande transforme une suite de chiffres en un \gls{entier}. On
trouve, en~\Ligne{18},  la commande \Macro{seq_pop_right+NNT} qui
prend la valeur la plus à droite dans la suite passée en 1\ier
argument pour la placer dans la \gls{variable} donnée en 2\ieme
argument. Si cette opération est possible \TO \CAD si la pile n'était
pas vide\TF le 3\ieme argument est développé.

La suite, après cette opération, est amputée de sa dernière
valeur. J'utilise ici la suite comme une pile \TO c'est le fameux
\textsc{lifo} pour \English{Last In First Out} \TF. \Expliii{} fournit
aussi des commandes qui permettent d'utiliser la suite comme une queue
\TO \textsc{fifo} pour \English{First In First Out}\TF.

Dans la boucle, j'utilise \Macro{seq_if_empty_p+N} qui prend la valeur
\VRAI{} si la \gls{sequence} est vide. Tant qu'elle n'est pas vide, je
multiplie par~\(10\) la valeur conservée dans la \gls{variable} donnée
en~|#2| et j'y ajoute la dernière valeur extraite de la pile.

De fait, le chiffre le plus à droite est celui de \emph{poids} le plus
grand. 

\medbreak

\Fragment[Kaprekar, remplissage]{24}{26}{./codes/kaprekar-def}

Cette commande assure que la suite contient le bon nombre de chiffres
en ajoutant des zéros si nécessaire. La seule nouveauté est
\Macro{seq_count+N} qui donne le nombre de valeurs contenues dans la
suite.

\medbreak

\Fragment[Kaprekar, tri]{28}{32}{./codes/kaprekar-def}

\Macro{seq_sort+Nn} trie la \gls{sequence} donnée \TO par son \og
nom\fg\TF en 1\ier argument à l'aide du code donnée en 2\ieme
argument. Ici, j'échange |##1| et |##2| si le premier est strictement
supérieur au second, autrement dit, je trie les éléments de la suite du
plus petit au plus grand. Les commandes \Macro{sort_return_swapped+}
\TO retourner en échangeant\TF et \Macro{sort_return_same+} \TO
retourner le même\TF assurent le tri.

\medbreak

\Fragment[Kaprekar, calcul du terme
suivant]{34}{44}{./codes/kaprekar-def} 

Le terme courant est donné par une \gls{variable} comme 2\ieme
argument, le 1\ier est la taille, le 3\ieme doit être la
\gls{variable} qui recevra le terme suivant.

\Macro{kaprun_seq} prend en~\Ligne{35} les chiffres du terme courant
puis on remplit puis on range dans l'ordre croissant.

Avec \Macro{seq_set_eq:NN}, en~\Ligne{39}, on crée, localement et sans
contrôle, \Macro{kaprdx_seq} comme copie de \Macro{kaprun_seq} puis,
en ligne suivante, on la range dans l'ordre décroissant.

Les trois dernières lignes traduisent l'algorithme de \textsc{Kaprekar}
tel qu'exposé en~\pageref{algokap}.

Il ne reste plus qu'à utiliser tout cela.

\subsection{Commande interne}

\Fragment[Kaprekar, commande interne]{46}{71}{./codes/kaprekar-def}

La commande \Macro{KAP_ecrire_tous_les_suivants+nn} prend la valeur
de~\(N\) \TO la \emph{taille}\TF en premier argument et le premier
terme de la suite comme 2\ieme argument.

Une fois de plus, tout se passe dans un groupe, ouvert en~\Ligne{47}
et fermé en~\Ligne{71}.

J'utilise la \gls{variable} booléenne locale \Macro{l_tmpa_bool}
fournie par \Expliii{}. Je lui donne la valeur \VRAI{}
en~\Ligne{48}. Quand elle prendra la valeur \FAUX{} il sera temps de
sortir de la boucle \TO lignes~\(54\) à~\(57\)\TF qui remplit la
\gls{sequence} \Macro{kaprekar_seq}.

Avec \Macro{seq_clear+N}, en~\Ligne{49}, je vide la suite
\Macro{kaprekar_seq}.

En \Ligne{51} je place le 1\ier terme de la suite dans la
\gls{sequence} \Macro{kaprekar_seq}. Cette fois je remplis par la
droite avec \Macro{seq_put_right+NV} \TO j'utilise la suite comme une
\emph{queue}\TF.

C'est parce que j'ai besoin que cette valeur soit placée dans la
\gls{variable} \Macro{l_tmpc_int} que j'utilise
\Macro{seq_put_right+NV}. S'il suffisait de placer la valeur
directement dans la suite, on pourrait utiliser
\Macro{seq_put_right+Nn} avec le code
\vspace{-\baselineskip}
\begin{Verbatim}[frame=lines,framesep=0.75\baselineskip]
\seq_put_right:Nn \kaprekar_seq {#2}
\end{Verbatim}

Je place dans \Macro{l_tmpc_int} le terme suivant calculé avec
\Macro{KAP_suivant+nNN} en~\Ligne{52} et si ce terme est déjà présent
dans la suite \TO cas ou \(u\sb{1} = u\sb{0}\)\TF je donne à
\Macro{l_tmpa_bool} la valeur \FAUX. C'est \Macro{seq_if_in+NVT} qui
permet de vérifier que la valeur~|#2| est présente dans la suite~|#1|
et de développer, si besoin le code placé dans~|#3|.

À la fin de la \Ligne{57} la suite a été créée.

J'ai laissé, commentée, la \Ligne{58} que j'ai utilisée à titre de
diagnostic: la commande \Macro{seq_show+N} permet d'écrire le
contenu de la \gls{sequence} dans le fichier~|.log|.

À la ligne suivante, la commande \Macro{seq_use+Nnnn} permet d'écrire
le contenu de la \gls{sequence}, donnée en 1\ier argument. Les trois
arguments suivants donnent, dans l'ordre, le séparateur du premier et
deuxième élément quand il n'y a que deux éléments; ce qui précède
chaque élément, sauf le dernier, quand il y plus de deux éléments;
enfin, ce qui sépare l'avant-dernier du dernier élément. Comme je veux
que les nombres soient écrits à l'aide de \Macro{np} je ne peux pas me
contenter de cette ligne d'où ce qui suit.

Après avoir placé dans un compteur le nombre d'éléments de la suite,
j'utilise \Macro{seq_map_inline+Nn} pour appliquer à chaque élément de
la suite passée en 1\ier argument le code contenu dans le 2\ieme
argument \TO \Ligne{62} à~\(70\)\TF.

Dans ce code, j'écris l'élément en en donnant la valeur à \Macro{np}
en mode mathématique en ligne puis je décrémente le compteur à l'aide
de \Macro{int_decr+N}.

Avec \Macro{int_case+nnF}, je sélectionne le séparateur à placer: si
le compteur vaut~\(1\) j'écris \og |~et~|\fg, s'il vaut~\(0\), je n'écris
rien \TO \Ligne{67}\TF et, dans tous les autres cas \TO \Ligne{69}\TF,
j'écris \og |~;~|\fg.

% \subsection{\texorpdfstring{\Kaprekar[2]{12}}{kaprekar 2-12}}

% \Fragment[Kaprekar]{4}{45}{./codes/kaprekar-def}


%%% Local Variables:
%%% mode: latex
%%% TeX-master: "dun19expl3"
%%% End:
