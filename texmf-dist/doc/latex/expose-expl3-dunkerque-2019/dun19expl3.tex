% Time-stamp: <2020-03-21 18:06:46 administrateur>
% Création : 2019-08-04T17:42:03+0200

\documentclass[french]{article}
\usepackage[T1]{fontenc}
\usepackage[a4paper]{geometry}
\usepackage{xspace}
\usepackage{array}
\usepackage{csquotes}
\usepackage[expert,oldstyle]{fourier}
\usepackage[CharPoCal=false,suite=false,taupe=false]{tdsfrmath}

\usepackage{fancybox}
\usepackage{etaremune}

\usepackage{fvextra}
\usepackage{xparse}
\usepackage{dun19codepres}

\usepackage[xindy]{imakeidx}
\makeindex

\usepackage[main=french]{babel}
\usepackage[np,autolanguage]{numprint}

\usepackage[hidelinks]{hyperref} % options à définir !

\usepackage[backend=biber]{biblatex}
\addbibresource{./bibliographie/dun19expl3.bib}

\usepackage[xindy,nogroupskip]{glossaries-extra}
\newglossary[klg]{cs}{kls}{klo}{Commandes, fonctions et variables}
\makeglossaries

\loadglsentries{./glosaires/termes-techniques}
\loadglsentries{./glosaires/les-macros}
\loadglsentries{./glosaires/erato-supp-glo}
\loadglsentries{./glosaires/kaprekar-sup-glo}

\title{%
  Quelques aspects de la programmation avec
  \Expliii{}\footnote{Version 1.2}}
\author{Yvon \textsc{Henel}}

\begin{document}
\newif\ifprime      \newif\ifunknown
\newcount\n         \newcount\p
\newcount\d         \newcount\a

\def \primes#1{2,~3\n=#1 \advance \n by-2  \p=5 
  \loop \ifnum \n>0 \printifprime \advance \p by2 \repeat}
  
\def \printp{, \ifnum \n=1 and~\fi 
  \number \p \advance \n by -1 }

\def \printifprime{\testprimality \ifprime\printp\fi}

\def \testprimality{{\d=3 \global \primetrue
   \loop \trialdivision \ifunknown \advance \d by2 \repeat}}
   
\def \trialdivision{\a=\p \divide \a by\d
  \ifnum \a>\d \unknowntrue \else \unknownfalse \fi
  \multiply \a by\d \ifnum \a=\p \global \primefalse \unknownfalse \fi}

\endinput
% Time-stamp: <2019-08-07 10:59:31 administrateur>
% Création : 2019-05-20T11:51:49+0200
\ExplSyntaxOn 

\cs_new:Nn \__yh_ecrire_nombre:n { \(\np{\int_to_arabic:n{ #1 }}\) }

\cs_new:Nn \yh_impair_est_premier:nT {
  \bool_set_true:N \yh_continue_bool
  \bool_set_true:N \yh_est_premier_bool
  \int_set:Nn \l_tmpa_int {1}
  % FIN INITIALISATION
  \bool_do_while:Nn \yh_continue_bool { % BOUCLE `TANTQUE'
    % INCRÉMENTATION 
    \int_add:Nn \l_tmpa_int { 2 }
    \bool_if:nTF {
    % TEST de divisibilité
      \int_compare_p:nNn { \int_mod:nn {#1} { \l_tmpa_int } } = { 0 }
    } { % SI VRAI
      \bool_set_false:N \yh_continue_bool
      \bool_gset_false:N \yh_est_premier_bool
    } { % SI FAUX
      \int_compare:nNnT { \l_tmpa_int * \l_tmpa_int } > { #1 } 
      { \bool_set_false:N \yh_continue_bool }
    } } % FIN DU SI ET FIN DE LA BOUCLE
  \bool_if:nT {\yh_est_premier_bool} {#2} }

\cs_new:Nn \yh_ecrire_separateur:n { } 

\cs_new:Nn \yh_ecrire_si_premier:n {
  \yh_impair_est_premier:nT { #1 }
  { \yh_ecrire_separateur:n { #1 }
    \__yh_ecrire_nombre:n { #1 } } }

\NewDocumentCommand { \ListePremiers } { m } {
  \group_begin:
  % INITIALISATION
  \int_new:N \l_compteur_int
  \int_new:N \l_nombre_requis
  \int_set:Nn \l_compteur_int { 2 }
  \int_set:Nn \l_nombre_requis { #1 }
  \cs_set:Nn \yh_ecrire_separateur:n {
    \int_compare:nNnTF { ##1 } < { \l_nombre_requis - 1} {,~} {~et~} }
  % LES DEUX PREMIERS NOMBRES PREMIERS NE SONT PAS CALCULÉS 
  \__yh_ecrire_nombre:n {2}
  \yh_ecrire_separateur:n {1}
  \__yh_ecrire_nombre:n {3} 
  % PRÉPARATION DE LA BOUCLE, \l_tmpb_int EST LE CANDIDAT À TESTER
  \int_set:Nn \l_tmpb_int { 3 }
  % BOUCLE `TANT QUE'
  \int_while_do:nNnn { \l_compteur_int } < { \l_nombre_requis } {
    \int_add:Nn \l_tmpb_int { 2 }
    \yh_impair_est_premier:nT { \l_tmpb_int }
    { \yh_ecrire_separateur:n { \l_compteur_int }
      \__yh_ecrire_nombre:n{ \l_tmpb_int }
      \int_incr:N \l_compteur_int } }
  % ON FAIT LE MÉNAGE AVANT DE SORTIR
  \cs_undefine:N  \l_compteur_int
  \cs_undefine:N  \l_nombre_requis
  \group_end: }

\ExplSyntaxOff

%%% Local Variables:
%%% mode: latex
%%% TeX-master: t
%%% End:

% Time-stamp: <2019-08-26 11:48:16 administrateur>
% ReCréation : 2019-08-23T10:40:31+0200
\ExplSyntaxOn

\NewDocumentCommand{ \eratosthene } { m } {
  \group_begin:
  \int_new:N \l_ERA_diviseur_max_int
  \int_set:Nn \l_ERA_diviseur_max_int { \fp_to_int:n { floor( sqrt ( #1 ))} }
  \intarray_new:Nn \__ERA_qt_ia { #1 }
  \intarray_new:Nn \__ERA_dv_ia { #1 }
  % commande auxiliaire --------------------------------------------------------
  \cs_set:Nn \ERA_marquer_de_a_par:nnn {
    \int_set:Nn \l_tmpa_int { ##1 / ##3}
    
    \int_step_inline:nnnn { ##1 } { ##3 } { ##2 } {
      \int_compare:nNnT { \intarray_item:Nn  \__ERA_dv_ia { ####1 } } = { 0 }
      { \intarray_gset:Nnn \__ERA_dv_ia  { ####1 } { ##3 }
        \intarray_gset:Nnn \__ERA_qt_ia  { ####1 } { \l_tmpa_int } }
      \int_incr:N \l_tmpa_int }
  }
  % ----------------------------------------------------------------------------
  % marquer les pairs
  \int_set:Nn \l_tmpa_int { 1 }

  \int_step_inline:nnnn { 2 } { 2 } { #1 }
  { \intarray_gset:Nnn \__ERA_dv_ia  { ##1 } { 2 }
    \intarray_gset:Nnn \__ERA_qt_ia  { ##1 } { \l_tmpa_int }
    \int_incr:N \l_tmpa_int }
  
  % marquer les impairs jusqu'à racine de #1
  \int_set:Nn \l_tmpb_int { 3 }

  \int_while_do:nNnn { \l_tmpb_int } < { \l_ERA_diviseur_max_int }
  { \ERA_marquer_de_a_par:nnn { \l_tmpb_int } { #1 } { \l_tmpb_int }
    \int_do_until:nNnn {\intarray_item:Nn  \__ERA_dv_ia { \l_tmpb_int }} = { 0 }
    { \int_add:Nn \l_tmpb_int { 2 } } }
  
  % marquer les derniers impairs
  \int_step_inline:nnnn { 3 } { 2 } { #1 }
  { \int_compare:nNnT { \intarray_item:Nn  \__ERA_dv_ia { ##1 } } = { 0 }
    { \intarray_gset:Nnn \__ERA_dv_ia  { ##1 } { ##1 }
      \intarray_gset:Nnn \__ERA_qt_ia  { ##1 } { 1 } } }
  
  % nettoyage
  \cs_undefine:N \l_ERA_diviseur_max_int
  \group_end: }

%%% -----------------------------------------------------------------------

\cs_new:Nn \ERA_presenter_nieme:n {
  \group_begin:
  \int_set:Nn \l_tmpa_int { \intarray_item:Nn \__ERA_qt_ia { #1 } }
  \int_set:Nn \l_tmpb_int { \int_mod:nn { #1 } { 10 } }
  
  \framebox[4em] {\strut \footnotesize
    \int_compare:nNnTF { \l_tmpa_int } = { 1 }
      {\textbf}           % premier
      {\textcolor{gray}}  % composé
    {\int_to_arabic:n { #1 }}
    \int_compare:nNnT { \l_tmpa_int } > { 1 }
    { \int_set:Nn \l_tmpa_int { \intarray_item:Nn \__ERA_dv_ia { #1 } }
      ~\({}^{\langle \int_to_arabic:n { \l_tmpa_int } \rangle}\) } }
  \kern-\fboxrule
  % saut de paragraphe tous les 10  
  \int_compare:nNnT { \l_tmpb_int } = { 0 }
  { \par \nointerlineskip \kern-\fboxrule \noindent }
  \group_end: }

\NewDocumentCommand { \EcrireCribleEratosthene } { m }{
  \par \noindent \int_step_inline:nn { #1 } { \ERA_presenter_nieme:n {##1} } }


\NewDocumentCommand { \EcrireDiviseurs } { s m } {
  \group_begin:
  \int_new:N \l_ERA_nv_qt_int         % quotient
  \int_set:Nn \l_ERA_nv_qt_int { #2 } 
  \int_new:N \l_ERA_vx_dv_int         % ancient diviseur
  \int_set:Nn \l_ERA_vx_dv_int { 0 }  
  \int_new:N \l_ERA_nv_dv_int         % nouveau diviseur
  \int_set:Nn \l_ERA_nv_dv_int { \intarray_item:Nn \__ERA_dv_ia { #2 } }

  \IfBooleanF{#1}{\par}
  \( \int_to_arabic:n { #2 } = 
  \int_while_do:nNnn { \l_ERA_nv_qt_int } > { 1 }
  { \int_compare:nNnT { \l_ERA_vx_dv_int } > { 0 } { \times }
    \int_to_arabic:n { \l_ERA_nv_dv_int }
    \int_set:Nn \l_ERA_vx_dv_int { \l_ERA_nv_dv_int }
    \int_set:Nn \l_ERA_nv_qt_int {
      \intarray_item:Nn \__ERA_qt_ia { \l_ERA_nv_qt_int } }
    \int_set:Nn \l_ERA_nv_dv_int {
      \intarray_item:Nn \__ERA_dv_ia { \l_ERA_nv_qt_int } } }
  \)
  \cs_undefine:N \l_ERA_nv_qt_int
  \cs_undefine:N \l_ERA_vx_dv_int
  \cs_undefine:N \l_ERA_nv_dv_int
  \group_end: }

\ExplSyntaxOff

%%% Local Variables:
%%% mode: latex
%%% TeX-master: t
%%% End:

% Time-stamp: <2019-08-28 11:48:38 administrateur>
% ReCréation : 2019-08-26T12:32:29+0200
\ExplSyntaxOn
\seq_new:N \kaprun_seq
\seq_new:N \kaprekar_seq
\int_new:N \l_tmpc_int


\cs_new:Nn \KAP_int_to_seq:NN { 
  \int_set:Nn \l_tmpa_int { #1 }
  \int_do_until:nNnn { \l_tmpa_int } = { 0 }
  { \int_set:Nn \l_tmpb_int { \int_mod:nn { \l_tmpa_int } { 10 } }
    \seq_put_left:NV #2 \l_tmpb_int 
    \int_set:Nn \l_tmpa_int
    { \int_div_truncate:nn { \l_tmpa_int - \l_tmpb_int } { 10 } } } }

\cs_new:Nn \KAP_seq_to_int:NN {
  \seq_pop_right:NNT #1 \l_tmpa_tl
  { \int_set:Nn #2 { \l_tmpa_tl }
    \bool_do_until:nn { \seq_if_empty_p:N #1 }
    { \seq_pop_right:NNT #1 \l_tmpa_tl
      \int_set:Nn #2 { 10 * #2 + \l_tmpa_tl } } } }

\cs_new:Nn \KAP_remplir_seq:Nn { 
  \int_while_do:nNnn { \seq_count:N #1 } < { #2 }
  { \seq_put_left:Nn #1 { 0 } } }

\cs_new:Nn \KAP_ranger_seq:N {
  \seq_sort:Nn #1
  { \int_compare:nNnTF { ##1 } > { ##2 }
    { \sort_return_swapped: }
    { \sort_return_same:    } } }

\cs_new:Nn \KAP_suivant:nNN {
  \KAP_int_to_seq:NN #2 \kaprun_seq 
  \KAP_remplir_seq:Nn \kaprun_seq { #1 }
  \KAP_ranger_seq:N \kaprun_seq
  
  \seq_set_eq:NN \kaprdx_seq \kaprun_seq
  \seq_reverse:N \kaprdx_seq

  \KAP_seq_to_int:NN \kaprdx_seq \l_tmpa_int
  \KAP_seq_to_int:NN \kaprun_seq \l_tmpb_int
  \int_set:Nn #3 { \l_tmpb_int - \l_tmpa_int } }

\cs_new:Nn \KAP_ecrire_tous_les_suivants:nn { 
  \group_begin:
  \bool_set_true:N \l_tmpa_bool 
  \seq_clear:N \kaprekar_seq
  \int_set:Nn \l_tmpc_int { #2 } 
  \seq_put_right:NV \kaprekar_seq \l_tmpc_int
  \KAP_suivant:nNN {#1} \l_tmpc_int \l_tmpc_int
  \seq_if_in:NVT \kaprekar_seq \l_tmpc_int {\bool_set_false:N \l_tmpa_bool}
  \bool_while_do:nn { \l_tmpa_bool }
  { \seq_put_right:NV \kaprekar_seq \l_tmpc_int
    \KAP_suivant:nNN {#1} \l_tmpc_int \l_tmpc_int
    \seq_if_in:NVT \kaprekar_seq \l_tmpc_int { \bool_set_false:N \l_tmpa_bool } }
  % \seq_show:N \kaprekar_seq
  % \seq_use:Nnnn \kaprekar_seq {~et~} {,~} {~et~}\par{}
  \int_set:Nn \l_tmpa_int { \seq_count:N \kaprekar_seq }
  \seq_map_inline:Nn \kaprekar_seq
  { \( \np{##1} \)
    \int_decr:N \l_tmpa_int
    \int_case:nnF { \l_tmpa_int }
    { 
      { 1 } {~et~}
      { 0 } { }
    }
    {~;~}
  }
  \group_end: }

\NewDocumentCommand{\Kaprekar}{ O{4} m }{
  \KAP_ecrire_tous_les_suivants:nn { #1 } { #2 } }

\ExplSyntaxOff
%%% Local Variables:
%%% coding: utf-8 
%%% mode: latex
%%% TeX-master: "../dun19expl3"
%%% End:


\makeunderlineletter
\DefineShortVerb{\|}
\maketitle{}

\begin{abstract}
  Quelques aspects de la programmation avec \Expliii{} présentés au
  travers de code calculant sur les entiers: écriture des \(n\)
  premiers entiers premiers; crible d'\textsc{Ératosthène}; suites de
  \textsc{Kaprekar}.
\end{abstract}

\vspace{4\baselineskip}

\begin{center}
  \shadowbox{%
    \begin{minipage}{0.8\linewidth}
      \centering \large \textbf{Remerciements}\par

      \vspace{\baselineskip}
      
      \normalsize
      Je remercie François \textsc{Pantigny} pour sa lecture attentive
      de la version~1.

      \vspace*{\baselineskip}
    \end{minipage}}
\end{center}


\newpage{}

\tableofcontents{}

\newpage{}

% Time-stamp: <2020-03-21 12:02:52 administrateur>
% Création : 2019-08-24T14:27:10+0200
\section{Introduction}
\label{sec:introduction}

À l'occasion du stage de Dunkerque 2019, j'ai eu envie, pour présenter
\Expliii{}, de revenir sur un code donné par \textsc{Knuth} dans le
\TeX book et que Denis \textsc{Roegel} avait décortiqué dans
l'article \og Anatomie d’une macro\fg paru dans les \emph{Cahiers
  Gutenberg}, \no 31 (1998), p. 19-28.

Cette macro, bien évidemment écrite en \TeX{}, calcule les
\(n\)~nombres premiers pour \(n\)~entier supérieur à~\(3\).

J'ai décidé de réécrire une macro produisant le même résultat à l'aide
de \Expliii{} pour utiliser le code comme base de mon exposé. Une fois
lancé, j'ai écrit d'autres petits morceaux de code pour présenter
quelques fonctionalités du langage que le premier exemple ne m'avait
pas permis d'exposer. C'est tout cela que je vais présenter dans ce
qui suit.

%%% Local Variables:
%%% mode: latex
%%% TeX-master: "dun19expl3"
%%% End:


\part{Écrire les premiers nombres premiers}
\label{part:premierspremiers}

% Time-stamp: <2019-09-24 10:47:32 administrateur>
% Création : 2019-08-24T14:29:55+0200
\section{Rappel}
\label{sec:rappel}

Je commence par redonner le code de Donald E. \textsc{Knuth} mais je
laisse le lecteur rechercher l'article précité de Denis \textsc{Roegel}
pour y trouver toutes les explications nécessaires.

\VerbatimInput[frame=lines, framesep=0.75\baselineskip, 
label={\textsc{Knuth} \textemdash{} \emph{définitions}},
labelposition=topline]{./codes/knuth-def.tex}

L'appel |\primes{15}| permet d'obtenir
\og \primes{15}\fg.

\section{Version avec \Expliii}
\label{sec:versionexpl3}


Commençons par quelques remarques: comme le code est écrit dans un
fichier \texttt{.tex} \TO et non dans un \gls{module} \TF je dois
commencer par enclencher la syntaxe propre à \Expliii{}, ce que je
fais dès la première ligne qui n'est pas un commentaire à usage
interne \TO d'où le décalage de numéro \TF dans le premier fragment
ci-dessous:
%
\Fragment{3}{5}{./codes/henel-def}

\subsection{Une macro privée pour écrire les nombres}
\label{sec:macro+ecrire+entier}

Sur la ligne~\(5\), on lit la définition d'une macro \emph{interne}
\CAD d'une macro qui n'a pas vocation à être appelée directement dans
un document rédigé par l'utilisateur final. On y voit déjà quelques
différences avec du code \LaTeXe{} que je vais préciser maintenant:
\begin{itemize}
\item les caractères |_| et~|:| sont promus au rang de lettres mais
  par l'arobase~|@| cher aux programmeurs en~\LaTeXe{};
  
\item les espaces du code ne donneront pas d'espace dans le
  document. Voilà qui soulage grandement le programmeur puisque
  disparait la phase de recherche des \TdSTrad{blancs
    inopportuns}{spurious blanks}.
\end{itemize}

Autre remarque, d'ordre typographique, la présentation que j'adopte
dans ce document \TO pour des raisons de place\TF n'est pas la
présentation recommandée par l'équipe du projet \LaTeX3. On devrait
plutôt, en effet, adopter la présentation qui suit\index{présentation
  du code}:
%
\begin{Verbatim}[frame=lines, framesep=0.75\baselineskip]
\cs_new:Nn \__yh_ecrire_nombre:n 
  { 
    \(\np{\int_to_arabic:n{ #1 }}\) 
  }
\end{Verbatim}
%
dont on m'accordera, j'espère, qu'elle occupe plus d'espace.

La commande que je définis s'appelle
\Macro{__yh_ecrire_nombre:n}. Ce nom comporte plusieurs parties:
\begin{enumerate}
  
\item les \og |__|\fg initiaux signalent que cette commande est
  \emph{privée} \CAD que son usage est réservée au \gls{module} \TO
  ici le simple fichier contenant le code\TF et n'est pas
  nécessairement documentée. Le programmeur ne s'engage à rien sur
  cette commande et un utilisateur du \gls{module} ne doit pas
  s'attendre à ce que le comportement de cette commande reste stable
  au fil des versions ni même à ce qu'elle soit toujours
  disponible. Il ne s'agit ici que d'une \textsb{convention} car
  \TeX{} est resté ce qu'il est et la gestion des espaces de noms
  n'est pas implémentée\footnote{L'équipe des développeurs de
    \Expliii{} nous dit, dans \autocite[p.~1]{expl3progr} que celà
    pourrait se faire mais avec un surcoût prohibitif.};
  
\item \og |yh|\fg remplace ici ce qui est \TO dans un \gls{module}\TF
  le nom du \gls{module} ou, à tout le moins, une chaine de caractères
  réservée pour ce \gls{module};
  
\item vient le nom explicite de la commande: \og |ecrire_nombre|\fg
  dans lequel le souligné~\og |_|\fg sert de séparateur de mots;
  
\item suivi du caractère~\og |:|\fg qui introduit la \gls{signature}
  de la commande qui est, ici, \og |n|\fg. Cela nous indique que la
  commande est une \gls{fonction} \TO c'est le vocabulaire défini par le
  projet \LaTeX3\TF qui attend un seul argument fourni entre
  accolades.
\end{enumerate}

La fonction \Macro{cs_new:Nn} utilise la \gls{signature} de la
commande qui suit pour déterminer le nombre et le type des arguments
que demande cette commande. Elle vérifie que la commande n'est pas
déjà définie et produit une erreur si c'est le cas. De plus, elle
définit la commande de manière globale \TO penser à \Macro{gdef} avec
les pouvoirs de \Macro{newcommand}\TF.

Elle possède plusieurs variantes dont, \PX{}, \Macro{cs_new:cn} qui
attend comme premier argument le \emph{nom} de la commande plutôt que
la commande elle-même. J'aurais obtenu le même résultat avec:
%
\begin{Verbatim}[frame=lines, framesep=0.75\baselineskip]
\cs_new:cn { __yh_ecrire_nombre:n } { \(\np{\int_to_arabic:n{ #1 }}\) }
\end{Verbatim}
%
code que je livre compacté.

Le 2\ieme argument de \Macro{cs_new:Nn} doit être fourni entre
accolades, c'est la définition de la commande. Ici j'utilise
\Macro{np} de l'extension \Pkg{numprint} en mode mathématique en
ligne \TO à l'aide du bien connu couple |\(| et~|\)| \TF  afin
d'obtenir \og \(\np{3257}\)\fg au lieu de \og \(3257\)\fg.

La \gls{fonction} \Macro{int_to_arabic:n} appartient au \gls{module}
\Mdl{int}, partie du noyau de \LaTeX3, qui regroupe les commandes
traitant les \glspl{entier}. Elle prend un argument dont elle fournit
la représentation en chiffres \emph{arabes}.

\subsection{Commande vérifiant qu'un entier impair est premier}
\label{sec:macro+impair+premier}

La commande \Macro{yh_impair_est_premier:nT} ne se déclare pas comme
privée, dans un \gls{module} elle devrait avoir une interface et une
sortie stable et être documentée car son nom ne commence pas par~|__|
mais ça aurait pu être le cas.

\(2\) étant le seul nombre pair premier, la question de la primarité
ne se pose vraiment que pour les entiers impairs. Je pourrai donc
n'essayer que des diviseurs impairs.

La \gls{signature} de cette \gls{fonction}, \CAD |nT|, nous indique
qu'elle attend deux arguments entre accolades, que le 1\ier sera un
test qui renverra soit~\VRAI{} soit~\FAUX{} et que, \emph{dans le seul
  cas} où le test est~\VRAI{}, le code contenu dans la 2\ieme paire
d'accolades sera considéré.

Regardons le code:
%
\Fragment{7}{25}{./codes/henel-def}

Dans la partie d'initialisation \TO lignes~\(8\) à~\(10\)\TF, je crée
deux \glspl{booleen} \TO \Macro{yh_continue_bool} et
\Macro{yh_est_premier_bool}\TF à l'aide de la \gls{fonction}
\Macro{bool_set_true:N} qui, dans le même temps, leur donne la
valeur~\VRAI.

Les \glspl{fonction} dont le nom contient |set| définissent les
commandes sans vérification et, à moins que le nom contienne |gset|
\TO |g| comme \og global\fg\TF, elles créent ces commandes
\emph{localement} \CAD{} dans le groupe où a lieu la définition.

Ici, contrairement à ma devise\footnote{À savoir: \og ceinture et
  bretelles\fg.}, je coure le risque que ces deux \glspl{booleen}
existent déjà, risque dont je pense qu'il est faible. On verra plus
bas ce que j'aurais pu faire pour sécuriser la définition.

Ces deux \glspl{booleen} ne sont pas des \glspl{fonction} mais des
\glspl{variable} \CAD{} des macros \TO au sens de \TeX{}\TF dont le
seul objet est de contenir une valeur et non pas de \emph{faire}
quelque chose. C'est pour cela que leurs noms n'ont pas de
\gls{signature} et ne se termine pas par~|:| qui signalerait une
\gls{fonction} sans argument.

Dernière étape de l'initialisation, avec \Macro{int_set:Nn} je donne
à la variable entière \textbf{l}ocale temporaire \Macro{l_tmpa_int}
la valeur~\(1\). Plusieurs \glspl{module} du noyau de \LaTeX3
fournissent deux variables locales dont les noms comportent |tmpa|
et~|tmpb| respectivement et deux variables \textbf{g}lobales \TO avec
les noms équivalents\TF.

Suivant la convention générale qui veut que le nom d'une
\gls{variable} se termine par son type \TO c'est le cas des deux
\glspl{booleen} que j'ai définis\TF, les variables temporaires
fournies par un \gls{module} du noyau ont une finale qui donne le type
comme \Macro{l_tmpa_int}. Cependant, contrairement à cette même
convention de nommage, les variables temporaires n'ont pas, en tête de
nom, le nom du module. On aura voulu éviter quelque chose comme
\Macro*{int_l_tmpa_int} dont on peut penser que c'est légèrement
redondant.

En ligne~\(12\), je commence une boucle \TANTQUE avec
\Macro{bool_do_while:Nn} qui attend un premier argument constitué d'un
seul lexème \TO ici \Macro{yh_continue_bool} qui est~\VRAI à l'entrée
dans la boucle\TF et un deuxième argument entre accolades: le corps de
la boucle.

Le corps de cette boucle commence avec l'accolade ouvrante du bout de
la ligne~\(12\) et s'achève avec la 2\ieme accolade fermante de la
ligne~\(24\).

Je commence par ajouter~\(2\) à la valeur courante de
\Macro{l_tmpa_int} à l'aide de la \gls{fonction} \Macro{int_add:Nn};
\Macro{l_tmpa_int} est ici le candidat diviseur.

\Macro{bool_if:nTF} est une forme de~\SIALORSSINON: on a bien les deux
branches de l'alternative puisque la signature se termine en~|TF|. Il
s'agit de savoir si le candidat est bien un diviseur de l'entier
représenté par~|#2|. Pour ce faire, j'utilise la \gls{fonction}
\Macro{int_compare_p:nNn} dont le 2\ieme argument est un opérateur de
comparaison \CAD |>|, |<| ou~|=|. Le 3\ieme argument est
simplement~\(0\). Le premier, lui, est le reste de la division de~|#2|
par \Macro{l_tmpa_int} que l'on obtient à l'aide de la \gls{fonction}
\Macro{int_mod:nn}.

Si le reste est nul, on exécute le code du 2\ieme argument de
\Macro{bool_if:nTF}. On positionne les deux \glspl{booleen} à~\FAUX{}
puisque |#2|, divisible par \Macro{l_tmpa_int}, n'est pas premier et
qu'il n'est donc pas nécessaire de reprendre la boucle.

Si le reste n'est pas nul, je compare l'entier~|#2| et le carré du
candidat \TO on peut utiliser les opérateurs \emph{classiques}
d'opérations: |+|, |-|, |*|, |/| car, derrière ce code, c'est le
mécanisme de la macro \Macro{numexpr} de \hologo{eTeX} qui est à
l'œuvre\TF. Si le carré est plus grand, c'est que |#2|~est premier et
qu'il est temps de s'arrêter d'où les valeurs données aux deux
\glspl{booleen}. Le 3\ieme argument du \Macro{bool_if:nTF} s'arrête
avec la 1\iere accolade fermante de la ligne~\(24\).

À la sortie de la boucle, si |#2| n'est pas premier il n'y a rien à
faire, s'il est premier, on le retourne. Cela est accompli à l'aide
de la \gls{fonction} \Macro{bool_if:nT} qui réalise un saut
conditionnel \SIALORS.

\subsubsection{Compléments}
\label{sec:complementbouclesaut}

Outre les deux \glspl{fonction} \Macro{bool_if:nTF} et
\Macro{bool_if:nT}, \Expliii fournit également la \gls{fonction}
\Macro{bool_if:nF} qui réalise un saut conditionnel \SISINON. Dans le
manuel, on verra que la signature se termine
par~\underline{\texttt{\emph{TF}}} pour signaler la présence des trois
signatures~|T|, |F| et~|TF|.

Par ailleurs, \Expliii fournit essentiellement quatre boucles:
\begin{itemize}
\item \Macro{bool_do_while:Nn}, déjà rencontrée. C'est un \TANTQUE
  dont le corps est exécuté puis le test effectué \CAD qu'en toutes
  circonstances, le corps est exécuté au moins une fois;

\item \Macro{bool_do_until:Nn}, qui inverse le sens du test. C'est un
  \JUSQUA dont le corps est exécuté au moins une fois;\label{booldountilNn}
  
\item \Macro{bool_while_do:Nn}, \TANTQUE qui commence par le
  test. Il se peut donc que le corps ne soit pas exécuté;
  
\item \Macro{bool_until_do:Nn}, \JUSQUA commençant par le
  test.
\end{itemize}

Ces quatres \glspl{fonction} ont des variantes de signatures~|cn|
et~|nn|. Avec~|cn|, le 1\ier argument est le nom d'un \gls{booleen};
avec~|n|, c'est une expression booléenne comme \TO exemple tiré de
\autocite{l3interfaces}\TF
%
\begin{Verbatim}[frame=lines, framesep=0.75\baselineskip]
\int_compare_p:n { 1 = 1 } &&
  (
    \int_compare_p:n { 2 = 3 } ||
    \int_compare_p:n { 4 <= 4 } ||
    \str_if_eq_p:nn { abc } { def }
  ) &&
! \int_compare_p:n { 2 = 4 }
\end{Verbatim}
%
qui permet de mesurer la complexité acceptée.

\subsection{Écrire la liste des nombres premiers}
\label{sec:ecrirelistepremiers}

Avant d'aborder la commande destinée à l'utilisateur \TO dont je dirai
que c'est une \emph{commande de document}, comme le suggère le nom de
la macro qui permettra de la créer tout à l'heure. L'extension
\Pkg{xparse}~\autocite{xparse} parle de \English{document-level
    command}\TF, je présente deux nouvelles macros utilitaires \TO
qui elles aussi auraient pu être \emph{privées}\TF à savoir
\Macro{yh_ecrire_separateur:n} et \Macro{yh_ecrire_si_premier:n}.

\subsubsection{Deux macros utilitaires}

\Fragment{27}{32}{./codes/henel-def}

On voit dans le code ci-dessus, en ligne~\(27\), une pré-déclaration
de \Macro{yh_ecrire_separateur:n} qui permet juste de s'assurer que ce
nom est bien disponible.

La macro suivante \Macro{yh_ecrire_si_premier:n} fait exactement ce
que son nom suggère. 

Passons maintenant à la commande principale.

\subsubsection{La commande de document}
\label{sec:commandededocument}

\Fragment{34}{59}{./codes/henel-def}

Le code permettant de définir cette commande utilise l'extension
\Pkg{xparse}~\autocite{xparse} et sa macro \Macro{NewDocumentCommand}.

La syntaxe de déclaration d'argument change quelque peu! Le |m|,
2\ieme argument de la macro, indique que l'on définit une commande
requérant un unique argument obligatoire fourni entre accolades \TO
|m| pour \English{mandatory} \CAD \og obligatoire\fg\TF.
% Le 1\ier argument est attendu \textsb{entre accolades} \TO pas de
% fantaisie à la \LaTeXe{}, du genre |\new|\0|com|\0|mand|
% |\truc|\dots{}, ici\TF. FAUX!
Le 1\ier argument est, comme on s'y attend, le nom de la commande
créée.

Le corps de la définition commence en ligne~\(35\) en ouvrant un
groupe avec \Macro{group_begin:}  et finit en ligne~\(59\) en fermant
le groupe avec \Macro{group_end:}.

\paragraph{Initialisation}

Dans le groupe, je crée en lignes~\(37\) et~\(38\) deux \glspl{entier}
locaux \TO d'où le |l| initial\TF mais qui seront créés
\emph{globalement}, car il n'y a pas moyen de faire autrement. Leur
caractère local est donc conventionnel mais signale que l'on ne doit
pas s'attendre à ce qu'ils aient une valeur pertinente à l'extérieur
du groupe.

Pour respecter ce caractère local, je donne à ces \glspl{entier}, une
valeur avec \Macro{int_set:Nn} et pas avec \Macro{int_gset:Nn} qui, de
fait, n'est qu'un alias de la première, à moins que ce soit
l'inverse\footnote{Je ne suis pas allé vérifier dans le
  source~\autocite{source3}.}.

Afin que, en sortie de commande, ces deux \glspl{entier} ne soient
plus accessibles \TO \textsl{Vous êtes priés de laisser ces lieux dans l'état
où vous auriez aimé les trouver en entrant!} \TF, je les annihile en
lignes~\(57\) et~\(58\) à l'aide de \Macro{cs_undefine:N}.

Il est temps à présent de définir, localement, avec \Macro{cs_set:Nn},
la fonction \Macro{yh_ecrire_separateur:n}, ce que je fais en
ligne~\(41\) et~\(42\). Comme la définition est faite dans une
définition, on doit, bien entendu, recourir au doublement des
\emph{dièses} d'où le |##1| dans le code de cette fonction.

Dans les 1\ier et 3\ieme arguments de \Macro{int_compare:nNnTF} on
peut écrire \emph{naturellement} des opérations sur les entiers. Je ne
m'en prive pas.

Les tildes~|~| que l'on voit apparaitre en ligne~\(42\) permettent de
coder un espace qui apparaitra dans le document final. Par contre, il
faut utiliser autre chose pour coder un espace insécable \TO la
solution est la macro \Macro{nobreakspace}\TF.

Je suis obligé de regarder si le séparateur est le dernier car, en
français\footnote{Encore que, si j'ai bien compris un article récent
  paru dans la presse britanique, même à Oxford on a fini par
  renoncer à la virgule devant le \emph{and} précédant le dernier mot
  d'une énumération. \textsl{Tout fout l'camp, ma bonne dame!}} il ne
doit pas y avoir de virgule devant le \emph{et}.

Comme l'annonce la ligne~\(43\) je ne calcule pas les deux premiers
nombres premiers \TO Donald \textsc{Knuth} ne le faisait pas\TF et je
ne vérifie pas que l'utilisateur demande au moins trois nombres \TO
même remarque\TF; pour faire comme dans les livres sérieux, je laisse
cette amélioration comme exercice pour le lecteur.

Je prépare maintenant la boucle en initialisant, localement, le
\emph{candidat premier} à~\(3\). On a déjà vu \Macro{int_set:Nn}.
Et je rentre dans la boucle en ligne~\(50\) pour en sortir en
ligne~\(55\). La seule nouveauté de ce fragment de code est l'emploi
de \Macro{int_incr:N} qui permet d'incrémenter localement le compteur
\Macro{l_compteur_int}.

J'ai déjà expliqué ci-dessus l'utilité des lignes~\(57\) à~\(59\).

\subsection{Utilisation de la commande de document}
\label{sec:utilisation}

Il me reste à utiliser cette commande dans ce document:
|\ListePremiers{15}| produit \og \ListePremiers{15}\fg.

Au passage, on notera avec satisfaction que |\primes| et
|\ListePremiers| sont d'accord quant aux quinze premiers nombres
premiers. C'est rassurant.

\subsection{Quelques commentaires subjectifs}
\label{sec:commentairessubjectifs}

J'accorde sans peine qu'avec \Expliii le code est plus verbeux qu'avec
\TeX{} tout seul. Je ne peux, cependant, m'empêcher de penser qu'il
est plus clair. Pour ma part, le gain énorme que je vois en passant
cette fois de \LaTeXe{} à \Expliii{} c'est le systématisme du nommage
des macros, \glspl{fonction} ou \glspl{variable}. L'équipe du projet
\LaTeX3 en fait d'ailleurs, elle-même, un ``\emph{argument de
  vente}''.

La puissance de \Macro{NewDocumentCommand} \TO que l'on a à peine
effleuré ci-dessus\TF, à elle seule, peut inciter à passer à
\Expliii{} pour créer ses propres macros.

%%% Local Variables:
%%% mode: latex
%%% TeX-master: "dun19expl3"
%%% End:


\part{Le crible d'Eratosthène}
\label{part:cribleerato}

% Time-stamp: <2019-09-24 10:45:10 administrateur>
% Création : 2019-08-24T14:40:52+0200

Il s'agit maintenant de fournir quelques commandes de document permettant de
créer un exemple du crible d'Eratosthène comme ceci:
\begin{center}\label{lecrible100}
  \eratosthene{100}
  \EcrireCribleEratosthene{100}
\end{center}

Ce sera l'occasion d'évoquer un des \og objets de haut niveau\fg
fourni par \Expliii. Non, je ne parle pas de programmation objet!

Une fois encore, je place le code dans un ``bête''
fichier~\texttt{.tex} d'où les bien connus désormais
\Macro{ExplSyntaxOn} et \Macro{ExplSyntaxOff} en début et fin de
fichier.

\section{Buts et algorithmes}
\label{sec:butsalgo}

L'objet dont il est question est appelé \English{intarray}. Il est
global et, une fois créé, ne peut être détruit\footnote{C'est du moins
  ce que je comprends à la lecture \TO que j'ai voulu rapide\TF du
  source. En tout cas \Expliii ne fournit pas d'outil pour ce
  faire.}. Dans certains langages on parle de \emph{vecteurs} mais il
s'agit ici d'un vecteur d'\glspl{entier} (relatifs) et d'entiers
seulement.  Sa taille doit être donnée à la création et ne peut pas
être changée.

Avec la commande \Macro{eratosthene}, on crée deux \emph{vecteurs
  d'entiers}. Le premier, \Macro{__ERA_dv_ia}, contient en \(n\)\ieme
place le plus petit diviseur premier de~\(n\). Le second,
\Macro{__ERA_qt_ia}, contient en \(n\)\ieme place le quotient de~\(n\)
et de ce diviseur premier dont je viens de parler.

La commande  \Macro{eratosthene} ne prend qu'un argument, un entier
naturel, qui est la taille des deux vecteurs sus-définis.

Une fois les vecteurs créés, on pourra les utiliser pour écrire le
crible d'Eratosthène comme vu précédemment ou, pour donner un autre
exemple, obtenir la décomposition d'un entier comme produit de ses
diviseurs premiers.

\section{Création des vecteurs}
\label{sec:creationvecteurs}

\Fragment[commande \texttt{eratosthene},
initialisation]{5}{10}{./codes/erato-def}

En \Ligne{6} j'ouvre un groupe avec \Macro{group_begin+} qui sera
fermé juste avant l'accolade fermante marquant la fin de la définition
de \Macro{eratosthene} en \Ligne{45}.

Dans ce groupe je vais créer un \emph{entier} avec
\Macro{int_new+N}. Il est créé globalement mais je le détruis en
\Ligne{44} avec \Macro{cs_undefine+N}. Il ne survivra donc pas à la
macro.

Je crée également les deux vecteurs en lignes~\(9\) et~\(10\) à l'aide
de \Macro{intarray_new+Nn} qui prend en 2\ieme argument la taille du
vecteur qui est l'argument passé à \Macro{eratosthene}. Initialement
ces vecteurs ne contiennent que des zéros.

Les noms des vecteurs devraient se terminer par \texttt{_intarray}
mais je me suis permis un raccourci en \texttt{_ia}. Leurs noms
commencent avec le double souligné puisque ce sont des
\glspl{variable} \emph{privées}. \texttt{ERA} est le sigle de ce
pseudo-module.

En \Ligne{8} je donne à la \gls{variable}
\Macro{l_ERA_diviseur_max_int} la partie entière de la racine carrée
de la taille des vecteurs \TO valeur donnée par~|#1|\TF en
\emph{transtypant} le résultat de l'opération |floor(| |sqrt (| |#1|
|))| qui est un \gls{flottant} en un \gls{entier}. 

\subsection{Une commande auxiliaire}

Je crée maintenant, à l'intérieur de la définition de
\Macro{eratosthene} la commande \Macro{ERA_marquer_de_a_par+nnn} que,
de fait, je n'utiliserai qu'une fois, en \Ligne{33}, et dont, par
conséquent, j'aurais pu faire l'économie, au risque d'embrouiller
considérablement le code suivant. 

\Fragment[commande \texttt{eratosthene},
auxiliaire]{12}{20}{./codes/erato-def}

Je crée cette commande localement grâce à \Macro{cs_set+Nn}, elle ne
survivra donc pas au groupe. Je commence, dans la définition de
\Macro{ERA_marquer_de_a_par+nnn}, par donner comme valeur à la
\gls{variable} \emph{locale} \Macro{l_tmpa_int} le quotient entier des
1\ier et 3\ieme arguments de la commande.

Puis, lignes~\(15\) à~\(19\), j'utilise \Macro{int_step_inline+nnnn}
pour accomplir le travail. C'est une boucle \POUR. L'index \TO
implicite et anonyme\TF parcourt les entiers en partant du premier
argument de cette dernière macro jusqu'au troisième argument avec un
pas donné par le deuxième. Le 4\ieme argument est le code \emph{en
  ligne} dans lequel |#1| est l'indice de la boucle. Enfin, |#1|
serait l'indice de la boucle si on utilisait directement la macro
\Macro{int_step_inline+nnnn} au niveau du document. Ici comme je
l'utilise dans une définition incluse elle-même dans une définition,
je dois doubler les dièses deux fois d'où le |####1| que l'on voit
dans le code.

En \Ligne{19}, j'incrémente \Macro{l_tmpa_int}. De la \Ligne{16} à la
\Ligne{18}, j'ai un test \SIALORS avec \Macro{int_compare+nNnT}:
les actions des lignes~\(17\) et~\(18\) ne seront accomplies que s'il
est vrai que la |####1|-ième composante du vecteur \Macro{__ERA_dv_ia}
est nulle \CAD que ce nombre n'a pas encore été traité. Dans ce cas,
je place la valeur du pas~|##3| \TO c'est le plus petit diviseur
premier du nombre considéré\TF à la bonne place dans le vecteur des
diviseurs et je place le quotient à la place idoine dans le vecteur
des quotients. L'incrémentation faite en-dessous me dispense de
calculer ce quotient à l'aide d'une division entière.

Comme le nom l'indique la macro \Macro{intarray_gset+Nnn} réalise une
affectation \textbf{g}lobale dans le vecteur donné en 1\ier argument. C'est
bien ce que je veux. Le 2\ieme argument donne l'indice, le 3\ieme la
valeur à y placer.

\subsection{Traitement des entiers pairs}
\label{sec:lespairs}

Je retourne maintenant au niveau de la définition de la commande
\Macro{eratosthene} et je commence par traiter les nombres pairs.

\Fragment[commande \texttt{eratosthene},
nombres pairs]{23}{28}{./codes/erato-def}

À ce moment de l'action, les deux  vecteurs ne contiennent que des
zéros, cela nous dispense du test tel qu'il apparaît dans la macro
auxiliaire vue ci-dessus. De plus, le premier quotient (\(2/2\)) est
clairement~\(1\) d'où la \Ligne{23}.

On retrouve ensuite \Macro{int_step_inline+nnnn} pour traiter les
nombres pairs en partant de~\(2\), avec un pas de~\(2\), et en
finissant à la taille des vecteurs~|#1|.

\subsection{Traitement des entiers impairs}
\label{sec:lesimpairs}

Le traitement des entiers impairs se fera en deux étapes, la deuxième
étant d'une simplicité déconcertante.

\Fragment[commande \texttt{eratosthene},
impairs]{31}{36}{./codes/erato-def}

Nous avons déjà vu tout ce que j'utilise ici. On notera, lignes~\(35\)
et~\(36\), la boucle permettant, en avançant de~\(2\) en~\(2\), de
trouver le prochain nombre impair non traité avec
\Macro{int_do_until+nNnn} analogue, pour réaliser une boucle sur les
entiers, de \Macro{bool_do_until+Nn} présentée page~\pageref{booldountilNn}. 

La dernière boucle traite les nombres impairs qui n'ont pas encore été
vus. On pourrait améliorer la chose en commençant avec le premier
impair non-vu, une fois encore je laisse ça en exercice!  

\Fragment[commande \texttt{eratosthene}]{39}{42}{./codes/erato-def}

\section{Deux exemples d'utilisation}
\label{sec:utilisationcrible}

Les deux vecteurs ayant été créés, on peut les utiliser. Je donne deux
exemples: l'écriture du \og crible\fg, l'écriture d'un nombre comme
produit de ses facteurs premiers \TO version améliorable\TF.

\subsection{Écriture du crible}
\label{sec:ecriturecrible}

Je présente maintenant la commande permettant de créer le tableau
présenté en page~\pageref{lecrible100}.

\Fragment[écriture du crible, commande de
document]{69}{70}{./codes/erato-def}

L'unique argument de \Macro{EcrireCribleEratosthene} doit être un
entier positif plus petit que la taille des deux vecteurs créés
auparavant sinon on aura une erreur d'accès aux vecteurs dû à un
indice hors bornes.

Bien entendu, cette commande se contente de refiler le boulot à la
commande interne \Macro{ERA_presenter_nieme+n} par l'intermédiaire de
la boucle \Macro{int_step_inline+nn}.

Soulevons le capot.

\Fragment[écriture du crible, commande interne]{50}{67}{./codes/erato-def}

Pour la création du tableau, je copie servilement le code que fourni
Christian \textsc{Tellechea} dans~\autocite{tellechea}. J'utilise
quand même la macro \Macro{framebox} fournie par \LaTeXe{} car
personne n'est obligé de réinventer la roue chaque matin!

\subsection{Décomposition d'un nombre en facteurs premiers}
\label{sec:facteurspremiers}

Regardons à présent la commande permettant d'obtenir:
\EcrireDiviseurs{78} \qquad \EcrireDiviseurs*{75} \qquad
\EcrireDiviseurs*{64}
\par\noindent
avec le code
\par\noindent
|\EcrireDiviseurs{78}| |\qquad| |\EcrireDiviseurs*{75}| |\qquad| |\EcrireDiviseurs*{64}|.

Voici le code: 
\Fragment[facteurs premiers]{73}{96}{./codes/erato-def}

Comme on l'aura remarqué, la commande \Macro{EcrireDiviseurs} admet
une variante étoilée. On l'obtient facilement avec
\Macro{NewDocumentCommand} en donnant comme descriptif de 1\ier
argument le~|s| que l'on voit en \Ligne{73}.

Avec \Macro{IfBooleanF} de la \Ligne{82}, on ne place un saut de
paragraphe \TO avec \Macro{par}\TF que s'il n'y a \textsb{pas}
d'étoile.

Pour réagir à la présence ou l'absence d'un lexème \TO une étoile~|*|
avec le descripteur d'argument~|s|, n'importe quel lexème avec~|t| \TF
on dispose également de \Macro{IfBooleanT} et de \Macro{IfBooleanTF}.

On retrouve les mêmes techniques de création d'\glspl{entier} dans un
groupe et leur destruction en sortie du corps de la définition de la
commande. J'évite les entiers de brouillon fournis par \Expliii{} car
je préfère, pour ne pas me perdre, des noms plus explicites. De toute
façon, j'ai besoin d'au moins trois entiers locaux. On devrait pouvoir
écrire la décomposition de \(75\) sous la forme \(75=3\times5^{2}\) à
l'aide d'un \gls{entier} supplémentaire. Là encore, l'exercice etc.


%%% Local Variables:
%%% mode: latex
%%% TeX-master: "dun19expl3"
%%% End:


\part{Les suites de Kaprekar}
\label{part:kaprekar}

% Time-stamp: <2019-09-22 16:32:18 administrateur>
% Création : 2019-08-26T12:37:33+0200

\section{Suites de \textsc{Kaprekar}}
\label{sec:suiteskaprekar}

Une suite de \textsc{Kaprekar} est définie par la donnée d'un nombre
entier, p.~ex. \(125\), d'une base \TO je prendrai systématiquement et
uniquement~\(10\) comme base de l'écriture des nombres entiers\TF et
d'un nombre~\(N\) de chiffres que j'appellerai la \emph{taille} du
nombre. On obtient un terme~\(u\sb{n+1}\) à partir du terme
précédent~\(u\sb{n}\) en appliquant l'algorithme suivant:
\begin{enumerate}\label{algokap}
\item on écrit \(u\sb{n}\) avec \(N\) chiffres dans la base choisie
  quitte à compléter avec des zéros; 
\item on range les \(N\) chiffres ainsi obtenus dans l'ordre
  décroissant pour obtenir \(v\sb{n}\); 
\item on range les \(N\) chiffres dans l'ordre croissant pour obtenir
  \(w\sb{n}\); 
\item enfin, \(u\sb{n+1} = v\sb{n} - w\sb{n}\).
\end{enumerate}

Une telle suite est évidemment finie. La commande que nous allons
détailler s'arrête sur le dernier nombre déjà obtenu.

En voici deux exemples.

On obtient \og \Kaprekar{125}\fg avec |\Kaprekar{125}| qui, par
défaut, travaille avec \(N=4\), et \og \Kaprekar[6]{125}\fg pour
lequel on a pris \(N=6\).

L'objet qui va nous servir à créer ces suites est la \gls{sequence}
\TO dont le sigle est \texttt{seq} \TF que fournit l'extension
\Pkg{l3seq}.


\section{Le code}
\label{sec:kaprekarcode}

Je rappelle que tout le code est placé entre \Macro{ExplSyntaxOn}
et \Macro{ExplSyntaxOff} pour les raisons déjà exposées.

La commande de document est \Macro{Kaprekar} dont voici le code.

\Fragment[Kaprekar, commande de
document]{73}{74}{./codes/kaprekar-def}

On y retrouve l'habituelle \Macro{NewDocumentCommand} mais un nouveau
descripteur d'argument, à savoir |O{4}|: le premier argument est
optionnel et sa valeur par défaut est~\(4\). Si on veut en fournir une
autre, on le placera entre crochet comme d'habitude.

La commande n'est, de fait et à part le truc de l'argument optionnel,
qu'un alias de \Macro{KAP_ecrire_tous_les_suivants+nn} que je vais
maintenant détailler. Mais commençons par la phase d'initialisation.

\subsection{Initialisation}
\label{sec:kaprinit}

\Fragment[Kaprekar, initialisation]{4}{6}{./codes/kaprekar-def}

Avec \Macro{seq_new+N} je crée deux \glspl{sequence}: \Macro{kaprun_seq}
et \Macro{kaprekar_seq}. La première \gls{variable} a vocation à
recevoir la suite des chiffres du nombre traité. La seconde contiendra
la suite de \textsc{Kaprekar}.

Je crée ensuite un \gls{entier} \emph{local} en suivant le modèle
de l'extension \Pkg{int}.

\subsection{De l'entier à la suite et retour}
\label{sec:entiersuiteretour}

J'ai besoin de commandes internes \TO qui pourraient être privées\TF
pour \emph{découper} un entier en la suite de ses chiffres et pour
créer un entier à partir d'une suite de chiffres. Le fait de
travailler en base fixée \TO et en base~\(10\) qui plus est\TF
facilite largement le travail. Là encore, si le cœur vous en dit,
exercice!

\Fragment[Kaprekar, de l'entier à la
suite]{9}{15}{./codes/kaprekar-def}

Le premier argument de \Macro{KAP_int_to_seq+NN} est le nombre à
découper, le second la suite qui contiendra ses chiffres. Dans les
deux cas, la commande attend une \gls{variable}.

Nous avons déjà rencontré la majorité des commandes utilisées dans la
définition. Avec \Macro{seq_put_left+NV} je place en première place
\TO à gauche \InEnglish{left}\TF la valeur contenue dans la
\gls{variable} \Macro{l_tmpb_int}. Le~|V| dans la signature assure que
la commande \Macro{l_tmpb_int} sera traitée de telle sorte que l'on
récupère bien son contenu.

Dans la boucle, on trouve \Macro{int_div_truncate+nn} qui permet de
récupérer le quotient entier de~|#1| par~|#2|.

\medbreak

\Fragment[Kaprekar, de la suite à
l'entier]{17}{22}{./codes/kaprekar-def}

Cette commande transforme une suite de chiffres en un \gls{entier}. On
trouve, en~\Ligne{18},  la commande \Macro{seq_pop_right+NNT} qui
prend la valeur la plus à droite dans la suite passée en 1\ier
argument pour la placer dans la \gls{variable} donnée en 2\ieme
argument. Si cette opération est possible \TO \CAD si la pile n'était
pas vide\TF le 3\ieme argument est développé.

La suite, après cette opération, est amputée de sa dernière
valeur. J'utilise ici la suite comme une pile \TO c'est le fameux
\textsc{lifo} pour \English{Last In First Out} \TF. \Expliii{} fournit
aussi des commandes qui permettent d'utiliser la suite comme une queue
\TO \textsc{fifo} pour \English{First In First Out}\TF.

Dans la boucle, j'utilise \Macro{seq_if_empty_p+N} qui prend la valeur
\VRAI{} si la \gls{sequence} est vide. Tant qu'elle n'est pas vide, je
multiplie par~\(10\) la valeur conservée dans la \gls{variable} donnée
en~|#2| et j'y ajoute la dernière valeur extraite de la pile.

De fait, le chiffre le plus à droite est celui de \emph{poids} le plus
grand. 

\medbreak

\Fragment[Kaprekar, remplissage]{24}{26}{./codes/kaprekar-def}

Cette commande assure que la suite contient le bon nombre de chiffres
en ajoutant des zéros si nécessaire. La seule nouveauté est
\Macro{seq_count+N} qui donne le nombre de valeurs contenues dans la
suite.

\medbreak

\Fragment[Kaprekar, tri]{28}{32}{./codes/kaprekar-def}

\Macro{seq_sort+Nn} trie la \gls{sequence} donnée \TO par son \og
nom\fg\TF en 1\ier argument à l'aide du code donnée en 2\ieme
argument. Ici, j'échange |##1| et |##2| si le premier est strictement
supérieur au second, autrement dit, je trie les éléments de la suite du
plus petit au plus grand. Les commandes \Macro{sort_return_swapped+}
\TO retourner en échangeant\TF et \Macro{sort_return_same+} \TO
retourner le même\TF assurent le tri.

\medbreak

\Fragment[Kaprekar, calcul du terme
suivant]{34}{44}{./codes/kaprekar-def} 

Le terme courant est donné par une \gls{variable} comme 2\ieme
argument, le 1\ier est la taille, le 3\ieme doit être la
\gls{variable} qui recevra le terme suivant.

\Macro{kaprun_seq} prend en~\Ligne{35} les chiffres du terme courant
puis on remplit puis on range dans l'ordre croissant.

Avec \Macro{seq_set_eq:NN}, en~\Ligne{39}, on crée, localement et sans
contrôle, \Macro{kaprdx_seq} comme copie de \Macro{kaprun_seq} puis,
en ligne suivante, on la range dans l'ordre décroissant.

Les trois dernières lignes traduisent l'algorithme de \textsc{Kaprekar}
tel qu'exposé en~\pageref{algokap}.

Il ne reste plus qu'à utiliser tout cela.

\subsection{Commande interne}

\Fragment[Kaprekar, commande interne]{46}{71}{./codes/kaprekar-def}

La commande \Macro{KAP_ecrire_tous_les_suivants+nn} prend la valeur
de~\(N\) \TO la \emph{taille}\TF en premier argument et le premier
terme de la suite comme 2\ieme argument.

Une fois de plus, tout se passe dans un groupe, ouvert en~\Ligne{47}
et fermé en~\Ligne{71}.

J'utilise la \gls{variable} booléenne locale \Macro{l_tmpa_bool}
fournie par \Expliii{}. Je lui donne la valeur \VRAI{}
en~\Ligne{48}. Quand elle prendra la valeur \FAUX{} il sera temps de
sortir de la boucle \TO lignes~\(54\) à~\(57\)\TF qui remplit la
\gls{sequence} \Macro{kaprekar_seq}.

Avec \Macro{seq_clear+N}, en~\Ligne{49}, je vide la suite
\Macro{kaprekar_seq}.

En \Ligne{51} je place le 1\ier terme de la suite dans la
\gls{sequence} \Macro{kaprekar_seq}. Cette fois je remplis par la
droite avec \Macro{seq_put_right+NV} \TO j'utilise la suite comme une
\emph{queue}\TF.

C'est parce que j'ai besoin que cette valeur soit placée dans la
\gls{variable} \Macro{l_tmpc_int} que j'utilise
\Macro{seq_put_right+NV}. S'il suffisait de placer la valeur
directement dans la suite, on pourrait utiliser
\Macro{seq_put_right+Nn} avec le code
\vspace{-\baselineskip}
\begin{Verbatim}[frame=lines,framesep=0.75\baselineskip]
\seq_put_right:Nn \kaprekar_seq {#2}
\end{Verbatim}

Je place dans \Macro{l_tmpc_int} le terme suivant calculé avec
\Macro{KAP_suivant+nNN} en~\Ligne{52} et si ce terme est déjà présent
dans la suite \TO cas ou \(u\sb{1} = u\sb{0}\)\TF je donne à
\Macro{l_tmpa_bool} la valeur \FAUX. C'est \Macro{seq_if_in+NVT} qui
permet de vérifier que la valeur~|#2| est présente dans la suite~|#1|
et de développer, si besoin le code placé dans~|#3|.

À la fin de la \Ligne{57} la suite a été créée.

J'ai laissé, commentée, la \Ligne{58} que j'ai utilisée à titre de
diagnostic: la commande \Macro{seq_show+N} permet d'écrire le
contenu de la \gls{sequence} dans le fichier~|.log|.

À la ligne suivante, la commande \Macro{seq_use+Nnnn} permet d'écrire
le contenu de la \gls{sequence}, donnée en 1\ier argument. Les trois
arguments suivants donnent, dans l'ordre, le séparateur du premier et
deuxième élément quand il n'y a que deux éléments; ce qui précède
chaque élément, sauf le dernier, quand il y plus de deux éléments;
enfin, ce qui sépare l'avant-dernier du dernier élément. Comme je veux
que les nombres soient écrits à l'aide de \Macro{np} je ne peux pas me
contenter de cette ligne d'où ce qui suit.

Après avoir placé dans un compteur le nombre d'éléments de la suite,
j'utilise \Macro{seq_map_inline+Nn} pour appliquer à chaque élément de
la suite passée en 1\ier argument le code contenu dans le 2\ieme
argument \TO \Ligne{62} à~\(70\)\TF.

Dans ce code, j'écris l'élément en en donnant la valeur à \Macro{np}
en mode mathématique en ligne puis je décrémente le compteur à l'aide
de \Macro{int_decr+N}.

Avec \Macro{int_case+nnF}, je sélectionne le séparateur à placer: si
le compteur vaut~\(1\) j'écris \og |~et~|\fg, s'il vaut~\(0\), je n'écris
rien \TO \Ligne{67}\TF et, dans tous les autres cas \TO \Ligne{69}\TF,
j'écris \og |~;~|\fg.

% \subsection{\texorpdfstring{\Kaprekar[2]{12}}{kaprekar 2-12}}

% \Fragment[Kaprekar]{4}{45}{./codes/kaprekar-def}


%%% Local Variables:
%%% mode: latex
%%% TeX-master: "dun19expl3"
%%% End:


\newpage{}

\part{Annexes}
\label{part:annexes}

\printbibliography{}

\printindex{}

\printglossaries

\newpage{}

\section*{Remarques sur le document et historique des corrections}
\label{sec:historique}

\subsection*{Remarques}
\label{sec:remarques}

Je tente \TO comme chacun sait il y a souvent loin de la coupe aux
lèvres\TF de respecter l'orthographe dite \emph{nouvelle}. On ne verra
donc plus d'accent circonflexe sur certains i! Il m'arrive, dans mon
ardeur de jeune révolutionnaire, de priver dudit accent certains o ou
autres a. On devra à quelque lectrice ou lecteur leur rétablissement.

Je tiens à remercier, d'avance, tout ceux qui prendront la peine de me
signaler erreur, omission, impropriété, obscurité et autres
maladresses.  

\subsection*{Historique des corrections}
\label{sec:histocorr}

\begin{etaremune}
\item 
  Seule une correction orthographique motive la version~\(1.2\),
  première version déposée sur le CTAN.

\item 
  La version~\(1.1\) intègre les corrections proposées par François
  \textsc{Pantigny} \TO courriel du 21 septembre 2019\TF même si je
  persiste à dépouiller les i de leurs si surannés chapeaux.
\end{etaremune}
\end{document}

%%% Local Variables:
%%% coding: utf-8 
%%% mode: latex
%%% TeX-master: t
%%% End:

