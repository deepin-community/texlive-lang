\documentclass[12pt,french]{article}
  %% ===== Ne rien modifier dans ce bloc =====================
  \usepackage{iftex}            % détection du moteur utilisé
  \iftutex                      % XeLaTeX
    \usepackage{fontspec}
  \else                         % pdfLaTeX
    \usepackage[utf8]{inputenc}
    \usepackage[T1]{fontenc}
  \fi
  %% =========================================================
  \usepackage{babel}

\begin{document}

Les commandes \LaTeX débutent par le symbole \verb=\= et se
terminent par le premier caractère autre qu'une lettre, y compris
l'espace. Cela a pour conséquence qu'une espace immédiatement après
une commande sans argument sera \emph{avalée} par la commande.

La portée d'une commande est \bfseries limitée à la zone entre accolades.

\begin{itemize}
\item L'environnement \texttt{enumerate} permet de créer une liste
  numérotée.
\item Les environnements de listes sont parmi les plus utilisés en
  \LaTeX.
\end{itemize}

\end{document}
