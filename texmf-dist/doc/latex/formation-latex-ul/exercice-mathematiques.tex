\documentclass[11pt,article,french]{memoir}
  %% ===== Ne rien modifier dans ce bloc =====================
  \usepackage{iftex}            % détection du moteur utilisé
  \iftutex                      % XeLaTeX
    \usepackage{fontspec}
  \else                         % pdfLaTeX
    \usepackage[utf8]{inputenc}
    \usepackage[T1]{fontenc}
  \fi
  %% =========================================================
  \usepackage{amsmath}
  \usepackage{babel}
  \usepackage[autolanguage]{numprint}

\begin{document}

Soit deux variables aléatoires stochastiquement indépendantes $X_1$ et
$X_2$. La loi de la première est une exponentielle de paramètre $1$.
Celle de la seconde est une gamma de paramètres $\alpha$ et $1$. Nous
allons déterminer la loi de $Y = \theta (X_1/X_2)$.

En premier lieu, la densité conjointe de $X_1$ et $X_2$ est
simplement le produit des densités marginales:
\begin{equation*}
  f_{X_1 X_2}(x_1, x_2) = \frac{1}{\Gamma(\alpha)}
  x_2^{\alpha - 1} e^{-(x_1 + x_2)}, \qquad
  x_1 > 0, x_2 > 0.
\end{equation*}

Nous allons utiliser la méthode du changement de variable (ou du
Jacobien). Pour ce faire, on pose les deux transformations
\begin{equation*}
  \begin{aligned}
    Y_1 &= \theta \left( \frac{X_1}{X_2} \right) \\
    Y_2 &= X_2
  \end{aligned}
  \qquad \Leftrightarrow \qquad
  \begin{aligned}
    X_1 &= \frac{Y_1 Y_2}{\theta} \\
    X_2 &= Y_2
  \end{aligned}
\end{equation*}
et donc
\begin{align*}
  \frac{\partial x_1}{\partial y_1} &= \frac{y_2}{\theta} &
  \frac{\partial x_1}{\partial y_2} &= \frac{y_1}{\theta} \\
  \frac{\partial x_2}{\partial y_1} &= 0 &
  \frac{\partial x_2}{\partial y_2} &= 1,
\end{align*}
d'où le Jacobien de la transformation est
\begin{equation*}
  J =
  \begin{vmatrix}
    y_2/\theta & y_1/\theta \\
    0 & 1
  \end{vmatrix}
  = \frac{y_2}{\theta}.
\end{equation*}
Le domaine de $Y_1$ et de $Y_2$ est $R^+$. On a donc
\begin{align*}
  f_{Y_1 Y_2}(y_1, y_2)
  &= f_{X_1 X_2}(y_1 y_2/\theta, y_2)\, \lvert y_2/\theta \rvert \\
  &= \frac{1}{\theta \Gamma(\alpha)} y_2^\alpha
    e^{-(y_1 y_2/\theta + y_2)}.
\end{align*}
Par conséquent,
\begin{align*}
  f_{Y_1}(y_1)
  &= \int_0^\infty f_{Y_1 Y_2}(y_1, y_2)\, dy_2 \\
  &= \frac{1}{\theta} \int_0^\infty
    \frac{1}{\Gamma(\alpha)} y_2^{\alpha + 1 - 1}
    e^{-[(y_1 + \theta)/\theta] y_2}\, dy_2 \\
  &= \frac{1}{\theta}
    \frac{\theta^{\alpha + 1}}{(y_1 + \theta)^{\alpha + 1}} \\
  &= \frac{\theta^\alpha}{%
    (y_1 + \theta)^{\alpha + 1}}, \qquad y_1 > 0,
\end{align*}
d'où la loi de $Y_1 = \theta (X_1/X_2)$ est une Pareto$(\alpha, \theta)$.

\end{document}
