\documentclass[11pt,french]{memoir}
  %% ===== Ne rien modifier dans ce bloc =====================
  \usepackage{iftex}            % détection du moteur utilisé
  \iftutex                      % XeLaTeX
    \usepackage{fontspec}
  \else                         % pdfLaTeX
    \usepackage[utf8]{inputenc}
    \usepackage[T1]{fontenc}
  \fi
  %% =========================================================
  \usepackage{babel}
  \usepackage[autolanguage]{numprint}
  \usepackage{icomma}           % charger vers la fin

  %% Activer le paquetage hyperref ci-dessous
  % \usepackage[colorlinks]{hyperref} % toujours charger en dernier

  \title{Initiation au système de mise en page \LaTeX}
  \author{Vincent Goulet}

\begin{document}

\maketitle

\chapter{Classes et paquetages}

\section{Marges variables}

La longueur des lignes en {\LaTeX} est ajustée automatiquement en
fonction de la taille de la police de manière à ce que le nombre de
caractères par ligne demeure à peu près constant. Cela a pour but
d'améliorer la lisibilité: lorsqu'une ligne de texte est trop longue,
notre œil a plus de difficulté à suivre celle-ci de gauche à droite.

\subsection{recto verso}

En passant à une classe de document recto verso, vous remarquerez que
les marges gauche et droite ne sont pas identiques. C'est afin de
tenir compte de la marge de reliure.

Le paquetage \textbf{babel} fournit les commandes \verb=\ier=,
\verb=\iere=, \verb=\ieme= pour écrire «premier», «première»,
«deuxième» en chiffres: 1{\ier}, 1{\iere}, 36{\ieme}.

En typographie française, on doit utiliser la virgule comme séparateur
décimal et l'espace fine comme séparateur des milliers.
\begin{displaymath}
  y = 1,2x + 5, \quad x = 0, 1, \dots, \nombre{10 000}.
\end{displaymath}

\end{document}
