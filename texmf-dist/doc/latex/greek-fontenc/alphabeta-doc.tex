% alphabeta-doc: Documentation and tests for alphabeta.sty
% ********************************************************
%
% :Copyright: © 2010, 2023 Günter Milde
% :Licence:   This work may be distributed and/or modified under the
%             conditions of the `LaTeX Project Public License`_, either
%             version 1.3 of this license or any later version.
%
% This LaTeX document can be compiled with 8-bit TeX (latex or pdflatex),
% XeTeX (xelatex), or LuaTeX (lualatex).
% As it contains tests for the different limitations, there will be warnings
% in the log, which can be safely ignored.

\documentclass[a4paper]{scrartcl}
\addtokomafont{disposition}{\rmfamily}

\usepackage{amssymb, amsmath}
\frenchspacing

\ifdefined \UnicodeEncodingName % set by LaTeX for Unicode-aware engines
  \ifdefined\XeTeXrevision
    \newcommand{\engine}{XeTeX}
  \fi
  \ifdefined\luatexversion
    \newcommand{\engine}{LuaTeX}
  \fi
  % Setup for Unicode fonts (Xe-/LuaTeX)
  \usepackage[no-math]{fontspec}
  % The Harfbuzz renderer provides selection of pre-composed characters
  % (NFC normalization) with LuaTeX
  \setmainfont[Renderer=Harfbuzz]{Linux Libertine O}
  \setsansfont[Renderer=Harfbuzz]{Linux Biolinum O}
  \setmonofont[Renderer=Harfbuzz,Scale=MatchLowercase]{Liberation Mono}
  \usepackage[libertine,slantedGreek]{newtxmath}
  % \usepackage{unicode-math} % package conflict
  \newcommand{\fontset}{fontspec with Unicode fonts}
\else
  \newcommand{\engine}{pdfTeX}
  \usepackage[utf8]{inputenc}
  \DeclareUnicodeCharacter{03DE}{\textKoppa} % in LGR mapped to \textkoppa
  \usepackage[LGR,T1]{fontenc}
  \usepackage{textcomp}
  \usepackage{lmodern}
  % \usepackage{libertine}
  % \usepackage{gfsdidot}
  % \usepackage{kerkis}
  % \usepackage{newtxtext,newtxmath}
  \usepackage{isomath}
  \newcommand{\fontset}{fontenc with TeX fonts}
\fi

% load alphabeta after math setup and encoding setup!
\usepackage[normalize-symbols]{alphabeta}

% Check, if loading hyperref after/before alphabeta works:
\usepackage[unicode,colorlinks=true,linkcolor=blue]{hyperref}
\usepackage{bookmark}
\hypersetup{colorlinks=true,linkcolor=blue,urlcolor=blue,pdfencoding=auto}

\DeclareTextCommandDefault{\,}{\thinspace}
\DeclareTextCommand{\,}{PU}{\9040\011}% U+2009 THIN SPACE


% correct upcasing requires Babel,
% the remainder should work without requiring Babel
\usepackage[greek.polutoniko,english]{babel}


% Fallback macros:

\newcommand*{\missing}{\ensuremath{\oslash}}
% varstigma only defined with 8-bit LGR fonts
\ProvideTextCommandDefault{\textvarstigma}{\missing}
% varkappa, only defined with newtxmath
\providecommand*{\varkappa}{\missing}
% varbeta only defined with additional packages
\providecommand*{\varbeta}{\missing}

% Semantic styling:

\newcommand{\file}{\texttt}
\newcommand{\cs}[1]{\texttt{\textbackslash#1}}
\newcommand{\pkgref}[1]{\emph{\href{https://ctan.org/pkg/#1}{#1}}}

% Auxiliary macros

% print current font encoding:
\makeatletter
\newcommand{\currentEncoding}{\f@encoding}
\makeatother

\begin{document}

\title{The \emph{alphabeta} package}
\author{Günter Milde}
\date{2020/10/30}
\maketitle

\begin{abstract}\noindent
The \emph{alphabeta} package makes the standard macros for Greek letters in
mathe­matical mode also available in text mode. This way, you can input Greek
letters ``by name'' everywhere in the document. The mode determines whether
the characters are taken from the text or math font.

With 8-bit TeX and \pkgref{greek-inputenc},
literal Unicode charactes can also be used in mathematical mode.%
\footnote{\label{compiler}%
  This document was compiled with \engine{} using the
  \href{https://ctan.org/pkg/encguide}{font encoding} \encodingdefault{}
  \ifdefined \UTFencname % defined by fontspec
     (Unicode fonts).
     For a version using 8-bit fonts, see
     \href{alphabeta-doc.pdf}{alphabeta-doc.pdf}.
  \else
     (8-bit fonts).
     For a version using Unicode fonts, see
     \href{alphabeta-doc-tu.pdf}{alphabeta-doc-tu.pdf}.
  \fi
}
\end{abstract}

\tableofcontents


\section{Requirements and Conflicts}

The \emph{alphabeta} package depends on
\emph{\href{textalpha-doc.pdf}{textalpha}}
(both are part of \pkgref{greek-fontenc}).
It can be used under 8-bit TeX as well as XeTeX/LuaTeX.\footref{compiler}
Depending on the converter and fonts, different
\hyperref[sec:limitations]{limitations} apply.

The package conflicts with \pkgref{unicode-math}.

It also fails, if the \texttt{utf8x} input encoding is selected.
(The interface to the \pkgref{ucs} package uses an incompatible
definition of \cs{DeclareUnicodeCharacter}.)


\section{Usage}

Load this package in the preamble of your document (after font and math
setup) with
\begin{verbatim}
      \usepackage{alphabeta}
\end{verbatim}
%
Now you can write a single Greek symbol (like \Psi{} or \mu{}) or
a \lambda\omicron\gamma\omicron\varsigma{} in non-Greek text as well as
ISO-conforming formulas with upright symbols\footnote{
  The \pkgref{isomath} documentation describes more alternatives
  for upright Greek symbols in math mode.}
for constants like
\[
   A = \text{\pi} r^2
\]
(instead of $A = \pi r^2$).
Just like Latin letters, the Greek counterparts are by default italic in
math mode%
\footnote{Capital Greek letters are upright in TeX unless a package selects
          the ``ISO'' math-style. This document uses \pkgref{isomath} with
          8-bit Tex and \pkgref{newtxmath} with the \texttt{slantedgreek}
          option with Xe/LuaTeX. See the \emph{isomath} documentation for a
          detailled discussion of math-styles.}
and upright in text:

\begin{quote}
  Text: L \Gamma{} l \gamma,
  mathematics: $ L \ \Gamma \ l \ \gamma $
\end{quote}
%
See the source of this document \texttt{alphabeta-doc.tex} for a setup and
usage example.


\subsection{options}

Package options are passed to the \href{textalpha-doc.pdf}{\emph{textalpha}}
package. Example call with options:

\begin{verbatim}
      \usepackage[normalize-symbols,keep-semicolon]{alphabeta}
\end{verbatim}

\begin{description}
\item[\texttt{normalize-symbols}] \label{item:normalize-symbols}
  merges ``letter'' and ``symbol`` variants of some Greek letters
  (cf. Table\,\ref{tab:symbol-variant-macros} and
  section~\ref{sec:symbol-variants} below) to the ``letter'' character:
  Without this option, the symbol variant characters cannot be used in text
  under 8-bit LaTeX, because they are not supported by the Greek 8-bit font
  encoding LGR.

  \textbf{Attention}: Be careful in cases where the distinction between the
  symbol variants may be important (e.g. in a mathematical or scientific
  context). Use XeTeX/LuaTeX with Unicode fonts or the respective characters
  in mathematical mode (e.g. $\pi$ vs. $\varpi$).

\item[\texttt{keep-semicolon}]
  prevents conversion of the semicolon to an \emph{ano teleia} in 8-bit TeX
  (see \emph{\href{textalpha-doc.pdf}{textalpha-doc}}).
\end{description}
%
Both options are ignored in text set using Unicode fonts.


\subsection{symbol variants \label{sec:symbol-variants}}

Mathematical notation uses variant shapes of some Greek letters as
additional symbols. The variations have no syntactic meaning in Greek text
and text fonts may use the variant shapes in place of the “regular” ones as
a stylistic choice.

Unicode defines separate code points for the symbol variants. TeX supports
some of the variant shape symbols in mathematical mode, but its concept of
“standard” vs. “variant” symbols differs from the distinction between
“GREEK LETTER ...” vs. “GREEK ... SYMBOL” in the Unicode standard.

The \emph{alphabeta} package defines generic macros for these variants that
are short forms of the set defined in \file{tuenc-greek.def}
(cf. \href{test-tuenc-greek.pdf}{test-tuenc-greek}):
\begin{quote}
  \verb|\<name>| selects the Unicode GREEK LETTER ... variant,

  \verb|\<name>symbol| selects the Unicode
     GREEK ... SYMBOL variant,

  \verb|\var<name>| selects the variant
    shape according to TeX's mathematical mode
\end{quote}
See Table\,\ref{tab:symbol-variant-macros} at the end of this document for
the full list.


\section{Limitations \label{sec:limitations}}

With 8-bit TeX, the limitations described in the
\href{textalpha-doc.pdf}{textalpha documentation} apply
(see also section~\ref{sec:diacritics}).
These limitations do not apply, if the text language is switched to ``greek''
with Babel\footnote{\label{footnote:babel-greek}
  Setting the correct language for Greek text parts with the
  \pkgref{babel} package additionally ensures correct hyphenation and
  upcasing.},
the text part is wrapped in \verb+\ensuregreek+,
or set using Unicode fonts with XeTeX/LuaTeX.

With XeTeX/LuaTeX and Unicode fonts, literal Unicode characters cannot be
used in formulas (the log file reports missing characters) This is a generic
TeX limitation which \emph{alphabeta} overcomes if used under 8-bit TeX.
Under XeTeX/LuaTeX it may be circumvented using the \pkgref{unicode-math}
package. Mind, that \emph{unicode-math} conflicts with \emph{alphabeta}.


\section{Tests and examples}

\subsection{Greek alphabet}

Greek letters via generic ``name'' macros without language/font-encoding
switch (active font encoding \encodingdefault):

\begin{quote}
  \Alpha{} \Beta{} \Gamma{} \Delta{} \Epsilon{} \Zeta{} \Eta{} \Theta{}
  \Iota{} \Kappa{} \Lambda{} \Mu{} \Nu{} \Xi{} \Omicron{} \Pi{} \Rho{}
  \Sigma{} \Tau{} \Upsilon{} \Phi{} \Chi{} \Psi{} \Omega{}
  \quad
  \Digamma{} \Stigma{} \Koppa
    \footnote{In LGR, there is no separate glyph for uppercase Koppa.}
  \Qoppa{} \Sampi{}
  \\
  \alpha{} \beta{} \gamma{} \delta{} \epsilon{} \zeta{} \eta{} \theta{}
  \iota{} \kappa{} \lambda{} \mu{} \nu{} \xi{} \omicron{} \pi{} \rho{}
  \sigma{} \finalsigma{} \tau{} \upsilon{} \phi{} \chi{} \psi{} \omega{}
  \quad
  \digamma{} \stigma{} \varstigma
    \footnote{There is no separate Unicode code point for a stigma variant
              symbol, \cs{varstigma} is not defined with
              Xe/LuaTeX and similar to \cs{stigma} in some fonts.}
   \koppa{} \qoppa{} \sampi{}

\end{quote}
%
Greek letters via Unicode literals (active font encoding \encodingdefault):

\begin{quote}
  Α Β Γ Δ Ε Ζ Η Θ Ι Κ Λ Μ Ν Ξ Ο Π Ρ Σ   Τ Υ Φ Χ Ψ Ω \quad Ϝ Ϛ Ϟ Ϙ Ϡ
  \\
  α β γ δ ε ζ η θ ι κ λ μ ν ξ ο π ρ σ ς τ υ φ χ ψ ω \quad ϝ ϛ ϟ ϙ ϡ
\end{quote}


\subsection{Diacritics \label{sec:diacritics}}

According to Greek typographical conventions, diacritics (except the
dialytika and sub iota) are placed before capital letters and
dropped in UPPERCASE.
Since 2022, \cs{MakeUppercase} only drops diacritics from Greek literals
when the text language is set to \texttt{greek} with Babel or Polyglossia.
Diacritics input as standard accent macros are only dropped if the Greek
language is defined with Babel (i.e. not in this document).
For an example using \pkgref{babel-greek} see
\href{char-list-alphabeta.pdf}{char-list-alphabeta.pdf}).
%
\begin{quote}
  \greekscript
  \newcommand{\sample}{\<{\alpha} \>{\epsilon} \"'{\iota} \>`{\eta}
                       \'<{\omicron} \~<{\upsilon} \~>{\omega}}
  \sample{} →   \MakeUppercase{\sample}

  \renewcommand{\sample}{\<{\Alpha} \>{\Epsilon} \"{\Iota} \>`{\Eta}
                         \'<{\Omicron} \~<{\Upsilon} \~>{\Omega}}
  \sample{} →   \MakeUppercase{\sample}
\end{quote}
%
Certain \textbf{limitations} apply \textbf{if} Greek LICRs are
\textbf{used in a non-Greek font encoding} (e.g. T1).%
\ifdefined \UnicodeEncodingName
  \footnote{This document is typeset using Unicode fonts,
  	    for details see the version using 8-bit fonts
	    \href{alphabeta-doc.pdf}{alphabeta-doc.pdf}.}
\else
  \footnote{This document is typeset using  8-bit fonts.}

  \begin{itemize}

  \item Composition of diacritics (like \verb|\>\'| or \verb|\accpsili\accoxia|)
        fails: \\
        \"'{\iota} \>`\eta{}
        \'<{\omicron} \~<{\upsilon} \~>{\omega} (\currentEncoding) vs.
        \ensuregreek{\"'{\iota} \>`\eta{}
        \'<{\omicron} \~<{\upsilon} \~>{\omega} (\currentEncoding)}

        Simple diacritics and long names (like \verb+\accdasiaoxia+) work in
        any font encoding.

  \item Accent macros do not select precomposed characters
        (the subtle difference becomes obvious if you drag-and-drop text from
        the PDF version of this document):
        \\
        \accdasiaoxia\alpha{} (\currentEncoding) vs.
        \ensuregreek{\accdasiaoxia\alpha{} (\currentEncoding)}

  \item Wrong placement of diacritics on capital letters:
        \'\Alpha{} \accdasiaoxia\Omega{} (\currentEncoding) vs.
        \ensuregreek{\'\Alpha{} \accdasiaoxia\Omega{} (\currentEncoding)}

  \item Uppercasing characters with diacritics leads to compilation errors
        unless the base letter is put in braces,
        e.g., \verb|\MakeUppercase{\'\alpha}| fails,\\
              \verb|\MakeUppercase{\'{\alpha}}| works.
  \end{itemize}
  The \cs{ensuregreek} macro can be used to avoid these problems.
  It sets its argument with a font encoding supporting Greek.%
  \footref{footnote:babel-greek}
\fi


\subsection{normalize-symbols}

The \texttt{normalize-symbols} option merges ``letters'' and ``symbol``
variants of some Greek letters to the ``letter'' character. It is ignored,
if the document uses Unicode fonts and is compiled with XeTeX or LuaTeX
(this document is compiled using \engine).
\begin{quote}
  The source of this quote uses both variants for beta (β|ϐ),
  epsilon (ε|ϵ), phi (φ|ϕ), kappa (κ|ϰ), pi (π|ϖ), rho (ρ|ϱ), theta (θ|ϑ),
  and Theta (Θ|ϴ).
\end{quote}


\subsection{\ensuregreek{%
  \<\Epsilon\lambda\lambda\eta\nu\iota\kappa\'\alpha} in PDF strings}

With the alphabeta package, you get Greek letters in both, the document body
and PDF metadata generated by \pkgref{hyperref} if the input uses Unicode
literals or macros. Wrapping in \verb+\ensuregreek+ ensures the right
placement of the accents and breathings (before, not above capital letters).


\subsection{Greek in maths $\Gamma = \sin\alpha / \cos{\beta}$}

In the main document, Greek in ``math'' mode should work as usual:

\[\Gamma = \frac{\sin\alpha}{\cos{\beta}}.
\]
Greek letters and symbols in math mode, input as macro:\footnote{
  There are no math macros for Greek letters wich exist
  with similar shape in the Latin alphabet.}
\begin{align*}
  &
  % \Alpha{} \Beta{}
  \Gamma{} \Delta{}
  % \Epsilon{} \Zeta{} \Eta{}
  \Theta{}
  % \Iota{} \Kappa{}
  \Lambda{}
  % \Mu{} \Nu{}
  \Xi{}
  % \Omicron{}
  \Pi{}
  % \Rho{}
  \Sigma{}
  % \Tau{}
  \Upsilon{} \Phi{}
  % \Chi{}
  \Psi{} \Omega{}
\\&
  \alpha{} \beta{} \gamma{} \delta{} \epsilon{} \zeta{} \eta{} \theta{}
  \iota{} \kappa{} \lambda{} \mu{} \nu{} \xi{}
  % \omicron{}
  \pi{} \rho{}
  \sigma{} \varsigma{} \tau{} \upsilon{} \phi{} \chi{} \psi{} \omega{}
\\&
  \vartheta \varphi \varpi \digamma{} \varrho \varepsilon
\end{align*}
%
PDF strings do not know math mode. The content of a formula or equation is
evaluated in text mode with non-valid commands discarded (and warnings
written to the log). This works resonably well for simple formulas (but not,
e.g., for super-/subscripts). With the \emph{alphabeta} package, it works
also for Greek letters.


\subsection{Greek Unicode characters in math}

With 8-bit TeX and \pkgref{greek-inputenc}, literal Greek Unicode characters
are supported also in mathematical mode.
%
\ifdefined\DeclareUnicodeCharacter
  \[
       Γ = \frac{\sin α}{\cos β}.
  \]
  Greek letters and symbols in math mode, input as Unicode literals:
  \begin{align*}
               & Γ ΔΘΛΞΠΣΥ ΦΨ Ω \\
                 & αβγδεζηθικλμνξπρσςτυφχψω \\
               & ϑϕϖϝϱϵ
  \end{align*}
\fi
This does not work with XeTeX/LuaTeX (unless in 8-bit emulation mode).
Here, \pkgref{unicode-math} can be used instead of \emph{alphabeta}.

The ``normal'' vs. ``variant'' shape of letters is used
so that the output matches the Unicode reference glyph
(cf. Table \ref{tab:symbol-variant-macros}).
This corresponds to the behaviour of \pkgref{unicode-math}.


\section{Character Tables}

\subsection{Greek and Coptic}

\newcommand*{\GreekAndCopticI}{%
  \ensuregreek{%
    \'{}             % ΄
    \"'{}            % ΅
    \'\Alpha{}       % Ά
    \textanoteleia{} % ·
    \'\Epsilon{}     % Έ
    \'\Eta{}         % Ή
    \'\Iota{}        % Ί
    \'\Omicron{}     % Ό
    \'\Upsilon{}     % Ύ
    \'\Omega{}       % Ώ
}}
\newcommand*{\GreekAndCopticII}{%
  \ensuregreek{%
    \'"\iota{} % ΐ
    \Alpha{}         % Α
    \Beta{}          % Β
    \Gamma{}         % Γ
    \Delta{}         % Δ
    \Epsilon{}       % Ε
    \Zeta{}          % Ζ
    \Eta{}           % Η
    \Theta{}         % Θ
    \Iota{}          % Ι
    \Kappa{}         % Κ
    \Lambda{}        % Λ
    \Mu{}            % Μ
    \Nu{}            % Ν
    \Xi{}            % Ξ
    \Omicron{}       % Ο
}}
\newcommand*{\GreekAndCopticIII}{%
  \ensuregreek{%
    \Pi{}            % Π
    \Rho{}           % Ρ
    \Sigma{}         % Σ
    \Tau{}           % Τ
    \Upsilon{}       % Υ
    \Phi{}           % Φ
    \Chi{}           % Χ
    \Psi{}           % Ψ
    \Omega{}         % Ω
    \"\Iota{}        % Ϊ
    \"\Upsilon{}     % Ϋ
    \'\alpha{}       % ά
    \'\epsilon{}     % έ
    \'\eta{}         % ή
    \'\iota{}        % ί
}}
\newcommand*{\GreekAndCopticIV}{%
  \ensuregreek{%
    \"'\upsilon{}    % ΰ
    \alpha{}         % α
    \beta{}          % β
    \gamma{}         % γ
    \delta{}         % δ
    \epsilon{}       % ε
    \zeta{}          % ζ
    \eta{}           % η
    \theta{}         % θ
    \iota{}          % ι
    \kappa{}         % κ
    \lambda{}        % λ
    \mu{}            % μ
    \nu{}            % ν
    \xi{}            % ξ
    \omicron{}       % ο
}}
\newcommand*{\GreekAndCopticV}{%
  \ensuregreek{%
    \pi{}            % π
    \rho{}           % ρ
    \finalsigma{}    % ς
    \sigma{}         % σ
    \tau{}           % τ
    \upsilon{}       % υ
    \phi{}           % φ
    \chi{}           % χ
    \psi{}           % ψ
    \omega{}         % ω
    \"\iota{}        % ϊ
    \"\upsilon{}     % ϋ
    \'\omicron{}     % ό
    \'\upsilon{}     % ύ
    \'\omega{}       % ώ
}}
\newcommand*{\GreekAndCopticVI}{%
  \ensuregreek{%
    % x03D0
    \betasymbol{}    % ϐ
    \thetasymbol{}   % ϑ
    \phisymbol{}     % ϕ
    \pisymbol{}      % ϖ
    \Qoppa{}         % Ϙ
    \qoppa{}         % ϙ
    \Stigma{}        % Ϛ
    \stigma{}        % ϛ
    \Digamma{}       % Ϝ
    \digamma{}       % ϝ
    \Koppa{}         % Ϟ
    \koppa{}         % ϟ
}}
\newcommand*{\GreekAndCopticVII}{%
  \ensuregreek{%
    % x03E0
    \Sampi{}         % Ϡ
    \sampi{}         % ϡ
    % x03F0
    \kappasymbol{}   % ϰ
    \rhosymbol{}     % ϱ
    \Thetasymbol{}   % ϴ
    \epsilonsymbol{} % ϵ
}}

\begin{minipage}{0.48\linewidth}
  symbol accent macros

  \GreekAndCopticI

  \GreekAndCopticII

  \GreekAndCopticIII

  \GreekAndCopticIV

  \GreekAndCopticV

  \GreekAndCopticVI

  \GreekAndCopticVII
\end{minipage}
\hfill
\begin{minipage}{0.5\linewidth}
\noindent
named accent macros

\ensuregreek{%
% x0384
\acctonos{}                  % ΄
\accdialytikatonos{}         % ΅
\acctonos\Alpha{}            % Ά
\textanoteleia{}             % ·
\acctonos\Epsilon{}          % Έ
\acctonos\Eta{}              % Ή
\acctonos\Iota{}             % Ί
\acctonos\Omicron{}          % Ό
\acctonos\Upsilon{}          % Ύ
\acctonos\Omega{}            % Ώ

% x0390
\accdialytikatonos\iota{}    % ΐ
\Alpha{}                     % Α
\Beta{}                      % Β
\Gamma{}                     % Γ
\Delta{}                     % Δ
\Epsilon{}                   % Ε
\Zeta{}                      % Ζ
\Eta{}                       % Η
\Theta{}                     % Θ
\Iota{}                      % Ι
\Kappa{}                     % Κ
\Lambda{}                    % Λ
\Mu{}                        % Μ
\Nu{}                        % Ν
\Xi{}                        % Ξ
\Omicron{}                   % Ο

% 0x3A0
\Pi{}                        % Π
\Rho{}                       % Ρ
\Sigma{}                     % Σ
\Tau{}                       % Τ
\Upsilon{}                   % Υ
\Phi{}                       % Φ
\Chi{}                       % Χ
\Psi{}                       % Ψ
\Omega{}                     % Ω
\accdialytika\Iota{}         % Ϊ
\accdialytika\Upsilon{}      % Ϋ
\acctonos\alpha{}            % ά
\acctonos\epsilon{}          % έ
\acctonos\eta{}              % ή
\acctonos\iota{}             % ί

% x03B0
\accdialytikatonos\upsilon{} % ΰ
\alpha{}                     % α
\beta{}                      % β
\gamma{}                     % γ
\delta{}                     % δ
\epsilon{}                   % ε
\zeta{}                      % ζ
\eta{}                       % η
\theta{}                     % θ
\iota{}                      % ι
\kappa{}                     % κ
\lambda{}                    % λ
\mu{}                        % μ
\nu{}                        % ν
\xi{}                        % ξ
\omicron{}                   % ο

% x03C0
\pi{}                        % π
\rho{}                       % ρ
\finalsigma{}                % ς
\sigma{}                     % σ
\tau{}                       % τ
\upsilon{}                   % υ
\phi{}                       % φ
\chi{}                       % χ
\psi{}                       % ψ
\omega{}                     % ω
\accdialytika\iota{}         % ϊ
\accdialytika\upsilon{}      % ϋ
\acctonos\omicron{}          % ό
\acctonos\upsilon{}          % ύ
\acctonos\omega{}            % ώ

% x03D0
\betasymbol{}    % ϐ
\thetasymbol{}   % ϑ
\phisymbol{}     % ϕ
\pisymbol{}      % ϖ
\Qoppa{}         % Ϙ
\qoppa{}         % ϙ
\Stigma{}        % Ϛ
\stigma{}        % ϛ
\Digamma{}       % Ϝ
\digamma{}       % ϝ
\Koppa{}         % Ϟ
\koppa{}         % ϟ

% x03E0
\Sampi{}         % Ϡ
\sampi{}         % ϡ
% x03F0
\kappasymbol{}   % ϰ
\rhosymbol{}     % ϱ
\Thetasymbol{}   % ϴ
\epsilonsymbol{} % ϵ
} % end \ensuregreek
\end{minipage}


\subsection{Greek Extended}

\newcommand*{\GreekExtendedI}{%
  \ensuregreek{%
    \>\alpha{}
    \<\alpha{}
    \>`\alpha{}
    \<`\alpha{}
    \>'\alpha{}
    \<'\alpha{}
    \~>\alpha{}
    \~<\alpha{}
    \>\Alpha{}
    \<\Alpha{}
    \>`\Alpha{}
    \<`\Alpha{}
    \>'\Alpha{}
    \<'\Alpha{}
    \~>\Alpha{}
    \~<\Alpha{}
}}
\newcommand*{\GreekExtendedII}{%
  \ensuregreek{%
    \>\epsilon{}
    \<\epsilon{}
    \>`\epsilon{}
    \<`\epsilon{}
    \>'\epsilon{}
    \<'\epsilon{}
    \>\Epsilon{}
    \<\Epsilon{}
    \>`\Epsilon{}
    \<`\Epsilon{}
    \>'\Epsilon{}
    \<'\Epsilon{}
}}
\newcommand*{\GreekExtendedIII}{%
  \ensuregreek{%
    \>\eta{}
    \<\eta{}
    \>`\eta{}
    \<`\eta{}
    \>'\eta{}
    \<'\eta{}
    \~>\eta{}
    \~<\eta{}
    \>\Eta{}
    \<\Eta{}
    \>`\Eta{}
    \<`\Eta{}
    \>'\Eta{}
    \<'\Eta{}
    \~>\Eta{}
    \~<\Eta{}
}}
\newcommand*{\GreekExtendedIV}{%
  \ensuregreek{%
    \>\iota{}
    \<\iota{}
    \>`\iota{}
    \<`\iota{}
    \>'\iota{}
    \<'\iota{}
    \~>\iota{}
    \~<\iota{}
    \>\Iota{}
    \<\Iota{}
    \>`\Iota{}
    \<`\Iota{}
    \>'\Iota{}
    \<'\Iota{}
    \~>\Iota{}
    \~<\Iota{}
}}
\newcommand*{\GreekExtendedV}{%
  \ensuregreek{%
    \>\omicron{}
    \<\omicron{}
    \>`\omicron{}
    \<`\omicron{}
    \>'\omicron{}
    \<'\omicron{}
    \>\Omicron{}
    \<\Omicron{}
    \>`\Omicron{}
    \<`\Omicron{}
    \>'\Omicron{}
    \<'\Omicron{}
}}
\newcommand*{\GreekExtendedVI}{%
  \ensuregreek{%
    \>\upsilon{}
    \<\upsilon{}
    \>`\upsilon{}
    \<`\upsilon{}
    \>'\upsilon{}
    \<'\upsilon{}
    \~>\upsilon{}
    \~<\upsilon{}
    \<\Upsilon{}
    \<`\Upsilon{}
    \<'\Upsilon{}
    \~<\Upsilon{}
}}
\newcommand*{\GreekExtendedVII}{%
  \ensuregreek{%
    \>\omega{}
    \<\omega{}
    \>`\omega{}
    \<`\omega{}
    \>'\omega{}
    \<'\omega{}
    \~>\omega{}
    \~<\omega{}
    \>\Omega{}
    \<\Omega{}
    \>`\Omega{}
    \<`\Omega{}
    \>'\Omega{}
    \<'\Omega{}
    \~>\Omega{}
    \~<\Omega{}
}}
\newcommand*{\GreekExtendedVIII}{%
  \ensuregreek{%
    \`\alpha{}
    \'\alpha{}
    \`\epsilon{}
    \'\epsilon{}
    \`\eta{}
    \'\eta{}
    \`\iota{}
    \'\iota{}
    \`\omicron{}
    \'\omicron{}
    \`\upsilon{}
    \'\upsilon{}
    \`\omega{}
    \'\omega{}
}}
\newcommand*{\GreekExtendedIX}{%
  \ensuregreek{%
    \>\alpha\ypogegrammeni{}
    \<\alpha\ypogegrammeni{}
    \>`\alpha\ypogegrammeni{}
    \<`\alpha\ypogegrammeni{}
    \>'\alpha\ypogegrammeni{}
    \<'\alpha\ypogegrammeni{}
    \~>\alpha\ypogegrammeni{}
    \~<\alpha\ypogegrammeni{}
    \>\Alpha\ypogegrammeni{}
    \<\Alpha\ypogegrammeni{}
    \>`\Alpha\ypogegrammeni{}
    \<`\Alpha\ypogegrammeni{}
    \>'\Alpha\ypogegrammeni{}
    \<'\Alpha\ypogegrammeni{}
    \~>\Alpha\ypogegrammeni{}
    \~<\Alpha\ypogegrammeni{}
}}
\newcommand*{\GreekExtendedX}{%
  \ensuregreek{%
    \>\eta\ypogegrammeni{}
    \<\eta\ypogegrammeni{}
    \>`\eta\ypogegrammeni{}
    \<`\eta\ypogegrammeni{}
    \>'\eta\ypogegrammeni{}
    \<'\eta\ypogegrammeni{}
    \~>\eta\ypogegrammeni{}
    \~<\eta\ypogegrammeni{}
    \>\Eta\ypogegrammeni{}
    \<\Eta\ypogegrammeni{}
    \>`\Eta\ypogegrammeni{}
    \<`\Eta\ypogegrammeni{}
    \>'\Eta\ypogegrammeni{}
    \<'\Eta\ypogegrammeni{}
    \~>\Eta\ypogegrammeni{}
    \~<\Eta\ypogegrammeni{}
}}
\newcommand*{\GreekExtendedXI}{%
  \ensuregreek{%
    \>\omega\ypogegrammeni{}
    \<\omega\ypogegrammeni{}
    \>`\omega\ypogegrammeni{}
    \<`\omega\ypogegrammeni{}
    \>'\omega\ypogegrammeni{}
    \<'\omega\ypogegrammeni{}
    \~>\omega\ypogegrammeni{}
    \~<\omega\ypogegrammeni{}
    \>\Omega\ypogegrammeni{}\,%
    \<\Omega\ypogegrammeni{}\,%
    \>`\Omega\ypogegrammeni{}\,%
    \<`\Omega\ypogegrammeni{}\,%
    \>'\Omega\ypogegrammeni{}\,%
    \<'\Omega\ypogegrammeni{}\,%
    \~>\Omega\ypogegrammeni{}\,%
    \~<\Omega\ypogegrammeni{}
}}
\newcommand*{\GreekExtendedXII}{%
  \ensuregreek{%
    \u\alpha{}
    \=\alpha{}
    \`\alpha\ypogegrammeni{}
    \alpha\ypogegrammeni{}
    \'\alpha\ypogegrammeni{}
    \~\alpha{}
    \~\alpha\ypogegrammeni{}
    \u\Alpha{}
    \=\Alpha{}
    \`\Alpha{}
    \'\Alpha{}
    \Alpha\ypogegrammeni{}
    \>{}
    \prosgegrammeni{}
    \>{}
}}
\newcommand*{\GreekExtendedXIII}{%
  \ensuregreek{%
    \~{}
    \"\~{}
    \`\eta\ypogegrammeni{}
    \eta\ypogegrammeni{}
    \'\eta\ypogegrammeni{}
    \~\eta{}
    \~\eta\ypogegrammeni{}
    \`\Epsilon{}
    \'\Epsilon{}
    \`\Eta{}
    \'\Eta{}
    \Eta\ypogegrammeni{}
    \>`{}
    \>'{}
    \~>{}
}}
\newcommand*{\GreekExtendedXIV}{%
  \ensuregreek{%
    \u\iota{}
    \=\iota{}
    \`"\iota{}
    \'"\iota{}
    \~\iota{}
    \~"\iota{}
    \u\Iota{}
    \=\Iota{}
    \`\Iota{}
    \'\Iota{}
    \<`{}
    \<'{}
    \~<{}
}}
\newcommand*{\GreekExtendedXV}{%
  \ensuregreek{%
    \u\upsilon{}
    \=\upsilon{}
    \`"\upsilon{}
    \'"\upsilon{}
    \>\rho{}
    \<\rho{}
    \~\upsilon{}
    \~"\upsilon{}
    \u\Upsilon{}
    \=\Upsilon{}
    \`\Upsilon{}
    \'\Upsilon{}
    \<\Rho{}
    \`"{}
    \'"{}
    \`{}
}}
\newcommand*{\GreekExtendedXVI}{%
  \ensuregreek{%
    \`\omega\ypogegrammeni{}
    \omega\ypogegrammeni{}
    \'\omega\ypogegrammeni{}
    \~\omega{}
    \~\omega\ypogegrammeni{}
    \`\Omicron{}
    \'\Omicron{}
    \`\Omega{}
    \'\Omega{}
    \Omega\ypogegrammeni{}
    \'{}
    \<{}
}}


\begin{minipage}{0.5\linewidth}
  symbol accent macros

  \GreekExtendedI

  \GreekExtendedII

  \GreekExtendedIII

  \GreekExtendedIV

  \GreekExtendedV

  \GreekExtendedVI

  \GreekExtendedVII

  \GreekExtendedVIII

  \GreekExtendedIX

  \GreekExtendedX

  \GreekExtendedXI

  \GreekExtendedXII

  \GreekExtendedXIII

  \GreekExtendedXIV

  \GreekExtendedXV

  \GreekExtendedXVI
\end{minipage}
\hfill
\begin{minipage}{0.48\linewidth}
\noindent
named accent macros

\ensuregreek{%
\accpsili\alpha{}
\accdasia\alpha{}
\accpsilivaria\alpha{}
\accdasiavaria\alpha{}
\accpsilioxia\alpha{}
\accdasiaoxia\alpha{}
\accpsiliperispomeni\alpha{}
\accdasiaperispomeni\alpha{}
\accpsili\Alpha{}
\accdasia\Alpha{}
\accpsilivaria\Alpha{}
\accdasiavaria\Alpha{}
\accpsilioxia\Alpha{}
\accdasiaoxia\Alpha{}
\accpsiliperispomeni\Alpha{}
\accdasiaperispomeni\Alpha{}

\accpsili\epsilon{}
\accdasia\epsilon{}
\accpsilivaria\epsilon{}
\accdasiavaria\epsilon{}
\accpsilioxia\epsilon{}
\accdasiaoxia\epsilon{}
\accpsili\Epsilon{}
\accdasia\Epsilon{}
\accpsilivaria\Epsilon{}
\accdasiavaria\Epsilon{}
\accpsilioxia\Epsilon{}
\accdasiaoxia\Epsilon{}

\accpsili\eta{}
\accdasia\eta{}
\accpsilivaria\eta{}
\accdasiavaria\eta{}
\accpsilioxia\eta{}
\accdasiaoxia\eta{}
\accpsiliperispomeni\eta{}
\accdasiaperispomeni\eta{}
\accpsili\Eta{}
\accdasia\Eta{}
\accpsilivaria\Eta{}
\accdasiavaria\Eta{}
\accpsilioxia\Eta{}
\accdasiaoxia\Eta{}
\accpsiliperispomeni\Eta{}
\accdasiaperispomeni\Eta{}

\accpsili\iota{}
\accdasia\iota{}
\accpsilivaria\iota{}
\accdasiavaria\iota{}
\accpsilioxia\iota{}
\accdasiaoxia\iota{}
\accpsiliperispomeni\iota{}
\accdasiaperispomeni\iota{}
\accpsili\Iota{}
\accdasia\Iota{}
\accpsilivaria\Iota{}
\accdasiavaria\Iota{}
\accpsilioxia\Iota{}
\accdasiaoxia\Iota{}
\accpsiliperispomeni\Iota{}
\accdasiaperispomeni\Iota{}

\accpsili\omicron{}
\accdasia\omicron{}
\accpsilivaria\omicron{}
\accdasiavaria\omicron{}
\accpsilioxia\omicron{}
\accdasiaoxia\omicron{}
\accpsili\Omicron{}
\accdasia\Omicron{}
\accpsilivaria\Omicron{}
\accdasiavaria\Omicron{}
\accpsilioxia\Omicron{}
\accdasiaoxia\Omicron{}

\accpsili\upsilon{}
\accdasia\upsilon{}
\accpsilivaria\upsilon{}
\accdasiavaria\upsilon{}
\accpsilioxia\upsilon{}
\accdasiaoxia\upsilon{}
\accpsiliperispomeni\upsilon{}
\accdasiaperispomeni\upsilon{}
\accdasia\Upsilon{}
\accdasiavaria\Upsilon{}
\accdasiaoxia\Upsilon{}
\accdasiaperispomeni\Upsilon{}

\accpsili\omega{}
\accdasia\omega{}
\accpsilivaria\omega{}
\accdasiavaria\omega{}
\accpsilioxia\omega{}
\accdasiaoxia\omega{}
\accpsiliperispomeni\omega{}
\accdasiaperispomeni\omega{}
\accpsili\Omega{}
\accdasia\Omega{}
\accpsilivaria\Omega{}
\accdasiavaria\Omega{}
\accpsilioxia\Omega{}
\accdasiaoxia\Omega{}
\accpsiliperispomeni\Omega{}
\accdasiaperispomeni\Omega{}

\accvaria\alpha{}
\accoxia\alpha{}
\accvaria\epsilon{}
\accoxia\epsilon{}
\accvaria\eta{}
\accoxia\eta{}
\accvaria\iota{}
\accoxia\iota{}
\accvaria\omicron{}
\accoxia\omicron{}
\accvaria\upsilon{}
\accoxia\upsilon{}
\accvaria\omega{}
\accoxia\omega{}

\accpsili\alpha\ypogegrammeni{}
\accdasia\alpha\ypogegrammeni{}
\accpsilivaria\alpha\ypogegrammeni{}
\accdasiavaria\alpha\ypogegrammeni{}
\accpsilioxia\alpha\ypogegrammeni{}
\accdasiaoxia\alpha\ypogegrammeni{}
\accpsiliperispomeni\alpha\ypogegrammeni{}
\accdasiaperispomeni\alpha\ypogegrammeni{}
\accpsili\Alpha\ypogegrammeni{}
\accdasia\Alpha\ypogegrammeni{}
\accpsilivaria\Alpha\ypogegrammeni{}
\accdasiavaria\Alpha\ypogegrammeni{}
\accpsilioxia\Alpha\ypogegrammeni{}
\accdasiaoxia\Alpha\ypogegrammeni{}
\accpsiliperispomeni\Alpha\ypogegrammeni{}
\accdasiaperispomeni\Alpha\ypogegrammeni{}

\accpsili\eta\ypogegrammeni{}
\accdasia\eta\ypogegrammeni{}
\accpsilivaria\eta\ypogegrammeni{}
\accdasiavaria\eta\ypogegrammeni{}
\accpsilioxia\eta\ypogegrammeni{}
\accdasiaoxia\eta\ypogegrammeni{}
\accpsiliperispomeni\eta\ypogegrammeni{}
\accdasiaperispomeni\eta\ypogegrammeni{}
\accpsili\Eta\ypogegrammeni{}
\accdasia\Eta\ypogegrammeni{}
\accpsilivaria\Eta\ypogegrammeni{}
\accdasiavaria\Eta\ypogegrammeni{}
\accpsilioxia\Eta\ypogegrammeni{}
\accdasiaoxia\Eta\ypogegrammeni{}
\accpsiliperispomeni\Eta\ypogegrammeni{}
\accdasiaperispomeni\Eta\ypogegrammeni{}

\accpsili\omega\ypogegrammeni{}
\accdasia\omega\ypogegrammeni{}
\accpsilivaria\omega\ypogegrammeni{}
\accdasiavaria\omega\ypogegrammeni{}
\accpsilioxia\omega\ypogegrammeni{}
\accdasiaoxia\omega\ypogegrammeni{}
\accpsiliperispomeni\omega\ypogegrammeni{}
\accdasiaperispomeni\omega\ypogegrammeni{}
\accpsili\Omega\ypogegrammeni{}\,%
\accdasia\Omega\ypogegrammeni{}\,%
\accpsilivaria\Omega\ypogegrammeni{}\,%
\accdasiavaria\Omega\ypogegrammeni{}\,%
\accpsilioxia\Omega\ypogegrammeni{}\,%
\accdasiaoxia\Omega\ypogegrammeni{}\,%
\accpsiliperispomeni\Omega\ypogegrammeni{}\,%
\accdasiaperispomeni\Omega\ypogegrammeni{}

\u\alpha{}
\=\alpha{}
\accvaria\alpha\ypogegrammeni{}
\alpha\ypogegrammeni{}
\accoxia\alpha\ypogegrammeni{}
\accperispomeni\alpha{}
\accperispomeni\alpha\ypogegrammeni{}
\u\Alpha{}
\=\Alpha{}
\accvaria\Alpha{}
\accoxia\Alpha{}
\Alpha\ypogegrammeni{}
\accpsili{}
\prosgegrammeni{}
\accpsili{}

\accperispomeni{}
\accdialytikaperispomeni{}
\accvaria\eta\ypogegrammeni{}
\eta\ypogegrammeni{}
\accoxia\eta\ypogegrammeni{}
\accperispomeni\eta{}
\accperispomeni\eta\ypogegrammeni{}
\accvaria\Epsilon{}
\accoxia\Epsilon{}
\accvaria\Eta{}
\accoxia\Eta{}
\Eta\ypogegrammeni{}
\accpsilivaria{}
\accpsilioxia{}
\accpsiliperispomeni{}

\u\iota{}
\=\iota{}
\accdialytikavaria\iota{}
\accdialytikatonos\iota{}
\accperispomeni\iota{}
\accdialytikaperispomeni\iota{}
\u\Iota{}
\=\Iota{}
\accvaria\Iota{}
\accoxia\Iota{}
\accdasiavaria{}
\accdasiaoxia{}
\accdasiaperispomeni{}

\u\upsilon{}
\=\upsilon{}
\accdialytikavaria\upsilon{}
\accdialytikatonos\upsilon{}
\accpsili\rho{}
\accdasia\rho{}
\accperispomeni\upsilon{}
\accdialytikaperispomeni\upsilon{}
\u\Upsilon{}
\=\Upsilon{}
\accvaria\Upsilon{}
\accoxia\Upsilon{}
\accdasia\Rho{}
\accdialytikavaria{}
\accdialytikatonos{}
\accvaria{}

\accvaria\omega\ypogegrammeni{}
\omega\ypogegrammeni{}
\accoxia\omega\ypogegrammeni{}
\accperispomeni\omega{}
\accperispomeni\omega\ypogegrammeni{}
\accvaria\Omicron{}
\accoxia\Omicron{}
\accvaria\Omega{}
\accoxia\Omega{}
\Omega\ypogegrammeni{}
\accoxia{}
\accdasia{}
} % end \ensuregreek
\end{minipage}

\section{PDF Strings}

Generic LICRs and symbol accent macros.
(Check the ToC in the PDF sidebar to see how PDF strings are handled.)

\subsection{Greek and Coptic}

\subsubsection{\GreekAndCopticI}
\subsubsection{\GreekAndCopticII}
\subsubsection{\GreekAndCopticIII}
\subsubsection{\GreekAndCopticIV}
\subsubsection{\GreekAndCopticV}
\subsubsection{\GreekAndCopticVI}
\subsubsection{\GreekAndCopticVII}

\subsection{Greek Extended}

\subsubsection{\GreekExtendedI}
\subsubsection{\GreekExtendedII}
\subsubsection{\GreekExtendedIII}
\subsubsection{\GreekExtendedIV}
\subsubsection{\GreekExtendedV}
\subsubsection{\GreekExtendedVI}
\subsubsection{\GreekExtendedVII}
\subsubsection{\GreekExtendedVIII}
\subsubsection{\GreekExtendedIX}
\subsubsection{\GreekExtendedX}
\subsubsection{\GreekExtendedXI}
\subsubsection{\GreekExtendedXII}
\subsubsection{\GreekExtendedXIII}
\subsubsection{\GreekExtendedXIV}
\subsubsection{\GreekExtendedXV}
\subsubsection{\GreekExtendedXVI}

If this document is compiled with a post 2022 LaTeX and \emph{Babel},
a test for \cs{MakeUppercase} follows.

\ifdefined \AddToNoCaseChangeList
  % skip \MakeUppercase tests for LaTeX older than 2022/6

  \subsection{MakeUppercase}

  \ifdefined \extrasgreek
    \selectlanguage{greek} % must be done before the \section command
  \fi

  \subsubsection{\MakeUppercase{\GreekAndCopticI}}
  \subsubsection{\MakeUppercase{\GreekAndCopticII}}
  \subsubsection{\MakeUppercase{\GreekAndCopticIII}}
  \subsubsection{\MakeUppercase{\GreekAndCopticIV}}
  \subsubsection{\MakeUppercase{\GreekAndCopticV}}
  \subsubsection{\MakeUppercase{\GreekAndCopticVI}}
  \subsubsection{\MakeUppercase{\GreekAndCopticVII}}

  \subsubsection{\MakeUppercase{\GreekExtendedI}}
  \subsubsection{\MakeUppercase{\GreekExtendedII}}
  \subsubsection{\MakeUppercase{\GreekExtendedIII}}
  \subsubsection{\MakeUppercase{\GreekExtendedIV}}
  \subsubsection{\MakeUppercase{\GreekExtendedV}}
  \subsubsection{\MakeUppercase{\GreekExtendedVI}}
  \subsubsection{\MakeUppercase{\GreekExtendedVII}}
  \subsubsection{\MakeUppercase{\GreekExtendedVIII}}
  \subsubsection{\MakeUppercase{\GreekExtendedIX}}
  \subsubsection{\MakeUppercase{\GreekExtendedX}}
  \subsubsection{\MakeUppercase{\GreekExtendedXI}}
  \subsubsection{\MakeUppercase{\GreekExtendedXII}}
  \subsubsection{\MakeUppercase{\GreekExtendedXIII}}
  \subsubsection{\MakeUppercase{\GreekExtendedXIV}}
  \subsubsection{\MakeUppercase{\GreekExtendedXV}}
  \subsubsection{\MakeUppercase{\GreekExtendedXVI}}

  \ifdefined \extrasgreek
    \selectlanguage{english}
  \fi
\fi

\begin{table}[bp]
  \centering
  \begin{tabular}[t]{lcc}
    \hline
    macro                 & text           & math            \\
    \hline                                                   \\
    \verb$\beta$          & \beta          & $\beta$         \\
    \verb$\varbeta$       & \varbeta       & $\varbeta$      \\
    \verb$\betasymbol$    & \betasymbol    & $\betasymbol$   \\
    \hline                                                   \\
    \verb$\epsilon$       & \epsilon       & $\epsilon$      \\
    \verb$\varepsilon$    & \varepsilon    & $\varepsilon$   \\
    \verb$\epsilonsymbol$ & \epsilonsymbol & $\epsilonsymbol$\\
    \hline                                                   \\
    \verb$\phi$           & \phi           & $\phi$          \\
    \verb$\varphi$        & \varphi        & $\varphi$       \\
    \verb$\phisymbol$     & \phisymbol     & $\phisymbol$    \\
    \hline                                                   \\
    \verb$\kappa$         & \kappa         & $\kappa$        \\
    \verb$\varkappa$      & \varkappa      & $\varkappa$     \\
    \verb$\kappasymbol$   & \kappasymbol   & $\kappasymbol$  \\
    \hline                                                   \\
    \verb$\pi$            & \pi            & $\pi$           \\
    \verb$\varpi$         & \varpi         & $\varpi$        \\
    \verb$\pisymbol$      & \pisymbol      & $\pisymbol$     \\
    \hline                                                   \\
  % \end{tabular}
  % \begin{tabular}[t]{lcc}
  %   \hline
  %   macro                 & text           & math            \\
    % \hline                                                   \\
    \verb$\rho$           & \rho           & $\rho$          \\
    \verb$\varrho$        & \varrho        & $\varrho$       \\
    \verb$\rhosymbol$     & \rhosymbol     & $\rhosymbol$    \\
    \hline                                                   \\
    \verb$\sigma$         & \sigma         & $\sigma$        \\
    \verb$\varsigma$      & \varsigma      & $\varsigma$     \\
    \verb$\finalsigma$    & \finalsigma    & $\finalsigma$   \\
    \hline                                                   \\
    \verb$\theta$         & \theta         & $\theta$        \\
    \verb$\vartheta$      & \vartheta      & $\vartheta$     \\
    \verb$\thetasymbol$   & \thetasymbol   & $\thetasymbol$  \\
    \hline                                                   \\
    \verb$\Theta$         & \Theta         & $\Theta$        \\
    \verb$\varTheta$      & \varTheta      & $\varTheta$     \\
    \verb$\Thetasymbol$   & \Thetasymbol   & \missing        \\
    \hline                                                   \\
  \end{tabular}
  \caption{Macros for Greek \hyperref[sec:symbol-variants]{symbol variants}
    (\missing = symbol only available with additional packages).
    With 8-bit TeX and the
    \hyperref[item:normalize-symbols]{\texttt{normalize-symbols}} option,
    the output for both variants in text mode is the same (8-bit Greek text
    fonts contain only one symbol variant). \label{tab:symbol-variant-macros}}
\end{table}


\end{document}
