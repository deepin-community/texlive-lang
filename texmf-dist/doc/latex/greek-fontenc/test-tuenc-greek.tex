\documentclass[a4paper]{scrartcl}

\usepackage[unicode,colorlinks=true,linkcolor=blue]{hyperref}
\usepackage{bookmark}
\usepackage{booktabs}
\usepackage{amssymb}

% It is possible to use 8-bit Greek text fonts in the LGR TeX font encoding
% also with XeTeX/LuaTeX, if the »fontenc« package is loaded before
% »fontspec« and »textalpha«:
\usepackage[LGR,TU]{fontenc}
% Prevent a font substitution warning:
\DeclareFontFamilySubstitution{LGR}{FreeSerif(0)}{cmr}

% Unicode font setup:
\usepackage{fontspec}
% \setmainfont{DejaVu Serif}
% \setsansfont{DejaVu Sans}
% \usepackage{libertineotf}
% \setmainfont{Linux Libertine O}
\setsansfont[Renderer=Harfbuzz]{Linux Biolinum O}
% \setmainfont{cmunrm.otf} % CMU Serif % many missing characters :(
\setmainfont[Renderer=Harfbuzz]{FreeSerif}%
% \setmainfont{Droid Sans}

% Declare the font encoding and Greek LICR definitions:
\usepackage[normalize-symbols]{textalpha}

% For the new \MakeUppercase, we need to set the text language to Greek
% with Babel or Polyglossia:
\usepackage[greek,english]{babel}

\newcommand{\Greek}{
  \ifdefined\extrasgreek
    \selectlanguage{greek}
  \fi
}

% select the LGR font encoding and CB fonts
\providecommand*{\lgrfont}[1]{\leavevmode{%
  \fontfamily{cmr}\fontencoding{LGR}\selectfont#1}%
}

\newcommand{\pkgref}[1]{\emph{\href{https://ctan.org/pkg/#1}{#1}}}


\begin{document}

\title{Font setup for Greek with XeTeX/LuaTeX}
\author{Günter Milde}
\date{2020/10/30}
\maketitle

\noindent The file \href{tuenc-greek.def.html}{tuenc-greek.def} provides
support for Greek \hyperref[licr]{LICR} macros and upcasing of text with
XeTeX and LuaTeX. It is loaded automatically by
\href{textalpha-doc.pdf}{\emph{textalpha}},
\href{alphabeta-doc.pdf}{\emph{alphabeta}},
and \pkgref{babel-greek} when used with Unicode fonts
(LuaTeX or XeTeX with \pkgref{fontspec}).

\tableofcontents

\section{Requirements}

\subsection{\emph{fontspec} and suitable Unicode fonts}

LaTeX sets up the TU Unicode text font encoding if it detects the XeTeX or
LuaTeX engines. The user must ensure that the selected font contains Greek
glyphs (the default Latin Modern fonts have only capital Greek letters).
\textbf{There are no errors for missing glyphs}, just warnings in the log
file (but not in the console output) and empty spaces in the output
document.

The \pkgref{fontspec} package is the standard tool to select fonts in
XeTeX/LuaTeX. \pkgref{babel} provides a front-end to set
up language-specific fonts with the \verb|\babelfont| command.

\section{Usage}

\texttt{tuenc-greek.def} is loaded automatically by
\href{textalpha-doc.pdf}{\emph{textalpha}},
\href{alphabeta-doc.pdf}{\emph{alphabeta}}, or \pkgref{Babel} (with the
language option \href{https://ctan.org/pkg/babel-greek}{greek}) if these
packages are used with Unicode-aware TeX engines (XeTeX or LuaTeX). This
provides support for Greek LICR input and fixes for MakeUppercase
(cf. section~\ref{sec:uppercase-and-lowercase}).


\section{LICR input%
         \label{licr}}

The LaTeX internal character representation (LICR) can be used as a verbose,
fail-safe 7-bit ASCII encoding. It works under both, 8-bit
TeX and XeTeX/LuaTeX. Use cases are macro definitions and generated text.

See the source of this document,
\href{test-tuenc-greek.tex}{\texttt{test-tuenc-greek.tex}} for the input used
in the examples below.

\subsection{Greek alphabet}

Greek letters via LICR macros:
\begin{quote}
  \textAlpha{} \textBeta{} \textGamma{} \textDelta{} \textEpsilon{}
  \textZeta{} \textEta{} \textTheta{} \textIota{} \textKappa{}
  \textLambda{} \textMu{} \textNu{} \textXi{} \textOmicron{} \textPi{}
  \textRho{} \textSigma{} \textTau{} \textUpsilon{} \textPhi{}
  \textChi{} \textPsi{} \textOmega{}

  \textalpha{} \textbeta{} \textgamma{} \textdelta{} \textepsilon{}
  \textzeta{} \texteta{} \texttheta{} \textiota{} \textkappa{}
  \textlambda{} \textmu{} \textnu{} \textxi{} \textomicron{} \textpi{}
  \textrho{} \textsigma{} \texttau{} \textupsilon{}
  \textphi{} \textchi{} \textpsi{} \textomega{}
\end{quote}
The small sigma is set with a different
glyph if it ends a word:
\begin{quote}
  \textsigma{}       \verb|\textsigma|\\
  \textfinalsigma{}  \verb|\textfinalsigma| or \verb|\textvarsigma|
\end{quote}
The \verb|\textautosigma| currently works only with 8-bit LGR fonts:
\textautosigma\textiota{} vs. \textiota\textautosigma.


\subsection{Diacritics}

Greek diacritics can be input by named macro or symbol macro:
\begin{quote}
  \acctonos\textalpha       \'\textalpha{}  \acctonos       x\'x
  \accvaria\textalpha       \`\textalpha{}  \accvaria       x\`x
  \accdialytika\textiota    \"\textiota{}   \accdialytika   x\"x
  \accperispomeni\textalpha \~\textalpha{}  \accperispomeni x\~x
  \accpsili\textalpha       \>\textalpha{}  \accpsili       x\>x
  \accdasia\textalpha       \<\textalpha{}  \accdasia       x\<x
\end{quote}
%
XeTeX and LuaTeX with the Harfbuzz renderer normalise combining diacritics
with the base letter to the corresponding pre-composed Unicode character
(if such a character exists).%
\footnote{\label{fnt:overlap}
  Composite diacritics may overlap when they are not normalised to a
  pre-composed character.
  However, this is no problem in normal use as there are pre-composed
  Unicode characters for all letters/diacritic combinations that are
  actually used in the Greek language.%
}
\begin{quote}
  % άάὰὰϊϊᾶᾶἀἀἁἁ
  \acctonos α       \'α
  \accvaria α       \`α
  \accdialytika ι   \"ι
  \accperispomeni α \~α
  \accpsili α       \>α
  \accdasia α       \<α
\end{quote}

\subsubsection{perispomeni vs. tilde}

The Greek \emph{perispomeni} has the look of a tilde but the semantic of
a circumflex accent. The Unicode standard provides distinct code points.
Only <Greek letter> + COMBINING GREEK PERISPOMENI is considered equivalent
to the pre-composed character <Greek letter> WITH PERISPOMENI.

The \verb|\accperispomeni| macro uses COMBINING GREEK PERISPOMENI,
while the standard tilde-accent macro \verb|\~| uses the COMBINING TILDE.
Composite definitions for \verb|\~| select the pre-composed character:

\begin{quote}
  accent + base char / literal:
  \~α/ᾶ, \~η/ῆ, \~ι/ῖ, \~υ/ῦ, \~ω\ypogegrammeni{}/ῷ

  There are no pre-composed capital letters with perispomeni:
  \~Α, \~Η, \~Ι, \~Υ, \~ῼ \footref{fnt:overlap}
\end{quote}


\subsubsection{composite diacritics}

Composite diacritics are defined using two combining diacritical characters.%
\footref{fnt:overlap}

\begin{quote}
\accdialytikatonos\textiota{}     \"'\textiota{} \"\'\textiota{}
\accdialytikatonos x              \"'x           \"\'x
\accdialytikavaria\textiota{}     \"`\textiota{} \"\`\textiota{}
\accdialytikavaria x              \"`x           \"\`x
\accdialytikaperispomeni\textiota{} \~"\textiota{} \~\"\textiota{}
\accdialytikaperispomeni x        \~"x           \~\"x

\accdasiaoxia\textiota{}          \<'\textiota{} \<\'\textiota{}
\accdasiaoxia x                   \<'x           \<\'x
\accdasiavaria\textiota{}         \<`\textiota{} \<\`\textiota{}
\accdasiavaria x                  \<`x           \<\`x
\accdasiaperispomeni\textiota{}   \~<\textiota{} \~\<\textiota{}
\accdasiaperispomeni x            \~<x           \~\<x

\accpsilioxia\textiota{}          \>'\textiota{} \>\'\textiota{}
\accpsilioxia x                   \>'x           \>\'x
\accpsilivaria\textiota{}         \>`\textiota{} \>\`\textiota{}
\accpsilivaria x                  \>`x           \>\`x
\accpsiliperispomeni\textiota{}   \~>\textiota{} \~\>\textiota{}
\accpsiliperispomeni x            \~>x           \~\>x
\end{quote}

\subsubsection{sub-iota}

The sub-iota is input after the base letter.

\begin{itemize}
\item
  \verb|\ypogegrammeni| sets a COMBINING GREEK YPOGEGRAMMENI:
  \textalpha\ypogegrammeni{} k\ypogegrammeni{}.

  A Greek capital letter followed by COMBINING GREEK YPOGEGRAMMENI is
  normalised to the corresponding Greek capital letter WITH [... AND]
  PROSGEGRAMMENI, if a mapping exists in the Unicode standard.

\item \verb|\prosgegrammeni| sets a spacing GREEK PROSGEGRAMMENI:
  \textAlpha\prosgegrammeni{} K\prosgegrammeni{}.

  Spacing is better with the pre-composed characters for Greek capital
  letters \ldots{} WITH PROSGEGRAMMENI.

  Compare Αι (small letter iota) vs. Αι (spacing prosgegrammeni) vs. ᾼ
  (pre-composed).


\end{itemize}
%
Test letters with ypogegrammeni and prosgegrammeni (literal/LICR):

\begin{quote}
  % \Greek % compilation error, see section TODO below
  \begin{tabular}{lll}
    unchanged & make lowercase & make uppercase. \\

    ᾳαι/\textalpha\ypogegrammeni \textalpha\prosgegrammeni{} &
    \MakeLowercase{ᾳαι/\textalpha\ypogegrammeni \textalpha\prosgegrammeni} &
    \MakeUppercase{ᾳαι/\textalpha\ypogegrammeni \textalpha\prosgegrammeni} \\

    ᾼΑι/\textAlpha\ypogegrammeni \textAlpha\prosgegrammeni{} &
    \MakeLowercase{ᾼΑι/\textAlpha\ypogegrammeni \textAlpha\prosgegrammeni} &
    \MakeUppercase{ᾼΑι/\textAlpha\ypogegrammeni \textAlpha\prosgegrammeni} \\

    ΛͅΛι/\textLambda\ypogegrammeni \textLambda\prosgegrammeni{} &
    \MakeLowercase{ΛͅΛι/\textLambda\ypogegrammeni \textLambda\prosgegrammeni} &
    \MakeUppercase{ΛͅΛι/\textLambda\ypogegrammeni \textLambda\prosgegrammeni}
  \end{tabular}
\end{quote}


\subsection{Additional Greek symbols}

\subsubsection{symbols for Greek numbers}

\begin{quote}
\raggedright
\textkoppa{}      textkoppa                 \\ % ϟ
\textKoppa{}      textKoppa                 \\ % Ϟ
\textqoppa{}      textqoppa (archaic koppa) \\ % ϙ
\textQoppa{}      textQoppa (archaic Koppa) \\ % Ϙ
\textstigma{}     textstigma                \\ % ϛ
% \textvarstigma{}  textvarstigma \\ % no separate Unicode character
\textStigma{}     textStigma (Sigma-Tau-Ligature in CB-fonts)%
\footnote{the name “stigma” originally applied to a medieval sigma-tau
         ligature, whose shape was confusingly similar to the cursive
         digamma}                           \\ % Ϛ
\textsampi{}      textsampi                 \\ % ϡ
\textSampi{}      textSampi                 \\ % Ϡ
\textdigamma{}    textdigamma               \\ % ϝ (\digamma used by amsmath!)
\textDigamma{}    textDigamma               \\ % Ϝ
% numeral signs: http://en.wikipedia.org/wiki/Greek_numerals
\textdexiakeraia{}    textdexiakeraia       \\ % ʹ
\textaristerikeraia{} textaristerikeraia    \\ % ͵
\end{quote}

\subsubsection{symbol variants}

Mathematical notation uses variant shapes of some Greek letters as
additional symbols. The variations have no syntactic meaning in Greek text
and text fonts may use the variant shapes in place of the “regular” ones as
a stylistic choice.

Unicode defines separate code points for the symbol variants. TeX supports
some of the variant shape symbols in mathematical mode, but its concept of
“standard” vs. “variant” symbols differs from the distinction between
“GREEK LETTER ...” vs. “GREEK ... SYMBOL” in the Unicode standard (see
Table \ref{tab:symbol-variants}).

\begin{table}[tbp]
  \centering
  \begin{tabular}{cccc}
  \hline
  \multicolumn{2}{c}{TeX math} & \multicolumn{2}{c}{Unicode} \\
  symbol & var symbol & “letter” & “symbol” \\
  \hline
  $\pi$      & $\varpi$       & π & ϖ \\
  $\rho$     & $\varrho$      & ρ & ϱ \\
  $\theta$   & $\vartheta$    & θ & ϑ \\
  \hline
  $\epsilon$ & $\varepsilon$  & ε & ϵ \\
  $\phi$     & $\varphi$      & φ & ϕ \\
  \hline
  $\beta$    & \emph{missing} & β & ϐ \\
  $\kappa$   & \emph{missing} & κ & ϰ \\
  $\Theta$   & \emph{missing} & Θ & ϴ \\
  \hline
  \end{tabular}
  \caption{Greek symbol variants in TeX and Unicode}
  \label{tab:symbol-variants}
\end{table}

\texttt{tuenc-greek.def} defines three TextCommands for each of these
letters:
\begin{quote}
  \verb|\text<name>| selects the Unicode GREEK LETTER ... variant,

  \verb|\text<name>symbol| selects the Unicode
     GREEK ... SYMBOL variant,

  \verb|\textvar<name>| selects the variant
    shape according to TeX's mathematical mode
\end{quote}
See Table \ref{tab:symbol-variant-macros} for the full list. The
\href{alphabeta-doc.pdf}{\emph{alphabeta}} package defines short macros that
work in text and math mode.

\begin{table}[tbp]
  \centering
  \begin{tabular}{lclc}
  \hline
  \multicolumn{2}{c}{text} & \multicolumn{2}{c}{mathematics} \\
  macro & output & macro & output \\
  \hline
  \verb$\textbeta$          & \textbeta          & \verb$\beta$       & $\beta$       \\
  \verb$\textvarbeta$       & \textvarbeta       & \emph{missing}     & \\
  \verb$\textbetasymbol$    & \textbetasymbol    &                    & \\
  \hline
  \verb$\textepsilon$       & \textepsilon       & \verb$\epsilon$    & $\epsilon$    \\
  \verb$\textvarepsilon$    & \textvarepsilon    & \verb$\varepsilon$ & $\varepsilon$ \\
  \verb$\textepsilonsymbol$ & \textepsilonsymbol &                    & \\
  \hline
  \verb$\texttheta$         & \texttheta         & \verb$\theta$      & $\theta$      \\
  \verb$\textvartheta$      & \textvartheta      & \verb$\vartheta$   & $\vartheta$   \\
  \verb$\textthetasymbol$   & \textthetasymbol   &                    & \\
  \hline
  \verb$\textTheta$         & \textTheta         & \verb$\Theta$      & $\Theta$      \\
  \verb$\textvarTheta$      & \textvarTheta      & \emph{missing}     & \\
  \verb$\textThetasymbol$   & \textThetasymbol   &                    & \\
  \hline
  \verb$\textkappa$         & \textkappa         & \verb$\kappa$      & $\kappa$      \\
  \verb$\textvarkappa$      & \textvarkappa      & \verb$\varkappa$   & $\varkappa$   \\
  \verb$\textkappasymbol$   & \textkappasymbol   &                    & \\
  \hline
  \verb$\textpi$            & \textpi            & \verb$\pi$         & $\pi$         \\
  \verb$\textvarpi$         & \textvarpi         & \verb$\varpi$      & $\varpi$      \\
  \verb$\textpisymbol$      & \textpisymbol      &                    & \\
  \hline
  \verb$\textrho$           & \textrho           & \verb$\rho$        & $\rho$        \\
  \verb$\textvarrho$        & \textvarrho        & \verb$\varrho$     & $\varrho$     \\
  \verb$\textrhosymbol$     & \textrhosymbol     &                    & \\
  \hline
  \verb$\textphi$           & \textphi           & \verb$\phi$        & $\phi$        \\
  \verb$\textvarphi$        & \textvarphi        & \verb$\varphi$     & $\varphi$     \\
  \verb$\textphisymbol$     & \textphisymbol     &                    & \\
  \hline
  \end{tabular}
  \caption{Macros for Greek symbol variants}
  \label{tab:symbol-variant-macros}
\end{table}

\subsubsection{Ancient Greek Numbers}

Ancient Greek Numbers are missing in most fonts (including Libertine and
Deja Vu). The “FreeSerif” font works fine:
\begin{quote}
\textpentedeka    % GREEK ACROPHONIC ATTIC FIFTY
\textpentehekaton % GREEK ACROPHONIC ATTIC FIVE HUNDRED
\textpenteqilioi  % GREEK ACROPHONIC ATTIC FIVE THOUSAND
\textpentemuria   % GREEK ACROPHONIC ATTIC FIFTY THOUSAND
\end{quote}
If the LGR font encoding is loaded via «fontenc» in the document preamble,
Ancient Greek Numbers (as well as any other character) from LGR encoded
8-bit TeX fonts can be used after a font-encoding switch. (This document
defines the \verb|\lgrfont| command for this purpose in the preamble.)
\begin{quote}
\lgrfont{
\textpentedeka    % GREEK ACROPHONIC ATTIC FIFTY
\textpentehekaton % GREEK ACROPHONIC ATTIC FIVE HUNDRED
\textpenteqilioi  % GREEK ACROPHONIC ATTIC FIVE THOUSAND
\textpentemuria   % GREEK ACROPHONIC ATTIC FIFTY THOUSAND
}
\end{quote}


\subsubsection{generic text symbols}

There are some LICR macros for symbols from the 8-bit font encoding LGR
that are not confined to Greek but missing in
\texttt{tuenc.def} [2021/04/29 v2.0v] in TeXLive 23.

\begin{quote}
  \textsemicolon{} textsemicolon\\
  \textmicro{} textmicro \\
  \textschwa{} textschwa
\end{quote}
The SI unit prefix MICRO SIGN is not upcased with MakeUppercase:

\begin{quote}
  textmu: \textmu{} $\mapsto$ \MakeUppercase{\textmu} but
  textmicro: \textmicro{} $\mapsto$ \MakeUppercase{\textmicro}.
\end{quote}


\section{Latin transliteration}

The Latin transliteration known from LGR encoded 8-bit fonts%
\footnote{See the \pkgref{teubner} package or \pkgref{babel-greek}
          for a description.}
does not work with Unicode fonts.

For LuaTeX, there is a \texttt{transliteration.omega} \emph{transform} that
applies the transliteration system devised by Yannis Haralambous for the
Omega system (cf. the Babel documentation for the
\href{https://latex3.github.io/babel/guides/locale-greek.html}{Greek locale}).

It is possible to set up LGR encoded fonts parallel to Unicode fonts (see
the preamble of the source file \url{test-tuenc-greek.tex} for an example).
After switching the font encoding to LGR, Greek text can be input
via a \emph{Latin transliteration}, e.g. «logos» becomes «\lgrfont{logos}»
and «\texttt{>aupn'ia}» becomes «\lgrfont{>aupn\'ia}».

Mark that you cannot use Unicode input with LGR encoded fonts except when
running in 8-bit compatibility mode.
LICR macros work in both, TU and LGR: compare
\>\textIota\textalpha\textnu\textomicron\textupsilon\textalpha
\textrho\acctonos\textiota\textomicron\textupsilon{} (TU) vs.
\lgrfont{\>\textIota\textalpha\textnu\textomicron\textupsilon\textalpha
         \textrho\acctonos\textiota\textomicron\textupsilon} (LGR).


\section{UPPERCASE and lowercase \label{sec:uppercase-and-lowercase}}

According to Greek typesetting conventions, Greek diacritics (except the
dialytika and sub-iota) are placed to the left (instead of above) capital
letters and dropped if text is set in UPPERCASE, e.g. \ensuregreek{μαΐστρος
$\mapsto$ \MakeUppercase{μαΐστρος}}.

The 2022 \verb|\MakeUppercase| implementation
(cf. \textsf{ltnews35.pdf}) relies on Unicode data.
It upcases literal characters according to the Greek conventions
\textbf{if the text language is set to Greek} with \pkgref{babel}
or \pkgref{polyglossia}.
For the pre-2022/06 implementation, \texttt{tuenc-greek.def} contains
uccode/lccode corrections (taken from Apostolos Syropoulos \pkgref{xgreek}
package) to get the same effect (independent of the text language).

In addition, \texttt{tuenc-greek.def} contains code to drop Greek diacritics
input as accent macros.
However, when using the symbol macros (\verb|\' \` \~|) for tonos, varia,
and perispomeni, special definitions from \pkgref{babel-greek} ≥ 1.13.3 are
required to distinguish them from acute, grave, and tilde accents that must
be kept on Latin letters.

Named Greek accents and symbol accents on Greek vs. Latin letter:
\begin{quote}
  \Greek
  \newcommand{\sample}{
   \acctonos\textupsilon\'\textupsilon{}  \acctonos u\'u
   \accvaria\textupsilon\`\textupsilon{}  \accvaria u\`u}
  \sample → \MakeUppercase{\sample}
\end{quote}
Standard symbol accents on Latin letters are kept (OK). With pre-2023 LaTeX
versions or babel-greek < 1.13.3, standard accents are also kept on Greek
letters (sic!).


\subsection{hiatus}

Tonos and psili mark a \emph{hiatus} (break-up of a diphthong) if placed on
the first of two consecutive vowels (άι, άυ, έι, ἀυ, ἄι, ἄυ, ἔι).
A dialytika must be placed on the second vowel if they are dropped:
\begin{quote}
  % from teubner: άυλος/ΑΫΛΟΣ
  \acctonos\textalpha\textupsilon λος $\mapsto$
  \MakeUppercase{\acctonos\textalpha\textupsilon λος},
  \accpsilioxia\textalpha\textupsilon λος $\mapsto$
  \MakeUppercase{\accpsilioxia\textalpha\textupsilon λος},

  % from http://diacritics.typo.cz/index.php?id=69  μάινα -> ΜΑΪΝΑ
  m\acctonos\textalpha\textiota να $\mapsto$
  \MakeUppercase{m\acctonos\textalpha\textiota να},
  % from  http://de.wikipedia.org/wiki/Neugriechische_Orthographie#Das_Trema
  % κέικ, ἀυπνία/αϋπνία
  \textkappa\acctonos\textepsilon\textiota\textkappa $\mapsto$
  \MakeUppercase{\textkappa\acctonos\textepsilon\textiota\textkappa},
  \accpsili\textalpha\textupsilon πνία $\mapsto$
  \MakeUppercase{\accpsili\textalpha\textupsilon πνία}.
  % from https://r12a.github.io/scripts/grek/el.html#transforms
  % νεράιδα -> ΝΕΡΑΪΔΑ
  νεράιδα → \MakeUppercase{νεράιδα}
\end{quote}
%
Since 2023/02/10, this works with all literals
\begin{quote}
  \Greek
  \newcommand*{\hiatest}{άι, άυ, έι, ἄι, ἄυ, ἔι, ἀυ}
  \hiatest{} $\mapsto$ \MakeUppercase{\hiatest},
\end{quote}
with named accent-macro + Unicode literals:
\begin{quote}
  \Greek
  \newcommand*{\hiatest}{
    \acctonos αι, \acctonos αυ, \acctonos ει,
    \accpsilioxia αι, \accpsilioxia αυ, \accpsilioxia ει, \accpsili αυ}
  \hiatest{} $\mapsto$ \MakeUppercase{\hiatest}
\end{quote}
and, with babel-greek 1.13.3 with short accent-macro + Unicode literals:
\begin{quote}
  \Greek
  \newcommand*{\hiatest}{
    \'αι, \'αυ, \'ει, \'>αι, \'>αυ, \'>ει, \>αυ}
  \hiatest{} $\mapsto$ \MakeUppercase{\hiatest}
\end{quote}

\section{Character Tables}

The file \texttt{char-list.tex} contains a listing of all Greek
Unicode characters supported by the \emph{greek-fontenc} package.
See \texttt{char-list-tu.tex} for the output with XeTeX/LuaTeX.

\section{TODO}

Compilation error with MakeUppercase and combining ypogegrammeni in Greek
locale.

% \begin{quote}
%   \Greek
%   Λͅ → \MakeUppercase{Λͅ} % fails with new XeTeX / LuaTeX
%
%   % \~α\ypogegrammeni{} →
%   % \MakeUppercase{\~α\ypogegrammeni{}} % fails with old LuaTeX (but not XeTeX)
% \end{quote}

\end{document}
