\documentclass[a4paper]{article}
% \usepackage{textcomp}
\usepackage[unicode,colorlinks=true,linkcolor=blue]{hyperref}
% \usepackage{bookmark}
% \usepackage{parskip}
\usepackage{booktabs}
\usepackage{amssymb}

% It is possible to use 8-bit Greek text fonts in the LGR TeX font encoding
% also with XeTeX/LuaTeX, if the »fontenc« package is loaded before
% »fontspec« and »textalpha«:
\usepackage[LGR]{fontenc}

% Unicode font setup:
\usepackage{fontspec}
% \setmainfont{DejaVu Serif}
% \setsansfont{DejaVu Sans}
% \usepackage{libertineotf}
\setmainfont{Linux Libertine O}
\setsansfont{Linux Biolinum O}
% \setmainfont{cmunrm.otf} % CMU Serif % many missing characters :(
% \setmainfont{FreeSerif}%
% \setmainfont{Droid Sans}

% Declare the font encoding and Greek LICR definitions:
\usepackage[normalize-symbols]{textalpha}

\begin{document}

\title{Font setup for Greek with XeTeX/LuaTeX}
\author{Günter Milde}
\date{2020/10/30}
\maketitle

\noindent The file \href{tuenc-greek.def.html}{tuenc-greek.def} provides
support for Greek \hyperref[licr]{LICR} macros and upcasing of text with
XeTeX and LuaTeX. It is loaded automatically by
\href{textalpha-doc.pdf}{\emph{textalpha}},
\href{alphabeta-doc.pdf}{\emph{alphabeta}}, and
\href{http://www.ctan.org/pkg/babel-greek}{\emph{babel-greek}} when used
with Unicode fonts (LuaTeX or XeTeX with
\href{http://www.ctan.org/pkg/fontspec}{\emph{fontspec}}).

\tableofcontents

\section{Requirements}

\subsection{\emph{fontspec} and suitable Unicode fonts}

LaTeX sets up the TU Unicode text font encoding if it detects the XeTeX or
LuaTeX engiges. The user must ensure that the selected font contains Greek
glyphs (the default Latin Modern fonts have only capital Greek letters).
\textbf{There are no errors for missing glyphs, just warnings in the log
file (but not in the console output) and empty spaces in the output
document.}

The \href{http://www.ctan.org/pkg/fontspec}{\emph{fontspec}} package is the
standard tool to select fonts in Xe/LuaLaTeX. Examples for suitable fonts
are given in the
\href{http://mirrors.ctan.org/language/greek/greek-fontenc/greek-fontenc.html#TU}
{greek-fontenc documentation}.


\section{Usage}

\texttt{tuenc-greek.def} is usually not loaded directly, but by one of
\href{textalpha-doc.pdf}{\emph{textalpha}},
\href{alphabeta-doc.pdf}{\emph{alphabeta}}, or \emph{Babel} (with the
language option \href{http://www.ctan.org/pkg/babel-greek}{greek}). If these
packages are used with Unicode-aware TeX engines (XeTeX or LuaTeX), Unicode
font setup is amended for use of the Greek script.


\section{LICR input%
         \label{licr}}

The LaTeX internal character representation (LICR) is a verbose,
fail-safe 7-bit ASCII encoding that can be used unaltered under both, 8-bit
TeX and XeTeX/LuaTeX. Use cases are macro definitions and generated text.

See the source of this document,
\href{test-tuenc-greek.tex}{\texttt{test-tuenc-greek.tex}} for the input used
in the examples below.

\subsection{Greek alphabet}

Greek letters via LICR macros:
\begin{quote}
  \textAlpha{} \textBeta{} \textGamma{} \textDelta{} \textEpsilon{}
  \textZeta{} \textEta{} \textTheta{} \textIota{} \textKappa{}
  \textLambda{} \textMu{} \textNu{} \textXi{} \textOmicron{} \textPi{}
  \textRho{} \textSigma{} \textTau{} \textUpsilon{} \textPhi{}
  \textChi{} \textPsi{} \textOmega{}

  \textalpha{} \textbeta{} \textgamma{} \textdelta{} \textepsilon{}
  \textzeta{} \texteta{} \texttheta{} \textiota{} \textkappa{}
  \textlambda{} \textmu{} \textnu{} \textxi{} \textomicron{} \textpi{}
  \textrho{} \textsigma{} \texttau{} \textupsilon{}
  \textphi{} \textchi{} \textpsi{} \textomega{}
\end{quote}
The small sigma is set with a different
glyph if it ends a word:
\begin{quote}
  \textsigma{}       \verb|textsigma|\\
  \textfinalsigma{}  \verb|textfinalsigma| or \verb|textvarsigma|
\end{quote}
The \verb|\textautosigma| macro, which automatically chooses the
glyph according to the position, is not implemented for Unicode fonts.

\subsection{Diacritics}

Greek diacritics can be input by named macro or symbol macro:
\begin{quote}
  \acctonos\textalpha       \'\textalpha{}  \acctonos       x\'x
  \accvaria\textalpha       \`\textalpha{}  \accvaria       x\`x
  \accdialytika\textiota    \"\textiota{}   \accdialytika   x\"x
  \accperispomeni\textalpha \~\textalpha{}  \accperispomeni x\~x
  \accpsili\textalpha       \>\textalpha{}  \accpsili       x\>x
  \accdasia\textalpha       \<\textalpha{}  \accdasia       x\<x
\end{quote}
%
XeTeX normalizes base letter and combining diacritics to the
corresponding pre-composed character if such a mapping is defined in the
Unicode standard.

\begin{quote}
  % άάὰὰϊϊᾶᾶἀἀἁἁ
  \acctonos α       \'α
  \accvaria α       \`α
  \accdialytika ι   \"ι
  \accperispomeni α \~α
  \accpsili α       \>α
  \accdasia α       \<α
\end{quote}


\subsubsection{perispomeni vs. tilde}

The Greek \emph{perispomeni} has the look of a tilde but the semantic
of a circumflex accent.
The ``named'' \verb|\accperispomeni| macro uses COMBINING GREEK PERISPOMENI,
while the standard tilde-accent macro \verb|\~| uses the COMBINING TILDE
which is not normalized to GREEK LETTER ... WITH PERISPOMENI
characters.

Composite definitions for \verb|\~| select the pre-composed character:

\begin{quote}
  \~α = ᾶ, \~η = ῆ, \~ι = ῖ, \~υ = ῦ, \~ω = ῶ
\end{quote}


\subsubsection{combined diacritics}

Combined accents are defined using combining diacritical characters.

\begin{quote}
\accdialytikatonos\textiota{}     \"'\textiota{} \"\'\textiota{}
\accdialytikatonos x              \"'x           \"\'x
\accdialytikavaria\textiota{}     \"`\textiota{} \"\`\textiota{}
\accdialytikavaria x              \"`x           \"\`x
\accdialytikaperispomeni\textiota{} \~"\textiota{} \~\"\textiota{}
\accdialytikaperispomeni x        \~"x           \~\"x

\accdasiaoxia\textiota{}          \<'\textiota{} \<\'\textiota{}
\accdasiaoxia x                   \<'x           \<\'x
\accdasiavaria\textiota{}         \<`\textiota{} \<\`\textiota{}
\accdasiavaria x                  \<`x           \<\`x
\accdasiaperispomeni\textiota{}   \~<\textiota{} \~\<\textiota{}
\accdasiaperispomeni x            \~<x           \~\<x

\accpsilioxia\textiota{}          \>'\textiota{} \>\'\textiota{}
\accpsilioxia x                   \>'x           \>\'x
\accpsilivaria\textiota{}         \>`\textiota{} \>\`\textiota{}
\accpsilivaria x                  \>`x           \>\`x
\accpsiliperispomeni\textiota{}   \~>\textiota{} \~\>\textiota{}
\accpsiliperispomeni x            \~>x           \~\>x
\end{quote}
Composite diacritics overlap when they are not normalized to a pre-composed
character. However, this is not a major problem in normal use as
pre-composed characters exist in Unicode for all letters that are
used with diacritics in (ancient, polytonic or monotonic) Greek.

\subsubsection{sub-iota}

The sub-iota is input after the base letter.

\begin{itemize}
\item 
  \verb|\ypogegrammeni| sets a COMBINING GREEK YPOGEGRAMMENI:
  \textalpha\ypogegrammeni{} k\ypogegrammeni{}.

  A Greek capital letter followed by COMBINING GREEK YPOGEGRAMMENI is
  normalized to the corresponding Greek capital letter WITH [... AND]
  PROSGEGRAMMENI, if a mapping exists in the Unicode standard (by XeTeX but
  not by LuaTeX)

\item \verb|\prosgegrammeni| sets a spacing GREEK PROSGEGRAMMENI:
  \textAlpha\prosgegrammeni{} K\prosgegrammeni{}.

  Spacing is better with the pre-composed characters for Greek capital
  letters \ldots{} WITH PROSGEGRAMMENI.

  Compare Αι (small letter iota) vs. Αι (spacing prosgegrammeni) vs. ᾼ
  (pre-composed).
  

\end{itemize}
%
Test letters with ypogegrammeni and prosgegrammeni (literal/LICR):

\begin{quote}
\begin{tabbing}

  unchanged \quad\=  make lowercase\quad\=  make uppercase. \\

  ᾳαι/\textalpha\ypogegrammeni \textalpha\prosgegrammeni{} \>
  \MakeLowercase{ᾳαι/\textalpha\ypogegrammeni \textalpha\prosgegrammeni} \>
  \MakeUppercase{ᾳαι/\textalpha\ypogegrammeni \textalpha\prosgegrammeni} \\

  ᾼΑι/\textAlpha\ypogegrammeni \textAlpha\prosgegrammeni{} \>
  \MakeLowercase{ᾼΑι/\textAlpha\ypogegrammeni \textAlpha\prosgegrammeni} \>
  \MakeUppercase{ᾼΑι/\textAlpha\ypogegrammeni \textAlpha\prosgegrammeni} \\

  ΛͅΛι/\textLambda\ypogegrammeni \textLambda\prosgegrammeni{} \>
  \MakeLowercase{ΛͅΛι/\textLambda\ypogegrammeni \textLambda\prosgegrammeni} \>
  \MakeUppercase{ΛͅΛι/\textLambda\ypogegrammeni \textLambda\prosgegrammeni}

\end{tabbing}
\end{quote}

\subsection{Additional Greek symbols}

\subsubsection{symbols for Greek numbers}

\begin{quote}
\textkoppa{}      textkoppa                 \\ % ϟ
\textKoppa{}      textKoppa                 \\ % Ϟ
\textqoppa{}      textqoppa (archaic koppa) \\ % ϙ
\textQoppa{}      textQoppa (archaic Koppa) \\ % Ϙ
\textstigma{}     textstigma                \\ % ϛ
% \textvarstigma{}  textvarstigma \\ % no separate Unicode character
\textStigma{}     textStigma (Sigma-Tau-Ligature in CB-fonts)%
\footnote{the name “stigma” originally applied to a medieval sigma-tau
         ligature, whose shape was confusingly similar to the cursive
         digamma}                      \\ % Ϛ
\textsampi{}      textsampi  \\ % ϡ
\textSampi{}      textSampi  \\ % Ϡ
\textdigamma{}    textdigamma  \\ % ϝ (\digamma used by amsmath!)
\textDigamma{}    textDigamma  \\ % Ϝ
% numeral signs: http://en.wikipedia.org/wiki/Greek_numerals
\textdexiakeraia{}    textdexiakeraia  \\ % ʹ
\textaristerikeraia{} textaristerikeraia \\ % ͵
\end{quote}

\subsubsection{symbol variants}

Mathematical notation uses variant shapes of some Greek letters as
additional symbols. The variations have no syntactic meaning in Greek text
and text fonts may use the variant shapes in place of the “regular” ones as
a stylistic choice.

Unicode defines separate code points for the symbol variants. TeX supports
some of the variant shape symbols in mathematical mode, but its concept of
“standard” vs. “variant” symbols differs from the distinction between
“GREEK LETTER ...” vs. “GREEK ... SYMBOL” in the Unicode standard (see
Table \ref{tab:symbol-variants}).

\begin{table}[tbp]
  \centering
  \begin{tabular}{cccc}
  \hline
  \multicolumn{2}{c}{TeX math} & \multicolumn{2}{c}{Unicode} \\
  symbol & var symbol & “letter” & “symbol” \\
  \hline
  $\pi$      & $\varpi$       & π & ϖ \\
  $\rho$     & $\varrho$      & ρ & ϱ \\
  $\theta$   & $\vartheta$    & θ & ϑ \\
  \hline
  $\epsilon$ & $\varepsilon$  & ε & ϵ \\
  $\phi$     & $\varphi$      & φ & ϕ \\
  \hline
  $\beta$    & \emph{missing} & β & ϐ \\
  $\kappa$   & \emph{missing} & κ & ϰ \\
  $\Theta$   & \emph{missing} & Θ & ϴ \\
  \hline
  \end{tabular}
  \caption{Greek symbol variants in TeX and Unicode}
  \label{tab:symbol-variants}
\end{table}

\texttt{tuenc-greek.def} defines three TextCommands for each of these
letters:
\begin{quote}
  \verb|\text<name>| selects the Unicode GREEK LETTER ... variant,

  \verb|\text<name>symbol| selects the Unicode
     GREEK ... SYMBOL variant,

  \verb|\textvar<name>| selects the variant
    shape according to TeX' mathematical mode
\end{quote}
See Table \ref{tab:symbol-variant-macros} for the full list. The
\href{alphabeta-doc.pdf}{\emph{alphabeta}} package defines short macros that
work in text and math mode.

\begin{table}[tbp]
  \centering
  \begin{tabular}{lclc}
  \hline
  \multicolumn{2}{c}{text} & \multicolumn{2}{c}{mathematics} \\
  macro & output & macro & output \\
  \hline
  \verb$\textpi$            & \textpi            & \verb$\pi$         & $\pi$         \\
  \verb$\textvarpi$         & \textvarpi         & \verb$\varpi$      & $\varpi$      \\
  \verb$\textpisymbol$      & \textpisymbol      &                    & \\
  \hline
  \verb$\textrho$           & \textrho           & \verb$\rho$        & $\rho$        \\
  \verb$\textvarrho$        & \textvarrho        & \verb$\varrho$     & $\varrho$     \\
  \verb$\textrhosymbol$     & \textrhosymbol     &                    & \\
  \hline
  \verb$\texttheta$         & \texttheta         & \verb$\theta$      & $\theta$      \\
  \verb$\textvartheta$      & \textvartheta      & \verb$\vartheta$   & $\vartheta$   \\
  \verb$\textthetasymbol$   & \textthetasymbol   &                    & \\
  \hline
  \verb$\textepsilon$       & \textepsilon       & \verb$\epsilon$    & $\epsilon$    \\
  \verb$\textvarepsilon$    & \textvarepsilon    & \verb$\varepsilon$ & $\varepsilon$ \\
  \verb$\textepsilonsymbol$ & \textepsilonsymbol &                    & \\
  \hline
  \verb$\textphi$           & \textphi           & \verb$\phi$        & $\phi$        \\
  \verb$\textvarphi$        & \textvarphi        & \verb$\varphi$     & $\varphi$     \\
  \verb$\textphisymbol$     & \textphisymbol     &                    & \\
  \hline
  \verb$\textbeta$          & \textbeta          & \verb$\beta$       & $\beta$       \\
  \verb$\textvarbeta$       & \textvarbeta       & \emph{missing}     & \\
  \verb$\textbetasymbol$    & \textbetasymbol    &                    & \\
  \hline
  \verb$\textkappa$         & \textkappa         & \verb$\kappa$      & $\kappa$      \\
  \verb$\textvarkappa$      & \textvarkappa      & \verb$\varkappa$   & $\varkappa$   \\
  \verb$\textkappasymbol$   & \textkappasymbol   &                    & \\
  \hline
  \verb$\textTheta$         & \textTheta         & \verb$\Theta$      & $\Theta$      \\
  \verb$\textvarTheta$      & \textvarTheta      & \emph{missing}     & \\
  \verb$\textThetasymbol$   & \textThetasymbol   &                    & \\
  \hline
  \end{tabular}
  \caption{Macros for Greek symbol variants}
  \label{tab:symbol-variant-macros}
\end{table}

\subsubsection{Ancient Greek Numbers}

Ancient Greek Numbers are missing in most fonts (including Libertine and
Deja Vu). The “FreeSerif” font works fine:
\begin{quote}
\textpentedeka    % GREEK ACROPHONIC ATTIC FIFTY
\textpentehekaton % GREEK ACROPHONIC ATTIC FIVE HUNDRED
\textpenteqilioi  % GREEK ACROPHONIC ATTIC FIVE THOUSAND
\textpentemuria   % GREEK ACROPHONIC ATTIC FIFTY THOUSAND
\end{quote}
If the LGR font encoding is loaded via «fontenc» in the document preamble,
Ancient Greek Numbers (as well as any other character) from LGR encoded
8-bit TeX fonts can be used after a font-encoding switch. babel-greek
defines the \verb|\textgreek| command for this purpose.
\providecommand*{\textgreek}[1]{\leavevmode{%
  \fontfamily{cmr}\fontencoding{LGR}\selectfont#1}%
}
\begin{quote}
\textgreek{
\textpentedeka    % GREEK ACROPHONIC ATTIC FIFTY
\textpentehekaton % GREEK ACROPHONIC ATTIC FIVE HUNDRED
\textpenteqilioi  % GREEK ACROPHONIC ATTIC FIVE THOUSAND
\textpentemuria   % GREEK ACROPHONIC ATTIC FIFTY THOUSAND
}
\end{quote}


\subsubsection{generic text symbols}

There are some LICR macros for some symbols from the 8-bit font encoding LGR
that are not confined to Greek but not defined in
\texttt{tuenc.def} [2018/08/11 v2.0j].

\begin{quote}
  \textsemicolon{} textsemicolon\\
  \textmicro{} textmicro \\
  \textschwa{} textschwa
\end{quote}
The SI unit prefix MICRO SIGN is not upcased with MakeUppercase:

\begin{quote}
  textmu: \textmu{} $\mapsto$ \MakeUppercase{\textmu} but
  textmicro: \textmicro{} $\mapsto$ \MakeUppercase{\textmicro}.
\end{quote}


\section{Latin transcription}

The Latin transcription known from LGR encoded 8-bit fonts%
\footnote{ See the \href{http://www.ctan.org/pkg/teubner}{teubner} package
or the file usage.pdf from the
\href{http://www.ctan.org/pkg/babel-greek}{\emph{babel-greek}} package for a
description.}
does not work with Unicode fonts.

It is possible to set up LGR encoded fonts parallel to Unicode fonts (see
the preamble of the source file \url{test-tuenc-greek.tex} for an example).
The \verb|\textgreek| macro can then be used for the input of Greek letters
via the \emph{Latin transcription}, e.g. «logos» becomes «\textgreek{logos}»
and «\verb|\>aupn\'ia|» becomes «\textgreek{\>aupn\'ia}».

Mark that you cannot use Unicode input with LGR encoded fonts except when
running in 8-bit compatibility mode.
LICR macros work in both, Unicode font encoding and LGR: compare
           \>\textIota\textalpha\textnu\textomicron\textupsilon\textalpha
           \textrho\acctonos\textiota\textomicron\textupsilon{}
(Unicode font set up via fontspec) vs.
\textgreek{\>\textIota\textalpha\textnu\textomicron\textupsilon\textalpha
           \textrho\acctonos\textiota\textomicron\textupsilon}
(LGR-encoded 8-bit font set up via NFSS commands).


\section{UPPERCASE and lowercase}

Capital Greek letters have Greek diacritics (except the dialytika and
sub-iota) to the left (instead of above) and drop them if text is set in
UPPERCASE, e.g.
\ensuregreek{μαΐστρος $\mapsto$ \MakeUppercase{μαΐστρος}}.

The uccode/lccode corrections (taken from Apostolos Syropoulos xgreek
package) ensure dropping of accents with \verb|\MakeUppercase| for literal
Unicode characters.

@uclclist additions ensure that upcasing also drops Greek diacritics.
However, when the tonos, varia, and perispomeni accents
are input using the symbol macros (\verb|\' \` \~|), this does not work, as
they cannot be distinguished from Latin acute, grave, and tilde accents.%
\footnote{This might be fixed with \textbackslash accACUTE,
          \textbackslash accGRAVE, and  \textbackslash accTILDE definitions
          with corresponding @uclclist entries and composite definitions.}
If these accents should be dropped by MakeUppercase, they must be input as
named macro:

\begin{quote}
\acctonos\textalpha       \'\textalpha{}  \acctonos a\'a
\accvaria\textalpha       \`\textalpha{}  \accvaria a\`a
$\mapsto$
\MakeUppercase{
\acctonos\textalpha       \'\textalpha{}  \acctonos a\'a
\accvaria\textalpha       \`\textalpha{}  \accvaria a\`a
}
\end{quote}

\subsubsection{hiatus}

Tonos and dasia mark a \emph{hiatus} (break-up of a diphthong) if placed on
the first vowel of a diphtong (άι, άυ, έι, ἄι, ἄυ, ἔι). A dialytika must be
placed on the second vowel if they are dropped.

The «hiatus» feature works with macro input:
\begin{quote}
  % from teubner: άυλος/ΑΫΛΟΣ
  \acctonos\textalpha\textupsilon λος $\mapsto$
  \MakeUppercase{\acctonos\textalpha\textupsilon λος},
  \accpsilioxia\textalpha\textupsilon λος $\mapsto$
  \MakeUppercase{\accpsilioxia\textalpha\textupsilon λος},

  % from http://diacritics.typo.cz/index.php?id=69  μάινα -> ΜΑΪΝΑ
  m\acctonos\textalpha\textiota να $\mapsto$
  \MakeUppercase{m\acctonos\textalpha\textiota να},
  % from  http://de.wikipedia.org/wiki/Neugriechische_Orthographie#Das_Trema
  % κέικ, ἀυπνία/αϋπνία
  \textkappa\acctonos\textepsilon\textiota\textkappa $\mapsto$
  \MakeUppercase{\textkappa\acctonos\textepsilon\textiota\textkappa},
  \accpsili\textalpha\textupsilon πνία $\mapsto$
  \MakeUppercase{\accpsili\textalpha\textupsilon πνία}.
\end{quote}

It does not work with Unicode literals:
\begin{quote}
  άι, άυ, έι, ἄι, ἄυ, ἔι $\mapsto$ \MakeUppercase{άι, άυ, έι, ἄι, ἄυ, ἔι}
\end{quote}

or accent-macro + Unicode literals (yet?):
\begin{quote}
  \acctonos αι, \acctonos αυ, \acctonos ει, \'>αι, \'>αυ, \'>ει
  $\mapsto$
  \MakeUppercase{\acctonos αι, \acctonos αυ, \acctonos ει, \'>αι, \'>αυ, \'>ει}
\end{quote}

\section{Character Tables}

The following tables list the Greek Unicode characters. In the input, the
LICR macro is followed by the corresponding literal Unicode character.

\subsection{Greek and Coptic Unicode block}

Seldom used characters that are not part of LGR encoded TeX fonts have no
LICR definition.

\newcommand{\greekandcoptic}{
% NR    Unicode Name                                       %   babel name, UCS name
% 0370  GREEK CAPITAL LETTER HETA                          % Ͱ
% 0371  GREEK SMALL LETTER HETA                            % ͱ
% 0372  GREEK CAPITAL LETTER ARCHAIC SAMPI                 % Ͳ
% 0373  GREEK SMALL LETTER ARCHAIC SAMPI                   % Ͳ
\textnumeralsigngreek ʹ % \anwtonos, \textdexiakeraia
\textnumeralsignlowergreek ͵ % \katwtonos, \textaristerikeraia,
% 0376  GREEK CAPITAL LETTER PAMPHYLIAN DIGAMMA            % Ͷ
% 0377  GREEK SMALL LETTER PAMPHYLIAN DIGAMMA              % ͷ
{ }\ypogegrammeni ͺ % \textsubiota{\empty}
% 037B  GREEK SMALL REVERSED LUNATE SIGMA SYMBOL           %
% 037C  GREEK SMALL DOTTED LUNATE SIGMA SYMBOL             %
% 037D  GREEK SMALL REVERSED DOTTED LUNATE SIGMA SYMBOL    %
\texterotimatiko ; % \textquestion

\acctonos{ } ΄
\"'{ } ΅
\'\textAlpha Ά
\textanoteleia · % \anoteleia
\'\textEpsilon Έ
\'\textEta Ή
\'\textIota Ί
\'\textOmicron Ό
\'\textUpsilon Ύ
\'\textOmega Ώ

\'"\textiota ΐ
\textAlpha Α
\textBeta Β
\textGamma Γ
\textDelta Δ
\textEpsilon Ε
\textZeta Ζ
\textEta Η
\textTheta Θ
\textIota Ι
\textKappa Κ
\textLambda Λ
\textMu Μ
\textNu Ν
\textXi Ξ
\textOmicron Ο

\textPi Π
\textRho Ρ
\textSigma Σ
\textTau Τ
\textUpsilon Υ
\textPhi Φ
\textChi Χ
\textPsi Ψ
\textOmega Ω
\"\textIota Ϊ
\"\textUpsilon Ϋ
\'\textalpha ά
\'\textepsilon έ
\'\texteta ή
\'\textiota ί

\"'\textupsilon ΰ
\textalpha α
\textbeta β
\textgamma γ
\textdelta δ
\textepsilon ε
\textzeta ζ
\texteta η
\texttheta θ
\textiota ι
\textkappa κ
\textlambda λ
\textmu μ % \textmugreek
\textnu ν
\textxi ξ
\textomicron ο

\textpi π
\textrho ρ
\textvarsigma ς
\textsigma σ
\texttau τ
\textupsilon υ
\textphi φ
\textchi χ
\textpsi ψ
\textomega ω
\"\textiota ϊ
\"\textupsilon ϋ
\'\textomicron ό
\'\textupsilon ύ
\'\textomega ώ

\textbetasymbol ϐ
\textthetasymbol ϑ
ϒ ϓ ϔ
\textphisymbol ϕ
\textpisymbol ϖ %ϗ

\textQoppa Ϙ
\textqoppa ϙ
\textStigma Ϛ
\textstigma ϛ
\textDigamma Ϝ
\textdigamma ϝ
\textKoppa Ϟ % \textKoppagreek
\textkoppa ϟ % \koppa, \textqoppa [sic!]

\textSampi Ϡ
\textsampi ϡ

% Ϣ ϣ Ϥ ϥ Ϧ ϧ Ϩ ϩ Ϫ ϫ Ϭ ϭ Ϯ ϯ

\textkappasymbol ϰ
\textrhosymbol ϱ
% ϲ
% ϳ
\textThetasymbol ϴ
\textepsilonsymbol ϵ
% ϶
% Ϸ ϸ
% Ϲ
% Ϻ ϻ ϼ
% Ͻ Ͼ Ͽ
}

\greekandcoptic

\noindent MakeUppercase (note: standard accents not dropped):

\MakeUppercase{\greekandcoptic}

\noindent MakeLowercase:

\MakeLowercase{\greekandcoptic}

The lowercase of Σ is σ (GREEK SMALL LETTER SIGMA).\footnote{%
With LICRs, it is \texttt{\textbackslash textautosigma}.}
The lowercase of Ϛ (GREEK LETTER STIGMA) is ϛ (GREEK SMALL LETTER STIGMA).


\subsection{Greek Extended Unicode block}

Note: There are no LICR definitions for spacing diacritical characters.
Use the corresponding accent macro with an empty argument or a space.

% \accvaria\textalpha    ὰ
% \accoxia\textalpha    ά

\medskip

\newcommand{\greekextended}{
\>\textalpha   ἀ
\<\textalpha   ἁ
\>`\textalpha  ἂ
\<`\textalpha  ἃ
\>'\textalpha  ἄ
\<'\textalpha  ἅ
\~>\textalpha  ἆ
\~<\textalpha  ἇ
\>\textAlpha   Ἀ
\<\textAlpha   Ἁ
\>`\textAlpha  Ἂ
\<`\textAlpha  Ἃ
\>'\textAlpha  Ἄ
\<'\textAlpha  Ἅ
\~>\textAlpha  Ἆ
\~<\textAlpha  Ἇ

\>\textepsilon   ἐ
\<\textepsilon   ἑ
\>`\textepsilon  ἒ
\<`\textepsilon  ἓ
\>'\textepsilon  ἔ
\<'\textepsilon  ἕ
\>\textEpsilon   Ἐ
\<\textEpsilon   Ἑ
\>`\textEpsilon  Ἒ
\<`\textEpsilon  Ἓ
\>'\textEpsilon  Ἔ
\<'\textEpsilon  Ἕ

\>\texteta   ἠ
\<\texteta   ἡ
\>`\texteta  ἢ
\<`\texteta  ἣ
\>'\texteta  ἤ
\<'\texteta  ἥ
\~>\texteta  ἦ
\~<\texteta  ἧ
\>\textEta   Ἠ
\<\textEta   Ἡ
\>`\textEta  Ἢ
\<`\textEta  Ἣ
\>'\textEta  Ἤ
\<'\textEta  Ἥ
\~>\textEta  Ἦ
\~<\textEta  Ἧ

\>\textiota   ἰ
\<\textiota   ἱ
\>`\textiota  ἲ
\<`\textiota  ἳ
\>'\textiota  ἴ
\<'\textiota  ἵ
\~>\textiota  ἶ
\~<\textiota  ἷ
\>\textIota   Ἰ
\<\textIota   Ἱ
\>`\textIota  Ἲ
\<`\textIota  Ἳ
\>'\textIota  Ἴ
\<'\textIota  Ἵ
\~>\textIota  Ἶ
\~<\textIota  Ἷ

\>\textomicron   ὀ
\<\textomicron   ὁ
\>`\textomicron  ὂ
\<`\textomicron  ὃ
\>'\textomicron  ὄ
\<'\textomicron  ὅ
\>\textOmicron   Ὀ
\<\textOmicron   Ὁ
\>`\textOmicron  Ὂ
\<`\textOmicron  Ὃ
\>'\textOmicron  Ὄ
\<'\textOmicron  Ὅ

\>\textupsilon   ὐ
\<\textupsilon   ὑ
\>`\textupsilon  ὒ
\<`\textupsilon  ὓ
\>'\textupsilon  ὔ
\<'\textupsilon  ὕ
\~>\textupsilon  ὖ
\~<\textupsilon  ὗ
\<\textUpsilon   Ὑ
\<`\textUpsilon  Ὓ
\<'\textUpsilon  Ὕ
\~<\textUpsilon  Ὗ

\>\textomega   ὠ
\<\textomega   ὡ
\>`\textomega  ὢ
\<`\textomega  ὣ
\>'\textomega  ὤ
\<'\textomega  ὥ
\~>\textomega  ὦ
\~<\textomega  ὧ
\>\textOmega   Ὠ
\<\textOmega   Ὡ
\>`\textOmega  Ὢ
\<`\textOmega  Ὣ
\>'\textOmega  Ὤ
\<'\textOmega  Ὥ
\~>\textOmega  Ὦ
\~<\textOmega  Ὧ

\accvaria\textalpha   ὰ
\accoxia\textalpha    ά
\accvaria\textepsilon ὲ
\accoxia\textepsilon  έ
\accvaria\texteta     ὴ
\accoxia\texteta      ή
\accvaria\textiota    ὶ
\accoxia\textiota     ί
\accvaria\textomicron ὸ
\accoxia\textomicron  ό
\accvaria\textupsilon ὺ
\accoxia\textupsilon  ύ
\accvaria\textomega   ὼ
\accoxia\textomega    ώ

\>\textalpha\ypogegrammeni   ᾀ
\<\textalpha\ypogegrammeni   ᾁ
\>`\textalpha\ypogegrammeni  ᾂ
\<`\textalpha\ypogegrammeni  ᾃ
\>'\textalpha\ypogegrammeni  ᾄ
\<'\textalpha\ypogegrammeni  ᾅ
\~>\textalpha\ypogegrammeni  ᾆ
\~<\textalpha\ypogegrammeni  ᾇ
\>\textAlpha\ypogegrammeni   ᾈ
\<\textAlpha\ypogegrammeni   ᾉ
\>`\textAlpha\ypogegrammeni  ᾊ
\<`\textAlpha\ypogegrammeni  ᾋ
\>'\textAlpha\ypogegrammeni  ᾌ
\<'\textAlpha\ypogegrammeni  ᾍ
\~>\textAlpha\ypogegrammeni  ᾎ
\~<\textAlpha\ypogegrammeni  ᾏ

\>\texteta\ypogegrammeni     ᾐ
\<\texteta\ypogegrammeni     ᾑ
\>`\texteta\ypogegrammeni    ᾒ
\<`\texteta\ypogegrammeni    ᾓ
\>'\texteta\ypogegrammeni    ᾔ
\<'\texteta\ypogegrammeni    ᾕ
\~>\texteta\ypogegrammeni    ᾖ
\~<\texteta\ypogegrammeni    ᾗ
\>\textEta\ypogegrammeni     ᾘ
\<\textEta\ypogegrammeni     ᾙ
\>`\textEta\ypogegrammeni    ᾚ
\<`\textEta\ypogegrammeni    ᾛ
\>'\textEta\ypogegrammeni    ᾜ
\<'\textEta\ypogegrammeni    ᾝ
\~>\textEta\ypogegrammeni    ᾞ
\~<\textEta\ypogegrammeni    ᾟ

\>\textomega\ypogegrammeni   ᾠ
\<\textomega\ypogegrammeni   ᾡ
\>`\textomega\ypogegrammeni  ᾢ
\<`\textomega\ypogegrammeni  ᾣ
\>'\textomega\ypogegrammeni  ᾤ
\<'\textomega\ypogegrammeni  ᾥ
\~>\textomega\ypogegrammeni  ᾦ
\~<\textomega\ypogegrammeni  ᾧ
\>\textOmega\ypogegrammeni   ᾨ
\<\textOmega\ypogegrammeni   ᾩ
\>`\textOmega\ypogegrammeni  ᾪ
\<`\textOmega\ypogegrammeni  ᾫ
\>'\textOmega\ypogegrammeni  ᾬ
\<'\textOmega\ypogegrammeni  ᾭ
\~>\textOmega\ypogegrammeni  ᾮ
\~<\textOmega\ypogegrammeni  ᾯ

\u\textalpha                      ᾰ
\=\textalpha                      ᾱ
\accvaria\textalpha\ypogegrammeni ᾲ
\textalpha\ypogegrammeni          ᾳ
\accoxia\textalpha\ypogegrammeni  ᾴ
\accperispomeni\textalpha         ᾶ
\accperispomeni\textalpha\ypogegrammeni ᾷ
\u\textAlpha                      Ᾰ
\=\textAlpha                      Ᾱ
\accvaria\textAlpha               Ὰ
\accoxia\textAlpha                Ά
\textAlpha\ypogegrammeni          ᾼ
\>{ }                             ᾽
{ }\prosgegrammeni                ι
\>{ }                             ᾿

\accperispomeni{ }                ῀
\"\~{ }                           ῁
\accvaria\texteta\ypogegrammeni   ῂ
\texteta\ypogegrammeni            ῃ
\accoxia\texteta\ypogegrammeni    ῄ
\accperispomeni\texteta           ῆ
\accperispomeni\texteta\ypogegrammeni ῇ
\accvaria\textEpsilon             Ὲ
\accoxia\textEpsilon              Έ
\accvaria\textEta                 Ὴ
\accoxia\textEta                  Ή
\textEta\ypogegrammeni            ῌ
\>`{ }                            ῍
\>'{ }                            ῎
\~>{ }                            ῏

\u\textiota                       ῐ
\=\textiota                       ῑ
\`"\textiota                      ῒ
\'"\textiota                      ΐ
\accperispomeni\textiota          ῖ
\accperispomeni"\textiota         ῗ
\u\textIota                       Ῐ
\=\textIota                       Ῑ
\accvaria\textIota                Ὶ
\accoxia\textIota                 Ί
  \<`{ }                          ῝
\>'{ }                            ῞
\~<{ }                            ῟

\u\textupsilon                    ῠ
\=\textupsilon                    ῡ
\`"\textupsilon                   ῢ
\'"\textupsilon                   ΰ
\>\textrho                        ῤ
\<\textrho                        ῥ
\accperispomeni\textupsilon       ῦ
\accperispomeni"\textupsilon      ῧ
\u\textUpsilon                    Ῠ
\=\textUpsilon                    Ῡ
\accvaria\textUpsilon             Ὺ
\accoxia\textUpsilon              Ύ
\<\textRho                        Ῥ
\`"{ }                            ῭
\'"{ }                            ΅
\accvaria{ }                      `

\accvaria\textomega\ypogegrammeni ῲ
\textomega\ypogegrammeni          ῳ
\accoxia\textomega\ypogegrammeni  ῴ
\accperispomeni\textomega         ῶ
\accperispomeni\textomega\ypogegrammeni ῷ
\accvaria\textOmicron             Ὸ
\accoxia\textOmicron              Ό
\accvaria\textOmega               Ὼ
\accoxia\textOmega                Ώ
\textOmega\ypogegrammeni          ῼ
\accoxia{ }                       ´
\<{ }                             ῾
}

\greekextended

\noindent MakeUppercase:

\MakeUppercase{\greekextended}

\noindent MakeLowercase:

\MakeLowercase{\greekextended}

\end{document}
