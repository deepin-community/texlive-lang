\documentclass[twoside]{report}
\pagestyle{headings}
\begin{document}
\chapter[Floodings at Venice]{Floodings at Venice over Centuries}
\section[A Problem's Solution]{Solution for Venice's Flooding Problems}
\subsection{Can Venice be Saved?}

Water and Venice have always a complicated relationship. 
The world's most famously wet city is also one of its most
famously endangered ones, forever being flooded by its
signature canals. Even since the 14th century, Italian
engineers have dreamed of ways to control the water. Now
a solution may be at hand: the Moses project, a vast series
of sea gates that may finally the sodden city dry.

The need for Venetian water control has never been greater.
Especially high tides have caused major floods 10 times in
the past 67 years alone, most diastrously in 1966 when the
water in parts of the city climbed to more than 1.83 m.
Compression of sediment under the city, along with rising sea
levels, often causes smaller floods, shutting down businesses
and making sidewalks and squares impossible.

The source of the problem is geography. Venice is primarily a
small cluster of interlocked islands set in the northern end of
a 536-sq.-km lagoon. A long ridge of land seperates the lagoon
from the far larger Gulf of Venice except at three major inlets.
These openings allow high gulf tides to become high Venetian tides,
with the water sometimes climbing far enough to swamp the city's
seawalls.

In 1984 a commission composed of Italy's 50 largest engeneering and
construction firms was formed to find a way to control the water
flow through the inlets, and Moses is it. Moses, an acronym for the
plan's technical name as well as a lyrical reference to the parting of
the Red Sea, calls for 78 hollow sea gates---each up to 5 m thick, 20 m
wide and 27.5 m long---to be hinged to foundations, or caissons, in the
seabed and to lie flat there. The gates would usually be filled with
water, but when tides rises to a height of 1 m or more, compressse air
would pump the water out. The free end of the gates would then float
upwards, breaking the surface after about 30 minutes and sealing off 
the inlets. Sea locks would permit permit ships to pass while the gates
are up.

The project---which would take as long as ten years and cost at least
\$2.7 billion---could still run into obstacles, especially given the
fickle nature of Italian politics. Concerned that gates would be raised
so frequently and remain there so long that they would cause water in the
lagoon to grow stagnant, Greens are making that argument in an
environmental-impact review that could delay or even scuttle construction.
Even so, this is the closest Venice has come to a permanent solution to 
its water problems in 700 years. By local bureaucratic standards, that's
not too bad.

%---By Jeffrey Kluger. Reported by Jeff Israely/Rome---Time, June 2, 2003 

\subsection{Additional Information to the Gates}

The gates are made of steel covered with a resistant coating to prevent
building of algae and crustaceans. Every five years they're are scheduled 
for removal and cleaning.

The 78 hollow sea gates are filled with water most of the time and remain
out of sight in a foundation or cassion. During especially high tides,
compressed air flushes out the seawater. Within 30 min.\ the gates rise
to the surface and block the inlets. When the danger passes, water is
admitted back into the gates, causing them to sink within 15 min. 

\chapter[Preface]{Preface to the \LaTeX-Guide}
\section{General Remarks}

A new edition to ``A Guide to \LaTeX'' begs the fundamental question:
Has \LaTeX\ changed so much since the appearance of the third edition in 1999
that a new release of this manual is justified?

The simple answer to that question is `Well \dots.' In 1994, the \LaTeX\
world was in upheaval with the issue of the new version \LaTeXe, and the
second edition of the `Guide' came out just then to act as the bridge
between the old and new versions. By 1998, the initial teething problems had
been worked out and corrected through semi-annual releases, and the third
edition could describe an established, working system. However, homage was
still paid to the older 2.09 version since many users still employed its
familiar syntax, although they were most likely to be using it in a \LaTeXe\
environment. \LaTeX\ has now reached a degree of stability that since 2000
the regular updates have been reduced to annual events, which often appear
months after the nominal date, something that does not worry anyone. The old
version 2.09 is obsolete and should no longer play any role in such a manual.%
\footnote{With the result, that all maintanance for \LaTeX\ 2.09 has been
stopped with the occurance of \LaTeXe\ as well as for any further
developement.}
In this fourth edition, it is reduced to an appendix just to document its
syntax and usage.

But if \LaTeX\ itself has not changed substantially since 1999, many of its
peripherals have. The rise of programs like `pdf\TeX'  and `dvipdfm' for
PDF output adds new possibilities, which are realized, not in \LaTeX\
directly, but by means of more modern `packages' to extend the basic
features. The distribution of \TeX/\LaTeX\ installations has changed, such
that most users are given a complete, ready-to-run setup, with all the
`extras' that one used to have to obtain oneself. Those extras include
user-contributed packages, many of which are now considered indispensable.
Today `the \LaTeX\ system' includes much more than the basic kernel by Leslie
Lamport, encompassing the contributions of hundreds of other people. This
edition reflects this increase in breadth.\footnote{In between many hundreds
of supplements and extensions in form of packages have been developed for
\LaTeXe. Only the most important will be presented in this book.}

\section[The Changes]{The Changes in Detail}
The changes to the fourth edition are mainly those of emphasis.

\begin{enumerate}
\item material has been reorganized into `Basics' and `Beyond the Basics'
  (`advanced' sounds too intimidating) while the appendices contain
  topics that really can be skipped by most everyday users. Exception:
  Appendix H is an alphabetized command summary that many
  people find extremely useful (including ourselves).

  This reorganizing is meant to stress certain aspects over others. For
  example, the section on graphics inclusion and color was originally
  treated as an exotic freak, relegated to an appendix on extensions; in the
  third edition, it moved up to be included in a front chapter along with the
  picture environment and floats; now it dominates
  Chapter 6 all on its own, the floats come in the following
  Chapter 7, and `picture' is banished to the later
  Chapter 13. This is not to say that the picture features are no good,
  but only that they are very specialized. We add
  descriptions of additional drawing possibilities there too.

\item It is stressed as much as possible that \LaTeX\ is a markup language,
  with separation of content and form. Typographical settings
  should be placed in the preamble, while the body contains only logical
  markup. This is in keeping with the modern ideas of XML, where form and
  content are radically segregated.

\item Throughout this edition, contributed packages are explained at that point
  in the text where they are most relevant. The `fancyhdr' package
  comes in the section on page styles, `natbib' where literature
  citations are explained. This stresses that these `extensions' are part of
  the \LaTeX\ system as a whole. However, to remind the user that they must
  still be explicitly loaded, a marginal note is placed at the start of their
  descriptions.

\item PDF output is taken for granted throughout the book, in addition to the
  classical DVI format. This means that the added possibilities of `pdf\TeX'
  and `dvipdfm' are explained where they are relevant. A separate
  Chapter 10 on PostScript and PDF is still necessary, and the
  best interface to PDF output, the `hyperref' package by Sebastian
  Rahtz, is explained in detail. PDF is also included in
  Chapter 15 on presentation material

  On the other hand, the other Web output formats, HTML and XML, are only
  dealt with briefly in Appendix E, since these are large topics
  treated in other books, most noticeably the `\LaTeX\ Web Companion'.

\item This book is being distributed with the \TeX Live CD, with the kind
  permission of Sebastian Rahtz who maintains it for the \TeX\ Users Group.
  It contains a full \TeX\ and \LaTeX\ installation for Windows, Macintosh,
  and Linux, plus many of the myriad extensions that exist.
\end{enumerate}

We once again express our hope that this \textsl{Guide} will prove more than
useful to all those who wish to find their way through the intricate world of
\LaTeX. And with the addition of the \TeX Live CD, that world is brought even
closer to their doorsteps.\footnote{Information for installation details are
given in the documentation for the \TeX Live CD.}

\section[Running heads]{Demonstration for running heads}
\subsection{An additional exercise}
One main reason for this exercise was the demonstration of running heads
at the top of the pages execept for the first pages after the 
chapter sectioning commands. The running heads on even pages consist
of the actual chapter number and the chapter title. For odd pages consist
the running heads of actual section command for these pages as here
demonstrated.

\newpage\thispagestyle{empty}\noindent
\texttt{chapter} = \arabic{chapter}\\
\texttt{section} = \arabic{section}\\
\texttt{subsection} = \arabic{subsection}\\
\texttt{page} = \arabic{page}\\
\texttt{footnote} = \arabic{footnote}\\
\setcounter{equation}{3}
\texttt{equation} = \arabic{equation}\\
\addtocounter{equation}{2}
\texttt{equation} = \arabic{equation}\\
\texttt{enumi} = \arabic{enumi}\\
\texttt{enumii} = \arabic{enumi}
\end{document}

