% !TeX root = ../install-latex-guide-zh-cn.tex

\chapter{在线的 \LaTeX{} 平台}

在特定场合,
有些用户并不需要也没条件在本地安装发行版,
因此这里额外补充 \href{www.overleaf.com}{Overleaf} 和 \href{https://www.texpage.com/}{TeXPage} 的相关内容.

\section{Overleaf}

\subsection{注册 Overleaf}

Overleaf 是全球范围内首屈一指的在线 \LaTeX{} 编辑平台.
它为每位用户提供了 Ubuntu 系统下的 \TeX{} Live.
它优秀的协作功能、丰富的模板仓库已吸引全球科研工作者成为它的用户.
Overleaf 提供了包括%
\href{https://cn.overleaf.com}{中文}%
在内的多种语言供用户使用.
2024年6月27日,
\href{https://www.overleaf.com/blog/tex-live-2024-is-now-available}{它将后台的 \TeX{} Live 升级为 2024 版本},
同时,
\href{https://www.overleaf.com/blog/new-feature-select-your-tex-live-compiler-version}{Overleaf 还允许用户自主选择项目中的 \TeX{} Live 版本}.

目前 Overleaf 允许用户%
\href{https://www.overleaf.com/learn/latex/Chinese}{使用中文},
并为用户预先准备了一些%
\href{https://www.overleaf.com/learn/latex/Questions/Which_OTF_or_TTF_fonts_are_supported_via_fontspec%3F#Fonts_for_CJK}{中文字体}.
这些中文字体可通过%
\href{https://www.overleaf.com/latex/templates/using-the-ctex-package-on-overleaf-zai-overleafping-tai-shang-shi-yong-ctex/gndvpvsmjcqx}{C\TeX{} 宏集}%
调用.
特别推荐用户使用%
\href{https://www.overleaf.com/latex/examples/demonstration-of-noto-serif-cjk-and-noto-sans-cjk-fonts/sgrwgcddtqsq}{思源宋体和思源黑体}%
这两种开源中文字体.

遗憾的是,
国内网络环境会对 Overleaf 所使用的 reCaptcha 造成影响.
这也使得很多用户在直接注册 Overleaf 时就遇到了问题.

目前,
比较好的替代方案是借助 \href{https://orcid.org}{ORCID} 来进行注册.
目前使用国内网络访问 ORCID 还比较流畅.
用户, 尤其是科研工作者, 可以先注册一个 ORCID 账号.
未来投稿时,
可将 ORCID 账号与自己的期刊网站账号进行绑定.
同时,
用户也可逐步将自己所发表论文列在 ORCID 网站以便管理.

\subsection{使用 Overleaf}

用户通过 ORCID 注册 Overleaf 后便可进入自己的项目列表页面进行使用.

新建项目是用户首先使用的功能.
Overleaf 提供了多种渠道为用户新建项目:
可以通过 Overleaf 中的模板,
也可以通过上传本地的 \textsf{zip} 压缩文件包.

新建项目后,
用户便可进入编辑界面.
在编辑界面用户需要先在左上角 \menu{Menu} 中选择合适的编译命令.
由于默认字体和文件编码等原因,
强烈建议用户在处理中文文档时使用 \texttt{XeLaTeX}.
编写文档后,
用户可通过鼠标点击按钮进行编译,
也可使用快捷键 \keys{ctrl + enter}.

另外,
Overleaf 在 9 月 30 日更新了新的功能,
\href{https://www.overleaf.com/blog/new-feature-stop-on-first-error-compilation-mode}{Stop on first error},
这个功能一旦开启,
出现报错就立即停止编译.
它可以让用户高度关注报错信息并改正,
也可以帮助纠正一些会导致超时 (timeout) 的代码错误,
例如 tikz 里的 \verb+\draw+ 忘了最后的 \verb+;+.

用户编写的文件会保存在网站.
编写完成后,
用户只需点击 \menu{Menu} 旁的箭头回到项目列表.
这时可以看到新增项目右侧有四个图标,
它们分别是 \menu{Copy}、\menu{Download}、\menu{Archive} 和 \menu{Trash}.
用户可根据自己的需求点击合适的图标.

\subsection{选择不同发行版版本}

Overleaf 将后台 \TeX{} Live 升级后,
用户新建项目默认使用 \TeX{} Live 2022,
而老项目还是使用 \TeX{} Live 的老版本.
如果用户打算使用 \TeX{} Live 2022,
只需在 \menu{Menu > Settings > TeX Live version} 选择版本即可.

\subsection{学习与帮助}

Overleaf 的%
\href{https://www.overleaf.com/latex/templates}{模板}和%
\href{https://www.overleaf.com/learn}{文档}%
对全网公开,
用户可以自行学习.
另外 Overleaf 有着专业的技术支援团队,
用户可发送邮件至 \href{mailto:support@overleaf.com}%
{\ttfamily support@overleaf.com}
咨询使用过程中遇到的问题,
在邮件中请注意文明用语.
部分问题会超出免费服务的范畴,
用户需谨记这点.

\section{TeXPage}

\subsection{注册 TeXPage}

TeXPage 是由国内公司开发的在线 \LaTeX{} 平台.
相较于 Overleaf 的注册困难,
TeXPage 的注册则要方便得多.
用户只需访问它的主页,
使用邮箱注册即可,
具体过程不再赘述.

\subsection{使用 TeXPage}

新用户注册 TeXPage 后会在页面看到一份使用教程.
这份教程简明扼要地概括了一般中文用户会遇到的常见问题,
然而从代码的角度而言,
我个人不欣赏在浮动体中使用 \texttt{H} 选项的做法.

在页面右上角的位置有许多菜单,
在\textsf{设置}菜单中,
用户可以选择默认的编译命令、发行版版本等等.
设置完毕,
再点击编译即可.

\subsection{文档和帮助}

TeXPage 有自己的 \href{https://www.texpage.com/docs/}{文档中心},
文档数量不多,
但涵盖了相当一部分基础知识.
同时,
它也提供了联系方式 \href{mailto:support@texpage.com}%
{\ttfamily support@texpage.com},
用户如果遇到了一些使用上的问题也可以直接发邮件咨询.

\subsection{TeXPage 的优势}

在全球范围内,
Overleaf 依然是在线 \LaTeX{} 平台的主流,
然而由于种种原因,
大陆地区用户体验不佳,
因此 TeXPage 就成为了一个稳定有效的新的选择.

目前在 TeXPage 上也具有一些颇具特色的模板,
例如建模比赛的一些模板就已经被收录在 TeXPage 当中.
如果未来 TeXPage 能够再尽可能收录大陆各个高校的毕业论文模板的话,
那可能是一件非常有趣的事情.

\section{其他在线平台}

除以上介绍的两个平台外,
大陆地区部分高校 (例如%
\href{https://overleaf.tsinghua.edu.cn/login}{清华大学}%
和%
\href{https://latex.ustc.edu.cn/login}{中国科学技术大学}) 也搭建了供内部师生使用的平台.

中国科学院计算机网络信息中心科技云运行与技术发展部曾发邮件宣称开始提供论文协同编辑服务,
据我猜测,
它利用了早年 Sharelatex 的部分内容.
试用可点击%
\href{https://www.cstcloud.cn/resources/452}{这里}.

南京一家公司筹建了
\href{https://www.slager.link/#/Home}{Slager} 在线 \LaTeX{} 编辑工具.
目前看来,
Slager 的用法也比较简单,
并且有关于起步的%
\href{https://www.slager.link/#/HelpCenter}{帮助文档}.

\href{https://online.latexstudio.net/}{latexstudio} 也建立了一个在线平台,
这个平台目前只有一台服务器,
响应有点慢,
但是会免费供用户使用.
