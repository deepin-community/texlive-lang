% Tekijä:   Teemu Likonen <tlikonen@iki.fi>
% Lisenssi: Creative Commons Nimeä-JaaSamoin 4.0 Kansainvälinen (CC BY-SA 4.0)
% https://creativecommons.org/licenses/by-sa/4.0/legalcode.fi

\documentclass[fleqn]{book}
\usepackage{geometry}
\usepackage{fontspec}
\usepackage{polyglossia}
\usepackage{amsmath}
\usepackage[math-style=ISO]{unicode-math}
\usepackage[finnish, showseconds=false]{datetime2}
\usepackage{ragged2e}
\usepackage[hang,bottom]{footmisc}
\usepackage[clearempty]{titlesec}
\usepackage{titletoc}
\usepackage{graphicx}
\usepackage{xcolor}
\usepackage{floatrow}
\usepackage{caption}
\usepackage{wrapfig}
\usepackage{fancyvrb}
\usepackage{tabularx}
\usepackage{booktabs}
\usepackage{multirow}
\usepackage[normalem]{ulem}
\usepackage[makeindex, splitindex]{indextools}
\usepackage{chngcntr}
\usepackage{realscripts}
\usepackage{csquotes}
\usepackage[style=authoryear, dashed=false, maxbibnames=99,
datezeros=false]{biblatex}
\usepackage{tikz}
\usepackage{totcount}
\usepackage[all]{nowidow}
\usepackage{hanging}
\usepackage{lettrine}
\usepackage{multicol}
\usepackage{textpos}
\usepackage[unicode,hyperfootnotes=false]{hyperref}
\usepackage[shortcuts]{extdash}

\geometry{ a5paper, twoside, hscale=.72, vscale=.77, hmarginratio=17:28,
  vmarginratio=20:32, footskip=12mm, footnotesep=13bp,
  marginparwidth=50bp, marginparsep=10bp }

% \geometry{ papersize={158mm, 220mm}, layout=a5paper, layoutoffset={5mm,
%     5mm}, showcrop }

\urlstyle{sf}
\newcommand{\kulmaurl}[1]
{\href{#1}{\guilsinglleft\nolinkurl{#1}\guilsinglright}}
\newcommand{\kulmasp}[1]
{\href{mailto:#1}{\guilsinglleft\nolinkurl{#1}\guilsinglright}}

\newcommand{\versio}{2024.1}


\newcommand{\otsikko}{Käytännöllistä Latexia}
\newcommand{\alaotsikko}{Latex-ladontajärjestelmän opas}
\newcommand{\tekija}{Teemu Likonen}
\newcommand{\tekijat}{Teemu Likonen \kulmasp{tlikonen@iki.fi}}

\hypersetup{ hidelinks, bookmarksnumbered,
  pdfinfo={
    Title={\otsikko},
    Subject={\alaotsikko, versio \versio},
    Author={\tekija},
    Keywords={Latex, tekstinvalmistus, ladonta, tekstinkäsittely,
      oppaat, atk-ohjelmat, kirjoittaminen, typografia}
  }
}

\setdefaultlanguage{finnish}
\setotherlanguage{english}
\newcommand{\englanti}[1]{\textenglish{#1}}
\newcommand{\englantik}[1]{\textenglish{\emph{#1}}}

\defaultfontfeatures[\ttfamily]{} % Nollataan oletukset: tavutus päälle

\setmainfont{Libertinus Serif}[
Numbers=Lowercase,
BoldFont={* Semibold},
BoldItalicFont={* Semibold Italic},
SlantedFont={* Regular}, SlantedFeatures={FakeSlant=0.2},
BoldSlantedFont={* Semibold}, BoldSlantedFeatures={FakeSlant=0.2},
SmallCapsFont={* Regular}, SmallCapsFeatures={Language=Default,
  Letters=SmallCaps},
SwashFont={TeX Gyre Chorus}]

\setsansfont{Libertinus Sans}[
Scale=MatchLowercase, Numbers=Lowercase,
SlantedFont={* Regular}, SlantedFeatures={FakeSlant=0.2},
BoldSlantedFont={* Bold}, BoldSlantedFeatures={FakeSlant=0.2},
SmallCapsFont={* Regular}, SmallCapsFeatures={Language=Default,
  Letters=SmallCaps}]

\setmonofont{Libertinus Mono}[
Scale=MatchLowercase, Ligatures={TeXOff, CommonOff},
FakeStretch=0.8, Numbers=SlashedZero]

\setmathfont{Libertinus Math}[Scale=MatchLowercase]

\renewcommand{\scriptsize}{\fontsize{7bp}{7bp}\selectfont}
\renewcommand{\footnotesize}{\fontsize{8bp}{9bp}\selectfont}
\renewcommand{\small}{\fontsize{9bp}{10bp}\selectfont}
\renewcommand{\normalsize}{\fontsize{10.5bp}{13bp}\selectfont}
\renewcommand{\large}{\fontsize{13bp}{15bp}\selectfont}
\renewcommand{\Large}{\fontsize{16bp}{18bp}\selectfont}
\renewcommand{\LARGE}{\fontsize{20bp}{22bp}\selectfont}
\renewcommand{\huge}{\fontsize{24bp}{26bp}\selectfont}
\linespread{1}
\normalsize

\newcommand{\gemenanum}{\addfontfeatures{Numbers=Lowercase}}
\newcommand{\versaalinum}{\addfontfeatures{Numbers=Uppercase}}
\newcommand{\murtoluku}[2]{{\addfontfeatures{Fractions=On}#1/#2}}

\setlength{\emergencystretch}{1.3em}

\setlength{\parindent}{1em}
\setlength{\bibhang}{\parindent}
\setlength{\bibitemsep}{.5ex plus .1ex minus .1ex}
\newlength{\sisennys}\setlength{\sisennys}{1.8em}
\setlength{\parskip}{0em}
\setlength{\footnotemargin}{.8em}
\setlength{\floatsep}{2ex plus 1ex minus .5ex}
\setlength{\textfloatsep}{4ex plus 1ex minus .5ex}
\setlength{\multicolsep}{0bp}
\setlength{\intextsep}{0bp}

\renewcommand{\topfraction}{.75}
\renewcommand{\floatpagefraction}{.7}

\addbibresource{kirjallisuutta.bib}
\nocite{*}
\renewcommand{\bibfont}{\RaggedRight}

\DeclareDelimFormat[bib]{nametitledelim}{\addcolon\space}
\DeclareDelimFormat[bib]{multinamedelim}{\space--\space}
\DeclareDelimFormat[bib]{finalnamedelim}{\space--\space}
\DeclareDelimFormat[textcite,parencite]{finalnamedelim}{\space\&\space}
\DeclareFieldFormat{url}{Saatavissa: \kulmaurl{#1}}
\DeclareNameAlias{sortname}{family-given}
\DeclareNameAlias{default}{family-given}
\DefineBibliographyStrings{finnish}{andothers = {ym.}}

\DeclareNewFloatType{esimerkki}{name=Esimerkki, within=chapter}

\indexsetup{level=\section*, toclevel=section, noclearpage}

\makeindex[name=paketit, title={Paketit}, columns=2, columnsep=1em]
\makeindex[name=komennot, title={Komennot}, columns=2, columnsep=1em]
\makeindex[name=ymparistot, title={Ympäristöt}, columns=2, columnsep=1em]
\makeindex[name=mitat, title={Mitat}, columns=2, columnsep=1em]
\makeindex[name=laskurit, title={Laskurit}, columns=2, columnsep=1em]
\makeindex[name=dokumenttiluokat, title={Dokumenttiluokat}, columns=2,
columnsep=1em]

\renewcommand{\theFancyVerbLine}
{\sffamily\versaalinum\fontsize{6bp}{7bp}\selectfont\arabic{FancyVerbLine}}

\DefineVerbatimEnvironment{koodilohko}{Verbatim}{ fontsize=\small,
  gobble=0, frame=single, framesep=.4em, numbers=left, numbersep=.3em,
  xleftmargin=0em, xrightmargin=0mm, baselinestretch=1 }

\DefineVerbatimEnvironment{koodilohkosis}{Verbatim}{
  fontsize=\small, gobble=0, frame=none, numbers=none,
  numbersep=0em, xleftmargin=\sisennys, xrightmargin=0mm,
  baselinestretch=1, samepage=true }

\newcommand{\seurausnuoli}{\textcolor[gray]{.5}{⇒}}

\newenvironment{tulos}{%
  \begin{textblock*}{1cm}(-2em,3bp)
    \small\seurausnuoli
  \end{textblock*}
  \begin{minipage}{\textwidth}
    \linespread{1}\small
  }{%
  \end{minipage}
  \par\addvspace{\baselineskip}
}

\newenvironment{tulossis}{%
  \begin{list}{}{
      \setlength{\leftmargin}{\sisennys}
      \small
    }\item[\seurausnuoli]}{%
  \end{list}}

\floatsetup{ style=plain, font={small}, justification=raggedright,
  margins=centering, captionskip=0ex, capposition=bottom }

\floatsetup[table]{ style=plain, captionskip=2ex }
\floatsetup[figure]{ style=plain, captionskip=2ex }

\captionsetup{ font={small, sf}, labelfont={bf}, textfont={},
  textformat=period, margin={.1em, 1.1em}, oneside,
  justification=RaggedRight, singlelinecheck=off }

\newcommand{\leijutlk}[2]{%
  \begin{table*}
    \floatbox{table}[\FBwidth]{\versaalinum #1}{#2}
  \end{table*}}

\newcommand{\leijukuva}[2]{%
  \begin{figure*}
    \floatbox{figure}{#1}{#2}
  \end{figure*}}

\newenvironment{nluetelma}{%
  \begin{list}{\arabic{enumi}.}{
      \usecounter{enumi}
      \setlength{\leftmargin}{1.3em}
      \setlength{\labelsep}{.3em}
      \setlength{\itemsep}{.2ex plus .2ex}
      \setlength{\parsep}{0em}
      \setlength{\topsep}{.2ex plus .2ex}
    }}{\end{list}}

\newenvironment{maaritelma}[1]{%
  \begin{list}{}{
    \setlength{\leftmargin}{\parindent}
    \setlength{\labelwidth}{\parindent}
    \setlength{\listparindent}{\parindent}
    \setlength{\labelsep}{1em}
    \setlength{\itemindent}{1em}
    \setlength{\itemsep}{.2ex plus .2ex}
    \setlength{\parsep}{0em}
    \setlength{\topsep}{.2ex plus .2ex}
    \renewcommand{\makelabel}[1]{#1}
  }}{\end{list}}

\newcolumntype{L}{>{\RaggedRight\arraybackslash}X}

\definecolor{tavu}{rgb}{1,0,0}
\definecolor{apuviiva}{gray}{.4}
\definecolor{mittanuoli}{rgb}{1,0,0}

\definecolor{luokka}{rgb}{0,.3,.3}
\definecolor{komento}{rgb}{0,0,.5}
\definecolor{mkomento}{rgb}{.4,0,.4}
\definecolor{ymparisto}{rgb}{0,.3,0}
\definecolor{mymparisto}{rgb}{.4,0,.4}
\definecolor{mitta}{rgb}{.4,0,0}
\definecolor{laskuri}{rgb}{.4,0,.4}
\definecolor{paketti}{rgb}{.35,.35,0}

\newcommand{\keno}{\textbackslash}
\newcommand{\marginaali}[1]{\marginpar{\RaggedRight\footnotesize #1}}

\newcommand{\koodi}[1]{\texttt{#1}}
\newcommand{\koodil}[1]{\enquote{\texttt{#1}}}
\newcommand{\yipilkku}{\textsuperscript*{,}}

\newcommand{\luokkax}[1]{\textcolor{luokka}{\textsf{#1}}}
\newcommand{\luokkai}[1]{\index[dokumenttiluokat]{#1@\luokkax{#1}}}
\newcommand{\luokka}[1]{\luokkax{#1}\luokkai{#1}}
\newcommand{\luokkactan}[1]{\luokka{#1}\avctan{#1}}

\newcommand{\komentox}[1]{\textcolor{komento}{\koodi{\keno #1}}}
\newcommand{\komentoi}[1]{\index[komennot]{#1@\komentox{#1}}}
\newcommand{\komento}[1]{\komentox{#1}\komentoi{#1}}
\newcommand{\komentojatko}[1]{\katk\textcolor{komento}{\koodi{#1}}}
\newcommand{\komentoarg}[1]{\komentojatko{\{#1\}}}
\newcommand{\komentoargv}[1]{\komentojatko{[#1]}}

\newcommand{\mkomentox}[1]{\textcolor{mkomento}{\koodi{\keno #1}}}
\newcommand{\mkomentoi}[1]{\index[komennot]{#1@\mkomentox{#1}}}
\newcommand{\mkomento}[1]{\mkomentox{#1}\mkomentoi{#1}}
\newcommand{\mkomentojatko}[1]{\katk\textcolor{mkomento}{\koodi{#1}}}
\newcommand{\mkomentoarg}[1]{\mkomentojatko{\{#1\}}}
\newcommand{\mkomentoargv}[1]{\mkomentojatko{[#1]}}

\newcommand{\ymparistox}[1]{\textcolor{ymparisto}{\koodi{#1}}}
\newcommand{\ymparistoi}[1]{\index[ymparistot]{#1@\ymparistox{#1}}}
\newcommand{\ymparisto}[1]{\ymparistox{#1}\ymparistoi{#1}}

\newcommand{\mymparistox}[1]{\textcolor{mymparisto}{\koodi{#1}}}
\newcommand{\mymparistoi}[1]{\index[ymparistot]{#1@\mymparistox{#1}}}
\newcommand{\mymparisto}[1]{\mymparistox{#1}\mymparistoi{#1}}

\newcommand{\mittax}[1]{\textcolor{mitta}{\koodi{\keno #1}}}
\newcommand{\mittai}[1]{\index[mitat]{#1@\mittax{#1}}}
\newcommand{\mitta}[1]{\mittax{#1}\mittai{#1}}

\newcommand{\laskurix}[1]{\textcolor{laskuri}{\koodi{#1}}}
\newcommand{\laskurii}[1]{\index[laskurit]{#1@\laskurix{#1}}}
\newcommand{\laskuri}[1]{\laskurix{#1}\laskurii{#1}}

\newcommand{\pakettix}[1]{\textcolor{paketti}{\textsf{#1}}}
\newcommand{\pakettii}[1]{\index[paketit]{#1@\pakettix{#1}}}
\newcommand{\paketti}[1]{\pakettix{#1}\pakettii{#1}}
\newcommand{\pakettictan}[1]{\paketti{#1}\avctan{#1}}

\newcommand{\tavukohta}{\textcolor{tavu}{\raisebox{-.2ex}{\rule{.6bp}{2ex}}}}
\newcommand{\uctunnus}[1]{\textsc{\englanti{#1}}}
\newcommand{\avctan}[1]{\footnote{\url{https://www.ctan.org/pkg/#1}}}

\newcommand{\ots}[1]{{\sffamily\bfseries #1}}
\newcommand{\otsrivi}[1]{{\sffamily #1}}
\newcommand{\katk}{\discretionary{}{}{}}

\addto{\captionsfinnish}{
  \renewcommand{\contentsname}{Sisällys}
}

\newcommand{\otsikkotyyli}{ \raggedright \sffamily \bfseries }

\titleformat{\chapter}
[display]
{\Large\bfseries}
{\chaptertitlename\hspace{.3em}\thechapter}
{1.5ex}
{\otsikkotyyli\huge}[]
\titlespacing*{\chapter}{0em}{*13}{*8}

\titleformat{\section}
{\otsikkotyyli\large}
{\thesection}
{.8em}
{}[]
\titlespacing*{\section}{0pt}{*4}{*2}

\titleformat{\subsection}
{\otsikkotyyli\normalsize}
{\thesubsection}
{.8em}
{}[]
\titlespacing*{\subsection}{0bp}{*2}{*1}

\titleformat{\subsubsection}
{\otsikkotyyli\mdseries\scshape\normalsize}
{\thesubsubsection}
{.8em}
{}[]
\titlespacing*{\subsubsection}{0bp}{*2}{*1}

\titlecontents{chapter}
[8mm]
{\addvspace{1.5ex}\rmfamily\bfseries\large}
{\contentslabel{8mm}}
{\hspace{-8mm}}
{\small\titlerule[0bp]\contentspage}
[\addvspace{.5ex}]

\titlecontents{section}
[8mm]
{\addvspace{.5ex}\rmfamily\normalsize}
{\contentslabel{8mm}}
{}
{~\small\titlerule*[3mm]{.}\contentspage}
[\addvspace{.2ex}]

\titlecontents{subsection}
[18mm]
{\rmfamily\small}
{\contentslabel{10mm}}
{}
{~\small\titlerule*[3mm]{.}\contentspage}
[]

\titlecontents*{subsubsection}
[18mm]
{\rmfamily\footnotesize}
{\thecontentslabel. }
{}
{ (\thecontentspage)}
[ -- ][.]

\regtotcounter{chapter}

\begin{document}

\hyphenation{
ab-so-lute
ab-stract
abst-ra-hoin-ti
abst-rak-te-ja
abst-rak-tio-ta-soil-la
abst-rak-tio-ta-sol-la
abst-rak-tio-ta-sol-taan
abst-rak-tio-ta-son
add-bib-re-source
add-con-tents-line
add-font-fea-tures
ad-dress
add-to
add-to-con-tents
add-to-counter
add-to-length
add-vspace
ajan-ilmauk-set
ajan-ilmauk-sia
ajan-ilmaus-ten
ajan-ilmaus-tyy-lit
akuut-ti-ak-sen-tin
ala-in-dek-seil-le
ala-in-dek-se-jä
ala-in-dek-sien
ala-in-dek-si-ko-men-not
ala-in-dek-sik-si
ala-in-dek-sin
ala-in-dek-sit
ala-in-dek-si-toi-min-to
ala-in-dek-siä
ala-osaan
ala-ot-sik-ko
ala-otsi-kon
alku-osaa
angle
an-tiik-va
an-tiik-vaa
an-tiik-vaan
an-tiik-van
an-tiik-va-per-heen
an-tiik-vas-sa
an-tiik-vat
an-tiik-vo-jen
ap-pendix
apu-ohjel-ma
array-back-slash
array-rule-width
array-stretch
ar-ti-cle
as-pect-ra-tio
author
babel
babel-font
babel-hy-phen-ation
babel-pro-vide
babel-short-hands
babel-tags
back-ground
back-mat-ter
base-line-skip
beamer
bf-se-ries
biber
bib-font
bib-hang
bib-init-sep
bib-item
bib-item-sep
bib-latex
bib-name
bib-name-sep
bib-section
bib-sep
big-break
big-skip
big-skip-amount
bi-nää-ri-ope-raat-to-rit
book-marks-num-ber-ed
book-marks-open
book-tabs
bottom-frac-tion
bottom-number
bottom-rule
break
brit-ti-eng-lan-nin
cap-tion
cap-tion-setup
cap-tions-finnish
cen-ter
chapter-lists-gaps
chapter-mark
chapter-name
chng-cntr
cite
clear-double-page
clear-page
cline
clos-ing
cmid-rule
color-box
column-break
column-sep
column-sep-rule
column-width
con-tents-label
con-tents-name
con-tents-page
Con-tin-ued-Float
con-trols
counter-with-in
counter-with-out
cs-quotes
cycle
date-time2
dbl-float-page-frac-tion
dbl-float-sep
dbl-text-float-sep
dbl-top-frac-tion
dbl-top-number
De-clare-Bib-li-og-ra-phy-Driv-er
De-clare-Float-Font
De-clare-New-Float-Type
De-clare-Sort-ing-Tem-plate
de-fault
default-font-fea-tures
de-fine-col-or
de-scrip-tion
de-si-maa-li-erot-ti-mek-si
de-si-maa-li-erot-ti-me-na
de-si-maa-li-osa
dia-esi-tyk-siä
dia-esi-tys-ten
dis-able-hy-phen-ation
dis-cre-tionary
dis-play-math
doc-u-ment
doc-u-ment-class
dot-fill
double-rule-sep
DTM-cur-rent-time
DTM-dis-play-date
DTM-dis-play-time
DTM-set-style
DTM-today
ed-i-tor
emer-gen-cy-stretch
empty
en-able-hy-phen-ation
Endash
end-tab-u-larx
en-glish-font
en-glish-font-sf
en-glish-font-tt
en-large-this-page
en-space
enu-mer-ate
enumi
enumii
enumiii
enumiv
equa-tion
erin-omai-sen
ero-tin-merkki-ase-tuk-sen
esit-te-ly-osaan
esit-te-ly-osak-si
esit-te-ly-osas-sa
esi-tys-gra-fii-kan
esi-tys-gra-fiik-kaan
esi-tys-gra-fiik-ka-ohjel-mil-la
ext-dash
false
fancy-hdr
fancyhf
fancy-page-style
fancy-vrb
fbox-rule
fbox-sep
FB-width
fcolor-box
fig-ure
fig-ure-name
fig-ures
first-page-style
float
Float-Bar-rier
float-box
float-page-frac-tion
float-row
float-sep
float-setup
flush-bottom
flush-columns
font-size
font-spec
font-ti-ase-tuk-set
font-ti-ase-tuk-sia
font-ti-ase-tuk-siin
font-ti-ase-tuk-sil-la
font-ti-ase-tuk-sil-le
font-ti-ase-tus
font-ti-ase-tus-ta
font-ti-ase-tus-ten
font-ti-asiat
font-ti-koko-ase-tus-ten
font-ti-koko-komen-to-jen
font-ti-komen-to-ja
foot-misc
foot-note
foot-note-lay-out
foot-note-mark
foot-note-sep
foot-note-text
foot-rule-width
foot-skip
fore-ground
for-eign-lan-guage
frame-box
front-mat-ter
geom-e-try
graph-icx
gra-vis-ak-sen-til-la
gra-vis-ak-sen-tin
gra-vis-ak-sent-ti
greek-font
gro-tes-ki
gro-tes-kia
gro-tes-kiin
gro-tes-kin
hai-tal-li-suus-arvo
hai-tal-li-suus-arvoa
hai-tal-li-suus-ar-voik-si
ha-luaa
hang-ing
hang-para
hang-paras
head-ings
head-rule-width
head-sep
hide-links
hline
hrule-fill
hspace
hyper-ref
hyper-setup
Hyph-dash
hy-phen-ation
if-then-else
ig-nore-spaces
ig-nore-spaces-af-ter-end
imake-idx
in-clude
in-clude-graph-ics
in-col-lec-tion
in-dex
index-name
index-pro-logue
index-setup
index-tools
in-put
in-sti-tute
in-text-sep
in-toc
item-indent
item-ize
item-sep
it-seis-arvo
it-seis-ar-voa
it-shape
jul-kai-su-ajan-koh-taa
jul-kai-su-oh-jel-miin
jul-kai-su-oh-jel-mis-sa
jus-ti-fi-ca-tion
kak-si-osai-nen
kak-sois-alle-vii-vauk-sen
kap-pa-le-ot-sik-ko
kap-pa-le-ot-si-koi-ta
kat-se-lu-oh-jel-mal-la
kel-lon-aika
kel-lon-ajan
kel-lon-ajois-sa
ker-roin-ase-tus-ta
keski-osa
kie-li-ai-nek-seen
kie-li-ase-tuk-set
kie-li-ase-tuk-set-kin
kie-li-ase-tuk-sia
kie-li-ase-tuk-siin
kie-li-ase-tuk-sil-la
kie-li-ase-tuk-sis-ta
kie-li-ase-tus-ten
kie-li-opil-li-sis-ta
kie-li-op-pin-sa
kir-jain-yh-dis-tel-mät
kir-ja-typo-gra-fian
kir-ja-typo-gra-fias-sa
kir-joi-tus-asu
kir-joi-tus-asu-ja
kir-joi-tus-asu-jen
kir-joi-tus-asus-sa
kir-joi-tus-asut
koko-ase-tus
koko-ase-tus-ten
koko-ero
koko-nais-luku-osa
ko-men-to-vaih-to-eh-to
konk-reet-ti-ses-ti
konk-reet-ti-sia
koodi-esi-mer-keis-sä
koodi-esi-merk-kien
kuva-esi-mer-keis-tä
käsi-ala-kir-joi-tus-ta
käsi-ohjel-mien
käyt-töön-oton
käyt-töön-otos-sa
käyt-töön-otto
käyt-töön-ottoa
käyt-töön-ot-toon
kään-täjä-ohjel-mat
kään-täjä-ohjel-milla
label
label-font
label-format
label-sep
label-width
la-don-ta-oh-jei-ta
la-don-ta-oh-jel-man
land-scape
large
La-tex-mk
La-tex-mk-rc
left-hy-phen-min
left-margin
left-mark
let-ter
lett-rine
level
line
line-spread
line-width
list
list-fig-ure-name
list-of
list-of-fig-ures
list-of-tables
list-par-indent
list-table-name
lisä-omi-nai-suu-den
lisä-vaih-to-eh-toa
lohko-lai-naus-ympä-ris-töt
long-table
Lua-latex
lue-tel-ma-ympy-rä
lue-tel-ma-ympy-räl-lä
luku-alueel-la
luku-aluei-ta
luku-arvot
main-mat-ter
make-box
make-index
make-label
make-labels
make-title
margin-note
margin-par
margin-par-push
margin-par-sep
margin-par-width
mark-both
mark-right
Match-Low-er-case
ma-te-ma-tiik-ka-ym-pä-ris-tö-jä
md-se-ries
med-break
med-skip
med-skip-amount
mem-oir
mid-rule
mini-page
mit-ta-yk-sik-kö
mit-ta-yk-si-köi-den
mit-ta-yk-si-köi-tä
mit-ta-yk-si-köl-le
mit-ta-yk-si-kön
mit-ta-yk-si-köt
mk-bib-name-fam-i-ly
mk-bib-name-giv-en
mk-bib-name-pre-fix
mk-bib-name-suf-fix
mk-bib-parens
mk-bib-quote
mp-foot-note
multi-col
multi-cols
multi-col-sep
mul-ti-ple
multi-row
muo-toi-lu-ase-tuk-set
my-head-ings
mää-rit-te-ly-omi-nai-suut-ta
name
name-ref
new-column-type
new-com-mand
new-counter
new-en-vi-ron-ment
new-float
new-font-face
new-font-fam-i-ly
new-ge-om-e-try
new-length
new-line
new-page
ni-men-omaan
no-cite
no-clear-page
no-club
node
no-in-dent
no-indent-after
no-link-url
no-new-page
no-page-break
no-page-color
nor-mal
normal-font
normal-margin-par
normal-size
no-rule
note
no-widow
nuo-li-ak-sen-tit
nyky-aikai-nen
nyky-aikai-set
nyky-aikai-sia
nyky-aika-na
nyky-ajal-le-kin
nyky-ajan
näp-päin-aset-te-lus-sa
obey-cr
oikein-kir-joi-tus-oppaas-sa
ole-tus-ar-vo
ole-tus-ar-voa
ole-tus-ar-voi-hin
ole-tus-ar-vot
ole-tus-ase-tuk-set
ole-tus-ase-tuk-sia
ole-tus-ase-tuk-siin
ole-tus-ase-tuk-sil-la
ole-tus-ase-tus
ole-tus-mitta-yk-sik-kö-nä
ole-tus-yk-sik-kö
ole-tus-yk-sik-köä
one-column
one-side
on-line
open-ing
op-tions
origin
otf-info
other-lan-guage
over-lay
over-left-ar-row
over-right-ar-row
page
page-break
page-color
page-number-ing
page-ref
pages
page-style
paper-height
pa-pe-ri-ar-keis-sa
pa-pe-ri-ark-kei-hin
pa-pe-ri-ark-kia
pa-pe-ri-ark-kien
paper-width
para-graph
par-box
paren-cite
par-ent
par-in-dent
par-sep
par-skip
part-name
par-top-sep
pause
pdf-book-mark
pdf-info
pdf-latex
pdf-lscape
per-page
pe-rus-aja-tus
pe-rus-ase-tuk-set
pe-rus-ase-tuk-sia
pe-rus-asiat
pe-rus-asioi-hin
pe-rus-asioi-ta
pe-rus-osas-ta
pe-rus-osat
pe-rus-osien
pe-rus-osiin
pe-rus-vaih-to-eh-dot
pe-rus-ympä-ris-tö
pg-hy-phen-ation
phan-tom-sec-tion
pien-aak-ko-set
pien-aak-kos-ten
pii-rus-tus-alus-tan
pik-seli-gra-fiik-kaa
place-ins
poly-glos-sia
por-trait
print-bib-li-og-ra-phy
print-index
pro-gram
pro-tect
pro-vide-com-mand
pub-lish-er
pys-ty-asen-nos-sa
pys-ty-asen-toi-sel-la
pys-ty-asen-toi-sen
pää-asial-li-nen
pää-asial-li-sek-si
pää-asial-li-sen
pää-asias-sa
pää-ot-sik-ko
pääte-oh-jel-mien
quo-ta-tion
quote
ragged-bottom
ragged-columns
ragged-left
ragged-right
raise-box
real-scripts
ref-name
ref-step-counter
reg-tot-counter
Ren-der-er
re-new-com-mand
re-new-en-vi-ron-ment
re-size-box
re-store-cr
re-store-ge-om-e-try
reverse-margin-par
right-hy-phen-min
right-margin
right-mark
rivin-osan
rm-fam-i-ly
ro-tate-box
rule
sak-ko-ar-voa
sa-man-aikai-ses-ti
sar-kain-ase-tuk-sia
Scale
scale-box
school-hy-phens
sc-shape
sec-num-depth
sec-tion
sec-tion-break
sec-tion-mark
see-also
select-font
select-lan-guage
set-beamer-color
set-beamer-font
set-beamer-tem-plate
set-cite-style
set-counter
set-default-lan-guage
set-lang-hyphen-mins
set-length
set-main-font
set-math-font
set-mono-font
set-no-club
set-no-widow
set-sans-font
set-to-depth
set-to-height
set-to-width
sf-fam-i-ly
short-en
show-hy-phen-ation
show-seconds
sig-na-ture
si-joit-te-lu-ase-tuk-set
si-joit-te-lu-ase-tuk-sia
single-line-check
sir-kum-flek-si
sir-kum-flek-sia
sir-kum-flek-sin
si-sen-nys-ase-tuk-set
si-sen-nys-asiat
si-säl-tö-ele-ment-tien
sivu-alue
si-vu-alueel-le
si-vu-alueen
si-vu-aluei-den
sivu-koko-ase-tuk-sia
skaa-laus-ar-gu-ment-ti
skaa-laus-ase-tus
slides
slope
sl-shape
small
small-break
small-skip
small-skip-amount
space-skip
split-index
stable
step-counter
still-sans-serif-large
still-sans-serif-math
still-sans-serif-small
still-sans-serif-text
style
sub-para-graph
sub-sec-tion
sub-section-mark
sub-sub-sec-tion
suo-jaus-omi-nai-suus
suur-aak-kos-ten
sw-shape
sym-bo-li-esi-tys
säh-kö-pos-ti-osoi-te
säh-kö-pos-ti-osoit-tei-den
säh-kö-pos-ti-osoit-tei-ta
tab-bing
tab-bing-sep
tab-col-sep
table
table-name
table-of-con-tents
tables
tabu-lar
tabu-larx
tabu-lary
tau-luk-ko-ympä-ris-tö
tau-luk-ko-ympä-ris-tön
tau-luk-ko-ympä-ris-töt
tau-luk-ko-ympä-ris-töä
ta-van-omais-ten
tavu-tus-algo-rit-mia
tavu-tus-vaih-to-eh-to
teks-ti-alue
teks-ti-alueen
teks-ti-aluees-ta
teks-ti-edi-to-ria
teks-ti-edi-to-rien
teks-ti-edi-to-ril-la
teks-ti-edi-to-rin
teks-ti-edi-to-ris-sa
teks-ti-edi-to-ris-sa-kin
teks-ti-edi-to-ris-ta
teks-ti-ele-men-til-tä
teks-ti-ele-men-tin
teks-ti-ele-ment-te-jä
teks-ti-ele-ment-ti
teks-ti-ele-ment-tien
teks-ti-ele-ment-tiin
teks-tin-kä-sit-tely-ohjel-mat
teks-tin-kä-sit-tely-ohjel-mis-sa
teks-tin-osan
teks-ti-osa
teks-ti-osaan
teks-ti-osan
teks-ti-osas-sa
text-back-slash
text-block
text-block-ori-gin
text-bullet
text-cite
text-color
text-comp-word-mark
text-el-lip-sis
text-en-glish
text-float-sep
text-font
text-frac-tion
text-greek
text-height
text-pos
text-quote-dbl
text-quote-single
text-sub-script
text-super-script
text-width
the-author
the-bib-li-og-ra-phy
the-chapter
the-date
the-equa-tion
the-foot-note
the-mp-foot-note
the-page
the-section
the-sub-section
the-table
the-title
this-page-style
tie-teen-aloil-la
tie-to-kone-avus-teis-ta
tie-to-kone-oh-jel-ma
tie-to-kone-oh-jel-mal-le
tie-to-kone-oh-jel-mas-ta
tie-to-kone-oh-jel-mat
tie-to-kone-oh-jel-mia
tii-vis-tel-mä-osan
tikz-pic-ture
title
title-con-tents
title-for-mat
title-label
title-rule
title-sec
title-spacing
title-toc
toc-depth
toc-level
top-frac-tion
top-number
top-rule
top-sep
total
total-height
total-number
tot-count
tot-pages
tot-value
tt-fam-i-ly
two-column
two-side
ty-po-gra-fia
ty-po-gra-fiaa
ty-po-gra-fiaan
ty-po-gra-fian
ty-po-gra-fias-sa
ty-po-gra-fias-ta
ty-po-gra-fi-seen
ty-po-gra-fi-sen
ty-po-gra-fi-se-na
ty-po-gra-fi-ses-ta
ty-po-gra-fi-ses-ti
ty-po-gra-fi-set
ty-po-gra-fi-sia
ty-po-gra-fi-sin
ty-po-gra-fi-sis-ta
ty-po-gra-fis-ten
tyy-li-ase-tuk-set
tyy-li-ase-tus
tyy-li-ase-tus-ten
tär-keys-as-teil-le
tär-keys-as-tei-sia
täys-ym-py-rän
UL-depth
ulko-asu
ulko-asua
ulko-asun
ulko-asus-ta
ulko-asus-taan
ulko-asu-tee-mo-ja
ulko-asuun
UL-thick-ness
under-line
Uni-code
Uni-coden
url-style
use-color-theme
use-counter
use-font-theme
use-inner-theme
use-outer-theme
use-pack-age
use-theme
vaih-to-eh-don
vaih-to-eh-dot
vaih-to-eh-to
vaih-to-eh-toi-ses-ti
vaih-to-eh-to-ja
vaih-to-eh-to-jen
vaih-to-eh-to-na
vai-ku-tus-alue
vai-ku-tus-alueena
vai-ku-tus-aluet-ta
value
va-paa-eh-toi-sia
va-paa-eh-tois-ta
vas-taan-otta-jal-le
vas-taan-otta-jil-le
vek-tori-gra-fiik-kaa
vek-tori-gra-fiik-ka-font-te-ja
vek-tori-gra-fiik-ka-ku-viot
ver-ba-tim
verse
vir-he-il-moi-tuk-seen
vir-he-il-moi-tuk-sen
vir-he-il-moi-tus
vir-he-il-moi-tus-ta
vline
vol-ume
vspace
vä-him-mäis-etäi-syys
väli-ai-kai-ses-ti
väli-ai-kais-tie-dos-toa
väli-ai-kais-tie-dos-to-jen
väli-ai-kais-tie-dos-ton
väli-ai-kais-tie-dos-toon
väli-ai-kais-tie-dos-tot
väli-ot-si-kois-sa
väli-ot-si-kois-ta
väli-ot-si-kon
väli-ot-si-kot
vä-ri-asioi-ta
wall-cal-en-dar
with-in
wrap-fig
wrap-fig-ure
wrap-table
Xe-latex
yh-dys-osas-ta
yh-dys-osien
yh-tä-lö-esi-mer-kin
yleis-esi-merk-ki
yleis-ohje
yleis-yhdys-merk-ki
ylä-in-dek-si
ylä-in-dek-si-nume-roi-den
ylä-in-dek-si-nume-rois-ta
ylä-in-dek-si-nume-rol-la
ylä-in-dek-si-nume-ron
ylä-in-dek-sit
ylä-osaan
ylä-osas-sa
ään-tö-asu
}


\pagestyle{empty}

\newgeometry{top=1cm, bottom=1.8cm, hmargin=1.3cm}

% \newgeometry{ top=1cm, bottom=1.8cm, hmargin=1.3cm, papersize={158mm,
%     220mm}, layout=a5paper, layoutoffset={5mm, 5mm}, showcrop }

\pdfbookmark[0]{Nimiö}{sivu/nimiö}
\DTMsetstyle{finnish-numeric}

\vspace*{.2\textheight}

{

  \setlength{\parindent}{0pt}

  \fontsize{16bp}{16bp}\rmfamily \tekija

  \fontsize{52bp}{52bp}\sffamily\bfseries%
  \hspace{-3bp}%
  {\addfontfeatures{LetterSpace=-4} Käytännöllistä}

  \fontsize{65bp}{65bp}\selectfont%
  \hspace{-5bp}%
  \LaTeX{}ia

}

\vfill

{

  \raggedleft
  \alaotsikko \\
  Versio \versio

}

\clearpage
\restoregeometry
\pdfbookmark[0]{Tekijänoikeus}{sivu/tekijänoikeus}

\null\vfill

{
  \setlength{\parindent}{0em}
  \setlength{\parskip}{1.2ex plus .1ex}

  \section*{\otsikko}

  \textsc{Tekijä:} \tekijat

  \textsc{Versio:} \versio

  \textsc{Päiväys:} \DTMtoday{} kello \DTMcurrenttime{} (ensijulkaisu:
  26.12.2021)

  \textsc{Saatavissa:} \url{https://github.com/tlikonen/latex-opas}

  \textsc{Lisenssi:} \emph{Creative Commons Nimeä-Jaa\-Samoin 4.0
    Kansainvälinen} (\textsc{cc by-sa} 4.0). Lisenssi antaa sinulle
  luvan kopioida ja levittää tätä teosta tai sen osia missä tahansa
  välineessä ja muodossa. Sisältöä saa muokata, ja sen pohjalta saa
  luoda uusia teoksia mihin tahansa tarkoitukseen, myös kaupallisesti.
  Ehdot ovat seuraavat:

  \begin{list}{\textbullet}{
      \setlength{\leftmargin}{1em}
      \setlength{\labelsep}{.35em}
      \setlength{\topsep}{0ex}
      \setlength{\partopsep}{0ex}
      \setlength{\itemsep}{0ex}
    }
  \item Sinun on mainittava tekijä(t) asianmukaisesti, tarjottava linkki
    lisenssin koko tekstiin (ks. alla) sekä mainittava, mikäli olet
    tehnyt muutoksia.
  \item Jos muokkaat teosta tai luot sen pohjalta uuden teoksen, sinun on
    jaettava muutoksiasi samalla lisenssillä kuin alkuperäistä teosta.
  \item Et saa asettaa sellaisia oikeudellisia ehtoja tai teknisiä
    estoja, jotka estävät muita tekemästä asioita, jotka tämä lisenssi
    sallii.
  \end{list}

  Lisenssin koko teksti: \\
  \url{https://creativecommons.org/licenses/by-sa/4.0/legalcode.fi}

}

\cleardoublepage

\pagestyle{plain}
\pdfbookmark[0]{Sisällys}{sivu/sisällys}
\setcounter{tocdepth}{3}
\tableofcontents

\setcounter{secnumdepth}{-1}
% Tekijä:   Teemu Likonen <tlikonen@iki.fi>
% Lisenssi: Creative Commons Nimeä-JaaSamoin 4.0 Kansainvälinen (CC BY-SA 4.0)
% https://creativecommons.org/licenses/by-sa/4.0/legalcode.fi

\chapter{Esipuhe}

Tex ja Latex kehitettiin alun perin parantamaan tietokoneavusteista
tekstin latomista. Ohjelmat syntyivät 1970\-/ luvun lopulla ja 1980\-/
luvun alkupuolella eli ajalla, jolloin tietokoneilla ei vielä kovin
hyvin osattu tuottaa laadukasta typografiaa ja painotuotteita. Texin
alkuperäinen luoja on Donald Knuth, ja Latexin kehitti Leslie Lamport.

Texin ja Latexin myötä tietokonetypografia parani: tekstieditorin ja
merkintäkielen avulla kirjoittaja pystyi varsin helposti tuottamaan
hyvää jälkeä niin kuin typografian ja ladonnan ammattilaiset aikoinaan.
Varsinkin akateemisten tekstien tuottamisessa Latex sai vankan
jalansijan. Kirjoittaja keskittyy lähinnä sisältöön, ja
tietokoneohjelmat hoitavat taittamisen eli ulkoasun suunnittelun (Latex)
ja lopullisen dokumentin latomisen (Tex). Typografinen ammattitaito on
ohjelmoitu sisään näihin tietokoneohjelmiin.

Sitten tulivat tekstinkäsittelyohjelmat, taitto\-/ ohjelmat ja uusi
käsite \textsc{wy\-si\-wyg} eli \englantik{what you see is what you
  get}. Suoraan tietokoneen ruudulta eli graafisesta tietokoneohjelmasta
näki lopullisen painotuotteen sellaisenaan. Mikä sen helpompaa? Jokainen
kirjoittaja oli nyt samassa hetkessä myös latoja, joka heittelee
kirjakkeita tietokoneen ruudulle peräkkäin: sanoiksi, riveiksi,
palstoiksi ja lopulta valmiiksi dokumentiksi. Tekstien parissa
työskentely helpottui, mutta typografiaa tämä kehitys ei parantanut,
koska yhä useammat kirjoittajat saivat vastuulleen myös ulkoasun
suunnittelun ja lopullisten dokumenttien valmistuksen. Vuosisatojen
aikana kehittynyt typografinen osaaminen ei enää sisältynyt
tietokoneohjelmaan.

Tässä tilanteessa olemme edelleenkin: lähes jokainen kirjoittaja vastaa
itse niin sisällöstä kuin typografiastakin eli lopullisesta ulkoasusta.
Jokainen voi olla tekstinsä julkaisija. Visuaalisten
tekstinkäsittely\-/\ ja taitto\-/ ohjelmien yleistyminen ei kuitenkaan
lopettanut Latexin tarinaa. Latexin kehitys ei jäänyt 1980\-/ luvulle,
vaan sen ympärillä tapahtuu jatkuvasti edelleenkin. Esimerkiksi
nykyaikainen fonttitekniikka ja tietokoneiden laajat merkistöt ovat myös
Latexin käyttäjän arkipäivää.

Jos siis typografia, laadukkaat julkaisut ja niihin liittyvä tekniikka
kiinnostavat, on Latexilla varmasti paljon annettavaa nykyajallekin.
Toivottavasti tästä oppaasta on apua tutustumismatkassa -- ja sen
jälkeenkin. Tervetuloa mukaan!

\setcounter{secnumdepth}{2}
% Tekijä:   Teemu Likonen <tlikonen@iki.fi>
% Lisenssi: Creative Commons Nimeä-JaaSamoin 4.0 Kansainvälinen (CC BY-SA 4.0)
% https://creativecommons.org/licenses/by-sa/4.0/legalcode.fi

\chapter{Valmistautuminen}

Tämän pääluvun tarkoituksena on johdatella uudet ihmiset Latexin pariin.
Tämän luettuasi sinun pitäisi olla valmis aloittamaan Latexin käyttö ja
opiskelemaan sen tekniikkaa eteenpäin. Jos taas olet jo tottunut
kääntämään Latex\-/ lähdetiedostoja, voit aivan hyvin hypätä tämän luvun
yli tai ehkä vain silmäillä tekstiä lisävinkkien toivossa.

\section{Käsitteet ja nimet}

Latex ja sen ympärille rakentuneet ohjelmistot ovat aika monimutkainen
kokonaisuus, johon kuuluu eri\-/ikäisiä ja abstraktiotasoltaan erilaisia
osia. Mukaan kuuluu tietenkin konkreettisia tietokoneohjelmia, jotka
tekevät konkreettisia, ennalta määriteltyjä asioita. Mukaan kuuluu
kuitenkin myös ihmisten luomia abstrakteja käsitteitä, jotka ovat
vaikeammin määriteltävissä.

Internetissä näkyy silloin tällöin termi- ja käsitekeskusteluja, jossa
ihmetellään, mihin mikäkin Latexiin liittyvä palikka kuuluu
käsitteellisesti. Millainen suhde joillakin uudemmilla osilla on
vanhempiin? Oikeiden termien ja käsitteiden osaamisesta on hyötyä
ainakin silloin, kun pyytää verkossa apua suurelta yleisöltä.
Viestintähän aina vaatii, että puhutaan suunnilleen samaa kieltä, joten
seuraavaksi selvennetään hieman Latexiin liittyviä peruskäsitteitä.

\subsection{Tex ja Latex}

Tex on tekstin ladontaan erikoistunut ohjelmointikieli. Kielen avulla
voidaan antaa ladontaohjeita ja ilmaista muuta siihen liittyvää
logiikkaa. Ohjelmointikielellä kirjoitettujen ohjeiden perusteella
tietokoneohjelma osaa latoa tekstidokumentin ihmisten luettavaksi.
Niinpä Tex on myös tietokoneohjelma (\koodi{tex}), joka osaa lukea
Tex\-/ ohjelmointikieltä sisältävän tekstitiedoston ja tuottaa sen
perusteella valmiin julkaistavan dokumentin. Yleensä Tex\-/
ohjelmointikieltä ei kuitenkaan käytetä suoraan tekstidokumenttien
toteuttamiseen. Ohjelmointikieltä kirjoittavat tavallisimmin vain ne,
jotka haluavat kehittää itse ladontajärjestelmää paremmaksi muita
ihmisiä varten.

Latex toimii korkeammalla abstraktiotasolla kuin Tex. Se on laaja
kokoelma toimintoja, jotka piilottavat monimutkaiset tekniset
yksityiskohdat ja tarjoavat ihmisille varsin helppokäyttöisen
merkintäkielen, jolla omat tekstidokumentit voi toteuttaa. Latex\-/
merkintäkielen kirjoittaminen ei ole ohjelmointia, vaan se on oman
dokumentin sisällön, rakenteen ja ulkoasun kuvailua tietynlaisten
merkintätapojen avulla.

Latex\-/ merkintäkielellä kuvatut dokumentit välitetään
tietokoneohjelmalle jatkokäsiteltäväksi. Niinpä Latex on myös
tietokoneohjelma (\koodi{latex}, \koodi{pdflatex}), jolla
merkintäkielinen lähdetiedosto käännetään julkaistavaksi dvi- tai pdf\-/
dokumentiksi.%
\footnote{\textsc{dvi} = device independent file format; \textsc{pdf} =
  portable document format.}

Latex\-/ järjestelmästä käytetään tällä hetkellä versiota Latex~2ε, joka
julkaistiin jo vuonna 1994 ja johon ilmestyy pieniä parannuksia
vuosittain. Perusosat ovat kuitenkin varsin muuttumattomia. Varsinainen
kiinnostava kehitys tapahtuukin esimerkiksi Latexin kääntäjissä
(Lualatex ja Xelatex) sekä eri tekijöiden dokumenttiluokissa (luku
\ref{luku/dokumenttiluokat}) ja laajennuspaketeissa. Viimeksi mainitut
laajentavat perus\-/ Latexia tuomalla niihin lisää helppokäyttöisiä
toimintoja.

\subsection{Lualatex ja Xelatex}

Nykyaikana Latex\-/ dokumentteja ei juuri käännetä alkuperäisillä
kääntäjillä (\koodi{latex}, \koodi{pdflatex}, \koodi{tex}) vaan
kehittyneemmillä kääntäjäohjelmilla.%
\footnote{Englannin kielellä Latexin kääntäjiä on tapana kutsua
  yleisnimellä \englantik{engine} 'kone, moottori'.} Niistä tärkeimmät
ovat Lualatex ja Xelatex.%
\footnote{Tietokoneohjelmat \koodi{lualatex} ja \koodi{luatex} sekä
  \koodi{xelatex} ja \koodi{xetex}.} Ne muun muassa osaavat lukea
Unicode\-/ merkistöllä kirjoitettuja lähdedokumentteja ja käyttää
nykyaikaisia \englanti{True Type}- ja \englanti{Open Type} \=/fontteja,
mitä alkuperäinen Latex ja Tex eivät osaa.

Lualatexilla ja Xelatexilla ei ole ohjelmien käyttäjän kannalta
suurtakaan eroa -- ei välttämättä mitään näkyvää eroa. Xelatex tehtiin
ensin, ja tarkoituksena oli saada Unicode\-/ merkistön tuki ja
fonttiasiat ajan tasalle. Myöhemmin jotkut ajattelivat, että Lua\-/
ohjelmointikieli täytyy saada mukaan, koska ominaisuudesta on hyötyä
joillekuille laajennuspakettien tekijöille. Lua\-/ kielen
sisällyttäminen oikeastaan pakotti kirjoittamaan koko kääntäjän koodin
uusiksi, ja syntyi Lualatex.

Kääntäjien toteutuksissa on muitakin sisäisiä eroja, esimerkiksi
fonttien käsittelyssä. Xelatex oli pitkään suositumpi ja paremmin tuettu
eri laajennuspaketeissa, mutta erot ovat sittemmin tasoittuneet. Lua\-/
ohjelmointikieli on alkanut vaikuttaa Lualatexin eduksi. On joitakin
laajennuspaketteja tai niiden yksittäisiä ominaisuuksia, jotka vaativat
toimiakseen Lualatexin. Sen tulevaisuus vaikuttaa valoisammalta.

Latexin käytön alkutaipaleella voi vaikka arpoa kolikon avulla, kumpaa
kääntäjää käyttää, sillä niiden erot eivät ihan helposti tule esiin.
Kääntäjän vaihtaminen on joka tapauksessa helppoa. Tässä oppaassa
mainitaan siellä täällä pari ominaisuutta, jotka toimivat vain toisella
kääntäjällä: toiset Lualatexilla, toiset Xelatexilla.

\subsection{Latex yläkäsitteenä}

Jotta kaikki olisi mahdollisimman sekavaa, \emph{Latex} toimii myös
yleisnimityksenä tälle kaikelle. Se esiintyy ilmauksissa kuten
''Toteutin dokumentin Latexilla'' tai ''Tämä artikkeli on tehty
Latexilla''. Ilmaukset sitten tarkoittavat suunnilleen seuraavanlaista:
Henkilöllä on asennettuna tietokoneelle jokin Latex\-/ jakelukokonaisuus
(kuten Tex Live). Hän on kirjoittanut tekstieditorilla (kuten
\textsc{gnu} Emacsilla) tekstitiedoston, jossa on dokumentin sisältö ja
Latex\-/ merkintäkielisiä komentoja mutta ehkä myös joitakin Tex\-/
komentoja. Sitten hän on kääntänyt eli ladotuttanut tekstitiedostonsa
pdf\-/ tiedostoksi Latex\-/ ladontaohjelman jollakin toteutuksella kuten
Lualatexilla tai Xelatexilla.

Meille taitaa riittää vain Latexista puhuminen, mutta siitäkin on
mainittava vielä yksi asia. Latexin harrastajat tykkäävät käyttää
dokumenttiensa leipätekstissä ladontajärjestelmän logoja kuten \TeX{} ja
\LaTeX{}. Usein teksteissä näkyy myös logojen pohjalta mukailtuja
kirjoitusasuja TeX ja LaTeX.

Kielenhuollon suositusten\footnote{Kotimaisten kielten keskus:
  \url{https://www.kotus.fi/}} mukaan logojen eikä erikoisten
kIRjoiTusAsuJen paikka ei ole asiatyylisten tekstilajien leipätekstissä.
Nimet ovat osa kielen järjestelmää ja käyttäytyvät normaalissa tekstissä
sen mukaisesti. Niinpä tässä oppaassa käytetään erisnimiä kielenhuollon
normien mukaisesti, esimerkiksi Tex ja Latex. Koodi ja komennot ovat
siinä muodossa kuin ne tietokoneelle annetaan, esimerkiksi
\koodi{lualatex}. Tasalevyinen, kirjoituskonetyylinen fontti on merkkinä
siitä, että kyse on tietokonekoodista.

\section{Asentaminen tietokoneelle}
\label{luku/asentaminen}

Tavallisin tapa Latexin käyttöönottoon on jonkin Latexin jakelupaketin
asentaminen. Jakelupaketti sisältää Latexin perusosien lisäksi paljon
laajennuspaketteja ja niiden ohjekirjoja. Kaikkea ei kukaan tarvitse,
mutta kun yllättävä tarve tulee tai lukee vinkkejä verkkokeskusteluista,
on mukavaa huomata, että paketti olikin itsellä jo valmiina. Siksi
kokonaisen jakelupaketin asentaminen on helpoin tapa.

Linuxissa ja muissa Unix\-/ tyyppisissä käyttöjärjestelmissä käytetään
yleensä Tex Live \=/nimistä jakelua. Se on todennäköisesti saatavilla
käyttöjärjestelmäjakelun pakettivarastoista. Esimerkiksi Debianiin%
\footnote{\url{https://www.debian.org/}} ja sen kaltaisiin järjestelmiin
on asennuspaketti ''texlive-full'', joka asentaa kaiken helposti ja
kerralla.

Windows\-/ käyttöjärjestelmälle on saatavilla Tex Liven lisäksi Miktex
ja Protext. Mac \textsc{os} \=/käyttöjärjestelmän kanssa käytettäneen
yleensä Mactex\-/ nimistä jakelua.

\section{Apuohjelmia}

\subsection{Tekstieditori}

Lähdetiedostot eli Latex\-/ merkintäkieltä sisältävät tiedostot (luku
\ref{luku/lähdetiedosto}) ovat puhdasta tekstiä, tekstitiedostoja, joita
kirjoitetaan ja muokataan tekstieditorilla. Kirjoittamiseen kannattaa
käyttää kunnollista tekstieditoria, koska se on tärkein työkalu ja sen
kanssa ollaan eniten tekemisissä.

Pyri löytämään sellainen editori, joka osaa värjätä tekstiä Latexin tai
Texin tekstipiirteiden mukaisesti. Väreillä ei sinänsä ole merkitystä,
mutta editorin laadusta se yleensä kertoo paljon. Jos editori tuntee
erilaisten ohjelmointi\-/{} ja merkintäkielten luonnetta ja osaa merkitä
kielen avainsanoja havainnollisilla väreillä, se todennäköisesti on
tehty tehokkaaseen ohjelmointiin ja muuhun vastaavaan työskentelyyn.
Ihan yksinkertaisiin editoreihin ei tuollaisia ominaisuuksia yleensä
tehdä.

\subsection{Pdf-katselin}

Latex\-/ kääntäjät eli \=/moottorit kuten Lualatex ja Xelatex tuottavat
pdf\-/ tiedoston, ja niiden katselemiseen tarvitaan tietenkin oma
ohjelmansa. Sellaisia on saatavilla paljon erilaisia, ja melkein mikä
tahansa kelpaa, mutta yksi tietty ominaisuus olisi toivottavaa olla:
muuttuneen pdf\-/ tiedoston automaattinen lataaminen.

Välillä työskentely on sitä, että tehdään Latex\-/ dokumenttiin pieni
muutos, käännetään se ja katsotaan pdf:ää. Lopputulos ei ehkä ihan
miellytä. Muokataan tekstiä tai asetuksia vähän, käännetään ja
katsotaan, miltä ladottu pdf nyt näyttää.

On suuri apu, jos pdf\-/ katselimessa ei tarvitse joka kerta valikoiden
kautta avata samaa tiedostoa uudelleen, vaan ohjelma itse huomaa, että
jo avattu tiedosto muuttui tiedostojärjestelmässä, ja lataa sen
automaattisesti uudelleen. Jotkin pdf\-/ ohjelmat osaavat tämän. Jotkin
ohjelmat eivät ihan osaa mutta osaavat sentään yhdellä
näppäinpainalluksella avata saman pdf:n uudelleen
tiedostojärjestelmästä.

Hyvän tekstieditorin ja pdf\-/ katselimen kanssa työskentely on sujuvaa.
Parhaimmillaan editorissa tietty näppäinkomento tallentaa ja kääntää
dokumentin, ja pian pdf\-/ katselin lataa muuttuneen pdf:n
automaattisesti näkyviin. Sekä editorin että pdf\-/ katselimen voi pitää
esillä samanaikaisesti.

\subsection{Latexmk}
\label{luku/latexmk}

Erinomaisen hyödyllinen apuohjelma on Latexmk, koska se helpottaa
dokumenttien kääntämistä ja muutakin työskentelyä. Varsin usein Latex\-/
dokumentit täytyy kääntää useita kertoja ennen kuin pdf\-/ tiedosto on
valmis. Tämä johtuu siitä, että dokumentit sisältävät usein
ristiviitteitä eli viittauksia dokumentin toisiin osiin. Latex ei saa
ristiviitteitä kohdalleen yhdellä kääntämisellä, vaan ensin se
kirjoittaa viittausten kohteet muistiin väliaikaistiedostoon ja
seuraavilla kääntökerroilla käyttää väliaikaistiedostoa apunaan.

Tavallinenkin Latexin kääntäjä kyllä huomauttaa tietokoneen käyttäjää,
kun uusintakäännös on tarpeen, mutta Latexmk\-/ ohjelma käynnistää
uusintakäännöksen itse, aina kun se on tarpeellista. Alla ovat
esimerkkikomennot Latex\-/ dokumentin kääntämiseen Lualatexilla ja
Xelatexilla.

\begin{koodilohkosis}
latexmk -lualatex teksti.tex
latexmk -xelatex  teksti.tex
\end{koodilohkosis}

\noindent
Työskentelyä erityisen paljon helpottava valitsin on \koodi{\=/pvc}. Kun
tuo valitsin on mukana komennossa, Latexmk jää tarkkailemaan annettua
Latex\-/ lähdetiedostoa, ja kun se huomaa tiedoston muuttuneen, se
kääntää tiedoston automaattisesti uudelleen. Kirjoittajan ei siis
tarvitse muuta kuin tallentaa tiedosto tekstieditorista, ja
tarkkailutilassa oleva Latexmk kääntää sen aina itsestään.

Muitakin hyödyllisiä toimintoja on mukana. Seuraavista esimerkeistä
ensimmäinen komento poistaa kääntämisen aikana luodut
väliaikaistiedostot\footnote{Kääntäjän luomien väliaikaistiedostojen
  nimien päätteitä: \koodi{log}, \koodi{aux}, \koodi{out} ym.}, ja
jälkimmäinen komento poistaa kaikki luodut tiedostot eli
väliaikaistiedostojen lisäksi myös valmiin pdf\-/ tiedoston.

\begin{koodilohkosis}
latexmk -c teksti.tex
latexmk -C teksti.tex
\end{koodilohkosis}

\noindent
Edellisissä esimerkeissä käsitellään lähdetiedostoa nimeltä
\koodi{teksti.\katk tex}, mutta jos lähdetiedostoa ei anna komennolle
lainkaan, käännetään kaikki nykyisessä hakemistossa olevat
\koodi{tex}\-/ päätteiset tiedostot.

Latexmk\-/ ohjelmalle voi tehdä asetustiedoston, johon voi kirjoittaa
omaan käyttöön sopivat asetukset. Asetustiedosto sijoitetaan
tiedostojärjestelmässä käyttäjän kotihakemistossa olevaan
asetustiedostohakemistoon (\koodi{\textasciitilde /.config}). Esimerkki
\ref{esim/latexmkrc} näyttää, mitä se voisi ehkä sisältää.

\begin{esimerkki*}
\begin{koodilohko}
$pdf_mode = 4;   # 4=lualatex, 5=xelatex
$lualatex = 'lualatex -interaction=nonstopmode -shell-escape %O %S';
$xelatex  = 'xelatex  -interaction=nonstopmode -shell-escape %O %S';
push @generated_exts, "run.xml";
push @generated_exts, "nav";
push @generated_exts, "snm";
$pdf_previewer = 'okular %S';
\end{koodilohko}
  \caption{Latexmk\-/ ohjelman asetustiedosto (\koodi{\textasciitilde
      /.config/\katk latexmk/\katk latexmkrc})}
  \label{esim/latexmkrc}
\end{esimerkki*}

Esimerkin ensimmäisen rivin asetus määrittää, miten pdf\-/ tiedostot
tuotetaan tai mitä kääntäjää käytetään oletuksena. Toisella ja
kolmannella rivillä määritellään, millä tavoin Lualatex ja Xelatex
suoritetaan. Tässä esimerkissä oletusasetuksiin on lisätty
\koodi{-inter\-action=\katk non\-stop\-mode}, joka estää kaiken
vuorovaikutteisen toiminnan. Asetus on tarpeen ainakin silloin, kun
kääntäjä käynnistetään toisesta ohjelmasta kuten tekstieditorista eikä
vuorovaikutus kääntäjän kanssa ole mahdollista. Valitsin
\koodi{-shell-escape} kytkee päälle ominaisuuden, jota tarvitaan
joidenkin laajennuspakettien toimintaan.\footnote{Ainakin
  asiahakemistopaketit \paketti{indextools} ja \paketti{imakeidx}
  tarvitsevat \koodi{-shell-escape}\-/ toiminnon (luku
  \ref{luku/asiasanat}).}

Esimerkin \ref{esim/latexmkrc} riveillä 4--6 on komennot, joilla
lisätään kääntämisen aikana mahdollisesti syntyvien
väliaikaistiedostojen päätteitä. Latexmk\-/ ohjelma tuntee yleisimmät
väliaikaistiedostot (\koodi{log}, \koodi{aux}, \koodi{out} ym.), mutta
näillä komennoilla mukaan voi lisätä harvinaisempia, joita se ei tunne.
Viimeinen rivi määrittää pdf\-/ katseluohjelman, joka käynnistetään, kun
käytetään valitsimia \koodi{\=/pv} tai \koodi{\=/pvc}.

\subsection{Texdoc}

Latexin kirjoittajan täytyy silloin tällöin lukea ohjekirjoja. Vaikka
Latexin perusosat joskus oppisikin ulkoa, ei voi koskaan muistaa
kaikkien hyödyllisten laajennuspakettien kaikkia ominaisuuksia.

Tex Live \=/jakelun (luku \ref{luku/asentaminen}) mukana tulee mainio
komentotulkissa toimiva komento \koodi{texdoc}, jolla voi hakea ja avata
omaan järjestelmään asennettuja Latex\-/ aiheisia ohjeita. Jos vaikka
haluaa tutustua esimerkissä \ref{esim/ensimmäinen} mainittavaan
\paketti{fontspec}\-/ pakettiin syvällisemmin, tarvitsee vain komentaa
\koodi{texdoc fontspec}, ja paketin pdf\-/ muotoinen ohjekirja avautuu.

\section{Lähdetiedostot}
\label{luku/lähdetiedosto}

Latexin lähdetiedostot eli lähdedokumentit ovat tekstitiedostoja eli
puhdasta tekstiä. Esimerkissä \ref{esim/ensimmäinen} on tyypillisen
dokumentin vähimmäissisältö, jota voi käyttää harjoitteluun sekä
myöhemminkin pohjana omille töille.

Tallenna esimerkin sisältö tekstieditorin avulla tiedostoon vaikkapa
nimellä \koodi{teksti.\katk tex}. Käännä eli lado se pdf\-/ tiedostoksi
käyttämällä jotakin seuraavista komennoista (valitse yksi):

\begin{koodilohkosis}
lualatex teksti.tex
xelatex  teksti.tex
latexmk -lualatex teksti.tex
latexmk -xelatex  teksti.tex
\end{koodilohkosis}

\noindent
Tuloksena pitäisi olla tiedosto \koodi{teksti.\katk pdf}, jota voi
ihailla jollakin pdf\-/ tiedostojen katseluohjelmalla. Kääntämisen
aikana syntyy automaattisesti muitakin tiedostoja, jotka on tarkoitettu
lähinnä kääntäjän omaan väliaikaiseen käyttöön. Niitä ei tarvitse
säilyttää.

Esimerkin \ref{esim/ensimmäinen} ensimmäisellä rivillä määritellään
dokumenttiluokka \luokka{article}, joka on tietynlainen sivupohja tai
asetusten kokoelma, jonka perustalle aletaan rakentaa omaa dokumenttia.
Luokka \luokka{article} on tyypillinen valinta lyhyehköille teksteille.
Lisätietoa dokumenttiluokista on luvussa \ref{luku/dokumenttiluokat}.

Riveillä 2--4 käytetään komentoa \komento{usepackage}, jonka avulla
otetaan käyttöön sivun asetuksista huolehtiva \paketti{geometry}\-/
paketti, fonttiasetuksia hoitava \paketti{fontspec}\-/ paketti ja
kieliasetuksista vastaava \paketti{polyglossia}\-/ paketti. Näitä kolmea
tarvitaan melkein joka kerta dokumenteissa, ja niihin palataan tarkemmin
luvuissa \ref{luku/sivuasetukset}, \ref{luku/kirjaintyypit} ja
\ref{luku/kieliasetukset}.

Seuraavilla riveillä asetetaan kieleksi suomi (\koodi{finnish}) ja
määritetään oletuksena käytettävä fontti tai oikeastaan kokonainen
kirjainperhe. \englanti{Latin Modern Roman} \=/kirjainperheen tilalle
voi toki asettaa jonkin muunkin. Fontin oletuskoko on 10 typografista
pistettä, mutta tässä esimerkissä se venytetään 1,3\-/ kertaiseksi eli
13 pisteeseen. Riviväliin liittyvä kerroin asetetaan rivillä 8.

\begin{esimerkki*}
  \komentoi{begin}
  \komentoi{documentclass}
  \komentoi{end}
  \komentoi{linespread}
  \komentoi{setdefaultlanguage}
  \komentoi{setmainfont}
  \komentoi{usepackage}
  \luokkai{article}
  \pakettii{fontspec}
  \pakettii{geometry}
  \pakettii{polyglossia}
  \ymparistoi{document}

\begin{koodilohko}
\documentclass{article}
\usepackage[a4paper,top=20mm,bottom=30mm,left=20mm,right=20mm]{geometry}
\usepackage{fontspec}
\usepackage{polyglossia}

\setdefaultlanguage{finnish}
\setmainfont{Latin Modern Roman}[Scale=1.3]
\linespread{1.4}

\begin{document}

Minun Latex-dokumenttini!

\end{document}
\end{koodilohko}
  \caption{Latex\-/ lähdedokumentin runko ja perusasetukset}
  \label{esim/ensimmäinen}
\end{esimerkki*}

Dokumentin alkuosaa riville 9 saakka kutsutaan esittelyosaksi
(\englanti{preamble}). Tässä osassa ladataan tarvittavat paketit ja
määritetään dokumentin asetuksia ja taustatietoja. Riviltä 10 alkaa
varsinainen tekstiosa eli dokumentin sivuille ladottava sisältö. Se osa
kirjoitetaan \ymparisto{document}\-/ ympäristön sisään eli riveillä 10
ja 14 olevien ympäristön aloitus\-/\ ja lopetuskomentojen väliin
(\komento{begin}, \komento{end}).

Tällaisen merkintäkielen avulla dokumentit kirjoitetaan. Osa
merkintäkielen komennoista tulee Latexin perusosasta ja osa tulee
erikseen ladattavista paketeista (\paketti{geometry},
\paketti{fontspec}, \paketti{polyglossia} ym.). Komentoja voi luoda
itsekin.

Myöhempää käyttöä varten voisi olla hyödyllistä tallentaa tämänkaltainen
pohjadokumentti. Välttyy samojen perusjuttujen kirjoittamiselta, kun voi
aloittaa työt valmiista dokumenttipohjasta.

Lähdetiedoston nimissä kannattaa pitäytyä melko suppeassa
merkkivalikoimassa, ja varsinkin välilyöntejä kannattaa välttää.
Nimittäin kääntämisen aikana Latex ja sen paketit luovat
väliaikaistiedostoja, joilla on sama nimen osa kuin lähdetiedostossa, ja
näitä tiedostoja saattavat käsitellä monet erilaiset taustalla
vaikuttavat työkaluohjelmat. Tiedoston nimissä olevat välilyönnit ja
ehkä muutkin erikoisemmat merkit aiheuttavat ongelmia.

Pitkä lähdedokumentti voi olla mielekästä jakaa useammaksi tiedostoksi.
Yhteen lähdetiedostoon voi sisällyttää toisen tiedoston käyttämällä
\komento{input}\-/ komentoa. Komennon argumentiksi annetaan ladattavan
lähdetiedoston nimi:

\komentoi{input}
\begin{koodilohkosis}
\input{toinen.tex}
\end{koodilohkosis}

\noindent
Hieman vastaava komento on \komento{include}, joka myös lisää
automaattisen sivunvaihdon (\komento{clearpage}, luku
\ref{luku/sivunvaihdot}) komennon kohdalle.

Nyt lienee sopiva aika alkaa opiskella itse Latexia eli merkintäkieltä
ja kokeilla sen ominaisuuksia itse. Tätä opasta ei tarvitse lukea
järjestyksessä luku luvulta eteenpäin, vaan eri aiheita voi vapaasti
opiskella mielenkiinnon ja tarpeiden mukaan. Onnea matkaan!

% Tekijä:   Teemu Likonen <tlikonen@iki.fi>
% Lisenssi: Creative Commons Nimeä-JaaSamoin 4.0 Kansainvälinen (CC BY-SA 4.0)
% https://creativecommons.org/licenses/by-sa/4.0/legalcode.fi

\chapter{Merkintäkieli ja perustekniikka}

Latex on merkintäkieli, mikä tarkoittaa, että se sisältää omat tapansa
dokumentin rakenteen ja sisällön kuvaamiseen. Kaikkea ei kirjoiteta
lähdedokumenttiin sellaisenaan, vaan täytyy käyttää tiettyjä kielen
sääntöjen mukaisia merkintätapoja tai komentoja.

Tässä luvussa käsitellään merkintäkielen perusasioita, joita on tarpeen
ymmärtää ennen kuin voi tehokkaasti toimia Latexin parissa. Kaikkea ei
tarvitse opetella ulkoa, mutta tähän lukuun on hyvä palata välillä
kertaamaan perustekniikkaa.

\section{Merkistö}

Latex\-/lähdetiedostoon voi kirjoittaa tekstiä Unicode\-/ merkistöllä ja
sen \textsc{utf}\=/8\-/ koodauksella, jos kääntäjänä on Unicoden osaava
ohjelma kuten Lualatex tai Xelatex. Pääasiassa siis merkit kirjoitetaan
sellaisenaan lähdetiedostoon, mutta on kuitenkin monenlaisia
poikkeuksia, ja niitä käsitellään tässä alaluvussa.

\subsection{Varatut erikoismerkit}

Muutamalla merkillä on perus Latexissa erikoismerkitys, eikä niitä voi
käyttää normaalilla tavalla. Merkit ovat seuraavat:

\begin{koodilohkosis}
% $ ^ _ # & { } ~ \
\end{koodilohkosis}

\noindent
Useimmat näistä merkeistä voi suojata erikoismerkitykseltään
kirjoittamalla niiden eteen kenoviivan (\koodi{\keno}). Tildeä
(\textasciitilde), sirkumfleksia (\textasciicircum) eikä kenoviivaa
itseään ei voi suojata pelkän kenoviivan avulla, koska kenoviivan kanssa
ne muodostavat eräitä muita komentoja. Taulukossa
\ref{tlk/merkkien-suojaus} on koottuna, kuinka edellä mainitut
erikoismerkit suojataan eli saadaan ladottua dokumenttiin sellaisenaan.

\leijutlk{
  \begin{tabular}{cll}
    \toprule
    \ots{Merkki} & \multicolumn{2}{l}{\ots{Kirjoittaminen}} \\
    \midrule
    \koodi{\%} & \komento{\%} \\
    \koodi{\$} & \komento{\$} & \komento{textdollar} \\
    \koodi{\^{}} & \komento{\^{}}\komentoarg{} & \komento{textasciicircum} \\
    \koodi{\_} & \komento{\_} & \komento{textunderscore} \\
    \koodi{\#} & \komento{\#} \\
    \koodi{\&} & \komento{\&} \\
    \koodi{\{} & \komento{\{} & \komento{textbraceleft} \\
    \koodi{\}} & \komento{\}} & \komento{textbraceright} \\
    \koodi{\~{}} & \komento{\~{}}\komentoarg{} & \komento{textasciitilde} \\
    \koodi{\keno} && \komento{textbackslash} \\
    \bottomrule
  \end{tabular}
}{
  \caption{Varattujen erikoismerkkien kirjoittaminen}
  \label{tlk/merkkien-suojaus}
}

Jotkin paketit määrittelevät muitakin erikoismerkkejä. Esimerkiksi
kieliasetuksiin (luku \ref{luku/kieliasetukset}) liittyvät
\paketti{polyglossia}\-/{} ja \paketti{babel}\-/paketit voivat
määritellä pari lainausmerkillä (\koodi{\textquotedbl}) alkavaa,
tavutuksen hallintaan liittyvää komentoa tai erikoismerkkiä (luvut
\ref{luku/tavutus-polyglossia} ja \ref{luku/tavutus-babel}).

\subsection{Sanaväli}
\label{luku/sanaväli}

Välilyönti, sarkainmerkki ja yksi rivinvaihto ovat kaikki tavallisia
sanavälejä Latex\-/dokumentissa, ja näillä kolmella on sama merkitys.
Esimerkiksi rivin lopussa oleva rivinvaihto tarkoittaa samaa kuin
sanojen välissä oleva välilyönti. Välilyöntejä ja sarkainmerkkejä voi
kirjoittaa useita peräkkäin, mutta ne ovat sama asia kuin yksi väli.

\begin{koodilohkosis}
Nämä      kaikki
     ovat            vain
sanoja  peräkkäin  ja               kuuluvat
    samaan kappaleeseen.
\end{koodilohkosis}

\begin{tulossis}
  Nämä kaikki ovat vain sanoja peräkkäin ja kuuluvat samaan
  kappaleeseen.
\end{tulossis}

\noindent
Sanavälien leveys ei ole vakio. Silloin kun tekstipalsta tasataan
molemmista reunoista -- kuten tämänkin oppaan leipätekstissä \==,
rivillä olevia sanavälejä venytetään sopivasti, jotta tekstipalstan
molemmat reunat saadaan tasaiseksi.

Sanavälit eivät kuitenkaan veny loputtomasti, ainakaan
oletusasetuksilla, koska kovin suuret sanavälit olisivat
tekstikappaleessa rumia. Jos sanaväleille haluaa antaa
''hätätilanteissa'' lisää venymisvaraa, täytyy käyttää mittaa
\mitta{emergencystretch}, jota käsitellään tarkemmin tekstikappaleiden
yhteydessä luvussa \ref{luku/kappale}. Mittoihin liittyvää tekniikkaa
käsitellään puolestaan luvussa \ref{luku/mitat}.

Matalatasoinen sanavälejä ja niiden venymistä säätelevä mitta on
\mitta{spaceskip}, johon voi asettaa haluamansa leveyden ja mahdolliset
venymisen rajat. Tätä mittaa ei ole suositeltavaa käyttää tavallisen
tekstin kanssa, mutta se sopii tilanteisiin, joihin tarvitaan hyvin
poikkeukselliset sanavälit.

\komentoi{setlength}
\mittai{spaceskip}
\begin{koodilohkosis}
\setlength{\spaceskip}{0.8em plus 0.3em minus 0.2em}
\end{koodilohkosis}

\noindent
Myös kirjainperheelle voi asettaa oman sanavälikertoimensa fontin
asetusten \koodi{Word\-Space}\-/valitsimella. Tätä asetusta käsitellään
fonttien yhteydessä luvussa \ref{luku/fontit-välistys}.

\subsection{Rivinvaihto lähdedokumentissa}
\label{luku/rivinvaihtomerkit}

Lähdedokumentissa olevat rivinvaihdot tulkitaan vain sanaväleiksi
eivätkä ne vaihda riviä lopullisessa dokumentissa. Jos ladottuun
dokumenttiin tarvitaan rivinvaihto, kirjoitetaan lähdedokumenttiin kaksi
kenoviivaa (\komento{\keno}). Tätä komentoa käsitellään tarkemmin
tekstikappaleiden yhteydessä luvussa \ref{luku/rivinvaihtokomennot}.

On kuitenkin mahdollista saada myös lähdedokumentin rivinvaihdot
toteutumaan automaattisesti ladotussa tekstissä. Sellainen tila
kytketään päälle komennolla \komento{obeycr}; normaaliin tilaan
palataan taas komennolla \komento{restorecr}. Näistä komennoista voi
olla hyötyä väliaikaisesti ja erityistilanteissa, mutta pysyväksi koko
dokumentin tilaksi \komento{obeycr} ei yleensä sovi.

\subsection{Kappaleen vaihtuminen}
\label{luku/kappaleen-vaihtuminen}

Tyhjä rivi lähdetiedostossa tarkoittaa kappaleen vaihtumista. Rivi on
tyhjä silloin, kun se ei sisällä mitään muuta kuin rivinvaihdon tai kun
se sisältää vain välilyöntejä tai sarkainmerkkejä ja lopuksi
rivinvaihdon. Tyhjiä rivejä voi olla useita peräkkäin, mutta ne
tarkoittavat samaa kuin yksi tyhjä rivi. Uuden tekstikappaleen voi
aloittaa myös komennolla \komento{par}.

\begin{koodilohkosis}
Nämä rivit kuuluvat
samaan kappaleeseen.

Tässä on toinen tekstikappale.
Nyt ei oteta kantaa siihen, miten
rivit ja kappaleet muotoillaan.
\end{koodilohkosis}

\noindent
Ladotuissa teksteissä uuden tekstikappaleen alkaminen ilmaistaan usein
sisennetyllä rivillä, mutta sisennyksiä eikä muitakaan muotoiluja ei
tehdä tekstieditorissa välien avulla. Kappaleiden muotoiluun on omat
keinonsa, ja niistä käsitellään luvussa \ref{luku/kappale}.

\subsection{Kommentit ja muistiinpanot}

Latex\-/dokumentissa prosentin merkki (\koodi{\%}) on kommenttimerkki,
jonka jälkeisen rivinosan kääntäjä jättää huomioimatta. Merkki on
tarkoitettu kirjoittajan omien kommenttien ja muistiinpanojen
kirjoittamiseen.

\begin{koodilohkosis}
% Nyt ei tosin ole
% mitään kommentoitavaa.
\end{koodilohkosis}

\noindent
Kommenttimerkki vaikuttaa kääntäjään myös siten, että se syö kaikki
välilyönnit ja sarkainmerkit, jotka tulevat kyseisen kommentin jälkeen.
Tämän vuoksi kommenttimerkin avulla voi yhdistää eri riveillä olevan
tekstin. Seuraava esimerkki tuottaa ladottuna ehjän sanan \emph{Latex}:

\begin{koodilohkosis}
La% Nämä rivit
  t% yhdistyvät.
    ex
\end{koodilohkosis}

\begin{tulossis}
  La% Nämä rivit
    t% yhdistyvät.
      ex
\end{tulossis}

\subsection{Aaltosulkeet}
\label{luku/aaltosulkeet}

Aaltosulkeet \mbox{\koodi{\{\}}} muodostavat eräänlaisen näkymättömän
ympäristön, jonka sisällä voi olla väliaikaisesti voimassa erilaiset
asetukset kuin ulkopuolella. Aaltosulkeiden sisällä suoritetut komennot,
uusien komentojen määrittelyt (luku \ref{luku/komennot}) tai asetetut
mittojen arvot (luku \ref{luku/mitat}) ovat voimassa vain kyseisen
ympäristön sisäpuolella. Seuraavassa esimerkissä aaltosulkeilla rajataan
kursivointikomennon \komento{itshape} vaikutusaluetta.

\begin{koodilohkosis}
tavallinen {\itshape kursiivi} tavallinen
\end{koodilohkosis}

\begin{tulossis}
  tavallinen {\itshape kursiivi} tavallinen
\end{tulossis}

\subsection{Sitova välilyönti}

Sitova välilyönti on samanlainen tyhjä merkki kuin tavallinenkin
välilyönti, mutta rivinvaihtoa ei sallita sen kohdalta. Sitovalla
välilyönnillä kannattaa estää esimerkiksi pienistä osista koostuvan
ilmauksen hajoaminen eri riveille (esimerkki: \emph{osa~5}). Latexissa
sitova välilyönti saadaan joko tildemerkillä (\koodi{\textasciitilde})
tai nimenomaan siihen tarkoitetulla merkillä, jonka Unicode\-/tunnus on
\uctunnus{u+00a0 no-break space}.

Nämä kaksi eri merkkiä, tilde ja \uctunnus{u+00a0}, toimivat hieman eri
tavoin. Molemmat estävät rivinvaihdon, mutta tildemerkki sallii välin
venymisen samalla tavalla kuin tavallinenkin sanaväli sallii (luku
\ref{luku/sanaväli}). Sen sijaan merkki \uctunnus{u+00a0} on
vakiolevyinen eikä siis veny muiden sanavälien tavoin. Merkkiä
\uctunnus{u+00a0} täytyy käyttää ainakin vuorosanaviivan (\==) ja sitä
seuraavan sanan välissä, koska se väli ei saa venyä.

\subsection{Ohuke}
\label{luku/ohuke}

Ohuke on tavallista sanaväliä kapeampi väli, ja se tehdään
komennolla~\komento{,} (kenoviiva ja pilkku). Ohukkeen leveys Latexissa
on \murtoluku{1}{6} typografisen neliön leveydestä eli em-mitasta (luku
\ref{luku/mitat}). Ohuke on tasalevyinen ja sitova, eli se ei veny
muiden sanavälien tavoin, ja se estää rivinvaihdon. Siksi ohuke sopii
esimerkiksi pitkien lukujen ja puhelinnumeroiden ryhmittelyyn paremmin
kun sanaväli.

\komentoi{,}
\begin{koodilohkosis}
12\,750\,000
J.\,R.\,R. Tolkien
\end{koodilohkosis}

\noindent
Myös henkilön etunimen alkukirjainten välissä voi käyttää ohuketta, jos
tavallinen sanaväli vie kirjaimet turhan kauas toisistaan. Sukunimi
erotetaan kuitenkin aina sanavälillä. Joskus myös päiväyksissä käytetään
ohuketta järjestysluvun pisteiden jälkeen. Taulukossa \ref{tlk/ohuke}
vertaillaan sanaväliä, ohuketta ja yhteen kirjoittamista.

\leijutlk{
  \komentoi{,}
  \begin{tabular}{lrll}
    \toprule
    & \ots{Luku} & \ots{Päiväys} & \ots{Nimi} \\
    \midrule
    \otsrivi{Sanaväli} & 12 750 000 & \sout{9. 5. 2020} & J. R. R. Tolkien \\
    \otsrivi{Ohuke} & 12\,750\,000 & 9.\,5.\,2020 & J.\,R.\,R. Tolkien \\
    \otsrivi{Yhteen} & 12750000 & 9.5.2020 & \sout{J.R.R. Tolkien} \\
    \bottomrule
  \end{tabular}
}{
  \caption{Sanavälin, ohukkeen ja yhteen kirjoittamisen vertailu. Suomen
    kielen vastaiset kirjoitusasut on viivattu yli}
  \label{tlk/ohuke}
}

\subsection{Lainausmerkit ja heittomerkki}
\label{luku/lainausmerkit}

Suomalaisessa näppäinasettelussa \textsc{shift} eli vaihtonäppäin ja 2
tuottaa yleislainausmerkin eli niin sanotun \textsc{ascii}\-/
lainausmerkin (\textquotedbl), mutta se ei taida olla minkään kielen
varsinainen lainausmerkki. On siis syytä käyttää oikeita
lainausmerkkejä, ja se käy Latexissa varsin helposti.

Eri kielissä lainausmerkkikäytännöt ovat erilaiset. Suomen kielessä
käytetään ''tällaisia'' lainausmerkkejä ja joskus >>tällaisia>>
kulmalainausmerkkejä. Jos lainauksen sisään tarvitaan lainaus, täytyy
sisempi lainaus kirjoittaa 'tällaisten' puolilainausmerkkien avulla.
Yksittäin käytettynä se on nimeltään heittomerkki. Englannin kielessä
lainauksen alussa ja lopussa on erilainen merkki, ja ``tässä'' siitä
esimerkki. Samoin on puolilainausmerkin kohdalla: `näin'.

Latexissa voi käyttää Unicode\-/merkistöä ja lähdedokumenttiin voi
kirjoittaa suoraan ne lainausmerkit, jotka halutaan ladottavaksi, mutta
edellä mainituille merkeille on myös omat merkintätapansa.
Näppäimistöltä kirjoitettava yleisheittomerkki (\koodi{'}) tuottaa
ladottuna automaattisesti oikean kaarevan heittomerkin ('). Kun
kirjoittaa kaksi heittomerkkiä peräkkäin (\koodi{''}), on
lopputuloksena yksi kaareva lainausmerkki (''). Kahdella suurempi kuin
\=/merkillä (\koodi{>>}) saadaan kulmalainausmerkki~(>>).

\begin{koodilohkosis}
''Lainaus, jonka 'sisällä' on lainaus.'' \\
>>Lainaus, jonka 'sisällä' on lainaus.>>
\end{koodilohkosis}

\begin{tulossis}
  ''Lainaus, jonka 'sisällä' on lainaus.'' \\
  >>Lainaus, jonka 'sisällä' on lainaus.>>
\end{tulossis}

\noindent
Edellä mainitut riittävät suomen kieleen, mutta englantia ja muita
kieliä varten tarvitaan myös toisinpäin oleva merkki (``), joka tehdään
kahdella gravisaksentilla (\koodi{``}). Vastaava puolilainausmerkki (`)
tehdään yhdellä aksentilla (\koodi{`}). Joissakin kielissä käytetään
erilaisia kulmalainausmerkkejä lainauksen alussa ja lopussa. Vasemmalle
osoittava merkki (<<) tehdään kahdella pienempi kuin \=/merkillä
(\koodi{<<}).

Joskus todella halutaan latoa yleislainausmerkki (\textquotedbl) tai
yleisheittomerkki (\textquotesingle). Ne saadaan komennoilla
\komento{textquotedbl} ja \komento{textquotesingle}. Yksittäinen
gravisaksentti (\`{}) tehdään komennolla \komento{`}\komentoarg{}.
Lainausmerkkien merkintätapoja ja komentoja on koottu taulukkoon
\ref{tlk/erikoismerkit-lainaus}
(s.~\pageref{tlk/erikoismerkit-lainaus}). Toisaalta kielikohtaiset
asetukset (luku \ref{luku/kieliasetukset}) voivat tuoda mukanaan myös
kielikohtaisia keinoja lainausmerkkien kirjoittamiseen.

Edellä kuvatut Latexin omat lainausmerkkien merkintätavat (\koodi{''},
\koodi{>>} ym.) eli niin sanotut Tex\-/ligatuurit voi kytkeä päälle ja
pois päältä fontin asetuksista eli \paketti{fontspec}\-/ pakettiin
kuuluvien toimintojen avulla. Fonteissa on yleensä oletuksena päällä
Tex\-/ligatuurit eli asetus \koodi{Liga\-tures=\katk TeX}, mutta sen saa
poistettua asetuksella \koodi{Liga\-tures=\katk TeX\-Off}. Asetusta
muutetaan kirjainperheen määrittelyn yhteydessä tai väliaikaisesti
komennolla \komento{addfontfeatures}.

\komentoi{addfontfeatures}
\begin{koodilohkosis}
{\addfontfeatures{Ligatures=TeXOff} `` '' >> '}
\end{koodilohkosis}

\begin{tulossis}
  {\addfontfeatures{Ligatures=TeXOff} `` '' >> '}
\end{tulossis}

\noindent
Tasalevyisessä fontissa Tex\-/ligatuurit eivät ole päällä oletuksena,
joten yksittäisissä sanoissa tai lyhyissä ilmauksissa voi estää
Tex\-/ligatuurit esimerkiksi komennolla \komento{texttt}. Samalla
tietysti fonttikin vaihtuu tasalevyiseksi. Fontteja käsitellään
tarkemmin luvussa \ref{luku/kirjaintyypit}.

\komentoi{texttt}
\begin{koodilohkosis}
\texttt{`` '' >> '}
\end{koodilohkosis}

\begin{tulossis}
  \texttt{`` '' >> '}
\end{tulossis}

\noindent
Paketti \pakettictan{csquotes} sisältää lainausmerkkeihin liittyviä
komentoja ja kielikohtaista logiikkaa. Paketissa olevan
\komento{enquote}\-/ komennon avulla voi jättää paketin huoleksi, miten
aloittava ja lopettava lainausmerkki tai ulommat ja sisemmät
lainausmerkit kirjoitetaan missäkin kielessä. Kielipaketti
\paketti{polyglossia} tai \paketti{babel} täytyy olla ladattuna.

\komentoi{usepackage}
\pakettii{polyglossia}
\pakettii{csquotes}
\komentoi{setdefaultlanguage}
\komentoi{enquote}
\begin{koodilohkosis}
\usepackage{polyglossia} \setdefaultlanguage{finnish}
\usepackage[autostyle=true]{csquotes}
% ...
\enquote{Lainauksen \enquote{sisällä} lainaus.}
\end{koodilohkosis}

\begin{tulossis}
  \enquote{Lainauksen \enquote{sisällä} lainaus.}
\end{tulossis}

\subsection{Yhdysmerkki, ajatusviiva ja miinusmerkki}
\label{luku/yhdys-ajatus-miinus}

Yhdyssanan osien välissä käytettävä yhdysmerkki on Latexissa tavallinen
näppäimistöltä saatava yleisyhdysmerkki (\=/). Merkillä on vaikutusta
myös sanan tavutukseen, josta on tarkempaa tietoa luvussa
\ref{luku/tavutus}.

Ajatusviivaa tarvitaan esimerkiksi äärikohtien (27--29,
Oulu--Rova\-niemi), luetelmien, vuorosanojen ja virkkeen irrallisen
lisäysten merkitseminen. Suomen kielessä käytetään yleensä vain lyhyttä
ajatusviivaa \mbox{(--)}, joka tehdään Latexissa kahdella peräkkäisellä
yhdysmerkillä (\koodi{--}). Pitkä ajatusviiva \mbox{(---)} tehdään
kolmella yhdysmerkillä (\koodi{---}). Ajatusviivat vaikuttavat sanan
tavutukseen samoin kuin yhdysmerkki.

\begin{koodilohkosis}
Oulu--Rovaniemi-yhteys
\end{koodilohkosis}

\begin{tulossis}
  Oulu--Rovaniemi-yhteys
\end{tulossis}

\noindent
Myös Unicoden ajatusviivamerkit \uctunnus{u+2013 en dash} ja
\uctunnus{u+2014 em dash} toimivat, mutta tavutuksen kannalta ne ovat
käyttäytyneet eri tavoin Lualatex\-/\ ja Xelatex\-/ kääntäjillä.
Yhteensopivuussyistä on parasta tehdä ajatusviivat Texin omilla
merkintätavoilla eikä Unicode\-/ merkeillä.

Silloin kun todella täytyy latoa kaksi tai kolme peräkkäistä
yhdysmerkkiä, voi käyttää tasalevyistä fonttia
(\komento{texttt}\komentoarg{--}), joka oletuksena kytkee pois Latexin
ajatusviivatoiminnon. Saman asetuksen saa kyllä mihin tahansa fonttiin,
kun poistaa fontista niin sanotut Tex\-/ligatuurit asetuksella
\koodi{Liga\-tures=\katk TeX\-Off}. Väliaikaisesti asetus tehdään
seuraavasti:

\komentoi{addfontfeatures}
\begin{koodilohkosis}
{\addfontfeatures{Ligatures=TeXOff} -- ---}
\end{koodilohkosis}

\begin{tulossis}
  {\addfontfeatures{Ligatures=TeXOff} -- ---}
\end{tulossis}

\noindent
Miinusmerkille (−) ei Latexissa ole erityistä merkintätapaa muuten kuin
matematiikkatilassa (luku \ref{luku/matematiikka}). Tavallisessa
tekstitilassa lyhyttä ajatusviivaa voi ja saa käyttää myös
miinusmerkkinä, mutta vielä parempi olisi käyttää varsinaista Unicoden
miinusmerkkiä \uctunnus{u+2212 minus sign}, koska se on fonteissa
suunniteltu typografisesti yhteensopivaksi muiden matemaattisten
merkkien kanssa.

\subsection{Kolme pistettä eli ellipsi}

Ajatuksen katkeamista ja muuta sellaista ilmaisevalle kolmelle pisteelle
eli ellipsille (\ldots) on oma merkkinsä, ja fontissa se saattaa näyttää
hieman erilaiselta kuin kolme peräkkäistä pistemerkkiä. Tyypillisesti
ellipsimerkissä pisteet ovat hieman harvemmassa ja erottuvat toisistaan
paremmin kuin kolmena erillisenä merkkinä ladotut pisteet. Ellipsi
tehdään Latexissa komennoilla \komento{dots}, \komento{ldots},
\komento{textellipsis} tai Unicode\-/merkillä \uctunnus{u+2026
  horizontal ellipsis}.

\subsection{Ylä- ja alaindeksi}
\label{luku/ylä-alaindeksit}

Yläindeksit (a\textsuperscript{2}) tehdään komennolla
\komento{textsuperscript} ja alaindeksit (a\textsubscript{2})
komennolla \komento{textsubscript}.

\komentoi{textsuperscript}
\komentoi{textsubscript}
\begin{koodilohkosis}
a\textsuperscript{2} a\textsubscript{2}
\end{koodilohkosis}

\noindent
Oletusasetuksilla Latex toteuttaa indeksit mekaanisesti pienentämällä
fonttia ja sijoittamalla pienennetyn tekstin alas peruslinjan tuntumaan
tai ylös gemenalinjan yläpuolelle.\footnote{Katso typografinen viivasto
  eli kuva \ref{kuva/kirjainmitat} sivulla \pageref{kuva/kirjainmitat}.}
Lopputulos ei ole typografisesti välttämättä kovin hyvä, koska fontin
pienentäminen ohentaa samalla merkkien viivoja ja ohuimmat hiusviivat
voivat lähes kadota.

\englanti{Open Type} \=/fontit sisältävät usein tuen oikeille ylä- ja
alaindekseille, jotka fontin suunnittelija on toteuttanut. Niitä
kannattaa käyttää, koska suunnittelija tuntee oman fonttinsa ja on
saanut todennäköisesti parempaa jälkeä kuin Latex mekaanisesti.
\englanti{Open Type} \=/fonttien indeksit on kätevintä ottaa käyttöön
\pakettictan{realscripts}\-/ paketin avulla.

Paketti \paketti{realscripts} määrittelee uudelleen Latexin ylä- ja
alaindeksikomennot, niin että ne ensisijaisesti pyrkivät käyttämään
\englanti{Open Type} \=/fontin ominaisuutta. Jos käytössä oleva fontti
ei sisällä haluttujen merkkien ylä- tai alaindeksiä,
\paketti{realscripts}\-/ paketin komennot käyttävät automaattisesti
Latexin mekaanista keinoa. Paketti määrittelee pari muutakin hyödyllistä
komentoa, muun muassa tähdelliset versiot edellä mainituista:
\komento{textsuperscript*} ja \komento{textsubscript*}. Nämä komennot
toteuttavat aina mekaanisen ylä- tai alaindeksin eli toimivat kuten
Latexin alkuperäiset komennot.

\leijutlk{
  \gemenanum
  \begin{tabular}{lll}
    \toprule
    \ots{Komento}
    & \ots{\englanti{Open Type}}
    & \ots{Mekaaninen} \\
    \midrule
    \komento{textsuperscript}
    & \Large x\textsuperscript{ab36}
    & \Large x\textsuperscript*{ab36} \\
    \midrule
    \komento{textsubscript}
    & \Large H\textsubscript{2}SO\textsubscript{4}
    & \Large H\textsubscript*{2}SO\textsubscript*{4} \\
    \bottomrule
  \end{tabular}
}{
  \caption{Ylä- ja alaindeksien vertailua. Oikeat Open Type \=/fonttien
    indeksit saadaan \paketti{realscripts}\-/ paketin avulla. Mekaaninen
    toteutus perustuu fontin pienentämiseen}
  \label{tlk/indeksien-vertailu}
}

Taulukossa \ref{tlk/indeksien-vertailu} vertaillaan oikeita ja
mekaanisia ylä- ja alaindeksejä. Taulukon esimerkit paljastavat, että
lopputuloksessa on eroa. Mekaaninen ylä- ja alaindeksitoiminto jättää
merkit turhan suurikokoisiksi mutta saattaa silti ohentaa merkkien
viivoja liian paljon. Se ei myöskään ymmärrä poistaa gemenanumeroita
(3624) käytöstä vaan latoo ne sellaisenaan suunnilleen oikeaan paikkaan.

Ylä- ja alaindeksejä käytettäessä on siis syytä ladata
\paketti{realscripts}\-/ paketti ja käyttää indeksit hallitsevaa
\englanti{Open Type} \=/fonttia. Fonttien ominaisuuksia voi tutkia
käyttöjärjestelmän komentotulkissa komennolla \koodi{otfinfo}. Toisaalta
fonttiin sisältyviä ylä- ja alaindeksejä voi myös kirjoittaa
Unicode\-/merkistön avulla sellaisenaan. Lopputulos on sama.

\subsection{Tavutusvihje}

Komento \komento{-} on tavutusvihje, joka neuvoo rivejä latovalle
algoritmille, että sanan voi katkaista tästä kohdasta rivin lopussa.
Tavutusvihje ei normaalisti näy ladotussa dokumentissa, mutta jos sana
katkaistaan sen kohdalta rivin lopussa, ladotaan yhdysmerkki~(-).
Tavutusvihjeen käyttö voi estää sanan katkaisemisen muista kohdista.

\begin{koodilohkosis}
tavutus\-algo\-ritmi
\end{koodilohkosis}

\noindent
Myös sanassa olevat yhdysmerkit ja ajatusviivat vaikuttavat sanan
tavuttamiseen. Perusteellisemmin tavutusta ja sen asetuksia käsitellään
luvussa \ref{luku/tavutus}.

\subsection{Tarkkeet ja erikoismerkit}
\label{luku/tarkkeet}

Latexissa on useita komentoja tarkkeellisten kirjainten
(\'a\,\v{s}\,\c{c}\,\~o) kirjoittamiseen sekä muille merkeille, joita ei
ehkä ihan helposti saa suoraan näppäimistöltä. Komentoja on koottu
taulukoihin \ref{tlk/tarkkeet}, \ref{tlk/erikoismerkit-lainaus} ja
\ref{tlk/erikoismerkit-muut}. Taulukon tarkekomennoissa on käytetty
a\=/kirjainta esimerkkinä, mutta tarke voi liittyä muihinkin kirjaimiin.
Merkit voi kirjoittaa Latex\-/lähdedokumenttiin myös sellaisenaan, eli
näiden komentojen käyttö ei ole välttämätöntä.

\leijutlk{
  \providecommand{\rivi}{}
  \renewcommand{\rivi}[4][]{#3 & \komento{#2}\komentojatko{#1} & #4}
  \begin{tabular}{*{2}{cll}}
    \toprule
    \multicolumn{2}{l}{\ots{Merkki}}
    & \ots{Merkitys}
    & \multicolumn{2}{l}{\ots{Merkki}}
    & \ots{Merkitys} \\
    \cmidrule(r){1-3}
    \cmidrule(l){4-6}
    \rivi[a]{`}{\`a}{gravis}
    & \rivi{O}{\O}{poikkiviiva-O} \\
    \rivi[a]{'}{\'a}{akuutti}
    & \rivi{o}{\o}{poikkiviiva-o} \\
    \rivi[a]{\^{}}{\^a}{sirkumfleksi}
    & \rivi{DJ}{\DJ}{poikkiviiva-D} \\
    \rivi[a]{\~{}}{\~a}{tilde}
    & \rivi{dj}{\dj}{poikkiviiva-d} \\
    \rivi[a]{\textquotedbl}{\"a}{treema}
    & \rivi{DH}{\DH}{versaali-eth} \\
    \rivi[ a]{H}{\H a}{kaksoisakuutti}
    & \rivi{dh}{\dh}{gemena-eth} \\
    \rivi[ a]{r}{\r a}{yläympyrä}
    & \rivi{NG}{\NG}{versaali-äng} \\
    \rivi[ a]{v}{\v a}{hattu}
    & \rivi{ng}{\ng}{gemena-äng} \\
    \rivi[ a]{u}{\u a}{lyhyysmerkki}
    & \rivi{SS}{\SS}{versaali kaksois-s} \\
    \rivi[a]{=}{\=a}{pituusmerkki}
    & \rivi{ss}{\ss}{gemena kaksois-s} \\
    \rivi[ a]{b}{\b a}{alaviiva}
    & \rivi{TH}{\TH}{versaali thorn} \\
    \rivi[ a]{c}{\c a}{sedilji}
    & \rivi{th}{\th}{gemena thorn} \\
    \rivi[a]{.}{\.a}{yläpiste}
    & \rivi{i}{\i}{pisteetön i} \\
    \rivi[ a]{d}{\d a}{alapiste}
    & \rivi{j}{\j}{pisteetön j} \\
    \rivi[ a]{k}{\k a}{ogonek}
    & \rivi{AE}{\AE}{AE-ligatuuri} \\
    % \rivi{t\{ae\}}{\t{ae}}{sidontakaari}
    &&& \rivi{ae}{\ae}{ae-ligatuuri} \\
    \rivi{L}{\L}{poikkiviiva-L}
    & \rivi{OE}{\OE}{OE-ligatuuri} \\
    \rivi{l}{\l}{poikkiviiva-l}
    & \rivi{oe}{\oe}{oe-ligatuuri} \\
    \bottomrule
  \end{tabular}
}{
  \caption{Komentoja tarkkeellisten ja muiden kirjainten
    kirjoittamiseen}
  \label{tlk/tarkkeet}
}

\leijutlk{
  \providecommand{\rivi}{}
  \renewcommand{\rivi}[3]{ #1 & \komento{#2} & #3 \\}

  \begin{tabular}{cll}
    \toprule
    \multicolumn{2}{l}{\ots{Merkki ja komennot}}
    & \ots{Merkitys} \\
    \midrule

    \textquotedblleft
    & \komento{textquotedblleft}, \koodi{`{}`}
    & vasen lainausmerkki \\

    \textquotedblright
    & \komento{textquotedblright}, \koodi{'{}'}
    & oikea lainausmerkki \\

    \rivi{\textquotedbl}{textquotedbl}{yleislainausmerkki}

    \textquoteleft
    & \komento{textquoteleft}, \komento{lq}, \koodi{`}
    & vasen puolilainausmerkki \\

    \textquoteright
    & \komento{textquoteright}, \komento{rq}, \koodi{'}
    & oikea puolilainausmerkki, heittomerkki \\

    \rivi{\textquotesingle}{textquotesingle}
    {yleispuolilainausmerkki ja -heittomerkki}

    \guillemotleft
    & \komento{guillemotleft}, \koodi{<{}<}
    & vasen kulmalainausmerkki \\

    \guillemotright
    & \komento{guillemotright}, \koodi{>{}>}
    & oikea kulmalainausmerkki \\

    \rivi{\guilsinglleft}{guilsinglleft}{vasen kulmapuolilainausmerkki}
    \rivi{\guilsinglright}{guilsinglright}
    {oikea kulmapuolilainausmerkki}
    \rivi{\quotedblbase}{quotedblbase}{rivinalinen lainausmerkki}
    \rivi{\quotesinglbase}{quotesinglbase}{rivinalinen puolilainausmerkki}

    \bottomrule
  \end{tabular}
}{
  \caption{Komentoja lainausmerkkien kirjoittamiseen}
  \label{tlk/erikoismerkit-lainaus}
}

\leijutlk{
  \providecommand{\rivi}{}
  \renewcommand{\rivi}[3]{ #1 & \komento{#2} & #3 \\}
  \providecommand{\textbigcirclekorvike}{}
  \renewcommand{\textbigcirclekorvike}{%
    \begin{tikzpicture}
      \draw (0,0) circle [radius=.65ex];
    \end{tikzpicture}}

  \begin{tabular}{cll}
    \toprule
    \multicolumn{2}{l}{\ots{Merkki ja komennot}}
    & \ots{Merkitys} \\
    \midrule

    \textendash
    & \komento{textendash}, \koodi{--}
    & lyhyt ajatusviiva \\

    \textemdash
    & \komento{textemdash}, \koodi{---}
    & pitkä ajatusviiva \\

    \textexclamdown
    & \komento{textexclamdown}, \koodi{!`}
    & ylösalainen huutomerkki \\

    \textquestiondown
    & \komento{textquestiondown}, \koodi{?`}
    & ylösalainen kysymysmerkki \\

    \rivi{\textgreater}{textgreater}{suurempi kuin -merkki}
    \rivi{\textless}{textless}{pienempi kuin -merkki}

    \textellipsis
    & \komento{textellipsis}, \komento{ldots}, \komento{dots}
    & kolme pistettä, ellipsi \\

    \rivi{\texteuro}{texteuro}{euron merkki}

    \textsterling
    & \komento{textsterling}, \komento{pounds}
    & punnan merkki \\

    \textdollar
    & \komento{textdollar}, \komento{\$}
    & dollarin merkki \\

    \textsection
    & \komento{textsection}, \komento{S}
    & pykälän merkki \\

    \textparagraph
    & \komento{textparagraph}, \komento{P}
    & kappaleen merkki \\

    \textcopyright
    & \komento{textcopyright}, \komento{copyright}
    & tekijänoikeusmerkki \\

    \rivi{\textregistered}{textregistered}{rekisteröity tavaramerkki}
    \rivi{\texttrademark}{texttrademark}{tavaramerkki}

    \textdagger
    & \komento{textdagger}, \komento{dag}
    & risti \\

    \textdaggerdbl
    & \komento{textdaggerdbl}, \komento{ddag}
    & kaksoisristi \\

    \textasciicircum
    & \komento{textasciicircum}, \komento{\^{}}\komentoarg{}
    & sirkumfleksi \\

    \textasciitilde
    & \komento{textasciitilde}, \komento{\~{}}\komentoarg{}
    & tilde \\

    \rivi{\textasteriskcentered}{textasteriskcentered}
    {rivinkeskinen asteriski, tähti}
    \rivi{\textbackslash}{textbackslash}{kenoviiva}
    \rivi{\textbar}{textbar}{pystyviiva}
    \rivi{\textbardbl}{textbardbl}{kaksoispystyviiva}

    \textbraceleft
    & \komento{textbraceleft}, \komento{\{}
    & vasen aaltosulje \\

    \textbraceright
    & \komento{textbraceright}, \komento{\}}
    & oikea aaltosulje \\

    \rivi{\textbullet}{textbullet}{luetelmaympyrä}
    \rivi{\textbigcirclekorvike}{textbigcircle}{suuri ympyrä}
    \rivi{\textleftarrow}{textleftarrow}{nuoli vasemmalle}
    \rivi{\textrightarrow}{textrightarrow}{nuoli oikealle}
    \rivi{\textordfeminine}{textordfeminine}
    {feminiininen järjestysluvun merkki}
    \rivi{\textordmasculine}{textordmasculine}
    {maskuliininen järjestysluvun merkki}
    \rivi{\textperiodcentered}{textperiodcentered}{rivinkeskinen piste}

    \textunderscore
    & \komento{textunderscore}, \komento{\_}
    & alaviiva \\

    \rivi{\textvisiblespace}{textvisiblespace}{näkyvä välilyönti}
    \bottomrule
  \end{tabular}
}{
  \caption{Komentoja erikoismerkkien kirjoittamiseen}
  \label{tlk/erikoismerkit-muut}
}

Tarke- ja erikoismerkkikomentojen lisäksi on olemassa sekalaisia muita
komentoja erikoisempien asioiden latomiseen. \marginaali{\TeX}
\marginaali{\LaTeX} Ladontajärjestelmän logojen kirjoittamiseen on
komennot \komento{TeX} ja \komento{LaTeX}. Suorakulmioita voi tehdä
\komento{rule}\-/ komennolla, jolle annetaan argumenteiksi ainakin kaksi
mittaa: leveys ja korkeus. Myös yksi hakasulkeissa annettu valinnainen
argumentti on mahdollinen. Sekin on mitta ja ilmaisee, kuinka paljon
suorakulmiota nostetaan tekstin peruslinjasta. Negatiivinen mitta laskee
suorakulmiota alaspäin. Latexin mittoja ja mittayksiköitä käsitellään
luvussa \ref{luku/mitat}.

\komentoi{rule}
\begin{koodilohkosis}
\rule{1ex}{1ex} abc \rule{3em}{.5bp} abc \rule[1ex]{3em}{.5bp}
\end{koodilohkosis}

\begin{tulossis}
  \rule{1ex}{1ex} abc \rule{3em}{.5bp} abc \rule[1ex]{3em}{.5bp}
\end{tulossis}

\noindent
Komento \komento{strut} latoo näkymättömän, leveydettömän merkin, jonka
korkeus on rivikorkeuden eli mitan \mitta{baselineskip} mukainen.
Komentoa voi tarvita joskus esimerkiksi laatikoiden sisällä (luku
\ref{luku/laatikot}).

Käytännössä \komento{strut}\-/ komento hyödyntää sisäisesti
\komento{rule}\-/ komentoa ja latoo sen avulla leveydettömän
suorakulmion. Tarkemmin sanottuna korkeus eli tekstin peruslinjan
yläpuolinen osa on 0,7 kertaa \mitta{baselineskip}\-/ mitta ja syvyys
eli peruslinjan alapuolinen osa on 0,3 kertaa \mitta{baselineskip}.

\komentoi{rule}
\mittai{baselineskip}
\begin{koodilohkosis}
\rule[-0.3\baselineskip]{0bp}{\baselineskip}
\end{koodilohkosis}

\section{Komennot}
\label{luku/komennot}

Latexin komennot alkavat kenoviivalla (\koodi{\textbackslash}), jonka
jälkeen tulee komennon nimi. Nimi koostuu yleensä pienistä tai isoista
kirjaimista, mutta komento voi koostua myös muista merkeistä.

Komennot voivat ottaa vastaan argumentteja eli lisätietoa, jota komento
käsittelee ja tarvitsee toimintaansa. Jotkin argumentit voivat olla
pakollisia ja jotkin valinnaisia. Pakolliset kirjoitetaan
aaltosulkeisiin \koodi{\{}\ldots\koodi{\}} ja valinnaiset hakasulkeisiin
\koodi{[}\ldots\koodi{]}.

\begin{koodilohkosis}
\komento
\komento{argu}{mentteja}
\komento[valinnainen]{argu}{mentteja}
\end{koodilohkosis}

\noindent
Jos pakolliseen argumenttiin haluaa sisällyttää aaltosulkeen, täytyy sen
eteen kirjoittaa kenoviiva \komento{\{}~\komento{\}}, tai voi myös
käyttää taulukossa \ref{tlk/erikoismerkit-muut} mainittuja komentoja
aaltosulkeiden tuottamiseen. Sama pätee aaltosulkeiden latomiseen
muutenkin.

Hakasulkeet sen sijaan ladotaan tekstiin normaalisti, eikä niiden kanssa
käytetä kenoviivaa. Poikkeustilanne on komennon valinnaisen argumentin
sisällä, koska valinnainen argumentti jo sinänsä kirjoitetaan
hakasulkeiden sisään. Hakasulkeita ei voi suojata kenoviivalla, koska
\komento{[} ja \komento{]} ovat jo muuhun tarkoitettuja komentoja:
niillä luodaan matematiikkatilassa (luku \ref{luku/matematiikka}) oleva
tekstilohko. Valinnaisen argumentin sisään saa hakasulkeen, kun sen
kirjoittaa aaltosulkeiden sisään. Esimerkiksi komennon
\komentox{komento}\komentoargv{\{]\}} valinnaiseksi argumentiksi tulee
lopulta yksi \koodi{]}-merkki.

Komennon yhteydessä sanavälejä käsitellään hieman poikkeuksellisesti.
Esimerkiksi komennon nimen perässä olevat sanavälit syödään pois, jos
komennolle ei anneta yhtään argumenttia. Seuraavassa esimerkissä sana
\emph{Latex} ladotaan ehjänä, jos vain \komentox{komento} itsessään ei
kirjoita mitään eikä vaikuta tekstin latomiseen.

\begin{koodilohkosis}
La\komento   tex
\end{koodilohkosis}

\begin{tulossis}
  Latex
\end{tulossis}

\noindent
Jos täytyy saada komennon nimen jälkeinen sanaväli näkyviin, täytyy
kirjoittaa komennon nimen perään aaltosulkeet
(\komentox{komento}\komentoarg{}) tai kenoviiva
(\komentox{komento}\komentojatko{\keno}).

Komennon nimen ja argumenttien välissä voi olla sanavälejä, ja ne kaikki
syödään pois. Komennon ja sen argumentit voi siis kirjoittaa vaikka
seuraavalla tavalla:

\begin{koodilohkosis}
\komento  [valinnainen]
   {argu-}   {mentteja}
\end{koodilohkosis}

\subsection{Omat komennot ja abstrahointi}
\label{luku/komennot-abst}

Omien komentojen tärkein tarkoitus on merkintätapojen abstrahointi eli
teknisen toteutuksen ja yksityiskohtien piilottaminen. Sopiva
abstrahointi helpottaa lähdedokumentin käsittelemistä.

Esimerkiksi jos kirjoittaa Latexia käsittelevää kirjaa, kannattaa heti
aluksi luoda komento, jolla merkitään kaikki Latexin komennot. Komennon
nimi voisi olla vaikka \komentox{komento}, ja sen voisi määritellä
siten, että se lisää automaattisesti komennon nimen alkuun kenoviivan
(\koodi{\keno}) ja latoo koko ilmauksen tasalevyisellä fontilla.
Seuraavassa on esimerkki tällaisen komennon määrittelemisestä ja
käytöstä:

\komentoi{newcommand}
\komentoi{texttt}
\komentoi{textbackslash}
\begin{koodilohkosis}
\newcommand{\komento}[1]{\texttt{\textbackslash #1}}

Komennolla \komento{section} tehdään otsikoita.
\end{koodilohkosis}

\begin{tulossis}
  Komennolla \texttt{\textbackslash section} tehdään otsikoita.
\end{tulossis}

\noindent
Edellä mainitun komennon määritelmään voi olla tarpeen lisätä myöhemmin
muitakin asioita. Latexin komentojen nimet ovat suunnilleen englantia,
ja jos nimen tavuttaa, täytyisi se tehdä englannin sääntöjen mukaisesti.
Esimerkkinä olleen \komentox{komento}\-/ komennon määritelmässä voisi
siis vaihtaa myös kielen komennolla \komento{textenglish}.%
\footnote{Kieliasetuksista käsitellään luvussa
  \ref{luku/kieliasetukset}.}

Jos ollaan kirjoittamassa laajaa tietoteosta, Latex\-/ komennot halutaan
ehkä lisätä automaattisesti kirjan lopussa olevaan asiahakemistoon (ks.
s.~\pageref{luku/asiahakemisto}). Niinpä komennon määritelmään lisätään
vielä sitäkin varten komento \komento{index}.%
\footnote{Asiahakemistoja käsitellään luvussa \ref{luku/asiasanat}.}
Lopulta Latex\-/ komentojen merkitsemiseen tarkoitettu
\komentox{komento} määriteltäisiin seuraavalla tavalla:

\komentoi{newcommand}
\komentoi{textbackslash}
\komentoi{texttt}
\komentoi{textenglish}
\komentoi{index}
\begin{koodilohkosis}
\newcommand{\komento}[1]{%
  \texttt{\textbackslash\textenglish{#1}}%
  \index[komennot]{#1@\texttt{\textbackslash #1}}}
\end{koodilohkosis}

% Asiahakemistot-luvussa on viittaus tähän lukuun ja maininta
% \index-komennon sisällyttämisestä toisen komennon määritelmään.

\noindent
Näin oma \komentox{komento} ilmaisee tiiviisti ja havainnollisesti
tarkoituksen eli sen, että kyseessä on Latex\-/ komento. Se piilottaa
monimutkaisen teknisen toteutuksen eli fontin ja kielen vaihtamiseen
sekä asiahakemistoon liittyvät toiminnot. Lisäksi komennon teknistä
toteutusta on helppoa muuttaa myöhemmin, koska komennon määrittely on
vain yhdessä paikassa lähdedokumentin alussa.

\subsection{Komentojen määrittely}
\label{luku/komennot-määrittely}

Komentojen määrittelyyn on kolme erilaista komentoa, ja niille kaikille
annetaan samanlaiset argumentit. Komennot ovat seuraavat:

\komentoi{newcommand}
\komentoi{renewcommand}
\komentoi{providecommand}
\begin{koodilohkosis}
\newcommand     {\nimi}[n][oletus]{määritelmä}
\renewcommand   {\nimi}[n][oletus]{määritelmä}
\providecommand {\nimi}[n][oletus]{määritelmä}
\end{koodilohkosis}

\noindent
Ensimmäinen pakollinen argumentti on komennon nimi (\komentox{nimi}), ja
se voi koostua vain kirjaimista. Komento \komento{newcommand}
määrittelee uuden komennon. Mikäli komento on jo olemassa, annetaan
virheilmoitus. Toinen komento \komento{renewcommand} määrittelee
olemassa olevan komennon uudelleen. Se antaa virheilmoituksen, jos
komentoa ei ollut olemassa. Kolmas komento \komento{providecommand}
puolestaan määrittelee uuden komennon vain siinä tapauksessa, että
sellaista ei ollut ennen olemassa. Se ei anna virheilmoitusta.

Komentojen toinen pakollinen argumentti \koodi{määritelmä} sisältää
komennon määritelmän eli mitä tahansa tekstiä tai komentoja.
Suoritusvaiheessa komento ikään kuin vaihdetaan sen määritelmäksi.

\komentoi{newcommand}
\begin{koodilohkosis}
\newcommand{\komento}{Minua komennettiin!}
\komento
\end{koodilohkosis}

\begin{tulossis}
  Minua komennettiin!
\end{tulossis}

\noindent
Komentojen ensimmäinen valinnainen argumentti \koodi{n} on luku, joka
kertoo, kuinka monta argumenttia määriteltävä komento käsittelee.
Määritelmässä voi käyttää parametreja \koodi{\#1}, \koodi{\#2},
\koodi{\#3} jne., ja ne korvautuvat komennon suoritusvaiheessa
ensimmäisellä, toisella, kolmannella jne. argumentilla.

\komentoi{newcommand}
\begin{koodilohkosis}
\newcommand{\komento}[2]{Sanoit #1 ja #2!}
\komento{hip}{hei}
\end{koodilohkosis}

\begin{tulossis}
  Sanoit hip ja hei!
\end{tulossis}

\noindent
Toinen valinnainen argumentti \koodi{oletus} -- jos se on mukana --
kertoo, että määriteltävän komennon ensimmäinen argumentti on
valinnainen ja että tämä on sen oletusarvo. Oletusarvoa käytetään
silloin, kun valinnaista argumenttia ei ole annettu.

\komentoi{newcommand}
\begin{koodilohkosis}
\newcommand{\komento}[3][tyyppi]{Hei #1, sanoit #2 ja #3!}
\komento{hip}{hei} \\
\komento[Leslie]{hip}{hei}
\end{koodilohkosis}

\begin{tulossis}
  Hei tyyppi, sanoit hip ja hei! \\
  Hei Leslie, sanoit hip ja hei!
\end{tulossis}

\noindent
Joskus yhden komennon määritelmä sisältää komennon
\komento{renewcommand}, joka sitten määrittelee uudelleen jonkin toisen
komennon. Silloin parametrit \koodi{\#1}, \koodi{\#2} jne. on
tarkoitettu ensimmäisen eli uloimman kerroksen käsiteltäväksi. Sisempi
kerros käyttää parametreja \koodi{\#\#1}, \koodi{\#\#2} jne.

Kaikista kolmesta komentojen määrittelykomennosta on olemassa
tähdellinen versio eli sellainen, jonka komennon nimen lopussa on tähti
(\koodi{*}). Latexin komennoissa on tapana, että tähdellinen versio --
jos sellainen on olemassa -- tarjoaa samaan asiaan jonkin toisenlaisen
näkökulman.

\komentoi{newcommand*}
\komentoi{renewcommand*}
\komentoi{providecommand*}
\begin{koodilohkosis}
\newcommand*     {\nimi}[n][oletus]{määritelmä}
\renewcommand*   {\nimi}[n][oletus]{määritelmä}
\providecommand* {\nimi}[n][oletus]{määritelmä}
\end{koodilohkosis}

\noindent
Komentojen määrittelyssä tähdelliset versiot toimivat muuten samalla
tavalla, mutta ne antavat virheilmoituksen, jos komennolle annetut
argumentit sisältävät enemmän kuin yhden tekstikappaleen. Niinpä
esimerkin \ref{esim/newcommand} koodi tuottaa käännettäessä virheen.

\begin{esimerkki*}
  \komentoi{newcommand*}
\begin{koodilohko}
\newcommand*{\komento}[1]{Teksti: #1}

\komento{
  Ensimmäinen tekstikappale.

  Toinen tekstikappale.
}
\end{koodilohko}
  \caption{\komento{newcommand*}\-/ komennolla määritelty komento ei
    salli argumentteja, joissa on useita tekstikappaleita. Tämä
    esimerkki tuottaa käännettäessä virheilmoituksen}
  \label{esim/newcommand}
\end{esimerkki*}

\komento{newcommand*}\-/komennolla määritelty \komentox{komento} ei siis
suostu ottamaan vastaan argumentteja, jotka sisältävät kappaleen
vaihtumisen eli enemmän kuin yhden tekstikappaleen. Tämä voi olla
hyödyllinen suojausominaisuus.

\subsection{Kestävät ja hauraat komennot}
\label{luku/komennot-hauraat}

Kaikki Latex\-/ komennot ovat joko kestäviä (\englanti{robust}) tai
hauraita (\englanti{fragile}). Tällä ei ole yleensä käytännön
merkitystä, mutta hauraat komennot eivät välttämättä toimi toisen
komennon argumentissa.

Hauraita komentoja ovat sellaiset, jotka sisältävät tietoa, jota Latex
kirjoittaa väliaikaistiedostoon ja lukee sieltä takaisin. Kyse on
esimerkiksi sisällysluetteloon tai muihin automaattisiin luetteloihin
kirjoitettavasta tiedosta. Hauraita ovat myös rivinvaihdot ja
valinnaisia argumentteja eli hakasulkeissa annettavia argumentteja
sisältävät komennot.

Käytännössä ongelmia aiheuttavat esimerkiksi otsikkokomennot
(\komento{section} ym.) sekä kuvatekstikomento \komento{caption}. Näiden
argumenttina oleva teksti (ja mahdolliset komennot) kirjoitetaan
tiedostoon ja ladataan sieltä myöhemmin takaisin. Hauraat komennot eivät
toimi edellä mainittujen komentojen argumentissa, ja lähdetiedoston
kääntäminen johtaa virheilmoitukseen. Ongelman voi ainakin joskus
korjata kirjoittamalla hauraan komennon eteen komennon
\komento{protect}.

\subsection{Muita vinkkejä}

Komennon viimeistä argumenttia ei välttämättä tarvitse kirjoittaa
aaltosulkeisiin, jos argumentiksi halutaan vain yksi merkki. Tällaisessa
tilanteessa komento poimii argumentiksi seuraavan merkin, joka ei ole
sanaväli.

\komentoi{newcommand}
\begin{koodilohkosis}
\newcommand{\x}[1]{Argumentti: <#1>}
\x abc \\
\x.abc
\end{koodilohkosis}

\begin{tulossis}
  Argumentti: <a>bc \\
  Argumentti: <.>abc
\end{tulossis}

\noindent
Mikäli argumenttina on kenoviivalla alkava komento, sitäkään ei tarvitse
kirjoittaa aaltosulkeisiin. Latex\-/koodin lukemisen kannalta tällainen
ei välttämättä ole hyvä käytäntö, koska joskus voi hämärtyä, onko kyse
kahdesta peräkkäisestä komennosta vai onko toinen komento vain
argumenttina toiselle.

\komentoi{newcommand}
\begin{koodilohkosis}
\newcommand{\x}[1]{Argumentti: <#1>}
\newcommand{\yyy}{abc}
\x\yyy
\end{koodilohkosis}

\begin{tulossis}
  Argumentti: <abc>
\end{tulossis}

\noindent
Tätä merkintätapaa esiintyy jokin verran komentojen määrittelyssä, niin
että jätetään \komento{newcommand}\-/komennon ensimmäisenä argumenttina
oleva komennon nimi ilman aaltosulkeita.

\komentoi{newcommand}
\begin{koodilohkosis}
\newcommand\yyy{abc}
\end{koodilohkosis}

\noindent
Komennon määrittelyssä on välillä hyötyä \komento{ignorespaces}\-/
komennosta, joka jättää sanavälit huomioimatta komennon jälkeen. Ilman
tätä komentoa tulisi seuraavassa esimerkissä sanojen väliin yksi
välilyönti.

\komentoi{newcommand}
\komentoi{ignorespaces}
\begin{koodilohkosis}
\newcommand{\komento}[1]{#1\ignorespaces}
\komento{yhdys}       sana
\end{koodilohkosis}

\begin{tulossis}
  yhdyssana
\end{tulossis}

\begin{esimerkki*}
  \komentoi{newcommand}
  \komentoi{renewcommand}
\begin{koodilohko}
\newcommand{\komento}{alkuperäinen}
\komento
{%
  \renewcommand{\komento}{\textit{muutettu}}
  \komento
}
\komento
\end{koodilohko}
\begin{tulos}
  alkuperäinen \textit{muutettu} alkuperäinen
\end{tulos}
\caption{Aaltosulkeilla voi rajata komennon määrittelyn
  vaikutusaluetta}
\label{esim/aaltosulkeet-rajaaminen}
\end{esimerkki*}

\noindent
Aaltosulkeilla (luku \ref{luku/aaltosulkeet}) voi rajata
komentomäärittelyn vaikutusaluetta. Esimerkin
\ref{esim/aaltosulkeet-rajaaminen} alussa asetetaan \komentox{komento}
tiettyyn alkuperäismääritelmään. Aaltosulkeiden sisällä se määritellään
väliaikaisesti uudestaan. Aaltosulkeilla rajatun ympäristön jälkeen
komennon uusi määritelmä lakkaa ja komento palautuu alkuperäiseksi.

Monimutkaisiin komentoihin voidaan tarvita ehtorakenteita. Ne saa
toteutettua \pakettictan{ifthen}\-/ paketin tarjoaman
\komento{ifthenelse}\-/ komennon avulla. Se on ohjelmointikielistä
tuttu ehtorakenne: jos annettu ehtolauseke on tosi, käsitellään
then\-/haara; muussa tapauksessa käsitellään else-haara.

\section{Ympäristöt}
\label{luku/ympäristöt}

Ympäristöt ovat rakenteita, joilla on aloittava \komento{begin}\-/
komento ja lopettava \komento{end}\-/ komento sekä nimi. Ympäristöjen
ajatuksena on, että jokin ominaisuus tai jotkin toiminnot ovat voimassa
vain ympäristön sisällä ja asiat palautuvat ennalleen ympäristön
jälkeen. Tässä mielessä ne toimivat samalla tavalla kuin aaltosulkeet
(luku \ref{luku/aaltosulkeet}). Jos esimerkiksi fonttiasetusta (luku
\ref{luku/kirjaintyypit}) muuttaa ympäristön sisäpuolella, asetus
palautuu ympäristön jälkeen samaksi kuin se oli ennen ympäristön alkua.
Samoin ympäristön sisällä määritellyt komennot ovat voimassa vain
kyseisessä ympäristössä.

\komentoi{begin}
\komentoi{end}
\begin{koodilohkosis}
\begin{nimi}
  % ympäristön
  % vaikutusalue
\end{nimi}
\end{koodilohkosis}

\noindent
Yleisin ympäristö on nimeltään \ymparisto{document}, jonka sisään koko
dokumentin sisältö kirjoitetaan. Muita ympäristöjä käytetään
tavallisesta leipätekstistä poikkeavien rakenteiden ilmaisemiseen,
esimerkiksi luetelmiin ja taulukoihin (luvut \ref{luku/luetelmat} ja
\ref{luku/taulukot}). Ympäristöjä voi tehdä itsekin mihin hyvänsä
tarkoitukseen. Niitä määritellään seuraavilla komennoilla:

\komentoi{newenvironment}
\komentoi{renewenvironment}
\begin{koodilohkosis}
\newenvironment   {nimi}[n][oletus]{aloitus}{lopetus}
\renewenvironment {nimi}[n][oletus]{aloitus}{lopetus}
\end{koodilohkosis}

\noindent
Komento \komento{newenvironment} määrittelee uuden ympäristön. Se antaa
virheilmoituksen, jos samanniminen ympäristö on jo olemassa. Komento
\komento{renewenvironment} puolestaan määrittelee uudelleen ympäristön,
joka on jo olemassa. Se antaa virheilmoituksen, jos ympäristöä ei
ollutkaan olemassa.

Argumentit ovat lähes samanlaiset kuin komentojen määrittelyssä (luku
\ref{luku/komennot-määrittely}). Ympäristöjen määrittelykomennoilla on
kolme pakollista argumenttia: ensimmäinen on ympäristön nimi, toinen on
ympäristön aloitusmääritelmä (\koodi{aloitus}) ja kolmas on
lopetusmääritelmä (\koodi{lopetus}).

\komentoi{newenvironment}
\komentoi{begin}
\komentoi{end}
\begin{koodilohkosis}
\newenvironment{ymp}{Tästä se alkaa.}{Tähän se päättyy.}

\begin{ymp}
  Ympäristön sisältöä.
\end{ymp}
\end{koodilohkosis}

\begin{tulossis}
  Tästä se alkaa. Ympäristön sisältöä. Tähän se päättyy.
\end{tulossis}

\noindent
Tavallisesti ympäristön aloitusmääritelmään kirjoitetaan jonkin toisen
ympäristön aloituskomento sekä mahdollisesti suoritetaan joitakin
asetuskomentoja. Vastaavasti lopetusmääritelmässä lopetetaan ympäristö
eli palataan normaaliin tilaan. Tarkoituksena on abstrahoida jokin
monimutkaisempi kokonaisuus eli tehdä uusi helppokäyttöinen ympäristö, joka
piilottaa teknisen toteutuksen.

\komentoi{newenvironment}
\komentoi{begin}
\komentoi{end}
\begin{koodilohkosis}
\newenvironment{ymp}
{\begin{mahtavuus}
    \omia\hienoja\asetuksia}
  {\end{mahtavuus}}
\end{koodilohkosis}

\noindent
Omille ympäristölle voi määritellä argumentteja samalla tavalla kuin
komennoillekin. Ensimmäinen valinnainen argumentti \koodi{n} on luku
joka kertoo, kuinka monta argumenttia määriteltävä ympäristö käsittelee.
Ympäristön aloitusmääritelmässä voi argumentteihin viitata parametreilla
\koodi{\#1}, \koodi{\#2}, \koodi{\#3} jne.

Jos toinen valinnainen argumentti \koodi{oletus} on mukana, se
ilmaisee, että määriteltävän ympäristön ensimmäinen argumentti on
valinnainen ja että tämä on sen oletusarvo. Oletusta käytetään silloin,
kun valinnaista argumenttia ei ole annettu. Argumentit annetaan
ympäristön aloittavan \komento{begin}\-/ komennon yhteydessä.

\komentoi{begin}
\komentoi{end}
\begin{koodilohkosis}
\begin{ymp}[valinnainen]{argu}{mentteja}
  % ympäristön
  % vaikutusalue
\end{ymp}
\end{koodilohkosis}

\noindent
Ympäristön määrittelykomennoille on myös tähdelliset versiot
\komento{newenvironment*} ja \komento{renewenvironment*}. Ne toimivat
samoin kuin edellä kuvatut tavallisetkin komentoversiot, mutta ne eivät
salli, että määritellylle ympäristölle annetut argumentit sisältävät
enemmän kuin yhden tekstikappaleen. Toiminta on siis sama kuin
komentojenkin määrittelyn tähdellisissä versioissa (luku
\ref{luku/komennot-määrittely}).

\komentoi{newenvironment*}
\komentoi{renewenvironment*}
\begin{koodilohkosis}
\newenvironment*   {nimi}[n][oletus]{aloitus}{lopetus}
\renewenvironment* {nimi}[n][oletus]{aloitus}{lopetus}
\end{koodilohkosis}

\begin{esimerkki*}
  \komentoi{newenvironment}
  \komentoi{ignorespaces}
  \komentoi{ignorespacesafterend}
  \komentoi{begin}
  \komentoi{end}

\begin{koodilohko}
\newenvironment{ymp}
{Yhdys\ignorespaces}
{esi\ignorespacesafterend}

\begin{ymp}
  sana% Kommentti poistaa seuraavan sanavälin.
\end{ymp}   merkki.
\end{koodilohko}
\begin{tulos}
  Yhdyssanaesimerkki.
\end{tulos}
\caption{Sanavälien käyttäytyminen ympäristöjen yhteydessä. Komennoilla
  \komento{ignorespaces} ja \komento{ignorespacesafterend} voi poistaa
  seuraavat sanavälit}
\label{esim/ignorespacesafterend}
\end{esimerkki*}

\noindent
Joskus ympäristöjen määrittelyyn on hyödyllistä sisällyttää komento
\komento{ignorespaces}, joka jättää huomioimatta tämän komennon
jälkeiset sanavälit. Toinen hyödyllinen on
\komento{ignorespacesafterend}, joka jättää huomioimatta ympäristön
lopettavan \komento{end}\-/ komennon jälkeiset sanavälit. Esimerkki
\ref{esim/ignorespacesafterend} selventää näiden toimintaa.

\section{Mitat}
\label{luku/mitat}

\subsection{Mittayksiköt}

Koska typografia on pitkälti teksti- ja muiden elementtien sijoittelua,
tarvitaan sitä varten mittavälineitä. Niinpä Texissäkin on pituusmittoja
ja useita pituuden mittayksiköitä. Taulukkoon \ref{tlk/mittayksiköt} on
koottu mittayksiköiden lyhenteet ja merkitykset. Teknisesti on
samantekevää, mitä yksiköitä käyttää, sillä ne ovat vain välineitä
pituuden ilmaisemiseen. Sisäisesti Tex käyttää sp-yksikköä, joka
ilmaisee samalla mittojen tarkkuuden: pienin jakamaton mitta on 1\,sp
(5,36\,nm).

\leijutlk{
  \begin{tabular}{ll}
    \toprule
    \ots{Lyh.} & \ots{Merkitys} \\
    \midrule
    bp & piste uudessa pica-järjestelmässä, 1/72 tuumaa, 0,3528 mm \\
    pt & piste vanhassa pica-järjestelmässä, 1/72,27 tuumaa, 0,3515 mm \\
    pc & pica eli 12 pt-pistettä \\
    sp & 1/65536 pt-pistettä (5,36\,nm), Texin sisäisesti käyttämä yksikkö \\
    dd & piste Didot-järjestelmässä, 0,376 mm \\
    cc & cicero eli 12 dd-pistettä \\
    mm & millimetri \\
    cm & senttimetri \\
    in & tuuma, 25,4 mm \\
    ex & nykyisen fontin x-korkeus, perus- ja gemenalinjan etäisyys \\
    em & typografisen neliön sivun pituus, sama kuin fontin koko \\
    \bottomrule
  \end{tabular}
}{
  \caption{Texin mittayksiköiden lyhenteet ja merkitykset}
  \label{tlk/mittayksiköt}
}

Usein tietyt yksiköt ovat vakiintuneet tiettyihin tilanteisiin.
Esimerkiksi fonttien kokoja ja rivikorkeuksia on tapana mitata
pistemittojen avulla. Nykyään typografiassa käytetään lähinnä
bp\-/yksikön mukaista pistettä, joka tuli käyttöön Post Script
\=/standardin myötä vuonna 1984 ja jota käytetään julkaisuohjelmissa.
Sivun mittoja kuten leveyttä, korkeutta ja marginaaleja ilmaistaan
tavallisesti metrijärjestelmän avulla eli senttimetreissä tai
millimetreissä.

Latexin mittojen mittaluvuissa desimaalierottimena on piste, ja
mittayksikön lyhenne kirjoitetaan kiinni mittalukuun. Seuraavassa
esimerkissä tehdään vaakasuuntaisia ja pystysuuntaisia välejä
komennoilla \komento{hspace} ja \komento{vspace}:

\komentoi{hspace}
\komentoi{vspace}
\begin{koodilohkosis}
Sanat\hspace{1.2cm}hassusti
\vspace{2mm}

\hspace{1.75em}erillään.
\end{koodilohkosis}

\begin{tulossis}
  Sanat\hspace{1.2cm}hassusti  \nopagebreak
  \vspace{2mm}

  \hspace{1.75em}erillään.
\end{tulossis}

\noindent
Jos pystysuuntaisen välin tekevä komento \komento{vspace} sattuu
sivunvaihdon kohdalle, väli jätetään kokonaan tekemättä. Komennosta on
olemassa tähtiversio \komento{vspace*}, joka tekee välin myös
sivunvaihdon kohdalle eli ennen tai jälkeen sivunvaihdon.

Vastaavalla tavalla toimii myös vaakasuuntainen väli \komento{hspace}.
Jos komento sattuu rivinvaihdon kohdalle, komentoa ei huomioida eli väli
jätetään tekemättä. Tähtiversio \komento{hspace*} tekee välin joka
tapauksessa.

\subsection{Mittakomennot ja typografinen viivasto}

Mittoja tallennetaan eräänlaisiin muuttujiin, jotka näyttävät päällepäin
komennoilta eli niiden alussa on kenoviiva (\koodi{\keno}) ja sitten
kirjaimista koostuva nimi. Esimerkiksi mitta \mitta{textwidth} on
tekstialueen leveys nykyisellä sivulla. Komentomaisesta ulkoasustaan
huolimatta mittoja ei voi suorittaa komentoina; ne sopivat vain komennon
argumentiksi, silloin kun tarvitaan mitta.

Uusia mittoja luodaan komennolla \komento{newlength} ja olemassa olevia
mittoja asetetaan esimerkiksi komennoilla \komento{setlength} ja
\komento{addtolength}. Omien mittoja on tarpeen luoda silloin, kun
halutaan määritellä tietynsuuruinen mitta, jota käytetään Latex\-/
koodissa useita kertoja.

\komentoi{newlength}
\komentoi{setlength}
\komentoi{addtolength}
\begin{koodilohkosis}
\newlength{\omamitta}         % Luodaan mitta.
\setlength{\omamitta}{2.3em}  % Asetetaan mitta.
\addtolength{\omamitta}{1em}  % Lisätään mittaan.
\addtolength{\omamitta}{-1em} % Vähennetään mitasta.
\end{koodilohkosis}

\noindent
Näin luotuja mittoja voi käyttää mittayksiköiden tavoin, eli niille voi
asettaa eteen kertoimen. Esimerkiksi seuraava \komento{hspace}\-/
komento luo vaakasuuntaisen välin, jonka pituus on 0,7 kertaa
\mittax{omamitta}:

\komentoi{hspace}
\begin{koodilohkosis}
\hspace{0.7\omamitta}
\end{koodilohkosis}

\noindent
Vaakasuuntaisten välien tekeminen on typografiassa varsin tavallista,
joten niitä varten on olemassa omia komentoja. Typografisen neliön
levyisen (1\,em) välin voi tehdä komennolla \komento{quad}. Sen
puolikkaan (\murtoluku{1}{2}\,em) saa komennolla \komento{enspace}.
Ohuke eli \murtoluku{1}{6}\,em-väli tehdään \komento{,}\-/ komennolla,
josta on tarkempaa tietoa luvussa \ref{luku/ohuke}.

Perus Latexissa ei voi tehdä laskutoimituksia mittojen avulla, mutta
\pakettictan{calc}\-/ paketti laajentaa \komento{setlength}\-/{} ja
\komento{addtolength}\-/ komentoja siten, että niiden argumentissa voi
toteuttaa yhteen\-/, vähennys\-/, kerto- ja jako\-las\-kuja
operaattoreilla \koodi{+-*/}. Myös sulkeita \koodi{()} voi käyttää
laskujärjestyksen muuttamiseen samoin kuin matemaattisissa lausekkeissa
tavallisestikin. Jos samassa lausekkeessa on sekä mittoja (esim.
\mitta{parindent} tai \koodi{1em}) että vakiotermejä (esim.~\koodi{2}),
täytyy mitta mainita lausekkeessa ensin. Seuraava esimerkki
havainnollistaa \paketti{calc}\-/ paketin laskutoimituksia:

\komentoi{setlength}
\mittai{parindent}
\begin{koodilohkosis}
\setlength{\omamitta}{\parindent + 1em}
\setlength{\omamitta}{\parindent * 2}  % ei: 2 * \parindent
\end{koodilohkosis}

\noindent
Latex sisältää myös komennot mittojen poimimiseen kirjaimista tai muista
merkeistä. Mittoja on kolme: leveys (\englanti{width}), korkeus
(\englanti{height}) ja syvyys (\englanti{depth}). Leveys on merkin
viemä tila leveyssuunnassa. Korkeus tarkoittaa merkin korkeutta
peruslinjan yläpuolella, ja syvyys on merkin korkeus peruslinjan
alapuolella. Kuva \ref{kuva/kirjainmitat} havainnollistaa näitä kolmea
mittaa sekä typografista viivastoa.

\leijukuva{
  \rmfamily
  \begin{tikzpicture}
    [viiva/.style={line width=.7bp, densely dotted, color=apuviiva},
    nuoli/.style={<->, line width=1bp, color=mittanuoli}, xscale=1.1,
    yscale=1.9, baseline=0pt]

    \node at (.04,.25) {\fontsize{120bp}{120bp}\selectfont p};

    \draw [viiva] (-1,.98) rectangle (1,-.5);
    \draw [viiva] (1,.98) -- (2.25,.98);
    \draw [viiva] (-1,.02) -- (2.25,.02);
    \draw [viiva] (1,-.5) -- (2.25,-.5);

    \draw [nuoli] (1.2,.02) -- (1.2,.98);
    \node at (1.9,.5) [color=mittanuoli] {korkeus};

    \draw [nuoli] (1.2,.02) -- (1.2,-.5);
    \node at (1.9,-.25) [color=mittanuoli] {syvyys};

    \draw [nuoli] (-1,1.1) -- (1,1.1);
    \node at (0,1.25) [color=mittanuoli] {leveys};

    \node at (3.13,1.64) {ylälinja};
    \node at (3.13,1.44) {versaalilinja};
    \node at (3.13,.98) {gemenalinja};
    \node at (3.13,.02) {peruslinja};
    \node at (3.13,-.5) {alalinja};

    \node at (6.52,.74) {\fontsize{120bp}{120bp}\selectfont H};

    \draw [viiva] (5.05,1.47) rectangle (7.95,.02);
    \draw [viiva] (5.05,.02) -- (4,.02);
    \draw [viiva] (5.05,1.47) -- (4,1.47);
    \draw [viiva] (7.95,1.62) -- (4,1.62);
    \draw [viiva] (7.95,.98) -- (4,.98);

    \draw [nuoli] (4.86,.02) -- (4.86,1.47);
    \node at (4.1,.5) [color=mittanuoli] {korkeus};

    \draw [nuoli] (5.05,-.1) -- (7.95,-.1);
    \node at (6.5,-.3) [color=mittanuoli] {leveys};
  \end{tikzpicture}
}{
  \caption{Kirjainten mitat ja typografinen viivasto}
  \label{kuva/kirjainmitat}
}

Gemenalinja sijaitsee gemenakirjainten eli pienaakkosten korkeudella.
Peruslinjan ja gemenalinjan etäisyys on fontin x\=/korkeus (1\,ex).
Optisen vaikutelman vuoksi gemenalinja ei tarkoita ehdotonta
gemenakirjainten ylintä kohtaa. Esimerkiksi p\=/kirjaimen yläpääte ja
silmukan yläkaari voivat yltää hieman gemenalinjan yläpuolelle ja
silmukan alakaari peruslinjan alapuolelle. Fonttien suunnittelijat
tekevät tällaisia optisia korjauksia, jotta kirjainten hahmot näyttävät
yhtä korkeilta ja tasapainoisilta yhdessä.

Versaalilinja sijaitsee versaalikirjainten eli suuraakkosten
korkeudella. Jotkin kirjaimet voivat yltää hieman versaalilinjan
yläpuolellekin. Esimerkiksi joidenkin fonttien Hk\=/kirjaimia
vertailemalla näkee, että k\=/kirjaimen yläpääte yltää H\=/kirjaimen eli
versaalilinjan yläpuolelle, ylälinjan tuntumaan. Myös tarkemerkit (luku
\ref{luku/tarkkeet}) voivat sijaita versaalilinjan yläpuolella.

Merkkien leveyden, korkeuden ja syvyyden voi mitata seuraavan esimerkin
komennoilla. Esimerkin ensimmäisellä rivillä luodaan uudet mitat
\mittax{leveys}, \mittax{korkeus} ja \mittax{syvyys}, joihin sitten
tallennetaan merkkien mitat.

\komentoi{newlength}
\komentoi{settowidth}
\komentoi{settoheight}
\komentoi{settodepth}
\begin{koodilohkosis}
\newlength{\leveys} \newlength{\korkeus} \newlength{\syvyys}
\settowidth{\leveys}{abc} % merkkien ”abc” leveys
\settoheight{\korkeus}{H} % merkin ”H” korkeus
\settodepth{\syvyys}{p}   % merkin ”p” syvyys
\end{koodilohkosis}

\noindent
Mitat saa näkyviin kirjoittamalla mitan eteen komennon \komento{the}.
Näin mitan pituus ladotaan dokumenttiin. Yksikkönä on typografinen piste
(pt).

\komentoi{the}
\begin{koodilohkosis}
Leveys: \the\leveys, korkeus: \the\korkeus, syvyys: \the\syvyys.
\end{koodilohkosis}

\begin{tulossis}
  Leveys: 14.62865pt, korkeus: 6.81897pt, syvyys: 2.4662pt.
\end{tulossis}

\noindent
Aiemmin mainituista optisista korjauksista johtuu, että gemenakirjaimet
eivät ole välttämättä täsmälleen samankorkuisia. Gemena\=/x asettuu
perus- ja gemenalinjojen väliin, mutta gemena\=/o voi yltää aavistuksen
verran linjojen ylä- ja alapuolelle. Ihmisen silmään ne näyttävät yhtä
korkeilta.

\subsection{Venyvät mitat ja välit}
\label{luku/venyvät-mitat}

Edellä on puhuttu vain kiinteistä mitoista, mutta Latex tuntee myös
venyvät mitat. Niiden ajatuksena on, että Latexille voi antaa luvan
kutistaa tai venyttää mittaa tietyissä rajoissa. Venyviä mittoja
käytettään usein pystysuuntaisissa väleissä, esimerkiksi väliotsikon
edellä ja jälkeen tai tekstikappaleiden välissä. Latex pystyy latomaan
sivut yleensä paremman näköiseksi, kun sille antaa venyvien mittojen
avulla hieman säätövaraa. Alla on esimerkki venyvän kappalevälin
(\mitta{parskip}) määrittämisestä. Samalla asetetaan kappaleen
ensimmäisen rivin sisennys (\mitta{parindent}) nollaan (\koodi{0em}).

\komentoi{setlength}
\mittai{parskip}
\mittai{parindent}
\begin{koodilohkosis}
\setlength{\parskip}{2ex plus 0.2ex minus 0.1ex}
\setlength{\parindent}{0em}
\end{koodilohkosis}

\noindent
Venyvissä mitoissa mainitaan ensin normaali pituus ja sitten sanoilla
\koodi{plus} ja \koodi{minus} ilmaistaan, kuinka paljon mitta voi venyä
tai kutistua. Molempia ei välttämättä tarvitse antaa.

Venyvät mitat voivat sisältää ''äärettömän'' mittayksikön \koodi{fill},
joka antaa luvan venyttää mittaa niin, että kaikki käytettävissä oleva
tila täyttyy. Äärettömän mittayksikön kanssa mittaluku ilmaisee
suhdeluvun muihin äärettömiin mittoihin. Seuraava esimerkki
havainnollistaa asiaa:

\komentoi{hspace}
\begin{koodilohkosis}
x\hspace{0mm plus 1fill}x\hspace{0mm plus 2fill}x
\end{koodilohkosis}

\begin{tulossis}
  x\hspace{0mm plus 1fill}x\hspace{0mm plus 2fill}x
\end{tulossis}

\noindent
Edellisen esimerkin mittojen luonnollinen pituus on nolla (\koodi{0mm}),
mutta \koodi{plus} ja \koodi{fill} \=/mitan vuoksi ne voivat venyä ja
täyttää koko käytettävissä olevan tilan. Ensin mainitulla on suhdeluku 1
(\koodi{1fill}) ja jälkimmäisellä suhdeluku 2 (\koodi{2fill}), joten
jälkimmäinen väli on kaksinkertainen ensimmäiseen verrattuna. Saman
äärettömästi venyvän mitan ja suhdeluvut voi ilmaista myös
\komento{stretch}\-/ mittakomennon avulla seuraavasti:

\komentoi{hspace}
\komentoi{stretch}
\begin{koodilohkosis}
x\hspace{\stretch{1}}x\hspace{\stretch{2}}x
\end{koodilohkosis}

\begin{tulossis}
  x\hspace{\stretch{1}}x\hspace{\stretch{2}}x
\end{tulossis}

\noindent
Äärettömästi venyvälle, koko tilan täyttävälle mittayksikölle on itse
asiassa kolme eri versiota: \koodi{fil}, \koodi{fill} ja \koodi{filll}.
Niiden erona on tärkeysjärjestys. Ensin mainittu \koodi{fil} on vähiten
tärkeä, ja viimeinen eli \koodi{filll} on tärkein. Jos samassa
yhteydessä käytetään eri tärkeysasteisia yksiköitä, ylemmäntasoiset
mitätöivät alemmantasoiset.

\komentoi{hspace}
\begin{koodilohkosis}
x\hspace{0mm plus 1filll}x\hspace{0mm plus 1fill}x
\end{koodilohkosis}

\begin{tulossis}
  x\hspace{0mm plus 1filll}x\hspace{0mm plus 1fill}x
\end{tulossis}

\noindent
Edellisessä esimerkissä ensimmäinen venyvä mitta \koodi{1filll} mitätöi
jälkimmäisen \koodi{1fill} kokonaan. Yleensä lienee parasta käyttää
keskimmäistä (\koodi{fill}), mutta eri tärkeysasteille voi olla käyttöä
esimerkiksi laajennuspaketin koodissa: \koodi{fil} sallii, että
kirjoittaja tai muu koodi syrjäyttää asetuksen; \koodi{filll} on ehdoton
sääntö, joka syrjäyttää muut.

Käytännöllisyyssyistä äärettömästi venyvä pituus \koodi{0mm plus 1fill}
on jo valmiiksi asetettu mittaan \mitta{fill}. Sitä on kätevää käyttää
esimerkiksi vaaka- ja pystysuuntaisia välejä latovien komentojen
\komento{hspace} ja \komento{vspace} kanssa:

\komentoi{hspace}
\komentoi{vspace}
\mittai{fill}
\begin{koodilohkosis}
\hspace{\fill} % ääretön vaakasuuntainen väli
\vspace{\fill} % ääretön pystysuuntainen väli
\end{koodilohkosis}

\noindent
Edellisen esimerkin komennoista on olemassa vieläkin lyhemmät versiot.
Vaakasuuntainen ääretön väli syntyy myös komennolla \komento{hfill} ja
pystysuuntainen väli komennolla \komento{vfill}. Seuraavassa toistetaan
eräs aiempi esimerkki yksinkertaisemmalla tavalla:

\komentoi{hfill}
\begin{koodilohkosis}
x\hfill x\hfill\hfill x
\end{koodilohkosis}

\begin{tulossis}
  x\hfill x\hfill\hfill x
\end{tulossis}

\section{Laskurit}
\label{luku/laskurit}

Laskureiden avulla Latex pitää kirjaa esimerkiksi sivunumeroista,
lukujen ja alalukujen sekä kuvien ja muiden elementtien numeroinnista.
Esimerkiksi nyt olemme pääluvun~\arabic{chapter}
alaluvussa~\arabic{section}, ja mainitut luvut tulevat Latexin
laskureista automaattisesti.

\leijutlk{
  \laskurii{part}
  \laskurii{paragraph}
  \laskurii{figure}
  \laskurii{enumi}
  \laskurii{chapter}
  \laskurii{subparagraph}
  \laskurii{table}
  \laskurii{enumii}
  \laskurii{section}
  \laskurii{page}
  \laskurii{footnote}
  \laskurii{enumiii}
  \laskurii{subsection}
  \laskurii{equation}
  \laskurii{mpfootnote}
  \laskurii{enumiv}
  \laskurii{subsubsection}

  \begin{tabular}{llll}
    \toprule
    \laskuri{part} & \laskuri{paragraph}
    & \laskuri{figure} & \laskuri{enumi} \\
    \laskuri{chapter} & \laskuri{subparagraph}
    & \laskuri{table} & \laskuri{enumii} \\
    \laskuri{section} & \laskuri{page}
    & \laskuri{footnote} & \laskuri{enumiii} \\
    \laskuri{subsection} & \laskuri{equation}
    & \laskuri{mpfootnote} & \laskuri{enumiv} \\
    \laskuri{subsubsection} \\
    \bottomrule
  \end{tabular}
}{
  \caption{Latexin laskurit}
  \label{tlk/latexin-laskurit}
}

Taulukossa \ref{tlk/latexin-laskurit} ovat Latexin perus laskurit. Monet
niistä ovat otsikoiden numerointia varten (\englanti{\laskuri{part},
  \laskuri{chapter}, \laskuri{section}, \laskuri{paragraph}}).
Sivunumeroita varten on \laskuri{page}\-/laskuri ja matemaattisten
kaavojen numerointiin \laskuri{equation}\-/laskuri. Leijuvista kuvista
ja taulukoista (luku \ref{luku/leijuosat}) pidetään kirjaa laskureissa
\laskuri{figure} ja \laskuri{table}, ja alaviitteiden (luku
\ref{luku/alaviitteet}) numerointi on laskureissa \laskuri{footnote} ja
\laskuri{mpfootnote}. Numeroidut luetelmat (luku \ref{luku/luetelmat})
käsitellään \laskurix{enum}\-/alkuisten laskureiden avulla siten, että
perustasolla käytetään laskuria \laskuri{enumi}, ja jos sen
luetelmakohta sisältää toisen numeroidun luetelman, käytetään siinä
laskuria \laskuri{enumii}, sen sisällä laskuria \laskuri{enumiii} jne.

Edellä mainituista laskureista ei tarvitse yleensä itse huolehtia, sillä
ne ovat vain tekniikkaa, joka toimii korkeamman tason toimintojen
taustalla. Joskus on kuitenkin käyttöä myös omille laskureille. Seuraava
esimerkki esittelee laskureiden käsittelyn peruskomennot. Niissä
käytetään laskuria nimeltä \laskurix{oma}. Laskurien nimet koostuvat
pelkistä kirjaimista, eikä niiden alussa ole kenoviivaa niin kuin
komentojen ja mittojen alussa.

\komentoi{newcounter}
\komentoi{setcounter}
\komentoi{addtocounter}
\begin{koodilohkosis}
\newcounter{oma}       % Luodaan uusi laskuri ”oma”.
\setcounter{oma}{3}    % Asetetaan laskurin arvoksi 3.
\addtocounter{oma}{1}  % Lisätään laskurin arvoon 1.
\addtocounter{oma}{-1} % Vähennetään laskurin arvosta 1.
\end{koodilohkosis}

\noindent
Laskuri on sisäisesti kokonaisluku, mutta sen arvon voi latoa eri
muodoissa: arabialaisena tai roomalaisena lukuna, kirjaimena tai
symbolien sarjana. Taulukossa \ref{tlk/laskurien-latominen} ovat komennot
laskurin arvon latomiseen. Komennon argumentiksi annetaan laskurin nimi.
Laskurin voi latoa kirjainmuodossa, jos sen arvo on 1--26; symbolit
toimivat vain lukualueella 1--9.

\leijutlk{
  \begin{tabular}{ll}
    \toprule
    \ots{Komento} & \ots{Merkitys} \\
    \midrule
    \komento{arabic} & arabialainen luku: 1, 2, 3\ldots \\
    \komento{roman} & roomalainen luku: i, ii, iii\ldots \\
    \komento{Roman} & roomalainen luku: I, II, III\ldots \\
    \komento{alph} & kirjain: a, b, c\ldots\ (vain 1--26) \\
    \komento{Alph} & kirjain: A, B, C\ldots\ (vain 1--26) \\
    \komento{fnsymbol} & symboli:
                         \textasteriskcentered
                         \textdagger
                         \textdaggerdbl\S\P\ldots\ (vain 1--9) \\
    \bottomrule
  \end{tabular}
}{
  \caption{Komennot laskurien arvon latomiseen}
  \label{tlk/laskurien-latominen}
}

Taulukon \ref{tlk/laskurien-latominen} komentojen lisäksi laskuriin
liittyvän arvon voi latoa komennolla, joka alkaa kirjaimilla
\komentox{the} ja jatkuu laskurin nimellä. Esimerkiksi komento
\komento{thepage} latoo sivunumeron eli tekee käytännössä saman asian
kuin komento \komento{arabic}\komentoarg{page}. Aina nämä eivät
kuitenkaan ole sama asia, ja \komentox{the}\-/alkuinen komento voi olla
määritelty toisellakin tavalla. Esimerkiksi tämän oppaan leijuvat
taulukot numeroidaan laskurilla \laskuri{table}, mutta komento
\komento{thetable} latoo ensin pääluvun ja sen perään pisteen ja
taulukon numeron. Viimeisin taulukko on numeroltaan \thetable{} eli
pääluvun~\arabic{chapter} taulukko~\arabic{table}.

Jos laskurin arvoa tarvitaan Latexin teknisessä tilanteessa eikä
tarkoituksena ole latoa sitä näkyviin itse dokumenttiin, täytyy käyttää
komentoa \komento{value}. Seuraava esimerkki luo laskurin nimeltä
\laskurix{mitta}, jonka arvoa (3) käytetään \komento{hspace}\-/ komennon
argumenttina mitan ilmaisemiseen (3\,em).

\komentoi{newcounter}
\komentoi{setcounter}
\komentoi{hspace}
\komentoi{value}
\begin{koodilohkosis}
\newcounter{mitta}
\setcounter{mitta}{3}
x\hspace{\value{mitta}em}x
\end{koodilohkosis}

\begin{tulossis}
  x\hspace{3em}x
\end{tulossis}

\noindent
Laskurien avulla voi suorittaa peruslaskutoimituksia, kun lataa paketin
\pakettictan{calc}. Paketti määrittelee uudelleen komennot
\komento{setcounter} ja \komento{addtocounter} siten, että komennon
argumentti voi sisältää yhteen\-/, vähennys\-/, kerto- ja
jako\-las\-kuja käyttämällä operaattoreita \koodi{+-*/}. Sulkeita
\koodi{()} voi käyttää lausekkeen laskujärjestyksen muuttamiseen.

\komentoi{setcounter}
\komentoi{value}
\laskurii{page}
\begin{koodilohkosis}
\setcounter{oma}{\value{page} + 3}
\end{koodilohkosis}

\subsection{Hierarkkiset laskurit}
\label{luku/hierarkkiset-laskurit}

Laskurit voivat olla hierarkkisia ja riippuvaisia toisista laskureista:
kun yhden laskurin arvo kasvaa, nollautuu jokin toinen laskuri
automaattisesti. Tällaista toimintoa käytetään dokumentin lukujen eli
otsikoiden numeroinnissa. Kirjan pääluvun~1 alaluvut ovat esimerkiksi
1.1, 1.2, 1.3 jne. Kun alkaa seuraava pääluku~2, nollautuu alaluvun
laskuri, jotta saadaan oikein numeroidut alaluvut 2.1, 2.2, 2.3 jne.
Latex huolehtii lukujen numeroinnista automaattisesti, mutta omille
laskureille täytyy määritellä riippuvuussuhteet itse. Seuraavassa
esimerkissä luodaan oma laskuri, joka nollautuu automaattisesti aina,
kun sivu vaihtuu eli laskurin \laskuri{page} arvo kasvaa.

\komentoi{newcounter}
\laskurii{page}
\begin{koodilohkosis}
\newcounter{oma}[page]
\end{koodilohkosis}

\noindent
Ylemmän tason laskureita ei kannata kasvattaa komennolla
\komento{addtocounter}, koska se vain muuttaa laskurin arvoa mutta ei
huolehdi alemman tason laskurien nollaamisesta. Hierarkkisten laskurien
kasvattamiseen on seuraavat kaksi komentoa:

\komentoi{stepcounter}
\komentoi{refstepcounter}
\begin{koodilohkosis}
\stepcounter{laskuri}
\refstepcounter{laskuri}
\end{koodilohkosis}

\noindent
Edellä mainitut komennot kasvattavat argumenttina annetun laskurin arvoa
yhdellä ja nollaavat siitä riippuvaiset alemman tason laskurit.
Jälkimmäinen komento \komento{refstepcounter} asettaa lisäksi
ristiviitteen numeron, joten tämän komennon jälkeen mahdollisesti tuleva
\komento{label}\-/ komento luo ristiviitekohteen. Kun tähän kohtaan
viitataan muualta käyttämällä \komento{ref}\-/ komentoa, viitteessä
näkyy laskurin numero tai oikeastaan \komentox{thelaskuri}\-/ komennon
latoma teksti. Ristiviitteitä käsitellään tarkemmin luvussa
\ref{luku/ristiviitteet}.

Paketti \pakettictan{chngcntr} helpottaa hierarkkisten laskurien
muuttamista jälkikäteen. Paketin komentojen avulla voi jälkikäteen
asettaa jonkin laskurin riippuvaiseksi toisesta laskurista tai poistaa
riippuvuuden.

\komentoi{newcounter}
\komentoi{counterwithin}
\komentoi{counterwithout}
\begin{koodilohkosis}
\newcounter{oma}           % Luodaan laskuri ”oma”.
\counterwithin {oma}{page} % Asetetaan riippuvuus page-laskurista.
\counterwithout{oma}{page} % Poistetaan riippuvuus.
\end{koodilohkosis}

\noindent
Edellä mainitut \paketti{chngcntr}\-/paketin komennot määrittelevät
uudelleen myös \komentox{the\-oma}\-/ komennon, jolla \laskurix{oma}\-/
laskurin arvon voi latoa. Se latoo mukaan myös ylemmäntasoisen laskurin
arvon. Jos tätä \komentox{the}\-/ alkuista komentoa ei halua määritellä
uudelleen, täytyy käyttää tähdellisiä komennon versioita:
\komento{counterwithin*} ja \komento{counterwithout*}. Laskuriin
liittyvän \komentox{the}\-/ komennon voi aina itsekin määritellä
haluamansa laiseksi komennolla \komento{renewcommand} (luku
\ref{luku/komennot-määrittely}), esimerkiksi seuraavalla tavalla:

\komentoi{renewcommand}
\komentoi{arabic}
\komentoi{alph}
\laskurii{page}
\begin{koodilohkosis}
\renewcommand{\theoma}{\arabic{page}/\alph{oma}}
\end{koodilohkosis}

\subsection{Kokonaislaskurit}

Latex ei tiedä etukäteen, mihin arvoon laskurit lopulta yltävät. Se vain
latoo sivuja peräkkäin eikä tiedä mitään tulevasta. Sen vuoksi
esimerkiksi dokumentin sivumäärää ei voi ihan yksinkertaisella
komennolla latoa dokumentin alkusivuille.

Tekninen vastaus tämäntyyppisiin ongelmiin löytyy ristiviitteistä (luku
\ref{luku/ristiviitteet}). Ne toimivat sisäisesti niin, että dokumentin
ensimmäisellä kääntökerralla kirjoitetaan väliaikaistiedostoon muistiin
tarpeellisia asioita ja seuraavalla kääntökerralla hyödynnetään
väliaikaistiedostoa. Dokumentin sivumäärän ja viimeisen sivunumeron
tallentamiseen on olemassa paketti \pakettictan{totpages}, joka lisää
viimeiselle sivulle automaattisesti ristiviitteen nimeltä
\koodi{Tot\-Pages}. Sivumäärän ja viimeisen sivun numeron voi latoa
komennoilla \komento{ref} ja \komento{pageref}. Esimerkki
\ref{esim/totpages} havainnollistaa niiden käyttöä.

\begin{esimerkki*}
  \luokkai{article}
  \komentoi{documentclass}
  \komentoi{usepackage}
  \pakettii{totpages}
  \komentoi{ref}
  \komentoi{pageref}
  \komentoi{addtocounter}
  \laskurii{page}
  \komentoi{newpage}

\begin{koodilohko}
\documentclass{article}
\usepackage{totpages}

\begin{document}
Sivumäärä: \ref{TotPages}. Viimeinen sivu: \pageref{TotPages}.

\addtocounter{page}{10} % Sivunumerot 11, 12, 13...
\newpage Jotain... \newpage ...sisältöä.
\end{document}
\end{koodilohko}
  \caption{Dokumentin sivumäärän ja viimeisen sivun numeron latominen}
  \label{esim/totpages}
\end{esimerkki*}

Muunlaisten kokonaislaskurien toteuttamiseen voi käyttää
\pakettictan{totcount}\-/ pakettia. Se tarjoaa komennon, jolla
rekisteröidään olemassa oleva laskuri kokonaislaskuriksi. Lisäksi on
komento, jolla ladotaan tai palautetaan laskurin lopullinen arvo.

\komentoi{newcounter}
\komentoi{regtotcounter}
\komentoi{addtocounter}
\komentoi{total}
\komentoi{totvalue}
\begin{koodilohkosis}
\newcounter{oma}      % Luodaan laskuri ”oma”.
\regtotcounter{oma}   % Rekisteröidään ”oma” kokonaislaskuriksi.
\addtocounter{oma}{1} % Kasvatetaan laskurin arvoa.
\total{oma}           % Ladotaan laskurin lopullinen arvo.
\totvalue{oma}        % Palautetaan laskurin lopullinen arvo.
\end{koodilohkosis}

\noindent
Edellä olevan esimerkin viimeistä komentoa \komento{totvalue} ei käytetä
laskurin arvon latomiseen itse dokumenttiin vaan sitä käytetään
teknisissä yhteyksissä kuten toisen Latex\-/komennon argumenttina.

Myös Latexin valmiita laskureita voi rekisteröidä kokonaislaskureiksi.
Tässä oppaassa on yhteensä \total{chapter} numeroitua päälukua, ja
mainittu lukumäärä saatiin laskettua seuraavasti:

\komentoi{regtotcounter}
\komentoi{total}
\laskurii{chapter}
\begin{koodilohkosis}
\regtotcounter{chapter} % Sijoitetaan dokumentin esittelyosaan.
\total{chapter}         % Latoo päälukujen (chapter) lukumäärän.
\end{koodilohkosis}

\section{Laatikot}
\label{luku/laatikot}

Tex ajattelee ladottavan tekstin ja muun sisällön laatikoina eli
suorakulmioina, joilla on leveys ja korkeus. Esimerkiksi kirjaimet ovat
teknisesti laatikoita, joita ladotaan peräkkäin. Samoin monet valmiit
ympäristöt luovat laatikon, joka sisältää jotakin ja vie tietyn verran
tilaa sivulta.

Dokumentin kirjoittajalle on välillä hyötyä luoda omia laatikoita, koska
laatikko toimii yhtenä kokonaisuutena, jonka reunoja muut elementit ja
latomiskoneisto kunnioittavat.

\subsection{Pienet laatikot}
\label{luku/laatikot-pienet}

Yksinkertainen peruslaatikko tehdään \komento{mbox}\-/ komennolla, joka
latoo argumenttina annetun tekstin näkymättömän laatikon sisään. Näin
luotu laatikko pysyy koossa, eli sen sisältämiä sanoja ei esimerkiksi
tavuteta. Komento sopiikin hyvin tavutuksen estämiseen yksittäisessä
tilanteessa.

Toinen laatikkokomento \komento{makebox} toimii samalla tavalla kuin
\komento{mbox} mutta hyväksyy yhden tai kaksi valinnaista argumenttia.
Komentojen muoto on seuraava:

\komentoi{makebox}
\begin{koodilohkosis}
\makebox[leveys]{teksti}
\makebox[leveys][sijainti]{teksti}
\end{koodilohkosis}

\noindent
Argumentti \koodi{leveys} on laatikon leveysmitta ja \koodi{sijainti} on
kirjainkoodi, joka ilmaisee laatikon sisällön vaakasuuntaisen
sijoittelun. Kirjain \koodi{c} (\englanti{center}) keskittää sisällön,
\koodi{l} (\englanti{left}) tasaa sisällön vasemmalle, \koodi{r}
(\englanti{right}) tasaa sisällön oikealle ja \koodi{s}
(\englanti{stretch}) tasaa sisällön molemmista reunoista eli venyttää
laatikon sisäisiä sanavälejä sopivasti.

\komentoi{makebox}
\begin{koodilohkosis}
\makebox[12em][s]{Laatikossa on tilaa.}
\end{koodilohkosis}

\begin{tulossis}
  \makebox[12em][s]{Laatikossa on tilaa.}
\end{tulossis}

\noindent
\komento{makebox}\-/ komennon \koodi{leveys}\-/ argumentissa voi käyttää
apuna mittoja \mitta{width} (leveys), \mitta{height} (korkeus),
\mitta{depth} (syvyys) ja \mitta{totalheight} (kokonaiskorkeus:
\mitta{height} + \mitta{depth}), jotka määräytyvät laatikon sisällön
luontaisten mittojen perusteella. Kuvassa \ref{kuva/kirjainmitat}
(s.~\pageref{kuva/kirjainmitat}) oleva typografinen viivasto
havainnollistaa näitä mittoja.

Seuraava esimerkki latoo laatikon, jonka leveys on 1,5\-/ kertainen
sisällön luontaiseen leveyteen nähden, ja keskittää tekstin. Pystyviivat
ovat mukana rajojen hahmottamisen vuoksi.

\komentoi{makebox}
\mittai{width}
\begin{koodilohkosis}
|\makebox[1.5\width][c]{Keskitetty}|
\end{koodilohkosis}

\begin{tulossis}
  |\makebox[1.5\width][c]{Keskitetty}|
\end{tulossis}

\noindent
Laatikkokomento \komento{fbox} toimii kuten \komento{mbox} mutta piirtää
laatikolle myös näkyvät kehykset. Komento \komento{framebox} on
puolestaan kehyksellinen versio \komento{makebox}\-/ komennosta.
Kehysviivan leveyden voi asettaa mitan \mitta{fboxrule} avulla, ja
kehysviivan etäisyys sisällöstä määritellään mitassa \mitta{fboxsep}.

\komentoi{fbox}
\komentoi{framebox}
\komentoi{setlength}
\mittai{fboxrule}
\mittai{fboxsep}
\mittai{width}
\begin{koodilohkosis}
\setlength{\fboxsep}{3bp}
\setlength{\fboxrule}{.5bp}
\framebox[1.2\width][c]{Laatikossa \fbox{laatikko}}
\end{koodilohkosis}

\begin{tulossis}
  \setlength{\fboxsep}{3bp}
  \setlength{\fboxrule}{.5bp}
  \framebox[1.2\width][c]{Laatikossa \fbox{laatikko}}
\end{tulossis}

\noindent
Tekstiä sisältävien laatikoiden korkeus vaihtelee kirjainten ja muiden
merkkien muodon mukaan. Esimerkiksi kirjainten ala- ja yläpidennykset
kasvattavat laatikon kokoa gemenalinjan yläpuolelle tai peruslinjan
alapuolelle. Seuraavasta esimerkistä voi vertailla tekstilaatikoiden
muodostumista:

\komentoi{setlength}
\komentoi{fbox}
\mittai{fboxsep}
\begin{koodilohkosis}
\setlength{\fboxsep}{0bp}
\fbox{sana} \fbox{kirja} \fbox{×}
\end{koodilohkosis}

\begin{tulossis}
  \setlength{\fboxsep}{0bp}
  \fbox{sana} \fbox{kirja} \fbox{×}
\end{tulossis}

\noindent
Edellisen esimerkin kaltaisissa tilanteissa voi olla tarpeen käyttää
laatikon sisällä komentoa \komento{strut}. Se latoo näkymättömän,
leveydettömän merkin, jonka korkeus on rivikorkeuden mukainen (mitta
\mitta{baselineskip}).

\komentoi{setlength}
\komentoi{fbox}
\komentoi{strut}
\mittai{fboxsep}
\begin{koodilohkosis}
\setlength{\fboxsep}{0bp}
\fbox{\strut sana} \fbox{\strut kirja} \fbox{\strut ×}
\end{koodilohkosis}

\begin{tulossis}
  \setlength{\fboxsep}{0bp}
  \fbox{\strut sana} \fbox{\strut kirja} \fbox{\strut ×}
\end{tulossis}

\subsection{Suuret laatikot}
\label{luku/laatikot-isot}

Jos laatikon täytyy sisältää useita tekstikappaleita tai muuta
pystysuuntaista sisältöä, tarvitaan esimerkiksi \komento{parbox}\-/
komentoa. Se tarvitsee ainakin kaksi argumenttia: laatikon leveyden ja
sisällön.

\komentoi{parbox}
\begin{koodilohkosis}
\parbox{leveys}{sisältö}
\end{koodilohkosis}

\noindent
\komento{parbox}\-/ laatikon sisällä teksti ladotaan samoin kuin teksti
normaalistikin eli rivit katkaistaan sanavälien kohdalta ja
mahdollisesti tavutuskohdista. Muutkin tekstikappaleen asetukset
pätevät, ja asetukset voi määrittää haluamikseen laatikon sisällä.
Lisätietoa tekstikappaleista on luvussa \ref{luku/kappale}.

Komento \komento{parbox} hyväksyy myös yhden, kaksi tai kolme
valinnaista argumenttia. Niillä määritellään laatikon pystysuuntainen
sijoittelu ympäristöönsä nähden, laatikon korkeus ja laatikon sisällön
pystysuuntainen sijoittelu. Valinnaiset argumentit toimivat seuraavasti:

\komentoi{parbox}
\begin{koodilohkosis}
\parbox[sijainti]{leveys}{sisältö}
\parbox[sijainti][korkeus]{leveys}{sisältö}
\parbox[sijainti][korkeus][sisäsijainti]{leveys}{sisältö}
\end{koodilohkosis}

\noindent
Argumentti \koodi{sijainti} on kirjain \koodi{c} (\englanti{center},
keskilinja), \koodi{t} (\englanti{top}, ylälinja) tai \koodi{b}
(\englanti{bottom}, alalinja). Näillä määritetään, miten laatikko
sijoitetaan pystysuunnassa suhteessa tekstiin. Seuraava esimerkki
havainnollistaa ensimmäisen valinnaisen argumentin toimintaa. Laatikon
sisällä on rivinvaihtokomentoja (\komento{\keno}) ja kolme tekstiriviä.

\komentoi{parbox}
\begin{koodilohkosis}
abc \parbox[c]{1em}{1 \\ 2 \\ 3}
abc \parbox[t]{1em}{1 \\ 2 \\ 3}
abc \parbox[b]{1em}{1 \\ 2 \\ 3} abc
\end{koodilohkosis}

\begin{tulossis}
  abc \parbox[c]{1em}{1 \\ 2 \\ 3}
  abc \parbox[t]{1em}{1 \\ 2 \\ 3}
  abc \parbox[b]{1em}{1 \\ 2 \\ 3} abc
\end{tulossis}

\noindent
Komennon toinen valinnainen argumentti \koodi{korkeus} määrittää
laatikon korkeuden. Oletuksena korkeus määräytyy sisällön korkeuden
mukaan, mutta tällä argumentilla korkeus asetetaan kiinteästi. Se voi
olla hyödyllistä varsinkin silloin, kun sisältö halutaan tasata laatikon
sisällä ylös, keskelle tai alas.

Kolmas valinnainen argumentti \koodi{sisäsijainti} nimittäin asettaa
laatikon sisällön pystysuuntaisen tasauksen. Argumentiksi annetaan
kirjain \koodi{c} (\englanti{center}, keskelle), \koodi{t}
(\englanti{top}, ylös), \koodi{b} (\englanti{bottom}, alas) tai
\koodi{s} (\englanti{stretch}, ylös ja alas). Viimeksi mainittu eli ylös
ja alas tasaaminen vaatii, että laatikon sisällä on venyviä
pystysuuntaisia välejä. Niitä käsitellään luvussa
\ref{luku/venyvät-mitat}.

Ympäristö \ymparisto{minipage} vastaa toiminnallisesti
\komento{parbox}\-/ komentoa. Se hyväksyy täsmälleen samat argumentit.
Erona on se, että \ymparisto{minipage}\-/ ympäristön sisällä voi käyttää
laatikon omia alaviitteitä (luku \ref{luku/alaviitteet}), jotka ladotaan
laatikon alaosaan.

\ymparistoi{minipage}
\begin{koodilohkosis}
\begin{minipage}[sijainti][korkeus][sisäsijainti]{leveys}
  ...
\end{minipage}
\end{koodilohkosis}

\subsection{Laatikoiden siirtely}

Komennolla \komento{raisebox} voi latoa laatikon tekstin ylemmäs tai
alemmas kuin se normaalisti sijoittuisi. Komennon ensimmäinen argumentti
on etäisyysmitta, kuinka paljon ylemmäs teksti ladotaan. Negatiivinen
mitta vie tekstin alemmas. Toinen argumentti on laatikon teksti.

\komentoi{raisebox}
\begin{koodilohkosis}
\raisebox{etäisyysmitta}{teksti}
\end{koodilohkosis}

\noindent
Argumentissa \koodi{etäisyysmitta} voi käyttää apuna mittoja
\mitta{width}, \mitta{height}, \mitta{depth} ja \mitta{totalheight},
joiden merkitys on sama kuin \komento{makebox}\-/ komennolla (luku
\ref{luku/laatikot-pienet}). \komento{raisebox}\-/ komennolle voi antaa
yhden tai kaksi valinnaista argumenttia, joilla määritellään laatikolle
uusi korkeus ja syvyys. Molemmat ovat mittoja, ja ne korvaavat siirretyn
laatikon tekstin luonnollisen korkeuden ja syvyyden. Komennon argumentit
annetaan seuraavasti:

\komentoi{raisebox}
\begin{koodilohkosis}
\raisebox{etäisyysmitta}[korkeus]{teksti}
\raisebox{etäisyysmitta}[korkeus][syvyys]{teksti}
\end{koodilohkosis}

\noindent
Grafiikkaan kuten kuviin ja väreihin erikoistunut paketti
\pakettictan{graphicx} sisältää hyödyllisiä laatikkokomentoja:
\komento{scalebox}, \komento{resizebox} ja \komento{rotatebox}. Komento
\komento{scalebox} skaalaa tekstin eri kokoiseksi vaaka- tai
pystysuunnassa.

\komentoi{scalebox}
\begin{koodilohkosis}
\scalebox{vaakaskaalaus}[pystyskaalaus]{teksti}
\end{koodilohkosis}

\noindent
Argumentit \koodi{vaakaskaalaus} ja \koodi{pystyskaalaus} ovat
kertoimia. Pystysuuntainen skaalausargumentti on valinnainen: jos sen
jättää pois, pystysuuntainen skaalaus on samansuuruinen kuin
vaakasuuntainen. Negatiivisella kertoimella voi kääntää tekstin
toisinpäin.

\komentoi{scalebox}
\begin{koodilohkosis}
\scalebox{2}[1]{leveä} \scalebox{-1}[1]{käännetty}
\end{koodilohkosis}

\begin{tulossis}
  \scalebox{2}[1]{leveä} \scalebox{-1}[1]{käännetty}
\end{tulossis}

\noindent
Samantapainen skaalaustoiminto saadaan myös komennolla
\komento{resizebox}. Sen argumentiksi ei anneta kertoimia vaan mittoja,
eli teksti skaalataan annettuihin mittoihin.

\komentoi{resizebox}
\komentoi{resizebox*}
\begin{koodilohkosis}
\resizebox {vaakamitta}{pystymitta}{teksti}
\resizebox*{vaakamitta}{pystymitta}{teksti}
\end{koodilohkosis}

\noindent
Komennon \komentoi{resizebox} normaali versio käsittelee
\koodi{pystymitta}\-/ argumentin siten, että se tulee olemaan tekstin
korkeus eli kirjainten peruslinjan yläpuolinen osa. Komennon
\komentoi{resizebox*} tähdellinen versio puolestaan toimii siten, että
pystymitta on tekstin kokonaiskorkeus eli peruslinjan yläpuolinen osa
(korkeus) ja alapuolinen osa (syvyys) yhdessä. Kuva
\ref{kuva/kirjainmitat} (s.~\pageref{kuva/kirjainmitat}) havainnollistaa
asiaa. \komento{resizebox}\-/ komennon argumenteissa voi käyttää apuna
mittoja \mitta{width}, \mitta{height}, \mitta{depth} ja
\mitta{totalheight} samoin kuin \komento{makebox}\-/ komennossa (luku
\ref{luku/laatikot-pienet}).

Laatikoiden pyörittely käy \komento{rotatebox}\-/ komennolla. Sen
argumentiksi täytyy antaa ainakin kääntökulma asteina ja käännettävä
teksti.

\komentoi{rotatebox}
\begin{koodilohkosis}
\rotatebox[asetuksia]{kääntökulma}{teksti}
\end{koodilohkosis}

\noindent
Valinnainen argumentti \koodi{asetuksia} voi sisältää asetusvalitsimia
ja niiden arvoja. Tärkein valitsin on \koodi{origin}, jolla määritetään
käännettävän tekstin kääntöpiste. Valitsimen arvoksi annetaan yksi tai
kaksi seuraavista kirjaimista: \koodi{c} (\englanti{center}, keski),
\koodi{l} (\englanti{left}, vasen), \koodi{r} (\englanti{right}, oikea),
\koodi{t} (\englanti{top}, ylä), \koodi{b} (\englanti{bottom}, ala) tai
\koodi{B} (\englanti{baseline}, peruslinja).

\komentoi{rotatebox}
\begin{koodilohkosis}
\rotatebox[origin=l] {25}{ylämäki}
\rotatebox[origin=r]{-25}{alamäki}
\end{koodilohkosis}

\begin{tulossis}
  \rotatebox[origin=l] {25}{ylämäki}
  \rotatebox[origin=r]{-25}{alamäki}
\end{tulossis}

\noindent
\komento{rotatebox}\-/ komennon valinnaisen argumentin valitsin
\koodi{units} määrittelee käytetyn kulmayksikön. Valitsimelle annetaan
arvoksi täysympyrän yksikkömäärä, ja oletusasetus on \koodi{units=\katk
  360}. Jos haluaa käyttää radiaaneja, sopiva asetus on
\koodi{units=\katk 6.283185} (2π\,rad).

\section{Tekstin ja laatikoiden värit}
\label{luku/värit}

Latexiin saa värit \pakettictan{xcolor}\-/ paketin
avulla.\footnote{Toinen vaihtoehto on yksinkertaisempi väripaketti
  \paketti{color}, joka sisältää värien käsittelyn perustoiminnot.} Se
tarjoaa muun muassa komentoja, joilla tekstin väriä, tekstin taustaväriä
tai sivun taustaväriä (luku \ref{luku/sivun-väri}) voi muuttaa. Lisäksi
paketti sisältää värien sekoittamiseen ja muuntamiseen liittyvää
tekniikkaa. Paketin komentojen avulla voi myös värittää taulukoiden
(luku \ref{luku/taulukot}) rivit vuorottelevin värein.

Omassa dokumentissa käytettävät värit on hyvä määritellä alussa.
Määrittely tarkoittaa sitä, että annetaan väreille nimet -- esimerkiksi
käyttötarkoituksen mukaan -- ja myöhemmin lähdedokumentissa viitataan
väreihin käyttämällä alussa määriteltyjä nimiä. Näin väri on määritelty
yhdessä paikassa ja sen muuttaminen on myöhemmin helppoa. On kyllä
mahdollista käyttää värejä ilman etukäteismäärittelyäkin.

\leijutlk{
  \providecommand{\rivi}{}
  \renewcommand{\rivi}[3]{\koodi{#1} & #2 & \koodi{#3}
    & \textcolor[#1]{#3}{\rule{.9em}{.9em}} \\}
  \begin{tabular}{llll}
    \toprule
    \ots{Malli}
    & \ots{Parametrit}
    & \multicolumn{2}{l}{\ots{Esimerkki}} \\
    \midrule
    \rivi{rgb}{punainen, vihreä, sininen}{1,.7,.3}
    \rivi{HTML}{punainen vihreä sininen}{33aaf3}
    \rivi{cmyk}{syaani, magenta, keltainen, musta}{.1,1,.5,0}
    \rivi{hsb}{sävy, kylläisyys, kirkkaus}{.2,1,.6}
    \rivi{gray}{harmaa}{.55}
    \bottomrule
  \end{tabular}
}{
  \caption{Erilaisia värimalleja ja niiden parametreja
    \paketti{xcolor}\-/ paketissa}
  \label{tlk/värimalleja}
}

Omat värit määritellään komennolla \komento{definecolor}. Se tarvitsee
kolme argumenttia: värin nimen, värimallin ja värikoodin. Komennon
argumenttien rakenne on seuraavanlainen:

\komentoi{definecolor}
\begin{koodilohkosis}
\definecolor{nimi}{välimalli}{parametrit}
\end{koodilohkosis}

\noindent
Komennon argumentti \koodi{nimi} on kirjoittajan itse valitsema nimi, ja
se voi liittyä esimerkiksi värin käyttötarkoitukseen (''otsikko'') tai
värin yleiseen nimeen (''keltainen''). Argumentti \koodi{väri\-malli}
tarkoittaa värin ilmaisemisen tapaa. Niitä on useita erilaisia, mutta
yleisimmät on koottu taulukkoon \ref{tlk/värimalleja}.

Eri värimalleilla on eri määrä parametreja, joilla ilmaistaan värin
ominaisuuksia. Kunkin parametrin arvo on yleensä desimaaliluku 0\==1,
mutta \koodi{HTML}\-/ värimallissa parametrit ilmaistaan yhtenä kolmen
kaksinumeroisen heksadesimaaliluvun sarjana \textsc{html}\-/
merkintäkielen tavoin. Värin parametrit kirjoitetaan komennon
argumenttiin \koodi{para\-metrit} ja ne erotetaan toisistaan pilkulla.
Taulukon \ref{tlk/värimalleja} Esimerkki\-/sarakkeessa on parametrien
antamisesta esimerkki. Muista värimalleista voi lukea
\paketti{xcolor}\-/ paketin ohjekirjasta.

Kun värit on määritelty, niitä voi käyttää eri komentojen yhteydessä.
Tekstin väri vaihdetaan toiseksi komennolla \komento{color}, josta on
kaksi erilaista versiota. Komento vaihtaa värin pysyvästi nykyisen
ympäristön (luku \ref{luku/ympäristöt}) tai aaltosulkeilla (luku
\ref{luku/aaltosulkeet}) rajatun alueen sisällä.

\komentoi{color}
\begin{koodilohkosis}
\color{nimi}                  % väriksi ennalta määritelty ”nimi”
\color[värimalli]{parametrit} % värimallin ja parametrien avulla
\end{koodilohkosis}

\noindent
Toinen vaihtoehto on käyttää komentoa \komento{textcolor}, jolle
annetaan edelliseen verrattuna vielä yksi argumentti lisää. Argumenttina
on teksti, johon värin halutaan vaikuttavan. Vaikutusalueena on siis
vain kyseinen teksti.

\komentoi{textcolor}
\begin{koodilohkosis}
\textcolor{nimi}{teksti}
\textcolor[värimalli]{parametrit}{teksti}
\end{koodilohkosis}

\noindent
Tekstin taustaväri vaihdetaan komennoilla \komento{colorbox} ja
\komento{fcolorbox}. Komennot luovat pienen laatikon (luku
\ref{luku/laatikot-pienet}), jonka sisällä taustaväri on voimassa. Ensin
mainittu komento vaihtaa vain taustavärin, ja jälkimmäinen sisältää
lisäksi värillisen kehyksen, eli sille määritellään kaksi väriä.

\begin{koodilohkosis}
\colorbox{nimi}{teksti} % taustaväriksi ennalta määritelty ”nimi”
\colorbox[värimalli]{parametrit}{teksti}
\fcolorbox{nimi/kehys}{nimi/tausta}{teksti}
\fcolorbox[värimalli]{parametrit/kehys}{parametrit/tausta}{teksti}
\end{koodilohkosis}

\noindent
Kehystetyn laatikon kehysviivan leveys on mitassa \mitta{fboxrule} ja
kehyksen etäisyys sisällöstä mitassa \mitta{fboxsep}. Seuraavassa on
esimerkki useiden edellä mainittujen komentojen käytöstä. Tästä tuskin
kannattaa ottaa mallia vakavaan typografiseen työhön.

\komentoi{colorbox}
\komentoi{color}
\komentoi{definecolor}
\komentoi{fcolorbox}
\komentoi{setlength}
\komentoi{textcolor}
\mittai{fboxrule}
\mittai{fboxsep}
\begin{koodilohkosis}
\definecolor{pun}{rgb}{1,.2,.2}
\setlength{\fboxrule}{2bp} \setlength{\fboxsep}{1bp}
\fcolorbox[gray]{.4}{.2}{\color[gray]{1}Tekstiä}
\textcolor{pun}{voi korostaa} \colorbox{pun}{väreillä}.
\end{koodilohkosis}

\begin{tulossis}
  \definecolor{pun}{rgb}{1,.2,.2}
  \setlength{\fboxrule}{2bp} \setlength{\fboxsep}{1bp}
  \fcolorbox[gray]{.4}{.2}{\color[gray]{1}Tekstiä}
  \textcolor{pun}{voi korostaa} \colorbox{pun}{väreillä}.
\end{tulossis}

% Tekijä:   Teemu Likonen <tlikonen@iki.fi>
% Lisenssi: Creative Commons Nimeä-JaaSamoin 4.0 Kansainvälinen (CC BY-SA 4.0)
% https://creativecommons.org/licenses/by-sa/4.0/legalcode.fi

\chapter{Asetukset}

Dokumentin yleiset asetukset koostuvat Latexissa dokumenttiluokan
valinnasta, sivun koon ja marginaalien määrittelystä, fonttien
määrittelystä sekä kieliasetuksista. Näitä kaikkia käsitellään tässä
luvussa.

Kirjoittajan ei tarvitse tehdä kaikkia dokumentin asetuksia kerralla. On
ehkä jopa suotavaakin keskittyä aluksi lähinnä dokumentin sisällön ja
rakenteen suunnitteluun ja tuottamiseen. Ulkoasuun kyllä ehtii tulla
kirjoittamisen aikana monenlaisia ajatuksia, joita ei olisi välttämättä
alussa osannut huomioida. Kirjoittamaan pääsee hyvinkin yksinkertaisen
dokumenttirungon avulla (esimerkki \ref{esim/ensimmäinen},
s.~\pageref{esim/ensimmäinen}), mutta tässä luvussa käsitellään
asetusten määrittelyä ja mahdollisuuksia melko perusteellisesti.

\section{Dokumenttiluokat}
\label{luku/dokumenttiluokat}

Latexin lähdedokumenttien alussa on aina samankaltainen rivi, joka
määrittelee käytettävän dokumenttiluokan ja mahdollisesti dokumentin
perusasetuksia. Dokumenttiluokka määritellään komennolla
\komento{documentclass} ja sen argumentiksi annetaan dokumenttiluokan
nimi. Valinnaisilla argumenteilla vaikutetaan asetuksiin.

\komentoi{documentclass}
\luokkai{article}
\begin{koodilohkosis}
\documentclass[a4paper, 12pt]{article}
\end{koodilohkosis}

\noindent
Dokumenttiluokka on eräänlainen pohjadokumentti eli ominaisuuksien ja
asetusten kokoelma, jonka varaan oma dokumentti kirjoitetaan. Eri
dokumenttiluokat sisältävät erilaisia ominaisuuksia ja erilaiset
oletusasetukset. Edellä olevassa esimerkissä käytettiin
\luokka{article}\-/ luokkaa, joka on yleiskäyttöinen luokka monenlaisten
dokumenttien kirjoittamiseen. Seuraavissa alaluvuissa käsitellään
tavallisimpia dokumenttiluokkia ja niiden asetuksia.

% Dokumenttiluokkia voi tehdä itsekin, ja sitä aihetta käsitellään
% luvussa \ref{luku/xxxx}.

\subsection{Perusdokumenttiluokat}
\label{luku/perusdokumenttiluokat}

Muutama dokumenttiluokka kuuluu Latexin perusvalikoimaan, eli ne ovat
aina saatavilla ja asennettuna. Niitä kutsutaan välillä standardeiksi
dokumenttiluokiksi, ja ne ovat ikään kuin muuttumattomia perusluokkia,
joiden varaan on turvallista rakentaa omia dokumentteja tai muita
dokumenttiluokkia. Muita dokumenttiluokkia kehitetään Latexin perusosien
ulkopuolella, ja niiden ominaisuudet voivat muuttua ja kehittyä
nopeammin ajan myötä.

Normaalit paperisivuihin perustuvat dokumentit tehdään luokkien
\luokka{article}, \luokka{report} tai \luokka{book} avulla. Ne ovat
keskenään hyvin samanlaisia, mutta niiden oletusasetukset poikkeavat
toisistaan. Karkeasti jaoteltuna \luokka{article}\-/ luokka on
tarkoitettu lyhyehköille artikkeleille ja yleiseksi perustaksi
monenlaisille dokumenteille. Sen sijaan \luokka{report} ja \luokka{book}
on tarkoitettu laajoihin dokumentteihin, ja niissä on kirjatypografian
piirteitä.

Suurin edellä mainittujen perusluokkien ero on otsikoinnissa ja
dokumentin jäsentämisessä: \luokka{report} ja \luokka{book} sisältävät
suuret pääluvut (\komento{chapter}) eli otsikot, jotka alkavat uudelta
sivulta; \luokka{article} ei sisällä samanlaisia päälukuja vaan
ainoastaan tavalliset leipätekstin mukana kulkevat väliotsikot
(\komento{section}, \komento{subsection} jne.). Tekstin jäsentämistä
käsitellään luvussa \ref{luku/jäsennys} ja otsikointia tarkemmin luvussa
\ref{luku/otsikot}.

Toinen ero perusluokkien välillä on se, että vain \luokka{book}\-/
luokka sisältää komennot \komento{frontmatter}, \komento{mainmatter} ja
\komento{backmatter}, joita voi käyttää ilmaisemaan tietokirjan
erityyppiset osat: esittely\-/, sisältö\-/{} ja liitesivut. Näitä
käsitellään luvussa \ref{luku/frontmainbackmatter}. Lisäksi
\luokka{report}\-/ luokka sisältää ympäristön \ymparisto{abstract}, joka
on tarkoitettu tutkimusraportin tai vastaavan dokumentin tiivistelmäosan
tekemiseen; \luokka{book}\-/ luokassa sitä ei ole.

Muut erot koskevat lähinnä oletusasetuksia, mutta asetukset ovat
tietenkin muutettavissa, eivätkä ne siten ole määräävä tekijä
valittaessa luokkien \luokka{article}, \luokka{report} ja \luokka{book}
välillä. Dokumenttiluokkien asetuksia käsitellään luvussa
\ref{luku/perusdokumenttiluokat-asetukset}.

Muita perusdokumenttiluokkia ovat \luokka{letter} ja \luokka{slides}.
Nimensä mukaisesti \luokka{letter} on kirjeiden kirjoittamista varten.
Se sisältää kirjeen tyypilliseen rakenteeseen liittyviä komentoja ja
soveltuu varsinkin tarpeisiin, joissa täytyy automaattisesti tuottaa
samanmuotoisia kirjeitä usealle vastaanottajalle. Sama
Latex\-/dokumentti voi sisältää useita kirjeitä, ja sivunumerointi ja
alaviitteiden numerointi alkaa joka kirjeessä alusta -- kuten on
tietysti toivottavaa. \luokka{letter}\-/ luokkaa käsitellään tarkemmin
luvussa \ref{luku/kirjeet}.

Piirtoheittimien läpinäkyvien kalvojen ja sen kaltaisten dokumenttien
tekemiseen on dokumenttiluokka \luokka{slides}. Piirtoheittimet ovat
jääneet menneisyyteen, ja nykyään jokseenkin vastaavanlainen rooli on
esitysgrafiikkaohjelmilla, joilla tehdään diaesityksiä esimerkiksi
esitelmien havaintomateriaaliksi. \luokka{slides}\-/ luokkaa ei varmaan
tarvita enää mihinkään, mutta esitysgrafiikkaan on olemassa erittäin
monipuolinen luokka \luokka{beamer}, jonka perusasioita käsitellään
luvussa \ref{luku/esitysgrafiikka}.

\subsection{Perusdokumenttiluokkien asetukset}
\label{luku/perusdokumenttiluokat-asetukset}

Taulukkoon \ref{tlk/luokkavalitsimet} on koottu perusdokumenttiluokkien
asetusten valitsimet. Sivun koko asetetaan valitsimilla
\koodi{a4\-paper} (210 × 297\,mm), \koodi{a5\-paper} (148 × 210\,mm),
\koodi{b5\-paper} (176 × 250\,mm), \koodi{execu\-tive\-paper} (7,25 ×
10,5 tuumaa), \koodi{legal\-paper} (8,5 × 14 tuumaa) tai
\koodi{letter\-paper} (8,5 × 11 tuumaa). Latexin oletuksena on
\koodi{letterpaper}, mutta oletusta voi olla muutettu Latexin
jakelupaketissa. Varminta on aina itse kirjoittaa haluttu sivukoko
dokumentin asetuksiin.

\leijutlk{
  \ttfamily
  \begin{tabular}{llll}
    \toprule
    a4paper & landscape & openright & 10pt \\
    a5paper & final & openany & 11pt \\
    b5paper & draft & onecolumn & 12pt \\
    executivepaper & oneside & twocolumn & titlepage \\
    legalpaper & twoside & fleqn & notitlepage \\
    letterpaper & openbib & leqno & clock \\
    \bottomrule
  \end{tabular}
}{
  \caption{Perusdokumenttiluokkien valitsimet}
  \label{tlk/luokkavalitsimet}
}

Sivu on oletuksena pystyasennossa, mutta valitsin \koodi{landscape}
asettaa sen vaaka\-/asentoon. Tämä asetus koskee dokumentin kaikkia
sivuja, mutta jos haluaa asettaa vain yksittäisiä sivuja
vaaka\-/asentoon, täytyy käyttää \pakettictan{pdflscape}\-/ pakettia ja
sen tarjoamaa ympäristöä.

Sivun koon ja muitakin mittoja voi määritellä monipuolisemmin
\paketti{geometry}\-/paketin toimintojen avulla (luku
\ref{luku/sivuasetukset}). Jos sitä pakettia käyttää, ei näitä
dokumenttiluokan sivukokoasetuksia tarvita välttämättä lainkaan.

Leipätekstin fontin koon voi määrittää valitsimilla \koodi{10pt},
\koodi{11pt} ja \koodi{12pt}, joista ensin mainittu on oletus. Lyhenne
pt tarkoittaa typografista pistemittaa, joita käsitellään luvussa
\ref{luku/mitat}. Nämä asetukset vaikuttavat myös sivun marginaaleihin,
koska fontin koko vaikuttaa tekstiriville mahtuvien merkkien määrään ja
Latex pyrkii pitämään rivit sopivan mittaisina lukemisen kannalta.

Vain kolme eri fonttikokoa leipätekstille on kovin vähän ja peräisin
ajalta, jolloin Latexin fontit olivat pikseligrafiikkaa eli muodostuivat
erillisistä kuvapisteistä. Nykyaikana fontit ovat vektorigrafiikkaa eli
matemaattisten kaavojen avulla muodostettavia kuvia, ja ne voi venyttää
mihin kokoon tahansa. Sen vuoksi nämä dokumenttiluokkien
fonttikokovalitsimet eivät ole kovin tarpeellisia eivätkä riitä kaikkiin
tarpeisiin nykyaikana. Voi niitä silti käyttää, jos niiden avulla saa
halutun lopputuloksen. Fontteja käsitellään tarkemmin luvussa
\ref{luku/kirjaintyypit}.

Dokumentista voi latoa luonnosversion käyttämällä dokumenttiluokan
valitsinta \koodi{draft}. Luonnoksena ladottuun dokumenttiin tulevat
merkityksi (musta suorakulmio) ainakin tekstipalstan yli pursuavat
rivit, mikä voi auttaa tekstikappaleiden ulkoasun korjailussa (luku
\ref{luku/kappale}). Eri paketit voivat hyödyntää \koodi{draft}\-/
valitsinta omalla tavallaan eli muuttaa toimintaansa sen perusteella.
Esimerkiksi grafiikkaan liittyvä paketti \paketti{graphicx} (luku
\ref{luku/grafiikka}) jättää luonnosversiosta kuvat pois, ja niiden
paikalla on vain suorakulmio. Pdf\-/tiedoston ristiviitteistä huolehtiva
\paketti{hyperref}\-/paketti jättää pdf:n sisäiset ristiviitteet ja
sisällysluettelon tekemättä (luku \ref{luku/ristiviitteet}). Latominen
voi nopeutua huomattavasti.

Lopullinen ladontatila on \koodi{final}, joka tarkoittaa, että
dokumenttiin ladotaan kaikki sisältö ja ominaisuudet niin kuin se on
tarkoitettu julkaistavaksi. Tätä valitsinta ei tarvitse käyttää, koska
se on oletustila.

Sivujen yksipuolisuus (\koodi{oneside}) ja kaksipuolisuus
(\koodi{twoside}) ovat asetuksia, jotka vaikuttavat sivun marginaalien
asetuksiin ja mahdollisesti uuden pääluvun (\komento{chapter})
aloitukseen. Oletus on yksipuolinen dokumentti kaikissa muissa
perusdokumenttiluokissa paitsi \luokka{book}\-/ luokassa, jonka oletus
on kaksipuolinen.

Yksipuolisessa dokumentissa kaikkien sivujen marginaalit ovat
lähtökohtaisesti samanlaisia ja asetuksissa puhutaan esimerkiksi
vasemmasta (\englanti{left}) ja oikeasta (\englanti{right})
marginaalista. Kaksipuolisessa dokumentissa sivut ajatellaan pareittain:
kaksi sivua muodostaa kirjan aukeaman, jonka vasemmalla puolella on
parillinen sivunumero ja oikealla pariton. Marginaalien asetuksissa
puhutaan sisämarginaaleista (\englanti{inner}) ja ulkomarginaaleista
(\englanti{outer}). Sisämarginaalit ovat oletuksena pienemmät, koska
niitä on aukeamalla aina kaksi vierekkäin. Ulkomarginaaleissa on tilaa
marginaalihuomautuksille (luku \ref{luku/marginaalihuomautukset}).
Marginaaleja ja muita sivun asetuksia käsitellään tarkemmin luvussa
\ref{luku/sivuasetukset}.

Jos dokumentti on määritetty kaksipuoliseksi (\koodi{twoside}), voidaan
\koodi{open\-right}\-/ valitsimella määrittää pääluvut
(\komento{chapter}) alkamaan aina oikeanpuoleiselta sivulta. Se onkin
oletus dokumenttiluokassa \luokka{book}. Valitsin \koodi{open\-any}
asettaa pääluvut alkamaan miltä tahansa seuraavalta sivulta.

Valitsin \koodi{open\-bib} liittyy automaattiseen lähdeluettelon
muotoiluun. Jos tämä valitsin on mukana, automaattisessa lähdeluettelon
muotoilussa lisätään rivinvaihtoja lähdemerkinnän eri osien kuten
teoksen tekijöiden ja teoksen nimen jälkeen. Lähdemerkintöjä ja
lähdeluettelon laatimista käsitellään luvussa \ref{luku/lähteet}.

Matemaattisten kaavojen ladontaan liittyvässä \ymparisto{equation}\-/
ympäristössä kaavat ladotaan normaalisti vaakasuunnassa sivun keskelle.
Dokumenttiluokan valitsinta \koodi{fleqn} käyttämällä kaavat ladotaan
kuitenkin sivun vasempaan reunaan. Kaavojen numerot sijoitetaan
oletuksena sivun oikeaan reunaan, mutta valitsimella \koodi{leqno} ne
saa ladottua sivun vasempaan reunaan. Matematiikkatilaa käsitellään
luvussa \ref{luku/matematiikka}.

Nimiösivun tai dokumentin perustietojen esitystapaan vaikutetaan
valitsimilla \koodi{title\-page} ja \koodi{no\-title\-page}. Latexissa
on yksinkertaiset komennot dokumentin perustietojen eli pääotsikon,
tekijöiden nimien ja päiväyksen latomiseen (luku
\ref{luku/dokumentin-perustiedot}), ja näillä valitsimilla vaikutetaan
siihen, ladotaanko perustiedot omalle sivulleen (\koodi{title\-page})
vaiko ensimmäisen sisältösivun alkuun (\koodi{no\-title\-page}).
Oletus\-asetus vaihtelee eri dokumenttiluokissa.

Oletuksena sivun tekstit ladotaan yhdelle palstalle
(\koodi{onecolumn}), mutta valitsimella \koodi{twocolumn} teksti
ladotaan kahdelle palstalle. Nämä valitsimet vaikuttavat myös sivun
marginaaleihin. Latexin perusosat eivät hallitse useampaa kuin kaksi
palstaa, mutta \paketti{multicol}\-/paketin avulla saa enemmänkin.
Palstoja käsitellään luvussa \ref{luku/palstat}.

Dokumenttiluokassa \luokka{slides} toimii \koodi{clock}\-/ valitsin,
joka latoo kellonajan piirtoheitinkalvon loppuun \komento{note}\-/
komennon yhteydessä. Tämä valitsin kuulunee samaan paikkaan kuin
piirtoheittimet ja \luokka{slides}\-/luokkakin: kierrätykseen tai
museoon.

\subsection{Kehittyneempi dokumentti: memoir}

Luokka \luokkactan{memoir} on Latexin \luokka{book}\-/ luokkaan
perustuva, ominaisuuksiltaan laajennettu luokka, joka sopii varsinkin
laajoihin dokumentteihin. Tähän luokkaan on sisällytetty monia
dokumentin rakenteeseen ja ulkoasuun liittyviä toimintoja, joihin
muutoin tarvitaan erillisiä paketteja tai lisäkoodia.

Saattaa siis olla helpompaa käyttää lähes pelkästään \luokka{memoir}\-/
luokkaa ja opiskella sen ohjekirjaa, kuin että ratkoisi dokumentin
rakenteen ja asettelun kysymyksiä erillisten pakettien avulla. Jos
päädyt käyttämään \luokka{memoir}\-/ luokkaa, kannattaa aina ensin
tutkia sen ohjekirjasta, saako tarvittavat asiat toteutettua luokan
omien ominaisuuksien avulla. Vasta sen jälkeen voi hakea neuvoa tästä
tai muista oppaista.

\section{Sivu}
\label{luku/sivuasetukset}

Latex\-/dokumentit muodostuvat aina peräkkäisistä sivuista, eli
taustalla kummittelee paperiarkkeihin perustuva julkaisumuoto, vaikka
dokumenttia ei varsinaisesti tulostettaisi tai painettaisi paperille.
Tämän vuoksi melkein aina täytyy määrittää sivun asetukset kuten
paperikoko ja marginaalit. Ne tehdään \paketti{geometry}\-/ paketin
avulla, jota käsitellään luvussa \ref{luku/sivun-mitat}. Joskus halutaan
muokata sivun ylä- tai alatunnistetta eli verrattain pysyviä
tunnistetietoja, jotka toistuvat joka sivulla ylä- tai alareunassa.
Niiden muokkaamiseen käytetään \paketti{fancyhdr}\-/ pakettia, jonka
ohjeita on luvussa \ref{luku/ylä-ala-tunnisteet}. Sivun taustavärin
asettamista käsitellään luvussa \ref{luku/korostus-värit}.

\subsection{Sivun koko ja marginaalit}
\label{luku/sivun-mitat}

Paperin eli sivun kokoon ja marginaaleihin pääsee vaikuttamaan
\pakettictan{geometry}\-/ paketin avulla. Halutut asetukset voi kertoa
paketin lataamisen yhteydessä. Seuraavassa esimerkissä asetetaan
paperikoko (\koodi{a4\-paper} eli \textsc{a4}-koko), ylä- ja
alamarginaalin pituus (\koodi{top}, \koodi{bottom}) sekä vasemman ja
oikean marginaalin pituus (\koodi{left}, \koodi{right}).

\komentoi{usepackage}
\begin{koodilohkosis}
\usepackage[a4paper, top=20mm, bottom=30mm,
  left=20mm, right=20mm]{geometry}
\end{koodilohkosis}

\noindent
Vaihtoehtoisesti asetukset voi ilmaista erikseen \komento{geometry}\-/
komennon avulla seuraavalla tavalla:

\komentoi{usepackage}
\komentoi{geometry}
\begin{koodilohkosis}
\usepackage{geometry}
\geometry{a4paper, top=20mm, bottom=30mm, left=20mm, right=20mm}
\end{koodilohkosis}

\noindent
Mikäli myöhemmin dokumentissa täytyy vaihtaa sivun asettelua, käytetään
komentoa \komento{newgeometry}. Alkuperäiset asetukset palautetaan
komennolla \komento{restoregeometry}.

\komentoi{newgeometry}
\komentoi{restoregeometry}
\begin{koodilohkosis}
\newgeometry{top=…, bottom=…, …} % Uudet asetukset.
\restoregeometry     % Palautetaan alkuperäiset asetukset.
\end{koodilohkosis}

\noindent
Valmiiksi määriteltyjä standardipaperikokoja on useita.
\textsc{iso}\-/standardin mukaiset koot \textsc{a0}--\textsc{a6}
valitaan valitsimilla kuten \koodi{a3paper}, \koodi{a4paper} tai
\koodi{a5paper}. Samoin käytetään myös kokoja \textsc{b0}--\textsc{b6}
ja \textsc{c0}--\textsc{c6}, esimerkiksi \koodi{b2paper} tai
\koodi{c6paper}. Lisäksi on valittavissa useita japanilaisia ja
yhdysvaltalaisia standardeja käyttämällä valitsimia kuten
\koodi{b0j}--\koodi{b6j}, \koodi{ansi\-a\-paper} tai
\koodi{letter\-paper}.

Mikäli standardit koot eivät riitä, voi sivun mitat määrittää vapaasti
\koodi{paper\-size}\-/ valitsimella, jolle annetaan arvoksi leveys- ja
korkeusmitta.

\komentoi{geometry}
\begin{koodilohkosis}
\geometry{papersize={10cm, 16cm}}
\end{koodilohkosis}

\noindent
Sivu on oletuksena pystyasennossa (\koodi{portrait}), mutta sen voi
asettaa vaaka\-/asentoon valitsimella \koodi{landscape}. Tämä asetus on
voimassa koko dokumentin ajan. Jos sen sijaan haluaa asettaa vain
yksittäisiä sivuja vaaka\-/asentoon, täytyy käyttää
\pakettictan{pdflscape}\-/ pakettia ja sen tarjoamaa
\ymparisto{landscape}\-/ ympäristöä. Ympäristön sisältö ladotaan
vaakasuuntaisille sivuille.

\ymparistoi{landscape}
\begin{koodilohkosis}
\begin{landscape}
  ...
\end{landscape}
\end{koodilohkosis}

\noindent
Sivun asetusten suunnittelussa voi olla avuksi valitsin
\koodi{show\-frame}, joka piirtää viivat marginaalien kohdalle ja
merkitsee myös ylä- ja alatunnisteiden sekä marginaalihuomautusten
rajoja.

Taulukkoon \ref{tlk/sivun-marginaalit} on koottu tärkeimpiä valitsimia
marginaalien mittojen asettamiseksi. Valitsimille annetaan arvoksi Texin
mittayksikkö, jotka ovat taulukossa \ref{tlk/mittayksiköt}
(s.~\pageref{tlk/mittayksiköt}). Kaksipuolisessa asettelussa
(\koodi{twoside=\katk true}) sivut muodostavat aukeaman eli on erikseen
vasemmanpuoleinen ja oikeanpuoleinen sivu. Tällöin vasen ja oikea
marginaali vuorottelevat, ja niitä on havainnollisempaa kutsua sisä- ja
ulkomarginaaliksi. Marginaalien määrittelyssä voi käyttää valitsimen
\koodi{left} sijasta valitsinta \koodi{inner}, ja vastaavasti
\koodi{right}\-/ valitsin voidaan korvata \koodi{outer}\-/ valitsimella.
Teknisesti näillä ei ole mitään eroa.

\leijutlk{
  \providecommand{\rivi}{}
  \renewcommand{\rivi}[3]{\koodi{#1} & \koodi{#2} & #3 \\}
  \begin{tabular}{lll}
    \toprule
    \multicolumn{2}{l}{\ots{Valitsin}} & \ots{Merkitys} \\
    \midrule
    \rivi{left}{inner}{vasen marginaali tai sisämarginaali}
    \rivi{right}{outer}{oikea marginaali tai ulkomarginaali}
    \rivi{top}{}{ylämarginaali}
    \rivi{bottom}{}{alamarginaali}
    \bottomrule
  \end{tabular}
}{
  \caption{\paketti{geometry}\-/paketin valitsimia sivun marginaalien
    määrittelemiseen}
  \label{tlk/sivun-marginaalit}
}

Sivun tekstialueen koon ja marginaalit voi määrittää myös
suhteellisesti. Voisi esimerkiksi määrittää, että leveyssuunnassa
tekstialue täyttää 0,7\=/kertaisesti (70\,\%) sivun leveyden ja loput
jää marginaaleille. Marginaalien keskinäiset suhteetkin voi ilmaista
suhdelukuna: esimerkiksi vasemman (sisä) ja oikean (ulko) marginaalin
suhde voisi olla 2:3. Suhteellisessa tavassa ei tarvitse ottaa kantaa
sivun kokoon eikä muihinkaan varsinaisiin mittoihin, vaan samat
tekstialueen ja marginaalien suhteet säilyvät, vaikka sivukokoa
muuttaisikin. Suhteellisia mittoja koskevia valitsimia on koottu
taulukkoon \ref{tlk/sivun-marginaalit-suhd}.

\leijutlk{
  \providecommand{\rivi}{}
  \renewcommand{\rivi}[2]{\koodi{#1} & #2 \\}
  \begin{tabular}{ll}
    \toprule
    \ots{Valitsin} & \ots{Merkitys} \\
    \midrule
    \rivi{hscale}{tekstialueen osuus sivun leveydestä}
    \rivi{vscale}{tekstialueen osuus sivun korkeudesta}
    \rivi{hmarginratio}{vasemman (sisä) ja oikean (ulko) marginaalin suhde}
    \rivi{vmarginratio}{ylä- ja alamarginaalin suhde}
    \bottomrule
  \end{tabular}
}{
  \caption{\paketti{geometry}\-/paketin valitsimia sivun tekstitilan ja
    marginaalien suhteiden määrittämiseen. Osuudet (\koodi{*scale})
    ilmaistaan prosenttikertoimella (esim. \koodi{0.7}). Suhteet
    (\koodi{*marginratio}) ilmaistaan suhdelukuna (esim. \koodi{2:3})}
  \label{tlk/sivun-marginaalit-suhd}
}

Klassisessa kirjatypografiassa, jossa teksti ladotaan yhdelle palstalle,
marginaalien suuruusjärjestys on suurimmasta pienimpään seuraavanlainen:
ala-, ulko-, ylä- ja sisämarginaali. Marginaalien suhdeluvut ovat
samassa järjestyksessä: 32, 28, 20, 17. \paketti{geometry}\-/paketin
valitsimilla tämä ilmaistaan seuraavasti:

\komentoi{geometry}
\begin{koodilohkosis}
\geometry{hmarginratio=17:28, vmarginratio=20:32}
\end{koodilohkosis}

\noindent
Klassisia suhteita ei nykyaikana yleensä noudateta kovin tarkasti, mutta
niistä kannattaa ymmärtää yleinen ajatus. Alamarginaalin pitäisi olla
hieman suurempi kuin ylämarginaali, koska muuten tekstialue tuntuu
pudonneen sivulla alas. Alamarginaalissa (alatunnisteessa) on yleensä
sivunumero. Kirjan sivuja katsotaan pareittain eli aukeamina, joten
sisämarginaaleja on kaksi vierekkäin. Siksi sisämarginaalit yksittäin
ajateltuna ovat pienemmät kuin ulkomarginaalit. Lisäksi ulkomarginaaleja
voidaan käyttää huomautusten kirjoittamiseen, mikä on melko yleinen
käytäntö tietokirjoissa.

Marginaalihuomautukset (luku \ref{luku/marginaalihuomautukset})
sijaitsevat oletuksena sivujen ulkomarginaalissa tai oikeanpuoleisessa
marginaalissa. Huomautuspalstan leveys asetetaan valitsimella
\koodi{marginparwidth}, ja palstan etäisyys sivun varsinaisesta
tekstialueesta määritellään valitsimella \koodi{marginparsep}. Kumpikin
valitsin tarvitsee argumentiksi mitan. Jos haluaa vaihtaa huomautukset
sivun vastakkaiseen marginaaliin, lisätään mukaan valitsin
\koodi{reversemarginpar}.

Perus Latex osaa latoa tekstin yhdelle tai kahdelle palstalle, ja
\paketti{geometry}\-/ paketin valitsimella \koodi{onecolumn} tai
\koodi{twocolumn} asetetaan, kumpi tila on oletuksena päällä.
Useampikin palsta on mahdollista laajennuspaketin avulla. Valitsimen
\koodi{columnsep} avulla asetetaan palstojen välinen etäisyys.
Käytännössä tämä valitsin on asettaa mitan \mitta{columnsep}, jota voi
muokata myös komennolla \komento{setlength}, kuten muitakin mittoja
(luku \ref{luku/mitat}). Tarkempaa tietoa palstoista on luvussa
\ref{luku/palstat}.

\subsection{Sivun mittoja}

\leijutlk{
  \providecommand{\rivi}{}
  \renewcommand{\rivi}[2]{\mitta{#1} & #2 \\}
  \begin{tabular}{ll}
    \toprule
    \ots{Mitta} & \ots {Merkitys} \\
    \midrule
    \rivi{paperwidth}{paperin eli sivun leveys}
    \rivi{paperheight}{paperin eli sivun korkeus}
    \rivi{textwidth}{tekstialueen leveys}
    \rivi{columnwidth}{nykyisen palstan leveys}
    \rivi{linewidth}{nykyisen rivin leveys}
    \rivi{textheight}{tekstialueen korkeus}
    \bottomrule
  \end{tabular}
}{
  \caption{Sivun mittoja}
  \label{tlk/sivun-mittoja}
}

Sivun kokoasetusten määrittämisen jälkeenkin voi dokumentissa olla
tarpeen hyödyntää joitakin sivun mittoja. Usein esimerkiksi halutaan
piirtää taulukko tai kuva, joka on sivun tekstialueen levyinen tai
siihen suhteutettu. Silloin on kätevää käyttää mittaa, jonka arvona on
juuri tekstialueen leveys. Tärkeimmät sivun mitat on koottu taulukkoon
\ref{tlk/sivun-mittoja}, mutta perusteellisemmin niitä käsitellään
\pakettictan{geometry}\-/ paketin ohjekirjassa.

Taulukossa \ref{tlk/sivun-mittoja} mainittu mitta \mitta{linewidth}
eroaa tekstialueen (\mitta{textwidth}) tai palstan (\mitta{columnwidth})
leveysmitasta esimerkiksi silloin, kun tekstikappaletta on sisennetty.
Kappaleiden ensimmäinen rivi voi olla sisennetty, ja sen vuoksi rivi ei
ole täysilevyinen. Muutkin sisennykset kuten lohkolainaukset tai
luetelmat vaikuttavat rivin leveysmittaan. Sisennysasiat liittyvät
tekstikappaleiden muotoiluun, jota käsitellään
luvussa~\ref{luku/kappale}. Luetelmia puolestaan käsitellään
luvussa~\ref{luku/luetelmat}.

\subsection{Leikkuuvarat}

Sivu voi olla ulkoisesti erikokoinen kuin sisäisesti. On siis
mahdollista asettaa sivu esimerkiksi \textsc{a4}\-/kokoiseksi ja
käsitellä marginaalit ja muut sivun mitat \textsc{a4}-koon mukaan, mutta
ulkoisesti tai fyysisesti sivu onkin osana suurempaa sivua tai
paperiarkkia. Tällaista tarvitaan ainakin silloin, kun halutaan
merkitä leikkuuvarat dokumentin painamista varten.

Painokoneen paperiarkkien leikkauskohta ei välttämättä osu täsmälleen
samaan kohtaan pdf\-/tiedoston sivun reunan kanssa, ja siksi
dokumentissa reunaan saakka yltävät kuvat asetetaan varmuuden vuoksi
hieman ylikokoiseksi. Sivun reunaan saakka aiotut kuvat siis yltävät
lähde-pdf:ssä pari millimetriä varsinaisen sivualueen ulkopuolelle eli
leikkuuvaran puolelle. Tällä varmistetaan, että painamisen jälkeen
leikatuissa paperiarkeissa kuva varmasti yltää reunaan saakka.

Jos dokumenttiin tarvitaan sivun ulkopuoliset leikkuuvarat, määritellään
dokumentin ulommaiset mitat edelleen samalla tavalla kuin
tavallisestikin eli esimerkiksi valitsimella \koodi{paper\-size} (luku
\ref{luku/sivun-mitat}). Sen sijaan sivun sisäiset mitat täytyy
määritellä toisella tavalla, käyttämällä valitsinta \koodi{lay\-out} tai
\koodi{layout\-size}.

\begin{esimerkki*}
  \komentoi{geometry}
\begin{koodilohko}
\geometry{
  papersize={220mm, 307mm},
  layout=a4paper,           % tai: layoutsize={210mm, 297mm}
  layoutoffset={5mm, 5mm},
  showcrop
}
\end{koodilohko}
  \caption{Sivun ulkoisten ja sisäisten mittojen sekä leikkuvaarojen
    määrittäminen}
  \label{esim/leikkuuvarat}
\end{esimerkki*}

Esimerkissä \ref{esim/leikkuuvarat} käytetään sisäisesti \textsc{a4}\-/
kokoista (210 × 297\,mm) sivua, mutta sivulle on määritetty joka
puolelle 5\,mm:n leikkuuvarat. Niinpä ulkoisesti sivu on 10\,mm leveämpi
ja korkeampi, eli ulkoiset mitat ovat 220 × 307\,mm. Valitsimella
\koodi{layout\-offset} asetetaan sisäisen sivun etäisyys ulkoisen sivun
vasemmasta ylänurkasta. Esimerkissä on mukana myös valitsin
\koodi{show\-crop}, joka merkitsee sisäisen ja ulkoisen sivun rajakohdan
eli leikkuuvaran rajan. Merkinnät näkyvät vain sivun nurkissa
leikkuuvaran puolella, joten ne eivät päädy lopulliseen
painotuotteeseen.

\subsection{Ylä- ja alatunnisteet}
\label{luku/ylä-ala-tunnisteet}

\paketti{geometry}\-/paketin asetuksiin kuuluu pari valitsinta, joilla
vaikutetaan ylä- ja alatunnisteiden mittoihin. Valitsimella \koodi{head}
ilmaistaan ylätunnisteen korkeus ja valitsimella \koodi{headsep} sen
etäisyys sivun tekstipalstasta. Alatunnisteen peruslinjan etäisyys
tekstipalstasta säädetään valitsimella \koodi{footskip}. Taulukkoon
\ref{tlk/ylä-ala-tunnistemitat} on koottu näiden valitsimien merkitys,
ja seuraavassa on niiden käyttämisestä esimerkki. Mukana on myös
valitsin \koodi{show\-frame}, joka piirtää sivulle apuviivoja. Se auttaa
sivun mittojen suunnittelussa.

\komentoi{geometry}
\begin{koodilohkosis}
\geometry{head=24bp, headsep=8bp, footskip=12mm, showframe}
\end{koodilohkosis}

\leijutlk{
  \begin{tabular}{ll}
    \toprule
    \ots{Valitsin} & \ots{Merkitys} \\
    \midrule
    \koodi{head} & ylätunnisteen korkeusmitta \\
    \koodi{headsep} & ylätunnisteen etäisyys tekstipalstasta \\
    \koodi{footskip} & alatunnisteen peruslinjan etäisyys tekstipalstasta \\
    \bottomrule
  \end{tabular}
}{
  \caption{\paketti{geometry}\-/paketin valitsimet ylä- ja
    alatunnisteiden mittojen asettamiseen}
  \label{tlk/ylä-ala-tunnistemitat}
}

\noindent
Latexin perusosat eivät sisällä kovin kummoista keinovalikoimaa ylä- ja
alatunnisteiden muokkaamiseen, mutta pari hyödyllistä sivutyyliä on
kuitenkin mukana. Ylä- ja alatunnisteet määräytyvät sivutyylin
perusteella, ja haluttu tyyli asetetaan voimaan komennolla
\komento{pagestyle}:

\komentoi{pagestyle}
\begin{koodilohkosis}
\pagestyle{plain}
\end{koodilohkosis}

\noindent
Edellä mainittu sivutyyli \koodi{plain} latoo alatunnisteeseen
sivunumeron. Se on yleensä oletustyyli. Sivunumero on peräisin
laskurista \laskuri{page} ja sen arvon tulostavasta komennosta
\komento{thepage} (luku \ref{luku/laskurit}). Toinen hyödyllinen tyyli
on \koodi{empty}, joka tarkoittaa tyhjää, eli ylä- eikä
alatunnisteeseen ei ladota mitään.

Yksittäiselle sivulle voi asettaa muusta dokumentista poikkeavan
sivutyylin komennolla \komento{thispagestyle}. Komento siis vaikuttaa
vain sillä hetkellä ladottavan sivun tyyliin, ja sen jälkeen palataan
taas voimassa olevaan tyyliin, joka on aiemmin määritelty komennolla
\komento{pagestyle}.

\komentoi{thispagestyle}
\begin{koodilohkosis}
\thispagestyle{empty}
\end{koodilohkosis}

\noindent
Sivutyyli \koodi{headings} latoo ylätunnisteeseen aukeaman
vasemmanpuoleisille sivuille esimerkiksi kirjan pääluvun nimen ja
oikeanpuoleisille sivuille meneillään olevan alaluvun nimen. Mainitut
lukujen nimet tulevat sivun sisäreunaan; ulkoreunaan ladotaan
sivunumero.

Teknisesti ja sisäisesti tämä on toteutettu siten, että pääluvun
aloittava otsikkokomento (esim. \komento{chapter}) automaattisesti
määrittelee joka kerta uudelleen komennon \komento{leftmark}, niin että
se sisältää pääluvun nimen. Sivuja ladottaessa Latex sitten latoo
ylätunnisteeseen sen, mitä \komento{leftmark}\-/komento sattuu
tulostamaan. Vastaavasti alaluvun otsikkokomento (esim.
\komento{section}) määrittelee uudelleen komennon \komento{rightmark},
niin että se sisältää alaluvun nimen. Tämän komennon tulostama teksti
ladotaan aukeaman oikeanpuoleisille sivuille.

Jos haluaa itse vaikuttaa ylätunnisteen tekstiin, voi käyttää sivutyyliä
\koodi{myheadings} sekä komentoa \komento{markboth}, jolla määritellään
aukeaman vasemmanpuoleisen ja oikeanpuoleisen sivun ylätunnisteen
teksti. Tämä komento on tarkoitettu suoritettavaksi päälukujen
yhteydessä. Komennolla \komento{markright} määritellään pelkästään
oikeanpuoleisen sivun teksti, ja komento on tarkoitettu suoritettavaksi
aina alalukujen yhteydessä.

\komentoi{markboth}
\komentoi{markright}
\begin{koodilohkosis}
\markboth{vasen}{oikea}  % määrittelee: \leftmark ja \rightmark
\markright{oikea}        % määrittelee: \rightmark
\end{koodilohkosis}

\noindent
Sivunumeroinnin tyyliin voi vaikuttaa esimerkiksi komennolla
\komento{pagenumbering}, jonka argumentiksi annetaan numerointityylin
nimi. Ne on koottu taulukkoon \ref{tlk/sivu-numerointityylit}.

\komentoi{pagenumbering}
\begin{koodilohkosis}
\pagenumbering{roman}
\end{koodilohkosis}

\noindent
Edellä mainittu komento määrittelee käytännössä uudelleen komennon
\komento{thepage}, joka on tarkoitettu juuri sivunumerolaskurin
latomiseen. Lisätietoa sivunumeroista ja muista laskureista on luvussa
\ref{luku/laskurit}.

\leijutlk{
  \providecommand{\rivi}{}
  \renewcommand{\rivi}[2]{\koodi{#1} & #2 \\}
  \begin{tabular}{ll}
    \toprule
    \ots{Tyyli} & \ots{Merkitys} \\
    \midrule
    \rivi{arabic}{arabialaiset luvut: 1, 2, 3\dots}
    \rivi{roman}{roomalaiset luvut: i, ii, iii\dots}
    \rivi{Roman}{roomalaiset luvut: I, II, III\dots}
    \rivi{alph}{kirjaimet: a, b, c\dots\ (vain 1--26)}
    \rivi{Alph}{kirjaimet: A, B, C\dots\ (vain 1--26)}
    \bottomrule
  \end{tabular}
}{
  \caption{Sivunumerointityylit \komento{pagenumbering}\-/ komennon
    argumentiksi}
  \label{tlk/sivu-numerointityylit}
}

Monipuolisemmin ylä- ja alatunnisteita voi muokata paketin
\pakettictan{fancyhdr} toimintojen avulla. Silloin sivutyylinä voi olla
myös \koodi{fancy}:

\komentoi{usepackage}
\komentoi{pagestyle}
\begin{koodilohkosis}
\usepackage{fancyhdr}
\pagestyle{fancy}
\end{koodilohkosis}

\noindent
Kun käytössä on sivutyyli \koodi{fancy}, voi ylä- ja alatunnisteiden
sisällön asettaa vapaasti komennolla \komento{fancyhf}. Komennon
argumenttien merkitys on seuraavanlainen:

\komentoi{fancyhf}
\begin{koodilohkosis}
\fancyhf[paikka]{sisältö}
\end{koodilohkosis}

\noindent
Valinnainen argumentti \koodi{paikka} kertoo, mihin paikkaan tai
paikkoihin \koodi{sisältö} sijoitetaan. Vaihtoehtoina on ylä- tai
alatunniste, pariton tai parillinen sivu, sivun vasen reuna, keskiosa
tai oikea reuna. Nämä vaihtoehdot ja niitä vastaavat valitsimet on
koottu taulukkoon \ref{tlk/fancyhf-paikat}. Saman komennon avulla voi
määrittää useitakin paikkoja, kun ne erottaa pilkulla, esimerkiksi
seuraavalla tavalla:

\leijutlk{
  \providecommand{\rivi}{}
  \renewcommand{\rivi}[2]{\koodi{#1} & #2 \\}
  \begin{tabular}{cl}
    \toprule
    \ots{Valitsin} & \ots{Merkitys} \\
    \midrule
    \rivi{H}{ylätunniste (header)}
    \rivi{F}{alatunniste (footer)}
    \rivi{E}{parillinen sivu, vasen (even)}
    \rivi{O}{pariton sivu, oikea (odd)}
    \rivi{L}{sivun vasen reuna (left)}
    \rivi{C}{sivun keskelle (center)}
    \rivi{R}{sivun oikea reuna (right)}
    \bottomrule
  \end{tabular}
}{
  \caption{\komento{fancyhf}\-/ komennon valitsimia ylä- tai
    alatunnisteen paikan määrittämiseen}
  \label{tlk/fancyhf-paikat}
}

\komentoi{fancyhf}
\begin{koodilohkosis}
\fancyhf{}  % Tyhjennetään ylä- ja alatunnisteet.
\fancyhf[HEL,HOR]{\thepage}
\end{koodilohkosis}

\noindent
Edellä mainittu komento sijoittaa sivunumeron (\komento{thepage})
ylätunnisteeseen (\koodi{H}) parillisten sivujen (\koodi{E}) vasempaan
reunaan (\koodi{L}) ja parittomien sivujen (\koodi{O}) oikeaan reunaan
(\koodi{R}). Käytännössä siis aukeaman ulkoreunoihin ladotaan
sivunumerot. Seuraava esimerkki sijoittaa ajatusviivoilla (\==)
reunustetun sivunumeron kaikille sivuille alatunnisteeseen (\koodi{F})
sivun keskelle (\koodi{C}):

\komentoi{fancyhf}
\komentoi{thepage}
\begin{koodilohkosis}
\fancyhf[FC]{-- \thepage\ --}
\end{koodilohkosis}

\noindent
Paketin \paketti{fancyhdr} avulla voi myös määritellä Latexin
sivutyylejä toisenlaiseksi tai luoda kokonaan omia sivutyylejä. Nämä
tehdään komennolla \komento{fancypagestyle}, jonka ensimmäinen
argumentti on sivutyylin nimi ja toinen argumentti on sivutyylin
määritelmä. Määritelmä sisältää tarvittavat \komento{fancyhf}\-/
komennot, joilla ylä- ja alatunnisteet määritellään.

Seuraava esimerkki tekee Latexin \koodi{plain}\-/ sivutyylistä saman
kuin \koodi{fancy}\-/tyylin, eli sekin noudattaa \komento{fancyhf}\-/
komennolla määriteltyjä ylä- ja alatunnisteita:

\komentoi{fancypagestyle}
\begin{koodilohkosis}
\fancypagestyle{plain}{}
\end{koodilohkosis}

\noindent
Seuraava esimerkki määrittelee kokonaan oman sivutyylin:

\komentoi{fancypagestyle}
\komentoi{fancyhf}
\begin{koodilohkosis}
\fancypagestyle{omatyyli}{
  \fancyhf{}
  \fancyhf[FEL,FOR]{\thepage}
}
\end{koodilohkosis}

\noindent
Ylä- ja alatunniste voidaan erottaa tekstipalstasta vaakasuuntaisella
viivalla, jonka leveyttä on mahdollista muuttaa määrittelemällä
uudelleen komennot \komento{headrulewidth} ja \komento{footrulewidth}.
Komennon määritelmäksi kirjoitetaan Texin mitta. Seuraavassa esimerkissä
asetetaan yhden typografisen pisteen (1\,bp) levyiset viivat. Mitan
voisi asettaa myös nollaksi (0\,bp), jolloin erotinviiva katoaa kokonaan
näkyvistä.

\komentoi{renewcommand}
\komentoi{headrulewidth}
\komentoi{footrulewidth}
\begin{koodilohkosis}
\renewcommand{\headrulewidth}{1bp} % ylätunnisteen erotinviiva
\renewcommand{\footrulewidth}{1bp} % alatunnisteen erotinviiva
\end{koodilohkosis}

\noindent
Mikäli haluaa omiin ylätunnisteisiin esimerkiksi päälukujen ja
alalukujen nimiä, täytyy tunnisteisiin sisällyttää aiemmin kuvatut
\komento{leftmark}- ja \komento{rightmark}\-/komennot. Ensin mainittu
sisältää pääluvun nimen ja jälkimmäinen alaluvun nimen.

\komentoi{fancyhf}
\komentoi{leftmark}
\komentoi{rightmark}
\begin{koodilohkosis}
\fancyhf[HEL]{\leftmark}
\fancyhf[HOR]{\rightmark}
\end{koodilohkosis}

\noindent
Latex latoo pää- ja alalukujen nimet oletuksena versaalikirjaimilla eli
isoilla kirjaimilla. Jos ne haluaa johonkin toiseen muotoon, täytyy itse
määritellä uudelleen komennot \komento{chaptermark},
\komento{sectionmark} tai \komento{subsectionmark} (vain kaksi näistä)
ja käyttää määritelmässä komentoja \komento{markboth} ja
\komento{markright}. Esimerkkiin \ref{esim/fancyhdr-koko} on koottu
varsin kokonaisvaltainen koodi omien ylä- ja alatunnisteiden
toteutukseen.

\begin{esimerkki*}
  \komentoi{documentclass}
  \luokkai{book}
  \komentoi{usepackage}
  \pakettii{fancyhdr}
  \komentoi{fancypagestyle}
  \komentoi{fancyhf}
  \komentoi{renewcommand}
  \komentoi{headrulewidth}
  \komentoi{footrulewidth}
  \komentoi{leftmark}
  \komentoi{rightmark}
  \komentoi{thepage}
  \komentoi{pagestyle}
  \komentoi{chaptermark}
  \komentoi{markboth}
  \komentoi{chaptername}
  \komentoi{thechapter}
  \komentoi{sectionmark}
  \komentoi{thesection}

\begin{koodilohko}
\documentclass{book}
\usepackage{fancyhdr}

% Päälukujen (\chapter) aloitussivu käyttää plain-sivutyyliä.
% Tässä määritellään se uudestaan.
\fancypagestyle{plain}{
  \fancyhf{}
  \fancyhf[FC]{-- \thepage\ --}
  \renewcommand{\headrulewidth}{0bp}
  \renewcommand{\footrulewidth}{0bp}
}

% fancy-sivutyylin asetukset:
\fancyhf{}
\fancyhf[HEL]{\leftmark}
\fancyhf[HOR]{\rightmark}
\fancyhf[FC]{-- \thepage\ --}
\renewcommand{\headrulewidth}{1bp}
\renewcommand{\footrulewidth}{0bp}

\begin{document}

\pagestyle{fancy}

% Päälukujen (\chapter) yhteydessä komento \markboth{…}{…} määrittää
% sekä \leftmark- että \rightmark-komennot. Tässä jälkimmäinen
% määritellään tyhjäksi.
\renewcommand{\chaptermark}[1]{%
  \markboth{\chaptername\ \thechapter: #1}{}}

% Alalukujen (\section) yhteydessä komento \markright{…} määrittää
% vain \rightmark-komennon.
\renewcommand{\sectionmark}[1]{\markright{\thesection\ #1}}
\end{koodilohko}
  \caption{Omien ylä- ja alatunnisteiden toteuttaminen}
  \label{esim/fancyhdr-koko}
\end{esimerkki*}

\section{Pdf-tiedosto}
\label{luku/pdf-asetukset}

Pdf-tiedostot voivat sisältää metatietoja kuten dokumentin nimen,
aiheen, tekijän ja päiväyksen. Tiedostoille voi määrittää myös erilaisia
asetuksia kuten ristiviitteiden ja linkkien ulkoasun tai sisäisen
sisällysluettelon ominaisuuksia. Pdf\-/tiedoston asetukset toteutetaan
\pakettictan{hyperref}\-/ paketin avulla. Paketti neuvotaan lataamaan
muiden pakettien jälkeen, koska se lisää ominaisuuksia muihin
komentoihin. Paketin voisi ladata esimerkiksi seuraavalla tavalla:

\komentoi{usepackage}
\pakettii{hyperref}
\begin{koodilohkosis}
% Muiden \usepackage-komentojen jälkeen.
\usepackage[unicode]{hyperref}
\end{koodilohkosis}

\noindent
Esimerkissä käytetty valitsin \koodi{unicode} aiheuttaa sen, että pdf\-/
tiedoston sisäisissä merkkijonoissa käytetään Unicode\-/ merkistöä ja
sen \textsc{utf\=/8}\-/ koodausta. Ilman tätä valitsinta pdf\-/
tiedoston sisäisen sisällysluettelon merkistö ei välttämättä näy oikein.

\paketti{hyperref}\-/ paketin asetuksia voi määritellä valitsimien
avulla \komento{usepackage}\-/ komennon yhteydessä mutta myös
erillisellä \komento{hypersetup}\-/ komennolla. Komennolle annetaan
yksi argumentti, joka sisältää pilkuilla erotettuna erilaisia valitsimia
ja niiden arvoja.

\begin{esimerkki*}
  \komentoi{hypersetup}

\begin{koodilohko}
\hypersetup{
  hidelinks, bookmarksopen, bookmarksnumbered,
  pdfinfo={
    Title={Laatikollinen lateksia},
    Subject={Opas lateksiin ja liisteriin},
    Author={Lauri Liisteri}
  }
}
\end{koodilohko}
  \caption{\komento{hypersetup}\-/ komennolla asetetaan
    \paketti{hyperref}\-/ paketin asetuksia, esimerkiksi pdf:n
    metatietoja}
  \label{esim/hypersetup}
\end{esimerkki*}

Esimerkki \ref{esim/hypersetup} havainnollistaa \komento{hypersetup}\-/
komennon käyttöä. Komennon argumentissa valitsin \koodi{hidelinks} saa
aikaan sen, että pdf:ssä olevia linkkejä ja ristiviitteitä ei merkitä
millään tavalla. Oletuksena linkit kehystetään eri väreillä riippuen
linkin tyypistä. Valitsin \koodi{bookmarksopen} näyttää pdf:n
sisällysluettelon kokonaan avattuna. Oletuksena alaluvut on
piilotettuna, ja ne joutuu avaamaan napsauttamalla hiirellä
ylemmäntasoisen otsikon avauspainiketta. Valitsin
\koodi{bookmarksnumbered} näyttää pdf:n sisällysluettelossa lukujen
numeroinnin (kuten 1.1, 1.2, 2.1 tms.). Valitsimella \koodi{pdfinfo}
määritellään pdf:n metatietoja kuten otsikko, aihe ja tekijä mutta
omiakin metatietoja voi lisätä. Paljon muitakin asetuksia on olemassa,
ja niistä voi lukea lisää \paketti{hyperref}\-/ paketin ohjekirjasta.

Pdf\-/ tiedoston sisäiseen sisällysluetteloon tulevat automaattisesti
mukaan samat otsikot (luku \ref{luku/otsikot}) kuin ladottavaan
sisällysluetteloonkin (luku \ref{luku/sisällysluettelo}). Pdf\-/
tiedoston luetteloon voi kuitenkin itse lisäillä omia otsikoita, jotka
eivät näy missään muualla. Se tehdään \paketti{hyperref}\-/ paketin
komennolla \komento{pdfbookmark}:

\komentoi{pdfbookmark}
\begin{koodilohkosis}
\pdfbookmark[taso]{Teksti}{tunniste}
\end{koodilohkosis}

\noindent
Komennon valinnainen argumentti \koodi{taso} on kokonaisluku, joka
tarkoittaa otsikon tasoa. Se on samankaltainen tasonumero kuin
otsikkotasojen taulukossa \ref{tlk/otsikkotasot}
(s.~\pageref{tlk/otsikkotasot}). Argumentti \koodi{Teksti} on pdf:n
sisällysluettelomerkinnän teksti, ja \koodi{tunniste} on mikä tahansa
yksilöllinen tekstimuotoinen tunniste kyseiselle luettelomerkinnälle.
Tunniste ei näy missään, mutta pdf tarvitsee sisäiseen toimintaansa
jonkin yksilöllisen tunnisteen.

\section{Fontit}
\label{luku/kirjaintyypit}

Fontit ja niiden asettaminen on Latexissa melko monimutkainen
kokonaisuus, koska fonteilla on paljon ominaisuuksia ja niihin
vaikutetaan monilla eri asetuksilla ja abstraktiotasoilla. Aika monta
asiaa pitää ymmärtää, jotta voi tehokkaasti työskennellä Latexin
fonttien kanssa.

Fontti jo itsessään on moniselitteinen käsite, joka vaatii typografiassa
usein täsmentäviä ilmauksia. Sana \emph{fontti} voi tarkoittaa
kokonaista kirjainperhettä eli yhteensopivien kirjainleikkausten
muodostamaa kokonaisuutta. Samaan kirjainperheeseen kuuluu yleensä
ainakin neljä eri leikkausta: tavallinen, \textit{kursiivi},
\textbf{lihavoitu} ja \textbf{\textit{lihavoitu kursiivi}}. Joihinkin
perheisiin kuuluu leikkauksia paljon enemmänkin, kuten useita eri
vahvuuksia. Joissakin puheissa sana \emph{fontti} tarkoittaa vain yhtä
kirjainleikkausta, ja silloin koko perheeseen viitataan ehkä sanalla
fonttiperhe.

Tässä oppaassa käytetään \emph{fontti}\-/sanaa yleisnimityksenä Latexin
kirjaintyyppeihin liittyville asetuksille. Se tarkoittaa kirjainperhettä
tai siihen kuuluvaa yksittäistä leikkausta sekä asetuksia, jotka
liittyvät niihin. Silloin kun merkitystä pitää täsmentää, käytetään
suomenkielisiä nimiä kirjainperhe ja kirjainleikkaus. Sen sijaan sana
\emph{kirjasin} on jätetty kokonaan pois. Se tarkoittaa vanhassa
metalliladonnassa ja mekaanisissa kirjoituskoneissa metallisen
ladontakappaleen eli kirjakkeen päähän valettua kirjaimen tai muun
merkin kohokuviota, joka painaa mustejäljen paperille.

Kuten Latexissa yleensäkin myös fonttien kanssa kannattaa käyttää
korkean abstraktiotason komentoja, jotka piilottavat yksityiskohdat ja
teknisen toteutuksen. Latexin fonttitoiminnot on suunniteltu juuri
siihen: matalan tason fonttiasetukset määritellään mieluiten vain kerran
dokumentin esittelyosassa, ja sen jälkeen käytetään pelkästään korkean
tason komentoja.

Latexin fonttitekniikka rakentuu eri\-/ikäisistä kerroksista ja
tekniikoista. Fontteja on aikoinaan tehty
\englanti{Metafont}\avctan{metafont}\-/ nimisellä kielellä, jolla
kuvataan merkkien muodot. \englanti{Metafont} on myös tietokoneohjelma,
joka tuottaa kuvauskielen perusteella bittikarttafontteja eli
pikseleistä koostuvia fontteja. On käytetty myös kehittyneempää
\englanti{Metapost}\avctan{metapost}\-/ kuvauskieltä ja \=/ohjelmaa,
joilla on tuotettu vektorigrafiikkafontteja \textsc{eps}- eli
\englanti{Encapsulated Post Script} \=/muodossa ja muutettu niitä
edelleen \englanti{Post Script Type~1} \=/fonteiksi. Myöhemmin mukaan
ovat tulleet nykyaikaiset \englanti{True Type}- ja \englanti{Open Type}
\=/fontit, ja niihin tämä opas keskittyy.

\subsection{Fonttien määrittäminen}
\label{luku/fontin-valinta}

Latexin fonttien perustoiminnot rakentuvat kolmen erityyppisen
kirjainperheen varaan:

\begin{nluetelma}
\item peruskirjainperhe eli dokumentin pääasiallinen kirjainperhe, joka
  on kirjatypografiassa usein antiikva eli pääteviivallinen
  (\englanti{serif, roman})%
  \footnote{Antiikva (lat. \emph{antiquus} 'vanha') perustuu antiikin
    Roomassa käytettyihin kirjainmuotoihin. Niissä on pääteviivat, ja
    viivojen vahvuus vaihtelee.}
\item groteski eli pääteviivaton (\englanti{sans serif, gothic})%
  \footnote{Groteskiin (ransk. \emph{grotesque} 'kummallinen') kuuluu
    pääteviivojen puuttumisen lisäksi lähes tasavahvuiset kirjainten
    viivat. Tämän oppaan groteskifontissa on kuitenkin selvästi
    antiikvamaiset kaksivahvuiset viivat, joten se on eräänlainen
    antiikvan ja groteskin välimuoto.}
\item tasalevyinen kirjoituskoneen kaltainen perhe
  (\englanti{type\-writer, mono\-spaced, tele\-type}).
\end{nluetelma}

\noindent
Kuvassa \ref{kuva/kirjainperhetyypit} ovat tässä oppaassa käytetyt kolme
eri kirjainperhettä. Leipätekstissä käytetään antiikvaa, otsikoissa ja
kuvateksteissä groteskia ja koodiesimerkeissä tasalevyistä.
Kirjoituskoneen kaltainen tasalevyinen kirjainperhe on tässä tapauksessa
tyypiltään antiikva eli pääteviivallinen, mutta se voisi olla muutakin.
Tasalevyisyys on sen kirjainperheen tärkein määrittävä tekijä Latexin
asetusten näkökulmasta.

\leijukuva{
  {\rmfamily\addfontfeatures{ScaleAgain=5}Amf}
  \hfill
  {\sffamily\addfontfeatures{Scale=5}Amf}
  \hfill
  {\ttfamily\addfontfeatures{FakeStretch=1, Scale=4.5}Amf}
}{
  \caption{Vasemmalla pääteviivallinen (antiikva), keskellä
    pääteviivaton (groteski) ja oikealla tasalevyinen kirjainperhe}
  \label{kuva/kirjainperhetyypit}
}

Joidenkin fonttien käyttöönottoon on tehty oma pakettinsa, joten
sellaiset fontit voi ladata dokumentin esittelyosassa komennolla
\komento{usepackage}. Fonttikohtaisia paketteja on olemassa
varsinkin vanhalle fonttitekniikalle (\englanti{Metafont, Post Script
  Type~1}) mutta myös matematiikkatilan fonttiasetuksille (luku
\ref{luku/matematiikka-fontit}) ja joillekin
kir\-jain\-perhe\-koko\-nai\-suuk\-sille.

Latexissa pisimmälle ''tuotteistettu'' kokonaisuus taitaa olla
Libertinus\-/ kirjainperhe, joka sisältää antiikvan, groteskin ja
tasalevyisen kirjainperheen sekä matematiikkatilan symboleita.
Libertinus\-/ kirjainperheet saa käyttöön lataamalla paketin
\pakettictan{libertinus}:

\komentoi{usepackage}
\pakettii{libertinus}
\begin{koodilohkosis}
\usepackage{libertinus}
\end{koodilohkosis}

\noindent
Valmiita paketteja on kuitenkin vain harvoille fonteille, ja käytännössä
lähes aina \englanti{True Type}- ja \englanti{Open Type} \=/muodossa
olevat fontit otetaan käyttöön \pakettictan{fontspec}\-/ paketin
komennoilla\footnote{Vaihtoehtoisesti fonttien asettamiseen voi käyttää
  \paketti{babel}\-/ paketin komentoja (luku \ref{luku/babel}).}
seuraavan esimerkin mukaisesti:

\komentoi{setmainfont}
\komentoi{setsansfont}
\komentoi{setmonofont}
\begin{koodilohkosis}
\setmainfont{TeX Gyre Termes}[Scale=1]
\setsansfont{TeX Gyre Heros} [Scale=MatchLowercase]
\setmonofont{TeX Gyre Cursor}[Scale=MatchLowercase]
\end{koodilohkosis}

\leijutlk{
  \begin{tabular}{ll}
    \toprule
    \ots{Kirjainperhe} & \ots{Tyyli} \\
    \midrule
    Garamond Libre & antiikva, renessanssi, ranskalainen \\
    TeX Gyre Pagella & antiikva, renessanssi, ranskalainen \\
    TeX Gyre Termes & antiikva, renessanssi, alankomainen \\
    Libertinus Serif & antiikva, barokki, siirtymäkausi \\
    Latin Modern Roman & antiikva, uusantiikva \\
    TeX Gyre Schola & antiikva, vahvapäätteinen \\
    TeX Gyre Bonum & antiikva, vahvapäätteinen \\
    DejaVu Serif & antiikva, vahvapäätteinen \\
    TeX Gyre Chorus & kalligrafinen \\
    \midrule
    Libertinus Sans & groteski, humanistinen, kaksivahvuinen \\
    TeX Gyre Heros & groteski, uusgroteski \\
    DejaVu Sans & groteski, uusgroteski \\
    TeX Gyre Adventor & groteski, geometrinen \\
    \midrule
    Libertinus Mono & tasalevyinen \\
    TeX Gyre Cursor & tasalevyinen \\
    DejaVu Sans Mono & tasalevyinen \\
    \bottomrule
  \end{tabular}
}{
  \caption{Vapaita fontteja ja kirjainperheitä, jotka toimitetaan Tex
    Live \=/jakelun tai käyttöjärjestelmän mukana. Tyylit viittaavat
    kirjainmuotoon tai historialliseen kirjaintyyliin}
  \label{tlk/vapaita-fontteja}
}

\noindent
Edellisessä esimerkissä \englanti{TeX Gyre Termes, Heros} ja
\englanti{Cursor} ovat kirjainperheiden nimiä. Fonttitiedostojen tulee
olla asennettuna käyttöjärjestelmän normaalien käytäntöjen mukaisesti
tai Latex\-/jakelun käytäntöjen mukaisesti. Taulukkoon
\ref{tlk/vapaita-fontteja} on koottu erityylisiä, vapaasti käytettävissä
olevia kirjainperheitä.

Kirjainperheiden käyttöönoton yhteydessä voi määritellä lukuisia
asetuksia kuten ligatuureja, gemenanumeroita, optisia kokoja ja muita
fontin ominaisuuksia. Edellisessä esimerkissä käytetään vain
\koodi{Scale}\-/ valitsinta, jolla fontin voi skaalata haluttuun kokoon.

Peruskirjainperheen (\komento{setmainfont}) skaalaukseksi asetetaan
esimerkissä \koodi{Scale=1}, eli sille ei tehdä mitään, ja koko
valitsimen voisi jättää pois. Sen sijaan kahdella muulla
kirjainperheellä (\komento{setsansfont}, \komento{setmonofont})
käytetään kerroinasetusta \koodi{MatchLowercase}, joka skaalaa fontin
siten, että gemenakirjaimet eli pienet kirjaimet ovat yhtä korkeita kuin
peruskirjainperheessä. Mikäli skaalausasetus \koodi{MatchLowercase} ei
tuota ihan toivottua tulosta, voi kirjainperheen skaalausta hienosäätää
vielä \koodi{Scale\-Again}\-/ valitsimella seuraavalla tavalla:

\komentoi{setmonofont}
\begin{koodilohkosis}
\setmonofont{TeX Gyre Cursor}
[Scale=MatchLowercase, ScaleAgain=.97]
\end{koodilohkosis}

\noindent
Kirjainperheen määrittelyn yhteydessä ei yleensä tarvitse antaa kuin
kirjainperheen nimi, sillä \paketti{fontspec}\-/paketti ja kääntäjät
osaavat automaattisesti ladata perheeseen sisältyviä eri
fonttitiedostoja kuten pystyasentoisen leikkauksen, kursiivin ja
lihavoinnin. On kuitenkin mahdollista määritellä kirjainperheeseen
kuuluvia leikkausten nimiä tai fonttitiedostoja erikseen. Tällainen on
tarpeen esimerkiksi silloin, kun kirjainperhe sisältää useita eri
vahvuuksia ja halutaan itse määritellä, mikä niistä tulee
perusvahvuudeksi ja mikä lihavaksi.

Esimerkki \ref{esim/fontit-leik-omin} selventää, kuinka kirjainperheen
eri leikkausten nimet tai fonttitiedostot määritellään. Kullekin
leikkaukselle voi määrittää myös omat asetuksensa \koodi{Fea\-tures}\-/
sanaan päättyvällä valitsimella. Samaa asiaa havainnollistetaan myös
konkreettisemmin esimerkissä \ref{esim/fontit-leik-omin-käyt}.
Leikkauksen nimessä voi käyttää tähteä (\koodi{*}), joka korvautuu
kirjainperheen nimellä (\englanti{Macklin Text}). Fonttien nimeämisestä
on lisätietoa luvussa \ref{luku/luatex-xetex-fonttitekn}.

\begin{esimerkki*}
  \komentoi{setmainfont}
\begin{koodilohko}
\setmainfont{…}[
  UprightFont={…},     UprightFeatures={…},
  ItalicFont={…},      ItalicFeatures={…},
  BoldFont={…},        BoldFeatures={…},
  BoldItalicFont={…},  BoldItalicFeatures={…},
  SlantedFont={…},     SlantedFeatures={…},
  BoldSlantedFont={…}, BoldSlantedFeatures={…},
  SmallCapsFont={…},   SmallCapsFeatures={…},
  SwashFont={…},       SwashFeatures={…},
  BoldSwashFont={…},   BoldSwashFeatures={…}]
\end{koodilohko}
  \caption{Kirjainperheeseen sisältyvien leikkausten nimien ja
    kirjainleikkauskohtaisten ominaisuuksien määrittely}
  \label{esim/fontit-leik-omin}
\end{esimerkki*}

\begin{esimerkki*}
  \komentoi{setmainfont}
\begin{koodilohko}
\setmainfont{Macklin Text}[
  UprightFont    = {* Light},
  ItalicFont     = {* Light Italic},
  BoldFont       = {* Medium},
  BoldItalicFont = {* Medium Italic}]
\end{koodilohko}
  \caption{Eri leikkausten nimien määrittely \englanti{Macklin Text}
    \=/kirjainperheelle. Leikkauksen nimessä tähti (\koodi{*}) korvautuu
    automaattisesti koko perheen nimellä}
  \label{esim/fontit-leik-omin-käyt}
\end{esimerkki*}

Jos edellä kuvatut kolme kirjainperhettä (\komento{setmainfont},
\komento{setsansfont} ja \komento{setmonofont}) eivät riitä, on
\paketti{fontspec}\-/paketissa komennot lisäperheiden ja \=/leikkausten
määrittämiseen. Uusi perhe määritellään seu\-raa\-vasti:

\komentoi{newfontfamily}
\begin{koodilohkosis}
\newfontfamily{\hienoperhe}{TeX Gyre Schola}[…]
\end{koodilohkosis}

\noindent
Komento \komento{newfontfamily} toimii samalla tavalla kuin aiemmin
esitellyt \komento{setmainfont} ym. komennot, mutta lisäksi
ensimmäisellä argumentilla nimetään komento, jolla kirjainperhe otetaan
käyttöön. Edellisessä esimerkissä luodaan komento
\komentox{hieno\-perhe}, joka kytkee päälle \englanti{TeX Gyre Schola}
\=/nimisen kirjainperheen.

Jos ei tarvita kokonaista perhettä vaan yksi leikkaus riittää, käytetään
komentoa \komento{newfontface}. Seuraavassa esimerkissä määriteltävä
komento \komentox{hieno\-leikkaus} ottaa käyttöön lihavoidun (bold)
kirjainleikkauksen perheestä \englanti{TeX Gyre Schola}.

\komentoi{newfontface}
\begin{koodilohkosis}
\newfontface{\hienoleikkaus}{TeX Gyre Schola Bold}[…]
\end{koodilohkosis}

\subsection{Fontin koko ja rivikorkeus}

Fonttien koot on tapana valita ja ilmaista typografisen pistemitan
avulla. Esimerkiksi 10--12 pistettä on tyypillinen leipätekstin
oletuskoko tekstinkäsittelyohjelmissa. Piste on typografiassa
mittayksikkö, jonka pituus on määritelty eri tavoin eri aikoina ja eri
kulttuureissa.

Myös Latexissa fonttien koot voi määritellä pistemittojen avulla. Niitä
ja muitakin Latexin mittayksiköitä käsitellään tarkemmin luvussa
\ref{luku/mitat}. Fonteissa oletusmittayksikkönä on vanha pica\-/
järjestelmän piste, jonka pituus on noin 0,3515 millimetriä. Sen lyhenne
Latexissa on~pt. Tämän oppaan esimerkeissä käytetään kuitenkin
\englanti{Post Script} \=/standardin mukaista, julkaisuohjelmiin
vakiintunutta uudempaa pica\-/ pistettä, joka on hieman edellistä
pidempi: noin 0,3528 millimetriä. Latexissa sen lyhenne on~bp. Ero
näiden kahden pistemitan välillä on hyvin pieni, tavallisilla
fonttiko'oilla käytännössä merkityksetön.

Kirjainleikkauksen koko mitataan merkistön ylimmän ja alimman kohdan
välillä, esimerkiksi k\=/kirjaimen ylimmän pisteen ja y\=/kirjaimen
alimman pisteen välillä. Lisäksi mittaan luetaan mukaan merkistön ylä-
ja alapuolella oleva pieni tyhjä tila, jonka fontin suunnittelija on
määritellyt.

Matalalla tasolla fonttien kokoon vaikuttaa Latexissa eräs yllättävä
asia. Nimittäin dokumenttiluokalle (luku \ref{luku/dokumenttiluokat})
voi antaa valitsimen, jolla fontin koko asetetaan. Vaihtoehtoja on
Latexin normaaleissa dokumenttiluokissa vain kolme: \koodi{10pt}
(oletus), \koodi{11pt} ja \koodi{12pt}. Dokumenttiluokan kokoasetus
vaikuttaa myös sivun marginaaleihin, koska Latex pyrkii pitämään rivin
merkkimäärän lukijalle sopivana: yhdelle riville ei kannata latoa ihan
mahdottomasti merkkejä, koska kovin pitkän rivin seuraaminen rasittaa
lukijaa ja vaatii enemmän keskittymistä.

Fontin koon määrittäminen dokumenttiluokan valitsimella ehkä kuuluu jo
vähän menneisyyteen, mutta voi sitä edelleen käyttää, jos se riittää ja
sillä saa halutun lopputuloksen. Yleensä lienee järkevää jättää
dokumenttiluokan fonttiasetus oletukseksi (\koodi{10pt}) ja käyttää koon
asettamiseen luvuissa \ref{luku/fontti-suhteellinen} ja
\ref{luku/fontti-absoluuttinen} kerrottuja tapoja. Sivun marginaalien ja
muiden mittojen määrittämiseen on ohjeita luvussa
\ref{luku/sivuasetukset}.

Fonttiasetuksiin kuuluu fontin koon lisäksi toinenkin mitta: rivikorkeus
(\mitta{baselineskip}). Se on peräkkäisten rivien peruslinjojen välinen
etäisyys. Fontin koko ja rivikorkeus määritellään samanaikaisesti, koska
ne ovat saman \komento{fontsize}\-/ komennon argumentteja. Esimerkki:

\komentoi{fontsize}
\komentoi{selectfont}
\begin{koodilohkosis}
\fontsize{10bp}{12bp} \selectfont
\end{koodilohkosis}

\noindent
Ensimmäinen argumentti on fontin kokomitta ja toinen on rivikorkeus.
Mittayksiköt voivat olla mitä tahansa Texin mittoja, ja oletuksena
käytetään pt\-/pistemittaa, jos yksikköä ei ole mainittu. Komento
\komento{selectfont} on mukana, koska vasta sen myötä matalan tason
fonttikomennot tulevat voimaan. Korkean tason fonttikomennot (luku
\ref{luku/fontit-korkea}) suorittavat sen automaattisesti.

Rivikorkeus on vähintään sama kuin fontin koko, mutta yleensä se
asetetaan pari pistettä suuremmaksi, jotta rivit eivät olisi liian
lähellä toisiaan. Esimerkissä \ref{esim/rivikorkeus} on kaksi erilaista
\komento{fontsize}\-/komentoa ja ladottu lopputulos.

\begin{esimerkki*}
  \komentoi{fontsize}
  \komentoi{selectfont}

\begin{koodilohko}
\fontsize{8bp}{11bp}\selectfont Tässä on pienehkö leipätekstin
fonttikoko ja suhteellisen suuri rivikorkeus. Pitkät rivit vaativat
suuremman rivikorkeuden kuin lyhyet rivit.

\fontsize{16bp}{17bp}\selectfont Tässä on melko suuri fontti ja
suhteellisen pieni rivikorkeus. Suuri fontti ja lyhyet rivit eivät
tarvitse kovin suurta rivikorkeutta.
\end{koodilohko}
  \begin{tulos}
    \fontsize{8bp}{11bp}\selectfont Tässä on pienehkö leipätekstin
    fonttikoko ja suhteellisen suuri rivikorkeus. Pitkät rivit vaativat
    suuremman rivikorkeuden kuin lyhyet rivit.

    \fontsize{16bp}{17bp}\selectfont Tässä on melko suuri fontti ja
    suhteellisen pieni rivikorkeus. Suuri fontti ja lyhyet rivit eivät
    tarvitse kovin suurta rivikorkeutta.
  \end{tulos}
  \caption{Fontin koon ja rivikorkeuden asettaminen ja vaikutus}
  \label{esim/rivikorkeus}
\end{esimerkki*}

Toinen tekstirivien peruslinjojen väliseen etäisyyteen vaikuttava asetus
on \komento{baselinestretch}. Se on desimaalilukukerroin, jolla nykyinen
rivikorkeus kerrotaan. Kerroin asetetaan helpoimmin komennolla
\komento{linespread}.\footnote{Toinen tapa: \koodi{\keno
    renewcommand\{\keno baselinestretch\}\{kerroin\}}}

\komentoi{fontsize}
\komentoi{linespread}
\komentoi{selectfont}
\begin{koodilohkosis}
\fontsize{10bp}{12bp} \linespread{1.3} \selectfont
\end{koodilohkosis}

\noindent
Edellä oleva esimerkki asettaa fontin kooksi 10 pistettä ja
rivikorkeudeksi 12 pistettä. \komento{linespread}\-/ komennolla asetetun
kertoimen vuoksi rivien peruslinjojen väliseksi etäisyydeksi tulee
lopulta 1,3 kertaa 12 pistettä eli 15,6 pistettä. Ei ole väliä, kummassa
järjestyksessä \komento{fontsize}- ja \komento{linespread}\-/ komennot
annetaan. Asetukset tulevat voimaan vasta \komento{selectfont}\-/
komennon jälkeen.

Käytännössä \komento{linespread} sopii rivikorkeuden yleistason
hienosäätöön esimerkiksi dokumentin esittelyosassa. Tilannekohtainen
rivikorkeus on parasta asettaa \komento{fontsize}\-/komennolla.

\subsection{Kirjainperheen ja -leikkauksen valitseminen}
\label{luku/fontit-korkea}

Latexissa on joukko korkean tason fonttikomentoja, jotka on tarkoitettu
käytettäväksi sen jälkeen, kun matalan tason asetukset on kerran
määritetty. Taulukoissa \ref{tlk/komennot-kirjainperhe} ja
\ref{tlk/komennot-kirjainleikk} ovat komennot kirjainperheen ja
kirjainleikkauksen valintaan. Joka rivillä ensin mainittu komento (esim.
\komento{rmfamily}) vaikuttaa tekstiin, joka tulee komennon jälkeen.
Vaikutusalue rajoittuu nykyisen ympäristön (luku \ref{luku/ympäristöt})
sisään tai aaltosulkeilla (luku \ref{luku/aaltosulkeet}) rajatun alueen
sisään. Rivillä toisena olevalle komennolle (esim. \komentox{textrm})
annetaan yksi argumentti, ja komennon vaikutus koskee vain argumenttina
olevaa tekstiä.

\leijutlk{
  \providecommand{\rivi}{}
  \renewcommand{\rivi}[3]{%
    \komento{#1} & \komento{#2}\komentoarg{\dots} & #3 \\}
  \begin{tabular}{lll}
    \toprule
    \multicolumn{2}{l}{\ots{Komento}} & \ots{Merkitys} \\

    \midrule
    \rivi {rmfamily} {textrm}
    {\textrm{perus, yleensä antiikva, serif, roman}}
    \rivi {sffamily} {textsf}
    {\textsf{groteski, sans serif, gothic}}
    \rivi {ttfamily} {texttt}
    {\texttt{tasalevyinen, typewriter}}

    \bottomrule
  \end{tabular}
}{
  \caption{Komennot kirjainperheen valintaan. Perustila on
    \komento{rmfamily}}
  \label{tlk/komennot-kirjainperhe}
}

\leijutlk{
  \providecommand{\rivi}{}
  \renewcommand{\rivi}[3]{%
    \komento{#1} & \komento{#2}\komentoarg{\dots} & #3 \\}
  \begin{tabular}{llll}
    \toprule
    \multicolumn{2}{l}{\ots{Komento}} & \ots{Merkitys} \\

    \midrule
    \rivi {mdseries} {textmd}
    {\textmd{tavallinen vahvuus, medium}}
    \rivi {bfseries} {textbf}
    {\textbf{lihavoitu, bold}}

    \midrule
    \rivi {upshape} {textup}
    {\textup{pystyasento, tavallinen}}
    \rivi {itshape} {textit}
    {\textit{kursiivi, italic}}
    \rivi {slshape} {textsl}
    {\textsl{kalteva, slanted, oblique}}
    \rivi {swshape} {textsw}
    {{\swashfontti Koristeellinen, swash}}
    \rivi {scshape} {textsc}
    {\textsc{pienversaali, kapiteeli, small caps}}

    \bottomrule
  \end{tabular}
}{
  \caption{Komennot kirjainleikkauksen valintaan saman kirjainperheen
    sisällä. Perustila on \komento{mdseries} ja \komento{upshape}}
  \label{tlk/komennot-kirjainleikk}
}

Taulukossa \ref{tlk/komennot-kirjainperhe} ovat kirjainperhekomennot,
jotka vaihtavat koko perheen kaikkine leikkauksineen. Taulukon
\ref{tlk/komennot-kirjainleikk} komennot puolestaan valitsevat toisen
leikkauksen samasta perheestä. Kirjainleikkauksen asetukset jaetaan
kahteen ryhmään: \englantik{series} 'sarja' ja \englantik{shape}
'muoto'. Kummastakin ryhmästä on valittuna aina yksi ominaisuus, eli
samanaikaisesti voi olla voimassa esimerkiksi \komento{bfseries}
(\textbf{lihavoitu}) ja \komento{itshape} (\textit{kursiivi}), ja
tuloksena on \textbf{\itshape lihavoitua kursiivia}.

Useimmissa fonteissa kursiivileikkaus (\komento{itshape}) ja kalteva
leikkaus (\komento{slshape}) tuottavat saman lopputuloksen, mutta
käsitteellisesti ne ovat eri asia. Kursiivi on aina muodoltaan erilainen
leikkaus, joka hieman mukailee käsialakirjoitusta, joskaan kirjaimia ei
ole sidottu toisiinsa. Sen sijaan kalteva leikkaus on tavallisen eli
pystyasentoisen leikkauksen kallistettu versio.

Jos kallistus on kirjainmuotoilijan piirtämä, kursiivista poikkeava
leikkaus eli ihan oma fonttitiedostonsa, kallistettu leikkaus täytynee
erikseen määrittää osaksi kirjainperhettä \koodi{Slanted\-Font}\-/
asetuksella (esimerkki \ref{esim/fontit-leik-omin}). Kallistuksen voi
tehdä myös keinotekoisesti Latexissa. Katso lisätietoa luvusta
\ref{luku/fontit-venytys}.

Jotkut \englanti{Open Type} \=/fontit sisältävät erillisen
koristeellisen tyylin (\komento{swshape}), joka voi sisältyä esimerkiksi
kursiivileikkaukseen mutta joka täytyy silti kytkeä erikseen päälle.
Sellaisen saa määritettyä osaksi kirjainperhettä käyttämällä valitsimia
\koodi{Swash\-Font} ja \koodi{Swash\-Fea\-tures} (esimerkki
\ref{esim/fontit-leik-omin}) seuraavalla tavalla:

\komentoi{setmainfont}
\begin{koodilohkosis}
\setmainfont{Garamond Premier Pro}[
  SwashFont={GaramondPremrPro-It},   % kursiivileikkaus
  SwashFeatures={Style=Swash}]       % koristeellinen swash-tyyli
\end{koodilohkosis}

\noindent
Komennot fontin koon valintaan ovat taulukossa
\ref{tlk/fonttikokokomennot}. Taulukko kertoo myös, mitä fontin
pistekokoa (pt) mikäkin komento tarkoittaa oletuksena. Oletus riippuu
Latexin dokumenttiluokkien (luku \ref{luku/perusdokumenttiluokat})
fonttikokovalitsimista \koodi{10pt}, \koodi{11pt} ja \koodi{12pt}.

\leijutlk{
  \begin{tabular}{lr@{}lr@{}lr@{}l}
    \toprule
    \ots{Komento}
    & \multicolumn{2}{c}{\ots{10pt}}
    & \multicolumn{2}{c}{\ots{11pt}}
    & \multicolumn{2}{c}{\ots{12pt}} \\
    \midrule
    \komento{tiny} & 5 && 6 && 6 \\
    \komento{scriptsize} & 7 && 8 && 8 \\
    \komento{footnotesize} & 8 && 9 && 10 \\
    \komento{small} & 9 && 10 && 10&,95 \\
    \komento{normalsize} & 10 && 10&,95 & 12 \\
    \komento{large} & 12 && 12 && 14&,4 \\
    \komento{Large} & 14&,4 & 14&,4 & 17&,28 \\
    \komento{LARGE} & 17&,28 & 17&,28 & 20&,74 \\
    \komento{huge} & 20&,74 & 20&,74 & 24&,88 \\
    \komento{Huge} & 24&,88 & 24&,88 & 24&,88 \\
    \bottomrule
  \end{tabular}
}{
  \caption{Fonttien oletuspistekoot dokumenttiluokkien valitsimilla
    \koodi{10pt}, \koodi{11pt} ja \koodi{12pt}}
  \label{tlk/fonttikokokomennot}
}

\ymparistoi{rmfamily}
\ymparistoi{sffamily}
\ymparistoi{ttfamily}
\ymparistoi{mdseries}
\ymparistoi{bfseries}
\ymparistoi{upshape}
\ymparistoi{itshape}
\ymparistoi{slshape}
\ymparistoi{swshape}
\ymparistoi{scshape}
\ymparistoi{tiny}
\ymparistoi{scriptsize}
\ymparistoi{footnotesize}
\ymparistoi{small}
\ymparistoi{normalsize}
\ymparistoi{large}
\ymparistoi{Large}
\ymparistoi{LARGE}
\ymparistoi{huge}
\ymparistoi{Huge}%

Kaikille korkean tason fonttikomennoille on olemassa myös samanniminen
ympäristönsä, esimerkiksi \ymparisto{rmfamily}, \ymparisto{bfseries},
\ymparisto{itshape} tai \ymparisto{small}. Seuraavassa esimerkissä on
kaksi fontteihin vaikuttavaa ympäristöä sisäkkäin.

\ymparistoi{footnotesize}
\ymparistoi{scshape}
\begin{koodilohkosis}
\begin{footnotesize}
  \begin{scshape}
    Tämä teksti on pientä pienversaalia.
  \end{scshape}
\end{footnotesize}
\end{koodilohkosis}

\begin{tulossis}
  \begin{footnotesize}
    \begin{scshape}
      Tämä teksti on pientä pienversaalia.
    \end{scshape}
  \end{footnotesize}
\end{tulossis}

\subsection{Fonttikoon määrittely suhteellisesti}
\label{luku/fontti-suhteellinen}

Dokumentin fonttien koot on helpointa määrittää siten, että asettaa
ensin peruskirjainperheen koon ja antaa muiden fonttien määräytyä
suhteessa siihen. Esimerkki \ref{esim/fontti-suhteellinen} selventää,
kuinka se tapahtuu. Alussa otetaan käyttöön dokumenttiluokka
\luokka{article} ja annetaan sille valitsin \koodi{10pt}, joka määrittää
fonttikooksi 10 pistettä. Se on dokumenttiluokan oletusasetus, jota ei
tarvitsisi edes kirjoittaa näkyviin. Esimerkin toisella rivillä otetaan
\paketti{fontspec}\-/paketti käyttöön.

Peruskirjainperheen (rivi~4) koko skaalataan 1,4\-/kertaiseksi, eli
pistekooksi tulee 1,4 kertaa 10 pistettä eli 14 pistettä (pt).
Normaalikokoinen peruskirjainperhe on ainoa, jonka pistekoko tiedetään.
Kaikkien muiden koot täytyisi selvittää laskemalla.

Groteski eli pääteviivaton kirjainperhe (rivi~5) ja tasalevyinen perhe
(rivi~6) skaalataan samankorkuiseksi kuin perusperhe. Vertailukohtana
ovat gemenat eli pienaakkoset (\koodi{MatchLowercase}). Näiden kahden
kirjainperheen pistekokoa ei tiedetä. Se ei välttämättä ole sama kuin
perusfontissa, koska fonttien pistekoko mitataan ylimmän ja alimman
kohdan välillä ja koska fonttien mittasuhteet ovat erilaisia.

\begin{esimerkki*}
  \komentoi{documentclass}
  \komentoi{usepackage}
  \pakettii{fontspec}
  \komentoi{setmainfont}
  \komentoi{setsansfont}
  \komentoi{setmonofont}
  \komentoi{linespread}
  \luokkai{article}

\begin{koodilohko}
\documentclass[10pt]{article} % 10pt on oletus
\usepackage{fontspec}

\setmainfont{TeX Gyre Termes}[Scale=1.4]
\setsansfont{TeX Gyre Heros} [Scale=MatchLowercase]
\setmonofont{TeX Gyre Cursor}[Scale=MatchLowercase]
\linespread{1.45}
\end{koodilohko}
  \caption{Fonttikokojen määrittäminen suhteessa peruskirjainperheeseen}
  \label{esim/fontti-suhteellinen}
\end{esimerkki*}

Viimeisellä rivillä oleva \komento{linespread}\-/ komento on tärkeä. Se
asettaa rivikorkeuden kertoimeksi 1,45. Kertoimen täytyy olla vähintään
yhtä suuri kuin peruskirjainperheen skaalauskerroin (1,4), jotta
rivivälit ovat riittävän suuret. Näiden asetusten jälkeen dokumentissa
käytetään korkeamman tason komentoja fonttien valintaan, esimerkiksi
fonttikoon valintakomentoja \komento{small}, \komento{normalsize},
\komento{large} (taulukko \ref{tlk/fonttikokokomennot}).

Edellä kuvatussa suhteellisessa kirjainperheiden koon määrittelyssä on
sellainen ongelma tai kummallisuus, että Latex koko ajan luulee, että
peruskirjainperhe on normaalikokoisena 10 pistettä (pt). Latexin matalan
tason fonttikomennot eivät tiedä kirjainperheen skaalauskertoimesta, ja
siksi esimerkiksi komentojen

\komentoi{fontsize}
\komentoi{selectfont}
\begin{koodilohkosis}
\fontsize{10bp}{12bp} \selectfont
\end{koodilohkosis}

\noindent
tuloksena ei todellisuudessa ole 10 pisteen (bp) fontti, vaan mukaan
lasketaan myös kirjainperheen skaalauskerroin. Tämän vuoksi
\komento{fontsize}\-/ komennon käyttö menee aika oudoksi. Argumenttina
annettu kokomitta ei pidä paikkaansa.

Jos korkean tason fonttikokokomentojen (taulukko
\ref{tlk/fonttikokokomennot}) lisäksi tarvitaan jotakin muuta kokoa,
voisi mahdollisesti \komento{fontsize}\-/ komennon sijasta käyttää
\paketti{fontspec}\-/paketin tarjoamaa komentoa ja tilanteeseen sopivaa
skaalauskerrointa esimerkiksi seuraavalla tavalla:

\komentoi{addfontfeatures}
\begin{koodilohkosis}
{\addfontfeatures{Scale=3.2} Poikkeuksellisen isoa tekstiä}
\end{koodilohkosis}

\noindent
Jos edellä mainitut kummallisuudet eivät häiritse eikä ole tarvetta
määritellä fontteja tarkasti tietyn pistekoon mukaiseksi, on
suhteellinen määrittelytapa todella helppo. Kaikki dokumentin fontit
määräytyvät perusfontin skaalauskertoimen kautta. Tämä tapa sopii hyvin
varsinkin dokumentin sisällön kirjoittamisvaiheeseen, jossa ehkä
halutaan vain nopeasti asettaa dokumentti suurin piirtein järkevän
näköiseksi. Myöhemmin voi määrittää koot tarkemmin niin sanotun
absoluuttisen menetelmän avulla, jota käsitellään seuraavassa
alaluvussa.

\subsection{Fonttikoon määrittely absoluuttisesti}
\label{luku/fontti-absoluuttinen}

Absoluuttinen fonttien koonmääritystapa tarkoittaa sitä, että koot
asetetaan tietyn kokoiseksi käyttämällä esimerkiksi pistemittoja ja että
kirjaimet myös päätyvät lopulliseen dokumenttiin juuri sen kokoisena.
Tämä tapa on myös teknisesti eheä, eli Latexin eri osat ovat samaa
mieltä siitä, minkäkokoisesta fontista on kyse. Näin ei ollut
suhteellisen tavan kanssa (luku \ref{luku/fontti-suhteellinen}).

Joskus oppilaitoksen, yhtiön tai muun julkaisijan ohjeissa määritellään
tarkasti, mitä fontteja käytetään ja mikä on leipätekstin ja otsikoiden
fonttikoko. Silloin tarvitaan tässä luvussa kuvattua tapaa fonttien
asettamiseen.

\begin{esimerkki*}
  \komentoi{documentclass}
  \komentoi{usepackage}
  \pakettii{fontspec}
  \komentoi{setmainfont}
  \komentoi{setsansfont}
  \komentoi{setmonofont}
  \komentoi{newfontfamily}
  \komentoi{linespread}
  \komentoi{renewcommand}
  \komentoi{footnotesize}
  \komentoi{small}
  \komentoi{normalsize}
  \komentoi{large}
  \komentoi{Large}

\begin{koodilohko}
\documentclass{article}
\usepackage{fontspec}

% Leipätekstiin samankokoiset fontit
\setmainfont{TeX Gyre Termes}
\setsansfont{TeX Gyre Heros} [Scale=MatchLowercase]
\setmonofont{TeX Gyre Cursor}[Scale=MatchLowercase]

% Muualle sans ja mono ilman skaalausta
\newfontfamily{\sffamilyabs}{TeX Gyre Heros}
\newfontfamily{\ttfamilyabs}{TeX Gyre Cursor}

\linespread{1} % ei välttämättä tarvita

% Kaikki tarvittavat fonttikoot ja komennot
\renewcommand{\footnotesize}{\fontsize{10bp}{12bp}\selectfont}
\renewcommand{\small}       {\fontsize{12bp}{14bp}\selectfont}
\renewcommand{\normalsize}  {\fontsize{14bp}{17bp}\selectfont}
\renewcommand{\large}       {\fontsize{17bp}{19bp}\selectfont}
\renewcommand{\Large}       {\fontsize{20bp}{22bp}\selectfont}

\normalsize % jotta tulee heti voimaan eikä vasta tekstiosassa
\end{koodilohko}
  \caption{Fonttikokojen määrittäminen pistekoon avulla}
  \label{esim/fontti-absoluuttinen}
\end{esimerkki*}

Esimerkistä \ref{esim/fontti-absoluuttinen} selviää perusajatus.
Peruskirjainperhe (rivi~5) otetaan käyttöön ilman skaalausta
(\koodi{Scale=1}), minkä vuoksi koon voi jatkossa asettaa täsmälleen
kohdalleen \komento{fontsize}\-/komennolla. Samaa ei tehdä groteskin
eikä tasalevyisen fontin kanssa (rivit 6--7), vaan käytetään skaalausta
\koodi{MatchLowercase}, jotta tekstikappaleessa kaikki kirjainperheet
näyttävät samankokoisilta. Tässä menetetään mahdollisuus määrittää
näiden kirjainperheiden koko täsmällisesti pistemitan avulla. Jos siihen
on tarvetta esimerkiksi otsikoissa, voidaan käyttää rivien 10--11
komentoja. Niillä luodaan uudet kirjainperheet, jotka ovat käytännössä
samoja mutta ilman skaalausta.

Uusien skaalaamattomien kirjainperheiden komentojen nimiksi on valittu
\komentox{sf\-fam\-i\-ly\-abs} ja \komentox{tt\-fam\-i\-ly\-abs} (vrt.
\komento{sffamily} ja \komento{ttfamily}, taulukko
\ref{tlk/komennot-kirjainperhe}), ja näillä komennoilla kirjainperheet
kytketään päälle. Jos esimerkiksi jonkin julkaisun vaatimuksiin kuuluu,
että otsikossa täytyy olla 20 pisteen lihavoitu TeX Gyre Heros
\=/kirjainleikkaus, voi esimerkissä \ref{esim/fontti-absoluuttinen}
olevien asetusten pohjalta antaa otsikon ulkoasun (luku
\ref{luku/otsikot-ulkoasu}) määrittelyn yhteydessä seuraavat komennot:

\komentoi{Large}
\komentoi{bfseries}
\begin{koodilohkosis}
\sffamilyabs\Large\bfseries
\end{koodilohkosis}

\noindent
Esimerkin \ref{esim/fontti-absoluuttinen} riveillä 16--20 määritellään
uudelleen Latexin korkean tason komennot, joilla fonttikoot asetetaan.
Oletusarvot tulevat dokumenttiluokasta (luku
\ref{luku/dokumenttiluokat}), mutta jos ne eivät ole sopivia, täytyy
vähintäänkin määritellä komento \komento{normalsize} mutta sen lisäksi
kaikki ne koot, joita omassa dokumentissa tarvitaan. Tässä esimerkissä
normaali koko asetetaan 14 pisteen kokoiseksi.

Jokaiselle fonttikoolle määritetään riveillä 16--20 myös oma
rivikorkeus, ja se on tarkoitus asettaa sopivaksi juuri kyseiselle
koolle. Rivikorkeuteen vaikuttaa myös kerroin \komento{baselinestretch},
joka asetetaan komennolla \komento{linespread}. Sitä ei välttämättä
tarvitse käyttää, koska kirjainperheitä ei ole skaalattu ja koska
rivikorkeus asetetaan aina \komento{fontsize}\-/komennolla.
\komento{linespread} on kuitenkin kätevä komento rivikorkeuden
säätämiseen yleisesti kaikkialla.

Fonttikokojen määrittelyn lopuksi rivillä 22 suoritetaan komento
\komento{normalsize}, jotta se tulee heti voimaan. Dokumentin
esittelyosassa voidaan käyttää fonttikokoon viittaavia mittayksiköitä em
ja ex, ja ne viittaavat nyt tähän kokoon. \komento{normalsize}\-/
komento suoritetaan kyllä myöhemmin automaattisesti dokumentin
tekstiosan eli \ymparistox{document}\-/ ympäristön alussa.

Edellä kuvatun absoluuttisen koonmääritystavan etuna on se, että
kirjoittaja hallitsee fonttien kokoa ja rivikorkeuksia tarkasti ja että
julkaisuun saadaan juuri ne mitat, jotka halutaan tai vaaditaan. Tapa on
myös teknisesti eheä eli toimii Latexin sisäisen logiikan näkökulmasta
oikein. Haittana voi pitää sitä, että kaikki koot täytyy määritellä
erikseen.

\subsection{Lualatex, Xelatex ja fonttitekniikka}
\label{luku/luatex-xetex-fonttitekn}

Kääntäjät Lualatex ja Xelatex (Luatex ja Xetex) käsittelevät fontteja
sisäisesti eri tavalla. Ne esimerkiksi hyväksyvät kirjainperheiden tai
\=/leikkausten nimet hieman toisistaan poikkeavalla tavalla, ja
kääntäjää vaihtaessa saattaa joskus huomata, ettei jotakin
kirjainperhettä tai yksittäistä leikkausta enää löydykään.

Ongelma korjaantuu nimeämällä kirjainperheen tai \=/leikkauksen eri
tavalla. Yleispätevää ohjetta nimeämiseen on vaikeaa antaa, mutta jos
tulee ongelmia, kannatta fonttitiedostoista tutkia kirjainperheelle tai
\=/leikkaukselle annettuja nimiä. Se onnistuu käyttöjärjestelmän
komentotulkissa \koodi{otfinfo}\-/ komennolla, joka tulostaa jotakin
esimerkin \ref{esim/otfinfo-fonttinimiä} kaltaista. Kokeilemalla
tulosteessa olevia nimiä saa kyllä fontit toimimaan.

\begin{esimerkki*}
\begin{koodilohko}
Family:              Garamond Premr Pro
Subfamily:           Regular
Full name:           GaramondPremrPro
PostScript name:     GaramondPremrPro
Preferred family:    Garamond Premier Pro
Mac font menu name:  Garamond Premr Pro
\end{koodilohko}
  \caption{\koodi{otfinfo}\-/ komennon tuloste kertoo fontista muun
    muassa kirjainperheen ja \=/leikkauksen nimiä}
  \label{esim/otfinfo-fonttinimiä}
\end{esimerkki*}

\begin{koodilohkosis}
otfinfo -i GaramondPremrPro.otf
\end{koodilohkosis}

\noindent
Lualatex (Luatex) käyttää ulkoisia tekniikoita fonttien latomiseen eli
''renderöimiseen'', ja näitä tekniikoita voi vaihtaa fontin
määrittelemisen yhteydessä. Eri vaihtoehtojen kokeilu voi olla tarpeen,
jos oletusasetuksilla ei synny toivottua jälkeä. Asetuksia muutetaan
käyttämällä valitsinta \koodi{Renderer} seuraavan esimerkin tavoin:

\komentoi{setmainfont}
\begin{koodilohkosis}
\setmainfont{…}[Renderer=Node]
\end{koodilohkosis}

\noindent
Tavallisimpia \koodi{Renderer}\-/ valitsimen arvoja ovat
\koodi{\mbox{Node}}, \koodi{Harf\-Buzz} ja \koodi{Open\-Type}.
Lisätietoa voi lukea \paketti{fontspec}\-/ paketin ohjekirjan luvusta,
joka käsittelee Luatexin erityispiirteitä. Myös Xelatexissa (Xetex) on
fontteihin liittyviä ominaisuuksia, joita ei muissa kääntäjissä ole.
Niistäkin kerrotaan \paketti{fontspec}\-/ paketin ohjekirjassa.

\subsection{Fonttien oletusasetuksia}
\label{luku/fontit-oletusasetukset}

Oletuksena Latex\-/dokumentin peruskirjainperheessä
(\komento{setmainfont}) ja pääteviivattomassa kirjainperheessä
(\komento{setsansfont}) ovat päällä muun muassa seuraavat asetukset:%

\begin{koodilohkosis}
Ligatures={TeX, Common}
HyphenChar=-
\end{koodilohkosis}

\noindent
\koodi{TeX}\-/ligatuurit tarkoittavat lainausmerkkien ja ajatusviivojen
tuottamiseen tarkoitettuja Texin merkintätapoja kuten \koodi{''} ja
\koodi{\==}, joita käsitellään luvuissa \ref{luku/lainausmerkit} ja
\ref{luku/yhdys-ajatus-miinus}. Ne eivät ole varsinaisia luonnollisen
kielen eivätkä typografisia ligatuureja vaan kuuluvat ainoastaan
Tex\-/kielen merkintätapoihin. \koodi{Common}\-/ ligatuurit sen sijaan
ovat oikeita typografisia ligatuureja kuten fi, ff ja fl, ja niitä
käsitellään tarkemmin luvussa \ref{luku/typo-liga}. Molemmat edellä
mainitut ligatuurityypit saa pois päältä seuraavalla asetuksella:

\begin{koodilohkosis}
Ligatures={TeXReset, NoCommon}
\end{koodilohkosis}

\noindent
Valitsin \koodi{Hyphen\-Char} asettaa tavutusmerkin kyseiselle
kirjainperheelle. Oletuksena se on yhdysmerkki (\=/), mutta merkin
vaihtaminen toimii vain Xelatex\-/kääntäjän kanssa, ei esimerkiksi
Lualatexilla. Sen sijaan kaikilla kääntäjillä tavutuksen voi kytkeä
kirjainperheeltä pois, kun antaa arvoksi \koodi{None}. Tavutusta
käsitellään perusteellisemmin luvussa \ref{luku/tavutus}.

Tasalevyinen kirjainperhe (\komento{setmonofont}) toimii oletuksena
hieman eri tavalla. Niissä muun muassa ei ole lainausmerkkien ja
ajatusviivojen kirjoittamiseen tarkoitetut Tex\-/ligatuurit päällä.
Oletusasetuksia ovat muun muassa seuraavat:

\begin{koodilohkosis}
Ligatures=Common
HyphenChar=None
\end{koodilohkosis}

\noindent
Yleensä tasalevyisestä kirjainperheestä kannattaa kytkeä typografiset
ligatuurit pois päältä asetuksella \koodi{Liga\-tures=\katk No\-Common}.
Tasalevyisen fontin ajatukseen nimenomaan kuuluu, että merkit ovat
samanlevyisiä eikä yhden merkin tilaan sovi sulloa useampaa kirjainta.
Kaikki tasalevyiset kirjainperheet eivät edes sisällä ligatuurimerkkejä,
joten ominaisuus näyttää olevan automaattisesti pois päältä. Varmuuden
vuoksi on kuitenkin hyvä lisätä asetus \koodi{Liga\-tures=\katk
  No\-Common} tasalevyiselle perheelle.

Oletuksena tavutus on kytketty pois päältä tasalevyiseltä
kirjainperheeltä, koska sitä käytetään tavallisesti tietokoneisiin
liittyvien koodien tai vastaavien ilmausten latomiseen, eikä niitä
haluta yleensä tavuttaa. Tavutuksen saa kuitenkin päälle Xelatexissa
määrittämällä tavutusmerkiksi esimerkiksi yhdysmerkin
(\koodi{Hyphen\-Char=-}) tai kaikilla kääntäjillä poistamalla
tasalevyisen kirjainperheen oletusasetukset ennen sen määrittelyä
seuraavasti:

\komentoi{defaultfontfeatures}
\komentoi{ttfamily}
\komentoi{setmonofont}
\begin{koodilohkosis}
\defaultfontfeatures[\ttfamily]{}
\setmonofont{…}[…]
\end{koodilohkosis}

\noindent
Komennolla \komento{defaultfontfeatures} voi asettaa (joidenkin)
kirjainperheiden oletusasetukset, kun komennon suorittaa ennen
kirjainperheiden määrittelyä.

\komentoi{defaultfontfeatures}
\komentoi{rmfamily}
\komentoi{sffamily}
\begin{koodilohkosis}
\defaultfontfeatures[\rmfamily,\sffamily]{Ligatures={TeX, Common},
  Numbers=Lowercase}
\end{koodilohkosis}

\noindent
Komennon valinnaisella argumentilla voi rajata, mitä kirjainperheitä
oletusasetukset koskevat. Valinnaiseen argumenttiin kirjoitetaan yksi
tai useampia pilkulla erotettuja komentoja, joilla kirjainperheet
kytketään päälle: \komento{rmfamily}, \komento{sffamily},
\komento{ttfamily} (taulukko \ref{tlk/komennot-kirjainperhe},
s.~\pageref{tlk/komennot-kirjainperhe}) tai komennolla
\komento{newfontfamily} määritelty fonttikomento (luku
\ref{luku/fontin-valinta}).

\subsection{Typografiset ligatuurit}
\label{luku/typo-liga}

Typografiset ligatuurit ovat fontissa olevia yhdistelmämerkkejä, joissa
on typografisista syistä yhdistetty kaksi tai useampia kirjaimia yhteen
merkkiin.\footnote{Typografisten ligatuurien lisäksi on olemassa myös
  luonnollisten kielten ligatuureja, jotka muodostavat kieleen kuuluvan
  kirjaimen, kuten æ norjan kielessä. Niillä on jokin luonnolliseen
  kieleen liittyvä merkitys, eikä eri merkkejä ei ole yhdistetty
  typografisista syistä.} Tällaisia ligatuureja käytettiin jo
metalliladonnassa, eli samassa metallikirjakkeessa saattoi olla enemmän
kuin yksi kirjain.

Typografisten ligatuurien tarkoituksena on tuottaa tyylikkäämpi
lopputulos kuin saataisiin erillisten kirjainten avulla. Syynä on
esimerkiksi se, että erilliset peräkkäiset kirjaimet eivät ole aina
keskenään yhteensopivia. Kirjainten välistys voi olla ongelmallista:
sopivalle etäisyydelle asetettaessa kirjainten osat voivat mennä
rumannäköisesti päällekkäin. Riippuu fontista, mitkä ligatuurit ovat
tarpeellisia ja mitkä ligatuurimerkit on ylipäätään toteutettu fontin
merkistöön. Tavallisia antiikvojen ligatuureja ovat fi, ff, ffi, fl,
ffl, fj ja ffj, mutta jotkin fontit sisältävät muitakin. Kuvaan
\ref{kuva/ligatuurit} on koottu esimerkkejä.

\leijukuva{
  \rmfamily
  \addfontfeatures{Scale=3.4, Ligatures={Common, Historic,
      Discretionary}}
  fi ff ffi fl ffl fj ffj Th st ct
}{
  \caption{Tavalliset f\=/alkuiset typografiset ligatuurit, vähän
    harvinaisempi Th sekä historialliset ligatuurit st ja ct}
  \label{kuva/ligatuurit}
}

Latex hoitaa tavallisten typografisten ligatuurien latomisen
automaattisesti, eli lähdedokumenttiin ei kannata kirjoittaa
Unicode\-/merkistön typografisia ligatuurimerkkejä (esimerkiksi
\uctunnus{u+fb01 latin small ligature fi}) vaan ihan tavallisia
erillisiä kirjaimia. \englanti{Open Type} \=/fontin suunnittelija on
päättänyt meidän puolestamme, mitkä kirjainyhdistelmät on parasta latoa
ligatuurin avulla, ja oletusasetuksilla Latex noudattaa niitä ohjeita.

Jos kuitenkin yksittäisen ligatuurin muodostumisen haluaa estää, voi
kirjainten väliin kirjoittaa komennon \komento{textcompwordmark}, joka
tekee näkymättömän, juuri tähän tarkoitukseen olevan merkin
\uctunnus{u+200c zero width non-joiner}.

\komentoi{textcompwordmark}
\begin{koodilohkosis}
fi f\textcompwordmark i
\end{koodilohkosis}

\begin{tulossis}
  fi f\textcompwordmark i
\end{tulossis}

\noindent
Mikäli tavallisia ligatuureja ei halua käyttöön lainkaan, on parasta
valita jo kirjainperheen tai \=/leikkauksen käyttöönotossa asetus
\koodi{Liga\-tures=\katk No\-Com\-mon}. Väliaikaisesti fontin asetuksiin
voi vaikuttaa komennolla \komento{addfontfeatures}, joka sekin on
\paketti{fontspec}\-/pakettiin sisältyvä komento.

\komentoi{addfontfeatures}
\begin{koodilohkosis}
{\addfontfeatures{Ligatures=NoCommon} fi fl}
\end{koodilohkosis}

\noindent
Tavallisten ligatuurien lisäksi \englanti{Open Type} \=/fontit voivat
sisältää myös harvinaisempia ligatuureja, jotka täytyy erikseen kytkeä
päälle. Niiden tarkoituksena on esimerkiksi erityinen koristeellisuus
tai historiallisen kirjainleikkauksen jäljittely. Harvinaisia
ligatuureja ei ole tarkoitus käyttää joka tilanteessa vaan ainoastaan
erityisestä syystä. Kuvassa \ref{kuva/ligatuurit} ja tässäkin
tekstikappaleessa olevat {%
  \newcommand{\hlig}[1]{{\addfontfeatures{Ligatures=Historic}#1}}%
  \hlig{st}- ja \hlig{ct}\-/ligatuurit%
} on saatu fontin asetuksella \koodi{Liga\-tures=\katk His\-toric}.
Joissakin fonteissa sama tehdään asetuksella \koodi{Liga\-tures=\katk
  Discretionary}. Nämä ligatuurit sopivat niin sanotun humanistisen
antiikvaperheen (mm. \englanti{Adobe Jenson}) ja renessanssityylisen
sisällön kanssa käytettäväksi.

Teknisesti on mahdollista kytkeä päälle useitakin ligatuurityyppejä
samanaikaisesti:

\begin{koodilohkosis}
Ligatures={TeX, Required, Common, Historic, Discretionary,
  Contextual}
\end{koodilohkosis}

\noindent
Tuskin mikään fontti sisältää kaikkia ligatuurityyppejä, eikä se ole
tarkoituskaan. Jotkin ligatuurit kuuluvat vain tiettyyn typografian
aikakauteen tai kirjaintyyliin. Open Type \=/fontin ominaisuuksia voi
selvittää käyttöjärjestelmän komentotulkissa komennolla \koodi{otfinfo}.
Komennon argumentiksi annetaan muun muassa fonttitiedoston nimi.

\subsection{Numeroiden muoto}
\label{luku/fontit-numerot}

Varsinkin kirjatypografiassa on tavallista käyttää pääasiassa
gemenanumeroita ({\gemenanum 1967}) eikä versaalinumeroita
({\versaalinum 1967}), koska gemenanumerot sopivat leipätekstin
gemenakirjainten kanssa paremmin yhteen. Versaalinumerot puolestaan
erottuvat paremmin ja sopivat esimerkiksi taulukoihin, joissa on paljon
lukuja. Näihin asetuksiin vaikutetaan fontin asetuksella
\koodi{Numbers}. Tosin kaikissa fonteissa ei gemenanumeroita edes ole.

\begin{koodilohkosis}
Numbers=Lowercase % gemenanumerot
Numbers=Uppercase % versaalinumerot (oletus)
\end{koodilohkosis}

\noindent
Jotkin koodi\-/ilmaukset sisältävät versaalikirjaimia ja numeroita
sekaisin. Gemenanumerot eivät kuitenkaan sovi yhteen versaalikirjainten
kanssa, koska merkkien kokoero on häiritsevän suuri. Ei siis näin:
RJ45, R2D2. Gemenanumeroiden kanssa täytyy käyttää pienversaalia eli
kapiteelia: \textsc{rj45}, \textsc{r2d2}. Toinen vaihtoehto on käyttää
pelkästään versaaleja: {\versaalinum RJ45, R2D2}.

Silloin kun käytetään numeroita teknisten koodien ilmaisemiseen, voi
olla tarpeen merkitä numero nolla poikkiviivalla
({\addfontfeatures{Numbers={Uppercase, SlashedZero}} 0}), jottei se
sekoitu O\=/kirjaimeen. Nollaan saadaan poikkiviiva seuraavalla
asetuksella, jos vain fontissa on tämä ominaisuus:

\begin{koodilohkosis}
Numbers=SlashedZero
\end{koodilohkosis}

\noindent
Kun numeroita ladotaan taulukkoon, voidaan haluta käyttää tasalevyisiä
numeroita, jotta ne sijoittuvat allekkain samalle linjalle. Joissakin
fonteissa on tarjolla tavallisten vaihtelevan levyisten numeroiden
lisäksi myös tasalevyiset. Numeroiden leveyteen vaikutetaan seuraavilla
asetuksilla:

\begin{koodilohkosis}
Numbers=Monospaced   % tasalevyiset numerot
Numbers=Proportional % vaihtelevan levyiset numerot (oletus)
\end{koodilohkosis}

\subsection{Välistykset: harvennus ja tiivistys}
\label{luku/fontit-välistys}

Fontin kanssa käytettäviä sanavälejä voi säätää \koodi{Word\-Space}\-/
valitsimella, jonka arvoksi annetaan desimaalilukukerroin eli suhdeluku
normaaliin verrattuna. Arvona voi olla myös kolme eri kerrointa, jolloin
mukana on lisäksi sanavälin venymisen rajat. Tämä asetus on tehtävä
kirjainperheen tai \=/leikkauksen määrittelyn yhteydessä, eli se ei
toimi esimerkiksi \komento{addfontfeatures}\-/ komennon kanssa. Katso
tietoa sanaväleistä myös luvusta \ref{luku/sanaväli} tai
tekstikappaleiden latomista käsittelevästä luvusta \ref{luku/kappale}.
\noclub[2]

Fonttiasetusten valitsimella \koodi{Letter\-Space} säädetään merkkien
välistystä eli niiden välistä tyhjää tilaa. Arvoksi annetaan
positiivinen tai negatiivinen prosenttiluku, joka ilmaisee alkuperäiseen
lisättävän osuuden. Asetus \koodi{Letter\-Space=\katk 6} tuottaa siis
kuusi prosenttia leveämmät merkkien välit. Välistyksen tiivistäminen ja
harvennuksen typografiaa käsitellään luvussa
\ref{luku/korostus-harvennus}, mutta seuraavassa on esimerkki, kuinka ne
teknisesti toteutetaan:

\komentoi{scshape}
\komentoi{addfontfeatures}
\begin{koodilohkosis}
\scshape {\addfontfeatures{LetterSpace=-2} tiivistys} \\
normaali \\ {\addfontfeatures{LetterSpace=6} harvennus}
\end{koodilohkosis}

\begin{tulossis}
  \scshape {\addfontfeatures{LetterSpace=-2} tiivistys} \\
  normaali \\ {\addfontfeatures{LetterSpace=6} harvennus}
\end{tulossis}

\noindent
Jos esimerkiksi harvennuksen haluaa automaattisesti mukaan
kirjainperheeseen kuuluvaan pienversaaliin, käytetään valitsinta
\koodi{Small\-Caps\-Features}:

\komentoi{setmainfont}
\begin{koodilohkosis}
\setmainfont{…}[SmallCapsFeatures={LetterSpace=6}]
\end{koodilohkosis}

\noindent
\koodi{Letter\-Space}\-/ ominaisuus ei ole aina toiminut kunnolla
kaikilla Lualatexin fontinladontatekniikoilla. Jos merkkien välistys
toimii epätasaisesti, kannattaa kokeilla vaihtaa tekniikkaa
\koodi{Renderer}\-/ valitsimella. Lisätietoa on luvussa
\ref{luku/luatex-xetex-fonttitekn}.

\subsection{Keinotekoinen venytys, lihavointi ja kallistus}
\label{luku/fontit-venytys}

Valitsimella \koodi{Fake\-Stretch} voi venyttää tai kutistaa merkkejä
leveyssuunnassa. Arvoksi annetaan desimaalilukukerroin, joka on
suhdeluku alkuperäiseen leveyteen nähden. Seuraavassa havainnollistava
esimerkki:

\komentoi{addfontfeatures}
\begin{koodilohkosis}
esimerkki \\
{\addfontfeatures{FakeStretch=1.3} esimerkki}
\end{koodilohkosis}

\begin{tulossis}
  esimerkki \\
  {\addfontfeatures{FakeStretch=1.3} esimerkki}
\end{tulossis}

\noindent
Kirjainleikkauksen leventäminen tekee tekstistä samalla lihavampaa,
koska kirjainten viivat tulevat vahvemmiksi leveyssuunnassa.
Korkeussuunnassa vahvuus säilyy ennallaan, joten kirjainten
viivakontrasti muuttuu. Kirjainleikkauksen kutistaminen vaikuttaa
lihavuuteen päinvastaisesti. Tällainen keinotekoinen fontin
korjaileminen ei välttämättä tuota typografisesti kovin hyvää jälkeä
mutta voi sopia lievästi käytettynä esimerkiksi otsikoihin. Tämän oppaan
tasalevyistä kirjainperhettä on hieman kutistettu, koska alkuperäinen on
kohtuuttoman leveä ({\ttfamily\addfontfeatures{FakeStretch=1}
  esimerkki}) verrattuna muiden kirjainperheiden leveyteen.

Jos kirjainperheeseen ei sisälly sopivaa lihavoitua tai kaltevaa
leikkausta, voi sellaiset koettaa tehdä myös keinotekoisesti
\koodi{Fake\-Bold}- ja \koodi{Fake\-Slant}\-/ valitsimien avulla.
Niille annetaan arvoksi desimaalilukukerroin. Valitettavasti
\koodi{Fake\-Bold} toimii vain Xelatex\-/kääntäjän kanssa.

\komentoi{addfontfeatures}
\begin{koodilohkosis}
pysty {\addfontfeatures{FakeSlant=.2} kalteva}
\end{koodilohkosis}

\begin{tulossis}
  pysty {\addfontfeatures{FakeSlant=.2} kalteva}
\end{tulossis}

\noindent
Keinotekoisen lihavoinnin tai kallistuksen saa osaksi kirjainperhettä,
kun asettaa kirjainperheen määrittelyn yhteydessä halutun
kirjainleikkauksen valitsimella \koodi{Bold\-Font} tai
\koodi{Slanted\-Font} ja määrittää niille erityisiä ominaisuuksia
valitsimella \koodi{Bold\-Fea\-tures} tai \koodi{Slanted\-Fea\-tures}.
Seuraava esimerkki asettaa dokumentin perusfontille kaltevan
leikkauksen. Tämän jälkeen komennot \komento{slshape} ja
\komento{textsl} valitsevat \textsl{kaltevan} leikkauksen, joka on siis
eri asia kuin \textit{kursiivi} (\komento{itshape}, \komento{textit}).

\komentoi{setmainfont}
\begin{koodilohkosis}
\setmainfont{TeX Gyre Termes}[           % kirjainperhe
  SlantedFont={TeX Gyre Termes Regular}, % pystyasentoinen leikkaus
  SlantedFeatures={FakeSlant=.2}]        % keinotekoinen kallistus
\end{koodilohkosis}

\subsection{Keinotekoinen pienversaali}
\label{luku/fontit-keinopienversaali}

Moniin fontteihin ei sisälly lainkaan pienversaalia. Jos sellaisen silti
haluaa omaan korostusvalikoimaan, voi yrittää pienentää
versaalikirjaimet sopivaan kokoon. Seuraavassa esimerkissä versaaleja
ensin pienennetään ja sitten venytetään hieman leveyssuunnassa. Lisäksi
merkkivälejä harvennetaan lievästi. Käytännössä versaaleja pienentämällä
ei saada tyylikästä pienversaalia aikaan, koska merkkien viivoista tulee
liian ohuita.

\komentoi{textsc}
\komentoi{addfontfeatures}
\begin{koodilohkosis}
\textsc{pienversaali} oikea \\
{\addfontfeatures{ScaleAgain=.68, FakeStretch=1.17, LetterSpace=4}
  PIENVERSAALI} keinotekoinen
\end{koodilohkosis}

\begin{tulossis}
  \textsc{pienversaali} oikea \\
  {\addfontfeatures{ScaleAgain=.68, FakeStretch=1.17, LetterSpace=4}
    PIENVERSAALI} keinotekoinen
\end{tulossis}

\subsection{Matematiikkatilan fontti}
\label{luku/matematiikka-fontit}

Latexin matematiikkatilan (luku \ref{luku/matematiikka}) fontti
asetetaan eri tavalla kuin normaalin tilan eli tekstitilan fontit.
Matematiikkatilaa varten tarvitaan paketti \pakettictan{unicode-math},
joka täytyy ladata muiden matematiikkaan tai fontteihin liittyvien
pakettien jälkeen. Paketti tuo komennon \komento{setmathfont}, joka on
matematiikkatilan vastine luvussa \ref{luku/fontin-valinta} esitellyille
fontinmäärittelykomennoille \komento{setmainfont}, \komento{setsansfont}
ja \komento{setmonofont}. Lähes aina on järkevää käyttää
matematiikkatilan fontille asetusta \koodi{Scale=\katk MatchLowercase}.
Se asettaa fontin samankokoiseksi kuin vastaava peruskirjainperhe.

\pakettii{fontspec}
\pakettii{unicode-math}
\komentoi{setmainfont}
\komentoi{setmathfont}
\begin{koodilohkosis}
\usepackage{fontspec}
\usepackage{unicode-math}
\setmainfont{TeX Gyre Pagella}     % tekstitilan perusfontti
\setmathfont{TeX Gyre Pagella Math}[Scale=MatchLowercase]
\end{koodilohkosis}

\leijutlk{
  \providecommand{\rivi}{}
  \renewcommand{\rivi}[3]{\koodi{#1} & \koodi{#2} & \koodi{#3} \\}

  \begin{tabular}{lll}
    \toprule
    \ots{Yleistyyli} & \ots{Lihavointi} & \ots{Pääteviivaton} \\
    \midrule
    \rivi{math-style=ISO}    {bold-style=ISO}    {sans-style=upright}
    \rivi{math-style=TeX}    {bold-style=TeX}    {sans-style=italic}
    \rivi{math-style=french} {bold-style=upright}{sans-style=literal}
    \rivi{math-style=upright}{}{}
    \rivi{math-style=literal}{}{}
    \bottomrule
  \end{tabular}
}{
  \caption{Matematiikkatilan tyyliasetuksia (\paketti{unicode-math}\-/
    paketti)}
  \label{tlk/unicode-math-style}
}

\noindent
Paketin \paketti{unicode-math} lataamisen yhteydessä voi
\komento{usepackage}\-/ komennon valinnaisen argumentin avulla vaikuttaa
matematiikkatilassa käytettyihin kirjainleikkauksiin. Argumenttiin
sopivia valitsimia ovat \koodi{math-style}, \koodi{bold-style} ja
\koodi{sans-style}, ja niille sopivia arvoja on koottu taulukkoon
\ref{tlk/unicode-math-style}. Asetukset vaikuttavat esimerkiksi siihen,
mitä kirjainleikkausta käytetään, kun ladotaan matematiikkatilan
latinalaisia tai kreikkalaisia kirjaimia.

\begin{koodilohkosis}
\usepackage[math-style=ISO]{unicode-math}
\end{koodilohkosis}

\noindent
Matematiikkatilan fontiksi ei kelpaa mikä tahansa, koska tarvittavien
matemaattisten symbolien täytyy sisältyä \englanti{Open Type} \=/fontin
merkkivalikoimaan. Matematiikkatukea ei todennäköisesti ole, jos sitä ei
fontin ominaisuuksien yhteydessä mainita. Fonttitiedoston ominaisuuksia
voi tutkia esimerkiksi käyttöjärjestelmän komentotulkissa
\koodi{otfinfo}\-/ komennolla:

\begin{koodilohkosis}
otfinfo -s texgyrepagella-math.otf
\end{koodilohkosis}

\leijutlk{
  \providecommand{\rivi}{}
  \renewcommand{\rivi}[2]{#1 & $\longleftrightarrow$ & #2 \\}

  \begin{tabular}{r@{\enspace}c@{\enspace}l}
    \toprule
    \ots{Leipäteksti} && \ots{Matematiikkatila} \\
    \midrule
    \rivi{Libertinus Serif}{Libertinus Math}
    \rivi{TeX Gyre Bonum}{TeX Gyre Bonum Math}
    \rivi{TeX Gyre Pagella}{TeX Gyre Pagella Math}
    \rivi{TeX Gyre Schola}{TeX Gyre Schola Math}
    \rivi{TeX Gyre Termes}{TeX Gyre Termes Math}
    \rivi{DejaVu Serif}{TeX Gyre Dejavu Math}
    \rivi{Garamond Libre}{Garamond Math}
    \rivi{Latin Modern Roman}{Latin Modern Math}
    \bottomrule
  \end{tabular}
}{
  \caption{Ulkoasultaan yhteensopivia leipätekstin ja matematiikkatilan
    fontteja}
  \label{tlk/matematiikka-fontteja}
}

\noindent
Taulukkoon \ref{tlk/matematiikka-fontteja} on koottu \englanti{Open
  Type} \=/fontteja, joissa on yhteensopiva leipätekstin ja
matematiikkatilan kirjainperhe. Teknisesti kaikkia voi käyttää kaikkien
kanssa, mutta on tietenkin toivottavaa, että kirjainleikkaukset sopivat
ulkoasultaan yhteen. Kaikki fontit ovat vapaita ja toimitetaan Tex Live
\=/jakelun tai käyttöjärjestelmän mukana.

\section{Kieli}
\label{luku/kieliasetukset}

Melkein aina Latex\-/dokumenttiin täytyy ladata kielipaketti ja sen
mukana asetukset tiettyjä kieliä varten. Kieliasetukset sisältävät
ainakin tavutussäännöt (luku \ref{luku/tavutus}) sekä kielelle
mukautettuja nimiä dokumentin eri osille. Esimerkiksi sisällysluettelon
ja kirjallisuusluettelon otsikot tulevat kieliasetuksista, samoin
leijuvien taulukoiden ja kuvien nimet ''\tablename'' ja ''\figurename''.
Myös muita asetuksia tai komentoja saattaa tulla kieliasetusten mukana,
mutta ne vaihtelevat eri kielissä.

Kielipaketteja on kaksi -- \paketti{babel} ja \paketti{polyglossia} \==,
ja kirjoittajan täytyy valita niistä jompikumpi. Pakettien historia ja
kehitys on suunnilleen seuraavanlainen: \paketti{babel} on paljon
vanhempi, ja monet vanhat oppaat ja esimerkit käsittelevät pelkästään
sitä. Kun Latex siirtyi Unicode\-/aikaan Xelatex\-/kääntäjän ja
\paketti{fontspec}\-/paketin myötä, \paketti{babel} ei pysynyt
kehityksessä mukana. Syntyi \paketti{polyglossia}, joka hallitsi
Unicoden sekä muitakin kuin latinalaiseen kirjaimistoon perustuvia
kieliä ja kirjoitusjärjestelmiä. \paketti{babel}\-/paketin kehitys ei
kuitenkaan pysähtynyt, ja sittemmin myös se on kehittynyt Unicode\-/
aikakaudelle.

Suomen kielen kannalta ei ole merkitystä, kumpaa kielipakettia käyttää,
mutta meidän näkökulmastamme ''erikoisemmat'' kielet ja
kirjoitusjärjestelmät voivat vaatia selvittämistä, kumpi kielipaketti
soveltuu paremmin. \paketti{babel} sisältää enemmän ominaisuuksia,
esimerkiksi omien komentojen ja muiden viritysten tekemiseen;
\paketti{polyglossia} on yksinkertaisempi paketti, jonka kehitys
tuntuisi keskittyvän vain ydintehtävään.

Seuraavissa alaluvuissa käsitellään kummankin kielipaketin tärkeimmät
toiminnot eli kielen valintaan liittyvät asiat. Yleinen ajatus on se,
että Latex\-/dokumentin esittelyosassa ladataan kielipaketti ja
määritellään dokumentin pääasiallinen kieli ja mahdolliset muut kielet.
Jos dokumentin tekstiosa sisältää muita kuin pääasiallista kieltä,
täytyy käyttää erityisiä komentoja tai ympäristöjä, joilla kerrotaan
Latexille, mistä kielestä on kyse. Tavutusta käsitellään kummankin
kielipaketin osalta luvussa \ref{luku/tavutus}.

\subsection{Polyglossia}
\label{luku/polyglossia}

Kielipaketti \pakettictan{polyglossia} vaatii toimiakseen Lualatex\-/\
tai Xelatex\-/ kääntäjän. Se ei siis toimi perinteisillä Latexin
kääntäjillä. Kielipaketin käyttöönotto dokumentin esittelyosassa näyttää
esimerkiksi seuraavanlaiselta:

\komentoi{usepackage}
\komentoi{setdefaultlanguage}
\komentoi{setotherlanguage}
\pakettii{polyglossia}
\begin{koodilohkosis}
\usepackage{polyglossia}
\setdefaultlanguage{finnish}
\setotherlanguage{english}
\setotherlanguage{greek}
\end{koodilohkosis}

\noindent
Eri kielillä on erilaisia valinnaisia asetuksia, jotka täytyy selvittää
\paketti{polyglossia}\-/paketin ohjekirjasta. Asetukset liittyvät
esimerkiksi kielen kirjoitusjärjestelmän alueelliseen tai
historialliseen vaihteluun.

Dokumentin tekstiosa käyttää kieltä, joka määriteltiin komennolla
\komento{setdefaultlanguage}. Muita dokumentin esittelyosassa
määriteltyjä kieliä voi käyttää väliaikaisesti komennolla, joka alkaa
kirjaimilla \komentox{text} ja jatkuu kielen nimellä, esimerkiksi
\komento{textenglish} tai \komento{textgreek}.

\komentoi{textenglish}
\begin{koodilohkosis}
Englannin sana \textenglish{shorthand} tarkoittaa 'pikakirjoitusta'.
\end{koodilohkosis}

\noindent
Toinen vaihtoehto on käyttää kielen nimen mukaista ympäristöä:

\begin{koodilohkosis}
\begin{greek}
  ...
\end{greek}
\end{koodilohkosis}

\noindent
Tietynkieliseksi merkitty teksti voi näyttää ladotussa dokumentissa ihan
samalta kuin muukin teksti. Tavutuksessa kuitenkin käytetään eri
kielissä eri asetuksia, ja joissakin kielissä voi olla myös pieniä
typografisia yksityiskohtia eri tavalla.

Oletuksena kaikilla kielillä käytetään samoja fontteja, mutta
kirjoittaja voi määritellä fontin myös kielikohtaisesti, niin että
kielen vaihtuessa fonttikin vaihtuu automaattisesti. Tietyn kielen
fontti pitää määrittää komentoon, jonka nimessä on ensin kielen nimi ja
lopuksi sana \koodi{font} (peruskirjainperhe), \koodi{fontsf}
(pääteviivaton) tai \koodi{fonttt} (tasalevyinen). Esimerkiksi
fonttikomennot englannin kielelle ja kolmelle eri kirjainperheelle ovat
\komento{englishfont}, \komento{englishfontsf} ja
\komento{englishfonttt}.

Näiden fonttikomentojen ja kirjainperheiden määrittämiseen kannattaa
käyttää \paketti{fontspec}\-/ paketin komentoa \komento{newfontfamily}
(luku \ref{luku/fontin-valinta}). Seuraavassa esimerkissä asetetaan
kaikki kolme kirjainperhettä kreikan kielelle:

\komentoi{newfontfamily}
\komentoi{greekfont}
\komentoi{greekfontsf}
\komentoi{greekfonttt}
\begin{koodilohkosis}
\newfontfamily{\greekfont}  {GFS Artemisia}  [Scale=MatchLowercase]
\newfontfamily{\greekfontsf}{GFS Neohellenic}[Scale=MatchLowercase]
\newfontfamily{\greekfonttt}{TeX Gyre Cursor}[Scale=MatchLowercase]
\end{koodilohkosis}

\noindent
Kirjoittajan ei tarvitse itse käyttää edellä mainittuja komentoja kuten
\komento{englishfont}, \komento{greekfont} jne. Kirjoittaja käyttää vain
kielenvaihtokomentoja, ja fonttikin vaihtuu samalla itsestään, jos vain
kohdekielelle ja \=/kirjainperheelle on määritelty sopiva fonttikomento.

\subsection{Babel}
\label{luku/babel}

Vanha kunnon \pakettictan{babel}\yipilkku\avctan{babel-finnish} toimii
useiden eri Latex\-/ kääntäjien kanssa. Ennen dokumentit kirjoitettiin
rajallisilla merkistöillä (kuten \textsc{iso-8859\=/1}), jotka
sisältävät vain reilut kaksisataa ihmiskielten kirjoitusmerkkiä.
\paketti{babel} kuitenkin toimii myös Lualatex\-/\ ja Xelatex\-/
kääntäjillä eli Unicode\-/ merkistön kanssa.

Kieliasetukset otetaan käyttöön \paketti{babel}\-/paketissa seuraavan
esimerkin tavoin. Paketin lataamisessa valinnaiseen argumenttiin
kirjoitetaan ladattavien kielten nimet ja viimeisenä mainitaan se kieli,
joka halutaan pääasialliseksi kieleksi. Toisaalta pääasiallisen kielen
voi valita myös \koodi{main}\-/ valitsimella. Seuraavassa on esimerkki
kummastakin kieltenvalintatavasta:

\komentoi{usepackage}
\pakettii{babel}
\begin{koodilohkosis}
\usepackage[english,greek,finnish]{babel}
\usepackage[main=finnish,english,greek]{babel}
\end{koodilohkosis}

\noindent
Dokumentin tekstissä yksittäiset vieraskieliset sanat tai ilmaukset
merkitään komennolla \komento{foreignlanguage}. Komennon ensimmäinen
argumentti on kielen nimi ja toinen on sillä kielellä ladottava teksti.

\komentoi{foreignlanguage}
\begin{koodilohkosis}
Englannin sana \foreignlanguage{english}{shorthand} tarkoittaa
'pikakirjoitusta'.
\end{koodilohkosis}

\noindent
Komennon sijasta voi käyttää ympäristöä \ymparisto{otherlanguage}, joka
vaihtaa kieltä ympäristön ajaksi.

\ymparistoi{otherlanguage}
\begin{koodilohkosis}
\begin{otherlanguage}{greek}
  ...
\end{otherlanguage}
\end{koodilohkosis}

\noindent
Kielen vaihtamista voi helpottaa \komento{babeltags}\-/ komennolla, joka
määrittelee lyhempiä komentoja ja ympäristöjä kielen vaihtamiseen.
Komentoa käytetään seuraavan esimerkin tavoin. Sen jälkeen voi käyttää
englanninkielisen tekstin merkitsemiseen komentoa \komentox{text\-eng}
tai ympäristöä \ymparistox{eng}.

\komentoi{babeltags}
\begin{koodilohkosis}
\babeltags{eng = english}
\end{koodilohkosis}

\noindent
Oletuskieli voidaan vaihtaa kesken dokumentin komennolla
\komento{selectlanguage}. Komennon argumentiksi annetaan kielen nimi,
esimerkiksi seuraavalla tavalla:

\komentoi{selectlanguage}
\begin{koodilohkosis}
\selectlanguage{english}
\end{koodilohkosis}

\noindent
Eri kielille on mahdollista asettaa eri kirjainperheet
\komento{babelfont}\-/ komennolla. Tämä komento korvaa
\paketti{fontspec}\-/ paketin fontinvalintakomennot (luku
\ref{luku/fontin-valinta}) ja muodostaa uuden korkeamman tason komennon,
joka sisältää myös kielen. Jos siis käyttää komentoa
\komento{babelfont}, ei saa samanaikaisesti käyttää komentoja
\komento{setmainfont}, \komento{setsansfont} eikä \komento{setmonofont}.

Oman dokumentin kirjainperheet voitaisiin määrittää
\komento{babelfont}\-/ komennolla esimerkin \ref{esim/babelfont} tavoin.
Ensin esimerkissä määritellään peruskirjainperhe (\koodi{rm}),
pääteviivaton perhe (\koodi{sf}) ja tasalevyinen perhe (\koodi{tt}). Sen
jälkeen määritellään kreikan kielessä (\koodi{greek}) käytettävä
peruskirjainperhe ja pääteviivaton perhe.

\begin{esimerkki*}
  \komentoi{babelfont}

\begin{koodilohko}
\babelfont{rm}{TeX Gyre Termes}
\babelfont{sf}[Scale=MatchLowercase]{TeX Gyre Heros}
\babelfont{tt}[Scale=MatchLowercase]{TeX Gyre Cursor}

\babelfont[greek]{rm}[Scale=MatchLowercase]{GFS Artemisia}
\babelfont[greek]{sf}[Scale=MatchLowercase]{GFS Neohellenic}
\end{koodilohko}
  \caption{\komento{babelfont}\-/komennon käyttö dokumentin
    kirjainperheiden valintaan ja kielikohtaisten kirjainperheiden
    valintaan}
  \label{esim/babelfont}
\end{esimerkki*}

Esimerkissä ennen fontin nimeä (esim. TeX Gyre Heros) oleva valinnainen
argumentti on sama kuin \paketti{fontspec}\-/paketin
fontinvalintakomennoissa. Sen avulla määritellään kyseisen
kirjainperheen asetuksia. Lisätietoa voi lukea fontteja käsittelevästä
luvusta \ref{luku/kirjaintyypit} ja \pakettictan{fontspec}\-/ paketin
ohjekirjasta.

\section{Tavutus}
\label{luku/tavutus}

Sanojen tavutus eli katkaiseminen rivin lopussa tavurajan kohdalta
kytkeytyy läheisesti kieliasetuksiin (luku \ref{luku/kieliasetukset}),
joihin on syytä perehtyä ennen tämän luvun lukemista.

Tex tavuttaa eli katkaisee sanat automaattisesti rivien lopussa, jotta
se saisi tekstikappaleet näyttämään tasapainoisilta. Tavutus perustuu
yksinkertaisiin kirjainpohjaisiin sääntöihin: millaisia kirjainten
yhdistelmiä tavuissa voi olla. Tavutussäännöt määräytyvät valitun kielen
perusteella ja joskus kieleen liittyvien muiden asetusten perusteella.

Automaattinen tavutus auttaa paljon, mutta se ei yksinään riitä. Se
tavuttaa välillä kielen kannalta väärin tai tuottaa muuten suositusten
vastaista jälkeä. Niinpä kirjoittajan täytyy auttaa välillä eli
kirjoittaa tavutusvihjeitä. Texissä itsessään on tavutukseen vaikuttavia
erikoismerkkejä ja toimintoja, mutta myös kielipaketeissa
\paketti{polyglossia} ja \paketti{babel} on omia tavutukseen vaikuttavia
erikoisuuksiaan, jotka voivat olla eri kielissä erilaisia.
Kielipakettien ominaisuuksia käsitellään luvuissa
\ref{luku/tavutus-polyglossia} ja \ref{luku/tavutus-babel}.

\subsection{Yleiset tavutusvihjeet}
\label{luku/hyphenation-komento}

Yksi tapa tavutusvihjeiden kirjoittamiseen on \komento{hyphenation}\-/
komento, jolla määritellään yksittäisten sanojen tavutuskohdat
kaikkialla dokumentissa. Seuraava esimerkki selventää komennon käyttöä:

\komentoi{hyphenation}
\begin{koodilohkosis}
\hyphenation{
  ala-indek-si alku-osa nimen-omaan
  typo-gra-fi-nen Latex
}
\end{koodilohkosis}

\noindent
Komennon argumentiksi eli aaltosulkeiden sisään kirjoitetaan sanoja,
jotka erotetaan toisistaan sanaväleillä. Sanoihin kirjoitetaan
yhdysmerkki niihin kohtiin, joista sanan katkaiseminen on sallittua. Jos
yhdysmerkkiä ei ole, sanaa ei tavuteta mistään kohdasta. Jos taas sana
on yhdyssana ja sen osien väliin kuuluu on yhdysmerkki, ei sen tavutusta
voi käsitellä tällä komennolla. Luvussa \ref{luku/tavutuksen-merkit}
kerrotaan muita tapoja.

\komento{hyphenation}\-/komennon voi sijoittaa dokumentin esittelyosaan
tai tekstiosaan, mutta sijainti vaikuttaa sen toimintaan. Jos komennon
sijoittaa dokumentin esittelyosaan ennen kuin mitään kieltä on ladattu
tai valittu, se vaikuttaa kaikkien sanojen tavutukseen kielestä
riippumatta. Jos komennon sijoittaa dokumentin tekstiosaan eli kielen
valitsemisen jälkeen, se vaikuttaa vain kyseisen kielen eli yleensä
dokumentin pääasiallisen kielen tavutukseen. Kielipaketit
\paketti{polyglossia} (luku \ref{luku/tavutus-polyglossia}) ja
\paketti{babel} (luku \ref{luku/tavutus-babel}) sisältävät komennon,
joilla voi asettaa yleisiä kielikohtaisia tavutusvihjeitä.

\subsection{Yksittäisten sanojen tavutus}
\label{luku/tavutuksen-merkit}

Tietyt sanassa mukana olevat merkit kytkevät tavalliset tavutussäännöt
pois päältä. Jos sanassa on mukana yksikin yhdysmerkki (\koodi{\=/}),
lyhyt ajatusviiva (\mbox{\koodi{--}}) tai pitkä ajatusviiva
(\mbox{\koodi{---}}), sana katkaistaan vain näiden merkkien jälkeen, eli
muut tavutuskohdat kytketään pois käytöstä.

Unicode\-/ merkistön ajastusviivat \uctunnus{u+2013 en dash} ja
\uctunnus{u+2014 em dash} ovat toimineet tavutuksessa eri tavalla
Lualatex\-/{} ja Xelatex\-/ kääntäjissä ja mahdollisesti eri tavalla
kuin edellä mainitut Texin omat merkintätavat. Yhteensopivuussyistä on
parasta kirjoittaa ajatusviivat Texin merkintätavoilla.

Myös tavutusvihje (\komento{-}) vaikuttaa sanan tavutuskohtiin: jos
sanassa on mukana yksikin tavutusvihje, tavalliset tavutuskohdat
kytkeytyvät pois käytöstä, ja sana katkaistaan vain tavutusvihjeiden
kohdalta sekä edellä mainittujen viivavälimerkkien jälkeen. Kielipaketti
\paketti{babel} (luku \ref{luku/tavutus-babel}) käyttäytyy kuitenkin
suomen kielessä eri tavoin. Se määrittelee ainakin suomen kielessä
tavutusvihjeen siten, että se sallii sanan tavutuksen muistakin kohdista
kuin tavutusvihjeen kohdalta.

\subsection{Tavutus sanan reunasta}
\label{luku/tavutus-reunasta}

Asetukset \komento{lefthyphenmin}\komentojatko{=N} ja
\komento{righthyphenmin}\komentojatko{=N} vaikuttavat tavutukseen sanan
reunoissa. Argumentti \koodi{N} on positiivinen kokonaisluku, joka
määrittelee, kuinka monta merkkiä vähintään sanan vasemmasta tai
oikeasta reunasta pidetään yhdessä. Oletusarvot ovat kielikohtaisia ja
ne on määritelty \paketti{polyglossia}\-/{} ja \paketti{babel}\-/
paketeissa. Suomen kielessä kumpikin asetus on kaksi (2) merkkiä.

Nämä asetukset alustetaan automaattisesti kielikohtaisiin oletusarvoihin
aina, kun kieliasetukset tulevat voimaan. Näin on esimerkiksi dokumentin
aloittavan \ymparistox{document}\-/ ympäristön alussa ja aina kielen
vaihtuessa. Jos kirjoittaja haluaa muuttaa asetuksia, täytyy omat
muutokset tehdä joka kerta edellä mainittujen asioiden jälkeen.
Toisaalta kätevämpää on sisällyttää omat asetukset kielikohtaisiin
alustuskomentoihin. Sillä tavoin omat asetukset tulevat voimaan
automaattisesti kielen vaihtuessa.

\komentoi{addto}
\komentoi{captionsfinnish}
\komentoi{lefthyphenmin}
\komentoi{righthyphenmin}
\begin{koodilohkosis}
\addto{\captionsfinnish}{
  \lefthyphenmin=3 \righthyphenmin=2
}
\end{koodilohkosis}

\noindent
Edellisessä esimerkissä oleva komento \komento{addto} on
\paketti{polyglossia}\-/{} ja \paketti{babel}\-/ paketin ominaisuus,
jolla lisätään omia komentoja kielikohtaisiin asetuksiin. Tässä
esimerkissä käsitellään suomen kielen asetuksia
(\komento{captionsfinnish}). Komento täytyy suorittaa lähdedokumentin
esittelyosassa.

\subsection{Polyglossia ja tavutus}
\label{luku/tavutus-polyglossia}

Kielipaketti \pakettictan{polyglossia} (myös luku
\ref{luku/polyglossia}) sisältää tavutukseen vaikuttavia asetuksia,
jotka ovat eri kielissä erilaisia. Suomen kielessä on hyödyllistä ottaa
käyttöön asetus \koodi{babelshorthands}, joka tuo mukaan muutaman
tavutukseen vaikuttavan erikoismerkin.

\komentoi{setdefaultlanguage}
\begin{koodilohkosis}
\setdefaultlanguage[babelshorthands]{finnish}
\end{koodilohkosis}

\noindent
Asetus \koodi{babelshorthands} lisää erikoismerkkivalikoimaan seuraavat
tavutukseen vaikuttavat merkit. Taulukossa \ref{tlk/polyglossia-tavutus}
on esimerkkejä tavutusvihjeiden, yhdysmerkin ja ajatusviivan sekä
\paketti{polyglossia}\-/ paketin erikoismerkkien vaikutuksesta.

\newcommand{\shorthandsyhdysmerkki}{%
\item [\textquotedbl-] Sitova yhdysmerkki (\=/), joka sitoo merkin
  seuraavaan sanaan eli estää katkaisemisen yhdysmerkin jälkeen. Tämä
  merkki sallii sanan tavuttamisen muista kohdista. Tämän sijasta voi
  käyttää Unicode\-/ merkistön sitovaa yhdysmerkkiä \uctunnus{u+2011
    non\-/ breaking hyphen}.}

\newcommand{\shorthandspystyviiva}{%
\item [\textquotedbl|] Estää typografisten ligatuurien (luku
  \ref{luku/typo-liga}) muodostumisen tähän kohtaan. Esimerkiksi fi\-/
  ligatuurin muodostumisen voi estää kirjoittamalla \koodi{f"|i}, ja
  tuloksena on kaksi erillistä merkkiä: f\textcompwordmark i.}

\newcommand{\shorthandslainausmerkki}{%
\item [\textquotedbl\textquotedbl] Tavutusvihje, joka ei lisää
  tavutuskohtaan yhdysmerkkiä. Tämä on hyödyllinen esimerkiksi
  teknisissä ilmauksissa, joissa sanaan tarvitaan mahdollinen
  katkaisukohta mutta ei haluta lisätä ylimääräistä yhdysmerkkiä
  sotkemaan ilmausta.}

\begin{maaritelma}{\koodi{#1}}
  \shorthandsyhdysmerkki

% Polyglossian tulevassa versiossa (jälkeen 30.12.2023).
% \item [\textquotedbl=] Sama kuin edellinen (\koodi{\textquotedbl-}).

  \shorthandspystyviiva

  \shorthandslainausmerkki

\item [\textquotedbl/] Vinoviiva (/), jonka jälkeen on tavutusvihje.
  Tavutuskohtaan ei lisätä yhdysmerkkiä. Tavutus sallitaan sanan muista
  kohdista.
\end{maaritelma}

\leijutlk{
  \begin{tabular}{lll}
    \toprule
    \ots{Lähde}
    & \ots{Tavutus}
    & \ots{Merkitys} \\
    \midrule

    \koodi{erikoisalalla}
    & eri\tavukohta koi\tavukohta sa\tavukohta lal\tavukohta la
    & tavutus kaikista kohdista (väärin) \\

    \koodi{erikois\komento{-}alalla}
    & erikois\tavukohta alalla
    & vain tavutusvihjeen kohdalta \\

    \koodi{matka-aika}
    & matka-\tavukohta aika
    & vain yhdysmerkin jälkeen \\

    \koodi{matka-ai\komento{-}ka}
    & matka-\tavukohta ai\tavukohta ka
    & vain yhdysmerkki ja tavutusvihje \\

    \koodi{Oulu--Rovaniemi}
    & Oulu--\tavukohta Rovaniemi
    & vain ajatusviivan jälkeen \\

    \koodi{Oulu--Rova\komento{-}niemi}
    & Oulu--\tavukohta Rova\tavukohta niemi
    & vain ajatusviiva ja tavutusvihje \\

    \koodi{matka-}
    & matka-
    & ei tavutuskohtia \\

    \koodi{-aika}
    & -\tavukohta aika
    & vain yhdysmerkin jälkeen (väärin) \\

    \koodi{\textquotedbl-aika}
    & -ai\tavukohta ka
    & ei sitovan yhdysmerkin jälkeen \\

    \bottomrule
  \end{tabular}
}{
  \caption{Kielipaketti \paketti{polyglossia} ja tavutusvihjeen,
    yhdysmerkin ja ajatusviivan sekä kielipaketin omien erikoismerkkien
    (\koodi{babelshorthands}) vaikutus tavutukseen}
  \label{tlk/polyglossia-tavutus}
}

\noindent
Yleisiä koko kieleen vaikuttavia tavutusvihjeitä voi määritellä
komennolla \komento{pghyphenation}. Se vastaa Texin
\komento{hyphenation}\-/ komentoa (luku \ref{luku/hyphenation-komento}),
mutta tavutusvihjeet määritellään kielikohtaisesti. Komennon ensimmäinen
argumentti on kielen nimi, ja toiseen argumenttiin luetellaan sanat ja
niiden sallitut tavutuskohdat yhdysmerkin avulla.

\komentoi{pghyphenation}
\begin{koodilohkosis}
\pghyphenation{finnish}{
  ala-indek-si alku-osa nimen-omaan typo-gra-fi-nen Latex
}
\end{koodilohkosis}

\noindent
Kielikohtaisesti voi määrittää myös sen, kuinka monta kirjainta sanan
reunasta jätetään tavuttamatta. \paketti{polyglossia}\-/ paketissa on
komento \komento{setlanghyphenmins}, jonka ensimmäinen argumentti on
kielen nimi. Toinen ja kolmas argumentti ovat kokonaislukuja, joilla
asetetaan sanan vasemman ja oikean reunan tavuttamaton merkkimäärä.

\komentoi{setlanghyphenmins}
\begin{koodilohkosis}
\setlanghyphenmins{finnish}{3}{2}
\end{koodilohkosis}

\noindent
Suomen kielellä vaihtoehtoisen tavutussäännöstön voi kytkeä päälle
käyttämällä kielen asettamisen yhteydessä valitsinta
\koodi{schoolhyphens}. Tämä sallii tavutuksen kaikista mahdollisista
kohdista. Lisätietoa suomen tavutuksen erityispiirteistä on luvussa
\ref{luku/suomi-tavutus}.

\komentoi{setdefaultlanguage}
\begin{koodilohkosis}
\setdefaultlanguage[schoolhyphens]{finnish}
\end{koodilohkosis}

\subsection{Babel ja tavutus}
\label{luku/tavutus-babel}

Kielipaketti \pakettictan{babel}\yipilkku\avctan{babel-finnish} (myös
luku \ref{luku/babel}) lisää automaattisesti moniin kieliin -- myös
suomen kieleen -- muutaman erikoismerkin, joilla voi vaikuttaa
tavutukseen. Erikoismerkit saa pois päältä käyttämällä paketin
lataamisen yhteydessä valitsinta \koodi{shorthands=\katk off}.
Seuraavassa on selitetty suomen kielen erikoismerkkejä.

\begin{maaritelma}{\koodi{#1}}
\item [\komento{-}] Tämä on Texin normaali tavutusvihje (luku
  \ref{luku/tavutuksen-merkit}), mutta \paketti{babel} määrittelee sen
  suomen kielessä uudelleen siten, että tavutus sallitaan sanassa myös
  muista kohdista tavutussääntöjen mukaan.

  \shorthandsyhdysmerkki

\item [\textquotedbl=] Sama kuin edellinen (\koodi{\textquotedbl-}).

  \shorthandspystyviiva

  \shorthandslainausmerkki

\end{maaritelma}

\leijutlk{
  \begin{tabular}{lll}
    \toprule
    \ots{Lähde}
    & \ots{Tavutus}
    & \ots{Merkitys} \\
    \midrule

    \koodi{erikoisalalla}
    & eri\tavukohta koi\tavukohta sa\tavukohta lal\tavukohta la
    & tavutus kaikista kohdista (väärin) \\

    \koodi{erikois\komento{-}alalla}
    & eri\tavukohta kois\tavukohta alal\tavukohta la
    & tavutusvihjeen kohdalta ja muualta \\

    \koodi{matka-aika}
    & matka-\tavukohta aika
    & vain yhdysmerkin jälkeen \\

    \koodi{matka-ai\komento{-}ka}
    & matka-\tavukohta ai\tavukohta ka
    & vain yhdysmerkki ja tavutusvihje \\

    \koodi{Oulu--Rovaniemi}
    & Oulu--\tavukohta Rovaniemi
    & vain ajatusviivan jälkeen \\

    \koodi{Oulu--Rova\komento{-}niemi}
    & Oulu--\tavukohta Rova\tavukohta nie\tavukohta mi
    & tavutusvihjeen kohdalta ja muualta \\

    \koodi{matka-}
    & matka-
    & ei tavutuskohtia \\

    \koodi{-aika}
    & -\tavukohta aika
    & vain yhdysmerkin jälkeen (väärin) \\

    \koodi{\textquotedbl-aika}
    & -ai\tavukohta ka
    & ei sitovan yhdysmerkin jälkeen \\

    \bottomrule
  \end{tabular}
}{
  \caption{Kielipaketti \paketti{babel} ja suomen kielellä
    tavutusvihjeen, yhdysmerkin ja ajatusviivan sekä kielipaketin omien
    erikoismerkkien vaikutus tavutukseen}
  \label{tlk/babel-tavutus}
}

\noindent
Taulukossa \ref{tlk/babel-tavutus} on esimerkkejä suomen kielen
tavutuksesta, kun sanassa on tavutusvihjeitä, yhdysmerkkejä,
ajatusviivoja tai \paketti{babel}\-/ paketin erikoismerkkejä.

Taulukon viimeinen rivi soittaa, että kun sanan alussa on yhdysmerkki,
täytyy suomen kielessä käyttää sitovaa yhdysmerkkiä, joka estää
tavutuksen heti yhdysmerkin jälkeen. Toisaalta \paketti{babel}\-/
paketissa on suomen kielelle asetus, joka käsittelee sanan alussa olevan
yhdysmerkin automaattisesti oikein. Ominaisuus kytketään päälle
kielipaketin lataamisen jälkeen \komento{babelprovide}\-/ komennolla
seuraavasti:

\komentoi{usepackage}
\pakettii{babel}
\komentoi{babelprovide}
\begin{koodilohkosis}
\babelprovide[transforms=prehyphen.nobreak]{finnish}
\end{koodilohkosis}

\noindent
Kun edellä mainittu asetus on päällä, voi lähdetiedostoon kirjoittaa
huoletta esimerkiksi \koodi{matkasuunnitelma ja \=/aika} (ilman sitovaa
yhdysmerkkiä), ja silti sanaan \emph{\mbox{-aika}} sisältyvä yhdysmerkki
pysyy kiinni sanassa eikä jää koskaan yksin rivin loppuun.

Yleisiä kaikkiin kieliin tai tiettyyn kieleen vaikuttavia
tavutusvihjeitä voi asettaa komennolla \komento{babelhyphenation}.
Komento vastaa Texin \komento{hyphenation}\-/ komentoa (luku
\ref{luku/hyphenation-komento}), mutta sille voi antaa valinnaisen
argumentin, joka rajaa vaikutuksen vain tiettyyn kieleen. Seuraavassa
esimerkissä asetetaan tavutusvihjeitä vain suomen kielen sanoille:

\komentoi{babelhyphenation}
\begin{koodilohkosis}
\babelhyphenation[finnish]{
  ala-indek-si alku-osa nimen-omaan typo-gra-fi-nen Latex
}
\end{koodilohkosis}

\noindent
Vaihtoehtoisen suomen kielen tavutussäännöstön voi kytkeä päälle
komennolla \komento{babelprovide}, seuraavan esimerkin mukaisesti. Tämä
vaihtoehto sallii sanan katkaisemisen kaikista mahdollisista suomen
kielen tavutuskohdista. Lisätietoa suomen tavutuksen erityispiirteistä
on luvussa \ref{luku/suomi-tavutus}.

\komentoi{babelprovide}
\begin{koodilohkosis}
\babelprovide[hyphenrules=schoolfinnish]{finnish}
\end{koodilohkosis}

\subsection{Viivojen erikoispaketti extdash}
\label{luku/tavutus-extdash}

Jos Texin ja kielipakettien välimerkit ja tavutusvihjeet eivät riitä,
mahdollisuuksia saa lisää \pakettictan{extdash}\-/ paketin avulla. Se
tuo mukanaan uusia komentoja viivavälimerkeille ja samalla lisää
vaihtoehtoja tavutuksen hallintaan. Komennot ovat sellaisia kuin
\komento{Hyphdash} ja \komento{Endash}, mutta niille on saatavilla myös
lyhyemmät vastineet, jos paketin lataa käyttämällä
\koodi{short\-cuts}\-/ valitsinta.

\komentoi{usepackage}
\pakettii{extdash}
\begin{koodilohkosis}
\usepackage[shortcuts]{extdash}
\end{koodilohkosis}

\noindent
Paketti sisältää kaksi lisävaihtoehtoa kolmelle viivavälimerkille eli
yhdysmerkille, lyhyelle ajatusviivalle ja pitkälle ajatusviivalle. Kun
Texin viivavälimerkit (luku \ref{luku/tavutuksen-merkit}) aina estävät
tavutuksen muualta kuin välimerkin jälkeen, \paketti{extdash}\-/ paketin
perusvaihtoehdot sallivat tavutuksen muualtakin. Lisäksi kaikille
kolmelle viivavälimerkille on sitova versio, joka estää tavutuksen
välimerkin jälkeen mutta sallii muualta.

Taulukossa \ref{tlk/extdash} ovat \paketti{extdash}\-/paketin tärkeimmät
komennot ja niiden merkitykset. Taulukossa \ref{tlk/extdash-vertailu}
vertaillaan \paketti{extdash}\-/paketin komentoja Texin vastaaviin.

\leijutlk{
  \begin{tabular}{ll}
    \toprule
    \ots{Komento} & \ots{Merkitys} \\
    \midrule
    \komento{-/} & tavutuksen salliva yhdysmerkki \\
    \komento{=/} & sitova, tavutuksen salliva yhdysmerkki \\
    \komento{--} & tavutuksen salliva lyhyt ajatusviiva \\
    \komento{==} & sitova, tavutuksen salliva lyhyt ajatusviiva \\
    \komento{---} & tavutuksen salliva pitkä ajatusviiva \\
    \komento{===} & sitova, tavutuksen salliva pitkä ajatusviiva \\
    \bottomrule
  \end{tabular}
}{
  \caption{\paketti{extdash}-paketin komentoja, kun paketti on ladattu
    käyttämällä asetusta \koodi{short\-cuts}}
  \label{tlk/extdash}
}

\leijutlk{
  \begin{tabular}{llll}
    \toprule
    \ots{Lähde} & \ots{Tavutus}
    & \ots{Lähde} & \ots{Tavutus} \\
    \cmidrule(r){1-2}
    \cmidrule(l){3-4}
    \koodi{matka-aika}
                & matka-\tavukohta aika
    & \koodi{Oulu--Rovaniemi}
                  & Oulu--\tavukohta Rovaniemi \\
    \koodi{matka\komento{-/}aika}
                & mat\tavukohta ka-\tavukohta ai\tavukohta ka
    & \koodi{Oulu\komento{--}Rovaniemi}
                  & Ou\tavukohta lu--\tavukohta Ro\tavukohta va\tavukohta
                    nie\tavukohta mi \\
    \koodi{matka\komento{=/}aika}
                & mat\tavukohta ka-ai\tavukohta ka
    & \koodi{Oulu\komento{==}Rovaniemi}
                  & Ou\tavukohta lu--Ro\tavukohta va\tavukohta
                    nie\tavukohta mi \\
    \koodi{matka-}
                & matka-
    & \koodi{-aika}
                  & -\tavukohta aika (väärin) \\
    \koodi{matka\komento{-/}}
                & mat\tavukohta ka-
    & \koodi{\komento{=/}aika}
                  & -ai\tavukohta ka \\
    \bottomrule
  \end{tabular}
}{
  \caption{Texin ja \paketti{extdash}-paketin viivamerkkien vertailua}
  \label{tlk/extdash-vertailu}
}

On tärkeää huomioida, että \paketti{extdash}\-/paketin komennot ovat
todellakin normaaleja komentoja. Se tarkoittaa, että komennon jälkeiset
sanavälit syödään pois. Tämä asia saattaa unohtua seuraavanlaisessa
tilanteessa, jossa käytetään tavutuksen sallivaa yhdysmerkkikomentoa:

\komentoi{-/}
\begin{koodilohkosis}
matka\-/ ja aika\-/arvio
\end{koodilohkosis}

\begin{tulossis}
  matka\-/ ja aika\-/arvio
\end{tulossis}

\noindent
Ensimmäisen yhdysmerkkikomennon jälkeinen sanaväli hävisi, ja syntyi
virheellinen sana \emph{matka-ja}. Sanavälin saa säilymään, kun
kirjoittaa komennon perään aaltosuljeparin tai kenoviivan sekä
sanavälin.

\komentoi{-/}
\begin{koodilohkosis}
matka\-/{} ja aika\-/arvio \\
matka\-/\ ja aika\-/arvio
\end{koodilohkosis}

\begin{tulossis}
  matka\-/{} ja aika\-/arvio \\
  matka\-/\ ja aika\-/arvio
\end{tulossis}

\subsection{Muita tavutusasetuksia}
\label{luku/tavutus-muut}

Varsinainen tavutuksen peruskomento on \komento{discretionary}, joka
mahdollistaa omanlaisten tavutuskohtien määrittelyn. Komento
kirjoitetaan sanassa juuri siihen kohtaan, johon tavutuskohta halutaan,
ja komennon argumenttien rakenne on seuraavanlainen:

\komentoi{discretionary}
\begin{koodilohkosis}
\discretionary{loppu}{alku}{katkaisematon}
\end{koodilohkosis}

\noindent
Komennon ensimmäinen argumentti \koodi{loppu} ilmaisee
katkaisutilanteessa rivin loppuun ladottavat merkit. Toinen argumentti
\koodi{alku} ilmaisee katkaisutilanteessa seuraavan rivin alkuun
ladottavat merkit, ja kolmas argumentti tarkoittaa katkaisemattomaan
sanaan ladottavia merkkejä. Normaali tavutuskohta sanaan \emph{tavu}
määriteltäisiin seuraavan esimerkin tavoin:

\komentoi{discretionary}
\begin{koodilohkosis}
ta\discretionary{-}{}{}vu
\end{koodilohkosis}

\noindent
Edellisessä esimerkissä komennon ensimmäinen argumentti on yhdysmerkki,
koska rivin loppuun tietenkin halutaan yhdysmerkki silloin, kun sana
katkaistaan tästä kohdasta. Toinen argumentti on tyhjä, koska seuraavan
rivin alkuun ei kirjoiteta suomen kielessä mitään ylimääräistä. Myös
kolmas argumentti on tyhjä, koska katkaisemattomaan sanaan ei haluta
mitään merkkiä tavujen väliin.

\komento{discretionary}\-/komennolla voi luoda myös tavutuskohtia,
joihin ei ilmesty yhdysmerkkiä eikä mitään muutakaan merkkiä
katkaisutilanteessa. Tämä vastaa kielipakettien tavutusvihjettä
\koodi{\textquotedbl\textquotedbl}. \komento{discretionary}\-/
komennolla sama toteutetaan jättämällä kaikki argumentit tyhjäksi:

\komentoi{discretionary}
\begin{koodilohkosis}
\discretionary{}{}{}
\end{koodilohkosis}

\noindent
Joissakin kielissä sanan katkaiseminen yhdysmerkin kohdalta vaatii, että
rivin loppuun kirjoitetaan yksi yhdysmerkki ja seuraavan rivin alkuun
toinen. Näin ilmaistaan, että sanassa on pysyvä yhdysmerkki eikä vain
väliaikaisesti katkaisun merkkinä. Tällaisia kieliä varten saattaa
kielipaketissa olla omat yhdysmerkkitoiminnot tai oikeanlainen
tavutuslogiikka voi jo sisältyä kieliasetuksiin, mutta tällaisen
tavutuskohdan saa myös seuraavalla tavalla:

\komentoi{discretionary}
\begin{koodilohkosis}
\discretionary{-}{-}{-}
\end{koodilohkosis}

\noindent
Tavutuksen suunnittelussa ja tutkimisessa voi auttaa
\pakettictan{showhyphenation}\-/ paketti. Kun paketti on ladattuna,
kaikkien sanojen mahdollisiin tavutuskohtiin ladotaan pieni merkki.
Paketti hyödyntää Lualatex\-/ kääntäjän ominaisuuksia, eikä se siis
toimi muiden kääntäjien kanssa.

\subsection{Suomen kielen tavutus}
\label{luku/suomi-tavutus}

Suomen kielelle on olemassa Latexissa kaksi erilaista tavutussäännöstöä.
Oletuksena toimivat säännöt kuvaavat tavallisimmat tavujen rakenteet
mutta pyrkivät huomioimaan myös typografisia suosituksia. Esimerkiksi
sanaa ei välttämättä katkaista vokaalien välistä, koska se ei ole
suositeltavaa. Myös jotkin yleiset vierasperäisten sanojen
konsonanttiyhdistelmät pidetään yhdessä, vaikka ne suomen tavurakenteen
mukaan kuuluisivat joskus eri tavuihin.

Toinen, vaihtoehtoinen tavutussäännöstö kuvaa pelkästään suomen tavujen
rakenteen, eli se tavuttaa sanat kaikista mahdollisista kohdista eikä
huomioi typografisia suosituksia mitenkään.%
\footnote{Lisätietoa:
  \url{https://github.com/hyphenation/basic-finnish}} Asetus sanan
reunasta tavuttamisesta on kuitenkin voimassa (luku
\ref{luku/tavutus-reunasta}). Tämä vaihtoehtoinen tavutustapa täytyy
kytkeä erikseen päälle kielipaketin asetuksista. Sen nimi on Latexissa
\emph{\englanti{school}}, koska se on ''peruskoulutavuttamista'' eikä
typografian mukaista tavuttamista. Katso kielipakettikohtaiset ohjeet
luvuista \ref{luku/tavutus-polyglossia} ja \ref{luku/tavutus-babel}.

Texin kirjainyhdistelmiin perustuvat tavutussäännöt eivät yksinään ole
tarpeeksi älykkäitä suomen kieleen, ja esimerkiksi yhdyssanojen
rajakohdat tuottavat usein ongelmia. Sana \emph{alku\-osa} katkaistaan
Latexissa kohdista \emph{al-kuo-sa}. Se on kyllä oikein tavurakenteen
kannalta mutta käytännössä ongelmallinen. Tässä ei ole kyse \emph{uo}\-/
diftongista eli samaan tavuun kuuluvista vokaaleista, vaan yhdyssanan
rajalla on myös tavutuskohta (\emph{al-ku-o\=/sa}). Lisäksi sanaa ei
saisi katkaista siten, että siitä jää yksittäinen kirjain eri riville
(ei: \emph{o\=/sa}).

Parasta olisi katkaista suomen kielen yhdyssanat vain yhdysosien välistä
(\emph{alku-osa}). Muualtakin voi katkaista (\emph{al-ku-osa}), kunhan
sanasta eikä sen yhdysosasta ei jää yksittäinen kirjain eri riville.
Mielellään ei katkaista myöskään kahden vokaalin välistä, jos ne
kuuluvat samaan sanaan (ei: \emph{kau-emmin}). Joskus halutaan välttää
myös niin sanottuja orpotavuja eli sitä, että tekstikappaleen
viimeiselle riville jää vain yksi tavu.

Käytännössä siis suomenkielinen teksti ja hyvä typografia vaativat
välillä tavutusvihjeiden kirjoittamista. Yhdyssanojen osien väliin
tarvitaan tavutusvihje silloin, kun jälkimmäinen osa alkaa vokaalilla
tai useammalla kuin yhdellä konsonantilla. Joitakin tällaisia tapauksia
Tex osaa tavuttaa oikein ilman tavutusvihjeitäkin.

\komentoi{-}
\begin{koodilohkosis}
alku\-osa pusku\-traktori
\end{koodilohkosis}

\noindent
Sitova yhdysmerkki on suomen kielessä tarpeellinen, kun sana alkaa
yhdysmerkillä eli ennen yhdysmerkkiä on sanaväli. Esimerkiksi
ilmauksessa \emph{matkasuunnitelma ja \mbox{-aika}} ei riviä saa
katkaista sanassa \emph{\mbox{-aika}} olevan yhdysmerkin jälkeen. Täytyy
siis käyttää sitovaa yhdysmerkkiä eli merkkiä, joka sitoo yhdysmerkin
kiinni seuraavaan sanaan. Sitovan yhdysmerkin käyttöä neuvotaan luvuissa
\ref{luku/tavutus-polyglossia}, \ref{luku/tavutus-babel} ja
\ref{luku/tavutus-extdash}.

Yhdysmerkin tai ajatusviivan sisältävät pitkät yhdyssanat voivat vaatia
tavutuskohtien lisäämistä, koska yhdysmerkki ja ajatusviivat estävät
tavutuksen muualta kuin näiden merkkien jälkeen. Ilman tavutuskohtien
lisäämistä Texillä ei ehkä ole riittävästi vaihtoehtoja tekstikappaleen
rivittämiseen. Voi syntyä liian suuria sanavälejä, tai joistakin
riveistä tulee ylipitkiä, eli ne yltävät marginaalin puolelle.

Tavutuskohtia voi lisätä sopiviin kohtiin tavutusvihjeillä.
Vaihtoehtoisesti voi käyttää \paketti{extdash}\-/ paketin
välimerkkikomentoja (taulukko \ref{tlk/extdash}), jotka sallivat
tavutuksen kaikista kielen tavutussääntöjen mukaisista kohdista.
Seuraavassa esimerkissä on tavutuksen hallintaa sanalle
\emph{Molo\-tov--Ribben\-trop-sopi\-mus}:

\komentoi{-}
\komentoi{--}
\komentoi{-/}
\begin{koodilohkosis}
Molo\-tov--Ribben\-trop-sopi\-mus % tavutusvihjeet
Molotov\--Ribben\-trop\-/sopimus  % tavutuksen sallivat välimerkit
\end{koodilohkosis}

\noindent
\komento{discretionary}\-/komentoa (luku \ref{luku/tavutus-muut}) voi
hyödyntää suomen kielen sanoissa silloin, kun niissä on heittomerkki
erottamassa kahta tavurajan molemmin puolin olevaa samaa vokaalia,
esimerkiksi sanoissa \emph{vaa'an} ja \emph{liu'uttaa}. Normaalisti Tex
ei katkaise heittomerkin kohdalta lainkaan, eikä se olisi suomen
kielessä suositeltavaakaan, koska vokaalien välistä ei mielellään
katkaista sanaa. Jos tavutuksen kuitenkin haluaa myös heittomerkin
kohdalle, täytyy huolehtia, että tavutustilanteessa heittomerkki poistuu
ja sen paikalle tulee yhdysmerkki rivin loppuun. Sanan \emph{vaa'an}
voisi siis kirjoittaa seuraavalla tavalla:

\komentoi{discretionary}
\begin{koodilohkosis}
vaa\discretionary{-}{}{'}an
\end{koodilohkosis}

\noindent
Komennon kolmas argumentti on heittomerkki, koska se pitää latoa
tavurajalle silloin, kun sanaa ei katkaista tästä kohdasta. Mikäli
tällaisia tarvitsee paljon, on järkevää määritellä sitä varten
yksinkertaisempi komento, jota sitten käytetään sanoissa heittomerkin
sijasta.

\komentoi{discretionary}
\begin{koodilohkosis}
\newcommand{\hm}{\discretionary{-}{}{'}}
vaa\hm an, liu\hm uttaa
\end{koodilohkosis}

\noindent
Suomen kielessä käytetään heittomerkkiä myös taivutuspäätteen, liitteen
tai johtimen edellä silloin, kun sanavartalon kirjoitusasu päättyy
konsonanttiin mutta ääntöasu vokaaliin, esimerkiksi \emph{show'ssa}.
Näissä tilanteissa ei ole kyse tavurajasta vaan morfeemirajasta eli
merkityksellisten osien rajakohdasta. Tavurajakin voi sattua samaan
paikkaan, mutta tavutettaessa heittomerkki säilyy: \emph{show'-hun}.%
\footnote{Asiaa ei yleensä mainita kielenhuolto\-/oppaissa. Tieto
  perustuu Kielikello\-/lehden 2/2006 artikkeliin:
  \kulmaurl{https://www.kielikello.fi/-/lainausmerkit-}. Viittauspäivä
  6.7.2020.} Mikäli tällainen tavutuskohta halutaan mukaan, käytetään
sanassa tavallista tavutusvihjettä: \koodi{show'}\komento{-}\koodi{hun}.
Mieluummin ei kuitenkaan katkaista sanoja heittomerkin kohdalta.

\subsection{Tavutus pois päältä}

Tavutuksen voi kytkeä kokonaan pois päältä \paketti{polyglossia}\-/
kielipaketin komennolla \komento{disablehyphenation}. Tavutuksen saa
takaisin päälle komennolla \komento{enablehyphenation}.

Toinen vaihtoehto on käyttää fontin asetusta \koodi{Hyphen\-Char=\katk
  None} kirjainperheen määrittelyn yhteydessä. Oletuksena tavutus on
pois päältä tasalevyisestä fontista, eli esimerkiksi \komento{texttt}\-/
komennon argumenttina olevaa tekstiä ei tavuteta. Fontteja ja niiden
asetuksia käsitellään luvussa \ref{luku/kirjaintyypit}.

Käytännössä tavutus menee pois päältä myös silloin, kun tekstikappaleet
tasaa vain vasempaan reunaan eli tekee oikealle liehureunan komennolla
\komento{raggedright}. Tekstikappaleiden tasaamista ja palstan muotoa
käsitellään luvussa \ref{luku/kappaleen-tasaus}.

% Tekijä:   Teemu Likonen <tlikonen@iki.fi>
% Lisenssi: Creative Commons Nimeä-JaaSamoin 4.0 Kansainvälinen (CC BY-SA 4.0)
% https://creativecommons.org/licenses/by-sa/4.0/legalcode.fi

\chapter{Rakenne ja sisältö}
\label{luku/rakenne}

Tämä luku on oppaan kaikista luvuista kenties käytännönläheisin. Luku
keskittyy asioihin, joita kirjoittaja miettii erityisesti dokumentin
sisällön kirjoitusvaiheessa. Kirjoittaja tekee esimerkiksi valintoja
otsikoinnin ja muun jäsentämisen osalta. Hän miettii tiedon esittämistä
paitsi tekstin avulla myös esimerkiksi luetelmien, kuvien ja taulukoiden
avulla. Kirjoittaja tekee myös typografisia valintoja tekstin muotoilun
ja korostuskeinojen näkökulmasta. Edellä mainittuja ja muitakin
dokumentin rakenteen ja sisällön asioita käsitellään niin tekniikan kuin
typografiankin näkökulmasta.

\section{Tekstikappaleet}
\label{luku/kappale}

Tekstikappale on tekstin osa, jonka pitäisi käsitellä suunnilleen yhtä
asiakokonaisuutta. Se voi olla esimerkiksi yksi aihe, näkökulma,
ajankohta tai henkilö. Tekstin seuraava kappale käsittelee jotakin
toista aihetta, näkökulmaa tms. Kappaleen vaihtuminen on lukijalle
merkki siitä, että tekstin sisällössäkin jokin muuttuu.

Latexin lähdetiedostoissa kappaleen vaihtuminen ilmaistaan
kirjoittamalla kappaleiden väliin vähintään yksi tyhjä rivi. Tätä
merkintäkielen piirrettä käsitellään myös luvussa
\ref{luku/kappaleen-vaihtuminen}. Kappale vaihtuu myös komennolla
\komento{par}, joka sopii käytettäväksi esimerkiksi komentojen
määrittelyssä (luku \ref{luku/komennot}), kun halutaan varmistaa
kappaleen vaihtuminen tietyssä kohdassa.

Ladotuissa teksteissä kuten kirjoissa ja lehdissä kappaleen vaihtuminen
ilmaistaan melkein aina siten, että uuden kappaleen ensimmäinen rivi
sisennetään hieman. Niin on tässäkin oppaassa. Toisinaan tekstikappaleet
erotetaan pystysuuntaisella välillä, ja silloin kappaleiden ensimmäistä
riviä ei sisennetä. Kappaleiden välejä, sisennyksiä, rivien tasaamista
ja muita asetuksia käsitellään seuraavissa alaluvuissa.

Monissa kappaleisiin liittyvissä asetuksissa tarvitaan Texin mittoja ja
mittayksiköitä. Mittoihin liittyvää tekniikkaa käsitellään tarkemmin
luvussa \ref{luku/mitat}, joka on syytä tuntea ennen tämän alaluvun
lukemista.

\subsection{Tasaaminen ja palstan muoto}
\label{luku/kappaleen-tasaus}

Perusdokumenttiluokissa (luku \ref{luku/perusdokumenttiluokat})
tekstikappaleet tasataan oletuksena palstan molempiin reunoihin, ja tätä
palstan muotoa kutsutaan tasapalstaksi. Se tarkoittaa samalla sitä, että
rivillä olevia sanavälejä venytetään sopivasti, jotta jokainen rivi
näyttäisi yhtä pitkältä ja palstan molemmat reunat tasaiselta.

Käytännössä sanavälien venymiselle on määritelty yläraja, jonka yli
niitä ei venytetä. Ylärajan tarkoituksena on estää liian suuret ja rumat
sanavälit. Rajoitus on sinänsä järkevä, mutta se voi myös johtaa siihen,
että Tex ei saa tasattua kaikkia tekstikappaleita palstan oikeasta
reunasta: jotkin rivit yltävät palstan reunan yli; jotkin rivit jäävät
vajaaksi. Näin käy usein varsinkin suomen kielessä, jonka sanat ovat
usein pitkiä ja riveillä on vähänlaisesti sanavälejä. Suomen kielessä
sanavälien venymisen yläraja on usein tarpeellista asettaa oletusarvoa
suuremmaksi. Se tehdään mitan \mitta{emergencystretch} avulla,
esimerkiksi seuraavasti:

\komentoi{setlength}
\mittai{emergencystretch}
\begin{koodilohkosis}
\setlength{\emergencystretch}{1em}
\end{koodilohkosis}

\noindent
Kaikenlaiset kappaleiden latomiseen liittyvät tekniset rajoitukset voi
poistaa tai asettaa hyvin suuriksi komennolla \komento{sloppy}. Komento
asettaa muun muassa sanavälien venymisen ylärajaksi 3\,em. Tämän
komennon käyttö ei ole kovin suositeltavaa, koska sillä on muitakin
seurauksia ja se voi vaikuttaa myös sellaisiin kappaleisiin, jotka
muuten saataisiin ladottua nätisti. Parempi on asettaa vain mitta
\mitta{emergencystretch} riittävän suureksi. Sanavälien venymiseen ja
kappaleiden tasaiseen latomiseen liittyvät asetukset voi palauttaa
oletusarvoihin komennolla \komento{fussy}.

Hyvin tavallista on tasata teksti pelkästään vasempaan reunaan, jolloin
rivien pituudet vaihtelevat ja oikealla on niin sanottu liehureuna.
Oikea liehureuna sopii pitkiin teksteihin yhtä hyvin kuin tasapalstakin,
mutta se on parempi valinta erityisesti silloin, kun palsta on kapea.
Nimittäin kapealla palstalla venyviä sanavälejä on käytettävissä hyvin
vähän ja oikean reunan tasaaminen vaatii sanavälien venyttämistä joskus
kohtuuttoman paljon. Tekstiin jää rumia aukkoja.

\leijutlk{
  \providecommand{\rivi}{}
  \renewcommand{\rivi}[3]{\komento{#1} & \ymparisto{#2} & #3 \\}
  \begin{tabular}{lll}
    \toprule
    \ots{Komento} & \ots{Ympäristö} & \ots{Merkitys} \\
    \midrule
    \rivi{raggedright}{flushleft}{vasen tasaus, oikea liehu}
    \rivi{raggedleft}{flushright}{oikea tasaus, vasen liehu}
    \rivi{centering}{center}{keskitetty}
    \midrule
    \rivi{RaggedRight}{FlushLeft}
    {vasen tasaus, oikea liehu, tavutus (\paketti{ragged2e})}
    \rivi{RaggedLeft}{FlushRight}
    {oikea tasaus, vasen liehu, tavutus (\paketti{ragged2e})}
    \rivi{Centering}{Center}{keskitetty, tavutus (\paketti{ragged2e})}
    \rivi{justifying}{justify}{tasapalsta, tavutus (\paketti{ragged2e})}
    \bottomrule
  \end{tabular}
}{
  \caption{Tekstikappaleen tasaamiseen ja palstan muotoon vaikuttavat
    komennot ja ympäristöt. Osa sisältyy \paketti{ragged2e}\-/pakettiin}
  \label{tlk/kappaleen-tasauskomennot}
}

Kappaleiden tasaamiseen ja palstan muotoon vaikuttavia komentoja ja
ympäristöjä on koottu taulukkoon \ref{tlk/kappaleen-tasauskomennot}.
Taulukossa on mainittu ensin Latexin omat komennot ja sitten
\pakettictan{ragged2e}\-/ paketin vastaavat. Latexin omat komennot
estävät sanojen tavuttamisen, kun taas \paketti{ragged2e}\-/ paketin
komennot sallivat tavutuksen normaalisti.

\subsection{Optinen tasaus}

Tekstin tasaamisessa halutaan toisinaan jättää jotkut merkit
tasauskohdan ulkopuolelle. Esimerkiksi suurikokoisissa otsikoissa on
joskus alussa lainausmerkki, joka halutaan latoa tasauskohdan vasemmalle
puolelle, jotta eri riveillä olevat sanat saadaan tasaan.

\begin{tulossis}
  \Large\makebox[0bp][r]{''}Optinen tasaus \\
  otsikossa''
\end{tulossis}

\noindent
Tasauskohdan ulkopuolinen lainausmerkki voidaan toteuttaa leveydettömän
laatikon avulla (luku \ref{luku/laatikot}). Lainausmerkki kirjoitetaan
laatikkoon, jonka leveys on nolla ja jonka sisältö tasataan laatikon
oikeaan reunaan. Tällöin laatikko ei vie yhtään tilaa, mutta sen sisältö
ladotaan kyseisen kohdan vasemmalle puolelle.

\komentoi{makebox}
\begin{koodilohkosis}
\makebox[0bp][r]{''}Optinen tasaus \\
otsikossa''
\end{koodilohkosis}

\noindent
Toisinaan keskitetyssä monirivisessä tekstissä tai otsikossa on rivin
lopussa yhdysmerkki, mutta keskitys ehkä halutaan toteuttaa vain
kirjainten perusteella ja jättää yhdysmerkki sen ulkopuolelle. Sellainen
yhdysmerkki voidaan kirjoittaa leveydettömään laatikkoon, jolloin sitä
ei huomioida keskittämisessä.

\komentoi{makebox}
\ymparistoi{center}
\begin{koodilohkosis}
\begin{center}
  \Large Latex\makebox[0bp][l]{-} \\ opas
\end{center}
\end{koodilohkosis}

\begin{tulossis}
  \begin{center}
    \Large Latex\makebox[0bp][l]{-} \\ opas
  \end{center}
\end{tulossis}

\subsection{Pystysuuntaiset välit}
\label{luku/pystysuuntaiset-välit}

Kappaleiden väliin ladottava pystysuuntainen tyhjä tila asetetaan mitan
\mitta{parskip} avulla. Se on oletuksena nolla, mutta pientä venymistä
kuitenkin sallitaan, eli joissakin tilanteissa kappaleiden väliin
voidaan latoa pieni tyhjä tila. Jos tyhjää tilaa ei haluta missään
tilanteessa, asetetaan mitta vain nollaksi:

\komentoi{setlength}
\mittai{parskip}
\begin{koodilohkosis}
\setlength{\parskip}{0ex}
\end{koodilohkosis}

\noindent
Seuraava esimerkkikomento asettaa kappaleväliksi 1,3\,ex. Lisäksi se
sallii kappalevälin venyä 0,2\,ex:n verran tai kutistua 0,1\,ex:n
verran.

\komentoi{setlength}
\mittai{parskip}
\begin{koodilohkosis}
\setlength{\parskip}{1.3ex plus .2ex minus .1ex}
\end{koodilohkosis}

\noindent
Silloin kun kappaleet ladotaan erilleen toisistaan, on yleensä hyvä
sallia kappalevälin venyä tai kutistua hieman, koska venyvät
pystysuuntaiset välit antavat Texille paremmat mahdollisuudet latoa
hyvännäköisiä sivuja. Venyvien välien avulla esimerkiksi sivujen
tekstialueen ylä- ja alareunat saadaan aina samalle kohdalle. Toisaalta
myös liian suuret ja toisistaan liiaksi poikkeavat kappalevälit voivat
olla rumannäköisiä.

Tavallista kappaleväliä suurempien pystysuuntaisten välien tekemiseen on
olemassa kolme valmista komentoa: suurimmasta pienimpään ne ovat
\komento{bigskip}, \komento{medskip} ja \komento{smallskip}. Ne
sopivat käytettäväksi yksittäisiin tilainteisiin, joissa normaali
kappaleväli on liian vähän. Jos sivunvaihto osuu näiden komentojen
kohdalle, mitään väliä ei ladota sivun loppuun eikä seuraavan alkuun.

Edellä mainittujen komentojen latoman välin suuruuteen voi vaikuttaa
mittojen \mitta{bigskipamount}, \mitta{medskipamount} ja
\mitta{smallskipamount} avulla. Seuraavassa on esimerkkikomennot
mittojen määrittelemiseen ja samalla niiden oletusarvot:

\komentoi{setlength}
\mittai{bigskipamount}
\mittai{medskipamount}
\mittai{smallskipamount}
\begin{koodilohkosis}
\setlength{\bigskipamount} {12pt plus 4pt minus 4pt}
\setlength{\medskipamount}  {6pt plus 2pt minus 2pt}
\setlength{\smallskipamount}{3pt plus 1pt minus 1pt}
\end{koodilohkosis}

\noindent
Komentojen \komento{bigskip}, \komento{medskip} ja \komento{smallskip}
sijasta voi käyttää myös komentoja \komento{bigbreak},
\komento{medbreak} tai \komento{smallbreak}. Nämä toimivat lähes
samalla tavalla, mutta niihin sisältyy myös sivunvaihtovihje. Toisin
sanoen ne vaikuttavat ladonta\-/algoritmiin siten, että komennon
kohdalla todennäköisyys sivun vaihtumiselle kasvaa suhteessa muihin
kohtiin. Sivu voi edelleen vaihtua muustakin kohdasta, jos algoritmi
löytää omasta mielestään vielä paremman paikan.

Pystysuuntaisten välien yleiskomento on \komento{vspace}, jolle
annetaan argumentiksi välin suuruus ja mahdolliset venymisen rajat.
Tämäkin komento jättää välin latomatta, jos se sattuu sivunvaihdon
kohdalle. Sen sijaan tähdellinen versio \komento{vspace*} latoo välin
joka tapauksessa, vaikka se olisi sivun lopussa tai alussa.

\komentoi{vspace}
\begin{koodilohkosis}
Tekstikappale.
\vspace{5ex plus 1ex minus .5ex}

Toinen tekstikappale.
\end{koodilohkosis}

\noindent
Komento \komento{addvspace} toimii lähes samoin kuin \komento{vspace},
mutta se huomioi mahdolliset peräkkäiset \komento{addvspace}\-/ komennot
ja varmistaa, että vain suurin väli toteutuu. Jos siis useita
\komento{addvspace}\-/ komentoja sattuu peräkkäin, niiden määrittämiä
välejä ei ladota peräkkäin vaan ainoastaan suurin niistä ladotaan.
Seuraava esimerkki latoo kappaleiden väliin 2\,ex:n suuruisen
pystysuuntaisen välin:

\komentoi{addvspace}
\begin{koodilohkosis}
Tekstikappale.

\addvspace{1ex} \addvspace{2ex} \addvspace{.5ex}
Toinen tekstikappale.
\end{koodilohkosis}

\noindent
Jos edellisessä esimerkissä olisi käytetty \komento{vspace}\-/ komentoa,
pystysuuntaisen välin suuruus olisi 3,5\,ex, joka on välien
yhteenlaskettu suuruus.

\komento{addvspace}\-/komento soveltuu hyvin komentojen ja ympäristöjen
määrittelyyn (luvut \ref{luku/komennot} ja \ref{luku/ympäristöt}).
Esimerkiksi itse määritellyn ympäristön alussa ja lopussa voi
\komento{addvspace}\-/ komennolla varmistaa tietynsuuruisen välin, mutta
jos sama tai muu vastaava ympäristö on dokumentissa kahdesti peräkkäin,
huomioidaan pystysuuntainen väli vain kerran eli suurimman välin mukaan.
Jotkin Latexin valmiit ympäristöt tekevät juuri näin eli käyttävät
\komento{addvspace}\-/komentoa välien asettamiseen.

Sivun alueella äärettömästi venyvän pystysuuntaisen välin saa komennolla
\komento{vfill}. Mitan luonnollinen arvo on nolla, mutta se voi venyä
niin, että se täyttää kaiken tyhjän tilan sivulla. \komento{vfill}\-/
komento tarkoittaa käytännössä samaa kuin
\komento{vspace}\komentoarg{0mm plus 1fill} \=/komento. Texin venyviä
mittoja ja välejä käsitellään tarkemmin luvussa
\ref{luku/venyvät-mitat}.

\subsection{Ensimmäisen rivin sisennys}
\label{luku/ensimmäisen-rivin-sisennys}

Kirjojen ja lehtien typografiassa kappaleen ensimmäinen rivi on tapana
sisentää merkiksi siitä, että alkaa uusi kappale. Kappaleiden välissä ei
ole pystysuuntaista tilaa, vaan pelkkä sisennys on lukijalle merkki
siitä, että tekstin sisällössä siirrytään seuraavaan asiaan.
Oletusasetuksilla Latex latoo sisennyksen automaattisesti kappaleiden
alkuun.

Ensimmäisen rivin sisennyksen suuruus asetetaan mitan \mitta{parindent}
avulla, seuraavan esimerkin mukaisesti. Sopiva mittayksikkö tähän
tarkoitukseen on em, koska se viittaa suoraan nykyisen fontin kokoon.

\komentoi{setlength}
\mittai{parindent}
\begin{koodilohkosis}
\setlength{\parindent}{1em}
\end{koodilohkosis}

\noindent
Edellä mainittu mitta pitäisi asettaa nollaan silloin, kun kappaleet
erotetaan pystysuuntaisen välin avulla. Välihän jo sinänsä ilmaisee,
että kappale vaihtuu, joten sisennys on turha.

\komentoi{setlength}
\mittai{parskip}
\mittai{parindent}
\begin{koodilohkosis}
\setlength{\parskip}{1.3ex plus .2ex minus .1ex}
\setlength{\parindent}{0em}  % Ei sisennystä.
\end{koodilohkosis}

\noindent
\mitta{parindent}\-/ mitan levyisen välin voi tehdä mihin tahansa
komennolla \komento{indent}. Tätä komentoa ei tavallisesti tarvita,
koska kappaleet alkavat automaattisesti sen suuruisella sisennyksellä.
Tarpeellisempi komento on sen vastakohta \komento{noindent}, joka
voidaan kirjoittaa kappaleen alkuun estämään kyseisen kappaleen alun
sisentäminen.

\komentoi{noindent}
\begin{koodilohkosis}
\noindent
Tämän tekstikappaleen ensimmäistä riviä ei sisennetä.
\end{koodilohkosis}

\noindent
Suomenkielisissä julkaisuissa on tavallista, että leipätekstin
kappaleessa ei ole sisennystä, jos sitä ennen on pystysuuntainen väli.
Tällainen tilanne on aina otsikoiden jälkeen mutta myös kokonaan
sisennetyn tekstikappaleen jälkeen (esim. lohkolainaus, luku
\ref{luku/lohkolainaukset}) tai kuvan, taulukon, luetelman tai muun
vastaavan osan jälkeen, jos nämä ovat osa tekstivirtaa eivätkä leijuvia
osia (luku \ref{luku/leijuosat}). Käytäntöön on joskus poikkeuksia
suomenkielisessäkin typografiassa, mutta eri kielten välillä käytäntö
voi vaihdella enemmänkin.

Latex estää sisennyksen automaattisesti otsikoiden jälkeen mutta latoo
sisennyksen kuitenkin kaikkien muiden elementtien ja pystysuuntaisen
välin jälkeen. Jos sisennys halutaan estää, pitäisi kyseiset kappaleet
aloittaa aina \komento{noindent}\-/komennolla.

Toinen vaihtoehto on käyttää \pakettictan{noindentafter}\-/ pakettia ja
määritellä sen tarjoaman komennon avulla, minkä ympäristöjen jälkeen ei
haluta sisennystä. Seuraava esimerkki poistaa sisennyksen aina
\ymparisto{list}- ja \ymparisto{tabular}\-/ ympäristöjen jälkeen (luvut
\ref{luku/list-ympäristö} ja \ref{luku/taulukot}).

\komentoi{NoIndentAfterEnv}
\ymparistoi{list}
\ymparistoi{tabular}
\begin{koodilohkosis}
\NoIndentAfterEnv{list}
\NoIndentAfterEnv{tabular}
\end{koodilohkosis}

\noindent
Tosin \paketti{noindentafter}\-/ paketti ei ole aina toiminut
luotettavasti yhdessä \paketti{polyglossia}\-/ kielipaketin kanssa. Jos
sisennyksen poistaminen ei tahdo toimia, kyse voi olla juuri tästä.

Kolmas keino kappaleen ensimmäisen rivin sisennyksen estämiseksi jonkin
ympäristön jälkeen on se, että aloittaa tekstikappaleen heti ympäristön
lopettavan \komento{end}\-/ komennon jälkeen -- ilman tyhjää riviä.

\subsection{Riippuva sisennys}
\label{luku/riippuva-sisennys}

Riippuva sisennys tarkoittaa tekstikappaleen muotoa, jossa sisennetään
kappaleen muita rivejä mutta ei ensimmäistä. Riippuvaa sisennystä
käytetään esimerkiksi kirjallisuus\-/{} ja lähdeluetteloissa, joissa on
tarpeellista saada henkilön nimi tai muu lähdemerkinnän hakusana
erottumaan selvästi vasemmassa reunassa. Tämän oppaan lopussa sivulla
\pageref{luku/kirjallisuutta} on esimerkki lähdemerkinnöistä.

Myös virallisten asiakirjojen muotoilussa käytetään riippuvaa
sisennystä. Niissä kappaleen ensimmäinen rivi voi sisältää otsikon, joka
on tasattu vasempaan reunaan. Otsikon perässä on sarkainhyppy
tekstikappaleen sisennyksen tasalle, ja kappaleen muut rivit on
sisennettynä samalla tasolle.

\begin{esimerkki*}
  \komentoi{hangpara}
  \pakettii{hanging}
  \komentoi{,}

\begin{koodilohko}
\hangpara{2cm}{1}Tässä tekstikappaleessa on riippuva sisennys. Kappale
alkaa yhdellä sisentämättömällä rivillä, ja kappaleen seuraavat rivit
on sisennetty 2\,cm. Ei ole kovin vaikeaa.
\end{koodilohko}
  \begin{tulos}
    \hangpara{2cm}{1}Tässä tekstikappaleessa on riippuva sisennys. Kappale
    alkaa yhdellä sisentämättömällä rivillä, ja kappaleen seuraavat rivit
    on sisennetty 2\,cm. Ei ole kovin vaikeaa.
  \end{tulos}
  \caption{Riippuva sisennys \paketti{hanging}\-/ paketin ja sen
    \komento{hangpara}\-/ komennon avulla}
  \label{esim/riippuva-sis-hangpara}
\end{esimerkki*}

Helpoin tapa riippuvien sisennysten toteuttamiseen lienee
\pakettictan{hanging}\-/ paketin käyttö. Paketti tuo uuden komennon
\komento{hangpara}, jonka käyttöä esimerkki
\ref{esim/riippuva-sis-hangpara} selventää. Komennon ensimmäinen
argumentti on sisennyksen mitta ja toinen argumentti määrittää, kuinka
monta riviä kappaleen alusta jätetään sisentämättä. Jos toinen
argumentti on negatiivinen luku, on merkitys päinvastainen eli luvun
itseisarvo määrittää, kuinka monta riviä kappaleen alusta sisennetään.

Komennon \komento{hangpara} vaihtoehtona on ympäristö
\ymparisto{hangparas}, jonka sisällä kaikki kappaleet sisennetään
riippuvalla tyylillä samojen asetusten mukaisesti. Ympäristölle annetaan
samat argumentit kuin \komento{hangpara}\-/ komennollekin.

\ymparistoi{hangparas}
\begin{koodilohkosis}
\begin{hangparas}{2cm}{1}
  ...
\end{hangparas}
\end{koodilohkosis}

\begin{esimerkki*}
  \komentoi{hangpara}
  \komentoi{makebox}
  \komentoi{,}

\begin{koodilohko}
\hangpara{2cm}{1}\makebox[2cm][l]{Otsikko}Tässä tekstikappaleessa on
riippuva sisennys. Kappale alkaa yhdellä sisentämättömällä rivillä,
joka sisältää näkymättömässä 2\,cm leveässä laatikossa olevan otsikon.
Kappaleen muut rivit on sisennetty 2\,cm.
\end{koodilohko}
  \begin{tulos}
    \hangpara{2cm}{1}\makebox[2cm][l]{Otsikko}Tässä tekstikappaleessa on
    riippuva sisennys. Kappale alkaa yhdellä sisentämättömällä rivillä,
    joka sisältää näkymättömässä 2\,cm leveässä laatikossa olevan otsikon.
    Kappaleen muut rivit on sisennetty 2\,cm.
  \end{tulos}
  \caption{Asiakirjan tyylisten tekstikappaleiden toteutus}
  \label{esim/riippuva-sis-asiakirja}
\end{esimerkki*}

\noindent
Asiakirjan tyylisen otsikon saa toteutettua \komento{makebox}\-/
komennon avulla esimerkin \ref{esim/riippuva-sis-asiakirja} tavoin.
Komento latoo näkymättömän laatikon, jonka leveys määritellään
sisennyksen levyiseksi ja jonka sisään kirjoitetaan otsikko. Jos
asiakirjatyylisiä tekstikappaleita tarvitaan useita, kannattaa
määritellä sarkainleveyttä ja sisennystä varten oma mitta ja
tekstikappaleen kirjoittamista varten oma komento. Seuraavassa on siitä
esimerkki:

\komentoi{newlength}
\komentoi{setlength}
\komentoi{newcommand}
\komentoi{makebox}
\komentoi{par}
\komentoi{hangpara}
\komentoi{ignorespaces}
\begin{koodilohkosis}
\newlength{\sarkain}
\setlength{\sarkain}{23mm}
\newcommand{\kappale}[1][]{\par\hangpara{2\sarkain}{1}%
  \makebox[2\sarkain][l]{\ignorespaces #1}\ignorespaces}
\end{koodilohkosis}

\noindent
Tämän jälkeen voi komennolla \komentox{kappale} aloittaa asiakirjan
sisennetyn tekstikappaleen. Komennolle voi antaa hakasulkeissa
valinnaisen argumentin, joka on kappaleen otsikko. \komento{makebox}\-/
komentoa ja muita laatikoita käsitellään tarkemmin luvussa
\ref{luku/laatikot}.

\begin{esimerkki*}
  \ymparistoi{list}
  \komentoi{setlength}
  \mittai{leftmargin}
  \mittai{itemindent}
  \komentoi{item}
  \komentoi{,}

\begin{koodilohko}
\begin{list}{}{
    \setlength{\leftmargin}{2cm}
    \setlength{\itemindent}{-2cm}
  }
\item Tässä tekstikappaleessa on riippuva sisennys. Kappale alkaa
  yhdellä sisentämättömällä rivillä, ja kappaleen muut rivit on
  sisennetty 2\,cm.
\end{list}
\end{koodilohko}
  \caption{Riippuvan sisennyksen toteuttaminen \ymparisto{list}\-/
    ympäristön avulla}
  \label{esim/riippuva-sis-list}
\end{esimerkki*}

Riippuvan sisennyksen voi toteuttaa myös \ymparisto{list}\-/ ympäristön
avulla. Se on tarkoitettu luetelmien tekemiseen, mutta sopivilla
asetuksilla yksi ''luetelman'' kohta on riippuvasti sisennetty kappale.
Tarkemmin \ymparisto{list}\-/ ympäristöä käsitellään luetelmien
yhteydessä luvussa \ref{luku/list-ympäristö}, mutta oheisessa
esimerkissä \ref{esim/riippuva-sis-list} on sopivat asetukset riippuvan
sisennyksen toteuttamiseen. Kappale alkaa \komento{item}\-/ komennolla,
koska kyseessä on ikään kuin luetelman kohta.

\subsection{Vasen ja oikea sisennys sekä lohkolainaukset}
\label{luku/lohkolainaukset}

Dokumentteihin tarvitaan välillä kokonaisia sisennettyjä
tekstikappaleita, koska leipätekstin ohessa halutaan näyttää
muuntyyppistä sisältöä. Kyse voi olla teksti- tai kuvaesimerkeistä,
esimerkiksi muualta lainatusta tekstistä. Tässä oppaassa käytetään
paljon sisennettyjä tekstikappaleita Latex\-/koodien esimerkkeihin.

Kokonaan sisennettyjä tekstikappaleita kutsutaan lohkolainauksiksi,
koska ne ovat lainauksia, jotka käsittävät kokonaisen tekstilohkon.
Lainausmerkkejä ei tarvitse käyttää, koska lainaus ilmaistaan
typografisin keinoin. Sisennyksen lisäksi varsin yleistä on käyttää
hieman pienempää kirjainleikkausta ja riviväliä kuin leipätekstissä.
Joskus vasemman reunan sisennyksen lisäksi sisennetään myös oikeasta
reunasta.

Latexissa on tavallisille lohkolainauksille kolme erilaista ympäristöä:
\ymparisto{quotation}, \ymparisto{quote} ja \ymparisto{verse}. Kaksi
ensin mainittua on tarkoitettu normaalilla tavalla juoksevalle
tekstille, kun taas kolmas on tarkoitettu runon säkeiden ja säkeistöjen
latomiseen.

\ymparisto{quotation}\-/ ympäristö sisentää tekstikappaleiden
ensimmäisen rivin 1,5\,em:n verran, eikä kappaleiden välissä ole
pystysuuntaista tilaa. \ymparisto{quote}\-/ ympäristö ei sisennä
kappaleiden ensimmäistä riviä, ja se puolestaan erottaa kappaleet
toisistaan pystysuuntaisen tilan avulla. \ymparisto{verse}\-/ ympäristöä
käytetään siten, että lähdedokumentissa runon säkeet lopetetaan
rivinvaihtokomentoon (\komento{\keno}), lukuun ottamatta säkeistön
viimeistä säettä. Säkeistöt erotetaan toisistaan tyhjällä rivillä, kuten
Latexissa tekstikappaleet muutenkin. Lopputuloksena on useimpiin
runoihin sopiva ladontatapa, jossa säkeistöjen vasen reuna on samalla
tasolla, oikealla on liehureuna ja säkeistöjen välissä on
pystysuuntaista tilaa.

Jos Latexin valmiit lohkolainausympäristöt eivät tuota haluttua
lopputulosta, voi sisennetyt tekstikappaleet toteuttaa myös luetelmien
tekemiseen tarkoitetun \ymparisto{list}\-/ ympäristön avulla (luku
\ref{luku/list-ympäristö}). Sopivilla asetuksilla ''luetelma'' sisältää
ihan tavallisen näköisiä tekstikappaleita, jotka vain on sisennetty
vasemmalta tai oikealta tai molemmista reunoista.

\begin{esimerkki*}
  \ymparistoi{list}
  \komentoi{setlength}
  \mittai{leftmargin}
  \mittai{rightmargin}
  \mittai{itemindent}
  \mittai{listparindent}
  \mittai{parindent}
  \mittai{parsep}
  \mittai{topsep}
  \mittai{partopsep}
  \komentoi{item}
  \komentoi{linespread}
  \komentoi{small}

\begin{koodilohko}
\newenvironment{lohkolainaus}{%
  \begin{list}{}{
      \setlength{\leftmargin}{1cm}
      \setlength{\rightmargin}{1cm}
      \setlength{\itemindent}{0bp}
      \setlength{\listparindent}{\parindent}
      \setlength{\parsep}{\parskip}
      \setlength{\topsep}{1em}
      \setlength{\partopsep}{0bp}
    }
  \item\linespread{1}\small
  }{\end{list}}
\end{koodilohko}
  \caption{Lohkolainausten eli tekstikappaleen vasemman ja oikean
    sisennyksen toteutus \ymparisto{list}\-/ ympäristön avulla.
    Esimerkkikoodi määrittelee uuden ympäristön nimeltä
    \ymparistox{lohkolainaus}}
  \label{esim/vasen-oikea-sisennys}
\end{esimerkki*}

Esimerkistä \ref{esim/vasen-oikea-sisennys} selviää, kuinka
\ymparisto{list}\-/ ympäristöä voi käyttää sisennyksen toteuttamiseen.
Esimerkki määrittelee uuden ympäristön nimeltä
\ymparistox{lohkolainaus}, jota voi hyödyntää myöhemmin dokumentissa.

\begin{koodilohkosis}
\begin{lohkolainaus}
  Tämä tekstikappale on sisennetty vasemmalta ja oikealta. Lisäksi
  kirjainleikkaus on hieman pienempi (\small) kuin leipätekstissä.
\end{lohkolainaus}
\end{koodilohkosis}

\noindent
Omien ympäristöjen määrittelyä käsitellään tarkemmin luvussa
\ref{luku/ympäristöt}. Esimerkin \ref{esim/vasen-oikea-sisennys} rivillä
11 oleva \komento{item}\-/ komento on pakollinen, koska se aloittaa
\ymparisto{list}\-/ ympäristöön kuuluvan luetelman kohdan. Sen perässä
olevat komennot \komento{linespread} ja \komento{small} sen sijaan ovat
vapaaehtoisia. Ne ovat mukana siksi, että on varsin tavallista latoa
lohkolainaukset pienemmällä rivivälillä (rivikorkeudella) ja pienemmällä
kirjainleikkauksella kuin leipäteksti.

\subsection{Rivinvaihtokomennot}
\label{luku/rivinvaihtokomennot}

Latex\-/lähdedokumentissa olevat rivinvaihdot tulkitaan sanaväleiksi
siinä missä välilyönnitkin, eli ne rivinvaihdot eivät päädy ladottuun
dokumenttiin (luvut \ref{luku/sanaväli} ja
\ref{luku/rivinvaihtomerkit}). Sen sijaan ladottuun dokumenttiin saadaan
rivinvaihto käyttämällä komentoa \komento{\keno} eli kaksi kenoviivaa.
Komennon ei tarvitse sijaita lähdedokumentissa rivin lopussa.

\komentoi{\keno}
\begin{koodilohkosis}
ensimmäinen \\ toinen \\
kolmas
\end{koodilohkosis}

\begin{tulossis}
  ensimmäinen \\* toinen \\* kolmas
\end{tulossis}

\noindent
Rivinvaihtokomennolle voi antaa hakasulkeissa valinnaisen argumentin,
joka ilmaisee rivien väliin ladottavan ylimääräisen pystysuuntaisen
tilan. Argumentin on siis oltava mitta.

\komentoi{\keno}
\begin{koodilohkosis}
ensimmäinen \\ toinen \\[1.3ex] kolmas
\end{koodilohkosis}

\begin{tulossis}
  ensimmäinen \\* toinen \\*[1.3ex] kolmas
\end{tulossis}

\noindent
Komennosta on olemassa tähtiversio \komento{\keno *}, joka edellisten
ominaisuuksien lisäksi estää sivun vaihtumisen tämän rivinvaihdon
kohdalla. Myös tähtiversiolle voi antaa valinnaiseksi argumentiksi
mitan, ja sen merkitys on sama kuin komennon normaalilla versiollakin.

Rivin voi vaihtaa myös komennolla \komento{newline}, mutta tämä komento
ei hyväksy valinnaista argumenttia eikä siitä ole tähdellistä versiota.
Komennot \komento{newline} ja \komento{\keno} käyttäytyvät eri tavoin
taulukoissa, joita käsitellään luvussa \ref{luku/taulukot}.

\subsection{Lesket ja orvot}

Leski- ja orporivit tarkoittavat typografiassa rumannäköisiä yksinäisiä
rivejä. Leskirivi (\englanti{widow}) on tekstikappaleen viimeinen rivi,
joka on yksinään sivun tai palstan yläreunassa. Orporivi
(\englanti{orphan}) puolestaan on tekstikappaleen ensimmäinen rivi, joka
on yksin sivun tai palstan alareunassa. Molemmat voivat näyttää
ikävältä, mutta yleensä orporivejä ei pidetä kovin vakavana virheenä;
leskien välttämisessä ollaan enemmän tosissaan.

Ulkoasun lisäksi lesket ja orvot voivat olla ikäviä myös lukemisen
kannalta. Kun tekstikappale vaihtuu, lukija pitää pienen tauon ja
valmistautuu uuteen kappaleeseen. Sivun tai palstan olisi sopivaa
vaihtua samassa kohdassa eli tekstikappaleiden välissä, mutta leski- ja
orporivit aiheuttavat kaksi taukoa melkein peräkkäin: sekä kappaleiden
välissä että sivun tai palstan vaihtumisen kohdalla.

Latexissa leski- tai orporivit lienee käytännöllisintä estää
\pakettictan{nowidow}\-/ paketin avulla. Paketin lataamisen jälkeen
käytetään komentoa \komento{setnowidow}, joka estää leskirivit eli
pitää tekstikappaleen lopusta vähintään kaksi riviä yhdessä sivun tai
palstan yläreunassa. Komennolle voi antaa hakasulkeissa valinnaisen
argumentin (kokonaisluvun), joka ilmaisee, kuinka monta riviä täytyy
vähintään pysyä yhdessä. Vastaavasti orporivit estetään komennolla
\komento{setnoclub}, joka toimii samalla tavalla. Molemmat komennot
vaikuttavat koko dokumenttiin eli kaikkiin tekstikappaleisiin.

\komentoi{usepackage}
\pakettii{nowidow}
\komentoi{setnowidow}
\komentoi{setnoclub}
\begin{koodilohkosis}
\usepackage{nowidow}
\setnowidow   % leskirivien esto
\setnoclub    % orporivien esto
\end{koodilohkosis}

\noindent
Paketti \paketti{nowidow} määrittelee myös komennot \komento{nowidow}
ja \komento{noclub}, joilla voi vaikuttaa yksittäisen tekstikappaleen
leski- ja orporiveihin. Nämä komennot täytyy sijoittaa tekstikappaleen
loppuun, ja myös niille voi antaa hakasulkeissa valinnaiseksi
argumentiksi luvun, joka kertoo yhdessä pidettävien rivien määrän.

Jos ei halua tai voi käyttää \paketti{nowidow}\-/pakettia, voi leski- ja
orporivit estää myös Texin matalan tason toimintojen avulla.
Ladonta\-/algoritmi käyttää leski- ja orporiveissä sisäisesti
haitallisuusarvoa tai sakkoarvoa (\englanti{penalty}), ja jos lesket ja
orvot halutaan estää, määritellään niiden haitallisuusarvo
mahdollisimman korkeaksi. Käytännössä arvo 10\,000 tarkoittaa samaa kuin
ääretön, eli silloin lesken tai orvon haitallisuus on niin suuri, ettei
sellaisia sallita.

\komentoi{widowpenalty}
\komentoi{clubpenalty}
\begin{koodilohkosis}
\widowpenalty=10000  % leskirivien esto
\clubpenalty=10000   % orporivien esto
\end{koodilohkosis}

\noindent
Leskien ja orpojen haitallisuusarvoiksi voi kokeilla hieman pienempiäkin
lukuja. Silloin leski- tai orporivit voidaan sallia joissakin
tilanteissa, jos ladonta\-/algoritmi ei löydä parempaakaan ratkaisua.

Orvoksi kutsutaan myös tavua, joka jää yksin kappaleen viimeiselle
riville. Se on häiritsevän näköinen ainakin silloin, kun tavu on
kapeampi kuin seuraavan kappaleen ensimmäisen rivin sisennys.
Orpotavujen estämiseen ei taida olla automaattisia keinoja, mutta
kappaleen sanamäärää ja sanajärjestystä muuttamalla voi tietenkin
vaikuttaa rivien latomiseen. Kappaleen viimeiseen sanaan voi kirjoittaa
myös tavutusvihjeitä (\komento{-}) mutta jättää vihje pois ennen
viimeistä tavua. Näin estetään lyhyen orpotavun muodostuminen.
Tavutusvihjeitä ja muita tavutuksen asetuksia käsitellään
luvussa~\ref{luku/tavutus}.

\subsection{Marginaalihuomautukset}
\label{luku/marginaalihuomautukset}

Klassisessa kirjatypografiassa -- jota tämäkin opas suunnilleen
noudattaa -- on aukeaman ulkoreunoissa melko suuret marginaalit. Niitä
voidaan käyttää eräänlaisina apupalstoina, joihin voi kirjoittaa
lyhyehköjä huomautuksia ja lisätietoa.

Marginaalihuomautukset tehdään komennolla \komento{marginpar}, jonka
argumentiksi annetaan marginaaliin tuleva teksti. Marginaalin teksti
ladotaan vasempaan tai oikeaan marginaaliin sen mukaan, kumpi on
aukeaman ulkoreunassa. Yksipuolisilla sivuilla teksti ladotaan
oletuksena oikeaan marginaaliin. Mikäli sivu on kaksipalstaisessa
tilassa (luku \ref{luku/palstat}), marginaalihuomautus ladotaan
lähempänä olevaan marginaaliin.

\komentoi{marginpar}
\begin{koodilohkosis}
Tämä on leipätekstin tekstikappale,
\marginpar{Tämä ladotaan marginaaliin.}
joka ladotaan sivun normaalille tekstialueelle.
\end{koodilohkosis}

% Miten valinnainen argumentti toimii? \marginpar[left]{right}

\noindent
Yleensä marginaalihuomautukset kannattaa latoa erilaisella
kirjainleikkauksella kuin leipäteksti, jotta ne erottuvat toisistaan.
Jos leipätekstissä käytetään antiikvaa, voisi marginaalissa käyttää
esimerkiksi pienempää groteskia. Tämän \marginpar{\sffamily
  \linespread{1} \scriptsize \RaggedRight Tässä on pienellä groteskilla
  ladottu huomautus.} tekstikappaleen vieressä on esimerkki edellä
mainitun tyylisestä marginaalihuomautuksesta. Kirjainleikkauksen on
syytä olla selvästi pienempi ja riittävän erilainen kuin leipätekstissä,
jotta huomautus ei häiritse liikaa leipätekstin lukemista. Tällaisen
toteutusta varten kannattaa määritellä uusi komento, vaikkapa nimellä
\komentox{huomautus}, jota käyttämällä marginaalihuomautukset saa
helposti yhdenmukaiseksi. Seuraavassa on esimerkki tällaisen komennon
määrittelystä:

\komentoi{newcommand}
\komentoi{marginpar}
\komentoi{sffamily}
\komentoi{scriptsize}
\komentoi{RaggedRight}
\begin{koodilohkosis}
\newcommand{\huomautus}[1]{%
  \marginpar{\sffamily\scriptsize\RaggedRight #1}}
\end{koodilohkosis}

\noindent
Omien komentojen määrittelyä käsitellään tarkemmin luvussa
\ref{luku/komennot} ja fonttiasetuksia luvussa~\ref{luku/kirjaintyypit}.
Edellä olevassa komennon määrittelyssä on mukana myös palstan muotoon
vaikuttava komento \komento{RaggedRight}, josta on lisätietoa luvussa
\ref{luku/kappaleen-tasaus}.

Sivun marginaalissa olevan huomautuspalstan leveys voidaan määrittää
sivun asetusten yhteydessä. Asetuksia hoitavan \pakettictan{geometry}\-/
paketin valitsimella \koodi{marginparwidth} asetetaan palstan leveys ja
valitsimella \koodi{marginparsep} palstan etäisyys leipätekstistä.
Valitsimella \koodi{reversemarginpar} siirretään marginaalihuomautukset
päinvastaiseen marginaaliin. Näitä ja muitakin sivun asetuksia
käsitellään luvussa~\ref{luku/sivuasetukset}.

Edellä mainittuja asetuksia voi muuttaa myös kesken dokumentin. Mitta
\mitta{marginparwidth} määrittää marginaalihuomautuspalstan leveyden ja
mitta \mitta{marginparsep} palstan etäisyyden leipätekstistä. Lisäksi on
mitta \mitta{marginparpush}, jolla asetetaan peräkkäisten
marginaalihuomautusten vähimmäisetäisyys toisistaan. Seuraavassa on
esimerkki näiden mittojen asettamisesta:

\komentoi{setlength}
\mittai{marginparwidth}
\mittai{marginparsep}
\mittai{marginparpush}
\begin{koodilohkosis}
\setlength{\marginparwidth}{50bp}
\setlength{\marginparsep}  {10bp}
\setlength{\marginparpush}  {6bp}
\end{koodilohkosis}

\noindent
Kesken dokumentin komennolla \komento{reversemarginpar} voidaan vaihtaa
marginaalihuomautukset vastakkaiseen marginaaliin ja komennolla
\komento{normalmarginpar} palautetaan oletusasetukset.

Marginaalihuomautuksen ensimmäisen rivin peruslinja sijoittuu oletuksena
samalle kohdalle kuin rivi, jolla \komento{marginpar}\-/ komento
sijaitsee. Huomautukset ovat kuitenkin ovat jossain määrin leijuvia, eli
samaan kohtaan sattuvat huomautukset ladotaan allekkain, ja niiden
välissä on mitan \mitta{marginparpush} suuruinen pystysuuntainen tila.

Valitettavasti Latex ei osaa aina latoa marginaalihuomautuksia
siististi. Esimerkiksi sivun lopussa oleva monirivinen huomautus saattaa
sijoittua ikävän alas alamarginaaliin. Parempi olisi latoa se
kokonaisena seuraavan sivun alkuun. Sivun loppuun sattuva
\komento{marginpar}\-/ komento saattaa myös aiheuttaa kaksipuolisessa
asettelussa (luku \ref{luku/sivun-mitat}) sen, että huomautus ladotaan
seuraavalle sivulle mutta väärään marginaaliin -- ei siis aukeaman
ulkomarginaaliin, kuten yleensä pitäisi. Käytännössä kirjoittaja ei voi
luottaa siihen, että huomautukset ladotaan aina tyylikkäästi, vaan
lopputulosta täytyy vähän tarkkailla.

Vaihtoehto Latexin omille marginaalihuomautukselle on
\pakettictan{marginnote}\-/ paketin komento \komento{marginnote}. Se ei
tee leijuvia huomautuksia, vaan huomautus ladotaan juuri siihen kohtaan
kuin missä komento sijaitsee. Kirjoittaja voi siis tarkemmin vaikuttaa
huomautuksen sijoitteluun. Toisaalta komennosta puuttuu samaan kohtaan
sattuvien huomautuksen automaattinen sijoittaminen allekkain, eli
huomautukset voivat sijoittua myös toistensa päälle.
\komento{marginnote}\-/ komennon etuna on se, että sitä voi käyttää
leijuvien osien sisällä (luku \ref{luku/leijuosat}) ja alaviitteissäkin
(luku \ref{luku/alaviitteet}). Paketin ohjekirjassa kerrotaan muutamasta
hyödyllisestä huomautusten sijoitteluun vaikuttavasta asetuksesta.

\subsection{Anfangit eli suurikokoiset alkukirjaimet}

\lettrine[lines=3, loversize=.06, lhang=.02, findent=-5bp, nindent=4bp,
slope=4bp]{A}{nfangit} ovat suurikokoisia tekstin alkukirjaimia.
Tyypillisesti kirjaimen korkeus on kahdesta viiteen riviä ja se
upotetaan kappaleen sisään. Joskus anfangi sijoitetaan kokonaan
marginaalin puolelle tai se sijaitsee normaalisti tekstin peruslinjalla.
Anfangikirjain voidaan latoa erilaisella kirjainleikkauksella kuin
leipäteksti. Esimerkiksi vanhassa ja vanhantyylisessä typografiassa
anfangiin voi sopia erityisen koristeellinen kirjain.

\medbreak

\lettrine[lines=2, lhang=1, nindent=0bp]{E\hspace{1.5bp}}{dellä
  olevaan}, tähän ja seuraavaan tekstikappaleeseen anfangit sopivat
huonosti, eikä niitä tavallisesti käytetä kirjan alalukujen alussa eikä
väliotsikon jälkeen. Tässä ne ovatkin vain esimerkin vuoksi. Anfangien
käytön varsinainen tarkoitus on osoittaa tekstin alkamiskohta
esimerkiksi artikkelin alussa tai kirjan päälukujen alussa. Varsinkin
aikakauslehden artikkelit sisältävät monenlaisia visuaalisia elementtejä
kuten pää- ja alaotsikon, erillisen johdantokappaleen eli ingressin,
kuvia ja tietolaatikoita. Anfangin avulla katse löytää artikkelin alun
helposti.

\lettrine[lines=1, loversize=.1, lhang=.02, findent=1.5bp]{K}{appaleen
  aloittava sana} tai pidempikin ilmaus ladotaan joskus lihavoidulla
kirjainleikkauksella tai pienversaalilla, kuten näissä esimerkeissä on
tehty. Tällainen kappaleen aloitus voi toimia hieman otsikon tavoin tai
sitten muuten vain ilmaisee uuden luvun alkamista.

Latexissa anfangien tekemistä varten on paketti \pakettictan{lettrine},
joka tarjoaa käyttöön samannimisen komennon. Jokainen anfangi on
erilainen, ja siksi \komento{lettrine}\-/ komentokin sisältää useita
asetuksia anfangin ulkoasun hienosäätöön. Komentoa käytetään seuraavalla
tavalla:

\komentoi{lettrine}
\begin{koodilohkosis}
\lettrine[lines=3, loversize=.06, lhang=.02, findent=-5bp,
  nindent=4bp, slope=4bp]{A}{nfangi} aloittaa tämän tekstikappaleen.
\end{koodilohkosis}

\noindent
Edellä oleva esimerkki tekee kolmerivisen upotetun A\-/anfangin, jonka
perässä oleva sanan osa \emph{nfangi} ladotaan pienversaalilla.
Valitsimella \koodi{lover\-size} kasvatetaan hieman anfangikirjaimen
kokoa, ja \koodi{lhang}\-/ valitsimella siirretään kirjainta hieman
vasemmalle marginaalin puolelle. Valitsin \koodi{findent} määrittää
anfangikirjaimen rinnalla olevien rivien sisennyksen määrän suhteessa
kirjaimeen. \koodi{nindent} määrittää toisen ja sitä seuraavien rivien
sisennyksen suhteessa ensimmäiseen riviin, ja \koodi{slope} määrittää
sisennysportaan, joka kertautuu joka rivillä kolmannesta rivistä alkaen.
Näillä asetuksilla saadaan A\-/kirjaimen viistoa reunaa mukailevat
rivien alut.

Ennen \paketti{lettrine}\-/paketin käyttämistä kannattaa lukea paketin
ohjekirjasta tietoa \komento{lettrine}\-/ komennon valitsimista ja myös
teknisistä rajoituksista. Anfangit eivät toimi Latexissa teknisesti joka
paikassa -- ja typografisesti vielä harvemmassa.

\section{Tekstin korostaminen}
\label{luku/korostus}

Tekstin yksittäisten sanojen korostukset toteutetaan tavallisimmin
siten, että käytetään leipätekstin kirjainperheestä jotakin poikkeavaa
kirjainleikkausta, esimerkiksi \textit{kursiivia},
\textsl{kallistettua}, \textbf{lihavoitua} tai \textsc{pienversaalia}.
Joskus vaihdetaan koko kirjainperhe toiseksi, esimerkiksi ladotaan
\textrm{antiikvan} sekaan \textsf{groteskia} tai \texttt{tasalevyistä}.

Korostukset siis liittyvät läheisesti fonttien valintaan ja asetuksiin,
ja siksi korostusten teknistä toteuttamista ja Latex\-/komentoja
käsitellään pääasiassa fonttien yhteydessä luvussa
\ref{luku/fontit-korkea}. Siellä muun muassa neuvotaan, kuinka
kirjainperheet ja niiden eri kirjainleikkaukset valitaan. Sen sijaan nyt
käsillä olevassa luvussa keskitytään tekstin korostamisen typografisiin
käytäntöihin ja suosituksiin. Joitakin uusia komentojakin esitellään.

Tekstissä sanoja korostetaan siksi, että halutaan niiden erottuvan
ympäristöstään. Tärkeän avainsanan pitää ehkä erottua tekstipalstasta
niin, että sen huomaa nopeasti tekstiä silmäilemällä. Se voi helpottaa
tiedon löytämistä.

Sanoja tai ilmauksia korostetaan toisinaan myös siksi, että ne eivät ole
kielen sanoja tavallisessa merkityksessään vaan lainauksia kielestä tai
toisesta tekstityypistä. Ihmiskieltä käsittelevissä teksteissä on
tavallista kursivoida sanat silloin, kun viitataan itse sanoihin eikä
varsinaisesti käytetä kyseisiä sanoja. Sanoihin ja muuhun kieliainekseen
viitatessa voi tosin käyttää myös lainausmerkkejä. Tietotekniikan
koodikielen esimerkit ladotaan usein leipätekstistä poikkeavalla
fontilla, jotta käy selväksi, että kyse on koodikielisestä lainauksesta.
Tällöin hyvin tavallista on tasalevyisen fontin käyttö, koska se
jäljittelee tekstieditorien ja komentotulkkia suorittavien
pääteohjelmien fonttia.

Sopivan korostustavan valinnassa on olennaista saavuttaa riittävä
kontrasti ympäröivään tekstiin. Liian lievä muutos ei erotu; toisaalta
liian räikeä muutos voi olla häiritsevä. Yhteensopivat korostuskeinot
saadaankin lähes aina saman kirjainperheen sisältä, eri leikkauksia
käyttämällä.

\subsection{Kursiivi, kallistus ja lihavointi}
\label{luku/peruskorostukset}

Antiikvatyyppisen kirjainleikkauksen korostamiseen sopii yleensä
parhaiten saman perheen \textit{kursiivi} (\komento{textit}).
Lihavointi toimii huonommin antiikvan kanssa, koska pelkkä merkkien
viivojen vahvistaminen ei tuo riittävää kontrastia.

\begin{tulossis}
  \rmfamily Antiikvaperheen \textit{kursiivileikkaus} tuo yleensä
  voimakkaan muotokontrastin ja on tyylillisesti yhteensopiva
  korostuskeino. Sen sijaan \textbf{lihavointi} erottuu antiikvassa
  huonommin.
\end{tulossis}

\noindent
\textbf{Lihavointi} (\komento{textbf}) toimii huomattavasti paremmin
groteskityyppisen kirjainperheen kanssa, koska lihavoitu leikkaus on
usein selvästi vahvempi normaaliin verrattuna. Groteskin kursiivi sen
sijaan ei monesti edes ole varsinainen kursiivileikkaus vaan pelkkä
lievä kallistus (\komento{textsl}). Sellainen ei välttämättä tuo
riittävää muotokontrastia.\footnote{Joissakin niin sanotuissa
  humanistisissa groteskiperheissä on mukana ihan käyttökelpoinen ja
  selvästi erottuva kursiivileikkaus.}

\begin{tulossis}
  \sffamily Groteskia on tavallista korostaa \textbf{lihavoinnilla} eli
  leikkauksella, joka on selvästi tavallista vahvempi. Sen sijaan
  \textit{kursiivi} tai \textsl{kallistus} ei kaikissa groteskiperheissä
  erotu riittävästi.
\end{tulossis}

\noindent
Joissakin kirjainperheissä on mukana useita erivahvuisia leikkauksia.
Silloin täytyy valita käyttöön riittävästi toisistaan poikkeavat
vahvuudet. Esimerkiksi kaksi vierekkäistä vahvuutta eivät välttämättä
poikkea toisistaan riittävästi, eikä selvää vahvuuskontrastia synny.

Mikäli leipätekstissä käytetään sulkeita ja niiden sisällä kursivoitua
ilmausta (\textit{kuten tässä}), ei itse suljemerkkejä pidä kursivoida,
koska ne kuuluvat ulompaan tekstiin. Jos sulkeiden ulkopuolellakin on
voimassa kursivointi, voi itse sulkeetkin kursivoida.

Latexissa kursiivin voi toteuttaa myös komennolla \komento{emph}, joka
huomioi ympärillä olevan korostustilan. Tavallisesti komento kursivoi
argumentiksi annetun tekstin, mutta jos kursivointi on jo päällä
komennon ympärillä, se latookin argumentiksi annetun tekstin
tavallisella pystyasentoisella kirjainleikkauksella. Komennon ajatuksena
on, että voidaan samalla komennolla korostaa sekä pystyasentoisen
tekstin että kursiivin keskellä, eikä tarvitse välittää tai tietää,
kumpi leikkaus on voimassa.

\komentoi{emph}
\begin{koodilohkosis}
tavallinen \emph{korostettu} tavallinen \\
\emph{korostettu \emph{kaksoiskorostettu} korostettu}
\end{koodilohkosis}

\begin{tulossis}
  tavallinen \emph{korostettu} tavallinen \\
  \emph{korostettu \emph{kaksoiskorostettu} korostettu}
\end{tulossis}

\subsection{Pienversaali eli kapiteeli}
\label{luku/korostus-pienversaali}

Melko rauhallinen typografinen keino on \textsc{pienversaali} eli
kapiteeli (\komento{textsc}), jonka kirjaimet näyttävät
versaalikirjaimilta mutta ovat suunnilleen gemenakirjainten eli pienten
kirjainten korkuisia. Pienversaali on fonttiin sisältyvä erillinen
suunnittelijan piirtämä leikkaus, eli se ei ole tietokoneen mekaanisesti
pienentämä versio versaalikirjaimista.

Jos fonttiin ei sisälly pienversaalia ja sellaisen silti haluaa omaan
keinovalikoimaansa, voi yrittää vähemmän tyylikästä mekaanista
toteutusta luvussa \ref{luku/fontit-keinopienversaali} kerrotuilla
asetuksilla. Luvuissa \ref{luku/fontit-välistys} ja
\ref{luku/korostus-harvennus} puolestaan käsitellään merkkivälien
{\scshape\addfontfeatures{LetterSpace=6} harvennusta}, joka voi
maltillisesti käytettynä sopia pienversaaleihin.

Korostustarkoituksessa pienversaalia käytetään esimerkiksi pienissä
väliotsikoissa, koska sen avulla syntyy selvä muotokontrasti
leipätekstin kanssa. Toisinaan tekstikappaleen ensimmäisiä sanoja
ladotaan pienversaalilla, jolloin se toimii otsikon kaltaisessa
tehtävässä. Myös kirjallisuus\-/{} ja lähdeluetteloissa käytetään
välillä pienversaalia tekijöiden nimissä, koska se helpottaa nimen
mukaan aakkostetun luettelon silmäilyä.

Pienversaalia käytetään toisinaan versaalikirjainten sijasta. Silloin
tarkoituksena ei ole tekstin korostamiseen vaan päinvastoin häiritsevän
korostuksen välttäminen. Esimerkiksi versaalikirjaimiset lyhenteet (SPR,
HTML) voivat paljon käytettynä erottua leipätekstistä ikävän selvästi,
eikä tekstipalsta näytä tasaiselta. Pienversaalit sulautuvat muuhun
tekstiin paremmin (\textsc{spr}, \textsc{html}).

Typografisesta estetiikasta juontuu sekin, että pienversaali on sopiva
pari gemenanumeroiden kanssa erilaisiin teknisiin koodeihin ja muihin
ilmauksiin, jotka koostuvat kirjaimista ja numeroista. Sen sijaan
versaalikirjaimia ja gemenanumeroita ei sovi käyttää yhdessä, koska
merkkien kokoero on häiritsevän suuri. Kuten taulukosta
\ref{tlk/versaali-pienversaali-vertailu} näkyy, versaalit ovat joskus
miltei kaksinkertaisia gemenanumeroihin verrattuna, ja muutenkin
gemenanumerot asettuvat keskimäärin eri korkeustasolle.

\leijutlk{
  \setlength{\tabcolsep}{3bp}
  \renewcommand{\arraystretch}{1.2}
  \begin{tabular}{llll}
    \toprule
    & \multicolumn{2}{l}{\ots{Esimerkit}}
    & \ots{Kirjainten ja numeroiden muoto} \\
    \midrule
    \otsrivi{Väärin:}
    & {\gemenanum ISO-8859-1}
    & {\gemenanum 2012 a.D.}
    & versaalikirjaimet ja gemenanumerot \\
    \otsrivi{Oikein:}
    & {\gemenanum\scshape iso-8859-1}
    & {\gemenanum\scshape 2012 a.d.}
    & pienversaalikirjaimet ja gemenanumerot \\
    \otsrivi{Oikein:}
    & {\versaalinum ISO-8859-1}
    & {\versaalinum 2012 A.D.}
    & versaalikirjaimet ja versaalinumerot \\
    \bottomrule
  \end{tabular}
}{
  \caption{Versaalikirjainten, pienversaalin sekä gemena\-/\ ja
    versaalinumeroiden vertailua. Versaalikirjaimet ja gemenanumerot
    eivät sovi yhteen}
  \label{tlk/versaali-pienversaali-vertailu}
}

Fontin näkökulmasta gemenanumerot ovat erillinen ominaisuus, jonka
fontin suunnittelija on toteuttanut. Ominaisuuden käyttöönottoa ja muita
numeroihin liittyviä fonttiasetuksia käsitellään luvussa
\ref{luku/fontit-numerot}.

\subsection{Alleviivaus ja yliviivaus}

Alleviivaus on käsin kirjoitetun tekstin ja mekaanisten
kirjoituskoneiden korostuskeino, mutta nykyajan typografiaan se ei
oikeastaan sovi. Alleviivaus nimittäin vaikeuttaa kirjainhahmojen
tunnistamista ja hidastaa lukemista. Parempia korostuskeinoja ovat
esimerkiksi kursiivi ja lihavointi. Alleviivaus voi kuitenkin joskus
olla kätevä keino ohjata lukijan huomio tiettyyn kohtaan esimerkiksi
pidemmässä esimerkkitekstissä.

Latexin oma alleviivauskomento \komento{underline} piirtää viivan
tekstin tai muun argumentiksi annetun sisällön alapuolelle. Komennon
huono puoli on, että erilliset alleviivaukset eivät välttämättä osu
samalle korkeudelle. Seuraavassa esimerkissä toisen sanan alleviivaus on
alempana j- ja y\=/kirjainten alapidennyksen vuoksi.

\komentoi{underline}
\begin{koodilohkosis}
\underline{kone}, \underline{öljy}
\end{koodilohkosis}

\begin{tulossis}
  \underline{kone}, \underline{öljy}
\end{tulossis}

\noindent
Toinen \komento{underline}\-/ komennon ongelma on, että se estää sanojen
tavuttamisen ja muutenkin tekstin katkaisemisen rivin lopussa. Komennon
argumentiksi annettua tekstiä ei katkaista edes sanavälien kohdalta.
Komennon hyvä puoli on, että se toimii myös matematiikkatilassa
(luku~\ref{luku/matematiikka}).

Käytännössä alleviivaukset on parempi toteuttaa esimerkiksi paketin
\pakettictan{ulem} avulla. Tavallisen alleviivauksen lisäksi se sisältää
kaksoisalleviivauksen, aaltoviivat, pisteviivat ja erilaisia
ylikirjoitusmerkintöjä. Taulukkoon \ref{tlk/ulem} on koottu paketin
sisältämiä komentoja.

\leijutlk{
  \providecommand{\rivi}{}
  \renewcommand{\rivi}[2]{\komento{#1} & #2}
  \renewcommand{\arraystretch}{1.2}
  \begin{tabular}{llll}
    \toprule
    \ots{Komento}
    & \ots{Merkitys} & \ots{Komento} & \ots{Merkitys} \\
    \cmidrule(r){1-2}
    \cmidrule(l){3-4}
    \rivi{uline}{\uline{alleviivaus}}
    & \rivi{uuline}{\uuline{kaksoisalleviivaus}} \\
    \rivi{uwave}{\uwave{aaltoviiva}}
    & \rivi{sout}{\sout{yliviivaus}} \\
    \rivi{dashuline}{\dashuline{katkoviiva}}
    & \rivi{xout}{\xout{poisto}} \\
    \rivi{dotuline}{\dotuline{pisteviiva}} \\
    \bottomrule
  \end{tabular}
}{
  \caption{Alleviivaus- ja ylikirjoitusmerkintöjä \paketti{ulem}\-/
    paketin komentojen avulla}
  \label{tlk/ulem}
}

Oletuksena \paketti{ulem}\-/ paketti määrittelee Latexin
\komento{emph}\-/ korostuskomennon (luku \ref{luku/peruskorostukset})
uudelleen siten, että komento tuottaa kursiivin sijasta alleviivattua
tekstiä. Ajatuksena lienee se, että paketin avulla voi helposti muuttaa
typografisen peruskorostusten eli kursiivin tilalle kirjoituskoneen
korostustavan eli alleviivauksen. Käytännössä tällaista tuskin halutaan
koskaan, ja onneksi paketin lataamisen yhteydessä voi käyttää valitsinta
\koodi{nor\-mal\-em}, joka estää Latexin komentojen uudelleen
määrittelyn.

\pakettii{ulem}
\komentoi{usepackage}
\begin{koodilohkosis}
\usepackage[normalem]{ulem}
\end{koodilohkosis}

\noindent
Paketin \paketti{ulem} myötä on käytettävissä mitta \mitta{ULdepth},
joka on alleviivauksen etäisyys tekstin peruslinjasta. Mitta asetetaan
\komento{setlength}\-/ komennolla (luku \ref{luku/mitat}). Viivan
paksuus puolestaan sisältyy komennon \komento{ULthickness} määritelmään,
ja komennon voi halutessaan määritellä uudestaan. Seuraavassa on näistä
esimerkki:

\komentoi{setlength}
\mittai{ULdepth}
\komentoi{renewcommand}
\komentoi{ULthickness}
\komentoi{uline}
\begin{koodilohkosis}
\setlength{\ULdepth}{.2ex}        % viivan etäisyys
\renewcommand{\ULthickness}{.1ex} % viivan paksuus
\uline{kone}, \uline{öljy}
\end{koodilohkosis}

\begin{tulossis}
  \setlength{\ULdepth}{.2ex}
  \renewcommand{\ULthickness}{.1ex}
  \uline{kone}, \uline{öljy}
\end{tulossis}

\noindent
Toinen paketti alleviivausten ja yliviivausten tekemiseen on
\pakettictan{soul}. Se ei sisällä erikoisempia korostustapoja kuten
aalto- ja katkoviivoja, mutta viivan etäisyyden ja paksuuden
määrittämiseen on hieman kätevämmät komennot.

Tekstin alleviivauksen lisäksi on olemassa pari komentoa, joilla saa
ladottua pelkän viivan ilman tekstiä. Komento \komento{hrulefill} latoo
kaiken tilan täyttävän alaviivan ja komento \komento{dotfill} kaiken
tilan täyttävän pisteviivan. Myös \komento{rule}\-/ komento voi soveltua
alaviivojen toteuttamiseen. Komentoa käsitellään erikoismerkkien
yhteydessä luvussa \ref{luku/tarkkeet}.

\komentoi{hrulefill}
\komentoi{dotfill}
\komentoi{rule}
\begin{koodilohkosis}
abc \hrulefill\ abc \dotfill\ abc \rule[-.3ex]{5em}{.4bp} abc
\end{koodilohkosis}

\begin{tulossis}
  abc \hrulefill\ abc \dotfill\ abc \rule[-.3ex]{5em}{.4bp} abc
\end{tulossis}

\subsection{Harvennus ja tiivistys}
\label{luku/korostus-harvennus}

Merkkivälien harventaminen on tyyli- tai korostuskeino, joka juontuu
antiikin roomalaisten kiveen hakatuista teksteistä. Historiansa vuoksi
harvennus sopii luontevimmin antiikvan eli pääteviivallisen
kirjainperheen versaali\-/\ tai pienversaalikirjainten (luku
\ref{luku/korostus-pienversaali}) kanssa. Antiikin aikana
gemenakirjaimet eivät olleet vielä käytössä. Tässä alaluvussa
keskitytään harventamisen ja tiivistämisen typografiaan; teknisiä
ohjeita on luvussa \ref{luku/fontit-välistys}.

\komentoi{large}
\komentoi{addfontfeatures}
\begin{koodilohkosis}
\large TAVALLINEN \\
{\addfontfeatures{LetterSpace=10} KLASSINEN HARVENNUS}
\end{koodilohkosis}

\begin{tulossis}
  \large
  TAVALLINEN \\
  {\addfontfeatures{LetterSpace=10} KLASSINEN HARVENNUS}
\end{tulossis}

\noindent
Klassisesti eli antiikin tyyliin harvennetut (n.~10\,\%)
versaalikirjaimet voivat sopia joihinkin otsikoihin. Sen sijaan
pienversaalit lievästi {\addfontfeatures{LetterSpace=6}
  \textsc{harvennettuna}} (esim.~6\,\%) voivat olla tyylikäs
tehostuskeino leipätekstiin.

Edellisiä esimerkkejä suurempi harvennus on huomiota herättävä
korostus\-/\ tai tehokeino, joka voi sopia yksittäisiin sanoihin
leipätekstin ulkopuolella. Jos kirjaimet ovat hyvinkin kaukana
toisistaan, täytyy sanan ympärillä olla runsaasti tyhjää tilaa, jotta
sanahahmo pysyy visuaalisesti koossa.

\komentoi{sffamily}
\komentoi{bfseries}
\komentoi{addfontfeatures}
\begin{koodilohkosis}
{\sffamily\bfseries\addfontfeatures{LetterSpace=40} SEIS!}
\end{koodilohkosis}

\begin{tulossis}
  {\sffamily\bfseries\addfontfeatures{LetterSpace=40} SEIS!}
\end{tulossis}

\noindent
Merkkivälien tiivistäminen on joskus tarpeen suurikokoisissa otsikoissa.
Fontin merkkivälit on ehkä suunniteltu ensisijaisesti pienten kokojen
ehdoilla, ja siksi reilusti suuremmilla fonttiko'oilla välit voivat olla
häiritsevän suuria. Erityisesti tämä pätee groteskiin eli
pääteviivattomaan kirjainperheeseen, kuten kuva
\ref{kuva/otsikon-tiivistys} osoittaa. Antiikvat eivät yleensä siedä
tiivistämistä, koska kirjainten pääteviivat asettuvat helposti
päällekkäin.

\leijukuva{
  \sffamily\fontsize{50bp}{50bp}\selectfont
  Tavallinen \\
  \addfontfeatures{LetterSpace=-5} Tiivistetty
}{
  \caption{Otsikoiden merkkivälien tiivistäminen voi olla tarpeen
    suurikokoisen groteskin kanssa -- varsinkin jos lihavointi ei ole
    käytössä}
  \label{kuva/otsikon-tiivistys}
}

\subsection{Tietokonekoodi ja tasalevyinen kirjainperhe}

Tietokonekoodi tai muu vastaava tekninen, koneelle syötettävä tai koneen
tulostama koodi on tapana esittää \texttt{tasalevyisellä} fontilla.
Kyseessä on ikään kuin lainaus toisenlaisesta tekstityypistä, joka ei
ole luonnollista kieltä lainkaan.

Tasalevyinen kirjainperhe määritellään fonttiasetusten yhteydessä (luku
\ref{luku/fontin-valinta}) ja valitaan käyttöön esimerkiksi komennolla
\komento{texttt}. Tasalevyisellä fontilla ladottava teksti on
tavallisesti tietokonekoodia, joten sille on asetettu hieman
toisenlaiset oletusasetukset kuin muille kirjainperheille. Oletuksena
tekstiä ei tavuteta, ja esimerkiksi \komento{texttt}\-/ komennon
argumentiksi annettu teksti katkaistaan vain sanavälien ja yhdysmerkkien
kohdalta. Myös niin sanotut Tex\-/ligatuurit (esim. \koodi{''}
ja~\koodi{--}) on kytketty pois päältä. Fonttien oletusasetuksia ja
niiden muuttamista käsitellään luvussa
\ref{luku/fontit-oletusasetukset}. Tex\-/ligatuureja eli Texin
merkintätapoja lainausmerkeille ja ajatusviivoille käsitellään luvuissa
\ref{luku/lainausmerkit} ja \ref{luku/yhdys-ajatus-miinus}.

Toinen vaihtoehto tasalevyisen tekstin tuottamiseen on
\komento{verb}\-/ komento. Komennon nimi on lyhennetty englannin kielen
sanasta \englantik{verbatim}, joka tarkoittaa 'kirjaimellisesti' tai
'sananmukaisesti'. Niin se juuri toimiikin, eli \keno{verb}\-/
komennolle annettu argumentti tulkitaan kirjaimellisesti.
Tex\-/järjestelmän omat merkintätavat kuten kenoviivalla (\koodi{\keno})
alkavat komennot eivät toimi tämän komennon argumentissa.

\komento{verb}\-/ komennon argumentti täytyy ilmaista tavallisesta
poikkeavalla tavalla. Välittömästi komennon nimen jälkeen tuleva merkki
on oleva argumentin aloitus- ja lopetusmerkki. Niiden välissä oleva
teksti tulkitaan kirjaimellisesti ja ladotaan tasalevyisellä
kirjainperheellä. Seuraavassa on tästä kaksi esimerkkiä:

\komentoi{verb}
\begin{koodilohkosis}
\verb.tasalevyinen.   % aloitus- ja lopetusmerkkinä piste
\verb|tasalevyinen|   % aloitus- ja lopetusmerkkinä pystyviiva
\end{koodilohkosis}

\begin{tulossis}
  \verb.tasalevyinen.
  \verb|tasalevyinen|
\end{tulossis}

\noindent
Yksittäisiä sanoja pidempien tietokonekoodien latomiseen on olemassa
\ymparisto{verbatim}\-/ ympäristö, jonka sisällä kaikki teksti
tulkitaan kirjaimellisesti -- myös välilyönnit ja rivinvaihdot.

\ymparistoi{verbatim}
\begin{koodilohkosis}
\begin{verbatim}
tieto-       kone-
       koodia
\end{verbatim}
\end{koodilohkosis}

\begin{tulossis}
\begin{verbatim}
tieto-       kone-
       koodia
\end{verbatim}
\end{tulossis}

\noindent
Paljon monipuolisemman ympäristön tarjoaa paketti
\pakettictan{fancyvrb}. Sen mukana tuleva \ymparisto{Verbatim}\-/
ympäristö ymmärtää monenlaisia asetuksia, joilla vaikutetaan
tietokonekoodin latomiseen. Koodiin saa mukaan esimerkiksi rivinumerot,
tai sen voi latoa kehyksen sisään. Myös koodin sisennykseen, fontin
kokoon ja väreihin voi vaikuttaa.

\subsection{Verkko-osoitteet}

Verkko\-/osoitteiden eli internetin linkkien latominen voi olla
pienoinen typografinen ongelma: osoitteet ovat pitkiä, eivätkä ne
sisällä välilyöntejä, joista ne voisi luontevasti katkaista rivin
vaihtumisen kohdalla. Tavuviivojen lisääminen rivin loppuun voi olla
ongelmallista, koska kyse on teknisestä merkkijonosta, joka pitäisi
tulkita kirjaimellisesti.

Parasta olisi latoa verkko\-/osoitteet leipäteksti ulkopuolelle,
esimerkiksi sivun alareunaan alaviitteeksi (luku
\ref{luku/alaviitteet}). Tällä tavoin pitkät ja joskus rumat osoitteet
eivät häiritse leipätekstin lukemista mutta ovat silti lukijan
saatavilla.

Latexissa verkko\-/osoitteet kannattaa toteuttaa siihen tarkoitetun
paketin avulla. Paketti \pakettictan{hyperref} sisältää toimintoja
paitsi verkko\-/ osoitteiden latomiseen mutta myös pdf\-/tiedoston
asetusten (luku \ref{luku/pdf-asetukset}) ja ristiviitteiden (luku
\ref{luku/ristiviitteet}) hallintaan. Paketin avulla valmiin pdf\-/
tiedoston verkkolinkit toimivat, eli pdf\-/ lukijassa linkkiä
napsauttamalla pitäisi avautua verkkoselain ja linkin osoittama sivu.

Paketin käyttämisen typografinen hyöty on se, että rivin vaihdon
kohdalle sattuvat verkko\-/ osoitteet katkaistaan automaattisesti
luontevista kohdista eli vain tiettyjen välimerkkien jälkeen.
Tavuviivoja ei lisätä rivin loppuun. Myös typografiset ligatuurit (luku
\ref{luku/typo-liga}) kytketään pois päältä.

\paketti{hyperref}\-/paketin ohjekirja neuvoo lataamaan paketin
viimeisenä eli muiden pakettien lataamisen jälkeen, koska se
muokkaa joidenkin muualla määriteltyjen komentojen ominaisuuksia.

\komentoi{usepackage}
\pakettii{hyperref}
\komentoi{hypersetup}
\komentoi{urlstyle}
\begin{koodilohkosis}
\usepackage{hyperref}  % paketin lataaminen
\hypersetup{hidelinks} % linkeistä pois kehykset
\urlstyle{same}        % linkkien fontti
\end{koodilohkosis}

\noindent
Edellisen esimerkin komennolla \komento{hypersetup} vaikutetaan
paketin asetuksiin. Oletuksena esimerkiksi linkkien ympärille
ladotaan värilliset kehykset, mutta valitsin \koodi{hide\-links} poistaa
ne. Valitsimella \koodi{url\-color} voisi vaihtaa verkko\-/osoitteiden
kehyksen väriä.

Esimerkin seuraava komento \komento{urlstyle} määrittää, mitä
kirjainperhettä käytetään verkko\-/ osoitteiden latomiseen. Komennon
argumentti \koodi{same} tarkoittaa, että käytetään samaa kirjainperhettä
kuin ympärillä olevassa tekstissä. Tietty kiinteä kirjainperhe valitaan
\komento{urlstyle}\-/ komennon argumentilla \koodi{rm} (antiikva,
roman), \koodi{sf} (groteski, sans serif) tai \koodi{tt} (tasalevyinen).
Oletusasetus on tasalevyinen kirjainperhe.

Verkko\-/ osoitteiden ilmaisemisen peruskomento on \komento{href},
jolle annetaan kaksi pakollista argumenttia: ensimmäinen on varsinainen
osoite ja toinen on dokumenttiin ladottava teksti. Linkin teksti voi
siis olla mitä hyvänsä tekstiä, ja pdf\-/ lukijassa tekstiä
napsauttamalla linkin osoittama sivusto avautuu.

\komentoi{href}
\begin{koodilohkosis}
\href{https://www.ctan.org/}{CTAN-sivusto}
\end{koodilohkosis}

\begin{tulossis}
  \href{https://www.ctan.org/}{CTAN-sivusto}
\end{tulossis}

\noindent
Kun halutaan latoa dokumenttiin itse verkko\-/osoite, käytetään komentoa
\komento{url}, joka osaa katkaista osoitteet järkevällä tavalla rivin
lopussa. Komennolle annetaan vain yksi argumentti, verkko\-/osoite.

\komentoi{url}
\begin{koodilohkosis}
\url{https://www.ctan.org/}
\end{koodilohkosis}

\begin{tulossis}
  \url{https://www.ctan.org/}
\end{tulossis}

\noindent
Joskus dokumenttiin voidaan tarvita verkko\-/osoite, joka ei ole pdf:ssä
napsautettava linkki. Silloin käytetään komentoa \komento{nolinkurl}.
Tämän komennon argumenttina oleva verkko\-/osoite ainoastaan ladotaan
osoitteiden tavoin.

Linkkinä toimivien sähköpostiosoitteiden latominen vaatii
\komento{href}- ja \komento{nolinkurl}\-/ komentojen yhdistämistä.
Linkin alkuun täytyy lisätä sana \koodi{mailto:}, mutta tuota sanaa
tuskin halutaan näkyviin itse dokumenttiin. Niinpä sähköpostiosoite
kannattaa kirjoittaa seuraavalla tavalla:

\komentoi{href}
\komentoi{nolinkurl}
\begin{koodilohkosis}
\href{mailto:tunnus@osoite.net}{\nolinkurl{tunnus@osoite.net}}
\end{koodilohkosis}

\noindent
Osoite pitää siis kirjoittaa kaksi kertaa: itse linkkiä varten ja
dokumenttiin ladottavaa tekstiä varten. Omaa työtä kannattaa helpottaa
määrittelemällä sähköpostiosoitteita varten oma komento, esimerkiksi
seuraavalla tavalla:

\komentoi{newcommand}
\komentoi{href}
\komentoi{nolinkurl}
\begin{koodilohkosis}
\newcommand{\sposti}[1]{\href{mailto:#1}{\nolinkurl{#1}}}
\end{koodilohkosis}

\subsection{Värit}

Eri värien käyttö voi sopia tekstin erilaisten osien korostamiseen.
Esimerkiksi otsikot (luku \ref{luku/otsikot}), taulukot (luku
\ref{luku/taulukot}) tai leijuvat osat (luku \ref{luku/leijuosat}) saa
erottumaan selvemmin, kun niiden toteutuksessa käyttää värejä. Sen
sijaan leipätekstissä värejä kannattaa käyttää hillitymmin, sillä
värikäs tekstipalsta näyttää rauhattomalta. Katse voi hakeutua
poikkeavan värisiin osiin liian helposti, mikä hidastaa lukemista.
Värien käsittelyn tekniikkaa käsitellään luvussa \ref{luku/värit} ja
sivun taustaväriä luvussa \ref{luku/sivun-väri}.

Tämän oppaan leipätekstissä on käytetty värejä Latexin komentojen ja
muiden rakenteiden ilmaisemiseen, esimerkiksi \komentox{komento},
\ymparistox{ympäristö} ja \mittax{mitta}. Värisävyt on melko tummia,
jotta mustaa tekstiä sisältävä tekstipalsta näyttäisi verrattain
tasaiselta eikä katse kiinnittyisi värillisiin kohtiin kovin helposti.
Samanlaista ajatusta kannattanee toteuttaa muidenkin elementtien
värittämisessä.

\section{Sivunvaihdot ja sivujen tasaaminen}
\label{luku/sivunvaihdot}

Kirjoittajan ei normaalisti tarvitse huolehtia sivun vaihtumisesta
lainkaan, vaan järjestelmä pyrkii latomaan hyvännäköisiä sivuja
automaattisesti. Välillä tietenkin halutaan aloittaa jokin uusi aihe
puhtaalta sivulta, ja sitä varten Latexissa on muutama komento.
Mainittakoon kuitenkin heti aluksi, että otsikoiden yhteyteen saa
sivunvaihdon kätevimmin keinoilla, jotka neuvotaan otsikoiden yhteydessä
luvussa \ref{luku/otsikot-ulkoasu}.

Komento \komento{clearpage} vaihtaa sivua, eli komennon jälkeinen
dokumentin osa alkaa aina puhtaalta sivulta. Lähes samanlainen komento
on \komento{cleardoublepage}, joka kaksipuolisessa dokumentissa
aloittaa uuden sisällön aina parittomilta sivuilta. Jos seuraavana ei
satu olemaan pariton sivu, komento tekee väliin tyhjän sivun. Tätä
ominaisuutta on käytetty tässä oppaassa: pääluvut alkavat aina
oikeanpuoleiselta eli parittomalta sivulta.

\komento{cleardoublepage}\-/komennon aiheuttamalle tyhjälle sivulle
saatetaan kuitenkin latoa ylä- ja alatunnisteet, jos sellaiset on
määritelty (luku \ref{luku/ylä-ala-tunnisteet}). Täysin tyhjän sivun saa
käyttämällä apuna pakettia \pakettictan{titlesec} ja sen valitsinta
\koodi{clear\-empty}:

\komentoi{usepackage}
\pakettii{titlesec}
\begin{koodilohkosis}
\usepackage[clearempty]{titlesec}
\end{koodilohkosis}

\noindent
Edellä mainittu \paketti{titlesec}\-/ paketti sisältää monenlaisia
toimintoja otsikoiden muotoiluun, joten sitä käsitellään tarkemmin
otsikoiden yhteydessä luvussa \ref{luku/otsikot-ulkoasu}.

Komennoilla \komento{clearpage} ja \komento{cleardoublepage} on lisäksi
sellainen ominaisuus, että ne pakottavat aiemmin määritellyt leijuvat
osat (luku \ref{luku/leijuosat}) ladottavaksi ennen uutta puhdasta
sivua.

Hieman toisenlainen sivunvaihtokomento on \komento{newpage}. Se sallii
aiempien leijuvien osien latomisen myös sivunvaihdon jälkeen.
\komento{newpage} eroaa edellisistä sivuvaihtokomennoista myös siten,
että kaksipalstaisessa tilassa (luku \ref{luku/palstat}) se aloittaa
vain uuden palstan, ei välttämättä uutta sivua.

Sivunvaihdon todennäköisyyteen tietyssä kohdassa voi vaikuttaa
komennoilla \komento{pagebreak} ja \komento{nopagebreak}. Kummallekin
voi antaa hakasulkeissa valinnaisen argumentin, joka on luku 0\==4.
Nolla tarkoittaa pienintä todennäköisyyttä sivunvaihdolle
(\komento{pagebreak}\komentoargv{0}) tai sen välttämiselle
(\komento{nopagebreak}\komentoargv{0}). Luku neljä (joka on oletusarvo)
tarkoittaa, että sivunvaihto pakotetaan tai estetään siinä kohdassa.
Näiden komentojen tarkoitus on kertoa ladonta\-/ algoritmille vihje,
kuinka toivottavaa sivunvaihto on tietyssä kohdassa. Jos ei käytä
suurinta arvoa neljä~(4), algoritmi voi löytää omasta mielestään
paremmankin sivunvaihtokohdan.

\komentoi{pagebreak}
\komentoi{nopagebreak}
\begin{koodilohkosis}
\pagebreak[3]  % Lisätään sivunvaihdon todennäköisyyttä.
\nopagebreak   % Estetään sivunvaihto (oletusarvo 4).
\end{koodilohkosis}

\noindent
Jos \komento{pagebreak}\-/ komentoa käyttää tekstikappaleen sisällä,
sivunvaihto ei välttämättä satu juuri komennon kohdalle. Sivu vaihdetaan
kohdalle osuvan rivin lopussa.

Oletuksena Latex pyrkii latomaan kaikkien sivujen ylä- ja alareunan
tasaiseksi. Toisin sanoen se saattaa venyttää tai kutistaa sivulla
olevia pystysuuntaisia välejä niin, että kaikki sivut näyttävät yhtä
korkeilta. Tämä tila voidaan asettaa päälle komennolla
\komento{flushbottom}, mutta se on jo oletuksena päällä ainakin Latexin
perusdokumenttiluokissa.

Vaihtoehtoinen tila asetetaan komennolla \komento{raggedbottom}.
Komennon seurauksena sivujen alareunoja ei tasata. Toisin sanoen sivun
pystysuuntaisiin väleihin ei kosketa, joten eri sivujen alareunat voivat
sattua eri korkeudelle.

Dokumentin viimeistelyvaiheessa sivujen hienosäätöön sopinee komento
\komento{enlargethispage}, jolla voi muuttaa käsillä olevan sivun
korkeutta, käytännössä \mitta{textheight}\-/mittaa. Komennolle annetaan
argumentiksi positiivinen tai negatiivinen mitta. Seuraavassa on
esimerkkejä:

\komentoi{enlargethispage}
\mittai{baselineskip}
\begin{koodilohkosis}
\enlargethispage{12bp}          % 12 pistettä korkemmaksi
\enlargethispage{-4bp}          % 4 pistettä matalammaksi
\enlargethispage{\baselineskip} % yksi tekstirivi lisää
\end{koodilohkosis}

\noindent
Komennon tähtiversio \komento{enlargethispage*} pyrkii lisäksi
tiivistämään sivun sisältöä pystysuunnassa niin paljon kuin mahdollista.
Sen tyypillisin käyttötarkoitus lienee se, että sivulle pyritään
mahduttamaan yksi ylimääräinen rivi. Tämäntyyppinen sivujen hienosäätö
kannattaa tehdä vasta viimeisenä eli siinä vaiheessa, kun kaikki sisältö
on valmiina ja muu typografinen asettelu tehty.

\section{Jäsennys}
\label{luku/jäsennys}

Tekstin jäsentäminen tarkoittaa sisällön järjestystä ja sen jakamista
sopiviin kokonaisuuksiin. Esimerkiksi laaja teos voi koostua osista,
osien sisäisistä pääluvuista ja niiden alaluvuista. Eri luvut ilmaistaan
eritasoisilla otsikoilla. Varsinaisten sisältölukujen jälkeen teoksessa
voi olla erilaisia liite- tai luettelosivuja, esimerkiksi lähdeluettelo
ja asiahakemisto. Toisaalta lyhyempi artikkeli sisältää ehkä vain yhden-
tai kahdentasoisia otsikoita -- muuta ei tarvita. Tässä alaluvussa
käsitellään juuri tämäntyyppisiä asioita eli tekstikappaletta suurempien
kokonaisuuksien jäsentämistä.

\subsection{Perustiedot, nimiösivu ja tiivistelmä}
\label{luku/dokumentin-perustiedot}

Latexissa on yksinkertainen toiminto dokumentin perustietojen latomiseen
dokumentin alkuun, mahdollisesti erilliselle nimiösivulle. Perustietoja
ovat teoksen nimi, tekijä ja päiväys. Lähdedokumentin alussa perustiedot
ilmaistaan seuraavilla komennoilla:

\komentoi{title}
\komentoi{author}
\komentoi{date}
\komentoi{and}
\komentoi{today}
\begin{koodilohkosis}
\title{Tieteellistä tiedettä}              % teoksen nimi
\author{Timo Tiedemies \and Tuija Tutkija} % tekijöiden nimet
\date{\today}                              % päiväys
\end{koodilohkosis}

\noindent
Edellisen esimerkin komentojen argumentissa voi käyttää komentoa
\komento{thanks}, joka on tarkoitettu kiitosten osoittamiseen.
Komennolle annetaan argumentiksi lyhyehkö teksti eli kiitosviesti, joka
tulisi näkymään sivulla alaviitteessä.

Komennot eivät itse vielä lado mitään tekstiä dokumenttiin. Ne vain
tallentavat perustiedot muistiin mahdollista myöhempää käyttöä varten.
Komennolla \komento{maketitle} voi latoa kaikki edellä mainitut tiedot.
Ne voivat näkyä ensimmäisen sisältösivun alussa tai erillisellä
nimiösivulla; se riippuu dokumenttiluokan asetuksista ja valitsimista
\koodi{titlepage} ja \koodi{notitlepage}. Se, kumpi asetus on oletuksena
päällä, vaihtelee eri dokumenttiluokissa (luku
\ref{luku/perusdokumenttiluokat-asetukset}). Yksittäiset perustiedot saa
ladottua seuraavilla komennoilla: \komento{thetitle},
\komento{theauthor} ja \komento{thedate}.

Perustietokomentoihin \komento{title}, \komento{author} ja
\komento{date} ei ole tarkoitus sisällyttää kovin monimutkaisia
muotoilukomentoja. Jos haluaa vaikuttaa perustietojen ulkoasuun,
kannattaa käyttää siihen tarkoitettua pakettia \pakettictan{titling}.
Toisaalta mikään ei estä muotoilemasta oman dokumentin perustietoja
ilman Latexin valmiita keinoja.

Hyödyllisiä ympäristöjä dokumentin alussa käytettäväksi voivat olla
\ymparisto{titlepage} ja \ymparisto{abstract}. Ensin mainittu luo
tyhjän sivun nimiösivun toteuttamista varten. Se myös aloittaa
sivunumeroinnin alusta eli nollaa \laskuri{page}\-/ laskurin arvon.
Ympäristö \ymparisto{abstract} toimii \luokka{article}\-/\ ja
\luokka{report}\-/ dokumenttiluokissa ja on tarkoitettu tieteellisen
artikkelin tiivistelmän kirjoittamiseen. Molempien edellä mainittujen
ympäristöjen käyttö on vapaaehtoista. Nimiösivun ja artikkelin
tiivistelmän voi aivan hyvin toteuttaa ilman näitäkin.

Teoksen nimiösivu -- jos sellainen on -- kannattanee lisätä pdf\-/
tiedoston sisäiseen sisällysluetteloon, jotta lukija voi helposti palata
alkuun myös luettelon kautta. Normaaliin ladottavaan sisällysluetteloon
tuskin koskaan merkitään nimiösivua. Pdf:n sisällysluetteloon saa oman
merkinnän esimerkiksi seuraavalla komennolla:

\komentoi{pdfbookmark}
\begin{koodilohkosis}
\pdfbookmark[0]{Nimiö}{sivu/nimiö}
\end{koodilohkosis}

\noindent
Edellä mainittu komento \komento{pdfbookmark} täytyy sijoittaa Latex\-/
lähdedokumentissa nimenomaan nimiösivulle, koska
sisällysluettelomerkintä on samalla linkki kyseiseen kohtaan
dokumentissa. Komento kuuluu \paketti{hyperref}\-/ pakettiin, ja sitä
käsitellään tarkemmin pdf:n asetusten yhteydessä luvussa
\ref{luku/pdf-asetukset}.

\subsection{Otsikointi}
\label{luku/otsikot}

Otsikoiden tehtävänä on tietenkin jakaa sisältö mielekkäisiin osiin ja
ilmaista tekstikokonaisuuksien aihe. Ne auttavat lukijaa ennakoimaan,
mitä on tulossa. Otsikot havainnollistavat myös asioiden
hierarkkisuutta: ylemmäntasoiset otsikot nimeävät yläkäsitteitä tai
laajempia kokonaisuuksia, alemmantasoiset otsikot alakäsitteitä ja
pienempiä kokonaisuuksia.

Jos kirjoittaa Latexilla vähänkään otsikoita, kannattaa ladata mukaan
paketti \paketti{hyperref}. Se huolehtii, että Latexiin merkityt otsikot
tuodaan mukaan pdf\-/ tiedoston sisäiseen sisällysluetteloon ja että
dokumenttiin mahdollisesti ladottavassa sisällysluettelossa otsikot ovat
automaattisesti myös linkkejä, joilla voi hypätä kyseiseen lukuun.
Pdf\-/ tiedoston selailu on näin mukavampaa. \paketti{hyperref}\-/
pakettia ja pdf\-/ tiedoston ominaisuuksia käsitellään tarkemmin luvussa
\ref{luku/pdf-asetukset}.

\leijutlk{
  \providecommand{\rivi}{}
  \renewcommand{\rivi}[4]{\komento{#1} & #2 & #3 & #4 \\}
  \begin{tabular}{lccl}
    \toprule
    \multirow{2}{*}{\ots{Komento}}
    & \ots{Taso}
    & \ots{Taso}
    & \multirow{2}{*}{\ots{Merkitys}} \\
    & (\luokka{article}) & (\luokka{book}, \luokka{report}) \\
    \midrule
    \rivi{part}{0}{-1}{teoksen osa}
    \rivi{chapter}{}{0}{pääluku}
    \rivi{section}{1}{1}{otsikko 1, luku}
    \rivi{subsection}{2}{2}{otsikko 2, alaluku}
    \rivi{subsubsection}{3}{3}{otsikko 3, alaluvun alaluku}
    \rivi{paragraph}{4}{4}{kappaleotsikko 1}
    \rivi{subparagraph}{5}{5}{kappaleotsikko 2}
    \bottomrule
  \end{tabular}
}{
  \caption{Otsikointikomennot ja niitä vastaavat tasot Latexin
    otsikkohierarkiassa. Dokumenttiluokassa \luokka{article} ei ole
    päälukuja (\komento{chapter})}
  \label{tlk/otsikkotasot}
}

Latexissa eri otsikoiden tasoille on omat komentonsa, jotka on koottu
taulukkoon \ref{tlk/otsikkotasot} -- suurimmasta otsikosta pienimpään.
Ylin otsikkotaso on teoksen osa, ja sitä merkitään komennolla
\komento{part}. Tämän komennon käyttö on vapaaehtoista, mutta sillä voi
osoittaa laajan teoksen osia. Osat numeroidaan oletuksena roomalaisilla
järjestysluvuilla: ''\partname~I'', ''\partname~II'', ''\partname~III''
jne. Ne eivät vaikuta alemmantasoisten otsikoiden numerointiin, eli osan
vaihtuminen ei oletusasetuksilla nollaa alemmantasoisten lukujen
laskuria.

Otsikkotaso \komento{chapter} tarkoittaa kirjan päälukua, joka alkaa
oletuksena puhtaalta sivulta. Otsikkoa ennen suoritetaan automaattisesti
sivunvaihtokomento \komento{clearpage} tai \komento{cleardoublepage}
(luku \ref{luku/sivunvaihdot}). \komento{chapter}\-/ tasoiset otsikot
saavat oletuksena numeroinnin ''\chaptername~1'', ''\chaptername~2'',
''\chaptername~3'' jne. Dokumenttiluokassa \luokka{article} ei tätä
otsikkotasoa ole lainkaan.

Seuraavat otsikkotasot \komento{section}, \komento{subsection} ja
\komento{subsubsection} ovat tavallisimpia tekstin väliotsikoita. Nekin
saavat oletuksena numeroinnin, jossa pisteellä erotetaan mahdolliset
alaluvut: 1.1, 1.2, 1.2.1, 1.2.2, 1.3 yms.

Otsikoinnissa on tärkeää säilyttää suora hierarkkinen suhde
otsikkotasojen välillä: alemmantasoista otsikkoa ei saa käyttää ennen
kuin on käytetty yhtä astetta ylempää otsikkotasoa. Täytyy siis olla
ensin \komento{section} ennen kuin voi olla \komento{subsection}\-/
komentoja. Muuten otsikoiden numerointi ei toimi oikein. Poikkeus on
\komento{part}, joka ei vaikuta numerointiin ja jonka käyttö on
muutenkin vapaaehtoista.

Otsikkotasot \komento{paragraph} ja \komento{subparagraph} eivät ole
ihan perinteisiä otsikoita, vaan ne ladotaan tekstikappaleen alkuun.
Nekin kyllä erottuvat leipätekstistä ja voivat toimia otsikon
kaltaisessa tehtävässä. Oletuksena kappaleotsikoita ei numeroida eivätkä
ne tule mukaan sisällysluetteloon. Oletusasetuksia on kuitenkin
mahdollista muuttaa, ja tarvittaessa \komento{paragraph} ja
\komento{subparagraph} voivat toimia normaaleina otsikoina tason
\komento{subsubsection} jälkeen. Kovin syvää otsikkorakennetta ei
yleensä suositeta, koska lukijan voi olla vaikeaa hahmottaa monitasoisia
kokonaisuuksia.

Otsikkokomennoille annetaan tyypillisesti vain yksi argumentti, joka on
kyseisen otsikon teksti. Otsikkoon ei ole tarkoitus kirjoittaa kovin
monimutkaisia fontti- eikä muotoilukomentoja, vaan mieluiten ainoastaan
otsikon teksti. Typografinen muotoilu tehdään muilla keinoilla, joita
esitellään luvussa \ref{luku/otsikot-ulkoasu}. Yksinkertaiset
korostuskomennot ja tavutusvihjeet (\komento{-}) kuitenkin toimivat
yleensä. Otsikkokomennon argumentissa olevien hauraiden komentojen (luku
\ref{luku/komennot-hauraat}) eteen täytyy kirjoittaa komento
\komento{protect}.

\komentoi{section}
\begin{koodilohkosis}
\section{Tämä teksti on otsikko}
\end{koodilohkosis}

\noindent
Otsikko tulee automaattisesti mukaan teoksen sisällysluetteloon, jos
sellainen on olemassa (luku \ref{luku/sisällysluettelo}). Otsikko näkyy
myös pdf\-/ tiedoston sisällysluettelossa, jos \paketti{hyperref}\-/
paketti on käytössä (luku \ref{luku/pdf-asetukset}). Lisäksi otsikon
tekstiä voi käyttää ristiviittauksissa (luku~\ref{luku/ristiviitteet}).

Otsikkokomennoille voi antaa yhden valinnaisen argumentin, jolla
ilmaistaan otsikosta lyhempi tai yksinkertaisempi versio. Tällöin
sisällysluettelossa ja ristiviitteissä näytetään lyhempi otsikko ja
tekstiin ladotaan varsinainen otsikko. Tätä ominaisuutta on syytä
hyödyntää myös silloin, kun varsinainen otsikko sisältää otsikkotekstin
lisäksi muotoilukomentoja. Silloin valinnaiseen argumenttiin
kirjoitetaan vain puhdasta tekstiä sisältävä versio otsikosta.

\komentoi{section}
\begin{koodilohkosis}
\section[Lyhyt otsikko]{Tämä tässä on pitkä otsikko}
\end{koodilohkosis}

\noindent
Kaikille otsikkokomennoille on olemassa myös tähtiversio eli
komentovaihtoehto, jonka nimen lopussa on tähti: \komento{part*},
\komento{chapter*}, \komento{section*}, \komento{subsection*},
\komento{subsubsection*}, \komento{paragraph*}, \komento{subparagraph*}.
Nämä vaihtoehdot latovat otsikon, jolla ei ole numerointia ja jota ei
lisätä sisällysluetteloon.

Pelkän otsikon numeroinnin poistamiseen ei ehkä kannata käyttää näitä
tähtiversioita, koska yleensä otsikko kuitenkin halutaan
sisällysluetteloon. Otsikoiden numeroinnin saa pois keinoilla, jotka
neuvotaan luvussa \ref{luku/otsikot-numerointi}. Jos kuitenkin haluaa
käyttää tähdellistä otsikkokomentoa ja haluaa otsikon myös
sisällysluetteloon, täytyy se tehdä seuraavalla tavalla:

\komentoi{section*}
\komentoi{addcontentsline}
\komentoi{phantomsection}
\begin{koodilohkosis}
\section*{Tässäpä numeroimaton otsikko}
\phantomsection
\addcontentsline{toc}{section}{Tässäpä numeroimaton otsikko}
\end{koodilohkosis}

\noindent
Edellä olevan esimerkin komento \komento{addcontentsline} lisää
merkinnän sisällysluetteloon. Komennon ensimmäinen argumentti on
\koodi{toc} (\englanti{table of contents}), joka on sisällysluetteloiden
tekemisessä tarvittavan väliaikaistiedoston pääte. Esimerkissä lisättävä
sisällysluettelon merkintä on tasoltaan \koodi{section} eli vastaa muita
\komento{section}\-/ komennolla syntyviä sisällysluettelon kohtia.
Komento \komento{phantomsection} kuuluu \paketti{hyperref}\-/
pakettiin, ja se lisää pdf\-/ tiedostoon näkymättömän ankkurin tähän
kohtaan dokumentissa. Sitä tarvitaan, jotta sisällysluettelon merkintä
viittaisi oikealle sivulle ja toimisi myös linkkinä, joka tuo juuri
tähän kohtaan.

\subsection{Otsikoiden numerointi}
\label{luku/otsikot-numerointi}

Oletusasetuksilla Latex numeroi otsikot (\komento{chapter},
\komento{section} jne.) automaattisesti, mutta kirjoittaja voi vaikuttaa
numerointiin monella tavalla, kuten siihen, mihin otsikkotasoon saakka
numerointi yltää ja millä tyylillä otsikoiden numerot ylipäätään
ladotaan.

Laskuri \laskuri{secnumdepth} on tarkoitettu kirjoittajan
asetettavaksi, ja sillä määritetään, mihin otsikkotasoon saakka otsikot
numeroidaan. Tasonumerot ovat taulukossa \ref{tlk/otsikkotasot}
(s.~\pageref{tlk/otsikkotasot}). Seuraava esimerkki määrittelee, että
vain tasoon~1 (\komento{section}) saakka otsikot numeroidaan.
Alemmantasoiset otsikot (\komento{subsection} ym.) eivät saa numeroa.

\komentoi{setcounter}
\laskurii{secnumdepth}
\begin{koodilohkosis}
\setcounter{secnumdepth}{1}
\end{koodilohkosis}

\noindent
Asetusta voi muuttaa kesken dokumentin. Muuttaminen voi olla mielekästä
esimerkiksi laajan teoksen lopussa, jossa on mahdollisesti
lähdeluettelo, asiahakemistoja tai liitteitä, joille ei ehkä haluta
samanlaista otsikkonumerointia kuin varsinaisille sisältöluvuille.

Latexin otsikoiden numerointi noudattaa yleistä tietokirjallisuuden
käytäntöä, eli numeroinnissa käytetään arabialaisia numeroita ja eri
tasoja ilmaisevat numerot erotetaan toisistaan pisteellä: 1.1, 1.2,
1.2.1, 1.2.2 jne. Sisäisesti Latex käyttää laskureita, joilla on sama
nimi kuin vastaavalla otsikointikomennolla: \laskuri{part},
\laskuri{chapter}, \laskuri{section}, \laskuri{subsection},
\laskuri{subsubsection}, \laskuri{paragraph}, \laskuri{subparagraph}.

Laskureiden toimintaan voi perehtyä tarkemmin luvun \ref{luku/laskurit}
avulla, mutta kerrataan tässä yhteydessä yksi asia. Laskurin arvon voi
latoa dokumenttiin käyttämällä komentoa, joka alkaa kirjaimilla
\komentox{the} ja jatkuu laskurin nimellä. Niin otsikkokomennot juuri
tekevät: esimerkiksi \komento{section}\-/ otsikon kohdalla suoritetaan
automaattisesti \komento{thesection}, joka tuottaa otsikon numeron.
Vastaavasti \komento{subsection}\-/ komento latoo numeron käyttämällä
komentoa \komento{thesubsection} jne.

Kirjoittaja voi halutessaan määritellä nämä laskurin latomiskomennot
uudelleen. Jos esimerkiksi teoksen osat (\komento{part}) halutaan
ilmaista arabialaisin numeroin eikä roomalaisin (kuten on oletus),
tehdään se seuraavan esimerkin mukaisesti. Esimerkissä käytettävä
komento \komento{arabic} latoo laskurin arvon arabialaisilla numeroilla.

\komentoi{renewcommand}
\komentoi{thepart}
\laskurii{part}
\komentoi{arabic}
\begin{koodilohkosis}
\renewcommand{\thepart}{\arabic{part}}
\end{koodilohkosis}

\noindent
Joskus ehkä halutaan, että otsikoissa eri tasojen numeroita ei eroteta
toisistaan pisteellä (2.1) vaan yhdysmerkillä tai ajatusviivalla
(2\==1). Tällainen asetus saadaan määrittelemällä alemmantasoisten
laskurien latomiskomennot uudelleen, esimerkiksi seuraavasti:

\komentoi{renewcommand}
\komentoi{thesection}
\komentoi{thesubsection}
\komentoi{thesubsubsection}
\laskurii{subsection}
\laskurii{subsubsection}
\komentoi{arabic}
\begin{koodilohkosis}
\renewcommand{\thesubsection}{\thesection--\arabic{subsection}}
\renewcommand{\thesubsubsection}{%
  \thesubsection--\arabic{subsubsection}}
\end{koodilohkosis}

\noindent
Otsikkonumeroinnin poistamista ei pidä toteuttaa siten, että määrittelee
edellä mainitut \komentox{the}\-/ alkuiset komennot tyhjäksi. Numerointi
poistetaan laskurin \laskuri{secnumdepth} avulla määrittämällä sen arvo
tarpeeksi pieneksi. Sekin on syytä muistaa, että nämä \komentox{the}\-/
alkuiset komennot on tarkoitettu vain laskurien arvon latomiseen.
Komennon määritelmään ei pidä kirjoittaa typografiseen muotoiluun
liittyviä komentoja. Otsikoiden ulkoasuun vaikutetaan muilla keinoilla.

\subsection{Otsikoiden ulkoasu}
\label{luku/otsikot-ulkoasu}

Otsikoiden typografiaan eli esimerkiksi kirjainperheen ja \=/leikkauksen
valintaan ja välistyksiin on kätevintä käyttää \pakettictan{titlesec}\-/
pakettia. Perus Latexissakin on kyllä otsikoiden muotoiluun tarkoitettu
komento \komentox{@startsection}, mutta sen käyttö on hankalampaa eikä
se toimi kaikilla otsikkotasoilla. Tässä oppaassa keskitytään
\paketti{titlesec}\-/ pakettiin.

Otsikoiden perusmuotoilu on helpointa toteuttaa
\komento{titleformat*}\-/ komennon avulla. Sille annetaan argumenteiksi
otsikkokomento ja sitä vastaavat muotoilukomennot. Seuraava esimerkki
vaihtaa otsikoihin groteskin kirjainperheen (\komento{sffamily}) ja
hieman toisistaan poikkeavan kirjainleikkauksen.

\komentoi{titleformat*}
\komentoi{section}
\komentoi{subsection}
\komentoi{subsubsection}
\komentoi{sffamily}
\komentoi{large}
\komentoi{Large}
\komentoi{normalsize}
\komentoi{bfseries}
\komentoi{itshape}
\begin{koodilohkosis}
\titleformat*{\section}      {\sffamily\bfseries\Large}
\titleformat*{\subsection}   {\sffamily\bfseries\itshape\large}
\titleformat*{\subsubsection}{\sffamily\bfseries\normalsize}
\end{koodilohkosis}

\noindent
Monipuolisemmat muotoiluasetukset saa toteutettua komennon tähdettömällä
versiolla \komento{titleformat}. Tämän komentoversion monista
argumenteista ja ominaisuuksista kannattaa lukea \paketti{titlesec}\-/
paketin ohjekirjasta, mutta esimerkistä \ref{esim/titleformat}
selvinnevät perustoiminnot.

\begin{esimerkki*}
  \komentoi{titleformat}
  \komentoi{section}
  \komentoi{sffamily}
  \komentoi{bfseries}
  \komentoi{Large}
  \komentoi{raggedright}
  \komentoi{thesection}

\begin{koodilohko}
\titleformat{\section}     % otsikkotaso: \section
[hang]                     % muoto: riippuva sisennys
{\sffamily\bfseries\Large  % ulkoasukomentoja
  \raggedright}
{\thesection}              % numerointi
{.8em}                     % numeroinnin ja otsikkotekstin väli (mitta)
{}                         % koodia ennen otsikkotekstiä
[]                         % koodia otsikon jälkeen
\end{koodilohko}
  \caption{Otsikoiden ulkoasuun voi vaikuttaa monipuolisesti
    \komento{titleformat}\-/ komennolla, joka kuuluu \paketti{titlesec}\-/
    pakettiin}
  \label{esim/titleformat}
\end{esimerkki*}

Luvun numeroinnin (1.1, 1.2 jne.) tyyliin voi vaikuttaa komennolla
\komento{titlelabel}. Komento suoritetaan yleensä vain kerran, ja se
vaikuttaa kaikentasoisiin otsikoihin. Tätä komentoa ei tarvita, jos
käyttää edellä mainittua ''tähdetöntä'' \komento{titleformat}\-/
komentoa, koska se sisältää jo toiminnot myös lukujen numeroinnin
asetuksiin.

\komentoi{titlelabel}
\komentoi{hspace}
\begin{koodilohkosis}
\titlelabel{\thetitle\hspace{1em}}
\end{koodilohkosis}

\noindent
Edellä olevan komennon argumentissa käytetty komento \komentox{thetitle}
on tarkoitettu vain tämän komennon ja \paketti{titlesec}\-/ paketin
sisäiseen käyttöön. Se latoo kyseisen luvun numeron, olipa otsikkotaso
mikä hyvänsä. Sen perään esimerkissä asetetaan 1\,em:n suuruinen väli
\komento{hspace}\-/ komennolla.

Otsikoiden yhteydessä ladottavaan sisennykseen ja pystysuuntaiseen
väliin vaikutetaan komennolla \komento{titlespacing*}. Sille annetaan
argumenteiksi otsikkokomento ja halutut mitat järjestyksessä vasen
sisennys, yläpuolinen väli ja alapuolinen väli.

\komentoi{titlespacing*}
\komentoi{section}
\begin{koodilohkosis}
\titlespacing*{\section}{0mm}{3ex plus 2bp minus 1bp}{2ex plus 1bp}
\end{koodilohkosis}

\noindent
Pystysuuntaiset mitat on yleensä hyvä asettaa venyviksi (luku
\ref{luku/venyvät-mitat}), jotta Tex saa pelivaraa sivujen latomiseen.
Otsikon yläpuolella pitäisi olla isompi väli kuin alapuolella, koska
otsikko kuuluu kiinteämmin sitä seuraavaan tekstiin. Pystysuuntaisten
välien määrittämisessä voi käyttää *\=/lyhennettä, joka tarkoittaa
oletuksena ex\-/mittayksikköä ja sallii hieman venymistä.

\komentoi{titlespacing*}
\komentoi{section}
\begin{koodilohkosis}
\titlespacing*{\section}{0mm}{*3}{*2}
\end{koodilohkosis}

\noindent
Tähdellinen \komento{titlespacing*}\-/ komento poistaa otsikon jälkeen
tulevasta tekstikappaleesta ensimmäisen rivin sisennyksen (luku
\ref{luku/ensimmäisen-rivin-sisennys}). Tähdetön komentoversio
\komento{titlespacing} sen sijaan säilyttää kappaleen sisennyksen. Sitä
ei liene tarvetta käyttää koskaan, sillä otsikoiden jälkeen tulevaa
tekstikappaletta ei kuulu sisentää.

Teoksen osien (\komento{part}) vaihtuessa sivulle ladotaan oletuksena
teksti kuten ''\partname~I'', ''\partname~II'' jne. Mainittu sana
''\partname'' tulee suomen kieliasetuksista ja komennosta
\komento{partname}. Kaikkien kielten asetukset nimittäin määrittelevät
komennon \komento{partname} siten, että se sisältää ja latoo kyseiseen
kieleen sopivan ilmauksen. Vastaavasti päälukujen (\komento{chapter})
yhteydessä ladotaan suomen kieliasetuksilla ilmaus ''\chaptername~1'',
''\chaptername~2'' jne. Sana ''\chaptername'' tulee komennosta
\komento{chaptername}.

Jos omassa dokumentissa haluaa kutsua teoksen osia tai päälukuja
joksikin muuksi, voi määritellä edellä mainitut komennot uudelleen.
Seuraavassa esimerkissä voisi olla jokin romaani, jossa teoksen osat
ovat \emph{kausia} ja pääluvut ovat \emph{päiviä}.

\komentoi{addto}
\komentoi{captionsfinnish}
\komentoi{renewcommand}
\komentoi{partname}
\komentoi{chaptername}
\begin{koodilohkosis}
\addto{\captionsfinnish}{
  \renewcommand{\partname}{Kausi}
  \renewcommand{\chaptername}{Päivä}
}
\end{koodilohkosis}

\noindent
Edellisessä esimerkissä \komento{renewcommand}\-/ komennot sijoitettiin
\komento{addto}\-/ komennon toiseen argumenttiin. Komento
\komento{addto} täytyy suorittaa lähdedokumentin esittelyosassa, ja
tässä esimerkissä se lisää uusia komentoja suomen kielen asetuksiin
(\komento{captionsfinnish}), niin että ne tulevat voimaan samalla kun
suomen kieliasetuksetkin. \komento{addto} kuuluu
\paketti{polyglossia}\-/\ ja \paketti{babel}\-/ paketteihin, joita
käsitellään tarkemmin kieliasetusten yhteydessä luvussa
\ref{luku/kieliasetukset}.

Otsikoiden yhteyteen saa automaattisen sivunvaihdon määrittämällä
komennon, jonka nimen alussa on otsikkokomennon nimi ja lopussa sana
\koodi{break}. Tavallisin lienee \komento{sectionbreak}. Komennon
määritelmään kirjoitetaan jokin sivuvaihtokomento kuten
\komento{clearpage}. Sivunvaihtokomentoja käsitellään tarkemmin luvussa
\ref{luku/sivunvaihdot}.

\komentoi{newcommand}
\komentoi{sectionbreak}
\komentoi{clearpage}
\begin{koodilohkosis}
\newcommand{\sectionbreak}{\clearpage}
\end{koodilohkosis}

\subsection{Esittely, pääluvut, liitteet ja luettelot}
\label{luku/frontmainbackmatter}

Dokumenttiluokassa \luokka{book} (luku \ref{luku/perusdokumenttiluokat})
on käytettävissä kolme lisäkomentoa dokumentin erilaisten osien
ilmaisemiseen: \komento{frontmatter}, \komento{mainmatter} ja
\komento{backmatter}. Niillä jaetaan laaja teos esittelysivuihin,
varsinaisiin sisältölukuihin ja loppuosaan kuten liitesivuihin ja
erilaisiin luetteloihin. Lisäksi muissakin dokumenttiluokissa on komento
\komento{appendix}, jolla voi aloittaa liitesivuja sisältävän osan
dokumentissa. Näiden komentojen käyttö on vapaaehtoista. Luvussa
\ref{luku/tyypillinen-tietokirja} käsitellään tyypillisen suomalaisen
tietokirjan rakennetta ja teknistä toteuttamista.

Dokumentin alkuun voi kirjoittaa komennon \komento{frontmatter}, joka
asettaa sivunumeroiden tyyliksi roomalaiset numerot (i, ii, iii jne.) ja
poistaa pääluvuilta (\komento{chapter}) numeroinnin. Tässä osassa ovat
teoksen nimiö ja perustiedot, sisällysluettelo, tietokirjan tai
tutkimuksen tiivistelmä ja mahdollisesti teoksen esipuhe tai muu
vastaava esittelyteksti, joka ei ole vielä varsinaista sisältöä.

Jos dokumentissa käytetään \komento{frontmatter}\-/ komentoa, täytynee
varsinaisten sisältölukujen alkuun kirjoittaa komento
\komento{mainmatter}, joka nollaa sivunumerolaskurin, asettaa
numeroinnin tyyliksi arabialaiset numerot (1, 2, 3 jne.) ja kytkee
päälle lukujen numeroinnin. Sen jälkeen ensimmäinen \komento{chapter}\-/
komennolla tehty luku on siis ''\chaptername~1''.

Sisältösivujen lopussa voi olla komento \komento{appendix}. Se
ensinnäkin nollaa \luokka{book}- ja \luokka{report}\-/ luokissa
päälukujen laskurin \laskuri{chapter}. Sen sijaan \luokka{article}\-/
luokassa se nollaa \laskuri{section}\-/ laskurin. Sen lisäksi
\komento{appendix}\-/ komento muuttaa lukujen numerointityyliksi
kirjaimet (A, B, C jne.) ja päälukujen nimeksi ''\appendixname'', joka
tulee komennosta \komento{appendixname}. Näin teoksen lopussa
\komento{appendix}\-/ komennon jälkeen pääluvut tulevat nimetyksi
uudella tavalla: ''\appendixname~A'', ''\appendixname~B'' jne.

Sisältösivujen ja mahdollisten liitteiden jälkeen voi olla hyödyllistä
käyttää komentoa \komento{backmatter}, joka lopettaa päälukujen
numeroinnin kokonaan. Tähän osaan voisi sijoittaa ainakin asiahakemistot
sekä kirjallisuus\-/\ tai lähdeluettelon.

\subsection{Tyypillinen tietokirjan rakenne}
\label{luku/tyypillinen-tietokirja}

Latexin komennot \komento{frontmatter} ja \komento{mainmatter} (luku
\ref{luku/frontmainbackmatter}) eivät ihan vastaa suomalaista
tietokirjojen käytäntöä. Suomessa teoksen alun sivunumeroita ei ole
tapana ilmaista roomalaisilla numeroilla eikä sivunumerointia nollata
sisältölukujen alkaessa. Esimerkkiin \ref{esim/tietokirjojen-rakenne} on
koottu varsin tyypillinen suomalainen tietokirjojen rakenne. Esimerkissä
ei käytetä lainkaan komentoja \komento{frontmatter},
\komento{mainmatter} eikä \komento{backmatter}, vaan lukujen
numerointiin vaikutetaan suoraan asettamalla laskuriin
\laskuri{secnumdepth} sopiva arvo. Tätä laskuria käsitellään otsikoiden
yhteydessä luvussa \ref{luku/otsikot-numerointi}.

\begin{esimerkki*}
  \komentoi{documentclass}
  \luokkai{book}
  \komentoi{pagestyle}
  \komentoi{cleardoublepage}
  \komentoi{setcounter}
  \laskurii{tocdepth}
  \laskurii{secnumdepth}
  \komentoi{tableofcontents}
  \komentoi{chapter}
  \komentoi{appendix}
  \komentoi{printbibliography}
  \komentoi{printindex}
  \ymparistoi{document}

\begin{koodilohko}
\documentclass{book}

\begin{document}

\pagestyle{empty} % sivunumerot pois näkyvistä
% nimiösivu yms.

\cleardoublepage
\setcounter{tocdepth}{2} % sisällysluettelon syvyys
\pagestyle{plain}        % sivunumerot näkyviin
\tableofcontents         % sisällysluettelo

\setcounter{secnumdepth}{-1} % lukujen numerointi pois

\chapter{Esipuhe}
% esipuheen teksti

\setcounter{secnumdepth}{2} % aloitetaan lukujen numerointi

\chapter{Ensimmäinen pääluku}
% ...
\chapter{Toinen pääluku}
% ...

% mahdollisesti liitteet
\appendix
\chapter{Tärkeä liite}
% ...
\chapter{Tosi tärkeä liite}
% ...

% Kirjallisuusluettelot, asiahakemistot yms.
\setcounter{secnumdepth}{-1} % lukujen numerointi pois

\chapter{Kirjallisuutta}
\printbibliography

\chapter{Asiahakemisto}
\printindex

\end{document}
\end{koodilohko}
  \caption{Tyypillinen suomalainen tietokirjojen rakenne sekä sivujen ja
    lukujen numerointikäytännöt}
  \label{esim/tietokirjojen-rakenne}
\end{esimerkki*}

Suomalaisissa kirjoissa sivunumeroinnin laskuri alkaa ensimmäiseltä
sivulta, johon on painettu jotakin, yleensä pelkkä teoksen nimi tai
kirjailijan nimi. Sivu~1 ei ole vielä kirjan kansi, koska kirjapainossa
kannet lisätään ikään kuin tekijöiden työn ympärille. Aikakauslehdissä
kansi on kuitenkin sivu~1.

Sivunumerot tulevat näkyviin hieman vaihtelevasti: joskus sivunumerot
näytetään sisällysluettelon alusta saakka; joskus ne näytetään vasta
sisällysluettelon jälkeen esipuheessa; joskus esipuhe on ennen
sisällysluetteloa ja sivunumerot näkyvät esipuheen alusta saakka.
Esimerkissä \ref{esim/tietokirjojen-rakenne} sivutyyli \koodi{plain}
kytketään päälle sisällysluettelon alkaessa, eli sivunumerotkin alkavat
sen myötä näkyä. Sivutyylejä ja muita ylä- ja alatunnisteisiin liittyviä
asioita käsitellään tarkemmin luvussa \ref{luku/ylä-ala-tunnisteet}.

Mikäli haluaa, että sisällysluettelossa ei vielä näytetä sivunumeroita,
joutuu dokumenttiluokissa \luokka{book} ja \luokka{report} käyttämään
vähän epätavallista keinoa halutun lopputuloksen saavuttamiseksi.
Sellainen neuvotaan sisällysluettelon käsittelyn yhteydessä esimerkissä
\ref{esim/sisällysluettelo-empty}
(s.~\pageref{esim/sisällysluettelo-empty}).

\section{Sisällysluettelo}
\label{luku/sisällysluettelo}

Dokumentin sisällysluettelo syntyy helposti yhdellä komennolla:
\komento{tableofcontents}. Otsikkokomennoilla (luku \ref{luku/otsikot})
tehdyt otsikot tulevat automaattisesti sisällysluetteloon. Jos käytössä
on paketti \paketti{hyperref} (luku \ref{luku/pdf-asetukset}),
sisällysluettelon otsikot ovat myös linkkejä, jotka vievät kyseiseen
kohtaan dokumentissa. Oletuksena syntyvä sisällysluettelo on tyyliltään
neutraali, ja se sopinee sellaisenaan useimpiin töihin. Ulkoasua voi
kyllä muokatakin varsin helposti.

\komentoi{tableofcontents}
\begin{koodilohkosis}
% Ladotaan sisällysluettelo tähän kohtaan.
\tableofcontents
\end{koodilohkosis}

\noindent
Laskurin \laskuri{tocdepth} avulla kirjoittaja voi määritellä, mihin
tasoon saakka sisällysluettelossa näytetään otsikoita. Esimerkiksi
jonkin kuvitteellisen romaanin (dokumenttiluokka \luokka{book})
sisällysluettelossa voisi näyttää ainoastaan osat (\komento{part}) ja
pääluvut (\komento{chapter}) asettamalla laskurin arvoksi nolla. Arvo
nolla viittaa nimenomaan \komento{chapter}\-/ tasoisiin otsikoihin.
Otsikoiden tasonumerot voi tarkistaa taulukosta \ref{tlk/otsikkotasot}
(s.~\pageref{tlk/otsikkotasot}).

\komentoi{setcounter}
\laskurii{tocdepth}
\begin{koodilohkosis}
\setcounter{tocdepth}{0}
\end{koodilohkosis}

\noindent
Sisällysluettelolle ladotaan automaattisesti otsikko, joka tulee
komennosta \komento{contentsname}. Komento on määritelty
kieliasetuksissa, ja se latoo suomen kielessä sanan ''\contentsname''.
Jos otsikkoon ei ole tyytyväinen, voi lähdedokumentin esittelyosassa
\komento{addto}\-/ komennon avulla muuttaa kieliasetuksia. Seuraavassa
esimerkissä muutetaan sisällysluettelon otsikko suomen kielen osalta:

\komentoi{addto}
\komentoi{captionsfinnish}
\komentoi{renewcommand}
\komentoi{contentsname}
\begin{koodilohkosis}
\addto{\captionsfinnish}{
  \renewcommand{\contentsname}{Sisällysluettelo}
}
\end{koodilohkosis}

\noindent
Dokumenttiluokissa \luokka{book} ja \luokka{report} (luku
\ref{luku/perusdokumenttiluokat}) on sellainen ominaisuus tai
kummallisuus, että sisällysluettelon alkusivulla suoritetaan
automaattisesti komento \komento{thispagestyle}\komentoarg{plain}. Tämän
vuoksi sisällysluettelon ensimmäinen sivu on aina \koodi{plain}\-/
tyylinen eli alatunnisteessa näkyy sivunumero, vaikka ennen
sisällysluetteloa olisikin asettanut sivutyyliksi jonkin muun.
Monisivuisen sisällysluettelon seuraavilla sivuilla sen sijaan
noudatetaan aiemmin \komento{pagestyle}\-/ komennolla asetettua
sivutyyliä.

\begin{esimerkki*}
  \komentoi{pagestyle}
  \komentoi{renewcommand}
  \komentoi{thispagestyle}
  \komentoi{tableofcontents}

\begin{koodilohko}
\pagestyle{empty}
{
  \renewcommand{\thispagestyle}[1]{} % tyhjä määritelmä
  \tableofcontents
}
\end{koodilohko}
  \caption{Sisällysluettelon sivutyylin muuttaminen kokonaan
    \koodi{empty}\-/ tyyliseksi. Komento \komento{thispagestyle} pitää
    määritellä väliaikaisesti uudestaan dokumenttiluokissa \luokka{report}
    ja \luokka{book}}
  \label{esim/sisällysluettelo-empty}
\end{esimerkki*}

Jotta edellä mainitusta kummallisuudesta pääsisi eroon, täytyy
määritellä väliaikaisesti \komento{thispagestyle}\-/ komento
toisenlaiseksi. Väliaikainen komennon määrittely saadaan aikaan
aaltosulkeilla (luku \ref{luku/aaltosulkeet}), joiden sisään komennon
uudelleen määrittely ja sisällysluettelon latomiskomento sijoitetaan.
Esimerkistä \ref{esim/sisällysluettelo-empty} selviää, miten se tehdään.
Tässä esimerkissä komennon määritelmä jätettiin tyhjäksi, koska sen ei
haluta tekevän mitään. Näin ollen aiemmin annettu \komento{pagestyle}\-/
komento vaikuttaa koko sisällysluetteloon.

Normaalit otsikot tulevat mukaan sisällysluetteloon itsestään, mutta
joskus voi joutua lisäämään luetteloon kohtia myös keinotekoisesti. Se
tehdään komennolla \komento{addcontentsline}, jonka käytöstä on
esimerkki luvussa \ref{luku/otsikot}. Pelkkään pdf\-/ tiedoston
sisäiseen sisällysluetteloon voi lisätä kohtia komennolla
\komento{pdfbookmark}, joka kuuluu \paketti{hyperref}\-/ pakettiin.
Tätä komentoa käsitellään luvussa \ref{luku/pdf-asetukset}.

\begin{esimerkki*}
  \komentoi{addvspace}
  \komentoi{bfseries}
  \komentoi{contentslabel}
  \komentoi{contentspage}
  \komentoi{hspace}
  \komentoi{large}
  \komentoi{normalsize}
  \komentoi{rmfamily}
  \komentoi{small}
  \komentoi{titlecontents}
  \komentoi{titlerule*}
  \komentoi{titlerule}

\begin{koodilohko}
\titlecontents{chapter}                     % otsikon taso: chapter
[8mm]                                       % vasen sisennys (mitta)
{\addvspace{1.5ex}\rmfamily\bfseries\large} % yläpuolinen koodi
{\contentslabel{8mm}}                       % numeroitu kohta
{\hspace{-8mm}}                             % numeroimaton kohta
{\small\titlerule[0bp]\contentspage}  % pisteviiva (pois) ja sivunumero
[\addvspace{.5ex}]                    % alapuolinen koodi

\titlecontents{section}                     % otsikon taso: section
[8mm]                                       % vasen sisennys (mitta)
{\addvspace{.5ex}\rmfamily\normalsize}      % yläpuolinen koodi
{\contentslabel{8mm}}                       % numeroitu kohta
{}                                          % numeroimaton kohta
{~\small\titlerule*[3mm]{.}\contentspage}   % pisteviiva ja sivunumero
[\addvspace{.2ex}]                          % alapuolinen koodi

\titlecontents{subsection}                  % otsikon taso: subsection
[18mm]                                      % vasen sisennys (mitta)
{\rmfamily\small}                           % yläpuolinen koodi
{\contentslabel{10mm}}                      % numeroitu kohta
{}                                          % numeroimaton kohta
{~\small\titlerule*[3mm]{.}\contentspage}   % pisteviiva ja sivunumero
[]                                          % alapuolinen koodi
\end{koodilohko}
  \caption{Sisällysluettelon ulkoasua muokataan \komento{titlecontents}\-/
    komennolla, joka on peräisin \paketti{titletoc}\-/ paketista}
  \label{esim/titlecontents}
\end{esimerkki*}

Sisällysluettelon ulkoasun muokkaamista varten on paketti
\pakettictan{titletoc}, jonka tärkeimmän komennon
(\komento{titlecontents}) käyttöä esitellään esimerkissä
\ref{esim/titlecontents}. Komennolla voi muokata paitsi
sisällysluettelon kohtia mutta myös leijuvien osien (luku
\ref{luku/leijuosat}) kuten kuvien ja taulukoiden luetteloita.

Esimerkistä selviää, miten \komento{titlecontents}\-/ komennolla
määritellään luettelokohdan vasemman sisennyksen mitta, yläpuolinen ja
alapuolinen väli (\komento{addvspace}, luku
\ref{luku/pystysuuntaiset-välit}), mahdolliset kirjainperheeseen ja
\=/leikkaukseen vaikuttavat komennot kuten \komento{rmfamily},
\komento{bfseries} ja \komento{large} (luku \ref{luku/fontit-korkea}).

\komento{titlecontents}\-/ komennon argumenteissa käytetään jonkin
verran muita komentoja, jotka on tarkoitettu nimenomaan tämän komennon
argumentteihin. Komento \komento{contentslabel} latoo kyseisen otsikon
numeron (1.2, 1.3 tms.) tietynlevyiseen tilaan ja sisennystason
vasemmalle puolelle (riippuva sisennys). Komento \komento{contentspage}
latoo sivunumeron. Komennolla \komento{titlerule} voi tehdä katsetta
ohjaavan viivan tai pisteviivan otsikkotekstin ja sivunumeron väliin.

Paketin \paketti{titletoc} ohjekirjassa on paljon muitakin
toimintoja ja esimerkkejä sisällysluetteloiden ehostamiseen. Ohjeisiin
kannattaa tutustua, mikäli kaipaa erikoisempia sisällysluettelomalleja.

\section{Luetelmat}
\label{luku/luetelmat}

Latexissa on helppokäyttöiset ympäristöt kolmelle erilaiselle luetelman
perustyypille: numeroimaton (\ymparisto{itemize}), numeroitu
(\ymparisto{enumerate}) ja määritelmäluetelma (\ymparisto{description}).
Ne riittänevät perustarpeisiin ja ovat hieman myös muokattavissa
typografisesti. Näitä luetelmien perusympäristöjä käsitellään luvussa
\ref{luku/luetelma-perus}.

Perusluetelmien lisäksi on olemassa yleinen luetelmien rakenteluun
tarkoitettu ympäristö \ymparisto{list}, jolla voi toteuttaa suunnilleen
mitä tahansa luetelmiin tai tekstikappaleisiin liittyviä rakenteita.
Teknisesti \ymparisto{list}\-/ ympäristö sisältää kaikki tarvittavat
luetelmiin liittyvät ominaisuudet, joten kirjoittaja voi aivan hyvin
käyttää pelkästään sitä. Ympäristöä käsitellään luvussa
\ref{luku/list-ympäristö}.

\subsection{Perusympäristöt}
\label{luku/luetelma-perus}

Ympäristö \ymparisto{itemize} tekee numeroimattoman luetelman, jonka
jokainen kohta merkitään samanlaisella merkillä, esimerkiksi
luetelmaympyrällä~(\textbullet). \ymparisto{enumerate} puolestaan on
numeroitu luetelma, eli luetelman kohdat saavat järjestysnumeron tai
muun järjestystä ilmaisevan merkinnän kuten kirjaimen tai roomalaisen
numeron. Ympäristö \ymparisto{description} luo määritelmäluetelman,
jossa luetellaan käsitteitä ja niiden määritelmiä. Tämä on käytännössä
sama kuin tekstikappaleen riippuva sisennys (luku
\ref{luku/riippuva-sisennys}), jossa kappale alkaa määriteltävällä
ilmauksella.

Lähdetiedostossa kaikkien luetelmatyyppien rakenne on samanlainen:
luetelmaympäristön sisällä käytetään \komento{item}\-/ komentoa, joka
aloittaa uuden luetelmakohdan. Komennolle ei tarvitse antaa argumenttia,
vaan komennon jälkeinen teksti muodostaa luetelman kohdan. Seuraavassa
on esimerkki \ymparisto{itemize}\-/ ympäristöstä, mutta
\ymparisto{enumerate}\-/ ympäristöäkin käytetään samalla tavalla:

\ymparistoi{itemize}
\komentoi{item}
\begin{koodilohkosis}
\begin{itemize}
\item Luetelman ensimmäinen kohta.
\item Toinen kohta heti perään.
\item Kolmaskin kohta on tarpeen.
\end{itemize}
\end{koodilohkosis}

\noindent
\komento{item}\-/ komennolle voi antaa hakasulkeissa valinnaisen
argumentin. Silloin ei ladota tavanomaista luetelmamerkkiä vaan
argumentin sisältämä teksti.

\komentoi{item}
\begin{koodilohkosis}
\item[--] Tässä kohdassa onkin ajatusviiva.
\end{koodilohkosis}

\begin{esimerkki*}
  \ymparistoi{description}
  \komentoi{item}

\begin{koodilohko}
\begin{description}
\item[Tex] Tekstidokumenttien ladontaan erikoistunut ohjelmointikieli.
\item[Latex] Dokumenttien kirjoittajille tarkoitettu merkintäkieli.
\item[Lualatex] Latex-muotoisten lähdedokumenttien kääntäjäohjelma.
  Sisältää Lua-nimisen ohjelmointikielen.
\end{description}
\end{koodilohko}

  \begin{tulos}
    \begin{description}
    \item[Tex] Tekstidokumenttien ladontaan erikoistunut ohjelmointikieli.
    \item[Latex] Dokumenttien kirjoittajille tarkoitettu merkintäkieli.
    \item[Lualatex] Latex-muotoisten lähdedokumenttien kääntäjäohjelma.
      Sisältää Lua-nimi\-sen ohjelmointikielen.
    \end{description}
  \end{tulos}
  \caption{Käsitteiden määritelmiä ja sen kaltaisia luetelmia voi
    toteuttaa \ymparisto{description}\-/ ympäristön avulla. Käsitteet
    kirjoitetaan \komento{item}\-/ komennon valinnaiseen argumenttiin}
  \label{esim/description}
\end{esimerkki*}

\noindent
Komennon valinnaista argumenttia hyödynnetään varsinkin
\ymparisto{description}\-/ ympäristössä, jossa luetellaan sanoja tai
käsitteitä ja niiden määritelmiä. Esimerkistä \ref{esim/description}
selviää, kuinka määriteltävät käsitteet ilmaistaan \komento{item}\-/
komennon valinnaisessa argumentissa.

Edellä mainittuja luetelmaympäristöjä voi kirjoittaa sisäkkäin: yksi
luetelman kohta (\komento{item}) voi siis aloittaa uuden
luetelmaympäristön. Latex osaa käsitellä neljä sisäkkäistä luetelmaa.

Latex sisentää sisäkkäiset luetelmat automaattisesti loogisella tavalla
ja vaihtaa eri tasoilla luetelmakohtien merkkiä tai numerointitapaa.
Esimerkiksi uloimmainen \ymparisto{itemize}\-/ ympäristö käyttää
luetelmaympyrää~(\textbullet), mutta sen sisällä oleva käyttää
ajatusviivaa~(\==). Uloimmainen \ymparisto{enumerate}\-/ ympäristö
käyttää arabialaisia numeroita (1, 2, 3 jne.) mutta sen sisällä oleva
kirjaimia (a, b, c jne.).

\begin{esimerkki*}
  \komentoi{renewcommand}
  \komentoi{labelitemi}
  \komentoi{labelitemii}
  \komentoi{labelitemiii}
  \komentoi{labelitemiv}
  \komentoi{textbullet}
  \komentoi{normalfont}
  \komentoi{bfseries}
  \komentoi{textendash}
  \komentoi{textasteriskcentered}
  \komentoi{textperiodcentered}
  \komentoi{labelenumi}
  \komentoi{labelenumii}
  \komentoi{labelenumiii}
  \komentoi{labelenumiv}
  \komentoi{arabic}
  \komentoi{alph}
  \komentoi{Alph}
  \komentoi{roman}
  \laskurii{enumi}
  \laskurii{enumii}
  \laskurii{enumiii}
  \laskurii{enumiv}

\begin{koodilohko}
% itemize-ympäristön luetelmamerkki
\renewcommand{\labelitemi}  {\textbullet} % uloimmainen luetelma
\renewcommand{\labelitemii} {\normalfont\bfseries\textendash}
\renewcommand{\labelitemiii}{\textasteriskcentered}
\renewcommand{\labelitemiv} {\textperiodcentered}

% enumerate-ympäristön numerointitapa
\renewcommand{\labelenumi}  {\arabic{enumi}.} % uloimmainen luetelma
\renewcommand{\labelenumii} {(\alph{enumii})}
\renewcommand{\labelenumiii}{\roman{enumiii}.}
\renewcommand{\labelenumiv} {\Alph{enumiv}.}
\end{koodilohko}
  \caption{Luetelmamerkkien ja numerointitapojen muuttaminen
    \ymparisto{itemize}\-/\ ja \ymparisto{enumerate}\-/ ympäristöissä.
    Esimerkissä näkyvät oletusarvot}
  \label{esim/labelitem-labelenum}
\end{esimerkki*}

\leijutlk{
  \providecommand{\rivi}{}
  \renewcommand{\rivi}[3]{#1 & \uctunnus{#2} & #3 \\}
  \begin{tabular}{cll}
    \toprule
    \multicolumn{2}{l}{\ots{Merkki ja Unicode-nimi}} & \ots{Komento} \\
    \midrule
    \rivi{\textbullet}{u+2022 bullet}{\komento{textbullet}}
    \rivi{◦}{u+25e6 white bullet}{}
    \rivi{$\circ$}{u+2218 ring operator}{\koodi{\$\mkomento{circ}\$}}
    % \rivi{⁃}{u+2043 hyphen bullet}{}
    % \rivi{▪}{u+25aa black small square}{}
    \rivi{\rule[.3ex]{.5ex}{.5ex}}{}
    {\komento{rule}\komentoargv{.3ex}\komentoarg{.5ex}\komentoarg{.5ex}}
    \rivi{‣}{u+2023 triangular bullet}{}
    \rivi{--}{u+2013 en dash}{\komento{textendash}, \koodi{--}}
    \rivi{$\star$}{u+22c6 star operator}{\koodi{\$\mkomento{star}\$}}
    \rivi{\textasteriskcentered}
    {u+2217 asterisk operator} {\komento{textasteriskcentered}}
    \rivi{\textperiodcentered}
    {u+00b7 middle dot}{\komento{textperiodcentered}}
    \bottomrule
  \end{tabular}
}{
  \caption{Erilaisia luetelmamerkkejä. Komento-sarakkeen
    \koodi{\$}\=/merkit ovat matematiikkatilan aloitus- ja
    lopetuskomentoja}
  \label{tlk/luetelmamerkkejä}
}

Luetelmamerkit ja luetelmien numerointitavat ovat kirjoittajan
muutettavissa. Ympäristössä \ymparisto{itemize} luetelmamerkit tulevat
komennosta, joka alkaa sanoilla \komentox{label\-item} ja jatkuu
luetelman tasoa ilmaisevalla roomalaisella numerolla
\koodi{i}\==\koodi{iv}. Komennot voi määritellä uudelleen haluamallaan
tavalla, kuten esimerkki \ref{esim/labelitem-labelenum} osoittaa.
Taulukkoon \ref{tlk/luetelmamerkkejä} on koottu erilaisia
luetelmamerkkejä sekä Latex\-/ komentoja, joilla niitä voi tuottaa.

\ymparisto{enumerate}\-/ ympäristössä numeroidut kohdat tulevat
komennosta, joka alkaa sanoilla \komentox{label\-enum} ja jatkuu
luetelman tasoa ilmaisevalla roomalaisella numerolla
\koodi{i}\==\koodi{iv}. Numeroidussa luetelmassa käytetään laskureita
\laskuri{enumi}, \laskuri{enumii}, \laskuri{enumiii} ja
\laskuri{enumiv}, joten luetelmakohtien täytyy latoa näiden laskurien
arvo jossakin muodossa. Esimerkki \ref{esim/labelitem-labelenum}
selventää asiaa.

Numeroiduista luetelmakohdista huolehtivat laskurit nollataan
automaattisesti aina ympäristön alussa. Joskus voi kuitenkin olla
tarpeen aloittaa luetelmakohdat jostain muusta kuin luvusta~1 tai
kirjaimesta~\textit{a}. Silloin kirjoittajan täytyy asettaa laskuri
haluamaansa arvoon \ymparisto{enumerate}\-/ ympäristön sisällä. Seuraava
esimerkki asettaa uloimman luetelman \laskuri{enumi}\-/ laskurin
arvoon~3, joten ensimmäiseksi luetelmakohdaksi tulee~4.

\ymparistoi{enumerate}
\komentoi{setcounter}
\komentoi{item}
\laskurii{enumi}
\begin{koodilohkosis}
\begin{enumerate}
  \setcounter{enumi}{3}
\item Tämä on luetelmakohta numero 4.
\end{enumerate}
\end{koodilohkosis}

\noindent
Kirjoittaja voi vaikuttaa myös luetelmakohtien sisennykseen ja
pystysuuntaisiin väleihin. Sisennykseen vaikuttavat mitat alkavat
sanoilla \mittax{leftmargin} ja jatkuvat roomalaisella numerolla
\koodi{i}\==\koodi{iv}, joka ilmaisee luetelman tasoa. Uloimmaisen
luetelman sisennys ilmaistaan suhteessa sivun tekstialueen reunaan;
sisempien luetelmien sisennys ilmaistaan suhteessa edellisen tason
sisennykseen. Oletuksena Latexin sisennykset ovat melko suuria, joten
niitä on usein tarpeen pienentää, esimerkiksi seuraavan esimerkin
mukaisesti:

\komentoi{setlength}
\mittai{leftmargini}
\mittai{leftmarginii}
\mittai{leftmarginiii}
\mittai{leftmarginiv}
\begin{koodilohkosis}
\setlength{\leftmargini}  {1.5em} % uloimmainen luetelma
\setlength{\leftmarginii} {1.5em}
\setlength{\leftmarginiii}{1.5em}
\setlength{\leftmarginiv} {1.5em}
\end{koodilohkosis}

\noindent
Edellä olevat sisennysasetukset toimivat ympäristöissä
\ymparisto{itemize}, \ymparisto{enumerate} ja \ymparisto{description}.
Mainituissa ympäristöissä voi säätää pystysuuntaisia välejä mittojen
\mitta{parsep}, \mitta{itemsep} ja \mitta{parskip} avulla. Mittojen
sijainti luetelmassa on nähtävissä kuvassa \ref{kuva/list-mitat}
(s.~\pageref{kuva/list-mitat}). Mittojen asettamiskomennot on syytä
sijoittaa luetelmaympäristön sisään heti \komento{begin}\-/ komennon
jälkeen.

\komentoi{setlength}
\mittai{parsep}
\mittai{itemsep}
\mittai{parskip}
\begin{koodilohkosis}
\setlength{\parsep} {0bp}   % kappaleiden väli
\setlength{\itemsep}{3bp}   % luetelmakohtien väli
\setlength{\parskip}{1.2ex} % väli luetelman alussa ja lopussa
\end{koodilohkosis}

\subsection{Omat luetelmat (list)}
\label{luku/list-ympäristö}

Ympäristö \ymparisto{list} on yleistyökalu, joka soveltuu moniin
asioihin: se on tarkoitettu erilaisten luetelmien rakentamiseen, mutta
sen avulla muotoilla mitä tahansa tekstikappaleita. Esimerkiksi luvussa
\ref{luku/luetelma-perus} käsitellyt luetelmat voi toteuttaa myös
\ymparisto{list}\-/ ympäristön avulla -- samoin luvuissa
\ref{luku/riippuva-sisennys} ja \ref{luku/lohkolainaukset} käsitellyt
riippuvat sisennykset ja lohkolainaukset.

Tyypillisesti \ymparisto{list}\-/ ympäristöä ei käytetä dokumentin
kirjoittamisessa sellaisenaan, vaan kirjoittaja määrittelee oman
helppokäyttöisen luetelma\-/\ tai muun ympäristön, ja
\ymparisto{list}\-/ ympäristöä hyödynnetään ympäristön
määrittelemisessä. Omien ympäristöjen tekemistä käsitellään yleisesti
luvussa \ref{luku/ympäristöt}, mutta tämän alaluvun loppupuolella on
esimerkki, joka liittyy nimenomaan \ymparisto{list}\-/ ympäristöön.

Seuraavassa on esimerkki \ymparisto{list}\-/ ympäristön rakenteesta. Se
muistuttaa luvussa \ref{luku/luetelma-perus} käsiteltyjä perusluetelmia,
mutta lisäksi sille täytyy antaa kaksi argumenttia. Niiden avulla
vaikutetaan luetelman asetuksiin.

\begin{koodilohkosis}
\begin{list}{merkki}{asetukset}
\item ...
\end{list}
\end{koodilohkosis}

\noindent
Edellisen esimerkin argumentti \koodi{merk\-ki} sisältää tekstiä tai
koodia, joka muodostaa luetelmakohdan merkin. Numeroimattomassa
luetelmassa siihen voi laittaa esimerkiksi luetelmaympyrän eli komennon
\komento{textbullet} tai ajatusviivan (\koodi{--}). Muita esimerkkejä on
taulukossa \ref{tlk/luetelmamerkkejä}. Numeroidussa luetelmassa siihen
kirjoitetaan komento, joka latoo jonkin laskurin arvon.

\ymparisto{list}\-/ ympäristön toinen argumentti \koodi{ase\-tuk\-set}
sisältää mitä hyvänsä koodia, jolla vaikutetaan ympäristön asetuksiin.
Tyypillisesti siinä määritellään käytettävä luetelmalaskuri sekä
asetetaan luetelman ulkoasuun vaikuttavat mitat sopivaksi. Mahdollisesti
siinä myös määritellään luetelmamerkkien latomiseen vaikuttava komento
\komento{makelabel}.

\begin{esimerkki*}
  \ymparistoi{list}
  \komentoi{arabic}
  \komentoi{setlength}
  \komentoi{usecounter}
  \mittai{leftmargin}
  \mittai{itemsep}
  \mittai{parsep}
  \laskurii{enumi}

\begin{koodilohko}
\begin{list}{\arabic{enumi})}{
    \usecounter{enumi}             % käytetään enumi-laskuria
    \setlength{\leftmargin}{1.5em} % vasen sisennys
    \setlength{\itemsep}{.2ex}     % luetelmakohtien väli
    \setlength{\parsep}{0ex}       % luetelmakappaleiden väli
  }
\item Luetelman ensimmäinen kohta.
\item Tässä on vielä toinen
\item ja kolmas kohta.
\end{list}
\end{koodilohko}
  \begin{tulos}
    \begin{list}{\arabic{enumi})}{
        \usecounter{enumi}             % käytetään enumi-laskuria
        \setlength{\leftmargin}{1.5em} % vasen sisennys
        \setlength{\itemsep}{.2ex}     % luetelmakohtien väli
        \setlength{\parsep}{0ex}       % luetelmakappaleiden väli
      }
    \item Luetelman ensimmäinen kohta.
    \item Tässä on vielä toinen
    \item ja kolmas kohta.
    \end{list}
  \end{tulos}
  \caption{\ymparisto{list}\-/ ympäristön argumenttien avulla vaikutetaan
    luetelmakohtien merkintätapaan ja mittoihin}
  \label{esim/list-perus}
\end{esimerkki*}

Esimerkissä \ref{esim/list-perus} on yksinkertainen numeroitu luetelma,
joka hyödyntää olemassa olevaa \laskuri{enumi}\-/ laskuria (luku
\ref{luku/luetelma-perus}). Laskurin arvo ladotaan arabialaisilla
numeroilla (\komento{arabic}). Laskurin \laskuri{enumi} sijasta voi
käyttää mitä tahansa omaakin laskuria, kunhan sen on luonut etukäteen
komennolla \komento{newcounter} (luku \ref{luku/laskurit}). Esimerkissä
asetetaan myös eräitä luetelman sisäisiä mittoja, joista on lisätietoa
myöhemmin tässä alaluvussa sekä kuvassa \ref{kuva/list-mitat}.

\begin{esimerkki*}
  \ymparistoi{list}
  \komentoi{renewcommand}
  \komentoi{makelabel}
  \komentoi{textsc}
  \komentoi{setlength}
  \komentoi{item}
  \mittai{leftmargin}
  \mittai{labelwidth}
  \mittai{labelsep}
  \mittai{itemindent}
  \mittai{itemsep}

\begin{koodilohko}
\begin{list}{}{
    \renewcommand{\makelabel}[1]{\textsc{#1:}}
    \setlength{\leftmargin}{1.5em}
    \setlength{\labelwidth}{1.5em}
    \setlength{\itemindent}{1em}
    \setlength{\labelsep}{1em}
    \setlength{\itemsep}{.2ex}
  }
\item[Tex] Tekstidokumenttien ladontaan erikoistunut ohjelmointikieli.
\item[Latex] Dokumenttien kirjoittajille tarkoitettu merkintäkieli.
\item[Lualatex] Latex-muotoisten lähdedokumenttien kääntäjäohjelma.
  Sisältää Lua-nimi\-sen ohjelmointikielen.
\end{list}
\end{koodilohko}
  \begin{tulos}
    \begin{list}{}{
        \renewcommand{\makelabel}[1]{\textsc{#1:}}
        \setlength{\leftmargin}{1.5em}
        \setlength{\labelwidth}{1.5em}
        \setlength{\itemindent}{1em}
        \setlength{\labelsep}{1em}
        \setlength{\itemsep}{.2ex}
      }
    \item[Tex] Tekstidokumenttien ladontaan erikoistunut ohjelmointikieli.
    \item[Latex] Dokumenttien kirjoittajille tarkoitettu merkintäkieli.
    \item[Lualatex] Latex-muotoisten lähdedokumenttien kääntäjäohjelma.
      Sisältää Lua-nimi\-sen ohjelmointikielen.
    \end{list}
  \end{tulos}
  \caption{Määritelmäluetelmien tekeminen \ymparisto{list}\-/ ympäristön
    avulla. Sisäisesti komento \komento{makelabel} huolehtii
    luetelmamerkkien eli tässä käsitteiden nimien latomisesta}
  \label{esim/list-makelabel}
\end{esimerkki*}

Luetelmamerkit \ymparisto{list}\-/ ympäristö latoo siten, että se
suorittaa aina komennon \komento{makelabel} ja antaa sille argumentiksi
kulloisenkin luetelmamerkin eli \ymparisto{list}\-/ ympäristön
ensimmäisenä argumenttina annetun tekstin tai \komento{item}\-/ komennon
valinnaisena argumenttina olevan tekstin. Luetelmamerkki ladotaan sille
varatun alueen oikeaan reunaan. Oletuksena \komento{makelabel}\-/
komento toimii ikään kuin se olisi määritelty seuraavalla tavalla:

\komentoi{renewcommand}
\komentoi{makelabel}
\komentoi{hfill}
\begin{koodilohkosis}
\renewcommand{\makelabel}[1]{\hfill #1}
\end{koodilohkosis}

\noindent
Kirjoittaja voi määritellä \komento{makelabel}\-/ komennon uudelleen
sellaiseksi kuin haluaa. Se on tarpeen esimerkiksi silloin, kun täytyy
latoa kaikki luetelmamerkit eri fontilla kuin muu teksti. Varsinkin
määritelmäluetelmissa (luku \ref{luku/luetelma-perus}), joissa
luetellaan käsitteitä ja niiden määritelmiä, on usein hyödyllistä, että
käsitteet erottuvat muusta tekstistä selvästi. Esimerkki
\ref{esim/list-makelabel} selventää tätä asiaa.

\leijukuva{
  \begin{tikzpicture}
    [x=.01\textwidth, y=.01\textwidth, line width=1bp, rounded corners=2bp]

    \newcommand{\kpl}[1]{\draw (15,#1) -- ++(70,0) -- ++(0,15)
      -- ++(-60,0) -- ++(0,-5) -- ++(-10,0) -- cycle}
    \newcommand{\nuoli}[2]{\draw [color=mittanuoli, <->] (#1) -- (#2)}
    \newcommand{\kyltti}[2]{\draw (#1) node [anchor=west] {#2}}

    \newcommand{\katko}[2]{\draw [color=apuviiva, densely dotted, line
      width=.7bp] (#1) -- (#2)}

    \katko{0,10}{0,110};
    \katko{100,10}{100,110};

    \kpl{80};
    \kpl{55};
    \kpl{25};

    \draw (0,10) rectangle ++(100,-7);
    \node at (50,6) {alapuolinen teksti};

    \draw (0,110) rectangle ++(100,7);
    \node at (50,113) {yläpuolinen teksti};

    \nuoli{50,110}{50,95};
    \kyltti{51,102.5}{\mitta{topsep} + \mitta{parskip} (+ \mitta{partopsep})};

    % Ylin kappale
    \node at (50,87) {luetelmakohta 1};
    \draw (5,91) rectangle ++(15,4);
    \katko{5,98}{5,95};
    \katko{20,98}{20,95};
    \katko{25,98}{25,95};
    \nuoli{5,98}{20,98}; \kyltti{2,102}{\mitta{labelwidth}};
    \nuoli{20,98}{25,98}; \kyltti{20,102}{\mitta{labelsep}};
    \katko{25,90}{25,87};
    \nuoli{15,87}{25,87}; \kyltti{15,84}{\mitta{itemindent}};

    \nuoli{50,70}{50,80};
    \kyltti{51,75}{\mitta{parsep}};

    % Keskimmäinen kappale
    \node at (50,62) {toinen tekstikappale};
    \nuoli{0,57}{15,57};
    \kyltti{0,52}{\mitta{leftmargin}};
    \nuoli{85,57}{100,57};
    \kyltti{82,52}{\mitta{rightmargin}};
    \katko{15,70}{15,65};
    \nuoli{15,70}{25,70};
    \kyltti{13,74}{\mitta{listparindent}};

    \nuoli{50,40}{50,55};
    \kyltti{51,47}{\mitta{itemsep} + \mitta{parsep}};

    % Alin kappale
    \node at (50,32) {luetelmakohta 2};
    \draw (5,36) rectangle ++(15,4);

    \nuoli{50,25}{50,10};
    \kyltti{51,17.5}{\mitta{topsep} + \mitta{parskip} (+ \mitta{partopsep})};
  \end{tikzpicture}
}{
  \caption{Luetelmien tekemiseen tarkoitetun \ymparisto{list}\-/
    ympäristön mitat}
  \label{kuva/list-mitat}
}

Luetelmiin liittyviä mittoja on useita, ja ne voi asettaa sopiviin
arvoihin \ymparisto{list}\-/ ympäristön toisen argumentin sisällä.
Mittojen sijainnit on koottu kuvaan \ref{kuva/list-mitat}, ja
seuraavassa on lisätietoa niiden merkityksestä.

\begin{maaritelma}{\mitta{#1}}
\item [parskip] Pystysuuntainen mitta ja väli, joka ei liity pelkästään
  \ymparisto{list}\-/ ympäristöön vaan on yleinen tekstikappaleiden
  välinen etäisyysmitta (luku \ref{luku/pystysuuntaiset-välit}). Mitta
  on vaikuttaa kuitenkin myös \ymparisto{list}\-/ ympäristöön, ja sen
  voi määritellä väliaikaisesti uudelleen ympäristön yhteydessä.
\item [topsep] Pystysuuntainen väli, joka ladotaan ennen ja jälkeen
  \ymparisto{list}\-/ ympäristön, yhdessä mittojen \mitta{parskip} ja
  \mitta{partopsep} kanssa.
\item [partopsep] Pystysuuntainen väli, joka lisätään
  \ymparisto{list}\-/ ympäristöä ennen ja sen jälkeen silloin, kun
  ympäristö aloittaa uuden tekstikappaleen eli sitä ennen on tyhjä rivi.
\item [parsep] Pystysuuntainen väli, joka tulee luetelman sisällä
  kaikkien luetelmakohtien väliin sekä saman luetelmakohdan eri
  tekstikappaleiden väliin.
\item [itemsep] Pystysuuntainen väli, joka tulee luetelman sisällä
  luetelmakohtien väliin.
\item [leftmargin] Luetelman kappaleiden vasen sisennys.
\item [rightmargin] Luetelman kappaleiden oikea sisennys. Oletusarvo on
  nolla.
\item [labelwidth] Luetelmamerkin leveys. Merkki ladotaan oletuksena
  tämän alueen oikeaan reunaan, mutta se riippuu \komento{makelabel}\-/
  komennon on määrittelystä. Jos luetelmamerkki on leveämpi kuin
  \mitta{labelwidth}\-/ mitta, luetelmamerkin jälkeinen teksti siirtyy
  oikealle siten, että merkki mahtuu kokonaan. Luetelmamerkin alueen
  vasen reuna sijaitsee kohdassa, jonka voi laskea seuraavasta kaavasta:
  \mitta{leftmargin} − \mitta{labelwidth} + \mitta{itemindent} −
  \mitta{labelsep}. Mikäli haluaa, että luetelmamerkin vasen reuna on
  sisennyksen nollakohdassa, täytyy edellä mainitut mitat asettaa siten,
  että mittojen yhteistulos on nolla. Esimerkissä
  \ref{esim/list-makelabel} on tehty juuri näin.
\item [labelsep] Vaakasuuntainen väli luetelmamerkin jälkeen, ennen
  luetelman sisältötekstiä. Oletusarvo on 0,5\,em.
\item [itemindent] Ylimääräinen luetelmakohtien ensimmäisen rivin
  sisennys. Tätä voi tarvita, kun haluaa säätää luetelmamerkin vasemman
  reunan tiettyyn kohtaan. Katso mitta \mitta{labelwidth} edellä.
  Oletusarvo on nolla.
\item [listparindent] Luetelmakohtien toisen ja sitä seuraavien
  tekstikappaleiden ensimmäisen rivin sisennys. Oletusarvo on nolla.
\end{maaritelma}

\noindent
Kun on lopulta saanut \ymparisto{list}\-/ ympäristön asetukset
kohdalleen ja luetelmat ovat sellaisia kuin pitääkin, on varmaankin hyvä
aika luoda oma ympäristö, joka piilottaa asetukset yhden
ympäristömäärittelyn sisään. Omaa ympäristöä on sitten helppoa käyttää
useitakin kertoja omassa lähdedokumentissa.

\begin{esimerkki*}
  \komentoi{newenvironment}
  \komentoi{setlength}
  \komentoi{textbullet}
  \mittai{itemsep}
  \mittai{labelsep}
  \mittai{leftmargin}
  \mittai{parsep}
  \ymparistoi{list}

\begin{koodilohko}
\newenvironment{numeroimaton}[1][\textbullet]{%
  \begin{list}{#1}{
      \setlength{\leftmargin}{1.1em}
      \setlength{\labelsep}{.2em}
      \setlength{\itemsep}{.4ex plus .1ex}
      \setlength{\parsep}{.2ex}
    }
  }{\end{list}}
\end{koodilohko}
  \caption{Oman numeroimattoman luetelman tekeminen \ymparisto{list}\-/
    ympäristön avulla}
  \label{esim/list-oma-numeroimaton}
\end{esimerkki*}

\begin{esimerkki*}
  \komentoi{arabic}
  \komentoi{bfseries}
  \komentoi{hfill}
  \komentoi{large}
  \komentoi{makelabel}
  \komentoi{newenvironment}
  \komentoi{renewcommand}
  \komentoi{setcounter}
  \komentoi{setlength}
  \komentoi{usecounter}
  \laskurii{enumi}
  \mittai{itemsep}
  \mittai{labelsep}
  \mittai{leftmargin}
  \mittai{parsep}
  \ymparistoi{list}

\begin{koodilohko}
\newenvironment{numeroitu}[1][0]{%
  \begin{list}{\arabic{enumi}.}{
      \usecounter{enumi}
      \setcounter{enumi}{#1}
      \renewcommand{\makelabel}[1]{\hfill\bfseries\large ##1}
      \setlength{\leftmargin}{2em}
      \setlength{\labelsep}{.5em}
      \setlength{\itemsep}{.4ex plus .1ex}
      \setlength{\parsep}{.2ex}
    }
  }{\end{list}}
\end{koodilohko}
  \caption{Oman numeroidun luetelman tekeminen \ymparisto{list}\-/
    ympäristön avulla}
  \label{esim/list-oma-numeroitu}
\end{esimerkki*}

Esimerkissä \ref{esim/list-oma-numeroimaton} on vinkkejä oman
numeroimattoman luetelman tekemiseen. Luetelma on nimeltään
\ymparistox{numeroimaton}, ja sille voi antaa hakasulkeissa valinnaisen
argumentin, jolla valitaan luetelmamerkki. Oletuksena se on
luetelmaympyrä (\komento{textbullet}).

Esimerkki \ref{esim/list-oma-numeroitu} tekee numeroidun ympäristön
nimeltä \ymparistox{numeroitu}. Tälle ympäristölle voi antaa valinnaisen
argumentin, jolla ilmaistaan numerointilaskurin alkukohta. Oletuksena se
on nolla, eli luetelmakohtien numerointi alkaa luvusta~1. Tämä ympäristö
määrittelee myös \komento{makelabel}\-/ komennon siten, että
luetelmanumerot ladotaan lihavoituna (\komento{bfseries}) ja suuremmalla
fontilla (\komento{large}). Numerot myös tasataan luetelmamerkille
varatun tilan oikeaan reunaan (\komento{hfill}). Huomaa, että parametri
\koodi{\#1} kuuluu \ymparistox{numeroitu}\-/ ympäristön määritelmään ja
parametri \koodi{\#\#1} kuuluu sisempään, \komento{makelabel}\-/
komennon määritelmään. Katso lisätietoa komentojen määrittelemistä
käsittelevästä luvusta \ref{luku/komennot-määrittely}.

\section{Taulukot}
\label{luku/taulukot}

Taulukoiden tekemiseen on perus Latexissa kaksi ympäristöä,
\ymparisto{tabular} ja \ymparisto{tabular*}, joskin jälkimmäisen tilalle
sopii yleensä paremmin \paketti{tabularx}\-/ paketin ympäristö
\ymparisto{tabularx}. Ennen kuin käsitellään taulukoiden teknisiä
ohjeita on kuitenkin syytä puhua niiden sijoittelusta ja typografiasta.

Taulukon voi sijoittaa dokumenttiin eri tavoin. Yksi mahdollisuus on
sijoittaa ne normaalin tekstivirran sekaan omiksi tekstikappaleikseen.
Pieni pystysuuntainen väli on silloin tarpeen ennen ja jälkeen taulukon.
Jos tekstikappaleiden välissä ei ole normaalisti väliä
(\mitta{parskip}), täytyy sellainen lisätä käsin. Pystysuuntaisia välejä
käsitellään luvussa \ref{luku/pystysuuntaiset-välit}. Taulukolle tulee
oletuksena sama sisennys kuin tekstikappaleiden ensimmäisellä rivillä on
eli mitta \mitta{parindent} (luku
\ref{luku/ensimmäisen-rivin-sisennys}). Jos sisennyksen haluaa pois,
täytyy käyttää komentoa \komento{noindent}.

Toinen tyypillinen vaihtoehto on sijoittaa taulukko vaakasuunnassa sivun
keskelle, jolloin se erottuu muusta tekstistä vielä selvemmin. Tämän voi
toteuttaa kirjoittamalla taulukkoympäristö \ymparisto{center}\-/
ympäristön sisään (luku \ref{luku/kappaleen-tasaus}).

Kolmas vaihtoehto on tehdä taulukosta leijuva eli antaa Latexin
sijoittaa se sopivaan paikkaan. Samalla taulukolle annetaan kuvateksti
ja yksilöllinen tunniste, niin että siihen voi viitata tekstistä.
Leijuvia osia ja ristiviitteitä käsitellään luvuissa
\ref{luku/leijuosat} ja \ref{luku/ristiviitteet}.

Taulukkojen ulkoasua ja helppolukuisuutta käsittelevät ohjeet yleensä
neuvovat, että pystyviivoja ei pitäisi juuri käyttää. Ne häiritsevät
vasemmalta oikealle lukemista ja vaikeuttavat katseen tarkentamista
solujen sisältöön. Tavallinen sarakkeiden välinen tyhjä tila on yleensä
riittävä erottamaan solut toisistaan.

Se on hyvä yleisohje, mutta sitä tuskin kannattaa pitää ihan ehdottomana
sääntönä. Joskus on tarpeen jakaa sarakkeet mielekkäisiin ryhmiin ja
nimenomaan ohjata katsetta pystysuunnassa tiettyjä sarakkeita pitkin.
Katseen ei välttämättä haluta lipsuvan tietyn kohdan yli, ja pystyviiva
sopii hyvin siihen tarkoitukseen. Tämän luvun esimerkeissä käytetään
paljon vaaka- ja pystyviivoja, jotta lukija hahmottaa helposti, miten
esimerkkikoodin solut ovat yhteydessä ladottuun taulukkoon. Käytännössä
viivoja ei kannata käyttää niin paljon, vaan pari vaakaviivaa yleensä
riittää.

\subsection{Perustoiminnot}

Taulukoiden perusympäristö on \ymparisto{tabular}, jolle täytyy antaa
ainakin yksi argumentti. Ympäristön rakenne on seuraavanlainen:

\komentoi{\keno}
\ymparistoi{tabular}
\begin{koodilohkosis}
\begin{tabular}[sijainti]{sarakkeet}
  solu 1 & solu 2 \\
  solu 3 & solu 4 \\
\end{tabular}
\end{koodilohkosis}

\noindent
Ympäristön valinnainen argumentti \koodi{sijainti} määrittelee
taulukkoympäristön sijainnin pystysuunnassa, jos se on samassa
kappaleessa muun tekstin kanssa. Oletusarvo on \koodi{c}
(\englanti{center}), joka tarkoittaa, että pystysuunnassa taulukon
peruslinja on sen keskellä. Muut vaihtoehdot ovat \koodi{t}
(\englanti{top}), eli taulukon peruslinja on yläreunassa, ja \koodi{b}
(\englanti{bottom}), eli taulukon peruslinja on sen alareunassa.

\leijutlk{
  \providecommand{\rivi}{}
  \renewcommand{\rivi}[2]{\koodi{#1} & #2 \\}
  \begin{tabular}{ll}
    \toprule
    \ots{Tyyppi} & \ots{Merkitys} \\
    \midrule
    \rivi{l}{vasemmalle tasattu sarake (\englanti{left})}
    \rivi{c}{keskitetty sarake (\englanti{center})}
    \rivi{r}{oikealle tasattu sarake (\englanti{right})}
    \rivi{p\{m\}}{sarake, jonka leveys on mitta \koodi{m}}
    \midrule
    \rivi{|}{sarakkeiden välissä pystyviiva}
    \rivi{@\{…\}}{sarakkeiden välissä olevaa tekstiä tai komentoja}
    \rivi{*\{n\}\{s\}}{toistetaan \koodi{n} kertaa sarakkeet \koodi{s}}
    \bottomrule
  \end{tabular}
}{
  \caption{Taulukoiden saraketyyppien ja sarakkeiden välien määrittely}
  \label{tlk/taulukko-sarakemerkit}
}

Pakollinen argumentti \koodi{sarakkeet} määrittelee taulukon sarakkeiden
määrän ja tyypin. Ne ilmaistaan tietyillä kirjaimilla tai muilla
merkeillä, jotka on koottu taulukkoon \ref{tlk/taulukko-sarakemerkit}.
Taulukon neljä ensimmäistä tyyppiä \koodi{lcrp} määrittelevät,
millaisesta sarakkeesta on kyse. Syntyvässä taulukossa tulee olemaan
niin monta saraketta kuin näitä on ympäristön \koodi{sarakkeet}\-/
argumentissa. Kolme ensin mainittua luovat sarakkeen, jonka leveys
määräytyy sarakkeen leveimmän solun perusteella. Neljäs eli saraketyyppi
\koodi{p} tarvitsee aaltosulkeissa argumentiksi mitan (luku
\ref{luku/mitat}), ja sillä ilmaistaan sarakkeen kiinteä leveys. Tässä
saraketyypissä teksti tasataan solun molemmista reunoista eli sanavälit
voivat venyä.

Merkit \koodi{|} ja \koodi{@} voi sijoittaa edellä mainittujen
saraketyyppien väliin tai ennen ensimmäistä saraketta tai viimeisen
sarakkeen jälkeen. Ensin mainittu tekee sarakkeiden väliin pystyviivan.
Viiva yltää taulukon kaikille riveille, mutta komennolla
\komento{multicolumn} voi tehdä rivikohtaisia poikkeuksia. Tätä komentoa
käsitellään myöhemmin. Merkki \koodi{@} tarvitsee argumentin
aaltosulkeissa. Se voi olla mitä hyvänsä tekstiä tai komentoja, jotka
halutaan suorittaa joka rivillä sarakkeiden välissä.

Sarakemerkki \koodi{*} on vain keino toistaa tiettyjä sarakemäärityksiä
useamman kerran. Se tarvitsee aaltosulkeissa kaksi argumenttia:
ensimmäinen on luku, ja toinen sisältää mitä hyvänsä edellä mainittuja
saraketyyppejä. Nämä sarakkeet toistuvat luvun ilmaiseman määrän.
Esimerkiksi \koodi{*\{3\}\{rl|\}} tarkoittaa samaa kuin
\koodi{rl|rl|rl|}.

\begin{esimerkki*}
  \komentoi{\keno}
  \komentoi{arraystretch}
  \komentoi{cline}
  \komentoi{hline}
  \komentoi{hspace}
  \komentoi{renewcommand}
  \ymparistoi{tabular}

\begin{koodilohko}
\begin{tabular}{|l@{\hspace{4em}}r|c|}
  \hline
  Vasen & Oikea & Keskitetty \\[.5ex]
  \hline
  16 & 32 & 11732 \\
  \cline{1-2}
  71 & 87235 & 2 \\
  \cline{3-3}
  1235 & 238 & 982 \\
  \hline
\end{tabular}
\end{koodilohko}

  \begin{tulos}
    \versaalinum
    \renewcommand{\arraystretch}{1.3}
    \begin{tabular}{|l@{\hspace{4em}}r|c|}
      \hline
      Vasen & Oikea & Keskitetty \\[.5ex]
      \hline
      16 & 32 & 11732 \\
      \cline{1-2}
      71 & 87235 & 2 \\
      \cline{3-3}
      1235 & 238 & 982 \\
      \hline
    \end{tabular}
  \end{tulos}

  \caption{Taulukoiden erilaisia saraketyyppejä ja viivoja}
  \label{esim/taulukkomäärityksiä}
\end{esimerkki*}

Esimerkki \ref{esim/taulukkomäärityksiä} havainnollistaa taulukon
ominaisuuksia ja eri saraketyyppejä. Sitä tuskin kannattaa pitää
typografisena esimerkkinä, mutta tarkoituksena on osoittaa eri
saraketyyppien toiminta käytännössä. Sarakkeiden välissä on yleensä
pystyviiva, mutta yhdessä välissä on \koodi{@\{\dots\}}\=/ merkeillä
määritelty komento \komento{hspace}, jolla tehdään poikkeuksellisen
leveä (4\,em) tyhjä tila sarakkeiden väliin.

Rivillä olevat peräkkäiset solut erotetaan toisistaan \koodi{\&}\=/
merkillä, ja taulukon rivi täytyy päättää aina \komento{\keno}\=/
komentoon. Tälle rivinvaihtokomennolle voi antaa hakasulkeissa yhden
argumentin. Se on mitta ja tarkoittaa, kuinka paljon ylimääräistä
pystysuuntaista tyhjää tilaa halutaan rivin jälkeen, esimerkiksi
\komento{\keno}\komentoargv{.5ex}.

Vaakasuuntainen, koko taulukon levyinen viiva tehdään komennolla
\komento{hline} ja määrämittaisia viivoja komennolla \komento{cline}.
Tälle komennolle täytyy antaa yksi argumentti, joka sisältää kaksi
yhdysmerkillä erotettua lukua. Luvuilla ilmaistaan, mistä sarakkeesta
mihin sarakkeeseen viiva yltää.

\subsection{Asetuksia}

Taulukon rivit saatetaan latoa oletuksena vähän liian lähelle toisiaan
-- varsinkin jos käytetään vaakaviivoja. Sen korjaamiseksi on
taulukoille olemassa oma rivivälikerroin, joka on komennossa
\komento{arraystretch}. Kerroin on desimaaliluku, jolla normaali
rivikorkeus kerrotaan taulukoiden sisällä. Sopiva kerroin lienee yleensä
1\==1,3. Oletusarvo on~1. Kerroinkomennon voi määritellä milloin hyvänsä
uudelleen, ja se tehdään seuraavan esimerkin mukaisesti:

\komentoi{renewcommand}
\komentoi{arraystretch}
\begin{koodilohkosis}
\renewcommand{\arraystretch}{1.3}
\end{koodilohkosis}

\noindent
Sarakkeiden määrittelyssä voi kirjoittaa useita pystyviivoja eli
\koodi{|}\=/ merkkejä peräkkäin. Silloin pystyviivoja myös ladotaan
taulukkoon useita vierekkäin. Ladottujen viivojen väliseen etäisyyteen
voi vaikuttaa mitan \mitta{doublerulesep} avulla. Sen oletusarvo on
2\,pt. Kaikkien viivoja paksuus puolestaan on mitassa
\mitta{arrayrulewidth}, jonka oletusarvo on 0,4\,pt.

Jos sarakkeiden välissä ei ole \koodi{@}\=/ merkillä tehtyjä
poikkeuksia, ladotaan joka solun vasempaan ja oikeaan reunaan väli, joka
on määritelty mitassa \mitta{tabcolsep}. Tämän mitan suuruinen väli
siis ladotaan kerran taulukon vasempaan ja oikeaan reunaan ja kaksi
kertaa jokaisen sarakkeen väliin. Oletusarvo on 6\,pt. Seuraavassa on
koonti edellä mainittujen mittojen asettamisesta ja niiden oletusarvot.

\komentoi{setlength}
\mittai{doublerulesep}
\mittai{arrayrulewidth}
\mittai{tabcolsep}
\begin{koodilohkosis}
\setlength{\doublerulesep}{2pt}
\setlength{\arrayrulewidth}{.4pt}
\setlength{\tabcolsep}{6pt}
\end{koodilohkosis}

\noindent
Sarakemääritelmässä voi \koodi{@}\=/ merkillä säätää sarakkeiden välit
haluamansa laiseksi. Merkeillä \koodi{@\{\}} sarakeväli poistetaan
kokonaan. Jos \koodi{@\{\dots\}}\=/ määrittelyyn halutaan mukaan
normaali sarakkeiden pystyviiva, täytyy käyttää komentoa
\komento{vline}, esimerkiksi seuraavasti:

\komentoi{\keno}
\komentoi{hspace}
\komentoi{vline}
\ymparistoi{tabular}
\begin{koodilohkosis}
\begin{tabular}{|l@{\hspace{1em}\vline\hspace{1em}}l|}
  ensimmäinen & toinen \\
\end{tabular}
\end{koodilohkosis}

\subsection{Poikkeuksellisia sarakkeita ja rivejä}

Taulukkoympäristön argumenteissa määritellään taulukon sarakkeiden määrä
ja tyypit, mutta niihin on mahdollista tehdä yksittäisiä poikkeuksia
taulukon sisällä. Sarakekohtaiset poikkeukset tehdään komennolla
\komento{multicolumn} ja rivikohtaiset poikkeukset komennolla
\komento{multirow}, joka kuuluu pakettiin \pakettictan{multirow}.

\begin{esimerkki*}
  \komentoi{\keno}
  \komentoi{multicolumn}
  \ymparistoi{tabular}

\begin{koodilohko}
\begin{tabular}{|r@{}l|}
  \multicolumn{2}{l}{Tulokset} \\
  \hline
  425 & ,34 \\
    4 & ,021 \\
   32 & \\
    0 & ,75 \\
  \hline
\end{tabular}
\end{koodilohko}

  \begin{tulos}
    \renewcommand{\arraystretch}{1.2}
    \versaalinum
    \begin{tabular}{|r@{}l|}
      \multicolumn{2}{l}{Tulokset} \\
      \hline
      425 & ,34 \\
      4 & ,021 \\
      32 & \\
      0 & ,75 \\
      \hline
    \end{tabular}
  \end{tulos}

  \caption{Desimaalilukujen tasaaminen pilkun kohdalta, ja
    poikkeuksellisten sarakkeiden tekeminen \komento{multicolumn}\-/
    komennolla}
  \label{esim/taulukko-desimaalipilkku}
\end{esimerkki*}

Esimerkissä \ref{esim/taulukko-desimaalipilkku} on taulukoitu lukuja,
jotka halutaan latoa samaan linjaan desimaalipilkun kohdalta.
Tasaamisessa hyödynnetään kahden sarakkeen rajakohtaa: luvun
kokonaislukuosa ja desimaaliosa tasataan tätä rajakohtaa vasten --
toinen oikealle~(\koodi{r}), toinen vasemmalle~(\koodi{l}). Sarakeväli
on poistettu tästä kohdasta kokonaan \koodi{@\{\}}\=/ merkeillä.

Esimerkkitaulukon otsikko ''Tulokset'' halutaan kuitenkin sijoittaa
desimaalilukujen koko leveydelle eli kahden sarakkeen päälle. Lisäksi
otsikkoriviltä halutaan poistaa pystyviivat, jotka muutoin ladotaan
taulukon reunoille. Näihin asioihin tarvitaan komentoa
\komento{multicolumn}, joka on tarkoitettu poikkeuksellisten
sarakemääritysten tekemiseen. Komentoa käytetään seuraavasti:

\komentoi{multicolumn}
\begin{koodilohkosis}
\multicolumn{n}{sarakkeet}{teksti}
\end{koodilohkosis}

\noindent
Komennon \komento{multicolumn} voi sijoittaa mihin hyvänsä taulukon
soluun. Komennon ensimmäinen argumentti \koodi{n} on luku ja ilmaisee,
kuinka monen sarakkeen alueeseen vaikutetaan. Toinen argumentti
\koodi{sarak\-keet} sisältää uuden sarakemäärittelyn tälle alueelle, ja
se tehdään taulukon \ref{tlk/taulukko-sarakemerkit} merkkien avulla.
Kolmas argumentti on teksti, joka ladotaan tähän uudella tavalla
määriteltyyn alueeseen.

Edellä on käsitelty vain soluja, joissa on yksirivinen teksti, mutta
tietyissä tilanteissa yhteen soluun voidaan latoa useampia rivejä.
Saraketyyppi~\koodi{p} (taulukko \ref{tlk/taulukko-sarakemerkit}) on
leveydeltään kiinteä, ja jos solun sisältö ei mahdu kyseiseen tilaan, se
jaetaan useammalle riville ja sanoja mahdollisesti katkaistaan
tavurajojen kohdalta. Myös \komento{newline}\-/ komennolla voi vaihtaa
riviä yksittäisen solun sisällä. Sekin toimii vain kiinteäleveyksisillä
\koodi{p}\-/ tyypin sarakkeilla.

Seuraavassa on esimerkki, kuinka yhden solun sisältö voi jakautua
kahdelle riville. Teknisessä mielessä taulukkorivien määrä ei kuitenkaan
kasva, vaan muihin saman taulukkorivin soluihin tulee tyhjää tilaa.

\komentoi{\keno}
\komentoi{hline}
\ymparistoi{tabular}
\begin{koodilohkosis}
\begin{tabular}{|l|p{6.5em}|l|}
  \hline solu & Tämä ei mahdu yhdelle riville. & solu \\ \hline
\end{tabular}
\end{koodilohkosis}

\begin{tulossis}
  \renewcommand{\arraystretch}{1.2}
  \begin{tabular}{|l|p{6.5em}|l|}
    \hline solu & Tämä ei mahdu yhdelle riville. & solu \\ \hline
  \end{tabular}
\end{tulossis}

\noindent
Joskus puolestaan halutaan, että yksi solu levittäytyy useamman rivin
alueelle verrattuna taulukon muihin riveihin. Tämän toteutukseen
tarvitaan \paketti{multirow}\-/ pakettia ja sen \komento{multirow}\-/
komentoa. Komennon pakolliset argumentit ovat seuraavat:

\komentoi{multirow}
\begin{koodilohkosis}
\multirow{n}{leveys}{teksti}
\end{koodilohkosis}

\noindent
Ensimmäinen argumentti \koodi{n} on luku, joka ilmaisee, kuinka monen
rivin alueelle solu levitetään. Toinen argumentti \koodi{leveys} on
tekstin leveyttä ilmaiseva mitta. Argumentiksi voi kirjoittaa pelkän
tähden (\koodi{*}), jolloin käytetään tekstin luonnollista leveyttä.
Kolmas argumentti on teksti, joka ladotaan tähän pystysuunnassa
laajennettuun soluun.

Komennolla on myös valinnaisia argumentteja, jotka ilmaistaan
hakasulkeiden avulla. Tässä niistä tärkeimmät:

\komentoi{multirow}
\begin{koodilohkosis}
\multirow[sijainti]{n}{leveys}[pysty]{teksti}
\end{koodilohkosis}

\noindent
Edellisen esimerkin valinnainen argumentti \koodi{sijainti} on kirjain,
jolla ilmaistaan solun sisällön pystysuuntainen asemointi ylös, keskelle
tai alas: \koodi{t} (\englanti{top}), \koodi{c} (\englanti{center},
oletus) tai \koodi{b} (\englanti{bottom}). Valinnainen argumentti
\koodi{pysty} on mitta, jolla hienosäädetään solun sisällön sijaintia
pystysuunnassa. Positiivinen mitta siirtää sisältöä ylöspäin,
negatiivinen alaspäin.

Komennon muista mahdollisuuksista kannattaa lukea \paketti{multirow}\-/
paketin ohjekirjasta. Esimerkki \ref{esim/taulukko-multirow}
havainnollistaa komennon peruskäyttöä. \komento{multirow}\-/ komento
kirjoitetaan siihen soluun, josta monirivinen solu alkaa. Sen
alapuolelta täytyy jättää riittävä määrä soluja tyhjäksi.

\begin{esimerkki*}
  \komentoi{\keno}
  \komentoi{cline}
  \komentoi{hline}
  \komentoi{multirow}
  \ymparistoi{tabular}

\begin{koodilohko}
\begin{tabular}{|l|l|l|}
  \hline
  \multirow{2}{*}{kaksirivinen solu} & tavallisia & soluja \\
  \cline{2-3}
                                     & kahdella   & rivillä \\
  \hline
\end{tabular}
\end{koodilohko}

  \begin{tulos}
    \renewcommand{\arraystretch}{1.3}
    \begin{tabular}{|l|l|l|}
      \hline
      \multirow{2}{*}[-1bp]{kaksirivinen solu} & tavallisia & soluja \\
      \cline{2-3}
                                               & kahdella & rivillä \\
      \hline
    \end{tabular}
  \end{tulos}

  \caption{\paketti{multirow}\-/ paketin \komento{multirow}\-/
    komennolla voi levittää yhden solun useamman rivin alueelle}
  \label{esim/taulukko-multirow}
\end{esimerkki*}

\subsection{Kiinteälevyiset taulukot}

Latexin toinen taulukkoympäristö \ymparisto{tabular*} toimii muuten
samalla tavalla kuin \ymparisto{tabular}, mutta se tarvitsee yhden
argumentin enemmän, taulukon leveysmitan:

\ymparistoi{tabular*}
\begin{koodilohkosis}
\begin{tabular*}{leveys}[sijainti]{sarakkeet}
  ...
\end{tabular*}
\end{koodilohkosis}

\noindent
Valitettavasti taulukon leveysmittaa voi hyödyntää vain rajallisesti. Se
vaikuttaa taulukon leveyteen silloin, kun sarakkeiden välit on
määritelty venyviksi käyttämällä seuraavanlaista sarakevälin määritystä:

\komentoi{extracolsep}
\mittai{fill}
\begin{koodilohkosis}
@{\extracolsep{\fill}}
\end{koodilohkosis}

\noindent
Edellisestä on tärkeää huomata, että sarakkeiden leveydet eivät veny
vaan ainoastaan niiden välissä oleva tyhjä tila. Usein olisi kuitenkin
hyödyllisempää, että joidenkin sarakkeiden leveys olisi venyvä ja ne
mukautuisivat koko taulukon leveyteen. Tämä ominaisuus saadaan
\pakettictan{tabularx}\-/ paketin avulla. Se määrittelee uuden
taulukkoympäristön \ymparisto{tabularx}, joka on käytännössä parempi
versio \ymparisto{tabular*}\-/ ympäristöstä. Sen käyttö on samanlaista:

\ymparistoi{tabularx}
\begin{koodilohkosis}
\begin{tabularx}{leveys}[sijainti]{sarakkeet}
  ...
\end{tabularx}
\end{koodilohkosis}

\noindent
Ympäristö \ymparisto{tabularx} sisältää yhden uuden
saraketyypin:~\koodi{X} (vrt. taulukko \ref{tlk/taulukko-sarakemerkit}).
Tämä saraketyyppi toimii kuten tyyppi~\koodi{p}, mutta sille ei anneta
argumentiksi mittaa, vaan sen leveys on äärettömästi venyvä.
Tämäntyyppiset sarakkeet täyttävät siis kaiken vapaana olevan tilan, ja
jos niitä on useita, ne ovat oletuksena keskenään yhtä leveitä. Tämä
ilmenee seuraavasta esimerkistä:

\komentoi{hline}
\mittai{linewidth}
\ymparistoi{tabularx}
\begin{koodilohkosis}
\begin{tabularx}{\linewidth}{|l|X|X|}
  \hline luonnollinen & venyvä & venyvä \\ \hline
\end{tabularx}
\end{koodilohkosis}

\begin{tulossis}
  \renewcommand{\arraystretch}{1.2}
  \begin{tabularx}{\linewidth}{|l|X|X|}
    \hline luonnollinen & venyvä & venyvä \\ \hline
  \end{tabularx}
\end{tulossis}

\noindent
Saraketyypissä \koodi{X} teksti tasataan oletuksena solujen
molemmista reunoista, mutta siihen on mahdollista vaikuttaa
sijoittamalla sarakemäärittelyssä \koodi{>}\=/ merkki ja sen
aaltosulkeissa olevat argumentit ennen \koodi{X}\=/ merkkiä.
Argumentiksi kirjoitetaan sopivia asetuskomentoja. Seuraava
sarakemäärittely tekee vasemmalle tasatun \koodi{X}\=/ sarakkeen:

\komentoi{raggedright}
\komentoi{arraybackslash}
\begin{koodilohkosis}
>{\raggedright\arraybackslash}X
\end{koodilohkosis}

\noindent
Sarakkeen solujen tasaaminen vasemmalle tulee komennon
\komento{raggedright} vaikutuksesta (luku \ref{luku/kappaleen-tasaus}).
Sen perään tarvitaan myös komento \komento{arraybackslash}, koska
\komento{raggedright} määrittelee \komento{\keno}\=/ rivinvaihtokomennon
uudelleen tavalla, joka on ristiriidassa taulukon rivien lopussa
tarvittavan vastaavan komennon kanssa. \komento{arraybackslash}
palauttaa määritelmän sellaiseksi, että se toimii taulukossa.

\paketti{tabularx}\-/ paketin myötä voi määritellä omiakin
saraketyyppejä. Hyödyllisiä tyyppejä voisivat olla esimerkiksi
\koodi{LCR}, jotka toimisivat lähes samoin kuin perus Latexin
saraketyypit \koodi{lcr} (taulukko \ref{tlk/taulukko-sarakemerkit})
mutta olisivat kuitenkin leveydeltään venyviä. Omia saraketyyppejä
tehdään komennolla \komento{newcolumntype}, ja edellä mainitut tyypit
voisi määritellä seuraavasti:

\komentoi{arraybackslash}
\komentoi{centering}
\komentoi{newcolumntype}
\komentoi{raggedleft}
\komentoi{raggedright}
\begin{koodilohkosis}
\newcolumntype{L}{>{\raggedright\arraybackslash}X}
\newcolumntype{C}{>{\centering\arraybackslash}X}
\newcolumntype{R}{>{\raggedleft\arraybackslash}X}
\end{koodilohkosis}

\noindent
Mitä hyvänsä saraketyyppejä voi tehdä vastaavalla tavalla. Esimerkiksi
taulukon ensimmäiseen, otsikoita ilmaisevaan saraketyyppiin \koodi{O}
voitaisiin laittaa fonttikomento, jotta koko sarake ladotaan
automaattisesti eri kirjainleikkauksella kuin muu taulukko.

\komentoi{newcolumntype}
\komentoi{bfseries}
\begin{koodilohkosis}
\newcolumntype{O}{>{\bfseries}l} % lihavoitu otsikkosarake
\end{koodilohkosis}

\noindent
Ympäristön \ymparisto{tabularx} käyttöön liittyy tiettyjä rajoituksia:
sitä ei voi käyttää normaalisti omien ympäristöjen määrittelyssä (luku
\ref{luku/ympäristöt}). Esimerkiksi seuraava ympäristön määrittely ei
käytännössä toimi:

\komentoi{newenvironment}
\ymparistoi{tabularx}
\mittai{linewidth}
\begin{koodilohkosis}
\newenvironment{omataulukko}[1]
{\begin{tabularx}{\linewidth}{#1}}
{\end{tabularx}}
\end{koodilohkosis}

\noindent
Edellä mainitun puutteen saa korjattua käyttämällä oman ympäristön
määrittelyssä \ymparisto{tabularx}\-/ ympäristön aloitukseen komentoa
\komento{tabularx} ja lopetukseen komentoa \komento{endtabularx}. Ei
siis käytetä normaaleja aloitus\-/\ ja lopetuskomentoja \komento{begin}
ja \komento{end}.

\komentoi{newenvironment}
\komentoi{tabularx}
\komentoi{endtabularx}
\mittai{linewidth}
\begin{koodilohkosis}
\newenvironment{omataulukko}[1]
{\tabularx{\linewidth}{#1}}
{\endtabularx}
\end{koodilohkosis}

\subsection{Muita paketteja}

\paketti{tabularx}\-/ paketin kanssa samankaltainen on paketti
\pakettictan{tabulary}, jossa on joitakin erilaisia ominaisuuksia. Se
muun muassa sisältää valmiiksi uudet saraketyypit \koodi{LCRJ}.

Edellä esitellyt taulukkoympäristöt ladotaan aina kokonaisena eli vain
yhden sivun alueelle. Useammalle sivulle jakautuvia taulukoita voi tehdä
\pakettictan{longtable}\-/ paketin taulukkoympäristön
\ymparisto{longtable} avulla. Sitä voi käyttää kuten
\ymparisto{tabular}\-/ ympäristöä, mutta lisäksi se sisältää komentoja,
joiden avulla saa joka sivulla toistumaan automaattisesti tietyt
otsikkorivit tai yhteenvetorivit. Toinen mahdollisuus yli sivurajan
yltävien taulukon kaltaisten rakenteiden tekemiseen ovat sarkaimet,
joita käsitellään luvussa \ref{luku/sarkaimet}.

Hyödyllinen taulukoiden apupaketti on \pakettictan{booktabs}, jonka
avulla saa erilaisia ja tyylikkäämpiä vaakaviivoja. Paketin komennot
\komento{toprule}, \komento{midrule}, \komento{bottomrule} ja
\komento{cmidrule} piirtävät rivien väliin erivahvuisia viivoja ja
huolehtivat myös niiden ylä- ja alapuolisesta välistä paremmin kuin
Latexin omat viivakomennot.

Taulukon rivien taustan voi värittää \pakettictan{xcolor}\-/ paketin
(luku \ref{luku/värit}) komentojen avulla. Varsinkin leveiden
taulukoiden lukemista voi helpottaa, jos rivien eri taustavärit
vuorottelevat. Tarkemmat tiedot selviävät paketin ohjekirjasta, mutta
seuraavan esimerkin avulla ominaisuutta pääsee kokeilemaan:

\komentoi{rowcolors}
\komentoi{usepackage}
\pakettii{xcolor}
\begin{koodilohkosis}
\usepackage[table]{xcolor}     % lähdedokumentin esittelyosaan
\rowcolors{1}{blue!10}{red!10} % ennen taulukkoympäristöjä
\end{koodilohkosis}

\section{Sarkaimet}
\label{luku/sarkaimet}

Sarkaimet ovat ennalta määritettyjä vaakasuuntaisia sisennys\-/\ tai
tasauskohtia. Niitä voi hyödyntää tekstin tasaamisessa tiettyihin
kohtiin rivillä. Suomenkielisissä asiakirjoissa sarkainkohdat ovat 23
millimetrin välein, mutta jos ei ole kirjoittamassa virallista
asiakirjaa, voi kirjoittaja määrittää sarkainkohdat niin kuin haluaa.

Ajatus sarkaimista on vanha, sillä mekaanisissa kirjoituskoneissakin oli
sarkainnäppäin (\englanti{tab, tabular}), joka siirsi telaa eteenpäin
seuraavaan sarkainkohtaan. Se helpotti tekstin tasaamista tiettyyn
kohtaan ja sen myötä taulukoiden kirjoittamista.

Latexissa on taulukoita varten omatkin ympäristönsä (luku
\ref{luku/taulukot}), mutta jos ei tarvitse varsinaista taulukkoa vaan
ainoastaan yksittäisiä tasauskohtia, voi olla kätevämpää käyttää
\ymparisto{tabbing}\-/ sarkainympäristöä tai rakentaa tasaukset
näkymättömien laatikoiden avulla. Näitä käsitellään seuraavissa
alaluvuissa.

\subsection{Sarkainympäristö}

Sarkaimet voi toteuttaa niitä varten tehdyn ympäristön eli
\ymparisto{tabbing}\-/ ympäristön avulla. Esimerkki
\ref{esim/tabbing-perus} havainnollistaa ympäristön peruskäyttöä.
Rivillä~2 asetetaan kaikki sarkainkohdat eli tehdään vaakasuuntaisia
välejä \komento{hspace}\-/ komennolla ja merkitään sarkainkohdat
\komento{=}\=/ komennoilla. Sen rivin lopussa on komento \komento{kill},
joka hylkää rivin eli jättää sen latomatta. Seuraavien rivien lopussa on
\komento{\keno}\-/ komento -- paitsi viimeisellä rivillä ennen
ympäristön lopetuskomentoa. Rivin sisällä seuraavaan sarkainkohtaan
hypätään \komento{>}\-/ komennolla.

\begin{esimerkki*}
  \ymparistoi{tabbing}
  \komentoi{hspace}
  \komentoi{kill}
  \komentoi{=}
  \komentoi{>}

\begin{koodilohko}
\begin{tabbing}
  \hspace{23mm} \= \hspace{23mm} \= \hspace{23mm} \= \kill
  Tässä \> on \\
  \> eri \> tavoin \\
  \>\> sisennettyä \\
  \>\>\> tekstiä.
\end{tabbing}
\end{koodilohko}

  \begin{tulos}
    \begin{tabbing}
      \hspace{23mm} \= \hspace{23mm} \= \hspace{23mm} \= \kill
      Tässä \> on \\
      \> eri \> tavoin \\
      \>\> sisennettyä \\
      \>\>\> tekstiä.
    \end{tabbing}
  \end{tulos}
  \caption{Sarkainkohtien määrittely ja käyttö \ymparisto{tabbing}\-/
    ympäristössä}
  \label{esim/tabbing-perus}
\end{esimerkki*}

\ymparisto{tabbing}\-/ ympäristön sisällä on käytettävissä muutama
muukin erityiskomento, joilla vaikutetaan sarkainten käsittelyyn.
Seuraavassa kerrotaan ympäristön kaikkien erityiskomentojen merkitys, ja
esimerkissä \ref{esim/tabbing-erikois} havainnollistetaan joidenkin
komentojen käyttöä.

\begin{maaritelma}{\komento{#1}}
\item [\keno] Rivinvaihto, joka täytyy kirjoittaa jokaisen rivin
  loppuun, lukuun ottamatta viimeistä riviä ennen ympäristön
  lopetuskomentoa \komento{end}.
\item [kill] Rivinvaihto, joka hylkää kyseisen rivin. Tämä on
  käyttökelpoinen sarkainkohtien määrittelyrivillä, jota itsessään ei
  haluta latoa näkyviin.
\item [=] Sarkainkohdan määrittäminen komennon kohdalle.
\item [>] Hyppää seuraavaan sarkainkohtaan nykyisellä rivillä.
\item [+] Siirtää seuraavan rivin aloituskohtaa yhden sarkaimen verran
  oikealle siitä, missä se oli ennestään.
\item [-] Siirtää seuraavan rivin aloituskohtaa yhden sarkaimen verran
  vasemmalle siitä, missä se oli ennestään.
\item [<] Siirtää nykyisen rivin aloituskohdan yhden sarkaimen verran
  vasemmalla. Tätä komentoa voi käyttää vain rivin alussa ja vain
  silloin, kun rivin aloituskohtaa on sisennetty aiemmin \komento{+}\-/
  komennolla. Tämä komento ei muuta pysyvästi rivin aloituskohdan
  asetusta, joka on tehty komennoilla \komento{+} ja \komento{-}.
\item ['] Latoo komentoa edeltävän tekstin nykyisen sarkainkohdan
  vasemmalle puolelle ja tasaa tekstin oikeasta reunasta. Tekstin oikean
  reunan ja sarkainkohdan väliin tulee mitan \mitta{tabbingsep}
  suuruinen väli.
\item [`] Tasaa komennon jälkeisen tekstin sivun oikeaan reunaan.
\item [pushtabs] Tallentaa nykyiset sarkainkohdat muistiin. Tämän
  jälkeen sarkainasetuksia voi muuttaa ja alkuperäiset asetukset
  palauttaa \komento{poptabs}\-/ komennolla.
\item [poptabs] Palauttaa aiemmin \komento{pushtabs}\-/ komennolla
  tallennetut sarkainkohdat käyttöön.
\item [a'] Latoo akuuttiaksentin (\'a) seuraavaan kirjaimeen. Vastaa
  \komento{'}\-/ komentoa \ymparisto{tabbing}\-/ ympäristön ulkopuolella
  (taulukko \ref{tlk/tarkkeet}, s.~\pageref{tlk/tarkkeet}).
\item [a`] Latoo gravisaksentin (\`a) seuraavaan kirjaimeen. Vastaa
  \komento{`}\-/ komentoa \ymparisto{tabbing}\-/ ympäristön
  ulkopuolella.
\item [a=] Latoo pituusmerkin (\=a) seuraavaan kirjaimeen. Vastaa
  \komento{=}\-/ komentoa \ymparisto{tabbing}\-/ ympäristön
  ulkopuolella.
\end{maaritelma}

\begin{esimerkki*}
  \komentoi{'}
  \komentoi{+}
  \komentoi{-}
  \komentoi{<}
  \komentoi{=}
  \komentoi{>}
  \komentoi{`}
  \komentoi{a'}
  \komentoi{a`}
  \komentoi{a=}
  \komentoi{hspace}
  \komentoi{kill}
  \komentoi{setlength}
  \mittai{tabbingsep}
  \ymparistoi{tabbing}

\begin{koodilohko}
\setlength{\tabbingsep}{0mm}
\begin{tabbing}
  \hspace{23mm} \= \hspace{23mm} \= \hspace{23mm} \= \kill
  tässäpä \> kaikki \> eri \> sarkainkohdat \+ \\
  valmiiksi sisennetty \+ \\
  vielä enemmän sisennetty \- \\
  vähemmän sisennetty \>\> oikealle\' \\
  \< poikkeuksellisesti sisennetty vähemmän \` täysin oikealla \\
  paluu aiempaan sisennystasoon ja tarkkeita: \a'{a}\a`{a}\a={a}
\end{tabbing}
\end{koodilohko}

  \begin{tulos}
    \setlength{\tabbingsep}{0mm}
    \begin{tabbing}
      \hspace{23mm} \= \hspace{23mm} \= \hspace{23mm} \= \kill
      tässäpä \> kaikki \> eri \> sarkainkohdat \+ \\
      valmiiksi sisennetty \+ \\
      vielä enemmän sisennetty \- \\
      vähemmän sisennetty \>\> oikealle\' \\
      \< poikkeuksellisesti sisennetty vähemmän \` täysin oikealla \\
      paluu aiempaan sisennystasoon ja tarkkeita: \a'{a}\a`{a}\a={a}
    \end{tabbing}
  \end{tulos}

  \caption{Erikoisempia sarkaintoimintoja \ymparisto{tabbing}\-/
    ympäristössä}
  \label{esim/tabbing-erikois}
\end{esimerkki*}

\subsection{Laatikkototeutus}

Sarkainkohtia sisältävän rivin voi varsin helposti toteuttaa laatikoiden
(luku \ref{luku/laatikot}) avulla. Varsinkin \komento{makebox}\-/
komento soveltuu hyvin, koska laatikolla ei ole reunaviivoja ja laatikon
leveyden ja sisällön tasauksen voi asettaa. Katso lisätietoa
\komento{makebox}\-/ komennon argumenteista luvusta
\ref{luku/laatikot-pienet}.

\begin{esimerkki*}
  \komentoi{newlength}
  \komentoi{setlength}
  \komentoi{newcommand}
  \komentoi{makebox}
  \komentoi{ignorespaces}
  \komentoi{noindent}

\begin{koodilohko}
\newlength{\sarkainleveys}
\setlength{\sarkainleveys}{23mm}
\newcommand{\sarkain}[3][l]{%
  \makebox[#2\sarkainleveys][#1]{#3}\ignorespaces}

\noindent
\sarkain{1}{ensin} \sarkain{1}{kaikki} \sarkain{1}{eri} sarkainkohdat \\
\sarkain{1}{} sarkaimen verran sisennetty rivi \\
\sarkain{2}{jotain tekstiä} \sarkain[r]{1}{oikealle} \\
\sarkain{1}{} \sarkain[c]{2}{keskitetty}
\end{koodilohko}

  \begin{tulos}
    \newlength{\sarkainleveys}
    \setlength{\sarkainleveys}{23mm}
    \newcommand{\sarkain}[3][l]{%
      \makebox[#2\sarkainleveys][#1]{#3}\ignorespaces}

    \noindent
    \sarkain{1}{ensin} \sarkain{1}{kaikki} \sarkain{1}{eri} sarkainkohdat \\
    \sarkain{1}{} sarkaimen verran sisennetty rivi \\
    \sarkain{2}{jotain tekstiä} \sarkain[r]{1}{oikealle} \\
    \sarkain{1}{} \sarkain[c]{2}{keskitetty}
  \end{tulos}

  \caption{Sarkainten toteutus laatikoiden avulla}
  \label{esim/sarkain-laatikot}
\end{esimerkki*}

Esimerkissä \ref{esim/sarkain-laatikot} määritetään riveillä 1--2
sarkainten leveysmitta \mittax{sarkainleveys} ja riveillä 3--4 komento
\komentox{sarkain}, jolla tekstiä voi kirjoittaa tietyn levyiseen
laatikkoon. Näin määritelty \komentox{sarkain}\-/ komento tarvitsee
ainakin kaksi argumenttia: ensimmäinen argumentti on kerroin, kuinka
monen sarkaimen levyinen laatikko tehdään; toinen argumentti on teksti,
joka ladotaan tähän laatikkoon. Ennen pakollisia argumentteja voi antaa
myös valinnaisen argumentin hakasulkeissa: se on laatikon sisällön
tasaus, ja oletus on vasempaan reunaan. Esimerkki
\ref{esim/sarkain-laatikot} havainnollistaa sarkainten toteuttamista
laatikoiden avulla.

\section{Leijuvat osat}
\label{luku/leijuosat}

Leijuvat osat ovat sellaisia dokumentin osia -- käytännössä
ympäristöjä~--, joita ei ladota normaalin leipätekstin sekaan tiettyyn
kohtaan. Ne ''leijuvat'' irrallaan tavallisesta tekstivirrasta, ja Latex
huolehtii niiden sijoittelusta. Leijuvilla osilla on yleensä kuvateksti,
jonka alussa on sen tyyppi ja numero, esimerkiksi ''\figurename~1'' tai
''\tablename\ 4.2''. Näiden perässä on varsinainen kuvateksti, joka
kertoo kyseisen leijuvan osan sisällöstä tai merkityksestä.
Leipätekstistä usein viitataan leijuviin osiin käyttämällä ilmauksia
kuten ''kuvassa~1''. Näissä hyödynnetään Latexin ristiviittaustoimintoja
(luku \ref{luku/ristiviitteet}).

Perus Latex sisältää kaksi erityyppistä leijuvaa osaa, kuvat ja taulukot
(luku \ref{luku/leijuosat-latex}), mutta \paketti{floatrow}\-/ paketin
avulla voi tehdä muunkinlaisia (luku \ref{luku/leijuosat-omat}).
Esimerkiksi tässä oppaassa on käytössä myös tyyppi ''Esimerkki'', jota
käytetään leijuvien Latex\-/ koodiesimerkkien toteuttamiseen.

Koska Latex huolehtii leijuvien osien sijoittamisesta, niiden lopullinen
järjestyskin voi hieman poiketa siitä, miten ne ovat lähdetiedostossa.
Tosin samantyyppiset leijuvat osat, esimerkiksi kuvat, ladotaan aina
siinä järjestyksessä kuin ne ovat lähdetiedostossa. Sen sijaan
erityyppiset leijuvat osat saatetaan latoa eri järjestyksessä:
esimerkiksi jokin taulukko saatetaan latoa ennen tiettyä kuvaa, vaikka
ne olisivat lähdedokumentissa päinvastaisessa järjestyksessä. Latex
pyrkii latomaan hyvännäköisiä sivuja, ja leijuvien osien sijoittelulla
se voi vaikuttaa asiaan.

\subsection{Leijuvat taulukot ja kuvat}
\label{luku/leijuosat-latex}

Käytännössä leijuvat osat ovat ympäristöjä, joiden sijoittelusta vastaa
Latex eikä kirjoittaja. Tosin kirjoittajakin voi hieman vaikuttaa
asiaan, ja tätä käsitellään luvussa \ref{luku/leijuosat-sijoittelu}.
Ympäristö \ymparisto{table} on tarkoitettu taulukoille, ja ympäristö
\ymparisto{figure} on kuville. Esimerkistä \ref{esim/leijuosat-perus}
ilmenee leijuvan taulukon toteuttamisen perusasiat. Muuntyyppiset
leijuvat osat tehdään samalla tavalla, ympäristö on vain eri.

\begin{esimerkki*}
  \komentoi{arraystretch}
  \komentoi{caption}
  \komentoi{hline}
  \komentoi{renewcommand}
  \ymparistoi{center}
  \ymparistoi{table}
  \ymparistoi{tabular}

\begin{koodilohko}
\begin{table}
  \begin{center}
    \renewcommand{\arraystretch}{1.3}
    \begin{tabular}{ll}
      Upeita & Lukuja \\
      \hline
      324 & 33 \\
      2   & 49 \\
      \hline
    \end{tabular}
  \end{center}
  \caption{Upea kuvateksti.}
\end{table}
\end{koodilohko}

  \caption{Leijuvan taulukon toteuttaminen \ymparisto{table}\-/
    ympäristön avulla. Varsinainen taulukko syntyy ympäristön
    \ymparisto{tabular} avulla}
  \label{esim/leijuosat-perus}
\end{esimerkki*}

Latex ei sinänsä ota kantaa siihen, mitä leijuva ympäristö sisältää, eli
\ymparisto{table}\-/ ympäristön sisällä ei tarvitse olla
taulukkoympäristöä (luku \ref{luku/taulukot}), eikä
\ymparisto{figure}\-/ ympäristön sisällä ole pakko olla kuvaa (luku
\ref{luku/grafiikka}). Valitut ympäristöt vaikuttavat kuitenkin
kuvatekstiin: \ymparisto{table}\-/ ympäristön kuvatekstiin tulee suomen
kieliasetuksilla sana ''\tablename'', ja \ymparisto{figure}\-/
ympäristössä se on ''\figurename''. Nimet tulevat komennoista
\komento{tablename} ja \komento{figurename}, jotka kirjoittaja voi
määrittää uudelleen. Se kannattaa tehdä dokumentin esittelyosassa
seuraavalla tavalla:

\komentoi{addto}
\komentoi{captionsfinnish}
\komentoi{figurename}
\komentoi{tablename}
\begin{koodilohkosis}
\addto{\captionsfinnish}{
  \renewcommand{\tablename}{Hökötys}
  \renewcommand{\figurename}{Himmeli}
}
\end{koodilohkosis}

\noindent
Edellä käytettiin \komento{addto}\-/ komentoa, joka kuuluu
kielipaketteihin \paketti{polyglossia} ja \paketti{babel}. Sillä
lisätään omia komentoja tietyn kielen asetuksiin, tässä suomen
kieliasetuksiin (\komento{captionsfinnish}). Kieliasetuksia käsitellään
tarkemmin luvussa \ref{luku/kieliasetukset}.

Kuvatekstit tehdään komennolla \komento{caption}, joka sijoitetaan
leijuvan ympäristön sisälle, kuten esimerkissä
\ref{esim/leijuosat-perus} on tehty. Komennolle annetaan yksi argumentti
eli haluttu kuvateksti. Siihen ei pidä kirjoittaa leijuvan osan tyyppiä
eikä numeroa (esim. ''\tablename~3''), sillä nämä Latex tekee
automaattisesti. \komento{caption}\-/ komennolle voi antaa myös
valinnaisen argumentin, jolla ilmaistaan kuvatekstistä lyhempi versio:

\komentoi{caption}
\begin{koodilohkosis}
\caption[Lyhyt kuvateksti.]{Varsinainen pitkä kuvateksti.}
\end{koodilohkosis}

\noindent
Varsinainen (pitkä) kuvateksti ladotaan aina leijuvan osan yhteyteen ja
lyhempää versiota käytetään mahdollisesti luettelossa (luku
\ref{luku/leijuosat-luettelot}) ja ristiviitteissä. Kuvatekstin eli
\komento{caption}\-/ komennon jälkeen leijuvaan ympäristöön lisätään
usein \komento{label}\-/ komennolla yksilöllinen tunniste
ristiviittausta varten. Komentoa käsitellään ristiviitteiden yhteydessä
luvussa \ref{luku/ristiviitteet}.

Leijuvista ympäristöistä on olemassa myös tähdelliset versiot,
\ymparisto{table*} ja \ymparisto{figure*}. Ympäristöjen eri versioilla
on merkitystä vain kaksipalstaisessa tilassa (luku \ref{luku/palstat}),
jossa normaaliversio (esim. \ymparisto{table}) ladotaan yhden palstan
sisään ja tähdellinen versio (esim. \ymparisto{table*}) sijoitetaan
palstojen ulkopuolelle, koko sivun tilaan.

\subsection{Muut leijuvat osat}
\label{luku/leijuosat-omat}

Omia, itse nimettyjä leijuvia osia tehdään \pakettictan{floatrow}\-/
paketin avustuksella. Se sisältää komennon on
\komento{DeclareNewFloatType}, jonka argumentit ovat seuraavanlaiset:

\komentoi{DeclareNewFloatType}
\begin{koodilohkosis}
\DeclareNewFloatType{tyyppi}{valitsimet}
\end{koodilohkosis}

\noindent
Komennon ensimmäinen argumentti \koodi{tyyppi} on leijuvan osan tyyppi
ja käytännössä myös ympäristön nimi. Jos tyypiksi antaa sanan
\koodi{esimerkki}, syntyy leijuvat ympäristöt \ymparistox{esimerkki} ja
\ymparistox{esimerkki*}. Argumentin \koodi{valitsimet} avulla voi
vaikuttaa tarkemmin leijuvan osan ominaisuuksiin. Mahdolliset valitsimet
on koottu taulukkoon \ref{tlk/declarenewfloat}. Eri valitsimet erotetaan
toisistaan pilkulla, eli komentoa käytetään esimerkin
\ref{esim/declarenewfloat} tavoin.

\leijutlk{
  \providecommand{\rivi}{}
  \renewcommand{\rivi}[2]{\koodi{\englanti{#1}} & #2 \\}
  \begin{tabular}{ll}
    \toprule
    \ots{Valitsin} & \ots{Merkitys} \\
    \midrule
    \rivi{name}{kuvateksteissä näkyvä tyyppi}
    \rivi{placement}{sijoitteluasetukset}
    \rivi{fileext}{väliaikaistiedoston pääte}
    \rivi{within}{ylemmäntasoinen laskuri}
    % \rivi{relatedcapstyle}{??}
    \bottomrule
  \end{tabular}
}{
  \caption{\komento{DeclareNewFloatType}\-/ komennon valitsimia, joilla
    vaikutetaan leijuvan osan ominaisuuksiin}
  \label{tlk/declarenewfloat}
}

\begin{esimerkki*}
\komentoi{DeclareNewFloatType}

\begin{koodilohko}
\DeclareNewFloatType{esimerkki}{
  name=Esimerkki,
  placement=tbp,
  fileext=loesim,
  within=chapter
}
\end{koodilohko}

\caption{\komento{DeclareNewFloatType}\-/ komennon käyttö}
\label{esim/declarenewfloat}
\end{esimerkki*}

Valitsimen \koodi{name} arvo näkyy kuvateksteissä leijuvan osan
tyyppinä. Valitsimella \koodi{placement} tehdään sijoitteluasetuksia,
joita käsitellään luvussa \ref{luku/leijuosat-sijoittelu}. Käytännössä
ne ovat tiettyjä kirjaimia, joilla vaikutetaan leijuvan osan
sijoitteluun. Oletusarvo on \koodi{tbp}.

Valitsin \koodi{file\-ext} määrittelee tiedoston päätteen. Nimittäin
leijuvien osien luetteloita tehtäessä Latex käyttää väliaikaistiedostoa,
ja sen päätteen voi valita tällä valitsimella. Luetteloissa on tapana
käyttää tiedostopäätettä, joka alkaa kirjaimilla \koodi{lo}
(\englanti{list of}). Latexin omissa leijuvissa osissa päätteet ovat
\koodi{lot} (\englanti{list of tables}) ja \koodi{lof} (\englanti{list
  of figures}). Oletuksena omille leijuville osille tulee päätteeksi
\koodi{lo}, jonka perässä on osan tyyppi.

Valitsimen \koodi{within} avulla vaikutetaan leijuvien osien numeroinnin
riippuvuuteen muista laskureista. Oletuksena leijuvan osan laskuri ei
ole riippuvainen muista laskureista eli osat numeroidaan dokumentin
alusta saakka yhdellä luvulla. Arvoksi voi antaa myös Latexin laskurin
nimen: mielekkäitä arvoja ovat esimerkiksi \laskuri{chapter} ja
\laskuri{section}. Vaikutus on se, että leijuvan osan laskuri on
riippuvainen näistä laskureista ja nollautuu aina, kun toisen laskurin
arvo kasvaa. Lisäksi kuvateksteissä näkyy kaksiosainen numerointi (4.1,
4.2 jne.), jossa ensimmäinen luku on ylemmäntasoisen laskurin arvo --
esimerkiksi kirjan pääluvun numero -- ja toinen on leijuvan osan
laskurin arvo. Tällaisia hierarkkisia laskureita käsitellään tarkemmin
luvussa \ref{luku/hierarkkiset-laskurit}. Leijuvan osan numeroinnin
ulkoasun muokkaamista käsitellään luvussa \ref{luku/leijuosat-ulkoasu}.

\subsection{Leijuvien osien luettelot}
\label{luku/leijuosat-luettelot}

Dokumentin leijuvista osista voi latoa automaattisesti luetteloita:
komennolla \komento{listoftables} syntyy taulukkoluettelo ja komennolla
\komento{listoffigures} kuvaluettelo. Omien leijuvien osien (luku
\ref{luku/leijuosat-omat}) luettelo tehdään \paketti{floatrow}\-/
paketin komennolla \komento{listof}, jolle annetaan argumentiksi
leijuvan osan tyyppi sekä luettelon otsikko.

\komentoi{listoftables}
\komentoi{listoffigures}
\komentoi{listof}
\begin{koodilohkosis}
\listoftables                  % taulukkoluettelo (table)
\listoffigures                 % kuvaluettelo (figure)
\listof{esimerkki}{Esimerkit}  % luettelo esimerkki-tyyppisistä osista
\end{koodilohkosis}

\noindent
Leijuvien osien luettelot muistuttavat sisällysluetteloita (luku
\ref{luku/sisällysluettelo}), eli niille tulee automaattisesti otsikko
kuten ''\listtablename'' tai ''\listfigurename'' ja automaattisesti
sisältö eli kaikki \komento{caption}\-/ komennolla ilmaistut (lyhyet)
kuvatekstit sekä tietysti sivunumerot.

\begin{esimerkki*}
  \komentoi{addvspace}
  \komentoi{contentslabel}
  \komentoi{contentspage}
  \komentoi{rmfamily}
  \komentoi{small}
  \komentoi{titlecontents}
  \komentoi{titlerule*}

\begin{koodilohko}
\titlecontents{table}                       % tyyppi: table
[8mm]                                       % vasen sisennys (mitta)
{\addvspace{3bp}\rmfamily\small}            % yläpuolinen koodi
{\contentslabel{8mm}}                       % numeroitu kohta
{}                                          % numeroimaton kohta
{~\small\titlerule*[3mm]{.}\contentspage}   % pisteviiva ja sivunumero
[]                                          % alapuolinen koodi
\end{koodilohko}
  \caption{Leijuvien osien luettelon ulkoasua muokataan
    \komento{titlecontents}\-/ komennolla, joka on peräisin
    \paketti{titletoc}\-/ paketista}
  \label{esim/leijuosat-titlecontents}
\end{esimerkki*}

Leijuvien osien luettelojen ulkoasua voi muokata samalla tavalla kuin
sisällysluetteloidenkin: käytetään pakettia \pakettictan{titletoc} ja
määritellään sen \komento{titlecontents}\-/ komennolla luettelokohtien
ulkoasu. Esimerkistä \ref{esim/leijuosat-titlecontents} selvinnee
perusajatus, ja lisätietoa on sisällysluetteloita käsittelevässä luvussa
\ref{luku/sisällysluettelo} sekä tietysti paketin ohjekirjassa.

Latexin omien leijuvien osien luetteloiden otsikot tulevat
kieliasetuksista ja komennoista \komento{listtablename} ja
\komento{listfigurename}, jotka kirjoittaja voi määritellä uudestaan.
Järkevää on tehdä se lähdedokumentin esittelyosassa ja käyttää
\komento{addto}\-/ komentoa, jolla asetukset lisätään tietyn kielen
asetuksiin.

\komentoi{addto}
\komentoi{captionsfinnish}
\komentoi{listfigurename}
\komentoi{listtablename}
\begin{koodilohkosis}
\addto{\captionsfinnish}{
  \renewcommand{\listtablename}{Taulukkoluettelo}
  \renewcommand{\listfigurename}{Kuvaluettelo}
}
\end{koodilohkosis}

\noindent
Luetteloihin saa lisättyä omia rivejä samoilla keinoilla kuin
sisällysluetteloihinkin eli komennolla \komento{addcontentsline}.
Komennon ensimmäinen argumentti on luettelon tyyppiä vastaavan tiedoston
pääte. Tyyppi \koodi{lot} (\englanti{list of tables}) on taulukoille ja
tyyppi \koodi{lof} (\englanti{list of figures}) on kuville. Omilla
leijuvilla osilla (luku \ref{luku/leijuosat-omat}) tiedoston pääte on
jotakin muuta ja valitaan osan määrittelyn yhteydessä.

\komentoi{addcontentsline}
\begin{koodilohkosis}
\addcontentsline{lot}{table}{Ylimääräinen taulukko}
\addcontentsline{lof}{figure}{Ylimääräinen kuva}
\end{koodilohkosis}

\noindent
Komennon toinen argumentti tarkoittaa luettelomerkinnän tasoa (vrt.
otsikkotasot). Leijuvien osien luetteloissa on vain yksi taso, ja siihen
kirjoitetaan kyseisen leijuvan osan tyyppi eli ympäristön nimi,
esimerkiksi \ymparisto{table} tai \ymparisto{figure}. Komennon kolmas
argumentti on luetteloon lisättävä merkintä.

\subsection{Sijoittelu sivulle}
\label{luku/leijuosat-sijoittelu}

Latex sijoittaa leijuvat osat yleensä sivun yläosaan, joskus alaosaan
tai omalle sivulleen, jossa ei ole leipätekstiä lainkaan. Yksittäisen
ympäristön sijoitteluun voi vaikuttaa valinnaisen argumentin avulla:

\ymparistoi{figure}
\begin{koodilohkosis}
\begin{figure}[sijainti]
  ...
\end{figure}
\end{koodilohkosis}

\leijutlk{
  \providecommand{\rivi}{}
  \renewcommand{\rivi}[2]{\koodi{#1} & #2 \\}
  \begin{tabular}{cl}
    \toprule
    \ots{Valitsin} & \ots{Merkitys} \\
    \midrule
    \rivi{t}{sivun yläosaan (\englanti{top})}
    \rivi{b}{sivun alaosaan (\englanti{bottom})}
    \rivi{p}{leijuvien osien sivulle (\englanti{page})}
    \rivi{h}{mieluiten tähän kohtaan (\englanti{here})}
    \rivi{H}{ehdottomasti tähän kohtaan (\englanti{here},
    \paketti{floatrow})}
    \rivi{!}{ei huomioida tiettyjä sijoittelurajoituksia}
    \bottomrule
  \end{tabular}
}{
  \caption{Leijuvien osien sijoitteluun voi vaikuttaa ympäristön
    valinnaisen argumentin avulla. Valitsin \koodi{H} on
    \paketti{floatrow}\-/ paketin ominaisuus}
  \label{tlk/leijuosat-sijoittelukirjaimet}
}

\noindent
Valinnainen argumentti \koodi{sijainti} on tietty merkki tai merkkien
yhdistelmä, jolla vaikutetaan leijuvan osan sijoitteluun. Taulukossa
\ref{tlk/leijuosat-sijoittelukirjaimet} ovat mahdolliset vaihtoehdot.
Oletusarvo on \koodi{tbp}, eli sen mukaan leijuva osa pyritään
sijoittamaan ensisijaisesti sivun yläosaan (\koodi{t}) mutta
mahdollisesti myös alaosaan (\koodi{b}) tai omalle sivulleen (\koodi{p})
mahdollisesti muiden leijuvien osien kanssa.

Vaihtoehto \koodi{h} pyrkii latomaan leijuvan osan siihen kohtaan kuin
se on lähdetiedostossa. Se on kuitenkin vain ehdotus eikä useinkaan
toteudu: Latex lisää automaattisesti aina \koodi{h}\=/ valitsimen perään
\koodi{t}:n eli sivun yläosaan sijoittamisen. Latex käyttää sitä, jos se
saa sillä tavoin omasta mielestään paremman lopputuloksen. Lataamalla
paketin \paketti{floatrow} voi käyttää myös myös valitsinta \koodi{H}.
Se tarkoittaa ehdotonta vaatimusta, että juuri tähän kohtaan leijuva osa
pitää sijoittaa. Tätä valitsinta ei voi yhdistää muiden valitsimien
kanssa.

Valitsinta \koodi{!} käytetään yhdessä edellä mainittujen valitsimien
\koodi{tbp} kanssa. Se poistaa tietyt sijoitteluun vaikuttavat säännöt
tai rajoitukset. Niitä käsitellään seuraavaksi.

Leijuvien osien automaattisessa sijoittelussa vaikuttaa kymmenen
erilaista parametria, joihin kirjoittaja voi vaikuttaa: kolme laskuria,
kolme mittaa ja neljä komentoa. Joillekin asetuksille on olemassa
rinnakkainen versio, jota käytetään kahden palstan tilassa (luku
\ref{luku/palstat}). Ne koskevat leijuvien ympäristöjen tähdellisiä
versiota, kuten \ymparisto{table*} ja \ymparisto{figure*}, eli
tilannetta, jossa leijuva osa ei ole palstojen sisällä, vaan koko sivun
leveys on käytettävissä.

Seuraavat kolme laskuria vaikuttavat sivulle ladottavien leijuvien osien
enimmäismäärään. Ne koskevat tilannetta, jossa sivulla on myös
tavallista leipätekstiä tai muuta sisältöä. Laskureille asetetaan uusi
arvo komennolla \komento{setcounter} (luku \ref{luku/laskurit}). Näiden
laskurien vaikutuksen voi poistaa kokonaan käyttämällä leijuvan
ympäristön valitsinta \koodi{!} (taulukko
\ref{tlk/leijuosat-sijoittelukirjaimet}).

\begin{maaritelma}{\laskuri{#1}}
\item [topnumber] Sivun yläosaan sijoitettavien leijuvien osien
  enimmäismäärä. Oletusarvo on~2. Kahden palstan tilassa käytetään
  laskuria \laskuri{dbltopnumber}.
\item [bottomnumber] Sivun alaosaan sijoitettavien leijuvien osien
  enimmäismäärä. Oletusarvo on~1. Kahden palstan tilassa leijuvia osia
  ei sijoiteta sivun alaosaan täysilevyisenä.
\item [totalnumber] Samalla sivulla olevien leijuvien osien
  enimmäismäärä, kun sivulla on myös tavallista leipätekstiä. Oletusarvo
  on~3.
\end{maaritelma}

\noindent
Seuraavat kolme mittaa vaikuttavat leijuvien osien pystysuuntaisiin
väleihin. Mitat asetetaan komennolla \komento{setlength} (luku
\ref{luku/mitat}).

\begin{maaritelma}{\mitta{#1}}
\item [floatsep] Samalla sivulla kahden peräkkäisen leijuvan osan
  pystysuuntainen väli. Oletusarvo on venyvä mitta \koodi{12pt plus 2pt
    minus 2pt}. Kahden palstan tilassa käytetään mittaa
  \mitta{dblfloatsep}.
\item [textfloatsep] Leijuvan osan ja leipätekstin välinen
  pystysuuntainen väli. Oletusarvo on venyvä mitta \koodi{20pt plus 2pt
    minus 4pt}. Kahden palstan tilassa käytetään mittaa
  \mitta{dbltextfloatsep}.
\item [intextsep] Pystysuuntainen väli leipätekstin ja leijuvan osan
  välissä silloin, kun leijuva osa sijoitetaan osaksi tekstivirtaa. Tämä
  koskee sijoitteluasetuksia \koodi{h} ja \koodi{H} (taulukko
  \ref{tlk/leijuosat-sijoittelukirjaimet}). Oletusarvo vaihtelee
  dokumenttiluokan fonttikokoasetusten (luku
  \ref{luku/perusdokumenttiluokat-asetukset}) perusteella. Jos
  dokumenttiluokan valitsin on \koodi{10pt} (oletus) tai \koodi{11pt},
  oletusmitta on \koodi{12pt plus 2pt minus 2pt}. Dokumenttiluokan
  valitsimella \koodi{12pt} oletus on \koodi{14pt plus 4pt minus 4pt}.
\end{maaritelma}

\noindent
Seuraavat neljä komentoa sisältävät pelkän desimaaliluvun 0\==1. Luku
kertoo leijuvien osien tai leipätekstin viemän suhteellisen tilan
samalla sivulla. Komennot määritellään uudelleen komennolla
\komento{renewcommand} (luku \ref{luku/komennot-määrittely}). Näiden
vaikutuksen voi poistaa kokonaan leijuvan ympäristön valitsimella
\koodi{!} (taulukko \ref{tlk/leijuosat-sijoittelukirjaimet}).

\begin{maaritelma}{\komento{#1}}
\item [topfraction] Sivun yläosaan sijoitettavien leijuvien osien viemä
  enimmäistila suhteessa sivun tekstialueen korkeuteen. Oletusarvo
  on~\koodi{0.7}, mikä tarkoittaa, että leijuvat osat voivat viedä
  korkeintaan 0,7\=/ kertaisesti (70\,\%) sivulla olevan pystysuuntaisen
  tilan. Kahden palstan tilassa käytetään komentoa
  \komento{dbltopfraction}.
\item [bottomfraction] Sivun alaosaan sijoitettavien leijuvien osien
  viemä enimmäistila suhteessa sivun tekstialueen korkeuteen. Oletusarvo
  on~\koodi{0.3} (30\,\%). Kahden palstan tilassa leijuvia osia ei
  koskaan sijoiteta sivun alaosaan.
\item [textfraction] Leipätekstin vähimmäistila sivulla, kun samalla
  sivulla on myös leijuvia osia. Oletusarvo on \koodi{0.2} (20\,\%).
\item [floatpagefraction] Omalle sivulleen sijoitettavien leijuvien
  osien vähimmäistila sivun tekstialueen korkeudesta. Toisin sanoen
  leijuvien osien täytyy viedä vähintään tämän verran tilaa sivulta,
  jotta ne voidaan sijoittaa omalle sivulleen, jossa ei ole leipätekstiä
  lainkaan. Oletusarvo on~\koodi{0.5} (50\,\%). Kahden palstan tilassa
  käytetään komentoa \komento{dblfloatpagefraction}.
\end{maaritelma}

\noindent
Sivunvaihtokomennot \komento{clearpage} ja \komento{cleardoublepage}
(luku \ref{luku/sivunvaihdot}) pakottavat kaikki aiemmin
lähdetiedostossa olleet leijuvat osat ladottavaksi. Osanvaihtokomento
\komento{part} ja otsikkokomento \komento{chapter} (luku
\ref{luku/otsikot}) tekevät saman automaattisesti joissakin
dokumenttiluokissa.

Paketin \pakettictan{placeins} avulla voi tehdä omiakin rajakohtia,
jonka yli leijuvat osat eivät voi siirtyä. Paketin komennolla
\komento{FloatBarrier} tehdään raja kyseiseen kohtaan lähdedokumentissa.
Jos paketin lataamisessa antaa argumentin \koodi{section}, asetetaan
tällainen raja automaattisesti \komento{section}\-/ otsikoille.

\komentoi{usepackage}
\pakettii{placeins}
\begin{koodilohkosis}
\usepackage[section]{placeins}
\end{koodilohkosis}

\subsection{Ulkoasu}
\label{luku/leijuosat-ulkoasu}

Leijuvan osan sisäisen asettelun, ulkoasun ja muun typografian
hallintaan ja muokkaamiseen tarvitaan yleensä paketteja
\paketti{floatrow} ja \pakettictan{caption}. Jälkimmäisen avulla
muokataan kuvatekstin fonttia ja rivitystä, ja sitä käsitellään
myöhemmin tässä alaluvussa.

Paketti \paketti{floatrow} tarjoaa runsaasti mahdollisuuksia leijuvien
osien sijoitteluun: osat voi kehystää laatikolla, kuvatekstin voi
sijoittaa ylös, alas tai sivulle, useita leijuvia osia voi sijoittaa
vierekkäin ym. Tässä oppaassa käsitellään ainoastaan peruskäyttöä.

Jotta \paketti{floatrow}\-/ paketin ominaisuudet saisi käyttöönsä,
täytyy leijuvat osat toteuttaa sen oman \komento{floatbox}\-/ komennon
avulla. Komento sijoitetaan leijuvan ympäristön sisään. Sen
argumenteiksi annetaan ainakin leijuvan osan tyyppi, kuvatekstikomento
ja varsinainen leijuva sisältö. Seuraavassa esimerkissä käytetään
\ymparisto{table}\-/ ympäristöä, mutta muut tyypit toimivat samalla
tavalla.

\komentoi{floatbox}
\ymparistoi{table}
\begin{koodilohkosis}
\begin{table}
  \floatbox{table}[leveys]{kuvatekstit}{sisältö}
\end{table}
\end{koodilohkosis}

\begin{esimerkki*}
  \komentoi{caption}
  \komentoi{floatbox}
  \komentoi{label}
  \mittai{FBwidth}
  \ymparistoi{table}
  \ymparistoi{tabular}

\begin{koodilohko}
\begin{table}
  \floatbox{table}[\FBwidth]{
    \caption{Upea taulukko}    % kuvateksti
    \label{tlk/upea-taulukko}  % tunniste ristiviittausta varten
  }{
    \begin{tabular}{ll}
      jotain & soluja \\
      esimerkin & vuoksi \\
    \end{tabular}
  }
\end{table}
\end{koodilohko}
  \caption{\komento{floatbox}\-/ komennon peruskäyttö}
  \label{esim/floatbox-perus}
\end{esimerkki*}

\noindent
Valinnaisen \koodi{leveys}\-/ argumentin voi jättää poiskin, jolloin
leijuva osa varaa itselleen sivun (tai palstan) leveyden verran tilaa.
Argumentiksi on joskus hyödyllistä antaa mitta \mitta{FBwidth}, joka on
leijuvan osan sisällön levyinen. Käyttämällä tätä mittaa kuvatekstit
rivitetään sisällön levyiseksi. Esimerkki \ref{esim/floatbox-perus}
selventää, miten \komento{floatbox}\-/ komento toimii käytännössä.

\begin{esimerkki*}
  \komentoi{caption}
  \komentoi{floatbox}
  \mittai{FBwidth}
  \ymparistoi{floatrow}
  \ymparistoi{table}

\begin{koodilohko}
\begin{table}
  \begin{floatrow}[2]  % kaksi rinnakkain
    \floatbox{table}[\FBwidth]{\caption{Vasen}}{…}
    \floatbox{table}[\FBwidth]{\caption{Oikea}}{…}
  \end{floatrow}
\end{table}
\end{koodilohko}
  \caption{Rinnakkaisten leijuvien osien toteutus
    \ymparisto{floatrow}\-/ ympäristön ja \komento{floatbox}\-/ komennon
    avulla. Kohdassa ''\koodi{\ldots}'' olisi varsinainen sisältö eli
    taulukon toteutus}
  \label{esim/floatrow-floatbox}
\end{esimerkki*}

Yksi leijuva ympäristö voi sisältää rinnakkain useampia taulukoita,
kuvia tms., joilla jokaisella on oma kuvatekstinsä. Se tehdään
ympäristöllä \ymparisto{floatrow}, jonka valinnainen argumentti
ilmaisee, kuinka monta kuvaa tms. ladotaan rinnakkain. Oletusarvo on~2.
Ympäristön sisällä täytyy olla yhtä monta \komento{floatbox}\-/
komentoa. Esimerkistä \ref{esim/floatrow-floatbox} selviää perusajatus
kahden taulukon latomiseksi rinnakkain.

Kun leijuvat osat on toteutettu \komento{floatbox}\-/ komennon avulla,
voi niiden ulkoasuun vaikuttaa komennolla \komento{floatsetup}. Sille
annetaan ainakin yksi argumentti, joka voi sisältää useita pilkulla
toisistaan erotettuja valitsimia ja niiden arvoja.

\komentoi{floatsetup}
\begin{koodilohkosis}
\floatsetup[tyyppi]{valitsimet}
\end{koodilohkosis}

\leijutlk{
  \providecommand{\rivi}{}
  \renewcommand{\rivi}[2]{\koodi{#1} & #2 \\}
  \begin{tabularx}{\textwidth}{lL}
    \toprule
    \ots{Valitsin} & \ots{Merkitys ja vaihtoehtoja} \\
    \midrule

    \rivi{style}{tyyli: \koodi{plain}, \koodi{ruled}, \koodi{boxed} ym.}

    \rivi{capposition}{kuvatekstin sijainti: \koodi{top}, \koodi{bottom}
      ym.}

    \rivi{captionskip}{kuvatekstin etäisyys sisällöstä (mitta)}

    \rivi{font}{fonttiasetukset: \koodi{rm}, \koodi{sf}, \koodi{tt},
      \koodi{md}, \koodi{bf}, \koodi{up}, \koodi{it}, \koodi{sl},
      \koodi{sc}, \koodi{scriptsize}, \koodi{footnotesize},
      \koodi{small}, \koodi{normalsize}, \koodi{large}, \koodi{Large}}

    \rivi{margins}{leijuvan osan sijainti: \koodi{centering},
      \koodi{raggedright}, \koodi{raggedleft} ym.}

    \rivi{justification}{sisällön tasaus: \koodi{justified},
      \koodi{centering}, \koodi{raggedright}, \koodi{raggedleft},
      \koodi{RaggedRight} (\paketti{ragged2e}), \koodi{RaggedLeft}
      (\paketti{ragged2e})}

    \bottomrule
  \end{tabularx}
}{
  \caption{Muutama \komento{floatsetup}\-/ komennon valitsin, joilla
    vaikutetaan leijuvien osien ulkoasuun. Osa asetuksista tarvitsee
    paketin \paketti{ragged2e}}
  \label{tlk/floatsetup-valitsimia}
}

\noindent
Valinnainen argumentti \koodi{tyyppi} on leijuvan osan tyyppi,
esimerkiksi \ymparisto{table} tai \ymparisto{figure}. Jos se on annettu,
asetukset vaikuttavat vain kyseiseen tyyppiin; muuten asetukset koskevat
kaikentyyppisiä leijuvia osia. Asetukset tehdään erilaisten valitsimien
avulla, joista muutama on koottu taulukkoon
\ref{tlk/floatsetup-valitsimia}.

Valitsin \koodi{font} asettaa leijuvan ympäristön fontin mutta ei
vaikuta kuvatekstiin. Fonttiasetukset ovat käytännössä avainsanoja,
jotka vastaavat lähes samannimisiä fonttikomentoja. Niitä käsitellään
luvussa \ref{luku/fontit-korkea}. Omiakin fonttien avainsanoja voi luoda
komennolla \komento{DeclareFloatFont}. Valitsimen \koodi{justification}
arvoksi annetaan myös tietty sana, joka vastaava samannimisiä palstan
tasauskomentoja (luku \ref{luku/kappaleen-tasaus}). Käytännössä leijuvan
osan asetukset voisivat näyttää vaikka seuraavanlaiselta:

\komentoi{floatsetup}
\begin{koodilohkosis}
\floatsetup{ style=plain, capposition=bottom, font={sf, small},
  justification=centering, captionskip=2ex }
\floatsetup[figure]{ style=boxed }
\end{koodilohkosis}

\noindent
Kuvatekstien fonttia ja rivittämistä hallitaan \paketti{caption}\-/
paketin avulla. Tärkein komento on \komento{captionsetup}, joka yleensä
sijoitetaan dokumentin esittelyosaan tai muuten alkuun, koska silloin se
vaikuttaa koko dokumentissa. Komennon voi sijoittaa myös yksittäisen
leijuvan ympäristön sisään, jolloin se vaikuttaa vain kyseisessä
ympäristössä.

\komentoi{captionsetup}
\begin{koodilohkosis}
\captionsetup[tyyppi]{valitsimet}
\end{koodilohkosis}

\leijutlk{
  \providecommand{\rivi}{}
  \renewcommand{\rivi}[2]{\koodi{#1} & #2 \\}
  \begin{tabularx}{\textwidth}{lL}
    \toprule
    \ots{Valitsin} & \ots{Merkitys ja vaihtoehtoja} \\
    \midrule

    \rivi{font}{kuvatekstin fontti: \koodi{normalfont}, \koodi{rm},
      \koodi{sf}, \koodi{tt}, \koodi{md}, \koodi{bf}, \koodi{up},
      \koodi{it}, \koodi{sl}, \koodi{sc}, \koodi{scriptsize},
      \koodi{footnotesize}, \koodi{small}, \koodi{normalsize},
      \koodi{large}, \koodi{Large}}

    \rivi{labelfont}{leijuvan osan tyypin ja numeroinnin fontti}

    \rivi{textfont}{varsinaisen kuvatekstin fontti}

    \rivi{format}{kappaleen tyyppi, tavallinen \koodi{plain} vai
      riippuva \koodi{hang}}

    \rivi{indentation}{sisennysmitta kuvatekstin toisesta rivistä
      alkaen}

    \rivi{labelformat}{leijuvan osan tyypin muoto: \koodi{default},
      \koodi{empty}, \koodi{simple}, \koodi{brace}, \koodi{parens}}

    \rivi{labelsep}{leijuvan osan tyypin erotinmerkki: \koodi{none},
      \koodi{colon}, \koodi{period}, \koodi{space}, \koodi{quad},
      \koodi{newline}, \koodi{endash}}

    \rivi{textformat}{kuvatekstin lopetusmerkki: \koodi{simple},
      \koodi{period}}

    \rivi{justification}{kuvatekstin tasaus: \koodi{justified},
      \koodi{centering}, \koodi{centerlast}, \koodi{centerfirst},
      \koodi{raggedright}, \koodi{RaggedRight} (\paketti{ragged2e}),
      \koodi{raggedleft}}

    \rivi{singlelinecheck}{yksirivisten kuvatekstien poikkeuksellinen
      tasaus: \koodi{true} (oletus), \koodi{false}}

    \rivi{margin}{kuvatekstin marginaalit: yksi mitta tai kaksi pilkulla
      erotettua mittaa (vasen ja oikea marginaali)}

    \rivi{width}{kuvatekstikappaleen leveys (mitta)}

    \bottomrule
  \end{tabularx}
}{
  \caption{Tärkeimpiä \komento{captionsetup}\-/ komennon valitsimia}
  \label{tlk/captionsetup-valitsimia}
}

\noindent
Komennon valinnainen \koodi{tyyppi}\-/ argumentti on leijuvan osan
tyyppi, esimerkiksi \ymparisto{table} tai \ymparisto{figure}, johon
halutaan vaikuttaa. Jos sen jättää pois, asetukset vaikuttavat kaikkien
leijuvien osien kuvateksteihin. Pakollinen argumentti on
\koodi{valitsimet}, johon kirjoitetaan valitsimia ja niiden arvoja.
Tärkeimpiä valitsimia on koottu taulukkoon
\ref{tlk/captionsetup-valitsimia}. Käytännössä komento voisi näyttää
esimerkiksi seuraavanlaiselta:

\komentoi{captionsetup}
\begin{koodilohkosis}
\captionsetup{ font={small, sf}, labelfont={bf}, textfont={},
  textformat=period, margin={.5em,0em}, justification=raggedright,
  singlelinecheck=false }
\end{koodilohkosis}

\noindent
Oletuksena Latex käsittelee yksiriviset kuvatekstit poikkeuksellisella
tavalla eli keskittää ne. Sen vuoksi tasausvalitsin
\koodi{justification} ei vaikuta yksirivisiin kuvateksteihin. Tämän
poikkeuksen saa pois päältä käyttämällä asetusta
\koodi{singlelinecheck=\katk false}, jolloin sama
\koodi{justification} pätee yhtä lailla niin yksi- kuin monirivisiinkin
kuvateksteihin.

\begin{esimerkki*}
  \ymparistoi{table}
  \komentoi{caption}
  \komentoi{ContinuedFloat}

\begin{koodilohko}
\begin{table}                 % kokonaisuuden 1. leijuva ympäristö
  \caption{Hieno taulukko}
  ...
\end{table}
...
\begin{table}                 % kokonaisuuden 2. leijuva ympäristö
  \ContinuedFloat             % sama numerointi kuin edelliselle
  \caption{Hieno taulukko (jatkuu)}
  ...
\end{table}
\end{koodilohko}

  \caption{Usealle leijuvalle osalle saa saman numeron käyttämällä
    jälkimmäisissä ympäristöissä komentoa \komento{ContinuedFloat}}
  \label{esim/continuedfloat}
\end{esimerkki*}

Joskus halutaan, että useampi taulukko, kuva tai muu leijuva osa
muodostaa kokonaisuuden, jolla on sama numero, esimerkiksi
''\tablename~3''. Ajatuksena on, että sama sisältö on jakautunut
useampaan osaan. Se onnistuu sijoittamalla \paketti{caption}\-/ paketin
komento \komento{ContinuedFloat} ensimmäisen jälkeisiin leijuviin
ympäristöihin. Komento pitäisi sijoittaa heti ympäristön alkuun.
Esimerkki \ref{esim/continuedfloat} havainnollistaa sen käyttöä.

Kuvatekstissä oleva leijuvan osan numero tulee laskurista, ja
erityyppisellä osilla on oma laskurinsa. Laskurin nimi on sama kuin
vastaavan leijuvan ympäristön nimi, eli esimerkiksi \ymparisto{table}\-/
ympäristöt numeroidaan laskurin \laskuri{table} avulla.

Käytännössä laskurien arvo ladotaan \komentox{the}\-/ alkuisella
komennolla, jonka perässä on laskurin nimi, esimerkiksi
\komento{thetable}. Oletuksena nämä komennot latovat laskurin
arabialaisilla numeroilla. Jos laskuri on riippuvainen teoksen
pääluvuista (\komento{chapter}), ladotaan ensin pääluvun numero,
erotinpiste ja leijuvan osan numero: 2.1, 2.2 jne.

Kirjoittaja voi määritellä laskurien latomiskomennon haluamallaan
tavalla. Seuraavassa esimerkissä leijuville kuville (\ymparisto{figure})
määritellään numerointitapa, jossa on pääluvun numero, ajatusviiva (\==)
ja leijuvan kuvan numero:

\komentoi{arabic}
\komentoi{renewcommand}
\komentoi{thechapter}
\komentoi{thefigure}
\laskurii{figure}
\begin{koodilohkosis}
\renewcommand{\thefigure}{\thechapter--\arabic{figure}}
\end{koodilohkosis}

\subsection{Upottaminen tekstipalstaan}

Pienikokoisia leijuvia osia on jotenkuten mahdollista upottaa
tekstipalstan vasempaan tai oikeaan reunaan siten, että teksti kiertää
kyseisen kuvan tai taulukon (kuva \ref{kuva/wrapfig-esimerkki}). Tähän
tarvitaan \pakettictan{wrapfig}\-/ pakettia ja sen ympäristöjä
\ymparisto{wrapfigure} (kuvat) tai \ymparisto{wraptable} (taulukot).
Niitä käytetään seuraavalla tavalla:

\ymparistoi{wrapfigure}
\komentoi{caption}
\begin{koodilohkosis}
\begin{wrapfigure}{sijainti}{leveys}
  % kuvan sisältökoodi
  \caption{Kuvateksti.}
\end{wrapfigure}
\end{koodilohkosis}

\begin{wrapfigure}{R}{5em}
  \begin{tikzpicture}[x=.5em, y=.5em,
    tyyli/.style={draw=black, fill=yellow, line width=.6bp}]
    \draw [tyyli] (0,0) ellipse [x radius=4, y radius=1.5, rotate=45];
    \draw [tyyli] (-2,2) circle [radius=1];
    \draw [tyyli] (2,-2) circle [radius=1];
  \end{tikzpicture}
  \caption{Tekstipalstaan upotettu kuva}
  \label{kuva/wrapfig-esimerkki}
\end{wrapfigure}

\noindent
Ympäristön pakollinen argumentti \koodi{sijainti} määrittää kuvan
sijoittelun, ja se on yhden kirjaimen koodi: \koodi{L} (\englanti{left},
vasen), \koodi{R} (\englanti{right}, oikea), \koodi{I}
(\englanti{inside}, sisäreuna) tai \koodi{O} (\englanti{outside},
ulkoreuna). Jos kuvaa ei halua leijuvaksi eli Latexin sijoittamaksi vaan
se halutaan sijoittaa täsmälleen ympäristön kohdalle, käytetään
vastaavia kirjainkoodeja mutta pieniä kirjaimia: \koodi{lrio}.

Toinen pakollinen ympäristölle annettava argumentti \koodi{leveys}
määrittää upotetun osan leveyden. Se on mitta, jonka levyinen tila
palstasta varataan ympäristön sisällölle ja mahdolliselle kuvatekstille.
Ympäristölle voi antaa myös valinnaisia argumentteja, jolloin
argumenttien rakenne on seuraavanlainen:

\ymparistoi{wrapfigure}
\begin{koodilohkosis}
\begin{wrapfigure}[rivit]{sijainti}[siirto]{leveys}
  ...
\end{wrapfigure}
\end{koodilohkosis}

\noindent
Ensimmäinen valinnainen argumentti \koodi{rivit} on rivimäärä: näin
monen tekstirivin korkuinen tila varataan upotettavalle kuvalle. Latex
laskee tämän automaattisesti, joten argumenttia ei peruskäytössä
tarvita. Toinen valinnainen argumentti \koodi{siirto} on mitta, jolla
ilmaistaan, kuinka paljon kuvaa siirretään sivun marginaalin suuntaan.
Tässä argumentissa voi käyttää apuna mittaa \mitta{width}, joka on
upotettavan kuvan leveys. Esimerkiksi \koodi{siirto}\-/ argumenttiin
voisi kirjoittaa \koodi{0.5\keno width}, jolloin kuva siirtyy puoliksi
marginaalin puolelle.

Tekstipalstaan upotetun kuvan ylä- ja alapuolelle jätetään mitan
\mitta{intextsep} suuruinen väli. Vasemmalle tai oikealle puolelle
jätetään mitan \mitta{columnsep} suuruinen väli. Samaa mittaa käytetään
palstojen välissä (luku \ref{luku/palstat}).

Valitettavasti Latexin tekniikka ei sovellu kovin hyvin tekstipalstaan
upotettavien elementtien käsittelyyn, ja siksi \paketti{wrapfig}\-/
paketinkaan avulla ei saa toteutettua mitä tahansa upotuksia.
Esimerkiksi luetelmien (luku \ref{luku/luetelmat}) tai muiden
erikoisempien ympäristöjen rinnalle ei voi upottaa kuvaa. Lähinnä vain
tavallinen teksti toimii upotuksen rinnalla. Paketin ohjekirjassa
kerrotaan rajoituksista tarkemmin.

\section{Ristiviittaukset}
\label{luku/ristiviitteet}

Ristiviittaukset tarkoittavat dokumentin sisäisiä viittauksia eli
viittauksia toisiin kohtiin samassa dokumentissa. Lukijalle ne ilmenevät
ilmauksina kuten ''luvussa 4.2'' tai ''kuvassa~5''. Kirjoittajan ei
kuitenkaan kannata naputella luvun numeroa (esim. 4.2) käsin
lähdedokumenttiin, koska lukujen järjestys ja numerointi voi muuttua
kirjoittamisen edetessä. Joutuisi korjaamaan muuttuneet numerot
mahdollisesti useitakin kertoja ennen kuin työ on valmis.

Latex numeroi dokumentin otsikot (luvut) ja leijuvat osat
automaattisesti, eli se tietää, minkä numeron mikäkin dokumentin osa
saa. Niinpä Latex osaa myös -- kirjoittajan pienellä avustuksella --
ylläpitää ajantasaisia ristiviittauksia dokumentin eri osiin. Teknisesti
tämä toteutetaan siten, että otsikoille ja leijuville osille annetaan
yksilöllinen tunniste ja viittaamisessa käytetään samoja tunnisteita.
Tunnisteet annetaan \komento{label}\-/ komennolla, joka sijoitetaan heti
otsikkokomennon (\komento{section}, \komento{subsection} ym.) tai
leijuvan osan kuvatekstikomennon (\komento{caption}) jälkeen:

\komentoi{section}
\komentoi{label}
\begin{koodilohkosis}
\section{Jokin otsikko}
\label{tunniste}
\end{koodilohkosis}

\noindent
\komento{label}\-/ komennon argumentti \koodi{tunniste} voi olla
suunnilleen mitä tahansa tekstiä. Sen täytyy olla yksilöllinen, eli sama
tunniste ei saa olla käytössä missään toisessa \komento{label}\-/
komennossa. Jos sama tunniste sattuu olemaan useassa paikassa, vain
viimeinen jää käytännössä voimaan.

Järkevää on kirjoittaa tunnisteeseen sana tai sanoja, jotka kertovat
jotakin kyseisen osan sisällöstä. Hyödyllistä voi olla myös liittää
mukaan tieto, onko kyse luvusta, taulukosta, kuvasta vai muusta
leijuvasta osasta, koska erityyppiset osat voivat käsitellä samaa
aihetta. Seuraavassa on esimerkkejä \komento{label}\-/ komennoista:

\komentoi{label}
\begin{koodilohkosis}
\label{luku/fonttien-valinta}    % otsikkokomennon jälkeen
\label{tlk/fonttikomentoja}      % taulukon \caption-komennon jälkeen
\label{kuva/kirjainleikkauksia}  % kuvan \caption-komennon jälkeen
\end{koodilohkosis}

\noindent
Ristiviittauksissa vähimmällä vaivalla selviää, kun huomioi kaksi asiaa:
1)~kirjoittaa \komento{label}\-/ komentoja vain niihin kohtiin, joihin
täytyy viitata muualta, 2)~yrittää valita tunnisteet siten, ettei niitä
tarvitse muuttaa enää sen jälkeen, kun ne on kerran valinnut.

Tavallisimmat ristiviittaukset tehdään \komento{ref}\-/ komennolla,
jonka argumentiksi annetaan sama tunniste kuin jossakin dokumentin
\komento{label}\-/ komennossa. Latex latoo \komento{ref}\-/ komentojen
kohdalle kyseisen luvun tai leijuvan osan numeron. Toinen tarpeellinen
ristiviittauskomento on \komento{pageref}, joka kääntämisvaiheessa
korvautuu kyseisen kohteen sivunumerolla. Näitä voi hyödyntää
lähdedokumentissa esimerkiksi seuraavasti:

\komentoi{ref}
\komentoi{pageref}
\begin{koodilohkosis}
Katso taulukko \ref{tlk/komennot-kirjainperhe} sivulla
\pageref{tlk/komennot-kirjainperhe}.
\end{koodilohkosis}

\begin{tulossis}
  Katso taulukko \ref{tlk/komennot-kirjainperhe} sivulla
  \pageref{tlk/komennot-kirjainperhe}.
\end{tulossis}

\noindent
Jos ladattuna on paketti \paketti{hyperref} (luku
\ref{luku/pdf-asetukset}), on ristiviittauksiin käytettävissä myös
komento \komento{nameref}. Sillekin annetaan argumentiksi jokin
\komento{label}\-/ komennolla määritetty tunniste, ja kääntämisvaiheessa
komennon tilalle ladotaan kyseisen kohteen teksti eli otsikko tai
kuvateksti. Ristiviittauksissa ladotaan otsikon tai kuvatekstin lyhempi
versio, jos sellainen on annettu. Otsikkokomentojen ja
kuvatekstikomennon argumentteja käsitellään luvuissa \ref{luku/otsikot}
(\nameref{luku/otsikot}) ja \ref{luku/leijuosat-latex}
(\nameref{luku/leijuosat-latex}).

Ristiviittaukset vaativat, että Latex\-/ lähdedokumentti käännetään
kahdesti. Ensimmäisellä kääntämiskerralla \komento{label}\-/ komennolla
mainittujen kohteiden tiedot kirjoitetaan muistiin väliaikaistiedostoon.
Toisella kääntämiskerralla hyödynnetään väliaikaistiedostoa ja
komentojen \komento{ref}, \komento{pageref} ja \komento{nameref} tilalle
ladotaan viittauskohteen oikeat tiedot.

\section{Alaviitteet}
\label{luku/alaviitteet}

Alaviitteet ovat sivun alareunassa olevia numeroituja tai muulla tavalla
merkittyjä huomautuksia. Niihin viitataan sanojen tai virkkeiden perässä
olevalla yläindeksinumerolla, \=/kirjaimella tai
\=/symbolilla.\footnote{Katso myös ylä- ja alaindeksejä käsittelevä luku
  \ref{luku/ylä-alaindeksit}. Siellä mainitun \paketti{realscripts}\-/
  paketin lataaminen on suositeltavaa, kun käyttää alaviitteitä.}
Alaviitteiden tarkoituksena on lisätä tekstiin lähes huomaamaton
lisätieto, joka ei häiritse lukemista. Se sopii asioille, joilla ei ole
suurta merkitystä useimmille lukijoille tai jotka muuten sopivat
huonosti leipätekstiin. Varsin usein myös tiedonlähteet ilmaistaan
alaviitteiden avulla.\footcites[162]{kt_oik}[127--128]{typokk}

Alaviite tehdään komennolla \komento{footnote}, jonka argumentiksi
kirjoitetaan alaviitteeseen tuleva teksti. Oletuksena Latex latoo
komennon paikalle yläindeksinumeron, joka tulee automaattisesti
laskurista \laskuri{footnote}. Komennolle voi antaa myös valinnaisen
argumentin, joka on kyseisen alaviitteen numero. Tässä tapauksessa
komento ei kasvata \laskuri{footnote}\-/ laskuria.

\komentoi{footnote}
\begin{koodilohkosis}
\footnote{Alaviitteen teksti.}
\footnote[numero]{Alaviitteen teksti.}
\end{koodilohkosis}

\noindent
Itse alaviite ladotaan samalla kirjaintyypillä kuin leipätekstikin mutta
selvästi pienempänä: sen koko tulee komennosta \komento{footnotesize}
(luku \ref{luku/fontit-korkea}). Alaviitteet erotetaan leipätekstistä
pienellä pystysuuntaisella välillä sekä oletuksena myös lyhyellä
vaakaviivalla. Samalle sivulle voi sattua useampikin alaviite. Jos
alaviitteen tekee leijuvaan osaan (luku \ref{luku/leijuosat}) tai
\ymparisto{minipage}\-/ ympäristöön (luku \ref{luku/laatikot-isot}), se
merkitään oletuksena kirjaimen avulla, ja alaviitteen teksti ladotaan
kyseisen osan alapuolelle.

\komento{footnote}\-/ komennon toiminnan voi jakaa kahteen erilliseen
osaan komentojen \komento{footnotemark} ja \komento{footnotetext}
avulla. Ensin mainittu komento latoo pelkän yläindeksinumeron, joka
tulee alaviitelaskurista. Komento ei lado varsinaista alaviitettä.
Jälkimmäinen komento puolestaan latoo pelkän alaviitteen. Kummallekin
komennolle voi antaa valinnaisen argumentin, jolla ilmaistaan kyseisen
alaviitteen numero.

\komentoi{footnotemark}
\komentoi{footnotetext}
\begin{koodilohkosis}
\footnotemark[numero]                      % yläindeksinumero
\footnotetext[numero]{Alaviitteen teksti.} % alaviite
\end{koodilohkosis}

\noindent
Alaviitteiden numerointi on laskurissa \laskuri{footnote}, mutta sen
arvo ladotaan komennon \komento{thefootnote} avulla. Tämän komennon
määritelmässä on oletuksena komento
\komento{arabic}\komentoarg{footnote}, eli laskuri ladotaan
arabialaisilla numeroilla. Kirjoittaja voi kuitenkin määritellä komennon
uudestaan. Seuraavassa esimerkissä määritellään komento siten, että
alaviitteet ilmaistaan symbolien avulla:

\komentoi{renewcommand}
\komentoi{thefootnote}
\komentoi{fnsymbol}
\laskurii{footnote}
\begin{koodilohkosis}
\renewcommand{\thefootnote}{\fnsymbol{footnote}}
\end{koodilohkosis}

\noindent
Laskurien symboliesitys toimii vain lukualueella 1\==9, joten samaan
dokumenttiin ei mahdu kovin monta alaviitettä, ellei laskuria nollaa
välillä. Laskurin voi määritellä riippuvaiseksi toisesta laskurista,
jolloin se nollautuu itsestään, kun toisen laskurin arvo kasvaa.
Esimerkiksi dokumenttiluokassa \luokka{book} alaviitteiden laskuri
nollautuu oletuksena aina päälukujen (\komento{chapter}) vaihtuessa.

Joskus voi olla tarpeen tehdä alaviitteiden laskuri sivukohtaiseksi eli
riippuvaiseksi \laskuri{page}\-/ laskurista. Sen voi toteuttaa
esimerkiksi paketin \paketti{footmisc} ja valitsimen \koodi{perpage}
avulla. Näitä käsitellään myöhemmin tässä luvussa. Toinen vaihtoehto
sivukohtaiseen alaviitenumerointiin on \paketti{chngcntr}\-/ paketin
komento \komento{counterwithin*} (luku
\ref{luku/hierarkkiset-laskurit}). Seuraavassa on siitä esimerkki:

\komentoi{counterwithin*}
\laskurii{footnote}
\laskurii{page}
\begin{koodilohkosis}
\counterwithin*{footnote}{page}
\end{koodilohkosis}

\noindent
Edellisissä esimerkeissä on käsitelty koko dokumentin alaviitelaskuria
\laskuri{footnote}, mutta leijuvissa osissa ja \ymparisto{minipage}\-/
ympäristöissä (luku \ref{luku/laatikot-isot}) alaviitteet numeroidaan
laskurin \laskuri{mpfootnote} avulla. Tämän laskurin arvo ladotaan
käytännössä komennolla \komento{thempfootnote}, jonka kirjoittaja voi
määritellä uudelleen tarpeen mukaan.

Leipäteksti ja alaviitteet erotetaan pystysuuntaisella välillä, jonka
voi asettaa sivun asetusten yhteydessä (luku \ref{luku/sivuasetukset}).
Asetus on \paketti{geometry}\-/ paketin valitsin \koodi{footnotesep},
jota käytetään esimerkiksi \komento{geometry}\-/ komennon kanssa:

\komentoi{geometry}
\begin{koodilohkosis}
\geometry{footnotesep=14bp}
\end{koodilohkosis}

\noindent
Välin täytyy olla ainakin niin suuri, ettei lukija sekoita alaviitteitä
leipätekstin kappaleeseen. Riittävä väli on noin yhden rivikorkeuden
verran tai vähän enemmän. Sopivan välin suuruus voi riippua siitäkin,
erotetaanko leipäteksti ja alaviitteet toisistaan myös vaakaviivalla.

Muunlaiset alaviitteisiin vaikuttavat asetukset täytyy tehdä paketin
\pakettictan{footmisc} avustuksella. Sen avulla voi esimerkiksi poistaa
alaviitteiden erotinviivan ja muotoilla alaviitekappaleita. Paketin
lataamisen yhteydessä voi antaa useita pilkulla erotettuja valitsimia,
joita on koottu taulukkoon \ref{tlk/footmisc-valitsimet}. Erotinviiva
poistetaan paketin valitsimella \koodi{norule}. Samalla pystysuuntainen
väli hieman kasvaa.

\leijutlk{
  \providecommand{\rivi}{}
  \renewcommand{\rivi}[2]{\koodi{#1} & #2 \\}
  \begin{tabular}{ll}
    \toprule
    \ots{Valitsin} & \ots{Merkitys} \\
    \midrule
    \rivi{norule}{vaakasuuntainen erotinviiva pois}
    \rivi{perpage}{numerointi sivukohtaiseksi}
    \rivi{bottom}{alaviite aina sivun alareunaan}
    \rivi{ragged}{alaviitteeseen automaattinen \komento{raggedright} tai
    \komento{RaggedRight}}
    \rivi{hang}{luetelmatyyppiset alaviitteet, riippuva sisennys}
    \rivi{multiple}{useat peräkkäiset yläindeksinumerot}
    \bottomrule
  \end{tabular}
}{
  \caption{\paketti{footmisc}\-/ paketin asetusten avulla muotoillaan
    alaviitteitä}
  \label{tlk/footmisc-valitsimet}
}

\pakettii{footmisc}
\komentoi{usepackage}
\begin{koodilohkosis}
\usepackage[norule]{footmisc}
\end{koodilohkosis}

\noindent
Valitsin \koodi{perpage} numeroi alaviitteet sivukohtaisesti, eli niiden
laskuri nollautuu aina sivun vaihtuessa. Valitsin \koodi{bottom}
pakottaa alaviitteen aina sivun alareunaan. Oletusasetuksilla alaviite
voi sijoittua joskus ylemmäksi, jos sivun alaosaan ladotaan leijuva osa
tai jos sivun alareunan tasaaminen on pois päältä
(\komento{raggedbottom}, luku \ref{luku/sivunvaihdot}).

Valitsin \koodi{ragged} latoo alaviitteen tekstikappaleet
liehureunaisena eli suorittaa automaattisesti komennon
\komento{raggedright} tai \komento{RaggedRight} (luku
\ref{luku/kappaleen-tasaus}). Monipuolisempi keino alaviitteen tyylin
muokkaamiseen on komento \komento{footnotelayout}. Komennon voi
määritellä itse haluamakseen, ja komennon määritelmään kirjoitetaan mitä
hyvänsä komentoja, jotka halutaan suorittaa automaattisesti jokaisen
alaviitetekstin alussa.

\komentoi{footnotelayout}
\komentoi{raggedright}
\begin{koodilohkosis}
\renewcommand{\footnotelayout}{\raggedright}
\end{koodilohkosis}

\noindent
Oletuksena alaviitekappaleet ladotaan samalla tavalla kuin
tekstikappaleet oletuksena muutenkin, eli alaviitteen ensimmäinen rivi
on sisennetty, ja sen alussa on alaviitteen numero. Seuraavia rivejä ei
sisennetä. Tällainen asetus sopii pitkien alaviitetekstien latomiseen.
Usein viitteet ovat kuitenkin lyhyitä, ja silloin voi sopia paremmin
luetelmatyyppiset alaviitekappaleet (riippuva sisennys), joissa
alaviitteen numero on omalla alueellaan ja varsinainen alaviiteteksti on
sisennetty. Tällainen asetus tehdään \paketti{footmisc}\-/ paketin
valitsimella \koodi{hang}. Sisennyksen suuruuteen voi vaikuttaa mitan
\mitta{footnotemargin} avulla:

\komentoi{setlength}
\komentoi{usepackage}
\mittai{footnotemargin}
\pakettii{footmisc}
\begin{koodilohkosis}
\usepackage[hang]{footmisc}
\setlength{\footnotemargin}{.8em}
\end{koodilohkosis}

\noindent
Valitsin \koodi{multiple} mahdollistaa peräkkäisten
\komento{footnote}\-/ komentojen kirjoittamisen helposti. Peräkkäin
olevien yläindeksinumeroiden väliin ladotaan automaattisesti pilkku,
kuten seuraava esimerkki osoittaa:

\komentoi{footnote}
\begin{koodilohkosis}
sana\footnote{…}\footnote{…}
\end{koodilohkosis}

\begin{tulossis}
  sana\textsuperscript{7}%
  \yipilkku%
  \textsuperscript{8}
\end{tulossis}

\noindent
Toisaalta yläindeksiin voi sijoittaa pilkun tai muuta tekstiä käsinkin
komennolla \komento{textsuperscript} (luku \ref{luku/ylä-alaindeksit}).
\koodi{multiple}\-/ valitsin vaatii toimiakseen tietyn
\paketti{hyperref}\-/ paketin valitsimen, joka käsitellään seuraavaksi.

Jos ladattuna on \paketti{hyperref}\-/ paketti (luku
\ref{luku/pdf-asetukset}), alaviitteisiin liittyvistä
yläindeksinumeroista tulee myös hiirellä napsautettavia linkkejä. Tästä
ominaisuudesta on harvoin mitään hyötyä, koska alaviite on melkein aina
samalla sivulla. Linkkitoiminto myös häiritsee \paketti{footmisc}\-/
paketin \koodi{multiple}\-/ valitsimen toimintaa. Alaviitteiden linkit
voi kytkeä pois \paketti{hyperref}\-/ paketin lataamisen yhteydessä
seuraavalla asetuksella:

\komentoi{usepackage}
\pakettii{hyperref}
\begin{koodilohkosis}
\usepackage[hyperfootnotes=false]{hyperref}
\end{koodilohkosis}

\noindent
\paketti{footmisc}\-/ paketti sisältää muitakin toimintoja alaviitteiden
ulkoasun säätämiseen. Lisätietoa saa paketin ohjekirjasta.

\section{Palstat}
\label{luku/palstat}

Latex pystyy usean palstan käsittelyn perusasioihin, mutta mihinkään
sanoma\-/\ tai aikakauslehden kaltaiseen monipuoliseen tekstin, kuvien
ja muiden osien sommitteluun se ei kykene. Verrattain yksinkertaiset
monipalstaiset tekstit kuitenkin onnistuvat Latexissakin varsin
mukavasti.

Perus Latexissa on kaksi palstoihin liittyvää tilaa: normaali eli yhden
palstan tila sekä kahden palstan tila, joka täytyy erikseen kytkeä
päälle. Tätä tilaa käsitellään luvussa \ref{luku/kahden-palstan-tila}.
Toinen palstojen toteuttamisen vaihtoehto on paketti
\paketti{multicol}. Sen kanssa käytetään Latexin yhden palstan tilaa ja
erityistä palstaympäristöä \ymparisto{multicols}. Ympäristön sisällä
teksti rivitetään kahdelle tai useammalle palstalle. Tätä vaihtoehtoa
käsitellään luvussa \ref{luku/multicol}.

\subsection{Kahden palstan tila}
\label{luku/kahden-palstan-tila}

Perus Latexin yhden tai kahden palstan tilan voi valita jo
dokumenttiluokan (luku \ref{luku/perusdokumenttiluokat-asetukset})
lataamisen yhteydessä tai sivun asetuksissa (luku
\ref{luku/sivuasetukset}) käyttämällä valitsinta \koodi{onecolumn}
(oletus) tai \koodi{twocolumn}. Dokumentin keskellä tilaa voi vaihtaa
komennoilla \komento{onecolumn} ja \komento{twocolumn}. Kumpikin komento
aloittaa uuden sivun, josta alkaen valittu tila on voimassa. Komennolle
\komento{twocolumn} voi antaa hakasulkeissa valinnaisen argumentin,
jolla ilmaistaan sivun alkuun ladottava koko sivun levyinen sisältö:

\komentoi{twocolumn}
\begin{koodilohkosis}
\twocolumn[yksipalstaista sisältöä]
\end{koodilohkosis}

\noindent
\komento{twocolumn}\-/ komennon valinnainen argumentti mahdollistaa sen,
että sivun alkuun voi sijoittaa esimerkiksi koko sivun levyisen otsikon,
kuvan tai muun osan. Sen jälkeen lähes kaikki sisältö on ladotaan
kahdelle palstalle, eli teksti täyttää ensin vasemmanpuoleisen palstan
ja sen jälkeen oikeanpuoleisen.

Leijuvat osat (luku \ref{luku/leijuosat}) ja ympäristöt kuten
\ymparisto{table} ja \ymparisto{figure} sijoitetaan palstan sisään, eli
niiden sisältö täytyy suunnitella yhden palstan levyiseksi. Ympäristöjen
tähdelliset versiot kuten \ymparisto{table*} ja \ymparisto{figure*}
ladotaan koko sivun levyisenä sivun yläosaan tai kokonaan omalle
sivulleen, jossa on vain leijuvia osia. Leijuvien osien sijoitteluun
kahden palstan tilassa voi vaikuttaa tiettyjen laskurien, mittojen ja
komentojen avulla. Niitä käsitellään luvussa
\ref{luku/leijuosat-sijoittelu}.

Palstojen välissä olevan tyhjän tilan leveys on mitassa
\mitta{columnsep}, ja sen voi asettaa haluamakseen
\komento{setlength}\-/ komennolla tai \paketti{geometry}\-/ paketin
asetuksista (luku \ref{luku/sivun-mitat}). Oletusasetus on 35\,pt.
Palstojen väliin saa myös pystyviivan, jonka leveys on mitassa
\mitta{columnseprule}. Sen oletusarvo on nolla, eli pystyviivaa ei
ladota.

\komentoi{setlength}
\mittai{columnsep}
\mittai{columnseprule}
\begin{koodilohkosis}
\setlength{\columnsep}{6mm}       % palstojen väli
\setlength{\columnseprule}{.2mm}  % pystyviivan leveys
\end{koodilohkosis}

\noindent
Latex laskee automaattisesti palstan leveyden ja asettaa sen mittaan
\mitta{columnwidth}. Tätä mittaa ei siis tarvitse eikä pidä itse
asettaa, mutta sitä voi hyödyntää esimerkiksi taulukoiden tai kuvien
sovittamisessa koko palstan levyiseksi. Käytännössä kahden palstan
tilassa palstan leveys on mittojen \mitta{textwidth} ja
\mitta{columnsep} erotus jaettuna kahdella. Tilannekohtainen rivin
pituus on mitassa \mitta{linewidth}.

\subsection{Palstaympäristö}
\label{luku/multicol}

Perus Latex osaa vain yhden ja kahden palstan tilan, ja ne ovat voimassa
koko sivulla. Paketti \pakettictan{multicol} tuo käyttöön myös
ympäristön \ymparisto{multicols}, jonka sisällä tietty palstamäärä on
voimassa. Samallakin sivulla voi olla eri tavoin palstoitettua sisältöä.
Esimerkki \ref{esim/multicols} havainnollistaa ympäristön käyttöä ja
ladottua lopputulosta. Oleelliset tiedot ovat seuraavat:

\ymparistoi{multicols}
\begin{koodilohkosis}
\begin{multicols}{palstamäärä}
  ...
\end{multicols}
\end{koodilohkosis}

\noindent
Argumentti \koodi{palstamäärä} on kokonaisluku, joka ilmaisee, kuinka
monelle palstalle ympäristön sisältö ladotaan. Palstojen korkeus
pyritään tasaamaan automaattisesti, niin että kaikki palstat ovat yhtä
korkeita. Käytännössä palstaympäristön viimeisen sivun viimeinen palsta
jää hieman vajaaksi. Jos käyttää ympäristön tähtiversiota
\ymparisto{multicols*}, ei viimeisen sivun palstojen korkeuksia tasata,
vaan ensin täytetään vasemmanpuoleinen palsta kokonaan, sitten seuraava
jne.

\begin{esimerkki*}
  \ymparistoi{multicols}
  \komentoi{noindent}

\begin{koodilohko}
\begin{multicols}{3}  % kolme palstaa
  \noindent Jukolan talo, eteläisessä Hämeessä, seisoo erään mäen
  pohjoisella rinteellä, liki Toukolan kylää. Sen läheisin ympäristö on
  kivinen tanner, mutta alempana alkaa pellot, joissa, ennenkuin talo
  oli häviöön mennyt, aaltoili teräinen vilja. Peltojen alla on niittu,
  apilaäyräinen, halkileikkaama monipolvisen ojan; ja runsaasti antoi se
  heiniä, ennenkuin joutui laitumeksi kylän karjalle. Muutoin on talolla
  avaria metsiä, soita ja erämaita, jotka, tämän tilustan ensimmäisen
  perustajan oivallisen toiminnan kautta, olivat langenneet sille osaksi
  jo ison jaon käydessä entisinä aikoina.
\end{multicols}
\end{koodilohko}
  \begin{tulos}
    \footnotesize
    \setlength{\columnsep}{1em}
    \setlength{\multicolsep}{0bp}
    \begin{multicols}{3}
      \noindent Jukolan talo, eteläisessä Hämeessä, seisoo erään mäen
      pohjoisella rinteellä, liki Toukolan kylää. Sen läheisin ympäristö
      on kivinen tanner, mutta alempana alkaa pellot, joissa, ennenkuin
      talo oli häviöön mennyt, aaltoili teräinen vilja. Peltojen alla on
      niittu, api\-la\-äyräi\-nen, halkileikkaama monipolvisen ojan; ja
      runsaasti antoi se heiniä, ennenkuin joutui laitumeksi kylän
      karjalle. Muutoin on talolla avaria metsiä, soita ja erämaita,
      jotka, tämän tilustan ensimmäisen perustajan oivallisen toiminnan
      kautta, olivat langenneet sille osaksi jo ison jaon käydessä
      entisinä aikoina.
    \end{multicols}
  \end{tulos}

  \caption{Palstojen toteutus \paketti{multicol}\-/ paketin
    \ymparisto{multicols}\-/ ympäristön avulla. Teksti on Aleksis Kiven
    \emph{Seitsemän veljestä} \=/romaanin alusta}
  \label{esim/multicols}
\end{esimerkki*}

Ympäristölle voi antaa hakasulkeissa valinnaisen argumentin, joka voi
sisältää mitä tahansa tekstiä tai koodia. Se on tarkoitettu otsikolle
(\komento{section} ym.) tai muulle sisällölle, joka ladotaan ennen
palstoja ja jonka täytyy päätyä samalle sivulle kuin palstasisällön
alku.

\ymparistoi{multicols}
\begin{koodilohkosis}
\begin{multicols}{palstamäärä}
  [tekstiä tai koodia ennen palstoja]
  ...
\end{multicols}
\end{koodilohkosis}

\noindent
Ennen palstaympäristöä ja sen jälkeen ladotaan mitan \mitta{multicolsep}
suuruinen pystysuuntainen väli. Väliä ei kuitenkaan ladota, jos se
sattuu sivunvaihdon kohdalle. Palstojen välissä oleva vaakasuuntainen
tyhjä tila asetetaan mitan \mitta{columnsep} avulla, ja väliin
mahdollisesti ladottavan pystyviivan leveys on mitassa
\mitta{columnseprule}. Tämän mitan oletusarvo on nolla, eli pystyviivaa
ei ladota.

\komentoi{setlength}
\mittai{multicolsep}
\mittai{columnsep}
\mittai{columnseprule}
\begin{koodilohkosis}
\setlength{\multicolsep}{.5ex plus .2ex}  % väli ennen ja jälkeen
\setlength{\columnsep}{6mm}               % palstojen väli
\setlength{\columnseprule}{.2mm}          % pystyviivan leveys
\end{koodilohkosis}

\noindent
Oletuksena palstojen alareuna pyritään joka sivulla pitämään tarkasti
samalla tasolla. Tämä toteutuu siten, että palstoilla olevia
pystysuuntaisia välejä voidaan venyttää, jotta niiden viimeiset rivit
saataisiin samalle tasolle. Tämä on yhdenmukainen sivujen
tasaamiskäytännön kanssa, jota käsitellään luvussa
\ref{luku/sivunvaihdot}. Palstojen alareunan tasaamisen ja
pystysuuntaisten välien venyttämisen saa pois komennolla
\komento{raggedcolumns}. Oletustilaan palataan komennolla
\komento{flushcolumns}. Palstanvaihdon voi pakottaa tiettyyn kohtaan
komennolla \komento{columnbreak}. Jos komento sijaitsee tekstikappaleen
sisällä, palsta vaihtuu nykyisen rivin lopussa.

Leijuvat osat (luku \ref{luku/leijuosat}) toimivat
\ymparisto{multicols}\-/ ympäristössä hieman toisin kuin muualla.
Leijuvien ympäristöjen tähdellisiä versioita kuten \ymparisto{table*} ja
\ymparisto{figure*} ei koskaan ladota samalle sivulle kuin missä ne
esiintyvät. Ensimmäinen mahdollinen paikka on seuraavan sivun yläosassa.

Leijuvien osien sijoitteluun vaikuttavat tavalliset yhden palstan tilan
asetukset kuten laskuri \laskuri{topnumber}, mitta \mitta{floatsep} ja
komento \komento{topfraction}. \ymparisto{multicols}\-/ ympäristössä ei
käytetä \koodi{dbl}\-/ alkuisia kahden palstan tilan laskureita, mittoja
eikä komentoja, kuten \laskuri{dbltopnumber}, \mitta{dblfloatsep} ja
\komento{dbltopfraction}. Lisätietoa leijuvien osien sijoittelusta on
luvussa \ref{luku/leijuosat-sijoittelu}.

\section{Lähdeluettelo ja lähdeviitteet}
\label{luku/lähteet}

Lähdeluettelossa mainittuihin teoksiin on tapana viitata muualta
tekstistä, ja Latex osaa auttaa viittausten hallinnassa. Ajatus on se,
että lähdeluettelo laaditaan tiettyjen komentojen avulla niin, että
jokainen lähdeteos saa yksilöllisen tunnisteen. Muualta tekstistä
viitataan lähdeteoksiin käyttämällä samoja tunnisteita, ja Latex osaa
automaattisesti poimia lähdeluettelosta esimerkiksi teoksen tekijöiden
nimet ja vuosiluvun.

Lähdeviittauksiin ja lähdemerkintöihin on useita käytäntöjä, jotka
vaihtelevat eri ammatti\-/\ ja tieteenaloilla, oppilaitoksilla tai
julkaisijoilla. Tässä yhteydessä käsitellään melko vakiintuneita
suomalaisia käytäntöjä, jotka kuvataan \emph{Kielitoimiston
  oikeinkirjoitusoppaassa} \parencite{kt_oik}. Samalla käsitellään
joitakin asetuksia, joilla kukin voi muokata viittausten ulkoasua omiin
tarpeisiin sopivaksi.

Latex sisältää lähdeluetteloiden ja \=/viittausten perustoiminnot, mutta
niiden avulla ei saa yleisen suomalaisen käytännön mukaisia
lähdeviittauksia. Siksi usein tarvitaan avuksi paketti, jolla
viittausten ja lähdeluettelon ulkoasuun voi vaikuttaa. Ensin käsiteltävä
paketti \paketti{natbib} (luku \ref{luku/natbib}) soveltuu
perustarpeisiin ja lienee sopivin valinta useimmille kirjoittajille.
Laajoja tieteellisiä teoksia kirjoittavan kannattanee opetella
käyttämään monipuolista \paketti{biblatex}\-/pakettia (luku
\ref{luku/biblatex}) ja ylläpitää yhteistä lähdeteosten tietokantaa,
josta tarvittavat teokset poimitaan kunkin dokumentin lähdeluetteloon
automaattisesti.

\subsection{Peruskäyttöön (natbib)}
\label{luku/natbib}

Paketti \pakettictan{natbib} laajentaa Latexin lähdeviittausten
perustoimintoja sen verran, että lähteisiin voidaan viitata teoksen
tekijöiden nimen ja vuosiluvun avulla. Seuraava esimerkki
havainnollistaa paketin käyttöönottoa ja asetuksia.

\komentoi{usepackage}
\pakettii{natbib}
\komentoi{setcitestyle}
\begin{koodilohkosis}
\usepackage{natbib}
\setcitestyle{authoryear,aysep={},notesep={: }}
\end{koodilohkosis}

\noindent
Edellisessä esimerkissä viittaustyyli valitaan \komento{setcitestyle}\-/
komennon argumentissa valitsimella \koodi{author\-year}
(tekijä\--vuosi). Valitsimella \koodi{aysep} määritetään, mikä
välimerkki ladotaan tekijän nimen ja vuosiluvun väliin. Tässä se
jätetään tyhjäksi. \koodi{note\-sep}\-/ valitsimella asetetaan merkit,
jotka ladotaan vuosiluvun ja sitä seuraavan huomautuksen kuten
sivunumeroiden väliin; tässä tapauksessa määritettiin kaksoispiste ja
väli \koodi{\{:~\}}, mutta pilkkukin on yleinen käytäntö.

\komento{setcitestyle}\-/ komennon valitsimet erotetaan toisistaan
pilkulla, eikä erotinpilkkujen ympärillä saa olla välilyöntejä.
Lopputuloksena lähdemerkinnät näyttävät esimerkiksi seuraavanlaisilta:

\komentoi{citet*}
\begin{koodilohkosis}
\citet*[27--29]{johdatus} % Viittaus teokseen ”johdatus”.
\end{koodilohkosis}

\begin{tulossis}
  Meikäläinen \& Teikäläinen (2020: 27--29)
\end{tulossis}

\noindent
Lähdeluettelo kirjoitetaan \ymparisto{thebibliography}\-/ ympäristön ja
\komento{bibitem}\-/ komentojen avulla esimerkin
\ref{esim/thebibliography} tavoin. Ympäristön aloittavan komennon
(rivi~1) yhteydessä on argumentti \koodi{00}, jolla ei ole tässä
yhteydessä merkitystä. Jos lähdeviittauksen tyylinä olisi
\koodi{numbers} (eikä \koodi{author\-year}), lähdeluettelon teokset
numeroitaisiin, ja silloin \ymparisto{thebibliography}\-/ ympäristön
argumentti ilmaisee, kuinka leveän sisennyksen numeroidut teokset
tarvitsevat. Argumentiksi voi kirjoittaa mitä tahansa merkkejä, ja Latex
mittaa niiden leveyden. Kannattaa kirjoittaa leveitä numeroita kuten
nollia (\koodi{0}) niin monta kappaletta kuin on numeroita suurimmassa
lähdemerkinnän luvussa. Yksi nolla riittää, jos lähteitä on 1\==9
kappaletta, kaksi jos lähteitä on kaksinumeroinen määrä eli 10\==99
kappaletta jne.

\begin{esimerkki*}
  \ymparistoi{thebibliography}
  \komentoi{bibitem}

\begin{koodilohko}
\begin{thebibliography}{00}

\bibitem[Meikäläinen ym.(2020)Meikäläinen \& Teikäläinen]{johdatus}
  Meikäläinen, Matti \& Teikäläinen, Teija (2020): Johdatus alkeiden
  perusteisiin. Toinen painos. Kustantaja oy.

\bibitem[Itkonen(2019)]{typografia} Itkonen, Markus (2019): Typografian
  käsikirja. Viides, tarkistettu painos. Typoteekki. Graafinen
  suunnittelu Markus Itkonen oy.

\end{thebibliography}
\end{koodilohko}
  \caption{Lähdeluettelon kirjoittaminen \ymparisto{thebibliography}\-/
    ympäristön ja \komento{bibitem}\-/ komentojen avulla.}
  \label{esim/thebibliography}
\end{esimerkki*}

Komennolla \komento{bibitem} tehdään varsinaiset teosmerkinnät. Samalla
määritetään teoksen yksilöllinen tunniste ja mitä tietoja
lähdeviittauksissa näytetään. Yleinen muoto on seuraavanlainen:

\komentoi{bibitem}
\begin{koodilohkosis}
\bibitem[lyhyt(vuosi)pitkä]{tunniste} Lähdeluettelon tekstit.
\end{koodilohkosis}

\noindent
Valinnaisen argumentin aluksi kirjoitetaan lähdeviittauksen lyhyt
merkintä, joka tulisi näkymään lähdeviittauksissa esimerkiksi muodossa
''Meikäläinen ym.''. Heti sen perään kirjoitetaan sulkeissa teoksen
vuosiluku ja sen perään vapaavalintainen lähdeviittauksen pitkä
merkintä, joka näkyisi esimerkiksi tekstinä ''Meikäläinen \&
Teikäläinen''. Vuosiluvun sulkeiden ympärillä ei saa olla välilyöntejä.

\komento{bibitem}\-/ komennon pakollinen argumentti on kyseisen
lähdeteoksen yksilöllinen tunniste, jonka avulla teokseen viitataan.
Komennon argumenttien jälkeen kirjoitetaan samaan tekstikappaleeseen
teksti, joka tulee näkymään lähdeluettelossa.

Lähdeteoksiin viittaamiseen on useita eri komentoja, jotka eroavat
toisistaan siinä, mitä tietoa lähdeviittauksessa näytetään ja onko
lähdeviittaus tai sen osa sulkeissa vai ei. Taulukossa
\ref{tlk/natbib-cite} on joitakin \paketti{natbib}\-/paketin
viittauskomentoja sekä esimerkki viittauksen ulkoasusta. Kuviteltu
esimerkkiteos on peräisin esimerkistä \ref{esim/thebibliography}.

\leijutlk{
  \providecommand{\rivi}{}
  \renewcommand{\rivi}[2]{\komento{#1} & #2 \\}
  \begin{tabular}{ll}
    \toprule
    \ots{Komento} & \ots{Esimerkki} \\
    \midrule
    \rivi{citet}{Meikäläinen ym. (2020)}
    \rivi{citet*}{Meikäläinen \& Teikäläinen (2020)}
    \rivi{citep}{(Meikäläinen ym. 2020)}
    \rivi{citep*}{(Meikäläinen \& Teikäläinen 2020)}
    \rivi{citealt}{Meikäläinen ym. 2020}
    \rivi{citealt*}{Meikäläinen \& Teikäläinen 2020}
    \rivi{citeauthor}{Meikäläinen ym.}
    \rivi{citeauthor*}{Meikäläinen \& Teikäläinen}
    \rivi{citeyear}{2020}
    \rivi{citeyearpar}{(2020)}
    \bottomrule
  \end{tabular}
}{
  \caption{\paketti{natbib}\-/paketin
    läh\-de\-viit\-taus\-ko\-men\-toja}
  \label{tlk/natbib-cite}
}

Lähdeluettelon ulkoasuun voi vaikuttaa mittojen \mitta{bibhang} ja
\mitta{bibsep} avulla. Ensin mainittu on lähdemerkinnän vaakasuuntaisen
riippuvan sisennyksen suuruus, ja jälkimmäinen on lähdemerkintöjen
välinen pystysuuntainen tila.

\komentoi{setlength}
\mittai{parindent}
\mittai{bibhang}
\mittai{bibsep}
\begin{koodilohkosis}
\setlength{\parindent}{1.1em} % tekstikappaleiden 1. rivin sisennys
\setlength{\bibhang}{\parindent}
\setlength{\bibsep}{.5ex plus .1ex minus .1ex}
\end{koodilohkosis}

\noindent
Lähdemerkintöjen fonttiin voi vaikuttaa määrittelemällä uudelleen
komennon \komento{bibfont} ja sijoittamalla halutut fontti- tai muut
komennot kyseisen komennon määritelmään.

\komentoi{renewcommand}
\komentoi{bibfont}
\begin{koodilohkosis}
\renewcommand{\bibfont}{\sffamily\small}
\end{koodilohkosis}

\noindent
Oletuksena \ymparisto{thebibliography}\-/ ympäristö latoo
lähdeluettelolle otsikon, ja otsikon teksti määräytyy kieliasetusten
(luku \ref{luku/kieliasetukset}) ja dokumenttiluokan perusteella (luku
\ref{luku/dokumenttiluokat}). Suomenkielisen lähdeluettelon otsikon voi
määrittää dokumentin esittelyosassa seuraavan esimerkin tavoin.
Esimerkissä hyödynnetään \komento{addto}\-/ komentoa, joka sisältyy
\paketti{polyglossia}\-/{} ja \paketti{babel}\-/ paketteihin.

\komentoi{addto}
\komentoi{captionsfinnish}
\komentoi{renewcommand}
\komentoi{refname}
\komentoi{bibname}
\luokkai{article}
\luokkai{book}
\luokkai{report}
\begin{koodilohkosis}
\addto{\captionsfinnish}{%
  \renewcommand{\refname}{Lähteet} % article-dokumenttiluokka
  \renewcommand{\bibname}{Lähteet} % report- ja book-luokat
}
\end{koodilohkosis}

\noindent
On myös mahdollista määritellä koko komentosarja, joka suoritetaan
lähdeluettelon otsikoinnin yhteydessä. Se tehdään määrittelemällä
uudelleen komento \komento{bibsection}.

\komentoi{renewcommand}
\komentoi{bibsection}
\komentoi{setcounter}
\laskurii{secnumdepth}
\komentoi{section}
\begin{koodilohkosis}
\renewcommand{\bibsection}{%
  \setcounter{secnumdepth}{-1}
  \section{Lähteet}
}
\end{koodilohkosis}

\noindent
Edellisessä esimerkissä komennolla \komento{setcounter} määritetään,
mille otsikkotasolle dokumentin otsikoiden eli lukujen numerointi yltää.
Pieni arvo \mbox{(\koodi{-1})} käytännössä tarkoittaa, että seuraaviin
otsikoihin ei tule numerointia. Komento \komento{section} tekee itse
otsikon.

Jos ei halua, että \ymparisto{thebibliography}\-/ ympäristö tekee
otsikon automaattisesti, voi \komento{bibsection}\-/ komennon määrittää
tyhjäksi.

\komentoi{renewcommand}
\komentoi{bibsection}
\begin{koodilohkosis}
\renewcommand{\bibsection}{}
\end{koodilohkosis}

\noindent
Paketti \paketti{natbib} sisältää edellä kuvattujen lisäksi muitakin
ominaisuuksia, joihin voi tutustua paketin ohjekirjan avulla. On muun
muassa mahdollista tehdä lähdeteoksista tietokanta Bibtex\-/
järjestelmän avulla. Jos kuitenkin siihen suuntaan haluaa edetä, ei ehkä
kannata käyttää \paketti{natbib}\-/ pakettia eikä vanhaa Bibtexiä vaan
monipuolisempaa pakettia \paketti{biblatex}, jota käsitellään
seuraavassa alaluvussa.

\subsection{Vaativaan käyttöön (biblatex)}
\label{luku/biblatex}

Suurten lähde- ja kirjallisuusluetteloiden ylläpito voi olla aika
työlästä: pitää jatkuvasti varmistaa, että kaikki viitatut teokset ovat
luettelossa ja että luettelo on pilkulleen yhdenmukainen. Paketti
\pakettictan{biblatex} on vastaus sellaisiin tarpeisiin.

Ajatuksena on, että kaikki tiedonlähteet ja kirjallisuus kirjoitetaan
tietokantaan, josta \paketti{biblatex}\-/paketin komennot hakevat tiedot
automaattisesti. Kirjoittaja tai työryhmä voi ylläpitää yhtä
kirjallisuustietokantaa, joka voi olla saatavilla oman laitoksen
verkkopalvelimella tai julkisella verkkosivullakin. Dokumentin tekstissä
viitataan teoksiin yksilöllisen tunnisteen avulla, ja pelkän viittauksen
perusteella oikeat teokset ilmestyvät lähdeluetteloon automaattisesti
aakkosjärjestyksessä ja yhdenmukaisessa muodossa. Yhtään tiedonlähdettä
ei tarvitse kirjoittaa lopulliseen lähdeluetteloon käsin.

\paketti{biblatex}\-/paketin käyttö vaatii hieman opettelua -- varsinkin
jos on tarve muokata lähdeluettelon ja lähdeviittausten ulkoasua.
Muutaman tiedonlähteen ylläpito on todennäköisesti paljon helpompaa ja
nopeampaa \paketti{natbib}\-/ paketin keinoilla (luku
\ref{luku/natbib}). Sen sijaan laajoja tieteellisiä artikkeleita
kirjoittaville \paketti{biblatex} voi olla suuri apu, koska
artikkeleissa on yleensä paljon lähteitä ja useissakin artikkeleissa
viitataan yleensä samoihin lähteisiin.

\subsubsection{Teostietokanta}

Lähdeteosten tietokanta on erillinen tekstitiedosto, joka tavallisesti
nimetään \koodi{bib}\-/päätteiseksi, esimerkiksi \koodi{teokset.bib}.
Tiedosto koostuu \koodi{@}\=/merkillä ja teostyypin nimellä alkavista
tietueista, joiden yleinen muoto on seuraavanlainen:

\begin{koodilohkosis}
@teostyyppi{tunniste,
  author = {...},
  title  = "..."
}
\end{koodilohkosis}

\noindent
Teostyypin nimen jälkeen aaltosulkeiden sisään kirjoitetaan teoksen
kaikki tiedot. Ne alkavat teoksen yksilöllisellä tunnisteella, jota
käytetään lähdeviittauksissa. Tunnisteen jälkeen tulevat muut kentät.
Eri kentät kuten \koodi{author} ja \koodi{title} erotetaan toisistaan
pilkulla. Kentän nimi ja sen sisältö erotetaan toisistaan
yhtäsuuruusmerkillä (\koodi{=}), ja kentän sisältö kirjoitetaan
aaltosulkeiden tai lainausmerkkien sisään, kuten edellinen esimerkki
näyttää.

\begin{esimerkki*}
\begin{koodilohko}
@book{itkonen_typogr,
  author = {Itkonen, Markus},
  title = {Typografian käsikirja},
  date = {2019},
  edition = {5},
  publisher = {Typoteekki. Graafinen suunnittelu Markus Itkonen Oy}
}

@incollection{likonen_teams,
  author = {Likonen, Teemu and Riskilä, Kaisa},
  title = {Verkkoyhteistyö Teams-ympäristössä},
  editor = {Tammi, Tuomo and Horila, Mikko},
  booktitle = {Oppimis- ja toimintaympäristöjen kehittäminen
    harjoittelukouluissa II},
  booksubtitle = {Tilat ja tekniikka pedagogisen kehittämisen tukena},
  publisher = {E-norssi. Opettajankouluttajien yhteistyöverkosto},
  date = {2020},
  pages = {85-92},
  url = {http://www.enorssi.fi/oppimisymparistojulkaisu2020/}
}

@article{likonen_tietokanta,
  author = {Likonen, Teemu},
  title = {Tietoa kantaan ja takaisin},
  journaltitle = {Skrolli},
  journalsubtitle = {Tietokonekulttuurin erikoislehti},
  date = {2015},
  volume = {2015},
  number = {4},
  pages = {52-55},
  url = {https://skrolli.fi/numerot/2015-4/}
}

@online{ctan,
  title = {Comprehensive TeX Archive Network},
  shorttitle = {CTAN},
  date = {1992/},
  url = {https://www.ctan.org/},
  urldate = {2021-07-14}
}
\end{koodilohko}
  \caption{Lähdeteosten tietokantatiedosto}
  \label{esim/bib-tiedosto}
\end{esimerkki*}

Todellista käyttöä vastaava tietokanta tai sen osa on esimerkissä
\ref{esim/bib-tiedosto}, jossa on neljä erityyppistä teostietuetta:
\koodi{book}, \koodi{incollection}, \koodi{article} ja \koodi{online}.
Ensin mainittu teostyyppi \koodi{book} sopii tavallisille kirjoille,
joissa tietyt tekijät (\koodi{author}) vastaavat suunnilleen koko
teoksen sisällöstä ja teoksella on jokin julkaisijataho
(\koodi{publisher}).

Teostyyppi \koodi{incollection} tarkoittaa esimerkiksi
artikkelikokoelmaa, jonka yksittäiseen artikkeliin (\koodi{title}) ja
sen kirjoittajaan (\koodi{author}) on tarkoitus viitata. Voidaan mainita
myös artikkelin alku- ja loppusivut (\koodi{pages}). Kokoelmalla on
toimittaja (\koodi{editor}) ja yhteinen nimi (\koodi{book\-title}).

Tyyppi \koodi{article} sopii säännöllisesti julkaistavan aikakaus- tai
muun lehden artikkeleihin. Viittauskohteena on yksittäinen artikkeli ja
sen kirjoittaja. Julkaisutiedoissa mainitaan lehden nimi
(\koodi{jour\-nal\-title}), julkaisukausi (\koodi{vol\-ume}), kauteen
kuuluvan julkaisun järjestysnumero (\koodi{num\-ber}) sekä mahdollisesti
artikkelin sivut (\koodi{pages}).

Verkkolähteiden merkitsemiseen sopii \koodi{online}\-/ teostyyppi,
joissa on tavanomaisten kenttien lisäksi ainakin verkko\-/osoite eli
\koodi{url}\-/kenttä ja mahdollisesti viittauspäivä (\koodi{url\-date})
osoittamassa, milloin viitatut tiedot olivat saatavilla.

Teostyyppejä ja teoksiin liittyviä tietokenttiä on olemassa paljon
muitakin. Niiden merkitystä ja käyttöä neuvotaan tarkemmin
\paketti{biblatex}\-/paketin ohjeissa. Seuraavassa on kuitenkin pari
huomiota tietokannan ja kenttien kieliopillisista asioista.

Tietueissa joidenkin kenttien sisältö voi koostua useasta osasta kuten
saman teoksen eri tekijöistä. Eri tekijöiden nimet erotetaan
\koodi{author}- ja \koodi{editor}\-/kentissä toisistaan
\koodi{and}\-/sanalla. Oletuksena \paketti{biblatex} katsoo, että
tekijät ovat henkilöitä, ja käsittelee esimerkiksi etu- ja sukunimet
tietyllä tavalla: jos mukana on pilkku, sitä ennen on sukunimi, ja
etunimet tulevat pilkun jälkeen; jos pilkkua ei ole, etunimet ovat
ensin, ja sukunimi on lopussa.

Jos kuitenkin teoksen tekijänä on yritys tai yhteisö, täytyy sen nimi
kirjoittaa kokonaan aaltosulkeisiin, jottei sitä tulkittaisi henkilön
nimeksi. Tällaisten aaltosulkeiden sisällä voi käyttää \koodi{and}\-/
sanaa normaalisti, eikä sitä tulkita eri tekijöiden erottimeksi.
Seuraavassa on näistä esimerkit:

\begin{koodilohkosis}
author = {Meikäläinen, Matti and Teikäläinen, Teija}
author = {{Org. of Latex and Typography} and Meikäläinen, Matti}
\end{koodilohkosis}

\noindent
Muunkinlaisia useasta osasta koostuvia kenttiä on olemassa.
Asiasanakentän (\koodi{keywords}) eri sanat erotetaan toisistaan
pilkulla, ja sivunumeroissa (\koodi{pages}) voi olla myös lukualueita,
jotka ilmaistaan yhdysmerkillä.

\begin{koodilohkosis}
keywords = {eri, sanoja, peräkkäin}
pages = {15-19}
\end{koodilohkosis}

\noindent
Teostietokantaan voi määrittää vakiosisältöisiä muuttujia käyttämällä
\koodi{@string}\-/ rakennetta. Vakioihin voi sitten viitata
teostietueiden kentistä esimerkin \ref{esim/bib-muuttujat} tavoin.
Vakiot ovat hyödyllisiä silloin, kun sama kentän sisältö toistuu useissa
teoksissa, kuten tässä esimerkissä sama tekijä (\koodi{author}) ja
aikakauslehden nimi (\koodi{jour\-nal\-title}). Vakioita voi yhdistää
saman kentän muuhun sisältöön käyttämällä \koodi{\#}\=/merkkiä, kuten
esimerkin rivillä 13 on tehty.

\begin{esimerkki*}
\begin{koodilohko}
@string{
  itse = {Meikäläinen, Matti},
  lehti = {Hienon hieno aikakauslehti}
}

@article{hieno_artikkeli,
  author = itse,
  journaltitle = lehti,
  ...
}

@article{toinen_artikkeli,
  author = itse # { and Teikäläinen, Teija},
  journaltitle = lehti,
  ...
}
\end{koodilohko}
  \caption{Muuttujien käyttö ja \koodi{@string}\-/rakenne}
  \label{esim/bib-muuttujat}
\end{esimerkki*}

\subsubsection{Käyttöönotto}

\paketti{biblatex}\-/paketti otetaan käyttöön esimerkin
\ref{esim/biblatex-käyttöönotto} rivien avulla. Mukana on myös
kielipaketti sekä \paketti{csquotes}, joka sisältää lainausmerkkeihin
liittyvää logiikkaa (luku \ref{luku/lainausmerkit}). Ilman sitä
\paketti{biblatex} ei saa eri kielten erilaisia lainausmerkkejä oikein
vaan käyttää pelkästään amerikkalaisia (``~'').

\begin{esimerkki*}
  \komentoi{usepackage}
  \pakettii{polyglossia}
  \pakettii{csquotes}
  \pakettii{biblatex}

\begin{koodilohko}
% Kielipaketti polyglossia tai babel on ladattava ennen biblatexia.
\usepackage{polyglossia} \setdefaultlanguage{finnish}
%\usepackage[main=finnish]{babel}

% Kielikohtaiset lainausmerkit oikein csquoten avulla.
\usepackage{csquotes}

\usepackage[style=authoryear]{biblatex}
\end{koodilohko}
  \caption{\paketti{biblatex}\-/ paketin käyttöönotto ja asetuksia}
  \label{esim/biblatex-käyttöönotto}
\end{esimerkki*}

Paketin asetuksissa käytetään valitsinta \koodi{style} ja sen asetusta
\koodi{author\-year}, joka asettaa lähdeviittausten ja lähdeluettelon
tyyliksi tekijän ja vuosiluvun. Se on yleinen käytäntö suomenkielisissä
teksteissä. Vastaavia tyylejä ovat myös \koodi{author\-year-comp},
\koodi{author\-year-ibid} ja \koodi{author\-year-icomp}, jotka lisäksi
tiivistävät peräkkäisiä lähdeviittauksia, jos teoksen tekijä on sama.

Numerointiin tai kirjainlyhenteisiin perustuvat lähdeluettelo\-/{} ja
viittaustyylit ovat nimeltään \koodi{nu\-mer\-ic} ja
\koodi{al\-pha\-bet\-ic}. Muitakin tyylejä on olemassa, mutta tämän
oppaan esimerkeissä käsitellään tekijä--vuosi-tyyliä.

Paketin omien lähdeluettelo\-/\ ja viittaustyylien lisäksi Latex\-/
jakelupaketissa on todennäköisesti mukana myös ulkopuolisten tahojen
tekemiä tyylejä. Tyylikokonaisuus nimeltä \pakettictan{biblatex-ext}
laajentaa \paketti{biblatex}\-/ paketin tavallisten tyylien
ominaisuuksia. Laajennustyylien käyttäminen ei vaadi erillisen paketin
lataamista, vaan tyylin saa käyttöön yksinkertaisesti vain
kirjoittamalla sen nimen \paketti{biblatex}\-/ paketin lataamisen
yhteydessä. Laajennetut tyylit alkavat kirjaimilla \mbox{\koodi{ext-},}
esimerkiksi \koodi{ext-author\-year} tai \koodi{ext-author\-year-comp}.

Kaikki käyttöön otettavat teostietokantatiedostot täytyy esitellä
komennolla \komento{addbibresource}. Tietokantatiedostoja voi olla
useampiakin, ja tietokanta voi olla myös verkko\-/ osoitteen takana
oleva tiedosto. \komento{addbibresource}\-/ komennot täytyy kirjoittaa
Latex\-/ lähdedokumentin esittelyosaan.

\komentoi{addbibresource}
\begin{koodilohkosis}
\addbibresource{teokset.bib}
\addbibresource{~/texmf/omat_kirjoitukset.bib}
\addbibresource[location=remote]{http://osoite.netissä/yhteiset.bib}
\end{koodilohkosis}

\noindent
Lähdeluettelo ladotaan dokumenttiin komennolla
\komento{printbibliography}. Komennolle voi antaa valinnaisen
argumentin, jonka valitsimilla vaikutetaan esimerkiksi lähdeluettelon
otsikon tekstiin tai poistetaan automaattinen otsikointi kokonaan. On
myös olemassa erilaisia lähdeteosten rajaamisvalitsimia, joiden avulla
voi määrittää, mitä teoksia kyseiseen luetteloon halutaan. Sen avulla
voidaan esimerkiksi rajata painetut lähteet yhteen luetteloon,
julkaisemattomat toiseen ja verkkolähteet kolmanteen. Seuraavassa on
erilaisia esimerkkejä:

\komentoi{printbibliography}
\begin{koodilohkosis}
\printbibliography
\printbibliography[title={Lähteet}]
\printbibliography[heading=none,  % Ei automaattista otsikkoa,
  type=online]           % ja rajataan vain online-tyyppisiin.
\end{koodilohkosis}

\noindent
Lähdeluetteloon tulevat mukaan vain ne teokset, joihin on viitattu.
Mitään ei siis näy, jos ei ole lähdeviittauksia. Seuraavassa alaluvussa
käsitellään lähdeviittauskomentoja ja myös ''näkymätöntä''
viittauskomentoa, jolla teoksia saadaan mukaan luetteloon ilman näkyvää
viittausta.

\subsubsection{Lähdeviittaukset}

\leijutlk{
  \providecommand{\rivi}{}
  \renewcommand{\rivi}[2]{\komento{#1} & #2 \\}
  \begin{tabular}{ll}
    \toprule
    \ots{Komento} & \ots{Esimerkki} \\
    \midrule
    \rivi{cite}{Meikäläinen 2020}
    \rivi{textcite}{Meikäläinen (2020)}
    \rivi{parencite}{(Meikäläinen 2020)}
    \rivi{citeauthor}{Meikäläinen}
    \rivi{citeyear}{2020}
    \rivi{citetitle}{[teoksen nimi]}
    \rivi{footcite}{Meikäläinen 2020 [alaviitteessä]}
    \rivi{nocite}{[näkymätön viittaus]}
    \bottomrule
  \end{tabular}
}{
  \caption{\paketti{biblatex}\-/paketin lähdeviittauskomentoja}
  \label{tlk/biblatex-cite}
}

Taulukkoon \ref{tlk/biblatex-cite} on koottu tavallisimpia
\paketti{biblatex}\-/paketin viittauskomentoja. Komennon argumentiksi
annetaan teoksen yksilöllinen tunniste. Komennoille voi antaa myös
valinnaisen argumentin, jolla kerrotaan täsmentävää tietoa
lähdeviittauksesta. Yleensä se on viitattavan teoksen sivunumero.
Viittaus näkyy dokumentissa esimerkiksi seuraavalla tavalla:

\komentoi{textcite}
\begin{koodilohkosis}
\textcite[27--29]{johdatus} % Viittaus teokseen ”johdatus”.
\end{koodilohkosis}

\begin{tulossis}
  Meikäläinen ja Teikäläinen (2020, s. 27--29)
\end{tulossis}

\noindent
Jos halutaan sisällyttää lähdeluetteloon teoksia, joihin ei ole
välttämättä viitattu, käytetään dokumentissa kerran ''näkymätöntä''
viittauskomentoa \komento{nocite}. Sille annetaan argumentiksi
tunnisteet niistä teoksista, jotka halutaan mukaan luetteloon.
Argumentti~\koodi{*} (tähti) valitsee kaikki teokset.

\komentoi{nocite}
\begin{koodilohkosis}
\nocite{meikäläinen, teikäläinen} % Nämä teokset mukaan.
\nocite{*}                        % Kaikki mukaan.
\end{koodilohkosis}

\subsubsection{Lähdetiedostojen kääntäminen}

Latexin kääntäjäohjelmat Lualatex tai Xelatex eivät yksinään riitä,
sillä teostietokanta ei ole tavallinen Latex\-/muotoinen tiedosto.
Tarvitaan myös Latex\-/jakelun mukana tulevaa komentoa \koodi{biber},
joka käsittelee teostietokantaan liittyviä tiedostoja. Lopulta
Latex\-/kääntäjääkin täytyy kutsua kaksi kolme kertaa, jotta kaikki
ristiviitteet saadaan kuntoon. Komentojen suoritusjärjestys on
seuraavanlainen:

\begin{koodilohkosis}
lualatex teksti.tex
biber teksti.bcf
lualatex teksti.tex
lualatex teksti.tex
\end{koodilohkosis}

\noindent
Edellisen esimerkin komennoissa voi tiedoston nimistä jättää päätteet
pois (\koodi{.tex}, \koodi{.bcf}). \koodi{lualatex}\-/ ohjelman paikalla
voi olla myös \koodi{xelatex}. Kääntäminen on vielä helpompaa, kun
käyttää \koodi{latexmk}\-/ ohjelmaa (luku \ref{luku/latexmk}), joka osaa
automaattisesti suorittaa myös \koodi{biber}\-/ ohjelman ja tarvittavat
uudelleen kääntämiset. Yksi komento riittää:

\begin{koodilohkosis}
latexmk -lualatex teksti.tex    % tai: -xelatex
\end{koodilohkosis}

\subsubsection{Lähdeluettelon mittoja}

Lähdeluettelon ulkoasuun voi vaikuttaa muutaman eri mitan avulla, joista
esitellään tässä yhteydessä vain osa. Lähdemerkinnän riippuvan
sisennyksen suuruus määräytyy mitan \mitta{bibhang} avulla. Yleensä
lienee sopivaa asettaa se samaksi kuin tekstikappaleiden ensimmäisen
rivin sisennys \mitta{parindent}.

\komentoi{setlength}
\mittai{parindent}
\mittai{bibhang}
\begin{koodilohkosis}
\setlength{\parindent}{1.1em} % tekstikappaleiden 1. rivin sisennys
\setlength{\bibhang}{\parindent}
\end{koodilohkosis}

\noindent
Mitta \mitta{bibitemsep} on lähdemerkintöjen välinen pystysuuntainen
tila. Sen avulla voi harventaa lähdeluetteloa, jolloin lähdemerkinnät
erottuvat paremmin toisistaan. Mitan \mitta{bibnamesep} avulla voi tehdä
suuremman pystysuuntaisen välin lähdemerkintöjen väliin silloin, kun
teoksen tekijä vaihtuu (\koodi{author} tai \koodi{editor}). Toisin
sanoen tämän mitan avulla voi ryhmitellä saman tekijän teokset
tiiviimmin yhteen ja jättää väliä seuraavan tekijän teoksiin.
Vastaavanlainen mitta on \mitta{bibinitsep}, jota käytetään silloin, kun
lähdemerkinnän aloittava kirjain vaihtuu. Tämän avulla voi ryhmitellä
lähdemerkinnät aakkosittain eli tehdä suuremman välin aina
lähdemerkinnän alkukirjaimen vaihtuessa.

\komentoi{setlength}
\mittai{bibitemsep}
\mittai{bibnamesep}
\mittai{bibinitsep}
\begin{koodilohkosis}
\setlength{\bibitemsep}{.5ex plus .1ex minus .1ex}
\setlength{\bibnamesep}{1em  plus .2ex minus .1ex}
\setlength{\bibinitsep}{2em  plus .2ex minus .1ex}
\end{koodilohkosis}

\subsubsection{Muita asetuksia}

Lähdemerkintöjen fonttiin voi vaikuttaa määrittelemällä uudelleen
komennon \komento{bibfont} ja sijoittamalla halutut fontti- tai muut
komennot kyseisen komennon määritelmään.

\komentoi{renewcommand}
\komentoi{bibfont}
\begin{koodilohkosis}
\renewcommand{\bibfont}{\sffamily\small}
\end{koodilohkosis}

\noindent
Lähdemerkinnät itsessään muodostetaan automaattisesti tiettyjen
tyyliasetusten perusteella. Omiakin tyylejä voi tehdä, mutta yleensä
riittää vain yksittäisen asetusten muuttaminen. Niistä käsitellään tässä
yhteydessä muutama. Asetusten muuttamiseen tarvitaan yleensä
\paketti{biblatex}\-/ paketin omia asetuskomentoja.

Lähdeluettelon nimet näkyvät oletusasetuksilla siten, että teoksen
ensimmäisen tekijän sukunimi mainitaan ensin (luettelon
aakkosjärjestyksen vuoksi) mutta saman teoksen muiden tekijöiden etunimi
mainitaan ensin. Tekijöiden nimet näkyvät siis seuraavalla tavalla:
''Meikäläinen, Matti ja Teija Teikäläinen''. Suomessa on kuitenkin
tapana kirjoittaa kaikki nimet samalla tavalla ja mainita sukunimi aina
ensin. Tämä saadaan toteutettua seuraavilla komennolla:

\komentoi{DeclareNameAlias}
\begin{koodilohkosis}
\DeclareNameAlias{default} {family-given}
\DeclareNameAlias{sortname}{family-given}
\end{koodilohkosis}

\begin{tulossis}
  Meikäläinen, Matti ja Teikäläinen, Teija (2020). --~--
\end{tulossis}

\noindent
Saman teoksen eri tekijöiden nimet erotetaan oletuksena toisistaan
pilkuilla paitsi kahden viimeisen nimen välissä on \textit{ja}-sana.
Usein on kuitenkin tapana käyttää \&\=/merkkiä ainakin lähdeluettelossa.
Seuraavat esimerkkikomennot asettavat lähdeluettelon kaikkien nimien
erottimeksi \&\=/merkin.

\komentoi{DeclareDelimFormat}
\begin{koodilohkosis}
\DeclareDelimFormat[bib]{multinamedelim}{\space\&\space}
\DeclareDelimFormat[bib]{finalnamedelim}{\space\&\space}
\end{koodilohkosis}

\begin{tulossis}
  Meikäläinen, Matti \& Teikäläinen, Teija \& Tutkija, Tuija (2020).
  --~--
\end{tulossis}

\noindent
Erotinmerkkiasetuksen nimi \koodi{multi\-name\-delim} tarkoittaa muiden
kuin kahden viimeisen tekijän nimen välissä olevaa erotinta. Kahden
viimeisen nimen erotin määritellään asetuksella
\koodi{final\-name\-delim}.

Edellisten esimerkkikomentojen valinnainen argumentti \koodi{bib}
tarkoittaa, että vaikutetaan vain lähdeluetteloon. Argumentti voi olla
myös esimerkiksi \koodi{cite}, \koodi{textcite} tai \koodi{parencite},
jolloin vaikutetaan samannimisillä komennoilla tehtyihin
lähdeviittauksiin: \komento{cite}, \komento{textcite} ja
\komento{parencite}. Katso lähdeviittauskomennot taulukosta
\ref{tlk/biblatex-cite}. Seuraava esimerkki vaihtaa
\komento{parencite}\-/ komentoon asetuksen, niin että kahden viimeisen
henkilön nimen väliin ladotaan \&\=/merkki. Oletuksena viittauksissa
käytetään \textit{ja}\-/sanaa.

\komentoi{DeclareDelimFormat}
\begin{koodilohkosis}
\DeclareDelimFormat[parencite]{finalnamedelim}{\space\&\space}
\end{koodilohkosis}

\begin{tulossis}
  (Meikäläinen, Teikäläinen \& Tutkija 2020)
\end{tulossis}

\noindent
Useiden saman teoksen tekijöiden luettelot lyhennetään automaattisesti
esimerkiksi muotoon ''Meikäläinen et~al.'', ja lyhentämisen säännöt
määritellään tiettyjen \koodi{max}- ja \koodi{min}\-/ alkuisten paketin
valitsimien avulla. Lähdeluettelossa teoksen tekijäluetteloon
vaikutetaan valitsimilla \koodi{max\-bib\-names} ja
\koodi{min\-bib\-names}, kun taas lähdeviittausten tekijäluetteloon
vaikutetaan valitsimilla \koodi{max\-cite\-names} ja
\koodi{min\-cite\-names}. Asetukset toimivat siten, että jos
enimmäismäärä (\englanti{max}) ylittyy, typistetään tekijäluettelo
vähimmäismäärään (\englanti{min}) ja lisätään ilmaus ''et al.'' tms.

Tekijäluetteloa ei kuitenkaan välttämättä lyhennetä, jos luettelosta
tulisi täsmälleen samanlainen kuin jollakin toisella teoksella. Tähän
asiaan puolestaan vaikutetaan valitsimella \koodi{unique\-list}, joka on
oletuksena päällä viittaustyylissä \koodi{author\-year}.

\komentoi{usepackage}
\pakettii{biblatex}
\begin{koodilohkosis}
\usepackage[style=authoryear, maxbibnames=99, minbibnames=3,
  maxcitenames=3, mincitenames=1, uniquelist=true]{biblatex}
\end{koodilohkosis}

\noindent
Kun halutaan näyttää lähdeluettelossa vain tekijän etunimen alkukirjain
eikä koko etunimeä, käytetään paketin valitsinta \koodi{given\-inits}.

\komentoi{usepackage}
\pakettii{biblatex}
\begin{koodilohkosis}
\usepackage[…, giveninits]{biblatex}
\end{koodilohkosis}

\begin{tulossis}
  Meikäläinen, M. (2020). -- --
\end{tulossis}

\noindent
Lähdeluettelossa näytetään teoksen tekijän nimen kohdalla ajatusviiva,
jos tekijä on sama kuin luettelon edelliselläkin teoksella. Mikäli tätä
(sinänsä yleistä) käytäntöä ei haluta, täytyy käyttää paketin asetusta
\koodi{dash\-ed=\katk false}.

\komentoi{usepackage}
\pakettii{biblatex}
\begin{koodilohkosis}
\usepackage[…, dashed=false]{biblatex}
\end{koodilohkosis}

\noindent
Joskus on tapana latoa lähdeluettelossa tekijöiden nimet esimerkiksi
pienversaalilla, jotta ne erottuvat luettelosta paremmin. Tällainen
muutos vaatii, että määritellään uudelleen henkilön nimiin liittyvät
latomiskomennot \komento{mkbibnamefamily}, \komento{mkbibnamegiven},
\komento{mkbibnameprefix} ja \komento{mkbibnamesuffix}. Se saadaan
automaattiseksi seuraavilla komennoilla:

\komentoi{AtBeginBibliography}
\komentoi{mkbibnamefamily}
\komentoi{mkbibnamegiven}
\komentoi{mkbibnameprefix}
\komentoi{mkbibnamesuffix}
\begin{koodilohkosis}
\AtBeginBibliography{%
  \renewcommand{\mkbibnamefamily}[1]{\textsc{#1}}
  \renewcommand{\mkbibnamegiven} [1]{\textsc{#1}}
  \renewcommand{\mkbibnameprefix}[1]{\textsc{#1}}
  \renewcommand{\mkbibnamesuffix}[1]{\textsc{#1}}
}
\end{koodilohkosis}

\begin{tulossis}
  \textsc{Meikäläinen}, \textsc{Matti} \& \textsc{Teikäläinen},
  \textsc{Teija} (2020). -- --
\end{tulossis}

\noindent
Lähdeluettelossa yhden lähdemerkinnän eri osien erottimena on piste.
Joskus kuitenkin tekijöiden nimien ja vuosiluvun jälkeen halutaan
kaksoispiste. Se saadaan toteutettua seuraavalla komennolla:

\komentoi{DeclareDelimFormat}
\begin{koodilohkosis}
\DeclareDelimFormat[bib]{nametitledelim}{\addcolon\space}
\end{koodilohkosis}

\begin{tulossis}
  Meikäläinen, Matti (2020): -- --
\end{tulossis}

\noindent
Oletuksena \paketti{biblatex} kursivoi \koodi{book}\-/ tyyppisten
teosten nimen (\koodi{title}). Sen sijaan artikkelikokoelmissa
(\koodi{incollection}) ja aikakauslehdissä (\koodi{article})
kursivoidaan julkaistun kokoelman nimi (\koodi{book\-title}) ja
aikakauslehden nimi (\koodi{jour\-nal\-title}). Näissä teostyypeissä
viitatun artikkelin nimi (\koodi{title}) kirjoitetaan lainausmerkkeihin.
Käytäntö tuntuu järkevältä, sillä kursivoituna on aina julkaistu
kokonainen teos eikä sen osa. Käytännössähän tiedonlähde joudutaan
hakemaan teoksen nimen perusteella. Joku voi silti haluta muuttaa näiden
ulkoasua ja esimerkiksi kursivoida aina viittauksen kohteena olevan
artikkelin. Seuraavassa on esimerkkikomennot edellä mainittujen
lähdeluettelon kenttien muuttamiseen.

\komentoi{DeclareFieldFormat}
\begin{koodilohkosis}
\DeclareFieldFormat[article,incollection]{title}{\emph{#1}}
\DeclareFieldFormat[article]{journaltitle}{#1}
\DeclareFieldFormat[incollection]{booktitle}{#1}
\end{koodilohkosis}

\noindent
Edellä olevissa esimerkkikomennoissa on valinnaisena argumenttina ne
teostyypit, joihin halutaan vaikuttaa. Jos valinnaisen argumentin jättää
pois, vaikutetaan kaikkiin teostyyppeihin, ellei tarkempaa
teostyyppikohtaista määritelmää ole olemassa. Ensimmäinen pakollinen
argumentti on kentän nimi teostietokannassa, ja toinen pakollinen
argumentti on sisältö, joka ladotaan lähdemerkintään kyseisen tiedon
kohdalle. Teostietokannasta tulevan kentän sisältö on parametrissa
\koodi{\#1}.

Mikäli haluaa jonkin teoksen tiedon lainausmerkkeihin tai sulkeisiin,
kannattaa käyttää komentoa \komento{mkbibquote} tai
\komento{mkbibparens}. Ne ymmärtävät ottaa huomioon eri kielten
lainausmerkkikäytännöt ja mahdolliset sisäkkäiset sulkeet.

\komentoi{DeclareFieldFormat}
\komentoi{mkbibquote}
\begin{koodilohkosis}
\DeclareFieldFormat[incollection]{booktitle}{\mkbibquote{#1}}
\end{koodilohkosis}

\noindent
Oletuksena teoksen vuosiluku tai muu päiväys ladotaan lähdeluetteloon
sulkeissa. Joissakin lähdeluettelokäytännöissä sulkeita ei kuitenkaan
ole, joten seuraavaksi käsitellään keino sulkeiden poistamiseen.
Tavallisessa \paketti{biblatex}\-/paketin lähdeluettelotyylissä
\koodi{author\-year} ei ole omaa asetusta teoksen päiväyksen ulkoasun
muuttamiseen, mutta jos käyttää tyyliä \koodi{ext-author\-year} (tms.),
sekin puute korjaantuu, ja voi käyttää \koodi{bib\-label\-date}\-/
asetusta.

\komentoi{usepackage}
\pakettii{biblatex}
\komentoi{DeclareFieldFormat}
\begin{koodilohkosis}
\usepackage[style=ext-authoryear]{biblatex}

\DeclareFieldFormat{biblabeldate}{#1}
\end{koodilohkosis}

\begin{tulossis}
  Meikäläinen, Matti 2020. -- --
\end{tulossis}

\noindent
Artikkelikokoelmissa (teostyyppi \koodi{incollection}) mainitaan
oletuksena kokoelman nimi ja toimittajat seuraavassa muodossa:
''Teoksessa: \textit{Hieno artikkelikokoelma.} Toim. Kirjailija,
Kaisa''. Ensin siis mainitaan julkaisun nimi ja sen jälkeen toimittajien
nimet. Suomessa on tapana kirjoittaa nämä tiedot toisinpäin ja laittaa
toimittajarooli sulkeisiin. Tällaiset asetukset saa käyttämällä tyyliä
\koodi{ext-author\-year} (tms.), paketin asetusta
\koodi{in\-name\-before\-title=\katk true} ja seuraavia komentoja:

\komentoi{usepackage}
\pakettii{biblatex}
\komentoi{DeclareFieldFormat}
\komentoi{DeclareDelimFormat}
\komentoi{mkbibparens}
\begin{koodilohkosis}
\usepackage[style=ext-authoryear, innamebeforetitle=true]{biblatex}

\DeclareFieldFormat{editortype}{\mkbibparens{#1}}
\DeclareDelimFormat{editortypedelim}{\addspace}
\end{koodilohkosis}

\begin{tulossis}
  -- -- Teoksessa: Kirjailija, Kaisa (toim.). \textit{Hieno
    artikkelikokoelma}. -- --
\end{tulossis}

\noindent
Lähdeviittauksissa vuosiluvun ja sivunumeroiden välissä käytetään
toisinaan pilkkua ja toisinaan kaksoispistettä. Sivunumeroiden
yhteydessä voi olla lyhenne ''s.'' tai se voidaan jättää pois.
Seuraavilla komennoilla vaikutetaan näihin asetuksiin:

\komentoi{DeclareFieldFormat}
\komentoi{DeclareDelimFormat}
\komentoi{textcite}
\begin{koodilohkosis}
\DeclareFieldFormat{postnote}{#1} % Lyhenne ”s.” pois.
\DeclareDelimFormat{postnotedelim}{\addcolon\space} % Kaksoispiste.

\textcite[15--16]{tunniste} toteaa artikkelissaan -- --
\end{koodilohkosis}

\begin{tulossis}
  Meikäläinen (2020: 15--16) toteaa artikkelissaan -- --
\end{tulossis}

\noindent
Kun saman teoksen usean tekijän luettelo lyhennetään, käytetään
oletuksena latinankielistä ilmausta pois jäävien nimien tilalla:
''Meikäläinen et al.''. Ilmauksen voi muuttaa suomenkieliseksi
seuraavalla komennolla:

\komentoi{DefineBibliographyStrings}
\begin{koodilohkosis}
\DefineBibliographyStrings{finnish}{
  andothers = {ym.},
}
\end{koodilohkosis}

\begin{tulossis}
  Meikäläinen ym. (2020)
\end{tulossis}

\noindent
\paketti{biblatex}\-/ paketti sisältää valtavan paljon asetuksia ja
mahdollisuuksia lähdeluettelon ja \=/viitteiden ulkoasun säätämiseen.
Esimerkiksi komennolla \komento{DeclareBibliographyDriver} voi ottaa
täysin haltuun, miten tietty teostyyppi ladotaan lähdeluetteloon.
Komennolla \komento{DeclareSortingTemplate} voi määritellä omia
aakkostustapoja. Lisätietoa saa paketin ohjekirjasta.

\section{Asiahakemistot}
\label{luku/asiasanat}

Asiahakemistot ovat asiasanojen eli tärkeiden käsitteiden tai
henkilöiden luetteloita, ja ne ovat yleensä tietokirjan viimeisillä
sivuilla. Kunkin hakemiston asiasanat ovat aakkosjärjestyksessä, ja
jokaisen sanan perässä luetellaan sivunumerot, joilla kyseistä asiaa
käsitellään. Asiahakemistojen tarkoituksena on helpottaa tiedon
löytämistä. Tämän oppaan luku \enquote{\nameref{luku/asiahakemisto}}
alkaa sivulta \pageref{luku/asiahakemisto}.

Latex tarvitsee asiahakemistojen tekemiseen paketin. Niitä on tehty
useampia, mutta parhaita taitavat olla \pakettictan{indextools} ja
\pakettictan{imakeidx}, joihin tämä opas keskittyy. Paketit ovat
käytännössä lähes samanlaisia: \paketti{indextools} perustuu
\paketti{imakeidx}\-/ pakettiin ja korjaa joitakin sen puutteita.

\begin{esimerkki*}
  \komentoi{documentclass}
  \komentoi{index}
  \komentoi{makeindex}
  \komentoi{printindex}
  \komentoi{usepackage}
  \luokkai{article}
  \pakettii{indextools}
  \ymparistoi{document}

\begin{koodilohko}
\documentclass{article}
\usepackage{indextools}
\makeindex

\begin{document}

\index{erilaisia} \index{asiasanoja}

...

\printindex

\end{document}
\end{koodilohko}
  \caption{Asiahakemistojen toteutuksen perusteet}
  \label{esim/asiasanat-perus}
\end{esimerkki*}

Asiahakemiston toteutuksen perusajatus on yksinkertainen: Lähdetiedoston
alussa komennolla \komento{makeindex} määritellään asiahakemistot.
Dokumentin sisältösivuilla \komento{index}\-/ komennolla lisätään
haluttuja asiasanoja hakemistoihin. Dokumentin lopussa komennolla
\komento{printindex} ladotaan varsinaiset hakemistot. Näitä komentoja
käsitellään tarkemmin erillisissä alaluvuissa, mutta esimerkissä
\ref{esim/asiasanat-perus} ovat perusasiat tiiviisti.

\subsection{Asiahakemistojen määrittely}
\label{luku/asiasanat-määrittely}

Ennen asiasanojen ja \=/hakemistojen käyttöä täytyy hakemistot
määritellä. Yksinkertaisimmillaan määrittelyksi riittää pelkkä yksi
\komento{makeindex}\-/ komento lähdedokumentin esittelyosassa, kuten
esimerkissä \ref{esim/asiasanat-perus} on tehty. Jos dokumenttiin
tarvitaan useampia asiahakemistoja tai jos haluaa vaikuttaa yksittäisten
hakemistojen asetuksiin, täytyy tietää \komento{makeindex}\-/ komennosta
enemmän. Sen sijaan yleisiä, kaikkien hakemistojen latomiseen liittyviä
asetuksia käsitellään luvussa \ref{luku/asiasanat-asetukset}.

\leijutlk{
  \providecommand{\rivi}{}
  \renewcommand{\rivi}[2]{\koodi{#1} & #2 \\}
  \begin{tabularx}{\textwidth}{lL}
    \toprule
    \ots{Valitsin} & \ots{Merkitys} \\
    \midrule
    \rivi{name}{asiahakemiston tekninen nimi}
    \rivi{title}{asiahakemistolle ladottava otsikko}
    \rivi{intoc}{sisällysluetteloon lisääminen}
    \rivi{columns}{asiahakemiston palstojen lukumäärä}
    \rivi{columnsep}{asiahakemiston palstojen väli (mitta)}
    \rivi{columnseprule}{pystyviiva palstojen väliin}
    \rivi{program}{asiahakemiston tekemisestä vastaava ohjelma:
      \koodi{makeindex} (oletus), \koodi{xindy} ym.}
    \rivi{options}{ohjelman komentorivin valitsimia}
    \bottomrule
  \end{tabularx}
}{
  \caption{\komento{makeindex}\-/ komennon valitsimia}
  \label{tlk/makeindex-valitsimia}
}

\komento{makeindex}\-/ komennolle voi antaa yhden valinnaisen
argumentin, joka voi sisältää useita pilkulla toisistaan erotettuja
valitsimia ja niiden arvoja. Valitsimia on koottu taulukkoon
\ref{tlk/makeindex-valitsimia}.

Valitsin \koodi{name} on asiahakemiston tekninen nimi, jota käytetään
Latexissa sisäisesti. Valitsinta ei tarvitse käyttää, jos tarvitaan vain
yksi asiahakemisto. Tällöin nimeksi tulee sama kuin Latex\-/
lähdedokumentin tiedoston nimi. Jos asiahakemistoja tarvitaan useita, on
järkevää nimetä jokainen erikseen \koodi{name}\-/ valitsimen avulla.
Seuraavassa esimerkissä määritellään kolme eri hakemistoa, jotka
voisivat olla Latexin komennoille, mitoille ja laskureille:

\komentoi{makeindex}
\begin{koodilohkosis}
\makeindex[name=komennot, title=Komennot, columns=2, intoc]
\makeindex[name=mitat,    title=Mitat,    columns=2, intoc]
\makeindex[name=laskurit, title=Laskurit, columns=2, intoc]
\end{koodilohkosis}

\noindent
\komento{makeindex}\-/ komennon valitsin \koodi{title} nimeää otsikon
hakemistolle, joka ladotaan \komento{printindex}\-/ komennolla (luku
\ref{luku/asiasanat-asetukset}). Jos \koodi{title}\-/ valitsinta ei ole
annettu, otsikko tulee komennosta \komento{indexname}, joka puolestaan
määräytyy kieliasetusten perusteella (luku \ref{luku/kieliasetukset}).
Se on suomen kieliasetuksilla ''\indexname''.

Jos valitsin \koodi{intoc} on mukana, kyseisen asiahakemiston otsikko
lisätään myös sisällysluetteloon. Valitsimella \koodi{columns} asetetaan
hakemiston palstojen lukumäärää ja valitsimella \koodi{columnsep} mitta,
joka on palstojen välinen tyhjä tila. Palstojen väliin saa pystyviivan
antamalla valitsimen \koodi{columnseprule}. Teknisesti palstat
toteutetaan \paketti{multicol}\-/ paketin avulla, joten sen paketin
mitat ovat käytettävissä myös asiahakemistojen latomisessa. Katso
lisätietoa palstoista ja \paketti{multicol}\-/ paketista luvusta
\ref{luku/multicol}.

Asiahakemiston järjestelystä vastaa erillinen tietokoneohjelma, joka
suoritetaan automaattisesti lähdedokumentin kääntämisen yhteydessä.
Ohjelmalle on olemassa pari vaihtoehtoa, ja sen voi valita valitsimella
\koodi{program}. Tärkeimmät vaihtoehdot ovat
\koodi{makeindex}\footnote{\koodi{program=\katk makeindex} on
  oletusasetus, mutta yleisen oletusasetuksen voi vaihtaa paketin
  lataamisen yhteydessä. Katso luku \ref{luku/asiasanat-asetukset}.} ja
\koodi{xindy}, joka käytännössä suorittaa \koodi{texindy}\-/ nimisen
ohjelman. Näillä ohjelmilla on omat ohjekirjansa, joita pääsee lukemaan
Linux\-/ järjestelmissä \koodi{man}\-/ komennolla. Jos käyttää
vaihtoehtoa \koodi{program=\katk xindy}, täytyy Latexin kääntäjälle
(\koodi{lualatex} tai \koodi{xelatex}) antaa komentorivivalitsin
\koodi{\=/shell-escape}, joka kytkee päälle erään
lisäominaisuuden.\footnote{Lisätietoa \koodi{tex}\-/ komennon
  ohjekirjasta, komennolla \koodi{man tex}.}

Valitsimella \koodi{options} annetaan edellä mainitulle ohjelmalle
komentorivivalitsimet. Tätä tarvitaan lähinnä silloin, kun kirjoittaja
on tehnyt oman tyylitiedoston tai muita erikoisempia asetuksia
asiahakemistolle. Varsinkin \koodi{xindy} on monipuolinen ohjelma, jonka
toimintaan voi vaikuttaa monella tavalla. Lisätietoa saa
\koodi{makeindex}\-/{} ja \koodi{xindy}\-/ ohjelmien ohjekirjoista.

\subsection{Asiasanojen lisääminen}
\label{luku/asiasanat-lisääminen}

Asiasanat lisätään komennolla \komento{index}, joka sijoitetaan
lähdedokumentissa juuri siihen kohtaan, jossa kyseistä asiaa
käsitellään. Komento itsessään ei lado mitään; se vain kirjaa asiasanan
ja sivunumeron muistiin. Komentoa käytetään esimerkiksi seuraavalla
tavalla:

\komentoi{index}
\begin{koodilohkosis}
Yhteen \index{kirjainperhe} kirjainperheeseen kuuluu tavallisesti
useita \index{kirjainleikkaus} leikkauksia.
\end{koodilohkosis}

\noindent
\komento{index}\-/ komennolle voi antaa yhden valinnaisen argumentin.
Sitä tarvitaan silloin, kun asiasana halutaan lisätä tiettyyn,
määrittelyn yhteydessä \koodi{name}\-/ valitsimella nimettyyn
hakemistoon (luku \ref{luku/asiasanat-määrittely}). Jos esimerkiksi
dokumentissa on määritelty asiahakemisto
\komento{makeindex}\komentoargv{name=\katk henkilöt}, lisättäisiin
asiasana kyseiseen hakemistoon seuraavasti:

\begin{koodilohkosis}
\index[henkilöt]{Waltari, Mika}
\end{koodilohkosis}

\noindent
Tämän luvun muissa esimerkeissä ei yleensä käytetä \komento{index}\-/
komennon valinnaista argumenttia. Argumentti on kuitenkin tarpeen aina,
kun halutaan lisätä asiasanoja tiettyyn, määrittelyn yhteydessä
nimettyyn hakemistoon.

Asiahakemistot aakkostetaan automaattisesti, mutta yksittäisten sanojen
aakkostustapaan voi vaikuttaa käyttämällä \koodi{@}\=/ merkkiä
\komento{index}\-/ komennon argumentissa. Ennen \koodi{@}\=/ merkkiä on
asiasana siinä muodossa kuin se aakkostetaan; \koodi{@}\=/ merkin
jälkeen annetaan asiasana siinä muodossa kuin se ladotaan
asiahakemistoon.

Seuraavassa esimerkissä asiahakemistoon lisätään kolme eri henkilöä,
joiden sukunimet ovat joko Vuori tai Wuori. Kaikki nimet aakkostetaan
nimen Vuori mukaisesti, eli \emph{v} ja \emph{w} katsotaan aakkostuksen
kannalta samaksi kirjaimeksi. Tämä on tavallinen käytäntö suomalaisten
sanojen ja nimien aakkostuksessa.

\komentoi{index}
\begin{koodilohkosis}
\index{Vuori, Lauri}
\index{Vuori, Anita@Wuori, Anita}
\index{Vuori, Yrjö@Wuori, Yrjö}
\end{koodilohkosis}

\begin{tulossis}
  Wuori, Anita, \arabic{page} \\*
  Vuori, Lauri, \arabic{page} \\*
  Wuori, Yrjö, \arabic{page}
\end{tulossis}

\noindent
Edellä mainittua aakkostustoimintoa tarvitaan myös silloin, kun
hakemiston sanojen latomisessa käytetään Latex\-/ komentoja eli asiasana
ei ole pelkkää tekstiä. Seuraavassa esimerkissä asiahakemistoon lisätään
Latex\-/ komento \komento{textbullet}. Se aakkostetaan sanan
\emph{\englanti{textbullet}} mukaan mutta hakemistoon se ladotaan
tasalevyisellä fontilla (\komento{texttt}) ja alkuun lisätään kenoviiva
(\komento{textbackslash}).

\komentoi{index}
\komentoi{texttt}
\komentoi{textbackslash}
\begin{koodilohkosis}
\index{textbullet@\texttt{\textbackslash textbullet}}
\end{koodilohkosis}

\noindent
Käytännössä \komento{index}\-/ komentoja ei kannata aina kirjoittaa
lähdedokumenttiin sellaisenaan, vaan komennon voi sisällyttää jonkin
toisen komennon määritelmään. Tästä on esimerkki luvussa
\ref{luku/komennot-abst}.

Asiasanat voi järjestää hakemistoon hierarkkisesti eli aihepiirien
mukaisesti. Se toteutetaan \komento{index}\-/ komennon argumentissa
\koodi{!}\=/ merkin avulla, jolla erotetaan ylemmäntasoiset käsitteet
alemmantasoisista. Seuraava esimerkki lisää asiahakemistoon sanan
\emph{fontit} ja sen alle kaksi asiasanaa sivunumeroineen:

\komentoi{index}
\begin{koodilohkosis}
\index{fontit!kirjainperhe}
\index{fontit!kirjainleikkaus}
\end{koodilohkosis}

\begin{tulossis}
  fontit \\*
  \makebox[\sisennys]{}kirjainleikkaus, \arabic{page} \\*
  \makebox[\sisennys]{}kirjainperhe, \arabic{page}
\end{tulossis}

\noindent
Asiasanahierarkiassa voi olla korkeintaan kolme tasoa, ja tasot
erotetaan toisistaan \koodi{!}\=/ merkillä. Jokainen \koodi{!}\=/
merkillä erotettu osa voi sisältää \koodi{@}\=/ merkin, jolla voi
vaikuttaa sanan aakkostamiseen: ensin mainitaan hakusanasta
aakkostamisessa käytetty versio ja \koodi{@}\=/ merkin jälkeen annetaan
hakemistoon ladottava versio.

Kirjoittaja voi vaikuttaa asiasanojen sivunumeroiden latomiseen
käyttämällä \komento{index}\-/ komennon argumentissa asiasanan perässä
\koodi{|}\=/ merkkiä ja antamalla sen jälkeen komennon nimen.
Seuraavassa esimerkissä asiasanan sivunumero ladotaan kursiivilla:

\komentoi{index}
\begin{koodilohkosis}
\index{asiasana|textit}
\end{koodilohkosis}

\begin{tulossis}
  asiasana, \textit{\arabic{page}}
\end{tulossis}

\noindent
\komento{index}\-/ komennon argumenttiin ei kirjoiteta varsinaista
komentoa (\komento{textit}) vaan ainoastaan komennon nimi \koodi{|}\=/
merkin jälkeen. Komento voi olla mikä hyvänsä ja sen pitäisi hyväksyä
ainakin yksi argumentti: sivunumero. Asiahakemistoissa on joskus tapana
merkitä asiasanan pääasiallinen käsittelysivu eri tavalla, esimerkiksi
kursiivilla tai lihavoinnilla, ja tämä \koodi{|}\=/ ominaisuus sopii
siihen.

Yksi hyödyllinen käyttökohde \koodi{|}\=/ ominaisuudelle on viittaus
toiseen asiasanaan. On määritelty jo valmiiksi komennot \komento{see} ja
\komento{seealso}, joita voi käyttää \komento{index}\-/ komennossa
seuraavasti:

\komentoi{index}
\begin{koodilohkosis}
\index{fontit}
\index{kirjainleikkaus|see{fontit}}
\index{kirjainperhe|seealso{fontit}}
\end{koodilohkosis}

\begin{tulossis}
  fontit, \arabic{page} \\*
  kirjainleikkaus, \textit{\seename} fontit \\*
  kirjainperhe, \textit{\alsoname} fontit
\end{tulossis}

\noindent
Molemmat komennot hyväksyvät kaksi argumenttia, mutta edellisessä
esimerkissä annettiin vain yksi: \koodi{fontit}. Viimeiseksi
argumentiksi tulee automaattisesti aina sivunumero, mutta nämä komennot
jättävät sen huomioimatta. Sivunumeroa ei haluta näkyviin, koska
tarkoitus on vain lisätä asiahakemistoon viittaus toiseen
asiasanaan.

Suomen kieliasetuksilla edellä mainitut komennot latovat kursivoidun
ilmauksen \textit{\seename} tai \textit{\alsoname}. Kielikohtaisen
määritelmän voi muuttaa kielipaketteihin \paketti{polyglossia} ja
\paketti{babel} kuuluvan \komento{addto}\-/ komennon avulla. Seuraavassa
esimerkissä vaikutetaan suomen kieliasetuksiin
(\komento{captionsfinnish}). Komennot täytyy sijoittaa dokumentin
esittelyosaan:

\komentoi{addto}
\komentoi{captionsfinnish}
\komentoi{renewcommand}
\komentoi{seename}
\komentoi{alsoname}
\begin{koodilohkosis}
\addto{\captionsfinnish}{
  \renewcommand{\seename}{katso}
  \renewcommand{\alsoname}{katso myös}
}
\end{koodilohkosis}

\noindent
Toisaalta kirjoittaja voi määritellä ihan oman komentonsa
asiasanaviittausten tekemiseen. Oman komennon avulla lopputulokseen voi
vaikuttaa enemmän. Esimerkissä \ref{esim/see-katso-komento} määritellään
komento \komentox{katso}, jolla voi toteuttaa viittauksia toisiin
asiasanoihin. Komento hyväksyy kaksi argumenttia mutta jälkimmäinen eli
sivunumero jätetään käsittelemättä.

\begin{esimerkki*}
  \komentoi{newcommand}
  \komentoi{textrightarrow}
  \komentoi{index}

\begin{koodilohko}
\newcommand{\katso}[2]{\textrightarrow\ #1}

% Seuraavat komennot tarvitaan vain kerran dokumentissa, koska ne vain
% viittaavat toisiin asiasanoihin.
\index{kirjainleikkaus|katso{fontit}}
\index{kirjainperhe|katso{fontit}}

% Seuraavassa on esimerkki varsinaisista asiasanoista. Komennot
% sijoitetaan paikkoihin, joissa aihetta käsitellään.
\index{fontit!kirjainleikkaus}
\index{fontit!kirjainperhe}
\end{koodilohko}

\begin{tulos}
  fontit \\*
  \makebox[\sisennys]{}kirjainleikkaus, \arabic{page} \\*
  \makebox[\sisennys]{}kirjainperhe, \arabic{page} \\*
  kirjainleikkaus, \textrightarrow\ fontit \\*
  kirjainperhe, \textrightarrow\ fontit
\end{tulos}

\caption{Oman \komentox{katso}\-/ komennon määrittely. Komennon nimeä
  käytetään \komento{index}\-/ komennon argumentissa ja sillä viitataan
  toisiin asiasanoihin}
\label{esim/see-katso-komento}
\end{esimerkki*}

Tavallisesti \komento{index} kirjaa muistiin vain sen sivun, jolla
komento esiintyy. Jos komento esiintyy usealla peräkkäisellä sivulla,
hakemistoon ladotaan automaattisesti sivualue, esimerkiksi 53--55.

Aina ei kuitenkaan ole mielekästä ilmaista aiheen käsittelyä
yksittäisten \komento{index}\-/ komentojen avulla. Toisinaan voi olla
sopivampaa määrittää aiheen käsittelyn alku- ja loppukohta. Alkukohta
ilmaistaan \komento{index}\-/ komennon argumentissa merkkien \koodi{|(}
avulla ja loppukohta merkkien \koodi{|)} avulla. Jos aloitus ja lopetus
sattuvat samalle sivulle, hakemistoon ladotaan vain yksi sivunumero.
Muussa tapauksessa ladotaan sivualue. Seuraavassa on esimerkki aiheen
käsittelyn alun ja lopun merkitsemiseen:

\komentoi{index}
\begin{koodilohkosis}
\index{asiasana|(}  % aiheen käsittely alkaa
...
\index{asiasana|)}  % aiheen käsittely loppuu
\end{koodilohkosis}

\begin{tulossis}
  asiasana, 53--55
\end{tulossis}

\noindent
Edellä mainittujen aloitusmerkkien \koodi{|(} perään voi kirjoittaa
komennon nimen, jos haluaa vaikuttaa sivunumeroiden latomiseen. Tätä
käsiteltiin jo edellä, mutta seuraavassa on käytännön esimerkki:

\komentoi{index}
\begin{koodilohkosis}
\index{asiasana|(textit}  % aiheen käsittely alkaa
...
\index{asiasana|)}        % aiheen käsittely loppuu
\end{koodilohkosis}

\begin{tulossis}
  asiasana, \textit{53--55}
\end{tulossis}

\noindent
Jos itse asiasanaan täytyy sisällyttää edellä mainittuja
\komento{index}\-/ komennon erikoismerkkejä (\koodi{@!|}), pitää niiden
eteen kirjoittaa yleislainausmerkki (\koodi{\textquotedbl}). Se estää
seuraavan merkin tulkitsemisen \komento{index}\-/ komennon
erikoismerkiksi.

\subsection{Hakemiston latominen ja asetukset}
\label{luku/asiasanat-asetukset}

Hakemistot ladotaan komennolla \komento{printindex}. Tyypillisesti
komento sijaitsee dokumentin lopussa eli esimerkiksi artikkelin tai
tietokirjan viimeisillä sivuilla. Jos komennon suorittaa ilman
argumentteja, se latoo oletushakemiston eli sen, jolle ei ole annettu
teknistä nimeä \komento{makeindex}\-/ komennon \koodi{name}\-/
valitsimella (luku \ref{luku/asiasanat-määrittely}). Komennolle voi
antaa hakasulkeissa yhden valinnaisen argumentin, joka ilmaisee
ladottavan hakemiston teknisen nimen, esimerkiksi:

\komentoi{printindex}
\begin{koodilohkosis}
\printindex            % Latoo nimeämättömän hakemiston.
\printindex[henkilöt]  % Latoo hakemiston nimeltä ”henkilöt”.
\end{koodilohkosis}

\noindent
Luvussa luku \ref{luku/asiasanat-määrittely} käsitellään yksittäisten
hakemistojen asetuksia, mutta \komento{indexsetup}\-/ komennolla voi
määrittää kaikkia hakemistoja koskevia yleisempiä asetuksia. Komennolle
annetaan yksi argumentti, joka voi sisältää useita pilkuilla erotettuja
valitsimia ja niiden arvoja. Valitsimia on koottu taulukkoon
\ref{tlk/indexsetup-valitsimia}, ja seuraavassa on käytännön esimerkki:

\begin{koodilohkosis}
\indexsetup{level=\section*, noclearpage}
\end{koodilohkosis}

\leijutlk{
  \providecommand{\rivi}{}
  \renewcommand{\rivi}[2]{\koodi{#1} & #2 \\}
  \begin{tabularx}{\textwidth}{lL}
    \toprule
    \ots{Valitsin} & \ots{Merkitys ja vaihtoehtoja} \\
    \midrule
    \rivi{level}{Hakemistojen otsikkokomento: \komento{chapter*},
      \komento{chapter}, \komento{section*}, \komento{section} ym.}
    \rivi{toclevel}{Hakemiston taso sisällysluettelossa:
      \koodi{chapter}, \koodi{section} ym.}
    \rivi{noclearpage}{Ei automaattista sivunvaihtoa hakemistojen
      alussa.}
    \rivi{firstpagestyle}{Hakemiston ensimmäisen sivun sivutyyli:
      \koodi{plain}, \koodi{empty} ym.}
    \bottomrule
  \end{tabularx}
}{
  \caption{\komento{indexsetup}\-/ komennon valitsimia, joilla
    vaikutetaan kaikkien asiahakemistojen latomiseen}
  \label{tlk/indexsetup-valitsimia}
}

\noindent
Valitsimen \koodi{level} arvoksi annetaan tyypillisesti otsikkokomento,
joka suoritetaan kunkin hakemiston alussa. Itse otsikon tekstin voi
määritellä \komento{makeindex}\-/ komennon \koodi{title}\-/ valitsimella
kullekin hakemistolle erikseen. \koodi{level}\-/ valitsimelle oletusarvo
on dokumenttiluokkakohtainen korkein otsikkotaso eli \komento{chapter*}
tai \komento{section*}. Jos arvoksi antaa tähdettömän otsikkokomennon
kuten \komento{chapter}, se luo automaattisesti merkinnän
sisällysluetteloon. Silloin ei kannata käyttää \komento{makeindex}\-/
komennon \koodi{intoc}\-/ valitsinta.

Käytännössä \koodi{level}\-/ valitsimen arvoksi voi antaa minkä hyvänsä
komennon, joka hyväksyy yhden argumentin eli kyseisen hakemiston
otsikkotekstin. Kirjoittaja voi siis asettaa tähän oman komentonsa,
jonka määritelmässä mahdollisesti suoritetaan muutakin kuin pelkkä
otsikkokomento. \koodi{toclevel}\-/ valitsimella ilmaistaan hakemistojen
taso sisällysluettelossa. Valitsimen arvoksi annetaan otsikkokomentojen
nimi kuten \koodi{section} tai \koodi{subsection}.

Oletuksena jokaisen asiahakemiston alussa suoritetaan sivunvaihtokomento
\komento{clearpage}. Käyttämällä valitsinta \koodi{noclearpage} estetään
automaattinen sivunvaihto. Valitsimella \koodi{firstpagestyle} asetetaan
kunkin asiahakemiston ensimmäisen sivun sivutyyli (luku
\ref{luku/ylä-ala-tunnisteet}). Oletusarvo on \koodi{plain} eli
tavallinen, oletuksena pelkän sivunumeron sisältävä sivutyyli. Tämä
valitsin ei ole käytössä, jos ladattuna on paketti \paketti{fancyhdr}.

Ennen \komento{printindex}\-/ komentoa voi käyttää komentoa
\komento{indexprologue}, jolla määritetään johdantoteksti asiahakemiston
alkuun. Komento vaikuttaa seuraavaksi ladottavaan asiahakemistoon, eli
se kannattaa sijoittaa juuri ennen \komento{printindex}\-/ komentoa.
Komentoa käytetään seuraavasti:

\komentoi{indexprologue}
\begin{koodilohkosis}
\indexprologue[väli]{teksti}
\end{koodilohkosis}

\noindent
Pakollinen argumentti \koodi{teksti} on hakemiston johdannoksi
tarkoitettu teksti. Valinnainen argumentti \koodi{väli} on tarkoitettu
komennolle, joka latoo pystysuuntaisen välin johdantotekstin jälkeen,
ennen asiasanojen luetteloa. Oletuksena väliksi tulee komennon
\komento{bigskip} latoma väli, mutta muunlaisen välin saa kirjoittamalla
valinnaiseen argumenttiin esimerkiksi \komento{vspace}\-/ komennon ja
sen argumentiksi halutun mitan.

Asiahakemistopaketin (\paketti{indextools} tai \paketti{imakeidx})
lataamisen yhteydessä voi määrittää asetuksia tiettyjen valitsimien
avulla. Asetusvalitsimet kirjoitetaan \komento{usepackage}\-/ komennon
valinnaiseen argumenttiin ja ne erotetaan toisistaan pilkulla.

Esimerkiksi valitsin \koodi{xindy} aiheuttaa sen, että
\komento{makeindex}\-/ komennoilla määritellyille asiahakemistoille
tulee oletuksena asetus \koodi{program=\katk xindy}. Muutoin oletus on
\koodi{program=\katk makeindex} (luku \ref{luku/asiasanat-määrittely}).

\komentoi{usepackage}
\pakettii{indextools}
\begin{koodilohkosis}
\usepackage[xindy]{indextools}
\end{koodilohkosis}

\noindent
Jos omassa dokumentissa tarvitaan useita asiahakemistoja, saatetaan
joskus tarvita valitsinta \koodi{splitindex}. Se kiertää erästä Latexin
tiedostojen käsittelyn puutetta. Nimittäin Latex ei voi kääntämisen
yhteydessä kirjoittaa kovin moneen tiedostoon, ja jokainen asiahakemisto
tarvitsisi omansa. Rajoitukset saattavat tulla vastaan useiden
asiahakemistojen käsittelyssä.

Käyttämällä valitsinta \koodi{splitindex} kierretään edellä mainittua
rajoitusta siten, että ensin kaikki asiahakemistot kirjoitetaan yhteen
väliaikaistiedostoon ja myöhemmin ne erotetaan automaattisesti
toisistaan suorittamalla \koodi{splitindex}\-/ niminen ohjelma. Tämän
valitsimen käyttö vaatii, että Latexin kääntäjäohjelmalle
(\koodi{lualatex} tai \koodi{xelatex}) annetaan komentorivivalitsin
\koodi{\=/shell-escape}.

\section{Kuvat}
\label{luku/grafiikka}

Perus Latex ei pysty juuri minkäänlaiseen kuvien eli tietokonegrafiikan
käsittelyyn, mutta avuksi on tehty monipuolisia paketteja. Niiden avulla
voidaan esimerkiksi ladata ja latoa kuvatiedostoja sekä piirtää
vektorigrafiikkakuvia Latex\-/ komentojen avulla.

\subsection{Kuvatiedostot}

Paketti \pakettictan{graphicx} sisältää toiminnot kuvatiedostojen
käsittelyyn sekä tekstin tai muun sisällön kääntelyyn ja skaalaamisen.
Kuvatiedostojen peruskäyttö sujuu komennolla \komento{includegraphics}
seuraavasti:

\komentoi{includegraphics}
\begin{koodilohkosis}
\includegraphics[width=6cm]{kuvatiedosto.jpg}
\end{koodilohkosis}

\noindent
Tällä tavoin lisätty kuvatiedosto käyttäytyy tekstikappaleessa kuin mikä
tahansa kirjain tai laatikko. Edellisen esimerkin pakollisessa
argumentissa oleva \koodi{kuvatiedosto.\katk jpg} on tiedoston nimi. Se
voi sisältää myös hakemistopolun, jos tiedosto ei sijaitse samassa
hakemistossa Latex\-/ lähdetiedostojen kanssa. Kaikki käyttöjärjestelmän
hyväksymät tiedostonnimet eivät Latexissa toimi, vaan kannattaa jättää
välilyönnit pois ja pitäytyä suppeassa latinalaisessa
merkkivalikoimassa.

\leijutlk{
  \providecommand{\rivi}{}
  \renewcommand{\rivi}[2]{\koodi{#1} & #2 \\}
  \begin{tabularx}{\textwidth}{lL}
    \toprule
    \ots{Valitsin} & \ots{Merkitys} \\
    \midrule
    \rivi{width}{leveysmitta}
    \rivi{height}{korkeusmitta}
    \rivi{angle}{kääntökulma asteina}
    \rivi{origin}{kääntämisen keskipiste:
      \koodi{c} (keskikohta),
      \koodi{l} (vasen),
      \koodi{r} (oikea),
      \koodi{t} (ylä),
      \koodi{b} (ala),
      \koodi{B} (rivin peruslinja)
    }

    \rivi{trim}{reunojen siirto: vasen, ala, oikea, ylä}
    \rivi{clip}{leikkaminen reunojen ulkopuolelta}
    \bottomrule
  \end{tabularx}
}{
  \caption{\komento{includegraphics}\-/ komennon valinnaiseen
    argumenttiin sopivia valitsimia}
  \label{tlk/includegraphics-valitsimia}
}

\komento{includegraphics}\-/ komennon valinnaiseen argumenttiin voi
kirjoittaa monenlaisia pilkulla toisistaan erotettuja valitsimia ja
niiden arvoja. Joitakin valitsimia on koottu taulukkoon
\ref{tlk/includegraphics-valitsimia}. Tavallisimpia ovat varmaankin
\koodi{width} ja \koodi{height}, joilla määritetään, minkä levyisenä tai
korkuisena kuva halutaan latoa. Näistä voi asettaa vain jommankumman, ja
toinen mitta määräytyy alkuperäiskuvan mittasuhteiden perusteella.
Asettamalla sekä leveyden että korkeuden kuva skaalataan juuri kyseisiin
mittoihin.

Kuvan kääntämisessä käytetään \koodi{angle}\-/ valitsinta ja astelukua.
Kuva kääntyy \koodi{origin}\-/ valitsimella määritetyn pisteen kautta,
ja valitsimen arvoksi annetaan taulukossa
\ref{tlk/includegraphics-valitsimia} mainittuja kirjaimia tai niiden
yhdistelmiä. Esimerkiksi \koodi{origin=\katk lt} tarkoittaa vasenta
(\koodi{l}) ylänurkkaa (\koodi{t}).

Kuvaa voi rajata käyttämällä yhdessä valitsimia \koodi{trim} ja
\koodi{clip}. \koodi{trim}\-/ valitsimen arvoksi annetaan neljä
välilyönnillä erotettua mittaa, joiden järjestys on seuraava: vasen,
ala, oikea, ylä. Tämä valitsin vain siirtää kuvan reunoja sisäänpäin,
mutta kun mukaan ottaa valitsimen \koodi{clip}, kuvasta myös leikataan
reunojen ulkopuolinen alue pois. Seuraavassa esimerkissä kuvaa leikataan
vasemmalta 1\,mm, alhaalta 2\,mm, oikealta 3\,mm ja ylhäältä 4\,mm.
Kuvan lopulliseksi leveydeksi tulee 40\,mm.

\komentoi{includegraphics}
\begin{koodilohkosis}
\includegraphics[width=40mm,
  clip, trim={1mm 2mm 3mm 4mm}]{kuvatiedosto.jpg}
\end{koodilohkosis}

\subsection{Vektorigrafiikka}

Paketti \pakettictan{tikz} tuo Latexiin monipuolisen apukielen, jolla
kirjoittaja voi piirtää vektorigrafiikkaa. Paketin taustalla toimii
toinen paketti ja grafiikkajärjestelmä \paketti{pgf} eli
\englantik{portable graphics format}, mutta kirjoittajan ei yleensä
tarvitse välittää taustalla olevasta matalan tason tekniikasta.

Tässä oppaassa käsitellään vain joitakin \paketti{tikz}\-/ paketin
perustoimintoja. Paketin omassa ohjekirjassa on yli 1\,300 sivua, eli
valtavan paljon hienoja ominaisuuksia jää lukijan itse selvitettäväksi.
Runsasta sivumäärää ei kannata säikähtää: ei opasta tarvitse kokonaan
lukea, ja alkupuolella teksti johdattelee lukijaa eteenpäin helppojen ja
käytännöllisten esimerkkien avulla.

Vektorigrafiikka toteutetaan ympäristön \ymparisto{tikzpicture} sisällä.
Se ei ole tavallinen Latexin ympäristö, vaan se on tarkoitettu vain
piirtokomennoille, joilla on oma, muusta Latexista hieman poikkeava
kielioppinsa. Seuraavassa on pieni esimerkki:

\ymparistoi{tikzpicture}
\komentoi{draw}
\begin{koodilohkosis}
\begin{tikzpicture}[x=1mm, y=1mm]
  \draw (0,0) -- (25,5) -- (50,0) -- cycle;
  \draw (25,0) -- (25,5);
\end{tikzpicture}
\end{koodilohkosis}

\begin{tulossis}
  \begin{tikzpicture}[x=1mm, y=1mm]
    \draw (0,0) -- (25,5) -- (50,0) -- cycle;
    \draw (25,0) -- (25,5);
  \end{tikzpicture}
\end{tulossis}

\noindent
Edellisessä esimerkissä ympäristön alussa määritellään hakasulkeissa
perusasetuksia. Tässä asetetaan x- ja y\=/ akselien suuntaiset
yksikkömitat 1\,mm:n pituiseksi. Piirtokomentojen koordinaatit käyttävät
sen jälkeen näitä yksiköitä. Jos yksiköitä ei määritellä, käytetään
oletusyksikköä, joka on 10\,mm. On myös mahdollista antaa
piirtokomentojen koordinaateissa suoraan Latexin mittayksiköt,
esimerkiksi \koodi{(25mm,5mm)} tai \koodi{(10bp,20bp)}, jolloin
koordinaatit tulevat juuri näiden mittojen mukaiseksi. Piirustusalustan
origo eli koordinaatti (0,~0) sijaitsee vasemmassa alanurkassa, ja
koordinaatisto kasvaa ylös ja oikealle.

Edellä olevassa esimerkissä piirretään komennolla \komento{draw}
viivakuvio, jonka välipisteiden koordinaatit ilmaistaan sulkeissa.
Lopussa oleva \koodi{cycle} piirtää viivan takaisin saman
\komento{draw}\-/ komennon alkupisteeseen. Tätä kokonaisuutta kutsutaan
poluksi, ja sen lopussa täytyy olla puolipiste
(\koodi{;}).\footnote{Polkujen luomisen peruskomento on \komento{path},
  joka ei piirrä mitään, ellei sille anna valinnaista argumenttia
  \koodi{draw}. Komento \komento{draw} on itse asiassa sama kuin
  \komento{path}\komentoargv{draw}.} Esimerkissä on toinenkin polku eli
toinen \komento{draw}\-/ komento, joka piirtää pystyviivan suuren
kolmion keskelle ja jakaa sen kahdeksi pienemmäksi kolmioksi.

Koordinaatit voi ilmaista suhteessa polun edelliseen pisteeseen
lisäämällä koordinaattisulkeiden eteen kaksi plusmerkkiä. Seuraavassa
esimerkissä havainnollistetaan sitä. Kuvion (polun) neljä ensimmäistä
pistettä ilmaistaan x- ja y\=/ koordinaateilla mutta viimeinen piste
ilmaistaan kulman (25°) ja pituuden (8) avulla.

\ymparistoi{tikzpicture}
\komentoi{draw}
\begin{koodilohkosis}
\begin{tikzpicture}[x=1mm, y=1mm]
  \draw (0,0) -- ++(0,5) -- ++(10,0) -- ++(0,-5) -- ++(25:8);
\end{tikzpicture}
\end{koodilohkosis}

\begin{tulossis}
  \begin{tikzpicture}[x=1mm, y=1mm]
    \draw (0,0) -- ++(0,5) -- ++(10,0) -- ++(0,-5) -- ++(25:8);
  \end{tikzpicture}
\end{tulossis}

\noindent
Viivojen lisäksi on olemassa komentoja muillekin geometrisille
peruskuvioille. Esimerkissä \ref{esim/tikz-kuvioita} piirretään
suorakulmio, ympyrä, ellipsi, sektorin kaari (0--90°) ja taivutettu
viiva. Viimeksi mainittu (rivi~6) sisältää alkupisteen, kaksi
näkymätöntä ohjauspistettä (\koodi{controls}, \koodi{and}), joiden
suuntaan viivaa taivutetaan, sekä loppupisteen.

\begin{esimerkki*}
  \ymparistoi{tikzpicture}
  \komentoi{draw}

\begin{koodilohko}
\begin{tikzpicture}[x=1mm, y=1mm]
  \draw (0,0) rectangle (10,5);
  \draw (20,2.5) circle [radius=2.5];
  \draw (35,2.5) ellipse [x radius=4, y radius=2, rotate=30];
  \draw (55,0) arc [start angle=0, end angle=90, radius=5];
  \draw (65,0) .. controls (75,7) and (80,7) .. (80,0);
\end{tikzpicture}
\end{koodilohko}

  \begin{tulos}
    \begin{tikzpicture}[x=1mm, y=1mm]
      \draw (0,0) rectangle (10,5);
      \draw (20,2.5) circle [radius=2.5];
      \draw (35,2.5) ellipse [x radius=4, y radius=2, rotate=30];
      \draw (55,0) arc [start angle=0, end angle=90, radius=5];
      \draw (65,0) .. controls (75,7) and (80,7) .. (80,0);
    \end{tikzpicture}
  \end{tulos}

  \caption{Erilaisia geometrisia kuvioita: suorakulmio, ympyrä, ellipsi,
    ympyrän kaari ja taivutettu viiva}
  \label{esim/tikz-kuvioita}
\end{esimerkki*}

Kuten esimerkki \ref{esim/tikz-kuvioita} osoittaa, piirtokomennoille voi
antaa hakasulkeissa lisätietoja. Mahdollisuuksia on valtavan paljon:
esimerkiksi värit, viivojen paksuus, nuolenkärjet ja kulmien pyöristys
ilmaistaan tällaisten lisätietojen avulla. Esimerkki
\ref{esim/tikz-asetuksia} havainnollistaa tavallisimpia valinnaisia
argumentteja. Värivalitsimien \koodi{draw} ja \koodi{fill} arvoksi voi
antaa mitä hyvänsä nimettyjä värejä, joita käsitellään tarkemmin luvussa
\ref{luku/värit}.

\begin{esimerkki*}
  \ymparistoi{tikzpicture}
  \komentoi{draw}

\begin{koodilohko}
\begin{tikzpicture}[x=1mm, y=1mm,
  nuoli/.style={line width=1bp, draw=red}]

  % Nuolia
  \draw [nuoli, ->] (0,0) -- (10,5);
  \draw [nuoli, <<-, draw=blue] (10,0) -- (20,5);
  \draw [nuoli, <->] (20,0) -- (30,5);

  % Täytetty nelikulmio, viivan paksuus ja kulmien pyöristys
  \draw [draw=black, fill=yellow, line width=1.5bp, rounded
    corners=3bp] (40,0) -- ++(3,5) -- ++(20,0) -- ++(3,-5) -- cycle;
\end{tikzpicture}
\end{koodilohko}

  \begin{tulos}
    \begin{tikzpicture}[x=1mm, y=1mm,
      nuoli/.style={line width=1bp, draw=red}]

      % Nuolia
      \draw [nuoli, ->] (0,0) -- (10,5);
      \draw [nuoli, <<-, draw=blue] (10,0) -- (20,5);
      \draw [nuoli, <->] (20,0) -- (30,5);

      % Täytetty nelikulmio, viivan paksuus ja kulmien pyöristys
      \draw [draw=black, fill=yellow, line width=1.5bp, rounded
      corners=3bp] (40,0) -- ++(3,5) -- ++(20,0) -- ++(3,-5) -- cycle;
    \end{tikzpicture}
  \end{tulos}

\caption{Erilaisia piirtokomentojen argumentteja: nuolenkärjet, värit,
  viivan paksuus ja kulmien pyöristys}
\label{esim/tikz-asetuksia}
\end{esimerkki*}

Esimerkissä \ref{esim/tikz-asetuksia} olevan \ymparisto{tikzpicture}\-/
ympäristön valinnaisessa argumentissa määritellään rivillä~2 oma
tyyliasetus nimeltä \koodi{nuoli}. Asetukseen kuuluu viivan leveys
(\koodi{line width}) ja väri (\koodi{draw}). Näin määriteltyä
\koodi{nuoli}\-/ tyyliä käytetään rivien 5\==7 \komento{draw}\-/
komennoissa. Tällä tavoin omia tyylejä määrittelemällä säästää
todennäköisesti aikaa ja vaivaa, koska tyyliasetuksia voi sen jälkeen
muuttaa yhdestä paikasta.

Usein tarvittavat \komento{draw}\-/ polut ja komentosarjat kannattanee
piilottaa omien komentojen sisään. Kirjoittaja voi siis tehdä
\komento{newcommand}\-/komennolla (luku \ref{luku/komennot}) ja muilla
vastaavilla omia, mahdollisesti kokeamman tason piirtokomentoja, jotka
piirtävät usein tarvittavia kuvan osia.

\begin{esimerkki*}
  \ymparistoi{tikzpicture}
  \komentoi{node}

\begin{koodilohko}
\begin{tikzpicture}[x=1mm, y=1mm]
  \node at (0,0) [draw=red, rectangle, rounded corners=3bp] {vasen};
  \node [color=blue] at (20,0) {keski};
  \node at (40,0) [draw, circle, inner sep=0bp, fill=yellow] {oikea};
\end{tikzpicture}
\end{koodilohko}

  \begin{tulos}
    \begin{tikzpicture}[x=1mm, y=1mm]
      \node at (0,0) [draw=red, rectangle, rounded corners=3bp] {vasen};
      \node [color=blue] at (20,0) {keski};
      \node at (40,0) [draw, circle, inner sep=0bp, fill=yellow] {oikea};
    \end{tikzpicture}
  \end{tulos}

  \caption{Solmuja tehdään \komento{node}\-/ komennolla}
  \label{esim/tikz-solmut}
\end{esimerkki*}

Kuvaan voi sisällyttää myös tekstiä tai muuta Latexin sisältöä. Se
tapahtuu helpoimmin komennolla \komento{node}, joka tekee niin sanottuja
solmuja. Komento tarvitsee ainakin solmun sijaintikoordinaatit ja
ladottavan sisällön. Esimerkissä \ref{esim/tikz-solmut} on kolme
erityyppistä solmua.

Kuten esimerkistä näkyy, solmulle voi määrittää kehykset ja niiden
värin, (\koodi{draw}), täyttövärin (\koodi{fill}), sisällön värin
(\koodi{color}), kehyksen kulmien pyöristyksen (\koodi{\englanti{rounded
    corners}}) ja sisäisen tyhjän tilan suuruuden (\koodi{inner sep}).
Kaikenlaista muutakin voi tehdä valinnaiseen argumenttiin
sisällytettävien valitsimien avulla, mutta niistä täytyy lukea lisää
paketin omasta ohjekirjasta.

\begin{esimerkki*}
  \ymparistoi{tikzpicture}
  \komentoi{node}
  \komentoi{draw}

\begin{koodilohko}
\begin{tikzpicture}[x=1mm, y=1mm]
  \node (ympyrä)      at  (0,0) [draw, circle]    {vasen};
  \node (suorakulmio) at (20,0) [draw, rectangle] {oikea};
  \draw [->, shorten >=1mm] (ympyrä) to [out=45, in=135] (suorakulmio);
\end{tikzpicture}
\end{koodilohko}

  \begin{tulos}
    \begin{tikzpicture}[x=1mm, y=1mm]
      \node (ympyrä) at (0,0) [draw, circle] {vasen};
      \node (suorakulmio) at (20,0) [draw, rectangle] {oikea};
      \draw [->, shorten >=1mm] (ympyrä) to [out=45, in=135] (suorakulmio);
    \end{tikzpicture}
  \end{tulos}

  \caption{Solmujen nimeäminen ja kytkeminen viivan avulla}
  \label{esim/tikz-solmujen-yhdistäminen}
\end{esimerkki*}

Solmulle voi määrittää yksilöllisen nimen, jota voi sitten käyttää
esimerkiksi viivojen piirtämiseen solmujen välille. Ei siis tarvitse
käyttää tavallisia \komento{draw}\-/ komennon koordinaatteja vaan voi
käyttää solmulle annettua nimeä. Tätä havainnollistetaan esimerkissä
\ref{esim/tikz-solmujen-yhdistäminen}, jossa esitellään myös
\komento{draw}\-/ komennon \koodi{to}\-/ operaatio. Se mahdollistaa
solmujen välisen viivan lähtö- ja tulokulman valinnan valitsimilla
\koodi{out} ja \koodi{in}. Esimerkissä myös lyhennetään nuolta
loppupäästä käyttämällä valitsinta \koodil{shorten~>}.

Tämän alaluvun ohjeiden avulla pääsee aika mukavasti alkuun
\paketti{tikz}\-/ paketin käytössä ja saa toteutettua tavallisimmat
vektorigrafiikkakuviot. Paketin omaan ohjekirjaan kannattaa silti
tutustua, sillä \ymparisto{tikzpicture}\-/ ympäristön mahdollisuudet
ovat valtavat.

\section{Matematiikka}
\label{luku/matematiikka}

Latexin matematiikkatila on suunniteltu matemaattisten kaavojen
latomiseen eli matematiikan syntaksia varten. Se on aivan oma
todellisuutensa, joka ei tunnu noudattavan samoja sääntöjä kuin
tekstitila. Matematiikkatilassa ovat voimassa eri komennot, eri fontit,
erilainen merkkien käyttäytyminen ja erilaiset välistykset. Tässä
luvussa käsitellään tilan käyttöä ja matemaattisen syntaksin
kirjoittamista. Fonttiasetuksia käsitellään luvussa
\ref{luku/matematiikka-fontit}.

\subsection{Matematiikkatilan käyttö}
\label{luku/matematiikka-käyttö}

Matematiikkatila voidaan kytkeä päälle joko tavallisen, tekstitilassa
toimivan rivin sisällä tai omassa tekstikappaleessaan. Tekstirivillä
matemaattiset kaavat lisätään komentojen \komento{(} ja \komento{)}
väliin, kahden \koodi{\$}\-/ merkin väliin tai ympäristön
\ymparisto{math} sisälle. Seuraavassa on esimerkki kaikista kolmesta:

\begin{koodilohkosis}
Kaava \( y = 2x + 3 \) on suoran yhtälö.
Kaava $y = 2x + 3$ on suoran yhtälö.
Kaava \begin{math} y = 2x + 3 \end{math} on suoran yhtälö.
\end{koodilohkosis}

\begin{tulossis}
Kaava \( y = 2x + 3 \) on suoran yhtälö.
Kaava $y = 2x + 3$ on suoran yhtälö.
Kaava \begin{math} y = 2x + 3 \end{math} on suoran yhtälö.
\end{tulossis}

\noindent
Kuten edellä olevasta esimerkistä näkyy, matematiikkatilassa kirjaimet
ladotaan kursiivilla. Tavalliset lähdetiedostoon kirjoitetut kirjaimet
on tarkoitettu muuttujien nimiksi, kuten tässä esimerkissä $x$ ja~$y$.

Jos matemaattiset kaavat ovat pitkiä tai vievät pystysuuntaista tilaa
enemmän kuin tavallisen rivikorkeuden verran, on parasta latoa ne omaksi
kappaleekseen. Se toteutetaan esimerkiksi kirjoittamalla kaavat
ympäristön \ymparisto{displaymath} sisään tai komentojen \komento{[} ja
\komento{]} väliin. Molemmat toimivat samalla tavalla.

\komentoi{[}
\komentoi{]}
\mkomentoi{left}
\mkomentoi{right}
\mkomentoi{frac}
\begin{koodilohkosis}
\[ \left(\frac{1}{x}\right)^2 = \frac{1}{x^2} \]
\end{koodilohkosis}
\[ \left(\frac{1}{x}\right)^2 = \frac{1}{x^2} \]

\noindent
Ympäristö \ymparisto{equation} toimii muuten samalla tavalla, mutta se
latoo kaavan viereen myös järjestysnumeron ristiviittauksia varten.
Ympäristön sisälle voi kirjoittaa \komento{label}\-/ komennon, jonka
argumentissa annetaan kaavalle yksilöllinen tunniste. Tekstistä voi
viitata kaavaan \komento{ref}\-/ komennolla, jonka argumenttina on
kaavan tunniste. Ristiviittauksia käsitellään tarkemmin luvussa
\ref{luku/ristiviitteet}.

Kaavojen numerot tulevat laskurista \laskuri{equation}, ja numeron
latomiseen voi vaikuttaa määrittelemällä uudelleen komennon
\komento{theequation}. Seuraavassa esimerkissä kaavan numeroon asetetaan
ensiksi \komento{section}\-/ tasoisen otsikon numero, piste erottimeksi
ja lopuksi kyseisen kaavan numero.

\komentoi{renewcommand}
\komentoi{theequation}
\komentoi{thesection}
\komentoi{arabic}
\laskurii{equation}
\begin{koodilohkosis}
\renewcommand{\theequation}{\thesection.\arabic{equation}}
\end{koodilohkosis}

\noindent
Latexin lähdedokumentissa matematiikkatilassa kaavojen täytyy sisältyä
yhteen kappaleeseen eli tyhjiä rivejä ei sallita. Matemaattiset kaavat
ladotaan oletuksena sivun keskelle vaakasuunnassa, mutta jos asettaa
Latexin dokumenttiluokalle (luku
\ref{luku/perusdokumenttiluokat-asetukset}) valitsimen \koodi{fleqn}, ne
ladotaan sivun vasempaan reunaan. Kaavojen numerot ladotaan oletuksena
sivun oikeaan reunaan, mutta vasemmalle ne saa käyttämällä
dokumenttiluokan valitsinta \koodi{leqno}.

Matematiikkaympäristössä on käytettävissä taulukkoympäristö
\mymparisto{array}, joka toimii pitkälti samalla tavalla kuin
tekstitilan taulukotkin (luku \ref{luku/taulukot}). Pelkästään
matematiikkatilassa toimiva \mymparisto{array}\-/ ympäristö mahdollistaa
useiden kaavarivien latomisen ja kohdistamisen vaakasuunnassa tietylle
kohdalle. Esimerkiksi yhtälöt on mielekästä sijoittaa samalle tasalle
yhtäsuuruusmerkin kohdalta.

\mymparistoi{array}
\begin{koodilohkosis}
\[ \begin{array}{r@{~}l}
     4x - 2 &= 6 \\
     x      &= 2 \\
   \end{array} \]
\end{koodilohkosis}
\[ \begin{array}{r@{~}l}
     4x - 2 &= 6 \\
     x      &= 2 \\
   \end{array} \]

\noindent
Pidemmälle kehitettyjä matematiikkaympäristöjä on paketissa
\pakettictan{amsmath}. Esimerkiksi \ymparisto{align}\-/\ ja
\ymparisto{align*}\-/ ympäristöt pystyvät tasaamaan allekkaiset kaavat
tietystä kohdasta, eikä taulukon sarakkeita tarvitse erikseen
määritellä. Tasauskohta ilmaistaan lähdedokumentissa \koodi{\&}\=/
merkillä normaalien taulukoiden tavoin. Ympäristön tähtiversio
\ymparisto{align*} ei lado kaavan numeroa mukaan. Edellä olleen
yhtälöesimerkin voi toteuttaa yksinkertaisemmin seuraavasti:

\ymparistoi{align*}
\begin{koodilohkosis}
\begin{align*}
  4x - 2 &= 6 \\
  x      &= 2
\end{align*}
\end{koodilohkosis}

\noindent
Edellä mainittuja \ymparisto{align}\-/\ ja \ymparisto{align*}\-/
ympäristöjä ei kirjoiteta komentojen \komento{[} ja \komento{]} sisään,
eli nämä ympäristöt on tarkoitettu tekstitilassa käytettäväksi.
Ympäristön sisältö on matematiikkatilassa.

Paketti \paketti{amsmath} sisältää paljon muitakin hyödyllisiä
ympäristöjä ja komentoja matematiikan latomiseen. Paketin ohjekirjaan on
erittäin suositeltavaa tutustua.

\subsection{Matematiikkatilan kielioppia}

Matematiikan kirjoittaminen Latexissa on tehty varsin luonnolliseksi.
Esimerkiksi tavalliset operaattorit ja kokonaisluvut kirjoitetaan
näppäimistöltä sellaisenaan, ja jos desimaalierottimena on piste, sekin
syötetään näppäimistöltä suoraan. Suomessa käytetään desimaalierottimena
kuitenkin pilkkua, joka toimii Latexin matematiikkatilassa välimerkkinä:
sen jälkeen ladotaan pieni väli. Suomalaisen desimaalierottimen saa
kirjoittamalla pilkun ympärille aaltosulkeet.

\mkomentoi{pi}
\mkomentoi{approx}
\begin{koodilohkosis}
\[ \pi \approx 3{,}142 \]
\end{koodilohkosis}
\[ \pi \approx 3{,}142 \]

\noindent
Toinen vaihtoehto pilkun käyttämiseen desimaalierottimana on ladata
paketti \pakettictan{icomma}. Paketti määrittelee matematiikkatilan
pilkun toimimaan jokseenkin älykkäästi: jos lähdetiedostossa on pilkun
jälkeen väli, pilkku ladotaan välimerkkinä; jos väliä ei ole, pilkku
katsotaan desimaalierottimeksi.

Plus-, jako- ja yhtäsuuruusmerkki sekä pienempi kuin ja suurempi kuin
\=/merkit ($+ : / = \; < \; >$) kirjoitetaan näppäimistöltä
sellaisenaan. Miinusmerkki ($-$) kirjoitetaan yhdysmerkin (\koodi{-})
avulla, eli yhdysmerkki ladotaan dokumenttiin Unicode\-/ merkistön
miinusmerkkinä \uctunnus{u+2212 minus sign}. Kertomerkit voi kirjoittaa
lähdedokumenttiin sellaisenaan mutta myös komennoilla \mkomento{cdot}
($\cdot$) ja \mkomento{times} ($\times$).

Pienet sulkeetkin voi kirjoittaa näppäimistöltä suoraan, mutta
matematiikassa tarvitaan usein erikokoisia, tilanteeseen mukautuvia
sulkeita. Ne tehdään komentojen \mkomento{left} (vasen) ja
\mkomento{right} (oikea) avulla. Komentojen jälkeen kirjoitetaan haluttu
suljemerkki, esimerkiksi \mkomento{left}\mkomentojatko{(} ja
\mkomento{right}\mkomentojatko{)}. Itseisarvoa merkitsevät pystyviivat
tehdään samoilla komennoilla, mutta suljemerkkien tilalle kirjoitetaan
pystyviiva~(\koodi{|}).

\mkomentoi{left}
\mkomentoi{right}
\mkomentoi{frac}
\begin{koodilohkosis}
\[ \left| a + \left( \frac{b}{c \left( d-1 \right)} \right) \right| \]
\end{koodilohkosis}
\[ \left| a + \left( \frac{b}{c \left( d-1 \right)} \right) \right| \]

\noindent
Komentoja \mkomento{left} ja \mkomento{right} täytyy käyttää pareittain,
jotta Latex osaa latoa oikeankokoiset sulkeet. Jos ei halua
suljemerkille paria, kirjoitetaan toisen suljekomennon ''sulkeeksi''
piste. Seuraavassa esimerkissä ladotaan vasemmalle aaltosulje
(\mkomento{left}\mkomentojatko{\keno\{}) mutta oikealle ei mitään
(\mkomento{right}\mkomentojatko{.}).

\mkomentoi{left}
\mkomentoi{right}
\mymparistoi{array}
\begin{koodilohkosis}
\[ \left\{ \begin{array}{r@{~}l}
             x &= 4 \\
             y &= -2 \\
           \end{array} \right. \]
\end{koodilohkosis}
\[ \left\{ \begin{array}{l@{~}l}
             x &= 4 \\
             y &= -2 \\
           \end{array} \right. \]

\noindent
Matematiikkatilan ylä- ja alaindeksit toteutetaan sirkumfleksin
(\koodi{\^{}}) ja alaviivan (\koodi{\_}) avulla. Välittömästä merkin
jälkeen oleva merkki ladotaan ylä- tai alaindeksiksi, ja jos indeksi
sisältää enemmän kuin yhden merkin, kirjoitetaan kokonaisuus
aaltosulkeisiin. Joidenkin matemaattisten operaattorikomentojen
yhteydessä indeksit ladotaan poikkeuksellisella tavalla, esimerkiksi
kokonaan operaattorin ylä- tai alapuolelle.

\mkomentoi{sum}
\mkomentoi{int}
\mkomentoi{lim}
\mkomentoi{to}
\mkomentoi{infty}
\mkomentoi{qquad}
\begin{koodilohkosis}
\[ x^{n-1} \qquad x_i \qquad \sum_{i=1}^n \qquad
  \int_0^\infty \qquad \lim_{n \to \infty} \]
\end{koodilohkosis}
\[ x^{n-1} \qquad x_i \qquad \sum_{i=1}^n \qquad
  \int_0^\infty \qquad \lim_{n \to \infty} \]

\noindent
Lausekkeen osia voi ryhmitellä ylä- tai alapuolisella aaltosulkeella,
jotka tehdään komennoilla \mkomento{overbrace} ja \mkomento{underbrace}.
Jos näiden komentojen jälkeen käyttää ylä- tai alaindeksiä, se ladotaan
sulkeen keskelle. Seuraavassa on esimerkki:

\mkomentoi{overbrace}
\mkomentoi{underbrace}
\begin{koodilohkosis}
\[ \overbrace{3x + x}^{4x} - \underbrace{5y - 2y}_{3y} = 4x - 3y \]
\end{koodilohkosis}
\[ \overbrace{3x + x}^{4x} - \underbrace{5y - 2y}_{3y} = 4x - 3y \]

\noindent
Muuttujien, funktion nimien tai muiden symbolien jäljessä oleva
$'$\=/merkki tehdään yleisheittomerkin (\koodi{'}) avulla.
Matematiikassa se voi tarkoittaa esimerkiksi funktion
derivaattafunktiota, ja Unicode\-/ merkistössä merkin tunnus
\uctunnus{u+2032 prime}.

\mkomentoi{quad}
\begin{koodilohkosis}
\[ f(x) = 3x^2 - 2x \quad f'(x) = 6x - 2 \]
\end{koodilohkosis}
\[ f(x) = 3x^2 - 2x \quad f'(x) = 6x - 2 \]

\noindent
Murtoluvuille ja jakoviivan latomiseen on komento \mkomento{frac}, jolle
annetaan kaksi argumenttia: osoittaja ja nimittäjä. Neliö- ja muut
juuret tehdään \mkomento{sqrt}\-/ komennolla, jolle annetaan ainakin
yksi argumentti. Komennolle voi antaa hakasulkeissa toisenkin
argumentin, joka ilmaisee juuriluvun. Seuraavassa on esimerkki
murtoluvun, neliöjuuren ja kuutiojuuren toteuttamisesta:

\mkomentoi{frac}
\mkomentoi{sqrt}
\mkomentoi{quad}
\begin{koodilohkosis}
\[ \frac{1}{3x + 1} \quad \sqrt{9} = \sqrt[3]{27} \]
\end{koodilohkosis}
\[ \frac{1}{3x + 1} \quad \sqrt{9} = \sqrt[3]{27} \]

\noindent
Vektorien nimien latomiseen eli merkkien yläpuoliselle nuolelle on oma
komento \mkomento{vec}, joka toimii yhden kirjaimen kanssa. Kahden
merkin mittaiseen nuoleen tarvitaan komentoa \mkomento{overrightarrow}
(tai \mkomento{overleftarrow}).

\mkomentoi{vec}
\mkomentoi{overrightarrow}
\mkomentoi{quad}
\begin{koodilohkosis}
\[ \vec{a} \quad \overrightarrow{AB} \]
\end{koodilohkosis}
\[ \vec{a} \quad \overrightarrow{AB} \]

\noindent
Matriisit voi toteuttaa luvussa \ref{luku/matematiikka-käyttö} esitellyn
\mymparisto{array}\-/ ympäristön avulla, mutta kätevämpää on käyttää
\paketti{amsmath}\-/ paketin ympäristöjä. Kullekin erilaiselle
suljetyypille on oma ympäristönsä: \mymparisto{matrix},
\mymparisto{pmatrix}~$(\,)$, \mymparisto{bmatrix}~$[\,]$,
\mymparisto{Bmatrix}~$\{\,\}$, \mymparisto{vmatrix}~$|$ ja
\mymparisto{Vmatrix}~$\|$. Ympäristön sisällä matriisin rivin solut
erotetaan toisistaan samoin kuin taulukoissakin eli \koodi{\&}\=/
merkillä ja rivinvaihto tehdään \mkomento{\keno}\=/ komennolla.
Seuraavassa on esimerkki kahdesta eri ympäristöstä:

\mymparistoi{matrix}
\mymparistoi{bmatrix}
\mkomentoi{qquad}
\begin{koodilohkosis}
\[ \begin{matrix} 1 & 2 \\ 3 & 4 \\ \end{matrix} \qquad
   \begin{bmatrix}
     1 & 2 & 3 \\ 4 & 5 & 6 \\ 7 & 8 & 9 \\
   \end{bmatrix} \]
\end{koodilohkosis}
\[ \begin{matrix} 1 & 2 \\ 3 & 4 \\ \end{matrix} \qquad
  \begin{bmatrix}
    1 & 2 & 3 \\ 4 & 5 & 6 \\ 7 & 8 & 9 \\
  \end{bmatrix} \]

\leijutlk{
  \providecommand{\rivi}{}
  \renewcommand{\rivi}[3]{\mkomento{#1} & $#2$ & #3 \\}

  \begin{tabular}{lll}
    \toprule
    \ots{Komento}
    & \ots{Esimerkki}
    & \ots{Merkitys} \\
    \midrule
    \rivi{mathrm}{\mathrm{ABC~abc}}{antiikva, serif, roman}
    \rivi{mathsf}{\mathsf{ABC~abc}}{groteski, sans serif, gothic}
    \rivi{mathtt}{\mathtt{ABC~abc}}{tasalevyinen, typewriter}
    \rivi{mathcal}{\mathcal{ABC}}{kalligrafinen}
    \midrule
    \rivi{mathbf}{\mathbf{ABC~abc}}{lihavoitu, bold}
    \rivi{mathit}{\mathit{ABC~abc}}{kursiivi, italic}
    \bottomrule
  \end{tabular}
}{
  \caption{Tekstin latominen matematiikkatilassa vaatii erityisen
    komennon}
  \label{tlk/matem-teksti}
}

\noindent
Kuten on jo todettu, tavalliset kirjaimet on matematiikkatilassa
tarkoitettu muuttujien nimiksi. Kun täytyy latoa varsinaista tekstiä --
tekstitilan tavoin -- täytyy käyttää erityisiä komentoja. Taulukkoon
\ref{tlk/matem-teksti} on koottu matematiikkatilan
tekstinlatomiskomentoja. Komentojen argumenttina oleva teksti ladotaan
kuin tekstitilassa.

Vaakasuuntaisten välien tekemiseen on matematiikkatilassa muutama
komento. Komento \mkomento{quad} latoo typografisen neliön (1\,em)
levyisen välin. Se on sama kuin nykyisen fontin koko. Komento
\mkomento{qquad} latoo 2\,em:n levyisen välin. Pienempiä välejä saa
komennoilla \mkomento{,} (\murtoluku{3}{18}\,em), \mkomento{:}
(\murtoluku{4}{18}\,em) ja \mkomento{;} (\murtoluku{5}{18}\,em). Komento
\mkomento{!} puolestaan tuottaa negatiivisen välin, jonka mitta on
−\murtoluku{3}{18}\,em. Negatiivista väliä voi käyttää liian suuren
välin pienentämiseen.

\subsection{Erikoismerkkejä}

Matematiikkatilassa voi käyttää Unicode\-/ merkistöä, eli monet
matemaattiset symbolit voi kirjoittaa lähdedokumenttiin sellaisenaan.
Voi silti olla helpompaa käyttää erityisiä komentoja sellaisten
kirjainten ja symbolien kirjoittamiseen, joita ei ihan helposti pysty
tuottamaan näppäimistöltä tai tekstieditorin toimintojen avulla.
Matematiikkatilan erikoismerkkejä on koottu oheisiin taulukoihin
(\ref{tlk/matem-kreikk}--). Lisää symboleja on paketeissa
\pakettictan{latexsym} ja \pakettictan{amsmath}.

\leijutlk{
  \providecommand{\rivi}{}
  \renewcommand{\rivi}[2]{$#1$ & \mkomento{#2}}

  \begin{tabular}{*{4}{cl}}
    \toprule

    \rivi{\alpha}{alpha}
    & \rivi{\Alpha}{Alpha}
    & \rivi{\beta}{beta}
    & \rivi{\Beta}{Beta} \\

    \rivi{\gamma}{gamma}
    & \rivi{\Gamma}{Gamma}
    & \rivi{\delta}{delta}
    & \rivi{\Delta}{Delta} \\

    \rivi{\epsilon}{epsilon}
    & \rivi{\varepsilon}{varepsilon}
    & \rivi{\Epsilon}{Epsilon}
    & \rivi{\zeta}{zeta} \\

    \rivi{\Zeta}{Zeta}
    & \rivi{\eta}{eta}
    & \rivi{\Eta}{Eta}
    & \rivi{\theta}{theta} \\

    \rivi{\vartheta}{vartheta}
    & \rivi{\Theta}{Theta}
    & \rivi{\iota}{iota}
    & \rivi{\Iota}{Iota} \\

    \rivi{\kappa}{kappa}
    & \rivi{\Kappa}{Kappa}
    & \rivi{\lambda}{lambda}
    & \rivi{\Lambda}{Lambda} \\

    \rivi{\mu}{mu}
    & \rivi{\Mu}{Mu}
    & \rivi{\nu}{nu}
    & \rivi{\Nu}{Nu} \\

    \rivi{\xi}{xi}
    & \rivi{\Xi}{Xi}
    & \rivi{\pi}{pi}
    & \rivi{\varpi}{varpi} \\

    \rivi{\Pi}{Pi}
    & \rivi{\rho}{rho}
    & \rivi{\varrho}{varrho}
    & \rivi{\Rho}{Rho} \\

    \rivi{\sigma}{sigma}
    & \rivi{\varsigma}{varsigma}
    & \rivi{\Sigma}{Sigma}
    & \rivi{\tau}{tau} \\

    \rivi{\Tau}{Tau}
    & \rivi{\upsilon}{upsilon}
    & \rivi{\Upsilon}{Upsilon}
    & \rivi{\phi}{phi} \\

    \rivi{\varphi}{varphi}
    & \rivi{\Phi}{Phi}
    & \rivi{\chi}{chi}
    & \rivi{\Chi}{Chi} \\

    \rivi{\psi}{psi}
    & \rivi{\Psi}{Psi}
    & \rivi{\omega}{omega}
    & \rivi{\Omega}{Omega} \\

    \bottomrule
  \end{tabular}
}{
  \caption{Kreikkalaisten kirjainten latominen matematiikkatilassa}
  \label{tlk/matem-kreikk}
}

\leijutlk{
  \providecommand{\rivi}{}
  \renewcommand{\rivi}[3][a]{$#2{#1}$ & \mkomento{#3}\mkomentoarg{#1}}

  \begin{tabular}{*{3}{cl}}
    \toprule

    \rivi{\grave}{grave}
    & \rivi{\ddot}{ddot}
    & \rivi{\hat}{hat} \\

    \rivi{\acute}{acute}
    & \rivi{\check}{check}
    & \rivi[aaa]{\widehat}{widehat} \\

    \rivi{\breve}{breve}
    & \rivi{\dot}{dot}
    & \rivi{\tilde}{tilde} \\

    \rivi{\bar}{bar}
    & \rivi{\mathring}{mathring}
    & \rivi[aaa]{\widetilde}{widetilde} \\

    \bottomrule
  \end{tabular}
}{
  \caption{Matematiikkatilan tarkekomentoja}
  \label{tlk/matem-tarkkeita}
}

\leijutlk{
  \providecommand{\rivi}{}
  \renewcommand{\rivi}[3][AB]{$#2{#1}$ & \mkomento{#3}\mkomentoarg{#1}}
  \renewcommand{\arraystretch}{1.5}

  \begin{tabular}{*{2}{cl}}
    \toprule

    \rivi{\overrightarrow}{overrightarrow}
    & \rivi{\overleftarrow}{overleftarrow} \\

    \rivi{\underrightarrow}{underrightarrow}
    & \rivi{\underleftarrow}{underleftarrow} \\

    \rivi{\overleftrightarrow}{overleftrightarrow}
    & \rivi{\underleftrightarrow}{underleftrightarrow} \\

    \rivi{\overline}{overline}
    & \rivi{\underline}{underline} \\

    \rivi[abc]{\overbrace}{overbrace}
    & \rivi[abc]{\underbrace}{underbrace} \\

    \bottomrule
  \end{tabular}
}{
  \caption{Ylä- ja alamerkintöjä}
  \label{tlk/matem-yla-ala-merk}
}

\leijutlk{
  \providecommand{\rivi}{}
  \renewcommand{\rivi}[2]{$#1$ & \mkomento{#2}}

  \begin{tabular}{*{4}{cl}}
    \toprule

    $<$
    & \koodi{<}
    & $>$
    & \koodi{>}
    & $=$
    & \koodi{=}
    & $:$
    & \koodi{:} \\

    $\leq$
    & \mkomento{leq}, \mkomento{le}
    & $\geq$
    & \mkomento{geq}, \mkomento{ge}
    & \rivi{\ll}{ll}
    & \rivi{\gg}{gg} \\

    \rivi{\equiv}{equiv}
    & \rivi{\doteq}{doteq}
    & \rivi{\prec}{prec}
    & \rivi{\preceq}{preceq} \\

    \rivi{\succ}{succ}
    & \rivi{\succeq}{succeq}
    & \rivi{\sim}{sim}
    & \rivi{\simeq}{simeq} \\

    \rivi{\subset}{subset}
    & \rivi{\subseteq}{subseteq}
    & \rivi{\supset}{supset}
    & \rivi{\supseteq}{supseteq} \\

    \rivi{\approx}{approx}
    & \rivi{\cong}{cong}
    & \rivi{\sqsubset}{sqsubset}
    & \rivi{\sqsubseteq}{sqsubseteq} \\

    \rivi{\sqsupset}{sqsupset}
    & \rivi{\sqsupseteq}{sqsupseteq}
    & \rivi{\bowtie}{bowtie}
    & \rivi{\in}{in} \\

    \rivi{\ni}{ni}
    & \rivi{\propto}{propto}
    & \rivi{\vdash}{vdash}
    & \rivi{\dashv}{dashv} \\

    \rivi{\models}{models}
    & \rivi{\mid}{mid}
    & \rivi{\parallel}{parallel}
    & \rivi{\perp}{perp} \\

    \rivi{\smile}{smile}
    & \rivi{\frown}{frown}
    & \rivi{\asymp}{asymp}
    & \rivi{\notin}{notin} \\

    $\neq$
    & \mkomento{neq}, \mkomento{ne} \\

    \bottomrule
  \end{tabular}
}{
  \caption{Relaatioita. Negaation saa kirjoittamalla komennon eteen
    \mkomento{not}}
  \label{tlk/matem-relaatioita}
}

\leijutlk{
  \providecommand{\rivi}{}
  \renewcommand{\rivi}[2]{$#1$ & \mkomento{#2}}
  \renewcommand{\arraystretch}{1.1}

  \begin{tabular}{*{4}{cl}}
    \toprule

    $+$
    & \koodi{+}
    & $-$
    & \koodi{-}
    & \rivi{\pm}{pm}
    & \rivi{\mp}{mp} \\

    \rivi{\cdot}{cdot}
    & \rivi{\div}{div}
    & \rivi{\times}{times}
    & \rivi{\setminus}{setminus} \\

    \rivi{\cup}{cup}
    & \rivi{\cap}{cap}
    & \rivi{\sqcup}{sqcup}
    & \rivi{\sqcap}{sqcap} \\

    \rivi{\vee}{vee}
    & \rivi{\wedge}{wedge}
    & \rivi{\oplus}{oplus}
    & \rivi{\ominus}{ominus} \\

    \rivi{\odot}{odot}
    & \rivi{\oslash}{oslash}
    & \rivi{\uplus}{uplus}
    & \rivi{\otimes}{otimes} \\

    \rivi{\sum}{sum}
    & \rivi{\prod}{prod}
    & \rivi{\coprod}{coprod}
    & \rivi{\int}{int} \\

    \rivi{\oint}{oint}
    & \rivi{\lim}{lim}
    & \rivi{\bigoplus}{bigoplus}
    & \rivi{\bigotimes}{bigotimes} \\

    \rivi{\bigodot}{bigodot}
    & \rivi{\bigvee}{bigvee}
    & \rivi{\bigwedge}{bigwedge}
    & \rivi{\bigcup}{bigcup} \\

    \rivi{\bigcap}{bigcap}
    & \rivi{\biguplus}{biguplus}
    & \rivi{\bigsqcup}{bigsqcup} \\

    \bottomrule
  \end{tabular}
}{
  \caption{Operaattoreita}
  \label{tlk/matem-operaattoreita}
}

\leijutlk{
  \providecommand{\rivi}{}
  \renewcommand{\rivi}[2]{$#1$ & \mkomento{#2}}

  \begin{tabular}{*{2}{cl}}
    \toprule

    $\leftarrow$
    & \mkomento{leftarrow}, \mkomento{gets}
    & $\rightarrow$
    & \mkomento{rightarrow}, \mkomento{to} \\

    \rivi{\longleftarrow}{longleftarrow}
    & \rivi{\longrightarrow}{longrightarrow} \\

    \rivi{\leftrightarrow}{leftrightarrow}
    & \rivi{\longleftrightarrow}{longleftrightarrow} \\

    \rivi{\Leftarrow}{Leftarrow}
    & \rivi{\Rightarrow}{Rightarrow} \\

    \rivi{\Longleftarrow}{Longleftarrow}
    & \rivi{\Longrightarrow}{Longrightarrow} \\

    \rivi{\Leftrightarrow}{Leftrightarrow}
    & \rivi{\Longleftrightarrow}{Longleftrightarrow} \\

    \rivi{\mapsto}{mapsto}
    & \rivi{\longmapsto}{longmapsto} \\

    \rivi{\hookleftarrow}{hookleftarrow}
    & \rivi{\hookrightarrow}{hookrightarrow} \\

    \rivi{\leftharpoonup}{leftharpoonup}
    & \rivi{\rightharpoonup}{rightharpoonup} \\

    \rivi{\leftharpoondown}{leftharpoondown}
    & \rivi{\rightharpoondown}{rightharpoondown} \\

    \rivi{\rightleftharpoons}{rightleftharpoons}
    & \rivi{\iff}{iff} \\

    \rivi{\uparrow}{uparrow}
    & \rivi{\downarrow}{downarrow} \\

    \rivi{\updownarrow}{updownarrow}
    & \rivi{\Updownarrow}{Updownarrow} \\

    \rivi{\Uparrow}{Uparrow}
    & \rivi{\Downarrow}{Downarrow} \\

    \rivi{\nearrow}{nearrow}
    & \rivi{\searrow}{searrow} \\

    \rivi{\swarrow}{swarrow}
    & \rivi{\nwarrow}{nwarrow} \\

    \bottomrule
  \end{tabular}
}{
  \caption{Nuolia}
  \label{tlk/matem-nuolia}
}

\leijutlk{
  \providecommand{\rivi}{}
  \renewcommand{\rivi}[2]{$#1$ & \mkomento{#2}}

  \begin{tabular}{*{4}{cl}}
    \toprule

    \rivi{\langle}{langle}
    & \rivi{\rangle}{rangle}
    & \rivi{\lfloor}{lfloor}
    & \rivi{\rfloor}{rfloor} \\

    \rivi{\rceil}{rceil}
    & \rivi{\lceil}{lceil}
    & $|$
    & \koodi{|}, \mkomento{vert}
    & $\|$
    & \mkomentox{|}, \mkomento{Vert} \\

    \rivi{\lgroup}{lgroup}
    & \rivi{\rgroup}{rgroup}
    & \rivi{\lmoustache}{lmoustache}
    & \rivi{\rmoustache}{rmoustache} \\

    $/$
    & \koodi{/}
    & \rivi{\backslash}{backslash} \\

    \bottomrule
  \end{tabular}
}{
  \caption{Sulkeita ja erotinmerkkejä}
  \label{tlk/matem-erottimia}
}

\leijutlk{
  \providecommand{\rivi}{}
  \renewcommand{\rivi}[2]{$#1$ & \mkomento{#2}}

  \begin{tabular}{*{3}{cl}}
    \toprule

    $\neg$
    & \mkomento{neg}, \mkomento{lnot}
    & \rivi{\angle}{angle}
    & \rivi{\emptyset}{emptyset} \\

    \rivi{\infty}{infty}
    & $'$
    & \koodi{'}
    & \rivi{\prime}{prime} \\

    \rivi{\forall}{forall}
    & \rivi{\exists}{exists}
    & \rivi{\wr}{wr} \\

    \rivi{\bot}{bot}
    & \rivi{\top}{top}
    & \rivi{\surd}{surd} \\

    \rivi{\dots}{dots}
    & \rivi{\cdots}{cdots}
    & \rivi{\vdots}{vdots} \\

    \rivi{\ddots}{ddots}
    & \rivi{\triangle}{triangle}
    & \rivi{\triangleleft}{triangleleft} \\

    \rivi{\triangleright}{triangleright}
    & \rivi{\nabla}{nabla}
    & \rivi{\star}{star} \\

    \rivi{\ast}{ast}
    & \rivi{\circ}{circ}
    & \rivi{\bigcirc}{bigcirc} \\

    \rivi{\bullet}{bullet}
    & \begin{tikzpicture} % \diamond puuttuu fontista
      \draw [rotate=45] (0,0) rectangle (.5ex,.5ex);
    \end{tikzpicture}
    & \mkomento{diamond}
    & \rivi{\amalg}{amalg} \\

    \rivi{\bigtriangleup}{bigtriangleup}
    & \rivi{\bigtriangledown}{bigtriangledown}
    & \rivi{\dagger}{dagger} \\

    \rivi{\ddagger}{ddagger}
    & \rivi{\diamondsuit}{diamondsuit}
    & \rivi{\heartsuit}{heartsuit} \\

    \rivi{\clubsuit}{clubsuit}
    & \rivi{\spadesuit}{spadesuit}
    & \rivi{\flat}{flat} \\

    \rivi{\natural}{natural}
    & \rivi{\sharp}{sharp}
    & \rivi{\hbar}{hbar} \\

    \rivi{\imath}{imath}
    & \rivi{\jmath}{jmath}
    & \rivi{\ell}{ell} \\

    \rivi{\Re}{Re}
    & \rivi{\Im}{Im}
    & \rivi{\aleph}{aleph} \\

    \rivi{\wp}{wp} \\

    \bottomrule
  \end{tabular}
}{
  \caption{Sekalaisia symboleja}
  \label{tlk/matem-sekalaisia}
}

% Tekijä:   Teemu Likonen <tlikonen@iki.fi>
% Lisenssi: Creative Commons Nimeä-JaaSamoin 4.0 Kansainvälinen (CC BY-SA 4.0)
% https://creativecommons.org/licenses/by-sa/4.0/legalcode.fi

\chapter{Erikoisdokumentit}

Kaikki dokumentit ja painotuotteet eivät ole samanlaista tekstivirtaa,
joka koostuu jäsennellystä rakenteesta, väliotsikoista ynnä muusta
sellaisesta. Tässä luvussa käsitellään dokumentteja, jotka vaativat
osittain toisenlaista rakennetta ja tekniikkaa kuin luvussa
\ref{luku/rakenne} on käsitelty.

\section{Esitysgrafiikka}
\label{luku/esitysgrafiikka}

Esitelmien ja muiden puhe\-/ esitysten tukena käytetään usein
esitysgrafiikkaohjelmia kuten \englanti{Microsoft Powerpointia} tai
\englanti{Libreoffice Impressiä}. Myös Latex sopii esitysgrafiikan
tekemiseen, ja se voi olla osaavissa käsissä jopa kaikkein nopein ja
typografisesti tyylikkäin työkalu esitysdiojen tuottamiseen.

Latexin tunnetuin ja monipuolisin esitysgrafiikkajärjestelmä on
\luokkactan{beamer}, joka on oma dokumenttiluokkansa. Sen ohjekirjassa
on reilu kaksisataa sivua, joten tässä oppaassa voidaan esitellä vain
perusasioita, joilla pääsee alkuun.

\begin{esimerkki*}
  \komentoi{author}
  \komentoi{date}
  \komentoi{documentclass}
  \komentoi{institute}
  \komentoi{item}
  \komentoi{maketitle}
  \komentoi{pause}
  \komentoi{section}
  \komentoi{setdefaultlanguage}
  \komentoi{title}
  \komentoi{usepackage}
  \luokkai{beamer}
  \pakettii{hyperref}
  \pakettii{polyglossia}
  \ymparistoi{enumerate}
  \ymparistoi{frame}

\begin{koodilohko}
\documentclass[aspectratio=169,t]{beamer}
\usepackage{polyglossia} \setdefaultlanguage{finnish}
\usepackage{hyperref}

\title{Esitysgrafiikkaa Latexilla}
\author{Teppo Tekijä}
\institute{Ladontatieteiden tiedekunta}
\date{13.9.2021}

\begin{document}

\section{Otsikkodia}

\begin{frame}
  \maketitle
\end{frame}

\section{Ensimmäinen väliotsikko}

\begin{frame}{Pääotsikko}{Alaotsikko}

  \pause Seuraavassa luetellaan jotakin: \pause

  \begin{enumerate}
  \item ensimmäinen \pause
  \item toinen \pause
  \item kolmas.
  \end{enumerate}

\end{frame}

\section{Toinen väliotsikko}

\begin{frame}[c]

  Sisältö on keskitetty pystysuunnassa.

\end{frame}

\end{document}
\end{koodilohko}

  \caption{\luokka{beamer}\-/ dokumenttiluokan avulla tehdyn
    diaesityksen runko}
  \label{esim/beamer-runko}
\end{esimerkki*}

\subsection{Diaesityksen rakentaminen}

Esimerkissä \ref{esim/beamer-runko} on kokonainen Latexin lähdetiedosto
ja diaesityksen runko. Dokumenttiluokan \luokka{beamer} lataamisen
yhteydessä (rivi~1) määritellään valitsimella \koodi{aspectratio}, mitkä
ovat diojen mittasuhteet. Tässä esimerkissä arvoksi annetaan
\koodi{169}, joka tarkoittaa leveyden ja korkeuden suhdetta 16:9.
Oletusarvo on \koodi{43} eli kuvasuhde 4:3. Muita mahdollisia arvoja
ovat esimerkiksi \koodi{1610} (16:10) ja \koodi{54} (5:4).

Esimerkin ensimmäisellä rivillä oleva toinen valitsin \koodi{t}
(\englanti{top}) tarkoittaa, että diojen sisältö sijoitetaan dian
yläreunaan. Oletusasetus on \koodi{c} (\englanti{center}), joka
keskittää dian sisällön pystysuunnassa. Tämän asetuksen voi muuttaa
kuhunkin diaan käyttämällä \ymparisto{frame}\-/ ympäristön valinnaista
argumenttia, kuten esimerkin rivillä 34 on tehty.

Riveillä 5--8 määritellään dokumentin perustiedot (luku
\ref{luku/dokumentin-perustiedot}), mutta perus Latexiin verrattuna
mukana on uusi komento \komento{institute}, jolla voi määritellä tekijän
edustaman laitoksen. Nämä perustiedot ladotaan näkyviin vasta
\komento{maketitle}\-/ komennolla, joka on esimerkin rivillä~15.

Jokainen yksittäinen sivu eli dia täytyy kirjoittaa \ymparisto{frame}\-/
ympäristön sisään. Ympäristö siis aloittaa aina puhtaan sivun.
Ympäristön alussa voi olla aaltosulkeissa sivun pääotsikko ja toisissa
aaltosulkeissa alaotsikko. Esimerkin rivillä 20 on tehty juuri näin.
Toisen tai molemmat otsikot voi jättää poiskin.

Sivun eli \ymparisto{frame}\-/ ympäristön sisällä voi olla tavallista
tekstiä ja käyttää suunnilleen normaaleja Latexin komentoja. Esimerkissä
\ref{esim/beamer-runko} on käytetty muutaman kerran \komento{pause}\-/
komentoa, joka tekee esitykseen tauon. Käytännössä se jakaa sivun
sisällön erillisiksi pdf\-/ sivuiksi, minkä avulla sisällön voi
paljastaa yleisölle vaihe vaiheelta. Esimerkin riveillä 24--28 käytetään
luetelmaa eli \ymparisto{enumerate}\-/ ympäristöä. Se toimii suunnilleen
samoin kuin perus Latexin vastaava ympäristö (luku
\ref{luku/luetelma-perus}), mutta \luokka{beamer}\-/ versiossa on hieman
enemmän ominaisuuksia.

Tästä esimerkistä puuttuu hyödyllinen \ymparisto{block}\-/ ympäristö,
jolla saa dian sisälle pienen väliotsikon ja siihen kuuluvan
tekstikokonaisuuden. Otsikon teksti annetaan ympäristön argumentissa:

\ymparistoi{block}
\begin{koodilohkosis}
\begin{block}{Väliotsikko}
  ...
\end{block}
\end{koodilohkosis}

\noindent
Otsikkokomennot \komento{chapter}, \komento{section},
\komento{subsection} ym. täytyy kirjoittaa \ymparisto{frame}\-/
ympäristöjen ulkopuolelle. \luokka{beamer}\-/ luokassa otsikkokomennot
eivät lado mitään itse dokumenttiin, vaan ne ainoastaan muodostavat
pdf\-/ tiedoston sisällysluettelon. Tosin sisällysluettelon kaltaista
rakennetietoa on mahdollista saada näkyviin itse diojen reunoillekin.
Niiden tarkoituksena on helpottaa esimerkiksi pitkän diaesityksen
seuraamista ja jäsentämistä. Näistä ominaisuuksista voi lukea lisää
\luokka{beamer}\-/ luokan ohjekirjasta.

Dioihin voi lisätä kuvatiedostoja tai vektorigrafiikkaa normaalin
Latexin tavoin eli luvun \ref{luku/grafiikka} ohjeilla. Palstoja ei
kuitenkaan kannata toteuttaa luvun \ref{luku/palstat} keinoilla vaan
\luokka{beamer}\-/ luokan oman \ymparisto{columns}\-/ ympäristön avulla.

\subsection{Ulkoasuteemat}

\luokka{beamer}\-/ dokumenttiluokka sisältää valmiita ulkoasuteemoja eli
ulkoasuun vaikuttavia asetusten kokonaisuuksia. Käyttämällä valmista
teemaa saa helposti käyttöönsä jonkun henkilön suunnitteleman tyylikkään
kokonaisuuden.

Yleinen teemojen valintakomento on \komento{usetheme}. Sillä valitaan
teema, joka voi määritellä vähän kaikkea diaesityksen ulkoasuun
liittyvää: fontit, värit, sisällön asettelua ja diojen reunoille
ladottavaa lisätietoa. Alemmantasoisilla teemakomennoilla valitaan vain
jonkin pienemmän osa-alueen teema. Esimerkiksi komennoilla
\komento{usecolortheme} ja \komento{usefonttheme} valitaan vain väri-
tai fonttiteema.

Fontit tai kirjainperheet sinänsä määritellään ja otetaan käyttöön
samalla tavalla kuin Latexissa muutenkin (luku
\ref{luku/kirjaintyypit}), mutta \luokka{beamer}\-/ luokan teema voi
määritellä, mitä kirjainperhettä tai \=/leikkausta käytetään missäkin
tilanteessa, esimerkiksi diojen otsikoissa tai leipätekstissä. Näiden
asetusten muuttaminen voi vaatia \luokka{beamer}\-/ luokan omia
komentoja, joista annetaan tietoa luvussa \ref{luku/beamer-asetuksia}.

Alemmantasoisia teemakomentoja ovat myös \komento{useinnertheme} ja
\komento{useoutertheme}. Niillä vaikutetaan erilaisiin diojen
sisältöelementtien (\englanti{inner}) ulkoasuun ja diojen reunoille
(\englanti{outer}) ladottavaan lisätietoon.

\leijutlk{
  \providecommand{\rivi}{}
  \renewcommand{\rivi}[2]{\midrule \komento{#1} & #2 \\}
  \providecommand{\teema}{}
  \renewcommand{\teema}[2][,]{\mbox{\koodi{#2}#1}}

  \begin{tabularx}{\linewidth}{lL}
    \toprule
    \ots{Komento} & \ots{Valmiita teemoja} \\
    \rivi{usetheme}{
      \teema{default}
      \teema{boxes}
      \teema{Bergen}
      \teema{Boadilla}
      \teema{Madrid}
      \teema{AnnArbor}
      \teema{CambridgeUS}
      \teema{EastLansing}
      \teema{Pittsburgh}
      \teema{Rochester}
      \teema{Antibes}
      \teema{JuanLesPins}
      \teema{Montpellier}
      \teema{Berkeley}
      \teema{PaloAlto}
      \teema{Goettingen}
      \teema{Marburg}
      \teema{Hannover}
      \teema{Berlin}
      \teema{Ilmenau}
      \teema{Dresden}
      \teema{Darmstadt}
      \teema{Frankfurt}
      \teema{Singapore}
      \teema{Szeged}
      \teema{Copenhagen}
      \teema{Luebeck}
      \teema{Malmoe}
      \teema[]{Warsaw}
    }

    \rivi{usecolortheme}{
      \teema{default}
      \teema{structure}
      \teema{sidebartab}
      \teema{albatross}
      \teema{beetle}
      \teema{crane}
      \teema{dove}
      \teema{fly}
      \teema{monarca}
      \teema{seagull}
      \teema{wolverine}
      \teema{beaver}
      \teema{spruce}
      \teema{lily}
      \teema{orchid}
      \teema{rose}
      \teema{whale}
      \teema{seahorse}
      \teema{dolphin}
      \teema[]{seahorse}
    }

    \rivi{usefonttheme}{
      \teema{default}
      \teema{serif}
      \teema{structurebold}
      \teema{structureitalicserif}
      \teema[]{structuresmallcapsserif}
    }

    \rivi{useinnertheme}{
      \teema{default}
      \teema{circles}
      \teema{rectangles}
      \teema{rounded}
      \teema[]{inmargin}
    }

    \rivi{useoutertheme}{
      \teema{default}
      \teema{infolines}
      \teema{miniframes}
      \teema{smoothbars}
      \teema{sidebar}
      \teema{split}
      \teema{shadow}
      \teema{tree}
      \teema[]{smoothtree}
    }
    \bottomrule
  \end{tabularx}
}{
  \caption{\luokka{beamer}\-/ luokan teemanvalintakomennot ja valmiita
    teemoja}
  \label{tlk/usetheme}
}

Taulukkoon \ref{tlk/usetheme} on koottu \luokka{beamer}\-/ luokan
teemakomennot ja valmiita teemoja. Taulukon ensimmäisessä sarakkeessa
ovat teeman valintakomennot ja toisessa sarakkeessa teemojen nimiä,
joita voi antaa komennolle argumentiksi. Ensimmäisenä mainittu
\koodi{default}\-/ teema on käytössä oletuksena. Teemat otetaan käyttöön
kirjoittamalla lähdedokumentin esittelyosaan teemakomentoja, esimerkiksi
seuraavalla tavalla:

\komentoi{usetheme}
\komentoi{usecolortheme}
\komentoi{usefonttheme}
\begin{koodilohkosis}
\usetheme{Bergen}
\usecolortheme{albatross}
\usefonttheme{serif}
\end{koodilohkosis}

\noindent
Teemanvalintakomennoille voi antaa hakasulkeissa valinnaisen argumentin,
jonka avulla määritetään kyseisen teeman asetuksia, jos teema sellaisia
tukee. Asetukset ovat teemakohtaisia, ja niistä voi lukea lisää
dokumenttiluokan ohjekirjasta. Seuraavassa on yleisesimerkki komentojen
rakenteesta:

\komentoi{usetheme}
\komentoi{useoutertheme}
\begin{koodilohkosis}
\usetheme[valitsin]{teeman nimi}
\useoutertheme[valitsin=arvo]{teeman nimi}
\end{koodilohkosis}

\noindent
Fonttiteema \koodi{serif} vaihtaa koko diaesityksen kirjainperheeksi
antiikvan. Sen myötä oletuksena olevaa groteskia (\englanti{sans serif})
ei siis käytetä enää missään. \koodi{serif}\-/ teema sisältää kuitenkin
muutaman valitsimen, joilla tähän voi tehdä pieniä hyödyllisiä
poikkeuksia. Valitsimia on koottu taulukkoon
\ref{tlk/beamer-serif-teema}, ja niitä voi antaa useampia kerralla.

Seuraava esimerkki asettaa groteskin suuriin elementteihin eli
otsikoihin ja pieniin elementteihin eli dian reunojen mahdollisiin
lisätietoihin. Muualla käytetään antiikvaa.

\komentoi{usefonttheme}
\begin{koodilohkosis}
\usefonttheme[stillsansseriflarge, stillsansserifsmall]{serif}
\end{koodilohkosis}

\leijutlk{
  \providecommand{\rivi}{}
  \renewcommand{\rivi}[2]{\koodi{#1} & #2 \\}

  \begin{tabularx}{\linewidth}{lL}
    \toprule
    \ots{Valitsin} & \ots{Merkitys} \\
    \midrule
    \rivi{stillsansseriflarge}{suuret elementit eli otsikot käyttävät
    groteskifonttia}
    \rivi{stillsansserifsmall}{pienet elementit kuten dian reuna\-/
    alueiden lisätiedot käyttävät groteskia}
    \rivi{stillsansseriftext}{normaali teksti groteskilla}
    \rivi{stillsansserifmath}{matematiikkatilassa groteski}
    \bottomrule
  \end{tabularx}
}{
  \caption{\luokka{beamer}\-/ luokan \koodi{serif}\-/ fonttiteeman
    asetusvalitsimia}
  \label{tlk/beamer-serif-teema}
}

\subsection{Muita asetuksia}
\label{luku/beamer-asetuksia}

Jos haluaa vaikuttaa \luokka{beamer}\-/ dokumenttien otsikoiden tai
muiden rakenteellisten osien fontteihin ja väreihin, täytyy käyttää
dokumenttiluokan omia komentoja. Niiden käyttöä neuvotaan tässä
alaluvussa. Sen sijaan diojen sisällä väliaikaiset kirjaintyypin tai
\=/leikkauksen muutokset tehdään samalla tavalla kuin Latexissa
muutenkin. Fonttien tekniikkaa käsitellään luvussa
\ref{luku/kirjaintyypit} ja tekstin korostamisen typografiaa luvussa
\ref{luku/korostus}.

\luokka{beamer}\-/ luokan erityisten tekstielementtien kuten otsikoiden
fonttiin voi vaikuttaa komennolla \komento{setbeamerfont}. Sen
ensimmäinen argumentti on tekstielementin nimi ja toinen argumentti
sisältää fonttiasetukset. Esimerkiksi seuraava komento muuttaa
aloitusdian (\komento{maketitle}) otsikon kirjainperheen ja
\=/leikkauksen:

\komentoi{setbeamerfont}
\begin{koodilohkosis}
\setbeamerfont{title}{family=\rmfamily, series=\bfseries,
  shape=\itshape, size=\Huge}
\end{koodilohkosis}

\leijutlk{
  \providecommand{\rivi}{}
  \renewcommand{\rivi}[2]{\koodi{#1} & #2 \\}

  \begin{tabular}{ll}
    \toprule
    \ots{Tekstielementti} & \ots{Merkitys} \\
    \midrule
    \rivi{title}{aloitusdian otsikko (\komento{maketitle})}
    \rivi{frametitle}{diojen otsikot}
    \rivi{framesubtitle}{diojen alaotsikot}
    \rivi{block title}{\ymparisto{block}\-/ ympäristön otsikot}
    \rivi{normal text}{normaali teksti diojen sisällä}
    \rivi{footnote}{alaviitteet}
    \rivi{item}{luetelmien luetelmamerkit}
    \bottomrule
  \end{tabular}
}{
  \caption{\luokka{beamer}\-/ luokan tekstielementtejä. Elementtien
    nimiä käytetään esimerkiksi komentojen \komento{setbeamerfont} ja
    \komento{setbeamercolor} kanssa}
  \label{tlk/beamer-tekstielementtejä}
}

\noindent
Aloitusdian otsikko on tekstielementti nimeltä \koodi{title}, ja siksi
se oli edellisen komennon ensimmäisenä argumenttina. Muita oleellisia
tekstielementtejä on koottu taulukkoon
\ref{tlk/beamer-tekstielementtejä}.

Tekstielementtien väreihin vaikutetaan komennolla
\komento{setbeamercolor}, joka toimii lähes samalla tavalla kuin edellä
esitelty komento \komento{setbeamerfont}. Värin asettamisessa käytetään
valitsimia \koodi{fg} ja \koodi{bg}, joista ensin mainittu vaihtaa
varsinaisen värin (\englanti{foreground}) ja jälkimmäinen taustavärin
(\englanti{background}). Seuraavassa esimerkissä vaihdetaan diojen
otsikon teksti valkeaksi ja tausta siniseksi:

\komentoi{setbeamercolor}
\begin{koodilohkosis}
\setbeamercolor{frametitle}{fg=white, bg=blue}
\end{koodilohkosis}

\noindent
Värien nimien täytyy olla ennalta määriteltyjä. Perusvärit
(\englanti{\koodi{white}, \koodi{blue}, \koodi{red}} ym.) on määritelty
valmiiksi, mutta lisää värejä voi määritellä ohjeilla, joita kerrotaan
luvussa \ref{luku/korostus-värit}.

Oletuksena luetelmaympäristöt \ymparisto{itemize} ja
\ymparisto{enumerate} latovat luetelmamerkit eri värillä kuin normaalin
tekstin. Väri riippuu käytetystä väriteemasta (\komento{usecolortheme}).
Jos haluaa, että luetelmamerkit ovat samalla värillä kuin normaali
teksti, kannattaa käyttää seuraavan esimerkin komentoa:

\komentoi{setbeamercolor}
\begin{koodilohkosis}
\setbeamercolor{item}{parent={normal text}}
\end{koodilohkosis}

\noindent
Edellisessä esimerkissä viitataan \koodi{item}\-/ nimiseen
tekstielementtiin, joka tarkoittaa luetelmamerkkejä. Valitsin
\koodi{parent} tarkoittaa, että väri peritään toiselta
tekstielementiltä, tässä tapauksessa normaalilta tekstiltä
(\koodi{\englanti{normal text}}). Toki tässäkin voi käyttää valitsimia
\koodi{fg} ja~\koodi{bg}.

Normaalin tekstin taustaväri tarkoittaa koko dian tekstialueen
taustaväriä. Seuraavassa on käytännön esimerkki, joka muuttaa diojen
otsikon taustan kirkkaan vihreäksi (70\,\%) ja tekstialueen taustan
vaalean vihreäksi (30\,\%).

\komentoi{setbeamercolor}
\begin{koodilohkosis}
\setbeamercolor{frametitle} {fg=black, bg=green!90}
\setbeamercolor{normal text}{fg=black, bg=green!30}
\end{koodilohkosis}

\noindent
Numeroimattomien luetelmien eli \ymparisto{itemize}\-/ ympäristön
luetelmamerkki on \luokka{beamer}\-/ dokumenttiluokassa oletuksena
kolmionmuotoinen, mutta asetusta voi muuttaa komennolla
\komento{setbeamertemplate}. Tällä komennolla tehdään sekalaisia diojen
asetuksia, joista annetaan tässä yhteydessä vain pari esimerkkiä.
Luetelmamerkki vaihdetaan seuraavan esimerkin komennoilla.

\komentoi{setbeamertemplate}
\begin{koodilohkosis}
\setbeamertemplate{itemize item}[circle]          % perustaso
\setbeamertemplate{itemize subitem}[square]       % toinen taso
\setbeamertemplate{itemize subsubitem}[triangle]  % kolmas taso
\end{koodilohkosis}

\noindent
Vielä monipuolisemmin voi luetelmamerkkeihin vaikuttaa, kun vaihtaa
hakasulkeissa olevan argumentin tilalle aaltosulkeet. Tällaiseen
argumenttiin voi kirjoittaa suunnilleen mitä hyvänsä Latex\-/ komentoja,
joilla luetelmamerkki tuotetaan. Seuraavassa esimerkissä tätä voimakasta
komentomuotoa käytetään maltillisesti pelkästään ajatusviivan
tuottamiseen.

\komentoi{setbeamertemplate}
\begin{koodilohkosis}
\setbeamertemplate{itemize item}{--}
\end{koodilohkosis}

\noindent
\komento{setbeamertemplate}\-/ komennolla voi asettaa myös diojen
oikeaan alareunaan ladottaviin navigointisymboleihin. Tässä yhteydessä
ei käsitellä niitä sen syvällisemmin, mutta navigointisymbolien
poistaminen onnistuu helposti seuraavalla komennolla:

\komentoi{setbeamertemplate}
\begin{koodilohkosis}
\setbeamertemplate{navigation symbols}{}
\end{koodilohkosis}

\section{Kirjeet}
\label{luku/kirjeet}

Latexin \luokka{letter}\-/ dokumenttiluokka on tarkoitettu kirjeiden
latomiseen. Tyylillisesti se soveltuu ehkä paremmin virallisiin
kirjeisiin kuin henkilökohtaisiin. Esimerkiksi joukkojakelukirjeiden
tekemiseen se voi olla käytännöllinen, koska tietokoneohjelman avulla
voi helposti tuottaa Latex\-/ koodia eli lähes samanlaisia kirjeitä eri
vastaanottajille.

\begin{esimerkki*}
  \komentoi{address}
  \komentoi{cc}
  \komentoi{closing}
  \komentoi{documentclass}
  \komentoi{encl}
  \komentoi{makelabels}
  \komentoi{opening}
  \komentoi{ps}
  \komentoi{signature}
  \komentoi{usepackage}
  \luokkai{letter}
  \ymparistoi{letter}

\begin{koodilohko}
\documentclass{letter}
\usepackage[a4paper]{geometry}
\usepackage{polyglossia}
\setdefaultlanguage{finnish}

\address{Liisa Lähettäjä \\ Katuosoite 1 \\ 00000 Kaupunki}
\signature{Liisa Lähettäjä}

\makelabels

\begin{document}

\begin{letter}{Virpi Vastaanottaja \\ Tiennimi 3 \\ 99999 Kunta}

  \opening{Hei!}

  Tässä kirjeessä ei ole kovin mielenkiintoista sisältöä, mutta tähän
  kohtaan se kirjoitettaisiin.

  \closing{Terveisin}

  \ps{Jk. Tässä on kirjeen jälkikirjoitus.}

  \cc{Mauno Muuhenkilö}
  \encl{lippu, lappu, paperi}

\end{letter}

\end{document}
\end{koodilohko}

  \caption{Latexin \luokka{letter}\-/ dokumenttiluokka on tarkoitettu
    kirjeiden latomiseen}
  \label{esim/letter}
\end{esimerkki*}

Esimerkissä \ref{esim/letter} on yhden kirjeen runko ja oleelliset
komennot. Dokumentin esittelyosassa olevalla \komento{address}\-/
komennolla määritellään lähettäjän nimi ja osoitetiedot. Ne ladotaan
jokaisen kirjeen alkuun. Myös komennolla \komento{signature} ilmaistaan
lähettäjän nimi eli allekirjoitus, joka ladotaan kirjeen loppuun.
Komento \komento{makelabels} aiheuttaa sen, että kaikkien kirjeiden
jälkeen ladotaan sivu (tai useampia), jossa ovat kaikkien
vastaanottajien osoitetiedot. Tämä on tarkoitettu osoitetarrojen
tulostamiseen.

Varsinainen kirjeen sisältö toteutetaan \ymparisto{letter}\-/ ympäristön
avulla. Ympäristölle annetaan yksi argumentti, jossa on kyseisen kirjeen
vastaanottajan nimi ja osoitetiedot. Itse kirje alkaa
\komento{opening}\-/ komennolla, jolla ilmaistaan tervehdys tai muut
kirjeen aloitussanat.

Kirjeen lopussa \komento{closing}\-/ komennolla on sopivaa ilmaista
lopputoivotus tai muu vastaava. Lähettäjän nimi ladotaan sen jälkeen
automaattisesti, jos lähettäjä on ilmaistu aiemmin
\komento{signature}\-/ komennolla. Mahdollisen jälkikirjoituksen voi
ilmaista komennolla \komento{ps}, kirjeen jakelutietoja komennolla
\komento{cc} ja liitteet komennolla \komento{encl}.

Yksi Latex\-/ lähdedokumentti voi sisältää useita \ymparisto{letter}\-/
ympäristöjä, ja jokainen niistä muodostaa erillisen, uudelta sivulta
alkavan kirjeen. Kaikkiin kirjeisiin ladotaan sama lähettäjä, ellei
lähettäjätietoja välillä vaihda \komento{address}\-/\ ja
\komento{signature}\-/ komennoilla.

\section{Muita esimerkkejä}

Pienempiin, rajattuihin erityistarpeisiin on olemassa useita luokkia.
Esimerkiksi yksisivuisten taitettavien lehtisten kuten tapahtuman
käsiohjelmien tekemiseen soveltuu \luokkactan{leaflet}\-/luokka.
\textsc{cd}\-/levyjen kansia voi tehdä \luokkactan{cd}\-/luokan avulla.
Kuvitettuja seinäkalentereita vartenkin on tehty oma luokkansa,
\luokka{wallcalendar}. Kitaran tablatuureja eli otelautakuvaan
perustuvia nuotteja voi tehdä \luokkactan{guitartabs}\-/ luokan avulla.

Saatavilla oleviin dokumenttiluokkiin voi tutustua esimerkiksi
\englanti{Comprehensive Tex Archive Network} (\textsc{ctan})
\=/verkkosivuston avulla.%
\footnote{\url{https://www.ctan.org/topic/class}}

% Tekijä:   Teemu Likonen <tlikonen@iki.fi>
% Lisenssi: Creative Commons Nimeä-JaaSamoin 4.0 Kansainvälinen (CC BY-SA 4.0)
% https://creativecommons.org/licenses/by-sa/4.0/legalcode.fi

\chapter{Muuta tekniikkaa}

Tähän päälukuun on koottu sekalaisia ohjeita, jotka eivät oikein sovi
mihinkään aiempaan päälukuun. Kyse on sellaisesta Latexin tekniikasta,
jota voi hyödyntää monenlaisissa dokumenteissa ja tilanteissa.

\section{Päiväykset ja kellonajat}

Latexissa on muutama komento, joilla voi latoa dokumenttiin
automaattisesti kääntämishetken päivämäärän. Sellaista voidaan tarvita
esimerkiksi dokumentin kansi- tai nimiösivulle ilmaisemaan
julkaisuajankohtaa. Yleisimmin käytetty komento lienee \komento{today},
joka latoo nykyisen kielivalinnan (luku \ref{luku/kieliasetukset})
mukaisen pitkän päivämäärän:

\komentoi{today}
\begin{koodilohkosis}
\today
\end{koodilohkosis}

\begin{tulossis}
  12. kesäkuuta 2023
\end{tulossis}

\noindent
Edellä on esimerkki suomen kielen kokonaisesta päiväysmerkinnästä.
Muilla kielillä komento tuottaa tietenkin jotain muuta. Päivämäärään
sisältyvät lukuarvot voi latoa kahden komennon yhdistelmillä: ensin
annetaan komento \komento{the} ja sen jälkeen päivää, kuukautta tai
vuotta ilmaiseva komento, kuten seuraavasta esimerkistä ilmenee.

\komentoi{the}
\komentoi{day}
\komentoi{month}
\komentoi{year}
\begin{koodilohkosis}
\the\day.\the\month.\the\year
\end{koodilohkosis}

\begin{tulossis}
  12.6.2023
\end{tulossis}

\noindent
Monipuolisempia ajanilmauksia voi latoa paketin \pakettictan{datetime2}
avulla. Se sisältää paljon komentoja ajanilmausten tuottamiseen eri
muodoissa ja eri kielillä. \paketti{datetime2}\-/ paketti pitäisi ladata
kieliasetusten (luku \ref{luku/kieliasetukset}) ja kielipaketin jälkeen,
ja lataamisen yhteydessä voi asettaa, minkä kielten mukaisesti
ajanilmaukset halutaan latoa. Seuraavassa esimerkissä ladataan suomen
(\koodi{finnish}) ja brittienglannin (\koodi{en-GB}) ajanilmaukset sekä
asetetaan, että kellonajoissa ei ole sekunteja mukana.

\pakettii{datetime2}
\begin{koodilohkosis}
% Ensin kielipaketti polyglossia tai babel. Sitten:
\usepackage[finnish, en-GB, showseconds=false]{datetime2}
\end{koodilohkosis}

\noindent
Ajanilmausten kielen tai muun tyylin voi muuttaa komennolla
\komento{DTMsetstyle}, kuten seuraava esimerkki osoittaa. Dokumentin
kääntämishetken päiväys ladotaan komennolla \komento{DTMtoday}, joka
vastaa perus Latexin \komento{today}\-/ komentoa. Nykyhetken kellonaika
saadaan komennolla \komento{DTMcurrenttime}.

\komentoi{DTMsetstyle}
\komentoi{DTMtoday}
\komentoi{DTMcurrenttime}
\begin{koodilohkosis}
\DTMsetstyle{finnish}
Tänään on \DTMtoday\ ja kello on \DTMcurrenttime. \\
\DTMsetstyle{en-GB}
Today is \DTMtoday\ and the time is \DTMcurrenttime.
\end{koodilohkosis}

\begin{tulossis}
  Tänään on 12. kesäkuuta 2023 ja kello on 10.46. \\
  Today is 12th June 2023 and the time is 10:46am.
\end{tulossis}

\noindent
Pitkän päivämäärän sijasta on mahdollista latoa myös pelkistä luvuista
koostuva päiväys. Tämä vaatii, että ajanilmausten tyyliä muutetaan
komennolla \komento{DTMsetstyle}. Esimerkiksi suomen kielelle on
olemassa tyyli \koodi{finnish-numeric} eli numeeristen päiväysten tyyli.

\komentoi{DTMsetstyle}
\komentoi{DTMtoday}
\begin{koodilohkosis}
\DTMsetstyle{finnish-numeric}
\DTMtoday
\end{koodilohkosis}

\begin{tulossis}
  12.6.2023
\end{tulossis}

\noindent
\paketti{datetime2}\-/ paketissa on myös komentoja minkä tahansa
päivämäärän tai kellonajan latomiseen, ei pelkästään nykyhetken.
Peruskomento päiväyksille on \komento{DTMdisplaydate}, jota käytetään
seuraavasti:

\komentoi{DTMdisplaydate}
\begin{koodilohkosis}
\DTMdisplaydate{vuosi}{kuukausi}{päivä}{viikonpäivä}
\end{koodilohkosis}

\noindent
Komennon kaikki argumentit ovat lukuja. Viimeinen argumentti on
viikonpäivän numero: \koodi{0}~on maanantai, \koodi{1}~on tiistai jne.
Kaikki ajanilmaustyylit tai \=/kielet eivät huomioi viikonpäivää
mitenkään. Esimerkiksi suomen kielellä sitä ei huomioida. Argumentiksi
voi asettaa \koodi{-1}, jolloin viikonpäivä jätetään aina huomioimatta.

\komentoi{DTMsetstyle}
\komentoi{DTMdisplaydate}
\begin{koodilohkosis}
\DTMsetstyle{finnish} Suomi julistautui itsenäiseksi
\DTMdisplaydate{1917}{12}{6}{-1}. \\
\DTMsetstyle{en-GB} Finland declared its independence on
\DTMdisplaydate{1917}{12}{6}{-1}.
\end{koodilohkosis}

\begin{tulossis}
  Suomi julistautui itsenäiseksi 6. joulukuuta 1917. \\
  Finland declared its independence on 6th December 1917.
\end{tulossis}

\noindent
Yleensä lienee kätevämpää kirjoittaa kiinteät päivämäärät ja muut
ajanilmaukset ihan normaalina tekstinä eikä minkään komennon avulla.
Komento \komento{DTMdisplaydate} voi kuitenkin olla hyödyllinen silloin,
kun saman Latex\-/ koodin avulla tuotetaan ajanilmauksia useiden
erikielisten dokumenttien osaksi. Tai ehkä kirjoittaja ei tunne
kohdekielen ajanilmausten merkintätapoja ja luottaa, että
\paketti{datetime2}\-/ paketin avulla ne saa oikeanlaiseksi~(?).

Mitä hyvänsä kellonaikoja saa ladottua komennolla
\komento{DTMdisplaytime}, joka huomioi voimassa olevat kieli- tai
tyyliasetukset. Komennolle annetaan kolme argumenttia: tunnit, minuutit
ja sekunnit. Sekunteja ei kuitenkaan näytetä, jos \paketti{datetime2}\-/
paketin lataamisen yhteydessä on käytetty asetusta
\koodi{showseconds=\katk false}.

\komentoi{DTMsetstyle}
\komentoi{DTMdisplaytime}
\begin{koodilohkosis}
\DTMsetstyle{finnish} \DTMdisplaytime{11}{43}{27}
\DTMsetstyle{en-GB}   \DTMdisplaytime{11}{43}{27}
\end{koodilohkosis}

\begin{tulossis}
   11.43 11:43am
\end{tulossis}

\noindent
Paketti \paketti{datetime2} sisältää paljon muitakin ominaisuuksia ja
komentoja ajanilmausten käsittelyyn ja latomiseen. Paketin hyödyt
korostuvat erityisesti silloin, kun tarvitaan automaattisia
ajanilmauksia eri kielillä. Tarkempaa tietoa on luettavissa paketin
ohjekirjasta, mutta seuraavassa on vielä yksi esimerkki tilanteesta,
jossa puhutaan jostakin kuukaudesta, ilman että kirjoitusvaiheessa
tiedetään, mistä kuukaudesta on kyse.

\komentoi{DTMsavenow}
\komentoi{DTMfinnishMonthname}
\komentoi{DTMfetchmonth}
\komentoi{DTMfetchyear}
\begin{koodilohkosis}
\DTMsavenow{nyt}
\DTMfinnishMonthname{\DTMfetchmonth{nyt}}ssa vuonna \DTMfetchyear{nyt}
eli tämän tekstin julkaisun aikaan -- --.
\end{koodilohkosis}

\begin{tulossis}
  \DTMsavenow{nyt}
  \DTMfinnishMonthname{\DTMfetchmonth{nyt}}ssa vuonna \DTMfetchyear{nyt}
  eli tämän tekstin julkaisun aikaan -- --.
\end{tulossis}

\section{Vapaasti sijoitettava sisältö}

Vapaasti sijoitettava sisältö tarkoittaa tässä sellaista tekstiä, kuvia
tai muuta sisältöä, joka ei ole osa normaalia tekstivirtaa.
Tavallisestihan Latex latoo kirjaimia peräkkäin riveiksi ja kappaleiksi
niin, että sisältö virtaa sujuvasti sivulta toiselle.

Paketin \pakettictan{textpos} avulla voi kuitenkin sijoittaa sivulle
sisältöä siten, että se on irrallaan normaalista tekstivirrasta.
Esimerkiksi tekstiä tai kuvia voi sommitella absoluuttisesti tiettyihin
kohtiin sivulle tai kuvan voi sijoittaa normaalin tekstin taakse.

Paketin lataamisen yhteydessä valitaan joko suhteellinen tai
absoluuttinen sisällön sijoittelu, eli käytetäänkö valitsinta
\koodi{absolute} vai ei. Myöhemmin selviää, mitä ne käytännössä
tarkoittavat.

\pakettii{textpos}
\begin{koodilohkosis}
\usepackage{textpos}            % oletus eli suhteellinen sijoittelu
\usepackage[absolute]{textpos}  % absoluuttinen sijoittelu
\end{koodilohkosis}

\noindent
\paketti{textpos}\-/ paketti määrittelee ympäristön nimeltä
\ymparisto{textblock}, jonka sisällä olevat asiat ovat kokonaan tai
osittain irrallaan tavallisesta tekstivirrasta. Ympäristö itse ei vie
sivulta tilaa lainkaan, eli normaalin tekstivirran näkökulmasta se on
näkymätön. Ympäristö vaatii tiettyjä argumentteja, joiden muoto on
seuraava:

\ymparistoi{textblock}
\begin{koodilohkosis}
\begin{textblock}{leveys}(x,y)
  ...
\end{textblock}
\end{koodilohkosis}

\noindent
Ympäristön ensimmäinen argumentti \koodi{leveys} on laatikon leveys, ja
ympäristön toinen argumentti \koodi{(x,y)} ilmaisee sisällön x- ja
y-koordinaatit. Nämä eivät ole mittoja vaan kertoimia eli mittalukuja,
joiden yksikkönä on mitta \mitta{TPHorizModule} (leveys ja
x-koordinaatti) ja \mitta{TPVertModule} (y-koordinaatti). Mitat voi
asettaa \komento{setlength}\-/ komennolla.

\komentoi{setlength}
\mittai{TPHorizModule}
\mittai{TPVertModule}
\begin{koodilohkosis}
\setlength{\TPHorizModule}{10mm}  % vaakasuuntainen yksikkö
\setlength{\TPVertModule} {10mm}  % pystysuuntainen yksikkö
\end{koodilohkosis}

\noindent
Edellä olevat komennot asettavat sekä vaaka- että pystysuuntaiseksi
yksiköksi 10\,mm, joten seuraavassa esimerkissä laatikon leveydeksi
tulee 50\,mm ja sen sijainti on 70\,mm oikealle ja 35\,mm alas.

\ymparistoi{textblock}
\begin{koodilohkosis}
\begin{textblock}{5}(7,3.5)
  Tämä teksti ei ole osa normaalia tekstivirtaa.
\end{textblock}
\end{koodilohkosis}

\noindent
Jos \paketti{textpos}\-/ paketti ladattiin suhteellisessa
sijoittelutilassa eli ilman \koodi{absolute}\-/ valitsinta,
\ymparisto{textblock}\-/ ympäristön x- ja y-koordinaattien origo on
juuri siinä kohdassa, missä ympäristö sijaitsee sivun normaalissa
tekstivirrassa. Niinpä esimerkiksi miinusmerkkinen y-koordinaatti
sijoittaa ympäristön sisältöä kyseisen kohdan yläpuolelle.

Jos paketti ladattiin absoluuttisessa tilassa eli käyttämällä
\koodi{absolute}\-/ valitsinta, on koordinaattien origo nykyisen sivun
vasemmassa ylänurkassa (oletuksena). Origon voi siirtää komennolla
\komento{textblockorigin}, ja sille argumentiksi annetut koordinaatit
siirtävät origon vasemmasta ylänurkasta oikealle ja alas. Seuraava
esimerkki siirtää origon leveyssuunnassa sivun keskelle ja
pystysuunnassa 5\,cm alas sivun yläreunasta.

\komentoi{textblockorigin}
\mittai{paperwidth}
\begin{koodilohkosis}
\textblockorigin{.5\paperwidth}{5cm}
\end{koodilohkosis}

\noindent
\ymparisto{textblock}\-/ ympäristön luoman laatikon rajat voi tarkistaa
antamalla paketin lataamisen yhteydessä valitsimen \koodi{showboxes}.
Näin laatikon reunaviivat piirretään näkyviin. Varsinkin sommittelutyön
alkuvaiheessa on usein hyödyllistä nähdä, missä laatikot tarkalleen
ovat.

\pakettii{textpos}
\begin{koodilohkosis}
\usepackage[showboxes]{textpos}
\end{koodilohkosis}

\noindent
Kuten edellä jo todettiin, ympäristö \ymparisto{textblock} ei käytä
argumenteissaan mittoja vaan kertoimia, jotka ovat suhteessa mittoihin
\mitta{TPHorizModule} ja \mitta{TPVertModule}. Ympäristöstä on kuitenkin
myös tähdellinen versio \ymparisto{textblock*}, jonka argumenteissa
käytetään nimenomaan mittoja.

\ymparistoi{textblock*}
\begin{koodilohkosis}
\begin{textblock*}{10cm}(3cm,2cm)
  Tämän laatikon leveys on 10 cm, ja se sijaitsee origosta 3 cm
  oikealle ja 2 cm alas.
\end{textblock*}
\end{koodilohkosis}

\noindent
Ympäristöt latovat sisältönsä sivulle kerroksittain siten, että sivulla
ensin oleva \ymparisto{textblock}\-/ sijaitsee alimpana ja seuraavat
ympäristöt ladotaan edellisten päälle -- mikäli niiden sisältö
ylipäätään sattuu päällekkäin.

Absoluuttisessa sijoittelutilassa (paketin \koodi{absolute}\-/ valitsin)
\ymparisto{textblock}\-/ ympäristöjen sisältö sijaitsee kaiken normaalin
tekstin alla. Tämän voi kuitenkin muuttaa käyttämällä paketin valitsinta
\koodi{overlay}, joka tuo ympäristöjen sisällön normaalin tekstin
päälle.

Paketin asetuksia voi muuttaa lataamisen jälkeen komennolla
\komento{TPoptions}. Komennon argumenttiin kirjoitetaan valitsimia, ja
niiden arvoksi annetaan joko \koodi{true} (kytke päälle) tai
\koodi{false} (kytke pois).

\komentoi{TPoptions}
\begin{koodilohkosis}
\TPoptions{absolute=false, showboxes=true}
\end{koodilohkosis}

\noindent
Tässä alaluvussa esiteltiin ehkä tärkeimmät \paketti{textpos}\-/ paketin
ominaisuudet ja niiden käyttäminen. Hieman muitakin ominaisuuksia on.
Esimerkiksi ympäristöjen sijoittelun kohdistusruudukon saa näkyviin ja
määriteltyä omalla komennollaan. Laatikoiden reunaviivojen leveyden ja
värin sekä sisällön marginaalin voi asettaa. Lisätietoa on luettavissa
paketin ohjekirjasta.

\setcounter{secnumdepth}{-1}

\chapter{Kirjallisuutta}
\label{luku/kirjallisuutta}

\printbibliography[heading=none]

\chapter{Asiahakemistot}
\label{luku/asiahakemisto}

\printindex[dokumenttiluokat]
\indexprologue{\noindent Tekstitilan ja matematiikkatilan komennot on
  ladottu eri väreillä: \komentox{teksti} ja \mkomentox{matematiikka}.}
\printindex[komennot]
\printindex[laskurit]
\printindex[mitat]
\printindex[paketit]
\indexprologue{\noindent Tekstitilan ja matematiikkatilan ympäristöt on
  ladottu eri väreillä: \ymparistox{teksti} ja
  \mymparistox{matematiikka}.}
\printindex[ymparistot]

\end{document}
