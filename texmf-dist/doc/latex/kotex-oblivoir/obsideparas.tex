\documentclass{oblivoir}

\ifLuaOrXeTeX
\setmainfont{STIX Two Text}
\setkomainfont(Noto Serif KR)[Scale=MatchUppercase]
\fi

\usepackage{fapapersize}
\usefapapersize{210mm,297mm,25mm,50mm,30mm,*}
\setlength\marginparwidth{\dimexpr\foremargin-5em\relax}

\usepackage{ragged2e}
\usepackage{etoolbox}
\newcounter{sidebarcnt}
\counterwithin*{sidebarcnt}{section}

\makeatletter
\renewcommand{\sidebar}[1]{%
  \refstepcounter{sidebarcnt}%
  \textsuperscript{\thesidebarcnt}%
  \insert\sideins{%
    \hsize\sidebarwidth
    \@parboxrestore
    \def\baselinestretch{\m@m@footnote@spacing}%
    \m@mwhich@margin{\m@msidebar@margin}%
    \sidebarform\sidebarfont
    \splittopskip=\ht\strutbox
    \splitmaxdepth=\dp\strutbox % doesn't do anything useful
    \allowbreak
    \prevdepth=\dp\strutbox    % supersedes a "top-strut"
    \vskip-\parskip
    \leavevmode\textsuperscript{\thesidebarcnt}#1%
    \ifvmode\else
      \unskip\@finalstrut\strutbox
    \fi\par
    \ifdim\prevdepth>\dp\strutbox \prevdepth=\dp\strutbox \fi
    \ifdim\prevdepth>99\p@
      \nobreak
      \vskip-\prevdepth
      \allowbreak
      \vskip\dp\strutbox
    \fi
    \vskip\sidebarvsep}}
\makeatother

\begin{document}

\title{oblivoir의 margin 문단}
\author{Nova de Hi}
\date{2021/10/30}
\maketitle

\begin{abstract}
oblivoir의 margin 문단을 소개한다. 이 margin 문단의 위치와 형태를
결정하는 여러 parameter에 대해서는 별도로 다룰 것이다(memoir 매뉴얼을
참고하라). 이 문서의 목적은 margin 문단의 종류와 개요를 이해하는 데 있다.

memoir에는 marginfigure나 margintable도 있지만 이것은 marginpar에
figure/table을 넣는 것이므로 특별한 것이 없어서 이 문서에서는 생략한다.
\end{abstract}

\tableofcontents*

\newpage

\footmarkstyle{\textsuperscript{#1}}
\setlength\footmarksep{0em}
\setlength\footmarkwidth{0pt}
\setlength\footparindent{0pt}
\footnotesinmargin

\section{footnotes in margin}

\begin{framed}
footnotes in margin (\cs{footnotesinmargin})은 \cs{footnote}를 하단이 아니라
margin에 두는 것이다. 스타일은 footnote 설정을 거의 그대로 따른다. 여기서는 footnote 번호가
찍히는 위치를 조금 바꾸었다. 
\end{framed}

경성학교 영어 교사 이형식\footnote{은 오후 두시 사년급 영어 시간을 마치고 내려쪼이는 유월 볕에 땀을 흘리면서 안동 김장로의 집으로 간다.} 김장로의 딸 선형(善馨)이가 명년 미국 유학을 가기 위하여 영어를 준비할 차로 이형식을 매일 한 시간씩 가정교사로 고빙\footnote{하여 오늘 오후 세시부터 수업을 시작하게 되었음이라.}

이형식은 아직 독신이라, 남의 \footnote{여자와 가까이 교제하여 본 적이 없고 이렇게 순결한 청년이 흔히 그러한 모양으로 젊은 여자를} 대하면 자연 수줍은 생각이 나서 얼굴이 확확 달며 고개가 저절로 숙여진다. 남자로 생겨나서 이러함이 못생겼다면 못생겼다고도 하려니와, 여자를 보면 아무러한 핑계를 얻어서라도 가까이 가려 하고, 말 한마디라도 하여 보려 하는 잘난 사람들보다는 나으리라.

형식은 여러 가지 생각을 한다. 우선 처음 만나서 어떻게 인사를 할까. 남자 남자 간에 하는 모양으로, ‘처음 보입니다. 저는 이형식이올시다’ 이렇게 할까. 그러나 잠시라도 나는 가르치는 자요, 저는 배우는 자라, 그러면 미상불 \footnote{무슨 차별이 있지나 아니할까.} 저편에서 먼저 내게 인사를 하거든 그제야 나도 인사를 하는 것이 마땅하지 아니할까. 그것은 그러려니와 교수하는 방법은 어떻게나 할는지. 어제 김장로에게 그 청탁을 들은 뒤로 지금껏 생각하건마는 무슨 묘방이 아니 생긴다. 가운데 책상을 하나 놓고, 거기 마주앉아서 가르칠까. 그러면 입김과 입김이 서로 마주치렷다. 혹 저편 히사시가미(양갈래로 딴 머릿단)가 내 이마에 스칠 때도 있으렷다. 책상 아래에서 무릎과 무릎이 가만히 마주 닿기도 하렷다. 이렇게 생각하고 형식은 얼굴이 붉어지며\footnote{혼자 빙긋 웃었다. 아니 아니? 그러다가 만일 마음으로라도 죄를 범하게 되면 어찌하게.} 옳다? 될 수 있는 대로 책상에서 멀리 떠나 앉겠다. 만일 저편 무릎이 내게 닿거든 깜짝 놀라며 내 무릎을 치우리라. 그러나 내 입에서 무슨 냄새가 나면 여자에게 대하여 실례라, 점심 후에는 아직 담배는 아니 먹었건마는, 하고 손으로 입을 가리우고 입김을 후 내어 불어 본다. 그 입김이 손바닥에 반사되어 코로 들어가면 냄새의 유무를 시험할 수 있음이라. 형식은, 아뿔싸! 내가 어찌하여 이러한 생각을 하는가, 내 마음이 이렇게 약하던가 하면서 두 주먹을 불끈 쥐고 전신에 힘을 주어 이러한 약한 생각을 떼어 버리려 하나, 가슴속에는 이상하게 불길이 확확 일어난다. 이때에,

“미스터 리, 어디로 가는가” \footnote{하는 소리에 깜짝 놀라 고개를 들었다.} 쾌활하기로 동류간에 유명한 신우선(申友善)이가 대팻밥 모자를 갖춰 쓰고 활개를 치며 내려온다. 형식은 자기 마음속을 꿰뚫어보지나 아니한가 하여 두 뺨이 한번 더 후끈하는 것을 겨우 참고 지어서 쾌활하게 웃으면서, 

\newpage

\section{marginpar}

\begin{framed}
\cs{marginpar}는 말 그대로 margin에 어떤 문단을 두는 것이다. 문단의 속성(폰트와 정렬 등)은
본문 속성을 그대로 가진다. marginpar는 float인데 충돌을 방지하기 위하여 marginpar가 overlap되면 이를 회피하는 설정을 가지고 있다. 경우에 따라 다음 페이지로 이동하기도 한다.
\end{framed}

경성학교 영어 교사 이형식\marginpar{은 오후 두시 사년급 영어 시간을 마치고 내려쪼이는 유월 볕에 땀을 흘리면서 안동 김장로의 집으로 간다.} 김장로의 딸 선형(善馨)이가 명년 미국 유학을 가기 위하여 영어를 준비할 차로 이형식을 매일 한 시간씩 가정교사로 고빙\marginpar{하여 오늘 오후 세시부터 수업을 시작하게 되었음이라.}

이형식은 아직 독신이라, 남의 \marginpar{여자와 가까이 교제하여 본 적이 없고 이렇게 순결한 청년이 흔히 그러한 모양으로 젊은 여자를} 대하면 자연 수줍은 생각이 나서 얼굴이 확확 달며 고개가 저절로 숙여진다. 남자로 생겨나서 이러함이 못생겼다면 못생겼다고도 하려니와, 여자를 보면 아무러한 핑계를 얻어서라도 가까이 가려 하고, 말 한마디라도 하여 보려 하는 잘난 사람들보다는 나으리라.

형식은 여러 가지 생각을 한다. 우선 처음 만나서 어떻게 인사를 할까. 남자 남자 간에 하는 모양으로, ‘처음 보입니다. 저는 이형식이올시다’ 이렇게 할까. 그러나 잠시라도 나는 가르치는 자요, 저는 배우는 자라, 그러면 미상불 \marginpar{무슨 차별이 있지나 아니할까.} 저편에서 먼저 내게 인사를 하거든 그제야 나도 인사를 하는 것이 마땅하지 아니할까. 그것은 그러려니와 교수하는 방법은 어떻게나 할는지. 어제 김장로에게 그 청탁을 들은 뒤로 지금껏 생각하건마는 무슨 묘방이 아니 생긴다. 가운데 책상을 하나 놓고, 거기 마주앉아서 가르칠까. 그러면 입김과 입김이 서로 마주치렷다. 혹 저편 히사시가미(양갈래로 딴 머릿단)가 내 이마에 스칠 때도 있으렷다. 책상 아래에서 무릎과 무릎이 가만히 마주 닿기도 하렷다. 이렇게 생각하고 형식은 얼굴이 붉어지며\marginpar{혼자 빙긋 웃었다. 아니 아니? 그러다가 만일 마음으로라도 죄를 범하게 되면 어찌하게.} 옳다? 될 수 있는 대로 책상에서 멀리 떠나 앉겠다. 만일 저편 무릎이 내게 닿거든 깜짝 놀라며 내 무릎을 치우리라. 그러나 내 입에서 무슨 냄새가 나면 여자에게 대하여 실례라, 점심 후에는 아직 담배는 아니 먹었건마는, 하고 손으로 입을 가리우고 입김을 후 내어 불어 본다. 그 입김이 손바닥에 반사되어 코로 들어가면 냄새의 유무를 시험할 수 있음이라. 형식은, 아뿔싸! 내가 어찌하여 이러한 생각을 하는가, 내 마음이 이렇게 약하던가 하면서 두 주먹을 불끈 쥐고 전신에 힘을 주어 이러한 약한 생각을 떼어 버리려 하나, 가슴속에는 이상하게 불길이 확확 일어난다. 이때에,

“미스터 리, 어디로 가는가” \marginpar{하는 소리에 깜짝 놀라 고개를 들었다.} 쾌활하기로 동류간에 유명한 신우선(申友善)이가 대팻밥 모자를 갖춰 쓰고 활개를 치며 내려온다. 형식은 자기 마음속을 꿰뚫어보지나 아니한가 하여 두 뺨이 한번 더 후끈하는 것을 겨우 참고 지어서 쾌활하게 웃으면서, 

\newpage

\sideparmargin{outer}
\renewcommand*\sideparfont{\footnotesize\sffamily}

\section{sidepar}

\begin{framed}
\cs{sidepar}는 여러 면에서 marginpar와 비슷하지만 marginpar의 floating 기능을
제외한 것이다. 이렇게 함으로써 최대한 원래 위치와 가까운 곳에 식자되기는 하지만 
overlap이 발생해도 그대로 찍는다. 또 하나의 차이는 ``좁은 행간''이 적용되고 \cs{raggedright} 정렬한다는 것이다. 이 설정은 물론 바꿀 수 있다.
\end{framed}

경성학교 영어 교사 이형식\sidepar{은 오후 두시 사년급 영어 시간을 마치고 내려쪼이는 유월 볕에 땀을 흘리면서 안동 김장로의 집으로 간다.} 김장로의 딸 선형(善馨)이가 명년 미국 유학을 가기 위하여 영어를 준비할 차로 이형식을 매일 한 시간씩 가정교사로 고빙\sidepar{하여 오늘 오후 세시부터 수업을 시작하게 되었음이라.}

이형식은 아직 독신이라, 남의 \sidepar{여자와 가까이 교제하여 본 적이 없고 이렇게 순결한 청년이 흔히 그러한 모양으로 젊은 여자를} 대하면 자연 수줍은 생각이 나서 얼굴이 확확 달며 고개가 저절로 숙여진다. 남자로 생겨나서 이러함이 못생겼다면 못생겼다고도 하려니와, 여자를 보면 아무러한 핑계를 얻어서라도 가까이 가려 하고, 말 한마디라도 하여 보려 하는 잘난 사람들보다는 나으리라.

형식은 여러 가지 생각을 한다. 우선 처음 만나서 어떻게 인사를 할까. 남자 남자 간에 하는 모양으로, ‘처음 보입니다. 저는 이형식이올시다’ 이렇게 할까. 그러나 잠시라도 나는 가르치는 자요, 저는 배우는 자라, 그러면 미상불 \sidepar{무슨 차별이 있지나 아니할까.} 저편에서 먼저 내게 인사를 하거든 그제야 나도 인사를 하는 것이 마땅하지 아니할까. 그것은 그러려니와 교수하는 방법은 어떻게나 할는지. 어제 김장로에게 그 청탁을 들은 뒤로 지금껏 생각하건마는 무슨 묘방이 아니 생긴다. 가운데 책상을 하나 놓고, 거기 마주앉아서 가르칠까. 그러면 입김과 입김이 서로 마주치렷다. 혹 저편 히사시가미(양갈래로 딴 머릿단)가 내 이마에 스칠 때도 있으렷다. 책상 아래에서 무릎과 무릎이 가만히 마주 닿기도 하렷다. 이렇게 생각하고 형식은 얼굴이 붉어지며\sidepar{혼자 빙긋 웃었다. 아니 아니? 그러다가 만일 마음으로라도 죄를 범하게 되면 어찌하게.} 옳다? 될 수 있는 대로 책상에서 멀리 떠나 앉겠다. 만일 저편 무릎이 내게 닿거든 깜짝 놀라며 내 무릎을 치우리라. 그러나 내 입에서 무슨 냄새가 나면 여자에게 대하여 실례라, 점심 후에는 아직 담배는 아니 먹었건마는, 하고 손으로 입을 가리우고 입김을 후 내어 불어 본다. 그 입김이 손바닥에 반사되어 코로 들어가면 냄새의 유무를 시험할 수 있음이라. 형식은, 아뿔싸! 내가 어찌하여 이러한 생각을 하는가, 내 마음이 이렇게 약하던가 하면서 두 주먹을 불끈 쥐고 전신에 힘을 주어 이러한 약한 생각을 떼어 버리려 하나, 가슴속에는 이상하게 불길이 확확 일어난다. 이때에,

“미스터 리, 어디로 가는가” \sidepar{하는 소리에 깜짝 놀라 고개를 들었다.} 쾌활하기로 동류간에 유명한 신우선(申友善)이가 대팻밥 모자를 갖춰 쓰고 활개를 치며 내려온다. 형식은 자기 마음속을 꿰뚫어보지나 아니한가 하여 두 뺨이 한번 더 후끈하는 것을 겨우 참고 지어서 쾌활하게 웃으면서, 

\newpage

\setlength{\sidebarwidth}{\dimexpr\foremargin-2em\relax}
\setlength{\sidebartopsep}{\dimexpr-\uppermargin+2\baselineskip\relax}
\setlength{\sidebarvsep}{.5ex}
\renewcommand*\sidebarfont{\footnotesize}
\renewcommand*\sidebarform{\RaggedRight}

\section{sidebar}

\begin{framed}
\cs{sidebar}는 \cs{sidepar}와 유사한데 side column의 상단에서부터 배열한다.
페이지가 넘어가지 않기 때문에 이 컬럼이 넘치면 이상한 결과가 될 수 있다.
이 보기에서는 마치 주석과 비슷하게 번호를 붙이고 시작 위치를 상단으로 더 올려붙이는
방법을 보였다. 이를 위하여 (preamble 참조) \cs{sidebar} 명령을 조금 수정했다.
\end{framed}

경성학교 영어 교사 이형식\sidebar{은 오후 두시 사년급 영어 시간을 마치고 내려쪼이는 유월 볕에 땀을 흘리면서 안동 김장로의 집으로 간다.} 김장로의 딸 선형(善馨)이가 명년 미국 유학을 가기 위하여 영어를 준비할 차로 이형식을 매일 한 시간씩 가정교사로 고빙\sidebar{하여 오늘 오후 세시부터 수업을 시작하게 되었음이라.}

이형식은 아직 독신이라, 남의 \sidebar{여자와 가까이 교제하여 본 적이 없고 이렇게 순결한 청년이 흔히 그러한 모양으로 젊은 여자를} 대하면 자연 수줍은 생각이 나서 얼굴이 확확 달며 고개가 저절로 숙여진다. 남자로 생겨나서 이러함이 못생겼다면 못생겼다고도 하려니와, 여자를 보면 아무러한 핑계를 얻어서라도 가까이 가려 하고, 말 한마디라도 하여 보려 하는 잘난 사람들보다는 나으리라.

형식은 여러 가지 생각을 한다. 우선 처음 만나서 어떻게 인사를 할까. 남자 남자 간에 하는 모양으로, ‘처음 보입니다. 저는 이형식이올시다’ 이렇게 할까. 그러나 잠시라도 나는 가르치는 자요, 저는 배우는 자라, 그러면 미상불 \sidebar{무슨 차별이 있지나 아니할까.} 저편에서 먼저 내게 인사를 하거든 그제야 나도 인사를 하는 것이 마땅하지 아니할까. 그것은 그러려니와 교수하는 방법은 어떻게나 할는지. 어제 김장로에게 그 청탁을 들은 뒤로 지금껏 생각하건마는 무슨 묘방이 아니 생긴다. 가운데 책상을 하나 놓고, 거기 마주앉아서 가르칠까. 그러면 입김과 입김이 서로 마주치렷다. 혹 저편 히사시가미(양갈래로 딴 머릿단)가 내 이마에 스칠 때도 있으렷다. 책상 아래에서 무릎과 무릎이 가만히 마주 닿기도 하렷다. 이렇게 생각하고 형식은 얼굴이 붉어지며\sidebar{혼자 빙긋 웃었다. 아니 아니? 그러다가 만일 마음으로라도 죄를 범하게 되면 어찌하게.} 옳다? 될 수 있는 대로 책상에서 멀리 떠나 앉겠다. 만일 저편 무릎이 내게 닿거든 깜짝 놀라며 내 무릎을 치우리라. 그러나 내 입에서 무슨 냄새가 나면 여자에게 대하여 실례라, 점심 후에는 아직 담배는 아니 먹었건마는, 하고 손으로 입을 가리우고 입김을 후 내어 불어 본다. 그 입김이 손바닥에 반사되어 코로 들어가면 냄새의 유무를 시험할 수 있음이라. 형식은, 아뿔싸! 내가 어찌하여 이러한 생각을 하는가, 내 마음이 이렇게 약하던가 하면서 두 주먹을 불끈 쥐고 전신에 힘을 주어 이러한 약한 생각을 떼어 버리려 하나, 가슴속에는 이상하게 불길이 확확 일어난다. 이때에,

“미스터 리, 어디로 가는가” \sidebar{하는 소리에 깜짝 놀라 고개를 들었다.} 쾌활하기로 동류간에 유명한 신우선(申友善)이가 대팻밥 모자를 갖춰 쓰고 활개를 치며 내려온다. 형식은 자기 마음속을 꿰뚫어보지나 아니한가 하여 두 뺨이 한번 더 후끈하는 것을 겨우 참고 지어서 쾌활하게 웃으면서, 

\newpage

\setsidefootheight{\textheight}
\setlength\sidefootwidth{\sidebarwidth}
\renewcommand*\sidefootform{\RaggedRight}

\section{sidefootnote}

\begin{framed}
이것은 \cs{footnotesinmargin}과는 다른 \cs{sidefootnote}라는 명령으로 식자한다.
페이지의 하단을 기준으로 배열하는 것이 특징이다. 실제로 해당 페이지의 주석을 페이지
우측 하단에 두려 할 때 쓴다. 여기서는 sidefootnote column의 길이를 늘리는 방법을 보였다.
\end{framed}

경성학교 영어 교사 이형식\sidefootnote{은 오후 두시 사년급 영어 시간을 마치고 내려쪼이는 유월 볕에 땀을 흘리면서 안동 김장로의 집으로 간다.} 김장로의 딸 선형(善馨)이가 명년 미국 유학을 가기 위하여 영어를 준비할 차로 이형식을 매일 한 시간씩 가정교사로 고빙\sidefootnote{하여 오늘 오후 세시부터 수업을 시작하게 되었음이라.}

이형식은 아직 독신이라, 남의 \sidefootnote{여자와 가까이 교제하여 본 적이 없고 이렇게 순결한 청년이 흔히 그러한 모양으로 젊은 여자를} 대하면 자연 수줍은 생각이 나서 얼굴이 확확 달며 고개가 저절로 숙여진다. 남자로 생겨나서 이러함이 못생겼다면 못생겼다고도 하려니와, 여자를 보면 아무러한 핑계를 얻어서라도 가까이 가려 하고, 말 한마디라도 하여 보려 하는 잘난 사람들보다는 나으리라.

형식은 여러 가지 생각을 한다. 우선 처음 만나서 어떻게 인사를 할까. 남자 남자 간에 하는 모양으로, ‘처음 보입니다. 저는 이형식이올시다’ 이렇게 할까. 그러나 잠시라도 나는 가르치는 자요, 저는 배우는 자라, 그러면 미상불 \sidefootnote{무슨 차별이 있지나 아니할까.} 저편에서 먼저 내게 인사를 하거든 그제야 나도 인사를 하는 것이 마땅하지 아니할까. 그것은 그러려니와 교수하는 방법은 어떻게나 할는지. 어제 김장로에게 그 청탁을 들은 뒤로 지금껏 생각하건마는 무슨 묘방이 아니 생긴다. 가운데 책상을 하나 놓고, 거기 마주앉아서 가르칠까. 그러면 입김과 입김이 서로 마주치렷다. 혹 저편 히사시가미(양갈래로 딴 머릿단)가 내 이마에 스칠 때도 있으렷다. 책상 아래에서 무릎과 무릎이 가만히 마주 닿기도 하렷다. 이렇게 생각하고 형식은 얼굴이 붉어지며\sidefootnote{혼자 빙긋 웃었다. 아니 아니? 그러다가 만일 마음으로라도 죄를 범하게 되면 어찌하게.} 옳다? 될 수 있는 대로 책상에서 멀리 떠나 앉겠다. 만일 저편 무릎이 내게 닿거든 깜짝 놀라며 내 무릎을 치우리라. 그러나 내 입에서 무슨 냄새가 나면 여자에게 대하여 실례라, 점심 후에는 아직 담배는 아니 먹었건마는, 하고 손으로 입을 가리우고 입김을 후 내어 불어 본다. 그 입김이 손바닥에 반사되어 코로 들어가면 냄새의 유무를 시험할 수 있음이라. 형식은, 아뿔싸! 내가 어찌하여 이러한 생각을 하는가, 내 마음이 이렇게 약하던가 하면서 두 주먹을 불끈 쥐고 전신에 힘을 주어 이러한 약한 생각을 떼어 버리려 하나, 가슴속에는 이상하게 불길이 확확 일어난다. 이때에,

“미스터 리, 어디로 가는가” \sidefootnote{하는 소리에 깜짝 놀라 고개를 들었다.} 쾌활하기로 동류간에 유명한 신우선(申友善)이가 대팻밥 모자를 갖춰 쓰고 활개를 치며 내려온다. 형식은 자기 마음속을 꿰뚫어보지나 아니한가 하여 두 뺨이 한번 더 후끈하는 것을 겨우 참고 지어서 쾌활하게 웃으면서, 


\end{document}
