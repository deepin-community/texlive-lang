% l2picfaq.tex
%
% (c) 2005 (06-10) by Dominik Bischoff.
% steht unter der "GNU free documentation license 1.3"
% Diese ist (falls nicht mitgeliefert) zu finden auf:
% 
% http://www.gnu.org/licenses/fdl.txt
%
%%%%%%%%%%%%%%%%%%%%%%%%%%%%%%%%%%%%%%%%%%%%%%%%%%%%%%%%%%%%%%%%%%%
% Übersicht und Codebeispiele für die im Zusammenhang mit Bildern
% und LaTeX am häufigsten auftauchenden Probleme
%%%%%%%%%%%%%%%%%%%%%%%%%%%%%%%%%%%%%%%%%%%%%%%%%%%%%%%%%%%%%%%%%%%

% aktuelle Version
\newcommand{\version}{Version 1.50 vom \today}

%++++++++++++++++++++++++++++++++++++++++++++++++++++++++++++++++++
%+++ Dokumentklasse +++++++++++++++++++++++++++++++++++++++++++++++
%++++++++++++++++++++++++++++++++++++++++++++++++++++++++++++++++++
% verwende KOMA-Scipt
% Doku: http://www.ctan.org/tex-archive/macros/latex/contrib/koma-script/scrguide.pdf
\documentclass[%
a4paper, % Seitenformat
12pt, % Schriftgrösse
DIV0, % Ränder und Co - automatisch berechnet
final, % Endversion; Zum Testen: ersetze final durch draft (schwarze Kästchen bei zu langen Zeilen)
halfparskip % Absatzabstand: halbe Zeile, kein Zeileneinzug
]{scrartcl} % verwende die scrartcl-Klasse

% verwendete Paketversionen im log-File anzeigen
\listfiles


%++++++++++++++++++++++++++++++++++++++++++++++++++++++++++++++++++
%+++ Sprache ++++++++++++++++++++++++++++++++++++++++++++++++++++++
%++++++++++++++++++++++++++++++++++++++++++++++++++++++++++++++++++
% neue Deutsche Rechtschreibung: Trennregeln!
\usepackage[ngerman]{babel}


%++++++++++++++++++++++++++++++++++++++++++++++++++++++++++++++++++
%+++ Encoding +++++++++++++++++++++++++++++++++++++++++++++++++++++
%++++++++++++++++++++++++++++++++++++++++++++++++++++++++++++++++++
% verwende Vektorschriften, falls vorhanden
\usepackage[T1]{fontenc}

% Encodierung der Quelldatei:
% UTF-8-Encoding
\usepackage[utf8]{inputenc}


%++++++++++++++++++++++++++++++++++++++++++++++++++++++++++++++++++
%+++ Schriften ++++++++++++++++++++++++++++++++++++++++++++++++++++
%++++++++++++++++++++++++++++++++++++++++++++++++++++++++++++++++++
% verwendete Schriften
% Doku: http://www.ctan.org/tex-archive/macros/latex/required/psnfss/psnfss2e.pdf
\usepackage{charter}
\usepackage[scaled=.92]{helvet}
\usepackage{courier}

% besseres Schriftbild
% Doku: http://www.ctan.org/tex-archive/macros/latex/contrib/microtype/microtype.pdf
\usepackage{microtype}


%++++++++++++++++++++++++++++++++++++++++++++++++++++++++++++++++++
%+++ Quellcode ++++++++++++++++++++++++++++++++++++++++++++++++++++
%++++++++++++++++++++++++++++++++++++++++++++++++++++++++++++++++++
% Quellcode-Anzeige
% Doku: http://www.ctan.org/tex-archive/macros/latex/contrib/listings/listings.pdf
\usepackage{listings}

% Deutsche Spezialzeichen
\lstset{literate=%
{Ö}{{\"O}}1
{Ä}{{\"A}}1
{Ü}{{\"U}}1
{ü}{{\"u}}1
{ä}{{\"a}}1
{ö}{{\"o}}1
}

%++++++++++++++++++++++++++++++++++++++++++++++++++++++++++++++++++
%+++ Bilder +++++++++++++++++++++++++++++++++++++++++++++++++++++++
%++++++++++++++++++++++++++++++++++++++++++++++++++++++++++++++++++
% Einbinden von Bildern
\usepackage{graphicx}


%++++++++++++++++++++++++++++++++++++++++++++++++++++++++++++++++++
%+++ mehrspaltiger Text +++++++++++++++++++++++++++++++++++++++++++
%++++++++++++++++++++++++++++++++++++++++++++++++++++++++++++++++++
\usepackage{multicol}
%\begin{multicols}{Spaltenzahl}[''titel''][''Abstand'']
%...
%\end{multicols}



%++++++++++++++++++++++++++++++++++++++++++++++++++++++++++++++++++
%+++ Kopf- und Fusszeile ++++++++++++++++++++++++++++++++++++++++++
%++++++++++++++++++++++++++++++++++++++++++++++++++++++++++++++++++
% verwende scrpage2 aus KOMA-Script
% Doku: http://www.ctan.org/tex-archive/macros/latex/contrib/koma-script/scrguide.pdf
\usepackage{scrpage2}

% Definition der Kopfzeilen
\ihead{\sffamily Bilder einfügen in \LaTeX{}: Ein How-To (l2picfaq.pdf)}
\chead{}
\ohead{\sffamily Seite \thepage{}}
\ifoot{\sffamily (c) 2005 (06-10) by Dominik Bischoff}
\cfoot{}
\ofoot{\sffamily \version}

\setheadsepline{0.5pt} %Dicke der Trennlinie Kopfzeile - Text
\setfootsepline{0.5pt} %Dicke der Trennlinie Fusszeile - Text

\pagestyle{scrheadings}

% Höhe der Kopfzeile (bzw: Abstand der Kopfzeile zum Seitenanfang)
\setlength{\headheight}{3\baselineskip}


%++++++++++++++++++++++++++++++++++++++++++++++++++++++++++++++++++
%+++ Listen +++++++++++++++++++++++++++++++++++++++++++++++++++++++
%++++++++++++++++++++++++++++++++++++++++++++++++++++++++++++++++++
% Abstände bei itemize-Umbgebung zwischen den Punkten
% um einzelne zu ändern: 
% \begin{itemize}\itemsep0pt
\let\origitemize\itemize
\def\itemize{\origitemize\itemsep0.1pt}
% selbiges für enumerate
\let\origenumerate\enumerate
\def\enumerate{\origenumerate\itemsep0.1em}


%++++++++++++++++++++++++++++++++++++++++++++++++++++++++++++++++++
%+++ Zeilenabstand ++++++++++++++++++++++++++++++++++++++++++++++++
%++++++++++++++++++++++++++++++++++++++++++++++++++++++++++++++++++
\usepackage{setspace}
% eineinhalbfacher Zeilenabstand. Dies ist nicht gleich wie Zeilenabstand
% 1.5 in üblicher Textverarbeitungssoftware!
\onehalfspacing


%++++++++++++++++++++++++++++++++++++++++++++++++++++++++++++++++++
%+++ Verlinkung +++++++++++++++++++++++++++++++++++++++++++++++++++
%++++++++++++++++++++++++++++++++++++++++++++++++++++++++++++++++++
% Doku: http://www.ctan.org/tex-archive/macros/latex/contrib/hyperref/hyperref.pdf
\usepackage[%
colorlinks, % verwende farbige Links
linkcolor=blue, % Linkfarbe ist blau
bookmarks, % erstelle Bookmarks der Links
bookmarksopen, % Bookmarks werden beim Öffnen des Dokumentes ebenfalls geöffnet
urlcolor=blue, % Hyperlinks sind blau 
bookmarksnumbered, % Bookmarks sind nummeriert
final % Endversion 
]{hyperref}


%++++++++++++++++++++++++++++++++++++++++++++++++++++++++++++++++++
%++++++++++++++++++++++++++++++++++++++++++++++++++++++++++++++++++
%+++ hier beginnt das eigenliche Dokument +++++++++++++++++++++++++
%++++++++++++++++++++++++++++++++++++++++++++++++++++++++++++++++++
%++++++++++++++++++++++++++++++++++++++++++++++++++++++++++++++++++
\begin{document}

%++++++++++++++++++++++++++++++++++++++++++++++++++++++++++++++++++
%+++ Einstellungen für die Codebeispiele ++++++++++++++++++++++++++
%++++++++++++++++++++++++++++++++++++++++++++++++++++++++++++++++++
\definecolor{mygray}{gray}{.82}

\lstset{numbers=none, % keine Zeilennummern
tabsize=3, % Tabulatorgrösse: 3 Zeichen
breaklines=true, % zu lange Zeilen werden umbrochen
aboveskip=1em, % Abstand nach oben
belowskip=0.3em, % Abstand nach unten
basicstyle=\small\ttfamily, % Schriftgrösse small, Typewriter-Font
framerule=0pt, % keinen Rand
backgroundcolor=\color{mygray}, % helles grau als Hintergrund
framexrightmargin=0.7em, % Hintergrund ragt leicht in den Seitenrand
framexleftmargin=0.7em, % Hintergrund ragt leicht in den Seitenrand
columns=fullflexible % damit Quellcode einfach rauskopiert werden kann
}


%++++++++++++++++++++++++++++++++++++++++++++++++++++++++++++++++++
%+++ Strafwerte +++++++++++++++++++++++++++++++++++++++++++++++++++
%++++++++++++++++++++++++++++++++++++++++++++++++++++++++++++++++++
% folgende Werte lockern die Regeln für den Textsatz ein wenig - 
% insbesonders werden breitere Leerräume erlaubt
% Doku: http://www.jr-x.de/publikationen/latex/tipps/zeilenumbruch.html
\clubpenalty=10000
\widowpenalty=10000
\displaywidowpenalty=10000
\emergencystretch=0.5em

%++++++++++++++++++++++++++++++++++++++++++++++++++++++++++++++++++
%+++ Titelseite +++++++++++++++++++++++++++++++++++++++++++++++++++
%++++++++++++++++++++++++++++++++++++++++++++++++++++++++++++++++++
\begin{titlepage}
\thispagestyle{empty}\enlargethispage{4.5em}
\vspace*{-1.5em}
\begin{center}
\begin{minipage}{0.42\linewidth}
	\includegraphics[width=\linewidth]{ctanlion}\\
	\begin{tiny}CTAN lion drawing by Duane Bibby; thanks to \url{www.ctan.org}\end{tiny}
\end{minipage}%
\hfill%
\begin{minipage}{0.5\linewidth}
	\begin{raggedright}
		\textbf{\Huge Bilder einfügen in\\[0.5em] \LaTeX: Ein How-To \normalfont}\\[2.5em]%
		\textit{\version}\vspace*{6.15em}
	\end{raggedright}
\end{minipage}\\[5em]
\newcommand{\fboxseptemp}{\fboxsep}
\setlength{\fboxsep}{1em}
\fbox{\begin{minipage}{13cm}\small
	Copyright (c)  2005 (06-10) by \textbf{Dominik Bischoff}.\\
	Permission is granted to copy, distribute and/or modify this document
	under the terms of the \textbf{GNU Free Documentation License, Version 1.3}
	or any later version published by the Free Software Foundation;
	with no Invariant Sections, no Front-Cover Texts, and no Back-Cover
	Texts.  A copy of the license is included in the section entitled "`GNU
	Free Documentation License"'.\normalfont
\end{minipage}}\\\vspace{2cm}
\setlength{\fboxsep}{\fboxseptemp}

Ohne die folgenden Personen\footnote{Alphabetisch geordnet. Falls jemand zusätzlich zu / anstatt seinem "`Nickname"' hier seinen richtigen Namen sehen möchte, so genügt eine Nachricht an den Autor.} wäre dieses Dokument in der hier sichtbaren Form nie zu Stande gekommen. Ich möchte mich bei allen für Tipps, Tricks und Korrekturen herzlich \textbf{bedanken:}\\[1em]
\textit{Atranis, Axel Sommerfeldt, bobmalaria, cookie170, countbela666, daswaldhorn (Carsten Gerlach), edico, etilli33, Frank Küster, Heiko Bauke, Herbert Voss, iii, Kerstin Schiebel, Markus Kohm, Matthias Pospiech, rais (Rainer Schnaack), Reiner Steib, red.iceman, Roland Geiger, Salnic, Simon Rutishauser, Ulrike Fischer, Uwe Siart}\\[1em]



\clearpage

\end{center}

\end{titlepage}



%++++++++++++++++++++++++++++++++++++++++++++++++++++++++++++++++++
%+++ Inhaltsverzeichnis +++++++++++++++++++++++++++++++++++++++++++
%++++++++++++++++++++++++++++++++++++++++++++++++++++++++++++++++++
\pdfbookmark[1]{Inhaltsverzeichnis}{toc}
\tableofcontents
\newpage



%++++++++++++++++++++++++++++++++++++++++++++++++++++++++++++++++++
%+++ Vorwort+++++++++++++++++++++++++++++++++++++++++++++++++++++++
%++++++++++++++++++++++++++++++++++++++++++++++++++++++++++++++++++
\section{Vorwort: Zu diesem Dokument}
Im \LaTeX-Board des Forums \url{www.mrunix.de} wurden immer wieder die gleichen Fragen zum Themengebiet Bilder gestellt. Es wurde immer wieder von den gleichen vermeintlichen Problemen berichtet. Auf die Idee von Benutzer \texttt{etilli33} hin ist dieses \textit{Bilder How-To} entstanden, welches folgenden Ansprüchen gerecht werden soll:
\begin{enumerate}
	\item Möglichst umfassend, so dass viele Problembereiche abgedeckt werden. Es ist allerdings nicht das Ziel, alle Pakete komplett vorzustellen - hierfür existieren die Paketdokumentationen.
	\item Möglichst kurz, um praxistauglich zu sein. Möglichst viele Codebeispiele.
	\item Für \LaTeX-Anfänger verständlich. 
\end{enumerate}

\subsection{Rückmeldungen}
Da es nicht die "`ideale Lösung"' gibt, bin ich jederzeit für Änderungsvorschläge und Ergänzungen offen.

Mail an: \url{walfisch@herr-der-mails.de}


\subsection{aktuelle Version des Dokumentes}
Aktuelle Versionen des Dokumentes sind zu finden auf:%
\begin{itemize}
\item \href{http://www.ctan.org/tex-archive/info/l2picfaq/german/}{l2picfaq @ CTAN}
%\item \href{http://homepage.sunrise.ch/mysunrise/dominikbischoff/l2picfaq/l2picfaq.pdf}{l2picfaq @ www.walfisch.ch.vu}
\end{itemize}


\clearpage




%++++++++++++++++++++++++++++++++++++++++++++++++++++++++++++++++++
%+++ Dokumentationen und Pakete +++++++++++++++++++++++++++++++++++
%++++++++++++++++++++++++++++++++++++++++++++++++++++++++++++++++++
\section{Dokumentationen und Pakete}
Sollte auf Ihrem Computer eines der hier vorgestellten Pakete nicht installiert sein, so kann dieses meist auf \href{http://www.ctan.org}{ctan.org} gefunden werden. Selbiges gilt für die Dokumentationen zu den entsprechenden Paketen.

\subsection{\texttt{epslatex.pdf}}
Bei \href{http://www.ctan.org/tex-archive/info/epslatex/english/epslatex.pdf}{\texttt{epslatex.pdf}} handelt es sich um eine sehr ausführliche Dokumentation (über 100 Seiten) zu Bildern, die es momentan allerdings nur in Englisch und Französisch gibt.

\subsection{Dokumentation des \texttt{graphics} Paketes}
Diese heisst \href{http://mirror.switch.ch/ftp/mirror/tex/macros/latex/required/graphics/grfguide.pdf}{\texttt{grfguide.pdf}}.




%++++++++++++++++++++++++++++++++++++++++++++++++++++++++++++++++++
%+++ Bilder einfügen in LaTeX +++++++++++++++++++++++++++++++++++++
%++++++++++++++++++++++++++++++++++++++++++++++++++++++++++++++++++
\section{Bilder einfügen in \LaTeX}





\subsection{Bildformate}

\subsubsection{kompilieren mittels \texttt{latex}}
Sollen Dokumente mittels des Kommandos \texttt{latex} kompiliert werden, so müssen die Grafiken im \texttt{*.eps} Format vorliegen. Folgendes Minimalbeispiel zeigt ein Bild an:

\begin{lstlisting}[frame=single]
\documentclass{article}
\usepackage{graphicx}
\begin{document}
\includegraphics{Bild}	
\end{document}
\end{lstlisting}

Hierbei ist folgendes zu beachten: Die Bilddatei muss \texttt{Bild.eps} heissen. Dabei unterscheidet \LaTeX{} je nach verwendetem Betriebssystem zwischen Gross- und Kleinschreibung im Dateinamen. Weiter sollte die Bilddatei im selben Ordner liegen, wie die zu kompilierende \LaTeX{}-Datei oder in einem Unterordner (siehe Abschnitt \ref{subsubsec:unterordner}). Schliesslich kann sie noch in einem beliebigen Ordner der Variable \texttt{\$TEXINPUTS} liegen, auf welche in diesem Dokument allerdings nicht näher eingegangen wird.

\textbf{Achtung:} Sonderzeichen (insbesonders Leerzeichen) im Dateinamen führen zu Fehlermeldungen!



\subsubsection{kompilieren mittels \texttt{pdflatex}}
Sollen die \LaTeX{}-Files mittels \texttt{pdflatex} kompiliert werden, so müssen die Bilddateien entweder als \texttt{*.pdf}, \texttt{*.png} oder als \texttt{*.jpg} vorliegen.

Zur Frage der Formatwahl:%
\begin{enumerate}
	\item Als Faustregel gilt: Falls die Grafik in einem der erwähnten Formate vorliegt, sollte sie so belassen werden.
	\item Für Zeichnungen oder Grafiken bietet sich \texttt{pdf} an, da dieses ein Vektorgrafikformat ist.
	\item Für Fotos bietet sich \texttt{jpg} aufgrund der Dateigrösse an.
	\item Für Bilder allgemeiner Art sollte \texttt{png} verwendet werden, da dieses verlustlos komprimiert. \texttt{jpg} erzeugt hier oftmals unschöne Kompressionsartefakte.
\end{enumerate}

Es kann wiederum dasselbe Minimalbeispiel verwendet werden:
\begin{lstlisting}[frame=single]
\documentclass{article}
\usepackage{graphicx}
\begin{document}
\includegraphics{Bild}	
\end{document}
\end{lstlisting}

In diesem Fall muss die Bilddatei \texttt{Bild.jpg}, \texttt{Bild.png} oder eben \texttt{Bild.pdf} heissen und sollte wiederum im selben Ordner wie das \LaTeX-File liegen.

\textbf{Achtung:} Sonderzeichen (insbesonders Leerzeichen) im Dateinamen führen zu Fehlermeldungen!


\subsubsection{sowohl mit \texttt{latex} als auch mit \texttt{pdflatex} kompilieren}
Manchmal ist es wünschenswert von einem Dokument sowohl eine \texttt{pdf}-Version als auch eine Version im  \texttt{ps}-Format zu erzeugen. Hierzu müssen folgende drei Bedingungen erfüllt sein:

\begin{enumerate}
	\item Sämtliche verwendeten Pakete müssen beide Varianten unterstützen.
	\item Sämtliche Bilder müssen doppelt vorhanden sein: Einmal als \texttt{eps} und einmal als \texttt{jpg} / \texttt{png} / \texttt{pdf}. Beide Varianten müssen (mit Ausnahme der Dateiendung) gleich heissen.
	\item Sämtliche Bilddateien müssen in \LaTeX{} ohne Dateiendung eingebunden werden.
\end{enumerate}



\subsubsection{Treiberangaben}
Treiber sollten im Normalfall nicht explizit angegeben werden, da diese beim Kompilieren automatisch richtig gewählt werden.

\begin{lstlisting}[frame=single]
\usepackage{graphicx} % -> richtig!
\usepackage[dvips]{graphicx} % -> falsch!
\end{lstlisting}



\subsubsection{Bilder in Unterordnern}
\label{subsubsec:unterordner}
Oftmals ist es sinnvoll, Bilder nicht im selben Ordner wie die \LaTeX-Datei zu speichern. Will man beispielsweise ein Bild einfügen, welches im Unterordner \texttt{Kapitel1} liegt, so verändert sich der Aufruf folgendermassen:

\begin{lstlisting}[frame=single]
\includegraphics{Kapitel1/Bild}	
\end{lstlisting}

\textbf{Achtung:} Sonderzeichen (insbesonders Leerzeichen) im Ordnernamen oder im Dateinamen des Bildes führen zu Fehlern!


\begin{samepage}
\hypertarget{lnk:BBox}{ }
\subsubsection{Probleme mit der Bounding-Box}
\label{subsubsec:bbox}\end{samepage}
In Verbindung mit \texttt{eps}-Grafikdateien erscheint oftmals folgende Fehlermeldung:

\begin{lstlisting}[frame=single]
!LaTeX Error: Cannot determine size of graphic in Bild.eps (no BoundingBox)
\end{lstlisting}

Die Ausgabe des Bildes im Dokument entspricht meist nicht dem erwarteten Verhalten. Es gibt folgende Lösungen für dieses Problem: 

\begin{enumerate}
\item Die \texttt{eps}-Grafik kann mittels "`Options"' -> "`EPS-CLIP"' im Programm \href{http://www.cs.wisc.edu/~ghost/gsview/}{\texttt{GhostView}} richtig zugeschnitten und neu abgespeichert werden.
\item Die \texttt{eps}-Grafik mittels des Kommandozeilentools \texttt{eps2eps} richtig umwandeln.
\item Das Bild mittels \verb|\includegraphics[bb=0 0 100 100]{Bild}| ins Dokument einbinden. Die richtigen Koordinaten (ersten zwei Zahlen entsprechen der linken unteren Ecke, die zweiten zwei der rechten oberen Ecke des anzuzeigenden Bildausschnittes) können in nahezu jedem \texttt{eps}-Viewer erhalten werden.
\end{enumerate}

\textbf{Anmerkung:} Diese Meldung kann auch auftauchen, wenn mittels \texttt{latex} kompiliert wird, allerdings eine Bilddatei mit falschem Format eingebunden wird.


\subsubsection{\texttt{eps}-Bilder werden falsch skaliert}
Wird ein Bild zwar angezeigt, jedoch viel zu klein oder zu weit nach unten gerutscht, so besteht meist ein Problem mit der \hyperlink{lnk:BBox}{Boundig-Box}. 


\subsubsection{\texttt{dvi}-Viewer zeigt Bilder nicht korrekt an!}
In \texttt{dvi}-Dateien werden die Bilder nicht eingebunden, sondern lediglich verlinkt. Dies hat mehrere Konsequenzen:
\begin{itemize}
  \item Wird die \texttt{dvi}-Datei an eine andere Stelle kopiert, so kann die verlinkte Grafik unter Umständen nicht mehr gefunden werden.
  \item Obwohl grosse Anstrengungen unternommen wurden, zeigen auch heute noch viele \texttt{dvi}-Viewer Grafiken fehlerhaft an. Wird die Grafik also im \texttt{dvi}-Viewer falsch angezeigt, so heisst dies noch lange nicht, dass der verwendete Code falsch ist. Zur Kontrolle kann eine \texttt{pdf}- oder \texttt{ps}-Datei erstellt werden. Um alle Zweifel aus der Welt zu räumen, kann diese danach auch noch ausgedruckt werden.
  \item \texttt{dvi} ist folglich ein Arbeitsformat (schnell, reverse search, \ldots), welches allerdings nicht für die Endausgabe verwendet werden sollte!
\end{itemize}




\subsubsection{\texttt{pdf} erstellen mit enthaltenem \texttt{ps}-Code}
Eines vorweg: Diese Methode kann zwar auch verwendet werden, um \texttt{eps}-Grafiken in ein \texttt{pdf}-Dokument einzufügen. In diesem Fall ist es allerdings schlauer, wenn die Grafik einmal ins \texttt{pdf}-Format konvertiert wird und nachher direkt verwendet wird!

Falls aber Post-Script-Pakete wie \hyperlink{lnk:pstricks}{\texttt{pstricks}} in einem \texttt{pdf}-Dokument verwendet werden sollen, kann diese Lösung verwendet werden:

\begin{lstlisting}[frame=single]
\usepackage{graphicx}
\usepackage{pst-pdf}
...
\begin{document}
...
\begin{postscript} pstricks-code \end{postscript}
...
\end{lstlisting}

Dabei ist zu beachten, dass bei dieser Paketkonfiguration speziell kompiliert werden muss:

\begin{lstlisting}[frame=single]
pdflatex Dokument.tex
  latex Dokument.tex
  dvips -o Dokument-pics.ps Dokument.dvi
  ps2pdf Dokument-pics.ps
pdflatex Dokument.tex
\end{lstlisting}

\textbf{Erklärung:} Die eingerückten Zeilen sorgen dafür, dass der Post-Script-Code in einer extra Datei gespeichert wird, aus welcher er anschliessend von \texttt{pdflatex} verwendet wird.

Es existieren auch Scripte, welche diese Schritte automatisieren. Für diverse Betriebssysteme findet man diese \href{http://www.ctan.org/tex-archive/macros/latex/contrib/pst-pdf/scripts/}{hier}.



\subsubsection{Sonderzeichen in Datei- und Pfadnamen}
Es wurde in diesem Dokument bereits mehrmals davor gewarnt, in Pfad- oder Datei\-namen Sonder- und Leerzeichen zu verwenden. Dies stellt eine gewisse Kompatibilität zwischen verschiedenen Programmen und Systemen sicher und verhindert somit allfällige Probleme. In Ausnahmefällen kann allerdings das Paket  \href{http://www.ctan.org/get/macros/latex/contrib/oberdiek/grffile.pdf}{\texttt{grffile}} verwendet werden, welches gewisse Sonderzeichen in Dateinamen erlaubt.




%---------------------------------------------------------------------------
%---------------------------------------------------------------------------
%---------------------------------------------------------------------------
%---------------------------------------------------------------------------




\subsection{Konvertierungstools}
Es folgt eine Übersicht verschiedener Tools, welche Bilddateien in ein anderes Format konvertieren können. Diese sind für viele Betriebssysteme im Internet zu finden.

\subsubsection{\texttt{pdf} -> \texttt{ps} / \texttt{eps}}
Für diesen Fall bieten sich die beiden Kommandozeilentools \texttt{pdftops} und \texttt{pdf2ps} an. Übergibt man an \texttt{pdftops} zusätzlich die Option \verb|-eps|, so lassen sich damit auch \texttt{eps}-Dateien erstellen.

\subsubsection{\texttt{ps} -> \texttt{pdf}}
Dafür ist das Kommandozeilentool \texttt{ps2pdf} zu gebrauchen.

\subsubsection{\texttt{eps} -> \texttt{pdf}}
Auch dafür gibt es zwei Kommandozeilentools: \texttt{eps2pdf} und \texttt{epstopdf}.

\subsubsection{\texttt{jpg} -> \texttt{ps} / \texttt{eps}}
Hier existiert das Kommandozeilentool \texttt{jpeg2ps}.

\subsubsection{\texttt{ps} -> \texttt{eps}}
Hierzu kann das Kommandozeilentool \texttt{ps2eps} verwendet werden.

\subsubsection{beliebig -> \texttt{ps}}
Möchte man aus einem beliebigen Programm heraus etwas identisch in sein \LaTeX-Dokument übernehmen, so führt der einfachste Weg über die Installation eines Treibers von einem postscriptfähigen Drucker. Geeignete Geräte sind alle etwas teureren Laserdrucker. Die Treiber können entweder übers Internet heruntergeladen oder oftmals sogar bereits auf dem Computer gefunden werden.

Danach kann im Normalfall über das Drucken-Menü in jedem Programm mit diesem neu installieren ("`virtuellen"') Drucker in eine \texttt{ps}-Datei gedruckt werden. Anschliessend muss noch die \hyperlink{lnk:BBox}{Bounding-Box} anpasst werden oder die Grafik direkt nach \texttt{pdf} konvertiert werden. Schon besitzt man ein qualitativ hochstehendes Bild!

\subsubsection{Multitalente}
Neben diversen Grafikprogrammen sind \href{http://www.gimp.org/}{\texttt{The Gimp}} mit grafischer Benutzer\-oberfläche und das Kommandozeilentool \href{http://www.imagemagick.org/}{\texttt{imagemagick}} sehr zu empfehlen. Beide sind opensource und beherrschen eine Vielzahl von Formaten.


\subsubsection{Probleme mit \texttt{eps}-Dateien}
Die Hauptprobleme äussern sich meist in riesigen Dateien oder Kompilierungsfehlern. Es gibt verschiedene Lösungsansätze, welche allerdings nicht immer zum Erfolg führen:
\begin{itemize}
  \item Kommandozeilentool \texttt{eps2eps}
  \item Grafik zuerst mit \texttt{epstopdf} gefolgt von \texttt{pdftoeps} umwandeln
\end{itemize}

Will man Speicherplatz sparen, so bietet es sich an, die \texttt{eps}-Dateien zusätzlich noch zu komprimieren: Hierzu kann ein Programm nach Wahl verwendet werden, einzige Bedingung ist, dass es das Format \texttt{.gz} (sprich: "`GeZip"') versteht. Da aus dieser komprimierten Datei die Bounding-Box nicht mehr korrekt ausgelesen werden kann, muss diese explizit angegeben werden.

\begin{lstlisting}[frame=single]
% ursprüngliche Datei: bild.eps
% komprimierte Datei: bild.eps.gz
\usepackage{graphicx}
...
\begin{document}
...
\begin{figure}[htb]
  \centering
    \includegraphics[bb=0 0 113 113]{Bild}
  \caption{Bildunterschrift}
\end{figure}
...
\end{lstlisting}

Für die korrekte Wahl der Bounding-Box siehe Abschnitt \ref{subsubsec:bbox}.



%---------------------------------------------------------------------------
%---------------------------------------------------------------------------
%---------------------------------------------------------------------------
%---------------------------------------------------------------------------


\subsection{Bilder manipulieren}

\subsubsection{Bilder skalieren}
Um die Grösse eines Bildes anzupassen, bieten sich folgende Kommandos an:

\begin{lstlisting}[frame=single]
\includegraphics[width=4cm]{Bild}
\includegraphics[height=4cm]{Bild}
\includegraphics[width=0.8\linewidth]{Bild}
\includegraphics[scale=0.5]{Bild}
\end{lstlisting}

Die ersten beiden Kommandos skalieren das Bild proportional auf  eine feste Breite beziehungsweise eine feste Höhe. Das dritte Kommando skaliert die Grafik abhängig von der Zeilenlänge; In diesem Fall auf 80\% einer Textzeile. Viertes Kommando skaliert das Bild auf die Hälfte der ursprünglichen Grösse.


\subsubsection{Bilder auf Maximalgrösse skalieren}
Falls ein Bild auf die maximale Grösse skaliert und dabei das Seitenverhältnis beibehalten werden soll:
\begin{lstlisting}[frame=single]
\noindent\includegraphics[width=\linewidth,height=\textheight,keepaspectratio]{Bild}
\end{lstlisting}




\subsubsection{Bilder nur dann skalieren, wenn sie breiter als die Seite sind}
Der folgende neue Befehl bindet Bilder in der Originalgrösse ein, falls sie weniger breit als die Seite sind. Sonst wird das Bild auf Seitenbreite skaliert.

\begin{lstlisting}[frame=single]
...
\usepackage{graphicx}
\makeatletter
\def\ScaleIfNeeded{%
  \ifdim\Gin@nat@width>\linewidth
    \linewidth
  \else
    \Gin@nat@width
  \fi
}
\makeatother
...
\begin{document}
...
\includegraphics[width=\ScaleIfNeeded]{Bild}
\end{lstlisting}

\textbf{Anmerkung:} Der Befehl funktioniert auch in mehrspaltigem Text. Das Bild wird dann auf die Spaltenbreite skaliert.



\subsubsection{Bilder drehen}
Bilder können mittels \LaTeX{} auch gedreht werden. Dies geschieht mit dem Kommando:

\begin{lstlisting}[frame=single]	
\includegraphics[angle={90}]{Bild}
\end{lstlisting}

Hierbei wird das Bild um 90 Grad im Gegenuhrzeigersinn gedreht.

\subsubsection{Bildausschnitte}
Um von einem bestehenden Bild nur einen Ausschnitt einzubinden, bietet sich die Option \texttt{trim} an:

\begin{lstlisting}[frame=single]	
\includegraphics[trim = 10mm 80mm 20mm 5mm, clip, width=3cm]{Bild}
\end{lstlisting}

Dieser Code schneidet links \texttt{10mm}, unten \texttt{80mm}, rechts \texttt{20mm} und oben \texttt{5mm} vom ursprünglichen Bild ab und skaliert anschliessend den sichtbaren Ausschnitt auf \texttt{3cm} Breite.





\subsubsection{gemischte Kommandos}

Selbstverständlich können eben genannte Kommandos auch gemischt werden. Dabei ist zu beachten, dass die an \texttt{\textbackslash includegraphics} übergebenen Optionen in der Reihenfolge ausgeführt werden, in der sie im Quellcode stehen! Es spielt also eine Rolle, ob ein Bild zuerst gedreht wird und danach auf eine Gesamtbreite skaliert, oder ob zuerst die Breite geändert und danach das Bild gedreht wird.



%---------------------------------------------------------------------------
%---------------------------------------------------------------------------
%---------------------------------------------------------------------------
%---------------------------------------------------------------------------



\subsection{Bilderumgebungen}

\subsubsection{Bilder gleiten lassen: \texttt{figure}-Umgebung}
Idealerweise werden Grafiken in \LaTeX{} mittels der \texttt{figure}-Umgebung eingefügt. Dies hat den grossen Vorteil, dass \LaTeX{} die Grafiken möglichst so platziert, dass grosse Lücken und ähnliche Unschönheiten verhindert werden können. Falls möglich werden zudem die Präferenzen des Autors berücksichtigt. 

Folgender Code fügt ein Bild ein und platziert es in folgender Reihenfolge: \textbf{h}ere - \textbf{t}op - \textbf{b}ottom - \textbf{p}age. Falls möglich wird das Bild an aktueller Stelle eingefügt. Als nächstes wird versucht, das Bild oben auf der Seite zu platzieren. Sollte dies immer noch nicht gelingen, so bleiben noch die Platzierung unten auf der Seite und das Platzieren auf einer eigenen Seite für die Abbildung übrig.

\begin{lstlisting}[frame=single]
\begin{figure}[htbp]
	\centering
	\includegraphics{Bild}%
	\caption{Hier steht die Beschreibung des Bildes}%
\end{figure}
\end{lstlisting}

\textbf{Anmerkung:} Wird zusätzlich noch ein Ausrufezeichen vorangestellt, so lockert \LaTeX{} seine Regeln und versucht unter allen Umständen das Bild wie gewünscht zu platzieren. Dies ist jedoch manchmal nicht möglich.

\begin{lstlisting}[frame=single]
\begin{figure}[!htbp]
\end{lstlisting}

\textbf{Anmerkung:} Das Paket \texttt{float} stellt zusätzlich die Option \texttt{H} zur Verfügung, welche das Einfügen der Grafik an aktueller Stelle erzwingt. Dieses Vorgehen wird allerdings im Normalfall nicht empfohlen, da es zu unschönen Lücken im Text kommen kann.

\begin{lstlisting}[frame=single]
\usepackage{float}
...
\begin{figure}[H]
\end{lstlisting}

\textbf{Achtung:} \verb|\restylefloat{figure}| und \verb|\restylefloat{table}| werden für das Funktionieren der Option \texttt{H} nicht benötigt und sollten im Allgemeinen nicht gebraucht werden (Kompatibilitätsprobleme mit anderen Paketen)!



\subsubsection{begrenztes Gleiten}
Sollen unbedingt alle bereits eingefügten Gleitumgebungen (\texttt{figure}, \texttt{table}, ...) vor einem bestimmten Punkt im Dokument erscheinen, so existieren zwei mögliche Ansätze:

Seitenumbruch erwünscht: Kommandos \texttt{\textbackslash clearpage} , \texttt{\textbackslash cleardoublepage}.

Seitenumbruch nicht erwünscht: Das Paket \texttt{placeins} bietet hier die Lösung.


\begin{lstlisting}[frame=single]
\usepackage{placeins}
...
%hier diverse figures
\FloatBarrier
...
\end{lstlisting}

\textbf{Anmerkung:} \verb|\usepackage[section]{placeins}| verhindert ein Gleiten der Grafiken ausserhalb der aktuellen "`section"'.

\begin{samepage}
\hypertarget{lnk:Captionof}{ }\end{samepage}
\subsubsection{Bild an aktueller Stelle einfügen}
Eines der häufigsten Problem betrifft Bilder, welche genau an der aktuellen Stelle einfügt werden sollen. Falls \texttt{figure} verwendet wird, funktioniert folgender Code meistens:

\begin{lstlisting}[frame=single]
\begin{figure}[!htb]
	\includegraphics{Bild}%	
	\caption{Bildunterschrift}%
\end{figure}
\end{lstlisting}

\LaTeX{} versucht dabei das Bild unter allen Umständen an der aktuellen Stelle zu platzieren. Dies kann jedoch manchmal unmöglich sein.

Um ein Bild ohne Bildunterschrift und somit auch ohne Eintrag ins Abbildungsverzeichnis einzubinden, kann das bereits mehrmals genannte Kommando verwendet werden:

\begin{lstlisting}[frame=single]
\includegraphics{Bild}
\end{lstlisting}

Soll das Bild allerdings eine Bildunterschrift erhalten, jedoch trotzdem immer an aktueller Stelle erscheinen, bietet sich das Paket \texttt{capt-of} an:

\begin{lstlisting}[frame=single]
\usepackage{capt-of}
% \usepackage{caption}
...
\begin{center}
	\begin{minipage}{\linewidth}
		\centering
		\includegraphics{Bild}%
		\captionof{figure}[kurze Bildunterschrift]{Bildunterschrift}%
	\end{minipage}	
\end{center}
...
\end{lstlisting}

\textbf{Anmerkung:} Neben dem \texttt{capt-of} Paket definiert auch das Paket \texttt{caption} den Befehl \texttt{captionof}. Grundsätzlich ist es sinnvoller, das \texttt{capt-of} Paket zu laden, da dieses kleiner ist. Wird allerdings das Paket \texttt{caption} sowieso geladen, so braucht \texttt{capt-of} nicht zusätzlich auch noch geladen zu werden. Ebenfalls gibt es einige wenige Ausnahmen, in welchen das \texttt{capt-of} Paket nicht funktioniert und stattdessen \texttt{caption} geladen werden muss. 

\textbf{Anmerkung:} Diese Methode kann auch verwendet werden, um Bilder in einer Tabelle mit einer Bildunterschrift zu versehen.

\textbf{Anmerkung:} Das \texttt{capt-of} Paket kann auch für Tabellen verwendet werden, um die \texttt{table}-Umgebung zu vermeiden und Tabellen an aktueller Stelle ins Dokument einzufügen: Der Befehl lautet dann: 

\begin{lstlisting}[frame=single]
\captionof{table}{Beschreibung}
\end{lstlisting}

\textbf{Anmerkung:} Grundsätzlich braucht es keine Bildunterschrift, wenn das Bild an aktueller Stelle eingefügt wird, da der Zusammenhang aus dem Text heraus klar werden sollte. Manchmal ist es dennoch wünschenswert, dem Bild eine Bildunterschrift zu geben -- sei es, um den Zusammenhang zu verdeutlichen oder um ein Abbildungsverzeichnis erstellen zu können. In diesem Fall sollte man sich allerdings überlegen, ob man das Bild nicht doch ein wenig gleiten lassen möchte, um einen schöneren Textsatz zu ermöglichen.




\subsubsection{textumflossene Bilder}
Soll Text um Bilder herumfliessen, so bietet sich das \href{http://tug.ctan.org/usergrps/uktug/baskervi/4_3/wrapfig.sty}{\texttt{wrapfig}}-Paket an. 

\textbf{Achtung:} \texttt{wrapfig} verursacht oftmals Probleme, wenn viele Bilder auf einer Seite platziert werden oder wenn auf der selben Seite zusätzlich Gleitumgebungen (\texttt{float}, \texttt{table}) vorhanden sind.

\begin{lstlisting}[frame=single]
\usepackage{wrapfig}
...
\begin{wrapfigure}{r}{5cm}
	\centering
	\includegraphics{Bild}%
	\caption{Hier steht die Beschreibung des Bildes}%
\end{wrapfigure}
...
\end{lstlisting}

Anstatt eines \texttt{r} (für rechts) kann man auch ein \texttt{l} nehmen, um Bilder am linken Seitenrand zu auszurichten. Die \texttt{5cm} sind die gewünschte Breite der \texttt{wrapfigure}.

Verwendet man anstatt des kleinen \texttt{r} ein grosses \texttt{R}, so erlaubt man der Abbildung zusätzlich noch zu gleiten. Analog verhält es sich mit \texttt{l} und \texttt{L}.

Sollte das Paket \texttt{wrapfig} nicht die erwünschten Erfolge bringen, so kann das Paket \href{http://www.ctan.org/tex-archive/macros/latex209/contrib/picins/}{\texttt{picins}} ausprobiert werden, welches ähnliche Funktionen bereitstellt.

Möchte man mit dem Text beispielsweise den Konturen eines runden Bildes folgen, so bietet sich das Paket  \href{http://www.ctan.org/tex-archive/macros/latex/contrib/shapepar/shapepar.pdf}{\texttt{shapepar}} an.


%---------------------------------------------------------------------------
%---------------------------------------------------------------------------
%---------------------------------------------------------------------------
%---------------------------------------------------------------------------
\subsection{Bilder in mehrspaltigem Text}

\subsubsection{Bild in die Spalte einfügen}

Wird das ganze Dokument in zweispaltigem Text gesetzt (Option \texttt{twocolumn}), so gibt es einzig zu beachten, dass die Bilder in die Spalte passen. Auch Gleitumgebungen sind erlaubt:

\begin{lstlisting}[frame=single]
\documentclass[a4paper,12pt,twocolumn]{scrartcl}
...
% etwas Text
\begin{figure}[htb]
\centering
\includegraphics[width=\linewidth]{Bild}%
\caption{Hier steht die Beschreibung des Bildes}%
\end{figure}
% etwas Text
\end{lstlisting}

Wird das \href{http://www.ctan.org/tex-archive/macros/latex/required/tools/multicol.pdf}{\texttt{multicol}} Paket verwendet (\texttt{float} funktioniert nicht!) oder möchte man ein Bild an der aktuellen Stelle einfügen, so geschieht dies über den \hyperlink{lnk:Captionof}{\texttt{\textbackslash captionof}} Befehl.



\subsubsection{Bild über ganze Seitenbreite}

Um in einem in Spalten gesetzten Text ein Bild über die ganze Seitenbreite einzufügen, kann folgender Code verwendet werden:

\begin{lstlisting}[frame=single]
% ... mehrspaltiger Text ...
\begin{figure*}[htb]
	\centering
	\includegraphics[width=\textwidth]{Bild}%
	\caption{Hier steht die Beschreibung des Bildes}%
\end{figure*}
% ... mehrspaltiger Text ...
\end{lstlisting}

\textbf{Anmerkung:} Wird das \href{http://www.ctan.org/tex-archive/macros/latex/required/tools/multicol.pdf}{\texttt{multicol}} Paket verwendet um mehrspaltigen Text zu erstellen, so kann alternativ die \texttt{multicols}-Umgebung beendet, das Bild eingefügt und mittels \hyperlink{lnk:Captionof}{\texttt{\textbackslash{}captionof}} beschriftet werden. Anschliessend muss die \texttt{multicols}-Umgebung erneut begonnen werden. 



\subsubsection{umflossene Bilder in mehrspaltigem Text}
Dies ist möglich, jedoch nicht gerade einfach und benötigt im Normalfall einiges and Handarbeit. Die Anleitung dazu findet man \href{http://tug.ctan.org/tex-archive/macros/latex/contrib/wrapfig/multiple-span.txt}{hier}.


%---------------------------------------------------------------------------
%---------------------------------------------------------------------------
%---------------------------------------------------------------------------
%---------------------------------------------------------------------------
\subsection{Abweichungen vom Standardlayout}

\subsubsection{Rahmen}
Um ein Bild einzurahmen, kann das Kommando \texttt{\textbackslash fbox} verwendet werden. Damit nicht unnötige Leerzeilen eingefügt werden, sollte nach jeder Zeile ein Prozentzeichen stehen:

\begin{lstlisting}[frame=single]
\begin{figure}[htb]
  \fbox{\begin{minipage}{6cm}%
  	\begin{center}%
    	\includegraphics[width=5cm]{Bild}%
    	\caption{Hier steht die Beschreibung des Bildes}%
  	\end{center}%
  \end{minipage}}%  
\end{figure}
\end{lstlisting}

Dabei ist zu beachten, dass die Breite der \texttt{minipage} grösser gewählt werden muss, als die Breite des Bildes.

Alternativ kann für farbige Ränder das Kommando \texttt{\textbackslash fcolorbox} verwendet werden.

\textbf{Achtung:} Ränder um Abbildungen wirken oftmals \textbf{unprofessionell}. Daher sollte man zuerst gut überlegen, ob man tatsächlich Ränder haben möchte.

Möchte man einen Rahmen ohne Abstand direkt um eine Grafik haben, so kann folgender Code verwendet werden:

\begin{lstlisting}[frame=single]
\frame{\includegraphics[width=5cm]{Bild}}
\end{lstlisting}



\subsubsection{Bilder neben Text / Tabellen: Minipages}
Um einen Text neben einem Bild zu platzieren, bietet sich folgender Trick an: 

\begin{lstlisting}[frame=single]
\begin{figure}
	\begin{minipage}{0.6\linewidth}
		\includegraphics[width=1.0\linewidth]{Bild}%
	\end{minipage}
	\begin{minipage}{0.3\linewidth}
		Hier folgt der Text...
	\end{minipage}
\end{figure}
\end{lstlisting}

Selbstverständlich kann das Bild auch mit einer Bildunterschrift versehen werden. Ebenfalls möglich ist es, die Bildunterschrift mit dieser Methode neben dem Bild zu platzieren. Dies geschieht mittels des bereits erwähnten \hyperlink{lnk:Captionof}{\texttt{\textbackslash captionof}}-Befehls.


\subsubsection{zwei Bilder nebeneinander}
Hierzu bietet sich das \href{http://www.ctan.org/tex-archive/macros/latex/contrib/subfig/subfig.pdf}{\texttt{subfig}}-Paket an:

\begin{lstlisting}[frame=single]
\usepackage{subfig}
...
\begin{figure}
	\centering
		\hfill %
		\subfloat[Titel 1 \label{pic:Bild1}]{\includegraphics{Bild1}}
		\hfill % alternativ auch \hspace{1cm} für genaue Angaben
		\subfloat[Titel 2 \label{pic:Bild2}]{\includegraphics{Bild2}}
		\hfill %
	\caption{Zwei Bilder: a) Bild1, b) Bild2}
	\label{Gesamtbild}	
\end{figure}
...
\end{lstlisting}

Dieser Code erzeugt eine Gleitumgebung, welche zwei Bilder mit eigenen Titeln enthält. Zusätzlich werden drei gleich grosse Abstände eingefügt (links vom ersten Bild, zwischen den Bildern, rechts vom zweiten Bild).

\textbf{Achtung:} Das Paket \texttt{subfigure} ist veraltet!

\textbf{Achtung:} Das \texttt{subfig}-Paket arbeitet nicht ohne weiteres mit dem \href{http://www.ctan.org/tex-archive/macros/latex/contrib/tocloft/tocloft.pdf}{\texttt{tocloft}}-Paket zusammen. Folgende Fehlermeldung erscheint:

\begin{lstlisting}[frame=single]
Command \c@lofdepth already defined. \newcounter{lofdepth}
Command \c@lotdepth already defined. \newcounter{lotdepth}
\end{lstlisting}

Durch Angabe einer zusätzlichen Option kann dieses Problem gelöst werden:

\begin{lstlisting}[frame=single]
...
\usepackage{subfig}
\usepackage[subfigure]{tocloft}
...
\end{lstlisting}

\textbf{Anmerkung:} Das \texttt{subfig} Paket bietet den Befehl \verb|\ContinuedFloat| an um eine \texttt{figure} mit mehreren \texttt{subfloats} auf zwei (oder mehr) Seiten zu verteilen.


\subsubsection{Bilder in Tabellen}
Benutzt man das normale \texttt{\textbackslash{}includegraphics} um ein Bild in einer Tabelle einzubinden, so stellt man fest, dass dieses nicht zentriert wird. Folgender Code schafft Abhilfe:

\begin{lstlisting}[frame=single]
...
% neuer Befehl: \includegraphicstotab[..]{..}
% Verwendung analog wie \includegraphics
\newlength{\myx} % Variable zum Speichern der Bildbreite
\newlength{\myy} % Variable zum Speichern der Bildhöhe
\newcommand\includegraphicstotab[2][\relax]{%
  % Abspeichern der Bildabmessungen
  \settowidth{\myx}{\includegraphics[{#1}]{#2}}%
  \settoheight{\myy}{\includegraphics[{#1}]{#2}}%
  % das eigentliche Einfügen
  \parbox[c][1.1\myy][c]{\myx}{%
    \includegraphics[{#1}]{#2}}%
}% Ende neuer Befehl
...
\begin{document}
...
\begin{tabular}{|c|c|c|}\hline
  text & text & text\\\hline
  text &  \includegraphicstotab[width=4cm]{Bild} & text\\\hline
  text & text & text\\\hline
\end{tabular}
...
\end{lstlisting}

\textbf{Anmerkung:} In vielen Fällen gelten vertikale Linien in Tabellen als typographisch falsch. Ob vertikale Linien benötigt werden oder nicht, sollte daher Fall zu Fall neu abgeklärt werden.

Ein weiteres Problem taucht auf, wenn man Bildunterschriften in einer \texttt{longtable} Umgebung benutzen möchte. Folgender Coder löst das Problem:
\begin{lstlisting}[frame=single]
...
\usepackage{caption}
\AtBeginDocument{\let\OrigCaption\caption}
\newcommand{\UseCaptionofInLongtable}{\let\caption\OrigCaption}
...
\begin{longtable}{...}
...
  \includegraphics{Bild}
  \UseCaptionofInLongtable
  \captionof{figure}{Bildunterschrift}%
...
\end{longtable}
...
\end{lstlisting}




\subsubsection{Bilder punktgenau platzieren}
Wer den Aufwand nicht scheut und ein spezielles Layout mit punktgenauer Platzierung der Elemente auf der Seite benötigt, der kann dies mit dem \href{http://www.ctan.org/tex-archive/macros/latex/contrib/textpos/textpos.pdf}{\texttt{textpos}}-Paket verwirklichen.



\subsubsection{Parameter für Gleitumgebungen: Seitenbelegung}
Die Grundeinstellungen von \LaTeX{} zum Platzieren von Grafiken sind eher konservativ. Folgende Werte können angepasst werden:

\begin{lstlisting}[frame=single]
\renewcommand{\topfraction}{.85} % maximaler Abstand von Seitenoberseite, bis zu welchem Gleitumgebungen noch plaziert werden dürfen
\end{lstlisting}
\begin{lstlisting}[frame=single]
\renewcommand{\bottomfraction}{.7} % maximaler Anteil welchen Gleitumgebungen am unteren Seitenrand einnehmen dürfen
\end{lstlisting}
\begin{lstlisting}[frame=single]
\renewcommand{\textfraction}{.15} % Anteil einer Seite mit Gleitumgebungen, welcher mindestens von Text belegt sein muss -> ansonsten kein Text auf der Seite
\end{lstlisting}
\begin{lstlisting}[frame=single]
\renewcommand{\floatpagefraction}{.66} % minimaler Seitenanteil welcher besetzt sein muss, bevor eine neue Seite für Gleitumgebungen angelegt wird
\end{lstlisting}
\begin{lstlisting}[frame=single]
\setcounter{topnumber}{3} % maximale Anzahl Gleitobjekte am oberen Seitenrand
\end{lstlisting}
\begin{lstlisting}[frame=single]
\setcounter{bottomnumber}{3} % maximale Anzahl Gleitobjekte am unteren Seitenrand
\end{lstlisting}
\begin{lstlisting}[frame=single]
\setcounter{totalnumber}{6} % maximale Anzahl Gleitobjekte pro Seite
\end{lstlisting}

\textbf{Anmerkung:} Zweispaltigen Textsatz: \verb|dbltopnumber|, \verb|dblfloatpagefraction| und \verb|dbltopfraction|.


\subsubsection{Parameter für Gleitumgebungen: vertikale Abstände}
Abstände zwischen Text und Gleitumgebung (am Kopf oder Fuss einer Seite):

\begin{lstlisting}[frame=single]
\setlength{\textfloatsep}{10pt plus5pt minus3pt}
\end{lstlisting}

Abstände zwischen Text und Gleitumgebung (im Text -- Option "`\texttt{h}"'):

\begin{lstlisting}[frame=single]
\setlength{\intextsep}{10pt plus5pt minus3pt}
\end{lstlisting}

\textbf{Anmerkung:} Zweispaltiger Text: \verb|dbltextfloatsep| (Abstand Text zu Grafik über Seitenbreite)

Analog gibt es noch die Abstände zwischen zwei Gleitobjekten: \verb|floatsep| oder für zweispaltigen Text \verb|dblfloatsep|.



%---------------------------------------------------------------------------
%---------------------------------------------------------------------------
%---------------------------------------------------------------------------
%---------------------------------------------------------------------------
\subsection{Spezialpakete}


\subsubsection{Paket \texttt{floatrow}}
Das Paket \href{http://www.ctan.org/tex-archive/macros/latex/contrib/floatrow/floatrow.pdf}{\texttt{floatrow}} bietet eine Menge interessanter Möglichkeiten: So lassen sich mit \texttt{floatrow} eigene Gleitumgebungen erstellen (um beispielsweise neben Abbildungen noch Diagramme getrennt einzufügen) und es lässt sich das Aussehen von Gleitungebungen global ändern (Rahmen, Abstände, ...). Zudem kann auch das Aussehen von einzelnen Abbildungen verändert werden (um beispielsweise die Bildunterschrift neben das Bild zu stellen). 

\textbf{Anmerkung:} Um möglichst einfach neben den Standardgleitumgebungen \texttt{float} und \texttt{tabular} noch weitere Gleitumgebungstypen (bspw: \texttt{diagram}) zu erstellen, kann auch das Paket  \href{http://www.ctan.org/get/macros/latex/contrib/trivfloat/trivfloat.pdf}{\texttt{trivfloat}} verwendet werden.


\subsubsection{ganzseitige PDF's einbinden}
Manchmal möchte man ganzseitige PDF-Dokumente in ein anderes Dokument einbinden. Dies geht mit dem Paket \href{http://www.ctan.org/tex-archive/macros/latex/contrib/pdfpages/pdfpages.pdf}{\texttt{pdfpages}}.

\begin{lstlisting}[frame=single]
\usepackage{pdfpages}
...
\includepdf[pages={1,3-5}]{Dateiname}
\end{lstlisting}

In diesem Fall werden vom Dokument \texttt{Dateiname.pdf} die Seiten 1 und 3 bis 5 eingebunden.


\begin{samepage}
\hypertarget{lnk:pstricks}{ }
\subsubsection{mit \LaTeX{} Zeichnungen erstellen}\end{samepage}
Hierzu bieten sich die beiden sehr umfangreichen Pakete \href{http://www.ctan.org/tex-archive/graphics/pstricks/base/doc/pstricks-doc.pdf}{\texttt{pstricks}} und \href{http://www.ctan.org/tex-archive/graphics/pgf/doc/generic/pgf/version-for-pdftex/en/pgfmanual.pdf}{\texttt{TikZ}} an.



\subsubsection{Hintergrundbilder}
Um auf einer oder mehreren Seiten ein Hintergrundbild zu platzieren, kann das \href{http://www.ctan.org/tex-archive/macros/latex/contrib/wallpaper/wallpapermanual.pdf}{\texttt{wallpaper}}-Paket verwendet werden:

\begin{lstlisting}[frame=single]
...
\usepackage{wallpaper}
...
\TileWallPaper{\paperwidth}{\paperheight}{bild}
...
\ClearWallPaper
...
\ThisTileWallPaper{\paperwidth}{\paperheight}{bild}
...
\end{lstlisting}

Mittels \texttt{\textbackslash TileWallPaper} wird das Hintergrundbild so lange angezeigt, bis es mittels \texttt{\textbackslash ClearWallPaper} wieder entfernt wird. Dieser Befehl darf \textbf{nicht} auf der selben Seite stehen, auf welcher das Hintergrundbild eingefügt wurde. 

Soll nur gerade auf einer Seite ein Hintergrundbild eingefügt werden, so kann der Befehl \texttt{\textbackslash ThisTileWallPaper} verwendet werden.


\subsubsection{Text über ein Bild legen / Grafiken beschriften}
Um Grafiken mit der selben Schriftart wie im restlichen Dokument zu beschriften, kann das \href{http://www.ctan.org/tex-archive/macros/latex/contrib/overpic/}{\texttt{overpic}}-Paket verwendet werden:

\begin{lstlisting}[frame=single]
...
\usepackage[percent]{overpic}
\usepackage{color}
...
\begin{overpic}[width=10cm,grid,tics=10]{Bild}
	\put(20,30){\textcolor{white}{etwas Text}}
\end{overpic}
...
\end{lstlisting}

\textbf{Anwendung:} Anstatt \texttt{\textbackslash includegraphics} kann eigentlich immer auch \texttt{over\-pic} verwendet werden. Die Angaben \texttt{grid} und \texttt{tics=10} sind einzig und alleine dazu da, auf dem Bild ein Raster anzuzeigen, von welchem die korrekten Koordinaten für die Beschriftung abgelesen werden können. Sobald die Beschriftung an der richtigen Stelle ist, können diese beiden Argumente entfernt werden.

Der \texttt{\textbackslash put}-Befehl fügt den eigentlichen Text ein: In diesem Fall an den Koordinaten 20 in horizontaler Richtung und 30 in vertikaler Richtung. Es ist auch möglich, ein Bild an mehreren Stellen zu beschriften. Dazu müssen lediglich mehrere \texttt{\textbackslash put}-Befehle hintereinander eingefügt werden. Damit die Beschriftung gut lesbar ist, empfiehlt es sich, eine passende Textfarbe zu wählen. Hierfür eignet sich das Paket \href{http://www.ctan.org/tex-archive/macros/latex/required/graphics/color.dtx}{\texttt{color}}.


\subsubsection{Bilder im Querformat einbinden}
Bei breiten Bildern kann es sinnvoll sein, diese auf einer einzelnen Seite im Querformat einzubinden. Hierzu stellt das Paket \href{http://www.ctan.org/tex-archive/macros/latex/contrib/rotating/}{\texttt{rotating}} eine spezielle Umgebung zur Verfügung:

\begin{lstlisting}[frame=single]
...
\usepackage{rotating}
...
\begin{document}
...
\begin{sidewaysfigure}
  \centering\includegraphics[scale=1]{Bild}
  \caption{Titel der Grafik}
\end{sidewaysfigure}
\end{lstlisting}




\subsubsection{transparente Grafiken}
Insbesonders für Bildschirmpräsentationen mit den Paketen \href{http://www.ctan.org/tex-archive/macros/latex/contrib/beamer/doc/beameruserguide.pdf}{\texttt{beamer}} oder \href{http://www.ctan.org/tex-archive/graphics/pgf/doc/generic/pgf/version-for-pdftex/en/pgfmanual.pdf}{\texttt{powerdot}} kann es nützlich sein, wenn gewisse Bereiche einer Grafik transparent sind. Dies kann mit dem Paket \href{http://www.ctan.org/tex-archive/graphics/pgf/doc/generic/pgf/version-for-pdftex/en/pgfmanual.pdf}{\texttt{pgf}} erreicht werden.



\subsubsection{Too many unprocessed floats}
Wenn viele Grafiken eingefügt werden, kann es vorkommen, dass \LaTeX{} mit folgendem Fehler abbricht:

\begin{lstlisting}[frame=single]
LaTeX Error: Too many unprocessed floats.
\end{lstlisting}

Das Problem ist, dass \LaTeX{} bei einer Gleitumgebung daran scheitert, diese vernünftig zu platzieren. Da die Reihenfolge der Grafiken nicht verändert werden darf, wird \LaTeX{} auch bei den folgenden Grafiken mit grosser Wahrscheinlichkeit scheitern. Das Problem kann meistens gelöst werden, indem an geeigneten Stellen ein \verb|\clearpage| eingefügt wird welches \LaTeX{} ermöglicht mit der Platzierung der folgenden Grafiken neu anzufangen. Besonders häufig tritt das Problem auf, wenn sehr viele Gleitumgebungen und sehr wenig Text im Dokument sind.

\textbf{Anmerkung:} Das Paket \texttt{morefloats} kann unter Umständen das Problem ebenfalls lösen, indem für die Aufgabe des Bilderplatzierens mehr Speicher zur Verfügung gestellt wird.





%---------------------------------------------------------------------------
%---------------------------------------------------------------------------
%---------------------------------------------------------------------------
%---------------------------------------------------------------------------
\subsection{Verweise, Links, Verzeichnisse, \ldots}

\subsubsection{kurze Bildunterschriften im Abbildungsverzeichnis}
Lange Bildunterschriften sehen im Abbildungsverzeichnis meist schlecht aus. Es gibt daher die Möglichkeit, neben einer ausführlichen Bildunterschrift für den Text einen weiteren kurzen Bildtitel fürs Abbildungsverzeichnis zu erstellen:


\begin{lstlisting}[frame=single]
...
\begin{figure}[htb]
	\centering
	\includegraphics{Bild}%
	\caption[Verzeichniseintrag]{viel, viel, viel zu lange Beschreibung}%
\end{figure}
...
\listoffigures
\end{lstlisting}



\subsubsection{Verweise auf Bilder}
\begin{lstlisting}[frame=single]
...
\begin{figure}[htb]
	\centering
	\includegraphics{Bild}%
	\caption{Titel}%
	\label{pic:DasBild}%
\end{figure}
...
Bild~\ref{pic:DasBild} zeigt ...
\end{lstlisting}

\textbf{Achtung:} Immer zuerst \texttt{\textbackslash caption} und danach erst \texttt{\textbackslash label}! Wird dies nicht beachtet, so stimmen die Verweise auf die Bilder nicht mit den Bildnummern überein.

\textbf{Anmerkung:} Die Tilde in der letzten Zeile sorgt dafür, dass an dieser Stelle nicht umbrochen wird und dass der Abstand konstant bleibt.


\subsubsection{Klicklinks im \texttt{pdf}}
Das Paket \href{http://www.ctan.org/tex-archive/macros/latex/contrib/hyperref/hyperref.pdf}{\texttt{hyperref}} erzeugt Links auf die Bildunterschriften. Normalerweise möchte man aber nicht die Bildunterschrift, sondern das Bild selbst verlinken. Hierzu muss das Paket \href{http://www.ctan.org/tex-archive/macros/latex/contrib/oberdiek/hypcap.pdf}{\texttt{hypcap}} geladen werden:

\begin{lstlisting}[frame=single]
...
\usepackage{hyperref} 
\usepackage[all]{hypcap}
\begin{document}
...
%Hier folgt das Bild
\end{lstlisting}

\textbf{Achtung:} \texttt{hypcap} ist eine der wenigen Ausnahmen, welche \textbf{nach} \texttt{hyperref} geladen werden müssen!

\textbf{Anmerkung:} Wird bereits das \texttt{caption}-Packet in einer Version neuer als \texttt{3.1} geladen, so sollte auf \texttt{hypcap} verzichtet werden.







\subsubsection{"`Abbildung"' umbenennen}
Das "`Abbildung"' in der Bildunterschrift ist manchmal nicht besonders passend. Möchte man beispielsweise "`Abbildung"'  im ganzen Dokument in "`Diagramm"' umändern, kann folgender Code verwendet werden:

\begin{lstlisting}[frame=single]
...
\usepackage[ngerman]{babel}
\addto\captionsngerman{\renewcommand\figurename{Diagramm}}
...
\begin{document}
...
\end{lstlisting}



\subsubsection{"`Abbildungsverzeichnis"' umbenennen}
Möchte man dem Abbildungsverzeichnis einen neuen Namen geben, so geschieht dies über folgenden Code:

\begin{lstlisting}[frame=single]
...
\usepackage[ngerman]{babel}
\addto\captionsngerman{\renewcommand\listfigurename{Diagrammverzeichnis}}
...
\begin{document}
...
\end{lstlisting}

\textbf{Anmerkung:} Es können auch andere Schlüsselwörter abgeändert werden. Die Namen der Schlüsselwörter für das Paket \texttt{babel} findet man beispielsweise beispielsweise \href{http://www.ctan.org/tex-archive/macros/latex/required/babel/ngermanb.dtx}{hier}.






\clearpage


%++++++++++++++++++++++++++++++++++++++++++++++++++++++++++++++++++
%+++ GNU FDL ++++++++++++++++++++++++++++++++++++++++++++++++++++++
%++++++++++++++++++++++++++++++++++++++++++++++++++++++++++++++++++

\section{GNU Free Documentation License}

\tiny
%\singlespacing
\begin{multicols}{2}[][2pt]
\input{gfdl}\end{multicols}
%\onehalfspacing


%++++++++++++++++++++++++++++++++++++++++++++++++++++++++++++++++++
%+++ History ++++++++++++++++++++++++++++++++++++++++++++++++++++++
%++++++++++++++++++++++++++++++++++++++++++++++++++++++++++++++++++
\section{History}
\footnotesize
In nahezu jeder neuen Version wurden einige Tippfehler behoben. Diese sind nicht explizit angegeben.

\newcommand{\myDate}[2]{\textbf{\ttfamily{#1 (v. #2)}:}}






\begin{itemize}
\item \myDate{29.08.2010}{1.50} neue Abschnitte: "`Parameter für Gleitumgebungen: Seitenbelegung"', "`Parameter für Gleitumgebungen: vertikale Abstände"', "`Bilder auf Maximalgrösse skalieren"',"`Too many unprocessed floats"',"`Dokumentation des \texttt{graphics} Paketes"'; "`begrenztes Gleiten"', "`Rahmen"', Befehl \verb|\ContinuedFloat| ergänzt; Neu unter GFDL 1.3 anstatt 1.2
\item \myDate{18.09.2008}{1.40} Codebeispiel zu Bildunterschriften in \texttt{longtable} im Abschnitt "`Bilder in Tabellen"'; Neuer Abschnitt: "`Sonderzeichen in Datei- und Pfadnamen"'; Abschnitt \texttt{floatrow}: Verweis auf das \texttt{trivfloat} Paket; Nahezu alle \verb|\textwidth| durch \verb|\linewidth| ersetzt. In einspaltigen Texten sind beide äquivalent, in mehrspaltigen Texten ist \verb|\linewidth| die Breite einer Spalte, während \verb|\textwidth| die Breite der Seite ist; Neuer Abschnitt: "`Bilder nur dann skalieren, wenn sie breiter als die Seite sind"'; Beispiel "`zwei Bilder nebeneinander"' verbessert. Text ergänzt; Titelseite überarbeitet: Nun endlich mit einem Bild :-)
\item \myDate{26.09.2007}{1.30} Vollständig überarbeitet (inklusive neuem Layout); Dokus zu den Paketen direkt verlinkt; neue Abschnitte, "`Bilder in Tabellen"' (vielen Dank an Rainer Schnaack für den Beispielcode!), "`Bilder im Querformat"', "`umflossene Bilder in mehrspaltigem Text"', \texttt{pdfpages}; Problem Kombination \texttt{subfig} und \texttt{tocloft}; Verweis auf Paket \texttt{shapepar}; Neu im Abschnitt "`Zeichen mit LaTeX"' ein Verweis auf \texttt{TikZ}; Umformulierung Abschnitt "`Treiber"'; Abschnitt "`Bilder gleiten lassen"' ergänzt (\texttt{restylefloat}); Verweis auf Paket \texttt{textpos}; Quellcode von \texttt{l2picfaq.tex} besser kommentiert (inklusive Doku-Links!)
\item \myDate{09.04.2007}{1.20} Fehler in Code "`Bild neben Text"' behoben; neuer Abschnitt \texttt{floatrow}-Paket
\item \myDate{23.03.2007}{1.16} Abschnitt "`Bilder gleiten lassen"' überarbeitet; Befehle mitten im Text neu mit Backslash
\item \myDate{15.03.2007}{1.15} Korrektur Abschnitt "`gemischte Kommandos"'; Korrektur Abschnitt \texttt{captionof}; Abschnitt "`Bilder in mehrspaltigem Text"' total überarbeitet
\item \myDate{09.03.2007}{1.14} Option \texttt{trim} ergänzt
\item \myDate{05.03.2007}{1.13} Mehrere Schreibfehler, Layoutfehler,... behoben; Neuer Abschnitt "`Bilder in mehrspaltigem Text"'
\item \myDate{15.01.2007}{1.12} Fehler im Abschnitt "`Bounding-Box"' behoben; Code Abschnitt "`<<Abbildung>> umbenennen"' verbessert
\item \myDate{27.11.2006}{1.11} Code in Abschnitt "`Abbildung umbenennen"' und "`Abbildungsverzeichnis umbennen"' sollte jetzt zuverlässiger arbeiten -- selbst wenn im Dokument die Sprache umgeschaltet wird; Umformulierung Abschnitt "`kompilieren mit latex"'; Code im Abschnitt "`Bild an aktueller Stelle einfügen"' angepasst
\item \myDate{23.07.2006}{1.10} Korrekte Übersetzung von "`float"' lautet "`gleiten"'; \texttt{epslatex.pdf} hinzugefügt; Codeschnipsel bei den Grundbeispielen durch voll funktionsfähige Minimalbeispiele ersetzt; Reihenfolge im Kapitel "`Bilder einfügen in \LaTeX"' geändert; Abschnitt \texttt[dvips] umformuliert; Umformulierung Abschnitt "`kompilieren mittels \texttt{latex}"'; Umformulierung Abschnitt "`Bounding-Box"'; Abschnitt "`\texttt{dvi}-Viewer"' vollständig umformuliert; Beispiel "`Rahmen um Bild"' wurde korrigiert; 
Bei Verwendung von \texttt{minipage} sollte \verb|\begin{center}| anstatt \verb|\centering| verwendet werden, da sonst die Abstände nicht stimmen (wurde korrigiert); Neuer Abschnitt "`Probleme mit \texttt{eps}-Dateien"'
\item \myDate{05.07.2006}{1.00} Vollständig überarbeitet (Tippfehler, unpassende Formulierungen, Textsatz, ...). \texttt{overpic}
\item \myDate{27.06.2006}{0.58} Neuer Abschnitt "`Probleme mit \texttt{hyperref}"'
\item \myDate{27.05.2006}{0.57} Druckfehler behoben.
\item \myDate{25.05.2006}{0.56} Erweiterung des Abschnittes \texttt{pst-pdf}. Beispiel zum Paket \texttt{float} wurde entfernt: Normale \texttt{figure}'s funktionieren auch ohne dieses Paket und \texttt{!htb} funktioniert im Normalfall besser als die Option \texttt{H}. \texttt{wrapfig} um Optionen \texttt{R} und \texttt{L} ergänzt.
\item \myDate{02.05.2006}{0.55} Paket \texttt{subfigure} in aktuelleres \texttt{subfig} geändert. \texttt{capt-of} durch \texttt{caption} ersetzt. \texttt{pst-pdf}. \texttt{ps2eps}. \texttt{picins}. \texttt{pgf}
\item \myDate{02.05.2006}{0.54} Hintergrundbild, Abschnitt "`fehlende Pakete"'
\item \myDate{31.03.2006}{0.53} Paket \texttt{caption}
\item \myDate{06.02.2006}{0.52} Gemischte Kommandos
\item \myDate{03.02.2006}{0.51} Einige Fehler behoben
\item \myDate{02.02.2006}{0.50} Diverse Fehler behoben, überarbeitet
\item \myDate{02.02.2006}{0.44} Paket \texttt{placeins} hinzugefügt
\item \myDate{24.01.2006}{0.43} Befehl \texttt{clearpage} hinzugefügt
\item \myDate{13.01.2006}{0.42} Verweise auf Bilder konkretisiert
\item \myDate{06.12.2005}{0.41} Fehler im Abschnitt \texttt{wrapfig} behoben
\item \myDate{04.12.2005}{0.40} Falsche Skalierung von \texttt{eps} und Skalierungstools
\item \myDate{03.11.2005}{0.31} Beispiel Verlinken verbessert
\item \myDate{23.10.2005}{0.30} Mehrere Bilder nebeneinander
\item \myDate{18.10.2005}{0.20} Abbildungsverzeichnis und Abbildung umbenennen
\item \myDate{16.10.2005}{0.10} Erste Version
\end{itemize}
% \end{multicols}

\end{document}