\documentclass[12pt,a5paper,landscape]{article}
\usepackage[utf8]{inputenc}
\usepackage{graphicx} % para inlcuir logo
\usepackage{xcolor} % para cor
\usepackage{bookman} % Fonte bookman

\usepackage{datatool} % para mala direta

\usepackage[lmargin=1cm,tmargin=1cm,bmargin=1.8cm,rmargin=1.8cm]{geometry} % margens
\usepackage{fancybox} % para colocar moldura na pagina

\usepackage{shadowtext} % texto sombreado
\usepackage[skins]{tcolorbox} % para moldura sombreada na página

\usepackage[none]{hyphenat} % sem hifenização

\sloppy % prefere underfull do que overfull

% para texto sombreado
\shadowoffset{2pt}
\shadowcolor{black!30}

% veja o uso de tcolorbox em
% https://tex.stackexchange.com/questions/223694/how-to-draw-a-text-box-with-shadow-borders-in-latex

\newcommand{\certificatebox}[1]{%
\tcbox[enhanced, boxsep=8pt, boxrule=2pt, colback=white, , shadow={2pt}{-2pt}{0pt}{black!30!white}, sharp corners]{#1}}

\fancypage{\setlength{\fboxsep}{8pt}\certificatebox}{}

\pagestyle{empty}

\begin{document}
\DTLsetseparator{,} % Separador de campos
\DTLsetdelimiter{"} % delimitador de celulas

% associa lista para o arquivo CSV
% \DTLloaddb[noheader,keys={Nome,Trabalho}]{lista}{latex-via-exemplos-lista-nomes.csv}
\DTLloaddb{lista}{latex-via-exemplos-lista-nomes.csv}

\DTLforeach{lista}{% processa cada item da lista
\personname=Nome,\worktitle=Trabalho}{%
% certificado

% incluindo a imagem de fundo (marca d'agua)
% \begin{flushleft}
%   \noindent
%   \unitlength 0.04\textwidth
%   \begin{picture}(0,0)(0,15)
%     \includegraphics[width=1.0\textwidth]{fundo}
%   \end{picture}
% \end{flushleft}

% \sffamily

% timbre
\begin{center}
%\includegraphics[width=0.1\textwidth]{logo-esquerda}
\hfill
\begin{minipage}[b]{0.7\textwidth}
\center
Universidade Federal de São Carlos \\
Centro de Ciências Tecnológicas e de Sustentabilidade \\
Departamento de Física, Quimica e Matemática
\end{minipage}
%\hfill
%\includegraphics[width=0.1\textwidth]{logo-direita}
\hfill~
\vfill
\end{center}

% titulo do certificado
\begin{center}
  \shadowtext{\scalebox{2}{\LARGE Certificado}}
\end{center}
\vfill

% corpo do certificado
\noindent
{\Large
Certificamos que \textit{\MakeUppercase\personname} apresentou o trabalho intitulado \textit{``\worktitle''} no \textsc{Nome do Congresso}, realizado no período de DATA, em LOCAL.}
\vfill

\begin{flushright}
{\Large LOCAL E DATA.}
\end{flushright}
\vfill

% campo de assinatura, etc
\begin{minipage}{0.45\textwidth}
Realização: ORGANIZADORES
\end{minipage}
%
\begin{minipage}{0.45\textwidth}
 \centering \noindent
    \underline{\hspace*{0.95\textwidth}} \\
{\Large Comissão Organizadora}
\end{minipage}

\newpage
} % \DTLforeach{lista}{%
\end{document}
