\documentclass[12pt]{article}
% 12pt torna 24pt apos dobrar a dimensao do documento usando jPDFTWeak ou poster
\usepackage[T1]{fontenc} % codificação da fonte em 8-bits
\usepackage[utf8]{inputenc} % acentuação direta
\usepackage[brazil]{babel} % em portugues brasileiro
\usepackage{lmodern} % Fonte Latin Modern (Computer Modern com extensao latin)
\usepackage{lipsum} % para preencher o espaco (para teste)
% acerto de margens usar metade da dimensao final
% para poster final de 100x120cm
% \usepackage[paperwidth=50cm,paperheight=60cm, margin=0.7cm,]{geometry}
% para poster final em A0
% \usepackage[a2paper, margin=0.7cm]{geometry}
% Poster em 90cmx120cm
\usepackage{geometry}
\geometry{paperwidth=45cm,paperheight=60cm,
lmargin=0.7cm,rmargin=0.7cm,tmargin=0.7cm,bmargin=0.7cm}
\usepackage{multicol} % usar varias colunas
\usepackage{graphicx} % usar graficos
\usepackage[usenames]{xcolor} % usar cores
\usepackage{fancybox} % para molduras adicionais nas caixas

% espacamento entre colunas
\setlength{\columnsep}{1cm}
% \setlength{\columnseprule}{1pt} % separador de colunas
% No poster, nao costuma usar a indentacao
\setlength{\parindent}{0pt} % sem indentacão

% comando para colocar titulo customizado
%----------------------------------------
\setlength{\fboxsep}{0pt} % bordas grudadas no conteudo
%% caixa de titulos: versao colorida.
\newcommand{\maintitlebox}[1]{\shadowbox{\colorbox{yellow}{\parbox{0.99\columnwidth}{#1}}}}
\newcommand{\titlebox}[1]{\fbox{\colorbox{yellow}{\parbox{1.0\columnwidth}{#1}}}}
%% caixa de titulos: versao monocromatica.
% \newcommand{\maintitlebox}[1]{\shadowbox{\parbox{0.99\columnwidth}{#1}}}
% \newcommand{\titlebox}[1]{\fbox{\parbox{1.0\columnwidth}{#1}}}

% \usepackage{cmbright} % computer modern compatible sans serif font for text and math

\renewcommand\familydefault{\sfdefault} % Usar sans serif por padrao
\pagestyle{empty} % sem enumeracao das paginas
\begin{document}
% \large % aumentar um pouco a letra (apesar de nao ser necessario)

\maintitlebox{
\begin{minipage}[t]{0.98\textwidth}
\begin{center}
\vspace{1pc}
\Huge
% podera acrescentar logo, usando inludegraphics
% \includegraphics[height=4pc]{logo-esquerda}
% \hfill
Poster de Teste
% \hfill
% \includegraphics[height=4pc]{logo-direita}
\end{center}
\vspace{1pc}
\end{minipage}
} % maintitlebox
\vspace{1pc}

\begin{center}
 {\huge Sadao Massago}

\

 {\large DFQM-UFSCar}  (Universidade Federal de São Carlos)

 web: \texttt{http://www.dm.ufscar.br/$\sim$sadao} \\
 e-mail: \texttt{sadao@ufscar.br}
\end{center}

\vspace{2pc}
% \hrule \vspace{1pc} \hrule

% \setlength\columnseprule{.4pt}
\begin{multicols}{3} % 3 colunas, por ter linha comprida 
\section*{\titlebox{Parte 1}}
\lipsum[1-2]
\section*{\titlebox{parte 2}}
\lipsum[1-5]
\section*{\titlebox{Parte 3}}
\lipsum[1-6]
\section*{\titlebox{Parte 4}}
\lipsum[1-4]
\section*{\titlebox{Parte 5}}
\lipsum[1-7]
\section*{\titlebox{Parte 6}}
\lipsum[1-3]
\end{multicols} % 2 colunas, por ter linha comprida 

\vspace{1pc}
\doublebox{
\begin{minipage}{0.99\textwidth}
\vspace{2pc}
\setlength\columnseprule{.5pt}

\begin{multicols}{2} % 2 colunas
\titlebox{Observação final:}
\lipsum[3]
\end{multicols} % 2 colunas

\vspace{1pc}
\end{minipage}
} % \doublebox{
\end{document}