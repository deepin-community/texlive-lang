\documentclass[12pt]{beamer} % para apresentacao.

\usepackage[T1]{fontenc} % codificação da fonte em 8-bits
\usepackage[utf8]{inputenc} % acentuação direta
\usepackage[brazil]{babel} % em portugues brasileiro

% \usetheme{Warsaw} % tema (modelo)
\usetheme{AnnArbor}
\usecolortheme{default} % tema de cores (Esquema de cores)
% para ver como fica as combinacoes de tema e esquema de cores,
% veja o site https://hartwork.org/beamer-theme-matrix/
% \usefonttheme[onlymath]{serif} % use serif for math
%\setbeamerfont{title}{family=\rm} % titulo em romano

\usepackage{amssymb,amsmath} % para incrementar fórmulas
\usepackage{tikz} % para cria degrade no fundo
%\usepackage[condensed,math]{iwona} % Mudando fontes
\usepackage{lipsum} % para gerar texto, para teste

% \usepackage{hyperref} % ja eh carregado pelo beamer
\hypersetup{% informacoes do PDF
  pdftitle={Slide beamer},%
  pdfauthor={Sadao Massago},
  pdfsubject={Exemplo de Slide},
  pdfkeywords={LaTeX, Slide}
} % \hypersetup

% Informacoes para criar titulo
\title[Exemplo de Slide]{Exemplo de Slide}
\author{%
  Sadao Massago\inst{1} % \and ???\inst{2}
  }
\institute[DFQM-UFSCar]{
  \inst{1}%
  Departamento de Física, Química e Matemática \\
  Universidade Federal de São Carlos
  % \and
  % \inst{2} ???
  }
\date[Março 2018]{\LaTeX{} Via Exemplos, 2018}

% fundo em degrade
\setbeamertemplate{background canvas}{%
\begin{tikzpicture}[remember picture,overlay]
%\shade[top color=red!10,bottom color=blue!10, middle color=white!10]
\shade[top color=red!10,bottom color=blue!10]
  (current page.north west) rectangle (current page.south east);
\end{tikzpicture}%     
}

\setbeamertemplate{navigation symbols}{} % desativa botao de navegacao

\begin{document}
\frame{\titlepage} % slide de titulos
\frame{\transdissolve\tableofcontents} % slide de sumario

\note{Em torno de 1 minuto para tópicos.} % notas

\section{Overlay (apresentando por etapas)} % section será usado no sumário e similares

\begin{frame} % slide
  \frametitle{Slides de apresentação}
\begin{itemize}
\item <1->\alert<1>{Usar letras grandes}
\item <2->\alert<2>{Cor do fundo deve criar contraste com texto}
\item <3->\alert<3>{Para apresentação, contraste pode ser pela cor}
\item <4->\alert<4>{Para imprimir, contraste deve ser claro/escuro}
\end{itemize}
\only<5->{\alert<5>{Escrever pouco e falar muito}}
\end{frame}

\begin{frame}
\frametitle{Blocos}
\begin{block}{}<1->
Este é um bloco sem título. Bloco aceita overlay.
\end{block}

\begin{block}{Segundo bloco}<2->
Este é um bloco com título.
\end{block}

``bloco'' é uma ``caixa'' com ou sem título e aceita o parâmetro de \texttt{overlay}.

\end{frame}

\section{Ambiente \texttt{verbatim} no slide} % outra entrada de sumário

%   Ambiente \texttt{verbatim} e similar requer opção \texttt{fragile}.
\begin{frame}[fragile] % outro slide
\frametitle{Verbatim}
Ambiente \texttt{verbatim} e similar requer opção \texttt{fragile}.
\begin{verbatim}
program teste;
begin
  writeln('Alô pessoal!');
end.
\end{verbatim}
\end{frame}

% Para listar o código do beamer, deverá criar o novo ambiente para evitar conflito de \end{frame}.
%% Crie um novo ambiente como segue, no preamble.
%\newenvironment{verbatimframe}
%{\begin{frame}[fragile,environment=verbatimframe]}
%{\end{frame}}
%% Depois use ele
%\begin{verbatimframe}
%  \frametitle{Verbatim com código beamer}
%Para listar o código do \texttt{beamer}, deverá criar o novo ambiente para evitar conflito de \verb+\end{frame}+.
%% \pause{}
%\begin{verbatim}
%\begin{frame}\frametitle{Título}
%...
%\end{frame}
%\end{verbatim}
%\end{verbatimframe}

%\begin{note} % outra nota: ambiente note so funciona fora do frame. Não funciona na versão 2016. Testar na versão mais recente.
\note{Sem a opção \texttt{fragile}, não pode usar o ambiente \texttt{verbatim} dentro do \texttt{frame}.
 Não pode usar ``\texttt{overlay}'' no ambiente \texttt{verbatim}
}
%\end{note}

\section{Quebra automática em frames} % mais uma entrada para sumário

\begin{frame}[allowframebreaks] % e mais um slide
\lipsum[1-2]
\end{frame}
\note{
Para que mude o frame automaticamente quando tornar cheio
coloque a opção \texttt{allowfreamebreaks} no frame
}

\section{Opção plain}
\begin{frame}[plain]
\frametitle{Opção \texttt{plain}}
\lipsum[1]
\end{frame}
\note{
Opção \texttt{plain} desativa cabeçalho e rodapé do frame para ter mais espaço.
Útil para colocar figura maior, por exemplo.
}

\end{document}