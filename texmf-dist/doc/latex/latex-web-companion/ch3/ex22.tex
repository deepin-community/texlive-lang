\documentclass{article}
\usepackage{html}
%begin{latexonly}
\usepackage[T1]{fontenc}
%end{latexonly}
\usepackage[dvips]{graphicx}
\usepackage{francais}
\newcommand{\Lcs}[1]{\texttt{\symbol{'134}#1}}
\begin{document}
\section{Exemple d'un tableau}
\label{sec-tableau}
Le \hyperref{tableau}{tableau }{}{tab-exa} montre
l'utilisation de l'environnement \texttt{table}.
\begin{table}
\centering
 \begin{tabular}{cccccc}
  \Lcs{primo}  & \primo & \Lcs{secundo} & \secundo & 
  \Lcs{tertio} & \tertio                           \\
  \Lcs{quatro} & \quatro& 1\Lcs{ier}    & 1\ier    & 
  1\Lcs{iere}  & 1\iere                            \\
  \Lcs{fprimo)}&\fprimo)& \Lcs{No} 10   & \No 10   & 
  \Lcs{no} 15  & \no 15                            \\
  \Lcs{og} a \Lcs{fg}&\og a \fg&3\Lcs{ieme}&3\ieme &
  10\Lcs{iemes}& 10\iemes 
 \end{tabular}
\caption{Quelques commandes de l'option 
  \texttt{french} de \texttt{babel}}\label{tab-exa}
\end{table}
\end{document}
