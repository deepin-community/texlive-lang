Omne qui habe probato ad exprime sentimentos plus profundo et plus delicato jam perveni
consciente aut inconsciente ad dilemma sequente.

Aut decide se pro illo tanto simplice, quod id pote es expresso logico-harmonico; aut
considera coh\ae{}rentia logico de lingua es inferiore et si necessario immola ad
harmonia plus alto de illo plus reale. In ultimo causa lingua es uso in toto altero modo.

Hic me dona imagine pro illustratione. Reproductione maximo completo de homocorpore
vivente es automato. Isto corresponde in forma, colore et etiam in motione ad originale.

Theorico isto es reproductione maximo bono. Sed automato composito maximo artificiale,
maximo ingenioso, maximo complicato et maximo functionante non pote repr\ae{}senta plus
multo quam illo toto simplice.

Contrario, ad spectatore sensitivo et experimentato aliquo lineas nigro super papyro
candido pote produc plus solido et plus justo effectu de vita. Tamen ignorante pr\ae{}fer
automato, nam isto respira et rota cum oculos \ae{}quale ad homo vivente. Secundum suo
obscrvatione rude isto es coh\ae{}rentia de realitate et pro isto corresponde maximo bono
per coh\ae{}rentia de reproductione. Isto es logico pro illo, contrario creta-lineas
nigro non possido sensu et es illogico.

Isto imagine es utile, si id es relicto ad tempore. Nos non es habituato ad automatos
pulcro et desiderato. Sed productione de lingua-structura logico-harmonico que
corresponde ad negotios simplice et generale, in que nos debe intercomprehende, habe
importantia grande.

Discursu ideale es illo mathematico. Move ex axiomas. Introduc solo terminos de valore
determinato et invariabile. Veni inductivo de illo simplice ad illo complicato. Progredi
synthetico, non plus rapido quam affirmatione completo de antecedente tolera. Sed solo in
causas abstracto tale discursu es rigoroso possibile. Introductione de termino concreto
fac que exactitudine fi simulante.

Termino <<substantia>>, termino <<es>>, <<natura>>, <<comprehende>> non es, et si
abstracto, tam exacto quam terminos <<tres>>, <<simile>>, <<dissimile>>, <<parallelo>>,
<<congruente>>. Tamen illos es plus abstracto, minus concreto et tunc minus reale pro
nostro intellectu quam terminos <<homo>>, <<\ae{}re>>, <<dolore>>, etc. Tunc gradu de
abstractione exsiste. Exemplo de gradu superiore es <<tres>>; de gradu inferiore es
<<dolore>> et solo dolore determinato de persona determinato in istante determinato;
inter isto gradu es: <<realitate>>. Discursu es plus exacto et suo effectu es plus certo,
secundum que suo terminos es minus concreto.

Illo concreto simula plus reale quam illo abstraoto. Nos concipe que <<libro>> es
exsistente, <<numero>> non. Etiam hic es gradatione. Objcctos materiale simula maximo
reale. Consideratione simplice et analysi de isto conceptione duc nos ad adopta non
objecto sed suo impressione sensuale, psyche-impressione, quam maximo reale. <<Lumine>>
quam impressione es primo realitate, maximo certo, – <<sol>> es conceptione facto ex
observationes.

Et si nos voca numeros quam objecto, nos cognosce que id non exsiste, si non quam idea.
<<Tres>> es nullo, non pote es <<tres>>, sed pote es tres objecto. Tunc toto mathematica
non applicato es irreale, contra reale. Id es modo de realitate repr\ae{}sentato aut
figurato in nostro cogitatione ab symbolos. Id es systema de relationes in multitudine
(numeros) aut in prolixitate, sed vacuo et impossibile, sine continentia de realitates.
Isto realitates debe es sentimento, nam nos non cognosce altero. In idea de mathematica
nos substitue continentia de unitate (in numero) ficto aut de linea et figura. In
mechanica continentia es <<fortia>>. Nos intellige que isto nonexistente, illo forma, es
minimo fallace, es maximo justo in suo coh\ae{}rentia.

Et in omne es gradatione. Es limite inter concreto, plus reale, sed minus exacto in
coh\ae{}rentia, et abstracto, irreale, sed exacto et stabile in coh\ae{}rentia.

Nos non vide limite pr\ae{}ciso et separatione pr\ae{}ciso.

Simula que concreto junge se evolutivo ad abstracto. Tamen consideratione pr\ae{}ciso
permitte ad nos admitte quo existe puncto ubi abstracto puro obtine continentia concreto.
Omne relatione ipso pote es deducto uno ex alio, initiante ab axioma, absoluto certo, –
sed continentia de relatione in nullo tempore pote es deducto ex relatione. Continentia
de relatione es observatione aut sentimento et pro isto non pote es deducto altero modo.

Concreto et abstracto es tunc simile duo fluvio, que non conflue, sed es parallelo in
coh\ae{}rentia perpetuo.

Et coh\ae{}rentia in tale modo que relationes fi plus complicato secundum suo continentia
fi plus concreto. Mathematica de mathematica es maximo simplice, mathematica de maximo
concreto, de emotiones, es infinito complicato.

Ad isto conditione de res, tam que nos senti id in nos ipso per reflexione, lingua
corresponde per suo reproductione et suo expressione. Lingua curre parallelo ad
situatione de psyche in suo gradus de concreto ad abstracto et illo conveniente.

Et tale modo que illo maximo abstracto es repr\ae{}sentato per vocabulos maximo
symbolico.

Quanto res, que nos debe exprime, es plus concreto, tanto plus lingua es sentimento et
sono, non symbolico, imitativo et figurativo.

Homo pote cogita que isto non apto coh\ae{}re pro nomine de objecto. Non objecto es re
maximo concreto, sed emotione. Et isto es expresso maximo bono per lingua minimo
symbolico, per sono-expressione directo, simplice, per lingua figurativo (p\oe{}si).

Exclamatione quale <<o>>, <<heu>> – vocabulo quale <<gaudio>>, <<triste>> es minimo
symbolico et significa re maximo concreto. Vocabulo de mathematica es symbolo puro.

Discursu es plus exacto et suo effectu es plus certo secundum que suo termine es minus
concreto. Tunc lingua-structura es minus fallibile secundum que id repr\ae{}senta notione
plus abstracto. Aut etiam, quanto plus puro symbolico, tanto plus certo lingua es. Lingua
absoluto symbolico, mathematica, es infallibile.

Et \ae{}quale quanto plus lingua muta de symbolico ad non-symbolico (imitativo,
figurativo) aut expressivo directo, repr\ae{}sentante notione plus abstracto, tanto plus
complicato suo relatione fi. Symbolica et repr\ae{}sentatione in lingua corresponde ad
differentia essentiale, sine limite visibile pr\ae{}ciso inter abstracto et concreto.

Hic etiam duo fluvio es, de quc limite non pote es percepto pr\ae{}ciso, sed que in nullo
tempore pote conflue. Mathematica es symbolico puro, p\oe{}si es figurativo quasi puro,
philosophia es semi-symbolico, semi-figurativo.

Noscente que repr\ae{}sentatione symbolico de re solo tolera positivitate et
coh\ae{}rentia perfecto: logica, homo habe probato exprime existente per lingua
symbolico. Tamen isto probationes es omne tempore vano, \ae{}quale quam deduc concreto ex
abstracto.

Illo duo fluvio curre parallelo, tam in re quam in suo reproductione per lingua. Ergo
lingua es uso in duo modo, que differ essentiale:
\begin{enumerate}
	\item in modo symbolico, repr\ae{}sentante relatione et notione abstracto.
	\item in modo figurativo, repr\ae{}sentante continentia de relatione, re concreto.
\end{enumerate}

Omne de isto modo de lingua-usu es independente. Nullo tempore illos pote substitue
reciproco, ce que homo pote exprime per versu illo non pote exprime per discursu, et
viceversa.

Solo in lingua puro symbolico logica, absoluto es et. poxsibile et necessario. Secundum
que lingua fi plus figurativo, id repr\ae{}senta re plus concreto, possibilitate et
necessitate de relatione exacto logico cessa.

Etiam re eoncreto es inter se subordinato ad \ae{}quale relatione quam illo abstracto,
sed in eomplicatione infinito. Lingua figurativo non exprime isto relatione. Isto non es
suo potentia nec es in suo intentione. Id vol dona solo repr\ae{}sentatione exacto de
illo concreto, de emotione aut de situatione de psyche, que es perceptibile. Ergo suo
postulato non es logica, sed harmonia pr\ae{}ciso cum situatione psychico.

Solo mathematica es symbolico puro. Tamen in scientia maximo exacto, in philosophia
maximo abstr\ae{}to lingua es uso occasionale figurativo.