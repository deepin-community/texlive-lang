\documentclass[a4paper]{article}

\usepackage[utf8x]{inputenc}
%\usepackage[utf8x]{luainputenc} %% in case using lualatex
\usepackage[L7x]{fontenc}
\usepackage[lithuanian]{babel}
\usepackage[pdftex,unicode]{hyperref}
%\usepackage{tgtermes}
%\usepackage{tgpagella}
%\usepackage{tgbonum}
%\usepackage{tgschola}

% Titulinio puslapio reikalai
\title{Ištrauka iš Wikipedijos}
\date{2006}

\begin{document}
\maketitle
\section{Istorija}


Lietuvių kalba kaip baltų kalbų grupės kalba yra glaudžiai susijusi su 
\href{http://lt.wikipedia.org/wiki/Latvi%C5%B3_kalba}{latvių kalba} 
ir išmirusia \href{http://lt.wikipedia.org/wiki/Pr%C5%ABs%C5%B3_kalba}{prūsų kalba}.

Lietuvių kalba kaip atskira rytų baltų šakos kalba pietinėje rytinių baltų dalyje ėmė klostytis 
nuo \href{http://lt.wikipedia.org/wiki/VII_am%C5%BEius}{VII a.} VI—VII a. latvių ir lietuvių kalbos 
atsiskyrė viena nuo kitos; vėliau ėmė skilti į tarmes. Manoma, kad apie XIII—XIV a. 
lietuvių kalboje ėmė išsiskirti pagrindinės 
\href{http://lt.wikipedia.org/w/index.php?title=Auk%C5%A1tai%C4%8Di%C5%B3_tarm%C4%97&amp;action=edit}{aukštaičių}
 ir \href{http://lt.wikipedia.org/wiki/%C5%BDemai%C4%8Di%C5%B3_tarm%C4%97"}{žemaičių tarmės}, 
kurios paskui dar smulkiau skaidėsi patarmėmis, o šios savo ruožtu — šnektomis bei pašnektėmis.

\href{http://lt.wikipedia.org/wiki/Auk%C5%A1taitija}{Aukštaičių} dabar yra trys pagrindinės patarmės:
 rytų, vakarų ir pietų aukštaičiai, arba dzūkai, 
o \href{http://lt.wikipedia.org/wiki/%C5%BDemaitija}{žemaičių} — taip pat trys: 
vakarų (arba klaipėdiškiai; donininkai), šiaurės vakarų (arba telšiškiai; dounininkai) 
ir pietų (arba raseiniškiai; dūnininkai).

Dabartinės literatūrinės kalbos pagrindas remiasi vakarų aukštaičių pietiečių (suvalkiečių) tarme, 
išlaikiusia senesnes fonetikos ir morfologijos lytis.

Seniausieji lietuvių kalbos paminklai siekia XVI a. pradžią. Pirmasis žinomas lietuviškas raštas — 
anoniminis poterių tekstas, ranka įrašytas į 1503 m. Štrasburge išleistos knygos 
„Tractatus sacerdotalis“ paskutinį puslapį. Tekstas remiasi 
\href{http://lt.wikipedia.org/wiki/Dz%C5%ABkai}{dzūkų} tarme ir veikiausiai yra nuorašas iš dar 
ankstesnio originalo. Nėra abejonės, kad bažnytinių lietuviškų rankraštinių tekstų būta ir anksčiau, 
gal net XIV a. pabaigoje, kadangi 1387 m. įvedus aukštaičiuose krikščionybę, tokių tekstų būtinai 
reikėjo religinei praktikai (istoriniuose šaltiniuose yra užuominų, kad pirmasis poterius į lietuvių 
kalbą esąs išvertęs \href{http://lt.wikipedia.org/wiki/Jogaila}{Jogaila}).

Tačiau seniausia žinoma spausdinta lietuviška knyga yra 
\href{http://lt.wikipedia.org/wiki/Martynas_Ma%C5%BEvydas}{Martyno Mažvydo} 1547 m. katekizmas, parašytas 
žemaičių tarmės pagrindu ir išspausdintas 
\href{http://lt.wikipedia.org/wiki/Kaliningradas}{Karaliaučiuje}. Šiame katekizme įdėtas ir pirmas lietuviškas 
elementorius „Pigus ir trumpas mokslas skaityti ir rašyti“, kuriame pateiktas lietuviškas raidynas ir vienas 
kitas autoriaus sukurtas gramatikos terminas (balsinė „balsė", sąbalsinė „priebalsė" ir kt.). 
Su M. Mažvydo katekizmu prasidėjo naujas etapas lietuvių kalbos istorijoje — ėmė kurtis ir plėtotis 
lietuvių literatūrinė kalba, reikšminga lietuvių tautos kultūrinio gyvenimo priemonė.

Apie 1620 m. pasirodė ir pirmasis lietuvių kalbos žodynas — 
\href{http://lt.wikipedia.org/wiki/Konstantinas_Sirvydas}{Konstantino Sirvydo} „Dictionarium trium linguarum“, 
susilaukęs penkių leidimų, o 1653 m. buvo išleista pirmoji lietuvių kalbos gramatika — Danieliaus Kleino 
„Grammatica Litvanica“. Taip XVII a. viduryje prasidėjo ir mokslinis lietuvių kalbos tyrinėjimas, kuris 
ypač suintensyvėjo XIX a., atsiradus lyginamajai istorinei kalbotyrai.

\section{Abėcėlė}

Lietuvių kalbos abėcėlę sudaro 32 raidės:

Aa Ąą Bb Cc Čč Dd Ee Ęę Ėė Ff Gg Hh Ii Įį Yy Jj Kk Ll Mm Nn Oo Pp Rr Ss Šš Tt Uu Ųų Ūū Vv Zz Žž

\MakeUppercase{Aa Ąą Bb Cc Čč Dd Ee Ęę Ėė Ff Gg Hh Ii Įį Yy Jj Kk Ll Mm Nn Oo Pp Rr Ss Šš Tt Uu Ųų Ūū Vv Zz Žž}

\MakeLowercase{Aa Ąą Bb Cc Čč Dd Ee Ęę Ėė Ff Gg Hh Ii Įį Yy Jj Kk Ll Mm Nn Oo Pp Rr Ss Šš Tt Uu Ųų Ūū Vv Zz Žž}

\textit{Aa Ąą Bb Cc Čč Dd Ee Ęę Ėė Ff Gg Hh Ii Įį Yy Jj Kk Ll Mm Nn Oo Pp Rr Ss Šš Tt Uu Ųų Ūū Vv Zz Žž}

\textsc{Aa Ąą Bb Cc Čč Dd Ee Ęę Ėė Ff Gg Hh Ii Įį Yy Jj Kk Ll Mm Nn Oo Pp Rr Ss Šš Tt Uu Ųų Ūū Vv Zz Žž}

{\fontshape{scsl}\selectfont{Aa Ąą Bb Cc Čč Dd Ee Ęę Ėė Ff Gg Hh Ii Įį Yy Jj Kk Ll Mm Nn Oo Pp Rr Ss Šš Tt Uu Ųų Ūū Vv Zz Žž}}

\textbf{Aa Ąą Bb Cc Čč Dd Ee Ęę Ėė Ff Gg Hh Ii Įį Yy Jj Kk Ll Mm Nn Oo Pp Rr Ss Šš Tt Uu Ųų Ūū Vv Zz Žž}

\textbf{\textit{Aa Ąą Bb Cc Čč Dd Ee Ęę Ėė Ff Gg Hh Ii Įį Yy Jj Kk Ll Mm Nn Oo Pp Rr Ss Šš Tt Uu Ųų Ūū Vv Zz Žž}}

\textsf{Aa Ąą Bb Cc Čč Dd Ee Ęę Ėė Ff Gg Hh Ii Įį Yy Jj Kk Ll Mm Nn Oo Pp Rr Ss Šš Tt Uu Ųų Ūū Vv Zz Žž}

\texttt{Aa Ąą Bb Cc Čč Dd Ee Ęę Ėė Ff Gg Hh Ii Įį Yy Jj Kk Ll Mm Nn Oo Pp Rr Ss Šš Tt Uu Ųų Ūū Vv Zz Žž}


\end{document}
