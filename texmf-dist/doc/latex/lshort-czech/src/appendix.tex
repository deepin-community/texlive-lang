\appendix
\renewcommand\thechapter{\Alph{chapter}.}
\renewcommand\thesection{\thechapter\arabic{section}.}
\renewcommand\thesubsection{\thesection\arabic{subsection}.}
\chapter{Instalace \LaTeX u}
\begin{intro}
Donald E.~Knuth publikoval zdrojové kódy \TeX u v~době, kdy ještě nikdo neslyšel
o~OpenSource a/nebo Free Software. Licence, kterou je \TeX{} opatřen,
dovoluje, abyste zdrojové kódy \TeX{}u použili jakýmkoliv způsobem. Pokud
ale zdrojové kódy upravíte a~vytvoříte z~nich nový systém, můžete ho nazývat \TeX{} jen
v~případě, že výsledný program uspěje v~sadě testů, které Knuth dal
také k~dispozici. Díky tomu máme dnes \TeX ové implementace zadarmo pro téměř
každý známý operační systém. 

V~této příloze naznačíme, co je pro zprovoznění \TeX u
na Linuxu, Mac~OS~X nebo Microsoft Windows potřeba nainstalovat.
\end{intro}

\section{Co instalovat}

Ať chcete \LaTeX{} používat na jakémkoliv počítači, budete potřebovat
čtyři programy:

\begin{enumerate}

\item Textový editor, v~kterém budete vytvářet a~upravovat vaše \LaTeX ové
zdrojové soubory.

\item Program \TeX{}/\LaTeX{}, který vaše \LaTeX ové zdrojové soubory
vysází, tj. převede do formátu PDF nebo DVI.

\item Prohlížeč PDF/DVI -- program, který umožňuje prohlížet a~tisknout
vaše dokumenty.

\item Program, který umožní přidávat do vašeho dokumentu Postscriptové
soubory a~obrázky.

\end{enumerate}

Pro každou platformu je k~dispozici mnoho takových programů. Zde zmíníme
jen ty, které jsme si vyzkoušeli a~líbí se nám.

\section{\TeX{} na Mac OS X}

\subsection{Volba editoru}

Založte své \LaTeX ové prostředí na editoru \wi{TextMate}! TextMate je
vysoce konfigurovatelný textový editor pro obecné použití. Navíc skvělým
způsobem podporuje \LaTeX{} a~je úzce zintegrován s~prohlížecím
programem \mbox{PDFView}. Tato kombinace vám umožní používat \LaTeX{}
šikovným \uv{Macovým} způsobem.
Zkušební verzi si můžete zadarmo stáhnout z~domovské stránky TextMate 
na adrese \url{http://macromates.com/}, plná verze stojí 39~EUR.
Znáte-li podobný OpenSource nástroj pro Mac, dejte nám prosím vědět.

\subsection{Obstarejte si \TeX ovou distribuci}

Pokud pro instalování programů na OS X používáte Macports nebo Fink,
použijte tyto manažery balíků i~na instalaci \LaTeX u. Uživatelé Macportu
nainstalují \LaTeX{} pomocí \framebox{\texttt{port install tetex}},
uživatelé Finku použijí příkaz \framebox{\texttt{fink install tetex}}.

Pokud Macports ani Fink nepoužíváte, stáhněte si \wib{MacTeX}{Mac\TeX}, což je
zkompilovaná \LaTeX ová distribuce pro OS X. \wib{MacTeX}{Mac\TeX} poskytuje
plnou \LaTeX ovou instalaci a~navíc řadu přídavných nástrojů. Mac\TeX
můžete získat na adrese \url{http://www.tug.org/mactex/}.

\subsection{Dopřejte si \wi{PDFView}}

Pro prohlížení PDF souborů vygenerovaných \LaTeX em použijte PDFView. Tento
program je úzce integrován s~vaším \LaTeX ovým textovým editorem. PDFView
je open-source aplikace a~můžete si ho stáhnout z~domovské stránky
\url{http://pdfview.sourceforge.net/}. Po nainstalování se ujistěte,
že v~dialogu \emph{preferences} programu je aktivováno
nastavení \emph{automatically reload documents} a~že je nastavená
podpora pro TextMate.

\section{\TeX{} na Windows}

\subsection{Obstarání \TeX u}

Nejdříve ze všeho si z~adresy \url{http://www.miktex.org/} obstarejte distribuci
\wib{MiKTeX}{MiK\TeX}. Obsahuje všechny základní programy a~soubory potřebné
pro přeložení \LaTeX ových dokumentů. Já osobně jsem unešen z~toho,
že MiK\TeX{} je schopen automaticky stáhnout a~nainstalovat chybějící \LaTeX ové balíky
během zpracování dokumentu.

\subsection{\LaTeX ový editor}

\LaTeX{} je programovacím jazykem pro textové dokumenty. \wib{TeXnicCenter}{\TeX nicCenter}
používá mnoho konceptů z~programovacího světa a~díky tomu poskytuje
elegantní a~efektivní prostředí pro editování \LaTeX ových zdrojových
souborů na Windows. Můžete si ho obstarat na~\url{http://www.toolscenter.org/}.
\TeX nicCenter je šikovně zintegrováno s~MiK\TeX em.

Další výbornou volbou je editor projektu LEd, dostupný na serveru \url{http://www.latexeditor.org/}.

\subsection{Práce s~grafikou}

Chcete-li v~dokumentu zpracovaném \LaTeX em použít grafiku vysoké kvality,
obrázky, které do dokumentu přidáváte musí být ve formátu Postscript (eps)
nebo PDF. Program GhostScript lze použít pro práci s~těmito obrázkovými formáty
a~můžete ho získat, spolu s~náhledovým programem \wi{GhostView}, z~adresy
\url{http://www.cs.wisc.edu/~ghost/}.

Pro bitmapovou grafiku (fotky nebo naskenované obrázky)
můžete použít program \wi{Gimp} -- open source alternativu k~programu
Photoshop. Program získáte na adrese \url{http://gimp-win.sourceforge.net/}.

Pro vektorovou grafiku zkuste Inkscape, \url{http://inkscape.org/}.

\section{\TeX{} na Linuxu}

\LaTeX{} bývá často instalován spolu s~operačním systémem Linux, nebo je
minimálně dostupný ve zdroji vaší instalace. Pomocí svého manažera balíků
si nainstalujte následující:

\begin{itemize}
\item tetex nebo texlive (texlive-full) -- základní \TeX ové/\LaTeX ové nastavení.
\item emacs (s~auctex) -- Linuxový editor, který je s~\LaTeX em úzce integrován pomocí přídavného balíku Auc\TeX.
\item ghostscript -- program na prohlížení a~tisknutí PostScriptových dokumentů.
\item xpdf a~acrobat -- program na prohlížení a~tisknutí PDF dokumentů.
\item imagemagick -- volně dostupný program pro konvertování bitmapových obrázků.
\item gimp -- volně dostupná alternativa k~programu Photoshop.
\item inkscape -- volně dostupná alternativa k~programům Illustrator a~Corel.
\end{itemize}

\section{Projekt \TeX onWeb}

Pokud nemůžete na počítači instalovat (např. nemáte administrátorská práva) ani si spustit \TeX{} z~DVD Live, vyzkoušejte projekt \TeX onWeb (autor Jan Přichystal, Brno), \url{http://tex.mendelu.cz/}.

