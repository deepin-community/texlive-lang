%%%%%%%%%%%%%%%%%%%%%%%%%%%%%%%%%%%%%%%%%%%%%%%%%%%%%%%%%%%%%%%%
% Contents: Math typesetting with LaTeX
% $Id: math.tex 534 2015-04-09 13:03:16Z oetiker $
%
% Changes by Stefan M. Moser: 2008/10/22
%
% -Section 2: "Single Equations": added comment about preference of
%  equation* over \[
% -Replaced (almost) all examples with \[ by equation*
% -New section 4: "Single Equations that are Too Long: multline"
% -New section 5: "Multiple Equations"
% -Section 6: "Arrays and Matrices": made a full section and added
%  some material
% -Section 9: "Theorems, Lemmas, ...": added a subsection about proofs
%  with new material
%
% Other Changes:
% -in lshort.sty:
%    *example environment adapted: changed in three places
%     \textwidth by \linewidth. This is necessary for
%     example-environment within a itemize-list.
%    *added \RequirePackage[retainorgcmds]{IEEEtrantools}
%
% THINGS TO DO:
% -adapt typesetting of new sections to rest of lshort, including all
%  the usual commands used so far. In particular, I guess we have to
%  get rid of the \verb-commands everywhere
% -include index-commands
%%%%%%%%%%%%%%%%%%%%%%%%%%%%%%%%%%%%%%%%%%%%%%%%%%%%%%%%%%%%%%%%%

\chapter{Valemite vormistamine}

\begin{intro}
Nüüd oled valmis! Selles peatükis ründame \TeX i peamist tugevust,
valemite ladumist. Kuid ole hoiatatud, see peatükk puudutab ainult
pealispinda. Kuigi siin selgitatud asjad on enamasti piisavad, ära heida
meelt, kui siit oma valemite vormistusprobleemile lahendust ei leia.
Väga tõenäoliselt on seda probleemi käsitletud \AmS-\LaTeX is.
\end{intro}

\section{\texorpdfstring{\AmS}{AMS}-\LaTeX i komplekt}

Soovides panna kirja (keerukamaid) \index{valem}valemeid, tuleks
kasutada \index{AmSLaTeX@\AmS-\LaTeX}\AmS-\LaTeX i. \AmS-\LaTeX{} on
kogum pakette ja klasse matemaatilise trükiladumise tarbeks. Enamasti
tegeleme selle komplekti paketiga \pai{amsmath}. \AmS-\LaTeX i on loonud
\wi{Ameerika Matemaatika Selts} ja seda kasutatakse laialdaselt
matemaatilise teksti vormistamiseks. Ka \LaTeX il endal on olemas mõned
elementaarsed vahendid ja keskkonnad valemite moodustamiseks, kuid need
on piiratud (või ehk vastupidigi: \AmS-\LaTeX{} on \emph{piiramatu}!) ja
mõnel juhul ebaühtlased.

\AmS-\LaTeX{} kuulub distributsiooni tuuma ja tuleb kaasa kõigi
hilisemate \LaTeX i distributsioonidega.\footnote{Kui see siiski
konkreetsest distributsioonist puudub, tuleks seda otsida aadressilt
\CTAN|pkg/amslatex|.} Selles peatükis eeldame, et \pai{amsmath} loetakse
sisse preambulis: \verb|\usepackage{amsmath}|.

\section{Üksikvalemid}

Valemi võib trükkida rea sees (\emph{\wi{tekstistiil}is}) või
tükeldada lõik ja trükkida valem eraldi real
(\emph{\wi{esitlusstiil}is}). Lõigu \emph{sees} olevad valemid
kirjutatakse \index{$@\texttt{\$}} %$
\texttt{\$} ja \texttt{\$} vahele:

\begin{example}
Liida $a$ ruudus ja $b$ ruudus
ning saad $c$ ruudus. Või
matemaatilisemalt kõneldes:
$a^2 + b^2 = c^2$
\end{example}
\begin{example}
\TeX{} hääldub kui
$\tau\epsilon\chi$\\[5pt]
100~m$^{3}$ vett\\[5pt]
See tuleb minu $\heartsuit$-st
\end{example}

Kui on vaja, et suurem valem trükitaks ülejäänud lõigust eraldi, siis on
mõistlik ta \emph{esile tõsta}. Selleks tuleks valem panna
spetsiaalsesse valemikeskkonda käskude
\verb|\begin{|\ei{equation}\verb|}| ja \verb|\end{equation}|
vahele.\footnote{See on \pai{amsmath}i keskkond. Kui mingil arusaamatul
põhjusel sellele paketile juurdepääs puudub, võib selle asemel kasutada
\LaTeX i enda keskkonda \ei{displaymath}.} Seejärel saab käsu \ci{label}
abil ära märkida valemi numbri ja viidata sellele mujalt tekstist käsuga
\ci{eqref}. Kui on tarvis anda valemile omaette nimi, võib selle määrata
käsuga \ci{tag}.
\begin{example}
Liida $a$ ruudus ja $b$ ruudus
ning saad $c$ ruudus. Või
matemaatilisemalt kõneldes:
 \begin{equation}
   a^2 + b^2 = c^2
 \end{equation}
Einstein ütleb:
 \begin{equation}
   E = mc^2 \label{tark}
 \end{equation}
Ta ei ütelnud
 \begin{equation}
  1 + 1 = 3 \tag{rumal}
 \end{equation}
See on viide valemile
\eqref{tark}.
\end{example}

Kui valemeid ei ole vaja automaatselt nummerdada, siis tuleks kasutada
keskkonna \ei{equation} tärniga varianti
\ei{equation*},\footnote{See on jällegi \pai{amsmath}. Standard-\LaTeX
is on olemas ainult keskkond \ei{equation} ilma
tärnita.}\index{keskkond!tärniga}\index{tärniga keskkond} või mis veel
lihtsam, panna valem \ci{[} ja \ci{]} vahele:
\begin{example}
Liida $a$ ruudus ja $b$ ruudus
ning saad $c$ ruudus. Või
matemaatilisemalt kõneldes:
 \begin{equation*}
   a^2 + b^2 = c^2
 \end{equation*}
või vähema sisestamistööga,
aga ikka sama tulemust saades:
 \[ a^2 + b^2 = c^2 \]
\end{example}
Kuigi \ci{[} on lühike ja kena, ei luba ta ümberlülitumist nummerdatud ja
nummerdamata stiili vahel nii lihtsasti kui \ei{equation} ja
\ei{equation*}.

Tasub tähele panna erinevust \wi{tekstistiil}is ja \wi{esitlusstiil}is
valemite vahel:
\begin{example}
See on tekstistiil:
$\lim_{n \to \infty}
 \sum_{k=1}^n \frac{1}{k^2}
 = \frac{\pi^2}{6}$.
Ja see on esitlusstiil:
 \begin{equation}
  \lim_{n \to \infty}
  \sum_{k=1}^n \frac{1}{k^2}
  = \frac{\pi^2}{6}
 \end{equation}
\end{example}

Tekstistiilis võib kõrged või sügavad matemaatilised avaldised või
alam"-avaldised anda käsu \ci{smash} argumendiks. Sellega ignoreerib
\LaTeX{} nende avaldiste püstsuunalist ulatust, mis hoiab reavahe
ühtlasena.

\begin{example}
Avaldis $a_{l_{l_a}}$, millele
järgneb teine avaldis
$\ddot u^{l^{e^s}}$. Võrdluseks
purustatud avaldis
\smash{$a_{l_{l_a}}$}, millele
järgneb avaldis
\smash{$\ddot u^{l^{e^s}}$}.
\end{example}

\subsection{Valemire\v{z}iim}

Erinevusi on ka \emph{\wi{valemire\v{z}iim}i} ja
\emph{tekstire\v{z}iimi} vahel. Näiteks on \emph{valemire\v{z}iimil}
järgmised iseärasused.

\begin{enumerate}

\item \index{vahe!valemis}\index{tühik!valemis}Enamikul tühikutel ja
reavahetustel puudub igasugune tähtsus, sest kõik tühikud kas
tuletatakse loogiliselt avaldisest endast või sisestatakse erikäskudega
nagu \ci{,}, \ci{quad} või \ci{qquad} (hiljem tuleme selle juurde
tagasi, vt jaotist~\ref{sec:math-spacing}).

\item Tühjad read ei ole lubatud. Ainult üks lõik valemi kohta.

\item Iga tähte käsitletakse muutujanimena ja vormistatakse vastavalt.
Kui valemi sees on vaja trükkida tavalist teksti (tavalises püstkirjas
ja tavaliste vahedega), siis tuleks see tekst valemisse lisada käsuga
\verb|\text{...}| (vt ka jaotist \ref{sec:fontsz} leheküljel
\pageref{sec:fontsz}).

\end{enumerate}
\begin{example}
$\forall x \in \mathbf{R}:
 \qquad x^{2} \geq 0$
\end{example}
\begin{example}
$x^{2} \geq 0\qquad
 \text{iga }x\in\mathbf{R}
 \text{ puhul}$
\end{example}

Matemaatikud võivad olla sümbolite suhtes väga pirtsakad: harilikult
kasutatakse siin "`\wi{tahvlipaks}u"' kirja,\index{paksud sümbolid} mis
saadakse käsuga \ci{mathbb} paketist \pai{amssymb}.\footnote{Pakett
\pai{amssymb} ei kuulu paketikogumikku
\AmS-\LaTeX{}, kuid installitud \LaTeX i
süsteemis on ta arvatavasti siiski olemas. Tasub kontrollida oma
distributsiooni või hankida see pakett aadressilt
\CTAN|pkg/amsfonts|.}\index{AmSLaTeX@\AmS-\LaTeX}
\ifx\mathbb\undefined\else Viimane näide teiseneb siis kujule
\begin{example}
$x^{2} \geq 0\qquad
 \text{iga } x
 \in \mathbb{R} \text{ puhul}$
\end{example}
\fi
Valemikirju leiab ka tabelist~\ref{mathalpha}
leheküljel~\pageref{mathalpha} ja tabelist~\ref{mathfonts}
leheküljel~\pageref{mathfonts}.

\section{Valemite ehituskivid}

Selles jaotises kirjeldame kõige tähtsamaid valemitrükkimiskäske. Enamik
selle jaotise käskudest ei nõua paketti \pai{amsmath} (kui nõuavad,
siis on see selgelt välja toodud), kuid nimetatud paketi võib ikkagi
sisse lugeda.

\textbf{Väikesed \wi{kreeka tähed}} sisestatakse kui \verb|\alpha|,
\verb|\beta|, \verb|\gamma|, \ldots{} ning suured tähed kui
\verb|\Gamma|, \verb|\Delta|, \ldots\footnote{Suurt alfat, beetat jne
\LaTeX{} ei defineeri, sest need näevad välja samasugused nagu ladina A,
B, \ldots}

Kreeka tähtede loend on tabelis~\ref{greekletters}
leheküljel~\pageref{greekletters}.
\begin{example}
$\lambda,\xi,\pi,\theta,
 \mu,\Phi,\Omega,\Delta$
\end{example}

\textbf{Astendajad}, \textbf{ülaindeksid} ja \textbf{alaindeksid}
määratakse
sümbolitega\index{astendaja}\index{ülaindeks}\index{alaindeks}
\verb|^|\index{^@\texttt{\^{}}} ja \verb|_|\index{_@\texttt{\_}}. Enamik
valemire\v{z}iimi käske mõjutab ainult järgmist märki, mistõttu tuleks
olukorras, kus käsu mõju peab ulatuma mitmele märgile, ühendada need
\index{rühm}rühmaks looksulgude \verb|{...}| abil.

Tabelis~\ref{binaryrel} leheküljel~\pageref{binaryrel} on loetletud
palju kahekohalisi relatsioone, nagu näiteks $\subseteq$ ja $\perp$.

\begin{example}
$p^3_{ij} \qquad
 m_\text{Knuth}\qquad
\sum_{k=1}^3 k \\[5pt]
 a^x+y \neq a^{x+y}\qquad
 e^{x^2} \neq {e^x}^2$
\end{example}

\textbf{\index{ruutjuur}Ruutjuur} saadakse kui \ci{sqrt}, $n$-s juur
sisestatakse konstruktsiooniga \verb|\sqrt[|$n$\verb|]|. Juure suuruse
valib \LaTeX{} automaatselt. Kui vaja on ainult juuremärki, siis selleks
on käsk \ci{surd}.

Tabelis~\ref{tab:arrows} leheküljel \pageref{tab:arrows} on kujutatud
mitmesugust liiki nooled, nagu $\hookrightarrow$ ja $\rightleftharpoons$.
\begin{example}
$\sqrt{x} \Leftrightarrow x^{1/2}
 \quad \sqrt[3]{2}
 \quad \sqrt{x^{2} + \sqrt{y}}
 \quad \surd[x^2 + y^2]$
\end{example}

\index{punktid}
\index{vertikaalpunktid}
\index{horisontaalpunktid}

Kuigi \textbf{\wi{korrutamispunkt}} jäetakse tavaliselt
kirjutamata, pannakse ta mõnikord siiski selleks, et aidata silmal
valemit rühmitada. Üksiku tsentreeritud punkti trükib käsk \ci{cdot}.
Tsentreeritud \index{kolm
punkti}\index{punktid}\index{mõttepunktid}\index{...@\ldots}\textbf{kolm
punkti} saab käsuga \ci{cdots} ning
madalal (alusjoonel) asuvad kolm punkti käsuga \ci{ldots}. Lisaks on
olemas käsud \ci{vdots} vertikaalsete ja \ci{ddots} \index{diagonaalsed
punktid}diagonaalsete punktide jaoks. Rohkem näiteid leiab
jaotisest~\ref{sec:arraymat}.
\begin{example}
$\Psi = v_1 \cdot v_2
 \cdot \ldots \qquad
 n! = 1 \cdot 2
 \cdots (n-1) \cdot n$
\end{example}

Käsud \ci{overline} ja \ci{underline} panevad avaldise kohale või alla
\textbf{horisontaaljoone}: \index{horisontaaljoon}
\index{joon!horisontaalne}
\begin{example}
$0{,}\overline{3} =
 \underline{\underline{1/3}}$
\end{example}

Käsud \ci{overbrace} ja \ci{underbrace} joonistavad avaldise kohale või
alla pika \textbf{horisontaalsulu}: \index{horisontaalsulg}
\index{sulg!horisontaalne}
\begin{example}
$\underbrace{\overbrace{a+b+c}^6
 \cdot \overbrace{d+e+f}^7}
 _\text{elu mõte} = 42$
\end{example}

\index{diakriitilised märgid!valemis}Tabelis~\ref{mathacc}
leheküljel~\pageref{mathacc} on loetletud käsud, millega saab muutujate
kohale lisada matemaatilisi lisamärke nagu \textbf{väikesed nooled} või
\textbf{\wi{tilde}}. Laiad katused ja tilded, mis haaravad mitu märki,
moodustatakse käskudega \ci{widehat} ja \ci{widetilde}. Tähele tasub
panna käskude \ci{hat} ja \ci{widehat} erinevust ning käsu \ci{bar}
ülakriipsu asukohta alaindeksiga muutuja puhul.
\index{apostroof}Apostroof \verb|'| annab
\wi{priim}i: % a dash is --
\begin{example}
$f(x) = x^2 \qquad f'(x)
 = 2x \qquad f''(x) = 2\\[5pt]
 \hat{XY} \quad \widehat{XY}
 \quad \bar{x_0} \quad \bar{x}_0$
\end{example}

\textbf{Vektorid}\index{vektor} määratakse sageli väikese
\index{noolemärk}noolemärgi lisamisega muutuja kohale, seda saab teha
käsuga \ci{vec}. Käskudega \ci{overrightarrow} ning \ci{overleftarrow}
saab märkida vektorit punktist $A$ punkti $B$:
\begin{example}
$\vec{a} \qquad
 \vec{AB} \qquad
 \overrightarrow{AB}$
\end{example}

Funktsioonide nimed vormistatakse harilikult püstkirjas, mitte kursiivis
nagu muutujad, seetõttu on \LaTeX is olemas järgmised käsud kõige
sagedasemate funktsiooninimede trükkimiseks: \index{funktsiooninimed}
\begin{center}
\begin{tabular}{llllll}
\ci{arccos} &  \ci{cos}  &  \ci{csc} &  \ci{exp} &  \ci{ker}    & \ci{limsup} \\
\ci{arcsin} &  \ci{cosh} &  \ci{deg} &  \ci{gcd} &  \ci{lg}     & \ci{ln}     \\
\ci{arctan} &  \ci{cot}  &  \ci{det} &  \ci{hom} &  \ci{lim}    & \ci{log}    \\
\ci{arg}    &  \ci{coth} &  \ci{dim} &  \ci{inf} &  \ci{liminf} & \ci{max}    \\
\ci{sinh}   & \ci{sup}   &  \ci{tan}  & \ci{tanh}&  \ci{min}    & \ci{Pr}     \\
\ci{sec}    & \ci{sin} \\
\end{tabular}
\end{center}

\begin{example}
\begin{equation*}
  \lim_{x \rightarrow 0}
  \frac{\sin x}{x}=1
\end{equation*}
\end{example}

Sellest nimekirjast puuduvaid funktsioone saab ise defineerida käsuga
\ci{DeclareMathOperator}. Sellest olemas ka tärniga versioon rajadega
funktsioonide jaoks.\index{käsk!tärniga}\index{tärniga käsk} See käsk
töötab ainult preambulis, nii et alltoodud näite kommenteeritud read
tuleb panna preambulisse:
\begin{example}
%\DeclareMathOperator{\argh}{argh}
%\DeclareMathOperator*{\nut}{Nut}
\begin{equation*}
  3\argh = 2\nut_{x=1}
\end{equation*}
\end{example}

\index{moodulifunktsioon}Moodulifunktsiooni saamiseks on kaks käsku:
\ci{bmod} kahekohalise operaatori $a \bmod b$ jaoks ja \ci{pmod}
avaldiste nagu $x\equiv a \pmod{b}$ jaoks:
\begin{example}
$a\bmod b \\
 x\equiv a \pmod{b}$
\end{example}

Mitmekorruselised \index{murd}\textbf{murrud} luuakse käsuga
\ci{frac}\verb|{...}{...}|.
\index{tekstistiil}\index{esitlusstiil}Tekstistiilis valemites
vormistatakse murd kokkusurutult, et ta reale ära mahuks. Esitlusstiilis
saab seda stiili jäljendada käsuga \ci{tfrac}. Vastupidine, st
esitlusstiilis valem teksti sees, moodustatakse käsuga \ci{dfrac}. Tihti
on eelistatum kaldkriipsuga kuju $1/2$, sest see näeb väikese koguse
"`murrulise materjali"' puhul parem välja:
\begin{example}
Esitlusstiilis:
\begin{equation*}
  3/8 \qquad \frac{3}{8}
  \qquad \tfrac{3}{8}
\end{equation*}
\end{example}

\begin{example}
Tekstistiilis:
$1\frac{1}{2}$~tundi \qquad
$1\dfrac{1}{2}$~tundi
\end{example}

Käsk \ci{partial} annab \wi{osatuletis}e märgi:
\begin{example}
\begin{equation*}
  \sqrt{\frac{x^2}{k+1}}\qquad
  x^\frac{2}{k+1}\qquad
  \frac{\partial^2f}
  {\partial x^2}
\end{equation*}
\end{example}

\index{binoomkordaja}Binoomkordajate või sarnaste struktuuride
trükkimiseks on käsk
\ci{binom} paketist \pai{amsmath}:
\begin{example}
Pascali valem on
\begin{equation*}
 \binom{n}{k} =\binom{n-1}{k}
 + \binom{n-1}{k-1}
\end{equation*}
\end{example}

Kahekohaliste relatsioonide puhul võib teinekord olla vaja paigutada
märke üksteise kohale. Käsk \ci{stackrel}\verb|{#1}{#2}| paneb
argumendis \verb|#1| antud märgi ülaindeksi suurusena argumendi
\verb|#2| kohale, mis ise trükitakse tavalisse asukohta.
\begin{example}
\begin{equation*}
 f_n(x) \stackrel{*}{\approx} 1
\end{equation*}
\end{example}

\textbf{\index{integraaloperaator}Integraaloperaatori} märk
moodustatakse käsuga \ci{int}, \textbf{\wi{summaoperaator}i} märk käsuga
\ci{sum} ja \textbf{\wi{korrutamisoperaator}i} märk käsuga \ci{prod}.
Ülemine ja alumine raja määratakse sümbolitega
\index{^@\texttt{\^{}}}\verb|^|~ja~\index{_@\texttt{\_}}\verb|_| nagu
ülaindeks ja alaindeks:
\begin{example}
\begin{equation*}
\sum_{i=1}^n \qquad
\int_0^{\frac{\pi}{2}} \qquad
\prod_\epsilon
\end{equation*}
\end{example}

Saavutamaks suuremat kontrolli indeksite paigutuse üle keerukamates
avaldistes, on paketis \pai{amsmath} defineeritud käsk \ci{substack}:
\begin{example}
\begin{equation*}
\sum^n_{\substack{0<i<n \\
        j\subseteq i}}
   P(i,j) = Q(i,j)
\end{equation*}
\end{example}

\LaTeX is on saadaval iga sorti märgid \textbf{\index{sulud}sulgude}
ja muude \textbf{\index{piirajad}piirajate} jaoks
(nt~$[\;\langle\;\|\;\updownarrow$). Ümar- ja nurksulge võib lisada
vastavate klahvidega ning looksulge käsuga \verb|\{|, kuid kõik
ülejäänud piirajad genereeritakse spetsiaalsete käskudega
(nt~\verb|\updownarrow|).
\begin{example}
\begin{equation*}
{a,b,c} \neq \{a,b,c\}
\end{equation*}
\end{example}

Kui avava piiraja ette lisada \ci{left} ja sulgeva piiraja ette
\ci{right}, siis valib \LaTeX{} automaatselt piiraja õige suuruse. Iga
\ci{left} jaoks peab olemas olema vastav sulgev \ci{right}. Kui sulgevat
piirajat pole, siis tuleb kasutada nähtamatut sulgejat \ci{right.}:
\begin{example}
\begin{equation*}
1 + \left(\frac{1}{1-x^{2}}
    \right)^3 \qquad
\left. \ddagger \frac{~}{~}\right)
\end{equation*}
\end{example}

Mõnel juhul tuleb seada valemis piiraja õige suurus käsitsi, selleks
võib enamiku piirajate ette lisada käsu \ci{big}, \ci{Big}, \ci{bigg}
või \ci{Bigg}:
\begin{example}
$\Big((x+1)(x-1)\Big)^{2}$\\
$\big( \Big( \bigg( \Bigg( \quad
\big\} \Big\} \bigg\} \Bigg\} \quad
\big\| \Big\| \bigg\| \Bigg\| \quad
\big\Downarrow \Big\Downarrow
\bigg\Downarrow \Bigg\Downarrow$
\end{example}
Kõigi võimalike piirajate nimekiri on toodud tabelis~\ref{tab:delimiters}
leheküljel \pageref{tab:delimiters}.

\section{Liiga pikad üksikvalemid: \ei{multline}}
\index{pikad valemid}\index{valem!pikk}
\label{sec:multline}

Kui valem on liiga pikk, tuleb teda mingit viisi murda. Paraku pole
murtud valemid tüüpiliselt nii lihtsasti loetavad kui murdmata valemid.
Loetavuse parandamiseks on olemas mõned reeglid, kuidas murdmist
sooritada.
\begin{enumerate}
\item Üldiselt tuleks alati murda valemit \emph{enne} võrdusmärki või
  tehtemärki.
\item Murdmine enne võrdusmärki on eelistatum võrreldes murdmisega enne
  ükskõik millist tehtemärki.
\item Murdmine enne pluss- või miinusmärki on eelistatum võrreldes
  murdmisega enne korrutamismärki.
\item Igasugust muud tüüpi murdmist tuleks võimalikult vältida.
\end{enumerate}
Kõige lihtsam viis valemit murda on kasutada keskkonda
\ei{multline}:\footnote{Keskkond \ei{multline} on defineeritud
paketis \pai{amsmath}.}
\begin{example}
\begin{multline}
  a + b + c + d + e + f
  + g + h + i \\
  = j + k + l + m + n
\end{multline}
\end{example}
\noindent
Erinevus keskkonnast \ei{equation} on see, et suvalistesse kohtadesse
saab lisada reavahetusi (või ka mitu reavahetust): panna \ci{\bs}
sinna, kust valemit on murda vaja. Sarnaselt keskkonnale \ei{equation*}
on olemas ka keskkonna variant \ei{multline*} valeminumbri
vältimiseks.\index{käsk!tärniga}\index{tärniga käsk}

Sageli annab paremaid tulemusi keskkond \ei{IEEEeqnarray} (vt
jaotist~\ref{sec:IEEEeqnarray}). Vaatleme järgmist olukorda:
\begin{example}
\begin{equation}
  a = b + c + d + e + f
  + g + h + i + j
  + k + l + m + n + o + p
  \label{eq:liiga_pikk_valem}
\end{equation}
\end{example}
\noindent
Siin on liiga pikk tegelikult võrduse parem pool, mis ei mahu ühele
reale ära. Keskkond \ei{multline} annab järgmise väljundi:
\begin{example}
\begin{multline}
  a = b + c + d + e + f
  + g + h + i + j \\
  + k + l + m + n + o + p
\end{multline}
\end{example}
\noindent See on parem kui \eqref{eq:liiga_pikk_valem}, kuid sellega
kaotab võrdusmärk oma loomuliku suurema tähtsuse tähe $k$ ees
oleva plussmärgi suhtes. Keskkonna \ei{IEEEeqnarray} pakutavat paremat
lahendust vaatleme üksikasjalisemalt järgmises jaotises.

\section{Mitu valemit}
\index{valem!mitu}
\label{sec:IEEEeqnarray}

Kõige üldisemalt on meil hulk valemeid, mis ei mahu ühele reale. Sel
juhul peame tegutsema vertikaaljoondusega, et valemite massiivi struktuur
saaks kena ja loetav.

Enne kui esitame oma ettepanekud, kuidas seda teha, alustame paari halva
näitega, mis demonstreerivad mõne levinud lahenduse suurimaid
puudujääke.

\subsection{Tavapäraste käskude probleemid}
\label{sec:problems_traditional}

Mitu valemit saab kokku võtta keskkonnas \ei{align}:\footnote{Keskkonna
\ei{align} abil võib paigutada ka mitu valemiplokki üksteise
kõrvale. See on veel üks hea keskkonna \ei{IEEEeqnarray} kasutusjuht:
argumendiks panna \texttt{\{rCl+rCl\}}.}
\begin{example}
\begin{align}
  a & = b + c \\
  & = d + e
\end{align}
\end{example}
\noindent See lähenemine ei tööta, kui üks rida on liiga pikk:
\begin{example}
\begin{align}
  a & = b + c \\
  & = d + e + f + g + h + i
  + j + k + l \nonumber \\
  & + m + n + o \\
  & = p + q + r + s
\end{align}
\end{example}
\noindent Siin peaks $+\:m$ asuma $d$ all, mitte võrdusmärgi all.
Loomulikult võib lisada veidi ruumi (\verb+\hspace{...}+), kuid see ei
anna kunagi täpset paigutust (ja on halb stiil \ldots).

Parema lahenduse pakub keskkond \ei{eqnarray}:
\begin{example}
\begin{eqnarray}
  a & = & b + c \\
  & = & d + e + f + g + h + i
  + j + k + l \nonumber \\
  && +\: m + n + o \\
  & = & p + q + r + s
\end{eqnarray}
\end{example}
\noindent Siiski ei ole ka see optimaalne, sest võrdusmärgi ümber on
vahed liiga suured. Seejuures, vahed \emph{ei ole} samad mis
keskkondades \ei{multline} ja \ei{equation}:
\begin{example}
\begin{eqnarray}
  a & = & a = a
\end{eqnarray}
\end{example}
\noindent Lisaks kattub avaldis mõnikord valemi numbriga, kuigi vasakul
oleks piisavalt ruumi:
\begin{example}
\begin{eqnarray}
  a & = & b + c \\
  & = & d + e + f + g + h^2
  + i^2 + j
  \label{eq:viganeeqnarray}
\end{eqnarray}
\end{example}

\noindent Keskkond \ei{eqnarray} tunnistab käsku \ci{lefteqn}, mida võib
kasutada siis, kui vasak pool on liiga pikk:
\begin{example}
\begin{eqnarray}
  \lefteqn{a + b + c + d
    + e + f + g + h}\nonumber \\
  & = & i + j + k + l + m \\
  & = & n + o + p + q + r + s
\end{eqnarray}
\end{example}
\noindent See ei ole samuti optimaalne, sest kui parem pool on liiga
kitsas, siis pole massiiv korralikult tsentreeritud:
\begin{example}
\begin{eqnarray}
  \lefteqn{a + b + c + d
    + e + f + g + h}\nonumber \\
  & = & i + j
\end{eqnarray}
\end{example}

\noindent Olles nüüd konkurente piisavalt maha teinud, võime leebelt
võtta suuna hiilgava lahenduse poole, milleks on \ldots

\subsection{Keskkond \ei{IEEEeqnarray}}
\label{sec:IEEEeqnarray_intro}

Keskkond \ei{IEEEeqnarray} on väga võimas ja paljude suvanditega. Siin
tutvustame ainult peamist funktsionaalsust, lisainfot leiab
manuaalist.\footnote{Ametlik manuaal on väljas aadressil
\CTAN|tex-archive/macros/latex/contrib/IEEEtran/IEEEtran_HOWTO.pdf|.
Keskkonda \ei{IEEEeqnarray} puudutav osa asub lisas~F.}

Keskkonna \ei{IEEEeqnarray} kasutamiseks tuleb dokumendi alguses sisse
lugeda pakett \pai{IEEEtrantools},\footnote{Pakett \pai{IEEEtrantools}
võib installatsioonist puududa, selle leiab CTANist.} lisades dokumendi
päisesse rea
\begin{lscommand}
\verb|\usepackage[retainorgcmds]{IEEEtrantools}|
\end{lscommand}

Keskkonna \ei{IEEEeqnarray} tugevuseks on võimalus määrata valemite
massiivi \emph{veergude} arv. Tavaliselt on spetsifikatsiooniks
\verb+{rCl}+, see tähendab, kolm veergu, esimene paremale joondatud,
keskmine tsentreeritud ja ümbritsevate väikeste vahedega (seepärast
kirjutame suure \texttt{C} väikese \texttt{c} asemel) ning kolmas
vasakule joondatud:
\begin{example}
\begin{IEEEeqnarray}{rCl}
  a & = & b + c \\
  & = & d + e + f + g + h
  + i + j + k \nonumber\\
  && \negmedspace {} + l + m
  + n + o \\
  & = & p + q + r + s
\end{IEEEeqnarray}
\end{example}
Veerge võib määrata ükskõik kui palju: \verb+{c}+ annab ainult ühe
veeru, kus kõik kirjed on tsentreeritud, \verb+{rCll}+ lisab neljanda,
vasakule joondatud veeru näiteks kommentaaride jaoks. Peale \texttt{l},
\texttt{c}, \texttt{r}, \texttt{L}, \texttt{C}, \texttt{R}
valemire\v{z}iimis kirjete joondamiseks on olemas ka \texttt{s},
\texttt{t}, \texttt{u} vasakule, keskele ja paremale joondatud
tekstire\v{z}iimis kirjete jaoks. Spetsifikaatoritega \texttt{.} ja
\texttt{/} ja \texttt{?} saab jätta lisavahesid kasvavas
järjekorras.\footnote{Vahede tüüpidest on veel juttu
jaotises~\ref{sec:putting-qed-right}.} Tähele tasub panna vahesid
võrdusmärgi ümber, vastandina keskkonna \ei{eqnarray} lisatavatele
vahedele.

\subsection{Tavakasutus}
\label{sec:common-usage}

Järgnevalt kirjeldame, kuidas lahendada levinumaid probleeme keskkonna
\ei{IEEEeqnarray} abil.

Kui rida kattub valemi numbriga nagu valemis \eqref{eq:viganeeqnarray},
siis aitab käsk \ci{IEEEeqnarraynumspace}. See käsk tuleb panna
vastavasse ritta ja ta nihutab tervet valemimassiivi valemi numbri
laiuse võrra vasakule (nihe sõltub numbri suurusest!). Seega
\begin{example}
\begin{IEEEeqnarray}{rCl}
  a & = & b + c \\
  & = & d + e + f + g + h
  + i + j + k \\
  & = & l + m + n
\end{IEEEeqnarray}
\end{example}
\noindent asemel saame
\begin{example}
\begin{IEEEeqnarray}{rCl}
  a & = & b + c \\
  & = & d + e + f + g + h
  + i + j + k
  \IEEEeqnarraynumspace \\
  & = & l + m + n.
\end{IEEEeqnarray}
\end{example}

Kui vasak pool on liiga pikk, pakub \ei{IEEEeqnarray} käsu
\ci{lefteqn} asemel käsku \ci{IEEEeqnarraymulticol}, mis töötab igas
olukorras:
\begin{example}
\begin{IEEEeqnarray}{rCl}
  \IEEEeqnarraymulticol{3}{l}{
    a + b + c + d + e + f + g + h
  }\nonumber\\
  \quad & = & i + j \\
  & = & k + l + m
\end{IEEEeqnarray}
\end{example}
\noindent Kasutamine sarnaneb keskkonna \ei{tabular} käsuga
\ci{multicolumns}. Esimene argument \verb+{3}+ määrab, et kolm veergu
ühendatakse üheks veeruks, teine argument \verb+{l}+ aga ütleb, et
saadud veerg joondatakse vasakule.

Käskude \ci{quad} ja \ci{qquad} lisamisega saab lihtsasti sättida
võrdusmärkide sügavust,\footnote{Kaugus üks \ci{quad} paistab hea
enamikul juhtudel.} nt
\begin{example}
\begin{IEEEeqnarray}{rCl}
  \IEEEeqnarraymulticol{3}{l}{
    a + b + c + d + e + f + g + h
  }\nonumber\\
  \qquad\qquad & = & i + j \\
  & = & k + l + m
\end{IEEEeqnarray}
\end{example}

Kui valem on jaotatud kahele või enamale reale, siis interpreteerib
\LaTeX{} esimest märki $+$ või $-$ liikme märgina, mitte kahekohalise
tehte tähisena. Seetõttu on vaja tehtemärgi ja liikme vahele lisaruumi:
seega
\begin{example}
\begin{IEEEeqnarray}{rCl}
  a & = & b + c \\
  & = & d + e + f + g + h
  + i + j + k \nonumber\\
  && + l + m + n + o \\
  & = & p + q + r + s
\end{IEEEeqnarray}
\end{example}
\noindent asemel peaksime kirjutama
\begin{example}
\begin{IEEEeqnarray}{rCl}
  a & = & b + c \\
  & = & d + e + f + g + h
  + i + j + k \nonumber\\
  && \negmedspace {} + l + m
  + n + o \\
  & = & p + q + r + s
\end{IEEEeqnarray}
\end{example}
\noindent Tasub tähele panna ruumi $+$ ja $l$ vahel!

Konstruktsioon \verb|{} + l| määrab, et \verb|+| on siin kahekohaline
tehtemärk, mitte lihtsalt arvu märk, ning sellest tuleneva soovimatu
tühiku \verb|{}| ja \verb|+| vahel kompenseerib negatiivne keskmise
pikkusega hüpe \ci{negmedspace}.

Valemi numbri saab rea lõppu trükkimata jätta käsuga \ci{nonumber} (või
\ci{IEEEnonumber}). Kui sellises reas on defineeritud märgend
\verb+\label{...}+, siis antakse see edasi järgmisele valeminumbrile,
mida pole ära keelatud. Märgend tuleks panna otse reavahetuse \verb+\\+
ette või valemi järele, mille juurde number kuulub. Lisaks algteksti
loetavuse parandamisele väldib see kompileerimisviga, kui käsk
\ci{IEEEmulticol} järgneb märgendi definitsioonile.

Keskkonnast on olemas ka tärniga versioon, kus kõik valeminumbrid on
keelatud.\index{keskkond!tärniga}\index{tärniga keskkond}
Sel juhul võib valeminumbril ilmuda lasta käsuga
\ci{IEEEyesnumber}:
\begin{example}
\begin{IEEEeqnarray*}{rCl}
  a & = & b + c \\
  & = & d + e \IEEEyesnumber\\
  & = & f + g
\end{IEEEeqnarray*}
\end{example}

Käsuga \ci{IEEEyessubnumber} saab lihtsasti moodustada alamnumbreid:
\begin{example}
\begin{IEEEeqnarray}{rCl}
  a & = & b + c
  \IEEEyessubnumber\\
  & = & d + e
  \nonumber\\
  & = & f + g
  \IEEEyessubnumber
\end{IEEEeqnarray}
\end{example}

\section{Massiivid ja maatriksid} \label{sec:arraymat}

\textbf{Massiivide} trükkimiseks on keskkond \ei{array}, mis töötab
sarnaselt keskkonnaga \ei{tabular}. Ridu murtakse käsu \ci{\bs}
abil:
\begin{example}
  \begin{equation*}
    \mathbf{X} = \left(
      \begin{array}{ccc}
        x_1 & x_2 & \ldots \\
        x_3 & x_4 & \ldots \\
        \vdots & \vdots & \ddots
      \end{array} \right)
  \end{equation*}
\end{example}

Keskkonnaga \ei{array} saab vormistada ka \index{harudega
funktsioonid}harudega funktsioone, lisades paremaks \ci{right} piirajaks
nähtamatu \verb|.|\,, nagu näiteks
\begin{example}
\begin{equation*}
  |x| = \left\{
    \begin{array}{rl}
      -x, & \text{kui } x < 0,\\
      0, & \text{kui } x = 0,\\
      x, & \text{kui } x > 0.
    \end{array} \right.
\end{equation*}
\end{example}
\noindent
Tähelepanu väärib ka keskkond \ei{cases} lihtsama süntaksi tõttu:
\begin{example}
  \begin{equation*}
    |x| =
    \begin{cases}
      -x, & \text{kui } x < 0,\\
      0, & \text{kui } x = 0,\\
      x, & \text{kui } x > 0.
    \end{cases}
\end{equation*}
\end{example}

Keskkonnaga \ei{array} saab moodustada maatrikseid\index{maatriks}, kuid
parema lahenduse pakub \pai{amsmath} oma erinevate
maatriksikeskkondadega. Neid on olemas kuus varianti eri piirajatega:
\ei{matrix} (pole), \ei{pmatrix}~$($\,, \ei{bmatrix}~$[$\,,
\ei{Bmatrix}~$\{$\,, \ei{vmatrix}~$\vert$ ja \ei{Vmatrix}~$\Vert$.
Erinevalt keskkonnast \ei{array} ei ole vaja määrata veergude arvu.
Maksimaalne veergude arv on 10, kuid see on seadistatav (kuigi 10 veergu
just väga tihti vaja ei lähe!):
\begin{example}
\begin{equation*}
  \begin{matrix}
    1 & 2 \\
    3 & 4
  \end{matrix} \qquad
  \begin{bmatrix}
    p_{11} & p_{12} & \ldots
    & p_{1n} \\
    p_{21} & p_{22} & \ldots
    & p_{2n} \\
    \vdots & \vdots & \ddots
    & \vdots \\
    p_{m1} & p_{m2} & \ldots
    & p_{mn}
  \end{bmatrix}
\end{equation*}
\end{example}

\section{Vahed valemire\v{z}iimis} \label{sec:math-spacing}

\index{vahe!valemis}\index{tühik!valemis} Kui \LaTeX i automaatselt
valitud vahed valemi sees ei sobi, siis saab neid ise sättida
spetsiaalseid vahekäske lisades: \ci{,} pikkusega
$\frac{3}{18}\:\textrm{em}$ (\demowidth{0.166em}), \ci{:} pikkusega
$\frac{4}{18}\: \textrm{em}$ (\demowidth{0.222em}) ja \ci{;} pikkusega
$\frac{5}{18}\: \textrm{em}$ (\demowidth{0.277em}). Langjoonega tühik
\verb*|\ | moodustab keskmise, sõnavahedega võrreldava pikkusega vahe
ning \ci{quad} (\demowidth{1em}) ja \ci{qquad} (\demowidth{2em})
moodustavad pikemad vahed. Vahe \ci{quad} suurus sõltub tähe M laiusest
kehtivas kirjas. Käsk \verb|\!|\cih{"!} moodustab negatiivse vahe
pikkusega $-\frac{3}{18}\:\textrm{em}$ ($-$\demowidth{0.166em}).

\begin{example}
\begin{equation*}
  \int_1^2 \ln x \mathrm{d}x
  \qquad
  \int_1^2 \ln x \,\mathrm{d}x
\end{equation*}
\end{example}

Diferentsiaali täht d tuleks standardi kohaselt kirjutada püstkirjas.
Järgmises näites defineerime uue käsu \ci{ud} (püstine d), mis annab
tulemuseks "`$\,\mathrm{d}$"' (enne tähte $\text{d}$ on vahe
\demowidth{0.166em}), nii et meil pole iga kord vaja vahe üle muret
tunda. Käsk \verb|\newcommand| pannakse preambulisse.

%  More on
% \ci{newcommand} in section~\ref{} on page \pageref{}. To Do: Add label and
% reference to "Customising LaTeX" -> "New Commands, Environments and Packages"
% -> "New Commands".
\begin{example}
\newcommand{\ud}{\,\mathrm{d}}

\begin{equation*}
 \int_a^b f(x)\ud x
\end{equation*}
\end{example}

Mitmekordsete integraalide trükkimisel ilmneb, et vahed
integraalimärkide vahel on liiga laiad. Neid võib korrigeerida käsuga
\ci{"!}, kuid paketis \pai{amsmath} on olemas lihtsam võimalus vahede
sättimiseks, nimelt käsud \ci{iint}, \ci{iiint}, \ci{iiiint} ja
\ci{idotsint}.

\begin{example}
\newcommand{\ud}{\,\mathrm{d}}

\begin{IEEEeqnarray*}{c}
  \int\int f(x)g(y)
                  \ud x \ud y \\
  \int\!\!\!\int
         f(x)g(y) \ud x \ud y \\
  \iint f(x)g(y)  \ud x \ud y
\end{IEEEeqnarray*}
\end{example}

Üksikasju võib vaadata elektroonilisest dokumendist
\texttt{testmath.tex} (tuleb kaasa \AmS-\LaTeX iga) või raamatu
\companion{} peatükist~8.

\subsection{Fantoomid}

Joondades käskudega \index{^@\texttt{\^{}}}\verb|^| ja
\index{_@\texttt{\_}}\verb|_| teksti vertikaalselt, on \LaTeX{} mõnikord
veidi liiga abivalmis. Käsuga \ci{phantom} saab reserveerida ruumi
märkide jaoks, mis lõppväljundis ei ilmu. Kõige lihtsam võimalus sellest
aru saada on vaadata näidet:
\begin{example}
\begin{equation*}
{}^{14}_{6}\text{C}
\qquad \text{võrreldes} \qquad
{}^{14}_{\phantom{1}6}\text{C}
\end{equation*}
\end{example}
Kui on tarvis trükkida palju selliseid isotoope nagu näites, siis on
pakett \pai{mhchem} väga kasulik nii isotoopide kui ka keemiliste
valmite vormistamiseks.

\section{Valemikirjade sättimine}\label{sec:fontsz}

Mitmesuguseid valemikirju on loetletud tabelis~\ref{mathalpha}
leheküljel \pageref{mathalpha}.
\begin{example}
 $\Re \qquad
  \mathcal{R} \qquad
  \mathfrak{R} \qquad
  \mathbb{R} \qquad $
\end{example}
\noindent Viimased kaks nõuavad paketti \pai{amssymb} või \pai{amsfonts}.

Mõnikord on vaja \LaTeX ile teada anda õige
kirjasuurus.\index{kirjasuurus} Valemire\v{z}iimis saab seda teha
järgmise nelja käsuga:
\begin{flushleft}
\ci{displaystyle}~($\displaystyle 123$),
\ci{textstyle}~($\textstyle 123$),
\ci{scriptstyle}~($\scriptstyle 123$) ja
\ci{scriptscriptstyle}~($\scriptscriptstyle 123$).
\end{flushleft}

Kui murru koosseisus esineb $\sum$, siis trükitakse see tekstistiilis,
välja arvatud juhul, kui nõuda \LaTeX ilt vastupidist:
\begin{example}
\begin{equation*}
 P = \frac{\displaystyle {
   \sum_{i=1}^n (x_i-x)
   (y_i-y)}}
   {\displaystyle{\left[
   \sum_{i=1}^n(x_i-x)^2
   \sum_{i=1}^n(y_i-y)^2
   \right]^{1/2}}}
\end{equation*}
\end{example}
\noindent Üldiselt mõjutab stiilide muutmine suurte operaatorite ja
rajade kujutamist.

% This is not a math accent, and no maths book would be set this way.
% mathop gets the spacing right.

\subsection{Paksud sümbolid}
\index{paksud sümbolid}

\LaTeX is on päris raske saada valemitesse pakse sümboleid; see on
arvatavasti meelega nii tehtud, sest amatöörkujundajad kipuvad neid üle
kasutama. Kirjamuutmise käsk \ci{mathbf} küll annab paksud tähed,
kuid need on püstkirjas, samas kui matemaatilised sümbolid on tavaliselt
kursiivis, ja samuti ei tööta see väikeste kreeka tähtedega. On olemas
käsk \ci{boldmath}, kuid seda saab kasutada ainult väljaspool
valemire\v{z}iimi. Siiski töötab see ka sümbolitega:
\begin{example}
$\mu, M \qquad
\mathbf{\mu}, \mathbf{M}$
\qquad \boldmath{$\mu, M$}
\end{example}

Pakett \pai{amsbsy} (sisaldub \pai{amsmath}is) ning ka pakett \pai{bm}
komplektist Tools teevad selle palju lihtsamaks, sest defineerivad käsu
\ci{boldsymbol}:
\begin{example}
$\mu, M \qquad
\boldsymbol{\mu}, \boldsymbol{M}$
\end{example}

\section{Teoreemid, lemmad, \ldots}

Matemaatilisi dokumente kirjutades on arvatavasti vaja vormistada
teoreeme, lemmasid, definitsioone, aksioome ja teisi analoogilisi
struktuure.
\begin{lscommand}
\ci{newtheorem}\verb|{|\emph{nimi}\verb|}[|\emph{loendur}\verb|]{|%
         \emph{tiitel}\verb|}[|\emph{jaotis}\verb|]|
\end{lscommand}
\noindent Argument \emph{nimi} on lühike võtmesõna, mille järgi
teoreemi identifitseeritakse. Argumendiga \emph{tiitel} määratakse
teoreemi tegelik nimi, nagu see trükitakse lõppdokumendis.

Nurksulgudes argumendid on valikulised. Mõlemad neist määravad
teoreemiga seotud nummerduse. Argument \emph{loendur} on varem
deklareeritud teoreemi \emph{nimi}. Uus teoreem nummerdatakse
siis samas jadas. Argument \emph{jaotis} on liigendusüksus, mille
piirides teoreemi number muutub.\pagebreak[3]

Pärast dokumendi päises käsu \ci{newtheorem} täitmist võib kirjutada:
\begin{code}
\verb|\begin{|\emph{nimi}\verb|}[|\emph{lisatekst}\verb|]|\\
See on minu põnev teoreem.\\
\verb|\end{|\emph{nimi}\verb|}|
\end{code}

Paketi \pai{amsthm} (kuulub \AmS-\LaTeX i) käsk
\ci{theoremstyle}\verb|{|\emph{stiil}\verb|}| võimaldab ette anda, mille
kohta teoreem käib, valides ühe kolmest eeldefineeritud stiilist:
\texttt{definition} (paks tiitel, püstkirjas sisu), \texttt{plain} (paks
tiitel, kursiivis sisu) või \texttt{remark} (kursiivis tiitel,
püstkirjas sisu).

Nüüd peaks teooriat olema piisavalt. Järgmised näited peaksid hajutama
viimasegi kahtluse ja tegema selgeks, et käsk \ci{newtheorem} on
arusaamiseks kaugelt liiga keeruline.

% actually define things
\theoremstyle{definition} \newtheorem{seadus}{Seadus}
\theoremstyle{plain}      \newtheorem{kohus}[seadus]{Kohus}
\theoremstyle{remark}     \newtheorem*{marg}{Margaret}

Kõigepealt defineerime teoreemid:

\begin{verbatim}
\theoremstyle{definition} \newtheorem{seadus}{Seadus}
\theoremstyle{plain}      \newtheorem{kohus}[seadus]{Kohus}
\theoremstyle{remark}     \newtheorem*{marg}{Margaret}
\end{verbatim}

\begin{example}
\begin{seadus} \label{seadus:pink}
Ära peida end kohtupingis!
\end{seadus}
\begin{kohus}[Kaksteist]
See võid olla sina! Vaata ette
ja loe seadust~\ref{seadus:pink}.
\end{kohus}
\begin{kohus}
Sa ignoreerid viimast ütlust.
\end{kohus}
\begin{marg}Ei, ei, ei\end{marg}
\begin{marg}Denis!\end{marg}
\end{example}

Teoreem \texttt{kohus} kasutab sama loendurit nagu teoreem
\texttt{seadus}, seetõttu nummerdatakse teda samas jadas teiste
"`seadustega"'. Nurksulgudes argumendiga määratakse teoreemi nimetus
vms.
\begin{example}
\newtheorem{mur}{Murphy}[section]

\begin{mur}
Kui millegi tegemiseks on kaks
viisi ja üks viis võib viia
katastroofini, siis keegi
kindlasti seda viisi kasutab.
\end{mur}
\end{example}

Teoreem \texttt{Murphy} saab numbri, mis on seotud jooksva jaotisega.
Liigendusüksus võib olla ka midagi muud, näiteks peatükk või alajaotis.

Kes soovib oma teoreeme seadistada viimase punktini, sellele annab
rohkelt võimalusi pakett \pai{ntheorem}.

\subsection{Tõestused ja tõestuse lõpumärk}
\label{sec:putting-qed-right}\index{tõestuse lõpumärk}

Pakett \pai{amsthm} defineerib ka keskkonna \ei{proof}.

\begin{example}
\begin{proof}
Triviaalne, kasuta valemit
 \begin{equation*}
   E=mc^2.
 \end{equation*}
\end{proof}
\end{example}

Käsuga \ci{qedhere} saab viia tõestuse lõpumärgi mujale olukorras,
kus see muidu jääks reale üksikuna.

\begin{example}
\begin{proof}
Triviaalne, kasuta valemit
 \begin{equation*}
   E=mc^2. \qedhere
 \end{equation*}
\end{proof}
\end{example}

Kahjuks ei tööta see keskkonnaga \ei{IEEEeqnarray}:
\begin{example}
\begin{proof}
  See on tõestus, mis lõpeb
  valemite massiiviga:
  \begin{IEEEeqnarray*}{rCl}
    a & = & b + c \\
    & = & d + e. \qedhere
  \end{IEEEeqnarray*}
\end{proof}
\end{example}
\noindent
Selle põhjuseks on \ei{IEEEeqnarray} siseehitus: massiivist
kummalegi poole lisatakse alati kaks nähtamatut veergu, mis
sisaldavad ainult venivat ruumi. Nii kindlustab \ei{IEEEeqnarray}, et
valemite massiiv on horisontaalselt joondatud keskele. Käsk \ci{qedhere}
tuleks panna sellest venivast ruumist \emph{väljapoole}, kuid seda ei
saa, sest need veerud on kasutajale
nähtamatud.

\enlargethispage{\baselineskip}%
Leidub aga väga lihtne väljapääs: veniva ruumi võib ette anda otse!
\begin{example}
\begin{proof}
  See on tõestus, mis lõpeb
  valemite massiiviga:
  \begin{IEEEeqnarray*}{+rCl+x*}
    a & = & b + c \\
    & = & d + e. & \qedhere
  \end{IEEEeqnarray*}
\end{proof}
\end{example}
\noindent
Argumendis \verb={+rCl+x*}= tähistab \verb=+= venivat ruumi, üks
valemist vasakul (mille \ei{IEEEeqnarray} paneb automaatselt, kui
seda pole määratud!) ja teine paremal. Kuid nüüd lisame paremale,
\emph{pärast} venivat veergu tühja veeru \verb=x=. Seda veergu on vaja
ainult viimases reas, kui sinna pannakse käsk \ci{qedhere}.
Lõppu kirjutame veel \verb=*=\,, mis määrab null-ruumi, et
\ei{IEEEeqnarray} ise teist soovimatut \verb=+=-ruumi ei lisaks.

Valemite nummerdamisel esineb sarnane probleem. Kui võrdleme
\begin{example}
\begin{proof}
  See on tõestus, mis lõpeb
  nummerdatud valemiga:
  \begin{equation}
    a = b + c.
  \end{equation}
\end{proof}
\end{example}
\noindent ja
\begin{example}
\begin{proof}
  See on tõestus, mis lõpeb
  nummerdatud valemiga:
  \begin{equation}
    a = b + c. \qedhere
  \end{equation}
\end{proof}
\end{example}
\noindent
siis näeme, et teises (õiges) variandis asub $\Box$ valemile palju
lähemal kui esimeses variandis.

Analoogiliselt, õige viis panna tõestuse lõpumärk nummerdatud valemite
massiivi lõppu on järgmine:
\begin{example}
\begin{proof}
  See on tõestus, mis lõpeb
  valemite massiiviga:
  \begin{IEEEeqnarray}{+rCl+x*}
    a & = & b + c \\
    & = & d + e. \\
    &&& \qedhere\nonumber
  \end{IEEEeqnarray}
\end{proof}
\end{example}
\noindent vastandina variandile
\begin{example}
\begin{proof}
  See on tõestus, mis lõpeb
  valemite massiiviga:
  \begin{IEEEeqnarray}{rCl}
    a & = & b + c \\
    & = & d + e.
  \end{IEEEeqnarray}
\end{proof}
\end{example}


%

% Local Variables:
% TeX-master: "lshort"
% mode: latex
% mode: flyspell
% End:

