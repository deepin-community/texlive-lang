%%%%%%%%%%%%%%%%%%%%%%%%%%%%%%%%%%%%%%%%%%%%%%%%%%%%%%%%%%%%%%%%%
% Contents: Who contributed to this Document
% $Id: overview.tex 456 2011-04-06 09:10:27Z oetiker $
%%%%%%%%%%%%%%%%%%%%%%%%%%%%%%%%%%%%%%%%%%%%%%%%%%%%%%%%%%%%%%%%%

% Because this introduction is the reader's first impression, I have
% edited very heavily to try to clarify and economize the language.
% I hope you do not mind! I always try to ask "is this word needed?"
% in my own writing but I don't want to impose my style on you...
% but here I think it may be more important than the rest of the book.
% --baron

\chapter{Eessõna}

\LaTeX{} \cite{manual} on küljendussüsteem, mis sobib väga hästi
tüpograafiliselt kõrge kvaliteediga teaduslike ja matemaatiliste
dokumentide loomiseks. Kuid ta sobib ka igasuguste muude tekstide
vormistamiseks, lihtsatest kirjadest täiemahuliste raamatuteni.
Trükiladumiseks kasutab \LaTeX{} programmi~\TeX{}~\cite{texbook}.

Käesolev lühike sissejuhatus kirjeldab süsteemi \LaTeXe{} ja peaks olema
piisav enamiku \LaTeX i-rakenduste jaoks. Täieliku ülevaate \LaTeX ist
võib leida raamatutest~\cite{manual,companion}.

\bigskip
\noindent See sissejuhatus jaguneb 6 peatükiks.
\begin{description}
\item[1. peatükk] kirjeldab \LaTeXe{} dokumentide põhistruktuuri, samuti
  puudutab veidi \LaTeX i ajalugu. Selle peatüki läbilugemisel peaks
  tekkima üldine ettekujutus, kuidas \LaTeX{} töötab.
\item[2. peatükk] süveneb dokumentide küljendamise üksikasjadesse ning
  tutvustab enamikku olulisemaid \LaTeX i käske ja keskkondi. Pärast
  selle peatüki lugemist saab hakata koostama esimesi dokumente.
\item[3. peatükk] selgitab, kuidas panna \LaTeX is kirja valemeid,
  illustreerides seda \LaTeX i ühte tugevaimat külge paljude näidetega.
  Peatüki lõpus asuvad tabelid, kuhu on koondatud kõik \LaTeX is
  kättesaadavad matemaatilised sümbolid.
\item[4. peatükk] tutvustab aineregistreid, kirjandusnimestiku
  genereerimist ja EPS-graafika lisamist. Siin käsitletakse ka
  PDF-dokumentide loomist pdf\LaTeX iga ning tuuakse välja mõned
  kasulikud lisapaketid.
\item[5. peatükk] näitab, kuidas \LaTeX iga luua graafikat. Selle
  asemel, et joonistada mõne joonistusprogrammiga pilt, salvestada see
  faili ja lisada dokumendile, võib \LaTeX ile ette anda pildi
  kirjelduse ja lasta tal endal selle järgi pilt valmis joonistada.
\item[6. peatükk] sisaldab veidi ohtlikuvõitu informatsiooni
  selle kohta, kuidas \LaTeX i standardset dokumendikujundust muuta.
  Siin selgitatakse, kuidas korraldada asju ümber nii, et
  \LaTeX i kaunis väljund muutuks koledaks või imeliseks, vastavalt
  kujundaja oskustele.
\end{description}
\bigskip
\noindent
Oluline on lugeda peatükke just selles järjekorras -- nii mahukas see
raamat ka pole. Hoolikalt tuleks läbi lugeda näited, sest palju
informatsiooni on koondatud raamatus leiduvatesse näidetesse.

\bigskip
\noindent \LaTeX{} on saadaval enamiku arvutite jaoks PC-st ja Macist
suurte UNIXi ja VMSi süsteemideni. Paljude ülikoolide arvutivõrkudes on
\LaTeX{} juba installitud ja kasutamiseks valmis. Juhiseid kohalikule
\LaTeX i-installatsioonile juurdepääsemise kohta annab \guide. Kui tekib
probleeme alustamisega, siis tasub küsida abi inimeselt, kes selle
raamatu andis. Käesoleva juhendi eesmärk \emph{ei ole} selgitada,
kuidas \LaTeX i installida ja üles seada, vaid õpetada, kuidas
kirjutada dokumente nii, et \LaTeX{} oskaks neid töödelda.

\bigskip
\noindent Ükskõik millise \LaTeX iga seotud materjali leidmiseks võib
esimesena vaadata mõnda CTANi (Comprehensive \TeX{} Archive
Network) saiti. CTANi kodulehekülg on \url{http://www.ctan.org}.

Siin raamatus leidub muidki viiteid CTANile, eeskätt allalaaditavatele
programmidele ja dokumentidele. Täieliku URLi asemel on nendes
aadressiks lihtsalt \texttt{CTAN:} koos järgneva asukohaga CTANi puus,
kuhu tuleks minna.

Oma arvutis \LaTeX i töölepanemiseks leiab materjali kataloogist
\CTAN|tex-archive/systems|.

\vspace{\stretch{1}}
\noindent Kui selle dokumendi kohta tekib mõtteid, st mida võiks lisada,
kustutada või muuta, siis palun need lahkesti mulle saata. Iseäranis
olen huvitatud tagasisidest algajatelt \LaTeX i-kasutajatelt selle
kohta, millised osad olid siin sissejuhatuses kergesti mõistetavad ja
mis võiks olla selgitatud paremini.

\bigskip
\begin{verse}
\contrib{Tobias Oetiker}{tobi@oetiker.ch}%
\noindent{OETIKER+PARTNER AG\\Aarweg 15\\4600 Olten\\\v{S}veits}
\end{verse}
\vspace{\stretch{1}}
\noindent Käesoleva dokumendi viimane versioon asub aadressil
\CTAN|tex-archive/info/lshort|.

\endinput



%

% Local Variables:
% TeX-master: "lshort2e"
% mode: latex
% mode: flyspell
% End:
