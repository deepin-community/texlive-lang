%%%%%%%%%%%%%%%%%%%%%%%%%%%%%%%%%%%%%%%%%%%%%%%%%%%%%%%%%%%%%%%%%
% Contents: Typesetting Part of LaTeX2e Introduction
% $Id: typeset.tex 537 2015-07-18 09:43:10Z oetiker $
%%%%%%%%%%%%%%%%%%%%%%%%%%%%%%%%%%%%%%%%%%%%%%%%%%%%%%%%%%%%%%%%%
\chapter{Teksti vormistamine}

\begin{intro}
  Eelmises peatükis tutvustasime \LaTeXe{} dokumentide põhilisi
  koostisosi.
  Selles peatükis täidame ülejäänud struktuuri, mida on
  tegelike materjalide koostamiseks vaja teada.
\end{intro}

\section{Teksti ja keele struktuur}
\secby{Hanspeter Schmid}{hanspi@schmid-werren.ch}
Teksti kirjutamise peamine mõte (osa modernset EIHE\footnote{Erinev iga
hinna eest, tõlge \v{S}veitsi saksakeelsest väljendist UVA (\emph{ums
Verrecken anders}).} kirjandust välja arvatud) on anda lugejale edasi
ideid, informatsiooni või teadmisi. Lugeja mõistab teksti paremini, kui
ideed on hästi struktureeritud, ning näeb ja tajub teose loogilist ja
semantilist ülesehitust palju selgemini, kui teose tüpograafiline vorm
seda peegeldab.

\LaTeX{} erineb muudest tekstivormistussüsteemidest selle poolest, et
talle tuleb ära kirjeldada teksti loogiline ja semantiline struktuur.
Seejärel tuletab \LaTeX{} dokumendiklassis ja mitmesugustes
stiilifailides paikapandud "`reeglite"' järgi ise teksti tüpograafilise
vormi.

Kõige tähtsam tekstiüksus \LaTeX is (ja üldse tüpograafias) on
\wi{lõik}. Me nimetame seda tekstiüksuseks, sest lõik on tüpograafiline
kogum, mis peaks väljendama ühte sidusat mõtet või ideed. Järgmistes
jaotistes õpetatakse, kuidas alustada uut rida, nt käsuga
\texttt{\bs\bs}, või lõiku, nt jättes algteksti tühja rea. Kui algab uus
mõte, siis tuleks alustada uut lõiku, ja kui mitte, siis piirduda ainult
ridade murdmisega. Lõiguvahe lisamise juures kahevahel olles võib
mõtelda tekstist kui ideede ja mõtete edasiandjast. Kui tekstis on
lõiguvahe, ent pärast seda vana mõte jätkub, siis tuleks lõiguvahe
eemaldada. Kui samas lõigus tuleb sisse täiesti uus mõttekäik, siis
tuleks lõiguvahe lisada.

Hästi paigutatud lõiguvahede tähtsust tihti alahinnatakse. Paljud ei
teagi, mida lõiguvahe tähendab, või, iseäranis \LaTeX is, jätavad sisse
lõiguvahesid ilma seda ise aimamata. Viimast viga on eriti lihtne teha
siis, kui tekstis esineb valemeid. Järgmisi näiteid vaadates tasub
mõtelda, miks mõnikord on enne või pärast valemit tühjad read
(lõiguvahed), mõnikord aga mitte. (Kui mõned käsud on siin veel
tundmatud, siis võib läbi lugeda selle ja järgmise peatüki ning seejärel
pöörduda käesoleva jaotise juurde tagasi.)

\begin{code}
\begin{verbatim}
% 1. näide
\ldots kui Einstein tõi sisse valemi
\begin{equation}
  E = m \cdot c^2 \; ,
\end{equation}
mis on kõige laiemalt tuntud ja samas
kõige vähem mõistetud füüsikavalem.
\end{verbatim}
\end{code}

\begin{code}
\begin{verbatim}
% 2. näide
\ldots kust järeldub Kirchhoffi voolutugevuste seadus
\begin{equation}
  \sum_{k=1}^{n} I_k = 0 \; .
\end{equation}

Kirchhoffi pingelangude seaduse võib tuletada \ldots
\end{verbatim}
\end{code}

\begin{code}
\begin{verbatim}
% 3. näide
\ldots millel on mitu eelist.

\begin{equation}
  I_D = I_F - I_R
\end{equation}
on aluseks hoopis teist laadi transistorimudelile. \ldots
\end{verbatim}
\end{code}

Väiksuse suunas järgmine tekstiüksus on lause. Ingliskeelsetes tekstides
pannakse lauset lõpetava punkti järele pikem tühik kui lühendit lõpetava
punkti järele. \LaTeX{} püüab ise aru saada, kumba on vaja. Kui \LaTeX{}
mõistab seda valesti, siis tuleb talle oma soovi selgitada; sellest on
juttu käesolevas peatükis edaspidi.

Teksti struktureerimine ulatub isegi lause osadele. Enamikus keeltes on
kirjavahemärkide reeglid väga keerulised, kuid paljudes keeltes
(sealhulgas inglise ja saksa keeles) saab peaaegu iga koma õigesti, kui
meenutada, mida see esitab: lühikest pausi keelevoos. Olles kahevahel,
kuhu koma panna, võib lauset lugeda valjusti ja teha pärast iga koma
väike hingetõmme. Kui mõni koht tundub selliselt kohmakas, siis
kustutada koma; kui mõnes teises kohas tekib vajadus hingata (või teha
lühike paus), siis lisada koma.\enlargethispage{\baselineskip}

Lõpuks peaksid tekstilõigud olema loogiliselt struktureeritud ka
kõrgemal tasemel, korrastatud peatükkideks, jaotisteks, alajaotisteks
jne. Kuid kirjutiste nagu \verb|\section{|\texttt{Keele ja teksti
struktuur}\verb|}| tüpograafiline efekt on nii ilmne, et on peaaegu
iseenesestmõistetav, kuidas selliseid kõrgema taseme struktuure tuleks
kasutada.

\section{Ridade murdmine ja lehekülgedeks jaotamine}

\subsection{Joondatud lõigud}

Raamatuid laotakse tihti nii, et kõik read on sama pikad. \LaTeX{} lisab
sõnade vahele vajalikud \index{rea murdmine}reamurdmised ja vahed,
optimeerides korraga terve lõigu sisu.\index{lõik} Kui tarvis, siis ta
isegi poolitab sõnu, mis sobivalt ühele reale ei mahu. See, kuidas lõike
laotakse, sõltub dokumendiklassist. Tavaliselt algab lõigu esimene rida
taandega ja kahe lõigu vahel pole lisaruumi. Täpsemat infot leiab
jaotisest~\ref{parsp}.

Erijuhtudel võib olla vaja anda \LaTeX ile ise reamurdmise käsk.
\begin{lscommand}
\ci{\bs} või \ci{newline}
\end{lscommand}
\noindent alustab uut rida ilma uut lõiku alustamata.
\begin{lscommand}
\ci{\bs*}
\end{lscommand}
\noindent keelab lisaks leheküljevahetuse pärast rea sundmurdmist.
\begin{lscommand}
\ci{newpage}
\end{lscommand}
\noindent alustab uut lehekülge.

\begin{lscommand}
\ci{linebreak}\verb|[|\emph{n}\verb|]|,
\ci{nolinebreak}\verb|[|\emph{n}\verb|]|,
\ci{pagebreak}\verb|[|\emph{n}\verb|]|,
\ci{nopagebreak}\verb|[|\emph{n}\verb|]|
\end{lscommand}
\noindent soovitavad kohti, kus murdmine võib (või ei või) toimuda.
Nende käskude toimet saab autor mõjutada valikulise argumendiga
\emph{n}, mis on täisarv nullist neljani. Kui \emph{n} on väiksem kui 4,
siis jääb \LaTeX ile võimalus käsku ignoreerida, kui tulemus näeks välja
väga halb. Neid "`break"'-käske ei tule segamini ajada eelnevate
"`new"'-käskudega. Isegi kui anda "`break"'-käsk, püüab \LaTeX{} ikkagi
ridade paremad servad ja lehekülje kogupikkuse ühtlaseks saada, nagu
järgmises jaotises kirjeldatud; see võib jätta teksti ebameeldivad
lüngad. Kui on tõesti vaja alustada uut rida või uut lehekülge, siis
tuleks kasutada vastavaid "`new"'-käske. Jälgi käskude nimesid!

\LaTeX{} püüab alati moodustada parima võimaliku reamurdmise. Kui ta ei
leia viisi murda ridu nii, nagu tema kõrgetele standarditele kohane,
jätab ta ühe rea lõigust paremale välja ulatuma ning kaebab sisendfaili
töötlemise ajal "`\wi{ületäitunud horisontaalkast}i"' ("`\index{overfull
hbox@\textit{overfull hbox}}\emph{overfull hbox}"') üle. Kõige
sagedamini juhtub see siis, kui \LaTeX{} ei suuda leida sobivat kohta
sõna poolitamiseks.\footnote{Kuigi \LaTeX{} annab sel juhul hoiatuse
(\texttt{Overfull \bs{}hbox}) ja kirjutab teda häiriva rea ekraanile,
pole selliseid ridu alati lihtne üles leida. Kui lisada käsule
\ci{documentclass} suvand \index{draft@\texttt{draft}}\texttt{draft},
siis tähistatakse need read paksu musta joonega lehekülje paremas
servas.} \LaTeX i saab instrueerida oma standardeid veidi alandama, kui
anda käsk \ci{sloppy}. See väldib taolisi üleulatuvaid ridu
sõnadevahelise ruumi suurendamise teel -- isegi kui lõppväljund ei ole
optimaalne. Sel juhul antakse kasutajale hoiatus "`\wi{alatäitunud
horisontaalkast}"' ("`\index{underfull hbox@\textit{underfull
hbox}}\emph{underfull hbox}"'). Enamikul juhtudel ei näe tulemus välja
väga hea. Käsk \ci{fussy} toob \LaTeX i normaalse tegutsemisviisi juurde
tagasi.

\subsection{Poolitamine} \label{hyph}

\LaTeX{} poolitab ise sõnu seal, kus vaja. Kui poolitamisalgoritm ei
leia õigeid poolituskohti, saab olukorra heastada järgmiste käskudega,
mis kirjeldavad \TeX ile erandeid.

Käsk
\begin{lscommand}
\ci{hyphenation}\verb|{|\emph{sõnade loend}\verb|}|
\end{lscommand}
\noindent lubab argumendis loetletud sõnu poolitada ainult märkidega
"`\verb|-|"' näidatud kohtadelt. Käsu argument peaks sisaldama ainult
sõnu, mis koosnevad tavalistest tähtedest, või täpsemini märkidest,
mida \LaTeX{} peab tavalisteks tähtedeks.
\index{poolitussoovitused}Poolitussoovitused jäävad kehtima keelele, mis
on aktiivne poolitussoovituste käsu täitmise hetkel. See tähendab, et
kui panna käsk dokumendi preambulisse, siis mõjutab see inglise keele
poolitust. Kui kasutada mõnda keeletoetuspaketti nagu \pai{babel} ja
anda see käsk pärast paketis keele valimist või pärast käsku
\verb|\begin{document}|, siis kehtivad poolitussoovitused keele jaoks,
mis paketis aktiveeritakse.

Järgmine näide lubab poolitada sõna \emph{poolitamine}, samuti sõna
\emph{Poolitamine}, ning keelab poolitada sõnu \emph{FORTRAN},
\emph{Fortran} ja \emph{fortran}. Argumendis ei tohi olla erimärke
ega -sümboleid.

Näide:
\begin{code}
\verb|\hyphenation{FORTRAN Poo-li-ta-mi-ne}|
\end{code}

Käsk \ci{-} lisab sõnasse valikulise poolituskoha. See saab ühtlasi
ainsaks punktiks, kust seda sõna võib poolitada. See käsk on iseäranis
kasulik spetsiaalsümboleid (nt võõrtäpitähti) sisaldavate sõnade puhul,
sest spetsiaalsümbolitega sõnu \LaTeX{} automaatselt ei poolita.
%\footnote{Unless you are using the new
%\wi{DC fonts}.}.

\begin{example}
Ma arvan, et see on: kuu\-li\-%
len\-nu\-tee\-tun\-%
ne\-li\-luuk
\end{example}

Mitut sõna saab hoida koos samal real käsuga
\begin{lscommand}
\ci{mbox}\verb|{|\emph{tekst}\verb|}|
\end{lscommand}
\noindent See hoiab argumendi üheskoos igas olukorras.

\begin{example}
Minu telefoninumber muutub varsti.
Uus number on \mbox{0116 291 2319}.

Parameeter
\mbox{\emph{failinimi}} peaks
sisaldama faili nime.
\end{example}

Käsk \ci{fbox} sarnaneb käsuga \ci{mbox}, kuid lisaks joonistab sisu
ümber nähtava raami.

\section{Valmisfraasid}

Eelnevatel lehekülgedel esines mõnes näites paar väga lihtsat \LaTeX i
käsku, mis on mõeldud kindlate tekstifraaside trükkimiseks:

\vspace{2ex}

\noindent
\begin{tabular}{@{}lll@{}}
Käsk&Näide&Kirjeldus\\
\hline
\ci{today} & \today   & Tänane kuupäev\\
\ci{TeX} & \TeX       & Lemmiktrükiladuja\\
\ci{LaTeX} & \LaTeX   & Mängu nimi\\
\ci{LaTeXe} & \LaTeXe & Praegune kehastus\\
\end{tabular}

\section{Erimärgid ja -sümbolid}

\subsection{Jutumärgid}

\index{jutumärgid}Jutumärke\index{""@\texttt{""}} \emph{ei tuleks}
sisestada märkidena \verb|"| nagu kirjutusmasinal. Trükinduses
kasutatakse spetsiaalseid avavaid ja sulgevaid jutumärke. Avavaid
inglisepäraseid jutumärke märgivad \LaTeX is kaks \textasciigrave{}
(graavis) ja sulgevaid jutumärke kaks \textquotesingle{} (apostroof).
Üksikjutumärkide saamiseks tuleb sisestada kumbagi üks.
\begin{example}
``Palun vajuta `x' klahvi.''
\end{example}
Kuigi visuaalne kuju pole ideaalne, on avav jutumärk siin tõesti graavis
(\textasciigrave) ja sulgev jutumärk apostroof (\textquotesingle),
olenemata sellest, kuidas see valitud kirjas võib paista.

\subsection{Kriipsud}

\LaTeX{} tunneb nelja tüüpi \wi{kriips}e. Kolme neist saab sisestada
erineva arvu järjestikuste sidekriipsudega. Neljas sümbol ei ole
tegelikult üldse kriips, vaid matemaatiline sümbol miinusmärk:

\begin{example}
üks-kaks-kolm, T-särk\\
leheküljed 13--67\\
jah --- või ei? \\
$0$, $1$ ja $-1$
\end{example}
Nende kriipsude nimed on: \index{-}"`-"' \wi{sidekriips},
\index{-@--}"`--"' \wi{enn-kriips}, \index{-@---}"`---"' \wi{emm-kriips}
ja \index{-@$-$}"`$-$"' \wi{miinusmärk}.

\subsection{Tilde ($\sim$)}
\index{URL-link}\index{tilde}
Üks märk, mida näeb sageli veebiaadressides, on tilde. \LaTeX is saab
seda moodustada käsuga \verb|\~{}|, kuid tulemus \~{} pole
võib-olla selline, nagu sooviks. Selle asemel võib proovida:

\begin{example}
http://www.rikas.ee/\~{}puhk \\
http://www.tark.ee/$\sim$demo
\end{example}

\subsection{Kaldkriips (/)}
\index{kaldkriips}
Kahe sõna vahele kaldkriipsu panemiseks võib selle lihtsalt sisestada,
näiteks \texttt{loe/kirjuta}, kuid nii käsitleb \LaTeX{} kahte sõna
ühena. Kummaski sõnas keelatakse poolitamine, nii et tekkida võib
ületäitumise vigu. Sellest võib üle saada käsuga \ci{slash}, näiteks
\verb|loe\slash kirjuta|, mis lubab poolitamist. Kuid
tavalise märgiga / saab esitada suhteid või ühikuid, nt
\texttt{5 MB/s}.

\subsection{Kraadimärk \texorpdfstring{($\circ$)}{}}

\index{kraadimärk}Kraadimärgi trükkimine puhtas \LaTeX is:
\begin{example}
Külma on $-30\,^{\circ}\mathrm{C}$.
Varsti muutun ma ülijuhtivaks.
\end{example}

Pakett \pai{textcomp} teeb kraadimärgi kättesaadavaks ka käsuna
\ci{textdegree} ja kombinatsioonis tähega C käsuna \ci{textcelsius}.

\begin{example}
30 \textcelsius{} on
86 \textdegree{}F.
\end{example}

\subsection{Euro märk \texorpdfstring{(\officialeuro)}{}}

\index{euro märk}Kirjutades tänapäeval rahast, läheb vaja euro märki.
Seda sisaldavad paljud kaasaegsed kirjapered. Lugedes dokumendi
preambulis sisse paketi \pai{textcomp}
\begin{lscommand}
\verb|\usepackage{textcomp}|
\end{lscommand}
\noindent saab euro märgi teksti lisada käsuga
\begin{lscommand}
\ci{texteuro}
\end{lscommand}

Kui kiri ei sisalda omaette euro märki või kui kirja euro märk ei
meeldi, siis on veel kaks valikut.

Esiteks võib kasutada paketti \pai{eurosym}, mis annab ametliku euro
märgi:%
\begin{lscommand}
\verb|\usepackage[official]{eurosym}|
\end{lscommand}
Kui eelistus on kirjaga kokkusobiv euro märk, siis tuleks suvandi
\texttt{official} asemele panna \texttt{gen}.

%If the Adobe Eurofonts are installed on your system (they are available for
%free from \url{ftp://ftp.adobe.com/pub/adobe/type/win/all}) you can use
%either the package \pai{europs} and the command \ci{EUR} (for a Euro symbol
%that matches the current font).
% does not work
% or the package
% \pai{eurosans} and the command \ci{euro} (for the ``official Euro'').

%The \pai{marvosym} package also provides many different symbols, including a
%Euro, under the name \ci{EURtm}. Its disadvantage is that it does not provide
%slanted and bold variants of the Euro symbol.

\begin{table}[!htbp]
\caption{Kotitäis euro märke} \label{eurosymb}
\begin{lined}{10cm}
\begin{tabular}{llccc}
LM+\pai{textcomp}  &\verb+\texteuro+ & \huge\texteuro &\huge\sffamily\texteuro
                                                &\huge\ttfamily\texteuro\\
\pai{eurosym}      &\verb+\euro+ & \huge\officialeuro &\huge\sffamily\officialeuro
                                                &\huge\ttfamily\officialeuro\\
\verb|[gen]|\pai{eurosym} &\verb+\euro+ & \huge\geneuro  &\huge\sffamily\geneuro
                                                &\huge\ttfamily\geneuro\\
%europs       &\verb+\EUR + & \huge\EURtm        &\huge\EURhv
%                                                &\huge\EURcr\\
%eurosans     &\verb+\euro+ & \huge\EUROSANS  &\huge\sffamily\EUROSANS
%                                             & \huge\ttfamily\EUROSANS \\
%marvosym     &\verb+\EURtm+  & \huge\mvchr101  &\huge\mvchr101
%                                               &\huge\mvchr101
\end{tabular}
\medskip
\end{lined}
\end{table}

\subsection{Mõttepunktid (\texorpdfstring{\ldots}{...})}

Kirjutusmasinal haarab koma või punkt enda alla sama palju ruumi kui
ükskõik milline muu täht. Raamatute trükkimisel aga võtavad need märgid
väga vähe ruumi ja nad laotakse tihedalt eelneva tähe kõrvale. Seetõttu
annab \wi{mõttepunktid}e sisestamine kolme punkti sisestamise teel vale
tulemuse. Selle asemel on mõttepunktide vormistamiseks omaette käsk,
mille nimi on

\begin{lscommand}
\ci{ldots} (madalad punktid)
\end{lscommand}
\index{mõttepunktid}\index{punktid}\index{kolm punkti}\index{...@\ldots}

\begin{example}
Mitte nii ... vaid nii:\\
New York, Tokyo, Budapest, \ldots
\end{example}

\subsection{Ligatuurid}

Mõnes keeles laotakse teatavad tähekombinatsioonid teinekord mitte
kahte eri tähte teineteise järele pannes, vaid iseseisvate märkidena:
\begin{code}
{\large ff fi fl ffi \ldots}\quad
mitte aga\quad {\large f{}f f{}i f{}l f{}f{}i \ldots}
\end{code}
Nende niinimetatud \wi{ligatuurid}e moodustamise saab keelata, kui
lisada kahe kõnealuse tähe vahele \ci{mbox}\verb|{}|. See võib olla
vajalik kahest sõnast koosnevate liitsõnade puhul.
\begin{example}
\Large Mitte shelfful,\\
vaid shelf\mbox{}ful
\end{example}

\subsection{Täpid ja erisümbolid}

\LaTeX{} toetab paljude keelte \index{diakriitilised
märgid}diakriitilisi märke ja \index{erisümbolid}erisümboleid.
Tabelis~\ref{accents} on loetletud iga sorti diakriitilised märgid
rakendatuna tähele o. Loomulikult töötavad ka teised tähed.

Selleks, et panna diakriitiline märk tähe i või j peale, tuleb sealt
enne täpp eemaldada. Selleks tuleks täht sisestada kujul
\index{i@\i{} (täpita i)}\verb|\i| või
\index{j@\j{} (täpita j)}\verb|\j|.

\begin{example}
H\^otel, na\"\i ve, \'el\`eve,\\
sm\o rrebr\o d, !`Se\~norita!,\\
Sch\"onbrunner Schlo\ss{}
Stra\ss e
\end{example}

\begin{table}[!hbp]
\caption{Diakriitilised märgid ja erisümbolid} \label{accents}
\begin{lined}{10cm}
\begin{tabular}{*4{cl}}
\A{\`o} & \A{\'o} & \A{\^o} & \A{\~o} \\
\A{\=o} & \A{\.o} & \A{\"o} & \B{\c}{c}\\[6pt]
\B{\u}{o} & \B{\v}{o} & \B{\H}{o} & \B{\c}{o} \\
\B{\d}{o} & \B{\b}{o} & \B{\t}{oo} \\[6pt]
\A{\oe}  &  \A{\OE} & \A{\ae} & \A{\AE} \\
\A{\aa} &  \A{\AA} \\[6pt]
\A{\o}  & \A{\O} & \A{\l} & \A{\L} \\
\A{\i}  & \A{\j} & !` & \verb|!`| & ?` & \verb|?`|
\end{tabular}
\index{täpita \i{} ja \j}\index{skandinaavia tähed}
\index{umlaut}\index{graavis}\index{akuut}
\index{ae@\ae}\index{oe@\oe}\index{a@\aa}\index{o@\o}\index{l@\l}

\bigskip
\end{lined}
\end{table}

\section{Rahvuskeelte tugi}
\index{rahvuskeel}Kirjutades dokumente muus \wi{keel}es kui inglise, on
kolm valdkonda, kus \LaTeX i tuleb sobivalt konfigureerida.

\begin{enumerate}
\item Uuele keelele tuleb kohandada kõik automaatselt genereeritavad
fraasid.\footnote{"`Sisukord"', "`Jooniste loetelu"', \ldots} Paljude
keelte puhul saab seda teha \index{Braams, Johannes}Johannes Braamsi
paketiga \pai{babel}.
\item \LaTeX{} peab tundma uue keele poolitamisreegleid.
Poolitamisreeglite lisamine \LaTeX ile on natuke keerukam. Selleks on
vaja uuesti genereerida vormingufail, andes ette teised poolitusmustrid.
Rohkem infot peaks selle kohta andma \guide.
\item Keelespetsiifilised tüpograafiareeglid. Näiteks prantsuse keeles
on enne iga koolonit (:) kohustuslik tühik.
\end{enumerate}

Kui süsteem on juba sobivalt konfigureeritud, saab paketi \pai{babel}
aktiveerida käsu
\begin{lscommand}
\verb|\usepackage[|\emph{keel}\verb|]{babel}|
\end{lscommand}
\noindent lisamisega pärast käsku \verb|\documentclass|. Iga kord, kui
kompilaator käivitub, kirjutab ta ekraanile keelte nimekirja, mis on
sellesse \LaTeX i süsteemi sisse ehitatud. Valitud \emph{keel}e jaoks
aktiveerib \pai{babel} automaatselt vastavad poolitamisreeglid. Kui
\LaTeX i vormingufail valitud keelt ei toeta, siis \pai{babel} küll
töötab, aga sõnu ei poolita, mis avaldab küljendatud dokumendile üsna
negatiivset mõju.

Samuti defineerib \pai{babel} mõne keele jaoks uued käsud, mis
lihtsustab erimärkide sisestamist. Näiteks saksa keel sisaldab
palju umlaute (\"a\"o\"u). Kui \pai{babel} on laaditud, saab \"o
sisestada \verb|\"o| asemel kujul \verb|"o|.

Kui \pai{babel} kutsutakse välja mitme keelega
\begin{lscommand}
\verb|\usepackage[|\emph{keelA}\verb|,|\emph{keelB}\verb|]{babel}|
\end{lscommand}
\noindent siis aktiivseks jääb suvandite loetelu viimane keel (st
\emph{keelB}). Aktiivset keelt saab muuta käsuga
\begin{lscommand}
\ci{selectlanguage}\verb|{|\emph{keelA}\verb|}|
\end{lscommand}

%Input Encoding
\newcommand{\ieih}[1]{%
% \index{kodeeringud!sisend!#1@\texttt{#1}}%
\index{sisendkodeering!#1@\texttt{#1}}%
\index{#1@\texttt{#1}}}
\newcommand{\iei}[1]{%
\ieih{#1}\texttt{#1}}
%Font Encoding
\newcommand{\feih}[1]{%
% \index{kodeeringud!kirja!#1@\texttt{#1}}%
\index{kirjakodeering!#1@\texttt{#1}}%
\index{#1@\texttt{#1}}}
\newcommand{\fei}[1]{%
\feih{#1}\texttt{#1}}

\pagebreak[2]Enamik kaasaegseid arvutisüsteeme lubab sisestada
rahvuskeelte tähestike tähti otse klaviatuurilt. Eri keelerühmades või
arvutiplatvormides kasutatavaid \wi{sisendkodeering}uid haldab \LaTeX{}
paketi \pai{inputenc} abil:%
\begin{lscommand}
\verb|\usepackage[|\emph{kodeering}\verb|]{inputenc}|
\end{lscommand}

Seda paketti kasutades tuleb arvestada, et ühes arvutis tehtud
fail\index{sisendfail} ei tarvitse olla teises arvutis vaadatav, sest
seal kehtib teine kodeering. Näiteks täht \"a kodeeritakse OS/2-s kui
132, Unixi süsteemides ISO-LATIN~1 kodeeringus kui 228, samas kui
Windowsi \index{kirillitsa}kirillitsa kodeeringus cp1251 see täht üldse
puudub; seepärast tuleks seda võimalust kasutada ettevaatlikult.
Sõltuvalt süsteemist võib kasu olla järgmistest
kodeeringusuvanditest.\footnote{Ladina tähestikus ja \wi{kirillitsa}s
kirjutatavaid keeli toetavaid sisendkodeeringuid tutvustatakse täpsemalt
failide \texttt{inputenc.dtx} ja \texttt{cyinpenc.dtx}
dokumentatsioonis. Paketi dokumentatsiooni genereerimisest on räägitud
jaotises~\ref{sec:Packages}.}

\begin{center}
\begin{tabular}{l | r | r }
Operatsiooni- & \multicolumn{2}{c}{kodeeringud}\\
süsteem  & lääne-ladina      & kirillitsa\\
\hline
Mac     &  \iei{applemac} & \iei{macukr}  \\
Unix    &  \iei{latin1}   & \iei{koi8-ru}  \\
Windows &  \iei{ansinew}  & \iei{cp1251}    \\
DOS, OS/2  &  \iei{cp850} & \iei{cp866nav}
\end{tabular}
\end{center}

\begin{lscommand}
\verb|\usepackage[|\iei{utf8}\verb|]{inputenc}|
\end{lscommand}
\noindent võimaldab luua \LaTeX i sisendfaile UTF-8 kodeeringus. See on
mitmebaidine kodeering, kus iga märk kodeeritakse vähemalt ühe ja
ülimalt nelja baidiga.

Alates sajandivahetusest on enamiku operatsioonisüsteemide (Windows XP,
MacOS X) põhikodeeringuks \wi{Unicode}. Seetõttu on soovitatav iga uue
projekti kodeeringuks valida UTF-8. Paketis \pai{inputenc} määratav
kodeering \iei{utf8} defineerib ainult need sümbolid, mis on olemas
kasutatavates kirjades. Kui vaja on rohkem (mitteladina) märke, siis
võib uurida Unicode'il põhinevat \TeX i mootorit \hologo{XeLaTeX}
jaotises \ref{sec:xetex}.

\index{kirjakodeering}Kirjakodeering on aga midagi muud. See määrab,
millisel positsioonil iga täht \TeX i kirjafailis asub. Mitu
sisendkodeeringut saab kujutada üheks kirjakodeeringuks, mis vähendab
vajaminevate kirjakomplektide arvu. Kirjakodeeringuid hallatakse paketi
\pai{fontenc} abil: \label{fontenc}
\begin{lscommand}
\verb|\usepackage[|\emph{kodeering}\verb|]{fontenc}| \index{kirjakodeering}
\end{lscommand}
\noindent kus \emph{kodeering} on kirjakodeering. On võimalik korraga
sisse lugeda mitu kodeeringut.

Vaikimisi kehtib \LaTeX is kirjakodeering \label{OT1}\fei{OT1}, mis on
kasutusel \TeX i originaalkirjades \index{kirjakomplekt!CM}Computer
Modern (CM). See kodeering sisaldab ainult 7-bitise ASCII märgitabeli
128 märki. Täpitähed moodustab \TeX{} tavalise tähe ja täppide
kombineerimise teel. Kuigi niimoodi saadakse pealtnäha korrektne
väljund, ei lase selline lähenemine automaatselt poolitada täpitähti
sisaldavaid sõnu. Peale selle, mõningaid ladina tähti ei olegi võimalik
saada tavalist tähte diakriitikuga kombineerides, rääkimata mitteladina
tähtedest, nagu kreeka ja kirillitsa tähed.

Nendest puudustest ülesaamiseks on loodud mitmeid 8-bitiseid
\index{kirjakomplekt!CM}CM-kir\-ja\-de taolisi kirjakomplekte. Näiteks
\fei{T1}-kodeeringus kirjad nimega \index{kirjakomplekt!EC}Extended
Cork (EC) sisaldavad harilikke tähti ja diakriitikutega tähti enamiku
Euroopa keelte jaoks, mis kasutavad ladina tähestikku. Kirjakomplekt
\index{kirjakomplekt!LH}LH
sisaldab tähti, mida on vaja tekstide vormistamiseks
\index{kirillitsa}kirillitsat kasutavates keeltes. Kirillitsa
tähemärkide suure arvu tõttu on märgid jaotatud nelja
kirjakodeeringusse: \fei{T2A}, \fei{T2B}, \fei{T2C} ja
\fei{X2}.\footnote{Keelte loetelu, mida igaüks neist kodeeringutest
toetab, leiab juhendist \cite{cyrguide}.} Kirjapakk
\index{kirjakomplekt!CB}CB sisaldab
\fei{LGR}-kodeeringus kirju \index{kreeka keel}kreekakeelse teksti
vormistamiseks.

Nende kirjade kasutamisel paraneb/avaneb poolitamine
mitteingliskeelsetes dokumentides. Uute kirjade teine eelis on see, et
neis on olemas \index{kirjakomplekt!CM}CM-kir"-ja"-pe"-re kirjad kõigis
kaaludes, kujudes ja optiliselt skaleeritud kirjasuurustes.

\subsection{Eesti keele tugi}

\secby{Reimo Palm}{reimo.palm@ut.ee}
\index{eesti keel}Eestikeelse dokumendi koostamiseks tuleks dokumendi
preambulisse panna
\begin{lscommand}
\verb|\usepackage[estonian]{babel}|\\
\verb|\usepackage[utf8]{inputenc}|\\
\verb|\usepackage[T1]{fontenc}|
\end{lscommand}
See aktiveerib eesti keele poolituse, kui \LaTeX i installatsioon on
vastavalt konfigureeritud, ja muudab kõik automaatselt genereeritavad
fraasid eestikeelseks. Sisendfaili kodeeringuks võetakse \iei{utf8} ja
kirjakodeeringuks \fei{T1}.

Lisaks teeb eesti keelepakett kättesaadavaks mõned lisakäsud
eestikeelsetes tekstides levinud vormistuselementide trükkimiseks.
Näiteks "`saksapäraseid"' jutumärke saab sisestada käskudega
\index{""`@\texttt{""}\texttt{`}}\verb|"`| ja
\index{""'@\texttt{""}\texttt{'}}\verb|"'| ning "<prantsusepäraseid">
jutumärke käskudega \index{""<@\texttt{""}\texttt{<}}\verb|"<| ja
\index{"">@\texttt{""}\texttt{>}}\verb|">|. Täienduseks standardsele
poolituskoha käsule \ci{-}, mis mujalt poolitamise keelab, saab
käsuga \index{""-@\texttt{""}\texttt{-}}\verb|"-| lisada poolituskoha,
mis lubab \LaTeX il sõna vajadusel ka mujalt poolitada. Paljud eesti
keelepaketi funktsioonid on laenatud saksa keelepaketist.

Eesti keele aktiveerimisel täidab eesti keelepakett käsu
\ci{frenchspacing}, mille mõjul vormistatakse lauset lõpetavad tühikud
sama pikana nagu sõnadevahelised tühikud.

\subsection{Portugali keele tugi}

\secby{Demerson Andre Polli}{polli@linux.ime.usp.br}
Poolituse aktiveerimiseks ja kõigi automaatsete fraaside tõlkimiseks
\wi{portugali keel}de tuleb anda käsk
\begin{lscommand}
\verb|\usepackage[portuguese]{babel}|
\end{lscommand}
\noindent või Brasiilias olles määrata keeleks \texttt{brazilian}.

Et portugali keeles on palju diakriitikuid, võib preambulisse panna
käsu
\begin{lscommand}
\verb|\usepackage[latin1]{inputenc}|
\end{lscommand}
\noindent mis võimaldab neid korrektselt sisestada, ning samuti käsu
\begin{lscommand}
\verb|\usepackage[T1]{fontenc}|
\end{lscommand}
\noindent et poolitamine oleks õige.

Tabelis~\ref{portuguese} on näidatud preambul, nagu see peaks portugali
keeles kirjutades olema. Selles näites on sisendkodeeringuks
\iei{latin1}. Kaasaegsetes süsteemides võiks selle asemel kasutada
kodeeringut \iei{utf8}.

\begin{table}[hbtp]
\caption{Portugalikeelse dokumendi preambul} \label{portuguese}
\begin{lined}{7cm}
\begin{verbatim}
\usepackage[portuguese]{babel}
\usepackage[latin1]{inputenc}
\usepackage[T1]{fontenc}
\end{verbatim}
\end{lined}
\end{table}

\subsection{Prantsuse keele tugi}

\secby{Daniel Flipo}{daniel.flipo@univ-lille1.fr}
Mõned soovitused \LaTeX iga \index{prantsuse keel}prantsuskeelsete
dokumentide loomiseks. Prantsuse keele tugi loetakse sisse käsuga

\begin{lscommand}
\verb|\usepackage[francais]{babel}|
\end{lscommand}
\noindent See aktiveerib prantsuse keele poolituse, kui \LaTeX i süsteem
on vastavalt konfigureeritud. Samuti muudab see kõik automaatsed fraasid
prantsuskeelseks: \verb+\chapter+ trükib "`Chapitre"', \verb+\today+
trükib tänase kuupäeva prantsuse keeles jne. Samuti muutub kättesaadavas
hulk uusi käske, mille abil saab prantsuskeelseid sisendfaile kirjutada
lihtsamalt. Inspiratsiooni leidmiseks võib vaadata tabelit
\ref{cmd-french}.

\begin{table}[!htbp]
\caption{Erikäsud prantsuse keele jaoks} \label{cmd-french}
\begin{lined}{9cm}
\selectlanguage{french}
\begin{tabular}{ll}
\verb+\og guillemets \fg{}+         \quad &\og guillemets \fg \\[1ex]
\verb+M\up{me}, D\up{r}+            \quad &M\up{me}, D\up{r}  \\[1ex]
\verb+1\ier{}, 1\iere{}, 1\ieres{}+ \quad &1\ier{}, 1\iere{}, 1\ieres{}\\[1ex]
\verb+2\ieme{} 4\iemes{}+           \quad &2\ieme{} 4\iemes{}\\[1ex]
\verb+\No 1, \no 2+                 \quad &\No 1, \no 2   \\[1ex]
\verb+20~\degres C, 45\degres+      \quad &20~\degres C, 45\degres \\[1ex]
\verb+\bsc{M. Durand}+              \quad &\bsc{M.~Durand} \\[1ex]
\verb+\nombre{1234,56789}+          \quad &\nombre{1234,56789}
\end{tabular}
\selectlanguage{english}
\bigskip
\end{lined}
\end{table}

Lülitudes ümber prantsuse keelele, muutub ka loendite vormistus. Rohkem
informatsiooni selle kohta, mida paketi \pai{babel} suvand
\texttt{francais} teeb ja kuidas selle toimimist seadistada, saab
siis, kui lasta \LaTeX ist läbi fail \texttt{frenchb.dtx} ja lugeda
tekkinud dokumenti \texttt{frenchb.dvi}.

Paketi \pai{frenchb} hilisemad versioonid realiseerivad käsu \ci{nombre}
paketi \pai{numprint} abil.

\subsection{Saksa keele tugi}

Mõned soovitused nendele, kes loovad \LaTeX iga
\index{saksa keel}saksakeelseid dokumente. Saksa keele
tugi loetakse sisse käsuga

\begin{lscommand}
\verb|\usepackage[german]{babel}|
\end{lscommand}
\noindent See aktiveerib saksa keele poolituse, kui \LaTeX i süsteem on
vastavalt konfigureeritud. Samuti muudab see kõik automaatsed fraasid
saksakeelseks, nt peatüki tiitliks saab "`Kapitel"', mitte "`Chapter"'.
Samuti muutub kättesaadavaks hulk uusi käske, mille abil on võimalik
saksakeelseid sisendfaile luua kiiremini, isegi kui paketti
\pai{inputenc} mitte kasutada. Inspiratsiooni leidmiseks võib vaadata
tabelit \ref{german}. Paketiga \pai{inputenc} muutub see kõik
ebavajalikuks, kuid siis on tekst ka lukustatud kindlasse kodeeringusse.

\begin{table}[!hbtp]
\index{""`@\texttt{""}\texttt{`}}
\index{""'@\texttt{""}\texttt{'}}
\index{""<@\texttt{""}\texttt{<}}
\index{"">@\texttt{""}\texttt{>}}
\caption{Saksa keele erimärgid} \label{german}
\begin{lined}{8cm}
\selectlanguage{german}
\begin{tabular}{*2{ll}}
\verb|"a| & "a \hspace*{1ex} & \verb|"s| & "s \\[1ex]
\verb|"`| & "` & \verb|"'| & "' \\[1ex]
\verb|"<| või \ci{flqq} & "<  & \verb|">| või \ci{frqq} & "> \\[1ex]
\ci{flq} & \flq & \ci{frq} & \frq \\[1ex]
\ci{dq} & " \\
\end{tabular}
\selectlanguage{english}
\bigskip
\end{lined}
\end{table}

Saksakeelsetes raamatutes esinevad tihti prantsuse jutumärgid (\flqq
guil\-le\-mets\frqq), ent saksa trükiladujad kasutavad neid teistmoodi.
Tsitaat saksakeelses raamatus näeb välja \frqq nii\flqq. \v{S}veitsi
saksakeelses osas kasutavad trükiladujad prantsuse jutumärke \flqq
guillemets\frqq{} samamoodi nagu prantslased.

Käskudega nagu \verb+\flq+ kaasneb üks suur probleem:
\fei{OT1}-kodeeringus (mis on vaikimisi kehtiv kodeering) näevad
prantsuse jutumärgid välja nii nagu matemaatiline sümbol $\ll$,
mille peale trükiladuja saab pahaseks. Samas \fei{T1}-kodeeringus kirjad
juba sisaldavad vajalikke märke. Sellepärast tuleks seda tüüpi
jutumärkide kasutamisel valida dokumendi kirjakodeeringuks \fei{T1}
(käsuga \verb|\usepackage[T1]{fontenc}|).

\subsection[Korea keele tugi]{Korea keele tugi\footnotemark}\label{support_korean}%
\footnotetext{%
Selle jaotise on kirjutanud Karnes Kim\mailto|karnes@ktug.org| ja
Kihwang Lee\mailto|leekh@ktug.org| Korea \TeX ikasutajate Ühingu ja
Korea \TeX i Seltsi nimel.}

\index{korea keel}\index{hangul}Hanguli\footnote{Hangul on korea
kirjasüsteem. Lisainfot leiab aadressilt
\url{http://en.wikipedia.org/wiki/Hangul}.} sümbolite töötlemiseks või
koreakeelse dokumendi vormistamiseks \LaTeX i abil tuleks dokumendi
preambulisse lisada järgmine rida:

\begin{lscommand}
\verb|\usepackage{kotex}|
\end{lscommand}

Seda deklaratsiooni sisaldavat dokumenti tuleb kompileerida pdf\LaTeX
iga, \hologo{XeLaTeX}iga või Lua\LaTeX iga. Tuleks jälgida, et
sisendfail oleks \wi{Unicode}'i UTF-8 kodeeringus. Paketikomplekti
\wi{ko.\TeX}\footnote{Loetakse "`Korea \TeX"'. ko.\TeX{} on pakettide kogum,
millesse kuuluvad teiste hulgas paketid \pai{cjk-ko}, \pai{kotex-utf},
\pai{xetexko} ja \pai{luatexko}.} arendavad pidevalt edasi Korea \TeX
i"-ka"-su"-ta"-ja"-te Ühing\footnote{\url{http://www.ktug.org}} ja Korea
\TeX i Selts ning seda kasutatakse laialdaselt igapäevaste koreakeelsete
dokumentide loomiseks. ko.\TeX\ on olnud CTANis kättesaadav alates 2014.
aastast ning ta kuulub ka \index{texlive@\TeX{} Live}\TeX{} Live'i,
\index{miktex@MiK\TeX}MiK\TeX i ja teiste kaasaegsete
\TeX idistributsioonide koosseisu. Seega on väga tõenäoline, et tööle
saab hakata kohe, ilma lisapakette installimata.

ko.\TeX\ ei kasuta paketti \pai{babel}. Paljusid korea keelega seotud
funktsioone saab aktiveerida paketi \pai{kotex} suvandite ja
seadistuskäskudega. Tegelike koreakeelsete dokumentide koostamiseks
on soovitatav tutvuda paketi dokumentatsiooniga (need dokumendid on
korea keeles).

ko.\TeX iga tuleb kaasa ka
\pai{oblivoir}\index{klass!oblivoir@\textsf{oblivoir}}, klassil
\pai{memoir}\index{klass!memoir@\textsf{memoir}} põhinev dokumendiklass, mis on
kohandatud koreakeelsetele dokumentidele. Koreakeelne dokument algab
seega järgmiselt:

\begin{lscommand}
\verb|\documentclass{oblivoir}|
\end{lscommand}

Koreakeelse dokumendi jaoks aineregistri genereerimiseks tuleks käsu
\texttt{makeindex} asemel anda käsk \texttt{komkindex}, mis on programmi
\wi{MakeIndex} korea keele töötlemiseks kohandatud variant.
Koreakeelsete registrikirjete leksikograafiliseks sortimiseks võib
kasutada ko.\TeX is olemasolevat registristiili \texttt{kotex.ist}
järgmiselt:

\begin{lscommand}
\verb|komkindex -s kotex foo.idx|
\end{lscommand}

Registrit saab genereerida ka programmiga \wi{Xindy}, sest
Xindy korea keele moodul on \TeX{} Live'is olemas.

On olemas veel üks pakett korea keele või \wi{hangul}i vormistamiseks:
\pai{CJK}. Nagu paketi nimi näitab, sisaldab see vahendeid hiina,
jaapani ja korea sümbolite trükkimiseks. See pakett toetab CJK sümbolite
puhul kasutatavaid mitmeseid kodeeringuid. Järgnevas on esitatud lihtne
näide UTF-8 kodeeringus hanguli vormistamisest paketiga \pai{CJK}. See
on kasulik käsikirja esitamisel akadeemilistele ajakirjadele, mis
lubavad autorite nimesid vormistada rahvuskeeltes.

\begin{code}
\begin{verbatim}
\usepackage{CJK}

\begin{CJK}{UTF8}{}
\CJKfamily{nanummj}
...
\end{CJK}
\end{verbatim}
\end{code}

\subsection{Kreeka keele tugi}
\secby{Nikolaos Pothitos}{pothitos@di.uoa.gr}
Tabelis~\ref{preamble-greek} on esitatud preambul, mida on vaja
\index{kreeka keel}kreeka keeles tekstide kirjutamiseks. See preambul
aktiveerib poolitamise ja muudab kõik automaatsed fraasid
kreekakeelseks.\footnote{Kui paketile \pai{inputenc} anda suvand
\iei{utf8x}, siis mõistab \LaTeX{} kreeka kirja ja polütoonilise
kreeka kirja Unicode'i tähti.} Kättesaadavaks muutub ka hulk uusi käske,
mille abil saab lihtsamalt kirjutada kreekakeelseid sisendfaile.
Ajutiselt lülituda inglise keelele ja vastupidi saab käskudega
\verb|\textlatin{|\emph{ingliskeelne tekst}\verb|}| ja
\verb|\textgreek{|\emph{kreekakeelne tekst}\verb|}|, millel mõlemal on
üks argument, mis trükitakse soovitud kirjakodeeringus. Muidu aga võib
kasutada käsku \verb|\selectlanguage{...}| nagu varem kirjeldatud.
Tabelis~\ref{sym-greek} on mõned kreeka keele kirjavahemärgid. Euro
märgi saab käsuga \verb|\euro|.

\begin{table}[!hbtp]
\caption{Kreekakeelse dokumendi preambul} \label{preamble-greek}
\begin{lined}{7cm}
\begin{verbatim}
\usepackage[english,greek]{babel}
\usepackage[iso-8859-7]{inputenc}
\end{verbatim}
\end{lined}
\end{table}
\begin{table}[tbp]
\caption{Kreeka keele erimärgid} \label{sym-greek}
\begin{lined}{4cm}
\selectlanguage{french}
\begin{tabular}{*2{ll}}
\verb|;| \hspace*{1ex}  &  $\cdot$ \hspace*{1ex}  &  \verb|?| \hspace*{1ex}&  ;   \\[1ex]
\verb|((|               &  \og                    &  \verb|))|&  \fg \\[1ex]
\verb|``|               &  `                      &  \verb|''| &  '   \\
\end{tabular}
\selectlanguage{english}
\bigskip
\end{lined}
\end{table}

\subsection{Kirillitsa tugi}

\secby{Maksym Polyakov}{polyama@myrealbox.com}
Paketis \pai{babel} on versioonist~3.7h alates olemas
\fei{T2*}-kodeeringute tugi ja võimalus kirjutada
\index{kirillitsa}kirillitsa tähtedega \index{bulgaaria keel}bulgaaria-,
\index{vene keel}vene- ja \index{ukraina keel}ukrainakeelseid tekste.

Kirillitsa tugi põhineb \LaTeX i standardmehhanismidel ning
pakettidel \pai{fontenc} ja \pai{inputenc}. Kuid kui vaja on kirillitsat
kasutada valemire\v{z}iimis, tuleb enne paketti \pai{fontenc} sisse
lugeda pakett \pai{mathtext}:\footnote{Kasutades \AmS-\LaTeX i pakette,
tuleb need samuti sisse lugeda enne pakette \pai{fontenc} ja
\pai{babel}.}
\begin{lscommand}
\verb+\usepackage{mathtext}+\\
\verb+\usepackage[T1+\verb+,+\fei{T2A}\verb+]{fontenc}+\\
\verb+\usepackage[+\iei{koi8-ru}\verb+]{inputenc}+\\
\verb+\usepackage[english,bulgarian,russian,ukranian]{babel}+
\end{lscommand}

Üldiselt valib \pai{babel} sobiva kirjakodeeringu automaatselt,
ülalnimetatud kolme keele puhul on selleks \fei{T2A}. Kuid dokumendid pole
piiratud üheainsa kirjakodeeringuga. Mitmekeelsetes dokumentides, kus on
kasutusel nii kirillitsaga kui ka ladina tähestikuga keeled, tuleks ära
määrata ka ladina kirjakodeering. Pakett \pai{babel} lülitub
automaatselt ümber õigele kirjakodeeringule, kui dokumendis valitakse
erinev keel.

Lisaks poolitamise võimaldamisele, automaatselt genereeritavate fraaside
tõlkimisele ja keelespetsiifiliste tüpograafiareeglite (nagu
\ci{frenchspacing}) aktiveerimisele teeb \pai{babel} kättesaadavaks ka
mõned käsud teksti trükkimiseks bulgaaria, vene või ukraina keele
standardite kohaselt.

Kõigi kolme keele jaoks on olemas keelespetsiifilised kirjavahemärgid:
kirillitsa kriips teksti jaoks (see on veidi kitsam kui ladina kriips ja
ümbritsetud väikeste vahedega), kriips otsekõne jaoks, jutumärgid ja
käsud poolitamise hõlbustamiseks; vt tabelit~\ref{Cyrillic}.

% Table borrowed from Ukrainian.dtx
\begin{table}[htb]
  \index{""-@\texttt{""}\texttt{-}}
  \index{""`@\texttt{""}\texttt{`}}
  \index{""'@\texttt{""}\texttt{'}}
  \index{"">@\texttt{""}\texttt{>}}
  \index{""<@\texttt{""}\texttt{<}}
  \index{bulgaaria keel}\index{vene keel}\index{ukraina keel}
  \caption[Bulgaaria, vene ja ukraina keel]{Paketi \pai{babel} bulgaaria,
  vene ja ukraina keelesuvandite täiendavad definitsioonid}\label{Cyrillic}
  \begin{lined}{\textwidth}
  \begin{tabular}{@{}p{.1\hsize}@{}p{.9\hsize}@{}}
   \verb="|= & keela ligatuur selles kohas               \\
   \verb|"-| & ilmutatud poolituskoht, mis lubab poolitamist
               ülejäänud sõnas                         \\
   \verb|"---| & kirillitsa emm-kriips tavatekstis                      \\
   \verb|"--~| & kirillitsa emm-kriips liitnimedes (perekonnanimedes)       \\
   \verb|"--*| & kirillitsa emm-kriips otsekõne tähistamiseks         \\
   \verb|""| & nagu \verb|"-|, aga ei moodusta poolitusmärki
               (sidekriipsuga liitsõnade jaoks, nt\ \verb|x-""y|
               või muude märkide jaoks nagu "`luba/keela"')     \\
   \verb|"~| & liitsõna märk ilma poolituskohata        \\
   \verb|"=| & liitsõna märk poolituskohaga, lubab liitunud sõnades
               poolitamist                   \\
   \verb|",| & väiketühik initsiaalides, poolituskohaga järgnevas
               perekonnanimes                                \\
   \verb|"`| & saksa vasakpoolsed jutumärgid
               (näeb välja nagu ,\kern-0.08em,)                     \\
   \verb|"'| & saksa parempoolsed jutumärgid (näeb välja nagu ``)       \\%''
   \verb|"<| & prantsuse vasakpoolsed jutumärgid (näeb välja nagu $<\!\!<$)  \\
   \verb|">| & prantsuse parempoolsed jutumärgid (näeb välja nagu $>\!\!>$) \\
  \end{tabular}
  \bigskip
  \end{lined}
\end{table}

Paketi \pai{babel} vene ja ukraina keelesuvand defineerivad käsud
\ci{Asbuk} ja \ci{asbuk}, mis töötavad nii nagu \ci{Alph} ja
\ci{alph}\footnote{Käsud, mis väljastavad loendurite väärtused kujul a,
b, c, \ldots}, kuid annavad tulemuseks vene või ukraina (vastavalt
sellele, mis on dokumendi aktiivne keel) tähestiku suured ja väikesed
tähed. Bulgaaria keelesuvandi puhul on olemas käsud \ci{enumBul} ja
\ci{enumLat} (\ci{enumEng}), mille toimel \ci{Alph} ja \ci{alph}
produtseerivad kas bulgaaria või ladina (inglise) tähestiku tähti.
Vaikimisi annavad \ci{Alph} ja \ci{alph} bulgaaria keelesuvandi puhul
bulgaaria tähestiku tähti.

%Finally, math alphabets are redefined and  as well as the commands for math
%operators according to Cyrillic typesetting traditions.

\subsection{Mongoolia keele tugi}

\index{mongoolia keel}Mongooliakeelsete tekstide trükkimisel on valida
kahe paketi vahel: mitmekeelne \pai{babel} ja \wi{Mon\TeX}, mille
autoriks on \index{Corff, Oliver}Oliver Corff.

Mon\TeX is on olemas tugi nii \index{kirillitsa}kirillitsa kui ka
traditsioonilise mongoolia kirja jaoks. Mon\TeX i käskude kasutamiseks
tuleb preambulisse lisada
\begin{lscommand}
\verb|\usepackage[|\emph{keel}\verb|,|\emph{kodeering}\verb|]{mls}|
\end{lscommand}
\noindent Suvandiks \emph{keel} tuleks panna \texttt{xalx}, see
genereerib päised ja kuupäevad kaasaegses mongoolia keeles. Dokumendi
kirjutamiseks traditsioonilises mongoolia kirjas tuleks suvandiks
\emph{keel} võtta \texttt{bicig}. Keelesuvand \texttt{bicig} aktiveerib
teksti sisestamiseks "`lihtsustatud transliteratsiooni"' meetodi.

Ladina transliteratsiooni saab sisse ja välja lülitada käskudega
\begin{lscommand}
\verb|\SetDocumentEncodingLMC|
\end{lscommand}
\noindent ja
\begin{lscommand}
\verb|\SetDocumentEncodingNeutral|
\end{lscommand}

Mon\TeX i kohta leiab rohkem infot veebiaadressilt
\CTAN|pkg/montex|.

Pakett \pai{babel} toetab mongoolia kirillitsat. Mongoolia keele tugi
aktiveeritakse järgmiste käskudega:

\begin{lscommand}
\verb|\usepackage[T2A]{fontenc}|\\
\verb|\usepackage[mn]{inputenc}|\\
\verb|\usepackage[mongolian]{babel}|
\end{lscommand}

\noindent kus \iei{mn} on sisendkodeering \iei{cp1251}. Kaasaegsema
lähenemise puhul tuleks kirjutada selle asemele \iei{utf8}.

\subsection{Unicode}

\secby{Axel Kielhorn}{A.Kielhorn@web.de}
\wi{Unicode} on loomulik valik siis, kui ühes dokumendis on koos mitu
keelt, eriti kui need keeled ei ole ladina tähestikus. On olemas kaks
\hologo{TeX}i mootorit, mis suudavad töödelda Unicode'is kirjutatud
sisendit.

\begin{description}
\item[\normalfont\hologo{XeTeX}] arendati välja MacOS X jaoks, kuid on
    nüüd olemas kõigi arhitektuuride jaoks. Avaldati esmakordselt \TeX{}
    Live 2007-s.\index{XeTeX@\hologo{XeTeX}}
\item[\normalfont\hologo{LuaTeX}] on pdf\TeX i järglane. Avaldati
    esmakordselt \TeX{} Live 2008-s.\index{LuaTeX@\hologo{LuaTeX}}
\end{description}

Järgnevas kirjeldame \hologo{XeLaTeX}i, nagu see on avaldatud \TeX{}
Live 2010-s.\index{XeLaTeX@\hologo{XeLaTeX}}

\subsubsection{Kiirstart}

Olemasoleva \LaTeX i faili konvertimiseks \hologo{XeLaTeX}i tuleb teha
järgmist.

\begin{enumerate}
  \item Salvestada fail UTF-8 kodeeringus.
  \item Eemaldada preambulist read
\begin{lscommand}
\verb|\usepackage{inputenc}|\\
\verb|\usepackage{fontenc}|\\
\verb|\usepackage{textcomp}|
\end{lscommand}
  \item Asendada käsk
\begin{lscommand}
\verb|\usepackage[|\emph{keelA}\verb|]{babel}|
\end{lscommand}
käskudega
\begin{lscommand}
\verb|\usepackage{polyglossia}|\\
\verb|\setdefaultlanguage[babelshorthands]{|\emph{keelA}\verb|}|
\end{lscommand}
  \item Lisada preambulisse
\begin{lscommand}
\verb|\usepackage[Ligatures=TeX]{fontspec}|
\end{lscommand}
\end{enumerate}

Pakett \pai{polyglossia}~\cite{polyglossia} asendab paketti \pai{babel}
ning hoolitseb poolitusmustrite ja automaatsete fraaside eest. Suvand
\verb|babelshorthands| aktiveerib \pai{babel}iga ühilduvad
kiirkombinatsioonid saksa ja katalaani keele jaoks.

Pakett \pai{fontspec}~\cite{fontspec} tegeleb kirjade laadimisega
\hologo{XeLaTeX}is ja \hologo{LuaTeX}is. Vaikimisi on kirjaks Latin
Modern Roman. Vähetuntud on fakt, et mõned \hologo{TeX}i käsud on
tegelikult \index{kirjakomplekt!CM}Computer Moderni kirjades
defineeritud \wi{ligatuurid}. Soovides neid kasutada mitte-\hologo{TeX}i
kirjaga, tuleb nad ise järele teha. Suvand \texttt{Ligatures=TeX}
defineerib järgmised ligatuurid:

\begin{tabular}{rr}
\verb|--|	& --\\
\verb|---|	& ---\\
\verb|''|	& ''\\
\verb|``|	& ``\\
\verb|!`|	& !`\\
\verb|?`|	& ?`\\
\verb|,,|	& ,,\\
\verb|<<|	& <<\\
\verb|>>|	& >>\\
\end{tabular}

\subsubsection{Minu jaoks on see nagu
$\kappa\rho\epsilon\epsilon\kappa\alpha$ keel}

Siiamaani pole Unicode'i \hologo{TeX}i mootori eelised veel välja
tulnud. See muutub, kui jätta ladina kiri selja taha ning liikuda mõne
huvitavama keele juurde, nagu kreeka või vene keel. Unicode'il põhinevas
süsteemis on võimalik lihtsalt\footnote{Lihtsa väikeste väärtuste
puhul.} tekstiredaktoris sisestada sümboleid ja \hologo{TeX} mõistab
neid.

Erinevates keeltes kirjutamiseks tuleb ainult preambulis keeled
määrata:

\begin{lscommand}
\verb|\setdefaultlanguage{english}|\\
\verb|\setotherlanguage[babelshorthands]{german}|
\end{lscommand}

Saksakeelse lõigu kirjutamiseks saab kasutada keskkonda \ei{german}:

\begin{code}
\verb|Harilik tekst.|\\
\verb|\begin{german}|\\
\verb|Deutscher Text.|\\
\verb|\end{german}|\\
\verb|Veel harilikku teksti.|
\end{code}

Kui vaja on ainult mõnda teiskeelset sõna, võib kasutada käsku
\verb|\text|\emph{keel}:

\begin{code}
\verb|Harilik tekst. \textgerman{Gesundheit} on|\\
\verb|tegelikult saksa sõna.|
\end{code}

See võib tunduda tarbetu, sest ainuke eelis on õige poolitus, kuid kui
teine keel on veidi eksootilisem, siis on asi vaeva väärt.

Mõnikord võivad põhiteksti kirjast puududa märgid, mida on teises keeles
vaja\footnote{Latin Modern ei sisalda \index{kirillitsa}kirillitsa
tähti.}. Lahendus on defineerida selle keele jaoks omaette kiri. Iga
kord, kui uus keel aktiveeritakse, kontrollib \pai{polyglossia}
kõigepealt, kas selle keele jaoks on kiri defineeritud.

\begin{lscommand}
\verb|\newfontfamily\russianfont[Script=Cyrillic,(...)]{(kiri)}|
\end{lscommand}

Nüüd võib kirjutada

\begin{code} \verb|\textrussian{Pravda} on Vene ajaleht.|
\end{code}
%
Väljundisse ilmub see fraas siis kirillitsa tähtedega.

Pakett \pai{xgreek}\index{kreeka keel}~\cite{xgreek} võimaldab panna
kirja tekste nii vanakreeka kui ka uuskreeka (monotooniline või
polütooniline) keeles.

\subsubsection{Paremalt vasakule kirjutatavad keeled}

Mõnesid keeli kirjutatakse vasakult paremale, teisi paremalt vasakule.
Viimaste toetamiseks on paketil \pai{polyglossia} vaja paketti
\pai{bidi}\footnote{Pakett \pai{bidi} ei toeta \hologo{LuaTeX}i.}
\cite{bidi}. Pakett \pai{bidi} peaks olema laaditavatest pakettidest
kõige viimane, asudes isegi pärast paketti \pai{hyperref}, mis
tavaliselt on viimane pakett. (Kuna \pai{polyglossia} loeb sisse paketi
\pai{bidi}, tähendab see, et \pai{polyglossia} peaks olema viimane
laaditav pakett.)

Pakett \pai{xepersian}\index{pärsia keel} \cite{xepersian} sisaldab
pärsia keele tuge. Seal on olemas pärsia \LaTeX i käsud, mille abil saab
sisestada käske nagu \verb|\section| pärsia keeles, mistõttu on see
pärsia keele rääkijatele väga atraktiivne. Pakett \pai{xepersian} on
ainuke pakett, mis toetab \wi{kashida}t \hologo{XeLaTeX}iga. Sarnast
algoritmi kasutav pakett süüria keele jaoks on arendamisel.

Iraani Info- ja Sidetehnoloogia Ülemnõukogu poolt kättesaadavaks tehtud
kirja \index{kiri!IranNastaliq}IranNastaliq saab alla laadida
organisatsiooni veebilehelt
\url{http://www.scict.ir/Portal/Home/Default.aspx}.

Pakett \pai{arabxetex} \cite{arabxetex} toetab mitut \index{araabia
kirjas keel}araabia kirja kasutavat keelt: araabia, pärsia, urdu,
sindhi, pu\v{s}tu, ottomani (türgi), kurdi, ka\v{s}miiri, malai (jawi),
uiguuri. Paketis on realiseeritud kirjavastavuste tabel, mis võimaldab
\hologo{XeLaTeX}il töödelda Arab\TeX i ASCII transkriptsioonis
kirjutatud sisendit.

Iraani Macikasutajate Ühendus on loonud kirjad, mis toetavad mitut
araabia tähestikuga keelt.
% \url{http://wiki.irmug.org/index.php/X_Series_2}.
% not working

\index{heebrea keel}Heebrea keele jaoks paketti pole, sest seda pole
vaja; paketi \pai{polyglossia} heebrea keele tugi peaks olema piisav.
Kuid tarvis on sobivat kirja täisväärtusliku Unicode'i heebrea
märgikomplektiga. Mittekommertseesmärkideks on vabalt kasutatav kiri
\index{kiri!SBL Hebrew}SBL
Hebrew, mis on saadaval aadressilt
\url{http://www.sbl-site.org/educational/biblicalfonts.aspx}. Teine
kiri, mida levitatakse Avatud Kirja Litsentsi alusel, on \index{kiri!Ezra SIL}Ezra SIL, mille
leiab aadressilt
\url{http://scripts.sil.org/cms/scripts/page.php?site_id=nrsi&id=ezrasil_home}.
Meeles tuleb pidada valida õige kirjasüsteem:
\begin{lscommand}
\verb|\newfontfamily\hebrewfont[Script=Hebrew]{SBL Hebrew}| \\
\verb|\newfontfamily\hebrewfont[Script=Hebrew]{Ezra SIL}|
\end{lscommand}

\subsubsection{Hiina, jaapani ja korea keel (CJK)}
\index{hiina keel}\index{jaapani keel}\index{korea keel}

Nende keelte puhul hoolitseb kirjavaliku ja kirjavahemärkide eest pakett
\pai{xeCJK} \cite{xecjk}.

\section{Sõnavahed}  \index{vahe!sõnade vahel}

Sirge parema serva saavutamiseks lisab \LaTeX{} sõnade vahele muutuvas
koguses ruumi. Ingliskeelset teksti vormistades lisab ta lause lõppu
ruumi natuke rohkem, sest see muudab teksti loetavamaks. \LaTeX{}
eeldab, et laused lõpevad punktiga, küsimärgiga või hüüumärgiga. Kui
punkt asub suurtähe järel, siis seda lause lõpuks ei loeta, sest
suurtähe järel esinevad punktid tavaliselt lühendites.

Igasugused kõrvalekalded nendest eeldustest tuleb määrata autoril.
Langjoon tühiku ees moodustab tühiku, mille suurus ei muutu.
Tilde~\verb|~| moodustab tühiku, mille suurus ei muutu ja mis lisaks
keelab rea murdmise. Käsk \verb|\@| punkti ees näitab, et see punkt
lõpetab lause, isegi kui ta järgneb suurtähele. \cih{"@} \index{~@
\verb.~.} \index{tilde}\index{tühik!pärast punkti}

\begin{example}
Hr.~Kask oli teda nähes rõõmus\\
Vt.~joon.~5\\
Mulle meeldib BASIC\@. Aga Sulle?
\end{example}

Lisaruumi panemise punktide järele võib ära keelata käsuga
\begin{lscommand}
\ci{frenchspacing}
\end{lscommand}
\noindent mis käsib \LaTeX il \emph{mitte} panna punkti järele rohkem
ruumi kui tavaliste tähemärkide järele. See on väga levinud
mitteingliskeelsetes tekstides, välja arvatud bibliograafiad. Käsu
\ci{frenchspacing} puhul pole käsk \verb|\@|\cih{"@} vajalik.

\section{Pealkirjad, peatükid ja jaotised}

Et lugeja leiaks paremini tee läbi teose, tuleks teos jagada
peatükkideks, jaotisteks ja alajaotisteks. \LaTeX{} toetab seda
spetsiaalsete käskudega, mille parameetriks on jaotise pealkiri. Autori
ülesanne on kasutada neid käske õiges järjekorras.

Klassis \pai{article} on olemas järgmised
\wi{jaotisekäsud}\index{liigendusüksused}:

\begin{lscommand}
\ci{section}\verb|{...}|\\
\ci{subsection}\verb|{...}|\\
\ci{subsubsection}\verb|{...}|\\
\ci{paragraph}\verb|{...}|\\
\ci{subparagraph}\verb|{...}|
\end{lscommand}

Soovides liigendada dokumenti osadeks ilma jaotiste või peatükkide
nummerdust mõjutamata, võib kasutada käsku
\begin{lscommand}
\ci{part}\verb|{...}|
\end{lscommand}

Klassides \pai{report} ja \pai{book} on olemas veel üks, kõige
ülemise taseme jaotisekäsk
\begin{lscommand}
\ci{chapter}\verb|{...}|
\end{lscommand}

Kuna klass \pai{article} ei tunne peatükke, saab artikleid lihtsasti
koondada raamatusse peatükkidena. Jaotiste vertikaalvahed, nummerduse
ja pealkirjade kirjasuuruse valib \LaTeX{} automaatselt.

Jaotisekäskudest on kaks käsku veidi erilised:
\begin{itemize}
\item käsk \ci{part} ei mõjuta peatükkide nummerdust;
\item käsul \ci{appendix} ei ole argumente, ta vaid muudab
peatükkide numbrid tähtedeks.\footnote{Artikliklassi puhul muudab
jaotiste numbreid.}
\end{itemize}

Eelmisest käivituskorrast võetud jaotiste pealkirjade ja
leheküljenumbrite põhjal loob \LaTeX{} sisukorra.\index{sisukord} Käsk
\begin{lscommand}
\ci{tableofcontents}
\end{lscommand}
\noindent laieneb oma esinemise kohas sisukorraks. Uut dokumenti tuleb
korrektse sisukorra saamiseks kompileerida ("`\LaTeX ida"') kaks korda.
Mõnikord võib olla vaja kompileerida dokumenti kolmandatki korda, sel
juhul annab \LaTeX{} sellest teada.

Kõigist ülaltoodud jaotisekäskudest on olemas ka tärniga variandid. Käsu
tärniga variant on käsu nimi, mille järele on lisatud tärn
\verb|*|.\index{käsk!tärniga}\index{tärniga käsk}
Nende abil saab moodustada jaotiste pealkirju, mida ei näidata
sisukorras ega nummerdata. Näiteks käsust \verb|\section{Abi}| saab
\verb|\section*{Abi}|.

Tavaliselt ilmuvad jaotiste pealkirjad sisukorras täpselt sellisel kujul
nagu tekstis kirjas. Mõnikord pole see aga võimalik, sest pealkiri on
sisukorda mahtumiseks liiga pikk. Siis võib sisukorrakirje määrata
valikulise argumendina enne tegelikku pealkirja.

\begin{code}
\verb|\chapter[Pealkiri sisukorra jaoks]{Pikk|\\
\verb|    ja eriti igav pealkiri, mida näidatakse tekstis}|
\end{code}

Kogu \wi{dokumendi tiitel}\index{tiitel} genereeritakse käsuga
\begin{lscommand}
\ci{maketitle}
\end{lscommand}
\noindent Tiitli sisu tuleb määrata käskudega
\begin{lscommand}
\ci{title}\verb|{...}|, \ci{author}\verb|{...}|
ja vajadusel \ci{date}\verb|{...}|
\end{lscommand}
\noindent enne käsu \verb|\maketitle| andmist. Käsu \ci{author}
argumendis võib olla mitu nime, sel juhul tuleb need üksteisest eraldada
käskudega \ci{and}.

Näide mõne ülalnimetatud käsu rakendamise kohta on toodud
joonisel~\ref{document} leheküljel~\pageref{document}.

Peale ülalvaadeldud jaotisekäskude on \LaTeX is olemas veel järgmised
käsud, mida kasutatakse koos klassiga \pai{book} ja mis aitavad trükist
liigendada. Need käsud muudavad peatükkide pealkirjade ja lehekülgede
nummerduse toimimist nii, nagu võiks oodata raamatult.
\begin{description}
\item[\normalfont\ci{frontmatter}] peaks olema kohe esimene käsk pärast
dokumendi sisu algust (\verb|\begin{document}|). Ta vormistab
leheküljenumbrid rooma numbritega ning jätab jaotiste pealkirjadest
numbrid ära, nagu oleks kasutatud tärniga jaotisekäske (nt
\verb|\chapter*{Eessõna}|),
\index{käsk!tärniga}\index{tärniga käsk}
kuid pealkirjad ilmuvad siiski sisukorda.
\item[\normalfont\ci{mainmatter}] tuleb vahetult enne raamatu esimest
peatükki. Ta lülitab sisse lehekülgede araabia numbrid ja alustab
lehekülgede loenduri suurendamist uuesti algusest.
\item[\normalfont\ci{appendix}] märgib raamatus lisamaterjali algust.
Pärast seda käsku nummerdatakse peatükke tähtedega.
\item[\normalfont\ci{backmatter}] tuleks lisada enne raamatu kõige
viimaseid üksusi, nagu kirjandusnimestikku või aineregistrit.
Standardsetes dokumendiklassides sellel käsul visuaalset efekti pole.
\end{description}

\section{Ristviited}

Raamatutes, aruannetes ja artiklites esineb tihti
\index{viited}\index{ristviited}viiteid
joonistele, tabelitele ja teistele tekstiosadele. Viidete jaoks pakub
\LaTeX{} järgmisi käske:
\begin{lscommand}
\ci{label}\verb|{|\emph{märgend}\verb|}|, \ci{ref}\verb|{|\emph{märgend}\verb|}|
ja \ci{pageref}\verb|{|\emph{märgend}\verb|}|
\end{lscommand}
\noindent kus \emph{märgend} on kasutaja valitud identifikaator. \LaTeX{}
asendab käsu \ci{ref} selle jaotise, alajaotise, joonise, tabeli või
teoreemi numbriga, mille järel anti vastav käsk \ci{label}. Käsk
\ci{pageref} trükib selle lehekülje numbri, kus esines vastav
käsk \ci{label}.\footnote{Need käsud pole teadlikud sellest, millele
nad viitavad. Käsk \ci{label} ainult salvestab viimase automaatselt
genereeritud numbri.} Nagu sisukorras jaotiste pealkirjade ja
leheküljenumbrite puhul, kasutatakse siingi väärtusi eelmisest
kompileerimistsüklist.

\begin{example}
Viide sellele alajaotisele
\label{jaot:see} näeb välja nii:
"`Vaata jaotist~\ref{jaot:see}
leheküljel~\pageref{jaot:see}"'.
\end{example}

\section{Allmärkused}

Käsuga
\begin{lscommand}
\ci{footnote}\verb|{|\emph{allmärkuse tekst}\verb|}|
\end{lscommand}
\noindent trükitakse käesoleva lehekülje alaäärde allmärkus. Allmärkused
tuleks alati panna\footnote{\emph{Panema} on üks levinumaid
eestikeelseid sõnu.} selle sõna või lause järele, millele nad viitavad.
Lausele või selle osale viitavad allmärkused tuleks seega panna koma või
punkti järele.\footnote{Allmärkused juhivad lugeja tähelepanu dokumendi
põhitekstist kõrvale. Tegelikult ju kõik loevad allmärkusi -- me oleme
uudishimulikud, seega miks mitte integreerida kõik, mida soovime öelda,
dokumendi põhiteksti?\footnotemark}\footnotetext{Teeviit ei lähe alati
sinna, kuhu viitab \texttt{:-)}}

\begin{example}
Allmärkusi\footnote{See on
  allmärkus.} kirjutavad
\LaTeX i kasutajad sageli.
\end{example}

\section{Rõhutatud sõnad}

Kirjutusmasinaga kirjutatud tekstis on kombeks \texttt{rõhutada olulisi
sõnu \underline{allajoonimisega}.}%
\begin{lscommand}
\ci{underline}\verb|{|\emph{tekst}\verb|}|
\end{lscommand}
\noindent Kuid trükitud raamatutes rõhutatakse sõnu
\emph{kursiivkirjaga}. Autoril ei tohiks vahet olla. Tähtis on \LaTeX
ile ütelda, et see tükk teksti on oluline ja seda tuleks rõhutada. Seega
käsk
\begin{lscommand}
\ci{emph}\verb|{|\emph{tekst}\verb|}|
\end{lscommand}
\noindent rõhutab teksti. Mida see käsk oma argumendiga tegelikult teeb,
sõltub kontekstist:

\begin{example}
\emph{Kui rõhutamist kasutada
rõhutatud teksti sees, siis
rõhutab \LaTeX{} teksti
\emph{tavalise} kirja} abil.
\end{example}

Kes soovib suuremat kontrolli kirja ja kirjasuuruse üle, leiab mõningat
inspiratsiooni jaotisest \ref{sec:fontsize} leheküljel
\pageref{sec:fontsize}.

\section{Keskkonnad} \label{env}\index{keskkond}

% To typeset special purpose text, \LaTeX{} defines many different
% \wi{environment}s for all sorts of formatting:
\begin{lscommand}
\ci{begin}\verb|{|\emph{keskkond}\verb|}|\quad
   \emph{tekst}\quad
\ci{end}\verb|{|\emph{keskkond}\verb|}|
\end{lscommand}
\noindent kus \emph{keskkond} on keskkonna nimi. Keskkondi võib
paigutada üksteise sisse, kui järgida õiget sisestusjärjekorda.
\begin{code}
\verb|\begin{aaa}...\begin{bbb}...\end{bbb}...\end{aaa}|
\end{code}

\noindent Järgnevates jaotistes tutvustatakse kõiki olulisi keskkondi.

\subsection{Keskkonnad \ei{itemize}, \ei{enumerate} ja \ei{description}}

Keskkond \ei{itemize} sobib lihtsate loetelude jaoks, keskkond
\ei{enumerate} nummerdatud loetelude jaoks ja keskkond \ei{description}
kirjelduste jaoks. \cih{item}

\begin{example}
\flushleft
\begin{enumerate}
\item Keskkondi võib paigutada
soovi järgi üksteise sisse.
\begin{itemize}
\item Kuid see võib paista
naljakas.
\item[-] Kriipsuga.
\end{itemize}
\item Seetõttu pea meeles:
\begin{description}
\item[rumalad] asjad ei muutu
targaks sellest, et nad on
loetelus;
\item[targad] asjad saab aga
kenasti esitada loetelus.
\end{description}
\end{enumerate}
\end{example}

\subsection{Keskkonnad \ei{flushleft}, \ei{flushright} ja \ei{center}}

Keskkonnad \ei{flushleft} ja \ei{flushright} moodustavad vastavalt
vasakule ja \wi{paremale joondatud} lõigud. \index{vasakule joondatud}
Keskkond \ei{center} moodustab tsentreeritud teksti. Kui pole määratud
reamurdmisi käskudega \ci{\bs}, siis valib \LaTeX{} reamurdmised
automaatselt.

\begin{example}
\begin{flushleft}
See tekst on\\ joondatud vasakule.
\LaTeX{} ei püüa teha
iga rida sama pikaks.
\end{flushleft}
\end{example}

\begin{example}
\begin{flushright}
See tekst on joondatud\\paremale.
\LaTeX{} ei püüa teha
iga rida sama pikaks.
\end{flushright}
\end{example}

\begin{example}
\begin{center}
Maailma\\keskpunktis.
\end{center}
\end{example}

\subsection{Keskkonnad \ei{quote}, \ei{quotation} ja \ei{verse}}

Keskkond \ei{quote} on kasulik tsitaatide, oluliste fraaside ja näidete
puhul.

\begin{example}
Tüpograafiline rusikareegel
tekstirea pikkuse jaoks on:
\begin{quote}
keskmiselt ei tohiks rida
olla pikem kui 66 sümbolit.
\end{quote}
See on põhjus, miks \LaTeX i
lehekülgedel on vaikimisi nii
laiad ääred ja miks ajalehti
trükitakse mitmeveeruliselt.
\end{example}

On olemas veel kaks sarnast keskkonda: \ei{quotation} ja \ei{verse}.
Keskkond \ei{quotation} sobib pikemate, mitmelõiguliste tsitaatide
jaoks, sest ta lisab iga lõigu esimesele reale taande. Keskkond
\ei{verse} sobib luuletuste jaoks, kus olulised on reapiirid.
Ridu murtakse käskudega \ci{\bs} ridade lõpus ja tühja reaga pärast iga
salmi.

\begin{example}
Ma tean peast ainult ühte
eestikeelset luuletust.
See räägib hanepoegadest.
\begin{flushleft}
\begin{verse}
Lumi tuli maha ja
valgeks läks maa,\\
kaks väikest hanepoega nüüd
välja ei saa.\\
Nad istuvad laudas, mis
teha, on talv\\
ja paljajalu käia on
lume peal halb.
\end{verse}
\end{flushleft}
\end{example}

\subsection{Sisukokkuvõte}

Teaduslikke publikatsioone alustatakse tavaliselt sisukokkuvõttega, mis
annab lugejale lühikese ülevaate, mida oodata. Selleks on \LaTeX is
olemas keskkond \ei{abstract}. Enamasti kasutatakse keskkonda
\ei{abstract} dokumentides, mille aluseks on artikli dokumendiklass.

\newenvironment{abstract}%
        {\begin{center}\begin{small}\begin{minipage}{0.8\textwidth}}%
        {\end{minipage}\end{small}\end{center}}
\begin{example}
\begin{abstract}
Kokkuvõttev kokkuvõte.
\end{abstract}
\end{example}

\subsection{Tähttäheline trükk}

Tekst, mis asub käskude \verb|\begin{|\ei{verbatim}\verb|}| ja
\verb|\end{verbatim}| vahel, trükitakse otse, nii nagu oleks ta
sisestatud kirjutusmasinal, koos kõigi reavahetuste ja tühikutega ning
ilma ühtegi \LaTeX i käsku täitmata.

Lõigu sees võib sama tulemuse saada käsuga
\begin{lscommand}
\ci{verb}\verb|+|\emph{tekst}\verb|+|
\end{lscommand}
\noindent Märk \verb|+| on lihtsalt eraldaja. Võib kasutada ükskõik
millist märki, välja arvatud tähed, \verb|*| ja tühik. Käesolevas
raamatukeses on paljud \LaTeX i näited vormistatud selle käsu abil.

\begin{example}
Käsk \verb|\ldots| \ldots

\begin{verbatim}
10 PRINT "TERE HOMMIKUST";
20 GOTO 10
\end{verbatim}
\end{example}

\begin{example}
\begin{verbatim*}
Keskkonna      verbatim
tärniga versioon
rõhutab    tekstis  tühikuid
\end{verbatim*}
\end{example}
\index{keskkond!tärniga}\index{tärniga keskkond}

Samamoodi võib koos tärniga kasutada käsku
\ci{verb}:\index{käsk!tärniga}\index{tärniga käsk}

\begin{example}
\verb*|nagu   niimoodi :-) |
\end{example}

Keskkond \ei{verbatim} ja käsk \verb|\verb| ei või asuda teiste
käskude argumentides.

\subsection{Keskkond \ei{tabular}}
\index{tabelid}

% \newcommand{\mfr}[1]{\framebox{\rule{0pt}{0.7em}\texttt{#1}}}
\newcommand{\mfr}[1]{\texttt{#1}}

Keskkonna \ei{tabular} abil saab vormistada keni tabeleid
valikuliste horisontaal- ja vertikaaljoontega. Veergude laiused valib
\LaTeX{} automaatselt.

Käsu
\begin{lscommand}
\verb|\begin{tabular}[|\emph{pos}\verb|]{|\emph{veerujoondused}\verb|}|
\end{lscommand}
\noindent argument \emph{veerujoondused} määrab tabeli vormi. Vasakule
joondatud veergu tähistab \mfr{l}, paremale joondatud veergu \mfr{r} ja
tsentreeritud veergu \mfr{c}; rajastatud ja murtavate ridadega teksti
sisaldavat veergu märgib \mfr{p\{}\emph{laius}\mfr{\}} ning
vertikaaljoont \mfr{|}.

Kui veeru tekst on lehekülje jaoks liiga lai, siis \LaTeX{} seda
automaatselt ei murra. Argumendiga \mfr{p\{}\emph{laius}\mfr{\}} saab
defineerida spetsiaalset liiki veeru, milles teksti murtakse nii, nagu
harilikus lõigus.

Argument \emph{pos} määratleb tabeli vertikaalse asendi ümbritseva
teksti alusjoone suhtes. Tähed \mfr{t}, \mfr{b} ja \mfr{c} suunavad
tabelit joonduma vastavalt üles, alla ja keskele.

Keskkonna \ei{tabular} sees tähistab \index{&@\texttt{\&}}\texttt{\&}
hüpet järgmisse veergu, \ci{\bs} uue rea algust ja \ci{hline}
horisontaaljoont. Osalisi jooni saab lisada käsuga
\ci{cline}\texttt{\{}$i$\texttt{-}$j$\texttt{\}}, kus $i$ ja $j$ on
veergude numbrid, üle mille joon ulatuma peab.

\begin{example}
\begin{tabular}{|r|l|}
\hline
7C0 & heksadetsimaalne\\
3700 & oktaalne \\ \cline{2-2}
11111000000 & binaarne \\
\hline \hline
1984 & detsimaalne\\
\hline
\end{tabular}
\end{example}

\begin{example}
\begin{tabular}{|p{4.7cm}|}
\hline
Tere tulemast kandilisse lõiku!
Loodan südamest, et te kõik
naudite etendust.\\
\hline
\end{tabular}
\end{example}

Veergude eraldaja võib määrata konstruktsiooniga
\mfr{@\{...\}}, mis tühistab senise veergudevahelise ruumi ja asendab
selle looksulgudes oleva materjaliga. Ühte selle käsu kasutusvõimalust
tutvustatakse allpool kümnendmurdude joondamise probleemi juures. Teine
võimalik rakendus on keelata käsuga \mfr{@\{\}} ära tabelit ümbritsevad
horisontaaltühikud.\index{tühik!tabeli ümber}

\begin{example}
\begin{tabular}{@{} l @{}}
\hline
ümbritsevaid tühikuid pole\\
\hline
\end{tabular}
\end{example}

\begin{example}
\begin{tabular}{l}
\hline
ümbritsevad tühikud
vasakul ja paremal\\
\hline
\end{tabular}
\end{example}

%
% This part by Mike Ressler
%

\index{kümnendmurdude joondamine} Kuna pole olemas sisseehitatud
võimalust joondada arvude veerge kümnendkoma järgi,\footnote{Kui
süsteemis on installitud paketikomplekt Tools, siis tasub vaadata
paketti \pai{dcolumn}.} siis võime sellest piirangust "`mööda hiilida"'
nii, et vormistame arvud kahes veerus: paremalt rajastatud täisosa ning
vasakult rajastatud murdosa. Reas \verb|\begin{tabular}| asendab
spetsifikaator \verb|@{,}| tavalise veergudevahelise ruumi märgiga
, jättes sedasi mulje ühest kümnendkoma järgi joondatud veerust.
Mitte unustada asendada arvudes kümnendkoma veergude eraldajaga
\verb|&|\,! Veerusildi saab arvude "`veeru"' kohale panna käsuga
\ci{multicolumn}.

\begin{example}
\begin{tabular}{c r @{,} l}
Piiavaldis          &
\multicolumn{2}{c}{Väärtus} \\
\hline
$\pi$               & 3&1416  \\
$\pi^{\pi}$         & 36&46   \\
$(\pi^{\pi})^{\pi}$ & 80662&7 \\
\end{tabular}
\end{example}

\begin{example}
\begin{tabular}{|c|c|}
\hline
\multicolumn{2}{|c|}{Trips} \\
\hline
Traps & Trull! \\
\hline
\end{tabular}
\end{example}

Keskkonnas \ei{tabular} vormistatud materjal jääb alati kokku ühele
leheküljele. Kui on vaja trükkida pikki tabeleid, siis saab seda teha
paketiga \pai{longtable}.

Mõnikord tunduvad \LaTeX i standardtabelid veidi liiga kokkusurutud.
Hingamisruumi juurdeandmiseks tuleks muuta parameetrite
\ci{arraystretch} ja \ci{tabcolsep} väärtused suuremaks.

\begin{example}
\begin{tabular}{|l|}
\hline
Need read\\\hline
on tihedalt\\\hline
\end{tabular}

{\renewcommand{\arraystretch}{1.5}
\renewcommand{\tabcolsep}{0.2cm}
\begin{tabular}{|l|}
\hline
natuke avaram\\\hline
tabeli kujundus\\\hline
\end{tabular}}
\end{example}

Kui on tarvis suurendada tabelis ainult ühe rea kõrgust, võib sobivasse
kohta lisada nähtamatu vertikaalkasti\footnote{Professionaalses
ladumises on selle nimi \index{strut@\textit{strut}}\emph{strut}.}. Selle triki saab
realiseerida käsuga \ci{rule}, võttes laiuseks nulli.

\begin{example}
\begin{tabular}{|c|}
\hline
\rule{1pt}{4ex}Props \ldots\\
\hline
\rule{0pt}{4ex}Tugi\\
\hline
\end{tabular}
\end{example}

Selles näites on \texttt{pt} ja \texttt{ex} \TeX i ühikud, mille kohta
leiab rohkem infot tabelist \ref{units} leheküljel \pageref{units}.

Paketis \pai{booktabs} on saadaval mõningad lisakäsud, mis
tabelikeskkonda laiendavad. Need muudavad professionaalse
väljanägemisega korrektsete vahedega tabelite loomise märksa lihtsamaks.

\section{Ujuvad elemendid}

Tänapäeval sisaldab enamik publikatsioone palju jooniseid ja tabeleid.
Need elemendid nõuavad erikohtlemist, sest neid ei saa murda üle
leheküljepiiride. Üks meetod oleks alustada iga kord, kui joonis või
tabel leheküljele ei mahu, uut lehekülge. Selline lähenemine jätab aga
leheküljed osaliselt tühjaks, mis näeb inetu välja.

Probleemi lahendus on lasta iga joonis või tabel, mis jooksvale
leheküljele ei mahu, "`ujuma"' hilisemale leheküljele, täites jooksva
lehekülje selle asemel põhitekstiga. \LaTeX is on
\index{ujuvelemendid}ujuvate elementide loomiseks kaks keskkonda, üks
tabelite ja teine jooniste jaoks. Et neist kahest keskkonnast täit kasu
saada, on oluline üldjoontes mõista, kuidas \LaTeX{} ujuvat materjali
sisemiselt käsitleb. Vastasel korral võivad ujuvelemendid muutuda
suureks frustratsiooni allikaks, sest \LaTeX{} ei pane neid kunagi
sinna, kus autor neid näha soovib.\index{tabelid}\index{joonised}

Vaatleme esmalt käske, mida \LaTeX{} ujuvelementide jaoks pakub.
Igasugust materjali, mis asub keskkonnas \ei{figure} või \ei{table},
käsitletakse ujuva materjalina. Mõlemal ujuval keskkonnal on valikuline
argument
\begin{lscommand}
\verb|\begin{figure}[|\emph{paigutuse spetsifikaator}\verb|]| või
\verb|\begin{table}[|\ldots\verb|]|
\end{lscommand}
\noindent nimega \emph{paigutuse spetsifikaator}, mille kaudu
antakse \LaTeX ile teada asukohad, kuhu ujuvelementi on lubatud
teisaldada. Paigutuse spetsifikaator konstrueeritakse \emph{ujuvelemendi
paigutusõiguste} järjendina, vt tabelit~\ref{tab:permiss}.

\begin{table}[tbp]
\caption{Ujuvelemendi paigutusõigused}\label{tab:permiss}
\noindent \begin{minipage}{\textwidth}
\medskip
\begin{center}
\begin{tabular}{@{}cp{8cm}@{}}
Spets.&Õigus paigutada ujuvelementi \ldots\\
\hline
\rule{0pt}{1.05em}\texttt{h} & \emph{Siia}, samale kohale tekstis, kus
ta esineb. See sobib enamasti väiksemate elementide puhul.\\[0.3ex]
\texttt{t} & Lehekülje \emph{ülaäärde}.\\[0.3ex]
\texttt{b} & Lehekülje \emph{alaäärde}.\\[0.3ex]
\texttt{p} & Eraldi \emph{leheküljele}, mis koosneb ainult
ujuv\-ele\-men\-ti\-dest.\\[0.3ex]
\texttt{!} & Arvestamata enamikku sisemisi parameetreid\footnote{Nagu
näiteks ühel leheküljel lubatud ujuvelementide maksimaalarv.}, mis
võivad muidu selle elemendi paigutamise välistada.
\end{tabular}
\end{center}
\end{minipage}
\end{table}

Näiteks võib tabelit alustada järgmise reaga
\begin{code}
\verb|\begin{table}[!hbp]|
\end{code}
\noindent \index{paigutuse spetsifikaator}Paigutuse spetsifikaator \verb|[!hbp]| lubab \LaTeX il
paigutada tabeli otse siia (\texttt{h}) või mõne lehekülje alaäärde
(\texttt{b}) või eraldi ujuvelementide leheküljele (\texttt{p}), ja
kõike seda ka juhul, kui tulemus ei paista välja väga hea (\texttt{!}).
Kui paigutuse spetsifikaator on määramata, siis võetakse
standardklassides selleks \verb|[tbp]|.

\LaTeX{} paigutab iga ujuvelemendi, mida ta kohtab, vastavalt autori
määratud paigutuse spetsifikaatorile. Kui elementi ei saa paigutada
jooksvale leheküljele, siis lisatakse ta kas \emph{jooniste} järjekorda
või \emph{tabelite} järjekorda.\footnote{Need on FIFO-järjekorrad
(esimesena sisse, esimesena välja)!} Kui algab uus lehekülg, siis
kontrollib \LaTeX{} kõigepealt, kas on võimalik luua järjekorras
olevatest elementidest omaette ujuvelementide lehekülg. Kui see pole
võimalik, siis vaadeldakse kummagi järjekorra esimest elementi nii, nagu
oleks see just tekstis esinenud: \LaTeX{} püüab teda uuesti paigutada
vastavalt elemendi paigutuse spetsifikaatorile (välja arvatud
\texttt{h}, mis pole enam võimalik). Kõik uued tekstis ettetulevad
ujuvelemendid lisatakse vastavatesse järjekordadesse. \LaTeX{} säilitab
rangelt kumbagi tüüpi ujuvelementide esialgse järjestuse. Seepärast
lükkab joonis, mida pole võimalik ära paigutada, kõik edasised joonised
dokumendi lõppu. Niisiis:

\begin{quote}
Kui \LaTeX{} ei paiguta ujuvelemente soovitud viisil, siis on sageli
põhjuseks üks ujuvelement, mis on ummistanud emma-kumma ujuvelementide
järjekorra.
\end{quote}

Kuigi \LaTeX ile on võimalik ette anda üksainus paigutuse
spetsifikaator, põhjustab see mõnikord probleeme. Kui ujuvelement
sellesse kohta ei mahu, jääb ta järjekorda kinni ja hakkab järgmisi
elemente blokeerima. Sealhulgas ei tohiks mitte kunagi kasutada üksinda
spetsifikaatorit \verb|h| -- see on nii halb, et \LaTeX i hilisemad
versioonid võtavad selle asemele automaatselt \verb|ht|.

Olles nüüd ära selgitanud keerulise osa, on veel mõned asjad, mida tasub
keskkondade \ei{table} ja \ei{figure} puhul mainida. Ujuvelemendi
pealkiri määratakse käsuga

\begin{lscommand}
\ci{caption}\verb|{|\emph{pealkirja tekst}\verb|}|
\end{lscommand}
\noindent \LaTeX{} lisab jooksva numbri koos sõnaga "`Joonis"' või
"`Tabel"'.

Käsud
\begin{lscommand}
\ci{listoffigures} ja \ci{listoftables}
\end{lscommand}
\noindent tegutsevad sarnaselt käsuga \verb|\tableofcontents|, trükkides
vastavalt jooniste ja tabelite loetelu. Neisse lähevad pealkirjad terves
mahus, nii et kui pealkirjad kipuvad olema pikad, tuleks moodustada
neist loetelude jaoks lühemad versioonid. Lühiversioon lisatakse käsu
\verb|\caption| järele nurksulgudesse.
\begin{code}
\verb|\caption[Lühi]{PPPPPiiiiiikkkkkkkkkk}|
\end{code}

Käskudega \ci{label} ja \ci{ref} saab luua tekstis ujuvelemendile viite.
Seejuures tuleb käsk \ci{label} panna käsu \ci{caption} \emph{järele},
sest viidata on vaja pealkirja numbrile.

Järgmine näide joonistab ruudu ja lisab selle dokumenti. Niimoodi saab
reserveerida ruumi jooniste jaoks, mis paigutatakse lõppdokumenti
hiljem.

\begin{code}
\begin{verbatim}
Joonis~\ref{valge} on näide pop-kunstist.
\begin{figure}[!hbtp]
\makebox[\textwidth]{\framebox[5cm]{\rule{0pt}{5cm}}}
\caption{Viis korda viis sentimeetrit\label{valge}}
\end{figure}
\end{verbatim}
\end{code}

\noindent Selles näites püüab \LaTeX{} \emph{tõesti
kõvasti}~(\texttt{!}) panna joonist otse
\emph{siia}~(\texttt{h}).\footnote{Eeldades, et jooniste järjekord on
tühi.} Kui see pole võimalik, püüab ta panna joonist lehekülje
\emph{alaäärde}~(\texttt{b}). Kui joonist ei õnnestu panna jooksvale
leheküljele, siis vaatab \LaTeX, kas on võimalik luua eraldi
ujuvelementide lehekülg, mis sisaldaks seda joonist ja võib-olla
mõningaid tabeleid tabelite järjekorrast. Kui ujuvelementide lehekülje
jaoks ei ole piisavalt materjali, alustab \LaTeX{} uut lehekülge ja
käsitleb uuesti joonist nii, nagu see oleks just tekstis ette tulnud.

Mõnes olukorras võib olla vaja anda käsk

\begin{lscommand}
\ci{clearpage} või isegi \ci{cleardoublepage}
\end{lscommand}
\noindent See käsib \LaTeX il paigutada kohe ära kõik järjekordadesse
kogunenud ujuv"-elemendid ja seejärel alustada uut lehekülge. Käsk
\ci{cleardoublepage} läheb isegi uuele
paremleheküljele.\enlargethispage{\baselineskip}

Hiljem õpetatakse käesolevas sissejuhatuses, kuidas lisada dokumenti \PSi
i jooniseid.

\section{Habraste käskude kaitsmine}

Käskude nagu \ci{caption} ja \ci{section} argumentides antud tekst võib
esineda dokumendis mitmes kohas (nt nii sisukorras kui ka dokumendi
põhitekstis). Mõned käsud lakkavad töötamast, kui nad panna
jaotisekäskude \ci{section} taoliste käskude argumentidesse, ja
dokumenti kompileerida ei õnnestu. Sellised käske nimetatakse
\index{haprad käsud}habrasteks käskudeks -- niisugused on näiteks
\ci{footnote} ja \ci{phantom}. Haprad käsud vajavad kaitsmist (kas seda
ei vaja me kõik?). Kaitsmiseks tuleks nende ette lisada käsk
\ci{protect}. Selliselt töötavad need käsud õigesti isegi siis, kui nad
esinevad liikuvates argumentides.

Käsk \ci{protect} mõjutab ainult järgmist käsku ja isegi mitte selle
argumente. Liigne \ci{protect} enamikul juhtudel probleeme ei tekita.

\begin{code}
\verb|\section{Ma olen hooliv|\\
\verb|      \protect\footnote{ja kaitsen oma allmärkusi}}|
\end{code}

% Local Variables:
% TeX-master: "lshort2e"
% mode: latex
% mode: flyspell
% End:

