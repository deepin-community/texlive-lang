%%%%%%%%%%%%%%%%%%%%%%%%%%%%%%%%%%%%%%%%%%%%%%%%%%%%%%%%%%%%%%%%%
% Contents: Customising LaTeX output
% $Id: custom.tex,v 1.2 2003/03/19 20:57:45 oetiker Exp $
%%%%%%%%%%%%%%%%%%%%%%%%%%%%%%%%%%%%%%%%%%%%%%%%%%%%%%%%%%%%%%%%%
\chapter{Prilagajanje \LaTeX{}a}

\begin{intro}
Dokumenti narejeni z ukazi, ki smo jih spoznali do tega trenutka, bodo sprejemljivi za
večji del občinstva. Kljub temu da njihov videz ni razkošen, se držijo vseh uveljavljenih 
pravil dobrega stavljenja, kar jih naredi lahko berljive in prijetnega videza.

Toda, obstajajo situacije v katerih \LaTeX{} nima na voljo ukaza ali okolja,
ki ustreza našim potrebam, ali pa rezultat dobljen z obstoječimi ukazi ne
izpolni naših želja.

V tem poglavju bomo dali nekaj namigov, kako lahko \LaTeX{} naučimo novih trikov in kako
lahko naredimo dokumente z drugačnim videzom od običajno privzetega.
\end{intro}

\section{Novi ukazi, okolja in paketi}

Verjetno ste že opazili, da so vsi ukazi, ko so prvič predstavljeni v knjigi, 
uokvirjeni, hkrati pa se pojavijo tudi v stvarnem kazalu na koncu knjige. 
Namesto direktne uporabe potrebnih \LaTeX{}ovih ukazov, s katerimi se da to doseči,
sem napisal \wi{paket}, v katerem so definirani novi ukazi in okolja za ta namen. 
Tako lahko preprosto napišemo:

\begin{example}
\begin{lscommand}
\ci{dum}
\end{lscommand}
\end{example}

V tem zgledu smo uporabili novo okolje \ei{lscommand}, ki nariše okvir okrog 
ukaza in nov ukaz \ci{ci}, ki zapiše ime ukaza, hkrati pa ustrezni podatek 
vstavi še v stvarno kazalo. To lahko preverite tako, da v stvarnem kazalu na 
koncu knjige poiščete geslo \ci{dum}.
Tam boste pri \ci{dum} našli referenco na vsako stran, kjer je v tekstu ukaz 
\ci{dum}.

Če se odločimo, da nočemo več imeti novih ukazov v okvirjih,
lahko preprosto popravimo definicijo okolja \texttt{lscommand} in naredimo nov videz. 
To je veliko enostavneje kot pa v celem dokumentu popravljati ustrezen tekst
na vseh mestih, kjer se direktno uporabljajo \LaTeX{}ovi ukazi za risanje okvirja okrog
besede.

\subsection{Novi ukazi}

Svoje nove ukaze lahko definiramo z ukazom 
\begin{lscommand}
\ci{newcommand}\verb|{|%
       \emph{ime}\verb|}[|\emph{num}\verb|]{|\emph{definicija}\verb|}|
\end{lscommand}
\noindent Ukaz potrebuje dva argumenta: \emph{ime} je ime ukaza, ki ga 
želimo definirati, \emph{definicija} pa je opis tega, kar želimo, da se izvede, zapisan 
z ustreznimi \LaTeX{} ukazi. Argument \emph{num} v oglatih oklepajih je neobvezen in 
določa število argumentov, ki jih potrebuje naš novi ukaz (maksimalno možno število parametrov je 9).
Če tega argumenta ni, se privzame vrednost $0$, kar pomeni, da gre za ukaz brez 
argumentov.

Naslednja dva zgleda bosta zadevo še bolj razjasnila. V prvem zgledu
definiramo nov ukaz z imenom \ci{xvec}. Ta ukaz nam pride prav vsakič ko je potrebno 
izpisati vektor $x_1,\ldots,x_n$. Namesto z \verb|$x_1,\ldots,x_n$| lahko sedaj
to naredimo z \verb|$\xvec$|.

\begin{example}
\newcommand{\xvec}{x_1,\ldots,x_n}
Vektor $\xvec$ \ldots
\end{example}

Naslednji zgled prikazuje, kako definiramo ukaz, ki potrebuje en argument. 
Denimo, da v tekstu poleg $x_1,\ldots,x_n$ večkrat potrebujemo tudi 
$y_1,\ldots,y_n$. Namesto definicije \verb|\yvec| definiramo splošni ukaz 
z enim argumentom. Značka \verb|#1| se zamenja z argumentom, ki ga podamo v zavitih oklepajih.
Če definiramo ukaz z več kot enim argumentom, potem je naslednji označen z \verb|#2| in tako
naprej.

\begin{example}
% v preambuli: 
\newcommand{\lvec}[1]
 {#1_1,\ldots,#1_n}
% v telesu dokumenta: 
Skalarni produkt vektorjev
$\lvec{x}$ in $\lvec{y}$ je 
\ldots
\end{example}

\LaTeX{} ne dovoli da definiramo nov ukaz, ki bi povozil že obstoječega. Če to na vsak način želimo
narediti, imamo na voljo ukaz \ci{renewcommand}.
Način uporabe je povsem enak kot pri ukazu \verb|\newcommand|.

V nekaterih primerih pride v poštev tudi ukaz \ci{providecommand}.
Deluje podobno kot \ci{newcommand}, toda če ukaz že obstaja, potem \LaTeXe{} tiho ignorira ukaz z definicijo 
in ohrani stari ukaz.

Tu se splača še enkrat spomniti, kako je s presledki za \LaTeX{} ukazi. Če se ne spomnite več, poglejte
na stran \pageref{whitespace}.


\subsection{Nova okolja}
Podobno kot ukaz \verb|\newcommand|, obstaja tudi ukaz, s katerim lahko definiramo 
nova okolja. Ukaz \ci{newenvironment} ima naslednjo obliko:

\begin{lscommand}
\ci{newenvironment}\verb|{|%
       \emph{ime}\verb|}[|\emph{num}\verb|]{|%
       \emph{preden}\verb|}{|\emph{potem}\verb|}|
\end{lscommand}

Podobno kot pri ukazu \verb|\newcommand|, lahko \ci{newenvironment} uporabljamo  z neobveznim argumentom
ali pa brez njega. To, kar navedemo v argumentu
\emph{preden}, se izvede pred procesiranjem teksta v okolju, vsebina \emph{potem} pa se procesira takrat,
ko srečamo ukaz \verb|\end{|\emph{ime}\verb|}|.

Spodnji primer prikazuje uporabo ukaza \ci{newenvironment}. 
\begin{example}
\newenvironment{kralj}
 {\rule{1ex}{1ex}%
      \hspace{\stretch{1}}}
 {\hspace{\stretch{1}}%
      \rule{1ex}{1ex}}

\begin{kralj} 
Moji skromni podložniki \ldots
\end{kralj}
\end{example}

Argument \emph{num} ima podoben pomen kot pri ukazu \verb|\newcommand|. 
\LaTeX{} poskrbi za to, da ne moremo definirati okolja, ki že obstaja. Če želimo spremeniti
kakšno že obstoječe okolje, uporabimo ukaz \ci{renewenvironment}, ki ima enak način uporabe kot 
ukaz \ci{newenvironment}.

Ukazi, uporabljeni v zgornjem primeru, bodo razloženi v nadaljevanju.
Za ukaz \ci{rule} poglejte stran \pageref{sec:rule}, za \ci{stretch} stran \pageref{cmd:stretch},
več informacij o \ci{hspace} pa dobite na strani \pageref{sec:hspace}.

\subsection{Presledki na začetku in po koncu okolja}

Ko kreiramo novo okolje, se nam lahko zgodi, da se na začetku ali na koncu brez 
kakšnega vidnega vzroka pojavijo
nezaželeni presledki. Denimo, da želimo definirati okolje za naslov, 
ki ne bo zamaknjen, prav tako pa ne bo zamaknjen prvi odstavek za 
naslovom. Ukaz \ci{ignorespaces} na začetku definicije okolja povzroči,
da se ignorirajo vsi morebitni presledki za ukazom 
\verb|\begin{|\emph{okolje}\verb|}|. Še bolj
zapleteno je poskrbeti za dogajanje po koncu bloka, saj na koncu okolja pride
do posebne obdelave. Z ukazom 
\ci{ignorespacesafterend} lahko \LaTeX{}u naročimo, naj za ustreznim
\verb|\end{|\emph{okolje}\verb|}| ukazom, ko bo posebne obdelave konec, sproži še ukaz 
\ci{ignorespaces}.

\begin{example}
\newenvironment{enostavno}%
 {\noindent}%
 {\par\noindent}

\begin{enostavno}
Pozor na presledek\\na levi strani.
\end{enostavno}
Enako\\tukaj.
\end{example}

\begin{example}
\newenvironment{pravilno}%
 {\noindent\ignorespaces}%
 {\par\noindent%
   \ignorespacesafterend}

\begin{pravilno}
Ni presledka\\na levi strani.
\end{pravilno}
Enako\\tukaj.
\end{example}

\subsection{Poganjanje \LaTeX{a} iz ukazne vrstice}

Če delate na operacijskem sistemu Unix oz.~podobnem, lahko za prevajanje
\LaTeX{} projektov uporabljate make datoteke. V tem primeru zna biti za
vas zanimivo, da lahko z uporabo dodatnih parametrov pri klicu prevajalnika
iz istega dokumenta dobite različne različice. Npr.~če v dokument dodamo
naslednjo strukturo:

\begin{verbatim}
\usepackage{ifthen}
\ifthenelse{\equal{\crnobelo}{true}}{
  % "črno bel" način; naredi nekaj..
}{
  % "barvni" način; naredi nekaj drugega..
}
\end{verbatim}

lahko potem dokument prevajamo iz ukazne vrstice z:
\begin{verbatim}
latex '\newcommand{\crnobelo}{true}

%English version
\newcommand{\TFRGB}[2]{#2} % #1 en francais #2 in english

\newcommand{\dft}{By default}
\newcommand{\SSCT}[2]{ \section{#2}}
\newcommand{\SbSSCT}[2]{ \subsection{#2}}
\newcommand{\Par}[2]{ \paragraph{#2}}
\newcommand{\SbSbSSCT}[2]{ \subsubsection{#2}}
\newcommand{\SSCTTC}[4]{ \section[#2]{#2}}
\newcommand{\SbSSCTTC}[4]{ \subsection[#2]{#2}}
\newcommand{\SbSbSSCTTC}[4]{ \subsubsection[#3]{#4}}

\newcommand{\maboite}[1]{\begin{center}  \tikz \draw node[draw,fill=yellow!20,inner sep=0.2cm,text centered,text width=.75\linewidth] {Load package : #1} ; \end{center} }

%\newcommand{\DW}[1]{\begin{center}  \tikz \draw node[draw,fill=red!50,inner sep=0.1cm,text centered] {\tiny don't work !} ; \end{center} }

\newcommand{\DW}[1]{\tikz[baseline=-1mm]  \draw node[draw,fill=red!50] {{\tiny don't work !}}; \index{\textbf{6 list of don't work }}} % for english version
%% Version française

\newcommand{\TFRGB}[2]{#1} % #1 en francais #2 in english
\newcommand{\dft}{Par défaut : }
\newcommand{\SSCT}[2]{\section{#1}}
\newcommand{\SbSSCT}[2]{\subsection{#1}}
\newcommand{\Par}[2]{ \paragraph{#1}}
\newcommand{\SbSbSSCT}[2]{\subsubsection{#1}}
\newcommand{\SSCTTC}[4]{\section[#1]{#2}}
\newcommand{\SbSSCTTC}[4]{\subsection[#1]{#2}}
\newcommand{\SbSbSSCTTC}[4]{\subsubsection[#1]{#2}}



\newcommand{\maboite}[1]{\begin{center}  \tikz \draw node[draw,fill=yellow!20,inner sep=0.2cm,text centered,text width=.75\linewidth] {Charger l'extension: #1} ; \end{center} }


\newcommand{\DW}[1]{\tikz[baseline=-1mm]  \draw node[draw,fill=red!50] {{\tiny ne fonctionne pas !}}; \index{\textbf{6 liste des non-fonctionnels }}} % pour la version française

 \documentclass[a4paper,10pt]{article}

 \usepackage{fontspec}
\usepackage[french,english]{babel}

%\TFRGB{\selectlanguage{french}}{\selectlanguage{english}}
\usepackage{tikzpeople}

\usepackage{amsmath,amsfonts,amssymb}

\usepackage{pdfpages}  


%\usepackage{pst-all}

\usepackage{graphicx} 
\usepackage{hyperref}

\usepackage{animate}
\usepackage{makeidx}
%\usepackage{wrapfig}
%\usepackage{tikz-dependency}
\usepackage{pgfplots} %<<<<<<<<<<<<<<<<<<<<<<<<<<<<< 
\usepackage{tikz}
\usepackage{tkz-tab}

 
%\usepgflibrary{shapes.callouts}
\usepackage{tikz-qtree}
\usepackage{tkz-tab}
\usepackage{csquotes}

  
\usetikzlibrary{angles}
\usetikzlibrary{arrows}

\usetikzlibrary{shadings}
\usetikzlibrary{calc}
\usetikzlibrary{backgrounds}
\usetikzlibrary{decorations.pathmorphing}

\usetikzlibrary{decorations.markings}
\usetikzlibrary{decorations.footprints}
\usetikzlibrary{decorations.shapes}
\usetikzlibrary{decorations.text}
\usetikzlibrary{decorations.fractals}
\usepgflibrary{shapes.geometric}
\usetikzlibrary{intersections}
\usetikzlibrary{scopes}
\usetikzlibrary{shapes.symbols}
\usetikzlibrary{shapes.arrows}
\usetikzlibrary{shapes.callouts}
\usetikzlibrary{shapes.misc}
\usepgflibrary{shapes.multipart}
\usetikzlibrary{plotmarks}
\usetikzlibrary{trees}
\usetikzlibrary{fadings}
\usetikzlibrary{arrows.meta}
\usetikzlibrary{bending}
\usetikzlibrary{fit}
%\usetikzlibrary{circuits}
\usetikzlibrary{circuits.ee.IEC}
\usetikzlibrary{circuits.logic.IEC}
\usetikzlibrary[circuits.logic.US]
\usetikzlibrary{circuits.logic.CDH}
%\usetikzlibrary{decorations}
\usetikzlibrary{shapes.gates.logic.IEC}
\usetikzlibrary{matrix}
\usetikzlibrary{chains}
%\usetikzlibrary{circuit.plc.sfc}
\usepackage{tikzsymbols}
\usetikzlibrary{datavisualization}
\usetikzlibrary{datavisualization.formats.functions}
%
\usepackage{tikzducks}

\usepackage{tikzrput}
\usepackage{pgfornament}

%\usetikzlibrary{babel}
\usetikzlibrary{math}
\usetikzlibrary{optics}
\usetikzlibrary{through}
\usetikzlibrary{turtle}
\usetikzlibrary{quotes}


\pgfplotsset{compat=1.8}
\usetikzlibrary{positioning}

\usepackage{geometry}
\geometry{a4paper,top={3cm}}

\usepackage{ifpdf}
\usepackage{ifluatex}
\usetikzlibrary{spy}




%====================================================================

\makeindex

\newcommand{\AC}[1]{\{#1\}}

\newcommand{\BS}[1]{$\backslash$#1}

\newcommand{\BSB}[1]{\textbf{\color{blue} {$\backslash$#1}}}


\newcommand{\BSR}[1]{\textbf{\color{red}  $\backslash$#1}}

%\newcommand{\RDDX}[2]{{\color{red}#1} \index{\textbf{3 Paramètres et options}!#2=#1}}


\newcommand{\RRR}[1]{\tikz[baseline=-1mm,inner sep=2pt]  \draw node[draw,fill=red!20] {{\footnotesize  PGFmanual section :  #1}} ; }

\newcommand{\RRP}[1]{\tikz[baseline=-1mm]  \draw node[draw,fill=red!20] {{\footnotesize  pgfplots section :  #1}} ; }

%\newcommand{\RRR}[1]{\tikz[baseline=-1mm]  \draw node[draw,fill=red!20] {{\footnotesize  PGFmanual section :  #1}} ;\index{\textbf{5 PGFmanual }!#1} }

\newcommand{\DFR}{ \tikzpicture[scale=.25]
\draw [fill=blue](0,0) rectangle (3,1.5);
\draw [fill=white](1,0) rectangle (2,1.5);
\draw [fill=red](2,0) rectangle (3,1.5);\endtikzpicture }

\newcommand{\DGE}{ \tikzpicture[scale=.25]
\draw [fill=yellow](0,0) rectangle (3,.5);
\draw [fill=red]((0,.5) rectangle (3,1);
\draw [fill=black](0,1) rectangle (3,1.5); \endtikzpicture }

\newcommand{\DGB}{ \tikzpicture[scale=.25]
\draw [fill=blue](0,0) rectangle (3,1.5);
\draw [white,line width=.1cm](0,0) -- (3,1.5);
\draw [white,line width=.1cm](0,1.5) -- (3,0);
\draw [white,line width=.1cm](1.5,0) -- (1.5,1.5);
\draw [white,line width=.1cm](0,0.75) -- (3,0.75);
\draw [red,line width=.05cm](0,0) -- (3,1.5);
\draw [red,line width=.05cm](0,1.5) -- (3,0);
\draw [red,line width=.05cm](1.5,0) -- (1.5,1.5);
\draw [red,line width=.05cm](0,0.75) -- (3,0.75);
\endtikzpicture }



\TFRGB{
\newcommand{\ESS}[1]{\textbf{\textbackslash begin\AC{#1}}\index{\textbf{1 Environnements}!#1}}

\newcommand{\BSS}[1]{\textbf{\textbackslash{#1}}\index{\textbf{2 Commandes}!#1 @\textbackslash{}#1}}


\newcommand{\DDD}[1]{{\color{red}  #1}\index{\textbf{3 Paramètres et options}!#1}}
\newcommand{\RDD}[1]{{\color{red}  #1}\index{\textbf{3 Paramètres et options}!#1}}
%4
\newcommand{\BDD}[1]{{\color{blue}  #1}\index{\textbf{4 Valeurs Tikz}!#1}}

\newcommand{\RDDX}[2]{{\color{red}#1} \index{\textbf{4 Valeurs Tikz}!#1 (#2)}}
%5
\newcommand{\FDD}[1]{{\color{red}  #1}\index{\textbf{5 Extrémit\'es}!#1}}
}
{
\newcommand{\ESS}[1]{\textbf{\textbackslash{#1}}\index{\textbf{1 Environments}!#1 @\textbackslash{}#1}}
%2
\newcommand{\BSS}[1]{\textbf{\textbackslash{#1}}\index{\textbf{2 Commands}!#1 @\textbackslash{}#1}}
%3
\newcommand{\RDD}[1]{{\color{red}  #1}\index{\textbf{3 Parameters and options}!#1}}

\newcommand{\DDD}[1]{{\color{red}  #1}\index{\textbf{3 Parameters and options}!#1}}

\newcommand{\RDDX}[2]{{\color{red}#1} \index{\textbf{4 Values Tikz}!#1 (#2)}}
%
\newcommand{\BDD}[1]{{\color{blue}  #1}\index{\textbf{4 Values Tikz}!#1}}

\newcommand{\FDD}[1]{{\color{red}  #1}\index{\textbf{5 Extremities}!#1}}
}

\newcommand{\rouge}[1] {{\color{red}  #1}}
\newcommand{\blll}[1] {{\color{blue}  #1}}



 \begin{document}


%\selectlanguage{english}
\selectlanguage{french}



%  
\author{{\Huge Jean Pierre Casteleyn } \\ {\Huge IUT Génie Thermique et \'Energie } \\ {\Huge Dunkerque, France }}

\DeclareFixedFont{\RM}{T1}{ptm}{b}{n}{2cm}

\DeclareFixedFont{\RMM}{T1}{ptm}{b}{n}{1cm}

\title{ {\RM Visual TikZ} \\ \vspace{1cm} {\RMM Version 0.66} }



\date{
\begin{center}
\begin{animateinline}[loop,autoplay]{12}%
 \multiframe{24}{iAngle=0+15,icol=0+5}{\begin{tikzpicture}[rotate=90]
    \draw  (0,0) node[fill=white,circle] {\includegraphics[width=4cm]{LogoIUT}}  (0,0) circle (1);
  \end{tikzpicture}} 
\end{animateinline}% 
\end{center}
{\LARGE \TFRGB{mis à jour le \today}{Updated on \today} 
}
}


\maketitle



 \begin{animateinline}[autoplay,loop]{12}%
 \multiframe{24}{iAngle=0+15,icol=0+5}{\begin{tikzpicture}
 [scale=1.8] %
   \draw[line width=0pt] (-2,-2) rectangle(6,2); %
   \draw  (0,0) node[fill=white,circle,rotate=\iAngle] {\includegraphics[width=2cm]{LogoIUT}}  (0,0) circle (1);
    \draw (0,0) circle (1);
    \coordinate (abc) at (${sqrt(9-sin(\iAngle)*sin(\iAngle))+cos(\iAngle)}*(1,0)$) ;
    \coordinate (xyz) at (\iAngle:1);
    \draw[ultra thick] (0,0) --(xyz); 
    \draw[ultra thick] (xyz) -- (abc) ;
    \fill[color=blue!\icol] (abc)++(0.5,-1) rectangle (5,1) ;
    \draw[ultra thick] (abc) ++(0,-1) rectangle ++(.5,2) ;
    \draw[ultra thick]  (1.5,1) -- (5,1) -- (5,-1) -- (1.5,-1);
    \fill[red] (xyz) circle (4pt);
    \fill[red] (abc) circle (4pt); 
  \end{tikzpicture}}
 \end{animateinline} 


 
\newpage
 
 
\TFRGB{
\textbf{Objectifs }: 

\begin{itemize}
\item Avoir une image par  commande ou par paramètre.
\item Avoir un texte réduit au strict minimum.
\item Etre le plus complet possible au fil de mises à jour régulières.
\item Garder la même structure que visuel pstricks
\end{itemize} 
}
{\textbf{Objectives }: 

\begin{itemize}
\item One image per command or parameter.
\item the minimum amount of text possible.
\item the most complete possible update after update.
\item keep the same structure as VisualPSTricks
\end{itemize}}


\vspace{1cm}

\TFRGB{
\textbf{Remarques }: Le code donné est minimal et ne sert qu'à montrer les commandes concernées. Les effets sont parfois exagérés pour bien les mettre en évidence. Pour en savoir plus, vous pouvez voir la documentation. Pour se faire j'ai indiqué le numéro de \tikz[baseline=-1mm]  \draw node[draw,fill=red!20] {Section de pgfmanual} ;
}
{\textbf{Remarks }:
Minimal code is given to show the effect of a command or a parameter. The effects are sometime exaggerated for clarity   .To consult the documentation, I have given the number of the  \tikz[baseline=-1mm]  \draw node[draw,fill=red!20] {Section in pgfmanual} ;
}
\vspace{1cm}


\TFRGB{
\textbf{Vous pouvez me contacter à}
 \href{mailto:jpcdk@yahoo.fr}{mon e-mail personnel} pour

\begin{itemize}
\item me signaler les erreurs que vous avez constatés (merci d'indiquer la page où vous l'avez constaté)
\item me faire part de vos commentaires, suggestions \dots
\end{itemize}}
{
\textbf{You can contact me at }
 \href{mailto:jpcdk@yahoo.fr}{my personal email} to

\begin{itemize}
\item let me know the mistakes found (please indicate the page)
\item give me your commentaries, your suggestions \dots
\end{itemize}}

\vspace{1cm}
\TFRGB{
\textbf{Quoi de neuf ! } :

\begin{itemize}
\item Ajout de la library  chains \pageref{lib-chains}
\item Ajout de la library  through \pageref{lib-through}
\item Ajout de la library  turtle \pageref{lib-turtle}
\item Ajout de la library positioning \pageref{lib-pos}
\item Ajout du module tikzsymbols \pageref{symbol}
\item mise à jour du module tikzducks \pageref{ducks}
\item mise à jour des modules shape \pageref{formes}
\end{itemize}

}
{
\textbf{What's new } :
\begin{itemize}
\item chains library added \pageref{lib-chains}
\item through library added \pageref{lib-through}
\item turtle library added \pageref{lib-turtle}
\item positioning library added \pageref{lib-pos}
\item Tikzsymbols package added \pageref{symbol}
\item Tikzducks package updated \pageref{ducks}
\item shapes packages updated \pageref{formes}
\end{itemize}
}



\vspace{1cm}
\textbf{Licence } :


This work may be distributed and/or modified under the conditions of the LaTeX Project Public License, either version 1.3 of this license or (at your option) any later version.

 The latest version of this license is in  http://www.latex-project.org/lppl.txt and version 1.3 or later is part of all distributions of LaTeX
version 2005/12/01 or later.

This work has the LPPL maintenance status `maintained'.

The Current Maintainer of this work is M. Jean Pierre Casteleyn.

\vspace{2cm}
\textbf{\TFRGB{Merci à }{Thanks to}}:

Till Tantau  ,
Alain Matthes ,
Jim Diamond ,
Falk Rühl ,
Axel Kielhorn ,
Nils Fleischhacker ,
Michel Fruchart ,
Ben Vitecek
 
\newpage


% \tableofcontents
%
%
%\newpage




\RRR{75-2 = Concept: Data Points and Data Formats}

\begin{tikzpicture}
\datavisualization [school book axes, visualize as smooth line]
data {
x, y
-1.5, 2.25
-1, 1
-.5, .25
0, 0
.5, .25
1, 1
1.5, 2.25
};
\end{tikzpicture}


\begin{tikzpicture}
\datavisualization [school book axes, visualize as smooth line]
data [format=function] {
var x : interval [-1.5:1.5] samples 7;
func y = \value x*\value x;
};
\end{tikzpicture}


\begin{tikzpicture}
\datavisualization [school book axes, visualize as smooth line]
data [format=function] {
var x : interval [-1.5:1.5] samples 3;
func y = \value x*\value x;
};
\end{tikzpicture}

Section 76 gives an in-depth coverage of the available data formats and explains how new data formats
can be defined.


\RRR{75-3 = Concept: Axes, Ticks, and Grids}


\begin{tikzpicture}
\datavisualization [
scientific axes,
x axis={length=3cm, ticks=few},
visualize as smooth line
]
data [format=function] {
var x : interval [-1.5:1.5] samples 20;
func y = \value x*\value x;
};
\end{tikzpicture}

\begin{tikzpicture}
\datavisualization [
scientific axes=clean,
x axis={length=3cm, ticks=few},
all axes={grid},
visualize as smooth line
]
data [format=function] {
var x : interval [-1.5:1.5] samples 7;
func y = \value x*\value x;
};
\end{tikzpicture}

Section 77 explains in more detail how axes, ticks, and grid lines can be chosen and configured.


\RRR {75-4 = Concept: Visualizers}

\begin{tikzpicture}
\datavisualization [
scientific axes=clean,
x axis={length=3cm, ticks=few},
visualize as smooth line
]
data [format=function] {
var x : interval [-1.5:1.5] samples 7;
func y = \value x*\value x;
};
\end{tikzpicture}

\begin{tikzpicture}
\datavisualization [
scientific axes=clean,
x axis={length=3cm, ticks=few},
visualize as scatter
]
data [format=function] {
var x : interval [-1.5:1.5] samples 7;
func y = \value x*\value x;
};
\end{tikzpicture}

Section 78 provides more information on visualizers as well as reference lists.

\RRR{75-5 = Concept: Style Sheets and Legends }

\begin{tikzpicture}[baseline]
\datavisualization [ scientific axes=clean,
y axis=grid,
visualize as smooth line/.list={sin,cos,tan},
style sheet=strong colors,
style sheet=vary dashing,
sin={label in legend={text=$\sin x$}},
cos={label in legend={text=$\cos x$}},
tan={label in legend={text=$\tan x$}},
data/format=function ]
data [set=sin] {
var x : interval [-0.5*pi:4];
func y = sin(\value x r);
}
data [set=cos] {
var x : interval [-0.5*pi:4];
func y = cos(\value x r);
}
data [set=tan] {
var x : interval [-0.3*pi:.3*pi];
func y = tan(\value x r);
};
\end{tikzpicture}


Section 79 details style sheets and legends.

\RRR{75-6 = Usage}

\subsection{/pgf/data/read from file=filename} (no default, initially empty)

If you set the source attribute to a non-empty hfilenamei, the data will be read from this file. In
this case, no hinline datai may be present, not even empty curly braces should be provided.
%\datavisualization ...
data [read from file=file1.csv]
data [read from file=file2.csv];
The other way round, if read from file is empty, the data must directly follow as hinline datai.
%\datavisualization ...
data {
x, y
1, 2
2, 3
};

The second important key is format, which is used to specify the data format:

\subsection{/pgf/data/format}

Use this key to locally set the format used for parsing the data, see Section 76 for a list of predefined
formats.

\tikz
\datavisualization [school book axes, visualize as line]
data [/data point/x=1] {
y
1
2
}
data [/data point/x=2] {
y
2
0
.5
};

\BS{datavisualization} . . . data point[options] . . . ;

\tikz \datavisualization [school book axes, visualize as line]
data point [x=1, y=1] data point [x=1, y=2]
data point [x=2, y=2] data point [x=2, y=0.5];

/tikz/data visualization/data point=options

\tikzdatavisualizationset{
horizontal/.style={
data point={x=#1, y=1}, data point={x=#1, y=2}},
}
\tikz \datavisualization
[ school book axes, visualize as line,
horizontal=1,
horizontal=2 ];

\BS{datavisualization} . . . data group[options]\AC{name}+=\AC{data specifications} . . . ;


\tikz \datavisualization data group {points} = {
data {
x, y
0, 1
1, 2
2, 2
5, 1
2, 0
1, 1
}
};

\tikz \datavisualization [school book axes, visualize as line] data group {points};
\qquad
\tikz \datavisualization [scientific axes=clean, visualize as line] data group {points};


\BS{datavisualization} . . . scope[options]{data specification} . . . ;

%\datavisualization...
%scope [/data point/experiment=7]
%{
%data [read from file=experiment007-part1.csv]
%data [read from file=experiment007-part2.csv]
%data [read from file=experiment007-part3.csv]
%}
%scope [/data point/experiment=23, format=foo]
%{
%data [read from file=experiment023-part1.foo]
%data [read from file=experiment023-part2.foo]
%};


\BS{datavisualization} . . . info[options]{code} . . . ;

\begin{tikzpicture}[baseline]
\datavisualization [ school book axes, visualize as line ]
data [format=function] {
var x : interval [-0.1*pi:2];
func y = sin(\value x r);
}
info {
\draw [red] (visualization cs: x={(.5*pi)}, y=1) circle [radius=1pt]
node [above,font=\footnotesize] {extremal point};
};
\end{tikzpicture}

\subsection{Coordinate system visualization}

\BS{datavisualization} . . . info’[options]{code} . . . ;

\begin{tikzpicture}[baseline]
\datavisualization [ school book axes, visualize as line ]
data [format=function] {
var x : interval [-0.1*pi:2];
func y = sin(\value x r);
}
info' {
\fill [red] (visualization cs: x={(.5*pi)}, y=1) circle [radius=2mm];
};
\end{tikzpicture}


\subsection{Predefined node data visualization bounding box}
This rectangle node stores a bounding box of the data visualization that is currently being constructed.
This node can be useful inside info commands or when labels need to be added.

\subsection{Predefined node data bounding box}
This rectangle node is similar to data visualization bounding box, but it keeps track only of the actual
data. The spaces needed for grid lines, ticks, axis labels, tick labels, and other all other information
that is not part of the actual data is not part of this box.


\RRR{75-7 = Advanced: Executing User Code During a Data Visualization}

\RRR{75-8 = Advanced: Creating New Objects}


\section{76 Providing Data for a Data Visualization}


%
%\newpage
%
%\RRR{17-2-1} fini
%
%
%/tikz/node font=font commands
%
%\begin{tikzpicture}
%\draw[node font=\itshape] (1,0) -- +(1,1) node[above] {italic};
%\end{tikzpicture}
%
%\tikz \node [node font=\tiny, minimum height=3em, draw] {tiny};
%\tikz \node [node font=\small, minimum height=3em, draw] {small};
%
%
%
%
%/tikz/node align header=
%
%
%\RRR{17-4-4} OK
%
%
%\RRR{17-5} Positioning Nodes
%
%
%
%\RRR{17-5-3 }Advanced Placement Options
%
%
%  
%
%\subsubsection{title}
%
%\begin{tikzpicture}[every node/.style={draw}]
%\draw[help lines](0,0) grid (3,2);
%\draw (1,0) node{A}
%(2,0) node[rotate=90,scale=1.5] {B};
%\draw[rotate=30] (1,0) node{A}
%(2,0) node[rotate=90,scale=1.5] {B};
%\draw[rotate=60] (1,0) node[transform shape] {A}
%(2,0) node[transform shape,rotate=90,scale=1.5] {B};
%\end{tikzpicture}
%
%m:::::::::::::::%\begin{tikzpicture}
%%% Install a nonlinear transformation:
%%\pgfsetcurvilinearbeziercurve
%%{\pgfpoint{0mm}{20mm}}
%%{\pgfpoint{10mm}{20mm}}
%%{\pgfpoint{10mm}{10mm}}
%%{\pgfpoint{20mm}{10mm}}
%%\pgftransformnonlinear{\pgfpointcurvilinearbezierorthogonal\pgf@x\pgf@y}%
%%% Draw something:
%%\draw [help lines] (0,-30pt) grid [step=10pt] (80pt,30pt);
%%\foreach \x in {0,20,...,80}
%%\node [fill=red!20] at (\x pt, -20pt) {\x};
%%\foreach \x in {0,20,...,80}
%%\node [fill=blue!20, transform shape nonlinear] at (\x pt, 20pt) {\x};
%%\end{tikzpicture}
%
%
%\newpage
%
%
%
%
\SbSSCT{Coordonnées}{Coordinates}
\begin{center}
\RRR{13-2-1}
\end{center}


\SbSbSSCT{Système de coordonnées \og canvas \fg}{Canvas coordinates}

\noindent


\tikzset{every picture/.style=blue,very thick,inner sep=0pt}

\begin{tabular}{|c|c|} \hline 
\TFRGB{Explicite}{explicit}  & \TFRGB{Implicite}{implicit}
\\ \hline
\begin{tikzpicture}
\draw[help lines] (0,0) grid (3,2);
\fill (canvas cs:x=2cm,y=1.5cm) circle (2pt);
\end{tikzpicture}
&
\begin{tikzpicture}
\draw[help lines] (0,0) grid (3,2);
\fill (2,1.5) circle (2pt);
\end{tikzpicture}

\\ \hline  
 \BS{fill} (\RDD{canvas cs}:\blll{x=2cm,y=1.5cm}) circle (2pt);
& \BS{fill} {\color{blue}(2cm,1.5cm)} circle (2pt);
\\ \hline 
\end{tabular} 


\SbSbSSCT{Système de coordonnées polaire \og canvas \fg}{Polar coordinates}

\noindent


\begin{tabular}{|c|c|c|} \hline
\TFRGB{Explicite}{explicit}  & \TFRGB{Implicite}{implicit}
\\ \hline
\begin{tikzpicture}
\draw[help lines] (0,0) grid (3,2);
\draw [dotted](0,2) arc (90 :0 :2);
\draw [dotted](0,0) --(2,2);
\fill (canvas polar cs:angle=45,radius=2cm) circle (2pt);
\end{tikzpicture}
&
\begin{tikzpicture}
\draw[help lines] (0,0) grid (3,2);
\draw [dotted](0,2) arc (90 :0 :2);
\draw [dotted](0,0) --(2,2);
\fill (45:2cm) circle (2pt);
\end{tikzpicture}
\\ \hline 
\BS{fill} (\RDD{canvas polar cs}:\RDD{angle}=45,\RDD{radius}=2cm) circle (2pt);
&
\BS{fill} {\color{blue}(45:2cm)} circle (2pt);
\\ \hline 
\end{tabular} 

\bigskip
\begin{tabular}{|c|} \hline  
\begin{tikzpicture}
\draw[help lines] (0,0) grid (3,2);
\draw [dotted](0,2) arc (90 :0 :3 and 2);
\draw [dotted](0,0) --(3,2);
\fill (canvas polar cs:angle=45,x radius=3cm,y radius=2cm) circle (2pt);
\end{tikzpicture}
\\ \hline  
\BS{fill} (canvas polar cs:angle=45,\RDD{x radius}=3cm,\RDD{y radius}=2cm) circle (2pt);
\\ \hline 
\end{tabular}


\SbSbSSCT{Système de coordonnées  xyz}{xyz coordinates}

\noindent


\begin{tabular}{|c|c|c|} \hline 
\begin{tikzpicture}[->]
\draw (0,0) -- (xyz cs:x=1);
\draw[red] (0,0) -- (xyz cs:y=1);
\draw[magenta] (0,0) -- (xyz cs:z=1);
\end{tikzpicture}
&
\begin{tikzpicture}[->]
\draw (0,0) -- (1,0,0);
\draw[red]  (0,0) -- (0,1,0);
\draw[magenta]  (0,0) -- (0,0,1);
\end{tikzpicture}
\\ \hline 
\BS{draw} (0,0) - - (\RDD{xyz cs}:x=1); & \BS{draw}  (0,0) - - (1,0,0); \\
\BS{draw}[red]  (0,0) - - (\RDD{xyz cs}:y=1); &  \BS{draw}[red] (0,0) - - (0,1,0); \\
\BS{draw}[magenta]  (0,0) - - (\RDD{xyz cs}:z=1); &  \BS{draw}[magenta]   (0,0) - - (0,0,1); 
\\ \hline 

\end{tabular} 

 
\newpage

\SbSbSSCT{Coordinate system xyz polar}{Coordinate system xyz polar}

\noindent

\begin{tabular}{|c|c|c|} \hline
\TFRGB{Explicite}{explicit}  & \TFRGB{Implicite}{implicit}
\\ \hline
\begin{tikzpicture}
\draw[help lines] (0,0) grid (3,2);
\draw [dotted](0,2) arc (90 :0 :2);
\draw [dotted](0,0) --(2,2);
\fill (xyz polar cs:angle=45,radius=2) circle (2pt);
\end{tikzpicture}
&
\begin{tikzpicture}
\draw[help lines] (0,0) grid (3,2);
\draw [dotted](0,2) arc (90 :0 :2);
\draw [dotted](0,0) --(2,2);
\fill (45:2) circle (2pt);
\end{tikzpicture}
\\ \hline 
\BS{fill} (\RDD{xyz polar cs}:\RDD{angle}=45,\RDD{radius}=2) circle (2pt);
&
\BS{fill} {\color{blue}(45:2cm)} circle (2pt);
\\ \hline 
\end{tabular} 

\bigskip
\begin{tabular}{|c|} \hline  
\begin{tikzpicture}
\draw[help lines] (0,0) grid (3,2);
\draw [dotted](0,2) arc (90 :0 :3 and 2);
\draw [dotted](0,0) --(3,2);
\fill (xyz polar cs:angle=45,x radius=3,y radius=2) circle (2pt);
\end{tikzpicture}
\\ \hline  
\BS{fill} (xyz polar cs:angle=45,\RDD{x radius}=3,\RDD{y radius}=2) circle (2pt);
\\ \hline 
\end{tabular} 

\bigskip

\begin{tabular}{|c|c|c|} \hline
\multicolumn{2}{|c|}{\BS{begin}\AC{tikzpicture}{\color{red}[x=1.5cm,y=1cm]} }
\\ \hline
\begin{tikzpicture}[x=1.5cm,y=1cm]
\draw[help lines] (0,0) grid (3,2);
\draw [dotted](0,2) arc (90 :0 :2);
\draw [dotted](0,0) --(2,2);
\fill (xyz polar cs:angle=45,radius=2) circle (2pt);
\end{tikzpicture}
&
\begin{tikzpicture}[x=1.5cm,y=1cm]
\draw[help lines] (0,0) grid (3,2);
\draw [dotted](0,2) arc (90 :0 :2);
\draw [dotted](0,0) --(2,2);
\fill (45:2) circle (2pt);
\end{tikzpicture}
\\ \hline 
\BS{fill} (\RDD{xyz polar cs}:\RDD{angle}=45,\RDD{radius}=2) circle (2pt);
&
\BS{fill} {\color{blue}(45:2cm)} circle (2pt);
\\ \hline 
\end{tabular} 
\bigskip

\begin{tabular}{|c|c|c|} \hline
\multicolumn{2}{|c|}{\BS{begin}\AC{tikzpicture}{\color{red}[x=\AC{(0cm,1cm)},y=\AC{(-1cm,0cm)}]} }
\\ \hline
\begin{tikzpicture}[x={(0cm,1cm)},y={(-1cm,0cm)}]
\draw[help lines] (0,0) grid (3,2);
\draw [dotted](0,2) arc (90 :0 :2);
\draw [dotted](0,0) --(2,2);
\fill (xyz polar cs:angle=45,radius=2) circle (2pt);
\end{tikzpicture}
&
\begin{tikzpicture}[x={(0cm,1cm)},y={(-1cm,0cm)}]
\draw[help lines] (0,0) grid (3,2);
\draw [dotted](0,2) arc (90 :0 :2);
\draw [dotted](0,0) --(2,2);
\fill (45:2) circle (2pt);
\end{tikzpicture}
\\ \hline 
\BS{fill} (\RDD{xyz polar cs}:\RDD{angle}=45,\RDD{radius}=2) circle (2pt);
&
\BS{fill} {\color{blue}(45:2cm)} circle (2pt);
\\ \hline 
\end{tabular} 

\SbSbSSCT{Coordonnées barycentriques}{Barycentric coordinates}

\begin{center}
\RRR{13-2-2}
\end{center}

\begin{tabular}{|c|c|c|} \hline
\multicolumn{3}{|c|}{  \BS{node} [circle,fill=red!20] at (\RDD{barycentric cs}:A=0.6,B=0.3 ) \AC{X};   }\\ 
\hline
\begin{tikzpicture}[scale=.6]
\draw[help lines] (0,0) grid (4,4);
\node[circle,fill=green!20,] (A) at (0,0) {A};
\node[circle,fill=green!20,] (B) at (4,0) {B};
\node[circle,fill=red!20] at (barycentric cs:A=0.3,B=0.3 ) {X};
\end{tikzpicture}
&
\begin{tikzpicture}[scale=.6]
\draw[help lines] (0,0) grid (4,4);
\node[circle,fill=green!20,] (A) at (0,0) {A};
\node[circle,fill=green!20,] (B) at (4,0) {B};
\node[circle,fill=green!20,] (C) at (4,4) {C};
\node[circle,fill=red!20] at (barycentric cs:A=0.4,B=0.4 ,C=.4) {X};
\end{tikzpicture}
&
\begin{tikzpicture}[scale=.6]
\draw[help lines] (0,0) grid (4,4);
\node[circle,fill=green!20,] (A) at (0,0) {A};
\node[circle,fill=green!20,] (B) at (4,0) {B};
\node[circle,fill=green!20,] (C) at (1,4) {C};
\node[circle,fill=green!20,] (D) at (4,4) {D};
\node[circle,fill=red!20] at (barycentric cs:A=0.5,B=0.5,C=.5,D=.5 ) {X};
\end{tikzpicture}
\\ \hline
A=0.3,B=0.3 & A=0.4,B=0.4 ,C=.4 & A=0.5,B=0.5,C=.5,D=.5 
\\ \hline
\begin{tikzpicture}[scale=.6]
\draw[help lines] (0,0) grid (4,4);
\node[circle,fill=green!20,] (A) at (0,0) {A};
\node[circle,fill=green!20,] (B) at (4,0) {B};
\node[circle,fill=red!20] at (barycentric cs:A=0.6,B=0.3 ) {X};
\end{tikzpicture}
&
\begin{tikzpicture}[scale=.6]
\draw[help lines] (0,0) grid (4,4);
\node[circle,fill=green!20,] (A) at (0,0) {A};
\node[circle,fill=green!20,] (B) at (4,0) {B};
\node[circle,fill=green!20,] (C) at (4,4) {C};
\node[circle,fill=red!20] at (barycentric cs:A=0.2,B=0.4 ,C=.6) {X};
\end{tikzpicture}
&
\begin{tikzpicture}[scale=.6]
\draw[help lines] (0,0) grid (4,4);
\node[circle,fill=green!20,] (A) at (0,0) {A};
\node[circle,fill=green!20,] (B) at (4,0) {B};
\node[circle,fill=green!20,] (C) at (1,4) {C};
\node[circle,fill=green!20,] (D) at (4,4) {D};
\node[circle,fill=red!20] at (barycentric cs:A=0.2,B=0.4,C=.6,D=.8 ) {X};
\end{tikzpicture}
\\ \hline
A=0.6,B=0.3 & A=0.2,B=0.4 ,C=.6 & A=0.2,B=0.4,C=.6,D=.8
\\ \hline
\end{tabular}

\SbSbSSCT{Coordonnées nominatives : n\oe ud}{Named coordinates: nodes}

\begin{center}
\RRR{13-2-3}
\end{center}

\begin{tabular}{|c|c|} \hline  
\begin{tikzpicture}[blue,very thick,baseline=1cm]
\draw[help lines] (0,0) grid (3,3);
\coordinate (centre) at (1.5,1.5) ;
\coordinate (A) at (.5,.5) ;
\coordinate (B) at (2.5,2.5) ;
\fill (centre) circle (3pt);
\draw[red] (A) rectangle (B) ;
\end{tikzpicture}
&  
\parbox[c]{8cm}{
\BSS{coordinate} {\color{blue}(centre)} at(1.5,1.5) ; \\
\BSS{coordinate} {\color{blue}(A)} at (.5,.5) ;\\
\BSS{coordinate} {\color{blue}(B)} at  (2.5,2.5) ;\\
\\
\BS{fill} {\color{blue}(centre)} circle (3pt);\\
\BS{draw}[red] {\color{blue}(A)} rectangle {\color{blue}(B)} ;\\
}
\\ \hline 
\end{tabular} 


\TFRGB{voir aussi}{see also} page \pageref{noeuds}


\SbSbSSCT{Coordonnées relatives à un noeud}{Coordinates relative to a node}

\noindent

\begin{tabular}{|c|c|c|c|} \hline
\multicolumn{4}{|l|}{  \BS{node} [draw,fill=green!20,] (A) at (1,1) \AC{\BS{huge}  noeud}; }\\ 
\multicolumn{4}{|l|}{  \BS{fill}[red] (\RDD{node cs}:\RDD{name}=A,\RDD{anchor}=south) circle (3pt);   }\\ 
\hline

\begin{tikzpicture}
\draw[help lines] (0,0) grid (2,2);
\node[draw,fill=green!20,] (A) at (1,1) {\huge noeud};
\fill[red] (node cs:name=A,anchor=south) circle (3pt);
\end{tikzpicture}
&
\begin{tikzpicture}
\draw[help lines] (0,0) grid (2,2);
\node[draw,fill=green!20,] (A) at (1,1) {\huge noeud};
\fill[red] (node cs:name=A,anchor=west) circle (3pt);
\end{tikzpicture}
&
\begin{tikzpicture}
\draw[help lines] (0,0) grid (2,2);
\node[draw,fill=green!20,] (A) at (1,1) {\huge noeud};
\fill[red] (node cs:name=A,anchor=north) circle (3pt);
\end{tikzpicture}
&
\begin{tikzpicture}
\draw[help lines] (0,0) grid (2,2);
\node[draw,fill=green!20,] (A) at (1,1) {\huge noeud};
\fill[red] (node cs:name=A,anchor=east) circle (3pt);
\end{tikzpicture}
\\ \hline
name=A,anchor=south & name=A,anchor=west & name=A,anchor=north & name=A,anchor=east
\\ \hline
\end{tabular}

\bigskip

\begin{tabular}{|c|c|c|c|} \hline
\multicolumn{4}{|l|}{  \BS{node} [draw,fill=green!20,] \blll{(A)} at (1,1) \AC{\BS{huge}  noeud}; }\\ 
\multicolumn{4}{|l|}{  \BS{fill}[red] (\blll{A}.south) circle (3pt);   }\\ 
\hline

\begin{tikzpicture}
\draw[help lines] (0,0) grid (2,2);
\node[draw,fill=green!20,] (A) at (1,1) {\huge noeud};
\fill[red] (A.south) circle (3pt);
\end{tikzpicture}
&
\begin{tikzpicture}
\draw[help lines] (0,0) grid (2,2);
\node[draw,fill=green!20,] (A) at (1,1) {\huge noeud};
\fill[red] (A.west) circle (3pt);
\end{tikzpicture}
&
\begin{tikzpicture}
\draw[help lines] (0,0) grid (2,2);
\node[draw,fill=green!20,] (A) at (1,1) {\huge noeud};
\fill[red] (A.north) circle (3pt);
\end{tikzpicture}
&
\begin{tikzpicture}
\draw[help lines] (0,0) grid (2,2);
\node[draw,fill=green!20,] (A) at (1,1) {\huge noeud};
\fill[red] (A.east) circle (3pt);
\end{tikzpicture}
\\ \hline
A.south & A.west & A.north & A.east
\\ \hline
\end{tabular}



\bigskip
\begin{tabular}{|c|c|c|c|} \hline
\multicolumn{4}{|c|}{  \BS{fill}[red] (node cs:\RDD{name}=A,\RDD{angle}=0) circle (3pt);  }\\ 
\hline

\begin{tikzpicture}
\draw[help lines] (0,0) grid (2,2);
\node[draw,fill=green!20,] (A) at (1,1) {\huge noeud};
\fill[red] (node cs:name=A,angle=0) circle (3pt);
\end{tikzpicture}
&
\begin{tikzpicture}
\draw[help lines] (0,0) grid (2,2);
\node[draw,fill=green!20,] (A) at (1,1) {\huge noeud};
\fill[red] (node cs:name=A,angle=-30) circle (3pt);
\end{tikzpicture}
&
\begin{tikzpicture}
\draw[help lines] (0,0) grid (2,2);
\node[draw,fill=green!20,] (A) at (1,1) {\huge noeud};
\fill[red] (node cs:name=A,angle=-90) circle (3pt);
\end{tikzpicture}
&
\begin{tikzpicture}
\draw[help lines] (0,0) grid (2,2);
\node[draw,fill=green!20,] (A) at (1,1) {\huge noeud};
\fill[red] (node cs:name=A,angle=-150) circle (3pt);
\end{tikzpicture}
\\ \hline
name=A,angle=0 & name=A,angle=-30 & nname=A,angle=-90 & name=A,angle=-150
\\ \hline
\end{tabular}

\bigskip


\begin{tabular}{|c|c|c|c|} \hline
\multicolumn{4}{|c|}{  \BS{fill}[red] (A.0) circle (3pt);  }\\ 
\hline

\begin{tikzpicture}
\draw[help lines] (0,0) grid (2,2);
\node[draw,fill=green!20,] (A) at (1,1) {\huge noeud};
\fill[red] (A.0) circle (3pt);
\end{tikzpicture}
&
\begin{tikzpicture}
\draw[help lines] (0,0) grid (2,2);
\node[draw,fill=green!20,] (A) at (1,1) {\huge noeud};
\fill[red] (A.-30) circle (3pt);
\end{tikzpicture}
&
\begin{tikzpicture}
\draw[help lines] (0,0) grid (2,2);
\node[draw,fill=green!20,] (A) at (1,1) {\huge noeud};
\fill[red] (A.-90) circle (3pt);
\end{tikzpicture}
&
\begin{tikzpicture}
\draw[help lines] (0,0) grid (2,2);
\node[draw,fill=green!20,] (A) at (1,1) {\huge noeud};
\fill[red] (A.-150) circle (3pt);
\end{tikzpicture}
\\ \hline
A.0 & A.-30 & A.-90 & A.-150
\\ \hline
\end{tabular}

\TFRGB{voir aussi}{see also} page \pageref{nomnoeud}


\newpage

\SbSbSSCT{Coordonnées relatives à deux points}{Coordinates relative to two points}
\begin{center}
\RRR{13-3-1}
\end{center}

\begin{tabular}{|c|c|} \hline
\multicolumn{2}{|c|}{  \BS{node} [circle,fill=red!20] at (1,1 {\color{red}|-} 3,3) \AC{X}   }\\ 
\hline
\begin{tikzpicture}
\draw[help lines] (0,0) grid (4,4);
\node[circle,fill=green!20,] (A) at (1,1) {A};
\node[circle,fill=green!20,] (B) at (3,3) {B};
\node[circle,fill=red!20] at (1,1 |- 3,3) {X};
\end{tikzpicture}
&
\begin{tikzpicture}
\draw[help lines] (0,0) grid (4,4);
\node[circle,fill=green!20,] (A) at (1,1) {A};
\node[circle,fill=green!20,] (B) at (3,3) {B};
\node[circle,fill=red!20] at (1,1 -| 3,3) {X};
\end{tikzpicture}
\\ \hline
at (1,1 {\color{red}|-} 3,3)
&
at (1,1 {\color{red}-|} 3,3)
\\ \hline
\end{tabular}



\SbSbSSCT{Coordonnée relative à une intersection}{Coordinates relative to an intersection}
\begin{center}
\RRR{13-3-2}
\end{center}

 \maboite{\BS{usetikzlibrary}\AC{intersections}}
\label{lib-intersections}


\begin{tabular}{|c|c|c|c|} \hline 
\multicolumn{4}{|l|}{  \BS{draw} [\RDD{name path}=XXX] (2,1) circle  (1cm);   }\\ 
\multicolumn{4}{|l|}{  \BS{draw} [\RDD{name path}=YYY] (0.5,0.5) rectangle +(3,1);   }\\ 
\multicolumn{4}{|l|}{ \BS{fill} [red,\RDD{ name intersections}=\AC{of=xxx and YYY}]
(\RDD{intersection}-1) circle (2pt)   }\\ 
\hline 
\begin{tikzpicture}[scale=.8]
\draw [help lines] grid (4,2);
\draw [name path=XXX] (2,1) circle  (1cm);
\draw [name path=YYY] (0.5,0.5) rectangle +(3,1);
\fill [red, name intersections={of=XXX and YYY}]
(intersection-1) circle (2pt)  ;
\end{tikzpicture}
& 
\begin{tikzpicture}[scale=.8]
\draw [help lines] grid (4,2);
\draw [name path=XXX] (2,1) circle  (1cm);
\draw [name path=YYY] (0.5,0.5) rectangle +(3,1);
\fill [red, name intersections={of=XXX and YYY}] (intersection-2) circle (2pt) ;
\end{tikzpicture} 
&  
\begin{tikzpicture}[scale=.8]
\draw [help lines] grid (4,2);
\draw [name path=XXX] (2,1) circle  (1cm);
\draw [name path=YYY] (0.5,0.5) rectangle +(3,1);
\fill [red, name intersections={of=XXX and YYY}] (intersection-3) circle (2pt) ;
\end{tikzpicture}
&  
\begin{tikzpicture}[scale=.8]
\draw [help lines] grid (4,2);
\draw [name path=XXX] (2,1) circle  (1cm);
\draw [name path=YYY] (0.5,0.5) rectangle +(3,1);
\fill [red, name intersections={of=XXX and YYY}] (intersection-4) circle (2pt) ;
\end{tikzpicture}
\\ 
\hline intersection-1 & intersection-2 &intersection-3  & intersection-4 \\ 
\hline 
\end{tabular} 

\bigskip

\begin{tabular}{|c|} \hline  
\BS{fill} [red, name intersections=\AC{of=XXX and YYY}] \\
(intersection-1) circle (2pt) {\color{red} node[black,above right] \AC{point a}} ;
\\ \hline  
\begin{tikzpicture}
\draw [help lines] grid (4,2);
\draw [name path=XXX] (2,1) circle  (1cm);
\draw [name path=YYY] (0.5,0.5) rectangle +(3,1);
\fill [red, name intersections={of=XXX and YYY}]
(intersection-1) circle (2pt) node[black,above right] {point a} ;
\end{tikzpicture} 
\\ \hline 
\end{tabular} 

\bigskip

\begin{tabular}{|c|} \hline 
\BS{fill} [red, name intersections=\AC{of=XXX and YYY, \RDD{name}=ZZZ}]; \\
\BS{draw} [red] (ZZZ-1) - - (ZZZ-3); \BS{draw} [green] (ZZZ-2) - - (ZZZ-4);
\\ \hline  
\begin{tikzpicture}
\draw [help lines] grid (4,2);
\draw [name path=XXX] (2,1) circle  (1cm);
\draw [name path=YYY] (0.5,0.5) rectangle +(3,1);
\fill [red, name intersections={of=XXX and YYY, name=ZZZ}];
\draw [red] (ZZZ-1) -- (ZZZ-3);
\draw [green] (ZZZ-2) -- (ZZZ-4);
\end{tikzpicture}
\\ \hline 
\end{tabular} 

\bigskip
\begin{tabular}{|c|} \hline  
\BS{fill} [red, name intersections=\AC{of=XXX and YYY , \RDD{by}=\AC{a,b,c,d}}]; \\
\BS{draw} [red] (a) - - (c); \hspace{1cm} \BS{draw} [green] (b) - - (d);
\\ \hline   
\begin{tikzpicture}
\draw [help lines] grid (4,2);
\draw [name path=XXX] (2,1) circle  (1cm);
\draw [name path=YYY] (0.5,0.5) rectangle +(3,1);
\fill [red, name intersections={of=XXX and YYY, by={a,b,c,d}}];
\draw [red] (a) -- (c);
\draw [green] (b) -- (d);
\end{tikzpicture}
\\ \hline 
\end{tabular} 

\bigskip

\begin{tabular}{|c|} \hline  
\BS{fill} [name intersections=\AC{of=XXX and YYY, name=i, \RDD{total}=\BS{t}}] [red] \\
\BS{foreach} \BS{s} in \AC{1,...,\BS{t}} \AC{(i-\BS{s}) circle (2pt) node[black,above right] \AC{\BS{s}}}
\\ \hline  
\begin{tikzpicture}
\draw [help lines] grid (4,2);
\draw [name path=XXX] (2,1) circle  (1cm);
\draw [name path=YYY] (0.5,0.5) rectangle +(3,1);
\fill [name intersections={of=XXX and YYY , name=i, total=\t}]
[red]
\foreach \s in {1,...,\t}{(i-\s) circle (2pt) node[black,above right] {\s}};
\end{tikzpicture}
\\ \hline 
\end{tabular} 



\newpage

\SbSbSSCT{Position calculée avec le module  \og  pgfmath \fg}{Calculated positions with  \og  pgfmath \fg }

\begin{center}
\RRR{13-2-1}
\end{center}

\TFRGB{Ce module est chargé automatiquement avec le module Tikz}{Package automatically loaded with Tikz} 

\begin{tabular}{|c|} \hline 
\begin{tikzpicture}
\draw[help lines] (0,0) grid (4,2);
\fill [red] (canvas cs:x=2cm+1.5cm,y=1.5cm-1cm) circle (3pt);
\fill [blue] (2cm,1.5cm) circle (3pt);
\draw[dashed] (2,1.5) -| (3.5,.5);
\end{tikzpicture}
\\ \hline 
\emph{\TFRGB{Explicite}{explicit}} 
 : \BS{fill} [red] (\RDD{canvas cs}:x=2cm+1.5cm,y=1.5cm-1cm) circle (3pt);
 \\  \hline 
\emph{\TFRGB{Implicite}{implicit}} :  \BS{fill} [red] {\color{red}(2cm+1.5cm,1.5cm-1cm)} circle (3pt);
\\ \hline 
\end{tabular} 

\bigskip
\begin{tabular}{|c|c|c|} \hline 
\begin{tikzpicture}[baseline=0pt]
\draw[help lines] (0,0) grid (4,4);
 \draw[dashed] (2,2) circle (2);
\fill[red](2+ 2*cos 30,2+2*sin 30) circle (3pt);
\fill[magenta](2+ 2*cos{(120)},2+2*sin{(120)}) circle (3pt);
\end{tikzpicture}
&
\parbox[c]{8cm}{
 \BS{draw}[dashed] (2,2) circle (2);\\
 \smallskip
 \BS{fill} [red]{\color{red}(2+ 2*cos 30 , 2+2*sin 30)} circle (3pt);\\
  \smallskip
 \BS{fill}[magenta] {\color{red}(2+2*cos\AC{(120)} , 2+2*sin\AC{(120)})} circle (3pt); 
 }
\\ \hline 
\end{tabular} 

\SbSbSSCT{Position calculée avec \og library calc \fg}{Calculated positions with \og  calc  library calc \fg}

\begin{center}
\RRR{13-5}
\end{center}
\label{lib-calc}

 \maboite{\BS{usetikzlibrary}\AC{calc}}
 
\begin{tabular}{|c|c|} \hline  
\begin{tikzpicture}[baseline=0pt]
\draw [help lines] (0,0) grid (3,2);
\node (a) at (1,1) {A};
\fill [red] ($(a) + 2/3*(1cm,0)$) circle (2pt);
\fill [red] ($(a) + 4/3*(1cm,0)$) circle (2pt);
\end{tikzpicture}
&
\parbox{8cm}{
\BS{node} (a) at (1,1) \AC{A}; \\
\BS{fill} [red] {\color{red} (\$(a) + 2/3*(1cm,0)\$)} circle (2pt); \\
\BS{fill} [red] {\color{red}(\$(a) + 4/3*(1cm,0)\$)} circle (2pt); \\
}
\\ 
\hline 
\end{tabular} 

\SbSbSSCT{Tangentes avec \og library calc \fg}{Tangents with  \og calc library  \fg}

\begin{center}
\RRR{13-2-4}
\end{center}

\begin{tabular}{|c|c|} \hline 
\multicolumn{2}{|l|}{\BS{node}[fill=green!20] (a) at (3,1.5) \AC{A}; } \\
\multicolumn{2}{|l|}{\BS{fill}[red] (\RDD{tangent cs}:\RDD{node}=c,\RDD{point}=\AC{(A)},\RDD{solution}=1);  }\\ 
\hline
\begin{tikzpicture}
\draw[help lines] (0,0) grid (4,2);
\node[fill=green!20] (A) at (3,1.5) {A};
\node [circle,draw] (c) at (1,1) [minimum size=1.5cm] {$c$};
\draw[red,dashed] (A) - -(tangent cs:node=c,point={(A)},solution=1) ;
\draw[red,dashed] (1,1) - -(tangent cs:node=c,point={(3,1.5)},solution=1) ;
\fill[red] (tangent cs:node=c,point={(A)},solution=1) circle (3pt);
\end{tikzpicture}
&
\begin{tikzpicture}
\draw[help lines] (0,0) grid (4,2);
\node[fill=green!20] (A) at (3,1.5) {A};
\node [circle,draw] (c) at (1,1) [minimum size=1.5cm] {$c$};
\draw[red,dashed] (A) - -(tangent cs:node=c,point={(A)},solution=2) ;
\draw[red,dashed] (1,1) - -(tangent cs:node=c,point={(A)},solution=2) ;
\fill[red] (tangent cs:node=c,point={(A)},solution=2) circle (3pt);
\end{tikzpicture}
\\ \hline
\RDD{solution}=1 & \RDD{solution}=2
\\ \hline
\end{tabular} 

\newpage

\SbSbSSCT{Point à pourcentage donné }{Percentage position }

\begin{center}
\RRR{13-5-3}
\end{center}


\begin{tabular}{|c|c|} \hline  
\multicolumn{2}{|c|}{\BS{fill}[red] ({\color{red}\$(0,1)!.25!(4,1)\$}) circle (4pt); } \\  \hline  

\begin{tikzpicture}
\draw [help lines] (0,0) grid (4,2);
\draw [line width= 3pt] (0,1) -- (4,1);
\fill[red] ($(0,1)!.25!(4,1)$) circle (4pt);
\end{tikzpicture}
&  
\begin{tikzpicture}
\draw [help lines] (0,0) grid (4,2);
\draw [line width= 3pt] (0,1) -- (4,1);
\fill[red] ($(0,1)!.75!(4,1)$) circle (4pt);
\end{tikzpicture}
\\ \hline (0,1)!{\color{red}0.25}!(4,1) & (0,1)!{\color{red}0.75}!(4,1) \\ 
\hline 
\end{tabular} 

\bigskip

\begin{tabular}{|c|} \hline  
\begin{tikzpicture}
\draw [help lines] (0,0) grid (4,3);
\draw [line width=2pt ](0,2) -- (4,2);
\draw[red] ($(0,2)!.75!(4,2)$) -- (0,0);
\fill[red] ($(0,2)!.75!(4,2)!.66!(0,0)$) circle (4pt);
\end{tikzpicture}
\\ \hline 
\BS{fill}[red] (\${\color{blue}(0,2)!0.75!(4,2)}!{\color{red}0.66!(0,0)}\$) circle (2pt);
\\ \hline 
\end{tabular} 


\SbSbSSCT{Point à distance donnée}{Position at a given distance }

\begin{center}
\RRR{13-5-4}
\end{center}

\begin{tabular}{|c|c|} \hline  
\multicolumn{2}{|c|}{\BS{fill}[red] ({\color{red}\$(0,1)!1.5cm!(4,1)\$}) circle (4pt); } \\  \hline  

\begin{tikzpicture}
\draw [help lines] (0,0) grid (4,2);
\draw [line width= 2pt] (0,1) -- (4,1);
\fill[red] ($(0,1)!1.5cm!(4,1)$) circle (4pt);
\end{tikzpicture}
&  
\begin{tikzpicture}
\draw [help lines] (0,0) grid (4,2);
\draw [line width= 2pt] (0,1) -- (4,1);
\fill[red] ($(0,1)!3cm!(4,1)$) circle (4pt);
\end{tikzpicture}
\\ \hline (0,1)!{\color{red}1.5cm}!(4,1) & (0,1)!{\color{red}3cm}!(4,1) \\ 
\hline 
\end{tabular} 

\bigskip

\begin{tabular}{|c|} \hline  
\begin{tikzpicture}
\draw [help lines] (0,0) grid (4,4);
\coordinate (a) at (1,0);
\coordinate (b) at (4,1);
\draw [line width= 3pt] (0,0) -- (4,1);
\draw [line width= 2pt,red](2,.5) -- ($ (2,.5)!2cm!90:(4,1) $);
\end{tikzpicture}
\\ \hline
\BS{draw} (2,.05) - - (\$ (2,0.5)!{\color{red}2cm!90:(4,1)} \$);
\\ \hline 
\end{tabular} 

\newpage

\SbSbSSCT{Coordonnées relatives}{Relative coordinates}


\Par{Cartésienne}{Cartesian coordinates}

\begin{center}
\RRR{13-4-1}
\end{center}

\begin{tabular}{|c|c|c|} \hline  
\TFRGB{relative à l'origine}{relative to the origin}  & \TFRGB{relative à une position}{relative to a position}  &  \TFRGB{relative à la dernière position}{relative to the last position}   
\\ \hline  
 
\begin{tikzpicture}
\draw[help lines] (0,-1) grid (3,1); 
 \draw[blue,very thick] (0,0) -- (1,0) - - (2,1) - - (2,-1);
 \fill[red] (0,0) circle (4pt);
\end{tikzpicture}
&
\begin{tikzpicture} %[scale=.8]
\draw[help lines] (0,-1) grid (4,1);
 \draw[blue,very thick] (0,0) - - (1,0) -- +(2,1) -- +(2,-1) ; %–- +(2,-1) ;
 \fill[red] (1,0) circle (4pt);
\end{tikzpicture}
&
\begin{tikzpicture} %[scale=.8]
\draw[help lines] (0,-1) grid (5,1);  
 \draw[blue,very thick] (0,0) -- (1,0)  - - ++(2,1) - - ++(2,-1);
 \fill[red] (1,0) circle (4pt);
 \fill[red] (3,1) circle (4pt);
\end{tikzpicture}
\\ \hline 
\tikz \fill node[fill=green!20,inner sep=0pt]{(0,0)}; - - (1,0) &
 (0,0) - - \tikz \fill node[fill=green!20,inner sep=0pt]{(1,0)};  & (0,0) - - \tikz \fill node[fill=green!20,inner sep=0pt]{(1,0)}; \\
 - - (2,1) - - (2,-1)  &
   - - +(2,1) - - +(2,-1) & - - ++\tikz \fill node[fill=green!20,inner sep=0pt]{(2,1)}; - - ++(2,-1)
\\ \hline 
\end{tabular} 

\bigskip

\begin{tabular}{|c|c|c|} \hline  
\begin{tikzpicture} [scale=.5]
\draw[help lines] (0,-1) grid (6,6);
 \draw[red,dotted,line width=2pt] (0,0) rectangle (2,2) ;
  \draw[green,dotted,line width=2pt] (0,0) rectangle (3,3) ;  
 \draw[blue,line width=2pt] (0,0) rectangle (1,1)  rectangle (2,2) rectangle (3,3);

\end{tikzpicture}

&  
\begin{tikzpicture} [scale=.5]
\draw[help lines] (0,-1) grid (6,6); 
  \draw[green,dotted,line width=2pt] (1,1) rectangle (4,4) ;   
 \draw[blue,line width=2pt] (0,0) rectangle (1,1)  rectangle +(2,2) rectangle +(3,3);
    \fill[red] (1,1) circle (4pt);
\end{tikzpicture}
&  
\begin{tikzpicture} [scale=.5]
\draw[help lines] (0,-1) grid (6,6);  
 \draw[blue,line width=2pt] (0,0) rectangle (1,1)  rectangle ++(2,2) rectangle ++(3,3);
    \fill[red] (1,1) circle (4pt);
     \fill[green] (3,3) circle (4pt); 
\end{tikzpicture}
\\ 
\hline 
\BS{draw} (0,0) rectangle (1,1)   &
\BS{draw} (0,0) rectangle (1,1)   & 
\BS{draw} (0,0) rectangle (1,1)  \\
rectangle (2,2) rectangle (3,3);  &
rectangle +(2,2) rectangle +(3,3);  &
rectangle ++(2,2) rectangle ++(3,3); \\
\hline 
\end{tabular}


\Par{Polaire }{Polar} {}

\bigskip


\noindent

\begin{tabular}{|c|c|c|c|} \hline
\TFRGB{relative à l'origine}{relative to the origin}  & \TFRGB{relative à une position}{relative to a position}  &  \TFRGB{relative à la dernière position}{relative to the last position}   
\\ \hline    
\begin{tikzpicture} %[scale=.8] 
\draw[help lines] (0,-1) grid (3,1);
 \fill[red] (0:0) circle (4pt);
 \draw[blue,very thick] (0:0)-- (0:1) -- (30:2) -- (-30:2);
\end{tikzpicture}
&
\begin{tikzpicture} %[scale=.8] 
\draw[help lines] (0,-1) grid (4,1);
 \fill[red] (1,0) circle (4pt);
 \draw[blue,very thick] (0:0) -- (0:1) -- +(30:2) -- +(-30:2);
\end{tikzpicture}
&
\begin{tikzpicture} %[scale=.8] 
\draw[help lines] (0,-1) grid (5,1);
 \fill[red] (1,0) circle (4pt);
 \fill[red] (2.732,1) circle (4pt);
 \draw[blue,very thick] (0:0)-- (0:1) -- ++(30:2) -- ++(-30:2);
\end{tikzpicture}
\\ \hline
\tikz \fill node[fill=green!20,inner sep=0pt] {(0:0)}; - - (0:1)&
 (0:0) - - \tikz \fill node[fill=green!20,inner sep=0pt] {(0:1)}; & (0:0)- - \tikz \fill node[fill=green!20,inner sep=0pt] {(0:1)}; \\
 - - (30:2) - - (-30:2)  &  - -  +(30:2) - - +(-30:2) & - -  ++\tikz \fill node[fill=green!20,inner sep=0pt] {(30:2)}; - - ++(-30:2)
\\ \hline 
\end{tabular} 

%\subsubsection{coordonnée relative en polaire}
\Par{coordonnée relative en polaire}{Relative polar coordinate}

\begin{center}
\RRR{13-4-2}
\end{center}
\bigskip

\begin{tabular}{|c|c|} \hline 
\multicolumn{2}{|c|}{ \BS{draw}[blue,very thick] (0,0) -- (2,1) -- ([turn]-45:1cm);}
 \\ \hline
\begin{tikzpicture} %[scale=.8] 
\draw[help lines] (0,0) grid (4,2);
 \draw[dotted] (0,0) -- (4,2);
 \draw[blue,very thick] (0,0) -- (2,1) -- ([turn]-45:1cm);
\end{tikzpicture}
&  
\begin{tikzpicture} %[scale=.8] 
\draw[help lines] (0,0) grid (4,2);
 \draw[dotted] (0,0) -- (4,2);
 \draw[blue,very thick] (0,0) -- (2,1) -- ([turn]45:1cm);
\end{tikzpicture}
\\ \hline ([\RDD{turn}]-45:1cm) & ([\RDD{turn}]45:1cm) \\ 
\hline 
\end{tabular}

\bigskip

\begin{tabular}{|c|c|} \hline  
\begin{tikzpicture}  
\draw[help lines] (-1,0) grid (4,3);
\draw [line width=2pt] (4,0) arc (0 :120 :2)  -- ([turn]90:2cm) ;

\end{tikzpicture}
&  
\begin{tikzpicture} %[scale=.8] 
\draw[help lines] (0,0) grid (4,3);
\draw [line width=2pt]  (0,0) to [bend left] (2,2) --  ([turn]0:2cm);
\fill [red](2,2) circle (4pt);
\end{tikzpicture}
\\ \hline  
\BS{draw} (4,0) arc (0 :120 :2)  - - ([\RDD{turn}]90:2cm) ;
& \BS{draw}  (0,0) to [bend left] (2,2) - -  ([\RDD{turn}]0:2cm); \\

\hline 
\end{tabular} 


%\bigskip 
%
%
%\tikz [delta angle=30, radius=1cm]
%\draw (0,0) arc [start angle=0] -- ([turn]0:1cm)
%arc [start angle=30] -- ([turn]0:1cm)
%arc [start angle=60] -- ([turn]30:1cm);



\bigskip

\begin{tabular}{|c|c|c|} \hline  
\multicolumn{3}{|c|}{ \BS{draw}(1,2)
.. controls ([turn]0:2cm) .. ([turn]-90:2cm); }
\\ \hline
\begin{tikzpicture} %[scale=.8] 
\draw[help lines] (0,0) grid (4,4);
 \draw [line width=2pt] (1,2)
.. controls ([turn]0:2cm) .. ([turn]-90:2cm);
\end{tikzpicture}
&  
\begin{tikzpicture} %[scale=.8] 
\draw[help lines] (0,0) grid (4,4);
 \draw [line width=2pt] (1,2)
.. controls ([turn]30:2cm) .. ([turn]-90:2cm);
\end{tikzpicture}
&  
\begin{tikzpicture} %[scale=.8] 
\draw[help lines] (-2,0) grid (2,4);
 \draw [line width=2pt] (1,2)
.. controls ([turn]0:2cm) .. ([turn]90:2cm);

\end{tikzpicture}
\\ \hline ([turn]0:2cm) .. ([turn]-90:2cm) & ([turn]30:2cm) .. ([turn]-90:2cm) & ([turn]0:2cm) .. ([turn]90:2cm) \\ 
\hline 
\end{tabular} 


\tikzset{every picture/.style=blue,very thick,inner sep=.3333em}

%
%
%
%\newpage
%
%\SSCT{Les n\oe uds }{Nodes}
%
%\SbSSCT{Définition des  n\oe uds}{Creation of nodes}
\tikzset{blue}

\label{noeuds}
\noindent

\begin{tabular}{|c | c | c | c | c |} \hline
\multicolumn{5}{|c|}{  \BS{draw} (1,1) node[\RDD{fill}=red!20] \AC{};   }\\ 
\hline 
\tikz \draw (0,0) grid (2,2) (1,1) node[fill=red!20] {};
&
\tikz \draw (0,0) grid (2,2) (1,1) node[fill=red!20,draw] {}; 
&
\tikz \draw (0,0) grid (2,2) (1,1) node[circle,fill=red!20] {};
&
\tikz \draw (0,0) grid (2,2) (1,1) node[circle,fill=red!20,draw] {};
&
\tikz \draw (0,0) grid (2,2) (1,1) node[coordinate] {};
\\  \hline
\dft
&
node[\RDD{draw}] 
&
 node[\RDD{circle}]  
&
 node[\RDD{circle},\RDD{draw}]
 &
  node[\RDD{coordinate}]
 \\  \hline
\end{tabular}
\bigskip

\begin{tabular}{|c | c | c | c | } \hline
\multicolumn{4}{|c|}{ \BSS{node} \RDD{at} (1,1) [fill=red!20] \AC{};   }\\ 
\hline 
 \begin{tikzpicture}
\draw (0,0) grid (2,2) ; 
\node at (1,1) [fill=red!20] {};
 \end{tikzpicture}
&
 \begin{tikzpicture}
\draw (0,0) grid (2,2) ; 
\node at (1,1) [draw] {};
 \end{tikzpicture}
&
 \begin{tikzpicture}
\draw (0,0) grid (2,2) ; 
\node at (1,1) [fill=red!20,circle] {};
 \end{tikzpicture}
&
 \begin{tikzpicture}
\draw (0,0) grid (2,2) ; 
\node at (1,1) [circle,draw] {};
 \end{tikzpicture}

\\  \hline
[fill=red!20]
&
[\RDD{draw}] 
&
[\RDD{circle},fill=red!20]
 &
[\RDD{circle},draw] 
 \\  \hline
\end{tabular}
\bigskip

\TFRGB{Autres types de n\oe uds voir page}{Other type of nodes see page} \pageref{noeudboite}
\bigskip


\begin{tabular}{|c|c|} \hline 
\BS{draw} (0,0) node at (1,0) \AC{1} node at (2,0) \AC{2} & \BS{draw}(0,0) node foreach \BS{x} in \AC{1,2,...,5}\\ 
node at (3,0) \AC{3} node at (4,0) \AC{4} node at (5,0) \AC{5}; &  at (\BS{x},0) \AC{\BS{x}};\\ 
\hline 
\tikz \draw (0,0) node at (1,0) {1} node at (2,0) {2} node at (3,0) {3} node at (4,0) {4} node at (5,0) {5};
&
\tikz \draw (0,0) node foreach \x in {1,2,...,5} at (\x,0) {\x};
\\ \hline 
\end{tabular} 



\bigskip

\begin{tabular}{|c|} \hline 
\BS{draw}[\rouge{every node/.style=\AC{draw,red}}](0,0) node foreach \BS{x} in \AC{1,2,...,5} at (\BS{x},0) \AC{\BS{x}};
\\ \hline 
\rule[-3pt]{0pt}{.8cm}\tikz \draw[every node/.style={draw,red}] (0,0) node foreach \x in {1,2,...,5} at (\x,0) {\x};
\\ \hline 
\end{tabular} 

\bigskip

\begin{tabular}{|c|} \hline 
\BS{draw}[\rouge{every rectangle node/.style=\AC{draw,red}},\\
\rouge{every circle node/.style=\AC{draw,double}}]\\ (0,0) node at (1,0) \AC{1} node[circle] at (2,0) \AC{2} \\ node[circle] at (3,0) \AC{3} node at (4,0) \AC{4} node at (5,0) \AC{5};
\\ \hline 
\rule[-3pt]{0pt}{1cm} \tikz \draw[every rectangle node/.style={draw,red},
every circle node/.style={draw,double}] (0,0) node at (1,0) {1} node[circle] at (2,0) {2} node[circle] at (3,0) {3} node at (4,0) {4} node at (5,0) {5};
\\ \hline 
\end{tabular} 

\SbSSCT{Nom des  n\oe uds}{Node name}


\begin{tabular}{|c|c|c|}
\hline 
\multicolumn{3}{|c|}{} \\ 
\hline 
\begin{tikzpicture}
\node[name=A,fill=red] at (0,0) {};
\draw  (-1,-1) grid (1,1) ;
\draw (A) circle (.5) ;
\end{tikzpicture} 
&  
\begin{tikzpicture}
\node[name=A,alias=B,fill=red] at (0,0) {} ;
\draw  (-1,-1) grid (1,1) ;
\draw (B) circle (.5) ;
\end{tikzpicture}
& 
\begin{tikzpicture}
\node[fill=red] (C) at (0,0) {};
\draw  (-1,-1) grid  (1,1) ;
\draw (C) circle (.5);
\end{tikzpicture} \\ 
\hline 
\BS{node}[\RDD{name}=A] at (0,0) \AC{}  & \BS{node}[\RDD{name}=A,\RDD{alias}=B] at (0,0) \AC{}  & 
\BS{node}\rouge {(C)} at (0,0) \AC{} \\ 
\BS{draw} (A) circle (.5); & \BS{draw}  (B) circle (.5); &\BS{draw} (C) circle (.5);
\\ \hline 
\end{tabular} 
\newpage

\SbSSCT{Contenu des  n\oe uds}{Node contents}
\tikzset{blue}

\begin{center}
\RRR{17-2-1}
\end{center}

\begin{tabular}{|c|c|} \hline 
\BS{node} at (1,1) [fill=red!20]\rouge { \AC{XXX} };
&  
\BS{node} at (1,1) [fill=red!20,\RDD{node contents}=XXX] \AC{};
\\  \hline 
 \begin{tikzpicture}
\draw (0,0) grid (2,2) ; 
\node at (1,1) [fill=red!20] {XXX};
\end{tikzpicture}
&  
\begin{tikzpicture}
\draw (0,0) grid (2,2) ; 
\node at (1,1) [fill=red!20,node contents=XXX] {};
\end{tikzpicture} 
\\ \hline 
\end{tabular} 

\bigskip

\begin{tabular}{|c|c|} \hline 
\BS{node}[red] at (1,1) [fill=blue!20] \AC{XXX} ;
&  
\BS{node}[red] at (1,1) [fill=blue20,node contents=XXX] \AC{};
\\  \hline 
 \begin{tikzpicture}
\draw (0,0) grid (2,2) ; 
\node[red] at (1,1) [fill=blue!20] {XXX};
\end{tikzpicture}
&  
\begin{tikzpicture}
\draw (0,0) grid (2,2) ; 
\node[red] at (1,1) [fill=blue!20,node contents=XXX] {};
\end{tikzpicture} 
\\ \hline 
\end{tabular} 


\SbSSCT{Premier ou arrière plan}{Behind or in front}

\begin{tabular}{|c|c|} \hline 
\multicolumn{2}{|l|}{\BS{tikz} \BS{fill} [fill=blue!50, draw=blue, very thick]
(0,0) } \\ 
\multicolumn{2}{|l|}{node [\RDD{behind path}, fill=red!50] \AC{XXXXX} }  \\
\multicolumn{2}{|l|}{- - (1.5,0) - - (1.5,1) - - (0,1) ;}
\\ \hline 
\tikz \fill [fill=blue!50, draw=blue, very thick]
(0,0) node [behind path, fill=red!50] {XXXXX}
-- (1.5,0) 
-- (1.5,1) 
-- (0,1) ;
&  
\tikz \fill [fill=blue!50, draw=blue, very thick]
(0,0) node [in front of path, fill=red!50] {XXXXX}
-- (1.5,0) 
-- (1.5,1) 
-- (0,1) ;
\\ \hline 
\RDD{behind path}
&  
\RDD{in front of path}
\\ \hline 
\end{tabular}



\SbSSCT{Noms à préfixe ou suffixe}{Name prefix or name suffix}


\begin{tabular}{|c|c|}
\hline 
\begin{tikzpicture}[every node/.style={draw},baseline=0pt]
\draw[name prefix = top-] node (A) at (1,1) {A} node (B) at (2,1) {B} node (C) at (3,1) {C};
\draw[name prefix = bottom-] node (1) at (1,0) {1} node (2) at (2,0) {2} node(3) at  (3,0) {3};
\draw [red] (top-A) -- (bottom-3);
\end{tikzpicture} 
&
\parbox{12cm}{
\BS{draw}[\RDD{name prefix} = \blll{top-} ] node (A) at (1,1) \AC{A} node (B) at (2,1) \AC{B} node (C) at (3,1) \AC{C}; \\
\BS{draw}[\RDD{name prefix} = \blll{bottom-}] node (1) at (1,0) \AC{1} node (2) at (2,0) \AC{2} node(3) at  (3,0) \AC{3}; \\
\BS{draw} [red] (\blll{top-}A) -- (\blll{bottom-}3);}
\\ \hline
\begin{tikzpicture}[every node/.style={draw},baseline=0pt]
\draw[name suffix= -top] node (A) at (1,1) {A} node (B) at (2,1) {B} node (C) at (3,1) {C};
\draw[name suffix=  -bottom] node (1) at (1,0) {1} node (2) at (2,0) {2} node(3) at  (3,0) {3};
\draw [red] (A-top) -- (3-bottom);
\end{tikzpicture}
&
\parbox{12cm}{
\BS{draw}[\RDD{name suffix} = \blll{-top}] node (A) at (1,1) \AC{A} node (B) at (2,1) \AC{B} node (C) at (3,1) \AC{C}; \\
\BS{draw}[\RDD{name suffix} = \blll{-bottom}] node (1) at (1,0) \AC{1} node (2) at (2,0) \AC{2} node(3) at  (3,0) \AC{3}; \\
\BS{draw} [red] (A \blll{-top}) - - (3 \blll{-bottom});}
\\ \hline 

\end{tabular} 


\SbSSCT{Liaisons}{Links}
\label{liaisons}

\begin{tabular}{|c|c|c|} \hline 
\multicolumn{3}{|l|}{\BS{node}[draw] (A) at (0,0) \AC{A}; \hspace{.5cm} \BS{node}[draw] (B) at (1.5,1.5) \AC{B}; \hspace{.5cm} \BS{draw} (A) - - (B) } \\ \hline 
\begin{tikzpicture}[blue]
\node[draw] (A) at (0,0) {A};
\node[draw] (B) at (1.5,1.5) {B};
\draw (A) -- (B);
\end{tikzpicture}
&  
\begin{tikzpicture}[blue]
\node[draw] (A) at (0,0) {A};
\node[draw] (B) at (1.5,1.5) {B};
\draw (A) |- (B);
\end{tikzpicture}
&  
\begin{tikzpicture}[blue]
\node[draw] (A) at (0,0) {A};
\node[draw] (B) at (1.5,1.5) {B};
\draw (A) -| (B);
\end{tikzpicture}
\\ \hline  
(A){\color{red} - -} (B) & (A) {\color{red}|-} (B) &  (A) {\color{red}-|} (B)
\\ \hline 
\begin{tikzpicture}[blue]
\node[draw] (A) at (0,0) {A};
\node[draw] (B) at (1.5,1.5) {B};
\draw (A) to [bend right] (B);
\end{tikzpicture}
&  
\begin{tikzpicture}[blue]
\node[draw] (A) at (0,0) {A};
\node[draw] (B) at (1.5,1.5) {B};
\draw (A) to [bend left] (B);
\end{tikzpicture}
&  
\begin{tikzpicture}[blue]
\node[draw] (A) at (0,0) {A};
\node[draw] (B) at (1.5,1.5) {B};
\draw (A) to[bend left=0] (B);
\end{tikzpicture}
\\ \hline  
(A) to [\RDD{bend right}] (B) & (A) to [\RDD{bend left}] (B) &  (A) to[\RDD{bend left}=0] (B)
\\ \hline 
\begin{tikzpicture}[blue]
\node[draw] (A) at (0,0) {A};
\node[draw] (B) at (1.5,1.5) {B};
\draw (A) to[bend left=120]  (B);
\end{tikzpicture}
&  
\begin{tikzpicture}[blue]
\node[draw] (A) at (0,0) {A};
\node[draw] (B) at (1.5,1.5) {B};
\draw (A) to[bend left=45] (B);
\end{tikzpicture}
&  
\begin{tikzpicture}[blue]
\node[draw] (A) at (0,0) {A};
\node[draw] (B) at (1.5,1.5) {B};
\draw (A) to[bend left=90] (B);
\end{tikzpicture}
\\ \hline  
(A)  to[\RDD{bend left}=120]  (B) & (A) to[\RDD{bend left}=45] (B) &  (A) to[\RDD{bend left}=90] (B)
\\ \hline 
\begin{tikzpicture}[blue]
\node[draw] (A) at (0,0) {A};
\node[draw] (B) at (1.5,1.5) {B};
\draw (A)  to[out=90]  (B);
\end{tikzpicture}
&  
\begin{tikzpicture}[blue]
\node[draw] (A) at (0,0) {A};
\node[draw] (B) at (1.5,1.5) {B};
\draw (A) to[out=30] (B);
\end{tikzpicture}
&  
\begin{tikzpicture}[blue]
\node[draw] (A) at (0,0) {A};
\node[draw] (B) at (1.5,1.5) {B};
\draw (A)  to[in=-90]  (B);
\end{tikzpicture}
\\ \hline  
(A)  to[\RDD{out}=90] (B) & (A) to[\RDD{out}=30]  (B) &  (A)  to[\RDD{in}=-90]  (B)
\\ \hline  
\end{tabular} 

\bigskip
\begin{tabular}{|c|c|c|} \hline  
\multicolumn{2}{|c|}{ \BS{draw} (A) .. controls +(right:2cm) and +(down:2cm) .. (B);  }\\ 
\hline  
\begin{tikzpicture}[blue]
\node[draw] (A) at (0,0) {A};
\node[draw] (B) at (2,2) {B};
\draw  (A) .. controls +(right:2cm) and +(down:2cm) .. (B);
\end{tikzpicture}
&
\begin{tikzpicture}[blue]
\node[draw] (A) at (0,0) {A};
\node[draw] (B) at (2,2) {B};
\draw  (A) .. controls +(up:1cm) and +(left:1cm) .. (B);
\end{tikzpicture}
\\ \hline 
controls +(right:2cm) and +(down:2cm)  &
controls +(up:1cm) and +(left:1cm)
\\ \hline 
\begin{tikzpicture}[blue]
\node[draw] (A) at (0,0) {A};
\node[draw] (B) at (2,2) {B};
\draw  (A) .. controls +(right:1cm) and +(right:2cm) .. (B);
\end{tikzpicture}
&
\begin{tikzpicture}[blue]
\node[draw] (A) at (0,0) {A};
\node[draw] (B) at (2,2) {B};
\draw  (A) .. controls +(up:1cm) and +(right:2cm) .. (B);
\end{tikzpicture}
\\ \hline 
controls +(right:1cm) and +(right:2cm)  &
controls +(up:1cm) and +(right:2cm) 
\\ \hline 
\begin{tikzpicture}[blue]
\node[draw] (A) at (0,0) {A};
\node[draw] (B) at (2,2) {B};
\draw  (A) .. controls +(120:2cm) and +(200:1cm) .. (B);
\end{tikzpicture}
 &
 \begin{tikzpicture}[blue]
 \node[draw] (A) at (0,0) {A};
 \node[draw] (B) at (2,2) {B};T
 \draw  (A) .. controls +(120:2cm) and +(200:1cm) .. (A);
 \end{tikzpicture}
\\  \hline  
controls +(120:2cm) and +(200:1cm) & controls +(120:2cm) and +(200:1cm) 
\\ \hline 
\begin{tikzpicture}[blue]
\node[draw] (A) at (0,0) {A};
\node[draw] (B) at (2,2) {B};
\node[draw] (C) at (0,1) {C};
\node[draw] (D) at (3,0) {D};
\draw  (A) .. controls +(C) and +(D) .. (B);
\end{tikzpicture}
&
\begin{tikzpicture}[blue]
\node[draw] (A) at (0,0) {A};
\node[draw] (B) at (2,2) {B};
\node[draw] (C) at (0,1) {C};
\node[draw] (D) at (3,0) {D};
\draw (A) .. controls +(D)  .. (B);
\end{tikzpicture}
\\ \hline 
controls +(C) and +(D) &
controls +(D) 
\\ \hline 
\end{tabular} 
 \bigskip
 
\begin{tabular}{|c|c|c|} \hline 
\multicolumn{3}{|l|}{ \BS{node}[draw] (A) at (0,0) \AC{A}  }\\

\multicolumn{3}{|l|}{ \BS{node}[draw] (B) at (2,2) \AC{B} \RDD{edge}  [->] (A);  }\\
\multicolumn{3}{|c|}{\RRR{17-12-1}}  \\
\hline 
 \begin{tikzpicture}
 \node[draw] (A) at (0,0) {A};
 \node[draw] (B) at (2,2) {B} edge [->] (A);
 \end{tikzpicture}
 &
 \begin{tikzpicture}
 \node[draw] (A) at (0,0) {A};
 \node[draw] (B) at (2,2) {B} edge [red]  (A);
 \end{tikzpicture}
 &
 \begin{tikzpicture}
 \node[draw] (A) at (0,0) {A};
 \node[draw] (B) at (2,2) {B} edge [dashed] (A);
 \end{tikzpicture}
\\ \hline 
[->] & [red]  & [dashed]
\\ \hline 
\end{tabular}

\SbSSCT{\'Etiquettes sur les n\oe uds}{Node labels}

\begin{tabular}{|c|c|c|c|} \hline
\multicolumn{4}{|c|}{  \BS{fill}(0,0) circle (2pt) node[\RDD{above}] \AC{texte} ; \RRR{17-5-2}   }\\ 
\hline 
  
\begin{tikzpicture} \draw[help lines] (-1,-1) grid (1,1) ;\fill (0,0) circle (2pt) node[above] {texte};\end{tikzpicture}
& 
\begin{tikzpicture} \draw[help lines] (-1,-1) grid (1,1) ;\fill (0,0) circle (2pt) node[below] {texte};\end{tikzpicture}
 &  
\begin{tikzpicture} \draw[help lines] (-1,-1) grid (1,1);\fill (0,0) circle (2pt) node[left] {texte};\end{tikzpicture}
 &  
\begin{tikzpicture} \draw[help lines] (-1,-1) grid (1,1); \fill (0,0) circle (2pt) node[right] {texte};\end{tikzpicture}
 \\  \hline 
 [\RDD{above}] & [\RDD{below}] & [\RDD{left}] &  [\RDD{right}]
 \\ \hline 
 \begin{tikzpicture} \draw[help lines] (-1,-1) grid (1,1) ;\fill (0,0) circle (2pt) node[above left] {texte};\end{tikzpicture}
 & 
 \begin{tikzpicture} \draw[help lines] (-1,-1) grid (1,1) ;\fill (0,0) circle (2pt) node[below left] {texte};\end{tikzpicture}
  &  
 \begin{tikzpicture} \draw[help lines] (-1,-1) grid (1,1);\fill (0,0) circle (2pt) node[above right] {texte};\end{tikzpicture}
  &  
 \begin{tikzpicture} \draw[help lines] (-1,-1) grid (1,1); \fill (0,0) circle (2pt) node[below right] {texte};\end{tikzpicture}
  \\  \hline 
  [\RDD{above left}] & [\RDD{below left}] & [\RDD{above right}] &  [\RDD{below right}]
  \\ \hline 
 \begin{tikzpicture} \draw[help lines] (-1,-1) grid (1,1) ;\fill (0,0) circle (2pt) node[anchor=south] {texte};\end{tikzpicture}
 & 
 \begin{tikzpicture} \draw[help lines] (-1,-1) grid (1,1) ;\fill (0,0) circle (2pt) node[anchor=west] {texte};\end{tikzpicture}
  &  
 \begin{tikzpicture} \draw[help lines] (-1,-1) grid (1,1);\fill (0,0) circle (2pt) node[anchor=north] {texte};\end{tikzpicture}
  &  
 \begin{tikzpicture} \draw[help lines] (-1,-1) grid (1,1); \fill (0,0) circle (2pt) node[anchor=east] {texte};\end{tikzpicture}
  \\  \hline 
  [\RDD{anchor}=south] & [\RDD{anchor}=west] & [\RDD{anchor}=north] & [\RDD{anchor}=east]                                                                                                                                                               ]
  \\ \hline 
 \begin{tikzpicture} \draw[help lines] (-1,-1) grid (1,1) ;\fill (0,0) circle (2pt) node[anchor=south east] {texte};\end{tikzpicture}
 & 
\begin{tikzpicture} \draw[help lines] (-1,-1) grid (1,1) ;\fill (0,0) circle (2pt) node[anchor=south west] {texte};\end{tikzpicture}
&  
\begin{tikzpicture} \draw[help lines] (-1,-1) grid (1,1);\fill (0,0) circle (2pt) node[anchor=north west] {texte};\end{tikzpicture}
&  
\begin{tikzpicture} \draw[help lines] (-1,-1) grid (1,1); \fill (0,0) circle (2pt) node[anchor=east] {texte};\end{tikzpicture}
\\  \hline 
[\RDD{anchor}=south east] & [\RDD{anchor}=south west] & [\RDD{anchor}=north west] & [\RDD{anchor==north east                                                                                                                                                       }]
  \\ \hline 
\end{tabular} 


\bigskip
\begin{tabular}{|c|c|c|c|} \hline
\multicolumn{4}{|c|}{  \BS{fill}(0,0) circle (2pt) node[\RDD{above}=.3cm] \AC{texte} ; \RRR{17-5-2}  }\\ 
\hline 
  
\begin{tikzpicture} \draw[help lines] (-1,-1) grid (1,1) ;\fill (0,0) circle (2pt) node[above=.3cm] {texte};\end{tikzpicture}
& 
\begin{tikzpicture} \draw[help lines] (-1,-1) grid (1,1) ;\fill (0,0) circle (2pt) node[below=.3cm] {texte};\end{tikzpicture}
 &  
\begin{tikzpicture} \draw[help lines] (-1,-1) grid (1,1);\fill (0,0) circle (2pt) node[left=.3cm] {texte};\end{tikzpicture}
 &  
\begin{tikzpicture} \draw[help lines] (-1,-1) grid (1,1); \fill (0,0) circle (2pt) node[right=.3cm] {texte};\end{tikzpicture}
 \\  \hline 
 [\RDD{above}=.3cm] & [\RDD{below}=.3cm] & [\RDD{left}=.3cm] &  [\RDD{right}=.3cm]]
 \\ \hline 
\begin{tikzpicture} \draw[help lines] (-1,-1) grid (1,1) ;\fill (0,0) circle (2pt) node[above left=.3cm] {texte};\end{tikzpicture}
& 
\begin{tikzpicture} \draw[help lines] (-1,-1) grid (1,1) ;\fill (0,0) circle (2pt) node[below left=.3cm] {texte};\end{tikzpicture}
 &  
\begin{tikzpicture} \draw[help lines] (-1,-1) grid (1,1);\fill (0,0) circle (2pt) node[above right=.3cm] {texte};\end{tikzpicture}
 &  
\begin{tikzpicture} \draw[help lines] (-1,-1) grid (1,1); \fill (0,0) circle (2pt) node[below right=.3cm] {texte};\end{tikzpicture}
 \\  \hline 
 [\RDD{above left}=.3cm] & [\RDD{below left}=.3cm] & [\RDD{above right}=.3cm] &  [\RDD{below right}=.3cm]]
 \\ \hline 
 
 \end{tabular} 

 
 \newpage
\selectlanguage{french}
 
 \begin{tabular}{|c|c|c|c|c|} \hline
 \multicolumn{5}{|l|}{ \BSS{shorthandoff}\AC{:} \footnotemark[1]  } \\
 \multicolumn{5}{|l|}{  \BS{node} [draw,\RDD{label}=right:texte] \AC{}   }\\
 \multicolumn{5}{|l|}{ \BSS{shorthandon}\AC{:} } \\ 
 \hline 
     \shorthandoff{:} 
 \tikz \node [draw,label=right:texte] {};
 \shorthandon{:}
 &
  \shorthandoff{:}
 \tikz \node [draw,label=left:texte] {};
 \shorthandon{:}
 &
  \shorthandoff{:}
 \tikz \node [draw,label=above:texte] {};
 \shorthandon{:}
 &
  \shorthandoff{:}
 \tikz \node [draw,label=below:texte] {};
 \shorthandon{:}
 &
  \shorthandoff{:}
 \tikz \node [draw,label=45:texte] {};
    \shorthandon{:}
   \\ \hline
  label=right & label=left &  label=above & label=below & label=45
    \\ \hline 
 \end{tabular}
 \footnotetext[1]{\TFRGB{désactivation et ré-activation de \og : \fg  conflit entre les modules Tikz et Babel en français}{Only useful when the package babel is loaded with the frenchb option    }}
 
 \bigskip
  \begin{tabular}{|c|c|c|c|c|} \hline
  \BS{fill}(0,0) circle (2pt) node[below right=.3cm,draw,label=45:étiquette] \AC{texte} ;
      \\ \hline 
  
  \shorthandoff{:}
\begin{tikzpicture} \draw[help lines] (-1,-1) grid (2,1); \fill (0,0) circle (2pt) node[below right=.3cm,draw,label=45:étiquette] {texte};\end{tikzpicture}
 \shorthandon{:}
 
    \\ \hline 
 \end{tabular}
\bigskip

 \shorthandoff{:}

\SbSSCT{\'Etiquettes épinglées}{The Pin Option} 

\begin{center}
\RRR{17-10-3}
\end{center}
 
\begin{tabular}{|c|c|c|} \hline
\multicolumn{3}{|c|}{  \BSS{shorthandoff}\AC{:} \BS{node}[circle,draw,blue,\RDD{pin}=texte] \AC{} ;   \BSS{shorthandon}\AC{:}  \footnotemark[1] }\\ 
\hline
\begin{tikzpicture} 
\node [circle,draw,blue,pin=texte] {};
\end{tikzpicture}
&
\begin{tikzpicture} 
\node [circle,draw,blue,pin=60:texte] {};
\end{tikzpicture}
&
\begin{tikzpicture} 
\node [circle,draw,blue,pin=right:texte] {};
\end{tikzpicture}
 \\ \hline
[circle,pin=texte] &   [circle,pin=60:texte] & [circle,pin=right:texte]
 \\ \hline 
\end{tabular}  

\bigskip
\begin{tabular}{|c|c|c|} \hline
\multicolumn{3}{|c|}{  \BS{tikz}[\RDD{pin position}=60] \BS{node} [circle,pin=texte] \AC{} ;   }\\ 
\hline 
\tikz[pin position=60] \node [circle,draw,blue,pin=texte] {};
&
\tikz[pin distance=0 cm] \node [circle,draw,blue,pin=60:texte] {};
&
\tikz[pin distance=2 cm] \node [circle,draw,blue,pin=60:texte,pin distance=0cm] {};
  \\ \hline
  [\RDD{pin position}=60] & [\RDD{pin distance}=0 cm] & [\RDD{pin distance}=2 cm]
    \\ \hline
  \dft{ : above} & \multicolumn{2}{|c|}{ \dft{ : 3 ex}}
      \\ \hline
\end{tabular}  

\newpage

   \shorthandon{:} 
   
\selectlanguage{english}   

\SbSSCT{ N\oe uds  sur un chemin}{Nodes on a path}

\RRR{17-8}

\begin{tabular}{|c|c|c|} \hline
\multicolumn{3}{|c|}{  \BS{draw}(0,0) .. controls (1,2) and (2,-1) .. (4,0) node[\RDD{at end}] \AC{texte} ;   }\\ 
\hline 
\tikz \draw (0,0) .. controls (1,2) and (2,-1) .. (4,0) node[pos=0] {texte}; 
&
\tikz \draw (0,0) .. controls (1,2) and (2,-1) .. (4,0) node[pos=.33] {texte}; 
&
\tikz \draw (0,0) .. controls (1,2) and (2,-1) .. (4,0) node[at end] {texte}; 
  \\ \hline 
\RDD{pos}{\color{red}  =0} & \RDD{pos}{\color{red}  =.33} & \RDD{at end} (pos=1)
  \\ \hline 

\tikz \draw (0,0) .. controls (1,2) and (2,-1) .. (4,0) node[very near end] {texte}; 
&
\tikz \draw (0,0) .. controls (1,2) and (2,-1) .. (4,0) node[near end] {texte}; 
&
\tikz \draw (0,0) .. controls (1,2) and (2,-1) .. (4,0) node[midway] {texte}; 
  \\ \hline 
\RDD{very near end} (pos=0.875.) & \RDD{ near end} (pos=0.75) & \RDD{midway} (pos=0.5)
  \\ \hline 
  
\tikz \draw (0,0) .. controls (1,2) and (2,-1) .. (4,0) node[near start] {texte}; 
&
\tikz \draw (0,0) .. controls (1,2) and (2,-1) .. (4,0) node[very near start] {texte}; 
&
\tikz \draw (0,0) .. controls (1,2) and (2,-1) .. (4,0) node[at start] {texte};
\\ \hline 
\RDD{near start} (pos=0.25) & \RDD{very near start} (pos=0.125) & \RDD{at start} (pos=0)
  \\ \hline 
  
\end{tabular} 

\bigskip
\begin{tabular}{|c|c|c|} \hline
\multicolumn{3}{|c|}{  \BS{draw}(0,0) .. controls (1,2) and (2,1) .. (4,0) node[\RDD{sloped},midway] \AC{texte} ;   }\\ 
\hline 
\tikz \draw (0,0) .. controls (1,2) and (2,-1) .. (4,0) node[sloped,midway] {texte};
&
\tikz \draw (0,0) .. controls (1,2) and (2,-1) .. (4,0) node[above,midway] {texte};
&
\tikz \draw (0,0) .. controls (1,2) and (2,-1) .. (4,0) node[below,midway] {texte};
  \\ \hline
\RDD{sloped} & \RDD{above} &\RDD{below}
  \\ \hline
\end{tabular}
\bigskip

\begin{tabular}{|c|c|c|} \hline
\multicolumn{3}{|c|}{  \BS{draw}(0,0) .. controls (1,2) and (2,1) .. (5,0) node[\RDD{sloped},midway,allow upside down] \AC{texte} ;   }\\ 
\hline 
\tikz \draw (0,0) .. controls (1,2) and (2,-1) .. (4,0) node[sloped,midway,allow upside down] {texte};
&
\tikz \draw (0,0) .. controls (1,2) and (2,-1) .. (4,0) node[above,midway,allow upside down] {texte};
&
\tikz \draw (0,0) .. controls (1,2) and (2,-1) .. (4,0) node[below,midway,allow upside down] {texte};
  \\ \hline
\RDD{sloped} & \RDD{above} &\RDD{below}
  \\ \hline
\end{tabular}  


\begin{tabular}{|c|c|c|} \hline
\multicolumn{3}{|c|}{  \BS{draw}(A)  to [bend right]  node [\RDD{bend right}] \AC{texte} (B);   }\\ 
\hline 
\begin{tikzpicture} 
\node[draw] (A) at (0,0) {A};
\node[draw] (B) at (2,2) {B};
\draw (A) to [bend right] node [bend right] {texte} (B);
\end{tikzpicture}
&
\begin{tikzpicture} 
\node[draw] (A) at (0,0) {A};
\node[draw] (B) at (2,2) {B};
\draw (A) to [bend right] node [auto,bend right] {texte} (B);
\end{tikzpicture}
&
\begin{tikzpicture} 
\node[draw] (A) at (0,0) {A};
\node[draw] (B) at (2,2) {B};
\draw (A) to[bend right] node [auto,swap,bend right] {texte} (B);
\end{tikzpicture}
  \\ \hline
[bend right]  & [\RDD{auto},bend right] & [auto,\RDD{swap},bend right] 
  \\ \hline
\end{tabular}  

\SbSSCT{ N\oe uds  sur un \og edge\fg}{Nodes on an edge}

\begin{tabular}{|c|c|c|}\hline  
\multicolumn{3}{|c|}{  \BS{draw}(0,0) edge \rouge{["abc", ->]} (4,0);  }\\ 
\multicolumn{3}{|c|}{  \RRR{17-12-2} }\\ 
\hline 
\begin{tikzpicture}[blue] 
\useasboundingbox  (0,-.5) rectangle (4,.5); 
\draw (0,0) edge ["abc", ->] (4,0);
\end{tikzpicture}
&
\begin{tikzpicture}[blue] 
\useasboundingbox  (0,-.5) rectangle (4,.5); 
\draw (0,0) edge ["abc", near start] (4,0);
\end{tikzpicture}
&
\begin{tikzpicture}[blue] 
\useasboundingbox  (0,-.5) rectangle (4,.5); 
\draw (0,0) edge ["abc", style={auto=right}] (4,0);
\end{tikzpicture}
\\ \hline 
["abc", ->]
& 
["abc", near start] &  ["abc", style=\AC{auto=right}] 
\\ \hline  
\begin{tikzpicture}[blue] 
\useasboundingbox  (0,-.5) rectangle (4,.5); 
\draw (0,0) edge [font=\Large,"abc" ] (4,0);
\end{tikzpicture}
&
\begin{tikzpicture}[blue] 
\useasboundingbox  (0,-.5) rectangle (4,.5); 
\draw (0,0) edge ["abc" color=red ] (4,0);
\end{tikzpicture}
&
\begin{tikzpicture}[blue] 
\useasboundingbox  (0,-.5) rectangle (4,.5); 
 \draw (0,0) edge ["abc" '] (4,0);
\end{tikzpicture}
\\ \hline 
[font=\BS{Large},"abc" ] & ["abc" color=red ]
&["abc" ' ]
\\ \hline 

\begin{tikzpicture}[blue] 
\useasboundingbox  (0,-.5) rectangle (4,.75); 
\draw (0,0) edge ["abc" draw ] (4,0);
\end{tikzpicture}
&
\begin{tikzpicture}[blue] 
\useasboundingbox  (0,-.5) rectangle (4,.5); 
\draw (0,0) edge ["abc" inner sep=0pt ] (4,0);
\end{tikzpicture}
&
\begin{tikzpicture}[blue] 
\useasboundingbox  (0,-.5) rectangle (4,.5); 
\draw (0,0) edge ["abc" fill ,fill=yellow ] (4,0);
\end{tikzpicture}
\\ \hline
["abc" draw ]
&
["abc" inner sep=0pt ]
&
["abc" fill ,fill=yellow ]
\\ \hline
\end{tabular} 



\bigskip

\begin{tabular}{|c|} \hline  
\BS{draw}[every edge quotes/.style=\AC{fill=yellow}] (0,0) edge ["abc"] (4,0);
\\ \hline  
\begin{tikzpicture}[blue] 
\useasboundingbox  (0,-.5) rectangle (4,.5); 
 \draw[every edge quotes/.style={fill=yellow}] (0,0) edge ["abc"] (4,0);
\end{tikzpicture}
\\ \hline 
\end{tabular} 

%
%\newpage
%
%\subsection{Positionnement relatif de n\oe uds}
\label{lib-pos}

\maboite{\BS{usetikzlibrary}\AC{positioning}}


\begin{center}
\RRR{17-5-3}
\end{center}

\begin{tabular}{|c|c|c|}  \hline 
\multicolumn{2}{|c|}{\BS{node} (a) at (1,0) [above=.4cm+.6cm,draw] \AC{XXX};} &  \\ \hline 
\begin{tikzpicture}
\draw[help lines] (0,0) grid (3,2);
\node (a) at (1,0) [above=.4cm+.6cm,draw] {XXX};
\draw[->,blue,line width=2pt,dotted] (1,0) -- (a.south) node [midway,right,draw=none,fill=red!10] {.4cm+.6cm} ;
\end{tikzpicture} 
&
\begin{tikzpicture}
\draw[help lines] (0,0) grid (3,2);
\node (a) at (1,0) [above=.5+sin(60),draw] {XXX};
\draw[->,blue,line width=2pt,dotted] (1,0) -- (a.south) node [midway,right,draw=none,fill=red!10] {.5+sin(60)} ;
\end{tikzpicture}  
&
\begin{tikzpicture}
\draw[help lines] (0,0) grid (2,2);
\node (a) at (1,0) [above=1,draw] {XXX};
\draw[->,blue,line width=2pt,dotted] (1,0) -- (a.south) node [midway,right,draw=none,fill=red!10] {1} ;
\end{tikzpicture}  
\\ \hline 
above = \rouge{0.4cm+0.6cm} & above = \rouge{.5+sin(60)}  & above = \rouge{1} \\ 
\hline 
\end{tabular} 

\bigskip

\begin{tabular}{|c|c|} \hline 
\multicolumn{2}{|c|}{\BS{node} (a) at (1,0) [\rouge{above right=3cm and 2cm},draw] \AC{XXX};} \\  \hline 
\begin{tikzpicture}
\draw[help lines] (0,0) grid (5,5);
\node (a) at (1,1) [above right=3cm and 2cm,draw] {XXX};
\draw[->,blue,line width=2pt,dotted] (1,1) |- (a.south west);
\end{tikzpicture}
&  
\begin{tikzpicture}
\draw[help lines] (0,0) grid (5,5);

\node (b) at (1,4) [below right=3cm and 2cm,draw] {XXX};
\draw[->,blue,line width=2pt,dotted] (1,4) |- (b.north west);
\end{tikzpicture}
\\ \hline 
\rouge{above right=3cm and 2cm} & \rouge{below right=3cm and 2cm}
\\ \hline 
\end{tabular}  

\bigskip
 
\begin{tabular}{|c|c|}  \hline 
\begin{tikzpicture}[every node/.style=draw,baseline=1.5cm]
\draw[help lines] (0,0) grid (5,4);
\node (a) at (1,1) {node a};
\node (b) [above=2cm of a.north east] {XXX};
\draw[->,blue,line width=2pt,dotted] (a.north) -- (b.south) node [midway,right,draw=none,fill=red!10] {2cm of a.north east} ;
\end{tikzpicture}
&  
\parbox{8cm}{
\BS{node} (a) at (1,1) \AC{node a}; \\
\BS{node} (b) [\rouge{above=2cm of a.north east}] \AC{XXX};}
\\ \hline 
\end{tabular} 

\bigskip

\begin{tabular}{|c|c|}  \hline 
\begin{tikzpicture}[every node/.style=draw]
\draw[help lines] (0,0) grid (2,3);
\node (a) at (1,0) {node a};
\node (b) [above=1cm of a] {node b};
\node (c) [above=1cm of b] {node c};
\draw[->,blue,line width=2pt,dotted] (a.north) -- (b.south) node [midway,right,draw=none,fill=red!10] {1cm} ;
\draw[->,blue,line width=2pt,dotted] (b.north) -- (c.south) node [midway,right,draw=none,fill=red!10] {1cm} ;
\end{tikzpicture}
&  
\begin{tikzpicture}[every node/.style=draw]
\draw[help lines] (0,0) grid (2,3);
\node (a) at (1,0) {node a };
\node (b) [on grid,above=1cm of a] {node b};
\node (c) [on grid,above=1cm of b] {node c};
\draw[->,blue,line width=2pt,dotted] (a.center) -- (b.center) node [midway,right,draw=none,fill=red!10] {1cm} ;
\draw[->,blue,line width=2pt,dotted] (b.center) -- (c.center) node [midway,right,draw=none,fill=red!10] {1cm} ;
\end{tikzpicture}
\\  \hline 
\BS{node} (a) at (1,0) \AC{node a};  &\BS{node} (a) at (1,0) \AC{node a};   \\ 
\BS{node} (b) [above=1cm of a] \AC{node b};  &\BS{node} (b) [\RDD{on grid},above=1cm of a] \AC{node b};   \\ 
\BS{node} (c) [above=1cm of b] \AC{node c};  &\BS{node} (c) [\RDD{on grid},above=1cm of b] \AC{node c};   \\ 
\hline 
\end{tabular} 

\begin{tabular}{|c|c|} \hline 
\begin{tikzpicture}[every node/.style=draw,node distance=1cm,baseline = 1.5cm]
\draw[help lines] (0,0) grid (2,3);
\node (a1) at (1,0) {node a};
\node (b) [above=of a] {node b};
\node (c) [above=of b] {node c};
\draw[->,blue,line width=2pt,dotted] (a.north) -- (b.south) node [midway,right,draw=none,fill=red!10] {1cm} ;
\draw[->,blue,line width=2pt,dotted] (b.north) -- (c.south) node [midway,right,draw=none,fill=red!10] {1cm} ;
\end{tikzpicture}
 & 
 \parbox{12cm}{ 
\BS{begin}\AC{tikzpicture}[every node/.style=draw,\RDD{node distance}=1mm] \\
\BS{node} (a1) at (1,0) \AC{node a}; \\
\BS{node} (b) [above=of a] \AC{node b}; \\
\BS{node} (c) [above=of b] \AC{node c}; \\
\BS{end}\AC{tikzpicture}
} 
 \\ 
\hline 
\end{tabular} 

\bigskip

\begin{tabular}{|l|l|} \hline 
\begin{tikzpicture}[node distance=2cm]
\draw[help lines] (0,-1) grid (6,1);
\huge
\node[draw] (X) at (0,0) {X};
\node[draw] (a) [right=of X] {a};
\node[draw] (y) [right=of a] {y};
\draw[->,blue,line width=2pt,dotted] (X.east) -- (a.west) node [midway,draw=none,fill=red!10] {\small{2cm}} ;
\draw[->,blue,line width=2pt,dotted] (a.east) -- (y.west) node [midway,draw=none,fill=red!10] {\small{2cm}} ;
\end{tikzpicture}
&  
\begin{tikzpicture}[node distance=2cm]
\draw[help lines] (0,-1) grid (6,1);
\huge
\node[draw] (X) at (0,0) {X};
\node[draw] (a) [base right=of X] {a};
\node[draw] (y) [base right=of a] {y};
\draw[->,blue,line width=2pt,dotted] (X.base east) -- (a.base west) node [midway,draw=none,fill=red!10] {\small{2cm}} ;
\draw[->,blue,line width=2pt,dotted] (a.base east) -- (y.base west) node [midway,draw=none,fill=red!10] {\small{2cm}} ;
\end{tikzpicture}
\\ \hline 
\BS{node}[draw] (X) at (0,0) \AC{X};
&  
\BS{node}[draw] (X) at (0,0) \AC{X};
\\
\BS{node}[draw] (a) [right=of X] \AC{a};
&
\BS{node}[draw] (a) [base right=of X] \AC{a};
\\
\BS{node}[draw] (y) [right=of a] \AC{y};
&
\BS{node}[draw] (y) [base right=of a] \AC{y};
\\ \hline 
\end{tabular} 


%
%\newpage
%
%\SSCT{Mettre du texte  en valeur}{Text highlighting}
%
%\label{ndbt}

\tikzset{blue}


\SbSSCT{Dans un n\oe ud de Tikz}{In a TikZ node}
\label{noeudboite}

\begin{tabular}{|c | c | c | c |} \hline
\multicolumn{4}{|c|}{ \BS{tikz} \BS{draw} (0,0) grid (2,2) (1,1) node[ fill=red!20 ] \AC{texte};   }\\ 
\hline 
\tikz \draw (0,0) grid (2,2) (1,1) node[fill=red!20] {texte};
&
\tikz \draw (0,0) grid (2,2) (1,1) node[fill=red!20,draw] {texte}; 
&
\tikz \draw (0,0) grid (2,2) (1,1) node[circle,fill=red!20] {texte};
&
\tikz \draw (0,0) grid (2,2) (1,1) node[circle,fill=red!20,draw] {texte};
\\  \hline
node[fill=red!20] 
&
node[fill=red!20,\RDD{draw}] 
&
 node[fill=red!20,\RDD{circle}]  
&
 node[fill=red!20,\RDD{circle},\RDD{draw}]
 \\  \hline
\end{tabular}
\bigskip


\subsubsection{Options}
\begin{tabular}{|c | c | c | c |c |c |c |c |} \hline
\multicolumn{8}{|c|}{ \BS{tikz} \BS{draw} node[draw,\RDD{double},blue] \AC{texte};   }\\ 
\hline 

\tikz \draw  node[draw,double,blue] {texte};
&
\tikz \draw  node[draw,rounded corners,blue] {texte};
&
\tikz \draw  node[draw,ultra thick,blue] {texte};
&
\tikz \draw  node[draw,dashed,blue] {texte};
&
\tikz \draw  node[draw,red] {texte};
&
\tikz \draw  node[draw,rotate=45,blue] {texte};
&
\tikz \draw  node[draw,shading=radial,blue] {texte};
&
\tikz \draw  node[draw,blue,text=red] {texte};
\\ \hline
\RDD{double} & \RDD{rounded corners} &  ultra thick & dashed & red & rotate=45 & shading=radial & text=red 
\\ \hline
\end{tabular}
\bigskip


\begin{tabular}{|c | c | c | c |c |} \hline
\multicolumn{4}{|c|}{ \BS{tikz} \BS{draw}  node[draw,\RDD{inner sep}=0pt] \AC{texte}; \RRR{17-2-3}  }\\ 
\hline 
\tikz \draw  node[draw,inner sep=0pt,blue] {texte};
&
\tikz \draw node[draw,inner sep=1cm,blue] {texte};
&
\tikz \draw  node[draw,inner xsep=1cm,blue] {texte};
&
\tikz \draw  node[draw,inner ysep=1cm,blue] {texte};
\\ \hline
 \RDD{inner sep}=0pt & \RDD{inner sep}=1cm & \RDD{inner xsep}=1cm & \RDD{inner ysep}=1cm
\\ \hline
\multicolumn{4}{|c|}{ \dft{} : 0.3333em }\\ 
\hline 

\end{tabular}

\bigskip

\begin{tabular}{|c | c | c | c |} \hline
\multicolumn{4}{|l|}{ \BS{node} [fill=red!20,\RDD{outer sep}=1cm] (A) at (1,1) \AC{texte}; \RRR{17-2-3} } \\ 
\multicolumn{4}{|l|}{ \BS{fill} (node cs:name=A,anchor=east) circle (3pt);  }\\ 
\multicolumn{4}{|l|}{ \BS{fill} (node cs:name=A,anchor=south) circle (3pt);  }\\ 
\hline 
\begin{tikzpicture}
\draw[help lines] (0,0) grid (3,2);
\node[fill=red!20,outer sep=1cm] (A) at (1,1) {texte};
\fill[red] (node cs:name=A,anchor=east) circle (3pt);
\fill[red] (node cs:name=A,anchor=south) circle (3pt);
\end{tikzpicture}
&
\begin{tikzpicture}
\draw[help lines] (0,0) grid (3,2);
\node[fill=red!20,outer sep=0pt] (A) at (1,1) {texte};
\fill[red] (node cs:name=A,anchor=east) circle (3pt);
\fill[red] (node cs:name=A,anchor=south) circle (3pt);
\end{tikzpicture}
&
\begin{tikzpicture}
\draw[help lines] (0,0) grid (3,2);
\node[fill=red!20,outer xsep=1cm] (A) at (1,1){texte};
\fill[red] (node cs:name=A,anchor=east) circle (3pt);
\fill[red] (node cs:name=A,anchor=south) circle (3pt);
\end{tikzpicture}
&
\begin{tikzpicture}
\draw[help lines] (0,0) grid (3,2);
\node[fill=red!20,outer ysep=1cm] (A) at (1,1) {texte};
\fill[red] (node cs:name=A,anchor=east) circle (3pt);
\fill[red] (node cs:name=A,anchor=south) circle (3pt);
\end{tikzpicture}
\\ \hline
 \RDD{outer sep}=1cm & \RDD{outer sep}=0pt & \RDD{outer xsep}=1cm & \RDD{outer ysep}=1cm
\\ \hline
\multicolumn{4}{|c|}{ \dft{} : 0.5\BS{pgflinewidth} }\\ 
\hline 
\end{tabular}

\SbSbSSCT{Taille minimale des noeuds}{Minimum size}

\begin{tabular}{|c|c|} \hline  
\multicolumn{2}{|c|}{  \BS{draw}((0,0) node[fill=blue!20,\RDD{minimum height}=1.5cm,draw]  \AC{texte} ;  \RRR{17-2-3}  }\\ 
\hline 
\tikz \draw (0,0) node[fill=red!20,minimum height=1.5cm,draw] {texte};
&  
\tikz \draw (0,0) node[fill=red!20,minimum width=3cm,draw] {texte};

\\ \hline  

\RDD{minimum height}=1.5cm
&  
\RDD{minimum width}=3cm
\\ \hline  
\tikz \draw (0,0) node[fill=red!20,minimum size=1.5cm,draw] {texte};
&  
\tikz \draw (0,0) node[fill=red!20,minimum size=1.5cm,draw,circle] {texte};

\\ \hline 
\RDD{minimum size}=1.5cm,draw
&  
\RDD{minimum size}=1.5cm,circle

\\ \hline 
\end{tabular} 

\newpage

\SbSSCT{Dans un n\oe ud à formes géométriques}{Geometric Shapes nodes}

\label{lib-geom}
\label{formes}


 \maboite{\BS{usetikzlibrary}\AC{shapes.geometric}}
 
 
\begin{center}
\RRR{67-3}
\end{center}

\SbSbSSCT{Formes disponibles}{Available shapes}

\label{nd1}

\begin{tabular}{|c|c|c|c|} \hline  
\multicolumn{4}{|l|}{ 2 syntaxes :   }\\ 
\multicolumn{4}{|l|}{ \BS{tikz} \BS{node}[fill=green!20,\RDD{shape}=diamond,draw,blue] \AC{texte};   }\\ 
\multicolumn{4}{|l|}{ \BS{tikz} \BS{node}[fill=green!20,\RDD{diamond},draw] \AC{texte};   }\\ 
\hline 
\tikz  \node[fill=green!20,diamond,draw] {texte}; 
&  
\tikz  \node[fill=green!20,ellipse,draw] {texte};
&  
\tikz  \node[fill=green!20,trapezium, regular polygon sides=6,draw] {texte};
&
\tikz  \node[fill=green!20,semicircle,draw] {texte}; 
\\ \hline 
diamond & ellipse  & trapezium & semicircle
\\ \hline 
\tikz  \node[fill=green!20,star,draw] {texte};
&  
\tikz  \node[fill=green!20,regular polygon,draw] {texte};
&  
\tikz  \node[fill=green!20,isosceles triangle,draw] {texte};
&
\tikz  \node[fill=green!20,kite,draw] {texte};
\\ \hline 
star & regular polygon  & isosceles triangle & kite 
\\ \hline 
\tikz  \node[fill=green!20,dart,draw] {texte};
&
\tikz  \node[fill=green!20,circular sector,draw] {texte};
&
\tikz  \node[fill=green!20,cylinder,draw] {texte};
&

\\ \hline 
dart & circular sector & cylinder &
\\ \hline 
\end{tabular} 

\subsubsection{Options}

\begin{tabular}{|c|c|c|} \hline
\multicolumn{3}{|c|}{  \BS{node} [trapezium,draw,\RDD{trapezium left angle}=90,draw,blue] \AC{texte};   }\\ 
\hline
\begin{tikzpicture}
\node[trapezium,draw,red,dashed] {texte};
\node[trapezium,draw,trapezium left angle=90,draw,blue] {texte};
\end{tikzpicture}
& 
\begin{tikzpicture}
\node[trapezium,draw,red,dashed] {texte};
\node[trapezium,draw,trapezium right angle=90,draw,blue] {texte};
\end{tikzpicture} 
& 
\begin{tikzpicture}
\node[trapezium,draw,red,dashed] {texte};
\node[trapezium,draw,trapezium angle=120,draw,blue] {texte};
\end{tikzpicture} 
\\ \hline
\RDD{trapezium left angle}=90  & \RDD{trapezium right angle}=90  & \RDD{trapezium  angle}=120 \\ 
\hline 
\begin{tikzpicture}
\node[trapezium,draw,red,dashed] {texte};
\node[trapezium,draw,minimum height=1.5cm,trapezium stretches=true,draw,blue] {texte};
\end{tikzpicture}
& 
\begin{tikzpicture}
\node[trapezium,draw,red,dashed] {texte};
\node[trapezium,draw,minimum height=1.5cm,trapezium stretches=false,draw,blue] {texte};
\end{tikzpicture} 
& 
\begin{tikzpicture}
\node[trapezium,draw,red,dashed] {texte};
\node[trapezium,draw,minimum width=3cm,trapezium stretches =false,draw,blue] {texte};
\end{tikzpicture} 

\\ \hline
minimum height=1.5cm & minimum height=1.5cm & minimum width=1.5cm \\
\RDD{trapezium stretches}=true & \RDD{trapezium stretches}=false & \RDD{trapezium stretches}  \\ 
\hline

\end{tabular} 


\bigskip
\begin{tabular}{|c|c|c|} \hline
\multicolumn{3}{|c|}{ \BS{tikz} \BS{node} [fill=green!20,star,\RDD{star points}=6,draw] \AC{texte};   }\\ 
\hline
\begin{tikzpicture}
\node[star,draw,red,dashed] {texte};
\node[star,star points=7,draw,blue] {texte};
\end{tikzpicture}
&  
\begin{tikzpicture}
\node[star,draw,red,dashed] {texte};
\node[star,star point height = 2cm,draw,blue] {texte};
\end{tikzpicture} 
&  
\begin{tikzpicture}
\node[star,draw,red,dashed] {texte};
\node[star,star point ratio = 3,draw,blue] {texte};
\end{tikzpicture} 
\\ \hline  
\RDD{star points}=7 & \RDD{star point height} = 2cm & \RDD{star point ratio} = 3 \\ \hline
\dft{5} & \dft.5cm &  \dft{1.5}\\ 
\hline 
\end{tabular} 
\bigskip

\begin{tabular}{|c|c|c|} \hline
\multicolumn{3}{|c|}{  \BS{node} [isosceles triangle,\RDD{isosceles triangle apex angle}=90,draw,blue] \AC{texte};   }\\ 
\multicolumn{3}{|c|}{  \BS{node} [regular polygon, \RDD{regular polygon sides}=6,draw,blue] \AC{texte};   }\\ 
\hline
\begin{tikzpicture}
\node[isosceles triangle,draw,red,dashed] {texte};
 \node[isosceles triangle,isosceles triangle apex angle=90,draw,blue] {texte};
\end{tikzpicture} 
& 
\begin{tikzpicture}
\node[isosceles triangle,draw,red,dashed] {texte};
 \node[isosceles triangle,isosceles triangle stretches=true,draw,blue] {texte};
\end{tikzpicture}
&  
\begin{tikzpicture}
\node[regular polygon,draw,red,dashed] {texte};
\node[regular polygon, regular polygon sides=6,draw,blue] {texte};
\end{tikzpicture} 
\\ \hline  
\RDD{isosceles triangle apex angle}=90 & \RDD{isosceles triangle stretches} & \RDD{regular polygon sides}=6 \\ 
\hline 
\end{tabular} 
\bigskip

\begin{tabular}{|c|c|c|} \hline 
\multicolumn{3}{|c|}{  \BS{node} [kite,\RDD{kite upper vertex angle}=90,draw,blue] \AC{texte};   }\\ 
\hline 
\begin{tikzpicture}
\node[red,kite,draw,dashed] {texte} ;
 \node[kite,kite upper vertex angle=90,draw,blue] {texte};
\end{tikzpicture} 
&  
\begin{tikzpicture}
\node[red,kite,draw,dashed] {texte} ;
 \node[kite,kite lower vertex angle=90,draw,blue] {texte};
\end{tikzpicture} 
&  
\begin{tikzpicture}
\node[red,kite,draw,dashed] {texte} ;
\node[kite,kite vertex angles=90,draw,blue] {texte};
\end{tikzpicture} 
\\ \hline  
\RDD{kite upper vertex angle}=90 & \RDD{kite lower vertex angle}=90 &\RDD{kite vertex angles}=90
\\ \hline 
initially 120 & initially 60 &  \\ 
\hline 
\end{tabular} 

\bigskip

\begin{tabular}{|c|c|c|} \hline
\multicolumn{3}{|c|}{  \BS{node} [dart,\RDD{dart tip angle}=90,draw,blue] \AC{texte};   }\\ 
\hline 
\begin{tikzpicture}
\node[dart,draw,red,dashed] {texte};
\node[dart,dart tip angle=90,draw,blue] {texte};
\end{tikzpicture} 
&  
\begin{tikzpicture}
\node[dart,draw,red,dashed] {texte};
\node[dart,dart tail angle=90,draw,blue] {texte};
\end{tikzpicture} 
&  
\begin{tikzpicture}
\node[,circular sector,draw,red,dashed] {texte};
\node[circular sector,circular sector angle=90,draw,blue] {texte};
\end{tikzpicture} 
\\ \hline  
\RDD{dart tip angle}=90 & \RDD{dart tail angle}=90  & \RDD{circular sector angle}=90
\\ \hline  
initially 45 & initially 135 & initially 60  \\ 
\hline 
\end{tabular} 

\bigskip

\begin{tabular}{|c|c|} \hline  
\multicolumn{2}{|c|}{  \BS{node} [cylinder,\RDD{aspect=2},draw,blue] \AC{texte};   }\\ 
\hline
\tikz  \node[cylinder,aspect=2,draw,blue] {texte};
& 
 \tikz  \node[cylinder,aspect=4,draw,blue] {texte};
\\ \hline 
\RDD{aspect}=2 & \RDD{aspect}=4 
\\ \hline
\tikz  \node[cylinder,cylinder uses custom fill, cylinder end fill=yellow,draw,blue] {texte};
&  
\tikz  \node[cylinder,cylinder uses custom fill, cylinder body fill=yellow,draw,blue] {texte};
\\ \hline
\RDD{cylinder uses custom fill}, & \RDD{cylinder uses custom fill}, \\ 
\RDD{cylinder end fill}=yellow & \RDD{cylinder body fill}=yellow  \\ 
\hline 
\end{tabular} 

\bigskip

\begin{tabular}{|c|c|c|c|} \hline 
\multicolumn{4}{|c|}{  \BS{draw}(0,0) node[\RDD{shape aspect}=1,diamond,draw]  \AC{texte} ;   }
\\ \hline
 
\tikz \draw (0,0) node[shape aspect=1,diamond,draw,blue] {texte};
&  
\tikz \draw (0,-2) node[shape aspect=2,diamond,draw,blue] {texte};
&
\tikz \draw (0,0) node[shape aspect=3,diamond,draw,blue] {texte};
&
\tikz \draw (0,0) node[shape aspect=4,diamond,draw,blue] {texte};
\\ \hline  
\RDD{shape aspect}=1
&  
\RDD{shape aspect}=2
&
\RDD{shape aspect}=3
&
\RDD{shape aspect}=4
\\ \hline 
\end{tabular} 

\bigskip

\begin{tabular}{|c|} \hline 
\BS{draw} node[\rouge {shape border rotate}=30,shape=dart, draw, \rouge {shape border uses incircle}] \AC{texte};
\\ \hline 
\tikz[] \draw node[shape border rotate=30,shape=dart, draw, shape border uses incircle] {texte};
\\ \hline 
\end{tabular} 

\newpage

\SbSSCT{Dans un n\oe ud en forme de symboles}{Symbol Shapes nodes}

\label{lib-symb}

\maboite{\BS{usetikzlibrary}\AC{shapes.symbols}}

\begin{center}
\RRR{67-4}
\end{center}

\SbSbSSCT{Formes disponibles}{Available shapes}

\label{nd2}

\begin{tabular}{|c|c|c|} \hline  
\tikz  \node[fill=green!20,forbidden sign,draw] {texte};
&  
\tikz  \node[fill=green!20,magnifying glass,draw] {texte};
&  
\tikz  \node[fill=green!20,cloud,draw] {texte};
\\ \hline 
forbidden sign & magnifying glass & cloud
\\ \hline  
\tikz  \node[fill=green!20,starburst,draw] {texte};
&  
\tikz  \node[fill=green!20,signal,draw] {texte};

&  
\tikz  \node[fill=green!20,tape,draw] {texte};
\\ \hline 
starburst & signal & tape
\\ \hline 
\end{tabular} 
\bigskip

\subsubsection{Options}

\begin{tabular}{|c|c|c|} \hline  
\multicolumn{3}{|c|}{  \BS{node}[magnifying glass,\RDD{magnifying glass handle angle}=45,draw,blue]  \AC{texte} ;   }
\\ \hline
\tikz  \node[magnifying glass,magnifying glass handle angle=45,draw,blue] {texte};
&  
\tikz  \node[,magnifying glass,magnifying glass handle aspect=3,draw,blue] {texte};
& 
\tikz  \node[magnifying glass,line width=1ex,draw,blue] {texte};

\\ \hline  
\RDD{magnifying glass handle angle}=45 & \RDD{magnifying glass handle aspect}=3  & line width=1ex  
\\ \hline 
\dft{ : -45} & \dft{ : 1.5}& 
\\ \hline 
\end{tabular} 

\bigskip

\begin{tabular}{|c|c|c|c|} \hline 
\multicolumn{4}{|c|}{  \BS{node} [cloud,\RDD{cloud puffs}=5,draw,blue] \AC{texte};   }\\ 
\hline 
\begin{tikzpicture}
\node[cloud,draw,red,dashed] {texte};
\node[cloud,cloud puffs=5,draw,blue] {texte};
\end{tikzpicture} 
&  
\begin{tikzpicture}
\node[cloud,draw,red,dashed] {texte};
\node[cloud,cloud puff arc=270,draw,blue] {texte};
\end{tikzpicture} 
&  
\begin{tikzpicture}
\node[cloud,draw,red,dashed] {texte};
\node[cloud,cloud ignores aspect=true,draw,blue] {texte};
\end{tikzpicture} 
&
\begin{tikzpicture}
\node[cloud,draw,red,dashed] {texte};
\node[cloud,cloud ignores aspect=false,draw,blue] {texte};
\end{tikzpicture} 
\\ \hline  
\RDD{cloud puffs}=5 & \RDD{cloud puff arc}=270 & \RDD{cloud ignores aspect}=false & \RDD{cloud ignores aspect}=true  \\ 
\hline 
\dft :  10 & \dft :  135 &\multicolumn{2}{|c|}{ \dft :  true } \\ \hline
\end{tabular} 

\bigskip

\begin{tabular}{|c|c|c|c|} \hline 
\multicolumn{4}{|c|}{  \BS{node} [starburst,\RDD{starburst points}=5,draw,blue] \AC{texte};   }\\ 
\hline  
\tikz  \node[starburst,starburst points=5,draw,blue] {texte};
&  
\tikz  \node[starburst,starburst point height=1cm,draw,blue] {texte};
&  
\tikz  \node[starburst,random starburst=50,draw,blue] {texte};
&
\tikz  \node[,starburst,random starburst=0,draw,blue] {texte};
\\ \hline  
\RDD{starburst points}=5 & \RDD{starburst point height}=1cm & \RDD{random starburst}=50 & \RDD{random starburst}=0  \\ 
\hline 
\end{tabular} 

\bigskip


\begin{tabular}{|c|c|c|} \hline 
\multicolumn{3}{|c|}{  \BS{node} [signal,\RDD{signal pointer angle}=45,draw,blue] \AC{texte};   }\\ 
\hline 
\tikz  \node[signal,signal pointer angle=45,draw,blue] {texte};
&
\tikz  \node[signal,signal pointer angle=10,draw,blue] {texte};
&
\tikz  \node[signal,signal pointer angle=300,draw,blue] {texte};
\\ \hline 
\RDD{signal pointer angle}=45
&
signal pointer angle=10
&
signal pointer angle=300
\\ \hline 
\multicolumn{3}{|c|}{  \dft{ : signal pointer angle= 90}  }
\\  \hline 

\end{tabular} 
\bigskip

\begin{tabular}{|c|c|c|c|c|} \hline 
\multicolumn{4}{|c|}{  \BS{node} [signal,\RDD{signal to}=above,draw,blue] \AC{texte};   }
\\ \hline 
\tikz  \node[signal,signal to=above,draw,blue] {texte};
&  
\tikz  \node[signal,signal to=below,draw,blue] {texte};
&
\tikz  \node[signal,signal to=right,draw,blue] {texte};
&
\tikz  \node[signal,signal to=above,draw,blue] {texte};
\\ \hline  
  \RDD{signal to}=above  & \RDD{signal to}=below & \RDD{signal to}=right  & \RDD{signal to}=above \\ 
\hline 
\end{tabular} 
\bigskip

\begin{tabular}{|c|c|c|c|c|} \hline 
\multicolumn{4}{|c|}{ \BS{tikz} [signal to=nowhere] \BS{node} [signal,\RDD{signal from=above}=45,draw,blue] \AC{texte};   }\\ 
\hline 
\tikz [signal to=nowhere] \node[signal,signal from=above,draw,blue] {texte};
&  
\tikz [signal to=nowhere] \node[signal,signal from=below,draw,blue] {texte};
&
\tikz [signal to=nowhere] \node[signal,signal from=right,draw,blue] {texte};
&
\tikz [signal to=nowhere] \node[signal,signal from=above,draw,blue] {texte};
\\ \hline  
  \RDD{signal from}=above  & \RDD{signal from}=below & \RDD{signal from}=right  & \RDD{signal from}=above \\ 
\hline 
\end{tabular} 

\bigskip
\begin{tabular}{|c|c|c|c|} \hline
\multicolumn{2}{|c|}{ \tikz  \node[draw,signal, signal from=east , signal to=west,blue] at (0,0) {texte};}
&
\multicolumn{2}{|c|}{ \tikz  \node[draw,signal,signal from=south, signal to=north,blue] at (0,0) {texte};}
\\ \hline 
\multicolumn{2}{|c|}{ \RDD{signal from}=east , \RDD{signal to}=west}
&
\multicolumn{2}{|c|}{\RDD{signal from}=south, \RDD{signal to}=north}

\\ \hline 
\end{tabular}
\bigskip

\begin{tabular}{|c | c | c | c |} \hline
\multicolumn{3}{|c|}{ \BS{tikz} \BS{node}  [tape, draw,\RDD{tape bend top}=out and in] \AC{texte};   }\\ 
\hline  
\tikz \node [tape, draw,tape bend top=out and in,blue] {texte};
&
\tikz \node [tape, draw, tape bend bottom=out and in,blue] {texte};
&
\tikz \node [tape, draw, tape bend bottom=in and in,blue] {texte};
 \\  \hline
 \RDD{tape bend top}=out and in & \RDD{tape bend bottom}=out and in &  \RDD{tape bend bottom}=in and in 
  \\  \hline
 \tikz \node [tape, draw, tape bend top=none,blue] {texte};
 &
 \tikz \node [tape, draw,tape bend top=out and in,tape bend bottom=out and in,blue] {texte};
 &
  \tikz \node [tape, draw,tape bend top=in and out,tape bend bottom=in and out,blue] {texte};
  \\  \hline
 \RDD{tape bend top}=none & \RDD{tape bend bottom}=out and in 	&  \RDD{tape bend bottom}=in and out  \\
 					& \RDD{tape bend top}=out and in 		& \RDD{tape bend top}=in and out  \\
 					& & (\dft{} ) 
  \\  \hline 
\end{tabular}
\bigskip

\begin{tabular}{|c | c | c | c |} \hline
\BS{tikz} \BS{node} [tape, draw, \RDD{tape bend height}=1cm,blue] \AC{texte}; 
  \\  \hline 
\tikz \node [tape, draw, tape bend height=1cm,blue] {texte};

  \\  \hline 
\dft{ : tape bend height = 5pt}
  \\  \hline 
\end{tabular}

\newpage

\SbSSCT{Dans un n\oe ud en forme de flèche}{Arrow Shapes nodes}

\label{lib-arr}

\maboite{\BS{usetikzlibrary}\AC{shapes.arrows}}

\begin{center}
\RRR{67-5}
\end{center}

\SbSbSSCT{Formes disponibles}{Available shapes}
\label{nd3}

\begin{tabular}{|c|c|c|} \hline  
\tikz \node[fill=green!20,single arrow,draw] {texte};
&  
\tikz  \node[fill=green!20,double arrow,draw] {texte};
&  
\tikz  \node[fill=green!20,arrow box,draw] {texte};
\\ \hline 
single arrow & double arrow & arrow box \\ 
\hline 
\end{tabular} 

\subsubsection{Options}

\begin{tabular}{|c|c|c|c|c|} \hline  
 \multicolumn{5}{|c|}{  \BS{node}[single arrow,draw,\RDD{single arrow tip angle}=45] \AC{texte};   }\\ 
  \multicolumn{5}{|c|}{  \BS{node}[single arrow,draw,\RDD{single arrow head extend}=.75cm] \AC{texte};   }\\
 \hline
\begin{tikzpicture}
 \node[single arrow,draw,red,dashed,text=black] {texte};
 \node[single arrow,draw,single arrow tip angle=45,blue] {texte};
\end{tikzpicture}
&
\begin{tikzpicture}
 \node[single arrow,draw,red,dashed,text=black] {texte};
\node[single arrow,draw,single arrow tip angle=120,blue] {texte};
\end{tikzpicture}
&
\begin{tikzpicture}
 \node[single arrow,draw,red,dashed,text=black] {texte};
 \node[single arrow,draw,single arrow head extend=.75cm,blue] {texte};
\end{tikzpicture}
&
\begin{tikzpicture}
 \node[single arrow,draw,red,dashed,text=black] {texte};
 \node[single arrow,draw,single arrow head extend=0cm,blue] {texte};
 \end{tikzpicture}
 &
 \begin{tikzpicture}
  \node[single arrow,draw,red,dashed,text=black] {texte};
  \node[single arrow,draw,single arrow head extend=-1mm,blue] {texte};
 \end{tikzpicture}

\\ \hline
angle=45 & angle=120 & extend=.75cm] & extend=0cm & extend=-1mm
\\ \hline 
\multicolumn{2}{|c|}{  \dft : single arrow tip angle= 90   }
&
\multicolumn{3}{|c|}{  \dft : single arrow head extend=0.5cm   }
\\ \hline 
\end{tabular} 
\bigskip


\begin{tabular}{|c|c|c|c|} \hline
 \multicolumn{4}{|c|}{  \BS{node}[minimum size=2cm,single arrow,draw,\RDD{single arrow head indent}=1cm,blue] \AC{texte};   }\\ 
 \hline   
\begin{tikzpicture}
 \node[minimum size=2cm,single arrow,draw,red,dashed,text=black] {texte};
\node[minimum size=2cm,single arrow,draw,single arrow head indent=1cm,blue] {texte};
\end{tikzpicture}
&
\begin{tikzpicture}
 \node[minimum size=2cm,single arrow,draw,red,dashed,text=black] {texte};
  \node[minimum size=2cm,single arrow,draw,single arrow head indent=10pt,blue] {texte};
  \end{tikzpicture}
&
\begin{tikzpicture}
 \node[minimum size=2cm,single arrow,draw,red,dashed,text=black] {texte};
  \node[minimum size=2cm,single arrow,draw,single arrow head indent=1ex,blue] {texte};
  \end{tikzpicture}
  &
  \begin{tikzpicture}
   \node[minimum size=2cm,single arrow,draw,red,dashed,text=black] {texte};
    \node[minimum size=2cm,single arrow,draw,single arrow head indent=-1ex,blue] {texte};
    \end{tikzpicture}
\\ \hline
indent=1cm & indent=10pt & indent=1ex & indent=-1ex
\\ \hline 
\end{tabular}
\bigskip

 



\begin{tabular}{|c|c|c|c|c|} \hline
 \multicolumn{5}{|c|}{  \BS{node}[minimum size=2cm,double arrow,draw,\RDD{double arrow tip angle}=45] \AC{texte};   }\\ 
  \multicolumn{5}{|c|}{  \BS{node}[minimum size=2cm,double arrow,draw,\RDD{double arrow head extend}=1ex] \AC{texte};   }\\
   \multicolumn{5}{|c|}{  \BS{node}[minimum size=2cm,double arrow,draw,\RDD{double arrow head indent}=1ex] \AC{texte};   }\\ 
 \hline  
\begin{tikzpicture}
\node[minimum size=2cm,double arrow,draw,red,dashed,text=black] {texte};
\node[minimum size=2cm,double arrow,draw,double arrow tip angle=45,blue] {texte};
\end{tikzpicture}
&
\begin{tikzpicture}
\node[minimum size=2cm,double arrow,draw,red,dashed,text=black] {texte};
\node[minimum size=2cm,double arrow,draw,double arrow tip angle=120,blue] {texte};
\end{tikzpicture}
&
\begin{tikzpicture}
 \node[minimum size=2cm,double arrow,draw,red,dashed,text=black] {texte};
 \node[minimum size=2cm,double arrow,draw,double arrow head extend=1ex,blue] {texte};
   \end{tikzpicture}
&
\begin{tikzpicture}
 \node[minimum size=2cm,double arrow,draw,red,dashed,text=black] {texte};
  \node[minimum size=2cm,double arrow,draw,double arrow head extend=0,blue] {texte};
    \end{tikzpicture}
&
\begin{tikzpicture}
 \node[minimum size=2cm,double arrow,draw,red,dashed,text=black] {texte};
  \node[,minimum size=2cm,double arrow,draw,double arrow head indent=1ex,blue] {texte};
    \end{tikzpicture}
\\ \hline 
angle=45 & angle=120 & extend=1ex & extend=0 & indent=1ex
\\ \hline
\end{tabular}

\bigskip

\begin{tabular}{|c|c|c|c|c|} \hline
\multicolumn{4}{|c|}{ \BS{node} [arrow box, draw, \RDD{arrow box arrows}=\AC{north:.25cm}] \AC{texte}; }\\ 
\hline 
\begin{tikzpicture}
\node[arrow box, draw,red,text=white,dashed] {texte};
\node[arrow box, draw, arrow box arrows={north:.25cm},blue] {texte};
\end{tikzpicture}
& 
\begin{tikzpicture}
\node[arrow box, draw,red,text=white,dashed] {texte};
\node[arrow box, draw, arrow box arrows={west:.25cm},blue] {texte};
\end{tikzpicture}
 &
 \begin{tikzpicture}
 \node[arrow box, draw,red,text=white,dashed] {texte};
 \node[arrow box, draw, arrow box arrows={south:.25cm},blue] {texte};
 \end{tikzpicture}
&
 \begin{tikzpicture}
 \node[arrow box, draw,red,text=white,dashed] {texte};
 \node[arrow box, draw, arrow box arrows={east:.25cm},blue] {texte};
 \end{tikzpicture}   
 \\ \hline
\AC{north:.25cm} & \AC{west:.25cm} & \AC{south:.25cm}& \AC{east:.25cm} 
\\ \hline
\multicolumn{4}{|c|}{  \dft{} : 0.5 cm}
 \\ \hline 
 \end{tabular}
 
 
 \bigskip
 
 \begin{tabular}{|c|c|} \hline
 \multicolumn{2}{|c|}{ \BS{node} [arrow box, draw, \RDD{arrow box tip angle}=45] \AC{texte}; }\\ 
 \hline 
  \begin{tikzpicture}
  \node[arrow box, draw,red,text=white,dashed] {texte};
  \node[arrow box, draw, arrow box tip angle=45,blue] {texte};
  \end{tikzpicture} 
  &
    \begin{tikzpicture}
   \node[arrow box, draw,red,text=white,dashed] {texte};
   \node[arrow box, draw, arrow box head extend=.25cm,blue] {texte};
   \end{tikzpicture}
\\ \hline  
\RDD{arrow box tip angle}=45 & \RDD{arrow box head extend}=.25cm
\\ \hline 
\dft : 90  & \dft : 0.125cm 
\\ \hline 
   \begin{tikzpicture}
   \node[arrow box, draw,red,text=white,dashed] {texte};
   \node[arrow box, draw, arrow box head indent=.25cm,blue] {texte};
   \end{tikzpicture} 
 &
    \begin{tikzpicture}
    \node[arrow box, draw,red,text=white,dashed] {texte};
    \node[arrow box, draw,arrow box shaft width=.25cm,blue] {texte};
    \end{tikzpicture} 
 \\ \hline 
\RDD{arrow box head indent}=.25cm  &  \RDD{arrow box shaft width}=.25cm
 \\ \hline  
 \dft{ : 0cm } &  \dft{ : 0.125cm }
 \\ \hline  
 \end{tabular}

\newpage

\SbSSCT{Dans un n\oe ud en forme de bulle}{Callout Shapes nodes}
\label{lib-call}

 \maboite{\BS{usetikzlibrary}\AC{shapes.callouts}}
 
\begin{center}
\RRR{67-7}
\end{center}

\SbSbSSCT{Formes disponibles}{Available shapes}

\begin{tabular}{|c|c|c|} \hline 
\tikz  \node[fill=green!20,ellipse callout,draw] {texte};
 &  
 \tikz  \node[fill=green!20,rectangle callout,draw] {texte};
  &  
  \tikz  \node[fill=green!20,cloud callout,draw] {texte};
 \\ \hline
 ellipse callout  &  rectangle callout  & cloud callout \\ 
\hline 
\end{tabular} 

\subsubsection{Options}


\begin{tabular}{|c | c | c | c |} \hline
\multicolumn{4}{|c|}{  \BS{node} [rectangle callout,draw,\RDD{callout absolute pointer}={(0,1)}] at (2,1) \AC{texte};   }\\ 
\hline 
\begin{tikzpicture} 
\draw [help lines] grid(3,3);
\node [rectangle callout,draw,blue, callout relative pointer={(0,1)}] at (2,1) {texte};
\end{tikzpicture}
&
\begin{tikzpicture} 
\draw [help lines] grid(3,3);
\node [ellipse callout,draw, callout relative pointer={(0,1)},blue] at (2,1) {texte};
\end{tikzpicture}
&
\begin{tikzpicture} 
\draw [help lines] grid(3,3);
\node [rectangle callout,draw,blue,callout absolute pointer={(0,1)}] at (2,1) {texte};
\end{tikzpicture}
&
\begin{tikzpicture} 
\draw [help lines] grid(3,3);
\node [ellipse callout,draw, callout absolute pointer={(0,1)},blue] at (2,1) {texte};
\end{tikzpicture}
 \\  \hline
\multicolumn{2}{|c|}{ \RDD{callout relative pointer}=\AC{(0,1)} } & 
\multicolumn{2}{|c|}{  \RDD{callout absolute pointer}=\AC{(0,1)} }
 \\  \hline 
 \begin{tikzpicture} 
 \draw [help lines] grid(3,3);
 \node [rectangle callout,draw, callout relative pointer={(0,1)},callout pointer shorten=.5cm,blue] at (2,1) {texte};
 \end{tikzpicture}
 &
  \begin{tikzpicture} 
  \draw [help lines] grid(3,3);
  \node [ellipse callout,draw, callout relative pointer={(0,1)},callout pointer shorten=.5cm,blue] at (2,1) {texte};
  \end{tikzpicture}
  &
 \begin{tikzpicture} 
 \draw [help lines] grid(3,3);
 \node [rectangle callout,draw, callout absolute pointer={(0,1)},callout pointer shorten=.5cm,blue] at (2,1) {texte};
 \end{tikzpicture}
  &
  \begin{tikzpicture} 
  \draw [help lines] grid(3,3);
  \node [ellipse callout,draw, callout absolute pointer={(0,1)},callout pointer shorten=.5cm,blue] at (2,1) {texte};
  \end{tikzpicture}
  \\  \hline
\multicolumn{4}{|c|}{ \RDD{callout pointer shorten}=.5cm} 
  \\  \hline 
\end{tabular}


\bigskip

\begin{tabular}{|c | c | c | c |} \hline
\multicolumn{3}{|c|}{  \BS{node} [ellipse callout,draw,\RDD{callout pointer arc}=1] at (0,1.5) \AC{texte};   }\\ 
\hline
\begin{tikzpicture}
\node[ellipse callout,draw, callout pointer arc=1,blue] at (0,1.5) {texte};
\end{tikzpicture}
&
\begin{tikzpicture}
\node[ellipse callout,draw, callout pointer arc=30,blue] at (0,1.5) {texte};
\end{tikzpicture}
 &
\begin{tikzpicture}
\node[ellipse callout,draw, callout pointer arc=90,blue] at (0,1.5) {texte};
\end{tikzpicture}
  \\  \hline 
   callout pointer arc=1 & callout pointer arc=30 & callout pointer arc=90
  \\  \hline  
  \multicolumn{3}{|c|}{  \dft{ : callout pointer arc=15}}
 \\  \hline  
 \end{tabular}

\bigskip

\begin{tabular}{|c | c | c | c |} \hline
\multicolumn{3}{|c|}{  \BS{node}[draw,cloud callout, aspect=2.5] \AC{texte};   }\\ 
\hline 
 \begin{tikzpicture}
  \node[draw,cloud callout, dashed,red,text=black] {texte};
 \node[draw,cloud callout, cloud puffs=5,blue] {texte};
 \end{tikzpicture}
&
 \begin{tikzpicture}
 \node[draw,cloud callout, dashed,red,text=black] {texte};
 \node[draw,cloud callout, aspect=2.5,blue] {texte};
 \end{tikzpicture}
&
  \begin{tikzpicture}
  \node[draw,cloud callout, dashed,red,text=black] {texte};
  \node[draw,cloud callout,cloud puff arc=120,blue] {texte};
  \end{tikzpicture}
   \\  \hline 
cloud puffs=5 & aspect=2.5 &  cloud puff arc=120
\\  \hline 
 \end{tabular}

\bigskip

\begin{tabular}{|c | c | c | c |c |} \hline
\multicolumn{3}{|c|}{  \BS{node} [draw,cloud callout,\RDD{callout pointer start size}=.1] \AC{texte};   }\\ 
\hline 
  \begin{tikzpicture}
  \node[draw,cloud callout, dashed,red,text=black] {texte};
  \node[draw,cloud callout,callout pointer start size=.1,blue] {texte};
  \end{tikzpicture}
&
  \begin{tikzpicture}
  \node[draw,cloud callout, dashed,red,text=black] {texte};
  \node[draw,cloud callout,callout pointer start size=.8cm,blue] {texte};
  \end{tikzpicture}
&
  \begin{tikzpicture}
  \node[draw,cloud callout, dashed,red,text=black] {texte};
 \node[draw,cloud callout,callout pointer start size=1cm and 0.1cm,blue] {texte};
  \end{tikzpicture}
\\  \hline 
\RDD{callout pointer start size}=.1 &start size=.8cm & start size=20pt and 1pt
\\  \hline 
\multicolumn{3}{|c|}{  \dft{} : callout pointer start size =.2 of callout  }
\\ 
\hline 
  \begin{tikzpicture}
  \node[draw,cloud callout, dashed,red,text=black] {texte};
  \node[draw,cloud callout,callout pointer end size=5,blue] {texte};
  \end{tikzpicture}
&
  \begin{tikzpicture}
  \node[draw,cloud callout, dashed,red,text=black] {texte};
  \node[draw,cloud callout,callout pointer end size=.8cm,blue] {texte};
  \end{tikzpicture}
&
    \begin{tikzpicture}
    \node[draw,cloud callout, dashed,red,text=black] {texte};
    \node[draw,cloud callout,callout pointer segments=3,blue] {texte};
    \end{tikzpicture}
\\  \hline 
\RDD{callout pointer end size}=.5 & \RDD{callout pointer end size}=.8cm & \RDD{callout pointer segments}=3
\\  \hline 
\multicolumn{2}{|c|}{  \dft{} : callout pointer start size = .1 of callout  }
& \dft{} : segments=2
\\  \hline  

 \end{tabular}

\newpage


\SbSSCT{Dans un n\oe ud en diverses formes  diverses}{Miscellaneous Shapes nodes}

\label{lib-misc}


 \maboite{\BS{usetikzlibrary}\AC{shapes.misc}}
 
\begin{center}
\RRR{67-8}
\end{center}

\SbSbSSCT{Formes disponibles}{Available shapes}

\begin{tabular}{|c|c|c|c|} \hline  
\tikz  \node[fill=green!20,cross out,draw] {texte};
&  
\tikz  \node[fill=green!20,strike out,draw] {texte};
&  
\tikz  \node[fill=green!20,rounded rectangle,draw] {texte};
&  
\tikz  \node[fill=green!20,chamfered rectangle,draw] {texte};
\\ \hline  
cross out & strike out & rounded rectangle & chamfered rectangle \\ 
\hline 
\end{tabular} 


\subsubsection{Options}

\paragraph{Options \TFRGB{pour}{for} \og rounded rectangle \fg} :


\begin{tabular}{|c|c|c|c|c|} \hline
\multicolumn{5}{|c|}{  \BS{node} [draw, rounded rectangle,\RDD{rounded rectangle arc length}=270] \AC{texte};   }\\ 

\hline 

\tikz \node[draw, rounded rectangle,rounded rectangle arc length=270,blue] {texte}; 
&
\tikz \node[draw, rounded rectangle,rounded rectangle arc length=180,blue]  {texte}; 
&
\tikz \node[draw, rounded rectangle,rounded rectangle arc length=120,blue] {texte}; 
&
\tikz \node[draw, rounded rectangle,rounded rectangle arc length=90,blue]  {texte}; 
&
\tikz \node[draw, rounded rectangle,rounded rectangle arc length=45,blue] {texte}; 
 \\ \hline 
270 & 180 & 120 & 90& 45 
\\ \hline 


\end{tabular} 

\bigskip


\begin{tabular}{|c|c|c|c|} \hline 
\multicolumn{4}{|c|}{  \BS{node} [draw, rounded rectangle,\RDD{rounded rectangle west arc}=concave] \AC{texte};   }\\ 
\multicolumn{4}{|c|}{  \BS{node} [draw, rounded rectangle,\RDD{rounded rectangle left arc}=concave] \AC{texte};   }\\ 
\hline 
\tikz \node[draw, rounded rectangle,rounded rectangle west arc=concave,blue] {texte}; 
&
\tikz \node[draw, rounded rectangle,rounded rectangle left arc=concave,blue] {texte}; 
&
\tikz \node[draw, rounded rectangle,rounded rectangle west arc=convex,blue] {texte}; 
&
\tikz \node[draw, rounded rectangle,rounded rectangle left arc=none,blue] {texte};
 \\\hline 
concave & convex & none 
 \\\hline 
\end{tabular} 

\bigskip

\begin{tabular}{|c|c|c|c|} \hline 
\multicolumn{3}{|c|}{  \BS{node} [draw, rounded rectangle,\RDD{rounded rectangle east arc}=concave] \AC{texte};   }\\ 
\multicolumn{3}{|c|}{  \BS{node} [draw, rounded rectangle,\RDD{rounded rectangle right arc}=concave] \AC{texte};   }\\ 

\hline 
\tikz \node[draw, rounded rectangle,rounded rectangle east arc=concave,blue] {texte}; 
&
\tikz \node[draw, rounded rectangle,rounded rectangle  east arc=convex,blue] {texte}; 
&
\tikz \node[draw, rounded rectangle,rounded rectangle right arc=none,blue] {texte};
 \\\hline 
concave & convex & none 
 \\\hline 
\end{tabular} 

\paragraph{Options  \TFRGB{pour}{for} \og chamfered rectangle \fg} :


\begin{tabular}{|c|c|c|c|} \hline 
\multicolumn{4}{|c|}{  \BS{node} [draw, chamfered rectangle,\RDD{chamfered rectangle angle}=30] \AC{texte};   }\\ 
\hline 
\tikz \node[draw, chamfered rectangle,chamfered rectangle angle=10,blue] {texte}; 
&
\tikz \node[draw, chamfered rectangle,chamfered rectangle angle=30,blue] {texte}; 
&
\tikz \node[draw,chamfered rectangle,chamfered rectangle angle=60,blue] {texte};
&
\tikz \node[draw,chamfered rectangle,chamfered rectangle angle=80,blue] {texte};
 \\ \hline 
10 & 30 & 60 & 80
\\ \hline 
\multicolumn{4}{|c|}{  \dft :  45 }
  \\\hline  

\end{tabular}

\bigskip

\begin{tabular}{|c|c|c|c|c|} \hline 
\multicolumn{5}{|c|}{  \BS{node} [draw, chamfered rectangle,\RDD{chamfered rectangle xsep}=10pt] \AC{texte};   }\\ 
\hline 
\tikz \node[draw, chamfered rectangle,chamfered rectangle xsep=0pt,blue] {texte}; 
&
\tikz \node[draw, chamfered rectangle,chamfered rectangle xsep=5pt,blue] {texte}; 
&
\tikz \node[draw, chamfered rectangle,chamfered rectangle xsep=10pt,blue] {texte}; 
&
\tikz \node[draw,chamfered rectangle,chamfered rectangle xsep=-10pt,blue] {texte};
&
\tikz \node[draw,chamfered rectangle,chamfered rectangle xsep=2cm,blue] {texte};
 \\\hline 
  xsep=0pt & xsep=5pt & xsep=10pt & xsep=-10pt  & xsep=2cm
  \\\hline  
\multicolumn{5}{|c|}{  \dft :  0.666ex }
  \\\hline   
\end{tabular}

\bigskip

\begin{tabular}{|c|c|c|c|c|} \hline 
\multicolumn{5}{|c|}{  \BS{node} [draw, chamfered rectangle,\RDD{chamfered rectangle ysep}=10pt] \AC{texte};   }\\ 
\hline 
\tikz \node[draw, chamfered rectangle,chamfered rectangle ysep=0pt,blue] {texte}; 
&
\tikz \node[draw, chamfered rectangle,chamfered rectangle ysep=5pt,blue] {texte}; 
&
\tikz \node[draw,chamfered rectangle,chamfered rectangle ysep=10pt,blue] {texte};
&
\tikz \node[draw,chamfered rectangle,chamfered rectangle ysep=-10pt,blue] {texte};
&
\tikz \node[draw,chamfered rectangle,chamfered rectangle ysep=1cm,blue] {texte};
 \\ \hline 
 ysep=0pt & ysep=5pt & ysep=10pt & ysep=-10pt & ysep=1cm
 \\\hline  
\end{tabular}

\bigskip

\begin{tabular}{|c|c|c|c|c|} \hline 
\multicolumn{5}{|c|}{  \BS{node} [draw, chamfered rectangle,\RDD{chamfered rectangle ysep}=10pt] \AC{texte};   }\\ 
\hline 
\tikz \node[draw, chamfered rectangle,chamfered rectangle sep=0pt,blue] {texte}; 
&
\tikz \node[draw, chamfered rectangle,chamfered rectangle sep=5pt,blue] {texte}; 
&
\tikz \node[draw, chamfered rectangle,chamfered rectangle sep=10pt,blue] {texte}; 

&
\tikz \node[draw, chamfered rectangle,chamfered rectangle sep=-10pt,blue] {texte}; 
&
\tikz \node[draw,chamfered rectangle,chamfered rectangle sep=1cm,blue] {texte};
 \\\hline 
 sep=0pt & sep=5pt & sep=10pt& sep=-10pt & sep=1cm
 \\\hline  
\end{tabular}

\bigskip

\begin{tabular}{|c|c|c|c|} \hline 
\multicolumn{3}{|c|}{  \BS{node} [draw, chamfered rectangle,\RDD{chamfered rectangle corners}=north west] \AC{texte};   }\\ 
\hline
\tikz \node[draw, chamfered rectangle,chamfered rectangle corners=north west,blue] {texte}; 
&
\tikz \node[draw, chamfered rectangle,chamfered rectangle corners={north east, south east},blue] {texte}; 
&
\tikz \node[draw,chamfered rectangle,chamfered rectangle corners={north east, south west},blue] {texte};
 \\ \hline 
 north west & \AC{north east, south east}  & \AC{north east, south west}
 \\ \hline 
\end{tabular}

\newpage

\SbSSCT{N\oe uds à plusieurs parties}{Shapes with Multiple Text Parts}

\label{lib-mult}


 \maboite{\BS{usetikzlibrary}\AC{shapes.multipart}}

\begin{center}
\RRR{67-6}
\end{center}



\begin{tabular}{|c|c|c|c|} \hline 
\multicolumn{4}{|c|}{  \BS{node} [\RDD{circle split},draw,fill=green!20]\AC{haut  \BSS{nodepart}\AC{lower} bas };   }\\ 
\hline 
 
\tikz  \node [circle split,draw,blue,fill=green!20] {haut  \nodepart{lower} bas }; 

&  
\tikz  \node [circle solidus,draw,blue,fill=green!20]{haut  \nodepart{lower} bas };
&  
\tikz  \node [ellipse split,draw,blue,fill=green!20]{texte haut  \nodepart{lower} texte bas };
& 
\tikz  \node [rectangle split,draw,blue,fill=green!20]{haut  \nodepart{lower} bas}; 

\\ \hline 
\RDD{circle split} & \RDD{circle solidus} & \RDD{ellipse split} & \RDD{rectangle split} \\ 
\hline 
\end{tabular} 

 \bigskip
 
 \begin{tabular}{|c|c|}  \hline  
 \begin{tikzpicture} [baseline=0pt]
 \node[rectangle split,rectangle split parts=5,draw,blue,fill=green!20] at(0,0)
 {texte 1
 \nodepart{second}
 texte 2
 \nodepart{four}
 texte 3};
 \end{tikzpicture}
&
\parbox[c]{10cm}{
 \BS{node}[rectangle split,\RDD{rectangle split parts}=5,\\
 draw] \\
 \AC{texte 1 \\
 \BSS{nodepart}\AC{second} texte 2 \\
 \BSS{nodepart}\AC{four} texte 3}; \\
 \\
\dft : rectangle split parts=4 }
 \\  \hline 
 \end{tabular} 
 
\bigskip

\begin{tabular}{|c|}\hline  
\BS{node} [rectangle split,rectangle split parts=3,\RDD{rectangle split horizontal},draw,blue] \\
\AC{texte1\BSS{nodepart}\AC{two}texte2\BSS{nodepart}\AC{three}texte3};
\\ \hline  
\tikz \node [rectangle split,rectangle split parts=3, rectangle split horizontal,draw,blue]
{texte 1\nodepart{two}texte 2\nodepart{three}texte 3}; 
\\ \hline 
\end{tabular} 
 
 \bigskip
 
% % % <<<<<<<<<<<<<<<<< A Voir rectangle split allocate boxes= >>>>>>>>>>>>>>>>>>>>>>>>>>>>>>>>

% \begin{tikzpicture} [baseline=0pt]%[every text node part/.style={text centered}]
% \node[rectangle split,draw,rectangle split parts=5,fill=green!20,rectangle split allocate boxes=3] at(0,0)
% {texte 1  \nodepart{second}  texte 2  \nodepart{four}  texte 3};
% \end{tikzpicture}
% 
 
\bigskip
 \begin{tabular}{|c|c|}  \hline  
\begin{tikzpicture}[baseline=0pt]
\node[rectangle split, rectangle split parts=3, draw,blue, text width=2.75cm]
{texte 1
\nodepart{two}
texte 2a \\
texte 2b \\
texte 2c
\nodepart{three}
texte 3a \\
texte 3b};
\end{tikzpicture}
&
\parbox{8cm}{
 \BS{node}[rectangle split,\RDD{rectangle split parts}=5, draw] \\
 \AC{texte 1 \\
 \BSS{nodepart}\AC{second} texte 2a  \BS{}\BS{}texte 2b  \BS{}\BS{}  texte 2c \\
 \BSS{nodepart}\AC{three} texte 3a \BS{}\BS{} texte 3b }; \\
}
 \\  \hline 
 \end{tabular} 
\bigskip


 \begin{tabular}{|c|c|}  \hline  
 \multicolumn{2}{|c|}{  \BS{node}[rectangle split, draw,blue,minimum size = 2cm,\RDD{rectangle split draw splits}= true] } \\
  \multicolumn{2}{|c|}{ 
  \AC{texte 1 \BS{nodepart}\AC{two} texte 2 \BS{nodepart}\AC{three} texte 3 \BS{nodepart}\AC{four} texte 4};   }\\ 
 \hline 
\tikz \node[rectangle split, draw,blue,minimum size = 2cm,rectangle split draw splits= true] {texte 1 \nodepart{two} texte 2 \nodepart{three} texte 3 \nodepart{four} texte 4};
&
\tikz \node[rectangle split, draw,blue,minimum size = 2cm,rectangle split draw splits= false] {texte 1 \nodepart{two} texte 2 \nodepart{three} texte 3 \nodepart{four} texte 4};
 \\ \hline
 \RDD{rectangle split draw splits}= true & \RDD{rectangle split draw splits}= false \\
 \dft &
 \\ \hline 
 \end{tabular}
 
\bigskip

 \begin{tabular}{|c|c|}  \hline  
\multicolumn{2}{|c|}{  
\BS{node} [rectangle split,rectangle split parts=3,draw,\RDD{rectangle split ignore empty parts}=false] }\\
 \multicolumn{2}{|c|}{ \AC{texte 1 \BS{nodepart}\AC{second} \BS{nodepart}\AC{third}texte 3};} 
\\ \hline  
\begin{tikzpicture} 
\node[rectangle split,rectangle split parts=3,draw,blue,rectangle split ignore empty parts=false] {texte 1 \nodepart{second} \nodepart{third}texte 3};
\end{tikzpicture}
&
\begin{tikzpicture}
\node[rectangle split,rectangle split parts=3,draw,blue,rectangle split ignore empty parts] 
{texte 1 \nodepart{second} \nodepart{third}texte 3};
\end{tikzpicture}
 \\  \hline 
\RDD{rectangle split ignore empty parts}=false & \RDD{rectangle split ignore empty parts}=true 
\\ \hline
 \end{tabular}
 
\bigskip

 \begin{tabular}{|c|c|}  \hline  
\multicolumn{2}{|c|}{  
\BS{node} [rectangle split,rectangle split parts=3,draw,\RDD{rectangle split empty part depth}=1cm] }\\
 \multicolumn{2}{|c|}{ \AC{texte 1 \BS{nodepart}\AC{second} \BS{nodepart}\AC{third}texte 3};} 
\\ \hline 
\begin{tikzpicture} 
\node[rectangle split,rectangle split parts=3,draw,blue,rectangle split empty part depth=1cm] {texte 1 \nodepart{second} \nodepart{third}texte 3};
\end{tikzpicture}
&
\begin{tikzpicture} 
\node[rectangle split,rectangle split parts=3,draw,blue,text depth=1cm] {texte 1 \nodepart{second} \nodepart{third}texte 3};
\end{tikzpicture}
\\ \hline 
\RDD{rectangle split empty part depth}=1cm & \RDD{text depth}=1cm
\\ \hline
\dft : 0ex & \dft : 0ex
\\ \hline 
\begin{tikzpicture}
\node[rectangle split,rectangle split parts=3,draw,blue,rectangle split empty part  height=1cm] 
{texte 1 \nodepart{second} \nodepart{third}texte 3};
\end{tikzpicture}
&
\begin{tikzpicture}
\node[rectangle split,rectangle split parts=3,draw,blue,text height=1cm] 
{texte 1 \nodepart{second} \nodepart{third}texte 3};
\end{tikzpicture}
\\  \hline 
\RDD{rectangle split empty part height}=1cm & \RDD{text height}=1cm
\\ \hline
\dft : 1ex & \dft : 1ex
\\ \hline 
 \end{tabular}
 
\bigskip



 \begin{tabular}{|c|c|}  \hline 
 \multicolumn{2}{|c|}{ 
 \BS{node} [rectangle split,rectangle split parts=3,draw,\RDD{rectangle split empty part width}=1cm]   \AC{};  } 
 \\ \hline 
\begin{tikzpicture} 
\node[rectangle split,rectangle split parts=3,draw,blue,rectangle split empty part width=2cm]{};
\end{tikzpicture}

&
\begin{tikzpicture} 
\node[rectangle split,rectangle split parts=3,draw,blue]{}; 
\end{tikzpicture}
\\  \hline 
 \RDD{rectangle split empty part width}=2cm  &  \dft : 1ex
\\ \hline
 \end{tabular} 
 
 \bigskip



% % % % <<<<<<<<<< A voir   /pgf/rectangle split use custom fill= (default true) <<<<<<<<<<<<<<<<<<<<<<<<<<<<
 


 \begin{tabular}{|c|c|}  \hline 
 \tikz[baseline=0pt] \node[rectangle split, draw,blue,minimum size = 2cm,rectangle split part align={center, left,right}] {texte 1 \nodepart{two} texte 2 \nodepart{three} texte 3 \nodepart{four} texte 4};
&
\parbox{8cm}{
\BS{node}[rectangle split, draw,blue,minimum size = 2cm,\\
\RDD{rectangle split part align}=\AC{center, left,right}]\\
 \AC{texte 1 \BS{nodepart}\AC{two} texte 2  \\
 \BS{nodepart}\AC{three} texte 3  \BS{nodepart}\AC{four} texte 4};
}
\\ \hline
 \tikz[baseline=0pt] \node[rectangle split, draw,blue,minimum size = 2cm, rectangle split horizontal,rectangle split part align={center,base, top,bottom}] {texte 1 \nodepart{two} texte 2 \nodepart{three} texte 3 \nodepart{four} texte 4};
 &
 \parbox{8cm}{
 \BS{node}[rectangle split, draw,blue,minimum size = 2cm,\\
 rectangle split horizontal,\\
 \RDD{rectangle split part align}=\AC{center,base, top,bottom}]\\
  \AC{texte 1 \BS{nodepart}\AC{two} texte 2  \\
  \BS{nodepart}\AC{three} texte 3  \BS{nodepart}\AC{four} texte 4};
 }
 \\ \hline
 \end{tabular}
 
\bigskip


 \begin{tabular}{|c|c|}  \hline  
\tikz[baseline=0pt] \node[rectangle split, draw,blue, minimum width=1cm,rectangle split part fill={red, green,cyan}]{};
&
\parbox{12cm}{
\BS{node}[rectangle split, draw,blue, minimum width=1cm,\\
 \RDD{rectangle split part fill}=\AC{red, green,cyan}]\AC{};}
\\ \hline
\end{tabular} 

\newpage

\SbSSCT{Mise en forme du texte}{Text attributes}

\subsubsection{Position}

\begin{center}
\RRR{17-4-3}
\end{center}

\begin{tabular}{|c|c|c|c|} \hline  
\multicolumn{4}{|l|}{ \BS{tikz} \BS{draw} (0,0) node[fill=blue!10,\RDD{text width}=2cm,\RDD{text justified}]   }\\ 

\multicolumn{4}{|l|}{ \AC{Ceci est une démonstration d'un texte  sur une largeur de 2cm};  }\\ 
\hline 
\tikz \draw (0,0) node[fill=blue!10,text width=2cm]
{Ceci est une démonstration d'un texte  sur une largeur de 2cm.};
&  
\tikz \draw (0,0) node[fill=blue!10,text width=2cm,text justified]
{Ceci est une démonstration d'un texte  sur une largeur de 2cm};
&  
\tikz \draw (0,0) node[fill=blue!10,text width=2cm,text centered]
{Ceci est une démonstration d'un texte  sur une largeur de 2cm .};
&  
\tikz \draw (0,0) node[fill=blue!10,text width=2cm,text ragged]
{Ceci est une démonstration d'un texte  sur une largeur de 2cm .};
\\  \hline  
\TFRGB{sans}{without} option & \RDD{text justified} & \RDD{text centered }& \RDD{text ragged}   
\\ \hline  
\tikz \draw (0,0) node[fill=blue!10,text width=2cm,text badly ragged]
{Ceci est une démonstration d'un texte  sur une largeur de 2cm.};
&  
\tikz \draw (0,0) node[fill=blue!10,text width=2cm,text badly centered]
{Ceci est une démonstration d'un texte  sur une largeur de 2cm .};
&
\tikz \draw (0,0) node[fill=blue!10,text width=2cm,align=center]
{Ceci est une démonstration d'un texte  sur une largeur de 2cm .};
&
\tikz \draw (0,0) node[fill=blue!10,text width=2cm,align=flush center]
{Ceci est une démonstration d'un texte  sur une largeur de 2cm .};
\\  \hline 
\RDD{text badly ragged} &  \RDD{text badly centered} &  \RDD{align}=center & \RDD{align}=flush center 
\\  \hline 
\tikz \draw (0,0) node[fill=blue!10,text width=2cm,align=justify]
{Ceci est une démonstration d'un texte  sur une largeur de 2cm .};
&
\tikz \draw (0,0) node[fill=blue!10,text width=2cm,align=flush right]
{Ceci est une démonstration d'un texte  sur une largeur de 2cm .};
&
\tikz \draw (0,0) node[fill=blue!10,text width=2cm,align=right]
{Ceci est une démonstration d'un texte  sur une largeur de 2cm .};
&
\tikz \draw (0,0) node[fill=blue!10,text width=2cm,align=flush left]
{Ceci est une démonstration d'un texte  sur une largeur de 2cm .};
\\ \hline 
\RDD{align}=justify & \RDD{align}=flush right &  \RDD{align}=right & \RDD{align}=flush left
\\ \hline 

\end{tabular} 
\bigskip

\begin{tabular}{|c|c|} \hline 
\tikz[baseline=0cm] \node [draw] {
\begin{tabular}{|c|c|} \hline
AAA & BBB \\ \hline
CCC & DDD \\ \hline
\end{tabular}
};
& 
\parbox{8cm}{
\BS{tikz} \BS{node} [draw] \AC{
\BS{begin}\AC{tabular}\AC{|c|c|} \BS{hline} \\
AAA \& BBB \BS{}\BS{} \BS{hline} \\
CCC \& DDD \BS{}\BS{} \BS{hline} \\
\BS{end}\AC{tabular}
};}
\\ \hline 
\end{tabular} 

\bigskip


\begin{tabular}{|c|c|c|}  \hline 
\multicolumn{3}{|c|}{\BS{tikz}[align=left] \BS{node}[draw] \AC{AAA \rouge{ \BS{}\BS{} } BBBBBBBB \rouge{ \BS{}\BS{} } CC};} \\ \hline
\tikz[align=left] \node[draw] {AAA\\BBBBBBBB\\CC};
&  
\tikz[align=center] \node[draw] {AAA\\BBBBBBBB\\CC};
&
\tikz[align=right] \node[draw] {AAA\\BBBBBBBB\\CC};
\\ \hline
[align=left]  & [align=center] &[align=right] 
\\ \hline
\end{tabular} 


\bigskip

\begin{tabular}{|c|c|} \hline 
\multicolumn{2}{|c|}{\BS{tikz}[align=left] \BS{node}[draw] \AC{AAA  \BS{}\BS{} \rouge{[1cm] } BBBBBBBB };} 
\\ \hline 
\rule[-1cm]{0pt}{1,5cm} \tikz[align=left] \node[draw] {AAA\\[1cm]BBBBBBBB\\}; 
& 
\tikz[align=left] \node[draw] {AAA\\[-1cm]BBBBBBBB\\}; 
\\ \hline 
\rouge{ [1cm] } & \rouge{[ -1cm] }
\\ \hline 
\end{tabular} 

\SbSbSSCT{Couleur et fontes }{Colors and Fonts}

\begin{tabular}{|c|c|c|c|c|c|} \hline  
\tikz \draw (0,0) node[text= red]{Texte.};
&
\tikz \draw (0,0) node[font=\itshape]{Texte.};
&
\tikz \draw (0,0) node[font=\slshape]{Texte.};
&
\tikz \draw (0,0) node[font=\scshape]{Texte.};
&
\tikz \draw (0,0) node[font=\upshape]{Texte.};
&
\tikz \draw (0,0) node[font=\bfseries]{Texte.};
\\ \hline 



[text= red] & [font=\BS{itshape}]  & [font=\BS{slshape}] & [font=\BS{scshape}] & [font=\BS{upshape}] & [font=\BS{bfseries}]
\\ \hline 
\end{tabular} 



\bigskip
 
\SbSbSSCT{Taille des fontes}{Font Sizes}

\begin{tabular}{|c|c|c|c|c|c|c|}\hline
\multicolumn{7}{|c|}{ \BS{tikz} \BS{draw} (0,0) node[\RDD{font}=\BS{tiny}]\AC{Texte.}   }
\\  \hline
\tikz \draw (0,0) node[font=\tiny]{Texte.};
&
\tikz \draw (0,0) node[font=\footnotesize]{Texte.};
&
\tikz \draw (0,0) node[font=\small]{Texte.};
&
\tikz \draw (0,0) node[font=\large]{Texte.};
&
\tikz \draw (0,0) node[font=\Large]{Texte.};
&
\tikz \draw (0,0) node[font=\huge]{Texte.};
&
\tikz \draw (0,0) node[font=\Huge]{Texte.};
\\ \hline \BS{tiny} & \BS{footnotesize}  & \BS{small} & \BS{large} & \BS{Large} & \BS{huge} & \BS{Huge} \\ 
\hline 
\end{tabular} 

\bigskip
\begin{center}
\RRR{17-4-4}
\end{center}

\begin{tabular}{|c|c|c|} \hline  
\tikz \draw (0,0) node[fill=blue!10,text height=1cm,draw]{Texte.};
&  
\tikz \draw (0,0) node[fill=blue!10,text depth=1cm,draw]{Texte.};
&  
\tikz \draw (0,0) node[fill=blue!10,text depth=0.5cm,,text height=.5cm,draw]{Texte.};
\\ \hline  
\RDD{text height}=1cm
&  
\RDD{text depth}=1cm
&
\RDD{text height}=0.5cm, \RDD{text depth}=0.5cm
\\ \hline 
\end{tabular} 

\newpage

\SbSSCT{Positions prédéfinies  sur un n\oe ud}{Positions on a node}
\label{nomnoeud}

\SbSbSSCT{pour l'ensemble des n\oe uds}{For all types of node}
\begin{center}
\RRR{17-5-1}
\end{center}

\begin{tabular}{|c|c|c|c|} \hline  
\begin{tikzpicture}
\node[rectangle,draw,minimum size=3cm] (A) at (1,1) {\Huge texte};
\fill[red] (node cs:name=A,anchor=north west) circle (3pt);
\end{tikzpicture}
&
\begin{tikzpicture}
\node[rectangle,draw,minimum size=3cm] (A) at (1,1) {\Huge texte};
\fill[red] (node cs:name=A,anchor=north) circle (3pt);
\end{tikzpicture}
&
\begin{tikzpicture}
\node[rectangle,draw,minimum size=3cm] (A) at (1,1) {\Huge texte};
\fill[red] (node cs:name=A,anchor=north east) circle (3pt);
\end{tikzpicture}
&
\begin{tikzpicture}
\node[rectangle,draw,minimum size=3cm] (A) at (1,1) {\Huge texte};
\fill[red] (node cs:name=A,anchor=text) circle (3pt);
\end{tikzpicture}
\\ \hline 
north west & north & north east & text
\\ \hline 

\begin{tikzpicture}
\node[rectangle,draw,minimum size=3cm] (A) at (1,1) {\Huge texte};
\fill[red] (node cs:name=A,anchor= west) circle (3pt);
\end{tikzpicture}
&
\begin{tikzpicture}
\node[rectangle,draw,minimum size=3cm] (A) at (1,1) {\Huge texte};
\fill[red] (node cs:name=A,anchor=mid  west) circle (3pt);
\end{tikzpicture}
&
\begin{tikzpicture}
\node[rectangle,draw,minimum size=3cm] (A) at (1,1) {\Huge texte};
\fill[red] (node cs:name=A,anchor= base west) circle (3pt);
\end{tikzpicture}
&
\begin{tikzpicture}
\node[rectangle,draw,minimum size=3cm] (A) at (1,1) {\Huge texte};
\fill[red] (node cs:name=A,anchor= base) circle (3pt);
\end{tikzpicture}
\\ \hline 
west & mid west & base west &  base
\\ \hline
 
\begin{tikzpicture}
\node[rectangle,draw,minimum size=3cm] (A) at (1,1) {\Huge texte};
\fill[red] (node cs:name=A,anchor=east) circle (3pt);
\end{tikzpicture}
&
\begin{tikzpicture}
\node[rectangle,draw,minimum size=3cm] (A) at (1,1) {\Huge texte};
\fill[red] (node cs:name=A,anchor=mid east) circle (3pt);
\end{tikzpicture}
&
\begin{tikzpicture}
\node[rectangle,draw,minimum size=3cm] (A) at (1,1) {\Huge texte};
\fill[red] (node cs:name=A,anchor=base east) circle (3pt);
\end{tikzpicture}
&
\begin{tikzpicture}
\node[rectangle,draw,minimum size=3cm] (A) at (1,1) {\Huge texte};
\fill[red] (node cs:name=A,anchor= mid) circle (3pt);
\end{tikzpicture}
\\ \hline 
east & mid esat & base east & mid
\\ \hline 

\begin{tikzpicture}
\node[rectangle,draw,minimum size=3cm] (A) at (1,1) {\Huge texte};
\fill[red] (node cs:name=A,anchor= south east) circle (3pt);
\end{tikzpicture}
&
\begin{tikzpicture}
\node[rectangle,draw,minimum size=3cm] (A) at (1,1) {\Huge texte};
\fill[red] (node cs:name=A,anchor= south) circle (3pt);
\end{tikzpicture}
&
\begin{tikzpicture}                                       
\node[rectangle,draw,minimum size=3cm] (A) at (1,1) {\Huge texte};
\fill[red] (node cs:name=A,anchor= south west) circle (3pt);
\end{tikzpicture}
&
\begin{tikzpicture}
\node[rectangle,draw,minimum size=3cm] (A) at (1,1) {\Huge texte};
\fill[red] (node cs:name=A,anchor=center ) circle (3pt);
\end{tikzpicture}
\\ \hline 
south east & south & south west & center
\\ \hline
 
\begin{tikzpicture}
\node[rectangle,draw,minimum size=3cm] (A) at (1,1) {\Huge texte};
\fill[red] (node cs:name=A,anchor=0) circle (3pt);
\end{tikzpicture}
&
\begin{tikzpicture}
\node[rectangle,draw,minimum size=3cm] (A) at (1,1) {\Huge texte};
\fill[red] (node cs:name=A,anchor=120) circle (3pt);
\end{tikzpicture}
&
\begin{tikzpicture}
\node[rectangle,draw,minimum size=3cm] (A) at (1,1) {\Huge texte};
\fill[red] (node cs:name=A,anchor=-60) circle (3pt);
\end{tikzpicture}
&


\\ \hline 
0 & 120 & -60 &  
\\ \hline 
\end{tabular}
 
\newpage 

\SbSbSSCT{spécifique à un n\oe ud}{Specific to a node}

\TFRGB{Consultez }{see} \RRR{67 }


\begin{tabular}{|c|c|} \hline 
shape=circle & shape=diamond
\\  \hline 
\begin{tikzpicture}[]
\node[circle,draw,minimum size=3.5cm] (A) at (1,1) {\Huge XXX};
\foreach \anchor/\placement in
{north west/above left, north/above, north east/above right,
west/above left, center/above, east/right,
mid west/left, mid/below right, mid east/right,
base west/below left, base/below, base east/below right,
south west/below left, south/below, south east/below right,
text/below, 20/right, 120/above}
\fill[blue,pin position=\placement] (node cs:name=A,anchor= \anchor) circle (2pt) node[blue,pin=\scriptsize{ \anchor} ] {} ;
\end{tikzpicture}
&
\begin{tikzpicture}[]
\node[diamond,draw,minimum size=3.5cm] (A) at (1,1) {\Huge XXX};
\foreach \anchor/\placement in
{north west/above left, north/above, north east/above right,
west/left, center/above, east/right,
mid/10,
base/below,
south west/below left, south/below, south east/below right,
text/left, 10/right, 120/above}
\fill[blue,pin position=\placement] (node cs:name=A,anchor= \anchor) circle (2pt) node[blue,pin=\scriptsize{ \anchor} ] {} ;
\end{tikzpicture}
\\ \hline 
\end{tabular} 

\bigskip

\begin{tabular}{|c|} \hline 
shape=ellipse
\\  \hline 
\begin{tikzpicture}[]
\node[ellipse,draw,minimum size=3.5cm] (A) at (1,1) {\Huge XXXXXXX};
\foreach \anchor/\placement in
{north west/above left, north/above, north east/above right, west/left, center/above, east/right,
mid west/left, mid/-75, mid east/right,
base west/200, base/-105, base east/-20,
south west/below left, south/below, south east/below right,
text/-75, 10/right, 130/above}
\fill[blue,pin position=\placement] (node cs:name=A,anchor= \anchor) circle (2pt) node[blue,pin=\scriptsize{ \anchor} ] {} ;
\end{tikzpicture}
\\ \hline 
\end{tabular}

\bigskip

\begin{tabular}{|c|} \hline 
shape=trapezium
\\  \hline 
\begin{tikzpicture}[]
\node[ trapezium,draw,minimum size=3cm] (A) at (1,1) {\Huge XXX};
\foreach \anchor/\placement in
{center/120, text/below, mid/-45, base/below, mid west/left, base west/-175, mid east/right, base east/-25,
west/175, east/above, north/-75, south/-60,
north west/above, north east/above,
south west/-150, south east/-30, 150/above}
\fill[blue,pin position=\placement] (node cs:name=A,anchor= \anchor) circle (2pt) node[blue,pin=\scriptsize{ \anchor} ] {} ;

\foreach \anchor/\placement in
{bottom left corner/below, top right corner/right,
top left corner/left, bottom right corner/below,
bottom side/-120, left side/left, right side/right, top side/above}
\fill[red,pin position=\placement] (node cs:name=A,anchor= \anchor) circle (2pt) node[blue,pin=\scriptsize{ \anchor} ] {} ;
\end{tikzpicture}
\\ \hline 
\end{tabular}

\bigskip

\begin{tabular}{|c|} \hline 
shape=semicircle,shape border rotate=0
\\  \hline 
\begin{tikzpicture}[]
\node[ semicircle,shape border rotate=0,draw,minimum size=3cm] (A) at (1,1) {\Huge XXX};
\foreach \anchor/\placement in
{center/above, base/-160, mid/-40, text/left, base west/-120, base east/-60, mid west/left, mid east/right, north/below, south/-75, east/60, west/120, north west/above left, north east/above right, south west/-140, south east/-60, 30/right}
\fill[blue,pin position=\placement] (node cs:name=A,anchor= \anchor) circle (2pt) node[blue,pin=\scriptsize{ \anchor} ] {} ;
\foreach \anchor/\placement in
{apex/above, arc start/-60, arc end/-120, chord center/-100}
\fill[red,pin position=\placement] (node cs:name=A,anchor= \anchor) circle (2pt) node[blue,pin=\scriptsize{ \anchor} ] {} ;
\end{tikzpicture}
\\ \hline 
\end{tabular}

\bigskip

\begin{tabular}{|c|} \hline 
shape=regular polygon
\\  \hline 
\begin{tikzpicture}[]
\node[ regular polygon,draw,minimum size=3cm] (A) at (1,1) {\Huge XXX};
\foreach \anchor/\placement in
{ center/97, text/97 , mid/-30, base/below, 75/above,
west/left, east/right, north/-87, south/-60,
north east/right, south east/right, north west/left, south west/left}
\fill[blue,pin position=\placement] (node cs:name=A,anchor= \anchor) circle (2pt) node[blue,pin=\scriptsize{ \anchor} ] {} ;

\foreach \anchor/\placement in
{corner 1/above, corner 2/left, corner 3/left, corner 4/right, corner 5/right,
side 1/above, side 2/left, side 3/-120, side 4/right, side 5/above}
\fill[red,pin position=\placement] (node cs:name=A,anchor= \anchor) circle (2pt) node[blue,pin=\scriptsize{ \anchor} ] {} ;
\end{tikzpicture}
\\ \hline 
\end{tabular}


\bigskip

\begin{tabular}{|c|} \hline 
shape=star
\\  \hline 
\begin{tikzpicture}[]
\node[  shape=star, star points=5, star point ratio=1.65,draw,minimum size=3cm] (A) at (1,1) {\Huge XXX};
\foreach \anchor/\placement in
{center/above, 
text/below, 
mid/-30, 
base/-80, 
75/above,
west/left, 
east/right, 
north/below, 
south/94,
north east/right, 
south east/right, 
north west/left, 
south west/left}
\fill[blue,pin position=\placement] (node cs:name=A,anchor= \anchor) circle (2pt) node[blue,pin=\scriptsize{ \anchor} ] {} ;

\foreach \anchor/\placement in
{inner point 1/above left, 
inner point 2/left, 
inner point 3/below, 
inner point 4/right,
inner point 5/above right, 
outer point 1/above, 
outer point 2/left, 
outer point 3/left,
outer point 4/right, 
outer point 5/right}
\fill[red,pin position=\placement] (node cs:name=A,anchor= \anchor) circle (2pt) node[blue,pin=\scriptsize{ \anchor} ] {} ;
\end{tikzpicture}
\\ \hline 
\end{tabular}



\bigskip

\begin{tabular}{|c|c|} \hline 
shape= isosceles triangle & shape= kite
\\  \hline 
\begin{tikzpicture}[]
\node[ shape=isosceles triangle,draw,minimum size=3cm] (A) at (1,1) {\Huge XXX};
\foreach \anchor/\placement in
{center/above,text/above,150/left,mid/-5, mid west/left, mid east/right,base/-120, base west/-150, 
base east/below right ,west/left, east/right,  north/above, north west/left, north east/above right,
south /-120 , south east/below right}
\fill[blue,pin position=\placement] (node cs:name=A,anchor= \anchor) circle (2pt) node[blue,pin=\scriptsize{ \anchor} ] {} ;
\foreach \anchor/\placement in
{apex/above, left corner/left, right corner/left,left side/above, right side/below, lower side/160}
\fill[red,pin position=\placement] (node cs:name=A,anchor= \anchor) circle (2pt) node[blue,pin=\scriptsize{ \anchor} ] {} ;
\end{tikzpicture}
&
\begin{tikzpicture}[]
\node[ shape=kite,draw,minimum size=3cm] (A) at (1,1) {\Huge XXX};
\foreach \anchor/\placement in
{center/above, text/85, mid/-85, base/-95,mid west/left, base west/-160, 
mid east/right, base east/-20,west/left, east/right, north/80, south/below left,north west/above left, north east/above right,south west/left, south east/right, 
110/above left}
\fill[blue,pin position=\placement] (node cs:name=A,anchor= \anchor) circle (2pt) node[blue,pin=\scriptsize{ \anchor} ] {} ;
\foreach \anchor/\placement in
{upper vertex/110, 
left vertex/left, 
lower vertex/below right,
right vertex/right, 
upper left side/left, 
upper right side/right,
lower left side/left, 
lower right side/below right}
\fill[red,pin position=\placement] (node cs:name=A,anchor= \anchor) circle (2pt) node[blue,pin=\scriptsize{ \anchor} ] {} ;
\end{tikzpicture}
\\ \hline 
\end{tabular}


\bigskip

\begin{tabular}{|c|c|} \hline 
shape= dart & shape= circular sector
\\  \hline 
\begin{tikzpicture}[]
\node[shape=dart, shape border rotate=90,,draw,minimum size=3cm] (A) at (1,1) {\Huge XXX};
\foreach \anchor/\placement in
{west/left  , east/above right , north/below,south/left,
north west/left, north east/right, south west/below, south east/below,110/above left}
\fill[blue,pin position=\placement] (node cs:name=A,anchor= \anchor) circle (2pt) node[blue,pin=\scriptsize{ \anchor} ] {} ;
\foreach \anchor/\placement in
{tip/above, tail center/right, right tail/below,
left tail/below, right tail/below, left side/above left, right side/above right}
\fill[red,pin position=\placement] (node cs:name=A,anchor= \anchor) circle (2pt) node[blue,pin=\scriptsize{ \anchor} ] {} ;
\end{tikzpicture}
&
\begin{tikzpicture}[]
\node[shape=circular sector,draw,minimum size=3cm] (A) at (1,1) {\Huge XXX};
\foreach \anchor/\placement in
{west/170  , east/right , north/above , south/below, north west/left, north east/above, south west/left, south east/below, 120/left}
\fill[blue,pin position=\placement] (node cs:name=A,anchor= \anchor) circle (2pt) node[blue,pin=\scriptsize{ \anchor} ] {} ;
\foreach \anchor/\placement in
{sector center/above, arc start/above, arc end/below, arc center/190}
\fill[red,pin position=\placement] (node cs:name=A,anchor= \anchor) circle (2pt) node[blue,pin=\scriptsize{ \anchor} ] {} ;
\end{tikzpicture}
\\ \hline 
\end{tabular}



\bigskip

\begin{tabular}{|c|c|} \hline 
shape=cylinder & shape=cloud
\\  \hline 
\begin{tikzpicture}[]
\node[shape=cylinder,draw,minimum size=3cm] (A) at (1,1) {\Huge XXX};
\foreach \anchor/\placement in
{west/170  , east/-10 , north/above , south/below, north west/left, north east/above, south west/left, south east/below, 120/left}
\fill[blue,pin position=\placement] (node cs:name=A,anchor= \anchor) circle (2pt) node[blue,pin=\scriptsize{ \anchor} ] {} ;
\foreach \anchor/\placement in
{before top/10 , top/10, after top/below right, before bottom/below left, bottom/190, after bottom/above left}
\fill[red,pin position=\placement] (node cs:name=A,anchor= \anchor) circle (2pt) node[blue,pin=\scriptsize{ \anchor} ] {} ;
\end{tikzpicture}
&
\begin{tikzpicture}[]
\node[shape=cloud,draw,minimum size=3cm] (A) at (1,1) {\Huge XXX};
\foreach \anchor/\placement in
{west/west  , east/east , north/below , south/below left, north west/left, north east/above right, south west/left, south east/right, 110/above}
\fill[blue,pin position=\placement] (node cs:name=A,anchor= \anchor) circle (2pt) node[blue,pin=\scriptsize{ \anchor} ] {} ;
\foreach \anchor/\placement in
{puff 1/above, puff 2/above left , puff 3/left, puff 4/left,
puff 5/below left, puff 6/below right, puff 7/below right, puff 8/right,
puff 9/right, puff 10/above}
\fill[red,pin position=\placement] (node cs:name=A,anchor= \anchor) circle (2pt) node[blue,pin=\scriptsize{ \anchor} ] {} ;
\end{tikzpicture}

\\ \hline 
\end{tabular}

\bigskip

\begin{tabular}{|c|} \hline 
shape=starburst
\\  \hline 
\begin{tikzpicture}[]
\node[shape=starburst, starburst points=9, starburst point height=2cm,draw,minimum size=3cm] (A) at (1,1) {\Huge XXX};
\foreach \anchor/\placement in
{west/west  , east/east , north/70 , south/above, north west/below , north east/below, south west/below left, south east/-85, 30/above right}
\fill[blue,pin position=\placement] (node cs:name=A,anchor= \anchor) circle (2pt) node[blue,pin=\scriptsize{ \anchor} ] {} ;
\foreach \anchor/\placement in
{outer point 1/105, outer point 2/above left , 
outer point 3/left, outer point 4/left, 
outer point 5/below, outer point 6/below, 
outer point 7/below, outer point 8/right, 
outer point 9/above,
inner point 1/93, inner point 2/160, 
inner point 3/190, inner point 4/below left, 
inner point 5/below, inner point 6/-85,
inner point 7/-30, inner point 8/above right, 
inner point 9/above}
\fill[red,pin position=\placement] (node cs:name=A,anchor= \anchor) circle (2pt) node[blue,pin=\scriptsize{ \anchor} ] {} ;
\end{tikzpicture}
\\ \hline 
\end{tabular}


\bigskip

\begin{tabular}{|c|} \hline 
shape=signal
\\  \hline 
\begin{tikzpicture}[]
\node[signal,signal from=west,draw,minimum size=3.5cm] (A) at (1,1) {\Huge XXX};
\foreach \anchor/\placement in
{north west/above left, north/above, north east/above right,
west/left, center/above, east/right,
mid west/left, mid/below right, mid east/right,
base west/-160, base/below, base east/below right,
south west/below left, south/below, south east/below right,
text/below, 20/right, 120/above}
\fill[blue,pin position=\placement] (node cs:name=A,anchor= \anchor) circle (2pt) node[blue,pin=\scriptsize{ \anchor} ] {} ;
\end{tikzpicture}
\\ \hline 
\end{tabular}


\bigskip

\begin{tabular}{|c|} \hline 
shape=tape
\\  \hline 
\begin{tikzpicture}[]
\node[tape, tape bend height=1cm,draw,minimum size=3.5cm] (A) at (1,1) {\Huge XXX};
\foreach \anchor/\placement in
{north west/above left, north/above, north east/above right,
west/left, center/above, east/right,
mid west/left, mid/below right, mid east/right,
base west/-160, base/110, base east/below right,
south west/below left, south/below, south east/below right,
text/110, 20/right, 120/above}
\fill[blue,pin position=\placement] (node cs:name=A,anchor= \anchor) circle (2pt) node[blue,pin=\scriptsize{ \anchor} ] {} ;
\end{tikzpicture}
\\ \hline 
\end{tabular}

\begin{tabular}{|c|} \hline 
shape=magnetic tape
\\  \hline 
\begin{tikzpicture}[]
\node[shape=magnetic tape,draw,minimum size=3cm] (A) at (1,1) {\Huge XXX};
\foreach \anchor/\placement in
{west/west  , east/east , north/above , south/below, north west/above left , north east/above right, south west/left, south east/below, 30/above right}
\fill[blue,pin position=\placement] (node cs:name=A,anchor= \anchor) circle (2pt) node[blue,pin=\scriptsize{ \anchor} ] {} ;
\foreach \anchor/\placement in
{tail east/right, tail south east/below right, tail north east/above right}
\fill[red,pin position=\placement] (node cs:name=A,anchor= \anchor) circle (2pt) node[blue,pin=\scriptsize{ \anchor} ] {} ;
\end{tikzpicture}
\\ \hline 
\end{tabular}



\bigskip

\begin{tabular}{|c|} \hline 
shape=single arrow
\\  \hline 
\begin{tikzpicture}[]
\node[shape=single arrow,draw,minimum size=3cm] (A) at (1,1) {\Huge XXXXXX};
\foreach \anchor/\placement in
{west/170  , east/below right , north/above , south/below, north west/above left, north east/above right, south west/below left, south east/below right, 30/east}
\fill[blue,pin position=\placement] (node cs:name=A,anchor= \anchor) circle (2pt) node[blue,pin=\scriptsize{ \anchor} ] {} ;
\foreach \anchor/\placement in
{tip/above right, before tip/above, after tip/below, before head/190 , after head/170, after tail/left, before tail/left, tail/190}
\fill[red,pin position=\placement] (node cs:name=A,anchor= \anchor) circle (2pt) node[blue,pin=\scriptsize{ \anchor} ] {} ;
\end{tikzpicture}
\\ \hline 
\end{tabular}


\bigskip

\begin{tabular}{|c|} \hline 
shape=double arrow
\\  \hline 
\begin{tikzpicture}[]
\node[shape=double arrow, double arrow head extend=1.5cm,,draw,minimum size=3cm] (A) at (1,1) {\Huge XXXXXXXXX};
\foreach \anchor/\placement in
{west/170  , east/-10 , north/above , south/below, north west/above left, north east/above right, south west/below left, south east/below right, 35/above right}
\fill[blue,pin position=\placement] (node cs:name=A,anchor= \anchor) circle (2pt) node[blue,pin=\scriptsize{ \anchor} ] {} ;
\foreach \anchor/\placement in
{before head 1/above right, before tip 1/above, 
tip 1/10, after tip 1/below, 
after head 1/below right, before head 2/below left, 
before tip 2/below left, tip 2/190, 
after tip 2/above left, after head 2/above left}
\fill[red,pin position=\placement] (node cs:name=A,anchor= \anchor) circle (2pt) node[blue,pin=\scriptsize{ \anchor} ] {} ;
\end{tikzpicture}
\\ \hline 
\end{tabular}


\bigskip

\begin{tabular}{|c|} \hline 
shape=arrow box
\\  \hline 
\begin{tikzpicture}[]
\node[shape=arrow box,draw,minimum size=3cm,arrow box arrows={north:2cm from border, south, east:2cm from border, west},arrow box shaft width=1cm,arrow box head extend=0.25cm] (A) at (1,1) {\Huge XXXXXXXXX};
\foreach \anchor/\placement in
{west/right  , east/left , north/below , south/above, north west/left, north east/right, south west/left, south east/right}
\fill[blue,pin position=\placement] (node cs:name=A,anchor= \anchor) circle (2pt) node[blue,pin=\scriptsize{ \anchor} ] {} ;
\foreach \anchor/\placement in
{north arrow tip/above,
south arrow tip/below, 
east arrow tip/right, 
west arrow tip/left,
before north arrow/above left, 
before north arrow head/110, 
before north arrow tip/left,
after north arrow tip/right, 
after north arrow head/70, 
after north arrow/above right,
before south arrow/below right, 
before south arrow head/-70, 
before south arrow tip/right,
after south arrow tip/left, 
after south arrow head/-110, 
after south arrow/below left,
before east arrow/above right, 
before east arrow head/right, 
before east arrow tip/right,
after east arrow tip/right, 
after east arrow head/right, 
after east arrow/below right,
before west arrow/below left, 
before west arrow head/left, 
before west arrow tip/left,
after west arrow tip/west, 
after west arrow head/left, 
after west arrow/above left}
\fill[red,pin position=\placement] (node cs:name=A,anchor= \anchor) circle (2pt) node[blue,pin=\scriptsize{ \anchor} ] {} ;
\end{tikzpicture}
\\ \hline 
\end{tabular}


\bigskip

\begin{tabular}{|c|} \hline 
shape=circle split
\\  \hline 
\begin{tikzpicture}[]
\node[shape=circle split,draw,minimum size=3.5cm](A) at (1,1) {XXX\nodepart{lower}YYY}  ;
\foreach \anchor/\placement in
{north west/above left, north/above, north east/above right,
west/left, center/above, east/right,
mid west/left, mid/below right, mid east/right,
base west/-160, base/110, base east/below right,
south west/below left, south/below, south east/below right,
text/110, 20/right, 120/above}
\fill[blue,pin position=\placement] (node cs:name=A,anchor= \anchor) circle (2pt) node[blue,pin=\scriptsize{ \anchor} ] {} ;
\foreach \anchor/\placement in
{text/left, lower/left}
\fill[red,pin position=\placement] (node cs:name=A,anchor= \anchor) circle (2pt) node[blue,pin=\scriptsize{ \anchor} ] {} ;
\end{tikzpicture}
\\ \hline 
\end{tabular}

\begin{tabular}{|c|} \hline 
shape=circle solidus
\\  \hline 
\begin{tikzpicture}[]
\node[shape=circle solidus,draw,minimum size=3.5cm](A) at (1,1) {XXX\nodepart{lower}YYY}  ;
\foreach \anchor/\placement in
{north west/above left, north/above, north east/above right,
west/left, center/above, east/right,
mid west/left, mid/below right, mid east/right,
base west/-160, base/110, base east/below right,
south west/below left, south/below, south east/below right,
text/110, 20/right, 120/above}
\fill[blue,pin position=\placement] (node cs:name=A,anchor= \anchor) circle (2pt) node[blue,pin=\scriptsize{ \anchor} ] {} ;
\foreach \anchor/\placement in
{text/left, lower/left}
\fill[red,pin position=\placement] (node cs:name=A,anchor= \anchor) circle (2pt) node[blue,pin=\scriptsize{ \anchor} ] {} ;
\end{tikzpicture}
\\ \hline 
\end{tabular}


\bigskip

\begin{tabular}{|c|} \hline 
shape=ellipse split
\\  \hline 
\begin{tikzpicture}[]
\node[shape=ellipse split,draw,minimum size=3.5cm](A) at (1,1) {XXX\nodepart{lower}YYY}  ;
\foreach \anchor/\placement in
{north west/above left, north/above, north east/above right,
west/left, center/above, east/right,
mid west/left, mid/below right, mid east/right,
base west/-160, base/110, base east/below right,
south west/below left, south/below, south east/below right,
text/110, 20/right, 120/above}
\fill[blue,pin position=\placement] (node cs:name=A,anchor= \anchor) circle (2pt) node[blue,pin=\scriptsize{ \anchor} ] {} ;
;
\end{tikzpicture}
\\ \hline 
\end{tabular}

\bigskip

\begin{tabular}{|c|} \hline 
shape=rectangle split
\\  \hline 
\begin{tikzpicture}[]
\node[name=s,shape=rectangle split, rectangle split parts=4,draw,inner ysep=0.75cm](A) at (1,1)
{\nodepart{text}XXXXXXXXXXXXXX\nodepart{two}YYY
\nodepart{three}ZZZ\nodepart{four}four};
\foreach \anchor/\placement in
{north/above, south/below, east/10, west/170,
north west/above, north east/above, south west/below, south east/below,
center/145, 20/right, mid/30, base/-145}
\fill[blue,pin position=\placement] (node cs:name=A,anchor= \anchor) circle (2pt) node[blue,pin=\scriptsize{ \anchor} ] {} ;
\foreach \anchor/\placement in
{text split/10, text split east/0, text split west/180,two split/30, two split east/right, two split west/left,
three split/30, three split east/east, three split west/west,text/-170, text east/east, text west/west,
two/left, two east/east, two west/west,
three/left, three east/east, three west/west,
four/west, four east/east, four west/west
}
\fill[red,pin position=\placement] (node cs:name=A,anchor= \anchor) circle (2pt) node[blue,pin=\scriptsize{ \anchor} ] {} ;
\end{tikzpicture}
\\ \hline 
\end{tabular}

\bigskip

\begin{tabular}{|c|} \hline 
shape=rectangle callout
\\  \hline 
\begin{tikzpicture}[]
\node[shape=rectangle callout, callout relative pointer={(1.5cm,-.5cm)},draw,
callout pointer width=2cm, inner xsep=1cm, inner ysep=.5cm] (A) at (1,1) {\Huge XXXXXXX};
\foreach \anchor/\placement in
{west/west  , east/east , north/above , south/below, north west/west , north east/right, south west/left, south east/right, 25/right}
\fill[blue,pin position=\placement] (node cs:name=A,anchor= \anchor) circle (2pt) node[blue,pin=\scriptsize{ \anchor} ] {} ;
\foreach \anchor/\placement in
{pointer/right}
\fill[red,pin position=\placement] (node cs:name=A,anchor= \anchor) circle (2pt) node[blue,pin=\scriptsize{ \anchor} ] {} ;
\end{tikzpicture}
\\ \hline 
\end{tabular}

\bigskip

\begin{tabular}{|c|} \hline 
shape=ellipse callout
\\  \hline 
\begin{tikzpicture}[]
\node[shape=ellipse callout,draw] (A) at (1,1) {\Huge XXXXXX};
\foreach \anchor/\placement in
{west/west  , east/right , north/above, south/below, north west/above left, north east/above right, south west/below left, south east/below right}
\fill[blue,pin position=\placement] (node cs:name=A,anchor= \anchor) circle (2pt) node[blue,pin=\scriptsize{ \anchor} ] {} ;
\foreach \anchor/\placement in
{pointer/below right}
\fill[red,pin position=\placement] (node cs:name=A,anchor= \anchor) circle (2pt) node[blue,pin=\scriptsize{ \anchor} ] {} ;
\end{tikzpicture}
\\ \hline 
\end{tabular}


\bigskip

\begin{tabular}{|c|} \hline 
shape=cloud callout
\\  \hline 
\begin{tikzpicture}[]
\node[shape=cloud callout,draw,aspect=1.5] (A) at (1,1) {\Huge XXXXXX};
\foreach \anchor/\placement in
{west/west  , east/right , north/below , south/above, north west/above left, north east/above right, south west/below left, south east/below right}
\fill[blue,pin position=\placement] (node cs:name=A,anchor= \anchor) circle (2pt) node[blue,pin=\scriptsize{ \anchor} ] {} ;
\foreach \anchor/\placement in
{puff 1/above, puff 2/above, puff 3/left, puff 4/left,
puff 5/below left, puff 6/below, puff 7/below right, puff 8/right,
puff 9/right, puff 10/above,pointer/below right}
\fill[red,pin position=\placement] (node cs:name=A,anchor= \anchor) circle (2pt) node[blue,pin=\scriptsize{ \anchor} ] {} ;
\end{tikzpicture}
\\ \hline 
\end{tabular}


\bigskip

\begin{tabular}{|c|} \hline 
shape=cross out
\\  \hline 
\begin{tikzpicture}[]
\node[shape=cross out,draw,minimum size=3cm] (A) at (1,1) {\Huge XXXXXXXXXX};
\foreach \anchor/\placement in
{west/west  , east/right , north/above , south/below, north west/above left, north east/above right, south west/below left, south east/below right}
\fill[blue,pin position=\placement] (node cs:name=A,anchor= \anchor) circle (2pt) node[blue,pin=\scriptsize{ \anchor} ] {} ;
\end{tikzpicture}
\\ \hline 
\end{tabular}

\bigskip

\begin{tabular}{|c|} \hline 
shape=rounded rectangle
\\  \hline 
\begin{tikzpicture}[]
\node[shape=rounded rectangle,draw,minimum size=3cm] (A) at (1,1) {\Huge XXXXXXXXXX};
\foreach \anchor/\placement in
{west/west  , east/right , north/above , south/below, north west/above left, north east/above right, south west/below left, south east/below right}
\fill[blue,pin position=\placement] (node cs:name=A,anchor= \anchor) circle (2pt) node[blue,pin=\scriptsize{ \anchor} ] {} ;

\end{tikzpicture}
\\ \hline 
\end{tabular}


\bigskip

\begin{tabular}{|c|} \hline 
shape=chamfered rectangle
\\  \hline 
\begin{tikzpicture}[]
\node[shape=chamfered rectangle,draw,minimum size=3cm, chamfered rectangle sep=.5cm,] (A) at (1,1) {\Huge XXXXXX};
\foreach \anchor/\placement in
{west/west  , east/right , north/above , south/below, north west/above left, north east/above right, south west/below left, south east/below right}
\fill[blue,pin position=\placement] (node cs:name=A,anchor= \anchor) circle (2pt) node[blue,pin=\scriptsize{ \anchor} ] {} ;
\foreach \anchor/\placement in
{before north east/above right, after north east/above right, before south east/below right,after south east/below right, before north west/above left, after north west/above left, before south west/below left,after south west/below left}
\fill[red,pin position=\placement] (node cs:name=A,anchor= \anchor) circle (2pt) node[blue,pin=\scriptsize{ \anchor} ] {} ;
\end{tikzpicture}
\\ \hline 
\end{tabular}

\normalsize


 

%
%%\begin{tikzpicture}
%%[spy using outlines={circle, magnification=4, size=2cm, connect spies}]
%%\draw [help lines] (0,0) grid (3,2);
%%\draw [decoration=Koch curve type 1]
%%decorate { decorate{ decorate{ decorate{ (0,0) -- (2,0) }}}};
%%\spy [red] on (1.6,0.3)
%%in node [left] at (3.5,-1.25);
%%\spy [blue, size=1cm] on (1,1)
%%in node [right] at (0,-1.25);
%%\end{tikzpicture}
%%
%%\begin{tikzpicture}
%%[spy using outlines={circle, magnification=4, size=2cm, connect spies}]
%%\draw [help lines] (0,0) grid (3,2);
%%\shadedraw[shading=Mandelbrot set ] (0,0) rectangle (2,2) ;
%%\spy [red] on (1,0.4)
%%in node [left] at (3.5,-1.25);
%%\spy [blue, size=1cm] on (.5,.8)
%%in node [right] at (0,-1.25);
%%\end{tikzpicture}
%
%
%%
 \maboite{\BS{usetikzlibrary}\AC{turtle}}
\label{lib-turtle}


\begin{center}
\RRR{ 73 }
\end{center}

\begin{tabular}{|c|c|c|c|} \hline 
\multicolumn{4}{|c|}{  \BS{draw} [blue,line width=3pt,turtle={home,forward}];} \\  \hline 
\begin{tikzpicture}
\draw[help lines] (-1.5,-2) grid (1.5,2) ; 
\draw [blue,line width=3pt,turtle={home,forward}];
\end{tikzpicture}
&  
\begin{tikzpicture}
\draw[help lines] (-1.5,-2) grid (1.5,2) ;  
\draw [blue,line width=3pt,turtle={home,forward=1.5cm}];
\end{tikzpicture}
&  
\begin{tikzpicture}
\draw[help lines] (-1.5,-2) grid (1.5,2) ;  
\draw [blue,line width=3pt,turtle={home,fd}];
\end{tikzpicture}
&  
\begin{tikzpicture}
\draw[help lines] (-1.5,-2) grid (1.5,2) ; 
\draw [blue,line width=3pt,turtle={home,fd=1.5cm}];
\end{tikzpicture}
\\ \hline 
turtle=\AC{home,forward}  & turtle=\AC{home,forward=1.5cm} & turtle=\AC{home,fd} & 
turtle=\AC{home,fd=1.5cm} \\ 
\hline 
\end{tabular} 

\bigskip


\begin{tabular}{|c|c|c|c|}
\hline 
\multicolumn{4}{|c|}{  \BS{draw} [blue,line width=3pt,turtle={home,left,fd];}} \\  \hline  
\hline 
\begin{tikzpicture}
\draw (-1,-1) grid (1,1) ; 
\draw [blue,line width=3pt,turtle={home,left,fd}];
\end{tikzpicture} 
&  
\begin{tikzpicture}
\draw (-1,-1) grid (1,1) ; 
\draw [blue,line width=3pt,turtle={home,left=45,fd}];
\end{tikzpicture}
&  
\begin{tikzpicture}
\draw (-1,-1) grid (1,1) ; 
\draw [blue,line width=3pt,turtle={home,lt,fd}];
\end{tikzpicture}
&  
\begin{tikzpicture}
\draw (-1,-1) grid (1,1) ; 
\draw [blue,line width=3pt,turtle={home,lt=45,fd}];
\end{tikzpicture}
\\ \hline
turtle=\AC{home,left,fd}  & turtle=\AC{home,left=45,fd} & turtle=\AC{home,lt,fd} & 
turtle=\AC{home,lt=45,fd} \\ 
\hline 
\end{tabular} 

\bigskip

\begin{tabular}{|c|c|c|c|}
\hline 
\multicolumn{4}{|c|}{  \BS{draw} [blue,line width=3pt,turtle={home,right,fd];}} \\  \hline  
\hline 
\begin{tikzpicture}
\draw (-1,-1) grid (1,1) ; 
\draw [blue,line width=3pt,turtle={home,right,fd}];
\end{tikzpicture} 
&  
\begin{tikzpicture}
\draw (-1,-1) grid (1,1) ; 
\draw [blue,line width=3pt,turtle={home,right=45,fd}];
\end{tikzpicture}
&  
\begin{tikzpicture}
\draw (-1,-1) grid (1,1) ; 
\draw [blue,line width=3pt,turtle={home,rt,fd}];
\end{tikzpicture}
&  
\begin{tikzpicture}
\draw (-1,-1) grid (1,1) ; 
\draw [blue,line width=3pt,turtle={home,rt=45,fd}];
\end{tikzpicture}
\\ \hline
turtle=\AC{home,right,fd}  & turtle=\AC{home,right=45,fd} & turtle=\AC{home,rt,fd} & 
turtle=\AC{home,rt=45,fd} \\ 
\hline 
\end{tabular} 

\bigskip

\begin{tabular}{|c|c|} \hline 
\tikz[blue,line width=3pt]
\draw [->,turtle={home,rt,fd,fd,lt,fd,lt,fd}];
&  
\tikz[blue,line width=3pt]
\draw [->,turtle/distance=2cm,turtle={home,rt,fd,fd,lt,fd,lt,fd}];
\\ \hline 
[->,turtle={home,rt,fd,fd,lt,fd,lt,fd}] & [->,turtle/distance=2cm,turtle={home,rt,fd,fd,lt,fd,lt,fd}] 
\\ \hline 
\end{tabular} 

\bigskip


\begin{tabular}{|c|} \hline 
\begin{tikzpicture}[turtle/distance=2cm]
\draw[help lines] (-1.5,-1) grid (6,3) ; 
\draw [blue,line width=3pt,dotted,turtle={home,forward,right,forward},fd];
\draw [red,line width=3pt,turtle={how/.style={bend left},home,fd,rt,fd,fd}] ;
\end{tikzpicture}
\\  \hline 
[red,turtle=\AC{\rouge{how/.style}=\AC{bend left},home,fd,rt,fd,fd}]
\\ \hline 
\end{tabular} 

\bigskip

\begin{tabular}{|c|c|}  \hline 
\tikz
\filldraw [turtle/distance=2cm,thick,blue,fill=red!20]
[turtle=home]
\foreach \i in {1,...,5}
{
[turtle={forward,right=144}]
}; 
& 
 
\parbox[b]{10cm}{
\BS{filldraw}[turtle/distance=2cm,thick,blue,fill=red!20] \\
$[$ turtle=home $]$ \\
\BS{foreach} \BS{i} in \AC{1,...,5} \\
{
[ turtle=\AC{forward,right=144} ]
};
}
\\ \hline  
\end{tabular} 

\bigskip



\begin{tabular}{|c|c|}  \hline 
\tikz \draw [thick,blue]
[turtle=home]
\foreach \i in {1,...,25}
{
[turtle={forward=\i/5,right=120}]
};
& 
 
\parbox[b]{10cm}{
\BS{draw}[thick,blue] \\
$[$ turtle=home $]$ \\
\BS{foreach} \BS{i} in \AC{1,...,25} \\
{
[turtle=\AC{forward=\BS{i}/5,right=120} ]
}; \\
\vspace{1cm}
}
\\ \hline  
\end{tabular}




%% 
%%
%% 
%%\newpage 
%%
%%\SbSSCT{Matrice de n\oe uds}{Matrices and Alignment}
%%
%%
%%
\label{matrix}
\begin{center}
\RRR{20}
\end{center}

\begin{tabular}{|c|c|} \hline  
\begin{tikzpicture}[baseline=1cm]
\draw[help lines] (0,0) grid (4,2);
\node [matrix,fill=red!20,draw=blue,very thick] (my matrix) at (2,1)
{
\draw (0,0) circle (4mm); & \node[rotate=45] {Hello}; \\
\draw (0.2,0) circle (2mm); & \fill[red] (0,0) circle (3mm); \\
};
\end{tikzpicture}
& 
\parbox{10cm}{
\BS{node} [\RDD{matrix},fill=red!10,draw=blue,very thick] at (2,1) \\
\{ \\
\BS{draw} (0,0) circle (4mm); \& \BS{node} [rotate=45] {Hello}; \BS{}\BS{} \\
\BS{draw}  (0.2,0) circle (2mm); \& \BS{fill}[red] (0,0) circle (3mm); \BS{}\BS{} \\
\}; \\
}
\\ \hline 
\end{tabular} 

\bigskip

\begin{tabular}{|c|c|} \hline  
\begin{tikzpicture}[baseline=0pt]
\matrix [fill=red!20,draw=blue,very thick] 
{
\draw (0,0) circle (4mm); & \node[rotate=45] {Hello}; \\
\draw (0.2,0) circle (2mm); & \fill[red] (0,0) circle (3mm); \\
};
\end{tikzpicture}
&  
\parbox{10cm}{
\BSS{matrix} [fill=red!10,draw=blue,very thick] \\
\{ \\
\BS{draw} (0,0) circle (4mm); \& \BS{node} [rotate=45] {Hello}; \BS{}\BS{} \\
\BS{draw}  (0.2,0) circle (2mm); \& \BS{fill}[red] (0,0) circle (3mm); \BS{}\BS{} \\
\}; \\
}
\\ \hline 
\end{tabular} 


\SbSbSSCT{Alignement des cellules}{Cell Pictures}


\begin{center}
\RRR{20-3}
\end{center}

\begin{tabular}{|c|c|c|} \hline  
\begin{tikzpicture}
[every node/.style={draw=black,font=\huge}]
\matrix [draw=red]
{
\node {a}; \fill[blue] (0,0) circle (2pt); &
\node {X}; \fill[blue] (0,0) circle (2pt); &
\node {g}; \fill[blue] (0,0) circle (2pt); \\
};
\end{tikzpicture}
&  
\begin{tikzpicture}
[every node/.style={draw=black,anchor=base,font=\huge}]
\matrix [draw=red]
{
\node {a}; \fill[blue] (0,0) circle (2pt); &
\node {X}; \fill[blue] (0,0) circle (2pt); &
\node {g}; \fill[blue] (0,0) circle (2pt); \\
};
\end{tikzpicture}
&  
\begin{tikzpicture}[every node/.style={draw=black}]
\matrix [draw=red,anchor=north,font=\huge]
{
\node {a}; \fill[blue] (0,0) circle (2pt); &
\node {X}; \fill[blue] (0,0) circle (2pt); &
\node {g}; \fill[blue] (0,0) circle (2pt); \\
};
\end{tikzpicture}
\\ \hline  
 & anchor=base &  anchor=north \\ \hline 
\end{tabular} 

\bigskip
\begin{tabular}{|c|c|c|} \hline  
\begin{tikzpicture}
[every node/.style={draw=black,font=\huge}]
\matrix [draw=red]
{

\node[left]  {X}; \fill[blue] (0,0) circle (2pt);  \\
};
\end{tikzpicture}
&  
\begin{tikzpicture}
[every node/.style={draw=black,anchor=base,font=\huge}]
\matrix [draw=red]
{
\node {a}; \fill[blue] (0,0) circle (2pt); ²\\
\node[right] {X}; \fill[blue] (0,0) circle (2pt);  \\
\node {g}; \fill[blue] (0,0) circle (2pt); \\
};
\end{tikzpicture}
&  
\begin{tikzpicture}[every node/.style={draw=black}]
\matrix [draw=red,anchor=north,font=\huge]
{
\node {a}; \fill[blue] (0,0) circle (2pt); &
\node[right] {X}; \fill[blue] (0,0) circle (2pt); &
\node {g}; \fill[blue] (0,0) circle (2pt); \\
};
\end{tikzpicture}
\\ \hline  
 & anchor=base &  anchor=north \\ \hline 
\end{tabular} 

\bigskip

\begin{tabular}{|c|c|} \hline  
\begin{tikzpicture}[baseline=0pt]
\matrix [draw=red,nodes=draw]
{
\node[left] {A}; \fill[blue] (0,0) circle (2pt); \\
\node {B}; \fill[blue] (0,0) circle (2pt); \\
\node[right] {C}; \fill[blue] (0,0) circle (2pt); \\
};
\end{tikzpicture}
&  
\parbox{12cm}{
\BS{matrix} [draw=red,nodes=draw]
\AC{\\
\BS{node}\rouge{[left]} {A}; \BS{fill}[blue] (0,0) circle (2pt); \BS{} \BS{} \\
\BS{node} {B}; \BS{fill}[blue] (0,0) circle (2pt);\BS{} \BS{} \\
\BS{node}\rouge{[right]} {C}; \BS{fill}[blue] (0,0) circle (2pt); \BS{} \BS{}\\
}; \\
}

\\ \hline 
\end{tabular} 

\bigskip

\begin{tabular}{|c|c|} \hline  
\multicolumn{2}{|c|}{\BS{matrix} [draw,\RDD{column  sep}=1cm,nodes=draw]} 
\\ \hline 
\begin{tikzpicture}
\matrix [draw,column sep=1cm,nodes=draw]
{
\node(a) {123}; & \node (b) {1}; & \node {1}; \\
\node {12}; & \node {12}; & \node {1}; \\
\node(c) {1}; & \node (d) {123}; & \node {1}; \\
};
\draw [red,thick] (a.east) -- (a.east |- c)
(d.west) -- (d.west |- b);
\draw [<->,red,thick] (a.east) -- (d.west |- b)
node [above,midway] {1cm};
\end{tikzpicture}
&  
\begin{tikzpicture}
\matrix [draw,column sep={1cm,between origins},nodes=draw]
{
\node(a) {123}; & \node (b) {1}; & \node {1}; \\
\node {12}; & \node {12}; & \node {1}; \\
\node {1}; & \node {123}; & \node {1}; \\
};
\draw [<->,red,thick] (a.center) -- (b.center) node [above,midway] {1cm};
\end{tikzpicture}
\\ \hline \RDD{column sep}=1cm & column sep=\AC{1cm,\RDD{between origins} } 
\\ \hline 
\end{tabular} 

\bigskip

\begin{tabular}{|c|c|} \hline
\multicolumn{2}{|c|}{\BS{matrix} [draw,\RDD{row sep}=1cm,nodes=draw]} 
\\ \hline 
\begin{tikzpicture}
\matrix [draw,row sep=1cm,nodes=draw]
{
\node (a) {123}; & \node {1}; & \node {1}; \\
\node (b) {12}; & \node {12}; & \node {1}; \\
\node {1}; & \node {123}; & \node {1}; \\
};
\draw [<->,red,thick] (a.south) -- (b.north) node [right,midway] {1cm};
\end{tikzpicture}
&
\begin{tikzpicture}
\matrix [draw,row sep={1cm,between origins},nodes=draw]
{
\node (a) {123}; & \node {1}; & \node {1}; \\
\node (b) {12}; & \node {12}; & \node {1}; \\
\node {1}; & \node {123}; & \node {1}; \\
};
\draw [<->,red,thick] (a.center) -- (b.center) node [right,midway] {1cm};
\end{tikzpicture}
\\  \hline 
\RDD{row sep}=1cm  & row sep=\AC{1cm,\RDD{between origins} } 
\\ \hline 


\end{tabular} 




\bigskip

\begin{tabular}{|c|c|} \hline  
\multicolumn{2}{|c|}{\BS{matrix} [ \rouge{row sep=5mm},draw,nodes=draw]} \\
\multicolumn{2}{|c|}{ \{ \BS{node} \AC{1}; \& \BS{node} \AC{2}; \& \BS{node} \AC{3}; \BS{}\BS{}  } \\
\multicolumn{2}{|c|}{ \BS{node} \AC{4} ; \& \BS{node}  \AC{5}; \& \BS{node}  \AC{6};  \BS{}\BS{} \rouge{[1cm]} } \\
\multicolumn{2}{|c|}{ \BS{node} \AC{7}; \& \BS{node}\AC{8}; \& \BS{node}\AC{9}; \BS{}\BS{} \}  } 
\\ \hline  
\begin{tikzpicture}
\matrix [row sep=5mm,draw,nodes=draw]
{
\node {1}; & \node {2};& \node {3}; \\
\node(a) {4} ; & \node {5}; & \node {6};\\[1cm]
\node(b) {7}; &\node {8}; & \node {9}; \\
};
\draw [<->,red,thick] (a.center) -- (b.center) node [right,midway] {1,5cm};
\end{tikzpicture}
&  
\begin{tikzpicture}
\matrix [row sep=5mm,draw,nodes=draw]
{
\node {1}; & \node {2};& \node {3}; \\
\node(a) {4} ; & \node {5}; & \node {6};\\[10mm,between origins]
\node(b) {7}; &\node {8}; & \node {9}; \\
};
\draw [<->,red,thick] (a.center) -- (b.center) node [right,midway] {1,5cm};
\end{tikzpicture}
\\ \hline 
\rouge{[1cm]} & \rouge{[1cm,between origins]}
\\ \hline 
\end{tabular} 

\bigskip

\begin{tabular}{|c|c|} \hline  
\multicolumn{2}{|c|}{\BS{matrix} [ \rouge{column sep=5mm},draw,nodes=draw]} \\
\multicolumn{2}{|c|}{ \{ \BS{node} \AC{1}; \& \BS{node} \AC{2}; \& \BS{node} \AC{3}; \BS{}\BS{}  } \\
\multicolumn{2}{|c|}{ \BS{node} \AC{4} ; \& \BS{node}  \AC{5}; \& \rouge{[1cm]}\BS{node}  \AC{6};  \BS{}\BS{}  } \\
\multicolumn{2}{|c|}{ \BS{node} \AC{7}; \& \BS{node}\AC{8}; \& \BS{node}\AC{9}; \BS{}\BS{} \}  } 
\\ \hline  

\begin{tikzpicture}
\matrix [draw,nodes=draw,column sep=5mm]
{
\node {1}; & \node(a) {2}; &[1cm] \node(b) {3}; \\
\node {4}; & \node{5}; & \node {6}; \\
\node {7}; & \node{8}; & \node {9}; \\
};
\draw [<->,red,thick] (a.east) -- (b.west) node [above,midway] {15mm};
\end{tikzpicture}
&  
\begin{tikzpicture}
\matrix [draw,nodes=draw,column sep=5mm]
{
\node {1}; &[2mm] \node(a){2}; &[1cm,between origins] \node(b){3}; \\
\node {4}; & \node {5}; & \node {6}; \\
\node {7}; & \node {8}; & \node {9}; \\
};
\draw [<->,red,thick] (a.center) -- (b.center) node [above,midway] {15mm};
\end{tikzpicture}
\\ \hline  
\rouge{[1cm]}
&  
\rouge{[1cm,between origins]}
\\ \hline 
\end{tabular} 




\bigskip

\begin{tikzpicture}
\matrix [draw,nodes=draw,column sep={1cm,between origins}]
{
\node (a) {8}; & \node (b) {1}; &[between borders] \node (c) {6}; \\
\node {3}; & \node {5}; & \node {7}; \\
\node {4}; & \node {9}; & \node {2}; \\
};
\draw [<->,red,thick] (a.center) -- (b.center) node [above,midway] {10mm};
\draw [<->,red,thick] (b.east) -- (c.west) node [above,midway] {1cm};
\end{tikzpicture}



\SbSbSSCT{Format des cellules}{Cell Styles and Options}

\noindent 

\begin{tabular}{|c|} \hline  
\BS{matrix} [nodes=draw,nodes=\AC{\rouge{fill}=blue!10\rouge{,minimum size}=1cm}]
\\ \hline  
\begin{tikzpicture}
\matrix [nodes=draw,nodes={fill=blue!10,minimum size=1cm}]
{
\node {1}; & \node{2}; & \node {3}; \\
\node {4}; & \node{5}; & \node {6}; \\
\node {7}; & \node{8}; & \node {9}; \\
};
\end{tikzpicture}
\\ \hline 
\end{tabular} 


\bigskip 


\begin{tabular}{|c|c|c|} \hline 
\multicolumn{3}{|c|}{\BS{matrix}[\rouge{row 2/.style}=\AC{red}]}
 \\ \hline 
\begin{tikzpicture}
\matrix[row 2/.style={red}]
{
\node {8}; & \node{1}; & \node {6}; \\
\node {3}; & \node{5}; & \node {7}; \\
\node {4}; & \node{9}; & \node {2}; \\
};
\end{tikzpicture}
&  
\begin{tikzpicture}
\matrix[column 2/.style={red}]
{
\node {8}; & \node{1}; & \node {6}; \\
\node {3}; & \node{5}; & \node {7}; \\
\node {4}; & \node{9}; & \node {2}; \\
};
\end{tikzpicture}
&  
\begin{tikzpicture}
\matrix[row 2 column 2/.style={red}]
{
\node {8}; & \node{1}; & \node {6}; \\
\node {3}; & \node{5}; & \node {7}; \\
\node {4}; & \node{9}; & \node {2}; \\
};
\end{tikzpicture}
\\ \hline 
row 2/.style=\AC{red} & column 2/.style=\AC{red}  & row 2 column 2/.style=\AC{red}\\ 
\hline 
\end{tabular} 

\bigskip 

\begin{tabular}{|c|c|c|} \hline 
\multicolumn{3}{|c|}{\BS{matrix}[column 1/.style=\AC{anchor=west}]}
 \\ \hline 
\begin{tikzpicture}
\matrix[column 1/.style={anchor=west}]
{
\node {12345};  & \node {67890}; \\
\node {123}; & \node{67};  \\
\node {1}; & \node{6}; & \\
};
\end{tikzpicture}
&  
\begin{tikzpicture}
\matrix[column 1/.style={anchor=east}]
{
\node {12345};  & \node {67890}; \\
\node {123}; & \node{67};  \\
\node {1}; & \node{6}; & \\
};
\end{tikzpicture}
&  
\begin{tikzpicture}
\matrix[column 1/.style={anchor=base}]
{
\node {12345};  & \node {67890}; \\
\node {123}; & \node{67};  \\
\node {1}; & \node{6}; & \\
};
\end{tikzpicture}
\\  \hline  
[\rouge{column 1/.style}={anchor=west}]& [\rouge{column 1/.style}={anchor=east}] & [\rouge{column 1/.style}={anchor=base}]\\ 
\hline 
\end{tabular} 

\bigskip

\begin{tabular}{|c|c|c|c|} \hline
\multicolumn{4}{|c|}{\BS{matrix}[matrix of nodes,\RDD{every odd column}/.style={red}]}
 \\ \hline 
\begin{tikzpicture}
\matrix [matrix of nodes,every odd column/.style={red}]
{
a & b & c & d \\
e & f & g & h \\
i & j & k & l \\
};
\end{tikzpicture}
&  
\begin{tikzpicture}
\matrix [matrix of nodes,every even column/.style={red}]
{
a & b & c & d \\
e & f & g & h \\
i & j & k & l \\
};
\end{tikzpicture}
&  
\begin{tikzpicture}
\matrix [matrix of nodes,every odd row/.style={red}]
{
a & b & c & d \\
e & f & g & h \\
i & j & k & l \\
};
\end{tikzpicture}
&  
\begin{tikzpicture}
\matrix [matrix of nodes,every even row/.style={red}]
{
a & b & c & d \\
e & f & g & h \\
i & j & k & l \\
};
\end{tikzpicture}
\\ 
\hline 
\RDD{every odd column} & \RDD{every even column} & \RDD{every odd row}  & \RDD{every even row} \\ 
\hline 
\end{tabular} 


\bigskip


\begin{tabular}{|c|} \hline  
\BS{matrix} [draw,matrix of nodes,\rouge{execute at begin cell}=\AC{(}]
\\ \hline  
\begin{tikzpicture}
\matrix [draw,matrix of nodes,execute at begin cell={(}]
{
1 & 2 &   \\
4 &   & 6 \\
  &   & 9 \\
};
\end{tikzpicture}
\\ \hline 
\end{tabular} 

\bigskip

\begin{tabular}{|c|} \hline  
\BS{tikz} 
[matrix of nodes/.style=\AC{
execute at begin cell=\BS{node}\BS{bgroup} , \\
\rouge{execute at end cell}=\$m\wedge 2\$\BS{egroup}; 
}] \\
\BS{matrix} [draw,matrix of nodes
]
\\ \hline  
\tikz 
[matrix of nodes/.style={
execute at begin cell=\node\bgroup ,
execute at end cell=$m^2$\egroup;
}]
\matrix [draw,matrix of nodes
]
{1 & 2 &  \\
4 &   & 6 \\
  & 8 & 9 \\
};
\\ \hline 
\end{tabular}

\bigskip

\begin{tabular}{|c|} \hline 

 \BS{matrix} [raw,matrix of nodes, \rouge {execute at empty cell}=\BS{node}\AC{- -}; ]
\\ \hline 
 
\begin{tikzpicture}
\matrix [draw,matrix of nodes,execute at empty cell=\node{--};]
{
1 & 2 & \\
4 & & 6 \\
& & 9 \\
};
\end{tikzpicture}
\\ \hline  
\end{tabular} 


\newpage
\SbSbSSCT{Points d'ancrage}{Anchoring a Matrix}

\begin{center}
\RRR{20-4}
\end{center}

\begin{tabular}{|c|c|c|} \hline 
\multicolumn{3}{|c|}{
\BS{matrix} [draw=red,nodes=draw,\RDD{matrix anchor}=east](XXX) at (1,1) }
\\ \hline  
\begin{tikzpicture}
\draw[help lines] (0,0) grid (3,3);
\matrix [draw=red,nodes=draw,matrix anchor=west](XXX) at (1,1)
{
\node {123}; \\ 
\node {12}; \\
\node {1}; \\
};
\fill[red](XXX.west) circle (3pt);
\end{tikzpicture}
&  
\begin{tikzpicture}
\draw[help lines] (0,0) grid (3,3);
\matrix [draw=red,nodes=draw,matrix anchor=east](XXX) at (1,1)
{
\node {123}; \\ 
\node {12}; \\
\node {1}; \\
};
\fill[red] (XXX.east) circle (3pt);
\end{tikzpicture}
&  
\begin{tikzpicture}
\draw[help lines] (0,0) grid (3,3);
\matrix [draw=red,nodes=draw,matrix anchor=south](XXX) at (1,1)
{
\node {123}; \\ 
\node {12}; \\
\node {1}; \\
};
\fill[red](XXX.south) circle (3pt);
\end{tikzpicture}

\\  \hline 
matrix anchor=west & matrix anchor=east & matrix anchor=south 
\\ \hline 
\end{tabular} 

\bigskip 
\begin{tabular}{|c|c|c|c|} \hline 
\multicolumn{2}{|c|}{\BS{matrix} [draw=red,nodes=draw,\rouge{anchor=west}] }
\\ \hline  
\begin{tikzpicture}
\matrix [draw=red,nodes=draw,anchor=west] 
{
\node {123}; & \node {abc}; \\ 
\node {12}; & \node {ab}; \\
\node {1}; & \node {a}; \\
};
\end{tikzpicture}
&  
\begin{tikzpicture}
\matrix [draw=red,nodes=draw,anchor=east] 
{
\node {123};& \node {abc}; \\ 
\node {12};  &\node {ab};\\
\node {1};  & \node {a}; \\
};
\end{tikzpicture}

\\ \hline  
anchor=west & anchor=east  \\ 
\hline 
\end{tabular} 

\bigskip 


\begin{tabular}{|c|c|}\hline  
\begin{tikzpicture}[baseline=1cm]
\draw[help lines] (0,0) grid (4,3);
\matrix[draw=red,nodes=draw ,matrix anchor=inner node.south,anchor=base, row sep=5mm, column sep=5mm] at (2,1)
{
\node {a}; & \node {b}; & \node {c}; & \node {d}; \\
\node {a}; & \node {b}; & \node(inner node){c}; & \node {d}; \\
\node {a}; & \node {b}; & \node {c}; & \node {d}; \\
};
\fill[red] (inner node.south) circle (3pt);
\end{tikzpicture}
&  
\parbox{10.5cm}{
\BS{matrix}[draw=red,nodes=draw, \\ 
\RDD{ matrix anchor}=\blll{inner node}.south, anchor=base, \\
  row sep=5mm,column sep=5mm] at (2,1) \\
\{ \\
\BS{node} \AC{a}; \& \BS{node} \AC{b}; \& \BS{node} \AC{c}; \& \BS{node} \AC{d};  \BS{}\BS{} \\
\BS{node} \AC{a}; \& \BS{node} \AC{b}; \& \BS{node}(\blll{inner node})\AC{c}; \& \BS{node} \AC{d};  \BS{}\BS{} \\
\BS{node}\AC{a}; \& \BS{node} \AC{b}; \& \BS{node}\AC{c}; \& \BS{node} \AC{d}; \BS{}\BS{}  \\
\};
}
\\ \hline 
\end{tabular} 


\SbSbSSCT{Changement du séparateur}{Considerations Concerning Active Characters}

\begin{center}
\RRR{20-5}
\end{center}

\begin{tabular}{|c|c|} \hline  
\tikz[baseline=0pt]
\matrix [ampersand replacement=\|]
{
\draw (0,0) circle (4mm); \| \node[rotate=10] {Hello}; \\
\draw (0.2,0) circle (2mm); \| \fill[red] (0,0) circle (3mm); \\
};
& 
\parbox{12cm}{ 
\BS{tikz}
\BS{matrix} [\RDD{ampersand replacement}=\blll{\BS{|}} ] \\
\{ \\
\BS{draw} (0,0) circle (4mm); \blll{\BS{|} }  \BS{node}[rotate=10] \AC{Hello}; \BS{}\BS{} \\
\BS{draw} (0.2,0) circle (2mm);  \blll{\BS{|} }   \BS{fill}[red] (0,0) circle (3mm); \BS{}\BS{} \\
\}; \\
}
\\ \hline 
\end{tabular} 


\SbSSCT{Matrice de n\oe uds (compléments) }{Matrix Library}

 \maboite{\BS{usetikzlibrary}\AC{matrix}}
\label{lib-matrix}


\begin{center}
\RRR{57-1}
\end{center}

\begin{tabular}{|c|c|} \hline  
\begin{tikzpicture}[baseline=0pt]
\matrix (XXX) [matrix of nodes]
{
1 & 2 & 3 \\
4 & 5 & 6 \\
7 & 8 & 9 \\
};
\end{tikzpicture}
& 
\parbox{10cm}{ 
\BS{begin}\AC{tikzpicture} \\
\BSS{matrix}  [matrix of nodes]\\
\{ \\
1 \hspace{3mm} \& \hspace{3mm}  2 \hspace{3mm} \& \hspace{3mm} 3 \hspace{3mm} \BS{}\BS{}   \\
4 \hspace{3mm} \& \hspace{3mm}  5 \hspace{3mm} \& \hspace{3mm} 6 \hspace{3mm} \BS{}\BS{}  \\
7 \hspace{3mm} \& \hspace{3mm}  8 \hspace{3mm} \& \hspace{3mm} 9 \hspace{3mm} \BS{}\BS{} \\
\}; \\
\BS{end}\AC{tikzpicture}
}
\\ \hline  
\end{tabular} 

\bigskip

\begin{tabular}{|c|c|} \hline  
\begin{tikzpicture}[baseline=0pt]
\matrix (XXX) [matrix of nodes,column sep=.5cm,row sep=.5cm,every node/.style=draw]
{
1 & 2 & 3 \\
4 & 5 & 6 \\
7 & 8 & 9 \\
};
\draw[thick,red,->] (XXX-1-1) -- (XXX-2-3);
\end{tikzpicture}
& 
\parbox{10cm}{ 
\BS{begin}\AC{tikzpicture} \\
\BSS{matrix} \blll{(XXX)} [matrix of nodes,column sep=.5cm,row sep=.5cm,every node/.style=draw]\\
\{ \\
1 \hspace{3mm} \& \hspace{3mm} 2 \hspace{3mm} \& \hspace{3mm} 3 \hspace{3mm} \BS{}\BS{}   \\
4 \hspace{3mm} \& \hspace{3mm} 5 \hspace{3mm} \& \hspace{3mm} 6 \hspace{3mm} \BS{}\BS{}  \\
7 \hspace{3mm} \& \hspace{3mm} 8 \hspace{3mm} \& \hspace{3mm} 9 \hspace{3mm} \BS{}\BS{} \\
\}; \\
\BS{draw}[thick,red,->] \blll{(XXX-1-1)} - - \blll{(XXX-2-3)} ; \\
\BS{end}\AC{tikzpicture}
}
\\ \hline  
\end{tabular} 

\bigskip


\begin{tabular}{|c|c|} \hline  
\begin{tikzpicture}
\matrix [matrix of nodes,column sep=.5cm,row sep=.5cm,every node/.style=draw]
{
8 & 1 & 6 \\
3 & 5 & |[red]| 7 \\
4 & 9 & 2 \\
};
\end{tikzpicture}
&  
\begin{tikzpicture}
\matrix [matrix of nodes]
{
1 & \& &  2 & \& &  3 				& \BS{}\BS{} \\
4 & \& & 5 	& \& & \rouge{ $|[$red$]|$} 6 & \BS{}\BS{} \\
7 & \& & 8 	& \& & 9 				& \BS{}\BS{} \\
};
\end{tikzpicture}
\\ \hline 
\end{tabular}  


\bigskip

\begin{tabular}{|c|c|} \hline 
\begin{tikzpicture}[baseline=-1cm] 
\matrix [matrix of nodes,column sep=.5cm,row sep=.5cm,every node/.style=draw]
{
AAA 			& |[circle]| BBB \\
CCC & |(d) [isosceles triangle]| DDD \\
| [ellipse]| EEE &  FFF \\
};
\end{tikzpicture}
& 
\begin{tikzpicture}
\matrix [matrix of nodes]
{
AAA & \& & \rouge{ $|[$circle$]|$} BBB &  \BS{}\BS{} \\
CCC & \& &\rouge{ $|[$isosceles triangle$]|$} DDD 	&  \BS{}\BS{} \\
\rouge{ $|[$ellipse$]|$} EEE & \& & FFF & \BS{}\BS{} \\
};
\end{tikzpicture}
\\ \hline 
\end{tabular} 


\bigskip

\begin{tabular}{|c|c|} \hline 
\begin{tikzpicture}[baseline=-2cm] 
\matrix [matrix of nodes,column sep=.5cm,row sep=.5cm,every node/.style=draw]
{
|(a)| AAA 	& |(b)| BBB \\
|(c)| CCC 	& |(d)| DDD \\
|(e)| EEE 	& |(f)| FFF \\
};
\draw (a) -- (d);
\draw (d) -- (f);
\end{tikzpicture}
&  
\begin{tikzpicture}
\node at (0,1.5) [text width=10cm]
{\BS{matrix} [matrix of nodes,column sep=.5cm,row sep=.5cm,every node/.style=draw] \\
\{ 
};
\matrix [matrix of nodes]
{
\rouge{ $|$(a)$|$} AAA & \& & \rouge{ $|$(b)$|$} BBB &  \BS{}\BS{} \\
\rouge{ $|$(c)$|$} CCC & \& & \rouge{ $|$(d)$|$} DDD 	&  \BS{}\BS{} \\
\rouge{ $|$(e)$|$} EEE & \& & \rouge{ $|$(f)$|$} FFF & \BS{}\BS{} \\
};

\node at (0,-1.2) [text width=10cm]
{  \}; \\ 
\BS{draw} (a) - - (d); \\ \BS{draw} (d) - - (f);
};
\end{tikzpicture}
\\ \hline 
\end{tabular} 

\bigskip


\begin{tabular}{|c|c|} \hline  
\begin{tikzpicture}
\matrix [matrix of nodes]
{
1 &[1cm] 2 &[5mm] |[red]| 3 \\
4 & 5 &  6 \\
7 & 8 & 9 \\
};
\end{tikzpicture}
&
\begin{tikzpicture}
\matrix [matrix of nodes]
{
1 & \& & \rouge{\lbrack 1cm \rbrack} 2 & \& &\rouge{\lbrack 5mm \rbrack} |[red]| 3 & \BS{}\BS{} \\
4 & \& & 5 & \& & 6 & \BS{}\BS{} \\
7 & \& & 8 & \& & 9 & \BS{}\BS{} \\
};
\end{tikzpicture}

\\ \hline 
\end{tabular} 



\bigskip

\begin{tabular}{|c|c|} \hline  
\begin{tikzpicture}[baseline=0pt]
\matrix [matrix of math nodes]
{
A_1 & A_2 & A_3 \\
a_4 & a_5 &  a_6 \\
a^7 & a^8 & a^9 \\
};
\end{tikzpicture}
&  
\parbox{8cm}{ 
\BSS{matrix}  [\rouge{ matrix of math nodes}]\\
\{ \\
A\_1 \hspace{2mm}  \& \hspace{2mm}  A\_2 \hspace{2mm}  \& \hspace{2mm}  A\_3 \hspace{2mm}   \BS{}\BS{}   \\
a\_4 \hspace{2mm}  \& \hspace{2mm}  a\_5 \hspace{2mm}  \&  \hspace{2mm}  a\_6 \hspace{2mm}  \BS{}\BS{}  \\
a\land 7 \hspace{2mm}  \& \hspace{2mm}  a\land 8 \hspace{2mm}  \& \hspace{2mm}  a\land 9 \hspace{2mm}   \BS{}\BS{} \\
\}; 
}
\\ \hline  
\end{tabular} 

\bigskip

\begin{tabular}{|c|c|} \hline  
\begin{tikzpicture}[baseline=0pt]
\matrix [matrix of math nodes,nodes={circle,draw}]
{
a_1 & & a_3 \\
a_4 & & a_6 \\
a_7 & a_8 & \\
};
\end{tikzpicture}
&  
\parbox{10cm}{ 
\BSS{matrix}  [matrix of math nodes,\rouge{nodes={circle,draw}}]\\
\{ \\
A\_1 \hspace{2mm}  \& \hspace{12mm}  \& \hspace{2mm}  A\_3 \hspace{2mm}   \BS{}\BS{}   \\
a\_4 \hspace{2mm}  \& \hspace{12mm}  \& \hspace{2mm}   a\_6 \hspace{2mm}  \BS{}\BS{}  \\
a\_ 7 \hspace{2mm}  \& \hspace{2mm}  a\_ 8 \hspace{2mm}  \& \hspace{12mm}    \BS{}\BS{} \\
\}; 
}
\\ \hline 
\end{tabular} 

\bigskip

\begin{tabular}{|c|c|} \hline  
\begin{tikzpicture}[baseline=0pt]
\matrix [matrix of math nodes,nodes={circle,draw},nodes in empty cells]
{
a_1 & & a_3 \\
a_4 & & a_6 \\
a_7 & a_8 & \\
};
\end{tikzpicture}
&  
\parbox{10cm}{ 
\BSS{matrix}  [matrix of math nodes,nodes={circle,draw} ,\rouge{nodes in empty cells}]\\
\{ \\
A\_1 \hspace{2mm}  \& \hspace{12mm}  \& \hspace{2mm}  A\_3 \hspace{2mm}   \BS{}\BS{}   \\
a\_4 \hspace{2mm}  \& \hspace{12mm}  \& \hspace{2mm}   a\_6 \hspace{2mm}  \BS{}\BS{}  \\
a\_ 7 \hspace{2mm}  \& \hspace{2mm}  a\_ 8 \hspace{2mm}  \& \hspace{12mm}    \BS{}\BS{} \\
\}; 
}
\\ \hline 
\end{tabular} 

\SbSbSSCT{Texte dans les n\oe uds}{Characters in Matrices of Nodes}

\begin{center}
\RRR{57-2}
\end{center}


\begin{tabular}{|c|c|} \hline  
\begin{tikzpicture}[baseline=0pt]
\matrix [matrix of nodes,nodes={text width=2cm,draw}]
{
aaa  & bbb \\ 
ccc \\
eee & fff\\
};
\end{tikzpicture}
&  
\parbox{10cm}{ 
\BSS{matrix}  [matrix of nodes,\rouge{nodes=\AC{text width=2cm,draw}} ]\\
\{ \\
aaa \&  bbb \BS{}\BS{}  \\
ccc \BS{}\BS{}  \\
eee \& fff \BS{}\BS{}  \\
\}; 
}
\\ \hline 
\end{tabular} 

\bigskip

\begin{tabular}{|c|c|}  \hline  
\begin{tikzpicture}[baseline=0cm]
\matrix [matrix of nodes,nodes={text width=2cm,draw}]
{
1 & {aaa \\ bbb \\ ccc } \\
2 & ddd \\
};
\end{tikzpicture}
&  
\parbox{10cm}{ 
\BSS{matrix}  [matrix of nodes,nodes=\AC{text width=2cm,draw} ]\\
\{ \\
1 \& \& \rouge { \AC{aaa \BS{}\BS{} bbb \BS{}\BS{} ccc } } \BS{}\BS{}   \\
2 \& \& ddd \BS{}\BS{}  \\
\}; 
}
\\ \hline 
\end{tabular} 

\bigskip

\SbSbSSCT{Délimiteurs}{Delimiters}


\begin{center}
\RRR{57-3}
\end{center}

\bigskip

\begin{tabular}{|c|c|c|c|} \hline 
\multicolumn{4}{|c|}{\BS{matrix} [matrix of math nodes,\RDD{left delimiter}=( ]}
\\ \hline  
\begin{tikzpicture}
\matrix [matrix of math nodes,left delimiter=( ]
{
a_1 & a_2 & a_3 \\
a_4 & a_5 & a_6 \\
a_7 & a_8 & a_9 \\
};
\end{tikzpicture}
&  
\begin{tikzpicture}
\matrix [matrix of math nodes,right delimiter=\}]
{
a_1 & a_2 & a_3 \\
a_4 & a_5 & a_6 \\
a_7 & a_8 & a_9 \\
};
\end{tikzpicture}
&
\begin{tikzpicture}
\matrix [matrix of math nodes,above delimiter=\| ]
{
a_1 & a_2 & a_3 \\
a_4 & a_5 & a_6 \\
a_7 & a_8 & a_9 \\
};
\end{tikzpicture}
&
\begin{tikzpicture}
\matrix [matrix of math nodes,below delimiter=\rmoustache ]
{
a_1 & a_2 & a_3 \\
a_4 & a_5 & a_6 \\
a_7 & a_8 & a_9 \\
};
\end{tikzpicture}


\\  \hline 
\RDD{left delimiter}=(  & \RDD{right delimiter}=\BS{\}} & \RDD{above delimiter}=\BS{|} & \RDD{below delimiter}=\BS{rmoustache}
\\  \hline
\end{tabular} 

\bigskip
\begin{tabular}{|c|} \hline  
\BS{tikz}
\BS{node} [fill=red!20,text width=2cm,\rouge{left delimiter}=\BS{\{} ] \\
\AC{Ceci est une démonstration d'un texte  sur une largeur de 2cm.};
\\ \hline  
\tikz
\node [fill=red!20,text width=2cm,left delimiter=\{]
{Ceci est une démonstration d'un texte  sur une largeur de 2cm.};
\\ \hline 
\end{tabular} 



%% 
%%\newpage 
%%
%%\SbSSCT{Chaine de n\oe uds}{Chains of nodes}
%%
%%\SbSbSSCT{Création d'une chaine de n\oe euds}{Starting and Continuing a Chain}

 \maboite{\BS{usetikzlibrary}\AC{chains}}
\label{lib-chains}


\begin{center}
\RRR{46-2}
\end{center}

\bigskip

\begin{tabular}{|l|} \hline  
\BS{begin}\AC{tikzpicture}[\RDD{start chain}] \\
\BS{node} [\RDD{on chain}] \AC{A};\\
\BS{node}  [\RDD{on chain}] \AC{B};\\
\BS{node}  [\RDD{on chain}] \AC{C};\\
\BS{end}\AC{tikzpicture} \\ \hline  
\begin{tikzpicture}[start chain]
\node [on chain] {A};
\node [on chain] {B};
\node [on chain] {C};
\end{tikzpicture}
\\ \hline 
\end{tabular} 

\bigskip

\begin{tabular}{|c|}  \hline  
\BS{begin}\AC{tikzpicture}[start chain, \RDD{node distance}= 0.5 cm] 
\\ \hline  
\begin{tikzpicture}[start chain, node distance= .5 cm]
\node [on chain] {A};
\node [on chain] {B};
\node [on chain] {C};
\end{tikzpicture}
\\ \hline 
\end{tabular} 

\bigskip

\begin{tabular}{|c|}  \hline 
\BS{begin}\AC{tikzpicture}[start chain=\rouge {going below} ]
\\   \hline 
\begin{tikzpicture}[start chain=going below]
\node [on chain] {A};
\node [on chain] {B};
\node [on chain] {C};
\end{tikzpicture}
\\   \hline 
\end{tabular} 

\bigskip

\begin{tabular}{|c|}  \hline 
\BS{begin}\AC{tikzpicture}[start chain=\rouge {going left} ] 
\\   \hline 
\rule[0cm]{0pt}{.7cm}  
\begin{tikzpicture}[start chain=going left]
\node [on chain] {A};
\node [on chain] {B};
\node [on chain] {C};
\end{tikzpicture} 
\\ \hline 
\end{tabular} 


\bigskip

\begin{tabular}{|c|}  \hline  
\BS{begin}\AC{tikzpicture}[start chain, \rouge{every node/.style=draw} ] 
\\ \hline 
\rule[0cm]{0pt}{.7cm}  
\begin{tikzpicture}[start chain, every node/.style=draw]
\node [on chain] {A};
\node [on chain] {B};
\node [on chain] {C};
\end{tikzpicture}
\\ \hline 
\end{tabular} 

\bigskip

\begin{tabular}{|c|c|}\hline
\begin{tikzpicture}[start chain=1 going right,
start chain=2 going left]
\node [draw,on chain=1] {A};
\node [draw,on chain=1] {B};
\node [draw,on chain=1] {C};
\node [draw,on chain=2] at (3,1) {0};
\node [draw,on chain=2] {1};
\node [draw,on chain=2] {2};
\node [draw,on chain=1] {D};
\end{tikzpicture} 
 &  
\parbox{10cm}{
\BS{begin}\AC{tikzpicture}[\rouge{start chain=1} going right , \\
\blll{start chain=2} going left] \\
\BS{node} [draw,\rouge{on chain=1}] \AC{A}; \\
\BS{node} [draw,\rouge{on chain=1}] \AC{B}; \\
\BS{node}[draw,\rouge{on chain=1}] \AC{C}; \\
\BS{node} [draw,\blll{on chain=2}] at (3,1) \AC{0}; \\
\BS{node} [draw,\blll{on chain=2}] \AC{1}; \\
\BS{node} [draw,\blll{on chain=2}] \AC{2}; \\
\BS{node}[draw,\rouge{on chain=1}] \AC{D}; \\
\BS{end}\AC{tikzpicture}} 
\\ \hline 
\end{tabular} 

\bigskip


\begin{tabular}{|c|c|} \hline  
\rule[-2cm]{0pt}{4cm} 
\begin{tikzpicture}[start chain=going right,baseline=-1.5cm]
\node [draw,on chain] {A};
\node [draw,on chain] {B};
\node [draw,continue chain=going below,on chain] {C};
\node [draw,on chain] {D};
\node [draw,continue chain=going right,on chain] {E};
\end{tikzpicture}
&  
\parbox{11cm}{
\BS{begin}\AC{tikzpicture}[start chain going right]
\BS{node} [draw,on chain] \AC{A}; \\
\BS{node} [draw,on chain] \AC{B}; \\
\BS{node} [draw,\RDD{continue chain}=going below,on chain] \AC{C}; \\
\BS{node}[draw,on chain] \AC{D}; \\
\BS{node} [draw,\RDD{continue chain}=going right,on chain] \AC{E}; \\
\BS{end}\AC{tikzpicture}} 
\\ \hline 
\end{tabular} 

\bigskip

\begin{tabular}{|c|c|}  \hline 
\begin{tikzpicture}[every node/.style=draw,baseline=-1.5cm]
{ [start chain=1]
\node [on chain] {A};
\node [on chain] {B};
\node [on chain] {C};
}
{ [start chain=2 going below]
\node [on chain=2] at (0.5,-.5) {0};
\node [on chain=2] {1};
\node [on chain=2] {2};
}
{ [continue chain=1]
\node [on chain] {D};
}
\end{tikzpicture}
&  
\parbox{10cm}{
\BS{begin}\AC{tikzpicture}[start chain going right] \\
\{ [\RDD{start chain}=1] \\
\BS{node} [draw,on chain] \AC{A}; \\
\BS{node} [draw,on chain] \AC{B}; \\
\BS{node} [draw,on chain] \AC{C}; \\
\} \\
\{ [\RDD{start chain}=2] \\
\BS{node}[draw,on chain=2] \AC{0}; \\
\BS{node}[draw,on chain=2] \AC{1}; \\
\BS{node}[draw,on chain=2] \AC{2}; \\
\} \\
\{ [\RDD{continue chain}=1] \\
\BS{node} [draw,on chain] \AC{D}; \\
\} \\
\BS{end}\AC{tikzpicture}} 
\\  \hline 
\end{tabular} 

\bigskip

\SbSbSSCT{N\oe uds sur la chaine}{Nodes on a Chain}

\begin{center}
\RRR{46-3} 
\end{center}

\bigskip

\begin{tabular}{|c|c|} \hline 
 \begin{tikzpicture}[start chain=XXX placed  {at=(\tikzchaincount*-30+90:1.5)},baseline=0pt]
 \foreach \i in {1,...,12}
 \node [on chain] {\i};
 \draw (0,0) -- (XXX-10);
 \draw (0,0) -- (XXX-2);
 \end{tikzpicture}
&
\parbox{11cm}{
\BS{begin}\AC{tikzpicture}[start chain=\blll{XXX} \RDD{placed} \\ \AC{at=(\BSS{tikzchaincount}*-30+90:1.5)}] \\
 \BS{foreach} \BS{i} in \AC{1,...,12} \\
\BS{node} [on chain] \AC{\BS{i}}; \\
\BS{draw }(0,0) -- \blll{(XXX-10)}; \\
\BS{draw }(0,0) -- \blll{(XXX-2)}; \\
\BS{end}\AC{tikzpicture}} 
\\ \hline 
\end{tabular} 

\bigskip


\begin{tabular}{|c|c|}  \hline 
\begin{tikzpicture}[start chain,baseline=-1cm]
\node [draw,on chain] {A};
\node [draw,on chain] {B};
\node [draw,on chain=going below] {C};
\node [draw,on chain] {D};
\node [draw,on chain] {E};
\end{tikzpicture}
&  
\parbox{11cm}{
\BS{begin}\AC{tikzpicture}[start chain] \\
\BS{node} [draw,on chain] \AC{A}; \\
\BS{node} [draw,on chain] \AC{B}; \\
\BS{node} [draw,on chain=\rouge{going below}] \AC{C}; \\
\BS{node} [draw,on chain] \AC{D}; \\
\BS{node} [draw,on chain] \AC{E}; \\
\BS{end}\AC{tikzpicture}} 
\\  \hline 
\end{tabular} 


\bigskip

\begin{tabular}{|c|c|} \hline 
\begin{tikzpicture}[start chain=going {at=(\tikzchainprevious),shift=(30:1)},baseline=1cm]
\node [draw,on chain] {A};
\node [draw,on chain] {B};
\node [draw,on chain] {C};
\node [draw,on chain] {D};
\end{tikzpicture}  
&  
\parbox{11cm}{
\BS{begin}\AC{tikzpicture}[start chain=going \\ \AC{at=(\BSS{tikzchainprevious},shift=(30:1)}] \\
\BS{node} [draw,on chain] \AC{A}; \\
\BS{node} [draw,on chain] \AC{B}; \\
\BS{node} [draw,on chain] \AC{C}; \\
\BS{node} [draw,on chain] \AC{D}; \\
\BS{end}\AC{tikzpicture}} 
\\ \hline 
\end{tabular} 

\bigskip

\begin{tabular}{|c|c|} \hline 
\begin{tikzpicture}[baseline=1cm]
\node[draw,red] (A) at (0,2) {A};
{ [start chain]
\node [draw,on chain] {B};
\node [draw,on chain] {C};
\chainin (A) [join];
\node [draw,on chain] {D};
\node [draw,on chain] {E};
}
\end{tikzpicture}
&  
\parbox{11cm}{
\BS{begin}\AC{tikzpicture} \\
\BS{node} [draw,red] (A) at (0,2)  \AC{A}; \\
\{ [start chain] \\
\BS{node} [draw,on chain] \AC{B}; \\
\BS{node} [draw,on chain] \AC{C}; \\
\BSS{chainin} (A) [join]; \\
\BS{node} [draw,on chain] \AC{D}; \\
\BS{node} [draw,on chain] \AC{E}; \\
\} \\
\BS{end}\AC{tikzpicture}} 
\\  \hline 
\end{tabular} 



\bigskip

\begin{tabular}{|c|c|} \hline 
\begin{tikzpicture}[baseline=-1cm]
\matrix [matrix of nodes,column sep=1cm,row sep=1cm,every node/.style=draw]
{
|(a) | A 	& |(b) |  B 	& |(c) | C \\
|(d) | D 	& |(e) | E 		& |(f) | F \\
};
{ [start chain,every on chain/.style={join=by ->}]
\chainin (a);
\chainin (b);
\chainin (d);
\chainin (c);
\chainin (f);
\chainin (e);
}
\end{tikzpicture}
&  
\parbox{11cm}{
\BS{begin}\AC{tikzpicture} \\
\BS{matrix} [matrix of nodes,column sep=5mm,row sep=5mm] ,every node/.style=draw \\
\{ \\
|(a) | A 	\& |(b) |  B 	\& |(c) | C \BS{}\BS{} \\
|(d) | D 	\& |(e) | E 	\& |(f) | F \BS{}\BS{} \\
\}; \\
\{ [start chain,every on chain/.style=\AC{join=by ->}] \\
\BSS{chainin} (a);
\BSS{chainin}(b);
\BSS{chainin}(d); \\
\BSS{chainin} (c);
\BSS{chainin}(f);
\BSS{chainin}(e);
\}
\BS{end}\AC{tikzpicture}
} 
\\ \hline 
\end{tabular} 

\bigskip

\SbSbSSCT{Jonction de n\oe uds}{Joining Nodes on a Chain}

\begin{center}
\RRR{46-4}
\end{center} 

\bigskip

\begin{tabular}{|c|c|} \hline 
\begin{tikzpicture}[start chain]
\node [draw,on chain] {A};
\node [draw,on chain,join] {B};
\node [draw,on chain] {C};
\node [draw,on chain,join] {D};
\end{tikzpicture}
&  
\parbox{11cm}{
\BS{begin}\AC{tikzpicture}[start chain] \\
\BS{node} [draw,on chain] \AC{A}; \\
\BS{node} [draw,on chain,\RDD{join}] \AC{B}; \\
\BS{node} [draw,on chain] \AC{C}; \\
\BS{node} [draw,on chain,\RDD{join}] \AC{D}; \\
\BS{end}\AC{tikzpicture}} 
\\ \hline 
\end{tabular} 

\bigskip

\begin{tabular}{|c|c|} \hline 
\begin{tikzpicture}[start chain, every on chain/.style=join, every join/.style=->]
\node [draw,on chain] {A};
\node [draw,on chain] {B};
\node [draw,on chain] {C};
\node [draw,on chain] {D};
\end{tikzpicture}
&  
\parbox{11cm}{
\BS{begin}\AC{tikzpicture}[start chain, \RDD{every on chain}/.style=join, \\ \RDD{every join}/.style=->] \\
\BS{node} [draw,on chain] \AC{A}; \\
\BS{node} [draw,on chain,\RDD{join}] \AC{B}; \\
\BS{node} [draw,on chain] \AC{C}; \\
\BS{node} [draw,on chain,\RDD{join}] \AC{D}; \\
\BS{end}\AC{tikzpicture}} 
\\ \hline 
\end{tabular} 

\bigskip

\begin{tabular}{|c|c|}  \hline 
\begin{tikzpicture}[start chain,baseline=-1cm]
\node [draw,on chain] {A};
\node [draw,on chain] {B};
\node [draw,on chain] {C};
\node [draw,on chain=going below,join=with chain-2 ] {D};
\end{tikzpicture} 
&  
\parbox{11cm}{
\BS{begin}\AC{tikzpicture}[start chain] \\
\BS{node} [draw,on chain] \AC{A}; \\
\BS{node} [draw,on chain] \AC{B}; \\
\BS{node} [draw,on chain] \AC{C}; \\
\BS{node} [draw,on chain=going below,\rouge{join=with chain-2} ] \AC{D}; \\
\BS{end}\AC{tikzpicture}} 
\\ \hline 
\begin{tikzpicture}[start chain,baseline=-1cm]
\node [draw,on chain] {A};
\node [draw,on chain] {B};
\node [draw,on chain] {C};
\node [draw,on chain=going below,join=with chain-1 by {blue,<-}] {D};
\end{tikzpicture}
&
\parbox{12cm}{
\BS{begin}\AC{tikzpicture}[start chain] \\
\BS{node} [draw,on chain] \AC{A}; \\
\BS{node} [draw,on chain] \AC{B}; \\
\BS{node} [draw,on chain] \AC{C}; \\
\BS{node} [draw,on chain=going below,join=with chain-1 \rouge{ by \AC{blue,<-}} ] \AC{D}; \\
\BS{end}\AC{tikzpicture}} 
\\ \hline 
\end{tabular} 



\bigskip

\SbSbSSCT{Branches}{Branches}

\begin{center}
\RRR{46-5}
\end{center} 


\bigskip

\begin{tabular}{|c|c|}  \hline 
\begin{tikzpicture} [baseline=-2cm]
{ [start chain=XXX]
\node [draw,on chain] {A};
\node [draw,on chain] {B};
{ [start branch=YYY going below]
\node [draw,on chain] {1};
\node [draw,on chain] {2};
\node [draw,on chain] {3};
}
\node [draw,on chain,join=with XXX/YYY-end,join=with XXX/YYY-2 ] {C};
}
\end{tikzpicture}
&  
\parbox{12cm}{
\BS{begin}\AC{tikzpicture}\\
\{ [start chain=\blll{XXX}] \\
\BS{node} [draw,on chain] \AC{A}; \\
\BS{node} [draw,on chain] \AC{B}; \\
\{ [\RDD{start branch}=\blll{YYY} going below] \\
\BS{node} [draw,on chain] \AC{1}; \\
\BS{node} [draw,on chain] \AC{2}; \\
\BS{node} [draw,on chain] \AC{3}; \\
\} \\
\BS{node} [ draw,on chain,join=with \blll{XXX/YYY}\rouge{-end}, \\ join=with \blll{XXX/YYY}\rouge{-2}]  \AC{C}; \\
\} \\
\BS{end}\AC{tikzpicture}   } 

\\ \hline 
\end{tabular} 

\bigskip

\begin{tabular}{|c|} \hline 
\BS{begin}\AC{tikzpicture}[ \RDD{node distance}=.2cm and 3cm]
\\ \hline 
\begin{tikzpicture}[ node distance=.2cm and 3cm]
{ [start chain=XXX]
\node [on chain] {A};
\node [on chain] {B};
{ [start branch=YYY going below]
\node [on chain] {1};
\node [on chain] {2};
\node [on chain] {3};
}
\node [on chain,join=with XXX/YYY-end] {C};
}
\end{tikzpicture}
\\ \hline 
\end{tabular} 

\bigskip

\begin{tabular}{|c|c|} \hline 
\begin{tikzpicture}[ node distance=2mm and 1cm,baseline=-2cm]
{ [start chain=XXX]
\node [draw,on chain] {A};
\node [draw,on chain] {B};
{ [start branch=YYY going below]
\node [draw,on chain] {1};
\node [draw,on chain] {2};
\node [draw,on chain] {3};
}
\node [draw,on chain,join=with XXX/YYY-end] {C};
{
[continue branch=YYY]
\node [draw,on chain] {4};
\node [draw,on chain] {5};
}
}
\end{tikzpicture}
&  
\parbox{12cm}{
\BS{begin}\AC{tikzpicture}[ node distance=2mm and 1cm]\\
\{ [start chain=\blll{XXX}] \\
\BS{node} [draw,on chain] \AC{A}; \\
\BS{node} [draw,on chain] \AC{B}; \\
\{ [start branch=\blll{YYY} going below] \\
\BS{node} [draw,on chain] \AC{1}; \\
\BS{node} [draw,on chain] \AC{2}; \\
\BS{node} [draw,on chain] \AC{3}; \} \\
\BS{node}  [draw,on chain,join=with \blll{XXX/YYY}-end]  \AC{C}; \\
\{ [\RDD{continue branch}=\blll{YYY}]\\
\BS{node} [on chain] \AC{4}; \\
\BS{node} [on chain] \AC{5}; \} \\
\} \\
\BS{end}\AC{tikzpicture}   } 
\\ \hline 
\end{tabular} 


\bigskip

\begin{tabular}{|c|c|} \hline 
\begin{tikzpicture}[node distance=2mm and 1cm, every node/.style=draw,baseline=-1cm]
{ [start chain]
\node [on chain] {1};
\node [on chain] {2};
{ [start branch=XXX going below] }
\node [on chain] {3};
{ [start branch=YYY going above] }
\node [on chain] {4};
{ [continue branch=XXX]
\node [on chain] {a};
\node [on chain] {b};
}{
[continue branch=YYY]
\node [on chain] {A};
\node [on chain] {B};
}
}
\end{tikzpicture}
&  
\parbox{12cm}{
\BS{begin}\AC{tikzpicture}[node distance=2mm and 1cm, every node/.style=draw]\\
\{ [start chain] \\
\BS{node} [on chain] \AC{1};  \\
\BS{node} [on chain] \AC{2}; \\
\{ [\RDD{start branch}=\blll{XXX} going below] \} \\
\BS{node} [on chain] \AC{3}; \\
\{ [\RDD{start branch}=\blll{YYY} going above] \} \\
\BS{node} [on chain] \AC{4}; \\
\{ [\RDD{continue branch}=\blll{XXX} ] \\
\BS{node} [on chain] \AC{a}; \\
\BS{node} [on chain] \AC{b};\} \\
\{ [\RDD{continue branch}=\blll{YYY} ] \\
\BS{node} [on chain] \AC{A}; \\
\BS{node} [on chain] \AC{B}; \}  }
\\ \hline 
\end{tabular} 



%%
%%
%%\bigskip
%
%%\begin{tikzpicture}[baseline=-1cm] 
%%\matrix [matrix of nodes,column sep=.5cm,row sep=.5cm,every node/.style=draw]
%%{
%%|(a) [red]|AAA 			& |(b) [circle]| BBB \\
%%|(c)| CCC 		& |(d) [isosceles triangle]| DDD \\
%%|(e) [ellipse]| EEE & |(f)| FFFF \\
%%};
%%\draw (a) -- (f);
%%\end{tikzpicture}
%
%
%\newpage
%
%
%
%%\section{symboles}
%%
%%
\SbSSCT{Coordonnées}{Coordinates}
\begin{center}
\RRR{13-2-1}
\end{center}


\SbSbSSCT{Système de coordonnées \og canvas \fg}{Canvas coordinates}

\noindent


\tikzset{every picture/.style=blue,very thick,inner sep=0pt}

\begin{tabular}{|c|c|} \hline 
\TFRGB{Explicite}{explicit}  & \TFRGB{Implicite}{implicit}
\\ \hline
\begin{tikzpicture}
\draw[help lines] (0,0) grid (3,2);
\fill (canvas cs:x=2cm,y=1.5cm) circle (2pt);
\end{tikzpicture}
&
\begin{tikzpicture}
\draw[help lines] (0,0) grid (3,2);
\fill (2,1.5) circle (2pt);
\end{tikzpicture}

\\ \hline  
 \BS{fill} (\RDD{canvas cs}:\blll{x=2cm,y=1.5cm}) circle (2pt);
& \BS{fill} {\color{blue}(2cm,1.5cm)} circle (2pt);
\\ \hline 
\end{tabular} 


\SbSbSSCT{Système de coordonnées polaire \og canvas \fg}{Polar coordinates}

\noindent


\begin{tabular}{|c|c|c|} \hline
\TFRGB{Explicite}{explicit}  & \TFRGB{Implicite}{implicit}
\\ \hline
\begin{tikzpicture}
\draw[help lines] (0,0) grid (3,2);
\draw [dotted](0,2) arc (90 :0 :2);
\draw [dotted](0,0) --(2,2);
\fill (canvas polar cs:angle=45,radius=2cm) circle (2pt);
\end{tikzpicture}
&
\begin{tikzpicture}
\draw[help lines] (0,0) grid (3,2);
\draw [dotted](0,2) arc (90 :0 :2);
\draw [dotted](0,0) --(2,2);
\fill (45:2cm) circle (2pt);
\end{tikzpicture}
\\ \hline 
\BS{fill} (\RDD{canvas polar cs}:\RDD{angle}=45,\RDD{radius}=2cm) circle (2pt);
&
\BS{fill} {\color{blue}(45:2cm)} circle (2pt);
\\ \hline 
\end{tabular} 

\bigskip
\begin{tabular}{|c|} \hline  
\begin{tikzpicture}
\draw[help lines] (0,0) grid (3,2);
\draw [dotted](0,2) arc (90 :0 :3 and 2);
\draw [dotted](0,0) --(3,2);
\fill (canvas polar cs:angle=45,x radius=3cm,y radius=2cm) circle (2pt);
\end{tikzpicture}
\\ \hline  
\BS{fill} (canvas polar cs:angle=45,\RDD{x radius}=3cm,\RDD{y radius}=2cm) circle (2pt);
\\ \hline 
\end{tabular}


\SbSbSSCT{Système de coordonnées  xyz}{xyz coordinates}

\noindent


\begin{tabular}{|c|c|c|} \hline 
\begin{tikzpicture}[->]
\draw (0,0) -- (xyz cs:x=1);
\draw[red] (0,0) -- (xyz cs:y=1);
\draw[magenta] (0,0) -- (xyz cs:z=1);
\end{tikzpicture}
&
\begin{tikzpicture}[->]
\draw (0,0) -- (1,0,0);
\draw[red]  (0,0) -- (0,1,0);
\draw[magenta]  (0,0) -- (0,0,1);
\end{tikzpicture}
\\ \hline 
\BS{draw} (0,0) - - (\RDD{xyz cs}:x=1); & \BS{draw}  (0,0) - - (1,0,0); \\
\BS{draw}[red]  (0,0) - - (\RDD{xyz cs}:y=1); &  \BS{draw}[red] (0,0) - - (0,1,0); \\
\BS{draw}[magenta]  (0,0) - - (\RDD{xyz cs}:z=1); &  \BS{draw}[magenta]   (0,0) - - (0,0,1); 
\\ \hline 

\end{tabular} 

 
\newpage

\SbSbSSCT{Coordinate system xyz polar}{Coordinate system xyz polar}

\noindent

\begin{tabular}{|c|c|c|} \hline
\TFRGB{Explicite}{explicit}  & \TFRGB{Implicite}{implicit}
\\ \hline
\begin{tikzpicture}
\draw[help lines] (0,0) grid (3,2);
\draw [dotted](0,2) arc (90 :0 :2);
\draw [dotted](0,0) --(2,2);
\fill (xyz polar cs:angle=45,radius=2) circle (2pt);
\end{tikzpicture}
&
\begin{tikzpicture}
\draw[help lines] (0,0) grid (3,2);
\draw [dotted](0,2) arc (90 :0 :2);
\draw [dotted](0,0) --(2,2);
\fill (45:2) circle (2pt);
\end{tikzpicture}
\\ \hline 
\BS{fill} (\RDD{xyz polar cs}:\RDD{angle}=45,\RDD{radius}=2) circle (2pt);
&
\BS{fill} {\color{blue}(45:2cm)} circle (2pt);
\\ \hline 
\end{tabular} 

\bigskip
\begin{tabular}{|c|} \hline  
\begin{tikzpicture}
\draw[help lines] (0,0) grid (3,2);
\draw [dotted](0,2) arc (90 :0 :3 and 2);
\draw [dotted](0,0) --(3,2);
\fill (xyz polar cs:angle=45,x radius=3,y radius=2) circle (2pt);
\end{tikzpicture}
\\ \hline  
\BS{fill} (xyz polar cs:angle=45,\RDD{x radius}=3,\RDD{y radius}=2) circle (2pt);
\\ \hline 
\end{tabular} 

\bigskip

\begin{tabular}{|c|c|c|} \hline
\multicolumn{2}{|c|}{\BS{begin}\AC{tikzpicture}{\color{red}[x=1.5cm,y=1cm]} }
\\ \hline
\begin{tikzpicture}[x=1.5cm,y=1cm]
\draw[help lines] (0,0) grid (3,2);
\draw [dotted](0,2) arc (90 :0 :2);
\draw [dotted](0,0) --(2,2);
\fill (xyz polar cs:angle=45,radius=2) circle (2pt);
\end{tikzpicture}
&
\begin{tikzpicture}[x=1.5cm,y=1cm]
\draw[help lines] (0,0) grid (3,2);
\draw [dotted](0,2) arc (90 :0 :2);
\draw [dotted](0,0) --(2,2);
\fill (45:2) circle (2pt);
\end{tikzpicture}
\\ \hline 
\BS{fill} (\RDD{xyz polar cs}:\RDD{angle}=45,\RDD{radius}=2) circle (2pt);
&
\BS{fill} {\color{blue}(45:2cm)} circle (2pt);
\\ \hline 
\end{tabular} 
\bigskip

\begin{tabular}{|c|c|c|} \hline
\multicolumn{2}{|c|}{\BS{begin}\AC{tikzpicture}{\color{red}[x=\AC{(0cm,1cm)},y=\AC{(-1cm,0cm)}]} }
\\ \hline
\begin{tikzpicture}[x={(0cm,1cm)},y={(-1cm,0cm)}]
\draw[help lines] (0,0) grid (3,2);
\draw [dotted](0,2) arc (90 :0 :2);
\draw [dotted](0,0) --(2,2);
\fill (xyz polar cs:angle=45,radius=2) circle (2pt);
\end{tikzpicture}
&
\begin{tikzpicture}[x={(0cm,1cm)},y={(-1cm,0cm)}]
\draw[help lines] (0,0) grid (3,2);
\draw [dotted](0,2) arc (90 :0 :2);
\draw [dotted](0,0) --(2,2);
\fill (45:2) circle (2pt);
\end{tikzpicture}
\\ \hline 
\BS{fill} (\RDD{xyz polar cs}:\RDD{angle}=45,\RDD{radius}=2) circle (2pt);
&
\BS{fill} {\color{blue}(45:2cm)} circle (2pt);
\\ \hline 
\end{tabular} 

\SbSbSSCT{Coordonnées barycentriques}{Barycentric coordinates}

\begin{center}
\RRR{13-2-2}
\end{center}

\begin{tabular}{|c|c|c|} \hline
\multicolumn{3}{|c|}{  \BS{node} [circle,fill=red!20] at (\RDD{barycentric cs}:A=0.6,B=0.3 ) \AC{X};   }\\ 
\hline
\begin{tikzpicture}[scale=.6]
\draw[help lines] (0,0) grid (4,4);
\node[circle,fill=green!20,] (A) at (0,0) {A};
\node[circle,fill=green!20,] (B) at (4,0) {B};
\node[circle,fill=red!20] at (barycentric cs:A=0.3,B=0.3 ) {X};
\end{tikzpicture}
&
\begin{tikzpicture}[scale=.6]
\draw[help lines] (0,0) grid (4,4);
\node[circle,fill=green!20,] (A) at (0,0) {A};
\node[circle,fill=green!20,] (B) at (4,0) {B};
\node[circle,fill=green!20,] (C) at (4,4) {C};
\node[circle,fill=red!20] at (barycentric cs:A=0.4,B=0.4 ,C=.4) {X};
\end{tikzpicture}
&
\begin{tikzpicture}[scale=.6]
\draw[help lines] (0,0) grid (4,4);
\node[circle,fill=green!20,] (A) at (0,0) {A};
\node[circle,fill=green!20,] (B) at (4,0) {B};
\node[circle,fill=green!20,] (C) at (1,4) {C};
\node[circle,fill=green!20,] (D) at (4,4) {D};
\node[circle,fill=red!20] at (barycentric cs:A=0.5,B=0.5,C=.5,D=.5 ) {X};
\end{tikzpicture}
\\ \hline
A=0.3,B=0.3 & A=0.4,B=0.4 ,C=.4 & A=0.5,B=0.5,C=.5,D=.5 
\\ \hline
\begin{tikzpicture}[scale=.6]
\draw[help lines] (0,0) grid (4,4);
\node[circle,fill=green!20,] (A) at (0,0) {A};
\node[circle,fill=green!20,] (B) at (4,0) {B};
\node[circle,fill=red!20] at (barycentric cs:A=0.6,B=0.3 ) {X};
\end{tikzpicture}
&
\begin{tikzpicture}[scale=.6]
\draw[help lines] (0,0) grid (4,4);
\node[circle,fill=green!20,] (A) at (0,0) {A};
\node[circle,fill=green!20,] (B) at (4,0) {B};
\node[circle,fill=green!20,] (C) at (4,4) {C};
\node[circle,fill=red!20] at (barycentric cs:A=0.2,B=0.4 ,C=.6) {X};
\end{tikzpicture}
&
\begin{tikzpicture}[scale=.6]
\draw[help lines] (0,0) grid (4,4);
\node[circle,fill=green!20,] (A) at (0,0) {A};
\node[circle,fill=green!20,] (B) at (4,0) {B};
\node[circle,fill=green!20,] (C) at (1,4) {C};
\node[circle,fill=green!20,] (D) at (4,4) {D};
\node[circle,fill=red!20] at (barycentric cs:A=0.2,B=0.4,C=.6,D=.8 ) {X};
\end{tikzpicture}
\\ \hline
A=0.6,B=0.3 & A=0.2,B=0.4 ,C=.6 & A=0.2,B=0.4,C=.6,D=.8
\\ \hline
\end{tabular}

\SbSbSSCT{Coordonnées nominatives : n\oe ud}{Named coordinates: nodes}

\begin{center}
\RRR{13-2-3}
\end{center}

\begin{tabular}{|c|c|} \hline  
\begin{tikzpicture}[blue,very thick,baseline=1cm]
\draw[help lines] (0,0) grid (3,3);
\coordinate (centre) at (1.5,1.5) ;
\coordinate (A) at (.5,.5) ;
\coordinate (B) at (2.5,2.5) ;
\fill (centre) circle (3pt);
\draw[red] (A) rectangle (B) ;
\end{tikzpicture}
&  
\parbox[c]{8cm}{
\BSS{coordinate} {\color{blue}(centre)} at(1.5,1.5) ; \\
\BSS{coordinate} {\color{blue}(A)} at (.5,.5) ;\\
\BSS{coordinate} {\color{blue}(B)} at  (2.5,2.5) ;\\
\\
\BS{fill} {\color{blue}(centre)} circle (3pt);\\
\BS{draw}[red] {\color{blue}(A)} rectangle {\color{blue}(B)} ;\\
}
\\ \hline 
\end{tabular} 


\TFRGB{voir aussi}{see also} page \pageref{noeuds}


\SbSbSSCT{Coordonnées relatives à un noeud}{Coordinates relative to a node}

\noindent

\begin{tabular}{|c|c|c|c|} \hline
\multicolumn{4}{|l|}{  \BS{node} [draw,fill=green!20,] (A) at (1,1) \AC{\BS{huge}  noeud}; }\\ 
\multicolumn{4}{|l|}{  \BS{fill}[red] (\RDD{node cs}:\RDD{name}=A,\RDD{anchor}=south) circle (3pt);   }\\ 
\hline

\begin{tikzpicture}
\draw[help lines] (0,0) grid (2,2);
\node[draw,fill=green!20,] (A) at (1,1) {\huge noeud};
\fill[red] (node cs:name=A,anchor=south) circle (3pt);
\end{tikzpicture}
&
\begin{tikzpicture}
\draw[help lines] (0,0) grid (2,2);
\node[draw,fill=green!20,] (A) at (1,1) {\huge noeud};
\fill[red] (node cs:name=A,anchor=west) circle (3pt);
\end{tikzpicture}
&
\begin{tikzpicture}
\draw[help lines] (0,0) grid (2,2);
\node[draw,fill=green!20,] (A) at (1,1) {\huge noeud};
\fill[red] (node cs:name=A,anchor=north) circle (3pt);
\end{tikzpicture}
&
\begin{tikzpicture}
\draw[help lines] (0,0) grid (2,2);
\node[draw,fill=green!20,] (A) at (1,1) {\huge noeud};
\fill[red] (node cs:name=A,anchor=east) circle (3pt);
\end{tikzpicture}
\\ \hline
name=A,anchor=south & name=A,anchor=west & name=A,anchor=north & name=A,anchor=east
\\ \hline
\end{tabular}

\bigskip

\begin{tabular}{|c|c|c|c|} \hline
\multicolumn{4}{|l|}{  \BS{node} [draw,fill=green!20,] \blll{(A)} at (1,1) \AC{\BS{huge}  noeud}; }\\ 
\multicolumn{4}{|l|}{  \BS{fill}[red] (\blll{A}.south) circle (3pt);   }\\ 
\hline

\begin{tikzpicture}
\draw[help lines] (0,0) grid (2,2);
\node[draw,fill=green!20,] (A) at (1,1) {\huge noeud};
\fill[red] (A.south) circle (3pt);
\end{tikzpicture}
&
\begin{tikzpicture}
\draw[help lines] (0,0) grid (2,2);
\node[draw,fill=green!20,] (A) at (1,1) {\huge noeud};
\fill[red] (A.west) circle (3pt);
\end{tikzpicture}
&
\begin{tikzpicture}
\draw[help lines] (0,0) grid (2,2);
\node[draw,fill=green!20,] (A) at (1,1) {\huge noeud};
\fill[red] (A.north) circle (3pt);
\end{tikzpicture}
&
\begin{tikzpicture}
\draw[help lines] (0,0) grid (2,2);
\node[draw,fill=green!20,] (A) at (1,1) {\huge noeud};
\fill[red] (A.east) circle (3pt);
\end{tikzpicture}
\\ \hline
A.south & A.west & A.north & A.east
\\ \hline
\end{tabular}



\bigskip
\begin{tabular}{|c|c|c|c|} \hline
\multicolumn{4}{|c|}{  \BS{fill}[red] (node cs:\RDD{name}=A,\RDD{angle}=0) circle (3pt);  }\\ 
\hline

\begin{tikzpicture}
\draw[help lines] (0,0) grid (2,2);
\node[draw,fill=green!20,] (A) at (1,1) {\huge noeud};
\fill[red] (node cs:name=A,angle=0) circle (3pt);
\end{tikzpicture}
&
\begin{tikzpicture}
\draw[help lines] (0,0) grid (2,2);
\node[draw,fill=green!20,] (A) at (1,1) {\huge noeud};
\fill[red] (node cs:name=A,angle=-30) circle (3pt);
\end{tikzpicture}
&
\begin{tikzpicture}
\draw[help lines] (0,0) grid (2,2);
\node[draw,fill=green!20,] (A) at (1,1) {\huge noeud};
\fill[red] (node cs:name=A,angle=-90) circle (3pt);
\end{tikzpicture}
&
\begin{tikzpicture}
\draw[help lines] (0,0) grid (2,2);
\node[draw,fill=green!20,] (A) at (1,1) {\huge noeud};
\fill[red] (node cs:name=A,angle=-150) circle (3pt);
\end{tikzpicture}
\\ \hline
name=A,angle=0 & name=A,angle=-30 & nname=A,angle=-90 & name=A,angle=-150
\\ \hline
\end{tabular}

\bigskip


\begin{tabular}{|c|c|c|c|} \hline
\multicolumn{4}{|c|}{  \BS{fill}[red] (A.0) circle (3pt);  }\\ 
\hline

\begin{tikzpicture}
\draw[help lines] (0,0) grid (2,2);
\node[draw,fill=green!20,] (A) at (1,1) {\huge noeud};
\fill[red] (A.0) circle (3pt);
\end{tikzpicture}
&
\begin{tikzpicture}
\draw[help lines] (0,0) grid (2,2);
\node[draw,fill=green!20,] (A) at (1,1) {\huge noeud};
\fill[red] (A.-30) circle (3pt);
\end{tikzpicture}
&
\begin{tikzpicture}
\draw[help lines] (0,0) grid (2,2);
\node[draw,fill=green!20,] (A) at (1,1) {\huge noeud};
\fill[red] (A.-90) circle (3pt);
\end{tikzpicture}
&
\begin{tikzpicture}
\draw[help lines] (0,0) grid (2,2);
\node[draw,fill=green!20,] (A) at (1,1) {\huge noeud};
\fill[red] (A.-150) circle (3pt);
\end{tikzpicture}
\\ \hline
A.0 & A.-30 & A.-90 & A.-150
\\ \hline
\end{tabular}

\TFRGB{voir aussi}{see also} page \pageref{nomnoeud}


\newpage

\SbSbSSCT{Coordonnées relatives à deux points}{Coordinates relative to two points}
\begin{center}
\RRR{13-3-1}
\end{center}

\begin{tabular}{|c|c|} \hline
\multicolumn{2}{|c|}{  \BS{node} [circle,fill=red!20] at (1,1 {\color{red}|-} 3,3) \AC{X}   }\\ 
\hline
\begin{tikzpicture}
\draw[help lines] (0,0) grid (4,4);
\node[circle,fill=green!20,] (A) at (1,1) {A};
\node[circle,fill=green!20,] (B) at (3,3) {B};
\node[circle,fill=red!20] at (1,1 |- 3,3) {X};
\end{tikzpicture}
&
\begin{tikzpicture}
\draw[help lines] (0,0) grid (4,4);
\node[circle,fill=green!20,] (A) at (1,1) {A};
\node[circle,fill=green!20,] (B) at (3,3) {B};
\node[circle,fill=red!20] at (1,1 -| 3,3) {X};
\end{tikzpicture}
\\ \hline
at (1,1 {\color{red}|-} 3,3)
&
at (1,1 {\color{red}-|} 3,3)
\\ \hline
\end{tabular}



\SbSbSSCT{Coordonnée relative à une intersection}{Coordinates relative to an intersection}
\begin{center}
\RRR{13-3-2}
\end{center}

 \maboite{\BS{usetikzlibrary}\AC{intersections}}
\label{lib-intersections}


\begin{tabular}{|c|c|c|c|} \hline 
\multicolumn{4}{|l|}{  \BS{draw} [\RDD{name path}=XXX] (2,1) circle  (1cm);   }\\ 
\multicolumn{4}{|l|}{  \BS{draw} [\RDD{name path}=YYY] (0.5,0.5) rectangle +(3,1);   }\\ 
\multicolumn{4}{|l|}{ \BS{fill} [red,\RDD{ name intersections}=\AC{of=xxx and YYY}]
(\RDD{intersection}-1) circle (2pt)   }\\ 
\hline 
\begin{tikzpicture}[scale=.8]
\draw [help lines] grid (4,2);
\draw [name path=XXX] (2,1) circle  (1cm);
\draw [name path=YYY] (0.5,0.5) rectangle +(3,1);
\fill [red, name intersections={of=XXX and YYY}]
(intersection-1) circle (2pt)  ;
\end{tikzpicture}
& 
\begin{tikzpicture}[scale=.8]
\draw [help lines] grid (4,2);
\draw [name path=XXX] (2,1) circle  (1cm);
\draw [name path=YYY] (0.5,0.5) rectangle +(3,1);
\fill [red, name intersections={of=XXX and YYY}] (intersection-2) circle (2pt) ;
\end{tikzpicture} 
&  
\begin{tikzpicture}[scale=.8]
\draw [help lines] grid (4,2);
\draw [name path=XXX] (2,1) circle  (1cm);
\draw [name path=YYY] (0.5,0.5) rectangle +(3,1);
\fill [red, name intersections={of=XXX and YYY}] (intersection-3) circle (2pt) ;
\end{tikzpicture}
&  
\begin{tikzpicture}[scale=.8]
\draw [help lines] grid (4,2);
\draw [name path=XXX] (2,1) circle  (1cm);
\draw [name path=YYY] (0.5,0.5) rectangle +(3,1);
\fill [red, name intersections={of=XXX and YYY}] (intersection-4) circle (2pt) ;
\end{tikzpicture}
\\ 
\hline intersection-1 & intersection-2 &intersection-3  & intersection-4 \\ 
\hline 
\end{tabular} 

\bigskip

\begin{tabular}{|c|} \hline  
\BS{fill} [red, name intersections=\AC{of=XXX and YYY}] \\
(intersection-1) circle (2pt) {\color{red} node[black,above right] \AC{point a}} ;
\\ \hline  
\begin{tikzpicture}
\draw [help lines] grid (4,2);
\draw [name path=XXX] (2,1) circle  (1cm);
\draw [name path=YYY] (0.5,0.5) rectangle +(3,1);
\fill [red, name intersections={of=XXX and YYY}]
(intersection-1) circle (2pt) node[black,above right] {point a} ;
\end{tikzpicture} 
\\ \hline 
\end{tabular} 

\bigskip

\begin{tabular}{|c|} \hline 
\BS{fill} [red, name intersections=\AC{of=XXX and YYY, \RDD{name}=ZZZ}]; \\
\BS{draw} [red] (ZZZ-1) - - (ZZZ-3); \BS{draw} [green] (ZZZ-2) - - (ZZZ-4);
\\ \hline  
\begin{tikzpicture}
\draw [help lines] grid (4,2);
\draw [name path=XXX] (2,1) circle  (1cm);
\draw [name path=YYY] (0.5,0.5) rectangle +(3,1);
\fill [red, name intersections={of=XXX and YYY, name=ZZZ}];
\draw [red] (ZZZ-1) -- (ZZZ-3);
\draw [green] (ZZZ-2) -- (ZZZ-4);
\end{tikzpicture}
\\ \hline 
\end{tabular} 

\bigskip
\begin{tabular}{|c|} \hline  
\BS{fill} [red, name intersections=\AC{of=XXX and YYY , \RDD{by}=\AC{a,b,c,d}}]; \\
\BS{draw} [red] (a) - - (c); \hspace{1cm} \BS{draw} [green] (b) - - (d);
\\ \hline   
\begin{tikzpicture}
\draw [help lines] grid (4,2);
\draw [name path=XXX] (2,1) circle  (1cm);
\draw [name path=YYY] (0.5,0.5) rectangle +(3,1);
\fill [red, name intersections={of=XXX and YYY, by={a,b,c,d}}];
\draw [red] (a) -- (c);
\draw [green] (b) -- (d);
\end{tikzpicture}
\\ \hline 
\end{tabular} 

\bigskip

\begin{tabular}{|c|} \hline  
\BS{fill} [name intersections=\AC{of=XXX and YYY, name=i, \RDD{total}=\BS{t}}] [red] \\
\BS{foreach} \BS{s} in \AC{1,...,\BS{t}} \AC{(i-\BS{s}) circle (2pt) node[black,above right] \AC{\BS{s}}}
\\ \hline  
\begin{tikzpicture}
\draw [help lines] grid (4,2);
\draw [name path=XXX] (2,1) circle  (1cm);
\draw [name path=YYY] (0.5,0.5) rectangle +(3,1);
\fill [name intersections={of=XXX and YYY , name=i, total=\t}]
[red]
\foreach \s in {1,...,\t}{(i-\s) circle (2pt) node[black,above right] {\s}};
\end{tikzpicture}
\\ \hline 
\end{tabular} 



\newpage

\SbSbSSCT{Position calculée avec le module  \og  pgfmath \fg}{Calculated positions with  \og  pgfmath \fg }

\begin{center}
\RRR{13-2-1}
\end{center}

\TFRGB{Ce module est chargé automatiquement avec le module Tikz}{Package automatically loaded with Tikz} 

\begin{tabular}{|c|} \hline 
\begin{tikzpicture}
\draw[help lines] (0,0) grid (4,2);
\fill [red] (canvas cs:x=2cm+1.5cm,y=1.5cm-1cm) circle (3pt);
\fill [blue] (2cm,1.5cm) circle (3pt);
\draw[dashed] (2,1.5) -| (3.5,.5);
\end{tikzpicture}
\\ \hline 
\emph{\TFRGB{Explicite}{explicit}} 
 : \BS{fill} [red] (\RDD{canvas cs}:x=2cm+1.5cm,y=1.5cm-1cm) circle (3pt);
 \\  \hline 
\emph{\TFRGB{Implicite}{implicit}} :  \BS{fill} [red] {\color{red}(2cm+1.5cm,1.5cm-1cm)} circle (3pt);
\\ \hline 
\end{tabular} 

\bigskip
\begin{tabular}{|c|c|c|} \hline 
\begin{tikzpicture}[baseline=0pt]
\draw[help lines] (0,0) grid (4,4);
 \draw[dashed] (2,2) circle (2);
\fill[red](2+ 2*cos 30,2+2*sin 30) circle (3pt);
\fill[magenta](2+ 2*cos{(120)},2+2*sin{(120)}) circle (3pt);
\end{tikzpicture}
&
\parbox[c]{8cm}{
 \BS{draw}[dashed] (2,2) circle (2);\\
 \smallskip
 \BS{fill} [red]{\color{red}(2+ 2*cos 30 , 2+2*sin 30)} circle (3pt);\\
  \smallskip
 \BS{fill}[magenta] {\color{red}(2+2*cos\AC{(120)} , 2+2*sin\AC{(120)})} circle (3pt); 
 }
\\ \hline 
\end{tabular} 

\SbSbSSCT{Position calculée avec \og library calc \fg}{Calculated positions with \og  calc  library calc \fg}

\begin{center}
\RRR{13-5}
\end{center}
\label{lib-calc}

 \maboite{\BS{usetikzlibrary}\AC{calc}}
 
\begin{tabular}{|c|c|} \hline  
\begin{tikzpicture}[baseline=0pt]
\draw [help lines] (0,0) grid (3,2);
\node (a) at (1,1) {A};
\fill [red] ($(a) + 2/3*(1cm,0)$) circle (2pt);
\fill [red] ($(a) + 4/3*(1cm,0)$) circle (2pt);
\end{tikzpicture}
&
\parbox{8cm}{
\BS{node} (a) at (1,1) \AC{A}; \\
\BS{fill} [red] {\color{red} (\$(a) + 2/3*(1cm,0)\$)} circle (2pt); \\
\BS{fill} [red] {\color{red}(\$(a) + 4/3*(1cm,0)\$)} circle (2pt); \\
}
\\ 
\hline 
\end{tabular} 

\SbSbSSCT{Tangentes avec \og library calc \fg}{Tangents with  \og calc library  \fg}

\begin{center}
\RRR{13-2-4}
\end{center}

\begin{tabular}{|c|c|} \hline 
\multicolumn{2}{|l|}{\BS{node}[fill=green!20] (a) at (3,1.5) \AC{A}; } \\
\multicolumn{2}{|l|}{\BS{fill}[red] (\RDD{tangent cs}:\RDD{node}=c,\RDD{point}=\AC{(A)},\RDD{solution}=1);  }\\ 
\hline
\begin{tikzpicture}
\draw[help lines] (0,0) grid (4,2);
\node[fill=green!20] (A) at (3,1.5) {A};
\node [circle,draw] (c) at (1,1) [minimum size=1.5cm] {$c$};
\draw[red,dashed] (A) - -(tangent cs:node=c,point={(A)},solution=1) ;
\draw[red,dashed] (1,1) - -(tangent cs:node=c,point={(3,1.5)},solution=1) ;
\fill[red] (tangent cs:node=c,point={(A)},solution=1) circle (3pt);
\end{tikzpicture}
&
\begin{tikzpicture}
\draw[help lines] (0,0) grid (4,2);
\node[fill=green!20] (A) at (3,1.5) {A};
\node [circle,draw] (c) at (1,1) [minimum size=1.5cm] {$c$};
\draw[red,dashed] (A) - -(tangent cs:node=c,point={(A)},solution=2) ;
\draw[red,dashed] (1,1) - -(tangent cs:node=c,point={(A)},solution=2) ;
\fill[red] (tangent cs:node=c,point={(A)},solution=2) circle (3pt);
\end{tikzpicture}
\\ \hline
\RDD{solution}=1 & \RDD{solution}=2
\\ \hline
\end{tabular} 

\newpage

\SbSbSSCT{Point à pourcentage donné }{Percentage position }

\begin{center}
\RRR{13-5-3}
\end{center}


\begin{tabular}{|c|c|} \hline  
\multicolumn{2}{|c|}{\BS{fill}[red] ({\color{red}\$(0,1)!.25!(4,1)\$}) circle (4pt); } \\  \hline  

\begin{tikzpicture}
\draw [help lines] (0,0) grid (4,2);
\draw [line width= 3pt] (0,1) -- (4,1);
\fill[red] ($(0,1)!.25!(4,1)$) circle (4pt);
\end{tikzpicture}
&  
\begin{tikzpicture}
\draw [help lines] (0,0) grid (4,2);
\draw [line width= 3pt] (0,1) -- (4,1);
\fill[red] ($(0,1)!.75!(4,1)$) circle (4pt);
\end{tikzpicture}
\\ \hline (0,1)!{\color{red}0.25}!(4,1) & (0,1)!{\color{red}0.75}!(4,1) \\ 
\hline 
\end{tabular} 

\bigskip

\begin{tabular}{|c|} \hline  
\begin{tikzpicture}
\draw [help lines] (0,0) grid (4,3);
\draw [line width=2pt ](0,2) -- (4,2);
\draw[red] ($(0,2)!.75!(4,2)$) -- (0,0);
\fill[red] ($(0,2)!.75!(4,2)!.66!(0,0)$) circle (4pt);
\end{tikzpicture}
\\ \hline 
\BS{fill}[red] (\${\color{blue}(0,2)!0.75!(4,2)}!{\color{red}0.66!(0,0)}\$) circle (2pt);
\\ \hline 
\end{tabular} 


\SbSbSSCT{Point à distance donnée}{Position at a given distance }

\begin{center}
\RRR{13-5-4}
\end{center}

\begin{tabular}{|c|c|} \hline  
\multicolumn{2}{|c|}{\BS{fill}[red] ({\color{red}\$(0,1)!1.5cm!(4,1)\$}) circle (4pt); } \\  \hline  

\begin{tikzpicture}
\draw [help lines] (0,0) grid (4,2);
\draw [line width= 2pt] (0,1) -- (4,1);
\fill[red] ($(0,1)!1.5cm!(4,1)$) circle (4pt);
\end{tikzpicture}
&  
\begin{tikzpicture}
\draw [help lines] (0,0) grid (4,2);
\draw [line width= 2pt] (0,1) -- (4,1);
\fill[red] ($(0,1)!3cm!(4,1)$) circle (4pt);
\end{tikzpicture}
\\ \hline (0,1)!{\color{red}1.5cm}!(4,1) & (0,1)!{\color{red}3cm}!(4,1) \\ 
\hline 
\end{tabular} 

\bigskip

\begin{tabular}{|c|} \hline  
\begin{tikzpicture}
\draw [help lines] (0,0) grid (4,4);
\coordinate (a) at (1,0);
\coordinate (b) at (4,1);
\draw [line width= 3pt] (0,0) -- (4,1);
\draw [line width= 2pt,red](2,.5) -- ($ (2,.5)!2cm!90:(4,1) $);
\end{tikzpicture}
\\ \hline
\BS{draw} (2,.05) - - (\$ (2,0.5)!{\color{red}2cm!90:(4,1)} \$);
\\ \hline 
\end{tabular} 

\newpage

\SbSbSSCT{Coordonnées relatives}{Relative coordinates}


\Par{Cartésienne}{Cartesian coordinates}

\begin{center}
\RRR{13-4-1}
\end{center}

\begin{tabular}{|c|c|c|} \hline  
\TFRGB{relative à l'origine}{relative to the origin}  & \TFRGB{relative à une position}{relative to a position}  &  \TFRGB{relative à la dernière position}{relative to the last position}   
\\ \hline  
 
\begin{tikzpicture}
\draw[help lines] (0,-1) grid (3,1); 
 \draw[blue,very thick] (0,0) -- (1,0) - - (2,1) - - (2,-1);
 \fill[red] (0,0) circle (4pt);
\end{tikzpicture}
&
\begin{tikzpicture} %[scale=.8]
\draw[help lines] (0,-1) grid (4,1);
 \draw[blue,very thick] (0,0) - - (1,0) -- +(2,1) -- +(2,-1) ; %–- +(2,-1) ;
 \fill[red] (1,0) circle (4pt);
\end{tikzpicture}
&
\begin{tikzpicture} %[scale=.8]
\draw[help lines] (0,-1) grid (5,1);  
 \draw[blue,very thick] (0,0) -- (1,0)  - - ++(2,1) - - ++(2,-1);
 \fill[red] (1,0) circle (4pt);
 \fill[red] (3,1) circle (4pt);
\end{tikzpicture}
\\ \hline 
\tikz \fill node[fill=green!20,inner sep=0pt]{(0,0)}; - - (1,0) &
 (0,0) - - \tikz \fill node[fill=green!20,inner sep=0pt]{(1,0)};  & (0,0) - - \tikz \fill node[fill=green!20,inner sep=0pt]{(1,0)}; \\
 - - (2,1) - - (2,-1)  &
   - - +(2,1) - - +(2,-1) & - - ++\tikz \fill node[fill=green!20,inner sep=0pt]{(2,1)}; - - ++(2,-1)
\\ \hline 
\end{tabular} 

\bigskip

\begin{tabular}{|c|c|c|} \hline  
\begin{tikzpicture} [scale=.5]
\draw[help lines] (0,-1) grid (6,6);
 \draw[red,dotted,line width=2pt] (0,0) rectangle (2,2) ;
  \draw[green,dotted,line width=2pt] (0,0) rectangle (3,3) ;  
 \draw[blue,line width=2pt] (0,0) rectangle (1,1)  rectangle (2,2) rectangle (3,3);

\end{tikzpicture}

&  
\begin{tikzpicture} [scale=.5]
\draw[help lines] (0,-1) grid (6,6); 
  \draw[green,dotted,line width=2pt] (1,1) rectangle (4,4) ;   
 \draw[blue,line width=2pt] (0,0) rectangle (1,1)  rectangle +(2,2) rectangle +(3,3);
    \fill[red] (1,1) circle (4pt);
\end{tikzpicture}
&  
\begin{tikzpicture} [scale=.5]
\draw[help lines] (0,-1) grid (6,6);  
 \draw[blue,line width=2pt] (0,0) rectangle (1,1)  rectangle ++(2,2) rectangle ++(3,3);
    \fill[red] (1,1) circle (4pt);
     \fill[green] (3,3) circle (4pt); 
\end{tikzpicture}
\\ 
\hline 
\BS{draw} (0,0) rectangle (1,1)   &
\BS{draw} (0,0) rectangle (1,1)   & 
\BS{draw} (0,0) rectangle (1,1)  \\
rectangle (2,2) rectangle (3,3);  &
rectangle +(2,2) rectangle +(3,3);  &
rectangle ++(2,2) rectangle ++(3,3); \\
\hline 
\end{tabular}


\Par{Polaire }{Polar} {}

\bigskip


\noindent

\begin{tabular}{|c|c|c|c|} \hline
\TFRGB{relative à l'origine}{relative to the origin}  & \TFRGB{relative à une position}{relative to a position}  &  \TFRGB{relative à la dernière position}{relative to the last position}   
\\ \hline    
\begin{tikzpicture} %[scale=.8] 
\draw[help lines] (0,-1) grid (3,1);
 \fill[red] (0:0) circle (4pt);
 \draw[blue,very thick] (0:0)-- (0:1) -- (30:2) -- (-30:2);
\end{tikzpicture}
&
\begin{tikzpicture} %[scale=.8] 
\draw[help lines] (0,-1) grid (4,1);
 \fill[red] (1,0) circle (4pt);
 \draw[blue,very thick] (0:0) -- (0:1) -- +(30:2) -- +(-30:2);
\end{tikzpicture}
&
\begin{tikzpicture} %[scale=.8] 
\draw[help lines] (0,-1) grid (5,1);
 \fill[red] (1,0) circle (4pt);
 \fill[red] (2.732,1) circle (4pt);
 \draw[blue,very thick] (0:0)-- (0:1) -- ++(30:2) -- ++(-30:2);
\end{tikzpicture}
\\ \hline
\tikz \fill node[fill=green!20,inner sep=0pt] {(0:0)}; - - (0:1)&
 (0:0) - - \tikz \fill node[fill=green!20,inner sep=0pt] {(0:1)}; & (0:0)- - \tikz \fill node[fill=green!20,inner sep=0pt] {(0:1)}; \\
 - - (30:2) - - (-30:2)  &  - -  +(30:2) - - +(-30:2) & - -  ++\tikz \fill node[fill=green!20,inner sep=0pt] {(30:2)}; - - ++(-30:2)
\\ \hline 
\end{tabular} 

%\subsubsection{coordonnée relative en polaire}
\Par{coordonnée relative en polaire}{Relative polar coordinate}

\begin{center}
\RRR{13-4-2}
\end{center}
\bigskip

\begin{tabular}{|c|c|} \hline 
\multicolumn{2}{|c|}{ \BS{draw}[blue,very thick] (0,0) -- (2,1) -- ([turn]-45:1cm);}
 \\ \hline
\begin{tikzpicture} %[scale=.8] 
\draw[help lines] (0,0) grid (4,2);
 \draw[dotted] (0,0) -- (4,2);
 \draw[blue,very thick] (0,0) -- (2,1) -- ([turn]-45:1cm);
\end{tikzpicture}
&  
\begin{tikzpicture} %[scale=.8] 
\draw[help lines] (0,0) grid (4,2);
 \draw[dotted] (0,0) -- (4,2);
 \draw[blue,very thick] (0,0) -- (2,1) -- ([turn]45:1cm);
\end{tikzpicture}
\\ \hline ([\RDD{turn}]-45:1cm) & ([\RDD{turn}]45:1cm) \\ 
\hline 
\end{tabular}

\bigskip

\begin{tabular}{|c|c|} \hline  
\begin{tikzpicture}  
\draw[help lines] (-1,0) grid (4,3);
\draw [line width=2pt] (4,0) arc (0 :120 :2)  -- ([turn]90:2cm) ;

\end{tikzpicture}
&  
\begin{tikzpicture} %[scale=.8] 
\draw[help lines] (0,0) grid (4,3);
\draw [line width=2pt]  (0,0) to [bend left] (2,2) --  ([turn]0:2cm);
\fill [red](2,2) circle (4pt);
\end{tikzpicture}
\\ \hline  
\BS{draw} (4,0) arc (0 :120 :2)  - - ([\RDD{turn}]90:2cm) ;
& \BS{draw}  (0,0) to [bend left] (2,2) - -  ([\RDD{turn}]0:2cm); \\

\hline 
\end{tabular} 


%\bigskip 
%
%
%\tikz [delta angle=30, radius=1cm]
%\draw (0,0) arc [start angle=0] -- ([turn]0:1cm)
%arc [start angle=30] -- ([turn]0:1cm)
%arc [start angle=60] -- ([turn]30:1cm);



\bigskip

\begin{tabular}{|c|c|c|} \hline  
\multicolumn{3}{|c|}{ \BS{draw}(1,2)
.. controls ([turn]0:2cm) .. ([turn]-90:2cm); }
\\ \hline
\begin{tikzpicture} %[scale=.8] 
\draw[help lines] (0,0) grid (4,4);
 \draw [line width=2pt] (1,2)
.. controls ([turn]0:2cm) .. ([turn]-90:2cm);
\end{tikzpicture}
&  
\begin{tikzpicture} %[scale=.8] 
\draw[help lines] (0,0) grid (4,4);
 \draw [line width=2pt] (1,2)
.. controls ([turn]30:2cm) .. ([turn]-90:2cm);
\end{tikzpicture}
&  
\begin{tikzpicture} %[scale=.8] 
\draw[help lines] (-2,0) grid (2,4);
 \draw [line width=2pt] (1,2)
.. controls ([turn]0:2cm) .. ([turn]90:2cm);

\end{tikzpicture}
\\ \hline ([turn]0:2cm) .. ([turn]-90:2cm) & ([turn]30:2cm) .. ([turn]-90:2cm) & ([turn]0:2cm) .. ([turn]90:2cm) \\ 
\hline 
\end{tabular} 


\tikzset{every picture/.style=blue,very thick,inner sep=.3333em}

%
%%\newpage
%%
%%\SbSSCT{Le peuple TikZ}{Tikzpeople}

\label{people}

 \maboite{\BS{usepackage}\AC{tikzpeople} \cite {tikzpeople} \footnote{ conflit \BS{usetikzlibrary}\AC{patterns} page \pageref{lib-patterns} : placer cette commande en premier} }

\bigskip
\begin{tabular}{|c|c|}\hline  
\BS{tikz} \BS{node}[\RDD{alice}] at (0,0) {};  &  \tikz \node[alice] at (0,0) {};\\ 
\hline 
\end{tabular} 
 
\SbSbSSCT{Personages disponibles}{available characters}

\noindent



\begin{tabular}{|c|c|c|c|c|c|c|}\hline 
\multicolumn{7}{|c|}{ \BS{tikz} \BS{node}[\RDD{alice},minimum size=1.5cm] at (0,0) {};  }
\\ \hline  
\tikz \node[alice,minimum size=1.5cm] at (0,0) {}; &  
\tikz \node[bob,minimum size=1.5cm] at (0,0) {}; &  
\tikz \node[bride,minimum size=1.5cm] at (0,0) {}; &  
\tikz \node[builder,minimum size=1.5cm] at (0,0) {}; &  
\tikz \node[businessman,minimum size=1.5cm] at (0,0) {}; &  
\tikz \node[charlie,minimum size=1.5cm] at (0,0) {}; &  
\tikz \node[chef,minimum size=1.5cm] at (0,0) {}; 
\\  \hline  
\RDD{alice} & \RDD{bob} & \RDD{bride} & \RDD{builder} & \RDD{businessman} & \RDD{charlie}  & \RDD{chef} 
\\  \hline  
\tikz \node[conductor,minimum size=1.5cm] at (0,0) {}; &  
\tikz \node[cowboy,minimum size=1.5cm] at (0,0) {}; &  
\tikz \node[criminal,minimum size=1.5cm] at (0,0) {}; &  
\tikz \node[dave,minimum size=1.5cm] at (0,0) {}; &  
\tikz \node[graduate,minimum size=1.5cm] at (0,0) {}; &  
\tikz \node[groom,minimum size=1.5cm] at (0,0) {}; & 
\tikz \node[guard,minimum size=1.5cm] at (0,0) {}; 
\\ \hline  
\RDD{conductor} & \RDD{cowboy} & \RDD{criminal} & \RDD{dave} & \RDD{graduate} & \RDD{groom} & \RDD{guard} 
\\ \hline  
\tikz \node[jester,minimum size=1.5cm] at (0,0) {}; &  
%\tikz \node[judge,minimum size=1.5cm] at (0,0) {}; 
&  
\tikz \node[mexican,minimum size=1.5cm] at (0,0) {}; &  
\tikz \node[nun,minimum size=1.5cm] at (0,0) {}; &  
\tikz \node[nurse,minimum size=1.5cm] at (0,0) {}; &  
\tikz \node[physician,minimum size=1.5cm] at (0,0) {}; &  
\tikz \node[pilot,minimum size=1.5cm] at (0,0) {};\\ \hline  
\RDD{jester} &  \RDD{judge} &  \RDD{mexican}  & \RDD{nun} &  \RDD{nurse} & \RDD{physician} 
&  \RDD{pilot}
\\ \hline  

\tikz \node[police,minimum size=1.5cm] at (0,0) {}; &  
\tikz \node[priest,minimum size=1.5cm] at (0,0) {}; &  
\tikz \node[sailor,minimum size=1.5cm] at (0,0) {}; &  
\tikz \node[santa,minimum size=1.5cm] at (0,0) {}; &  
\tikz \node[surgeon,minimum size=1.5cm] at (0,0) {};&  &  \\ 
\hline \RDD{police} & \RDD{priest}  & \RDD{sailor} & \RDD{santa} & \RDD{surgeon} &  &  \\ 
\hline 
\end{tabular} 

\subsubsection{Options}

\noindent

\begin{tabular}{|c|c|c|c|c|}\hline
\multicolumn{5}{|c|}{ \BS{tikz} \BS{node}[businessman,\RDD{evil},minimum size=1.5cm] at (0,0) {};  }
\\ \hline  
\tikz \node[businessman,evil,minimum size=1.5cm] at (0,0) {}; &  
\tikz \node[businessman,female,minimum size=1.5cm] at (0,0) {}; &  
\tikz \node[businessman,good,minimum size=1.5cm] at (0,0) {}; &  
\tikz \node[businessman,mirrored,minimum size=1.5cm] at (0,0) {}; &  
\tikz \node[businessman,monitor,minimum size=1.5cm] at (0,0) {};  
\\  \hline
\RDD{evil} & \RDD{female} & \RDD{good} & \RDD{mirrored} & \RDD{monitor}

\\  \hline 
\end{tabular}

\SbSbSSCT{Point d'ancrage spécifique}{Anchor specific}

\noindent

\begin{tabular}{|c|c|} \hline  
\begin{tikzpicture}[baseline=0pt,blue]
 \node[name=a,shape=bob,minimum
size=1.5cm] {};
 \node at (1.25,.5) [ellipse callout, draw,
callout absolute pointer={(a.mouth)},
font=\tiny] {Hey!};
\end{tikzpicture}
&  
\parbox{12cm}{
\BS{begin}\AC{tikzpicture}[blue] \\
\BS{node}[name=a,shape=bob,minimum size=1.5cm] \AC{};\\
\BS{node} at (1.25,.5) [ellipse callout, draw,
callout absolute pointer\AC{(a.\RDD{mouth})},
font=\BS{tiny}] {Hey!};\\
\BS{end}\AC{tikzpicture} \\
}
\\ \hline 
\end{tabular} 


\SbSbSSCT{Couleurs }{Colors}

\noindent


\begin{tabular}{|c|c|c|c|}\hline
\multicolumn{4}{|c|}{ \BS{tikz} \BS{node}[\blll{alice},\RDD{hair}=red,minimum size=1.5cm] at (0,0) {};  }
\\ \hline    
\tikz \node[alice,hair=red,minimum size=1.5cm] at (0,0) {}; &  
\tikz \node[alice,skin=red,minimum size=1.5cm] at (0,0) {}; &  
\tikz \node[alice,shirt=red,minimum size=1.5cm] at (0,0) {}; &  
\tikz \node[alice,undershirt=red,minimum size=1.5cm] at (0,0) {};  
\\  \hline
\RDD{hair}=red & \RDD{skin}=red & \RDD{shirt}=red & \RDD{details}=red 

\\  \hline 
\end{tabular}

\bigskip
\begin{tabular}{|c|c|c|c|}\hline 
\multicolumn{4}{|c|}{ \BS{tikz} \BS{node}[\blll{bob},\RDD{hair}=red,minimum size=1.5cm] at (0,0) {};  }
\\ \hline   
\tikz \node[bob,hair=red,minimum size=1.5cm] at (0,0) {}; &  
\tikz \node[bob,skin=red,minimum size=1.5cm] at (0,0) {}; &  
\tikz \node[bob,shirt=red,minimum size=1.5cm] at (0,0) {}; &  
\tikz \node[bob,details=red,minimum size=1.5cm] at (0,0) {};  
\\  \hline
\RDD{hair}=red & \RDD{skin}=red & \RDD{shirt}=red & \RDD{details}=red 
\\  \hline 
\end{tabular}

\bigskip
\begin{tabular}{|c|c|c|c|c|}\hline 
\multicolumn{5}{|c|}{ \BS{tikz} \BS{node}[\blll{bride},\RDD{hair}=red,minimum size=1.5cm] at (0,0) {};  }
\\ \hline    
\tikz \node[bride,hair=red,minimum size=1.5cm] at (0,0) {}; &  
\tikz \node[bride,skin=red,minimum size=1.5cm] at (0,0) {}; &  
\tikz \node[bride,shirt=red,minimum size=1.5cm] at (0,0) {}; &  
\tikz \node[bride,pearls=red,minimum size=1.5cm] at (0,0) {}; &
\tikz \node[bride,veil=red,minimum size=1.5cm] at (0,0) {};  
\\  \hline
\RDD{hair}=red & \RDD{skin}=red & \RDD{shirt}=red & \RDD{pearls}=red & \RDD{veil}=red 

\\  \hline 
\end{tabular}

\bigskip
\begin{tabular}{|c|c|c|c|c|}\hline
\multicolumn{5}{|c|}{ \BS{tikz} \BS{node}[\blll{builder},\RDD{hair}=red,minimum size=1.5cm] at (0,0) {};  }
\\ \hline    
\tikz \node[builder,hair=red,minimum size=1.5cm] at (0,0) {}; &  
\tikz \node[builder,skin=red,minimum size=1.5cm] at (0,0) {}; &  
\tikz \node[builder,shirt=red,minimum size=1.5cm] at (0,0) {}; &  
\tikz \node[builder,trousers=red,minimum size=1.5cm] at (0,0) {}; &
\tikz \node[builder,hat=red,minimum size=1.5cm] at (0,0) {};  
\\  \hline
\RDD{hair}=red & \RDD{skin}=red & \RDD{shirt}=red & \RDD{trousers}=red & \RDD{hat}=red   
\\  \hline 
\end{tabular}

\bigskip
\begin{tabular}{|c|c|c|c|c|c|}\hline
\multicolumn{6}{|c|}{ \BS{tikz} \BS{node}[\blll{businessman},\RDD{hair}=red,minimum size=1.5cm] at (0,0) {};  }
\\ \hline    
\tikz \node[businessman,hair=red,minimum size=1.5cm] at (0,0) {}; &  
\tikz \node[businessman,skin=red,minimum size=1.5cm] at (0,0) {}; &  
\tikz \node[businessman,shirt=red,minimum size=1.5cm] at (0,0) {}; &  
\tikz \node[businessman,tie=red,minimum size=1.5cm] at (0,0) {}; &
\tikz \node[businessman,undershirt=red,minimum size=1.5cm] at (0,0) {};  &
\tikz \node[businessman,monogram=red,minimum size=1.5cm] at (0,0) {};
\\  \hline
\RDD{hair}=red & \RDD{skin}=red & \RDD{shirt}=red & \RDD{tie}=red & \RDD{undershirt}=red & \RDD{monogram}=red 
\\  \hline 
\end{tabular}


\bigskip
\begin{tabular}{|c|c|c|c|}\hline
\multicolumn{4}{|c|}{ \BS{tikz} \BS{node}[\blll{charlie},\RDD{hair}=red,minimum size=1.5cm] at (0,0) {};  }
\\ \hline    
\tikz \node[charlie,hair=red,minimum size=1.5cm] at (0,0) {}; &  
\tikz \node[charlie,skin=red,minimum size=1.5cm] at (0,0) {}; &  
\tikz \node[charlie,shirt=red,minimum size=1.5cm] at (0,0) {}; &  
\tikz \node[charlie,buttons=red,minimum size=1.5cm] at (0,0) {}; 

\\  \hline
\RDD{hair}=red & \RDD{skin}=red & \RDD{shirt}=red & \RDD{buttons}=red 
\\  \hline 
\end{tabular}


\bigskip
\begin{tabular}{|c|c|c|c|c|}\hline
\multicolumn{5}{|c|}{ \BS{tikz} \BS{node}[\blll{chef},\RDD{hair}=red,minimum size=1.5cm] at (0,0) {};  }
\\ \hline 
\tikz \node[chef,hair=red,minimum size=1.5cm] at (0,0) {}; &  
\tikz \node[chef,skin=red,minimum size=1.5cm] at (0,0) {}; &  
\tikz \node[chef,shirt=red,minimum size=1.5cm] at (0,0) {}; &  
\tikz \node[chef,hat=red,minimum size=1.5cm] at (0,0) {}; &
\tikz \node[chef,details=red,minimum size=1.5cm] at (0,0) {};  
\\  \hline
\RDD{hair}=red & \RDD{skin}=red & \RDD{shirt}=red & \RDD{hat}=red & \RDD{details}=red 
\\  \hline 
\end{tabular}


\bigskip
\begin{tabular}{|c|c|c|c|c|}\hline
\multicolumn{5}{|c|}{ \BS{tikz} \BS{node}[\blll{conductor},\RDD{hair}=red,minimum size=1.5cm] at (0,0) {};  }
\\ \hline 
\tikz \node[conductor,hair=red,minimum size=1.5cm] at (0,0) {}; &  
\tikz \node[conductor,skin=red,minimum size=1.5cm] at (0,0) {}; &  
\tikz \node[conductor,shirt=red,minimum size=1.5cm] at (0,0) {}; &  \tikz \node[conductor,hat=red,minimum size=1.5cm] at (0,0) {}; &
\tikz \node[conductor,hatshield=red,minimum size=1.5cm] at (0,0) {};  
\\  \hline
\RDD{hair}=red & \RDD{skin}=red & \RDD{shirt}=red & \RDD{hat}=red & \RDD{hatshield}=red  
\\  \hline 
\tikz \node[conductor,undershirt=red,minimum size=1.5cm] at (0,0) {}; &  
\tikz \node[conductor,tie=red,minimum size=1.5cm] at (0,0) {}; &  
\tikz \node[conductor,hatbadge=red,minimum size=1.5cm] at (0,0) {}; &
\tikz \node[conductor,badge=red,minimum size=1.5cm] at (0,0) {}; &
\\  \hline 
\RDD{undershirt}=red &  \RDD{shirt}=red & \RDD{hatbadge}=red & \RDD{badge}=red &
\\  \hline 
\end{tabular}

\bigskip
\begin{tabular}{|c|c|c|c|}\hline
\multicolumn{4}{|c|}{ \BS{tikz} \BS{node}[\blll{cowboy},\RDD{hair}=red,minimum size=1.5cm] at (0,0) {};  }
\\ \hline 
\tikz \node[cowboy,hair=red,minimum size=1.5cm] at (0,0) {}; &  
\tikz \node[cowboy,skin=red,minimum size=1.5cm] at (0,0) {}; &  
\tikz \node[cowboy,shirt=green,minimum size=1.5cm] at (0,0) {}; &  
\tikz \node[cowboy,hat=red,minimum size=1.5cm] at (0,0) {};   
\\  \hline
\RDD{hair}=red & \RDD{skin}=red & \RDD{shirt}=green & \RDD{hat}=red 
\\  \hline 
\tikz \node[cowboy,patches=red,minimum size=1.5cm] at (0,0) {}; &  
\tikz \node[cowboy,tie=green ,minimum size=1.5cm] at (0,0) {}; &  
\tikz \node[cowboy,stitching=red,minimum size=1.5cm] at (0,0) {}; &
\tikz \node[cowboy,vest=red,minimum size=1.5cm] at (0,0) {}; 
\\  \hline 
\RDD{patches}=red &  \RDD{tie}=green & \RDD{stitching}=red & \RDD{vest}=red
\\  \hline 
\end{tabular}


\bigskip
\begin{tabular}{|c|c|c|c|}\hline 
\multicolumn{4}{|c|}{ \BS{tikz} \BS{node}[\blll{criminal},\RDD{hat}=red,minimum size=1.5cm] at (0,0) {};  }
\\ \hline 
\tikz \node[criminal,hat=red,minimum size=1.5cm] at (0,0) {}; &  
\tikz \node[criminal,skin=red,minimum size=1.5cm] at (0,0) {}; &  
\tikz \node[criminal,shirt=red,minimum size=1.5cm] at (0,0) {}; &  
\tikz \node[criminal,details=red,minimum size=1.5cm] at (0,0) {}; 
\\  \hline
\RDD{hat}=red & \RDD{skin}=red & \RDD{shirt}=red & \RDD{details}=red 
\\  \hline 
\end{tabular}

\bigskip
\begin{tabular}{|c|c|c|c|c|}\hline
\multicolumn{5}{|c|}{ \BS{tikz} \BS{node}[\blll{dave},\RDD{hair}=red,minimum size=1.5cm] at (0,0) {};  }
\\ \hline 
\tikz \node[dave,hair=red,minimum size=1.5cm] at (0,0) {}; &  
\tikz \node[dave,skin=red,minimum size=1.5cm] at (0,0) {}; &  
\tikz \node[dave,shirt=red,minimum size=1.5cm] at (0,0) {}; &  
\tikz \node[dave,undershirt=green,minimum size=1.5cm] at (0,0) {}; &
\tikz \node[dave,tie=green,minimum size=1.5cm] at (0,0) {};
\\  \hline
\RDD{hair}=red & \RDD{skin}=red & \RDD{shirt}=red & \RDD{undershirt}=green & \RDD{tie}=green
\\  \hline 
\end{tabular}

\bigskip
\begin{tabular}{|c|c|c|c|c|c|}\hline
\multicolumn{6}{|c|}{ \BS{tikz} \BS{node}[\blll{graduate},\RDD{hair}=red,minimum size=1.5cm] at (0,0) {};  }
\\ \hline 
\tikz \node[graduate,hair=red,minimum size=1.5cm] at (0,0) {}; &  
\tikz \node[graduate,skin=red,minimum size=1.5cm] at (0,0) {}; &  
\tikz \node[graduate,shirt=red,minimum size=1.5cm] at (0,0) {}; &  
\tikz \node[graduate,undershirt=red,minimum size=1.5cm] at (0,0) {}; &
\tikz \node[graduate,stripes=red,minimum size=1.5cm] at (0,0) {};
&
\tikz \node[graduate,hat=red,minimum size=1.5cm] at (0,0) {};
\\  \hline
\RDD{hair}=red & \RDD{skin}=red & \RDD{shirt}=red & \RDD{undershirt}=red & \RDD{stripes}=red & \RDD{hat}=red
\\  \hline 
\end{tabular}

\bigskip
\begin{tabular}{|c|c|c|c|c|c|}\hline
\multicolumn{6}{|c|}{ \BS{tikz} \BS{node}[\blll{groom},\RDD{hair}=red,minimum size=1.5cm] at (0,0) {};  }
\\ \hline
\tikz \node[groom,hair=red,minimum size=1.5cm] at (0,0) {}; &  
\tikz \node[groom,skin=red,minimum size=1.5cm] at (0,0) {}; &  
\tikz \node[groom,shirt=red,minimum size=1.5cm] at (0,0) {}; &  
\tikz \node[groom,undershirt=green,minimum size=1.5cm] at (0,0) {}; &
\tikz \node[groom,tie=green,minimum size=1.5cm] at (0,0) {}; &
\tikz \node[groom,hat=red,minimum size=1.5cm] at (0,0) {};
\\  \hline
\RDD{hair}=red & \RDD{skin}=red & \RDD{shirt}=red & \RDD{undershirt}=green & \RDD{tie}=green & \RDD{hat}=red
\\  \hline 
\end{tabular}


\bigskip
\begin{tabular}{|c|c|c|c|c|c|}\hline 
\multicolumn{6}{|c|}{ \BS{tikz} \BS{node}[\blll{guard},\RDD{hat}=red,minimum size=1.5cm] at (0,0) {};  }
\\ \hline
\tikz\node[guard,hat=red,minimum size=1.5cm] at (0,0) {}; &  
\tikz \node[guard,skin=red,minimum size=1.5cm] at (0,0) {}; &  
\tikz \node[guard,shirt=red,minimum size=1.5cm] at (0,0) {}; &  
\tikz \node[guard,collar=red,minimum size=1.5cm] at (0,0) {}; &
\tikz \node[guard,lining=red,minimum size=1.5cm] at (0,0) {}; &
\tikz \node[guard,details=red,minimum size=1.5cm] at (0,0) {};
\\  \hline
\RDD{hat}=red & \RDD{skin}=red & \RDD{shirt}=red & \RDD{collar}=red & \RDD{lining}=red & \RDD{details}=red
\\  \hline 
\end{tabular}


\bigskip
\begin{tabular}{|c|c|c|c|c|c|}\hline
\multicolumn{6}{|c|}{ \BS{tikz} \BS{node}[\blll{jester},\RDD{hat}=red,minimum size=1.5cm] at (0,0) {};  }
\\ \hline
\tikz \node[jester,hair=red,minimum size=1.5cm] at (0,0) {}; &  
\tikz \node[jester,skin=red,minimum size=1.5cm] at (0,0) {}; &  
\tikz \node[jester,shirt=yellow,minimum size=1.5cm] at (0,0) {}; &  
\tikz \node[jester,hat=red,minimum size=1.5cm] at (0,0) {}; &
%\tikz \node[jester,pattern=yellow,minimum size=1.5cm] at (0,0) {};
&
\tikz \node[jester,details=blue,minimum size=1.5cm] at (0,0) {};
\\  \hline
\RDD{hair}=red & \RDD{skin}=red & \RDD{shirt}=yellow & \RDD{hat}=red & \RDD{pattern}=yellow \footnote{voir confit} & \RDD{details}=blue
\\  \hline 
\end{tabular}

\bigskip
\begin{tabular}{|c|c|c|c|c|}\hline
\multicolumn{5}{|c|}{ \BS{tikz} \BS{node}[\blll{judge},\RDD{hair}=red,minimum size=1.5cm] at (0,0) {};  }
\\ \hline
%\tikz \node[judge,hair=red,minimum size=1.5cm] at (0,0) {}; 
&  
%\tikz \node[judge,skin=red,minimum size=1.5cm] at (0,0) {}; &  
%\tikz \node[judge,shirt=red,minimum size=1.5cm] at (0,0) {}; &  
%\tikz \node[judge,undershirt=red,minimum size=1.5cm] at (0,0) {}; &
%\tikz \node[judge,hairshadow=red,minimum size=1.5cm] at (0,0) {};

\\  \hline
\RDD{hair}=red & \RDD{skin}=red & \RDD{shirt}=red & \RDD{undershirt}=red & \RDD{hairshadow}=red 
\\  \hline 
\end{tabular}

\bigskip
\begin{tabular}{|c|c|c|c|c|c|c|}\hline
\multicolumn{7}{|c|}{ \BS{tikz} \BS{node}[\blll{mexican},\RDD{hair}=red,minimum size=1.5cm] at (0,0) {};  }
\\ \hline
\tikz \node[mexican,hair=red,minimum size=1.5cm] at (0,0) {}; &  
\tikz \node[mexican,skin=red,minimum size=1.5cm] at (0,0) {}; &  
\tikz \node[mexican,shirt=red,minimum size=1.5cm] at (0,0) {}; &  
\tikz \node[mexican,hat=green,minimum size=1.5cm] at (0,0) {}; &
\tikz \node[mexican,ringtop=red,minimum size=1.5cm] at (0,0) {};
&
\tikz \node[mexican,ringmid=red,minimum size=1.5cm] at (0,0) {};
&
\tikz \node[mexican,ringbot=yellow,minimum size=1.5cm] at (0,0) {};
\\  \hline
\RDD{hair}=red & \RDD{skin}=red & \RDD{shirt}=red & \RDD{hat}=green & \RDD{ringtop}=red &\RDD{ringmid}=red & \RDD{ringbot}=yellow
\\  \hline 
\end{tabular}


\bigskip

\begin{tabular}{|c|c|c|}\hline 
\multicolumn{3}{|c|}{ \BS{tikz} \BS{node}[\blll{nun},\RDD{plaid}=red,minimum size=1.5cm] at (0,0) {};  }
\\ \hline
\tikz \node[nun,plaid=red,minimum size=1.5cm] at (0,0) {}; &  
\tikz \node[nun,skin=red,minimum size=1.5cm] at (0,0) {}; &  
\tikz \node[nun,shirt=red,minimum size=1.5cm] at (0,0) {}; 
\\  \hline
\RDD{plaid}=red & \RDD{skin}=red & \RDD{shirt}=red 
\\  \hline 
\end{tabular}


\bigskip
\begin{tabular}{|c|c|c|c|c|c|c|}\hline 
\multicolumn{7}{|c|}{ \BS{tikz} \BS{node}[\blll{nurse},\RDD{hair}=red,minimum size=1.5cm] at (0,0) {};  }
\\ \hline
\tikz \node[nurse,hair=red,minimum size=1.5cm] at (0,0) {}; &  
\tikz \node[nurse,skin=red,minimum size=1.5cm] at (0,0) {}; &  
\tikz \node[nurse,shirt=red,minimum size=1.5cm] at (0,0) {}; &  
\tikz \node[nurse,badgeclip=green,minimum size=1.5cm] at (0,0) {}; &
\tikz \node[nurse,redcross=green,minimum size=1.5cm] at (0,0) {};
&
\tikz \node[nurse,badge=red,minimum size=1.5cm] at (0,0) {};
&
\tikz \node[nurse,badgename=red,minimum size=1.5cm] at (0,0) {};
\\  \hline
\RDD{hair}=red & \RDD{skin}=red & \RDD{shirt}=red & \RDD{badgeclip}=green & \RDD{redcross}=green & \RDD{badge}=red &  \RDD{badgename}=red
\\  \hline 
\end{tabular}


\bigskip
\begin{tabular}{|c|c|c|c|c|c|}\hline
\multicolumn{6}{|c|}{ \BS{tikz} \BS{node}[\blll{physician},\RDD{hair}=red,minimum size=1.5cm] at (0,0) {};  }
\\ \hline
\tikz \node[physician,hair=red,minimum size=1.5cm] at (0,0) {}; &  
\tikz \node[physician,skin=red,minimum size=1.5cm] at (0,0) {}; &  
\tikz \node[physician,shirt=red,minimum size=1.5cm] at (0,0) {}; &  
\tikz \node[physician,hat=red,minimum size=1.5cm] at (0,0) {}; &
\tikz \node[physician,stethoscope=red,minimum size=1.5cm] at (0,0) {};
&
\tikz \node[physician,tube=red,minimum size=1.5cm] at (0,0) {};
\\  \hline
\RDD{hair}=red & \RDD{skin}=red & \RDD{shirt}=red & \RDD{hat}=red & \RDD{stethoscope}=red &   \RDD{tube}=red
\\  \hline 
\end{tabular}


\bigskip
\begin{tabular}{|c|c|c|c|c|c|c|}\hline
\multicolumn{7}{|c|}{ \BS{tikz} \BS{node}[\blll{pilot},\RDD{hat}=red,minimum size=1.5cm] at (0,0) {};  }
\\ \hline
\tikz \node[pilot,hat=red,minimum size=1.5cm] at (0,0) {}; &  
\tikz \node[pilot,skin=red,minimum size=1.5cm] at (0,0) {}; &  
\tikz \node[pilot,shirt=red,minimum size=1.5cm] at (0,0) {}; &  
\tikz \node[pilot,undershirt=red,minimum size=1.5cm] at (0,0) {}; &
\tikz \node[pilot,visor=red,minimum size=1.5cm] at (0,0) {}; &
\tikz \node[pilot,straps=red,minimum size=1.5cm] at (0,0) {}; &
%\tikz \node[pilot,decoration=red,minimum size=1.5cm] at (0,0) {};
\\  \hline
\RDD{hat}=red & \RDD{skin}=red & \RDD{shirt}=red & \RDD{undershirt}=red & \RDD{visor}=red & \RDD{straps}=red & \RDD{decoration}=red
\\  \hline 
\end{tabular}


\bigskip
\begin{tabular}{|c|c|c|c|}\hline
\multicolumn{4}{|c|}{ \BS{tikz} \BS{node}[\blll{police},\RDD{hair}=red,minimum size=1.5cm] at (0,0) {};  }
\\ \hline 
\tikz \node[police,hair=red,minimum size=1.5cm] at (0,0) {}; &  
\tikz \node[police,skin=red,minimum size=1.5cm] at (0,0) {}; &  
\tikz \node[police,shirt=red,minimum size=1.5cm] at (0,0) {}; &  
\tikz \node[police,hat=red,minimum size=1.5cm] at (0,0) {};   
\\  \hline
\RDD{hair}=red & \RDD{skin}=red & \RDD{shirt}=red & \RDD{hat}=red
\\  \hline
\tikz \node[police,badge=red,minimum size=1.5cm] at (0,0) {}; &  
\tikz \node[police,hatbadge=red ,minimum size=1.5cm] at (0,0) {}; &  
\tikz \node[police,hatshield=red,minimum size=1.5cm] at (0,0) {}; &
\tikz \node[police,undershirt=red,minimum size=1.5cm] at (0,0) {}; 
\\  \hline 
\RDD{badge}=red &  \RDD{hatbadge}=red & \RDD{hatshield}=red & \RDD{undershirt}=red
\\  \hline 
\end{tabular}



\bigskip
\begin{tabular}{|c|c|c|c|c|c|}\hline
\multicolumn{6}{|c|}{ \BS{tikz} \BS{node}[\blll{priest},\RDD{hair}=red,minimum size=1.5cm] at (0,0) {};  }
\\ \hline
\tikz \node[priest,hair=red,minimum size=1.5cm] at (0,0) {}; &  
\tikz \node[priest,skin=red,minimum size=1.5cm] at (0,0) {}; &  
\tikz \node[priest,shirt=red,minimum size=1.5cm] at (0,0) {}; &  
\tikz \node[priest,hat=red,minimum size=1.5cm] at (0,0) {}; &
\tikz \node[priest,collar=red,minimum size=1.5cm] at (0,0) {}; &
\tikz \node[priest,cross=red,minimum size=1.5cm] at (0,0) {};
\\  \hline
\RDD{hair}=red & \RDD{skin}=red & \RDD{shirt}=red & \RDD{hat}=red & \RDD{collar}=red &   \RDD{cross}=red
\\  \hline 
\end{tabular}


\bigskip
\begin{tabular}{|c|c|c|c|c|c|c|}\hline 
\multicolumn{7}{|c|}{ \BS{tikz} \BS{node}[\blll{sailor},\RDD{hair}=red,minimum size=1.5cm] at (0,0) {};  }
\\ \hline
\tikz \node[sailor,hair=red,minimum size=1.5cm] at (0,0) {}; &  
\tikz \node[sailor,skin=red,minimum size=1.5cm] at (0,0) {}; &  
\tikz \node[sailor,shirt=red,minimum size=1.5cm] at (0,0) {}; &  
\tikz \node[sailor,hat=red,minimum size=1.5cm] at (0,0) {}; &
\tikz \node[sailor,undershirt=red,minimum size=1.5cm] at (0,0) {};
&
\tikz \node[sailor,stripes=red,minimum size=1.5cm] at (0,0) {};
&
\tikz \node[sailor,details=red,minimum size=1.5cm] at (0,0) {};
\\  \hline
\RDD{hair}=red & \RDD{skin}=red & \RDD{shirt}=red & \RDD{hat}=red & \RDD{undershirt}=red & \RDD{stripes}=red &  \RDD{details}=red
\\  \hline 
\end{tabular}


\bigskip
\begin{tabular}{|c|c|c|c|c|}\hline 
\multicolumn{5}{|c|}{ \BS{tikz} \BS{node}[\blll{santa},\RDD{hat}=green,minimum size=1.5cm] at (0,0) {};  }
\\ \hline
\tikz \node[santa,hat=green,minimum size=1.5cm] at (0,0) {}; &  
\tikz \node[santa,skin=green,minimum size=1.5cm] at (0,0) {}; &  
\tikz \node[santa,shirt=green,minimum size=1.5cm] at (0,0) {}; &  
\tikz \node[santa,beard=green,minimum size=1.5cm] at (0,0) {}; &
\tikz \node[santa,details=green,minimum size=1.5cm] at (0,0) {};

\\  \hline
\RDD{hat}=green & \RDD{skin}=green& \RDD{shirt}=green & \RDD{beard}=green & \RDD{details}=green 
\\  \hline 
\end{tabular}


\bigskip
\begin{tabular}{|c|c|c|c|c|}\hline 
\multicolumn{5}{|c|}{ \BS{tikz} \BS{node}[\blll{surgeon},\RDD{hat}=red,minimum size=1.5cm] at (0,0) {};  }
\\ \hline
\tikz \node[surgeon,hat=red,minimum size=1.5cm] at (0,0) {}; &  
\tikz \node[surgeon,skin=red,minimum size=1.5cm] at (0,0) {}; &  
\tikz \node[surgeon,shirt=red,minimum size=1.5cm] at (0,0) {}; &  
\tikz \node[surgeon,hair=red,minimum size=1.5cm] at (0,0) {}; &
\tikz \node[surgeon,mask=red,minimum size=1.5cm] at (0,0) {};

\\  \hline
\RDD{hat}=red & \RDD{skin}=red & \RDD{shirt}=red & \RDD{hair}=red & \RDD{mask}=red 
\\  \hline 
\end{tabular}


%
%%\newpage
%%
%%\SbSSCT{Canards}{Ducks}
%
%%\label{ducks}

 \maboite{\BS{usepackage}\AC{tikzducks} \cite {tikzducks}}


\begin{center}
\begin{tabular}{|c|}\hline  
\BS{tikz} \BSS{duck} ;
\\ \hline  
\tikz \duck ;
\\ \hline 
\end{tabular} 
\end{center}


\subsubsection{Options}

\noindent

\begin{tabular}{|c|c|c|c|c|} \hline 
\multicolumn{4}{|c|}{\BS{tikz} \BS{duck}[\RDD{body}=red] ;} 
\\ \hline
\tikz \duck[body=red] ;
&  
\tikz \duck[head=red] ;
&
\tikz \duck[bill=red] ;
  &
  \tikz \duck[eye=red] ;
    \\ 
\hline  
[\RDD{body}=red] & [\RDD{head}=red] & [\RDD{bill}=red] & [\RDD{eye}=red] \\ 
\hline 
\end{tabular} 
\begin{tabular}{|c|} \hline
\BS{tikz}  \BS{duck}[\RDD{grumpy}] ;
\\ \hline   
\tikz  \duck[grumpy] ;
\\ \hline  

\end{tabular} 



\bigskip

\begin{tabular}{|c|c|c|c|c|c|} \hline  
\tikz \duck[longhair] ;
&  
\tikz \duck[shorthair] ;
&
\tikz \duck[crazyhair] ;
&
\tikz \duck[recedinghair] ;
&
\tikz \duck[mohican] ;
&
\tikz \duck[mullet] ;
\\ \hline  
[\RDD{longhair}] & [\RDD{shorthair}] & [\RDD{crazyhair}] & [\RDD{recedinghair}] &  [\RDD{mohican}] &  [\RDD{mullet}]\\ 
\hline
\tikz \duck[longhair=red] ;
&  
\tikz \duck[shorthair=red] ;
&
\tikz \duck[crazyhair=red] ;
  &
  \tikz \duck[recedinghair=red] ;
  &
  \tikz \duck[mohican=red] ;
  &
  \tikz \duck[mullet=red] ;
    \\ 
\hline  
[longhair=red] & [shorthair=red] & [crazyhair=red] & [recedinghair=red] &  [mohican=red] &  [mullet=red]  \\ 
\hline 
\end{tabular}

\bigskip





\begin{tabular}{|c|c|c|c|c|} \hline  
\tikz \duck[eyebrow] ;
&  
\tikz \duck[eyebrow=red] ;
&
\tikz \duck[beard] ;
  &
  \tikz \duck[beard=red] ;
    \\ 
\hline  
[\RDD{eyebrow}] & [eyebrow=red] & [\RDD{beard}] & [beard=red] \\ 
\hline
 
\end{tabular}

\bigskip

\begin{tabular}{|c|c|c|c|c|} \hline  
\tikz \duck[tshirt] ;
&  
\tikz \duck[tie] ;
&
\tikz \duck[jacket] ;
&
\tikz \duck[cape] ;
&
\tikz \duck[tshirt,tie ,jacket ,cape] ;
\\ \hline
[\RDD{tshirt}] & [\RDD{tie}] & [\RDD{jacket}] & [\RDD{cape}]& [tshirt,tie ,jacket ,cape]
\\ \hline
\dft{white} & \dft{blue} & \dft{blue} & \dft{red}&
\\ \hline
\tikz \duck[tshirt=red] ;
&  
\tikz \duck[tie=red] ;
&
\tikz \duck[jacket=red] ;
&
\tikz \duck[cape=blue] ;
&

\\ \hline
[tshirt=red] & [tie=red] & [jacket=red] & [cape=blue]& 
\\ \hline
\end{tabular}

\bigskip

\begin{tabular}{|c|c|c|c|c|} \hline 
\tikz \duck[water];
&
\tikz \duck[alien];
&
\tikz \duck[hat];
&
\tikz \duck[tophat];
&
\tikz \duck[cap];
\\ \hline
[\RDD{water}] & [\RDD{alien}] & [\RDD{hat}]& [\RDD{tophat}] & [\RDD{cap}]
\\ \hline
\tikz \duck[santa];
&
\tikz \duck[graduate];
&
\tikz \duck[graduate,tassel];
&
\tikz \duck[beret];
&
\tikz \duck[peakedcap];
\\ \hline
[\RDD{santa}] & [\RDD{graduate}] & [graduate,\RDD{tassel}] & [\RDD{beret}] & [\RDD{peakedcap}]
\\ \hline
\tikz \duck[crown];
&
\tikz \duck[queencrown];
&
\tikz \duck[kingcrown];
&
\tikz \duck[sheep];
&
\tikz \duck[horsetail];
\\ \hline
[\RDD{crown}] &[\RDD{queencrown}]&[\RDD{kingcrown}] & [\RDD{sheep}] &[\RDD{horsetail}]
\\ \hline

\tikz \duck[crozier];
&
\tikz \duck[unicorn];
&

\tikz \duck[bunny];
&
\tikz \duck[bunny=red,inear=blue];
&
\tikz \duck[witch];
\\ \hline
 [\RDD{crozier}] & [\RDD{unicorn}] &[\RDD{bunny}] & [bunny=red,\RDD{inear}=blue] & [\RDD{witch}]
\\ \hline
\tikz \duck[magicwand];
&
\tikz \duck[magichat];
&
\tikz \duck[magichat=teal,
magicstars=blue!30!cyan,
magicwand];
&
\tikz \duck[glasses];
&
\tikz \duck[sunglasses];
\\ \hline
[\RDD{magicwand}] & [\RDD{magichat}] & \parbox{3cm}{[magichat,\\ \RDD{magicstars}]} & [\RDD{glasses}] & [\RDD{sunglasses}]
\\ \hline
\end{tabular}  


\begin{tabular}{|c|c|c|c|c|} \hline

\tikz \duck[squareglasses];
&
\tikz \duck[signpost=42];
&
\tikz \duck[signpost=XXX,signcolour=green];
&
\tikz \duck[signpost=XXX,signback=green];
&
\tikz \duck[speech={XXX}];
\\ \hline
[\RDD{squareglasses}] & [\RDD{signpost}=42] & \parbox{3cm}{[signpost=XXX,\\ \RDD{signcolour}=green]} & \parbox{3cm}{[signpost=XXX, \\ \RDD{signback}=green]} & [\RDD{speech}=\AC{XXX}] 
\\ \hline 
\end{tabular}


\begin{tabular}{|c|c|c|c|c|} \hline 
\tikz \duck[speech={XXX},bubblecolour=green];
&
\tikz \duck[think={XXX}];
&
\tikz \duck[think=XXX,bubblecolour=green];
&
\tikz \duck[book={XXX}];
\\ \hline
\parbox{3cm}{[speech={XXX},\\ \RDD{bubblecolour}=green]} & [\RDD{think}=\AC{XXX}] & 
\parbox{3cm}{[think={XXX},\\ \RDD{bubblecolour}=green]}
&[\RDD{book}=\AC{XXX}] 
\\ \hline
%\end{tabular}  
%
%\begin{tabular}{|c|c|c|c|c|} \hline
\tikz \duck[book=XXX,bookcolour=green];
&
\tikz \duck[book=\scalebox{0.5}{XXX}];
&
\multicolumn{2}{|c|}{\tikz 
\duck[signpost=\scalebox{0.4}{
\parbox{2cm}{
\centering XXX \\ XXXXX}}]
;}
\\ \hline

\parbox{3cm}{[book={XXX},\\ \RDD{bookcolour}=green]} 
&
\parbox{3.5cm}{\BS{tikz} \BS{duck}[book=\\ \BS{scalebox}\AC{0.5}\AC{XXX}]; }
&
\multicolumn{2}{|c|}{ \parbox{7cm}{ \BS{tikz} 
\BS{duck}[signpost=\BS{scalebox}\AC{0.4}\AC{ \\
\BS{parbox}\AC{2cm}{  
\BS{centering} XXX \ XXXXX}}]}
;}
\\ \hline
\end{tabular}


\begin{tabular}{|c|c|c|c|c|} \hline 
\tikz \duck[cricket];
&
\tikz \duck[hockey];
&
\tikz \duck[football];
&
\tikz \duck[lightsaber];
&
\tikz \duck[torch];
\\ \hline
[\RDD{cricket}]& [\RDD{hockey}] & [\RDD{football}] & [\RDD{lightsaber}] & [\RDD{torch}]
\\ \hline
\tikz \duck[prison];
&
\tikz \duck[necklace];
&
\tikz \duck[icecream];
&
\tikz \duck[icecream,flavoura=green];
&
\tikz \duck[icecream,flavourb=green];
\\ \hline
[\RDD{prison}] & [\RDD{necklace}] & [\RDD{icecream}] &
\parbox{3cm}{[icecream,\\ \RDD{flavoura}=green]} & 
\parbox{3cm}{[icecream,\\ \RDD{flavourb}=green]}
\\ \hline
\tikz \duck[icecream,flavourc=green];
&
\tikz \duck[chef];
&
\tikz \duck[rollingpin];
&
\tikz \duck[cake];
&
\tikz \duck[pizza];
\\ \hline
\parbox{3cm}{[icecream,\\ \RDD{flavourc}=green]} &
 [\RDD{chef}] & [\RDD{rollingpin}] & [\RDD{cake}] & [\RDD{pizza}]
\\ \hline
\tikz \duck[baguette];
&
\tikz \duck[milkshake];
&
\tikz \duck[wine];
&
\tikz \duck[mask];
&
\tikz \duck[buttons];
\\ \hline
  [\RDD{baguette}] & [\RDD{milkshake}] & [\RDD{wine}] & [\RDD{mask}] & [\RDD{buttons}]
\\ \hline
\end{tabular}

\begin{tabular}{|c|c|c|c|c|} \hline 
\tikz \duck[basket];
&
\tikz \duck[easter];
&
\tikz \duck[easter,egga=red];
&
\tikz \duck[easter,eggb=red];
&
\tikz \duck[easter,eggc=red];
\\ \hline
[\RDD{basket}]& [\RDD{easter}] & [easter,\RDD{egga}=red] & [easter,\RDD{eggb}=red] & [easter,\RDD{eggc}=red]
\\ \hline

\end{tabular}

\bigskip

\begin{tabular}{|c|c|c|c|} \hline 
\multicolumn{4}{|c|}{\BS{tikz} \BS{duck} \BS{path}[preaction=\AC{fill,green},pattern=dots, pattern  color=red]  \BSS{duckpathbody} ;} 
\\ \hline
\tikz \duck
\path[preaction={fill,green},pattern=dots, pattern  color=red]  \duckpathbody;
&
\tikz   \duck
\path[preaction={fill,green},pattern=dots, pattern  color=red]  \duckpathgrumpybill;
&
\tikz   \duck
\path[preaction={fill,green},pattern=dots, pattern  color=red]  \duckpathbill;
&
\tikz   \duck
\path[preaction={fill,green},pattern=dots, pattern  color=red]  \duckpathtshirt;
\\ \hline 
\BSS{duckpathbody } & \BSS{duckpathgrumpybill} & \BSS{duckpathbill} & \BSS{duckpathtshirt} 
\\ \hline
\tikz   \duck
\path[preaction={fill,green},pattern=dots, pattern  color=red]  \duckpathjacket;
&
\tikz   \duck
\path[preaction={fill,green},pattern=dots, pattern  color=red]  \duckpathcape ;
&
\tikz   \duck
\path[preaction={fill,green},pattern=dots, pattern  color=red]  \duckpathshorthair ;
&
\tikz   \duck
\path[preaction={fill,green},pattern=dots, pattern  color=red]  \duckpathlonghair;
\\ \hline 
\BSS{duckpathjacket} & \BSS{duckpathcape}  & \BSS{duckpathshorthair} & \BSS{duckpathlonghair} 
\\ \hline 
\tikz   \duck
\path[preaction={fill,green},pattern=dots, pattern  color=red]  \duckpathcrazyhair;
&
\tikz   \duck
\path[preaction={fill,green},pattern=dots, pattern  color=red]  \duckpathrecedinghair;
&
\tikz   \duck
\path[preaction={fill,green},pattern=dots, pattern  color=red]  \duckpathcrown ;
&
\tikz   \duck
\path[preaction={fill,green},pattern=dots, pattern  color=red]   \duckpathmohican ;
\\ \hline 
\BSS{duckpathcrazyhair} & \BSS{duckpathrecedinghair} & \BSS{duckpathcrown} & \BSS{duckpathmohican}
\\ \hline 
\tikz   \duck
\path[preaction={fill,green},pattern=dots, pattern  color=red]   \duckpathmullet;
&
\tikz   \duck
\path[preaction={fill,green},pattern=dots, pattern  color=red]   \duckpathqueencrown ;
&
\tikz   \duck
\path[preaction={fill,green},pattern=dots, pattern  color=red]   \duckpathkingcrown ;
&
\tikz   \duck
\path[preaction={fill,green},pattern=dots, pattern  color=red]  \duckpathdarthvader ;
\\ \hline 
\BSS{duckpathmullet} & \BSS{duckpathqueencrown} & \BSS{duckpathkingcrown} & \BSS{duckpathdarthvader}
\\ \hline 
\tikz   \duck
\path[preaction={fill,green},pattern=dots, pattern  color=red]  \duckpathhorsetail ;
& & & 
\\ \hline 
\BSS{duckpathhorsetail}& & & 
\\ \hline 
\end{tabular}


\SbSbSSCT{Canards aléatoires}{Random ducks}

\noindent

\begin{tabular}{|c|}\hline  
\BS{tikz} \BSS{randuck} ; \BS{tikz} \BSS{randuck} ; \BS{tikz} \BSS{randuck} ; \BS{tikz} \BSS{randuck} ; \BS{tikz} \BSS{randuck} ; 
\\ \hline    
\tikz \randuck ; \tikz \randuck  ; \tikz  \randuck  ; \tikz  \randuck  ; \tikz  \randuck ;
\\ \hline 
\end{tabular} 

\bigskip

\begin{tabular}{|c|} \hline  
\BS{tikz} \BSS{shuffleducks} \BS{duck}[\BSS{randomhead}] ;
\\ \hline  
\tikz \shuffleducks \duck[\randomhead] ; \tikz \shuffleducks \duck[\randomhead] ; \tikz \shuffleducks \duck[\randomhead] ; \tikz \shuffleducks \duck[\randomhead] ;
\tikz \shuffleducks \duck[\randomhead] ;
\\ \hline 
\end{tabular} 

\bigskip

\begin{tabular}{|c|} \hline  
\BS{tikz} \BSS{shuffleducks} \BS{duck}[\BSS{randomaccessories}] ;
\\ \hline  
\tikz \shuffleducks \duck[\randomaccessories] ; \tikz \shuffleducks \duck[\randomaccessories] ; \tikz \shuffleducks \duck[\randomaccessories] ; \tikz \shuffleducks \duck[\randomaccessories] ; \tikz \shuffleducks \duck[\randomaccessories] ; 
\\ \hline 
\end{tabular} 


\SbSbSSCT{Coordonnées}{Coordinates}

\noindent


\begin{tabular}{|c|c|c|} \hline 
\multicolumn{3}{|c|}{\BS{tikz} \BS{duck} \BS{fill}[red] (wing) circle (3pt);}
\\ \hline  
\tikz \duck \fill[red] (wing) circle (3pt);
&
\tikz \duck \fill[red] (head) circle (3pt);
&
\tikz \duck \fill[red] (bill) circle (3pt);
\\ \hline wing &  head & bill \\ 
\hline 
\end{tabular} 

\bigskip

\begin{tabular}{|c|} \hline
\BS{tikz} \BS{duck}[\RDD{name}=XXX] \\ \BS{begin}\AC{scope} [xshift=4cm] \BS{duck}[\RDD{name}=YYY] 
\BS{end}\AC{scope} \\ \BS{draw}[red] (XXX-wing) - - (YYY-bill) ;
\\ \hline
\tikz \duck[name=XXX] \begin{scope} [xshift=4cm] 
\duck[name=YYY]
\end{scope}  \draw[red] (XXX-wing) -- (YYY-bill);
\\ \hline
\end{tabular} 


\SbSbSSCT{Rayures}{Stripes}

\noindent

\begin{tabular}{|c|c|} \hline  
\tikz \duck \stripes ;
&  
\tikz \duck[stripes] ;
\\ \hline  
\BS{tikz} \BS{duck} \BSS{stripes} ; 
&  
\BS{tikz} \BS{}duck[\RDD{stripes}] ;
\\ \hline 
\end{tabular} 

\bigskip

\begin{tabular}{|c|c|} \hline  
\tikz \duck[rollingpin] \stripes ;
&  
\tikz  \duck[rollingpin,stripes] ;
\\ \hline  
\BS{tikz} \BS{duck}[rollingpin] \BS{stripes} ;
&  
\BS{tikz}  \BS{duck}[rollingpin,stripes] ;
\\ \hline 
\end{tabular} 



\bigskip

\begin{tabular}{|c|c|c|c|}\hline 
\multicolumn{4}{|c|}{\BS{tikz} \BS[duck] \BS{stripes}[\RDD{color}=red];}
\\ \hline  
\tikz \duck \stripes[color=red];
& 
\tikz \duck \stripes[distance=.5]; 
&  
\tikz \duck \stripes[width=.05];
&
\tikz \duck \stripes[height=1];  
\\ \hline 
[\RDD{color}=red] & [\RDD{distance}=.5] & [\RDD{width}=.05] & [\RDD{height}=1] 
\\ \hline 
\dft{black} & \dft{0.3}  & \dft{0.15}  &  \dft{2.7} \\ 
\hline  

\tikz \duck \stripes[rotate=45]; & \tikz \duck \stripes[initialx=1]; & \tikz \duck \stripes[initialy=1]; &  
\\ \hline 
[\RDD{rotate=}45] & [\RDD{initialx}=1] & [\RDD{initialy}=1] &
\\ \hline 
\dft{-10} & \dft{0.1} & \dft{-0.3} & 
\\ \hline 
\end{tabular} 


\bigskip

\begin{tabular}{|c|c|c|} \hline
\multicolumn{3}{|c|}{\BS{tikz} \BS[duck] \BS{stripes}[\RDD{emblem}=XXX];}
\\ \hline 
\tikz \duck \stripes[emblem={XXX}];
&  
\tikz \duck \stripes[emblem={\includegraphics[width=6mm]{LogoIUT}}];
&  
\tikz \duck \stripes[emblem={\DFR}];
\\ \hline  
[emblem=XXX]
& \parbox{5cm}{ [emblem=\{\BS{includegraphics}  [width=6mm]\AC{LogoIUT} \} ] }  
&  [emblem=\AC{\BS{DFR}}  ] 
\\ \hline 
& &  \BS{DFR} : \TFRGB{voir}{see} page \pageref{DFR} 
\\ \hline   
\end{tabular} 

\bigskip

\begin{tabular}{|c|c|c|} \hline
\tikz \duck[stripes={ \stripes \stripes[rotate=45]}] ;
\\ \hline 
\BS{tikz}
\BS{duck}[stripes=\AC{
\BS{stripes}
\BS{stripes}[rotate=45] } ]
;

\\ \hline  
\end{tabular}



%%
%%\newpage
%
%
%\newpage
%
%\section{A Voir}
%
%
%%\RRR{75-2 = Concept: Data Points and Data Formats}

\begin{tikzpicture}
\datavisualization [school book axes, visualize as smooth line]
data {
x, y
-1.5, 2.25
-1, 1
-.5, .25
0, 0
.5, .25
1, 1
1.5, 2.25
};
\end{tikzpicture}


\begin{tikzpicture}
\datavisualization [school book axes, visualize as smooth line]
data [format=function] {
var x : interval [-1.5:1.5] samples 7;
func y = \value x*\value x;
};
\end{tikzpicture}


\begin{tikzpicture}
\datavisualization [school book axes, visualize as smooth line]
data [format=function] {
var x : interval [-1.5:1.5] samples 3;
func y = \value x*\value x;
};
\end{tikzpicture}

Section 76 gives an in-depth coverage of the available data formats and explains how new data formats
can be defined.


\RRR{75-3 = Concept: Axes, Ticks, and Grids}


\begin{tikzpicture}
\datavisualization [
scientific axes,
x axis={length=3cm, ticks=few},
visualize as smooth line
]
data [format=function] {
var x : interval [-1.5:1.5] samples 20;
func y = \value x*\value x;
};
\end{tikzpicture}

\begin{tikzpicture}
\datavisualization [
scientific axes=clean,
x axis={length=3cm, ticks=few},
all axes={grid},
visualize as smooth line
]
data [format=function] {
var x : interval [-1.5:1.5] samples 7;
func y = \value x*\value x;
};
\end{tikzpicture}

Section 77 explains in more detail how axes, ticks, and grid lines can be chosen and configured.


\RRR {75-4 = Concept: Visualizers}

\begin{tikzpicture}
\datavisualization [
scientific axes=clean,
x axis={length=3cm, ticks=few},
visualize as smooth line
]
data [format=function] {
var x : interval [-1.5:1.5] samples 7;
func y = \value x*\value x;
};
\end{tikzpicture}

\begin{tikzpicture}
\datavisualization [
scientific axes=clean,
x axis={length=3cm, ticks=few},
visualize as scatter
]
data [format=function] {
var x : interval [-1.5:1.5] samples 7;
func y = \value x*\value x;
};
\end{tikzpicture}

Section 78 provides more information on visualizers as well as reference lists.

\RRR{75-5 = Concept: Style Sheets and Legends }

\begin{tikzpicture}[baseline]
\datavisualization [ scientific axes=clean,
y axis=grid,
visualize as smooth line/.list={sin,cos,tan},
style sheet=strong colors,
style sheet=vary dashing,
sin={label in legend={text=$\sin x$}},
cos={label in legend={text=$\cos x$}},
tan={label in legend={text=$\tan x$}},
data/format=function ]
data [set=sin] {
var x : interval [-0.5*pi:4];
func y = sin(\value x r);
}
data [set=cos] {
var x : interval [-0.5*pi:4];
func y = cos(\value x r);
}
data [set=tan] {
var x : interval [-0.3*pi:.3*pi];
func y = tan(\value x r);
};
\end{tikzpicture}


Section 79 details style sheets and legends.

\RRR{75-6 = Usage}

\subsection{/pgf/data/read from file=filename} (no default, initially empty)

If you set the source attribute to a non-empty hfilenamei, the data will be read from this file. In
this case, no hinline datai may be present, not even empty curly braces should be provided.
%\datavisualization ...
data [read from file=file1.csv]
data [read from file=file2.csv];
The other way round, if read from file is empty, the data must directly follow as hinline datai.
%\datavisualization ...
data {
x, y
1, 2
2, 3
};

The second important key is format, which is used to specify the data format:

\subsection{/pgf/data/format}

Use this key to locally set the format used for parsing the data, see Section 76 for a list of predefined
formats.

\tikz
\datavisualization [school book axes, visualize as line]
data [/data point/x=1] {
y
1
2
}
data [/data point/x=2] {
y
2
0
.5
};

\BS{datavisualization} . . . data point[options] . . . ;

\tikz \datavisualization [school book axes, visualize as line]
data point [x=1, y=1] data point [x=1, y=2]
data point [x=2, y=2] data point [x=2, y=0.5];

/tikz/data visualization/data point=options

\tikzdatavisualizationset{
horizontal/.style={
data point={x=#1, y=1}, data point={x=#1, y=2}},
}
\tikz \datavisualization
[ school book axes, visualize as line,
horizontal=1,
horizontal=2 ];

\BS{datavisualization} . . . data group[options]\AC{name}+=\AC{data specifications} . . . ;


\tikz \datavisualization data group {points} = {
data {
x, y
0, 1
1, 2
2, 2
5, 1
2, 0
1, 1
}
};

\tikz \datavisualization [school book axes, visualize as line] data group {points};
\qquad
\tikz \datavisualization [scientific axes=clean, visualize as line] data group {points};


\BS{datavisualization} . . . scope[options]{data specification} . . . ;

%\datavisualization...
%scope [/data point/experiment=7]
%{
%data [read from file=experiment007-part1.csv]
%data [read from file=experiment007-part2.csv]
%data [read from file=experiment007-part3.csv]
%}
%scope [/data point/experiment=23, format=foo]
%{
%data [read from file=experiment023-part1.foo]
%data [read from file=experiment023-part2.foo]
%};


\BS{datavisualization} . . . info[options]{code} . . . ;

\begin{tikzpicture}[baseline]
\datavisualization [ school book axes, visualize as line ]
data [format=function] {
var x : interval [-0.1*pi:2];
func y = sin(\value x r);
}
info {
\draw [red] (visualization cs: x={(.5*pi)}, y=1) circle [radius=1pt]
node [above,font=\footnotesize] {extremal point};
};
\end{tikzpicture}

\subsection{Coordinate system visualization}

\BS{datavisualization} . . . info’[options]{code} . . . ;

\begin{tikzpicture}[baseline]
\datavisualization [ school book axes, visualize as line ]
data [format=function] {
var x : interval [-0.1*pi:2];
func y = sin(\value x r);
}
info' {
\fill [red] (visualization cs: x={(.5*pi)}, y=1) circle [radius=2mm];
};
\end{tikzpicture}


\subsection{Predefined node data visualization bounding box}
This rectangle node stores a bounding box of the data visualization that is currently being constructed.
This node can be useful inside info commands or when labels need to be added.

\subsection{Predefined node data bounding box}
This rectangle node is similar to data visualization bounding box, but it keeps track only of the actual
data. The spaces needed for grid lines, ticks, axis labels, tick labels, and other all other information
that is not part of the actual data is not part of this box.


\RRR{75-7 = Advanced: Executing User Code During a Data Visualization}

\RRR{75-8 = Advanced: Creating New Objects}


\section{76 Providing Data for a Data Visualization}

%
%\begin{tikzpicture}
%\draw[help lines] (0,0) grid (2,2);
%\node[draw,fill=green!20,] (A) at (1,1) {\huge noeud};
%\fill[red] (A.-30) circle (3pt);
%\end{tikzpicture}
%
%
%
%\begin{tikzpicture}
%\draw[help lines] (0,0) grid (3,2);
%\draw (0,0) -- (1,1);
%\draw[red] (0,0) -- ([xshift=3pt] 1,1);
%\draw (1,0) -- +(30:2cm);
%\draw[red] (1,0) -- +([shift=(135:5pt)] 30:2cm);
%\end{tikzpicture}
%
%
%
%\section{Problèmes a Voir}
%
%%\tikz \node[jester,pattern=yellow,minimum size=1.5cm] at (0,0) {};
%
%\newpage
%
%
%
\begin{tabular}{|c|c|l c|}\hline 
\multicolumn{4}{|c|}{ \textbf{\TFRGB{module de base TikZ}{Basic TikZ package} : } }
\\ \hline

\TFRGB{nom}{name} & \TFRGB{A insérer dans le préambule}{Load package}& documentation \footnotemark[1] 	& \\  \hline 
tikz & \BS{usepackage}\AC{tikz}  	& pgfmanual.pdf			& \DGB \\

\hline 
\end{tabular} 

\bigskip

\begin{tabular}{|c|c|l c|}\hline 
\multicolumn{4}{|c|}{ \textbf{\TFRGB{Autres modules}{Other packages}} }
\\ \hline
\TFRGB{nom}{name} & \TFRGB{voir page}{see page} & documentation  \footnotemark[2] 	& \\  \hline 
animate 	& \pageref{anim} 	& animate.pdf 			& \DGB \\
tikz-optics 	& \pageref{optics} 	& tikz-optics.pdf 			& \DFR \\
pgfplots 	& \pageref{pgfplots} & pgfplots.pdf 		& \DGB \\
tikzpeople  & \pageref{people} 	& tikzpeople.pdf 		& \DGB \\
tikzducks  & \pageref{ducks} 	& tikzducks-doc.pdf 		& \DGB \\
tikzsymbols  & \pageref{symbol} 	& tikzsymbols.pdf 		& \DGB \\
tkz-tab  	& \pageref{tabl} 	& tkz-tab-screen.pdf 	& \DFR \\
\hline 
\end{tabular} 
\bigskip



\begin{tabular}{|l|c|l|}\hline 
\multicolumn{3}{|c|}{ \textbf{\TFRGB{Compléments optionnels}{Optional library} (documentation : pgfmanual.pdf)} }
\\ \hline
\TFRGB{nom}{name} 				& \TFRGB{voir page}{see page}						& \TFRGB{A insérer dans le préambule}{Load package}\\ \hline 
angles & \pageref{lib-angles} &  \BS{usetikzlibrary}\AC{angles}\\
arrows.meta	& \pageref{lib-arrows.meta}	&  \BS{usetikzlibrary}\AC{arrows.meta}\\
bending				& \pageref{lib-bending}			&  \BS{usetikzlibrary}\AC{bending}
\\
backgrounds			& \pageref{lib-bkgd} 			&  \BS{usetikzlibrary}\AC{backgrounds}
\\
calc				& \pageref{lib-calc}			&  \BS{usetikzlibrary}\AC{calc}
\\
chains			& \pageref{lib-chains} 			& \BS{usetikzlibrary}\AC{chains} 
\\
circuits.ee.IEC				& \pageref{lib-ee}			&  \BS{usetikzlibrary}\AC{circuits.ee.IEC}
\\
circuits.logic.IEC	& \pageref{lib-gate}			&  \BS{usetikzlibrary}\AC{circuits.logic.IEC}
\\ 
circuits.logic.US	& \pageref{lib-gate}			&  \BS{usetikzlibrary}\AC{circuits.logic.US}
\\ 
circuits.logic.CDH	& \pageref{lib-gate}			&  \BS{usetikzlibrary}\AC{circuits.logic.CDH}
\\ 
fit & \pageref{lib-fit} 	& \BS{usetikzlibrary}\AC{fit} 
\\
decorations.footprints & \pageref{lib-footprints} 	& \BS{usetikzlibrary}\AC{decorations.footprints} 
\\
decorations.fractals & \pageref{lib-fractals} 		& \BS{usetikzlibrary}\AC{decorations.fractals} 
\\
decorations.markings & \pageref{lib-mark} 			& \BS{usetikzlibrary}\AC{decorations.markings} 
\\
decorations.pathmorphing  & \pageref{lib-morph}		& \BS{usetikzlibrary}\AC{decorations.pathmorphing}
\\
decorations.pathreplacing & \pageref{lib-replac}	& \BS{usetikzlibrary}\AC{decorations.pathreplacing} 
\\
decorations.shapes & \pageref{lib-shapes} 			& \BS{usetikzlibrary}\AC{decorations.shapes} 
\\
decorations.text & \pageref{lib-text} 				& \BS{usetikzlibrary}\AC{decorations.text} 
\\
fadings 			& \pageref{lib-fadings}			&  \BS{usetikzlibrary}\AC{fadings }
\\
intersections		& \pageref{lib-intersections}	&  \BS{usetikzlibrary}\AC{intersections}
\\
matrix			& \pageref{lib-matrix} 			& \BS{usetikzlibrary}\AC{matrix} 
\\
patterns			& \pageref{lib-patterns}		&  \BS{usetikzlibrary}\AC{patterns}
\\
plotmarks			& \pageref{plotmarks} 			&  \BS{usetikzlibrary}\AC{plotmarks}
\\
positioning			& \pageref{lib-pos} 			&  \BS{usetikzlibrary}\AC{positioning}
\\ 
scopes				& \pageref{lib-scopes}			&  \BS{usetikzlibrary}\AC{scopes}
\\
shadings			& \pageref{lib-shadings}		&  \BS{usetikzlibrary}\AC{shadings}
\\
shapes.arrows		& \pageref{lib-arr}				&\BS{usetikzlibrary}\AC{shapes.arrows} 
\\shapes.callouts		& \pageref{lib-call}			& \BS{usetikzlibrary}\AC{shapes.callouts} 
\\
shapes.geometric	& \pageref{lib-geom} 			& \BS{usetikzlibrary}\AC{shapes.geometric}
\\

shapes.misc			& \pageref{lib-misc} 			& \BS{usetikzlibrary}\AC{shapes.misc} 
\\
shapes.multipart	& \pageref{lib-mult} 			& \BS{usetikzlibrary}\AC{shapes.multipart} 
\\
shapes.symbols		& \pageref{lib-symb}			& \BS{usetikzlibrary}\AC{shapes.symbols} 
\\
through				& \pageref{lib-through}			&  \BS{usetikzlibrary}\AC{through}
\\ 
trees				& \pageref{lib-trees}
\BS{usetikzlibrary}\AC{trees}
\\ 
through				& \pageref{lib-turtle}			&  \BS{usetikzlibrary}\AC{turtle}
\\ 
\hline
 \end{tabular} 

\TFRGB{ 
\footnotetext[1]{voir dans le répertoire :  \BS{texlive}\BS{2016}\BS{tesmf-dist}\BS{doc}\BS{generic}\BS{pgf}}
\footnotetext[2]{chercher  dans le répertoire  :  \BS{texlive}\BS{2016}\BS{tesmf-dist}\BS{doc}\BS{latex}} }
{ 
\footnotetext[1]{look in repertory :  \BS{texlive}\BS{2016}\BS{tesmf-dist}\BS{doc}\BS{generic}\BS{pgf}}
\footnotetext[2]{search in repertory :  \BS{texlive}\BS{2016}\BS{tesmf-dist}\BS{doc}\BS{latex}} }

\bigskip



\begin{tabular}{|l|c|}\hline
\multicolumn{2}{|c|}{ \TFRGB{dans une prochaine mise à jour}{In a a future update } }
\\ \hline
automata			& \RRR{41} \\
babel				& \RRR{42} \\
calendar			& \RRR{45} \\
%chains				& \RRR{46} \\ 
%circuits.ee		& \RRR{47-4} \\ 
 
circular graph drawing library 				& \RRR{32} \\
curvilinear library 						& \RRR{103-4-7} \\
datavisualization library					& \RRR{75} \\
datavisualization.formats.functions library & \RRR{76-4} \\
datavisualization.polar library 			& \RRR{80}  \\
 er 										& \RRR{49}  \\
examples graph drawing library 				& \RRR{35-8} \\ 
external 									& \RRR{50}  \\  
%fit 										& \RRR{52} \\ 
fixedpointarithmetic 						& \RRR{53} \\ 
folding 									& \RRR{59} \\
force graph drawing library 				& \RRR{31}  \\
fpu											& \RRR{54}  \\
graph.standard library 						& \RRR{19-10}\\
graphdrawing library 						& \RRR{27} \\
graphs library 								& \RRR{19} \\ 
layered graph drawing library 				& \RRR{30}  \\
lindenmayersystems							& \RRR{55} \\  
mindmap										& \RRR{58} \\ 
petri										& \RRR{61}  \\ 
phylogenetics graph drawing library 		& \RRR{33} \\
plothandlers								& \RRR{62}  \\  
profiler									& \RRR{64}   \\ 
quotes library 								& \RRR{17-10-4} \\
routing graph drawing library 				& \RRR{34} \\
shadows										& \RRR{66}   \\ 
 
spy											&  \RRR{68} \\ 
svg.path									&  \RRR{69} \\ 
%through										&  \RRR{71} \\ 
topaths										&  \RRR{70} \\ 
trees graph drawing library					& 
\\ \hline
\end{tabular}  


%\newpage
%%
%% \tableofcontents
%\renewcommand{\bibname}{Sources}
%
\label{sources}
%\input{bib}

\newpage

\begin{thebibliography}{99}
\bibitem{pgfmanual} pgfmanual.pdf  	\hspace{1cm}	version 3.0.1a \hspace{1cm} 	1161 pages 	\hspace{1cm}	\DGB
\bibitem{pgfplots} pgfplots.pdf 	\hspace{1cm}	version 1.80 \hspace{1cm} 	439 pages 	\hspace{1cm}	\DGB
\bibitem{tikstab} tkz-tab-screen.pdf 	\hspace{1cm}	version 1.1c \hspace{1cm} 	83 pages 	\hspace{1cm}	\DFR
\bibitem{tikzpeople} tikzpeople.pdf 	\hspace{1cm}	 \hspace{1cm} 	19 pages 	\hspace{1cm}	\DGB

\bibitem{tikzducks} tikzducks-doc.pdf 	\hspace{1cm}	version 0.6  \hspace{1cm} 	28 pages 	\hspace{1cm}	\DGB

\bibitem{tikzsymbols} tikzsymbols.pdf 	\hspace{1cm}	version sept 2017  \hspace{1cm} 	15 pages 	\hspace{1cm}	\DGB

\bibitem{animate} animate.pdf 	\hspace{1cm}	 \hspace{1cm} 	26 pages 	\hspace{1cm}	\DGB

\bibitem{optics} tikz-optics.pdf	\hspace{1cm}	version 0.2.2  \hspace{1cm} 	39 pages 	\hspace{1cm}	\DFR
\end{thebibliography}


%
%\newpage 
%
%%\printindex 
  
\end{document}:'
\end{verbatim}

Na ta način se najprej definira ukaz \verb|\blackandwhite|, potem pa 
se prevede še izbrana datoteka in v njej se prevedejo deli dokumenta v
črno belem načinu. Če bi namesto tega uporabili 
\begin{verbatim}
latex '\newcommand{\crnobelo}{false}

%English version
\newcommand{\TFRGB}[2]{#2} % #1 en francais #2 in english

\newcommand{\dft}{By default}
\newcommand{\SSCT}[2]{ \section{#2}}
\newcommand{\SbSSCT}[2]{ \subsection{#2}}
\newcommand{\Par}[2]{ \paragraph{#2}}
\newcommand{\SbSbSSCT}[2]{ \subsubsection{#2}}
\newcommand{\SSCTTC}[4]{ \section[#2]{#2}}
\newcommand{\SbSSCTTC}[4]{ \subsection[#2]{#2}}
\newcommand{\SbSbSSCTTC}[4]{ \subsubsection[#3]{#4}}

\newcommand{\maboite}[1]{\begin{center}  \tikz \draw node[draw,fill=yellow!20,inner sep=0.2cm,text centered,text width=.75\linewidth] {Load package : #1} ; \end{center} }

%\newcommand{\DW}[1]{\begin{center}  \tikz \draw node[draw,fill=red!50,inner sep=0.1cm,text centered] {\tiny don't work !} ; \end{center} }

\newcommand{\DW}[1]{\tikz[baseline=-1mm]  \draw node[draw,fill=red!50] {{\tiny don't work !}}; \index{\textbf{6 list of don't work }}} % for english version
%% Version française

\newcommand{\TFRGB}[2]{#1} % #1 en francais #2 in english
\newcommand{\dft}{Par défaut : }
\newcommand{\SSCT}[2]{\section{#1}}
\newcommand{\SbSSCT}[2]{\subsection{#1}}
\newcommand{\Par}[2]{ \paragraph{#1}}
\newcommand{\SbSbSSCT}[2]{\subsubsection{#1}}
\newcommand{\SSCTTC}[4]{\section[#1]{#2}}
\newcommand{\SbSSCTTC}[4]{\subsection[#1]{#2}}
\newcommand{\SbSbSSCTTC}[4]{\subsubsection[#1]{#2}}



\newcommand{\maboite}[1]{\begin{center}  \tikz \draw node[draw,fill=yellow!20,inner sep=0.2cm,text centered,text width=.75\linewidth] {Charger l'extension: #1} ; \end{center} }


\newcommand{\DW}[1]{\tikz[baseline=-1mm]  \draw node[draw,fill=red!50] {{\tiny ne fonctionne pas !}}; \index{\textbf{6 liste des non-fonctionnels }}} % pour la version française

 \documentclass[a4paper,10pt]{article}

 \usepackage{fontspec}
\usepackage[french,english]{babel}

%\TFRGB{\selectlanguage{french}}{\selectlanguage{english}}
\usepackage{tikzpeople}

\usepackage{amsmath,amsfonts,amssymb}

\usepackage{pdfpages}  


%\usepackage{pst-all}

\usepackage{graphicx} 
\usepackage{hyperref}

\usepackage{animate}
\usepackage{makeidx}
%\usepackage{wrapfig}
%\usepackage{tikz-dependency}
\usepackage{pgfplots} %<<<<<<<<<<<<<<<<<<<<<<<<<<<<< 
\usepackage{tikz}
\usepackage{tkz-tab}

 
%\usepgflibrary{shapes.callouts}
\usepackage{tikz-qtree}
\usepackage{tkz-tab}
\usepackage{csquotes}

  
\usetikzlibrary{angles}
\usetikzlibrary{arrows}

\usetikzlibrary{shadings}
\usetikzlibrary{calc}
\usetikzlibrary{backgrounds}
\usetikzlibrary{decorations.pathmorphing}

\usetikzlibrary{decorations.markings}
\usetikzlibrary{decorations.footprints}
\usetikzlibrary{decorations.shapes}
\usetikzlibrary{decorations.text}
\usetikzlibrary{decorations.fractals}
\usepgflibrary{shapes.geometric}
\usetikzlibrary{intersections}
\usetikzlibrary{scopes}
\usetikzlibrary{shapes.symbols}
\usetikzlibrary{shapes.arrows}
\usetikzlibrary{shapes.callouts}
\usetikzlibrary{shapes.misc}
\usepgflibrary{shapes.multipart}
\usetikzlibrary{plotmarks}
\usetikzlibrary{trees}
\usetikzlibrary{fadings}
\usetikzlibrary{arrows.meta}
\usetikzlibrary{bending}
\usetikzlibrary{fit}
%\usetikzlibrary{circuits}
\usetikzlibrary{circuits.ee.IEC}
\usetikzlibrary{circuits.logic.IEC}
\usetikzlibrary[circuits.logic.US]
\usetikzlibrary{circuits.logic.CDH}
%\usetikzlibrary{decorations}
\usetikzlibrary{shapes.gates.logic.IEC}
\usetikzlibrary{matrix}
\usetikzlibrary{chains}
%\usetikzlibrary{circuit.plc.sfc}
\usepackage{tikzsymbols}
\usetikzlibrary{datavisualization}
\usetikzlibrary{datavisualization.formats.functions}
%
\usepackage{tikzducks}

\usepackage{tikzrput}
\usepackage{pgfornament}

%\usetikzlibrary{babel}
\usetikzlibrary{math}
\usetikzlibrary{optics}
\usetikzlibrary{through}
\usetikzlibrary{turtle}
\usetikzlibrary{quotes}


\pgfplotsset{compat=1.8}
\usetikzlibrary{positioning}

\usepackage{geometry}
\geometry{a4paper,top={3cm}}

\usepackage{ifpdf}
\usepackage{ifluatex}
\usetikzlibrary{spy}




%====================================================================

\makeindex

\newcommand{\AC}[1]{\{#1\}}

\newcommand{\BS}[1]{$\backslash$#1}

\newcommand{\BSB}[1]{\textbf{\color{blue} {$\backslash$#1}}}


\newcommand{\BSR}[1]{\textbf{\color{red}  $\backslash$#1}}

%\newcommand{\RDDX}[2]{{\color{red}#1} \index{\textbf{3 Paramètres et options}!#2=#1}}


\newcommand{\RRR}[1]{\tikz[baseline=-1mm,inner sep=2pt]  \draw node[draw,fill=red!20] {{\footnotesize  PGFmanual section :  #1}} ; }

\newcommand{\RRP}[1]{\tikz[baseline=-1mm]  \draw node[draw,fill=red!20] {{\footnotesize  pgfplots section :  #1}} ; }

%\newcommand{\RRR}[1]{\tikz[baseline=-1mm]  \draw node[draw,fill=red!20] {{\footnotesize  PGFmanual section :  #1}} ;\index{\textbf{5 PGFmanual }!#1} }

\newcommand{\DFR}{ \tikzpicture[scale=.25]
\draw [fill=blue](0,0) rectangle (3,1.5);
\draw [fill=white](1,0) rectangle (2,1.5);
\draw [fill=red](2,0) rectangle (3,1.5);\endtikzpicture }

\newcommand{\DGE}{ \tikzpicture[scale=.25]
\draw [fill=yellow](0,0) rectangle (3,.5);
\draw [fill=red]((0,.5) rectangle (3,1);
\draw [fill=black](0,1) rectangle (3,1.5); \endtikzpicture }

\newcommand{\DGB}{ \tikzpicture[scale=.25]
\draw [fill=blue](0,0) rectangle (3,1.5);
\draw [white,line width=.1cm](0,0) -- (3,1.5);
\draw [white,line width=.1cm](0,1.5) -- (3,0);
\draw [white,line width=.1cm](1.5,0) -- (1.5,1.5);
\draw [white,line width=.1cm](0,0.75) -- (3,0.75);
\draw [red,line width=.05cm](0,0) -- (3,1.5);
\draw [red,line width=.05cm](0,1.5) -- (3,0);
\draw [red,line width=.05cm](1.5,0) -- (1.5,1.5);
\draw [red,line width=.05cm](0,0.75) -- (3,0.75);
\endtikzpicture }



\TFRGB{
\newcommand{\ESS}[1]{\textbf{\textbackslash begin\AC{#1}}\index{\textbf{1 Environnements}!#1}}

\newcommand{\BSS}[1]{\textbf{\textbackslash{#1}}\index{\textbf{2 Commandes}!#1 @\textbackslash{}#1}}


\newcommand{\DDD}[1]{{\color{red}  #1}\index{\textbf{3 Paramètres et options}!#1}}
\newcommand{\RDD}[1]{{\color{red}  #1}\index{\textbf{3 Paramètres et options}!#1}}
%4
\newcommand{\BDD}[1]{{\color{blue}  #1}\index{\textbf{4 Valeurs Tikz}!#1}}

\newcommand{\RDDX}[2]{{\color{red}#1} \index{\textbf{4 Valeurs Tikz}!#1 (#2)}}
%5
\newcommand{\FDD}[1]{{\color{red}  #1}\index{\textbf{5 Extrémit\'es}!#1}}
}
{
\newcommand{\ESS}[1]{\textbf{\textbackslash{#1}}\index{\textbf{1 Environments}!#1 @\textbackslash{}#1}}
%2
\newcommand{\BSS}[1]{\textbf{\textbackslash{#1}}\index{\textbf{2 Commands}!#1 @\textbackslash{}#1}}
%3
\newcommand{\RDD}[1]{{\color{red}  #1}\index{\textbf{3 Parameters and options}!#1}}

\newcommand{\DDD}[1]{{\color{red}  #1}\index{\textbf{3 Parameters and options}!#1}}

\newcommand{\RDDX}[2]{{\color{red}#1} \index{\textbf{4 Values Tikz}!#1 (#2)}}
%
\newcommand{\BDD}[1]{{\color{blue}  #1}\index{\textbf{4 Values Tikz}!#1}}

\newcommand{\FDD}[1]{{\color{red}  #1}\index{\textbf{5 Extremities}!#1}}
}

\newcommand{\rouge}[1] {{\color{red}  #1}}
\newcommand{\blll}[1] {{\color{blue}  #1}}



 \begin{document}


%\selectlanguage{english}
\selectlanguage{french}



%  
\author{{\Huge Jean Pierre Casteleyn } \\ {\Huge IUT Génie Thermique et \'Energie } \\ {\Huge Dunkerque, France }}

\DeclareFixedFont{\RM}{T1}{ptm}{b}{n}{2cm}

\DeclareFixedFont{\RMM}{T1}{ptm}{b}{n}{1cm}

\title{ {\RM Visual TikZ} \\ \vspace{1cm} {\RMM Version 0.66} }



\date{
\begin{center}
\begin{animateinline}[loop,autoplay]{12}%
 \multiframe{24}{iAngle=0+15,icol=0+5}{\begin{tikzpicture}[rotate=90]
    \draw  (0,0) node[fill=white,circle] {\includegraphics[width=4cm]{LogoIUT}}  (0,0) circle (1);
  \end{tikzpicture}} 
\end{animateinline}% 
\end{center}
{\LARGE \TFRGB{mis à jour le \today}{Updated on \today} 
}
}


\maketitle



 \begin{animateinline}[autoplay,loop]{12}%
 \multiframe{24}{iAngle=0+15,icol=0+5}{\begin{tikzpicture}
 [scale=1.8] %
   \draw[line width=0pt] (-2,-2) rectangle(6,2); %
   \draw  (0,0) node[fill=white,circle,rotate=\iAngle] {\includegraphics[width=2cm]{LogoIUT}}  (0,0) circle (1);
    \draw (0,0) circle (1);
    \coordinate (abc) at (${sqrt(9-sin(\iAngle)*sin(\iAngle))+cos(\iAngle)}*(1,0)$) ;
    \coordinate (xyz) at (\iAngle:1);
    \draw[ultra thick] (0,0) --(xyz); 
    \draw[ultra thick] (xyz) -- (abc) ;
    \fill[color=blue!\icol] (abc)++(0.5,-1) rectangle (5,1) ;
    \draw[ultra thick] (abc) ++(0,-1) rectangle ++(.5,2) ;
    \draw[ultra thick]  (1.5,1) -- (5,1) -- (5,-1) -- (1.5,-1);
    \fill[red] (xyz) circle (4pt);
    \fill[red] (abc) circle (4pt); 
  \end{tikzpicture}}
 \end{animateinline} 


 
\newpage
 
 
\TFRGB{
\textbf{Objectifs }: 

\begin{itemize}
\item Avoir une image par  commande ou par paramètre.
\item Avoir un texte réduit au strict minimum.
\item Etre le plus complet possible au fil de mises à jour régulières.
\item Garder la même structure que visuel pstricks
\end{itemize} 
}
{\textbf{Objectives }: 

\begin{itemize}
\item One image per command or parameter.
\item the minimum amount of text possible.
\item the most complete possible update after update.
\item keep the same structure as VisualPSTricks
\end{itemize}}


\vspace{1cm}

\TFRGB{
\textbf{Remarques }: Le code donné est minimal et ne sert qu'à montrer les commandes concernées. Les effets sont parfois exagérés pour bien les mettre en évidence. Pour en savoir plus, vous pouvez voir la documentation. Pour se faire j'ai indiqué le numéro de \tikz[baseline=-1mm]  \draw node[draw,fill=red!20] {Section de pgfmanual} ;
}
{\textbf{Remarks }:
Minimal code is given to show the effect of a command or a parameter. The effects are sometime exaggerated for clarity   .To consult the documentation, I have given the number of the  \tikz[baseline=-1mm]  \draw node[draw,fill=red!20] {Section in pgfmanual} ;
}
\vspace{1cm}


\TFRGB{
\textbf{Vous pouvez me contacter à}
 \href{mailto:jpcdk@yahoo.fr}{mon e-mail personnel} pour

\begin{itemize}
\item me signaler les erreurs que vous avez constatés (merci d'indiquer la page où vous l'avez constaté)
\item me faire part de vos commentaires, suggestions \dots
\end{itemize}}
{
\textbf{You can contact me at }
 \href{mailto:jpcdk@yahoo.fr}{my personal email} to

\begin{itemize}
\item let me know the mistakes found (please indicate the page)
\item give me your commentaries, your suggestions \dots
\end{itemize}}

\vspace{1cm}
\TFRGB{
\textbf{Quoi de neuf ! } :

\begin{itemize}
\item Ajout de la library  chains \pageref{lib-chains}
\item Ajout de la library  through \pageref{lib-through}
\item Ajout de la library  turtle \pageref{lib-turtle}
\item Ajout de la library positioning \pageref{lib-pos}
\item Ajout du module tikzsymbols \pageref{symbol}
\item mise à jour du module tikzducks \pageref{ducks}
\item mise à jour des modules shape \pageref{formes}
\end{itemize}

}
{
\textbf{What's new } :
\begin{itemize}
\item chains library added \pageref{lib-chains}
\item through library added \pageref{lib-through}
\item turtle library added \pageref{lib-turtle}
\item positioning library added \pageref{lib-pos}
\item Tikzsymbols package added \pageref{symbol}
\item Tikzducks package updated \pageref{ducks}
\item shapes packages updated \pageref{formes}
\end{itemize}
}



\vspace{1cm}
\textbf{Licence } :


This work may be distributed and/or modified under the conditions of the LaTeX Project Public License, either version 1.3 of this license or (at your option) any later version.

 The latest version of this license is in  http://www.latex-project.org/lppl.txt and version 1.3 or later is part of all distributions of LaTeX
version 2005/12/01 or later.

This work has the LPPL maintenance status `maintained'.

The Current Maintainer of this work is M. Jean Pierre Casteleyn.

\vspace{2cm}
\textbf{\TFRGB{Merci à }{Thanks to}}:

Till Tantau  ,
Alain Matthes ,
Jim Diamond ,
Falk Rühl ,
Axel Kielhorn ,
Nils Fleischhacker ,
Michel Fruchart ,
Ben Vitecek
 
\newpage


% \tableofcontents
%
%
%\newpage




\RRR{75-2 = Concept: Data Points and Data Formats}

\begin{tikzpicture}
\datavisualization [school book axes, visualize as smooth line]
data {
x, y
-1.5, 2.25
-1, 1
-.5, .25
0, 0
.5, .25
1, 1
1.5, 2.25
};
\end{tikzpicture}


\begin{tikzpicture}
\datavisualization [school book axes, visualize as smooth line]
data [format=function] {
var x : interval [-1.5:1.5] samples 7;
func y = \value x*\value x;
};
\end{tikzpicture}


\begin{tikzpicture}
\datavisualization [school book axes, visualize as smooth line]
data [format=function] {
var x : interval [-1.5:1.5] samples 3;
func y = \value x*\value x;
};
\end{tikzpicture}

Section 76 gives an in-depth coverage of the available data formats and explains how new data formats
can be defined.


\RRR{75-3 = Concept: Axes, Ticks, and Grids}


\begin{tikzpicture}
\datavisualization [
scientific axes,
x axis={length=3cm, ticks=few},
visualize as smooth line
]
data [format=function] {
var x : interval [-1.5:1.5] samples 20;
func y = \value x*\value x;
};
\end{tikzpicture}

\begin{tikzpicture}
\datavisualization [
scientific axes=clean,
x axis={length=3cm, ticks=few},
all axes={grid},
visualize as smooth line
]
data [format=function] {
var x : interval [-1.5:1.5] samples 7;
func y = \value x*\value x;
};
\end{tikzpicture}

Section 77 explains in more detail how axes, ticks, and grid lines can be chosen and configured.


\RRR {75-4 = Concept: Visualizers}

\begin{tikzpicture}
\datavisualization [
scientific axes=clean,
x axis={length=3cm, ticks=few},
visualize as smooth line
]
data [format=function] {
var x : interval [-1.5:1.5] samples 7;
func y = \value x*\value x;
};
\end{tikzpicture}

\begin{tikzpicture}
\datavisualization [
scientific axes=clean,
x axis={length=3cm, ticks=few},
visualize as scatter
]
data [format=function] {
var x : interval [-1.5:1.5] samples 7;
func y = \value x*\value x;
};
\end{tikzpicture}

Section 78 provides more information on visualizers as well as reference lists.

\RRR{75-5 = Concept: Style Sheets and Legends }

\begin{tikzpicture}[baseline]
\datavisualization [ scientific axes=clean,
y axis=grid,
visualize as smooth line/.list={sin,cos,tan},
style sheet=strong colors,
style sheet=vary dashing,
sin={label in legend={text=$\sin x$}},
cos={label in legend={text=$\cos x$}},
tan={label in legend={text=$\tan x$}},
data/format=function ]
data [set=sin] {
var x : interval [-0.5*pi:4];
func y = sin(\value x r);
}
data [set=cos] {
var x : interval [-0.5*pi:4];
func y = cos(\value x r);
}
data [set=tan] {
var x : interval [-0.3*pi:.3*pi];
func y = tan(\value x r);
};
\end{tikzpicture}


Section 79 details style sheets and legends.

\RRR{75-6 = Usage}

\subsection{/pgf/data/read from file=filename} (no default, initially empty)

If you set the source attribute to a non-empty hfilenamei, the data will be read from this file. In
this case, no hinline datai may be present, not even empty curly braces should be provided.
%\datavisualization ...
data [read from file=file1.csv]
data [read from file=file2.csv];
The other way round, if read from file is empty, the data must directly follow as hinline datai.
%\datavisualization ...
data {
x, y
1, 2
2, 3
};

The second important key is format, which is used to specify the data format:

\subsection{/pgf/data/format}

Use this key to locally set the format used for parsing the data, see Section 76 for a list of predefined
formats.

\tikz
\datavisualization [school book axes, visualize as line]
data [/data point/x=1] {
y
1
2
}
data [/data point/x=2] {
y
2
0
.5
};

\BS{datavisualization} . . . data point[options] . . . ;

\tikz \datavisualization [school book axes, visualize as line]
data point [x=1, y=1] data point [x=1, y=2]
data point [x=2, y=2] data point [x=2, y=0.5];

/tikz/data visualization/data point=options

\tikzdatavisualizationset{
horizontal/.style={
data point={x=#1, y=1}, data point={x=#1, y=2}},
}
\tikz \datavisualization
[ school book axes, visualize as line,
horizontal=1,
horizontal=2 ];

\BS{datavisualization} . . . data group[options]\AC{name}+=\AC{data specifications} . . . ;


\tikz \datavisualization data group {points} = {
data {
x, y
0, 1
1, 2
2, 2
5, 1
2, 0
1, 1
}
};

\tikz \datavisualization [school book axes, visualize as line] data group {points};
\qquad
\tikz \datavisualization [scientific axes=clean, visualize as line] data group {points};


\BS{datavisualization} . . . scope[options]{data specification} . . . ;

%\datavisualization...
%scope [/data point/experiment=7]
%{
%data [read from file=experiment007-part1.csv]
%data [read from file=experiment007-part2.csv]
%data [read from file=experiment007-part3.csv]
%}
%scope [/data point/experiment=23, format=foo]
%{
%data [read from file=experiment023-part1.foo]
%data [read from file=experiment023-part2.foo]
%};


\BS{datavisualization} . . . info[options]{code} . . . ;

\begin{tikzpicture}[baseline]
\datavisualization [ school book axes, visualize as line ]
data [format=function] {
var x : interval [-0.1*pi:2];
func y = sin(\value x r);
}
info {
\draw [red] (visualization cs: x={(.5*pi)}, y=1) circle [radius=1pt]
node [above,font=\footnotesize] {extremal point};
};
\end{tikzpicture}

\subsection{Coordinate system visualization}

\BS{datavisualization} . . . info’[options]{code} . . . ;

\begin{tikzpicture}[baseline]
\datavisualization [ school book axes, visualize as line ]
data [format=function] {
var x : interval [-0.1*pi:2];
func y = sin(\value x r);
}
info' {
\fill [red] (visualization cs: x={(.5*pi)}, y=1) circle [radius=2mm];
};
\end{tikzpicture}


\subsection{Predefined node data visualization bounding box}
This rectangle node stores a bounding box of the data visualization that is currently being constructed.
This node can be useful inside info commands or when labels need to be added.

\subsection{Predefined node data bounding box}
This rectangle node is similar to data visualization bounding box, but it keeps track only of the actual
data. The spaces needed for grid lines, ticks, axis labels, tick labels, and other all other information
that is not part of the actual data is not part of this box.


\RRR{75-7 = Advanced: Executing User Code During a Data Visualization}

\RRR{75-8 = Advanced: Creating New Objects}


\section{76 Providing Data for a Data Visualization}


%
%\newpage
%
%\RRR{17-2-1} fini
%
%
%/tikz/node font=font commands
%
%\begin{tikzpicture}
%\draw[node font=\itshape] (1,0) -- +(1,1) node[above] {italic};
%\end{tikzpicture}
%
%\tikz \node [node font=\tiny, minimum height=3em, draw] {tiny};
%\tikz \node [node font=\small, minimum height=3em, draw] {small};
%
%
%
%
%/tikz/node align header=
%
%
%\RRR{17-4-4} OK
%
%
%\RRR{17-5} Positioning Nodes
%
%
%
%\RRR{17-5-3 }Advanced Placement Options
%
%
%  
%
%\subsubsection{title}
%
%\begin{tikzpicture}[every node/.style={draw}]
%\draw[help lines](0,0) grid (3,2);
%\draw (1,0) node{A}
%(2,0) node[rotate=90,scale=1.5] {B};
%\draw[rotate=30] (1,0) node{A}
%(2,0) node[rotate=90,scale=1.5] {B};
%\draw[rotate=60] (1,0) node[transform shape] {A}
%(2,0) node[transform shape,rotate=90,scale=1.5] {B};
%\end{tikzpicture}
%
%m:::::::::::::::%\begin{tikzpicture}
%%% Install a nonlinear transformation:
%%\pgfsetcurvilinearbeziercurve
%%{\pgfpoint{0mm}{20mm}}
%%{\pgfpoint{10mm}{20mm}}
%%{\pgfpoint{10mm}{10mm}}
%%{\pgfpoint{20mm}{10mm}}
%%\pgftransformnonlinear{\pgfpointcurvilinearbezierorthogonal\pgf@x\pgf@y}%
%%% Draw something:
%%\draw [help lines] (0,-30pt) grid [step=10pt] (80pt,30pt);
%%\foreach \x in {0,20,...,80}
%%\node [fill=red!20] at (\x pt, -20pt) {\x};
%%\foreach \x in {0,20,...,80}
%%\node [fill=blue!20, transform shape nonlinear] at (\x pt, 20pt) {\x};
%%\end{tikzpicture}
%
%
%\newpage
%
%
%
%
\SbSSCT{Coordonnées}{Coordinates}
\begin{center}
\RRR{13-2-1}
\end{center}


\SbSbSSCT{Système de coordonnées \og canvas \fg}{Canvas coordinates}

\noindent


\tikzset{every picture/.style=blue,very thick,inner sep=0pt}

\begin{tabular}{|c|c|} \hline 
\TFRGB{Explicite}{explicit}  & \TFRGB{Implicite}{implicit}
\\ \hline
\begin{tikzpicture}
\draw[help lines] (0,0) grid (3,2);
\fill (canvas cs:x=2cm,y=1.5cm) circle (2pt);
\end{tikzpicture}
&
\begin{tikzpicture}
\draw[help lines] (0,0) grid (3,2);
\fill (2,1.5) circle (2pt);
\end{tikzpicture}

\\ \hline  
 \BS{fill} (\RDD{canvas cs}:\blll{x=2cm,y=1.5cm}) circle (2pt);
& \BS{fill} {\color{blue}(2cm,1.5cm)} circle (2pt);
\\ \hline 
\end{tabular} 


\SbSbSSCT{Système de coordonnées polaire \og canvas \fg}{Polar coordinates}

\noindent


\begin{tabular}{|c|c|c|} \hline
\TFRGB{Explicite}{explicit}  & \TFRGB{Implicite}{implicit}
\\ \hline
\begin{tikzpicture}
\draw[help lines] (0,0) grid (3,2);
\draw [dotted](0,2) arc (90 :0 :2);
\draw [dotted](0,0) --(2,2);
\fill (canvas polar cs:angle=45,radius=2cm) circle (2pt);
\end{tikzpicture}
&
\begin{tikzpicture}
\draw[help lines] (0,0) grid (3,2);
\draw [dotted](0,2) arc (90 :0 :2);
\draw [dotted](0,0) --(2,2);
\fill (45:2cm) circle (2pt);
\end{tikzpicture}
\\ \hline 
\BS{fill} (\RDD{canvas polar cs}:\RDD{angle}=45,\RDD{radius}=2cm) circle (2pt);
&
\BS{fill} {\color{blue}(45:2cm)} circle (2pt);
\\ \hline 
\end{tabular} 

\bigskip
\begin{tabular}{|c|} \hline  
\begin{tikzpicture}
\draw[help lines] (0,0) grid (3,2);
\draw [dotted](0,2) arc (90 :0 :3 and 2);
\draw [dotted](0,0) --(3,2);
\fill (canvas polar cs:angle=45,x radius=3cm,y radius=2cm) circle (2pt);
\end{tikzpicture}
\\ \hline  
\BS{fill} (canvas polar cs:angle=45,\RDD{x radius}=3cm,\RDD{y radius}=2cm) circle (2pt);
\\ \hline 
\end{tabular}


\SbSbSSCT{Système de coordonnées  xyz}{xyz coordinates}

\noindent


\begin{tabular}{|c|c|c|} \hline 
\begin{tikzpicture}[->]
\draw (0,0) -- (xyz cs:x=1);
\draw[red] (0,0) -- (xyz cs:y=1);
\draw[magenta] (0,0) -- (xyz cs:z=1);
\end{tikzpicture}
&
\begin{tikzpicture}[->]
\draw (0,0) -- (1,0,0);
\draw[red]  (0,0) -- (0,1,0);
\draw[magenta]  (0,0) -- (0,0,1);
\end{tikzpicture}
\\ \hline 
\BS{draw} (0,0) - - (\RDD{xyz cs}:x=1); & \BS{draw}  (0,0) - - (1,0,0); \\
\BS{draw}[red]  (0,0) - - (\RDD{xyz cs}:y=1); &  \BS{draw}[red] (0,0) - - (0,1,0); \\
\BS{draw}[magenta]  (0,0) - - (\RDD{xyz cs}:z=1); &  \BS{draw}[magenta]   (0,0) - - (0,0,1); 
\\ \hline 

\end{tabular} 

 
\newpage

\SbSbSSCT{Coordinate system xyz polar}{Coordinate system xyz polar}

\noindent

\begin{tabular}{|c|c|c|} \hline
\TFRGB{Explicite}{explicit}  & \TFRGB{Implicite}{implicit}
\\ \hline
\begin{tikzpicture}
\draw[help lines] (0,0) grid (3,2);
\draw [dotted](0,2) arc (90 :0 :2);
\draw [dotted](0,0) --(2,2);
\fill (xyz polar cs:angle=45,radius=2) circle (2pt);
\end{tikzpicture}
&
\begin{tikzpicture}
\draw[help lines] (0,0) grid (3,2);
\draw [dotted](0,2) arc (90 :0 :2);
\draw [dotted](0,0) --(2,2);
\fill (45:2) circle (2pt);
\end{tikzpicture}
\\ \hline 
\BS{fill} (\RDD{xyz polar cs}:\RDD{angle}=45,\RDD{radius}=2) circle (2pt);
&
\BS{fill} {\color{blue}(45:2cm)} circle (2pt);
\\ \hline 
\end{tabular} 

\bigskip
\begin{tabular}{|c|} \hline  
\begin{tikzpicture}
\draw[help lines] (0,0) grid (3,2);
\draw [dotted](0,2) arc (90 :0 :3 and 2);
\draw [dotted](0,0) --(3,2);
\fill (xyz polar cs:angle=45,x radius=3,y radius=2) circle (2pt);
\end{tikzpicture}
\\ \hline  
\BS{fill} (xyz polar cs:angle=45,\RDD{x radius}=3,\RDD{y radius}=2) circle (2pt);
\\ \hline 
\end{tabular} 

\bigskip

\begin{tabular}{|c|c|c|} \hline
\multicolumn{2}{|c|}{\BS{begin}\AC{tikzpicture}{\color{red}[x=1.5cm,y=1cm]} }
\\ \hline
\begin{tikzpicture}[x=1.5cm,y=1cm]
\draw[help lines] (0,0) grid (3,2);
\draw [dotted](0,2) arc (90 :0 :2);
\draw [dotted](0,0) --(2,2);
\fill (xyz polar cs:angle=45,radius=2) circle (2pt);
\end{tikzpicture}
&
\begin{tikzpicture}[x=1.5cm,y=1cm]
\draw[help lines] (0,0) grid (3,2);
\draw [dotted](0,2) arc (90 :0 :2);
\draw [dotted](0,0) --(2,2);
\fill (45:2) circle (2pt);
\end{tikzpicture}
\\ \hline 
\BS{fill} (\RDD{xyz polar cs}:\RDD{angle}=45,\RDD{radius}=2) circle (2pt);
&
\BS{fill} {\color{blue}(45:2cm)} circle (2pt);
\\ \hline 
\end{tabular} 
\bigskip

\begin{tabular}{|c|c|c|} \hline
\multicolumn{2}{|c|}{\BS{begin}\AC{tikzpicture}{\color{red}[x=\AC{(0cm,1cm)},y=\AC{(-1cm,0cm)}]} }
\\ \hline
\begin{tikzpicture}[x={(0cm,1cm)},y={(-1cm,0cm)}]
\draw[help lines] (0,0) grid (3,2);
\draw [dotted](0,2) arc (90 :0 :2);
\draw [dotted](0,0) --(2,2);
\fill (xyz polar cs:angle=45,radius=2) circle (2pt);
\end{tikzpicture}
&
\begin{tikzpicture}[x={(0cm,1cm)},y={(-1cm,0cm)}]
\draw[help lines] (0,0) grid (3,2);
\draw [dotted](0,2) arc (90 :0 :2);
\draw [dotted](0,0) --(2,2);
\fill (45:2) circle (2pt);
\end{tikzpicture}
\\ \hline 
\BS{fill} (\RDD{xyz polar cs}:\RDD{angle}=45,\RDD{radius}=2) circle (2pt);
&
\BS{fill} {\color{blue}(45:2cm)} circle (2pt);
\\ \hline 
\end{tabular} 

\SbSbSSCT{Coordonnées barycentriques}{Barycentric coordinates}

\begin{center}
\RRR{13-2-2}
\end{center}

\begin{tabular}{|c|c|c|} \hline
\multicolumn{3}{|c|}{  \BS{node} [circle,fill=red!20] at (\RDD{barycentric cs}:A=0.6,B=0.3 ) \AC{X};   }\\ 
\hline
\begin{tikzpicture}[scale=.6]
\draw[help lines] (0,0) grid (4,4);
\node[circle,fill=green!20,] (A) at (0,0) {A};
\node[circle,fill=green!20,] (B) at (4,0) {B};
\node[circle,fill=red!20] at (barycentric cs:A=0.3,B=0.3 ) {X};
\end{tikzpicture}
&
\begin{tikzpicture}[scale=.6]
\draw[help lines] (0,0) grid (4,4);
\node[circle,fill=green!20,] (A) at (0,0) {A};
\node[circle,fill=green!20,] (B) at (4,0) {B};
\node[circle,fill=green!20,] (C) at (4,4) {C};
\node[circle,fill=red!20] at (barycentric cs:A=0.4,B=0.4 ,C=.4) {X};
\end{tikzpicture}
&
\begin{tikzpicture}[scale=.6]
\draw[help lines] (0,0) grid (4,4);
\node[circle,fill=green!20,] (A) at (0,0) {A};
\node[circle,fill=green!20,] (B) at (4,0) {B};
\node[circle,fill=green!20,] (C) at (1,4) {C};
\node[circle,fill=green!20,] (D) at (4,4) {D};
\node[circle,fill=red!20] at (barycentric cs:A=0.5,B=0.5,C=.5,D=.5 ) {X};
\end{tikzpicture}
\\ \hline
A=0.3,B=0.3 & A=0.4,B=0.4 ,C=.4 & A=0.5,B=0.5,C=.5,D=.5 
\\ \hline
\begin{tikzpicture}[scale=.6]
\draw[help lines] (0,0) grid (4,4);
\node[circle,fill=green!20,] (A) at (0,0) {A};
\node[circle,fill=green!20,] (B) at (4,0) {B};
\node[circle,fill=red!20] at (barycentric cs:A=0.6,B=0.3 ) {X};
\end{tikzpicture}
&
\begin{tikzpicture}[scale=.6]
\draw[help lines] (0,0) grid (4,4);
\node[circle,fill=green!20,] (A) at (0,0) {A};
\node[circle,fill=green!20,] (B) at (4,0) {B};
\node[circle,fill=green!20,] (C) at (4,4) {C};
\node[circle,fill=red!20] at (barycentric cs:A=0.2,B=0.4 ,C=.6) {X};
\end{tikzpicture}
&
\begin{tikzpicture}[scale=.6]
\draw[help lines] (0,0) grid (4,4);
\node[circle,fill=green!20,] (A) at (0,0) {A};
\node[circle,fill=green!20,] (B) at (4,0) {B};
\node[circle,fill=green!20,] (C) at (1,4) {C};
\node[circle,fill=green!20,] (D) at (4,4) {D};
\node[circle,fill=red!20] at (barycentric cs:A=0.2,B=0.4,C=.6,D=.8 ) {X};
\end{tikzpicture}
\\ \hline
A=0.6,B=0.3 & A=0.2,B=0.4 ,C=.6 & A=0.2,B=0.4,C=.6,D=.8
\\ \hline
\end{tabular}

\SbSbSSCT{Coordonnées nominatives : n\oe ud}{Named coordinates: nodes}

\begin{center}
\RRR{13-2-3}
\end{center}

\begin{tabular}{|c|c|} \hline  
\begin{tikzpicture}[blue,very thick,baseline=1cm]
\draw[help lines] (0,0) grid (3,3);
\coordinate (centre) at (1.5,1.5) ;
\coordinate (A) at (.5,.5) ;
\coordinate (B) at (2.5,2.5) ;
\fill (centre) circle (3pt);
\draw[red] (A) rectangle (B) ;
\end{tikzpicture}
&  
\parbox[c]{8cm}{
\BSS{coordinate} {\color{blue}(centre)} at(1.5,1.5) ; \\
\BSS{coordinate} {\color{blue}(A)} at (.5,.5) ;\\
\BSS{coordinate} {\color{blue}(B)} at  (2.5,2.5) ;\\
\\
\BS{fill} {\color{blue}(centre)} circle (3pt);\\
\BS{draw}[red] {\color{blue}(A)} rectangle {\color{blue}(B)} ;\\
}
\\ \hline 
\end{tabular} 


\TFRGB{voir aussi}{see also} page \pageref{noeuds}


\SbSbSSCT{Coordonnées relatives à un noeud}{Coordinates relative to a node}

\noindent

\begin{tabular}{|c|c|c|c|} \hline
\multicolumn{4}{|l|}{  \BS{node} [draw,fill=green!20,] (A) at (1,1) \AC{\BS{huge}  noeud}; }\\ 
\multicolumn{4}{|l|}{  \BS{fill}[red] (\RDD{node cs}:\RDD{name}=A,\RDD{anchor}=south) circle (3pt);   }\\ 
\hline

\begin{tikzpicture}
\draw[help lines] (0,0) grid (2,2);
\node[draw,fill=green!20,] (A) at (1,1) {\huge noeud};
\fill[red] (node cs:name=A,anchor=south) circle (3pt);
\end{tikzpicture}
&
\begin{tikzpicture}
\draw[help lines] (0,0) grid (2,2);
\node[draw,fill=green!20,] (A) at (1,1) {\huge noeud};
\fill[red] (node cs:name=A,anchor=west) circle (3pt);
\end{tikzpicture}
&
\begin{tikzpicture}
\draw[help lines] (0,0) grid (2,2);
\node[draw,fill=green!20,] (A) at (1,1) {\huge noeud};
\fill[red] (node cs:name=A,anchor=north) circle (3pt);
\end{tikzpicture}
&
\begin{tikzpicture}
\draw[help lines] (0,0) grid (2,2);
\node[draw,fill=green!20,] (A) at (1,1) {\huge noeud};
\fill[red] (node cs:name=A,anchor=east) circle (3pt);
\end{tikzpicture}
\\ \hline
name=A,anchor=south & name=A,anchor=west & name=A,anchor=north & name=A,anchor=east
\\ \hline
\end{tabular}

\bigskip

\begin{tabular}{|c|c|c|c|} \hline
\multicolumn{4}{|l|}{  \BS{node} [draw,fill=green!20,] \blll{(A)} at (1,1) \AC{\BS{huge}  noeud}; }\\ 
\multicolumn{4}{|l|}{  \BS{fill}[red] (\blll{A}.south) circle (3pt);   }\\ 
\hline

\begin{tikzpicture}
\draw[help lines] (0,0) grid (2,2);
\node[draw,fill=green!20,] (A) at (1,1) {\huge noeud};
\fill[red] (A.south) circle (3pt);
\end{tikzpicture}
&
\begin{tikzpicture}
\draw[help lines] (0,0) grid (2,2);
\node[draw,fill=green!20,] (A) at (1,1) {\huge noeud};
\fill[red] (A.west) circle (3pt);
\end{tikzpicture}
&
\begin{tikzpicture}
\draw[help lines] (0,0) grid (2,2);
\node[draw,fill=green!20,] (A) at (1,1) {\huge noeud};
\fill[red] (A.north) circle (3pt);
\end{tikzpicture}
&
\begin{tikzpicture}
\draw[help lines] (0,0) grid (2,2);
\node[draw,fill=green!20,] (A) at (1,1) {\huge noeud};
\fill[red] (A.east) circle (3pt);
\end{tikzpicture}
\\ \hline
A.south & A.west & A.north & A.east
\\ \hline
\end{tabular}



\bigskip
\begin{tabular}{|c|c|c|c|} \hline
\multicolumn{4}{|c|}{  \BS{fill}[red] (node cs:\RDD{name}=A,\RDD{angle}=0) circle (3pt);  }\\ 
\hline

\begin{tikzpicture}
\draw[help lines] (0,0) grid (2,2);
\node[draw,fill=green!20,] (A) at (1,1) {\huge noeud};
\fill[red] (node cs:name=A,angle=0) circle (3pt);
\end{tikzpicture}
&
\begin{tikzpicture}
\draw[help lines] (0,0) grid (2,2);
\node[draw,fill=green!20,] (A) at (1,1) {\huge noeud};
\fill[red] (node cs:name=A,angle=-30) circle (3pt);
\end{tikzpicture}
&
\begin{tikzpicture}
\draw[help lines] (0,0) grid (2,2);
\node[draw,fill=green!20,] (A) at (1,1) {\huge noeud};
\fill[red] (node cs:name=A,angle=-90) circle (3pt);
\end{tikzpicture}
&
\begin{tikzpicture}
\draw[help lines] (0,0) grid (2,2);
\node[draw,fill=green!20,] (A) at (1,1) {\huge noeud};
\fill[red] (node cs:name=A,angle=-150) circle (3pt);
\end{tikzpicture}
\\ \hline
name=A,angle=0 & name=A,angle=-30 & nname=A,angle=-90 & name=A,angle=-150
\\ \hline
\end{tabular}

\bigskip


\begin{tabular}{|c|c|c|c|} \hline
\multicolumn{4}{|c|}{  \BS{fill}[red] (A.0) circle (3pt);  }\\ 
\hline

\begin{tikzpicture}
\draw[help lines] (0,0) grid (2,2);
\node[draw,fill=green!20,] (A) at (1,1) {\huge noeud};
\fill[red] (A.0) circle (3pt);
\end{tikzpicture}
&
\begin{tikzpicture}
\draw[help lines] (0,0) grid (2,2);
\node[draw,fill=green!20,] (A) at (1,1) {\huge noeud};
\fill[red] (A.-30) circle (3pt);
\end{tikzpicture}
&
\begin{tikzpicture}
\draw[help lines] (0,0) grid (2,2);
\node[draw,fill=green!20,] (A) at (1,1) {\huge noeud};
\fill[red] (A.-90) circle (3pt);
\end{tikzpicture}
&
\begin{tikzpicture}
\draw[help lines] (0,0) grid (2,2);
\node[draw,fill=green!20,] (A) at (1,1) {\huge noeud};
\fill[red] (A.-150) circle (3pt);
\end{tikzpicture}
\\ \hline
A.0 & A.-30 & A.-90 & A.-150
\\ \hline
\end{tabular}

\TFRGB{voir aussi}{see also} page \pageref{nomnoeud}


\newpage

\SbSbSSCT{Coordonnées relatives à deux points}{Coordinates relative to two points}
\begin{center}
\RRR{13-3-1}
\end{center}

\begin{tabular}{|c|c|} \hline
\multicolumn{2}{|c|}{  \BS{node} [circle,fill=red!20] at (1,1 {\color{red}|-} 3,3) \AC{X}   }\\ 
\hline
\begin{tikzpicture}
\draw[help lines] (0,0) grid (4,4);
\node[circle,fill=green!20,] (A) at (1,1) {A};
\node[circle,fill=green!20,] (B) at (3,3) {B};
\node[circle,fill=red!20] at (1,1 |- 3,3) {X};
\end{tikzpicture}
&
\begin{tikzpicture}
\draw[help lines] (0,0) grid (4,4);
\node[circle,fill=green!20,] (A) at (1,1) {A};
\node[circle,fill=green!20,] (B) at (3,3) {B};
\node[circle,fill=red!20] at (1,1 -| 3,3) {X};
\end{tikzpicture}
\\ \hline
at (1,1 {\color{red}|-} 3,3)
&
at (1,1 {\color{red}-|} 3,3)
\\ \hline
\end{tabular}



\SbSbSSCT{Coordonnée relative à une intersection}{Coordinates relative to an intersection}
\begin{center}
\RRR{13-3-2}
\end{center}

 \maboite{\BS{usetikzlibrary}\AC{intersections}}
\label{lib-intersections}


\begin{tabular}{|c|c|c|c|} \hline 
\multicolumn{4}{|l|}{  \BS{draw} [\RDD{name path}=XXX] (2,1) circle  (1cm);   }\\ 
\multicolumn{4}{|l|}{  \BS{draw} [\RDD{name path}=YYY] (0.5,0.5) rectangle +(3,1);   }\\ 
\multicolumn{4}{|l|}{ \BS{fill} [red,\RDD{ name intersections}=\AC{of=xxx and YYY}]
(\RDD{intersection}-1) circle (2pt)   }\\ 
\hline 
\begin{tikzpicture}[scale=.8]
\draw [help lines] grid (4,2);
\draw [name path=XXX] (2,1) circle  (1cm);
\draw [name path=YYY] (0.5,0.5) rectangle +(3,1);
\fill [red, name intersections={of=XXX and YYY}]
(intersection-1) circle (2pt)  ;
\end{tikzpicture}
& 
\begin{tikzpicture}[scale=.8]
\draw [help lines] grid (4,2);
\draw [name path=XXX] (2,1) circle  (1cm);
\draw [name path=YYY] (0.5,0.5) rectangle +(3,1);
\fill [red, name intersections={of=XXX and YYY}] (intersection-2) circle (2pt) ;
\end{tikzpicture} 
&  
\begin{tikzpicture}[scale=.8]
\draw [help lines] grid (4,2);
\draw [name path=XXX] (2,1) circle  (1cm);
\draw [name path=YYY] (0.5,0.5) rectangle +(3,1);
\fill [red, name intersections={of=XXX and YYY}] (intersection-3) circle (2pt) ;
\end{tikzpicture}
&  
\begin{tikzpicture}[scale=.8]
\draw [help lines] grid (4,2);
\draw [name path=XXX] (2,1) circle  (1cm);
\draw [name path=YYY] (0.5,0.5) rectangle +(3,1);
\fill [red, name intersections={of=XXX and YYY}] (intersection-4) circle (2pt) ;
\end{tikzpicture}
\\ 
\hline intersection-1 & intersection-2 &intersection-3  & intersection-4 \\ 
\hline 
\end{tabular} 

\bigskip

\begin{tabular}{|c|} \hline  
\BS{fill} [red, name intersections=\AC{of=XXX and YYY}] \\
(intersection-1) circle (2pt) {\color{red} node[black,above right] \AC{point a}} ;
\\ \hline  
\begin{tikzpicture}
\draw [help lines] grid (4,2);
\draw [name path=XXX] (2,1) circle  (1cm);
\draw [name path=YYY] (0.5,0.5) rectangle +(3,1);
\fill [red, name intersections={of=XXX and YYY}]
(intersection-1) circle (2pt) node[black,above right] {point a} ;
\end{tikzpicture} 
\\ \hline 
\end{tabular} 

\bigskip

\begin{tabular}{|c|} \hline 
\BS{fill} [red, name intersections=\AC{of=XXX and YYY, \RDD{name}=ZZZ}]; \\
\BS{draw} [red] (ZZZ-1) - - (ZZZ-3); \BS{draw} [green] (ZZZ-2) - - (ZZZ-4);
\\ \hline  
\begin{tikzpicture}
\draw [help lines] grid (4,2);
\draw [name path=XXX] (2,1) circle  (1cm);
\draw [name path=YYY] (0.5,0.5) rectangle +(3,1);
\fill [red, name intersections={of=XXX and YYY, name=ZZZ}];
\draw [red] (ZZZ-1) -- (ZZZ-3);
\draw [green] (ZZZ-2) -- (ZZZ-4);
\end{tikzpicture}
\\ \hline 
\end{tabular} 

\bigskip
\begin{tabular}{|c|} \hline  
\BS{fill} [red, name intersections=\AC{of=XXX and YYY , \RDD{by}=\AC{a,b,c,d}}]; \\
\BS{draw} [red] (a) - - (c); \hspace{1cm} \BS{draw} [green] (b) - - (d);
\\ \hline   
\begin{tikzpicture}
\draw [help lines] grid (4,2);
\draw [name path=XXX] (2,1) circle  (1cm);
\draw [name path=YYY] (0.5,0.5) rectangle +(3,1);
\fill [red, name intersections={of=XXX and YYY, by={a,b,c,d}}];
\draw [red] (a) -- (c);
\draw [green] (b) -- (d);
\end{tikzpicture}
\\ \hline 
\end{tabular} 

\bigskip

\begin{tabular}{|c|} \hline  
\BS{fill} [name intersections=\AC{of=XXX and YYY, name=i, \RDD{total}=\BS{t}}] [red] \\
\BS{foreach} \BS{s} in \AC{1,...,\BS{t}} \AC{(i-\BS{s}) circle (2pt) node[black,above right] \AC{\BS{s}}}
\\ \hline  
\begin{tikzpicture}
\draw [help lines] grid (4,2);
\draw [name path=XXX] (2,1) circle  (1cm);
\draw [name path=YYY] (0.5,0.5) rectangle +(3,1);
\fill [name intersections={of=XXX and YYY , name=i, total=\t}]
[red]
\foreach \s in {1,...,\t}{(i-\s) circle (2pt) node[black,above right] {\s}};
\end{tikzpicture}
\\ \hline 
\end{tabular} 



\newpage

\SbSbSSCT{Position calculée avec le module  \og  pgfmath \fg}{Calculated positions with  \og  pgfmath \fg }

\begin{center}
\RRR{13-2-1}
\end{center}

\TFRGB{Ce module est chargé automatiquement avec le module Tikz}{Package automatically loaded with Tikz} 

\begin{tabular}{|c|} \hline 
\begin{tikzpicture}
\draw[help lines] (0,0) grid (4,2);
\fill [red] (canvas cs:x=2cm+1.5cm,y=1.5cm-1cm) circle (3pt);
\fill [blue] (2cm,1.5cm) circle (3pt);
\draw[dashed] (2,1.5) -| (3.5,.5);
\end{tikzpicture}
\\ \hline 
\emph{\TFRGB{Explicite}{explicit}} 
 : \BS{fill} [red] (\RDD{canvas cs}:x=2cm+1.5cm,y=1.5cm-1cm) circle (3pt);
 \\  \hline 
\emph{\TFRGB{Implicite}{implicit}} :  \BS{fill} [red] {\color{red}(2cm+1.5cm,1.5cm-1cm)} circle (3pt);
\\ \hline 
\end{tabular} 

\bigskip
\begin{tabular}{|c|c|c|} \hline 
\begin{tikzpicture}[baseline=0pt]
\draw[help lines] (0,0) grid (4,4);
 \draw[dashed] (2,2) circle (2);
\fill[red](2+ 2*cos 30,2+2*sin 30) circle (3pt);
\fill[magenta](2+ 2*cos{(120)},2+2*sin{(120)}) circle (3pt);
\end{tikzpicture}
&
\parbox[c]{8cm}{
 \BS{draw}[dashed] (2,2) circle (2);\\
 \smallskip
 \BS{fill} [red]{\color{red}(2+ 2*cos 30 , 2+2*sin 30)} circle (3pt);\\
  \smallskip
 \BS{fill}[magenta] {\color{red}(2+2*cos\AC{(120)} , 2+2*sin\AC{(120)})} circle (3pt); 
 }
\\ \hline 
\end{tabular} 

\SbSbSSCT{Position calculée avec \og library calc \fg}{Calculated positions with \og  calc  library calc \fg}

\begin{center}
\RRR{13-5}
\end{center}
\label{lib-calc}

 \maboite{\BS{usetikzlibrary}\AC{calc}}
 
\begin{tabular}{|c|c|} \hline  
\begin{tikzpicture}[baseline=0pt]
\draw [help lines] (0,0) grid (3,2);
\node (a) at (1,1) {A};
\fill [red] ($(a) + 2/3*(1cm,0)$) circle (2pt);
\fill [red] ($(a) + 4/3*(1cm,0)$) circle (2pt);
\end{tikzpicture}
&
\parbox{8cm}{
\BS{node} (a) at (1,1) \AC{A}; \\
\BS{fill} [red] {\color{red} (\$(a) + 2/3*(1cm,0)\$)} circle (2pt); \\
\BS{fill} [red] {\color{red}(\$(a) + 4/3*(1cm,0)\$)} circle (2pt); \\
}
\\ 
\hline 
\end{tabular} 

\SbSbSSCT{Tangentes avec \og library calc \fg}{Tangents with  \og calc library  \fg}

\begin{center}
\RRR{13-2-4}
\end{center}

\begin{tabular}{|c|c|} \hline 
\multicolumn{2}{|l|}{\BS{node}[fill=green!20] (a) at (3,1.5) \AC{A}; } \\
\multicolumn{2}{|l|}{\BS{fill}[red] (\RDD{tangent cs}:\RDD{node}=c,\RDD{point}=\AC{(A)},\RDD{solution}=1);  }\\ 
\hline
\begin{tikzpicture}
\draw[help lines] (0,0) grid (4,2);
\node[fill=green!20] (A) at (3,1.5) {A};
\node [circle,draw] (c) at (1,1) [minimum size=1.5cm] {$c$};
\draw[red,dashed] (A) - -(tangent cs:node=c,point={(A)},solution=1) ;
\draw[red,dashed] (1,1) - -(tangent cs:node=c,point={(3,1.5)},solution=1) ;
\fill[red] (tangent cs:node=c,point={(A)},solution=1) circle (3pt);
\end{tikzpicture}
&
\begin{tikzpicture}
\draw[help lines] (0,0) grid (4,2);
\node[fill=green!20] (A) at (3,1.5) {A};
\node [circle,draw] (c) at (1,1) [minimum size=1.5cm] {$c$};
\draw[red,dashed] (A) - -(tangent cs:node=c,point={(A)},solution=2) ;
\draw[red,dashed] (1,1) - -(tangent cs:node=c,point={(A)},solution=2) ;
\fill[red] (tangent cs:node=c,point={(A)},solution=2) circle (3pt);
\end{tikzpicture}
\\ \hline
\RDD{solution}=1 & \RDD{solution}=2
\\ \hline
\end{tabular} 

\newpage

\SbSbSSCT{Point à pourcentage donné }{Percentage position }

\begin{center}
\RRR{13-5-3}
\end{center}


\begin{tabular}{|c|c|} \hline  
\multicolumn{2}{|c|}{\BS{fill}[red] ({\color{red}\$(0,1)!.25!(4,1)\$}) circle (4pt); } \\  \hline  

\begin{tikzpicture}
\draw [help lines] (0,0) grid (4,2);
\draw [line width= 3pt] (0,1) -- (4,1);
\fill[red] ($(0,1)!.25!(4,1)$) circle (4pt);
\end{tikzpicture}
&  
\begin{tikzpicture}
\draw [help lines] (0,0) grid (4,2);
\draw [line width= 3pt] (0,1) -- (4,1);
\fill[red] ($(0,1)!.75!(4,1)$) circle (4pt);
\end{tikzpicture}
\\ \hline (0,1)!{\color{red}0.25}!(4,1) & (0,1)!{\color{red}0.75}!(4,1) \\ 
\hline 
\end{tabular} 

\bigskip

\begin{tabular}{|c|} \hline  
\begin{tikzpicture}
\draw [help lines] (0,0) grid (4,3);
\draw [line width=2pt ](0,2) -- (4,2);
\draw[red] ($(0,2)!.75!(4,2)$) -- (0,0);
\fill[red] ($(0,2)!.75!(4,2)!.66!(0,0)$) circle (4pt);
\end{tikzpicture}
\\ \hline 
\BS{fill}[red] (\${\color{blue}(0,2)!0.75!(4,2)}!{\color{red}0.66!(0,0)}\$) circle (2pt);
\\ \hline 
\end{tabular} 


\SbSbSSCT{Point à distance donnée}{Position at a given distance }

\begin{center}
\RRR{13-5-4}
\end{center}

\begin{tabular}{|c|c|} \hline  
\multicolumn{2}{|c|}{\BS{fill}[red] ({\color{red}\$(0,1)!1.5cm!(4,1)\$}) circle (4pt); } \\  \hline  

\begin{tikzpicture}
\draw [help lines] (0,0) grid (4,2);
\draw [line width= 2pt] (0,1) -- (4,1);
\fill[red] ($(0,1)!1.5cm!(4,1)$) circle (4pt);
\end{tikzpicture}
&  
\begin{tikzpicture}
\draw [help lines] (0,0) grid (4,2);
\draw [line width= 2pt] (0,1) -- (4,1);
\fill[red] ($(0,1)!3cm!(4,1)$) circle (4pt);
\end{tikzpicture}
\\ \hline (0,1)!{\color{red}1.5cm}!(4,1) & (0,1)!{\color{red}3cm}!(4,1) \\ 
\hline 
\end{tabular} 

\bigskip

\begin{tabular}{|c|} \hline  
\begin{tikzpicture}
\draw [help lines] (0,0) grid (4,4);
\coordinate (a) at (1,0);
\coordinate (b) at (4,1);
\draw [line width= 3pt] (0,0) -- (4,1);
\draw [line width= 2pt,red](2,.5) -- ($ (2,.5)!2cm!90:(4,1) $);
\end{tikzpicture}
\\ \hline
\BS{draw} (2,.05) - - (\$ (2,0.5)!{\color{red}2cm!90:(4,1)} \$);
\\ \hline 
\end{tabular} 

\newpage

\SbSbSSCT{Coordonnées relatives}{Relative coordinates}


\Par{Cartésienne}{Cartesian coordinates}

\begin{center}
\RRR{13-4-1}
\end{center}

\begin{tabular}{|c|c|c|} \hline  
\TFRGB{relative à l'origine}{relative to the origin}  & \TFRGB{relative à une position}{relative to a position}  &  \TFRGB{relative à la dernière position}{relative to the last position}   
\\ \hline  
 
\begin{tikzpicture}
\draw[help lines] (0,-1) grid (3,1); 
 \draw[blue,very thick] (0,0) -- (1,0) - - (2,1) - - (2,-1);
 \fill[red] (0,0) circle (4pt);
\end{tikzpicture}
&
\begin{tikzpicture} %[scale=.8]
\draw[help lines] (0,-1) grid (4,1);
 \draw[blue,very thick] (0,0) - - (1,0) -- +(2,1) -- +(2,-1) ; %–- +(2,-1) ;
 \fill[red] (1,0) circle (4pt);
\end{tikzpicture}
&
\begin{tikzpicture} %[scale=.8]
\draw[help lines] (0,-1) grid (5,1);  
 \draw[blue,very thick] (0,0) -- (1,0)  - - ++(2,1) - - ++(2,-1);
 \fill[red] (1,0) circle (4pt);
 \fill[red] (3,1) circle (4pt);
\end{tikzpicture}
\\ \hline 
\tikz \fill node[fill=green!20,inner sep=0pt]{(0,0)}; - - (1,0) &
 (0,0) - - \tikz \fill node[fill=green!20,inner sep=0pt]{(1,0)};  & (0,0) - - \tikz \fill node[fill=green!20,inner sep=0pt]{(1,0)}; \\
 - - (2,1) - - (2,-1)  &
   - - +(2,1) - - +(2,-1) & - - ++\tikz \fill node[fill=green!20,inner sep=0pt]{(2,1)}; - - ++(2,-1)
\\ \hline 
\end{tabular} 

\bigskip

\begin{tabular}{|c|c|c|} \hline  
\begin{tikzpicture} [scale=.5]
\draw[help lines] (0,-1) grid (6,6);
 \draw[red,dotted,line width=2pt] (0,0) rectangle (2,2) ;
  \draw[green,dotted,line width=2pt] (0,0) rectangle (3,3) ;  
 \draw[blue,line width=2pt] (0,0) rectangle (1,1)  rectangle (2,2) rectangle (3,3);

\end{tikzpicture}

&  
\begin{tikzpicture} [scale=.5]
\draw[help lines] (0,-1) grid (6,6); 
  \draw[green,dotted,line width=2pt] (1,1) rectangle (4,4) ;   
 \draw[blue,line width=2pt] (0,0) rectangle (1,1)  rectangle +(2,2) rectangle +(3,3);
    \fill[red] (1,1) circle (4pt);
\end{tikzpicture}
&  
\begin{tikzpicture} [scale=.5]
\draw[help lines] (0,-1) grid (6,6);  
 \draw[blue,line width=2pt] (0,0) rectangle (1,1)  rectangle ++(2,2) rectangle ++(3,3);
    \fill[red] (1,1) circle (4pt);
     \fill[green] (3,3) circle (4pt); 
\end{tikzpicture}
\\ 
\hline 
\BS{draw} (0,0) rectangle (1,1)   &
\BS{draw} (0,0) rectangle (1,1)   & 
\BS{draw} (0,0) rectangle (1,1)  \\
rectangle (2,2) rectangle (3,3);  &
rectangle +(2,2) rectangle +(3,3);  &
rectangle ++(2,2) rectangle ++(3,3); \\
\hline 
\end{tabular}


\Par{Polaire }{Polar} {}

\bigskip


\noindent

\begin{tabular}{|c|c|c|c|} \hline
\TFRGB{relative à l'origine}{relative to the origin}  & \TFRGB{relative à une position}{relative to a position}  &  \TFRGB{relative à la dernière position}{relative to the last position}   
\\ \hline    
\begin{tikzpicture} %[scale=.8] 
\draw[help lines] (0,-1) grid (3,1);
 \fill[red] (0:0) circle (4pt);
 \draw[blue,very thick] (0:0)-- (0:1) -- (30:2) -- (-30:2);
\end{tikzpicture}
&
\begin{tikzpicture} %[scale=.8] 
\draw[help lines] (0,-1) grid (4,1);
 \fill[red] (1,0) circle (4pt);
 \draw[blue,very thick] (0:0) -- (0:1) -- +(30:2) -- +(-30:2);
\end{tikzpicture}
&
\begin{tikzpicture} %[scale=.8] 
\draw[help lines] (0,-1) grid (5,1);
 \fill[red] (1,0) circle (4pt);
 \fill[red] (2.732,1) circle (4pt);
 \draw[blue,very thick] (0:0)-- (0:1) -- ++(30:2) -- ++(-30:2);
\end{tikzpicture}
\\ \hline
\tikz \fill node[fill=green!20,inner sep=0pt] {(0:0)}; - - (0:1)&
 (0:0) - - \tikz \fill node[fill=green!20,inner sep=0pt] {(0:1)}; & (0:0)- - \tikz \fill node[fill=green!20,inner sep=0pt] {(0:1)}; \\
 - - (30:2) - - (-30:2)  &  - -  +(30:2) - - +(-30:2) & - -  ++\tikz \fill node[fill=green!20,inner sep=0pt] {(30:2)}; - - ++(-30:2)
\\ \hline 
\end{tabular} 

%\subsubsection{coordonnée relative en polaire}
\Par{coordonnée relative en polaire}{Relative polar coordinate}

\begin{center}
\RRR{13-4-2}
\end{center}
\bigskip

\begin{tabular}{|c|c|} \hline 
\multicolumn{2}{|c|}{ \BS{draw}[blue,very thick] (0,0) -- (2,1) -- ([turn]-45:1cm);}
 \\ \hline
\begin{tikzpicture} %[scale=.8] 
\draw[help lines] (0,0) grid (4,2);
 \draw[dotted] (0,0) -- (4,2);
 \draw[blue,very thick] (0,0) -- (2,1) -- ([turn]-45:1cm);
\end{tikzpicture}
&  
\begin{tikzpicture} %[scale=.8] 
\draw[help lines] (0,0) grid (4,2);
 \draw[dotted] (0,0) -- (4,2);
 \draw[blue,very thick] (0,0) -- (2,1) -- ([turn]45:1cm);
\end{tikzpicture}
\\ \hline ([\RDD{turn}]-45:1cm) & ([\RDD{turn}]45:1cm) \\ 
\hline 
\end{tabular}

\bigskip

\begin{tabular}{|c|c|} \hline  
\begin{tikzpicture}  
\draw[help lines] (-1,0) grid (4,3);
\draw [line width=2pt] (4,0) arc (0 :120 :2)  -- ([turn]90:2cm) ;

\end{tikzpicture}
&  
\begin{tikzpicture} %[scale=.8] 
\draw[help lines] (0,0) grid (4,3);
\draw [line width=2pt]  (0,0) to [bend left] (2,2) --  ([turn]0:2cm);
\fill [red](2,2) circle (4pt);
\end{tikzpicture}
\\ \hline  
\BS{draw} (4,0) arc (0 :120 :2)  - - ([\RDD{turn}]90:2cm) ;
& \BS{draw}  (0,0) to [bend left] (2,2) - -  ([\RDD{turn}]0:2cm); \\

\hline 
\end{tabular} 


%\bigskip 
%
%
%\tikz [delta angle=30, radius=1cm]
%\draw (0,0) arc [start angle=0] -- ([turn]0:1cm)
%arc [start angle=30] -- ([turn]0:1cm)
%arc [start angle=60] -- ([turn]30:1cm);



\bigskip

\begin{tabular}{|c|c|c|} \hline  
\multicolumn{3}{|c|}{ \BS{draw}(1,2)
.. controls ([turn]0:2cm) .. ([turn]-90:2cm); }
\\ \hline
\begin{tikzpicture} %[scale=.8] 
\draw[help lines] (0,0) grid (4,4);
 \draw [line width=2pt] (1,2)
.. controls ([turn]0:2cm) .. ([turn]-90:2cm);
\end{tikzpicture}
&  
\begin{tikzpicture} %[scale=.8] 
\draw[help lines] (0,0) grid (4,4);
 \draw [line width=2pt] (1,2)
.. controls ([turn]30:2cm) .. ([turn]-90:2cm);
\end{tikzpicture}
&  
\begin{tikzpicture} %[scale=.8] 
\draw[help lines] (-2,0) grid (2,4);
 \draw [line width=2pt] (1,2)
.. controls ([turn]0:2cm) .. ([turn]90:2cm);

\end{tikzpicture}
\\ \hline ([turn]0:2cm) .. ([turn]-90:2cm) & ([turn]30:2cm) .. ([turn]-90:2cm) & ([turn]0:2cm) .. ([turn]90:2cm) \\ 
\hline 
\end{tabular} 


\tikzset{every picture/.style=blue,very thick,inner sep=.3333em}

%
%
%
%\newpage
%
%\SSCT{Les n\oe uds }{Nodes}
%
%\SbSSCT{Définition des  n\oe uds}{Creation of nodes}
\tikzset{blue}

\label{noeuds}
\noindent

\begin{tabular}{|c | c | c | c | c |} \hline
\multicolumn{5}{|c|}{  \BS{draw} (1,1) node[\RDD{fill}=red!20] \AC{};   }\\ 
\hline 
\tikz \draw (0,0) grid (2,2) (1,1) node[fill=red!20] {};
&
\tikz \draw (0,0) grid (2,2) (1,1) node[fill=red!20,draw] {}; 
&
\tikz \draw (0,0) grid (2,2) (1,1) node[circle,fill=red!20] {};
&
\tikz \draw (0,0) grid (2,2) (1,1) node[circle,fill=red!20,draw] {};
&
\tikz \draw (0,0) grid (2,2) (1,1) node[coordinate] {};
\\  \hline
\dft
&
node[\RDD{draw}] 
&
 node[\RDD{circle}]  
&
 node[\RDD{circle},\RDD{draw}]
 &
  node[\RDD{coordinate}]
 \\  \hline
\end{tabular}
\bigskip

\begin{tabular}{|c | c | c | c | } \hline
\multicolumn{4}{|c|}{ \BSS{node} \RDD{at} (1,1) [fill=red!20] \AC{};   }\\ 
\hline 
 \begin{tikzpicture}
\draw (0,0) grid (2,2) ; 
\node at (1,1) [fill=red!20] {};
 \end{tikzpicture}
&
 \begin{tikzpicture}
\draw (0,0) grid (2,2) ; 
\node at (1,1) [draw] {};
 \end{tikzpicture}
&
 \begin{tikzpicture}
\draw (0,0) grid (2,2) ; 
\node at (1,1) [fill=red!20,circle] {};
 \end{tikzpicture}
&
 \begin{tikzpicture}
\draw (0,0) grid (2,2) ; 
\node at (1,1) [circle,draw] {};
 \end{tikzpicture}

\\  \hline
[fill=red!20]
&
[\RDD{draw}] 
&
[\RDD{circle},fill=red!20]
 &
[\RDD{circle},draw] 
 \\  \hline
\end{tabular}
\bigskip

\TFRGB{Autres types de n\oe uds voir page}{Other type of nodes see page} \pageref{noeudboite}
\bigskip


\begin{tabular}{|c|c|} \hline 
\BS{draw} (0,0) node at (1,0) \AC{1} node at (2,0) \AC{2} & \BS{draw}(0,0) node foreach \BS{x} in \AC{1,2,...,5}\\ 
node at (3,0) \AC{3} node at (4,0) \AC{4} node at (5,0) \AC{5}; &  at (\BS{x},0) \AC{\BS{x}};\\ 
\hline 
\tikz \draw (0,0) node at (1,0) {1} node at (2,0) {2} node at (3,0) {3} node at (4,0) {4} node at (5,0) {5};
&
\tikz \draw (0,0) node foreach \x in {1,2,...,5} at (\x,0) {\x};
\\ \hline 
\end{tabular} 



\bigskip

\begin{tabular}{|c|} \hline 
\BS{draw}[\rouge{every node/.style=\AC{draw,red}}](0,0) node foreach \BS{x} in \AC{1,2,...,5} at (\BS{x},0) \AC{\BS{x}};
\\ \hline 
\rule[-3pt]{0pt}{.8cm}\tikz \draw[every node/.style={draw,red}] (0,0) node foreach \x in {1,2,...,5} at (\x,0) {\x};
\\ \hline 
\end{tabular} 

\bigskip

\begin{tabular}{|c|} \hline 
\BS{draw}[\rouge{every rectangle node/.style=\AC{draw,red}},\\
\rouge{every circle node/.style=\AC{draw,double}}]\\ (0,0) node at (1,0) \AC{1} node[circle] at (2,0) \AC{2} \\ node[circle] at (3,0) \AC{3} node at (4,0) \AC{4} node at (5,0) \AC{5};
\\ \hline 
\rule[-3pt]{0pt}{1cm} \tikz \draw[every rectangle node/.style={draw,red},
every circle node/.style={draw,double}] (0,0) node at (1,0) {1} node[circle] at (2,0) {2} node[circle] at (3,0) {3} node at (4,0) {4} node at (5,0) {5};
\\ \hline 
\end{tabular} 

\SbSSCT{Nom des  n\oe uds}{Node name}


\begin{tabular}{|c|c|c|}
\hline 
\multicolumn{3}{|c|}{} \\ 
\hline 
\begin{tikzpicture}
\node[name=A,fill=red] at (0,0) {};
\draw  (-1,-1) grid (1,1) ;
\draw (A) circle (.5) ;
\end{tikzpicture} 
&  
\begin{tikzpicture}
\node[name=A,alias=B,fill=red] at (0,0) {} ;
\draw  (-1,-1) grid (1,1) ;
\draw (B) circle (.5) ;
\end{tikzpicture}
& 
\begin{tikzpicture}
\node[fill=red] (C) at (0,0) {};
\draw  (-1,-1) grid  (1,1) ;
\draw (C) circle (.5);
\end{tikzpicture} \\ 
\hline 
\BS{node}[\RDD{name}=A] at (0,0) \AC{}  & \BS{node}[\RDD{name}=A,\RDD{alias}=B] at (0,0) \AC{}  & 
\BS{node}\rouge {(C)} at (0,0) \AC{} \\ 
\BS{draw} (A) circle (.5); & \BS{draw}  (B) circle (.5); &\BS{draw} (C) circle (.5);
\\ \hline 
\end{tabular} 
\newpage

\SbSSCT{Contenu des  n\oe uds}{Node contents}
\tikzset{blue}

\begin{center}
\RRR{17-2-1}
\end{center}

\begin{tabular}{|c|c|} \hline 
\BS{node} at (1,1) [fill=red!20]\rouge { \AC{XXX} };
&  
\BS{node} at (1,1) [fill=red!20,\RDD{node contents}=XXX] \AC{};
\\  \hline 
 \begin{tikzpicture}
\draw (0,0) grid (2,2) ; 
\node at (1,1) [fill=red!20] {XXX};
\end{tikzpicture}
&  
\begin{tikzpicture}
\draw (0,0) grid (2,2) ; 
\node at (1,1) [fill=red!20,node contents=XXX] {};
\end{tikzpicture} 
\\ \hline 
\end{tabular} 

\bigskip

\begin{tabular}{|c|c|} \hline 
\BS{node}[red] at (1,1) [fill=blue!20] \AC{XXX} ;
&  
\BS{node}[red] at (1,1) [fill=blue20,node contents=XXX] \AC{};
\\  \hline 
 \begin{tikzpicture}
\draw (0,0) grid (2,2) ; 
\node[red] at (1,1) [fill=blue!20] {XXX};
\end{tikzpicture}
&  
\begin{tikzpicture}
\draw (0,0) grid (2,2) ; 
\node[red] at (1,1) [fill=blue!20,node contents=XXX] {};
\end{tikzpicture} 
\\ \hline 
\end{tabular} 


\SbSSCT{Premier ou arrière plan}{Behind or in front}

\begin{tabular}{|c|c|} \hline 
\multicolumn{2}{|l|}{\BS{tikz} \BS{fill} [fill=blue!50, draw=blue, very thick]
(0,0) } \\ 
\multicolumn{2}{|l|}{node [\RDD{behind path}, fill=red!50] \AC{XXXXX} }  \\
\multicolumn{2}{|l|}{- - (1.5,0) - - (1.5,1) - - (0,1) ;}
\\ \hline 
\tikz \fill [fill=blue!50, draw=blue, very thick]
(0,0) node [behind path, fill=red!50] {XXXXX}
-- (1.5,0) 
-- (1.5,1) 
-- (0,1) ;
&  
\tikz \fill [fill=blue!50, draw=blue, very thick]
(0,0) node [in front of path, fill=red!50] {XXXXX}
-- (1.5,0) 
-- (1.5,1) 
-- (0,1) ;
\\ \hline 
\RDD{behind path}
&  
\RDD{in front of path}
\\ \hline 
\end{tabular}



\SbSSCT{Noms à préfixe ou suffixe}{Name prefix or name suffix}


\begin{tabular}{|c|c|}
\hline 
\begin{tikzpicture}[every node/.style={draw},baseline=0pt]
\draw[name prefix = top-] node (A) at (1,1) {A} node (B) at (2,1) {B} node (C) at (3,1) {C};
\draw[name prefix = bottom-] node (1) at (1,0) {1} node (2) at (2,0) {2} node(3) at  (3,0) {3};
\draw [red] (top-A) -- (bottom-3);
\end{tikzpicture} 
&
\parbox{12cm}{
\BS{draw}[\RDD{name prefix} = \blll{top-} ] node (A) at (1,1) \AC{A} node (B) at (2,1) \AC{B} node (C) at (3,1) \AC{C}; \\
\BS{draw}[\RDD{name prefix} = \blll{bottom-}] node (1) at (1,0) \AC{1} node (2) at (2,0) \AC{2} node(3) at  (3,0) \AC{3}; \\
\BS{draw} [red] (\blll{top-}A) -- (\blll{bottom-}3);}
\\ \hline
\begin{tikzpicture}[every node/.style={draw},baseline=0pt]
\draw[name suffix= -top] node (A) at (1,1) {A} node (B) at (2,1) {B} node (C) at (3,1) {C};
\draw[name suffix=  -bottom] node (1) at (1,0) {1} node (2) at (2,0) {2} node(3) at  (3,0) {3};
\draw [red] (A-top) -- (3-bottom);
\end{tikzpicture}
&
\parbox{12cm}{
\BS{draw}[\RDD{name suffix} = \blll{-top}] node (A) at (1,1) \AC{A} node (B) at (2,1) \AC{B} node (C) at (3,1) \AC{C}; \\
\BS{draw}[\RDD{name suffix} = \blll{-bottom}] node (1) at (1,0) \AC{1} node (2) at (2,0) \AC{2} node(3) at  (3,0) \AC{3}; \\
\BS{draw} [red] (A \blll{-top}) - - (3 \blll{-bottom});}
\\ \hline 

\end{tabular} 


\SbSSCT{Liaisons}{Links}
\label{liaisons}

\begin{tabular}{|c|c|c|} \hline 
\multicolumn{3}{|l|}{\BS{node}[draw] (A) at (0,0) \AC{A}; \hspace{.5cm} \BS{node}[draw] (B) at (1.5,1.5) \AC{B}; \hspace{.5cm} \BS{draw} (A) - - (B) } \\ \hline 
\begin{tikzpicture}[blue]
\node[draw] (A) at (0,0) {A};
\node[draw] (B) at (1.5,1.5) {B};
\draw (A) -- (B);
\end{tikzpicture}
&  
\begin{tikzpicture}[blue]
\node[draw] (A) at (0,0) {A};
\node[draw] (B) at (1.5,1.5) {B};
\draw (A) |- (B);
\end{tikzpicture}
&  
\begin{tikzpicture}[blue]
\node[draw] (A) at (0,0) {A};
\node[draw] (B) at (1.5,1.5) {B};
\draw (A) -| (B);
\end{tikzpicture}
\\ \hline  
(A){\color{red} - -} (B) & (A) {\color{red}|-} (B) &  (A) {\color{red}-|} (B)
\\ \hline 
\begin{tikzpicture}[blue]
\node[draw] (A) at (0,0) {A};
\node[draw] (B) at (1.5,1.5) {B};
\draw (A) to [bend right] (B);
\end{tikzpicture}
&  
\begin{tikzpicture}[blue]
\node[draw] (A) at (0,0) {A};
\node[draw] (B) at (1.5,1.5) {B};
\draw (A) to [bend left] (B);
\end{tikzpicture}
&  
\begin{tikzpicture}[blue]
\node[draw] (A) at (0,0) {A};
\node[draw] (B) at (1.5,1.5) {B};
\draw (A) to[bend left=0] (B);
\end{tikzpicture}
\\ \hline  
(A) to [\RDD{bend right}] (B) & (A) to [\RDD{bend left}] (B) &  (A) to[\RDD{bend left}=0] (B)
\\ \hline 
\begin{tikzpicture}[blue]
\node[draw] (A) at (0,0) {A};
\node[draw] (B) at (1.5,1.5) {B};
\draw (A) to[bend left=120]  (B);
\end{tikzpicture}
&  
\begin{tikzpicture}[blue]
\node[draw] (A) at (0,0) {A};
\node[draw] (B) at (1.5,1.5) {B};
\draw (A) to[bend left=45] (B);
\end{tikzpicture}
&  
\begin{tikzpicture}[blue]
\node[draw] (A) at (0,0) {A};
\node[draw] (B) at (1.5,1.5) {B};
\draw (A) to[bend left=90] (B);
\end{tikzpicture}
\\ \hline  
(A)  to[\RDD{bend left}=120]  (B) & (A) to[\RDD{bend left}=45] (B) &  (A) to[\RDD{bend left}=90] (B)
\\ \hline 
\begin{tikzpicture}[blue]
\node[draw] (A) at (0,0) {A};
\node[draw] (B) at (1.5,1.5) {B};
\draw (A)  to[out=90]  (B);
\end{tikzpicture}
&  
\begin{tikzpicture}[blue]
\node[draw] (A) at (0,0) {A};
\node[draw] (B) at (1.5,1.5) {B};
\draw (A) to[out=30] (B);
\end{tikzpicture}
&  
\begin{tikzpicture}[blue]
\node[draw] (A) at (0,0) {A};
\node[draw] (B) at (1.5,1.5) {B};
\draw (A)  to[in=-90]  (B);
\end{tikzpicture}
\\ \hline  
(A)  to[\RDD{out}=90] (B) & (A) to[\RDD{out}=30]  (B) &  (A)  to[\RDD{in}=-90]  (B)
\\ \hline  
\end{tabular} 

\bigskip
\begin{tabular}{|c|c|c|} \hline  
\multicolumn{2}{|c|}{ \BS{draw} (A) .. controls +(right:2cm) and +(down:2cm) .. (B);  }\\ 
\hline  
\begin{tikzpicture}[blue]
\node[draw] (A) at (0,0) {A};
\node[draw] (B) at (2,2) {B};
\draw  (A) .. controls +(right:2cm) and +(down:2cm) .. (B);
\end{tikzpicture}
&
\begin{tikzpicture}[blue]
\node[draw] (A) at (0,0) {A};
\node[draw] (B) at (2,2) {B};
\draw  (A) .. controls +(up:1cm) and +(left:1cm) .. (B);
\end{tikzpicture}
\\ \hline 
controls +(right:2cm) and +(down:2cm)  &
controls +(up:1cm) and +(left:1cm)
\\ \hline 
\begin{tikzpicture}[blue]
\node[draw] (A) at (0,0) {A};
\node[draw] (B) at (2,2) {B};
\draw  (A) .. controls +(right:1cm) and +(right:2cm) .. (B);
\end{tikzpicture}
&
\begin{tikzpicture}[blue]
\node[draw] (A) at (0,0) {A};
\node[draw] (B) at (2,2) {B};
\draw  (A) .. controls +(up:1cm) and +(right:2cm) .. (B);
\end{tikzpicture}
\\ \hline 
controls +(right:1cm) and +(right:2cm)  &
controls +(up:1cm) and +(right:2cm) 
\\ \hline 
\begin{tikzpicture}[blue]
\node[draw] (A) at (0,0) {A};
\node[draw] (B) at (2,2) {B};
\draw  (A) .. controls +(120:2cm) and +(200:1cm) .. (B);
\end{tikzpicture}
 &
 \begin{tikzpicture}[blue]
 \node[draw] (A) at (0,0) {A};
 \node[draw] (B) at (2,2) {B};T
 \draw  (A) .. controls +(120:2cm) and +(200:1cm) .. (A);
 \end{tikzpicture}
\\  \hline  
controls +(120:2cm) and +(200:1cm) & controls +(120:2cm) and +(200:1cm) 
\\ \hline 
\begin{tikzpicture}[blue]
\node[draw] (A) at (0,0) {A};
\node[draw] (B) at (2,2) {B};
\node[draw] (C) at (0,1) {C};
\node[draw] (D) at (3,0) {D};
\draw  (A) .. controls +(C) and +(D) .. (B);
\end{tikzpicture}
&
\begin{tikzpicture}[blue]
\node[draw] (A) at (0,0) {A};
\node[draw] (B) at (2,2) {B};
\node[draw] (C) at (0,1) {C};
\node[draw] (D) at (3,0) {D};
\draw (A) .. controls +(D)  .. (B);
\end{tikzpicture}
\\ \hline 
controls +(C) and +(D) &
controls +(D) 
\\ \hline 
\end{tabular} 
 \bigskip
 
\begin{tabular}{|c|c|c|} \hline 
\multicolumn{3}{|l|}{ \BS{node}[draw] (A) at (0,0) \AC{A}  }\\

\multicolumn{3}{|l|}{ \BS{node}[draw] (B) at (2,2) \AC{B} \RDD{edge}  [->] (A);  }\\
\multicolumn{3}{|c|}{\RRR{17-12-1}}  \\
\hline 
 \begin{tikzpicture}
 \node[draw] (A) at (0,0) {A};
 \node[draw] (B) at (2,2) {B} edge [->] (A);
 \end{tikzpicture}
 &
 \begin{tikzpicture}
 \node[draw] (A) at (0,0) {A};
 \node[draw] (B) at (2,2) {B} edge [red]  (A);
 \end{tikzpicture}
 &
 \begin{tikzpicture}
 \node[draw] (A) at (0,0) {A};
 \node[draw] (B) at (2,2) {B} edge [dashed] (A);
 \end{tikzpicture}
\\ \hline 
[->] & [red]  & [dashed]
\\ \hline 
\end{tabular}

\SbSSCT{\'Etiquettes sur les n\oe uds}{Node labels}

\begin{tabular}{|c|c|c|c|} \hline
\multicolumn{4}{|c|}{  \BS{fill}(0,0) circle (2pt) node[\RDD{above}] \AC{texte} ; \RRR{17-5-2}   }\\ 
\hline 
  
\begin{tikzpicture} \draw[help lines] (-1,-1) grid (1,1) ;\fill (0,0) circle (2pt) node[above] {texte};\end{tikzpicture}
& 
\begin{tikzpicture} \draw[help lines] (-1,-1) grid (1,1) ;\fill (0,0) circle (2pt) node[below] {texte};\end{tikzpicture}
 &  
\begin{tikzpicture} \draw[help lines] (-1,-1) grid (1,1);\fill (0,0) circle (2pt) node[left] {texte};\end{tikzpicture}
 &  
\begin{tikzpicture} \draw[help lines] (-1,-1) grid (1,1); \fill (0,0) circle (2pt) node[right] {texte};\end{tikzpicture}
 \\  \hline 
 [\RDD{above}] & [\RDD{below}] & [\RDD{left}] &  [\RDD{right}]
 \\ \hline 
 \begin{tikzpicture} \draw[help lines] (-1,-1) grid (1,1) ;\fill (0,0) circle (2pt) node[above left] {texte};\end{tikzpicture}
 & 
 \begin{tikzpicture} \draw[help lines] (-1,-1) grid (1,1) ;\fill (0,0) circle (2pt) node[below left] {texte};\end{tikzpicture}
  &  
 \begin{tikzpicture} \draw[help lines] (-1,-1) grid (1,1);\fill (0,0) circle (2pt) node[above right] {texte};\end{tikzpicture}
  &  
 \begin{tikzpicture} \draw[help lines] (-1,-1) grid (1,1); \fill (0,0) circle (2pt) node[below right] {texte};\end{tikzpicture}
  \\  \hline 
  [\RDD{above left}] & [\RDD{below left}] & [\RDD{above right}] &  [\RDD{below right}]
  \\ \hline 
 \begin{tikzpicture} \draw[help lines] (-1,-1) grid (1,1) ;\fill (0,0) circle (2pt) node[anchor=south] {texte};\end{tikzpicture}
 & 
 \begin{tikzpicture} \draw[help lines] (-1,-1) grid (1,1) ;\fill (0,0) circle (2pt) node[anchor=west] {texte};\end{tikzpicture}
  &  
 \begin{tikzpicture} \draw[help lines] (-1,-1) grid (1,1);\fill (0,0) circle (2pt) node[anchor=north] {texte};\end{tikzpicture}
  &  
 \begin{tikzpicture} \draw[help lines] (-1,-1) grid (1,1); \fill (0,0) circle (2pt) node[anchor=east] {texte};\end{tikzpicture}
  \\  \hline 
  [\RDD{anchor}=south] & [\RDD{anchor}=west] & [\RDD{anchor}=north] & [\RDD{anchor}=east]                                                                                                                                                               ]
  \\ \hline 
 \begin{tikzpicture} \draw[help lines] (-1,-1) grid (1,1) ;\fill (0,0) circle (2pt) node[anchor=south east] {texte};\end{tikzpicture}
 & 
\begin{tikzpicture} \draw[help lines] (-1,-1) grid (1,1) ;\fill (0,0) circle (2pt) node[anchor=south west] {texte};\end{tikzpicture}
&  
\begin{tikzpicture} \draw[help lines] (-1,-1) grid (1,1);\fill (0,0) circle (2pt) node[anchor=north west] {texte};\end{tikzpicture}
&  
\begin{tikzpicture} \draw[help lines] (-1,-1) grid (1,1); \fill (0,0) circle (2pt) node[anchor=east] {texte};\end{tikzpicture}
\\  \hline 
[\RDD{anchor}=south east] & [\RDD{anchor}=south west] & [\RDD{anchor}=north west] & [\RDD{anchor==north east                                                                                                                                                       }]
  \\ \hline 
\end{tabular} 


\bigskip
\begin{tabular}{|c|c|c|c|} \hline
\multicolumn{4}{|c|}{  \BS{fill}(0,0) circle (2pt) node[\RDD{above}=.3cm] \AC{texte} ; \RRR{17-5-2}  }\\ 
\hline 
  
\begin{tikzpicture} \draw[help lines] (-1,-1) grid (1,1) ;\fill (0,0) circle (2pt) node[above=.3cm] {texte};\end{tikzpicture}
& 
\begin{tikzpicture} \draw[help lines] (-1,-1) grid (1,1) ;\fill (0,0) circle (2pt) node[below=.3cm] {texte};\end{tikzpicture}
 &  
\begin{tikzpicture} \draw[help lines] (-1,-1) grid (1,1);\fill (0,0) circle (2pt) node[left=.3cm] {texte};\end{tikzpicture}
 &  
\begin{tikzpicture} \draw[help lines] (-1,-1) grid (1,1); \fill (0,0) circle (2pt) node[right=.3cm] {texte};\end{tikzpicture}
 \\  \hline 
 [\RDD{above}=.3cm] & [\RDD{below}=.3cm] & [\RDD{left}=.3cm] &  [\RDD{right}=.3cm]]
 \\ \hline 
\begin{tikzpicture} \draw[help lines] (-1,-1) grid (1,1) ;\fill (0,0) circle (2pt) node[above left=.3cm] {texte};\end{tikzpicture}
& 
\begin{tikzpicture} \draw[help lines] (-1,-1) grid (1,1) ;\fill (0,0) circle (2pt) node[below left=.3cm] {texte};\end{tikzpicture}
 &  
\begin{tikzpicture} \draw[help lines] (-1,-1) grid (1,1);\fill (0,0) circle (2pt) node[above right=.3cm] {texte};\end{tikzpicture}
 &  
\begin{tikzpicture} \draw[help lines] (-1,-1) grid (1,1); \fill (0,0) circle (2pt) node[below right=.3cm] {texte};\end{tikzpicture}
 \\  \hline 
 [\RDD{above left}=.3cm] & [\RDD{below left}=.3cm] & [\RDD{above right}=.3cm] &  [\RDD{below right}=.3cm]]
 \\ \hline 
 
 \end{tabular} 

 
 \newpage
\selectlanguage{french}
 
 \begin{tabular}{|c|c|c|c|c|} \hline
 \multicolumn{5}{|l|}{ \BSS{shorthandoff}\AC{:} \footnotemark[1]  } \\
 \multicolumn{5}{|l|}{  \BS{node} [draw,\RDD{label}=right:texte] \AC{}   }\\
 \multicolumn{5}{|l|}{ \BSS{shorthandon}\AC{:} } \\ 
 \hline 
     \shorthandoff{:} 
 \tikz \node [draw,label=right:texte] {};
 \shorthandon{:}
 &
  \shorthandoff{:}
 \tikz \node [draw,label=left:texte] {};
 \shorthandon{:}
 &
  \shorthandoff{:}
 \tikz \node [draw,label=above:texte] {};
 \shorthandon{:}
 &
  \shorthandoff{:}
 \tikz \node [draw,label=below:texte] {};
 \shorthandon{:}
 &
  \shorthandoff{:}
 \tikz \node [draw,label=45:texte] {};
    \shorthandon{:}
   \\ \hline
  label=right & label=left &  label=above & label=below & label=45
    \\ \hline 
 \end{tabular}
 \footnotetext[1]{\TFRGB{désactivation et ré-activation de \og : \fg  conflit entre les modules Tikz et Babel en français}{Only useful when the package babel is loaded with the frenchb option    }}
 
 \bigskip
  \begin{tabular}{|c|c|c|c|c|} \hline
  \BS{fill}(0,0) circle (2pt) node[below right=.3cm,draw,label=45:étiquette] \AC{texte} ;
      \\ \hline 
  
  \shorthandoff{:}
\begin{tikzpicture} \draw[help lines] (-1,-1) grid (2,1); \fill (0,0) circle (2pt) node[below right=.3cm,draw,label=45:étiquette] {texte};\end{tikzpicture}
 \shorthandon{:}
 
    \\ \hline 
 \end{tabular}
\bigskip

 \shorthandoff{:}

\SbSSCT{\'Etiquettes épinglées}{The Pin Option} 

\begin{center}
\RRR{17-10-3}
\end{center}
 
\begin{tabular}{|c|c|c|} \hline
\multicolumn{3}{|c|}{  \BSS{shorthandoff}\AC{:} \BS{node}[circle,draw,blue,\RDD{pin}=texte] \AC{} ;   \BSS{shorthandon}\AC{:}  \footnotemark[1] }\\ 
\hline
\begin{tikzpicture} 
\node [circle,draw,blue,pin=texte] {};
\end{tikzpicture}
&
\begin{tikzpicture} 
\node [circle,draw,blue,pin=60:texte] {};
\end{tikzpicture}
&
\begin{tikzpicture} 
\node [circle,draw,blue,pin=right:texte] {};
\end{tikzpicture}
 \\ \hline
[circle,pin=texte] &   [circle,pin=60:texte] & [circle,pin=right:texte]
 \\ \hline 
\end{tabular}  

\bigskip
\begin{tabular}{|c|c|c|} \hline
\multicolumn{3}{|c|}{  \BS{tikz}[\RDD{pin position}=60] \BS{node} [circle,pin=texte] \AC{} ;   }\\ 
\hline 
\tikz[pin position=60] \node [circle,draw,blue,pin=texte] {};
&
\tikz[pin distance=0 cm] \node [circle,draw,blue,pin=60:texte] {};
&
\tikz[pin distance=2 cm] \node [circle,draw,blue,pin=60:texte,pin distance=0cm] {};
  \\ \hline
  [\RDD{pin position}=60] & [\RDD{pin distance}=0 cm] & [\RDD{pin distance}=2 cm]
    \\ \hline
  \dft{ : above} & \multicolumn{2}{|c|}{ \dft{ : 3 ex}}
      \\ \hline
\end{tabular}  

\newpage

   \shorthandon{:} 
   
\selectlanguage{english}   

\SbSSCT{ N\oe uds  sur un chemin}{Nodes on a path}

\RRR{17-8}

\begin{tabular}{|c|c|c|} \hline
\multicolumn{3}{|c|}{  \BS{draw}(0,0) .. controls (1,2) and (2,-1) .. (4,0) node[\RDD{at end}] \AC{texte} ;   }\\ 
\hline 
\tikz \draw (0,0) .. controls (1,2) and (2,-1) .. (4,0) node[pos=0] {texte}; 
&
\tikz \draw (0,0) .. controls (1,2) and (2,-1) .. (4,0) node[pos=.33] {texte}; 
&
\tikz \draw (0,0) .. controls (1,2) and (2,-1) .. (4,0) node[at end] {texte}; 
  \\ \hline 
\RDD{pos}{\color{red}  =0} & \RDD{pos}{\color{red}  =.33} & \RDD{at end} (pos=1)
  \\ \hline 

\tikz \draw (0,0) .. controls (1,2) and (2,-1) .. (4,0) node[very near end] {texte}; 
&
\tikz \draw (0,0) .. controls (1,2) and (2,-1) .. (4,0) node[near end] {texte}; 
&
\tikz \draw (0,0) .. controls (1,2) and (2,-1) .. (4,0) node[midway] {texte}; 
  \\ \hline 
\RDD{very near end} (pos=0.875.) & \RDD{ near end} (pos=0.75) & \RDD{midway} (pos=0.5)
  \\ \hline 
  
\tikz \draw (0,0) .. controls (1,2) and (2,-1) .. (4,0) node[near start] {texte}; 
&
\tikz \draw (0,0) .. controls (1,2) and (2,-1) .. (4,0) node[very near start] {texte}; 
&
\tikz \draw (0,0) .. controls (1,2) and (2,-1) .. (4,0) node[at start] {texte};
\\ \hline 
\RDD{near start} (pos=0.25) & \RDD{very near start} (pos=0.125) & \RDD{at start} (pos=0)
  \\ \hline 
  
\end{tabular} 

\bigskip
\begin{tabular}{|c|c|c|} \hline
\multicolumn{3}{|c|}{  \BS{draw}(0,0) .. controls (1,2) and (2,1) .. (4,0) node[\RDD{sloped},midway] \AC{texte} ;   }\\ 
\hline 
\tikz \draw (0,0) .. controls (1,2) and (2,-1) .. (4,0) node[sloped,midway] {texte};
&
\tikz \draw (0,0) .. controls (1,2) and (2,-1) .. (4,0) node[above,midway] {texte};
&
\tikz \draw (0,0) .. controls (1,2) and (2,-1) .. (4,0) node[below,midway] {texte};
  \\ \hline
\RDD{sloped} & \RDD{above} &\RDD{below}
  \\ \hline
\end{tabular}
\bigskip

\begin{tabular}{|c|c|c|} \hline
\multicolumn{3}{|c|}{  \BS{draw}(0,0) .. controls (1,2) and (2,1) .. (5,0) node[\RDD{sloped},midway,allow upside down] \AC{texte} ;   }\\ 
\hline 
\tikz \draw (0,0) .. controls (1,2) and (2,-1) .. (4,0) node[sloped,midway,allow upside down] {texte};
&
\tikz \draw (0,0) .. controls (1,2) and (2,-1) .. (4,0) node[above,midway,allow upside down] {texte};
&
\tikz \draw (0,0) .. controls (1,2) and (2,-1) .. (4,0) node[below,midway,allow upside down] {texte};
  \\ \hline
\RDD{sloped} & \RDD{above} &\RDD{below}
  \\ \hline
\end{tabular}  


\begin{tabular}{|c|c|c|} \hline
\multicolumn{3}{|c|}{  \BS{draw}(A)  to [bend right]  node [\RDD{bend right}] \AC{texte} (B);   }\\ 
\hline 
\begin{tikzpicture} 
\node[draw] (A) at (0,0) {A};
\node[draw] (B) at (2,2) {B};
\draw (A) to [bend right] node [bend right] {texte} (B);
\end{tikzpicture}
&
\begin{tikzpicture} 
\node[draw] (A) at (0,0) {A};
\node[draw] (B) at (2,2) {B};
\draw (A) to [bend right] node [auto,bend right] {texte} (B);
\end{tikzpicture}
&
\begin{tikzpicture} 
\node[draw] (A) at (0,0) {A};
\node[draw] (B) at (2,2) {B};
\draw (A) to[bend right] node [auto,swap,bend right] {texte} (B);
\end{tikzpicture}
  \\ \hline
[bend right]  & [\RDD{auto},bend right] & [auto,\RDD{swap},bend right] 
  \\ \hline
\end{tabular}  

\SbSSCT{ N\oe uds  sur un \og edge\fg}{Nodes on an edge}

\begin{tabular}{|c|c|c|}\hline  
\multicolumn{3}{|c|}{  \BS{draw}(0,0) edge \rouge{["abc", ->]} (4,0);  }\\ 
\multicolumn{3}{|c|}{  \RRR{17-12-2} }\\ 
\hline 
\begin{tikzpicture}[blue] 
\useasboundingbox  (0,-.5) rectangle (4,.5); 
\draw (0,0) edge ["abc", ->] (4,0);
\end{tikzpicture}
&
\begin{tikzpicture}[blue] 
\useasboundingbox  (0,-.5) rectangle (4,.5); 
\draw (0,0) edge ["abc", near start] (4,0);
\end{tikzpicture}
&
\begin{tikzpicture}[blue] 
\useasboundingbox  (0,-.5) rectangle (4,.5); 
\draw (0,0) edge ["abc", style={auto=right}] (4,0);
\end{tikzpicture}
\\ \hline 
["abc", ->]
& 
["abc", near start] &  ["abc", style=\AC{auto=right}] 
\\ \hline  
\begin{tikzpicture}[blue] 
\useasboundingbox  (0,-.5) rectangle (4,.5); 
\draw (0,0) edge [font=\Large,"abc" ] (4,0);
\end{tikzpicture}
&
\begin{tikzpicture}[blue] 
\useasboundingbox  (0,-.5) rectangle (4,.5); 
\draw (0,0) edge ["abc" color=red ] (4,0);
\end{tikzpicture}
&
\begin{tikzpicture}[blue] 
\useasboundingbox  (0,-.5) rectangle (4,.5); 
 \draw (0,0) edge ["abc" '] (4,0);
\end{tikzpicture}
\\ \hline 
[font=\BS{Large},"abc" ] & ["abc" color=red ]
&["abc" ' ]
\\ \hline 

\begin{tikzpicture}[blue] 
\useasboundingbox  (0,-.5) rectangle (4,.75); 
\draw (0,0) edge ["abc" draw ] (4,0);
\end{tikzpicture}
&
\begin{tikzpicture}[blue] 
\useasboundingbox  (0,-.5) rectangle (4,.5); 
\draw (0,0) edge ["abc" inner sep=0pt ] (4,0);
\end{tikzpicture}
&
\begin{tikzpicture}[blue] 
\useasboundingbox  (0,-.5) rectangle (4,.5); 
\draw (0,0) edge ["abc" fill ,fill=yellow ] (4,0);
\end{tikzpicture}
\\ \hline
["abc" draw ]
&
["abc" inner sep=0pt ]
&
["abc" fill ,fill=yellow ]
\\ \hline
\end{tabular} 



\bigskip

\begin{tabular}{|c|} \hline  
\BS{draw}[every edge quotes/.style=\AC{fill=yellow}] (0,0) edge ["abc"] (4,0);
\\ \hline  
\begin{tikzpicture}[blue] 
\useasboundingbox  (0,-.5) rectangle (4,.5); 
 \draw[every edge quotes/.style={fill=yellow}] (0,0) edge ["abc"] (4,0);
\end{tikzpicture}
\\ \hline 
\end{tabular} 

%
%\newpage
%
%\subsection{Positionnement relatif de n\oe uds}
\label{lib-pos}

\maboite{\BS{usetikzlibrary}\AC{positioning}}


\begin{center}
\RRR{17-5-3}
\end{center}

\begin{tabular}{|c|c|c|}  \hline 
\multicolumn{2}{|c|}{\BS{node} (a) at (1,0) [above=.4cm+.6cm,draw] \AC{XXX};} &  \\ \hline 
\begin{tikzpicture}
\draw[help lines] (0,0) grid (3,2);
\node (a) at (1,0) [above=.4cm+.6cm,draw] {XXX};
\draw[->,blue,line width=2pt,dotted] (1,0) -- (a.south) node [midway,right,draw=none,fill=red!10] {.4cm+.6cm} ;
\end{tikzpicture} 
&
\begin{tikzpicture}
\draw[help lines] (0,0) grid (3,2);
\node (a) at (1,0) [above=.5+sin(60),draw] {XXX};
\draw[->,blue,line width=2pt,dotted] (1,0) -- (a.south) node [midway,right,draw=none,fill=red!10] {.5+sin(60)} ;
\end{tikzpicture}  
&
\begin{tikzpicture}
\draw[help lines] (0,0) grid (2,2);
\node (a) at (1,0) [above=1,draw] {XXX};
\draw[->,blue,line width=2pt,dotted] (1,0) -- (a.south) node [midway,right,draw=none,fill=red!10] {1} ;
\end{tikzpicture}  
\\ \hline 
above = \rouge{0.4cm+0.6cm} & above = \rouge{.5+sin(60)}  & above = \rouge{1} \\ 
\hline 
\end{tabular} 

\bigskip

\begin{tabular}{|c|c|} \hline 
\multicolumn{2}{|c|}{\BS{node} (a) at (1,0) [\rouge{above right=3cm and 2cm},draw] \AC{XXX};} \\  \hline 
\begin{tikzpicture}
\draw[help lines] (0,0) grid (5,5);
\node (a) at (1,1) [above right=3cm and 2cm,draw] {XXX};
\draw[->,blue,line width=2pt,dotted] (1,1) |- (a.south west);
\end{tikzpicture}
&  
\begin{tikzpicture}
\draw[help lines] (0,0) grid (5,5);

\node (b) at (1,4) [below right=3cm and 2cm,draw] {XXX};
\draw[->,blue,line width=2pt,dotted] (1,4) |- (b.north west);
\end{tikzpicture}
\\ \hline 
\rouge{above right=3cm and 2cm} & \rouge{below right=3cm and 2cm}
\\ \hline 
\end{tabular}  

\bigskip
 
\begin{tabular}{|c|c|}  \hline 
\begin{tikzpicture}[every node/.style=draw,baseline=1.5cm]
\draw[help lines] (0,0) grid (5,4);
\node (a) at (1,1) {node a};
\node (b) [above=2cm of a.north east] {XXX};
\draw[->,blue,line width=2pt,dotted] (a.north) -- (b.south) node [midway,right,draw=none,fill=red!10] {2cm of a.north east} ;
\end{tikzpicture}
&  
\parbox{8cm}{
\BS{node} (a) at (1,1) \AC{node a}; \\
\BS{node} (b) [\rouge{above=2cm of a.north east}] \AC{XXX};}
\\ \hline 
\end{tabular} 

\bigskip

\begin{tabular}{|c|c|}  \hline 
\begin{tikzpicture}[every node/.style=draw]
\draw[help lines] (0,0) grid (2,3);
\node (a) at (1,0) {node a};
\node (b) [above=1cm of a] {node b};
\node (c) [above=1cm of b] {node c};
\draw[->,blue,line width=2pt,dotted] (a.north) -- (b.south) node [midway,right,draw=none,fill=red!10] {1cm} ;
\draw[->,blue,line width=2pt,dotted] (b.north) -- (c.south) node [midway,right,draw=none,fill=red!10] {1cm} ;
\end{tikzpicture}
&  
\begin{tikzpicture}[every node/.style=draw]
\draw[help lines] (0,0) grid (2,3);
\node (a) at (1,0) {node a };
\node (b) [on grid,above=1cm of a] {node b};
\node (c) [on grid,above=1cm of b] {node c};
\draw[->,blue,line width=2pt,dotted] (a.center) -- (b.center) node [midway,right,draw=none,fill=red!10] {1cm} ;
\draw[->,blue,line width=2pt,dotted] (b.center) -- (c.center) node [midway,right,draw=none,fill=red!10] {1cm} ;
\end{tikzpicture}
\\  \hline 
\BS{node} (a) at (1,0) \AC{node a};  &\BS{node} (a) at (1,0) \AC{node a};   \\ 
\BS{node} (b) [above=1cm of a] \AC{node b};  &\BS{node} (b) [\RDD{on grid},above=1cm of a] \AC{node b};   \\ 
\BS{node} (c) [above=1cm of b] \AC{node c};  &\BS{node} (c) [\RDD{on grid},above=1cm of b] \AC{node c};   \\ 
\hline 
\end{tabular} 

\begin{tabular}{|c|c|} \hline 
\begin{tikzpicture}[every node/.style=draw,node distance=1cm,baseline = 1.5cm]
\draw[help lines] (0,0) grid (2,3);
\node (a1) at (1,0) {node a};
\node (b) [above=of a] {node b};
\node (c) [above=of b] {node c};
\draw[->,blue,line width=2pt,dotted] (a.north) -- (b.south) node [midway,right,draw=none,fill=red!10] {1cm} ;
\draw[->,blue,line width=2pt,dotted] (b.north) -- (c.south) node [midway,right,draw=none,fill=red!10] {1cm} ;
\end{tikzpicture}
 & 
 \parbox{12cm}{ 
\BS{begin}\AC{tikzpicture}[every node/.style=draw,\RDD{node distance}=1mm] \\
\BS{node} (a1) at (1,0) \AC{node a}; \\
\BS{node} (b) [above=of a] \AC{node b}; \\
\BS{node} (c) [above=of b] \AC{node c}; \\
\BS{end}\AC{tikzpicture}
} 
 \\ 
\hline 
\end{tabular} 

\bigskip

\begin{tabular}{|l|l|} \hline 
\begin{tikzpicture}[node distance=2cm]
\draw[help lines] (0,-1) grid (6,1);
\huge
\node[draw] (X) at (0,0) {X};
\node[draw] (a) [right=of X] {a};
\node[draw] (y) [right=of a] {y};
\draw[->,blue,line width=2pt,dotted] (X.east) -- (a.west) node [midway,draw=none,fill=red!10] {\small{2cm}} ;
\draw[->,blue,line width=2pt,dotted] (a.east) -- (y.west) node [midway,draw=none,fill=red!10] {\small{2cm}} ;
\end{tikzpicture}
&  
\begin{tikzpicture}[node distance=2cm]
\draw[help lines] (0,-1) grid (6,1);
\huge
\node[draw] (X) at (0,0) {X};
\node[draw] (a) [base right=of X] {a};
\node[draw] (y) [base right=of a] {y};
\draw[->,blue,line width=2pt,dotted] (X.base east) -- (a.base west) node [midway,draw=none,fill=red!10] {\small{2cm}} ;
\draw[->,blue,line width=2pt,dotted] (a.base east) -- (y.base west) node [midway,draw=none,fill=red!10] {\small{2cm}} ;
\end{tikzpicture}
\\ \hline 
\BS{node}[draw] (X) at (0,0) \AC{X};
&  
\BS{node}[draw] (X) at (0,0) \AC{X};
\\
\BS{node}[draw] (a) [right=of X] \AC{a};
&
\BS{node}[draw] (a) [base right=of X] \AC{a};
\\
\BS{node}[draw] (y) [right=of a] \AC{y};
&
\BS{node}[draw] (y) [base right=of a] \AC{y};
\\ \hline 
\end{tabular} 


%
%\newpage
%
%\SSCT{Mettre du texte  en valeur}{Text highlighting}
%
%\label{ndbt}

\tikzset{blue}


\SbSSCT{Dans un n\oe ud de Tikz}{In a TikZ node}
\label{noeudboite}

\begin{tabular}{|c | c | c | c |} \hline
\multicolumn{4}{|c|}{ \BS{tikz} \BS{draw} (0,0) grid (2,2) (1,1) node[ fill=red!20 ] \AC{texte};   }\\ 
\hline 
\tikz \draw (0,0) grid (2,2) (1,1) node[fill=red!20] {texte};
&
\tikz \draw (0,0) grid (2,2) (1,1) node[fill=red!20,draw] {texte}; 
&
\tikz \draw (0,0) grid (2,2) (1,1) node[circle,fill=red!20] {texte};
&
\tikz \draw (0,0) grid (2,2) (1,1) node[circle,fill=red!20,draw] {texte};
\\  \hline
node[fill=red!20] 
&
node[fill=red!20,\RDD{draw}] 
&
 node[fill=red!20,\RDD{circle}]  
&
 node[fill=red!20,\RDD{circle},\RDD{draw}]
 \\  \hline
\end{tabular}
\bigskip


\subsubsection{Options}
\begin{tabular}{|c | c | c | c |c |c |c |c |} \hline
\multicolumn{8}{|c|}{ \BS{tikz} \BS{draw} node[draw,\RDD{double},blue] \AC{texte};   }\\ 
\hline 

\tikz \draw  node[draw,double,blue] {texte};
&
\tikz \draw  node[draw,rounded corners,blue] {texte};
&
\tikz \draw  node[draw,ultra thick,blue] {texte};
&
\tikz \draw  node[draw,dashed,blue] {texte};
&
\tikz \draw  node[draw,red] {texte};
&
\tikz \draw  node[draw,rotate=45,blue] {texte};
&
\tikz \draw  node[draw,shading=radial,blue] {texte};
&
\tikz \draw  node[draw,blue,text=red] {texte};
\\ \hline
\RDD{double} & \RDD{rounded corners} &  ultra thick & dashed & red & rotate=45 & shading=radial & text=red 
\\ \hline
\end{tabular}
\bigskip


\begin{tabular}{|c | c | c | c |c |} \hline
\multicolumn{4}{|c|}{ \BS{tikz} \BS{draw}  node[draw,\RDD{inner sep}=0pt] \AC{texte}; \RRR{17-2-3}  }\\ 
\hline 
\tikz \draw  node[draw,inner sep=0pt,blue] {texte};
&
\tikz \draw node[draw,inner sep=1cm,blue] {texte};
&
\tikz \draw  node[draw,inner xsep=1cm,blue] {texte};
&
\tikz \draw  node[draw,inner ysep=1cm,blue] {texte};
\\ \hline
 \RDD{inner sep}=0pt & \RDD{inner sep}=1cm & \RDD{inner xsep}=1cm & \RDD{inner ysep}=1cm
\\ \hline
\multicolumn{4}{|c|}{ \dft{} : 0.3333em }\\ 
\hline 

\end{tabular}

\bigskip

\begin{tabular}{|c | c | c | c |} \hline
\multicolumn{4}{|l|}{ \BS{node} [fill=red!20,\RDD{outer sep}=1cm] (A) at (1,1) \AC{texte}; \RRR{17-2-3} } \\ 
\multicolumn{4}{|l|}{ \BS{fill} (node cs:name=A,anchor=east) circle (3pt);  }\\ 
\multicolumn{4}{|l|}{ \BS{fill} (node cs:name=A,anchor=south) circle (3pt);  }\\ 
\hline 
\begin{tikzpicture}
\draw[help lines] (0,0) grid (3,2);
\node[fill=red!20,outer sep=1cm] (A) at (1,1) {texte};
\fill[red] (node cs:name=A,anchor=east) circle (3pt);
\fill[red] (node cs:name=A,anchor=south) circle (3pt);
\end{tikzpicture}
&
\begin{tikzpicture}
\draw[help lines] (0,0) grid (3,2);
\node[fill=red!20,outer sep=0pt] (A) at (1,1) {texte};
\fill[red] (node cs:name=A,anchor=east) circle (3pt);
\fill[red] (node cs:name=A,anchor=south) circle (3pt);
\end{tikzpicture}
&
\begin{tikzpicture}
\draw[help lines] (0,0) grid (3,2);
\node[fill=red!20,outer xsep=1cm] (A) at (1,1){texte};
\fill[red] (node cs:name=A,anchor=east) circle (3pt);
\fill[red] (node cs:name=A,anchor=south) circle (3pt);
\end{tikzpicture}
&
\begin{tikzpicture}
\draw[help lines] (0,0) grid (3,2);
\node[fill=red!20,outer ysep=1cm] (A) at (1,1) {texte};
\fill[red] (node cs:name=A,anchor=east) circle (3pt);
\fill[red] (node cs:name=A,anchor=south) circle (3pt);
\end{tikzpicture}
\\ \hline
 \RDD{outer sep}=1cm & \RDD{outer sep}=0pt & \RDD{outer xsep}=1cm & \RDD{outer ysep}=1cm
\\ \hline
\multicolumn{4}{|c|}{ \dft{} : 0.5\BS{pgflinewidth} }\\ 
\hline 
\end{tabular}

\SbSbSSCT{Taille minimale des noeuds}{Minimum size}

\begin{tabular}{|c|c|} \hline  
\multicolumn{2}{|c|}{  \BS{draw}((0,0) node[fill=blue!20,\RDD{minimum height}=1.5cm,draw]  \AC{texte} ;  \RRR{17-2-3}  }\\ 
\hline 
\tikz \draw (0,0) node[fill=red!20,minimum height=1.5cm,draw] {texte};
&  
\tikz \draw (0,0) node[fill=red!20,minimum width=3cm,draw] {texte};

\\ \hline  

\RDD{minimum height}=1.5cm
&  
\RDD{minimum width}=3cm
\\ \hline  
\tikz \draw (0,0) node[fill=red!20,minimum size=1.5cm,draw] {texte};
&  
\tikz \draw (0,0) node[fill=red!20,minimum size=1.5cm,draw,circle] {texte};

\\ \hline 
\RDD{minimum size}=1.5cm,draw
&  
\RDD{minimum size}=1.5cm,circle

\\ \hline 
\end{tabular} 

\newpage

\SbSSCT{Dans un n\oe ud à formes géométriques}{Geometric Shapes nodes}

\label{lib-geom}
\label{formes}


 \maboite{\BS{usetikzlibrary}\AC{shapes.geometric}}
 
 
\begin{center}
\RRR{67-3}
\end{center}

\SbSbSSCT{Formes disponibles}{Available shapes}

\label{nd1}

\begin{tabular}{|c|c|c|c|} \hline  
\multicolumn{4}{|l|}{ 2 syntaxes :   }\\ 
\multicolumn{4}{|l|}{ \BS{tikz} \BS{node}[fill=green!20,\RDD{shape}=diamond,draw,blue] \AC{texte};   }\\ 
\multicolumn{4}{|l|}{ \BS{tikz} \BS{node}[fill=green!20,\RDD{diamond},draw] \AC{texte};   }\\ 
\hline 
\tikz  \node[fill=green!20,diamond,draw] {texte}; 
&  
\tikz  \node[fill=green!20,ellipse,draw] {texte};
&  
\tikz  \node[fill=green!20,trapezium, regular polygon sides=6,draw] {texte};
&
\tikz  \node[fill=green!20,semicircle,draw] {texte}; 
\\ \hline 
diamond & ellipse  & trapezium & semicircle
\\ \hline 
\tikz  \node[fill=green!20,star,draw] {texte};
&  
\tikz  \node[fill=green!20,regular polygon,draw] {texte};
&  
\tikz  \node[fill=green!20,isosceles triangle,draw] {texte};
&
\tikz  \node[fill=green!20,kite,draw] {texte};
\\ \hline 
star & regular polygon  & isosceles triangle & kite 
\\ \hline 
\tikz  \node[fill=green!20,dart,draw] {texte};
&
\tikz  \node[fill=green!20,circular sector,draw] {texte};
&
\tikz  \node[fill=green!20,cylinder,draw] {texte};
&

\\ \hline 
dart & circular sector & cylinder &
\\ \hline 
\end{tabular} 

\subsubsection{Options}

\begin{tabular}{|c|c|c|} \hline
\multicolumn{3}{|c|}{  \BS{node} [trapezium,draw,\RDD{trapezium left angle}=90,draw,blue] \AC{texte};   }\\ 
\hline
\begin{tikzpicture}
\node[trapezium,draw,red,dashed] {texte};
\node[trapezium,draw,trapezium left angle=90,draw,blue] {texte};
\end{tikzpicture}
& 
\begin{tikzpicture}
\node[trapezium,draw,red,dashed] {texte};
\node[trapezium,draw,trapezium right angle=90,draw,blue] {texte};
\end{tikzpicture} 
& 
\begin{tikzpicture}
\node[trapezium,draw,red,dashed] {texte};
\node[trapezium,draw,trapezium angle=120,draw,blue] {texte};
\end{tikzpicture} 
\\ \hline
\RDD{trapezium left angle}=90  & \RDD{trapezium right angle}=90  & \RDD{trapezium  angle}=120 \\ 
\hline 
\begin{tikzpicture}
\node[trapezium,draw,red,dashed] {texte};
\node[trapezium,draw,minimum height=1.5cm,trapezium stretches=true,draw,blue] {texte};
\end{tikzpicture}
& 
\begin{tikzpicture}
\node[trapezium,draw,red,dashed] {texte};
\node[trapezium,draw,minimum height=1.5cm,trapezium stretches=false,draw,blue] {texte};
\end{tikzpicture} 
& 
\begin{tikzpicture}
\node[trapezium,draw,red,dashed] {texte};
\node[trapezium,draw,minimum width=3cm,trapezium stretches =false,draw,blue] {texte};
\end{tikzpicture} 

\\ \hline
minimum height=1.5cm & minimum height=1.5cm & minimum width=1.5cm \\
\RDD{trapezium stretches}=true & \RDD{trapezium stretches}=false & \RDD{trapezium stretches}  \\ 
\hline

\end{tabular} 


\bigskip
\begin{tabular}{|c|c|c|} \hline
\multicolumn{3}{|c|}{ \BS{tikz} \BS{node} [fill=green!20,star,\RDD{star points}=6,draw] \AC{texte};   }\\ 
\hline
\begin{tikzpicture}
\node[star,draw,red,dashed] {texte};
\node[star,star points=7,draw,blue] {texte};
\end{tikzpicture}
&  
\begin{tikzpicture}
\node[star,draw,red,dashed] {texte};
\node[star,star point height = 2cm,draw,blue] {texte};
\end{tikzpicture} 
&  
\begin{tikzpicture}
\node[star,draw,red,dashed] {texte};
\node[star,star point ratio = 3,draw,blue] {texte};
\end{tikzpicture} 
\\ \hline  
\RDD{star points}=7 & \RDD{star point height} = 2cm & \RDD{star point ratio} = 3 \\ \hline
\dft{5} & \dft.5cm &  \dft{1.5}\\ 
\hline 
\end{tabular} 
\bigskip

\begin{tabular}{|c|c|c|} \hline
\multicolumn{3}{|c|}{  \BS{node} [isosceles triangle,\RDD{isosceles triangle apex angle}=90,draw,blue] \AC{texte};   }\\ 
\multicolumn{3}{|c|}{  \BS{node} [regular polygon, \RDD{regular polygon sides}=6,draw,blue] \AC{texte};   }\\ 
\hline
\begin{tikzpicture}
\node[isosceles triangle,draw,red,dashed] {texte};
 \node[isosceles triangle,isosceles triangle apex angle=90,draw,blue] {texte};
\end{tikzpicture} 
& 
\begin{tikzpicture}
\node[isosceles triangle,draw,red,dashed] {texte};
 \node[isosceles triangle,isosceles triangle stretches=true,draw,blue] {texte};
\end{tikzpicture}
&  
\begin{tikzpicture}
\node[regular polygon,draw,red,dashed] {texte};
\node[regular polygon, regular polygon sides=6,draw,blue] {texte};
\end{tikzpicture} 
\\ \hline  
\RDD{isosceles triangle apex angle}=90 & \RDD{isosceles triangle stretches} & \RDD{regular polygon sides}=6 \\ 
\hline 
\end{tabular} 
\bigskip

\begin{tabular}{|c|c|c|} \hline 
\multicolumn{3}{|c|}{  \BS{node} [kite,\RDD{kite upper vertex angle}=90,draw,blue] \AC{texte};   }\\ 
\hline 
\begin{tikzpicture}
\node[red,kite,draw,dashed] {texte} ;
 \node[kite,kite upper vertex angle=90,draw,blue] {texte};
\end{tikzpicture} 
&  
\begin{tikzpicture}
\node[red,kite,draw,dashed] {texte} ;
 \node[kite,kite lower vertex angle=90,draw,blue] {texte};
\end{tikzpicture} 
&  
\begin{tikzpicture}
\node[red,kite,draw,dashed] {texte} ;
\node[kite,kite vertex angles=90,draw,blue] {texte};
\end{tikzpicture} 
\\ \hline  
\RDD{kite upper vertex angle}=90 & \RDD{kite lower vertex angle}=90 &\RDD{kite vertex angles}=90
\\ \hline 
initially 120 & initially 60 &  \\ 
\hline 
\end{tabular} 

\bigskip

\begin{tabular}{|c|c|c|} \hline
\multicolumn{3}{|c|}{  \BS{node} [dart,\RDD{dart tip angle}=90,draw,blue] \AC{texte};   }\\ 
\hline 
\begin{tikzpicture}
\node[dart,draw,red,dashed] {texte};
\node[dart,dart tip angle=90,draw,blue] {texte};
\end{tikzpicture} 
&  
\begin{tikzpicture}
\node[dart,draw,red,dashed] {texte};
\node[dart,dart tail angle=90,draw,blue] {texte};
\end{tikzpicture} 
&  
\begin{tikzpicture}
\node[,circular sector,draw,red,dashed] {texte};
\node[circular sector,circular sector angle=90,draw,blue] {texte};
\end{tikzpicture} 
\\ \hline  
\RDD{dart tip angle}=90 & \RDD{dart tail angle}=90  & \RDD{circular sector angle}=90
\\ \hline  
initially 45 & initially 135 & initially 60  \\ 
\hline 
\end{tabular} 

\bigskip

\begin{tabular}{|c|c|} \hline  
\multicolumn{2}{|c|}{  \BS{node} [cylinder,\RDD{aspect=2},draw,blue] \AC{texte};   }\\ 
\hline
\tikz  \node[cylinder,aspect=2,draw,blue] {texte};
& 
 \tikz  \node[cylinder,aspect=4,draw,blue] {texte};
\\ \hline 
\RDD{aspect}=2 & \RDD{aspect}=4 
\\ \hline
\tikz  \node[cylinder,cylinder uses custom fill, cylinder end fill=yellow,draw,blue] {texte};
&  
\tikz  \node[cylinder,cylinder uses custom fill, cylinder body fill=yellow,draw,blue] {texte};
\\ \hline
\RDD{cylinder uses custom fill}, & \RDD{cylinder uses custom fill}, \\ 
\RDD{cylinder end fill}=yellow & \RDD{cylinder body fill}=yellow  \\ 
\hline 
\end{tabular} 

\bigskip

\begin{tabular}{|c|c|c|c|} \hline 
\multicolumn{4}{|c|}{  \BS{draw}(0,0) node[\RDD{shape aspect}=1,diamond,draw]  \AC{texte} ;   }
\\ \hline
 
\tikz \draw (0,0) node[shape aspect=1,diamond,draw,blue] {texte};
&  
\tikz \draw (0,-2) node[shape aspect=2,diamond,draw,blue] {texte};
&
\tikz \draw (0,0) node[shape aspect=3,diamond,draw,blue] {texte};
&
\tikz \draw (0,0) node[shape aspect=4,diamond,draw,blue] {texte};
\\ \hline  
\RDD{shape aspect}=1
&  
\RDD{shape aspect}=2
&
\RDD{shape aspect}=3
&
\RDD{shape aspect}=4
\\ \hline 
\end{tabular} 

\bigskip

\begin{tabular}{|c|} \hline 
\BS{draw} node[\rouge {shape border rotate}=30,shape=dart, draw, \rouge {shape border uses incircle}] \AC{texte};
\\ \hline 
\tikz[] \draw node[shape border rotate=30,shape=dart, draw, shape border uses incircle] {texte};
\\ \hline 
\end{tabular} 

\newpage

\SbSSCT{Dans un n\oe ud en forme de symboles}{Symbol Shapes nodes}

\label{lib-symb}

\maboite{\BS{usetikzlibrary}\AC{shapes.symbols}}

\begin{center}
\RRR{67-4}
\end{center}

\SbSbSSCT{Formes disponibles}{Available shapes}

\label{nd2}

\begin{tabular}{|c|c|c|} \hline  
\tikz  \node[fill=green!20,forbidden sign,draw] {texte};
&  
\tikz  \node[fill=green!20,magnifying glass,draw] {texte};
&  
\tikz  \node[fill=green!20,cloud,draw] {texte};
\\ \hline 
forbidden sign & magnifying glass & cloud
\\ \hline  
\tikz  \node[fill=green!20,starburst,draw] {texte};
&  
\tikz  \node[fill=green!20,signal,draw] {texte};

&  
\tikz  \node[fill=green!20,tape,draw] {texte};
\\ \hline 
starburst & signal & tape
\\ \hline 
\end{tabular} 
\bigskip

\subsubsection{Options}

\begin{tabular}{|c|c|c|} \hline  
\multicolumn{3}{|c|}{  \BS{node}[magnifying glass,\RDD{magnifying glass handle angle}=45,draw,blue]  \AC{texte} ;   }
\\ \hline
\tikz  \node[magnifying glass,magnifying glass handle angle=45,draw,blue] {texte};
&  
\tikz  \node[,magnifying glass,magnifying glass handle aspect=3,draw,blue] {texte};
& 
\tikz  \node[magnifying glass,line width=1ex,draw,blue] {texte};

\\ \hline  
\RDD{magnifying glass handle angle}=45 & \RDD{magnifying glass handle aspect}=3  & line width=1ex  
\\ \hline 
\dft{ : -45} & \dft{ : 1.5}& 
\\ \hline 
\end{tabular} 

\bigskip

\begin{tabular}{|c|c|c|c|} \hline 
\multicolumn{4}{|c|}{  \BS{node} [cloud,\RDD{cloud puffs}=5,draw,blue] \AC{texte};   }\\ 
\hline 
\begin{tikzpicture}
\node[cloud,draw,red,dashed] {texte};
\node[cloud,cloud puffs=5,draw,blue] {texte};
\end{tikzpicture} 
&  
\begin{tikzpicture}
\node[cloud,draw,red,dashed] {texte};
\node[cloud,cloud puff arc=270,draw,blue] {texte};
\end{tikzpicture} 
&  
\begin{tikzpicture}
\node[cloud,draw,red,dashed] {texte};
\node[cloud,cloud ignores aspect=true,draw,blue] {texte};
\end{tikzpicture} 
&
\begin{tikzpicture}
\node[cloud,draw,red,dashed] {texte};
\node[cloud,cloud ignores aspect=false,draw,blue] {texte};
\end{tikzpicture} 
\\ \hline  
\RDD{cloud puffs}=5 & \RDD{cloud puff arc}=270 & \RDD{cloud ignores aspect}=false & \RDD{cloud ignores aspect}=true  \\ 
\hline 
\dft :  10 & \dft :  135 &\multicolumn{2}{|c|}{ \dft :  true } \\ \hline
\end{tabular} 

\bigskip

\begin{tabular}{|c|c|c|c|} \hline 
\multicolumn{4}{|c|}{  \BS{node} [starburst,\RDD{starburst points}=5,draw,blue] \AC{texte};   }\\ 
\hline  
\tikz  \node[starburst,starburst points=5,draw,blue] {texte};
&  
\tikz  \node[starburst,starburst point height=1cm,draw,blue] {texte};
&  
\tikz  \node[starburst,random starburst=50,draw,blue] {texte};
&
\tikz  \node[,starburst,random starburst=0,draw,blue] {texte};
\\ \hline  
\RDD{starburst points}=5 & \RDD{starburst point height}=1cm & \RDD{random starburst}=50 & \RDD{random starburst}=0  \\ 
\hline 
\end{tabular} 

\bigskip


\begin{tabular}{|c|c|c|} \hline 
\multicolumn{3}{|c|}{  \BS{node} [signal,\RDD{signal pointer angle}=45,draw,blue] \AC{texte};   }\\ 
\hline 
\tikz  \node[signal,signal pointer angle=45,draw,blue] {texte};
&
\tikz  \node[signal,signal pointer angle=10,draw,blue] {texte};
&
\tikz  \node[signal,signal pointer angle=300,draw,blue] {texte};
\\ \hline 
\RDD{signal pointer angle}=45
&
signal pointer angle=10
&
signal pointer angle=300
\\ \hline 
\multicolumn{3}{|c|}{  \dft{ : signal pointer angle= 90}  }
\\  \hline 

\end{tabular} 
\bigskip

\begin{tabular}{|c|c|c|c|c|} \hline 
\multicolumn{4}{|c|}{  \BS{node} [signal,\RDD{signal to}=above,draw,blue] \AC{texte};   }
\\ \hline 
\tikz  \node[signal,signal to=above,draw,blue] {texte};
&  
\tikz  \node[signal,signal to=below,draw,blue] {texte};
&
\tikz  \node[signal,signal to=right,draw,blue] {texte};
&
\tikz  \node[signal,signal to=above,draw,blue] {texte};
\\ \hline  
  \RDD{signal to}=above  & \RDD{signal to}=below & \RDD{signal to}=right  & \RDD{signal to}=above \\ 
\hline 
\end{tabular} 
\bigskip

\begin{tabular}{|c|c|c|c|c|} \hline 
\multicolumn{4}{|c|}{ \BS{tikz} [signal to=nowhere] \BS{node} [signal,\RDD{signal from=above}=45,draw,blue] \AC{texte};   }\\ 
\hline 
\tikz [signal to=nowhere] \node[signal,signal from=above,draw,blue] {texte};
&  
\tikz [signal to=nowhere] \node[signal,signal from=below,draw,blue] {texte};
&
\tikz [signal to=nowhere] \node[signal,signal from=right,draw,blue] {texte};
&
\tikz [signal to=nowhere] \node[signal,signal from=above,draw,blue] {texte};
\\ \hline  
  \RDD{signal from}=above  & \RDD{signal from}=below & \RDD{signal from}=right  & \RDD{signal from}=above \\ 
\hline 
\end{tabular} 

\bigskip
\begin{tabular}{|c|c|c|c|} \hline
\multicolumn{2}{|c|}{ \tikz  \node[draw,signal, signal from=east , signal to=west,blue] at (0,0) {texte};}
&
\multicolumn{2}{|c|}{ \tikz  \node[draw,signal,signal from=south, signal to=north,blue] at (0,0) {texte};}
\\ \hline 
\multicolumn{2}{|c|}{ \RDD{signal from}=east , \RDD{signal to}=west}
&
\multicolumn{2}{|c|}{\RDD{signal from}=south, \RDD{signal to}=north}

\\ \hline 
\end{tabular}
\bigskip

\begin{tabular}{|c | c | c | c |} \hline
\multicolumn{3}{|c|}{ \BS{tikz} \BS{node}  [tape, draw,\RDD{tape bend top}=out and in] \AC{texte};   }\\ 
\hline  
\tikz \node [tape, draw,tape bend top=out and in,blue] {texte};
&
\tikz \node [tape, draw, tape bend bottom=out and in,blue] {texte};
&
\tikz \node [tape, draw, tape bend bottom=in and in,blue] {texte};
 \\  \hline
 \RDD{tape bend top}=out and in & \RDD{tape bend bottom}=out and in &  \RDD{tape bend bottom}=in and in 
  \\  \hline
 \tikz \node [tape, draw, tape bend top=none,blue] {texte};
 &
 \tikz \node [tape, draw,tape bend top=out and in,tape bend bottom=out and in,blue] {texte};
 &
  \tikz \node [tape, draw,tape bend top=in and out,tape bend bottom=in and out,blue] {texte};
  \\  \hline
 \RDD{tape bend top}=none & \RDD{tape bend bottom}=out and in 	&  \RDD{tape bend bottom}=in and out  \\
 					& \RDD{tape bend top}=out and in 		& \RDD{tape bend top}=in and out  \\
 					& & (\dft{} ) 
  \\  \hline 
\end{tabular}
\bigskip

\begin{tabular}{|c | c | c | c |} \hline
\BS{tikz} \BS{node} [tape, draw, \RDD{tape bend height}=1cm,blue] \AC{texte}; 
  \\  \hline 
\tikz \node [tape, draw, tape bend height=1cm,blue] {texte};

  \\  \hline 
\dft{ : tape bend height = 5pt}
  \\  \hline 
\end{tabular}

\newpage

\SbSSCT{Dans un n\oe ud en forme de flèche}{Arrow Shapes nodes}

\label{lib-arr}

\maboite{\BS{usetikzlibrary}\AC{shapes.arrows}}

\begin{center}
\RRR{67-5}
\end{center}

\SbSbSSCT{Formes disponibles}{Available shapes}
\label{nd3}

\begin{tabular}{|c|c|c|} \hline  
\tikz \node[fill=green!20,single arrow,draw] {texte};
&  
\tikz  \node[fill=green!20,double arrow,draw] {texte};
&  
\tikz  \node[fill=green!20,arrow box,draw] {texte};
\\ \hline 
single arrow & double arrow & arrow box \\ 
\hline 
\end{tabular} 

\subsubsection{Options}

\begin{tabular}{|c|c|c|c|c|} \hline  
 \multicolumn{5}{|c|}{  \BS{node}[single arrow,draw,\RDD{single arrow tip angle}=45] \AC{texte};   }\\ 
  \multicolumn{5}{|c|}{  \BS{node}[single arrow,draw,\RDD{single arrow head extend}=.75cm] \AC{texte};   }\\
 \hline
\begin{tikzpicture}
 \node[single arrow,draw,red,dashed,text=black] {texte};
 \node[single arrow,draw,single arrow tip angle=45,blue] {texte};
\end{tikzpicture}
&
\begin{tikzpicture}
 \node[single arrow,draw,red,dashed,text=black] {texte};
\node[single arrow,draw,single arrow tip angle=120,blue] {texte};
\end{tikzpicture}
&
\begin{tikzpicture}
 \node[single arrow,draw,red,dashed,text=black] {texte};
 \node[single arrow,draw,single arrow head extend=.75cm,blue] {texte};
\end{tikzpicture}
&
\begin{tikzpicture}
 \node[single arrow,draw,red,dashed,text=black] {texte};
 \node[single arrow,draw,single arrow head extend=0cm,blue] {texte};
 \end{tikzpicture}
 &
 \begin{tikzpicture}
  \node[single arrow,draw,red,dashed,text=black] {texte};
  \node[single arrow,draw,single arrow head extend=-1mm,blue] {texte};
 \end{tikzpicture}

\\ \hline
angle=45 & angle=120 & extend=.75cm] & extend=0cm & extend=-1mm
\\ \hline 
\multicolumn{2}{|c|}{  \dft : single arrow tip angle= 90   }
&
\multicolumn{3}{|c|}{  \dft : single arrow head extend=0.5cm   }
\\ \hline 
\end{tabular} 
\bigskip


\begin{tabular}{|c|c|c|c|} \hline
 \multicolumn{4}{|c|}{  \BS{node}[minimum size=2cm,single arrow,draw,\RDD{single arrow head indent}=1cm,blue] \AC{texte};   }\\ 
 \hline   
\begin{tikzpicture}
 \node[minimum size=2cm,single arrow,draw,red,dashed,text=black] {texte};
\node[minimum size=2cm,single arrow,draw,single arrow head indent=1cm,blue] {texte};
\end{tikzpicture}
&
\begin{tikzpicture}
 \node[minimum size=2cm,single arrow,draw,red,dashed,text=black] {texte};
  \node[minimum size=2cm,single arrow,draw,single arrow head indent=10pt,blue] {texte};
  \end{tikzpicture}
&
\begin{tikzpicture}
 \node[minimum size=2cm,single arrow,draw,red,dashed,text=black] {texte};
  \node[minimum size=2cm,single arrow,draw,single arrow head indent=1ex,blue] {texte};
  \end{tikzpicture}
  &
  \begin{tikzpicture}
   \node[minimum size=2cm,single arrow,draw,red,dashed,text=black] {texte};
    \node[minimum size=2cm,single arrow,draw,single arrow head indent=-1ex,blue] {texte};
    \end{tikzpicture}
\\ \hline
indent=1cm & indent=10pt & indent=1ex & indent=-1ex
\\ \hline 
\end{tabular}
\bigskip

 



\begin{tabular}{|c|c|c|c|c|} \hline
 \multicolumn{5}{|c|}{  \BS{node}[minimum size=2cm,double arrow,draw,\RDD{double arrow tip angle}=45] \AC{texte};   }\\ 
  \multicolumn{5}{|c|}{  \BS{node}[minimum size=2cm,double arrow,draw,\RDD{double arrow head extend}=1ex] \AC{texte};   }\\
   \multicolumn{5}{|c|}{  \BS{node}[minimum size=2cm,double arrow,draw,\RDD{double arrow head indent}=1ex] \AC{texte};   }\\ 
 \hline  
\begin{tikzpicture}
\node[minimum size=2cm,double arrow,draw,red,dashed,text=black] {texte};
\node[minimum size=2cm,double arrow,draw,double arrow tip angle=45,blue] {texte};
\end{tikzpicture}
&
\begin{tikzpicture}
\node[minimum size=2cm,double arrow,draw,red,dashed,text=black] {texte};
\node[minimum size=2cm,double arrow,draw,double arrow tip angle=120,blue] {texte};
\end{tikzpicture}
&
\begin{tikzpicture}
 \node[minimum size=2cm,double arrow,draw,red,dashed,text=black] {texte};
 \node[minimum size=2cm,double arrow,draw,double arrow head extend=1ex,blue] {texte};
   \end{tikzpicture}
&
\begin{tikzpicture}
 \node[minimum size=2cm,double arrow,draw,red,dashed,text=black] {texte};
  \node[minimum size=2cm,double arrow,draw,double arrow head extend=0,blue] {texte};
    \end{tikzpicture}
&
\begin{tikzpicture}
 \node[minimum size=2cm,double arrow,draw,red,dashed,text=black] {texte};
  \node[,minimum size=2cm,double arrow,draw,double arrow head indent=1ex,blue] {texte};
    \end{tikzpicture}
\\ \hline 
angle=45 & angle=120 & extend=1ex & extend=0 & indent=1ex
\\ \hline
\end{tabular}

\bigskip

\begin{tabular}{|c|c|c|c|c|} \hline
\multicolumn{4}{|c|}{ \BS{node} [arrow box, draw, \RDD{arrow box arrows}=\AC{north:.25cm}] \AC{texte}; }\\ 
\hline 
\begin{tikzpicture}
\node[arrow box, draw,red,text=white,dashed] {texte};
\node[arrow box, draw, arrow box arrows={north:.25cm},blue] {texte};
\end{tikzpicture}
& 
\begin{tikzpicture}
\node[arrow box, draw,red,text=white,dashed] {texte};
\node[arrow box, draw, arrow box arrows={west:.25cm},blue] {texte};
\end{tikzpicture}
 &
 \begin{tikzpicture}
 \node[arrow box, draw,red,text=white,dashed] {texte};
 \node[arrow box, draw, arrow box arrows={south:.25cm},blue] {texte};
 \end{tikzpicture}
&
 \begin{tikzpicture}
 \node[arrow box, draw,red,text=white,dashed] {texte};
 \node[arrow box, draw, arrow box arrows={east:.25cm},blue] {texte};
 \end{tikzpicture}   
 \\ \hline
\AC{north:.25cm} & \AC{west:.25cm} & \AC{south:.25cm}& \AC{east:.25cm} 
\\ \hline
\multicolumn{4}{|c|}{  \dft{} : 0.5 cm}
 \\ \hline 
 \end{tabular}
 
 
 \bigskip
 
 \begin{tabular}{|c|c|} \hline
 \multicolumn{2}{|c|}{ \BS{node} [arrow box, draw, \RDD{arrow box tip angle}=45] \AC{texte}; }\\ 
 \hline 
  \begin{tikzpicture}
  \node[arrow box, draw,red,text=white,dashed] {texte};
  \node[arrow box, draw, arrow box tip angle=45,blue] {texte};
  \end{tikzpicture} 
  &
    \begin{tikzpicture}
   \node[arrow box, draw,red,text=white,dashed] {texte};
   \node[arrow box, draw, arrow box head extend=.25cm,blue] {texte};
   \end{tikzpicture}
\\ \hline  
\RDD{arrow box tip angle}=45 & \RDD{arrow box head extend}=.25cm
\\ \hline 
\dft : 90  & \dft : 0.125cm 
\\ \hline 
   \begin{tikzpicture}
   \node[arrow box, draw,red,text=white,dashed] {texte};
   \node[arrow box, draw, arrow box head indent=.25cm,blue] {texte};
   \end{tikzpicture} 
 &
    \begin{tikzpicture}
    \node[arrow box, draw,red,text=white,dashed] {texte};
    \node[arrow box, draw,arrow box shaft width=.25cm,blue] {texte};
    \end{tikzpicture} 
 \\ \hline 
\RDD{arrow box head indent}=.25cm  &  \RDD{arrow box shaft width}=.25cm
 \\ \hline  
 \dft{ : 0cm } &  \dft{ : 0.125cm }
 \\ \hline  
 \end{tabular}

\newpage

\SbSSCT{Dans un n\oe ud en forme de bulle}{Callout Shapes nodes}
\label{lib-call}

 \maboite{\BS{usetikzlibrary}\AC{shapes.callouts}}
 
\begin{center}
\RRR{67-7}
\end{center}

\SbSbSSCT{Formes disponibles}{Available shapes}

\begin{tabular}{|c|c|c|} \hline 
\tikz  \node[fill=green!20,ellipse callout,draw] {texte};
 &  
 \tikz  \node[fill=green!20,rectangle callout,draw] {texte};
  &  
  \tikz  \node[fill=green!20,cloud callout,draw] {texte};
 \\ \hline
 ellipse callout  &  rectangle callout  & cloud callout \\ 
\hline 
\end{tabular} 

\subsubsection{Options}


\begin{tabular}{|c | c | c | c |} \hline
\multicolumn{4}{|c|}{  \BS{node} [rectangle callout,draw,\RDD{callout absolute pointer}={(0,1)}] at (2,1) \AC{texte};   }\\ 
\hline 
\begin{tikzpicture} 
\draw [help lines] grid(3,3);
\node [rectangle callout,draw,blue, callout relative pointer={(0,1)}] at (2,1) {texte};
\end{tikzpicture}
&
\begin{tikzpicture} 
\draw [help lines] grid(3,3);
\node [ellipse callout,draw, callout relative pointer={(0,1)},blue] at (2,1) {texte};
\end{tikzpicture}
&
\begin{tikzpicture} 
\draw [help lines] grid(3,3);
\node [rectangle callout,draw,blue,callout absolute pointer={(0,1)}] at (2,1) {texte};
\end{tikzpicture}
&
\begin{tikzpicture} 
\draw [help lines] grid(3,3);
\node [ellipse callout,draw, callout absolute pointer={(0,1)},blue] at (2,1) {texte};
\end{tikzpicture}
 \\  \hline
\multicolumn{2}{|c|}{ \RDD{callout relative pointer}=\AC{(0,1)} } & 
\multicolumn{2}{|c|}{  \RDD{callout absolute pointer}=\AC{(0,1)} }
 \\  \hline 
 \begin{tikzpicture} 
 \draw [help lines] grid(3,3);
 \node [rectangle callout,draw, callout relative pointer={(0,1)},callout pointer shorten=.5cm,blue] at (2,1) {texte};
 \end{tikzpicture}
 &
  \begin{tikzpicture} 
  \draw [help lines] grid(3,3);
  \node [ellipse callout,draw, callout relative pointer={(0,1)},callout pointer shorten=.5cm,blue] at (2,1) {texte};
  \end{tikzpicture}
  &
 \begin{tikzpicture} 
 \draw [help lines] grid(3,3);
 \node [rectangle callout,draw, callout absolute pointer={(0,1)},callout pointer shorten=.5cm,blue] at (2,1) {texte};
 \end{tikzpicture}
  &
  \begin{tikzpicture} 
  \draw [help lines] grid(3,3);
  \node [ellipse callout,draw, callout absolute pointer={(0,1)},callout pointer shorten=.5cm,blue] at (2,1) {texte};
  \end{tikzpicture}
  \\  \hline
\multicolumn{4}{|c|}{ \RDD{callout pointer shorten}=.5cm} 
  \\  \hline 
\end{tabular}


\bigskip

\begin{tabular}{|c | c | c | c |} \hline
\multicolumn{3}{|c|}{  \BS{node} [ellipse callout,draw,\RDD{callout pointer arc}=1] at (0,1.5) \AC{texte};   }\\ 
\hline
\begin{tikzpicture}
\node[ellipse callout,draw, callout pointer arc=1,blue] at (0,1.5) {texte};
\end{tikzpicture}
&
\begin{tikzpicture}
\node[ellipse callout,draw, callout pointer arc=30,blue] at (0,1.5) {texte};
\end{tikzpicture}
 &
\begin{tikzpicture}
\node[ellipse callout,draw, callout pointer arc=90,blue] at (0,1.5) {texte};
\end{tikzpicture}
  \\  \hline 
   callout pointer arc=1 & callout pointer arc=30 & callout pointer arc=90
  \\  \hline  
  \multicolumn{3}{|c|}{  \dft{ : callout pointer arc=15}}
 \\  \hline  
 \end{tabular}

\bigskip

\begin{tabular}{|c | c | c | c |} \hline
\multicolumn{3}{|c|}{  \BS{node}[draw,cloud callout, aspect=2.5] \AC{texte};   }\\ 
\hline 
 \begin{tikzpicture}
  \node[draw,cloud callout, dashed,red,text=black] {texte};
 \node[draw,cloud callout, cloud puffs=5,blue] {texte};
 \end{tikzpicture}
&
 \begin{tikzpicture}
 \node[draw,cloud callout, dashed,red,text=black] {texte};
 \node[draw,cloud callout, aspect=2.5,blue] {texte};
 \end{tikzpicture}
&
  \begin{tikzpicture}
  \node[draw,cloud callout, dashed,red,text=black] {texte};
  \node[draw,cloud callout,cloud puff arc=120,blue] {texte};
  \end{tikzpicture}
   \\  \hline 
cloud puffs=5 & aspect=2.5 &  cloud puff arc=120
\\  \hline 
 \end{tabular}

\bigskip

\begin{tabular}{|c | c | c | c |c |} \hline
\multicolumn{3}{|c|}{  \BS{node} [draw,cloud callout,\RDD{callout pointer start size}=.1] \AC{texte};   }\\ 
\hline 
  \begin{tikzpicture}
  \node[draw,cloud callout, dashed,red,text=black] {texte};
  \node[draw,cloud callout,callout pointer start size=.1,blue] {texte};
  \end{tikzpicture}
&
  \begin{tikzpicture}
  \node[draw,cloud callout, dashed,red,text=black] {texte};
  \node[draw,cloud callout,callout pointer start size=.8cm,blue] {texte};
  \end{tikzpicture}
&
  \begin{tikzpicture}
  \node[draw,cloud callout, dashed,red,text=black] {texte};
 \node[draw,cloud callout,callout pointer start size=1cm and 0.1cm,blue] {texte};
  \end{tikzpicture}
\\  \hline 
\RDD{callout pointer start size}=.1 &start size=.8cm & start size=20pt and 1pt
\\  \hline 
\multicolumn{3}{|c|}{  \dft{} : callout pointer start size =.2 of callout  }
\\ 
\hline 
  \begin{tikzpicture}
  \node[draw,cloud callout, dashed,red,text=black] {texte};
  \node[draw,cloud callout,callout pointer end size=5,blue] {texte};
  \end{tikzpicture}
&
  \begin{tikzpicture}
  \node[draw,cloud callout, dashed,red,text=black] {texte};
  \node[draw,cloud callout,callout pointer end size=.8cm,blue] {texte};
  \end{tikzpicture}
&
    \begin{tikzpicture}
    \node[draw,cloud callout, dashed,red,text=black] {texte};
    \node[draw,cloud callout,callout pointer segments=3,blue] {texte};
    \end{tikzpicture}
\\  \hline 
\RDD{callout pointer end size}=.5 & \RDD{callout pointer end size}=.8cm & \RDD{callout pointer segments}=3
\\  \hline 
\multicolumn{2}{|c|}{  \dft{} : callout pointer start size = .1 of callout  }
& \dft{} : segments=2
\\  \hline  

 \end{tabular}

\newpage


\SbSSCT{Dans un n\oe ud en diverses formes  diverses}{Miscellaneous Shapes nodes}

\label{lib-misc}


 \maboite{\BS{usetikzlibrary}\AC{shapes.misc}}
 
\begin{center}
\RRR{67-8}
\end{center}

\SbSbSSCT{Formes disponibles}{Available shapes}

\begin{tabular}{|c|c|c|c|} \hline  
\tikz  \node[fill=green!20,cross out,draw] {texte};
&  
\tikz  \node[fill=green!20,strike out,draw] {texte};
&  
\tikz  \node[fill=green!20,rounded rectangle,draw] {texte};
&  
\tikz  \node[fill=green!20,chamfered rectangle,draw] {texte};
\\ \hline  
cross out & strike out & rounded rectangle & chamfered rectangle \\ 
\hline 
\end{tabular} 


\subsubsection{Options}

\paragraph{Options \TFRGB{pour}{for} \og rounded rectangle \fg} :


\begin{tabular}{|c|c|c|c|c|} \hline
\multicolumn{5}{|c|}{  \BS{node} [draw, rounded rectangle,\RDD{rounded rectangle arc length}=270] \AC{texte};   }\\ 

\hline 

\tikz \node[draw, rounded rectangle,rounded rectangle arc length=270,blue] {texte}; 
&
\tikz \node[draw, rounded rectangle,rounded rectangle arc length=180,blue]  {texte}; 
&
\tikz \node[draw, rounded rectangle,rounded rectangle arc length=120,blue] {texte}; 
&
\tikz \node[draw, rounded rectangle,rounded rectangle arc length=90,blue]  {texte}; 
&
\tikz \node[draw, rounded rectangle,rounded rectangle arc length=45,blue] {texte}; 
 \\ \hline 
270 & 180 & 120 & 90& 45 
\\ \hline 


\end{tabular} 

\bigskip


\begin{tabular}{|c|c|c|c|} \hline 
\multicolumn{4}{|c|}{  \BS{node} [draw, rounded rectangle,\RDD{rounded rectangle west arc}=concave] \AC{texte};   }\\ 
\multicolumn{4}{|c|}{  \BS{node} [draw, rounded rectangle,\RDD{rounded rectangle left arc}=concave] \AC{texte};   }\\ 
\hline 
\tikz \node[draw, rounded rectangle,rounded rectangle west arc=concave,blue] {texte}; 
&
\tikz \node[draw, rounded rectangle,rounded rectangle left arc=concave,blue] {texte}; 
&
\tikz \node[draw, rounded rectangle,rounded rectangle west arc=convex,blue] {texte}; 
&
\tikz \node[draw, rounded rectangle,rounded rectangle left arc=none,blue] {texte};
 \\\hline 
concave & convex & none 
 \\\hline 
\end{tabular} 

\bigskip

\begin{tabular}{|c|c|c|c|} \hline 
\multicolumn{3}{|c|}{  \BS{node} [draw, rounded rectangle,\RDD{rounded rectangle east arc}=concave] \AC{texte};   }\\ 
\multicolumn{3}{|c|}{  \BS{node} [draw, rounded rectangle,\RDD{rounded rectangle right arc}=concave] \AC{texte};   }\\ 

\hline 
\tikz \node[draw, rounded rectangle,rounded rectangle east arc=concave,blue] {texte}; 
&
\tikz \node[draw, rounded rectangle,rounded rectangle  east arc=convex,blue] {texte}; 
&
\tikz \node[draw, rounded rectangle,rounded rectangle right arc=none,blue] {texte};
 \\\hline 
concave & convex & none 
 \\\hline 
\end{tabular} 

\paragraph{Options  \TFRGB{pour}{for} \og chamfered rectangle \fg} :


\begin{tabular}{|c|c|c|c|} \hline 
\multicolumn{4}{|c|}{  \BS{node} [draw, chamfered rectangle,\RDD{chamfered rectangle angle}=30] \AC{texte};   }\\ 
\hline 
\tikz \node[draw, chamfered rectangle,chamfered rectangle angle=10,blue] {texte}; 
&
\tikz \node[draw, chamfered rectangle,chamfered rectangle angle=30,blue] {texte}; 
&
\tikz \node[draw,chamfered rectangle,chamfered rectangle angle=60,blue] {texte};
&
\tikz \node[draw,chamfered rectangle,chamfered rectangle angle=80,blue] {texte};
 \\ \hline 
10 & 30 & 60 & 80
\\ \hline 
\multicolumn{4}{|c|}{  \dft :  45 }
  \\\hline  

\end{tabular}

\bigskip

\begin{tabular}{|c|c|c|c|c|} \hline 
\multicolumn{5}{|c|}{  \BS{node} [draw, chamfered rectangle,\RDD{chamfered rectangle xsep}=10pt] \AC{texte};   }\\ 
\hline 
\tikz \node[draw, chamfered rectangle,chamfered rectangle xsep=0pt,blue] {texte}; 
&
\tikz \node[draw, chamfered rectangle,chamfered rectangle xsep=5pt,blue] {texte}; 
&
\tikz \node[draw, chamfered rectangle,chamfered rectangle xsep=10pt,blue] {texte}; 
&
\tikz \node[draw,chamfered rectangle,chamfered rectangle xsep=-10pt,blue] {texte};
&
\tikz \node[draw,chamfered rectangle,chamfered rectangle xsep=2cm,blue] {texte};
 \\\hline 
  xsep=0pt & xsep=5pt & xsep=10pt & xsep=-10pt  & xsep=2cm
  \\\hline  
\multicolumn{5}{|c|}{  \dft :  0.666ex }
  \\\hline   
\end{tabular}

\bigskip

\begin{tabular}{|c|c|c|c|c|} \hline 
\multicolumn{5}{|c|}{  \BS{node} [draw, chamfered rectangle,\RDD{chamfered rectangle ysep}=10pt] \AC{texte};   }\\ 
\hline 
\tikz \node[draw, chamfered rectangle,chamfered rectangle ysep=0pt,blue] {texte}; 
&
\tikz \node[draw, chamfered rectangle,chamfered rectangle ysep=5pt,blue] {texte}; 
&
\tikz \node[draw,chamfered rectangle,chamfered rectangle ysep=10pt,blue] {texte};
&
\tikz \node[draw,chamfered rectangle,chamfered rectangle ysep=-10pt,blue] {texte};
&
\tikz \node[draw,chamfered rectangle,chamfered rectangle ysep=1cm,blue] {texte};
 \\ \hline 
 ysep=0pt & ysep=5pt & ysep=10pt & ysep=-10pt & ysep=1cm
 \\\hline  
\end{tabular}

\bigskip

\begin{tabular}{|c|c|c|c|c|} \hline 
\multicolumn{5}{|c|}{  \BS{node} [draw, chamfered rectangle,\RDD{chamfered rectangle ysep}=10pt] \AC{texte};   }\\ 
\hline 
\tikz \node[draw, chamfered rectangle,chamfered rectangle sep=0pt,blue] {texte}; 
&
\tikz \node[draw, chamfered rectangle,chamfered rectangle sep=5pt,blue] {texte}; 
&
\tikz \node[draw, chamfered rectangle,chamfered rectangle sep=10pt,blue] {texte}; 

&
\tikz \node[draw, chamfered rectangle,chamfered rectangle sep=-10pt,blue] {texte}; 
&
\tikz \node[draw,chamfered rectangle,chamfered rectangle sep=1cm,blue] {texte};
 \\\hline 
 sep=0pt & sep=5pt & sep=10pt& sep=-10pt & sep=1cm
 \\\hline  
\end{tabular}

\bigskip

\begin{tabular}{|c|c|c|c|} \hline 
\multicolumn{3}{|c|}{  \BS{node} [draw, chamfered rectangle,\RDD{chamfered rectangle corners}=north west] \AC{texte};   }\\ 
\hline
\tikz \node[draw, chamfered rectangle,chamfered rectangle corners=north west,blue] {texte}; 
&
\tikz \node[draw, chamfered rectangle,chamfered rectangle corners={north east, south east},blue] {texte}; 
&
\tikz \node[draw,chamfered rectangle,chamfered rectangle corners={north east, south west},blue] {texte};
 \\ \hline 
 north west & \AC{north east, south east}  & \AC{north east, south west}
 \\ \hline 
\end{tabular}

\newpage

\SbSSCT{N\oe uds à plusieurs parties}{Shapes with Multiple Text Parts}

\label{lib-mult}


 \maboite{\BS{usetikzlibrary}\AC{shapes.multipart}}

\begin{center}
\RRR{67-6}
\end{center}



\begin{tabular}{|c|c|c|c|} \hline 
\multicolumn{4}{|c|}{  \BS{node} [\RDD{circle split},draw,fill=green!20]\AC{haut  \BSS{nodepart}\AC{lower} bas };   }\\ 
\hline 
 
\tikz  \node [circle split,draw,blue,fill=green!20] {haut  \nodepart{lower} bas }; 

&  
\tikz  \node [circle solidus,draw,blue,fill=green!20]{haut  \nodepart{lower} bas };
&  
\tikz  \node [ellipse split,draw,blue,fill=green!20]{texte haut  \nodepart{lower} texte bas };
& 
\tikz  \node [rectangle split,draw,blue,fill=green!20]{haut  \nodepart{lower} bas}; 

\\ \hline 
\RDD{circle split} & \RDD{circle solidus} & \RDD{ellipse split} & \RDD{rectangle split} \\ 
\hline 
\end{tabular} 

 \bigskip
 
 \begin{tabular}{|c|c|}  \hline  
 \begin{tikzpicture} [baseline=0pt]
 \node[rectangle split,rectangle split parts=5,draw,blue,fill=green!20] at(0,0)
 {texte 1
 \nodepart{second}
 texte 2
 \nodepart{four}
 texte 3};
 \end{tikzpicture}
&
\parbox[c]{10cm}{
 \BS{node}[rectangle split,\RDD{rectangle split parts}=5,\\
 draw] \\
 \AC{texte 1 \\
 \BSS{nodepart}\AC{second} texte 2 \\
 \BSS{nodepart}\AC{four} texte 3}; \\
 \\
\dft : rectangle split parts=4 }
 \\  \hline 
 \end{tabular} 
 
\bigskip

\begin{tabular}{|c|}\hline  
\BS{node} [rectangle split,rectangle split parts=3,\RDD{rectangle split horizontal},draw,blue] \\
\AC{texte1\BSS{nodepart}\AC{two}texte2\BSS{nodepart}\AC{three}texte3};
\\ \hline  
\tikz \node [rectangle split,rectangle split parts=3, rectangle split horizontal,draw,blue]
{texte 1\nodepart{two}texte 2\nodepart{three}texte 3}; 
\\ \hline 
\end{tabular} 
 
 \bigskip
 
% % % <<<<<<<<<<<<<<<<< A Voir rectangle split allocate boxes= >>>>>>>>>>>>>>>>>>>>>>>>>>>>>>>>

% \begin{tikzpicture} [baseline=0pt]%[every text node part/.style={text centered}]
% \node[rectangle split,draw,rectangle split parts=5,fill=green!20,rectangle split allocate boxes=3] at(0,0)
% {texte 1  \nodepart{second}  texte 2  \nodepart{four}  texte 3};
% \end{tikzpicture}
% 
 
\bigskip
 \begin{tabular}{|c|c|}  \hline  
\begin{tikzpicture}[baseline=0pt]
\node[rectangle split, rectangle split parts=3, draw,blue, text width=2.75cm]
{texte 1
\nodepart{two}
texte 2a \\
texte 2b \\
texte 2c
\nodepart{three}
texte 3a \\
texte 3b};
\end{tikzpicture}
&
\parbox{8cm}{
 \BS{node}[rectangle split,\RDD{rectangle split parts}=5, draw] \\
 \AC{texte 1 \\
 \BSS{nodepart}\AC{second} texte 2a  \BS{}\BS{}texte 2b  \BS{}\BS{}  texte 2c \\
 \BSS{nodepart}\AC{three} texte 3a \BS{}\BS{} texte 3b }; \\
}
 \\  \hline 
 \end{tabular} 
\bigskip


 \begin{tabular}{|c|c|}  \hline  
 \multicolumn{2}{|c|}{  \BS{node}[rectangle split, draw,blue,minimum size = 2cm,\RDD{rectangle split draw splits}= true] } \\
  \multicolumn{2}{|c|}{ 
  \AC{texte 1 \BS{nodepart}\AC{two} texte 2 \BS{nodepart}\AC{three} texte 3 \BS{nodepart}\AC{four} texte 4};   }\\ 
 \hline 
\tikz \node[rectangle split, draw,blue,minimum size = 2cm,rectangle split draw splits= true] {texte 1 \nodepart{two} texte 2 \nodepart{three} texte 3 \nodepart{four} texte 4};
&
\tikz \node[rectangle split, draw,blue,minimum size = 2cm,rectangle split draw splits= false] {texte 1 \nodepart{two} texte 2 \nodepart{three} texte 3 \nodepart{four} texte 4};
 \\ \hline
 \RDD{rectangle split draw splits}= true & \RDD{rectangle split draw splits}= false \\
 \dft &
 \\ \hline 
 \end{tabular}
 
\bigskip

 \begin{tabular}{|c|c|}  \hline  
\multicolumn{2}{|c|}{  
\BS{node} [rectangle split,rectangle split parts=3,draw,\RDD{rectangle split ignore empty parts}=false] }\\
 \multicolumn{2}{|c|}{ \AC{texte 1 \BS{nodepart}\AC{second} \BS{nodepart}\AC{third}texte 3};} 
\\ \hline  
\begin{tikzpicture} 
\node[rectangle split,rectangle split parts=3,draw,blue,rectangle split ignore empty parts=false] {texte 1 \nodepart{second} \nodepart{third}texte 3};
\end{tikzpicture}
&
\begin{tikzpicture}
\node[rectangle split,rectangle split parts=3,draw,blue,rectangle split ignore empty parts] 
{texte 1 \nodepart{second} \nodepart{third}texte 3};
\end{tikzpicture}
 \\  \hline 
\RDD{rectangle split ignore empty parts}=false & \RDD{rectangle split ignore empty parts}=true 
\\ \hline
 \end{tabular}
 
\bigskip

 \begin{tabular}{|c|c|}  \hline  
\multicolumn{2}{|c|}{  
\BS{node} [rectangle split,rectangle split parts=3,draw,\RDD{rectangle split empty part depth}=1cm] }\\
 \multicolumn{2}{|c|}{ \AC{texte 1 \BS{nodepart}\AC{second} \BS{nodepart}\AC{third}texte 3};} 
\\ \hline 
\begin{tikzpicture} 
\node[rectangle split,rectangle split parts=3,draw,blue,rectangle split empty part depth=1cm] {texte 1 \nodepart{second} \nodepart{third}texte 3};
\end{tikzpicture}
&
\begin{tikzpicture} 
\node[rectangle split,rectangle split parts=3,draw,blue,text depth=1cm] {texte 1 \nodepart{second} \nodepart{third}texte 3};
\end{tikzpicture}
\\ \hline 
\RDD{rectangle split empty part depth}=1cm & \RDD{text depth}=1cm
\\ \hline
\dft : 0ex & \dft : 0ex
\\ \hline 
\begin{tikzpicture}
\node[rectangle split,rectangle split parts=3,draw,blue,rectangle split empty part  height=1cm] 
{texte 1 \nodepart{second} \nodepart{third}texte 3};
\end{tikzpicture}
&
\begin{tikzpicture}
\node[rectangle split,rectangle split parts=3,draw,blue,text height=1cm] 
{texte 1 \nodepart{second} \nodepart{third}texte 3};
\end{tikzpicture}
\\  \hline 
\RDD{rectangle split empty part height}=1cm & \RDD{text height}=1cm
\\ \hline
\dft : 1ex & \dft : 1ex
\\ \hline 
 \end{tabular}
 
\bigskip



 \begin{tabular}{|c|c|}  \hline 
 \multicolumn{2}{|c|}{ 
 \BS{node} [rectangle split,rectangle split parts=3,draw,\RDD{rectangle split empty part width}=1cm]   \AC{};  } 
 \\ \hline 
\begin{tikzpicture} 
\node[rectangle split,rectangle split parts=3,draw,blue,rectangle split empty part width=2cm]{};
\end{tikzpicture}

&
\begin{tikzpicture} 
\node[rectangle split,rectangle split parts=3,draw,blue]{}; 
\end{tikzpicture}
\\  \hline 
 \RDD{rectangle split empty part width}=2cm  &  \dft : 1ex
\\ \hline
 \end{tabular} 
 
 \bigskip



% % % % <<<<<<<<<< A voir   /pgf/rectangle split use custom fill= (default true) <<<<<<<<<<<<<<<<<<<<<<<<<<<<
 


 \begin{tabular}{|c|c|}  \hline 
 \tikz[baseline=0pt] \node[rectangle split, draw,blue,minimum size = 2cm,rectangle split part align={center, left,right}] {texte 1 \nodepart{two} texte 2 \nodepart{three} texte 3 \nodepart{four} texte 4};
&
\parbox{8cm}{
\BS{node}[rectangle split, draw,blue,minimum size = 2cm,\\
\RDD{rectangle split part align}=\AC{center, left,right}]\\
 \AC{texte 1 \BS{nodepart}\AC{two} texte 2  \\
 \BS{nodepart}\AC{three} texte 3  \BS{nodepart}\AC{four} texte 4};
}
\\ \hline
 \tikz[baseline=0pt] \node[rectangle split, draw,blue,minimum size = 2cm, rectangle split horizontal,rectangle split part align={center,base, top,bottom}] {texte 1 \nodepart{two} texte 2 \nodepart{three} texte 3 \nodepart{four} texte 4};
 &
 \parbox{8cm}{
 \BS{node}[rectangle split, draw,blue,minimum size = 2cm,\\
 rectangle split horizontal,\\
 \RDD{rectangle split part align}=\AC{center,base, top,bottom}]\\
  \AC{texte 1 \BS{nodepart}\AC{two} texte 2  \\
  \BS{nodepart}\AC{three} texte 3  \BS{nodepart}\AC{four} texte 4};
 }
 \\ \hline
 \end{tabular}
 
\bigskip


 \begin{tabular}{|c|c|}  \hline  
\tikz[baseline=0pt] \node[rectangle split, draw,blue, minimum width=1cm,rectangle split part fill={red, green,cyan}]{};
&
\parbox{12cm}{
\BS{node}[rectangle split, draw,blue, minimum width=1cm,\\
 \RDD{rectangle split part fill}=\AC{red, green,cyan}]\AC{};}
\\ \hline
\end{tabular} 

\newpage

\SbSSCT{Mise en forme du texte}{Text attributes}

\subsubsection{Position}

\begin{center}
\RRR{17-4-3}
\end{center}

\begin{tabular}{|c|c|c|c|} \hline  
\multicolumn{4}{|l|}{ \BS{tikz} \BS{draw} (0,0) node[fill=blue!10,\RDD{text width}=2cm,\RDD{text justified}]   }\\ 

\multicolumn{4}{|l|}{ \AC{Ceci est une démonstration d'un texte  sur une largeur de 2cm};  }\\ 
\hline 
\tikz \draw (0,0) node[fill=blue!10,text width=2cm]
{Ceci est une démonstration d'un texte  sur une largeur de 2cm.};
&  
\tikz \draw (0,0) node[fill=blue!10,text width=2cm,text justified]
{Ceci est une démonstration d'un texte  sur une largeur de 2cm};
&  
\tikz \draw (0,0) node[fill=blue!10,text width=2cm,text centered]
{Ceci est une démonstration d'un texte  sur une largeur de 2cm .};
&  
\tikz \draw (0,0) node[fill=blue!10,text width=2cm,text ragged]
{Ceci est une démonstration d'un texte  sur une largeur de 2cm .};
\\  \hline  
\TFRGB{sans}{without} option & \RDD{text justified} & \RDD{text centered }& \RDD{text ragged}   
\\ \hline  
\tikz \draw (0,0) node[fill=blue!10,text width=2cm,text badly ragged]
{Ceci est une démonstration d'un texte  sur une largeur de 2cm.};
&  
\tikz \draw (0,0) node[fill=blue!10,text width=2cm,text badly centered]
{Ceci est une démonstration d'un texte  sur une largeur de 2cm .};
&
\tikz \draw (0,0) node[fill=blue!10,text width=2cm,align=center]
{Ceci est une démonstration d'un texte  sur une largeur de 2cm .};
&
\tikz \draw (0,0) node[fill=blue!10,text width=2cm,align=flush center]
{Ceci est une démonstration d'un texte  sur une largeur de 2cm .};
\\  \hline 
\RDD{text badly ragged} &  \RDD{text badly centered} &  \RDD{align}=center & \RDD{align}=flush center 
\\  \hline 
\tikz \draw (0,0) node[fill=blue!10,text width=2cm,align=justify]
{Ceci est une démonstration d'un texte  sur une largeur de 2cm .};
&
\tikz \draw (0,0) node[fill=blue!10,text width=2cm,align=flush right]
{Ceci est une démonstration d'un texte  sur une largeur de 2cm .};
&
\tikz \draw (0,0) node[fill=blue!10,text width=2cm,align=right]
{Ceci est une démonstration d'un texte  sur une largeur de 2cm .};
&
\tikz \draw (0,0) node[fill=blue!10,text width=2cm,align=flush left]
{Ceci est une démonstration d'un texte  sur une largeur de 2cm .};
\\ \hline 
\RDD{align}=justify & \RDD{align}=flush right &  \RDD{align}=right & \RDD{align}=flush left
\\ \hline 

\end{tabular} 
\bigskip

\begin{tabular}{|c|c|} \hline 
\tikz[baseline=0cm] \node [draw] {
\begin{tabular}{|c|c|} \hline
AAA & BBB \\ \hline
CCC & DDD \\ \hline
\end{tabular}
};
& 
\parbox{8cm}{
\BS{tikz} \BS{node} [draw] \AC{
\BS{begin}\AC{tabular}\AC{|c|c|} \BS{hline} \\
AAA \& BBB \BS{}\BS{} \BS{hline} \\
CCC \& DDD \BS{}\BS{} \BS{hline} \\
\BS{end}\AC{tabular}
};}
\\ \hline 
\end{tabular} 

\bigskip


\begin{tabular}{|c|c|c|}  \hline 
\multicolumn{3}{|c|}{\BS{tikz}[align=left] \BS{node}[draw] \AC{AAA \rouge{ \BS{}\BS{} } BBBBBBBB \rouge{ \BS{}\BS{} } CC};} \\ \hline
\tikz[align=left] \node[draw] {AAA\\BBBBBBBB\\CC};
&  
\tikz[align=center] \node[draw] {AAA\\BBBBBBBB\\CC};
&
\tikz[align=right] \node[draw] {AAA\\BBBBBBBB\\CC};
\\ \hline
[align=left]  & [align=center] &[align=right] 
\\ \hline
\end{tabular} 


\bigskip

\begin{tabular}{|c|c|} \hline 
\multicolumn{2}{|c|}{\BS{tikz}[align=left] \BS{node}[draw] \AC{AAA  \BS{}\BS{} \rouge{[1cm] } BBBBBBBB };} 
\\ \hline 
\rule[-1cm]{0pt}{1,5cm} \tikz[align=left] \node[draw] {AAA\\[1cm]BBBBBBBB\\}; 
& 
\tikz[align=left] \node[draw] {AAA\\[-1cm]BBBBBBBB\\}; 
\\ \hline 
\rouge{ [1cm] } & \rouge{[ -1cm] }
\\ \hline 
\end{tabular} 

\SbSbSSCT{Couleur et fontes }{Colors and Fonts}

\begin{tabular}{|c|c|c|c|c|c|} \hline  
\tikz \draw (0,0) node[text= red]{Texte.};
&
\tikz \draw (0,0) node[font=\itshape]{Texte.};
&
\tikz \draw (0,0) node[font=\slshape]{Texte.};
&
\tikz \draw (0,0) node[font=\scshape]{Texte.};
&
\tikz \draw (0,0) node[font=\upshape]{Texte.};
&
\tikz \draw (0,0) node[font=\bfseries]{Texte.};
\\ \hline 



[text= red] & [font=\BS{itshape}]  & [font=\BS{slshape}] & [font=\BS{scshape}] & [font=\BS{upshape}] & [font=\BS{bfseries}]
\\ \hline 
\end{tabular} 



\bigskip
 
\SbSbSSCT{Taille des fontes}{Font Sizes}

\begin{tabular}{|c|c|c|c|c|c|c|}\hline
\multicolumn{7}{|c|}{ \BS{tikz} \BS{draw} (0,0) node[\RDD{font}=\BS{tiny}]\AC{Texte.}   }
\\  \hline
\tikz \draw (0,0) node[font=\tiny]{Texte.};
&
\tikz \draw (0,0) node[font=\footnotesize]{Texte.};
&
\tikz \draw (0,0) node[font=\small]{Texte.};
&
\tikz \draw (0,0) node[font=\large]{Texte.};
&
\tikz \draw (0,0) node[font=\Large]{Texte.};
&
\tikz \draw (0,0) node[font=\huge]{Texte.};
&
\tikz \draw (0,0) node[font=\Huge]{Texte.};
\\ \hline \BS{tiny} & \BS{footnotesize}  & \BS{small} & \BS{large} & \BS{Large} & \BS{huge} & \BS{Huge} \\ 
\hline 
\end{tabular} 

\bigskip
\begin{center}
\RRR{17-4-4}
\end{center}

\begin{tabular}{|c|c|c|} \hline  
\tikz \draw (0,0) node[fill=blue!10,text height=1cm,draw]{Texte.};
&  
\tikz \draw (0,0) node[fill=blue!10,text depth=1cm,draw]{Texte.};
&  
\tikz \draw (0,0) node[fill=blue!10,text depth=0.5cm,,text height=.5cm,draw]{Texte.};
\\ \hline  
\RDD{text height}=1cm
&  
\RDD{text depth}=1cm
&
\RDD{text height}=0.5cm, \RDD{text depth}=0.5cm
\\ \hline 
\end{tabular} 

\newpage

\SbSSCT{Positions prédéfinies  sur un n\oe ud}{Positions on a node}
\label{nomnoeud}

\SbSbSSCT{pour l'ensemble des n\oe uds}{For all types of node}
\begin{center}
\RRR{17-5-1}
\end{center}

\begin{tabular}{|c|c|c|c|} \hline  
\begin{tikzpicture}
\node[rectangle,draw,minimum size=3cm] (A) at (1,1) {\Huge texte};
\fill[red] (node cs:name=A,anchor=north west) circle (3pt);
\end{tikzpicture}
&
\begin{tikzpicture}
\node[rectangle,draw,minimum size=3cm] (A) at (1,1) {\Huge texte};
\fill[red] (node cs:name=A,anchor=north) circle (3pt);
\end{tikzpicture}
&
\begin{tikzpicture}
\node[rectangle,draw,minimum size=3cm] (A) at (1,1) {\Huge texte};
\fill[red] (node cs:name=A,anchor=north east) circle (3pt);
\end{tikzpicture}
&
\begin{tikzpicture}
\node[rectangle,draw,minimum size=3cm] (A) at (1,1) {\Huge texte};
\fill[red] (node cs:name=A,anchor=text) circle (3pt);
\end{tikzpicture}
\\ \hline 
north west & north & north east & text
\\ \hline 

\begin{tikzpicture}
\node[rectangle,draw,minimum size=3cm] (A) at (1,1) {\Huge texte};
\fill[red] (node cs:name=A,anchor= west) circle (3pt);
\end{tikzpicture}
&
\begin{tikzpicture}
\node[rectangle,draw,minimum size=3cm] (A) at (1,1) {\Huge texte};
\fill[red] (node cs:name=A,anchor=mid  west) circle (3pt);
\end{tikzpicture}
&
\begin{tikzpicture}
\node[rectangle,draw,minimum size=3cm] (A) at (1,1) {\Huge texte};
\fill[red] (node cs:name=A,anchor= base west) circle (3pt);
\end{tikzpicture}
&
\begin{tikzpicture}
\node[rectangle,draw,minimum size=3cm] (A) at (1,1) {\Huge texte};
\fill[red] (node cs:name=A,anchor= base) circle (3pt);
\end{tikzpicture}
\\ \hline 
west & mid west & base west &  base
\\ \hline
 
\begin{tikzpicture}
\node[rectangle,draw,minimum size=3cm] (A) at (1,1) {\Huge texte};
\fill[red] (node cs:name=A,anchor=east) circle (3pt);
\end{tikzpicture}
&
\begin{tikzpicture}
\node[rectangle,draw,minimum size=3cm] (A) at (1,1) {\Huge texte};
\fill[red] (node cs:name=A,anchor=mid east) circle (3pt);
\end{tikzpicture}
&
\begin{tikzpicture}
\node[rectangle,draw,minimum size=3cm] (A) at (1,1) {\Huge texte};
\fill[red] (node cs:name=A,anchor=base east) circle (3pt);
\end{tikzpicture}
&
\begin{tikzpicture}
\node[rectangle,draw,minimum size=3cm] (A) at (1,1) {\Huge texte};
\fill[red] (node cs:name=A,anchor= mid) circle (3pt);
\end{tikzpicture}
\\ \hline 
east & mid esat & base east & mid
\\ \hline 

\begin{tikzpicture}
\node[rectangle,draw,minimum size=3cm] (A) at (1,1) {\Huge texte};
\fill[red] (node cs:name=A,anchor= south east) circle (3pt);
\end{tikzpicture}
&
\begin{tikzpicture}
\node[rectangle,draw,minimum size=3cm] (A) at (1,1) {\Huge texte};
\fill[red] (node cs:name=A,anchor= south) circle (3pt);
\end{tikzpicture}
&
\begin{tikzpicture}                                       
\node[rectangle,draw,minimum size=3cm] (A) at (1,1) {\Huge texte};
\fill[red] (node cs:name=A,anchor= south west) circle (3pt);
\end{tikzpicture}
&
\begin{tikzpicture}
\node[rectangle,draw,minimum size=3cm] (A) at (1,1) {\Huge texte};
\fill[red] (node cs:name=A,anchor=center ) circle (3pt);
\end{tikzpicture}
\\ \hline 
south east & south & south west & center
\\ \hline
 
\begin{tikzpicture}
\node[rectangle,draw,minimum size=3cm] (A) at (1,1) {\Huge texte};
\fill[red] (node cs:name=A,anchor=0) circle (3pt);
\end{tikzpicture}
&
\begin{tikzpicture}
\node[rectangle,draw,minimum size=3cm] (A) at (1,1) {\Huge texte};
\fill[red] (node cs:name=A,anchor=120) circle (3pt);
\end{tikzpicture}
&
\begin{tikzpicture}
\node[rectangle,draw,minimum size=3cm] (A) at (1,1) {\Huge texte};
\fill[red] (node cs:name=A,anchor=-60) circle (3pt);
\end{tikzpicture}
&


\\ \hline 
0 & 120 & -60 &  
\\ \hline 
\end{tabular}
 
\newpage 

\SbSbSSCT{spécifique à un n\oe ud}{Specific to a node}

\TFRGB{Consultez }{see} \RRR{67 }


\begin{tabular}{|c|c|} \hline 
shape=circle & shape=diamond
\\  \hline 
\begin{tikzpicture}[]
\node[circle,draw,minimum size=3.5cm] (A) at (1,1) {\Huge XXX};
\foreach \anchor/\placement in
{north west/above left, north/above, north east/above right,
west/above left, center/above, east/right,
mid west/left, mid/below right, mid east/right,
base west/below left, base/below, base east/below right,
south west/below left, south/below, south east/below right,
text/below, 20/right, 120/above}
\fill[blue,pin position=\placement] (node cs:name=A,anchor= \anchor) circle (2pt) node[blue,pin=\scriptsize{ \anchor} ] {} ;
\end{tikzpicture}
&
\begin{tikzpicture}[]
\node[diamond,draw,minimum size=3.5cm] (A) at (1,1) {\Huge XXX};
\foreach \anchor/\placement in
{north west/above left, north/above, north east/above right,
west/left, center/above, east/right,
mid/10,
base/below,
south west/below left, south/below, south east/below right,
text/left, 10/right, 120/above}
\fill[blue,pin position=\placement] (node cs:name=A,anchor= \anchor) circle (2pt) node[blue,pin=\scriptsize{ \anchor} ] {} ;
\end{tikzpicture}
\\ \hline 
\end{tabular} 

\bigskip

\begin{tabular}{|c|} \hline 
shape=ellipse
\\  \hline 
\begin{tikzpicture}[]
\node[ellipse,draw,minimum size=3.5cm] (A) at (1,1) {\Huge XXXXXXX};
\foreach \anchor/\placement in
{north west/above left, north/above, north east/above right, west/left, center/above, east/right,
mid west/left, mid/-75, mid east/right,
base west/200, base/-105, base east/-20,
south west/below left, south/below, south east/below right,
text/-75, 10/right, 130/above}
\fill[blue,pin position=\placement] (node cs:name=A,anchor= \anchor) circle (2pt) node[blue,pin=\scriptsize{ \anchor} ] {} ;
\end{tikzpicture}
\\ \hline 
\end{tabular}

\bigskip

\begin{tabular}{|c|} \hline 
shape=trapezium
\\  \hline 
\begin{tikzpicture}[]
\node[ trapezium,draw,minimum size=3cm] (A) at (1,1) {\Huge XXX};
\foreach \anchor/\placement in
{center/120, text/below, mid/-45, base/below, mid west/left, base west/-175, mid east/right, base east/-25,
west/175, east/above, north/-75, south/-60,
north west/above, north east/above,
south west/-150, south east/-30, 150/above}
\fill[blue,pin position=\placement] (node cs:name=A,anchor= \anchor) circle (2pt) node[blue,pin=\scriptsize{ \anchor} ] {} ;

\foreach \anchor/\placement in
{bottom left corner/below, top right corner/right,
top left corner/left, bottom right corner/below,
bottom side/-120, left side/left, right side/right, top side/above}
\fill[red,pin position=\placement] (node cs:name=A,anchor= \anchor) circle (2pt) node[blue,pin=\scriptsize{ \anchor} ] {} ;
\end{tikzpicture}
\\ \hline 
\end{tabular}

\bigskip

\begin{tabular}{|c|} \hline 
shape=semicircle,shape border rotate=0
\\  \hline 
\begin{tikzpicture}[]
\node[ semicircle,shape border rotate=0,draw,minimum size=3cm] (A) at (1,1) {\Huge XXX};
\foreach \anchor/\placement in
{center/above, base/-160, mid/-40, text/left, base west/-120, base east/-60, mid west/left, mid east/right, north/below, south/-75, east/60, west/120, north west/above left, north east/above right, south west/-140, south east/-60, 30/right}
\fill[blue,pin position=\placement] (node cs:name=A,anchor= \anchor) circle (2pt) node[blue,pin=\scriptsize{ \anchor} ] {} ;
\foreach \anchor/\placement in
{apex/above, arc start/-60, arc end/-120, chord center/-100}
\fill[red,pin position=\placement] (node cs:name=A,anchor= \anchor) circle (2pt) node[blue,pin=\scriptsize{ \anchor} ] {} ;
\end{tikzpicture}
\\ \hline 
\end{tabular}

\bigskip

\begin{tabular}{|c|} \hline 
shape=regular polygon
\\  \hline 
\begin{tikzpicture}[]
\node[ regular polygon,draw,minimum size=3cm] (A) at (1,1) {\Huge XXX};
\foreach \anchor/\placement in
{ center/97, text/97 , mid/-30, base/below, 75/above,
west/left, east/right, north/-87, south/-60,
north east/right, south east/right, north west/left, south west/left}
\fill[blue,pin position=\placement] (node cs:name=A,anchor= \anchor) circle (2pt) node[blue,pin=\scriptsize{ \anchor} ] {} ;

\foreach \anchor/\placement in
{corner 1/above, corner 2/left, corner 3/left, corner 4/right, corner 5/right,
side 1/above, side 2/left, side 3/-120, side 4/right, side 5/above}
\fill[red,pin position=\placement] (node cs:name=A,anchor= \anchor) circle (2pt) node[blue,pin=\scriptsize{ \anchor} ] {} ;
\end{tikzpicture}
\\ \hline 
\end{tabular}


\bigskip

\begin{tabular}{|c|} \hline 
shape=star
\\  \hline 
\begin{tikzpicture}[]
\node[  shape=star, star points=5, star point ratio=1.65,draw,minimum size=3cm] (A) at (1,1) {\Huge XXX};
\foreach \anchor/\placement in
{center/above, 
text/below, 
mid/-30, 
base/-80, 
75/above,
west/left, 
east/right, 
north/below, 
south/94,
north east/right, 
south east/right, 
north west/left, 
south west/left}
\fill[blue,pin position=\placement] (node cs:name=A,anchor= \anchor) circle (2pt) node[blue,pin=\scriptsize{ \anchor} ] {} ;

\foreach \anchor/\placement in
{inner point 1/above left, 
inner point 2/left, 
inner point 3/below, 
inner point 4/right,
inner point 5/above right, 
outer point 1/above, 
outer point 2/left, 
outer point 3/left,
outer point 4/right, 
outer point 5/right}
\fill[red,pin position=\placement] (node cs:name=A,anchor= \anchor) circle (2pt) node[blue,pin=\scriptsize{ \anchor} ] {} ;
\end{tikzpicture}
\\ \hline 
\end{tabular}



\bigskip

\begin{tabular}{|c|c|} \hline 
shape= isosceles triangle & shape= kite
\\  \hline 
\begin{tikzpicture}[]
\node[ shape=isosceles triangle,draw,minimum size=3cm] (A) at (1,1) {\Huge XXX};
\foreach \anchor/\placement in
{center/above,text/above,150/left,mid/-5, mid west/left, mid east/right,base/-120, base west/-150, 
base east/below right ,west/left, east/right,  north/above, north west/left, north east/above right,
south /-120 , south east/below right}
\fill[blue,pin position=\placement] (node cs:name=A,anchor= \anchor) circle (2pt) node[blue,pin=\scriptsize{ \anchor} ] {} ;
\foreach \anchor/\placement in
{apex/above, left corner/left, right corner/left,left side/above, right side/below, lower side/160}
\fill[red,pin position=\placement] (node cs:name=A,anchor= \anchor) circle (2pt) node[blue,pin=\scriptsize{ \anchor} ] {} ;
\end{tikzpicture}
&
\begin{tikzpicture}[]
\node[ shape=kite,draw,minimum size=3cm] (A) at (1,1) {\Huge XXX};
\foreach \anchor/\placement in
{center/above, text/85, mid/-85, base/-95,mid west/left, base west/-160, 
mid east/right, base east/-20,west/left, east/right, north/80, south/below left,north west/above left, north east/above right,south west/left, south east/right, 
110/above left}
\fill[blue,pin position=\placement] (node cs:name=A,anchor= \anchor) circle (2pt) node[blue,pin=\scriptsize{ \anchor} ] {} ;
\foreach \anchor/\placement in
{upper vertex/110, 
left vertex/left, 
lower vertex/below right,
right vertex/right, 
upper left side/left, 
upper right side/right,
lower left side/left, 
lower right side/below right}
\fill[red,pin position=\placement] (node cs:name=A,anchor= \anchor) circle (2pt) node[blue,pin=\scriptsize{ \anchor} ] {} ;
\end{tikzpicture}
\\ \hline 
\end{tabular}


\bigskip

\begin{tabular}{|c|c|} \hline 
shape= dart & shape= circular sector
\\  \hline 
\begin{tikzpicture}[]
\node[shape=dart, shape border rotate=90,,draw,minimum size=3cm] (A) at (1,1) {\Huge XXX};
\foreach \anchor/\placement in
{west/left  , east/above right , north/below,south/left,
north west/left, north east/right, south west/below, south east/below,110/above left}
\fill[blue,pin position=\placement] (node cs:name=A,anchor= \anchor) circle (2pt) node[blue,pin=\scriptsize{ \anchor} ] {} ;
\foreach \anchor/\placement in
{tip/above, tail center/right, right tail/below,
left tail/below, right tail/below, left side/above left, right side/above right}
\fill[red,pin position=\placement] (node cs:name=A,anchor= \anchor) circle (2pt) node[blue,pin=\scriptsize{ \anchor} ] {} ;
\end{tikzpicture}
&
\begin{tikzpicture}[]
\node[shape=circular sector,draw,minimum size=3cm] (A) at (1,1) {\Huge XXX};
\foreach \anchor/\placement in
{west/170  , east/right , north/above , south/below, north west/left, north east/above, south west/left, south east/below, 120/left}
\fill[blue,pin position=\placement] (node cs:name=A,anchor= \anchor) circle (2pt) node[blue,pin=\scriptsize{ \anchor} ] {} ;
\foreach \anchor/\placement in
{sector center/above, arc start/above, arc end/below, arc center/190}
\fill[red,pin position=\placement] (node cs:name=A,anchor= \anchor) circle (2pt) node[blue,pin=\scriptsize{ \anchor} ] {} ;
\end{tikzpicture}
\\ \hline 
\end{tabular}



\bigskip

\begin{tabular}{|c|c|} \hline 
shape=cylinder & shape=cloud
\\  \hline 
\begin{tikzpicture}[]
\node[shape=cylinder,draw,minimum size=3cm] (A) at (1,1) {\Huge XXX};
\foreach \anchor/\placement in
{west/170  , east/-10 , north/above , south/below, north west/left, north east/above, south west/left, south east/below, 120/left}
\fill[blue,pin position=\placement] (node cs:name=A,anchor= \anchor) circle (2pt) node[blue,pin=\scriptsize{ \anchor} ] {} ;
\foreach \anchor/\placement in
{before top/10 , top/10, after top/below right, before bottom/below left, bottom/190, after bottom/above left}
\fill[red,pin position=\placement] (node cs:name=A,anchor= \anchor) circle (2pt) node[blue,pin=\scriptsize{ \anchor} ] {} ;
\end{tikzpicture}
&
\begin{tikzpicture}[]
\node[shape=cloud,draw,minimum size=3cm] (A) at (1,1) {\Huge XXX};
\foreach \anchor/\placement in
{west/west  , east/east , north/below , south/below left, north west/left, north east/above right, south west/left, south east/right, 110/above}
\fill[blue,pin position=\placement] (node cs:name=A,anchor= \anchor) circle (2pt) node[blue,pin=\scriptsize{ \anchor} ] {} ;
\foreach \anchor/\placement in
{puff 1/above, puff 2/above left , puff 3/left, puff 4/left,
puff 5/below left, puff 6/below right, puff 7/below right, puff 8/right,
puff 9/right, puff 10/above}
\fill[red,pin position=\placement] (node cs:name=A,anchor= \anchor) circle (2pt) node[blue,pin=\scriptsize{ \anchor} ] {} ;
\end{tikzpicture}

\\ \hline 
\end{tabular}

\bigskip

\begin{tabular}{|c|} \hline 
shape=starburst
\\  \hline 
\begin{tikzpicture}[]
\node[shape=starburst, starburst points=9, starburst point height=2cm,draw,minimum size=3cm] (A) at (1,1) {\Huge XXX};
\foreach \anchor/\placement in
{west/west  , east/east , north/70 , south/above, north west/below , north east/below, south west/below left, south east/-85, 30/above right}
\fill[blue,pin position=\placement] (node cs:name=A,anchor= \anchor) circle (2pt) node[blue,pin=\scriptsize{ \anchor} ] {} ;
\foreach \anchor/\placement in
{outer point 1/105, outer point 2/above left , 
outer point 3/left, outer point 4/left, 
outer point 5/below, outer point 6/below, 
outer point 7/below, outer point 8/right, 
outer point 9/above,
inner point 1/93, inner point 2/160, 
inner point 3/190, inner point 4/below left, 
inner point 5/below, inner point 6/-85,
inner point 7/-30, inner point 8/above right, 
inner point 9/above}
\fill[red,pin position=\placement] (node cs:name=A,anchor= \anchor) circle (2pt) node[blue,pin=\scriptsize{ \anchor} ] {} ;
\end{tikzpicture}
\\ \hline 
\end{tabular}


\bigskip

\begin{tabular}{|c|} \hline 
shape=signal
\\  \hline 
\begin{tikzpicture}[]
\node[signal,signal from=west,draw,minimum size=3.5cm] (A) at (1,1) {\Huge XXX};
\foreach \anchor/\placement in
{north west/above left, north/above, north east/above right,
west/left, center/above, east/right,
mid west/left, mid/below right, mid east/right,
base west/-160, base/below, base east/below right,
south west/below left, south/below, south east/below right,
text/below, 20/right, 120/above}
\fill[blue,pin position=\placement] (node cs:name=A,anchor= \anchor) circle (2pt) node[blue,pin=\scriptsize{ \anchor} ] {} ;
\end{tikzpicture}
\\ \hline 
\end{tabular}


\bigskip

\begin{tabular}{|c|} \hline 
shape=tape
\\  \hline 
\begin{tikzpicture}[]
\node[tape, tape bend height=1cm,draw,minimum size=3.5cm] (A) at (1,1) {\Huge XXX};
\foreach \anchor/\placement in
{north west/above left, north/above, north east/above right,
west/left, center/above, east/right,
mid west/left, mid/below right, mid east/right,
base west/-160, base/110, base east/below right,
south west/below left, south/below, south east/below right,
text/110, 20/right, 120/above}
\fill[blue,pin position=\placement] (node cs:name=A,anchor= \anchor) circle (2pt) node[blue,pin=\scriptsize{ \anchor} ] {} ;
\end{tikzpicture}
\\ \hline 
\end{tabular}

\begin{tabular}{|c|} \hline 
shape=magnetic tape
\\  \hline 
\begin{tikzpicture}[]
\node[shape=magnetic tape,draw,minimum size=3cm] (A) at (1,1) {\Huge XXX};
\foreach \anchor/\placement in
{west/west  , east/east , north/above , south/below, north west/above left , north east/above right, south west/left, south east/below, 30/above right}
\fill[blue,pin position=\placement] (node cs:name=A,anchor= \anchor) circle (2pt) node[blue,pin=\scriptsize{ \anchor} ] {} ;
\foreach \anchor/\placement in
{tail east/right, tail south east/below right, tail north east/above right}
\fill[red,pin position=\placement] (node cs:name=A,anchor= \anchor) circle (2pt) node[blue,pin=\scriptsize{ \anchor} ] {} ;
\end{tikzpicture}
\\ \hline 
\end{tabular}



\bigskip

\begin{tabular}{|c|} \hline 
shape=single arrow
\\  \hline 
\begin{tikzpicture}[]
\node[shape=single arrow,draw,minimum size=3cm] (A) at (1,1) {\Huge XXXXXX};
\foreach \anchor/\placement in
{west/170  , east/below right , north/above , south/below, north west/above left, north east/above right, south west/below left, south east/below right, 30/east}
\fill[blue,pin position=\placement] (node cs:name=A,anchor= \anchor) circle (2pt) node[blue,pin=\scriptsize{ \anchor} ] {} ;
\foreach \anchor/\placement in
{tip/above right, before tip/above, after tip/below, before head/190 , after head/170, after tail/left, before tail/left, tail/190}
\fill[red,pin position=\placement] (node cs:name=A,anchor= \anchor) circle (2pt) node[blue,pin=\scriptsize{ \anchor} ] {} ;
\end{tikzpicture}
\\ \hline 
\end{tabular}


\bigskip

\begin{tabular}{|c|} \hline 
shape=double arrow
\\  \hline 
\begin{tikzpicture}[]
\node[shape=double arrow, double arrow head extend=1.5cm,,draw,minimum size=3cm] (A) at (1,1) {\Huge XXXXXXXXX};
\foreach \anchor/\placement in
{west/170  , east/-10 , north/above , south/below, north west/above left, north east/above right, south west/below left, south east/below right, 35/above right}
\fill[blue,pin position=\placement] (node cs:name=A,anchor= \anchor) circle (2pt) node[blue,pin=\scriptsize{ \anchor} ] {} ;
\foreach \anchor/\placement in
{before head 1/above right, before tip 1/above, 
tip 1/10, after tip 1/below, 
after head 1/below right, before head 2/below left, 
before tip 2/below left, tip 2/190, 
after tip 2/above left, after head 2/above left}
\fill[red,pin position=\placement] (node cs:name=A,anchor= \anchor) circle (2pt) node[blue,pin=\scriptsize{ \anchor} ] {} ;
\end{tikzpicture}
\\ \hline 
\end{tabular}


\bigskip

\begin{tabular}{|c|} \hline 
shape=arrow box
\\  \hline 
\begin{tikzpicture}[]
\node[shape=arrow box,draw,minimum size=3cm,arrow box arrows={north:2cm from border, south, east:2cm from border, west},arrow box shaft width=1cm,arrow box head extend=0.25cm] (A) at (1,1) {\Huge XXXXXXXXX};
\foreach \anchor/\placement in
{west/right  , east/left , north/below , south/above, north west/left, north east/right, south west/left, south east/right}
\fill[blue,pin position=\placement] (node cs:name=A,anchor= \anchor) circle (2pt) node[blue,pin=\scriptsize{ \anchor} ] {} ;
\foreach \anchor/\placement in
{north arrow tip/above,
south arrow tip/below, 
east arrow tip/right, 
west arrow tip/left,
before north arrow/above left, 
before north arrow head/110, 
before north arrow tip/left,
after north arrow tip/right, 
after north arrow head/70, 
after north arrow/above right,
before south arrow/below right, 
before south arrow head/-70, 
before south arrow tip/right,
after south arrow tip/left, 
after south arrow head/-110, 
after south arrow/below left,
before east arrow/above right, 
before east arrow head/right, 
before east arrow tip/right,
after east arrow tip/right, 
after east arrow head/right, 
after east arrow/below right,
before west arrow/below left, 
before west arrow head/left, 
before west arrow tip/left,
after west arrow tip/west, 
after west arrow head/left, 
after west arrow/above left}
\fill[red,pin position=\placement] (node cs:name=A,anchor= \anchor) circle (2pt) node[blue,pin=\scriptsize{ \anchor} ] {} ;
\end{tikzpicture}
\\ \hline 
\end{tabular}


\bigskip

\begin{tabular}{|c|} \hline 
shape=circle split
\\  \hline 
\begin{tikzpicture}[]
\node[shape=circle split,draw,minimum size=3.5cm](A) at (1,1) {XXX\nodepart{lower}YYY}  ;
\foreach \anchor/\placement in
{north west/above left, north/above, north east/above right,
west/left, center/above, east/right,
mid west/left, mid/below right, mid east/right,
base west/-160, base/110, base east/below right,
south west/below left, south/below, south east/below right,
text/110, 20/right, 120/above}
\fill[blue,pin position=\placement] (node cs:name=A,anchor= \anchor) circle (2pt) node[blue,pin=\scriptsize{ \anchor} ] {} ;
\foreach \anchor/\placement in
{text/left, lower/left}
\fill[red,pin position=\placement] (node cs:name=A,anchor= \anchor) circle (2pt) node[blue,pin=\scriptsize{ \anchor} ] {} ;
\end{tikzpicture}
\\ \hline 
\end{tabular}

\begin{tabular}{|c|} \hline 
shape=circle solidus
\\  \hline 
\begin{tikzpicture}[]
\node[shape=circle solidus,draw,minimum size=3.5cm](A) at (1,1) {XXX\nodepart{lower}YYY}  ;
\foreach \anchor/\placement in
{north west/above left, north/above, north east/above right,
west/left, center/above, east/right,
mid west/left, mid/below right, mid east/right,
base west/-160, base/110, base east/below right,
south west/below left, south/below, south east/below right,
text/110, 20/right, 120/above}
\fill[blue,pin position=\placement] (node cs:name=A,anchor= \anchor) circle (2pt) node[blue,pin=\scriptsize{ \anchor} ] {} ;
\foreach \anchor/\placement in
{text/left, lower/left}
\fill[red,pin position=\placement] (node cs:name=A,anchor= \anchor) circle (2pt) node[blue,pin=\scriptsize{ \anchor} ] {} ;
\end{tikzpicture}
\\ \hline 
\end{tabular}


\bigskip

\begin{tabular}{|c|} \hline 
shape=ellipse split
\\  \hline 
\begin{tikzpicture}[]
\node[shape=ellipse split,draw,minimum size=3.5cm](A) at (1,1) {XXX\nodepart{lower}YYY}  ;
\foreach \anchor/\placement in
{north west/above left, north/above, north east/above right,
west/left, center/above, east/right,
mid west/left, mid/below right, mid east/right,
base west/-160, base/110, base east/below right,
south west/below left, south/below, south east/below right,
text/110, 20/right, 120/above}
\fill[blue,pin position=\placement] (node cs:name=A,anchor= \anchor) circle (2pt) node[blue,pin=\scriptsize{ \anchor} ] {} ;
;
\end{tikzpicture}
\\ \hline 
\end{tabular}

\bigskip

\begin{tabular}{|c|} \hline 
shape=rectangle split
\\  \hline 
\begin{tikzpicture}[]
\node[name=s,shape=rectangle split, rectangle split parts=4,draw,inner ysep=0.75cm](A) at (1,1)
{\nodepart{text}XXXXXXXXXXXXXX\nodepart{two}YYY
\nodepart{three}ZZZ\nodepart{four}four};
\foreach \anchor/\placement in
{north/above, south/below, east/10, west/170,
north west/above, north east/above, south west/below, south east/below,
center/145, 20/right, mid/30, base/-145}
\fill[blue,pin position=\placement] (node cs:name=A,anchor= \anchor) circle (2pt) node[blue,pin=\scriptsize{ \anchor} ] {} ;
\foreach \anchor/\placement in
{text split/10, text split east/0, text split west/180,two split/30, two split east/right, two split west/left,
three split/30, three split east/east, three split west/west,text/-170, text east/east, text west/west,
two/left, two east/east, two west/west,
three/left, three east/east, three west/west,
four/west, four east/east, four west/west
}
\fill[red,pin position=\placement] (node cs:name=A,anchor= \anchor) circle (2pt) node[blue,pin=\scriptsize{ \anchor} ] {} ;
\end{tikzpicture}
\\ \hline 
\end{tabular}

\bigskip

\begin{tabular}{|c|} \hline 
shape=rectangle callout
\\  \hline 
\begin{tikzpicture}[]
\node[shape=rectangle callout, callout relative pointer={(1.5cm,-.5cm)},draw,
callout pointer width=2cm, inner xsep=1cm, inner ysep=.5cm] (A) at (1,1) {\Huge XXXXXXX};
\foreach \anchor/\placement in
{west/west  , east/east , north/above , south/below, north west/west , north east/right, south west/left, south east/right, 25/right}
\fill[blue,pin position=\placement] (node cs:name=A,anchor= \anchor) circle (2pt) node[blue,pin=\scriptsize{ \anchor} ] {} ;
\foreach \anchor/\placement in
{pointer/right}
\fill[red,pin position=\placement] (node cs:name=A,anchor= \anchor) circle (2pt) node[blue,pin=\scriptsize{ \anchor} ] {} ;
\end{tikzpicture}
\\ \hline 
\end{tabular}

\bigskip

\begin{tabular}{|c|} \hline 
shape=ellipse callout
\\  \hline 
\begin{tikzpicture}[]
\node[shape=ellipse callout,draw] (A) at (1,1) {\Huge XXXXXX};
\foreach \anchor/\placement in
{west/west  , east/right , north/above, south/below, north west/above left, north east/above right, south west/below left, south east/below right}
\fill[blue,pin position=\placement] (node cs:name=A,anchor= \anchor) circle (2pt) node[blue,pin=\scriptsize{ \anchor} ] {} ;
\foreach \anchor/\placement in
{pointer/below right}
\fill[red,pin position=\placement] (node cs:name=A,anchor= \anchor) circle (2pt) node[blue,pin=\scriptsize{ \anchor} ] {} ;
\end{tikzpicture}
\\ \hline 
\end{tabular}


\bigskip

\begin{tabular}{|c|} \hline 
shape=cloud callout
\\  \hline 
\begin{tikzpicture}[]
\node[shape=cloud callout,draw,aspect=1.5] (A) at (1,1) {\Huge XXXXXX};
\foreach \anchor/\placement in
{west/west  , east/right , north/below , south/above, north west/above left, north east/above right, south west/below left, south east/below right}
\fill[blue,pin position=\placement] (node cs:name=A,anchor= \anchor) circle (2pt) node[blue,pin=\scriptsize{ \anchor} ] {} ;
\foreach \anchor/\placement in
{puff 1/above, puff 2/above, puff 3/left, puff 4/left,
puff 5/below left, puff 6/below, puff 7/below right, puff 8/right,
puff 9/right, puff 10/above,pointer/below right}
\fill[red,pin position=\placement] (node cs:name=A,anchor= \anchor) circle (2pt) node[blue,pin=\scriptsize{ \anchor} ] {} ;
\end{tikzpicture}
\\ \hline 
\end{tabular}


\bigskip

\begin{tabular}{|c|} \hline 
shape=cross out
\\  \hline 
\begin{tikzpicture}[]
\node[shape=cross out,draw,minimum size=3cm] (A) at (1,1) {\Huge XXXXXXXXXX};
\foreach \anchor/\placement in
{west/west  , east/right , north/above , south/below, north west/above left, north east/above right, south west/below left, south east/below right}
\fill[blue,pin position=\placement] (node cs:name=A,anchor= \anchor) circle (2pt) node[blue,pin=\scriptsize{ \anchor} ] {} ;
\end{tikzpicture}
\\ \hline 
\end{tabular}

\bigskip

\begin{tabular}{|c|} \hline 
shape=rounded rectangle
\\  \hline 
\begin{tikzpicture}[]
\node[shape=rounded rectangle,draw,minimum size=3cm] (A) at (1,1) {\Huge XXXXXXXXXX};
\foreach \anchor/\placement in
{west/west  , east/right , north/above , south/below, north west/above left, north east/above right, south west/below left, south east/below right}
\fill[blue,pin position=\placement] (node cs:name=A,anchor= \anchor) circle (2pt) node[blue,pin=\scriptsize{ \anchor} ] {} ;

\end{tikzpicture}
\\ \hline 
\end{tabular}


\bigskip

\begin{tabular}{|c|} \hline 
shape=chamfered rectangle
\\  \hline 
\begin{tikzpicture}[]
\node[shape=chamfered rectangle,draw,minimum size=3cm, chamfered rectangle sep=.5cm,] (A) at (1,1) {\Huge XXXXXX};
\foreach \anchor/\placement in
{west/west  , east/right , north/above , south/below, north west/above left, north east/above right, south west/below left, south east/below right}
\fill[blue,pin position=\placement] (node cs:name=A,anchor= \anchor) circle (2pt) node[blue,pin=\scriptsize{ \anchor} ] {} ;
\foreach \anchor/\placement in
{before north east/above right, after north east/above right, before south east/below right,after south east/below right, before north west/above left, after north west/above left, before south west/below left,after south west/below left}
\fill[red,pin position=\placement] (node cs:name=A,anchor= \anchor) circle (2pt) node[blue,pin=\scriptsize{ \anchor} ] {} ;
\end{tikzpicture}
\\ \hline 
\end{tabular}

\normalsize


 

%
%%\begin{tikzpicture}
%%[spy using outlines={circle, magnification=4, size=2cm, connect spies}]
%%\draw [help lines] (0,0) grid (3,2);
%%\draw [decoration=Koch curve type 1]
%%decorate { decorate{ decorate{ decorate{ (0,0) -- (2,0) }}}};
%%\spy [red] on (1.6,0.3)
%%in node [left] at (3.5,-1.25);
%%\spy [blue, size=1cm] on (1,1)
%%in node [right] at (0,-1.25);
%%\end{tikzpicture}
%%
%%\begin{tikzpicture}
%%[spy using outlines={circle, magnification=4, size=2cm, connect spies}]
%%\draw [help lines] (0,0) grid (3,2);
%%\shadedraw[shading=Mandelbrot set ] (0,0) rectangle (2,2) ;
%%\spy [red] on (1,0.4)
%%in node [left] at (3.5,-1.25);
%%\spy [blue, size=1cm] on (.5,.8)
%%in node [right] at (0,-1.25);
%%\end{tikzpicture}
%
%
%%
 \maboite{\BS{usetikzlibrary}\AC{turtle}}
\label{lib-turtle}


\begin{center}
\RRR{ 73 }
\end{center}

\begin{tabular}{|c|c|c|c|} \hline 
\multicolumn{4}{|c|}{  \BS{draw} [blue,line width=3pt,turtle={home,forward}];} \\  \hline 
\begin{tikzpicture}
\draw[help lines] (-1.5,-2) grid (1.5,2) ; 
\draw [blue,line width=3pt,turtle={home,forward}];
\end{tikzpicture}
&  
\begin{tikzpicture}
\draw[help lines] (-1.5,-2) grid (1.5,2) ;  
\draw [blue,line width=3pt,turtle={home,forward=1.5cm}];
\end{tikzpicture}
&  
\begin{tikzpicture}
\draw[help lines] (-1.5,-2) grid (1.5,2) ;  
\draw [blue,line width=3pt,turtle={home,fd}];
\end{tikzpicture}
&  
\begin{tikzpicture}
\draw[help lines] (-1.5,-2) grid (1.5,2) ; 
\draw [blue,line width=3pt,turtle={home,fd=1.5cm}];
\end{tikzpicture}
\\ \hline 
turtle=\AC{home,forward}  & turtle=\AC{home,forward=1.5cm} & turtle=\AC{home,fd} & 
turtle=\AC{home,fd=1.5cm} \\ 
\hline 
\end{tabular} 

\bigskip


\begin{tabular}{|c|c|c|c|}
\hline 
\multicolumn{4}{|c|}{  \BS{draw} [blue,line width=3pt,turtle={home,left,fd];}} \\  \hline  
\hline 
\begin{tikzpicture}
\draw (-1,-1) grid (1,1) ; 
\draw [blue,line width=3pt,turtle={home,left,fd}];
\end{tikzpicture} 
&  
\begin{tikzpicture}
\draw (-1,-1) grid (1,1) ; 
\draw [blue,line width=3pt,turtle={home,left=45,fd}];
\end{tikzpicture}
&  
\begin{tikzpicture}
\draw (-1,-1) grid (1,1) ; 
\draw [blue,line width=3pt,turtle={home,lt,fd}];
\end{tikzpicture}
&  
\begin{tikzpicture}
\draw (-1,-1) grid (1,1) ; 
\draw [blue,line width=3pt,turtle={home,lt=45,fd}];
\end{tikzpicture}
\\ \hline
turtle=\AC{home,left,fd}  & turtle=\AC{home,left=45,fd} & turtle=\AC{home,lt,fd} & 
turtle=\AC{home,lt=45,fd} \\ 
\hline 
\end{tabular} 

\bigskip

\begin{tabular}{|c|c|c|c|}
\hline 
\multicolumn{4}{|c|}{  \BS{draw} [blue,line width=3pt,turtle={home,right,fd];}} \\  \hline  
\hline 
\begin{tikzpicture}
\draw (-1,-1) grid (1,1) ; 
\draw [blue,line width=3pt,turtle={home,right,fd}];
\end{tikzpicture} 
&  
\begin{tikzpicture}
\draw (-1,-1) grid (1,1) ; 
\draw [blue,line width=3pt,turtle={home,right=45,fd}];
\end{tikzpicture}
&  
\begin{tikzpicture}
\draw (-1,-1) grid (1,1) ; 
\draw [blue,line width=3pt,turtle={home,rt,fd}];
\end{tikzpicture}
&  
\begin{tikzpicture}
\draw (-1,-1) grid (1,1) ; 
\draw [blue,line width=3pt,turtle={home,rt=45,fd}];
\end{tikzpicture}
\\ \hline
turtle=\AC{home,right,fd}  & turtle=\AC{home,right=45,fd} & turtle=\AC{home,rt,fd} & 
turtle=\AC{home,rt=45,fd} \\ 
\hline 
\end{tabular} 

\bigskip

\begin{tabular}{|c|c|} \hline 
\tikz[blue,line width=3pt]
\draw [->,turtle={home,rt,fd,fd,lt,fd,lt,fd}];
&  
\tikz[blue,line width=3pt]
\draw [->,turtle/distance=2cm,turtle={home,rt,fd,fd,lt,fd,lt,fd}];
\\ \hline 
[->,turtle={home,rt,fd,fd,lt,fd,lt,fd}] & [->,turtle/distance=2cm,turtle={home,rt,fd,fd,lt,fd,lt,fd}] 
\\ \hline 
\end{tabular} 

\bigskip


\begin{tabular}{|c|} \hline 
\begin{tikzpicture}[turtle/distance=2cm]
\draw[help lines] (-1.5,-1) grid (6,3) ; 
\draw [blue,line width=3pt,dotted,turtle={home,forward,right,forward},fd];
\draw [red,line width=3pt,turtle={how/.style={bend left},home,fd,rt,fd,fd}] ;
\end{tikzpicture}
\\  \hline 
[red,turtle=\AC{\rouge{how/.style}=\AC{bend left},home,fd,rt,fd,fd}]
\\ \hline 
\end{tabular} 

\bigskip

\begin{tabular}{|c|c|}  \hline 
\tikz
\filldraw [turtle/distance=2cm,thick,blue,fill=red!20]
[turtle=home]
\foreach \i in {1,...,5}
{
[turtle={forward,right=144}]
}; 
& 
 
\parbox[b]{10cm}{
\BS{filldraw}[turtle/distance=2cm,thick,blue,fill=red!20] \\
$[$ turtle=home $]$ \\
\BS{foreach} \BS{i} in \AC{1,...,5} \\
{
[ turtle=\AC{forward,right=144} ]
};
}
\\ \hline  
\end{tabular} 

\bigskip



\begin{tabular}{|c|c|}  \hline 
\tikz \draw [thick,blue]
[turtle=home]
\foreach \i in {1,...,25}
{
[turtle={forward=\i/5,right=120}]
};
& 
 
\parbox[b]{10cm}{
\BS{draw}[thick,blue] \\
$[$ turtle=home $]$ \\
\BS{foreach} \BS{i} in \AC{1,...,25} \\
{
[turtle=\AC{forward=\BS{i}/5,right=120} ]
}; \\
\vspace{1cm}
}
\\ \hline  
\end{tabular}




%% 
%%
%% 
%%\newpage 
%%
%%\SbSSCT{Matrice de n\oe uds}{Matrices and Alignment}
%%
%%
%%
\label{matrix}
\begin{center}
\RRR{20}
\end{center}

\begin{tabular}{|c|c|} \hline  
\begin{tikzpicture}[baseline=1cm]
\draw[help lines] (0,0) grid (4,2);
\node [matrix,fill=red!20,draw=blue,very thick] (my matrix) at (2,1)
{
\draw (0,0) circle (4mm); & \node[rotate=45] {Hello}; \\
\draw (0.2,0) circle (2mm); & \fill[red] (0,0) circle (3mm); \\
};
\end{tikzpicture}
& 
\parbox{10cm}{
\BS{node} [\RDD{matrix},fill=red!10,draw=blue,very thick] at (2,1) \\
\{ \\
\BS{draw} (0,0) circle (4mm); \& \BS{node} [rotate=45] {Hello}; \BS{}\BS{} \\
\BS{draw}  (0.2,0) circle (2mm); \& \BS{fill}[red] (0,0) circle (3mm); \BS{}\BS{} \\
\}; \\
}
\\ \hline 
\end{tabular} 

\bigskip

\begin{tabular}{|c|c|} \hline  
\begin{tikzpicture}[baseline=0pt]
\matrix [fill=red!20,draw=blue,very thick] 
{
\draw (0,0) circle (4mm); & \node[rotate=45] {Hello}; \\
\draw (0.2,0) circle (2mm); & \fill[red] (0,0) circle (3mm); \\
};
\end{tikzpicture}
&  
\parbox{10cm}{
\BSS{matrix} [fill=red!10,draw=blue,very thick] \\
\{ \\
\BS{draw} (0,0) circle (4mm); \& \BS{node} [rotate=45] {Hello}; \BS{}\BS{} \\
\BS{draw}  (0.2,0) circle (2mm); \& \BS{fill}[red] (0,0) circle (3mm); \BS{}\BS{} \\
\}; \\
}
\\ \hline 
\end{tabular} 


\SbSbSSCT{Alignement des cellules}{Cell Pictures}


\begin{center}
\RRR{20-3}
\end{center}

\begin{tabular}{|c|c|c|} \hline  
\begin{tikzpicture}
[every node/.style={draw=black,font=\huge}]
\matrix [draw=red]
{
\node {a}; \fill[blue] (0,0) circle (2pt); &
\node {X}; \fill[blue] (0,0) circle (2pt); &
\node {g}; \fill[blue] (0,0) circle (2pt); \\
};
\end{tikzpicture}
&  
\begin{tikzpicture}
[every node/.style={draw=black,anchor=base,font=\huge}]
\matrix [draw=red]
{
\node {a}; \fill[blue] (0,0) circle (2pt); &
\node {X}; \fill[blue] (0,0) circle (2pt); &
\node {g}; \fill[blue] (0,0) circle (2pt); \\
};
\end{tikzpicture}
&  
\begin{tikzpicture}[every node/.style={draw=black}]
\matrix [draw=red,anchor=north,font=\huge]
{
\node {a}; \fill[blue] (0,0) circle (2pt); &
\node {X}; \fill[blue] (0,0) circle (2pt); &
\node {g}; \fill[blue] (0,0) circle (2pt); \\
};
\end{tikzpicture}
\\ \hline  
 & anchor=base &  anchor=north \\ \hline 
\end{tabular} 

\bigskip
\begin{tabular}{|c|c|c|} \hline  
\begin{tikzpicture}
[every node/.style={draw=black,font=\huge}]
\matrix [draw=red]
{

\node[left]  {X}; \fill[blue] (0,0) circle (2pt);  \\
};
\end{tikzpicture}
&  
\begin{tikzpicture}
[every node/.style={draw=black,anchor=base,font=\huge}]
\matrix [draw=red]
{
\node {a}; \fill[blue] (0,0) circle (2pt); ²\\
\node[right] {X}; \fill[blue] (0,0) circle (2pt);  \\
\node {g}; \fill[blue] (0,0) circle (2pt); \\
};
\end{tikzpicture}
&  
\begin{tikzpicture}[every node/.style={draw=black}]
\matrix [draw=red,anchor=north,font=\huge]
{
\node {a}; \fill[blue] (0,0) circle (2pt); &
\node[right] {X}; \fill[blue] (0,0) circle (2pt); &
\node {g}; \fill[blue] (0,0) circle (2pt); \\
};
\end{tikzpicture}
\\ \hline  
 & anchor=base &  anchor=north \\ \hline 
\end{tabular} 

\bigskip

\begin{tabular}{|c|c|} \hline  
\begin{tikzpicture}[baseline=0pt]
\matrix [draw=red,nodes=draw]
{
\node[left] {A}; \fill[blue] (0,0) circle (2pt); \\
\node {B}; \fill[blue] (0,0) circle (2pt); \\
\node[right] {C}; \fill[blue] (0,0) circle (2pt); \\
};
\end{tikzpicture}
&  
\parbox{12cm}{
\BS{matrix} [draw=red,nodes=draw]
\AC{\\
\BS{node}\rouge{[left]} {A}; \BS{fill}[blue] (0,0) circle (2pt); \BS{} \BS{} \\
\BS{node} {B}; \BS{fill}[blue] (0,0) circle (2pt);\BS{} \BS{} \\
\BS{node}\rouge{[right]} {C}; \BS{fill}[blue] (0,0) circle (2pt); \BS{} \BS{}\\
}; \\
}

\\ \hline 
\end{tabular} 

\bigskip

\begin{tabular}{|c|c|} \hline  
\multicolumn{2}{|c|}{\BS{matrix} [draw,\RDD{column  sep}=1cm,nodes=draw]} 
\\ \hline 
\begin{tikzpicture}
\matrix [draw,column sep=1cm,nodes=draw]
{
\node(a) {123}; & \node (b) {1}; & \node {1}; \\
\node {12}; & \node {12}; & \node {1}; \\
\node(c) {1}; & \node (d) {123}; & \node {1}; \\
};
\draw [red,thick] (a.east) -- (a.east |- c)
(d.west) -- (d.west |- b);
\draw [<->,red,thick] (a.east) -- (d.west |- b)
node [above,midway] {1cm};
\end{tikzpicture}
&  
\begin{tikzpicture}
\matrix [draw,column sep={1cm,between origins},nodes=draw]
{
\node(a) {123}; & \node (b) {1}; & \node {1}; \\
\node {12}; & \node {12}; & \node {1}; \\
\node {1}; & \node {123}; & \node {1}; \\
};
\draw [<->,red,thick] (a.center) -- (b.center) node [above,midway] {1cm};
\end{tikzpicture}
\\ \hline \RDD{column sep}=1cm & column sep=\AC{1cm,\RDD{between origins} } 
\\ \hline 
\end{tabular} 

\bigskip

\begin{tabular}{|c|c|} \hline
\multicolumn{2}{|c|}{\BS{matrix} [draw,\RDD{row sep}=1cm,nodes=draw]} 
\\ \hline 
\begin{tikzpicture}
\matrix [draw,row sep=1cm,nodes=draw]
{
\node (a) {123}; & \node {1}; & \node {1}; \\
\node (b) {12}; & \node {12}; & \node {1}; \\
\node {1}; & \node {123}; & \node {1}; \\
};
\draw [<->,red,thick] (a.south) -- (b.north) node [right,midway] {1cm};
\end{tikzpicture}
&
\begin{tikzpicture}
\matrix [draw,row sep={1cm,between origins},nodes=draw]
{
\node (a) {123}; & \node {1}; & \node {1}; \\
\node (b) {12}; & \node {12}; & \node {1}; \\
\node {1}; & \node {123}; & \node {1}; \\
};
\draw [<->,red,thick] (a.center) -- (b.center) node [right,midway] {1cm};
\end{tikzpicture}
\\  \hline 
\RDD{row sep}=1cm  & row sep=\AC{1cm,\RDD{between origins} } 
\\ \hline 


\end{tabular} 




\bigskip

\begin{tabular}{|c|c|} \hline  
\multicolumn{2}{|c|}{\BS{matrix} [ \rouge{row sep=5mm},draw,nodes=draw]} \\
\multicolumn{2}{|c|}{ \{ \BS{node} \AC{1}; \& \BS{node} \AC{2}; \& \BS{node} \AC{3}; \BS{}\BS{}  } \\
\multicolumn{2}{|c|}{ \BS{node} \AC{4} ; \& \BS{node}  \AC{5}; \& \BS{node}  \AC{6};  \BS{}\BS{} \rouge{[1cm]} } \\
\multicolumn{2}{|c|}{ \BS{node} \AC{7}; \& \BS{node}\AC{8}; \& \BS{node}\AC{9}; \BS{}\BS{} \}  } 
\\ \hline  
\begin{tikzpicture}
\matrix [row sep=5mm,draw,nodes=draw]
{
\node {1}; & \node {2};& \node {3}; \\
\node(a) {4} ; & \node {5}; & \node {6};\\[1cm]
\node(b) {7}; &\node {8}; & \node {9}; \\
};
\draw [<->,red,thick] (a.center) -- (b.center) node [right,midway] {1,5cm};
\end{tikzpicture}
&  
\begin{tikzpicture}
\matrix [row sep=5mm,draw,nodes=draw]
{
\node {1}; & \node {2};& \node {3}; \\
\node(a) {4} ; & \node {5}; & \node {6};\\[10mm,between origins]
\node(b) {7}; &\node {8}; & \node {9}; \\
};
\draw [<->,red,thick] (a.center) -- (b.center) node [right,midway] {1,5cm};
\end{tikzpicture}
\\ \hline 
\rouge{[1cm]} & \rouge{[1cm,between origins]}
\\ \hline 
\end{tabular} 

\bigskip

\begin{tabular}{|c|c|} \hline  
\multicolumn{2}{|c|}{\BS{matrix} [ \rouge{column sep=5mm},draw,nodes=draw]} \\
\multicolumn{2}{|c|}{ \{ \BS{node} \AC{1}; \& \BS{node} \AC{2}; \& \BS{node} \AC{3}; \BS{}\BS{}  } \\
\multicolumn{2}{|c|}{ \BS{node} \AC{4} ; \& \BS{node}  \AC{5}; \& \rouge{[1cm]}\BS{node}  \AC{6};  \BS{}\BS{}  } \\
\multicolumn{2}{|c|}{ \BS{node} \AC{7}; \& \BS{node}\AC{8}; \& \BS{node}\AC{9}; \BS{}\BS{} \}  } 
\\ \hline  

\begin{tikzpicture}
\matrix [draw,nodes=draw,column sep=5mm]
{
\node {1}; & \node(a) {2}; &[1cm] \node(b) {3}; \\
\node {4}; & \node{5}; & \node {6}; \\
\node {7}; & \node{8}; & \node {9}; \\
};
\draw [<->,red,thick] (a.east) -- (b.west) node [above,midway] {15mm};
\end{tikzpicture}
&  
\begin{tikzpicture}
\matrix [draw,nodes=draw,column sep=5mm]
{
\node {1}; &[2mm] \node(a){2}; &[1cm,between origins] \node(b){3}; \\
\node {4}; & \node {5}; & \node {6}; \\
\node {7}; & \node {8}; & \node {9}; \\
};
\draw [<->,red,thick] (a.center) -- (b.center) node [above,midway] {15mm};
\end{tikzpicture}
\\ \hline  
\rouge{[1cm]}
&  
\rouge{[1cm,between origins]}
\\ \hline 
\end{tabular} 




\bigskip

\begin{tikzpicture}
\matrix [draw,nodes=draw,column sep={1cm,between origins}]
{
\node (a) {8}; & \node (b) {1}; &[between borders] \node (c) {6}; \\
\node {3}; & \node {5}; & \node {7}; \\
\node {4}; & \node {9}; & \node {2}; \\
};
\draw [<->,red,thick] (a.center) -- (b.center) node [above,midway] {10mm};
\draw [<->,red,thick] (b.east) -- (c.west) node [above,midway] {1cm};
\end{tikzpicture}



\SbSbSSCT{Format des cellules}{Cell Styles and Options}

\noindent 

\begin{tabular}{|c|} \hline  
\BS{matrix} [nodes=draw,nodes=\AC{\rouge{fill}=blue!10\rouge{,minimum size}=1cm}]
\\ \hline  
\begin{tikzpicture}
\matrix [nodes=draw,nodes={fill=blue!10,minimum size=1cm}]
{
\node {1}; & \node{2}; & \node {3}; \\
\node {4}; & \node{5}; & \node {6}; \\
\node {7}; & \node{8}; & \node {9}; \\
};
\end{tikzpicture}
\\ \hline 
\end{tabular} 


\bigskip 


\begin{tabular}{|c|c|c|} \hline 
\multicolumn{3}{|c|}{\BS{matrix}[\rouge{row 2/.style}=\AC{red}]}
 \\ \hline 
\begin{tikzpicture}
\matrix[row 2/.style={red}]
{
\node {8}; & \node{1}; & \node {6}; \\
\node {3}; & \node{5}; & \node {7}; \\
\node {4}; & \node{9}; & \node {2}; \\
};
\end{tikzpicture}
&  
\begin{tikzpicture}
\matrix[column 2/.style={red}]
{
\node {8}; & \node{1}; & \node {6}; \\
\node {3}; & \node{5}; & \node {7}; \\
\node {4}; & \node{9}; & \node {2}; \\
};
\end{tikzpicture}
&  
\begin{tikzpicture}
\matrix[row 2 column 2/.style={red}]
{
\node {8}; & \node{1}; & \node {6}; \\
\node {3}; & \node{5}; & \node {7}; \\
\node {4}; & \node{9}; & \node {2}; \\
};
\end{tikzpicture}
\\ \hline 
row 2/.style=\AC{red} & column 2/.style=\AC{red}  & row 2 column 2/.style=\AC{red}\\ 
\hline 
\end{tabular} 

\bigskip 

\begin{tabular}{|c|c|c|} \hline 
\multicolumn{3}{|c|}{\BS{matrix}[column 1/.style=\AC{anchor=west}]}
 \\ \hline 
\begin{tikzpicture}
\matrix[column 1/.style={anchor=west}]
{
\node {12345};  & \node {67890}; \\
\node {123}; & \node{67};  \\
\node {1}; & \node{6}; & \\
};
\end{tikzpicture}
&  
\begin{tikzpicture}
\matrix[column 1/.style={anchor=east}]
{
\node {12345};  & \node {67890}; \\
\node {123}; & \node{67};  \\
\node {1}; & \node{6}; & \\
};
\end{tikzpicture}
&  
\begin{tikzpicture}
\matrix[column 1/.style={anchor=base}]
{
\node {12345};  & \node {67890}; \\
\node {123}; & \node{67};  \\
\node {1}; & \node{6}; & \\
};
\end{tikzpicture}
\\  \hline  
[\rouge{column 1/.style}={anchor=west}]& [\rouge{column 1/.style}={anchor=east}] & [\rouge{column 1/.style}={anchor=base}]\\ 
\hline 
\end{tabular} 

\bigskip

\begin{tabular}{|c|c|c|c|} \hline
\multicolumn{4}{|c|}{\BS{matrix}[matrix of nodes,\RDD{every odd column}/.style={red}]}
 \\ \hline 
\begin{tikzpicture}
\matrix [matrix of nodes,every odd column/.style={red}]
{
a & b & c & d \\
e & f & g & h \\
i & j & k & l \\
};
\end{tikzpicture}
&  
\begin{tikzpicture}
\matrix [matrix of nodes,every even column/.style={red}]
{
a & b & c & d \\
e & f & g & h \\
i & j & k & l \\
};
\end{tikzpicture}
&  
\begin{tikzpicture}
\matrix [matrix of nodes,every odd row/.style={red}]
{
a & b & c & d \\
e & f & g & h \\
i & j & k & l \\
};
\end{tikzpicture}
&  
\begin{tikzpicture}
\matrix [matrix of nodes,every even row/.style={red}]
{
a & b & c & d \\
e & f & g & h \\
i & j & k & l \\
};
\end{tikzpicture}
\\ 
\hline 
\RDD{every odd column} & \RDD{every even column} & \RDD{every odd row}  & \RDD{every even row} \\ 
\hline 
\end{tabular} 


\bigskip


\begin{tabular}{|c|} \hline  
\BS{matrix} [draw,matrix of nodes,\rouge{execute at begin cell}=\AC{(}]
\\ \hline  
\begin{tikzpicture}
\matrix [draw,matrix of nodes,execute at begin cell={(}]
{
1 & 2 &   \\
4 &   & 6 \\
  &   & 9 \\
};
\end{tikzpicture}
\\ \hline 
\end{tabular} 

\bigskip

\begin{tabular}{|c|} \hline  
\BS{tikz} 
[matrix of nodes/.style=\AC{
execute at begin cell=\BS{node}\BS{bgroup} , \\
\rouge{execute at end cell}=\$m\wedge 2\$\BS{egroup}; 
}] \\
\BS{matrix} [draw,matrix of nodes
]
\\ \hline  
\tikz 
[matrix of nodes/.style={
execute at begin cell=\node\bgroup ,
execute at end cell=$m^2$\egroup;
}]
\matrix [draw,matrix of nodes
]
{1 & 2 &  \\
4 &   & 6 \\
  & 8 & 9 \\
};
\\ \hline 
\end{tabular}

\bigskip

\begin{tabular}{|c|} \hline 

 \BS{matrix} [raw,matrix of nodes, \rouge {execute at empty cell}=\BS{node}\AC{- -}; ]
\\ \hline 
 
\begin{tikzpicture}
\matrix [draw,matrix of nodes,execute at empty cell=\node{--};]
{
1 & 2 & \\
4 & & 6 \\
& & 9 \\
};
\end{tikzpicture}
\\ \hline  
\end{tabular} 


\newpage
\SbSbSSCT{Points d'ancrage}{Anchoring a Matrix}

\begin{center}
\RRR{20-4}
\end{center}

\begin{tabular}{|c|c|c|} \hline 
\multicolumn{3}{|c|}{
\BS{matrix} [draw=red,nodes=draw,\RDD{matrix anchor}=east](XXX) at (1,1) }
\\ \hline  
\begin{tikzpicture}
\draw[help lines] (0,0) grid (3,3);
\matrix [draw=red,nodes=draw,matrix anchor=west](XXX) at (1,1)
{
\node {123}; \\ 
\node {12}; \\
\node {1}; \\
};
\fill[red](XXX.west) circle (3pt);
\end{tikzpicture}
&  
\begin{tikzpicture}
\draw[help lines] (0,0) grid (3,3);
\matrix [draw=red,nodes=draw,matrix anchor=east](XXX) at (1,1)
{
\node {123}; \\ 
\node {12}; \\
\node {1}; \\
};
\fill[red] (XXX.east) circle (3pt);
\end{tikzpicture}
&  
\begin{tikzpicture}
\draw[help lines] (0,0) grid (3,3);
\matrix [draw=red,nodes=draw,matrix anchor=south](XXX) at (1,1)
{
\node {123}; \\ 
\node {12}; \\
\node {1}; \\
};
\fill[red](XXX.south) circle (3pt);
\end{tikzpicture}

\\  \hline 
matrix anchor=west & matrix anchor=east & matrix anchor=south 
\\ \hline 
\end{tabular} 

\bigskip 
\begin{tabular}{|c|c|c|c|} \hline 
\multicolumn{2}{|c|}{\BS{matrix} [draw=red,nodes=draw,\rouge{anchor=west}] }
\\ \hline  
\begin{tikzpicture}
\matrix [draw=red,nodes=draw,anchor=west] 
{
\node {123}; & \node {abc}; \\ 
\node {12}; & \node {ab}; \\
\node {1}; & \node {a}; \\
};
\end{tikzpicture}
&  
\begin{tikzpicture}
\matrix [draw=red,nodes=draw,anchor=east] 
{
\node {123};& \node {abc}; \\ 
\node {12};  &\node {ab};\\
\node {1};  & \node {a}; \\
};
\end{tikzpicture}

\\ \hline  
anchor=west & anchor=east  \\ 
\hline 
\end{tabular} 

\bigskip 


\begin{tabular}{|c|c|}\hline  
\begin{tikzpicture}[baseline=1cm]
\draw[help lines] (0,0) grid (4,3);
\matrix[draw=red,nodes=draw ,matrix anchor=inner node.south,anchor=base, row sep=5mm, column sep=5mm] at (2,1)
{
\node {a}; & \node {b}; & \node {c}; & \node {d}; \\
\node {a}; & \node {b}; & \node(inner node){c}; & \node {d}; \\
\node {a}; & \node {b}; & \node {c}; & \node {d}; \\
};
\fill[red] (inner node.south) circle (3pt);
\end{tikzpicture}
&  
\parbox{10.5cm}{
\BS{matrix}[draw=red,nodes=draw, \\ 
\RDD{ matrix anchor}=\blll{inner node}.south, anchor=base, \\
  row sep=5mm,column sep=5mm] at (2,1) \\
\{ \\
\BS{node} \AC{a}; \& \BS{node} \AC{b}; \& \BS{node} \AC{c}; \& \BS{node} \AC{d};  \BS{}\BS{} \\
\BS{node} \AC{a}; \& \BS{node} \AC{b}; \& \BS{node}(\blll{inner node})\AC{c}; \& \BS{node} \AC{d};  \BS{}\BS{} \\
\BS{node}\AC{a}; \& \BS{node} \AC{b}; \& \BS{node}\AC{c}; \& \BS{node} \AC{d}; \BS{}\BS{}  \\
\};
}
\\ \hline 
\end{tabular} 


\SbSbSSCT{Changement du séparateur}{Considerations Concerning Active Characters}

\begin{center}
\RRR{20-5}
\end{center}

\begin{tabular}{|c|c|} \hline  
\tikz[baseline=0pt]
\matrix [ampersand replacement=\|]
{
\draw (0,0) circle (4mm); \| \node[rotate=10] {Hello}; \\
\draw (0.2,0) circle (2mm); \| \fill[red] (0,0) circle (3mm); \\
};
& 
\parbox{12cm}{ 
\BS{tikz}
\BS{matrix} [\RDD{ampersand replacement}=\blll{\BS{|}} ] \\
\{ \\
\BS{draw} (0,0) circle (4mm); \blll{\BS{|} }  \BS{node}[rotate=10] \AC{Hello}; \BS{}\BS{} \\
\BS{draw} (0.2,0) circle (2mm);  \blll{\BS{|} }   \BS{fill}[red] (0,0) circle (3mm); \BS{}\BS{} \\
\}; \\
}
\\ \hline 
\end{tabular} 


\SbSSCT{Matrice de n\oe uds (compléments) }{Matrix Library}

 \maboite{\BS{usetikzlibrary}\AC{matrix}}
\label{lib-matrix}


\begin{center}
\RRR{57-1}
\end{center}

\begin{tabular}{|c|c|} \hline  
\begin{tikzpicture}[baseline=0pt]
\matrix (XXX) [matrix of nodes]
{
1 & 2 & 3 \\
4 & 5 & 6 \\
7 & 8 & 9 \\
};
\end{tikzpicture}
& 
\parbox{10cm}{ 
\BS{begin}\AC{tikzpicture} \\
\BSS{matrix}  [matrix of nodes]\\
\{ \\
1 \hspace{3mm} \& \hspace{3mm}  2 \hspace{3mm} \& \hspace{3mm} 3 \hspace{3mm} \BS{}\BS{}   \\
4 \hspace{3mm} \& \hspace{3mm}  5 \hspace{3mm} \& \hspace{3mm} 6 \hspace{3mm} \BS{}\BS{}  \\
7 \hspace{3mm} \& \hspace{3mm}  8 \hspace{3mm} \& \hspace{3mm} 9 \hspace{3mm} \BS{}\BS{} \\
\}; \\
\BS{end}\AC{tikzpicture}
}
\\ \hline  
\end{tabular} 

\bigskip

\begin{tabular}{|c|c|} \hline  
\begin{tikzpicture}[baseline=0pt]
\matrix (XXX) [matrix of nodes,column sep=.5cm,row sep=.5cm,every node/.style=draw]
{
1 & 2 & 3 \\
4 & 5 & 6 \\
7 & 8 & 9 \\
};
\draw[thick,red,->] (XXX-1-1) -- (XXX-2-3);
\end{tikzpicture}
& 
\parbox{10cm}{ 
\BS{begin}\AC{tikzpicture} \\
\BSS{matrix} \blll{(XXX)} [matrix of nodes,column sep=.5cm,row sep=.5cm,every node/.style=draw]\\
\{ \\
1 \hspace{3mm} \& \hspace{3mm} 2 \hspace{3mm} \& \hspace{3mm} 3 \hspace{3mm} \BS{}\BS{}   \\
4 \hspace{3mm} \& \hspace{3mm} 5 \hspace{3mm} \& \hspace{3mm} 6 \hspace{3mm} \BS{}\BS{}  \\
7 \hspace{3mm} \& \hspace{3mm} 8 \hspace{3mm} \& \hspace{3mm} 9 \hspace{3mm} \BS{}\BS{} \\
\}; \\
\BS{draw}[thick,red,->] \blll{(XXX-1-1)} - - \blll{(XXX-2-3)} ; \\
\BS{end}\AC{tikzpicture}
}
\\ \hline  
\end{tabular} 

\bigskip


\begin{tabular}{|c|c|} \hline  
\begin{tikzpicture}
\matrix [matrix of nodes,column sep=.5cm,row sep=.5cm,every node/.style=draw]
{
8 & 1 & 6 \\
3 & 5 & |[red]| 7 \\
4 & 9 & 2 \\
};
\end{tikzpicture}
&  
\begin{tikzpicture}
\matrix [matrix of nodes]
{
1 & \& &  2 & \& &  3 				& \BS{}\BS{} \\
4 & \& & 5 	& \& & \rouge{ $|[$red$]|$} 6 & \BS{}\BS{} \\
7 & \& & 8 	& \& & 9 				& \BS{}\BS{} \\
};
\end{tikzpicture}
\\ \hline 
\end{tabular}  


\bigskip

\begin{tabular}{|c|c|} \hline 
\begin{tikzpicture}[baseline=-1cm] 
\matrix [matrix of nodes,column sep=.5cm,row sep=.5cm,every node/.style=draw]
{
AAA 			& |[circle]| BBB \\
CCC & |(d) [isosceles triangle]| DDD \\
| [ellipse]| EEE &  FFF \\
};
\end{tikzpicture}
& 
\begin{tikzpicture}
\matrix [matrix of nodes]
{
AAA & \& & \rouge{ $|[$circle$]|$} BBB &  \BS{}\BS{} \\
CCC & \& &\rouge{ $|[$isosceles triangle$]|$} DDD 	&  \BS{}\BS{} \\
\rouge{ $|[$ellipse$]|$} EEE & \& & FFF & \BS{}\BS{} \\
};
\end{tikzpicture}
\\ \hline 
\end{tabular} 


\bigskip

\begin{tabular}{|c|c|} \hline 
\begin{tikzpicture}[baseline=-2cm] 
\matrix [matrix of nodes,column sep=.5cm,row sep=.5cm,every node/.style=draw]
{
|(a)| AAA 	& |(b)| BBB \\
|(c)| CCC 	& |(d)| DDD \\
|(e)| EEE 	& |(f)| FFF \\
};
\draw (a) -- (d);
\draw (d) -- (f);
\end{tikzpicture}
&  
\begin{tikzpicture}
\node at (0,1.5) [text width=10cm]
{\BS{matrix} [matrix of nodes,column sep=.5cm,row sep=.5cm,every node/.style=draw] \\
\{ 
};
\matrix [matrix of nodes]
{
\rouge{ $|$(a)$|$} AAA & \& & \rouge{ $|$(b)$|$} BBB &  \BS{}\BS{} \\
\rouge{ $|$(c)$|$} CCC & \& & \rouge{ $|$(d)$|$} DDD 	&  \BS{}\BS{} \\
\rouge{ $|$(e)$|$} EEE & \& & \rouge{ $|$(f)$|$} FFF & \BS{}\BS{} \\
};

\node at (0,-1.2) [text width=10cm]
{  \}; \\ 
\BS{draw} (a) - - (d); \\ \BS{draw} (d) - - (f);
};
\end{tikzpicture}
\\ \hline 
\end{tabular} 

\bigskip


\begin{tabular}{|c|c|} \hline  
\begin{tikzpicture}
\matrix [matrix of nodes]
{
1 &[1cm] 2 &[5mm] |[red]| 3 \\
4 & 5 &  6 \\
7 & 8 & 9 \\
};
\end{tikzpicture}
&
\begin{tikzpicture}
\matrix [matrix of nodes]
{
1 & \& & \rouge{\lbrack 1cm \rbrack} 2 & \& &\rouge{\lbrack 5mm \rbrack} |[red]| 3 & \BS{}\BS{} \\
4 & \& & 5 & \& & 6 & \BS{}\BS{} \\
7 & \& & 8 & \& & 9 & \BS{}\BS{} \\
};
\end{tikzpicture}

\\ \hline 
\end{tabular} 



\bigskip

\begin{tabular}{|c|c|} \hline  
\begin{tikzpicture}[baseline=0pt]
\matrix [matrix of math nodes]
{
A_1 & A_2 & A_3 \\
a_4 & a_5 &  a_6 \\
a^7 & a^8 & a^9 \\
};
\end{tikzpicture}
&  
\parbox{8cm}{ 
\BSS{matrix}  [\rouge{ matrix of math nodes}]\\
\{ \\
A\_1 \hspace{2mm}  \& \hspace{2mm}  A\_2 \hspace{2mm}  \& \hspace{2mm}  A\_3 \hspace{2mm}   \BS{}\BS{}   \\
a\_4 \hspace{2mm}  \& \hspace{2mm}  a\_5 \hspace{2mm}  \&  \hspace{2mm}  a\_6 \hspace{2mm}  \BS{}\BS{}  \\
a\land 7 \hspace{2mm}  \& \hspace{2mm}  a\land 8 \hspace{2mm}  \& \hspace{2mm}  a\land 9 \hspace{2mm}   \BS{}\BS{} \\
\}; 
}
\\ \hline  
\end{tabular} 

\bigskip

\begin{tabular}{|c|c|} \hline  
\begin{tikzpicture}[baseline=0pt]
\matrix [matrix of math nodes,nodes={circle,draw}]
{
a_1 & & a_3 \\
a_4 & & a_6 \\
a_7 & a_8 & \\
};
\end{tikzpicture}
&  
\parbox{10cm}{ 
\BSS{matrix}  [matrix of math nodes,\rouge{nodes={circle,draw}}]\\
\{ \\
A\_1 \hspace{2mm}  \& \hspace{12mm}  \& \hspace{2mm}  A\_3 \hspace{2mm}   \BS{}\BS{}   \\
a\_4 \hspace{2mm}  \& \hspace{12mm}  \& \hspace{2mm}   a\_6 \hspace{2mm}  \BS{}\BS{}  \\
a\_ 7 \hspace{2mm}  \& \hspace{2mm}  a\_ 8 \hspace{2mm}  \& \hspace{12mm}    \BS{}\BS{} \\
\}; 
}
\\ \hline 
\end{tabular} 

\bigskip

\begin{tabular}{|c|c|} \hline  
\begin{tikzpicture}[baseline=0pt]
\matrix [matrix of math nodes,nodes={circle,draw},nodes in empty cells]
{
a_1 & & a_3 \\
a_4 & & a_6 \\
a_7 & a_8 & \\
};
\end{tikzpicture}
&  
\parbox{10cm}{ 
\BSS{matrix}  [matrix of math nodes,nodes={circle,draw} ,\rouge{nodes in empty cells}]\\
\{ \\
A\_1 \hspace{2mm}  \& \hspace{12mm}  \& \hspace{2mm}  A\_3 \hspace{2mm}   \BS{}\BS{}   \\
a\_4 \hspace{2mm}  \& \hspace{12mm}  \& \hspace{2mm}   a\_6 \hspace{2mm}  \BS{}\BS{}  \\
a\_ 7 \hspace{2mm}  \& \hspace{2mm}  a\_ 8 \hspace{2mm}  \& \hspace{12mm}    \BS{}\BS{} \\
\}; 
}
\\ \hline 
\end{tabular} 

\SbSbSSCT{Texte dans les n\oe uds}{Characters in Matrices of Nodes}

\begin{center}
\RRR{57-2}
\end{center}


\begin{tabular}{|c|c|} \hline  
\begin{tikzpicture}[baseline=0pt]
\matrix [matrix of nodes,nodes={text width=2cm,draw}]
{
aaa  & bbb \\ 
ccc \\
eee & fff\\
};
\end{tikzpicture}
&  
\parbox{10cm}{ 
\BSS{matrix}  [matrix of nodes,\rouge{nodes=\AC{text width=2cm,draw}} ]\\
\{ \\
aaa \&  bbb \BS{}\BS{}  \\
ccc \BS{}\BS{}  \\
eee \& fff \BS{}\BS{}  \\
\}; 
}
\\ \hline 
\end{tabular} 

\bigskip

\begin{tabular}{|c|c|}  \hline  
\begin{tikzpicture}[baseline=0cm]
\matrix [matrix of nodes,nodes={text width=2cm,draw}]
{
1 & {aaa \\ bbb \\ ccc } \\
2 & ddd \\
};
\end{tikzpicture}
&  
\parbox{10cm}{ 
\BSS{matrix}  [matrix of nodes,nodes=\AC{text width=2cm,draw} ]\\
\{ \\
1 \& \& \rouge { \AC{aaa \BS{}\BS{} bbb \BS{}\BS{} ccc } } \BS{}\BS{}   \\
2 \& \& ddd \BS{}\BS{}  \\
\}; 
}
\\ \hline 
\end{tabular} 

\bigskip

\SbSbSSCT{Délimiteurs}{Delimiters}


\begin{center}
\RRR{57-3}
\end{center}

\bigskip

\begin{tabular}{|c|c|c|c|} \hline 
\multicolumn{4}{|c|}{\BS{matrix} [matrix of math nodes,\RDD{left delimiter}=( ]}
\\ \hline  
\begin{tikzpicture}
\matrix [matrix of math nodes,left delimiter=( ]
{
a_1 & a_2 & a_3 \\
a_4 & a_5 & a_6 \\
a_7 & a_8 & a_9 \\
};
\end{tikzpicture}
&  
\begin{tikzpicture}
\matrix [matrix of math nodes,right delimiter=\}]
{
a_1 & a_2 & a_3 \\
a_4 & a_5 & a_6 \\
a_7 & a_8 & a_9 \\
};
\end{tikzpicture}
&
\begin{tikzpicture}
\matrix [matrix of math nodes,above delimiter=\| ]
{
a_1 & a_2 & a_3 \\
a_4 & a_5 & a_6 \\
a_7 & a_8 & a_9 \\
};
\end{tikzpicture}
&
\begin{tikzpicture}
\matrix [matrix of math nodes,below delimiter=\rmoustache ]
{
a_1 & a_2 & a_3 \\
a_4 & a_5 & a_6 \\
a_7 & a_8 & a_9 \\
};
\end{tikzpicture}


\\  \hline 
\RDD{left delimiter}=(  & \RDD{right delimiter}=\BS{\}} & \RDD{above delimiter}=\BS{|} & \RDD{below delimiter}=\BS{rmoustache}
\\  \hline
\end{tabular} 

\bigskip
\begin{tabular}{|c|} \hline  
\BS{tikz}
\BS{node} [fill=red!20,text width=2cm,\rouge{left delimiter}=\BS{\{} ] \\
\AC{Ceci est une démonstration d'un texte  sur une largeur de 2cm.};
\\ \hline  
\tikz
\node [fill=red!20,text width=2cm,left delimiter=\{]
{Ceci est une démonstration d'un texte  sur une largeur de 2cm.};
\\ \hline 
\end{tabular} 



%% 
%%\newpage 
%%
%%\SbSSCT{Chaine de n\oe uds}{Chains of nodes}
%%
%%\SbSbSSCT{Création d'une chaine de n\oe euds}{Starting and Continuing a Chain}

 \maboite{\BS{usetikzlibrary}\AC{chains}}
\label{lib-chains}


\begin{center}
\RRR{46-2}
\end{center}

\bigskip

\begin{tabular}{|l|} \hline  
\BS{begin}\AC{tikzpicture}[\RDD{start chain}] \\
\BS{node} [\RDD{on chain}] \AC{A};\\
\BS{node}  [\RDD{on chain}] \AC{B};\\
\BS{node}  [\RDD{on chain}] \AC{C};\\
\BS{end}\AC{tikzpicture} \\ \hline  
\begin{tikzpicture}[start chain]
\node [on chain] {A};
\node [on chain] {B};
\node [on chain] {C};
\end{tikzpicture}
\\ \hline 
\end{tabular} 

\bigskip

\begin{tabular}{|c|}  \hline  
\BS{begin}\AC{tikzpicture}[start chain, \RDD{node distance}= 0.5 cm] 
\\ \hline  
\begin{tikzpicture}[start chain, node distance= .5 cm]
\node [on chain] {A};
\node [on chain] {B};
\node [on chain] {C};
\end{tikzpicture}
\\ \hline 
\end{tabular} 

\bigskip

\begin{tabular}{|c|}  \hline 
\BS{begin}\AC{tikzpicture}[start chain=\rouge {going below} ]
\\   \hline 
\begin{tikzpicture}[start chain=going below]
\node [on chain] {A};
\node [on chain] {B};
\node [on chain] {C};
\end{tikzpicture}
\\   \hline 
\end{tabular} 

\bigskip

\begin{tabular}{|c|}  \hline 
\BS{begin}\AC{tikzpicture}[start chain=\rouge {going left} ] 
\\   \hline 
\rule[0cm]{0pt}{.7cm}  
\begin{tikzpicture}[start chain=going left]
\node [on chain] {A};
\node [on chain] {B};
\node [on chain] {C};
\end{tikzpicture} 
\\ \hline 
\end{tabular} 


\bigskip

\begin{tabular}{|c|}  \hline  
\BS{begin}\AC{tikzpicture}[start chain, \rouge{every node/.style=draw} ] 
\\ \hline 
\rule[0cm]{0pt}{.7cm}  
\begin{tikzpicture}[start chain, every node/.style=draw]
\node [on chain] {A};
\node [on chain] {B};
\node [on chain] {C};
\end{tikzpicture}
\\ \hline 
\end{tabular} 

\bigskip

\begin{tabular}{|c|c|}\hline
\begin{tikzpicture}[start chain=1 going right,
start chain=2 going left]
\node [draw,on chain=1] {A};
\node [draw,on chain=1] {B};
\node [draw,on chain=1] {C};
\node [draw,on chain=2] at (3,1) {0};
\node [draw,on chain=2] {1};
\node [draw,on chain=2] {2};
\node [draw,on chain=1] {D};
\end{tikzpicture} 
 &  
\parbox{10cm}{
\BS{begin}\AC{tikzpicture}[\rouge{start chain=1} going right , \\
\blll{start chain=2} going left] \\
\BS{node} [draw,\rouge{on chain=1}] \AC{A}; \\
\BS{node} [draw,\rouge{on chain=1}] \AC{B}; \\
\BS{node}[draw,\rouge{on chain=1}] \AC{C}; \\
\BS{node} [draw,\blll{on chain=2}] at (3,1) \AC{0}; \\
\BS{node} [draw,\blll{on chain=2}] \AC{1}; \\
\BS{node} [draw,\blll{on chain=2}] \AC{2}; \\
\BS{node}[draw,\rouge{on chain=1}] \AC{D}; \\
\BS{end}\AC{tikzpicture}} 
\\ \hline 
\end{tabular} 

\bigskip


\begin{tabular}{|c|c|} \hline  
\rule[-2cm]{0pt}{4cm} 
\begin{tikzpicture}[start chain=going right,baseline=-1.5cm]
\node [draw,on chain] {A};
\node [draw,on chain] {B};
\node [draw,continue chain=going below,on chain] {C};
\node [draw,on chain] {D};
\node [draw,continue chain=going right,on chain] {E};
\end{tikzpicture}
&  
\parbox{11cm}{
\BS{begin}\AC{tikzpicture}[start chain going right]
\BS{node} [draw,on chain] \AC{A}; \\
\BS{node} [draw,on chain] \AC{B}; \\
\BS{node} [draw,\RDD{continue chain}=going below,on chain] \AC{C}; \\
\BS{node}[draw,on chain] \AC{D}; \\
\BS{node} [draw,\RDD{continue chain}=going right,on chain] \AC{E}; \\
\BS{end}\AC{tikzpicture}} 
\\ \hline 
\end{tabular} 

\bigskip

\begin{tabular}{|c|c|}  \hline 
\begin{tikzpicture}[every node/.style=draw,baseline=-1.5cm]
{ [start chain=1]
\node [on chain] {A};
\node [on chain] {B};
\node [on chain] {C};
}
{ [start chain=2 going below]
\node [on chain=2] at (0.5,-.5) {0};
\node [on chain=2] {1};
\node [on chain=2] {2};
}
{ [continue chain=1]
\node [on chain] {D};
}
\end{tikzpicture}
&  
\parbox{10cm}{
\BS{begin}\AC{tikzpicture}[start chain going right] \\
\{ [\RDD{start chain}=1] \\
\BS{node} [draw,on chain] \AC{A}; \\
\BS{node} [draw,on chain] \AC{B}; \\
\BS{node} [draw,on chain] \AC{C}; \\
\} \\
\{ [\RDD{start chain}=2] \\
\BS{node}[draw,on chain=2] \AC{0}; \\
\BS{node}[draw,on chain=2] \AC{1}; \\
\BS{node}[draw,on chain=2] \AC{2}; \\
\} \\
\{ [\RDD{continue chain}=1] \\
\BS{node} [draw,on chain] \AC{D}; \\
\} \\
\BS{end}\AC{tikzpicture}} 
\\  \hline 
\end{tabular} 

\bigskip

\SbSbSSCT{N\oe uds sur la chaine}{Nodes on a Chain}

\begin{center}
\RRR{46-3} 
\end{center}

\bigskip

\begin{tabular}{|c|c|} \hline 
 \begin{tikzpicture}[start chain=XXX placed  {at=(\tikzchaincount*-30+90:1.5)},baseline=0pt]
 \foreach \i in {1,...,12}
 \node [on chain] {\i};
 \draw (0,0) -- (XXX-10);
 \draw (0,0) -- (XXX-2);
 \end{tikzpicture}
&
\parbox{11cm}{
\BS{begin}\AC{tikzpicture}[start chain=\blll{XXX} \RDD{placed} \\ \AC{at=(\BSS{tikzchaincount}*-30+90:1.5)}] \\
 \BS{foreach} \BS{i} in \AC{1,...,12} \\
\BS{node} [on chain] \AC{\BS{i}}; \\
\BS{draw }(0,0) -- \blll{(XXX-10)}; \\
\BS{draw }(0,0) -- \blll{(XXX-2)}; \\
\BS{end}\AC{tikzpicture}} 
\\ \hline 
\end{tabular} 

\bigskip


\begin{tabular}{|c|c|}  \hline 
\begin{tikzpicture}[start chain,baseline=-1cm]
\node [draw,on chain] {A};
\node [draw,on chain] {B};
\node [draw,on chain=going below] {C};
\node [draw,on chain] {D};
\node [draw,on chain] {E};
\end{tikzpicture}
&  
\parbox{11cm}{
\BS{begin}\AC{tikzpicture}[start chain] \\
\BS{node} [draw,on chain] \AC{A}; \\
\BS{node} [draw,on chain] \AC{B}; \\
\BS{node} [draw,on chain=\rouge{going below}] \AC{C}; \\
\BS{node} [draw,on chain] \AC{D}; \\
\BS{node} [draw,on chain] \AC{E}; \\
\BS{end}\AC{tikzpicture}} 
\\  \hline 
\end{tabular} 


\bigskip

\begin{tabular}{|c|c|} \hline 
\begin{tikzpicture}[start chain=going {at=(\tikzchainprevious),shift=(30:1)},baseline=1cm]
\node [draw,on chain] {A};
\node [draw,on chain] {B};
\node [draw,on chain] {C};
\node [draw,on chain] {D};
\end{tikzpicture}  
&  
\parbox{11cm}{
\BS{begin}\AC{tikzpicture}[start chain=going \\ \AC{at=(\BSS{tikzchainprevious},shift=(30:1)}] \\
\BS{node} [draw,on chain] \AC{A}; \\
\BS{node} [draw,on chain] \AC{B}; \\
\BS{node} [draw,on chain] \AC{C}; \\
\BS{node} [draw,on chain] \AC{D}; \\
\BS{end}\AC{tikzpicture}} 
\\ \hline 
\end{tabular} 

\bigskip

\begin{tabular}{|c|c|} \hline 
\begin{tikzpicture}[baseline=1cm]
\node[draw,red] (A) at (0,2) {A};
{ [start chain]
\node [draw,on chain] {B};
\node [draw,on chain] {C};
\chainin (A) [join];
\node [draw,on chain] {D};
\node [draw,on chain] {E};
}
\end{tikzpicture}
&  
\parbox{11cm}{
\BS{begin}\AC{tikzpicture} \\
\BS{node} [draw,red] (A) at (0,2)  \AC{A}; \\
\{ [start chain] \\
\BS{node} [draw,on chain] \AC{B}; \\
\BS{node} [draw,on chain] \AC{C}; \\
\BSS{chainin} (A) [join]; \\
\BS{node} [draw,on chain] \AC{D}; \\
\BS{node} [draw,on chain] \AC{E}; \\
\} \\
\BS{end}\AC{tikzpicture}} 
\\  \hline 
\end{tabular} 



\bigskip

\begin{tabular}{|c|c|} \hline 
\begin{tikzpicture}[baseline=-1cm]
\matrix [matrix of nodes,column sep=1cm,row sep=1cm,every node/.style=draw]
{
|(a) | A 	& |(b) |  B 	& |(c) | C \\
|(d) | D 	& |(e) | E 		& |(f) | F \\
};
{ [start chain,every on chain/.style={join=by ->}]
\chainin (a);
\chainin (b);
\chainin (d);
\chainin (c);
\chainin (f);
\chainin (e);
}
\end{tikzpicture}
&  
\parbox{11cm}{
\BS{begin}\AC{tikzpicture} \\
\BS{matrix} [matrix of nodes,column sep=5mm,row sep=5mm] ,every node/.style=draw \\
\{ \\
|(a) | A 	\& |(b) |  B 	\& |(c) | C \BS{}\BS{} \\
|(d) | D 	\& |(e) | E 	\& |(f) | F \BS{}\BS{} \\
\}; \\
\{ [start chain,every on chain/.style=\AC{join=by ->}] \\
\BSS{chainin} (a);
\BSS{chainin}(b);
\BSS{chainin}(d); \\
\BSS{chainin} (c);
\BSS{chainin}(f);
\BSS{chainin}(e);
\}
\BS{end}\AC{tikzpicture}
} 
\\ \hline 
\end{tabular} 

\bigskip

\SbSbSSCT{Jonction de n\oe uds}{Joining Nodes on a Chain}

\begin{center}
\RRR{46-4}
\end{center} 

\bigskip

\begin{tabular}{|c|c|} \hline 
\begin{tikzpicture}[start chain]
\node [draw,on chain] {A};
\node [draw,on chain,join] {B};
\node [draw,on chain] {C};
\node [draw,on chain,join] {D};
\end{tikzpicture}
&  
\parbox{11cm}{
\BS{begin}\AC{tikzpicture}[start chain] \\
\BS{node} [draw,on chain] \AC{A}; \\
\BS{node} [draw,on chain,\RDD{join}] \AC{B}; \\
\BS{node} [draw,on chain] \AC{C}; \\
\BS{node} [draw,on chain,\RDD{join}] \AC{D}; \\
\BS{end}\AC{tikzpicture}} 
\\ \hline 
\end{tabular} 

\bigskip

\begin{tabular}{|c|c|} \hline 
\begin{tikzpicture}[start chain, every on chain/.style=join, every join/.style=->]
\node [draw,on chain] {A};
\node [draw,on chain] {B};
\node [draw,on chain] {C};
\node [draw,on chain] {D};
\end{tikzpicture}
&  
\parbox{11cm}{
\BS{begin}\AC{tikzpicture}[start chain, \RDD{every on chain}/.style=join, \\ \RDD{every join}/.style=->] \\
\BS{node} [draw,on chain] \AC{A}; \\
\BS{node} [draw,on chain,\RDD{join}] \AC{B}; \\
\BS{node} [draw,on chain] \AC{C}; \\
\BS{node} [draw,on chain,\RDD{join}] \AC{D}; \\
\BS{end}\AC{tikzpicture}} 
\\ \hline 
\end{tabular} 

\bigskip

\begin{tabular}{|c|c|}  \hline 
\begin{tikzpicture}[start chain,baseline=-1cm]
\node [draw,on chain] {A};
\node [draw,on chain] {B};
\node [draw,on chain] {C};
\node [draw,on chain=going below,join=with chain-2 ] {D};
\end{tikzpicture} 
&  
\parbox{11cm}{
\BS{begin}\AC{tikzpicture}[start chain] \\
\BS{node} [draw,on chain] \AC{A}; \\
\BS{node} [draw,on chain] \AC{B}; \\
\BS{node} [draw,on chain] \AC{C}; \\
\BS{node} [draw,on chain=going below,\rouge{join=with chain-2} ] \AC{D}; \\
\BS{end}\AC{tikzpicture}} 
\\ \hline 
\begin{tikzpicture}[start chain,baseline=-1cm]
\node [draw,on chain] {A};
\node [draw,on chain] {B};
\node [draw,on chain] {C};
\node [draw,on chain=going below,join=with chain-1 by {blue,<-}] {D};
\end{tikzpicture}
&
\parbox{12cm}{
\BS{begin}\AC{tikzpicture}[start chain] \\
\BS{node} [draw,on chain] \AC{A}; \\
\BS{node} [draw,on chain] \AC{B}; \\
\BS{node} [draw,on chain] \AC{C}; \\
\BS{node} [draw,on chain=going below,join=with chain-1 \rouge{ by \AC{blue,<-}} ] \AC{D}; \\
\BS{end}\AC{tikzpicture}} 
\\ \hline 
\end{tabular} 



\bigskip

\SbSbSSCT{Branches}{Branches}

\begin{center}
\RRR{46-5}
\end{center} 


\bigskip

\begin{tabular}{|c|c|}  \hline 
\begin{tikzpicture} [baseline=-2cm]
{ [start chain=XXX]
\node [draw,on chain] {A};
\node [draw,on chain] {B};
{ [start branch=YYY going below]
\node [draw,on chain] {1};
\node [draw,on chain] {2};
\node [draw,on chain] {3};
}
\node [draw,on chain,join=with XXX/YYY-end,join=with XXX/YYY-2 ] {C};
}
\end{tikzpicture}
&  
\parbox{12cm}{
\BS{begin}\AC{tikzpicture}\\
\{ [start chain=\blll{XXX}] \\
\BS{node} [draw,on chain] \AC{A}; \\
\BS{node} [draw,on chain] \AC{B}; \\
\{ [\RDD{start branch}=\blll{YYY} going below] \\
\BS{node} [draw,on chain] \AC{1}; \\
\BS{node} [draw,on chain] \AC{2}; \\
\BS{node} [draw,on chain] \AC{3}; \\
\} \\
\BS{node} [ draw,on chain,join=with \blll{XXX/YYY}\rouge{-end}, \\ join=with \blll{XXX/YYY}\rouge{-2}]  \AC{C}; \\
\} \\
\BS{end}\AC{tikzpicture}   } 

\\ \hline 
\end{tabular} 

\bigskip

\begin{tabular}{|c|} \hline 
\BS{begin}\AC{tikzpicture}[ \RDD{node distance}=.2cm and 3cm]
\\ \hline 
\begin{tikzpicture}[ node distance=.2cm and 3cm]
{ [start chain=XXX]
\node [on chain] {A};
\node [on chain] {B};
{ [start branch=YYY going below]
\node [on chain] {1};
\node [on chain] {2};
\node [on chain] {3};
}
\node [on chain,join=with XXX/YYY-end] {C};
}
\end{tikzpicture}
\\ \hline 
\end{tabular} 

\bigskip

\begin{tabular}{|c|c|} \hline 
\begin{tikzpicture}[ node distance=2mm and 1cm,baseline=-2cm]
{ [start chain=XXX]
\node [draw,on chain] {A};
\node [draw,on chain] {B};
{ [start branch=YYY going below]
\node [draw,on chain] {1};
\node [draw,on chain] {2};
\node [draw,on chain] {3};
}
\node [draw,on chain,join=with XXX/YYY-end] {C};
{
[continue branch=YYY]
\node [draw,on chain] {4};
\node [draw,on chain] {5};
}
}
\end{tikzpicture}
&  
\parbox{12cm}{
\BS{begin}\AC{tikzpicture}[ node distance=2mm and 1cm]\\
\{ [start chain=\blll{XXX}] \\
\BS{node} [draw,on chain] \AC{A}; \\
\BS{node} [draw,on chain] \AC{B}; \\
\{ [start branch=\blll{YYY} going below] \\
\BS{node} [draw,on chain] \AC{1}; \\
\BS{node} [draw,on chain] \AC{2}; \\
\BS{node} [draw,on chain] \AC{3}; \} \\
\BS{node}  [draw,on chain,join=with \blll{XXX/YYY}-end]  \AC{C}; \\
\{ [\RDD{continue branch}=\blll{YYY}]\\
\BS{node} [on chain] \AC{4}; \\
\BS{node} [on chain] \AC{5}; \} \\
\} \\
\BS{end}\AC{tikzpicture}   } 
\\ \hline 
\end{tabular} 


\bigskip

\begin{tabular}{|c|c|} \hline 
\begin{tikzpicture}[node distance=2mm and 1cm, every node/.style=draw,baseline=-1cm]
{ [start chain]
\node [on chain] {1};
\node [on chain] {2};
{ [start branch=XXX going below] }
\node [on chain] {3};
{ [start branch=YYY going above] }
\node [on chain] {4};
{ [continue branch=XXX]
\node [on chain] {a};
\node [on chain] {b};
}{
[continue branch=YYY]
\node [on chain] {A};
\node [on chain] {B};
}
}
\end{tikzpicture}
&  
\parbox{12cm}{
\BS{begin}\AC{tikzpicture}[node distance=2mm and 1cm, every node/.style=draw]\\
\{ [start chain] \\
\BS{node} [on chain] \AC{1};  \\
\BS{node} [on chain] \AC{2}; \\
\{ [\RDD{start branch}=\blll{XXX} going below] \} \\
\BS{node} [on chain] \AC{3}; \\
\{ [\RDD{start branch}=\blll{YYY} going above] \} \\
\BS{node} [on chain] \AC{4}; \\
\{ [\RDD{continue branch}=\blll{XXX} ] \\
\BS{node} [on chain] \AC{a}; \\
\BS{node} [on chain] \AC{b};\} \\
\{ [\RDD{continue branch}=\blll{YYY} ] \\
\BS{node} [on chain] \AC{A}; \\
\BS{node} [on chain] \AC{B}; \}  }
\\ \hline 
\end{tabular} 



%%
%%
%%\bigskip
%
%%\begin{tikzpicture}[baseline=-1cm] 
%%\matrix [matrix of nodes,column sep=.5cm,row sep=.5cm,every node/.style=draw]
%%{
%%|(a) [red]|AAA 			& |(b) [circle]| BBB \\
%%|(c)| CCC 		& |(d) [isosceles triangle]| DDD \\
%%|(e) [ellipse]| EEE & |(f)| FFFF \\
%%};
%%\draw (a) -- (f);
%%\end{tikzpicture}
%
%
%\newpage
%
%
%
%%\section{symboles}
%%
%%
\SbSSCT{Coordonnées}{Coordinates}
\begin{center}
\RRR{13-2-1}
\end{center}


\SbSbSSCT{Système de coordonnées \og canvas \fg}{Canvas coordinates}

\noindent


\tikzset{every picture/.style=blue,very thick,inner sep=0pt}

\begin{tabular}{|c|c|} \hline 
\TFRGB{Explicite}{explicit}  & \TFRGB{Implicite}{implicit}
\\ \hline
\begin{tikzpicture}
\draw[help lines] (0,0) grid (3,2);
\fill (canvas cs:x=2cm,y=1.5cm) circle (2pt);
\end{tikzpicture}
&
\begin{tikzpicture}
\draw[help lines] (0,0) grid (3,2);
\fill (2,1.5) circle (2pt);
\end{tikzpicture}

\\ \hline  
 \BS{fill} (\RDD{canvas cs}:\blll{x=2cm,y=1.5cm}) circle (2pt);
& \BS{fill} {\color{blue}(2cm,1.5cm)} circle (2pt);
\\ \hline 
\end{tabular} 


\SbSbSSCT{Système de coordonnées polaire \og canvas \fg}{Polar coordinates}

\noindent


\begin{tabular}{|c|c|c|} \hline
\TFRGB{Explicite}{explicit}  & \TFRGB{Implicite}{implicit}
\\ \hline
\begin{tikzpicture}
\draw[help lines] (0,0) grid (3,2);
\draw [dotted](0,2) arc (90 :0 :2);
\draw [dotted](0,0) --(2,2);
\fill (canvas polar cs:angle=45,radius=2cm) circle (2pt);
\end{tikzpicture}
&
\begin{tikzpicture}
\draw[help lines] (0,0) grid (3,2);
\draw [dotted](0,2) arc (90 :0 :2);
\draw [dotted](0,0) --(2,2);
\fill (45:2cm) circle (2pt);
\end{tikzpicture}
\\ \hline 
\BS{fill} (\RDD{canvas polar cs}:\RDD{angle}=45,\RDD{radius}=2cm) circle (2pt);
&
\BS{fill} {\color{blue}(45:2cm)} circle (2pt);
\\ \hline 
\end{tabular} 

\bigskip
\begin{tabular}{|c|} \hline  
\begin{tikzpicture}
\draw[help lines] (0,0) grid (3,2);
\draw [dotted](0,2) arc (90 :0 :3 and 2);
\draw [dotted](0,0) --(3,2);
\fill (canvas polar cs:angle=45,x radius=3cm,y radius=2cm) circle (2pt);
\end{tikzpicture}
\\ \hline  
\BS{fill} (canvas polar cs:angle=45,\RDD{x radius}=3cm,\RDD{y radius}=2cm) circle (2pt);
\\ \hline 
\end{tabular}


\SbSbSSCT{Système de coordonnées  xyz}{xyz coordinates}

\noindent


\begin{tabular}{|c|c|c|} \hline 
\begin{tikzpicture}[->]
\draw (0,0) -- (xyz cs:x=1);
\draw[red] (0,0) -- (xyz cs:y=1);
\draw[magenta] (0,0) -- (xyz cs:z=1);
\end{tikzpicture}
&
\begin{tikzpicture}[->]
\draw (0,0) -- (1,0,0);
\draw[red]  (0,0) -- (0,1,0);
\draw[magenta]  (0,0) -- (0,0,1);
\end{tikzpicture}
\\ \hline 
\BS{draw} (0,0) - - (\RDD{xyz cs}:x=1); & \BS{draw}  (0,0) - - (1,0,0); \\
\BS{draw}[red]  (0,0) - - (\RDD{xyz cs}:y=1); &  \BS{draw}[red] (0,0) - - (0,1,0); \\
\BS{draw}[magenta]  (0,0) - - (\RDD{xyz cs}:z=1); &  \BS{draw}[magenta]   (0,0) - - (0,0,1); 
\\ \hline 

\end{tabular} 

 
\newpage

\SbSbSSCT{Coordinate system xyz polar}{Coordinate system xyz polar}

\noindent

\begin{tabular}{|c|c|c|} \hline
\TFRGB{Explicite}{explicit}  & \TFRGB{Implicite}{implicit}
\\ \hline
\begin{tikzpicture}
\draw[help lines] (0,0) grid (3,2);
\draw [dotted](0,2) arc (90 :0 :2);
\draw [dotted](0,0) --(2,2);
\fill (xyz polar cs:angle=45,radius=2) circle (2pt);
\end{tikzpicture}
&
\begin{tikzpicture}
\draw[help lines] (0,0) grid (3,2);
\draw [dotted](0,2) arc (90 :0 :2);
\draw [dotted](0,0) --(2,2);
\fill (45:2) circle (2pt);
\end{tikzpicture}
\\ \hline 
\BS{fill} (\RDD{xyz polar cs}:\RDD{angle}=45,\RDD{radius}=2) circle (2pt);
&
\BS{fill} {\color{blue}(45:2cm)} circle (2pt);
\\ \hline 
\end{tabular} 

\bigskip
\begin{tabular}{|c|} \hline  
\begin{tikzpicture}
\draw[help lines] (0,0) grid (3,2);
\draw [dotted](0,2) arc (90 :0 :3 and 2);
\draw [dotted](0,0) --(3,2);
\fill (xyz polar cs:angle=45,x radius=3,y radius=2) circle (2pt);
\end{tikzpicture}
\\ \hline  
\BS{fill} (xyz polar cs:angle=45,\RDD{x radius}=3,\RDD{y radius}=2) circle (2pt);
\\ \hline 
\end{tabular} 

\bigskip

\begin{tabular}{|c|c|c|} \hline
\multicolumn{2}{|c|}{\BS{begin}\AC{tikzpicture}{\color{red}[x=1.5cm,y=1cm]} }
\\ \hline
\begin{tikzpicture}[x=1.5cm,y=1cm]
\draw[help lines] (0,0) grid (3,2);
\draw [dotted](0,2) arc (90 :0 :2);
\draw [dotted](0,0) --(2,2);
\fill (xyz polar cs:angle=45,radius=2) circle (2pt);
\end{tikzpicture}
&
\begin{tikzpicture}[x=1.5cm,y=1cm]
\draw[help lines] (0,0) grid (3,2);
\draw [dotted](0,2) arc (90 :0 :2);
\draw [dotted](0,0) --(2,2);
\fill (45:2) circle (2pt);
\end{tikzpicture}
\\ \hline 
\BS{fill} (\RDD{xyz polar cs}:\RDD{angle}=45,\RDD{radius}=2) circle (2pt);
&
\BS{fill} {\color{blue}(45:2cm)} circle (2pt);
\\ \hline 
\end{tabular} 
\bigskip

\begin{tabular}{|c|c|c|} \hline
\multicolumn{2}{|c|}{\BS{begin}\AC{tikzpicture}{\color{red}[x=\AC{(0cm,1cm)},y=\AC{(-1cm,0cm)}]} }
\\ \hline
\begin{tikzpicture}[x={(0cm,1cm)},y={(-1cm,0cm)}]
\draw[help lines] (0,0) grid (3,2);
\draw [dotted](0,2) arc (90 :0 :2);
\draw [dotted](0,0) --(2,2);
\fill (xyz polar cs:angle=45,radius=2) circle (2pt);
\end{tikzpicture}
&
\begin{tikzpicture}[x={(0cm,1cm)},y={(-1cm,0cm)}]
\draw[help lines] (0,0) grid (3,2);
\draw [dotted](0,2) arc (90 :0 :2);
\draw [dotted](0,0) --(2,2);
\fill (45:2) circle (2pt);
\end{tikzpicture}
\\ \hline 
\BS{fill} (\RDD{xyz polar cs}:\RDD{angle}=45,\RDD{radius}=2) circle (2pt);
&
\BS{fill} {\color{blue}(45:2cm)} circle (2pt);
\\ \hline 
\end{tabular} 

\SbSbSSCT{Coordonnées barycentriques}{Barycentric coordinates}

\begin{center}
\RRR{13-2-2}
\end{center}

\begin{tabular}{|c|c|c|} \hline
\multicolumn{3}{|c|}{  \BS{node} [circle,fill=red!20] at (\RDD{barycentric cs}:A=0.6,B=0.3 ) \AC{X};   }\\ 
\hline
\begin{tikzpicture}[scale=.6]
\draw[help lines] (0,0) grid (4,4);
\node[circle,fill=green!20,] (A) at (0,0) {A};
\node[circle,fill=green!20,] (B) at (4,0) {B};
\node[circle,fill=red!20] at (barycentric cs:A=0.3,B=0.3 ) {X};
\end{tikzpicture}
&
\begin{tikzpicture}[scale=.6]
\draw[help lines] (0,0) grid (4,4);
\node[circle,fill=green!20,] (A) at (0,0) {A};
\node[circle,fill=green!20,] (B) at (4,0) {B};
\node[circle,fill=green!20,] (C) at (4,4) {C};
\node[circle,fill=red!20] at (barycentric cs:A=0.4,B=0.4 ,C=.4) {X};
\end{tikzpicture}
&
\begin{tikzpicture}[scale=.6]
\draw[help lines] (0,0) grid (4,4);
\node[circle,fill=green!20,] (A) at (0,0) {A};
\node[circle,fill=green!20,] (B) at (4,0) {B};
\node[circle,fill=green!20,] (C) at (1,4) {C};
\node[circle,fill=green!20,] (D) at (4,4) {D};
\node[circle,fill=red!20] at (barycentric cs:A=0.5,B=0.5,C=.5,D=.5 ) {X};
\end{tikzpicture}
\\ \hline
A=0.3,B=0.3 & A=0.4,B=0.4 ,C=.4 & A=0.5,B=0.5,C=.5,D=.5 
\\ \hline
\begin{tikzpicture}[scale=.6]
\draw[help lines] (0,0) grid (4,4);
\node[circle,fill=green!20,] (A) at (0,0) {A};
\node[circle,fill=green!20,] (B) at (4,0) {B};
\node[circle,fill=red!20] at (barycentric cs:A=0.6,B=0.3 ) {X};
\end{tikzpicture}
&
\begin{tikzpicture}[scale=.6]
\draw[help lines] (0,0) grid (4,4);
\node[circle,fill=green!20,] (A) at (0,0) {A};
\node[circle,fill=green!20,] (B) at (4,0) {B};
\node[circle,fill=green!20,] (C) at (4,4) {C};
\node[circle,fill=red!20] at (barycentric cs:A=0.2,B=0.4 ,C=.6) {X};
\end{tikzpicture}
&
\begin{tikzpicture}[scale=.6]
\draw[help lines] (0,0) grid (4,4);
\node[circle,fill=green!20,] (A) at (0,0) {A};
\node[circle,fill=green!20,] (B) at (4,0) {B};
\node[circle,fill=green!20,] (C) at (1,4) {C};
\node[circle,fill=green!20,] (D) at (4,4) {D};
\node[circle,fill=red!20] at (barycentric cs:A=0.2,B=0.4,C=.6,D=.8 ) {X};
\end{tikzpicture}
\\ \hline
A=0.6,B=0.3 & A=0.2,B=0.4 ,C=.6 & A=0.2,B=0.4,C=.6,D=.8
\\ \hline
\end{tabular}

\SbSbSSCT{Coordonnées nominatives : n\oe ud}{Named coordinates: nodes}

\begin{center}
\RRR{13-2-3}
\end{center}

\begin{tabular}{|c|c|} \hline  
\begin{tikzpicture}[blue,very thick,baseline=1cm]
\draw[help lines] (0,0) grid (3,3);
\coordinate (centre) at (1.5,1.5) ;
\coordinate (A) at (.5,.5) ;
\coordinate (B) at (2.5,2.5) ;
\fill (centre) circle (3pt);
\draw[red] (A) rectangle (B) ;
\end{tikzpicture}
&  
\parbox[c]{8cm}{
\BSS{coordinate} {\color{blue}(centre)} at(1.5,1.5) ; \\
\BSS{coordinate} {\color{blue}(A)} at (.5,.5) ;\\
\BSS{coordinate} {\color{blue}(B)} at  (2.5,2.5) ;\\
\\
\BS{fill} {\color{blue}(centre)} circle (3pt);\\
\BS{draw}[red] {\color{blue}(A)} rectangle {\color{blue}(B)} ;\\
}
\\ \hline 
\end{tabular} 


\TFRGB{voir aussi}{see also} page \pageref{noeuds}


\SbSbSSCT{Coordonnées relatives à un noeud}{Coordinates relative to a node}

\noindent

\begin{tabular}{|c|c|c|c|} \hline
\multicolumn{4}{|l|}{  \BS{node} [draw,fill=green!20,] (A) at (1,1) \AC{\BS{huge}  noeud}; }\\ 
\multicolumn{4}{|l|}{  \BS{fill}[red] (\RDD{node cs}:\RDD{name}=A,\RDD{anchor}=south) circle (3pt);   }\\ 
\hline

\begin{tikzpicture}
\draw[help lines] (0,0) grid (2,2);
\node[draw,fill=green!20,] (A) at (1,1) {\huge noeud};
\fill[red] (node cs:name=A,anchor=south) circle (3pt);
\end{tikzpicture}
&
\begin{tikzpicture}
\draw[help lines] (0,0) grid (2,2);
\node[draw,fill=green!20,] (A) at (1,1) {\huge noeud};
\fill[red] (node cs:name=A,anchor=west) circle (3pt);
\end{tikzpicture}
&
\begin{tikzpicture}
\draw[help lines] (0,0) grid (2,2);
\node[draw,fill=green!20,] (A) at (1,1) {\huge noeud};
\fill[red] (node cs:name=A,anchor=north) circle (3pt);
\end{tikzpicture}
&
\begin{tikzpicture}
\draw[help lines] (0,0) grid (2,2);
\node[draw,fill=green!20,] (A) at (1,1) {\huge noeud};
\fill[red] (node cs:name=A,anchor=east) circle (3pt);
\end{tikzpicture}
\\ \hline
name=A,anchor=south & name=A,anchor=west & name=A,anchor=north & name=A,anchor=east
\\ \hline
\end{tabular}

\bigskip

\begin{tabular}{|c|c|c|c|} \hline
\multicolumn{4}{|l|}{  \BS{node} [draw,fill=green!20,] \blll{(A)} at (1,1) \AC{\BS{huge}  noeud}; }\\ 
\multicolumn{4}{|l|}{  \BS{fill}[red] (\blll{A}.south) circle (3pt);   }\\ 
\hline

\begin{tikzpicture}
\draw[help lines] (0,0) grid (2,2);
\node[draw,fill=green!20,] (A) at (1,1) {\huge noeud};
\fill[red] (A.south) circle (3pt);
\end{tikzpicture}
&
\begin{tikzpicture}
\draw[help lines] (0,0) grid (2,2);
\node[draw,fill=green!20,] (A) at (1,1) {\huge noeud};
\fill[red] (A.west) circle (3pt);
\end{tikzpicture}
&
\begin{tikzpicture}
\draw[help lines] (0,0) grid (2,2);
\node[draw,fill=green!20,] (A) at (1,1) {\huge noeud};
\fill[red] (A.north) circle (3pt);
\end{tikzpicture}
&
\begin{tikzpicture}
\draw[help lines] (0,0) grid (2,2);
\node[draw,fill=green!20,] (A) at (1,1) {\huge noeud};
\fill[red] (A.east) circle (3pt);
\end{tikzpicture}
\\ \hline
A.south & A.west & A.north & A.east
\\ \hline
\end{tabular}



\bigskip
\begin{tabular}{|c|c|c|c|} \hline
\multicolumn{4}{|c|}{  \BS{fill}[red] (node cs:\RDD{name}=A,\RDD{angle}=0) circle (3pt);  }\\ 
\hline

\begin{tikzpicture}
\draw[help lines] (0,0) grid (2,2);
\node[draw,fill=green!20,] (A) at (1,1) {\huge noeud};
\fill[red] (node cs:name=A,angle=0) circle (3pt);
\end{tikzpicture}
&
\begin{tikzpicture}
\draw[help lines] (0,0) grid (2,2);
\node[draw,fill=green!20,] (A) at (1,1) {\huge noeud};
\fill[red] (node cs:name=A,angle=-30) circle (3pt);
\end{tikzpicture}
&
\begin{tikzpicture}
\draw[help lines] (0,0) grid (2,2);
\node[draw,fill=green!20,] (A) at (1,1) {\huge noeud};
\fill[red] (node cs:name=A,angle=-90) circle (3pt);
\end{tikzpicture}
&
\begin{tikzpicture}
\draw[help lines] (0,0) grid (2,2);
\node[draw,fill=green!20,] (A) at (1,1) {\huge noeud};
\fill[red] (node cs:name=A,angle=-150) circle (3pt);
\end{tikzpicture}
\\ \hline
name=A,angle=0 & name=A,angle=-30 & nname=A,angle=-90 & name=A,angle=-150
\\ \hline
\end{tabular}

\bigskip


\begin{tabular}{|c|c|c|c|} \hline
\multicolumn{4}{|c|}{  \BS{fill}[red] (A.0) circle (3pt);  }\\ 
\hline

\begin{tikzpicture}
\draw[help lines] (0,0) grid (2,2);
\node[draw,fill=green!20,] (A) at (1,1) {\huge noeud};
\fill[red] (A.0) circle (3pt);
\end{tikzpicture}
&
\begin{tikzpicture}
\draw[help lines] (0,0) grid (2,2);
\node[draw,fill=green!20,] (A) at (1,1) {\huge noeud};
\fill[red] (A.-30) circle (3pt);
\end{tikzpicture}
&
\begin{tikzpicture}
\draw[help lines] (0,0) grid (2,2);
\node[draw,fill=green!20,] (A) at (1,1) {\huge noeud};
\fill[red] (A.-90) circle (3pt);
\end{tikzpicture}
&
\begin{tikzpicture}
\draw[help lines] (0,0) grid (2,2);
\node[draw,fill=green!20,] (A) at (1,1) {\huge noeud};
\fill[red] (A.-150) circle (3pt);
\end{tikzpicture}
\\ \hline
A.0 & A.-30 & A.-90 & A.-150
\\ \hline
\end{tabular}

\TFRGB{voir aussi}{see also} page \pageref{nomnoeud}


\newpage

\SbSbSSCT{Coordonnées relatives à deux points}{Coordinates relative to two points}
\begin{center}
\RRR{13-3-1}
\end{center}

\begin{tabular}{|c|c|} \hline
\multicolumn{2}{|c|}{  \BS{node} [circle,fill=red!20] at (1,1 {\color{red}|-} 3,3) \AC{X}   }\\ 
\hline
\begin{tikzpicture}
\draw[help lines] (0,0) grid (4,4);
\node[circle,fill=green!20,] (A) at (1,1) {A};
\node[circle,fill=green!20,] (B) at (3,3) {B};
\node[circle,fill=red!20] at (1,1 |- 3,3) {X};
\end{tikzpicture}
&
\begin{tikzpicture}
\draw[help lines] (0,0) grid (4,4);
\node[circle,fill=green!20,] (A) at (1,1) {A};
\node[circle,fill=green!20,] (B) at (3,3) {B};
\node[circle,fill=red!20] at (1,1 -| 3,3) {X};
\end{tikzpicture}
\\ \hline
at (1,1 {\color{red}|-} 3,3)
&
at (1,1 {\color{red}-|} 3,3)
\\ \hline
\end{tabular}



\SbSbSSCT{Coordonnée relative à une intersection}{Coordinates relative to an intersection}
\begin{center}
\RRR{13-3-2}
\end{center}

 \maboite{\BS{usetikzlibrary}\AC{intersections}}
\label{lib-intersections}


\begin{tabular}{|c|c|c|c|} \hline 
\multicolumn{4}{|l|}{  \BS{draw} [\RDD{name path}=XXX] (2,1) circle  (1cm);   }\\ 
\multicolumn{4}{|l|}{  \BS{draw} [\RDD{name path}=YYY] (0.5,0.5) rectangle +(3,1);   }\\ 
\multicolumn{4}{|l|}{ \BS{fill} [red,\RDD{ name intersections}=\AC{of=xxx and YYY}]
(\RDD{intersection}-1) circle (2pt)   }\\ 
\hline 
\begin{tikzpicture}[scale=.8]
\draw [help lines] grid (4,2);
\draw [name path=XXX] (2,1) circle  (1cm);
\draw [name path=YYY] (0.5,0.5) rectangle +(3,1);
\fill [red, name intersections={of=XXX and YYY}]
(intersection-1) circle (2pt)  ;
\end{tikzpicture}
& 
\begin{tikzpicture}[scale=.8]
\draw [help lines] grid (4,2);
\draw [name path=XXX] (2,1) circle  (1cm);
\draw [name path=YYY] (0.5,0.5) rectangle +(3,1);
\fill [red, name intersections={of=XXX and YYY}] (intersection-2) circle (2pt) ;
\end{tikzpicture} 
&  
\begin{tikzpicture}[scale=.8]
\draw [help lines] grid (4,2);
\draw [name path=XXX] (2,1) circle  (1cm);
\draw [name path=YYY] (0.5,0.5) rectangle +(3,1);
\fill [red, name intersections={of=XXX and YYY}] (intersection-3) circle (2pt) ;
\end{tikzpicture}
&  
\begin{tikzpicture}[scale=.8]
\draw [help lines] grid (4,2);
\draw [name path=XXX] (2,1) circle  (1cm);
\draw [name path=YYY] (0.5,0.5) rectangle +(3,1);
\fill [red, name intersections={of=XXX and YYY}] (intersection-4) circle (2pt) ;
\end{tikzpicture}
\\ 
\hline intersection-1 & intersection-2 &intersection-3  & intersection-4 \\ 
\hline 
\end{tabular} 

\bigskip

\begin{tabular}{|c|} \hline  
\BS{fill} [red, name intersections=\AC{of=XXX and YYY}] \\
(intersection-1) circle (2pt) {\color{red} node[black,above right] \AC{point a}} ;
\\ \hline  
\begin{tikzpicture}
\draw [help lines] grid (4,2);
\draw [name path=XXX] (2,1) circle  (1cm);
\draw [name path=YYY] (0.5,0.5) rectangle +(3,1);
\fill [red, name intersections={of=XXX and YYY}]
(intersection-1) circle (2pt) node[black,above right] {point a} ;
\end{tikzpicture} 
\\ \hline 
\end{tabular} 

\bigskip

\begin{tabular}{|c|} \hline 
\BS{fill} [red, name intersections=\AC{of=XXX and YYY, \RDD{name}=ZZZ}]; \\
\BS{draw} [red] (ZZZ-1) - - (ZZZ-3); \BS{draw} [green] (ZZZ-2) - - (ZZZ-4);
\\ \hline  
\begin{tikzpicture}
\draw [help lines] grid (4,2);
\draw [name path=XXX] (2,1) circle  (1cm);
\draw [name path=YYY] (0.5,0.5) rectangle +(3,1);
\fill [red, name intersections={of=XXX and YYY, name=ZZZ}];
\draw [red] (ZZZ-1) -- (ZZZ-3);
\draw [green] (ZZZ-2) -- (ZZZ-4);
\end{tikzpicture}
\\ \hline 
\end{tabular} 

\bigskip
\begin{tabular}{|c|} \hline  
\BS{fill} [red, name intersections=\AC{of=XXX and YYY , \RDD{by}=\AC{a,b,c,d}}]; \\
\BS{draw} [red] (a) - - (c); \hspace{1cm} \BS{draw} [green] (b) - - (d);
\\ \hline   
\begin{tikzpicture}
\draw [help lines] grid (4,2);
\draw [name path=XXX] (2,1) circle  (1cm);
\draw [name path=YYY] (0.5,0.5) rectangle +(3,1);
\fill [red, name intersections={of=XXX and YYY, by={a,b,c,d}}];
\draw [red] (a) -- (c);
\draw [green] (b) -- (d);
\end{tikzpicture}
\\ \hline 
\end{tabular} 

\bigskip

\begin{tabular}{|c|} \hline  
\BS{fill} [name intersections=\AC{of=XXX and YYY, name=i, \RDD{total}=\BS{t}}] [red] \\
\BS{foreach} \BS{s} in \AC{1,...,\BS{t}} \AC{(i-\BS{s}) circle (2pt) node[black,above right] \AC{\BS{s}}}
\\ \hline  
\begin{tikzpicture}
\draw [help lines] grid (4,2);
\draw [name path=XXX] (2,1) circle  (1cm);
\draw [name path=YYY] (0.5,0.5) rectangle +(3,1);
\fill [name intersections={of=XXX and YYY , name=i, total=\t}]
[red]
\foreach \s in {1,...,\t}{(i-\s) circle (2pt) node[black,above right] {\s}};
\end{tikzpicture}
\\ \hline 
\end{tabular} 



\newpage

\SbSbSSCT{Position calculée avec le module  \og  pgfmath \fg}{Calculated positions with  \og  pgfmath \fg }

\begin{center}
\RRR{13-2-1}
\end{center}

\TFRGB{Ce module est chargé automatiquement avec le module Tikz}{Package automatically loaded with Tikz} 

\begin{tabular}{|c|} \hline 
\begin{tikzpicture}
\draw[help lines] (0,0) grid (4,2);
\fill [red] (canvas cs:x=2cm+1.5cm,y=1.5cm-1cm) circle (3pt);
\fill [blue] (2cm,1.5cm) circle (3pt);
\draw[dashed] (2,1.5) -| (3.5,.5);
\end{tikzpicture}
\\ \hline 
\emph{\TFRGB{Explicite}{explicit}} 
 : \BS{fill} [red] (\RDD{canvas cs}:x=2cm+1.5cm,y=1.5cm-1cm) circle (3pt);
 \\  \hline 
\emph{\TFRGB{Implicite}{implicit}} :  \BS{fill} [red] {\color{red}(2cm+1.5cm,1.5cm-1cm)} circle (3pt);
\\ \hline 
\end{tabular} 

\bigskip
\begin{tabular}{|c|c|c|} \hline 
\begin{tikzpicture}[baseline=0pt]
\draw[help lines] (0,0) grid (4,4);
 \draw[dashed] (2,2) circle (2);
\fill[red](2+ 2*cos 30,2+2*sin 30) circle (3pt);
\fill[magenta](2+ 2*cos{(120)},2+2*sin{(120)}) circle (3pt);
\end{tikzpicture}
&
\parbox[c]{8cm}{
 \BS{draw}[dashed] (2,2) circle (2);\\
 \smallskip
 \BS{fill} [red]{\color{red}(2+ 2*cos 30 , 2+2*sin 30)} circle (3pt);\\
  \smallskip
 \BS{fill}[magenta] {\color{red}(2+2*cos\AC{(120)} , 2+2*sin\AC{(120)})} circle (3pt); 
 }
\\ \hline 
\end{tabular} 

\SbSbSSCT{Position calculée avec \og library calc \fg}{Calculated positions with \og  calc  library calc \fg}

\begin{center}
\RRR{13-5}
\end{center}
\label{lib-calc}

 \maboite{\BS{usetikzlibrary}\AC{calc}}
 
\begin{tabular}{|c|c|} \hline  
\begin{tikzpicture}[baseline=0pt]
\draw [help lines] (0,0) grid (3,2);
\node (a) at (1,1) {A};
\fill [red] ($(a) + 2/3*(1cm,0)$) circle (2pt);
\fill [red] ($(a) + 4/3*(1cm,0)$) circle (2pt);
\end{tikzpicture}
&
\parbox{8cm}{
\BS{node} (a) at (1,1) \AC{A}; \\
\BS{fill} [red] {\color{red} (\$(a) + 2/3*(1cm,0)\$)} circle (2pt); \\
\BS{fill} [red] {\color{red}(\$(a) + 4/3*(1cm,0)\$)} circle (2pt); \\
}
\\ 
\hline 
\end{tabular} 

\SbSbSSCT{Tangentes avec \og library calc \fg}{Tangents with  \og calc library  \fg}

\begin{center}
\RRR{13-2-4}
\end{center}

\begin{tabular}{|c|c|} \hline 
\multicolumn{2}{|l|}{\BS{node}[fill=green!20] (a) at (3,1.5) \AC{A}; } \\
\multicolumn{2}{|l|}{\BS{fill}[red] (\RDD{tangent cs}:\RDD{node}=c,\RDD{point}=\AC{(A)},\RDD{solution}=1);  }\\ 
\hline
\begin{tikzpicture}
\draw[help lines] (0,0) grid (4,2);
\node[fill=green!20] (A) at (3,1.5) {A};
\node [circle,draw] (c) at (1,1) [minimum size=1.5cm] {$c$};
\draw[red,dashed] (A) - -(tangent cs:node=c,point={(A)},solution=1) ;
\draw[red,dashed] (1,1) - -(tangent cs:node=c,point={(3,1.5)},solution=1) ;
\fill[red] (tangent cs:node=c,point={(A)},solution=1) circle (3pt);
\end{tikzpicture}
&
\begin{tikzpicture}
\draw[help lines] (0,0) grid (4,2);
\node[fill=green!20] (A) at (3,1.5) {A};
\node [circle,draw] (c) at (1,1) [minimum size=1.5cm] {$c$};
\draw[red,dashed] (A) - -(tangent cs:node=c,point={(A)},solution=2) ;
\draw[red,dashed] (1,1) - -(tangent cs:node=c,point={(A)},solution=2) ;
\fill[red] (tangent cs:node=c,point={(A)},solution=2) circle (3pt);
\end{tikzpicture}
\\ \hline
\RDD{solution}=1 & \RDD{solution}=2
\\ \hline
\end{tabular} 

\newpage

\SbSbSSCT{Point à pourcentage donné }{Percentage position }

\begin{center}
\RRR{13-5-3}
\end{center}


\begin{tabular}{|c|c|} \hline  
\multicolumn{2}{|c|}{\BS{fill}[red] ({\color{red}\$(0,1)!.25!(4,1)\$}) circle (4pt); } \\  \hline  

\begin{tikzpicture}
\draw [help lines] (0,0) grid (4,2);
\draw [line width= 3pt] (0,1) -- (4,1);
\fill[red] ($(0,1)!.25!(4,1)$) circle (4pt);
\end{tikzpicture}
&  
\begin{tikzpicture}
\draw [help lines] (0,0) grid (4,2);
\draw [line width= 3pt] (0,1) -- (4,1);
\fill[red] ($(0,1)!.75!(4,1)$) circle (4pt);
\end{tikzpicture}
\\ \hline (0,1)!{\color{red}0.25}!(4,1) & (0,1)!{\color{red}0.75}!(4,1) \\ 
\hline 
\end{tabular} 

\bigskip

\begin{tabular}{|c|} \hline  
\begin{tikzpicture}
\draw [help lines] (0,0) grid (4,3);
\draw [line width=2pt ](0,2) -- (4,2);
\draw[red] ($(0,2)!.75!(4,2)$) -- (0,0);
\fill[red] ($(0,2)!.75!(4,2)!.66!(0,0)$) circle (4pt);
\end{tikzpicture}
\\ \hline 
\BS{fill}[red] (\${\color{blue}(0,2)!0.75!(4,2)}!{\color{red}0.66!(0,0)}\$) circle (2pt);
\\ \hline 
\end{tabular} 


\SbSbSSCT{Point à distance donnée}{Position at a given distance }

\begin{center}
\RRR{13-5-4}
\end{center}

\begin{tabular}{|c|c|} \hline  
\multicolumn{2}{|c|}{\BS{fill}[red] ({\color{red}\$(0,1)!1.5cm!(4,1)\$}) circle (4pt); } \\  \hline  

\begin{tikzpicture}
\draw [help lines] (0,0) grid (4,2);
\draw [line width= 2pt] (0,1) -- (4,1);
\fill[red] ($(0,1)!1.5cm!(4,1)$) circle (4pt);
\end{tikzpicture}
&  
\begin{tikzpicture}
\draw [help lines] (0,0) grid (4,2);
\draw [line width= 2pt] (0,1) -- (4,1);
\fill[red] ($(0,1)!3cm!(4,1)$) circle (4pt);
\end{tikzpicture}
\\ \hline (0,1)!{\color{red}1.5cm}!(4,1) & (0,1)!{\color{red}3cm}!(4,1) \\ 
\hline 
\end{tabular} 

\bigskip

\begin{tabular}{|c|} \hline  
\begin{tikzpicture}
\draw [help lines] (0,0) grid (4,4);
\coordinate (a) at (1,0);
\coordinate (b) at (4,1);
\draw [line width= 3pt] (0,0) -- (4,1);
\draw [line width= 2pt,red](2,.5) -- ($ (2,.5)!2cm!90:(4,1) $);
\end{tikzpicture}
\\ \hline
\BS{draw} (2,.05) - - (\$ (2,0.5)!{\color{red}2cm!90:(4,1)} \$);
\\ \hline 
\end{tabular} 

\newpage

\SbSbSSCT{Coordonnées relatives}{Relative coordinates}


\Par{Cartésienne}{Cartesian coordinates}

\begin{center}
\RRR{13-4-1}
\end{center}

\begin{tabular}{|c|c|c|} \hline  
\TFRGB{relative à l'origine}{relative to the origin}  & \TFRGB{relative à une position}{relative to a position}  &  \TFRGB{relative à la dernière position}{relative to the last position}   
\\ \hline  
 
\begin{tikzpicture}
\draw[help lines] (0,-1) grid (3,1); 
 \draw[blue,very thick] (0,0) -- (1,0) - - (2,1) - - (2,-1);
 \fill[red] (0,0) circle (4pt);
\end{tikzpicture}
&
\begin{tikzpicture} %[scale=.8]
\draw[help lines] (0,-1) grid (4,1);
 \draw[blue,very thick] (0,0) - - (1,0) -- +(2,1) -- +(2,-1) ; %–- +(2,-1) ;
 \fill[red] (1,0) circle (4pt);
\end{tikzpicture}
&
\begin{tikzpicture} %[scale=.8]
\draw[help lines] (0,-1) grid (5,1);  
 \draw[blue,very thick] (0,0) -- (1,0)  - - ++(2,1) - - ++(2,-1);
 \fill[red] (1,0) circle (4pt);
 \fill[red] (3,1) circle (4pt);
\end{tikzpicture}
\\ \hline 
\tikz \fill node[fill=green!20,inner sep=0pt]{(0,0)}; - - (1,0) &
 (0,0) - - \tikz \fill node[fill=green!20,inner sep=0pt]{(1,0)};  & (0,0) - - \tikz \fill node[fill=green!20,inner sep=0pt]{(1,0)}; \\
 - - (2,1) - - (2,-1)  &
   - - +(2,1) - - +(2,-1) & - - ++\tikz \fill node[fill=green!20,inner sep=0pt]{(2,1)}; - - ++(2,-1)
\\ \hline 
\end{tabular} 

\bigskip

\begin{tabular}{|c|c|c|} \hline  
\begin{tikzpicture} [scale=.5]
\draw[help lines] (0,-1) grid (6,6);
 \draw[red,dotted,line width=2pt] (0,0) rectangle (2,2) ;
  \draw[green,dotted,line width=2pt] (0,0) rectangle (3,3) ;  
 \draw[blue,line width=2pt] (0,0) rectangle (1,1)  rectangle (2,2) rectangle (3,3);

\end{tikzpicture}

&  
\begin{tikzpicture} [scale=.5]
\draw[help lines] (0,-1) grid (6,6); 
  \draw[green,dotted,line width=2pt] (1,1) rectangle (4,4) ;   
 \draw[blue,line width=2pt] (0,0) rectangle (1,1)  rectangle +(2,2) rectangle +(3,3);
    \fill[red] (1,1) circle (4pt);
\end{tikzpicture}
&  
\begin{tikzpicture} [scale=.5]
\draw[help lines] (0,-1) grid (6,6);  
 \draw[blue,line width=2pt] (0,0) rectangle (1,1)  rectangle ++(2,2) rectangle ++(3,3);
    \fill[red] (1,1) circle (4pt);
     \fill[green] (3,3) circle (4pt); 
\end{tikzpicture}
\\ 
\hline 
\BS{draw} (0,0) rectangle (1,1)   &
\BS{draw} (0,0) rectangle (1,1)   & 
\BS{draw} (0,0) rectangle (1,1)  \\
rectangle (2,2) rectangle (3,3);  &
rectangle +(2,2) rectangle +(3,3);  &
rectangle ++(2,2) rectangle ++(3,3); \\
\hline 
\end{tabular}


\Par{Polaire }{Polar} {}

\bigskip


\noindent

\begin{tabular}{|c|c|c|c|} \hline
\TFRGB{relative à l'origine}{relative to the origin}  & \TFRGB{relative à une position}{relative to a position}  &  \TFRGB{relative à la dernière position}{relative to the last position}   
\\ \hline    
\begin{tikzpicture} %[scale=.8] 
\draw[help lines] (0,-1) grid (3,1);
 \fill[red] (0:0) circle (4pt);
 \draw[blue,very thick] (0:0)-- (0:1) -- (30:2) -- (-30:2);
\end{tikzpicture}
&
\begin{tikzpicture} %[scale=.8] 
\draw[help lines] (0,-1) grid (4,1);
 \fill[red] (1,0) circle (4pt);
 \draw[blue,very thick] (0:0) -- (0:1) -- +(30:2) -- +(-30:2);
\end{tikzpicture}
&
\begin{tikzpicture} %[scale=.8] 
\draw[help lines] (0,-1) grid (5,1);
 \fill[red] (1,0) circle (4pt);
 \fill[red] (2.732,1) circle (4pt);
 \draw[blue,very thick] (0:0)-- (0:1) -- ++(30:2) -- ++(-30:2);
\end{tikzpicture}
\\ \hline
\tikz \fill node[fill=green!20,inner sep=0pt] {(0:0)}; - - (0:1)&
 (0:0) - - \tikz \fill node[fill=green!20,inner sep=0pt] {(0:1)}; & (0:0)- - \tikz \fill node[fill=green!20,inner sep=0pt] {(0:1)}; \\
 - - (30:2) - - (-30:2)  &  - -  +(30:2) - - +(-30:2) & - -  ++\tikz \fill node[fill=green!20,inner sep=0pt] {(30:2)}; - - ++(-30:2)
\\ \hline 
\end{tabular} 

%\subsubsection{coordonnée relative en polaire}
\Par{coordonnée relative en polaire}{Relative polar coordinate}

\begin{center}
\RRR{13-4-2}
\end{center}
\bigskip

\begin{tabular}{|c|c|} \hline 
\multicolumn{2}{|c|}{ \BS{draw}[blue,very thick] (0,0) -- (2,1) -- ([turn]-45:1cm);}
 \\ \hline
\begin{tikzpicture} %[scale=.8] 
\draw[help lines] (0,0) grid (4,2);
 \draw[dotted] (0,0) -- (4,2);
 \draw[blue,very thick] (0,0) -- (2,1) -- ([turn]-45:1cm);
\end{tikzpicture}
&  
\begin{tikzpicture} %[scale=.8] 
\draw[help lines] (0,0) grid (4,2);
 \draw[dotted] (0,0) -- (4,2);
 \draw[blue,very thick] (0,0) -- (2,1) -- ([turn]45:1cm);
\end{tikzpicture}
\\ \hline ([\RDD{turn}]-45:1cm) & ([\RDD{turn}]45:1cm) \\ 
\hline 
\end{tabular}

\bigskip

\begin{tabular}{|c|c|} \hline  
\begin{tikzpicture}  
\draw[help lines] (-1,0) grid (4,3);
\draw [line width=2pt] (4,0) arc (0 :120 :2)  -- ([turn]90:2cm) ;

\end{tikzpicture}
&  
\begin{tikzpicture} %[scale=.8] 
\draw[help lines] (0,0) grid (4,3);
\draw [line width=2pt]  (0,0) to [bend left] (2,2) --  ([turn]0:2cm);
\fill [red](2,2) circle (4pt);
\end{tikzpicture}
\\ \hline  
\BS{draw} (4,0) arc (0 :120 :2)  - - ([\RDD{turn}]90:2cm) ;
& \BS{draw}  (0,0) to [bend left] (2,2) - -  ([\RDD{turn}]0:2cm); \\

\hline 
\end{tabular} 


%\bigskip 
%
%
%\tikz [delta angle=30, radius=1cm]
%\draw (0,0) arc [start angle=0] -- ([turn]0:1cm)
%arc [start angle=30] -- ([turn]0:1cm)
%arc [start angle=60] -- ([turn]30:1cm);



\bigskip

\begin{tabular}{|c|c|c|} \hline  
\multicolumn{3}{|c|}{ \BS{draw}(1,2)
.. controls ([turn]0:2cm) .. ([turn]-90:2cm); }
\\ \hline
\begin{tikzpicture} %[scale=.8] 
\draw[help lines] (0,0) grid (4,4);
 \draw [line width=2pt] (1,2)
.. controls ([turn]0:2cm) .. ([turn]-90:2cm);
\end{tikzpicture}
&  
\begin{tikzpicture} %[scale=.8] 
\draw[help lines] (0,0) grid (4,4);
 \draw [line width=2pt] (1,2)
.. controls ([turn]30:2cm) .. ([turn]-90:2cm);
\end{tikzpicture}
&  
\begin{tikzpicture} %[scale=.8] 
\draw[help lines] (-2,0) grid (2,4);
 \draw [line width=2pt] (1,2)
.. controls ([turn]0:2cm) .. ([turn]90:2cm);

\end{tikzpicture}
\\ \hline ([turn]0:2cm) .. ([turn]-90:2cm) & ([turn]30:2cm) .. ([turn]-90:2cm) & ([turn]0:2cm) .. ([turn]90:2cm) \\ 
\hline 
\end{tabular} 


\tikzset{every picture/.style=blue,very thick,inner sep=.3333em}

%
%%\newpage
%%
%%\SbSSCT{Le peuple TikZ}{Tikzpeople}

\label{people}

 \maboite{\BS{usepackage}\AC{tikzpeople} \cite {tikzpeople} \footnote{ conflit \BS{usetikzlibrary}\AC{patterns} page \pageref{lib-patterns} : placer cette commande en premier} }

\bigskip
\begin{tabular}{|c|c|}\hline  
\BS{tikz} \BS{node}[\RDD{alice}] at (0,0) {};  &  \tikz \node[alice] at (0,0) {};\\ 
\hline 
\end{tabular} 
 
\SbSbSSCT{Personages disponibles}{available characters}

\noindent



\begin{tabular}{|c|c|c|c|c|c|c|}\hline 
\multicolumn{7}{|c|}{ \BS{tikz} \BS{node}[\RDD{alice},minimum size=1.5cm] at (0,0) {};  }
\\ \hline  
\tikz \node[alice,minimum size=1.5cm] at (0,0) {}; &  
\tikz \node[bob,minimum size=1.5cm] at (0,0) {}; &  
\tikz \node[bride,minimum size=1.5cm] at (0,0) {}; &  
\tikz \node[builder,minimum size=1.5cm] at (0,0) {}; &  
\tikz \node[businessman,minimum size=1.5cm] at (0,0) {}; &  
\tikz \node[charlie,minimum size=1.5cm] at (0,0) {}; &  
\tikz \node[chef,minimum size=1.5cm] at (0,0) {}; 
\\  \hline  
\RDD{alice} & \RDD{bob} & \RDD{bride} & \RDD{builder} & \RDD{businessman} & \RDD{charlie}  & \RDD{chef} 
\\  \hline  
\tikz \node[conductor,minimum size=1.5cm] at (0,0) {}; &  
\tikz \node[cowboy,minimum size=1.5cm] at (0,0) {}; &  
\tikz \node[criminal,minimum size=1.5cm] at (0,0) {}; &  
\tikz \node[dave,minimum size=1.5cm] at (0,0) {}; &  
\tikz \node[graduate,minimum size=1.5cm] at (0,0) {}; &  
\tikz \node[groom,minimum size=1.5cm] at (0,0) {}; & 
\tikz \node[guard,minimum size=1.5cm] at (0,0) {}; 
\\ \hline  
\RDD{conductor} & \RDD{cowboy} & \RDD{criminal} & \RDD{dave} & \RDD{graduate} & \RDD{groom} & \RDD{guard} 
\\ \hline  
\tikz \node[jester,minimum size=1.5cm] at (0,0) {}; &  
%\tikz \node[judge,minimum size=1.5cm] at (0,0) {}; 
&  
\tikz \node[mexican,minimum size=1.5cm] at (0,0) {}; &  
\tikz \node[nun,minimum size=1.5cm] at (0,0) {}; &  
\tikz \node[nurse,minimum size=1.5cm] at (0,0) {}; &  
\tikz \node[physician,minimum size=1.5cm] at (0,0) {}; &  
\tikz \node[pilot,minimum size=1.5cm] at (0,0) {};\\ \hline  
\RDD{jester} &  \RDD{judge} &  \RDD{mexican}  & \RDD{nun} &  \RDD{nurse} & \RDD{physician} 
&  \RDD{pilot}
\\ \hline  

\tikz \node[police,minimum size=1.5cm] at (0,0) {}; &  
\tikz \node[priest,minimum size=1.5cm] at (0,0) {}; &  
\tikz \node[sailor,minimum size=1.5cm] at (0,0) {}; &  
\tikz \node[santa,minimum size=1.5cm] at (0,0) {}; &  
\tikz \node[surgeon,minimum size=1.5cm] at (0,0) {};&  &  \\ 
\hline \RDD{police} & \RDD{priest}  & \RDD{sailor} & \RDD{santa} & \RDD{surgeon} &  &  \\ 
\hline 
\end{tabular} 

\subsubsection{Options}

\noindent

\begin{tabular}{|c|c|c|c|c|}\hline
\multicolumn{5}{|c|}{ \BS{tikz} \BS{node}[businessman,\RDD{evil},minimum size=1.5cm] at (0,0) {};  }
\\ \hline  
\tikz \node[businessman,evil,minimum size=1.5cm] at (0,0) {}; &  
\tikz \node[businessman,female,minimum size=1.5cm] at (0,0) {}; &  
\tikz \node[businessman,good,minimum size=1.5cm] at (0,0) {}; &  
\tikz \node[businessman,mirrored,minimum size=1.5cm] at (0,0) {}; &  
\tikz \node[businessman,monitor,minimum size=1.5cm] at (0,0) {};  
\\  \hline
\RDD{evil} & \RDD{female} & \RDD{good} & \RDD{mirrored} & \RDD{monitor}

\\  \hline 
\end{tabular}

\SbSbSSCT{Point d'ancrage spécifique}{Anchor specific}

\noindent

\begin{tabular}{|c|c|} \hline  
\begin{tikzpicture}[baseline=0pt,blue]
 \node[name=a,shape=bob,minimum
size=1.5cm] {};
 \node at (1.25,.5) [ellipse callout, draw,
callout absolute pointer={(a.mouth)},
font=\tiny] {Hey!};
\end{tikzpicture}
&  
\parbox{12cm}{
\BS{begin}\AC{tikzpicture}[blue] \\
\BS{node}[name=a,shape=bob,minimum size=1.5cm] \AC{};\\
\BS{node} at (1.25,.5) [ellipse callout, draw,
callout absolute pointer\AC{(a.\RDD{mouth})},
font=\BS{tiny}] {Hey!};\\
\BS{end}\AC{tikzpicture} \\
}
\\ \hline 
\end{tabular} 


\SbSbSSCT{Couleurs }{Colors}

\noindent


\begin{tabular}{|c|c|c|c|}\hline
\multicolumn{4}{|c|}{ \BS{tikz} \BS{node}[\blll{alice},\RDD{hair}=red,minimum size=1.5cm] at (0,0) {};  }
\\ \hline    
\tikz \node[alice,hair=red,minimum size=1.5cm] at (0,0) {}; &  
\tikz \node[alice,skin=red,minimum size=1.5cm] at (0,0) {}; &  
\tikz \node[alice,shirt=red,minimum size=1.5cm] at (0,0) {}; &  
\tikz \node[alice,undershirt=red,minimum size=1.5cm] at (0,0) {};  
\\  \hline
\RDD{hair}=red & \RDD{skin}=red & \RDD{shirt}=red & \RDD{details}=red 

\\  \hline 
\end{tabular}

\bigskip
\begin{tabular}{|c|c|c|c|}\hline 
\multicolumn{4}{|c|}{ \BS{tikz} \BS{node}[\blll{bob},\RDD{hair}=red,minimum size=1.5cm] at (0,0) {};  }
\\ \hline   
\tikz \node[bob,hair=red,minimum size=1.5cm] at (0,0) {}; &  
\tikz \node[bob,skin=red,minimum size=1.5cm] at (0,0) {}; &  
\tikz \node[bob,shirt=red,minimum size=1.5cm] at (0,0) {}; &  
\tikz \node[bob,details=red,minimum size=1.5cm] at (0,0) {};  
\\  \hline
\RDD{hair}=red & \RDD{skin}=red & \RDD{shirt}=red & \RDD{details}=red 
\\  \hline 
\end{tabular}

\bigskip
\begin{tabular}{|c|c|c|c|c|}\hline 
\multicolumn{5}{|c|}{ \BS{tikz} \BS{node}[\blll{bride},\RDD{hair}=red,minimum size=1.5cm] at (0,0) {};  }
\\ \hline    
\tikz \node[bride,hair=red,minimum size=1.5cm] at (0,0) {}; &  
\tikz \node[bride,skin=red,minimum size=1.5cm] at (0,0) {}; &  
\tikz \node[bride,shirt=red,minimum size=1.5cm] at (0,0) {}; &  
\tikz \node[bride,pearls=red,minimum size=1.5cm] at (0,0) {}; &
\tikz \node[bride,veil=red,minimum size=1.5cm] at (0,0) {};  
\\  \hline
\RDD{hair}=red & \RDD{skin}=red & \RDD{shirt}=red & \RDD{pearls}=red & \RDD{veil}=red 

\\  \hline 
\end{tabular}

\bigskip
\begin{tabular}{|c|c|c|c|c|}\hline
\multicolumn{5}{|c|}{ \BS{tikz} \BS{node}[\blll{builder},\RDD{hair}=red,minimum size=1.5cm] at (0,0) {};  }
\\ \hline    
\tikz \node[builder,hair=red,minimum size=1.5cm] at (0,0) {}; &  
\tikz \node[builder,skin=red,minimum size=1.5cm] at (0,0) {}; &  
\tikz \node[builder,shirt=red,minimum size=1.5cm] at (0,0) {}; &  
\tikz \node[builder,trousers=red,minimum size=1.5cm] at (0,0) {}; &
\tikz \node[builder,hat=red,minimum size=1.5cm] at (0,0) {};  
\\  \hline
\RDD{hair}=red & \RDD{skin}=red & \RDD{shirt}=red & \RDD{trousers}=red & \RDD{hat}=red   
\\  \hline 
\end{tabular}

\bigskip
\begin{tabular}{|c|c|c|c|c|c|}\hline
\multicolumn{6}{|c|}{ \BS{tikz} \BS{node}[\blll{businessman},\RDD{hair}=red,minimum size=1.5cm] at (0,0) {};  }
\\ \hline    
\tikz \node[businessman,hair=red,minimum size=1.5cm] at (0,0) {}; &  
\tikz \node[businessman,skin=red,minimum size=1.5cm] at (0,0) {}; &  
\tikz \node[businessman,shirt=red,minimum size=1.5cm] at (0,0) {}; &  
\tikz \node[businessman,tie=red,minimum size=1.5cm] at (0,0) {}; &
\tikz \node[businessman,undershirt=red,minimum size=1.5cm] at (0,0) {};  &
\tikz \node[businessman,monogram=red,minimum size=1.5cm] at (0,0) {};
\\  \hline
\RDD{hair}=red & \RDD{skin}=red & \RDD{shirt}=red & \RDD{tie}=red & \RDD{undershirt}=red & \RDD{monogram}=red 
\\  \hline 
\end{tabular}


\bigskip
\begin{tabular}{|c|c|c|c|}\hline
\multicolumn{4}{|c|}{ \BS{tikz} \BS{node}[\blll{charlie},\RDD{hair}=red,minimum size=1.5cm] at (0,0) {};  }
\\ \hline    
\tikz \node[charlie,hair=red,minimum size=1.5cm] at (0,0) {}; &  
\tikz \node[charlie,skin=red,minimum size=1.5cm] at (0,0) {}; &  
\tikz \node[charlie,shirt=red,minimum size=1.5cm] at (0,0) {}; &  
\tikz \node[charlie,buttons=red,minimum size=1.5cm] at (0,0) {}; 

\\  \hline
\RDD{hair}=red & \RDD{skin}=red & \RDD{shirt}=red & \RDD{buttons}=red 
\\  \hline 
\end{tabular}


\bigskip
\begin{tabular}{|c|c|c|c|c|}\hline
\multicolumn{5}{|c|}{ \BS{tikz} \BS{node}[\blll{chef},\RDD{hair}=red,minimum size=1.5cm] at (0,0) {};  }
\\ \hline 
\tikz \node[chef,hair=red,minimum size=1.5cm] at (0,0) {}; &  
\tikz \node[chef,skin=red,minimum size=1.5cm] at (0,0) {}; &  
\tikz \node[chef,shirt=red,minimum size=1.5cm] at (0,0) {}; &  
\tikz \node[chef,hat=red,minimum size=1.5cm] at (0,0) {}; &
\tikz \node[chef,details=red,minimum size=1.5cm] at (0,0) {};  
\\  \hline
\RDD{hair}=red & \RDD{skin}=red & \RDD{shirt}=red & \RDD{hat}=red & \RDD{details}=red 
\\  \hline 
\end{tabular}


\bigskip
\begin{tabular}{|c|c|c|c|c|}\hline
\multicolumn{5}{|c|}{ \BS{tikz} \BS{node}[\blll{conductor},\RDD{hair}=red,minimum size=1.5cm] at (0,0) {};  }
\\ \hline 
\tikz \node[conductor,hair=red,minimum size=1.5cm] at (0,0) {}; &  
\tikz \node[conductor,skin=red,minimum size=1.5cm] at (0,0) {}; &  
\tikz \node[conductor,shirt=red,minimum size=1.5cm] at (0,0) {}; &  \tikz \node[conductor,hat=red,minimum size=1.5cm] at (0,0) {}; &
\tikz \node[conductor,hatshield=red,minimum size=1.5cm] at (0,0) {};  
\\  \hline
\RDD{hair}=red & \RDD{skin}=red & \RDD{shirt}=red & \RDD{hat}=red & \RDD{hatshield}=red  
\\  \hline 
\tikz \node[conductor,undershirt=red,minimum size=1.5cm] at (0,0) {}; &  
\tikz \node[conductor,tie=red,minimum size=1.5cm] at (0,0) {}; &  
\tikz \node[conductor,hatbadge=red,minimum size=1.5cm] at (0,0) {}; &
\tikz \node[conductor,badge=red,minimum size=1.5cm] at (0,0) {}; &
\\  \hline 
\RDD{undershirt}=red &  \RDD{shirt}=red & \RDD{hatbadge}=red & \RDD{badge}=red &
\\  \hline 
\end{tabular}

\bigskip
\begin{tabular}{|c|c|c|c|}\hline
\multicolumn{4}{|c|}{ \BS{tikz} \BS{node}[\blll{cowboy},\RDD{hair}=red,minimum size=1.5cm] at (0,0) {};  }
\\ \hline 
\tikz \node[cowboy,hair=red,minimum size=1.5cm] at (0,0) {}; &  
\tikz \node[cowboy,skin=red,minimum size=1.5cm] at (0,0) {}; &  
\tikz \node[cowboy,shirt=green,minimum size=1.5cm] at (0,0) {}; &  
\tikz \node[cowboy,hat=red,minimum size=1.5cm] at (0,0) {};   
\\  \hline
\RDD{hair}=red & \RDD{skin}=red & \RDD{shirt}=green & \RDD{hat}=red 
\\  \hline 
\tikz \node[cowboy,patches=red,minimum size=1.5cm] at (0,0) {}; &  
\tikz \node[cowboy,tie=green ,minimum size=1.5cm] at (0,0) {}; &  
\tikz \node[cowboy,stitching=red,minimum size=1.5cm] at (0,0) {}; &
\tikz \node[cowboy,vest=red,minimum size=1.5cm] at (0,0) {}; 
\\  \hline 
\RDD{patches}=red &  \RDD{tie}=green & \RDD{stitching}=red & \RDD{vest}=red
\\  \hline 
\end{tabular}


\bigskip
\begin{tabular}{|c|c|c|c|}\hline 
\multicolumn{4}{|c|}{ \BS{tikz} \BS{node}[\blll{criminal},\RDD{hat}=red,minimum size=1.5cm] at (0,0) {};  }
\\ \hline 
\tikz \node[criminal,hat=red,minimum size=1.5cm] at (0,0) {}; &  
\tikz \node[criminal,skin=red,minimum size=1.5cm] at (0,0) {}; &  
\tikz \node[criminal,shirt=red,minimum size=1.5cm] at (0,0) {}; &  
\tikz \node[criminal,details=red,minimum size=1.5cm] at (0,0) {}; 
\\  \hline
\RDD{hat}=red & \RDD{skin}=red & \RDD{shirt}=red & \RDD{details}=red 
\\  \hline 
\end{tabular}

\bigskip
\begin{tabular}{|c|c|c|c|c|}\hline
\multicolumn{5}{|c|}{ \BS{tikz} \BS{node}[\blll{dave},\RDD{hair}=red,minimum size=1.5cm] at (0,0) {};  }
\\ \hline 
\tikz \node[dave,hair=red,minimum size=1.5cm] at (0,0) {}; &  
\tikz \node[dave,skin=red,minimum size=1.5cm] at (0,0) {}; &  
\tikz \node[dave,shirt=red,minimum size=1.5cm] at (0,0) {}; &  
\tikz \node[dave,undershirt=green,minimum size=1.5cm] at (0,0) {}; &
\tikz \node[dave,tie=green,minimum size=1.5cm] at (0,0) {};
\\  \hline
\RDD{hair}=red & \RDD{skin}=red & \RDD{shirt}=red & \RDD{undershirt}=green & \RDD{tie}=green
\\  \hline 
\end{tabular}

\bigskip
\begin{tabular}{|c|c|c|c|c|c|}\hline
\multicolumn{6}{|c|}{ \BS{tikz} \BS{node}[\blll{graduate},\RDD{hair}=red,minimum size=1.5cm] at (0,0) {};  }
\\ \hline 
\tikz \node[graduate,hair=red,minimum size=1.5cm] at (0,0) {}; &  
\tikz \node[graduate,skin=red,minimum size=1.5cm] at (0,0) {}; &  
\tikz \node[graduate,shirt=red,minimum size=1.5cm] at (0,0) {}; &  
\tikz \node[graduate,undershirt=red,minimum size=1.5cm] at (0,0) {}; &
\tikz \node[graduate,stripes=red,minimum size=1.5cm] at (0,0) {};
&
\tikz \node[graduate,hat=red,minimum size=1.5cm] at (0,0) {};
\\  \hline
\RDD{hair}=red & \RDD{skin}=red & \RDD{shirt}=red & \RDD{undershirt}=red & \RDD{stripes}=red & \RDD{hat}=red
\\  \hline 
\end{tabular}

\bigskip
\begin{tabular}{|c|c|c|c|c|c|}\hline
\multicolumn{6}{|c|}{ \BS{tikz} \BS{node}[\blll{groom},\RDD{hair}=red,minimum size=1.5cm] at (0,0) {};  }
\\ \hline
\tikz \node[groom,hair=red,minimum size=1.5cm] at (0,0) {}; &  
\tikz \node[groom,skin=red,minimum size=1.5cm] at (0,0) {}; &  
\tikz \node[groom,shirt=red,minimum size=1.5cm] at (0,0) {}; &  
\tikz \node[groom,undershirt=green,minimum size=1.5cm] at (0,0) {}; &
\tikz \node[groom,tie=green,minimum size=1.5cm] at (0,0) {}; &
\tikz \node[groom,hat=red,minimum size=1.5cm] at (0,0) {};
\\  \hline
\RDD{hair}=red & \RDD{skin}=red & \RDD{shirt}=red & \RDD{undershirt}=green & \RDD{tie}=green & \RDD{hat}=red
\\  \hline 
\end{tabular}


\bigskip
\begin{tabular}{|c|c|c|c|c|c|}\hline 
\multicolumn{6}{|c|}{ \BS{tikz} \BS{node}[\blll{guard},\RDD{hat}=red,minimum size=1.5cm] at (0,0) {};  }
\\ \hline
\tikz\node[guard,hat=red,minimum size=1.5cm] at (0,0) {}; &  
\tikz \node[guard,skin=red,minimum size=1.5cm] at (0,0) {}; &  
\tikz \node[guard,shirt=red,minimum size=1.5cm] at (0,0) {}; &  
\tikz \node[guard,collar=red,minimum size=1.5cm] at (0,0) {}; &
\tikz \node[guard,lining=red,minimum size=1.5cm] at (0,0) {}; &
\tikz \node[guard,details=red,minimum size=1.5cm] at (0,0) {};
\\  \hline
\RDD{hat}=red & \RDD{skin}=red & \RDD{shirt}=red & \RDD{collar}=red & \RDD{lining}=red & \RDD{details}=red
\\  \hline 
\end{tabular}


\bigskip
\begin{tabular}{|c|c|c|c|c|c|}\hline
\multicolumn{6}{|c|}{ \BS{tikz} \BS{node}[\blll{jester},\RDD{hat}=red,minimum size=1.5cm] at (0,0) {};  }
\\ \hline
\tikz \node[jester,hair=red,minimum size=1.5cm] at (0,0) {}; &  
\tikz \node[jester,skin=red,minimum size=1.5cm] at (0,0) {}; &  
\tikz \node[jester,shirt=yellow,minimum size=1.5cm] at (0,0) {}; &  
\tikz \node[jester,hat=red,minimum size=1.5cm] at (0,0) {}; &
%\tikz \node[jester,pattern=yellow,minimum size=1.5cm] at (0,0) {};
&
\tikz \node[jester,details=blue,minimum size=1.5cm] at (0,0) {};
\\  \hline
\RDD{hair}=red & \RDD{skin}=red & \RDD{shirt}=yellow & \RDD{hat}=red & \RDD{pattern}=yellow \footnote{voir confit} & \RDD{details}=blue
\\  \hline 
\end{tabular}

\bigskip
\begin{tabular}{|c|c|c|c|c|}\hline
\multicolumn{5}{|c|}{ \BS{tikz} \BS{node}[\blll{judge},\RDD{hair}=red,minimum size=1.5cm] at (0,0) {};  }
\\ \hline
%\tikz \node[judge,hair=red,minimum size=1.5cm] at (0,0) {}; 
&  
%\tikz \node[judge,skin=red,minimum size=1.5cm] at (0,0) {}; &  
%\tikz \node[judge,shirt=red,minimum size=1.5cm] at (0,0) {}; &  
%\tikz \node[judge,undershirt=red,minimum size=1.5cm] at (0,0) {}; &
%\tikz \node[judge,hairshadow=red,minimum size=1.5cm] at (0,0) {};

\\  \hline
\RDD{hair}=red & \RDD{skin}=red & \RDD{shirt}=red & \RDD{undershirt}=red & \RDD{hairshadow}=red 
\\  \hline 
\end{tabular}

\bigskip
\begin{tabular}{|c|c|c|c|c|c|c|}\hline
\multicolumn{7}{|c|}{ \BS{tikz} \BS{node}[\blll{mexican},\RDD{hair}=red,minimum size=1.5cm] at (0,0) {};  }
\\ \hline
\tikz \node[mexican,hair=red,minimum size=1.5cm] at (0,0) {}; &  
\tikz \node[mexican,skin=red,minimum size=1.5cm] at (0,0) {}; &  
\tikz \node[mexican,shirt=red,minimum size=1.5cm] at (0,0) {}; &  
\tikz \node[mexican,hat=green,minimum size=1.5cm] at (0,0) {}; &
\tikz \node[mexican,ringtop=red,minimum size=1.5cm] at (0,0) {};
&
\tikz \node[mexican,ringmid=red,minimum size=1.5cm] at (0,0) {};
&
\tikz \node[mexican,ringbot=yellow,minimum size=1.5cm] at (0,0) {};
\\  \hline
\RDD{hair}=red & \RDD{skin}=red & \RDD{shirt}=red & \RDD{hat}=green & \RDD{ringtop}=red &\RDD{ringmid}=red & \RDD{ringbot}=yellow
\\  \hline 
\end{tabular}


\bigskip

\begin{tabular}{|c|c|c|}\hline 
\multicolumn{3}{|c|}{ \BS{tikz} \BS{node}[\blll{nun},\RDD{plaid}=red,minimum size=1.5cm] at (0,0) {};  }
\\ \hline
\tikz \node[nun,plaid=red,minimum size=1.5cm] at (0,0) {}; &  
\tikz \node[nun,skin=red,minimum size=1.5cm] at (0,0) {}; &  
\tikz \node[nun,shirt=red,minimum size=1.5cm] at (0,0) {}; 
\\  \hline
\RDD{plaid}=red & \RDD{skin}=red & \RDD{shirt}=red 
\\  \hline 
\end{tabular}


\bigskip
\begin{tabular}{|c|c|c|c|c|c|c|}\hline 
\multicolumn{7}{|c|}{ \BS{tikz} \BS{node}[\blll{nurse},\RDD{hair}=red,minimum size=1.5cm] at (0,0) {};  }
\\ \hline
\tikz \node[nurse,hair=red,minimum size=1.5cm] at (0,0) {}; &  
\tikz \node[nurse,skin=red,minimum size=1.5cm] at (0,0) {}; &  
\tikz \node[nurse,shirt=red,minimum size=1.5cm] at (0,0) {}; &  
\tikz \node[nurse,badgeclip=green,minimum size=1.5cm] at (0,0) {}; &
\tikz \node[nurse,redcross=green,minimum size=1.5cm] at (0,0) {};
&
\tikz \node[nurse,badge=red,minimum size=1.5cm] at (0,0) {};
&
\tikz \node[nurse,badgename=red,minimum size=1.5cm] at (0,0) {};
\\  \hline
\RDD{hair}=red & \RDD{skin}=red & \RDD{shirt}=red & \RDD{badgeclip}=green & \RDD{redcross}=green & \RDD{badge}=red &  \RDD{badgename}=red
\\  \hline 
\end{tabular}


\bigskip
\begin{tabular}{|c|c|c|c|c|c|}\hline
\multicolumn{6}{|c|}{ \BS{tikz} \BS{node}[\blll{physician},\RDD{hair}=red,minimum size=1.5cm] at (0,0) {};  }
\\ \hline
\tikz \node[physician,hair=red,minimum size=1.5cm] at (0,0) {}; &  
\tikz \node[physician,skin=red,minimum size=1.5cm] at (0,0) {}; &  
\tikz \node[physician,shirt=red,minimum size=1.5cm] at (0,0) {}; &  
\tikz \node[physician,hat=red,minimum size=1.5cm] at (0,0) {}; &
\tikz \node[physician,stethoscope=red,minimum size=1.5cm] at (0,0) {};
&
\tikz \node[physician,tube=red,minimum size=1.5cm] at (0,0) {};
\\  \hline
\RDD{hair}=red & \RDD{skin}=red & \RDD{shirt}=red & \RDD{hat}=red & \RDD{stethoscope}=red &   \RDD{tube}=red
\\  \hline 
\end{tabular}


\bigskip
\begin{tabular}{|c|c|c|c|c|c|c|}\hline
\multicolumn{7}{|c|}{ \BS{tikz} \BS{node}[\blll{pilot},\RDD{hat}=red,minimum size=1.5cm] at (0,0) {};  }
\\ \hline
\tikz \node[pilot,hat=red,minimum size=1.5cm] at (0,0) {}; &  
\tikz \node[pilot,skin=red,minimum size=1.5cm] at (0,0) {}; &  
\tikz \node[pilot,shirt=red,minimum size=1.5cm] at (0,0) {}; &  
\tikz \node[pilot,undershirt=red,minimum size=1.5cm] at (0,0) {}; &
\tikz \node[pilot,visor=red,minimum size=1.5cm] at (0,0) {}; &
\tikz \node[pilot,straps=red,minimum size=1.5cm] at (0,0) {}; &
%\tikz \node[pilot,decoration=red,minimum size=1.5cm] at (0,0) {};
\\  \hline
\RDD{hat}=red & \RDD{skin}=red & \RDD{shirt}=red & \RDD{undershirt}=red & \RDD{visor}=red & \RDD{straps}=red & \RDD{decoration}=red
\\  \hline 
\end{tabular}


\bigskip
\begin{tabular}{|c|c|c|c|}\hline
\multicolumn{4}{|c|}{ \BS{tikz} \BS{node}[\blll{police},\RDD{hair}=red,minimum size=1.5cm] at (0,0) {};  }
\\ \hline 
\tikz \node[police,hair=red,minimum size=1.5cm] at (0,0) {}; &  
\tikz \node[police,skin=red,minimum size=1.5cm] at (0,0) {}; &  
\tikz \node[police,shirt=red,minimum size=1.5cm] at (0,0) {}; &  
\tikz \node[police,hat=red,minimum size=1.5cm] at (0,0) {};   
\\  \hline
\RDD{hair}=red & \RDD{skin}=red & \RDD{shirt}=red & \RDD{hat}=red
\\  \hline
\tikz \node[police,badge=red,minimum size=1.5cm] at (0,0) {}; &  
\tikz \node[police,hatbadge=red ,minimum size=1.5cm] at (0,0) {}; &  
\tikz \node[police,hatshield=red,minimum size=1.5cm] at (0,0) {}; &
\tikz \node[police,undershirt=red,minimum size=1.5cm] at (0,0) {}; 
\\  \hline 
\RDD{badge}=red &  \RDD{hatbadge}=red & \RDD{hatshield}=red & \RDD{undershirt}=red
\\  \hline 
\end{tabular}



\bigskip
\begin{tabular}{|c|c|c|c|c|c|}\hline
\multicolumn{6}{|c|}{ \BS{tikz} \BS{node}[\blll{priest},\RDD{hair}=red,minimum size=1.5cm] at (0,0) {};  }
\\ \hline
\tikz \node[priest,hair=red,minimum size=1.5cm] at (0,0) {}; &  
\tikz \node[priest,skin=red,minimum size=1.5cm] at (0,0) {}; &  
\tikz \node[priest,shirt=red,minimum size=1.5cm] at (0,0) {}; &  
\tikz \node[priest,hat=red,minimum size=1.5cm] at (0,0) {}; &
\tikz \node[priest,collar=red,minimum size=1.5cm] at (0,0) {}; &
\tikz \node[priest,cross=red,minimum size=1.5cm] at (0,0) {};
\\  \hline
\RDD{hair}=red & \RDD{skin}=red & \RDD{shirt}=red & \RDD{hat}=red & \RDD{collar}=red &   \RDD{cross}=red
\\  \hline 
\end{tabular}


\bigskip
\begin{tabular}{|c|c|c|c|c|c|c|}\hline 
\multicolumn{7}{|c|}{ \BS{tikz} \BS{node}[\blll{sailor},\RDD{hair}=red,minimum size=1.5cm] at (0,0) {};  }
\\ \hline
\tikz \node[sailor,hair=red,minimum size=1.5cm] at (0,0) {}; &  
\tikz \node[sailor,skin=red,minimum size=1.5cm] at (0,0) {}; &  
\tikz \node[sailor,shirt=red,minimum size=1.5cm] at (0,0) {}; &  
\tikz \node[sailor,hat=red,minimum size=1.5cm] at (0,0) {}; &
\tikz \node[sailor,undershirt=red,minimum size=1.5cm] at (0,0) {};
&
\tikz \node[sailor,stripes=red,minimum size=1.5cm] at (0,0) {};
&
\tikz \node[sailor,details=red,minimum size=1.5cm] at (0,0) {};
\\  \hline
\RDD{hair}=red & \RDD{skin}=red & \RDD{shirt}=red & \RDD{hat}=red & \RDD{undershirt}=red & \RDD{stripes}=red &  \RDD{details}=red
\\  \hline 
\end{tabular}


\bigskip
\begin{tabular}{|c|c|c|c|c|}\hline 
\multicolumn{5}{|c|}{ \BS{tikz} \BS{node}[\blll{santa},\RDD{hat}=green,minimum size=1.5cm] at (0,0) {};  }
\\ \hline
\tikz \node[santa,hat=green,minimum size=1.5cm] at (0,0) {}; &  
\tikz \node[santa,skin=green,minimum size=1.5cm] at (0,0) {}; &  
\tikz \node[santa,shirt=green,minimum size=1.5cm] at (0,0) {}; &  
\tikz \node[santa,beard=green,minimum size=1.5cm] at (0,0) {}; &
\tikz \node[santa,details=green,minimum size=1.5cm] at (0,0) {};

\\  \hline
\RDD{hat}=green & \RDD{skin}=green& \RDD{shirt}=green & \RDD{beard}=green & \RDD{details}=green 
\\  \hline 
\end{tabular}


\bigskip
\begin{tabular}{|c|c|c|c|c|}\hline 
\multicolumn{5}{|c|}{ \BS{tikz} \BS{node}[\blll{surgeon},\RDD{hat}=red,minimum size=1.5cm] at (0,0) {};  }
\\ \hline
\tikz \node[surgeon,hat=red,minimum size=1.5cm] at (0,0) {}; &  
\tikz \node[surgeon,skin=red,minimum size=1.5cm] at (0,0) {}; &  
\tikz \node[surgeon,shirt=red,minimum size=1.5cm] at (0,0) {}; &  
\tikz \node[surgeon,hair=red,minimum size=1.5cm] at (0,0) {}; &
\tikz \node[surgeon,mask=red,minimum size=1.5cm] at (0,0) {};

\\  \hline
\RDD{hat}=red & \RDD{skin}=red & \RDD{shirt}=red & \RDD{hair}=red & \RDD{mask}=red 
\\  \hline 
\end{tabular}


%
%%\newpage
%%
%%\SbSSCT{Canards}{Ducks}
%
%%\label{ducks}

 \maboite{\BS{usepackage}\AC{tikzducks} \cite {tikzducks}}


\begin{center}
\begin{tabular}{|c|}\hline  
\BS{tikz} \BSS{duck} ;
\\ \hline  
\tikz \duck ;
\\ \hline 
\end{tabular} 
\end{center}


\subsubsection{Options}

\noindent

\begin{tabular}{|c|c|c|c|c|} \hline 
\multicolumn{4}{|c|}{\BS{tikz} \BS{duck}[\RDD{body}=red] ;} 
\\ \hline
\tikz \duck[body=red] ;
&  
\tikz \duck[head=red] ;
&
\tikz \duck[bill=red] ;
  &
  \tikz \duck[eye=red] ;
    \\ 
\hline  
[\RDD{body}=red] & [\RDD{head}=red] & [\RDD{bill}=red] & [\RDD{eye}=red] \\ 
\hline 
\end{tabular} 
\begin{tabular}{|c|} \hline
\BS{tikz}  \BS{duck}[\RDD{grumpy}] ;
\\ \hline   
\tikz  \duck[grumpy] ;
\\ \hline  

\end{tabular} 



\bigskip

\begin{tabular}{|c|c|c|c|c|c|} \hline  
\tikz \duck[longhair] ;
&  
\tikz \duck[shorthair] ;
&
\tikz \duck[crazyhair] ;
&
\tikz \duck[recedinghair] ;
&
\tikz \duck[mohican] ;
&
\tikz \duck[mullet] ;
\\ \hline  
[\RDD{longhair}] & [\RDD{shorthair}] & [\RDD{crazyhair}] & [\RDD{recedinghair}] &  [\RDD{mohican}] &  [\RDD{mullet}]\\ 
\hline
\tikz \duck[longhair=red] ;
&  
\tikz \duck[shorthair=red] ;
&
\tikz \duck[crazyhair=red] ;
  &
  \tikz \duck[recedinghair=red] ;
  &
  \tikz \duck[mohican=red] ;
  &
  \tikz \duck[mullet=red] ;
    \\ 
\hline  
[longhair=red] & [shorthair=red] & [crazyhair=red] & [recedinghair=red] &  [mohican=red] &  [mullet=red]  \\ 
\hline 
\end{tabular}

\bigskip





\begin{tabular}{|c|c|c|c|c|} \hline  
\tikz \duck[eyebrow] ;
&  
\tikz \duck[eyebrow=red] ;
&
\tikz \duck[beard] ;
  &
  \tikz \duck[beard=red] ;
    \\ 
\hline  
[\RDD{eyebrow}] & [eyebrow=red] & [\RDD{beard}] & [beard=red] \\ 
\hline
 
\end{tabular}

\bigskip

\begin{tabular}{|c|c|c|c|c|} \hline  
\tikz \duck[tshirt] ;
&  
\tikz \duck[tie] ;
&
\tikz \duck[jacket] ;
&
\tikz \duck[cape] ;
&
\tikz \duck[tshirt,tie ,jacket ,cape] ;
\\ \hline
[\RDD{tshirt}] & [\RDD{tie}] & [\RDD{jacket}] & [\RDD{cape}]& [tshirt,tie ,jacket ,cape]
\\ \hline
\dft{white} & \dft{blue} & \dft{blue} & \dft{red}&
\\ \hline
\tikz \duck[tshirt=red] ;
&  
\tikz \duck[tie=red] ;
&
\tikz \duck[jacket=red] ;
&
\tikz \duck[cape=blue] ;
&

\\ \hline
[tshirt=red] & [tie=red] & [jacket=red] & [cape=blue]& 
\\ \hline
\end{tabular}

\bigskip

\begin{tabular}{|c|c|c|c|c|} \hline 
\tikz \duck[water];
&
\tikz \duck[alien];
&
\tikz \duck[hat];
&
\tikz \duck[tophat];
&
\tikz \duck[cap];
\\ \hline
[\RDD{water}] & [\RDD{alien}] & [\RDD{hat}]& [\RDD{tophat}] & [\RDD{cap}]
\\ \hline
\tikz \duck[santa];
&
\tikz \duck[graduate];
&
\tikz \duck[graduate,tassel];
&
\tikz \duck[beret];
&
\tikz \duck[peakedcap];
\\ \hline
[\RDD{santa}] & [\RDD{graduate}] & [graduate,\RDD{tassel}] & [\RDD{beret}] & [\RDD{peakedcap}]
\\ \hline
\tikz \duck[crown];
&
\tikz \duck[queencrown];
&
\tikz \duck[kingcrown];
&
\tikz \duck[sheep];
&
\tikz \duck[horsetail];
\\ \hline
[\RDD{crown}] &[\RDD{queencrown}]&[\RDD{kingcrown}] & [\RDD{sheep}] &[\RDD{horsetail}]
\\ \hline

\tikz \duck[crozier];
&
\tikz \duck[unicorn];
&

\tikz \duck[bunny];
&
\tikz \duck[bunny=red,inear=blue];
&
\tikz \duck[witch];
\\ \hline
 [\RDD{crozier}] & [\RDD{unicorn}] &[\RDD{bunny}] & [bunny=red,\RDD{inear}=blue] & [\RDD{witch}]
\\ \hline
\tikz \duck[magicwand];
&
\tikz \duck[magichat];
&
\tikz \duck[magichat=teal,
magicstars=blue!30!cyan,
magicwand];
&
\tikz \duck[glasses];
&
\tikz \duck[sunglasses];
\\ \hline
[\RDD{magicwand}] & [\RDD{magichat}] & \parbox{3cm}{[magichat,\\ \RDD{magicstars}]} & [\RDD{glasses}] & [\RDD{sunglasses}]
\\ \hline
\end{tabular}  


\begin{tabular}{|c|c|c|c|c|} \hline

\tikz \duck[squareglasses];
&
\tikz \duck[signpost=42];
&
\tikz \duck[signpost=XXX,signcolour=green];
&
\tikz \duck[signpost=XXX,signback=green];
&
\tikz \duck[speech={XXX}];
\\ \hline
[\RDD{squareglasses}] & [\RDD{signpost}=42] & \parbox{3cm}{[signpost=XXX,\\ \RDD{signcolour}=green]} & \parbox{3cm}{[signpost=XXX, \\ \RDD{signback}=green]} & [\RDD{speech}=\AC{XXX}] 
\\ \hline 
\end{tabular}


\begin{tabular}{|c|c|c|c|c|} \hline 
\tikz \duck[speech={XXX},bubblecolour=green];
&
\tikz \duck[think={XXX}];
&
\tikz \duck[think=XXX,bubblecolour=green];
&
\tikz \duck[book={XXX}];
\\ \hline
\parbox{3cm}{[speech={XXX},\\ \RDD{bubblecolour}=green]} & [\RDD{think}=\AC{XXX}] & 
\parbox{3cm}{[think={XXX},\\ \RDD{bubblecolour}=green]}
&[\RDD{book}=\AC{XXX}] 
\\ \hline
%\end{tabular}  
%
%\begin{tabular}{|c|c|c|c|c|} \hline
\tikz \duck[book=XXX,bookcolour=green];
&
\tikz \duck[book=\scalebox{0.5}{XXX}];
&
\multicolumn{2}{|c|}{\tikz 
\duck[signpost=\scalebox{0.4}{
\parbox{2cm}{
\centering XXX \\ XXXXX}}]
;}
\\ \hline

\parbox{3cm}{[book={XXX},\\ \RDD{bookcolour}=green]} 
&
\parbox{3.5cm}{\BS{tikz} \BS{duck}[book=\\ \BS{scalebox}\AC{0.5}\AC{XXX}]; }
&
\multicolumn{2}{|c|}{ \parbox{7cm}{ \BS{tikz} 
\BS{duck}[signpost=\BS{scalebox}\AC{0.4}\AC{ \\
\BS{parbox}\AC{2cm}{  
\BS{centering} XXX \ XXXXX}}]}
;}
\\ \hline
\end{tabular}


\begin{tabular}{|c|c|c|c|c|} \hline 
\tikz \duck[cricket];
&
\tikz \duck[hockey];
&
\tikz \duck[football];
&
\tikz \duck[lightsaber];
&
\tikz \duck[torch];
\\ \hline
[\RDD{cricket}]& [\RDD{hockey}] & [\RDD{football}] & [\RDD{lightsaber}] & [\RDD{torch}]
\\ \hline
\tikz \duck[prison];
&
\tikz \duck[necklace];
&
\tikz \duck[icecream];
&
\tikz \duck[icecream,flavoura=green];
&
\tikz \duck[icecream,flavourb=green];
\\ \hline
[\RDD{prison}] & [\RDD{necklace}] & [\RDD{icecream}] &
\parbox{3cm}{[icecream,\\ \RDD{flavoura}=green]} & 
\parbox{3cm}{[icecream,\\ \RDD{flavourb}=green]}
\\ \hline
\tikz \duck[icecream,flavourc=green];
&
\tikz \duck[chef];
&
\tikz \duck[rollingpin];
&
\tikz \duck[cake];
&
\tikz \duck[pizza];
\\ \hline
\parbox{3cm}{[icecream,\\ \RDD{flavourc}=green]} &
 [\RDD{chef}] & [\RDD{rollingpin}] & [\RDD{cake}] & [\RDD{pizza}]
\\ \hline
\tikz \duck[baguette];
&
\tikz \duck[milkshake];
&
\tikz \duck[wine];
&
\tikz \duck[mask];
&
\tikz \duck[buttons];
\\ \hline
  [\RDD{baguette}] & [\RDD{milkshake}] & [\RDD{wine}] & [\RDD{mask}] & [\RDD{buttons}]
\\ \hline
\end{tabular}

\begin{tabular}{|c|c|c|c|c|} \hline 
\tikz \duck[basket];
&
\tikz \duck[easter];
&
\tikz \duck[easter,egga=red];
&
\tikz \duck[easter,eggb=red];
&
\tikz \duck[easter,eggc=red];
\\ \hline
[\RDD{basket}]& [\RDD{easter}] & [easter,\RDD{egga}=red] & [easter,\RDD{eggb}=red] & [easter,\RDD{eggc}=red]
\\ \hline

\end{tabular}

\bigskip

\begin{tabular}{|c|c|c|c|} \hline 
\multicolumn{4}{|c|}{\BS{tikz} \BS{duck} \BS{path}[preaction=\AC{fill,green},pattern=dots, pattern  color=red]  \BSS{duckpathbody} ;} 
\\ \hline
\tikz \duck
\path[preaction={fill,green},pattern=dots, pattern  color=red]  \duckpathbody;
&
\tikz   \duck
\path[preaction={fill,green},pattern=dots, pattern  color=red]  \duckpathgrumpybill;
&
\tikz   \duck
\path[preaction={fill,green},pattern=dots, pattern  color=red]  \duckpathbill;
&
\tikz   \duck
\path[preaction={fill,green},pattern=dots, pattern  color=red]  \duckpathtshirt;
\\ \hline 
\BSS{duckpathbody } & \BSS{duckpathgrumpybill} & \BSS{duckpathbill} & \BSS{duckpathtshirt} 
\\ \hline
\tikz   \duck
\path[preaction={fill,green},pattern=dots, pattern  color=red]  \duckpathjacket;
&
\tikz   \duck
\path[preaction={fill,green},pattern=dots, pattern  color=red]  \duckpathcape ;
&
\tikz   \duck
\path[preaction={fill,green},pattern=dots, pattern  color=red]  \duckpathshorthair ;
&
\tikz   \duck
\path[preaction={fill,green},pattern=dots, pattern  color=red]  \duckpathlonghair;
\\ \hline 
\BSS{duckpathjacket} & \BSS{duckpathcape}  & \BSS{duckpathshorthair} & \BSS{duckpathlonghair} 
\\ \hline 
\tikz   \duck
\path[preaction={fill,green},pattern=dots, pattern  color=red]  \duckpathcrazyhair;
&
\tikz   \duck
\path[preaction={fill,green},pattern=dots, pattern  color=red]  \duckpathrecedinghair;
&
\tikz   \duck
\path[preaction={fill,green},pattern=dots, pattern  color=red]  \duckpathcrown ;
&
\tikz   \duck
\path[preaction={fill,green},pattern=dots, pattern  color=red]   \duckpathmohican ;
\\ \hline 
\BSS{duckpathcrazyhair} & \BSS{duckpathrecedinghair} & \BSS{duckpathcrown} & \BSS{duckpathmohican}
\\ \hline 
\tikz   \duck
\path[preaction={fill,green},pattern=dots, pattern  color=red]   \duckpathmullet;
&
\tikz   \duck
\path[preaction={fill,green},pattern=dots, pattern  color=red]   \duckpathqueencrown ;
&
\tikz   \duck
\path[preaction={fill,green},pattern=dots, pattern  color=red]   \duckpathkingcrown ;
&
\tikz   \duck
\path[preaction={fill,green},pattern=dots, pattern  color=red]  \duckpathdarthvader ;
\\ \hline 
\BSS{duckpathmullet} & \BSS{duckpathqueencrown} & \BSS{duckpathkingcrown} & \BSS{duckpathdarthvader}
\\ \hline 
\tikz   \duck
\path[preaction={fill,green},pattern=dots, pattern  color=red]  \duckpathhorsetail ;
& & & 
\\ \hline 
\BSS{duckpathhorsetail}& & & 
\\ \hline 
\end{tabular}


\SbSbSSCT{Canards aléatoires}{Random ducks}

\noindent

\begin{tabular}{|c|}\hline  
\BS{tikz} \BSS{randuck} ; \BS{tikz} \BSS{randuck} ; \BS{tikz} \BSS{randuck} ; \BS{tikz} \BSS{randuck} ; \BS{tikz} \BSS{randuck} ; 
\\ \hline    
\tikz \randuck ; \tikz \randuck  ; \tikz  \randuck  ; \tikz  \randuck  ; \tikz  \randuck ;
\\ \hline 
\end{tabular} 

\bigskip

\begin{tabular}{|c|} \hline  
\BS{tikz} \BSS{shuffleducks} \BS{duck}[\BSS{randomhead}] ;
\\ \hline  
\tikz \shuffleducks \duck[\randomhead] ; \tikz \shuffleducks \duck[\randomhead] ; \tikz \shuffleducks \duck[\randomhead] ; \tikz \shuffleducks \duck[\randomhead] ;
\tikz \shuffleducks \duck[\randomhead] ;
\\ \hline 
\end{tabular} 

\bigskip

\begin{tabular}{|c|} \hline  
\BS{tikz} \BSS{shuffleducks} \BS{duck}[\BSS{randomaccessories}] ;
\\ \hline  
\tikz \shuffleducks \duck[\randomaccessories] ; \tikz \shuffleducks \duck[\randomaccessories] ; \tikz \shuffleducks \duck[\randomaccessories] ; \tikz \shuffleducks \duck[\randomaccessories] ; \tikz \shuffleducks \duck[\randomaccessories] ; 
\\ \hline 
\end{tabular} 


\SbSbSSCT{Coordonnées}{Coordinates}

\noindent


\begin{tabular}{|c|c|c|} \hline 
\multicolumn{3}{|c|}{\BS{tikz} \BS{duck} \BS{fill}[red] (wing) circle (3pt);}
\\ \hline  
\tikz \duck \fill[red] (wing) circle (3pt);
&
\tikz \duck \fill[red] (head) circle (3pt);
&
\tikz \duck \fill[red] (bill) circle (3pt);
\\ \hline wing &  head & bill \\ 
\hline 
\end{tabular} 

\bigskip

\begin{tabular}{|c|} \hline
\BS{tikz} \BS{duck}[\RDD{name}=XXX] \\ \BS{begin}\AC{scope} [xshift=4cm] \BS{duck}[\RDD{name}=YYY] 
\BS{end}\AC{scope} \\ \BS{draw}[red] (XXX-wing) - - (YYY-bill) ;
\\ \hline
\tikz \duck[name=XXX] \begin{scope} [xshift=4cm] 
\duck[name=YYY]
\end{scope}  \draw[red] (XXX-wing) -- (YYY-bill);
\\ \hline
\end{tabular} 


\SbSbSSCT{Rayures}{Stripes}

\noindent

\begin{tabular}{|c|c|} \hline  
\tikz \duck \stripes ;
&  
\tikz \duck[stripes] ;
\\ \hline  
\BS{tikz} \BS{duck} \BSS{stripes} ; 
&  
\BS{tikz} \BS{}duck[\RDD{stripes}] ;
\\ \hline 
\end{tabular} 

\bigskip

\begin{tabular}{|c|c|} \hline  
\tikz \duck[rollingpin] \stripes ;
&  
\tikz  \duck[rollingpin,stripes] ;
\\ \hline  
\BS{tikz} \BS{duck}[rollingpin] \BS{stripes} ;
&  
\BS{tikz}  \BS{duck}[rollingpin,stripes] ;
\\ \hline 
\end{tabular} 



\bigskip

\begin{tabular}{|c|c|c|c|}\hline 
\multicolumn{4}{|c|}{\BS{tikz} \BS[duck] \BS{stripes}[\RDD{color}=red];}
\\ \hline  
\tikz \duck \stripes[color=red];
& 
\tikz \duck \stripes[distance=.5]; 
&  
\tikz \duck \stripes[width=.05];
&
\tikz \duck \stripes[height=1];  
\\ \hline 
[\RDD{color}=red] & [\RDD{distance}=.5] & [\RDD{width}=.05] & [\RDD{height}=1] 
\\ \hline 
\dft{black} & \dft{0.3}  & \dft{0.15}  &  \dft{2.7} \\ 
\hline  

\tikz \duck \stripes[rotate=45]; & \tikz \duck \stripes[initialx=1]; & \tikz \duck \stripes[initialy=1]; &  
\\ \hline 
[\RDD{rotate=}45] & [\RDD{initialx}=1] & [\RDD{initialy}=1] &
\\ \hline 
\dft{-10} & \dft{0.1} & \dft{-0.3} & 
\\ \hline 
\end{tabular} 


\bigskip

\begin{tabular}{|c|c|c|} \hline
\multicolumn{3}{|c|}{\BS{tikz} \BS[duck] \BS{stripes}[\RDD{emblem}=XXX];}
\\ \hline 
\tikz \duck \stripes[emblem={XXX}];
&  
\tikz \duck \stripes[emblem={\includegraphics[width=6mm]{LogoIUT}}];
&  
\tikz \duck \stripes[emblem={\DFR}];
\\ \hline  
[emblem=XXX]
& \parbox{5cm}{ [emblem=\{\BS{includegraphics}  [width=6mm]\AC{LogoIUT} \} ] }  
&  [emblem=\AC{\BS{DFR}}  ] 
\\ \hline 
& &  \BS{DFR} : \TFRGB{voir}{see} page \pageref{DFR} 
\\ \hline   
\end{tabular} 

\bigskip

\begin{tabular}{|c|c|c|} \hline
\tikz \duck[stripes={ \stripes \stripes[rotate=45]}] ;
\\ \hline 
\BS{tikz}
\BS{duck}[stripes=\AC{
\BS{stripes}
\BS{stripes}[rotate=45] } ]
;

\\ \hline  
\end{tabular}



%%
%%\newpage
%
%
%\newpage
%
%\section{A Voir}
%
%
%%\RRR{75-2 = Concept: Data Points and Data Formats}

\begin{tikzpicture}
\datavisualization [school book axes, visualize as smooth line]
data {
x, y
-1.5, 2.25
-1, 1
-.5, .25
0, 0
.5, .25
1, 1
1.5, 2.25
};
\end{tikzpicture}


\begin{tikzpicture}
\datavisualization [school book axes, visualize as smooth line]
data [format=function] {
var x : interval [-1.5:1.5] samples 7;
func y = \value x*\value x;
};
\end{tikzpicture}


\begin{tikzpicture}
\datavisualization [school book axes, visualize as smooth line]
data [format=function] {
var x : interval [-1.5:1.5] samples 3;
func y = \value x*\value x;
};
\end{tikzpicture}

Section 76 gives an in-depth coverage of the available data formats and explains how new data formats
can be defined.


\RRR{75-3 = Concept: Axes, Ticks, and Grids}


\begin{tikzpicture}
\datavisualization [
scientific axes,
x axis={length=3cm, ticks=few},
visualize as smooth line
]
data [format=function] {
var x : interval [-1.5:1.5] samples 20;
func y = \value x*\value x;
};
\end{tikzpicture}

\begin{tikzpicture}
\datavisualization [
scientific axes=clean,
x axis={length=3cm, ticks=few},
all axes={grid},
visualize as smooth line
]
data [format=function] {
var x : interval [-1.5:1.5] samples 7;
func y = \value x*\value x;
};
\end{tikzpicture}

Section 77 explains in more detail how axes, ticks, and grid lines can be chosen and configured.


\RRR {75-4 = Concept: Visualizers}

\begin{tikzpicture}
\datavisualization [
scientific axes=clean,
x axis={length=3cm, ticks=few},
visualize as smooth line
]
data [format=function] {
var x : interval [-1.5:1.5] samples 7;
func y = \value x*\value x;
};
\end{tikzpicture}

\begin{tikzpicture}
\datavisualization [
scientific axes=clean,
x axis={length=3cm, ticks=few},
visualize as scatter
]
data [format=function] {
var x : interval [-1.5:1.5] samples 7;
func y = \value x*\value x;
};
\end{tikzpicture}

Section 78 provides more information on visualizers as well as reference lists.

\RRR{75-5 = Concept: Style Sheets and Legends }

\begin{tikzpicture}[baseline]
\datavisualization [ scientific axes=clean,
y axis=grid,
visualize as smooth line/.list={sin,cos,tan},
style sheet=strong colors,
style sheet=vary dashing,
sin={label in legend={text=$\sin x$}},
cos={label in legend={text=$\cos x$}},
tan={label in legend={text=$\tan x$}},
data/format=function ]
data [set=sin] {
var x : interval [-0.5*pi:4];
func y = sin(\value x r);
}
data [set=cos] {
var x : interval [-0.5*pi:4];
func y = cos(\value x r);
}
data [set=tan] {
var x : interval [-0.3*pi:.3*pi];
func y = tan(\value x r);
};
\end{tikzpicture}


Section 79 details style sheets and legends.

\RRR{75-6 = Usage}

\subsection{/pgf/data/read from file=filename} (no default, initially empty)

If you set the source attribute to a non-empty hfilenamei, the data will be read from this file. In
this case, no hinline datai may be present, not even empty curly braces should be provided.
%\datavisualization ...
data [read from file=file1.csv]
data [read from file=file2.csv];
The other way round, if read from file is empty, the data must directly follow as hinline datai.
%\datavisualization ...
data {
x, y
1, 2
2, 3
};

The second important key is format, which is used to specify the data format:

\subsection{/pgf/data/format}

Use this key to locally set the format used for parsing the data, see Section 76 for a list of predefined
formats.

\tikz
\datavisualization [school book axes, visualize as line]
data [/data point/x=1] {
y
1
2
}
data [/data point/x=2] {
y
2
0
.5
};

\BS{datavisualization} . . . data point[options] . . . ;

\tikz \datavisualization [school book axes, visualize as line]
data point [x=1, y=1] data point [x=1, y=2]
data point [x=2, y=2] data point [x=2, y=0.5];

/tikz/data visualization/data point=options

\tikzdatavisualizationset{
horizontal/.style={
data point={x=#1, y=1}, data point={x=#1, y=2}},
}
\tikz \datavisualization
[ school book axes, visualize as line,
horizontal=1,
horizontal=2 ];

\BS{datavisualization} . . . data group[options]\AC{name}+=\AC{data specifications} . . . ;


\tikz \datavisualization data group {points} = {
data {
x, y
0, 1
1, 2
2, 2
5, 1
2, 0
1, 1
}
};

\tikz \datavisualization [school book axes, visualize as line] data group {points};
\qquad
\tikz \datavisualization [scientific axes=clean, visualize as line] data group {points};


\BS{datavisualization} . . . scope[options]{data specification} . . . ;

%\datavisualization...
%scope [/data point/experiment=7]
%{
%data [read from file=experiment007-part1.csv]
%data [read from file=experiment007-part2.csv]
%data [read from file=experiment007-part3.csv]
%}
%scope [/data point/experiment=23, format=foo]
%{
%data [read from file=experiment023-part1.foo]
%data [read from file=experiment023-part2.foo]
%};


\BS{datavisualization} . . . info[options]{code} . . . ;

\begin{tikzpicture}[baseline]
\datavisualization [ school book axes, visualize as line ]
data [format=function] {
var x : interval [-0.1*pi:2];
func y = sin(\value x r);
}
info {
\draw [red] (visualization cs: x={(.5*pi)}, y=1) circle [radius=1pt]
node [above,font=\footnotesize] {extremal point};
};
\end{tikzpicture}

\subsection{Coordinate system visualization}

\BS{datavisualization} . . . info’[options]{code} . . . ;

\begin{tikzpicture}[baseline]
\datavisualization [ school book axes, visualize as line ]
data [format=function] {
var x : interval [-0.1*pi:2];
func y = sin(\value x r);
}
info' {
\fill [red] (visualization cs: x={(.5*pi)}, y=1) circle [radius=2mm];
};
\end{tikzpicture}


\subsection{Predefined node data visualization bounding box}
This rectangle node stores a bounding box of the data visualization that is currently being constructed.
This node can be useful inside info commands or when labels need to be added.

\subsection{Predefined node data bounding box}
This rectangle node is similar to data visualization bounding box, but it keeps track only of the actual
data. The spaces needed for grid lines, ticks, axis labels, tick labels, and other all other information
that is not part of the actual data is not part of this box.


\RRR{75-7 = Advanced: Executing User Code During a Data Visualization}

\RRR{75-8 = Advanced: Creating New Objects}


\section{76 Providing Data for a Data Visualization}

%
%\begin{tikzpicture}
%\draw[help lines] (0,0) grid (2,2);
%\node[draw,fill=green!20,] (A) at (1,1) {\huge noeud};
%\fill[red] (A.-30) circle (3pt);
%\end{tikzpicture}
%
%
%
%\begin{tikzpicture}
%\draw[help lines] (0,0) grid (3,2);
%\draw (0,0) -- (1,1);
%\draw[red] (0,0) -- ([xshift=3pt] 1,1);
%\draw (1,0) -- +(30:2cm);
%\draw[red] (1,0) -- +([shift=(135:5pt)] 30:2cm);
%\end{tikzpicture}
%
%
%
%\section{Problèmes a Voir}
%
%%\tikz \node[jester,pattern=yellow,minimum size=1.5cm] at (0,0) {};
%
%\newpage
%
%
%
\begin{tabular}{|c|c|l c|}\hline 
\multicolumn{4}{|c|}{ \textbf{\TFRGB{module de base TikZ}{Basic TikZ package} : } }
\\ \hline

\TFRGB{nom}{name} & \TFRGB{A insérer dans le préambule}{Load package}& documentation \footnotemark[1] 	& \\  \hline 
tikz & \BS{usepackage}\AC{tikz}  	& pgfmanual.pdf			& \DGB \\

\hline 
\end{tabular} 

\bigskip

\begin{tabular}{|c|c|l c|}\hline 
\multicolumn{4}{|c|}{ \textbf{\TFRGB{Autres modules}{Other packages}} }
\\ \hline
\TFRGB{nom}{name} & \TFRGB{voir page}{see page} & documentation  \footnotemark[2] 	& \\  \hline 
animate 	& \pageref{anim} 	& animate.pdf 			& \DGB \\
tikz-optics 	& \pageref{optics} 	& tikz-optics.pdf 			& \DFR \\
pgfplots 	& \pageref{pgfplots} & pgfplots.pdf 		& \DGB \\
tikzpeople  & \pageref{people} 	& tikzpeople.pdf 		& \DGB \\
tikzducks  & \pageref{ducks} 	& tikzducks-doc.pdf 		& \DGB \\
tikzsymbols  & \pageref{symbol} 	& tikzsymbols.pdf 		& \DGB \\
tkz-tab  	& \pageref{tabl} 	& tkz-tab-screen.pdf 	& \DFR \\
\hline 
\end{tabular} 
\bigskip



\begin{tabular}{|l|c|l|}\hline 
\multicolumn{3}{|c|}{ \textbf{\TFRGB{Compléments optionnels}{Optional library} (documentation : pgfmanual.pdf)} }
\\ \hline
\TFRGB{nom}{name} 				& \TFRGB{voir page}{see page}						& \TFRGB{A insérer dans le préambule}{Load package}\\ \hline 
angles & \pageref{lib-angles} &  \BS{usetikzlibrary}\AC{angles}\\
arrows.meta	& \pageref{lib-arrows.meta}	&  \BS{usetikzlibrary}\AC{arrows.meta}\\
bending				& \pageref{lib-bending}			&  \BS{usetikzlibrary}\AC{bending}
\\
backgrounds			& \pageref{lib-bkgd} 			&  \BS{usetikzlibrary}\AC{backgrounds}
\\
calc				& \pageref{lib-calc}			&  \BS{usetikzlibrary}\AC{calc}
\\
chains			& \pageref{lib-chains} 			& \BS{usetikzlibrary}\AC{chains} 
\\
circuits.ee.IEC				& \pageref{lib-ee}			&  \BS{usetikzlibrary}\AC{circuits.ee.IEC}
\\
circuits.logic.IEC	& \pageref{lib-gate}			&  \BS{usetikzlibrary}\AC{circuits.logic.IEC}
\\ 
circuits.logic.US	& \pageref{lib-gate}			&  \BS{usetikzlibrary}\AC{circuits.logic.US}
\\ 
circuits.logic.CDH	& \pageref{lib-gate}			&  \BS{usetikzlibrary}\AC{circuits.logic.CDH}
\\ 
fit & \pageref{lib-fit} 	& \BS{usetikzlibrary}\AC{fit} 
\\
decorations.footprints & \pageref{lib-footprints} 	& \BS{usetikzlibrary}\AC{decorations.footprints} 
\\
decorations.fractals & \pageref{lib-fractals} 		& \BS{usetikzlibrary}\AC{decorations.fractals} 
\\
decorations.markings & \pageref{lib-mark} 			& \BS{usetikzlibrary}\AC{decorations.markings} 
\\
decorations.pathmorphing  & \pageref{lib-morph}		& \BS{usetikzlibrary}\AC{decorations.pathmorphing}
\\
decorations.pathreplacing & \pageref{lib-replac}	& \BS{usetikzlibrary}\AC{decorations.pathreplacing} 
\\
decorations.shapes & \pageref{lib-shapes} 			& \BS{usetikzlibrary}\AC{decorations.shapes} 
\\
decorations.text & \pageref{lib-text} 				& \BS{usetikzlibrary}\AC{decorations.text} 
\\
fadings 			& \pageref{lib-fadings}			&  \BS{usetikzlibrary}\AC{fadings }
\\
intersections		& \pageref{lib-intersections}	&  \BS{usetikzlibrary}\AC{intersections}
\\
matrix			& \pageref{lib-matrix} 			& \BS{usetikzlibrary}\AC{matrix} 
\\
patterns			& \pageref{lib-patterns}		&  \BS{usetikzlibrary}\AC{patterns}
\\
plotmarks			& \pageref{plotmarks} 			&  \BS{usetikzlibrary}\AC{plotmarks}
\\
positioning			& \pageref{lib-pos} 			&  \BS{usetikzlibrary}\AC{positioning}
\\ 
scopes				& \pageref{lib-scopes}			&  \BS{usetikzlibrary}\AC{scopes}
\\
shadings			& \pageref{lib-shadings}		&  \BS{usetikzlibrary}\AC{shadings}
\\
shapes.arrows		& \pageref{lib-arr}				&\BS{usetikzlibrary}\AC{shapes.arrows} 
\\shapes.callouts		& \pageref{lib-call}			& \BS{usetikzlibrary}\AC{shapes.callouts} 
\\
shapes.geometric	& \pageref{lib-geom} 			& \BS{usetikzlibrary}\AC{shapes.geometric}
\\

shapes.misc			& \pageref{lib-misc} 			& \BS{usetikzlibrary}\AC{shapes.misc} 
\\
shapes.multipart	& \pageref{lib-mult} 			& \BS{usetikzlibrary}\AC{shapes.multipart} 
\\
shapes.symbols		& \pageref{lib-symb}			& \BS{usetikzlibrary}\AC{shapes.symbols} 
\\
through				& \pageref{lib-through}			&  \BS{usetikzlibrary}\AC{through}
\\ 
trees				& \pageref{lib-trees}
\BS{usetikzlibrary}\AC{trees}
\\ 
through				& \pageref{lib-turtle}			&  \BS{usetikzlibrary}\AC{turtle}
\\ 
\hline
 \end{tabular} 

\TFRGB{ 
\footnotetext[1]{voir dans le répertoire :  \BS{texlive}\BS{2016}\BS{tesmf-dist}\BS{doc}\BS{generic}\BS{pgf}}
\footnotetext[2]{chercher  dans le répertoire  :  \BS{texlive}\BS{2016}\BS{tesmf-dist}\BS{doc}\BS{latex}} }
{ 
\footnotetext[1]{look in repertory :  \BS{texlive}\BS{2016}\BS{tesmf-dist}\BS{doc}\BS{generic}\BS{pgf}}
\footnotetext[2]{search in repertory :  \BS{texlive}\BS{2016}\BS{tesmf-dist}\BS{doc}\BS{latex}} }

\bigskip



\begin{tabular}{|l|c|}\hline
\multicolumn{2}{|c|}{ \TFRGB{dans une prochaine mise à jour}{In a a future update } }
\\ \hline
automata			& \RRR{41} \\
babel				& \RRR{42} \\
calendar			& \RRR{45} \\
%chains				& \RRR{46} \\ 
%circuits.ee		& \RRR{47-4} \\ 
 
circular graph drawing library 				& \RRR{32} \\
curvilinear library 						& \RRR{103-4-7} \\
datavisualization library					& \RRR{75} \\
datavisualization.formats.functions library & \RRR{76-4} \\
datavisualization.polar library 			& \RRR{80}  \\
 er 										& \RRR{49}  \\
examples graph drawing library 				& \RRR{35-8} \\ 
external 									& \RRR{50}  \\  
%fit 										& \RRR{52} \\ 
fixedpointarithmetic 						& \RRR{53} \\ 
folding 									& \RRR{59} \\
force graph drawing library 				& \RRR{31}  \\
fpu											& \RRR{54}  \\
graph.standard library 						& \RRR{19-10}\\
graphdrawing library 						& \RRR{27} \\
graphs library 								& \RRR{19} \\ 
layered graph drawing library 				& \RRR{30}  \\
lindenmayersystems							& \RRR{55} \\  
mindmap										& \RRR{58} \\ 
petri										& \RRR{61}  \\ 
phylogenetics graph drawing library 		& \RRR{33} \\
plothandlers								& \RRR{62}  \\  
profiler									& \RRR{64}   \\ 
quotes library 								& \RRR{17-10-4} \\
routing graph drawing library 				& \RRR{34} \\
shadows										& \RRR{66}   \\ 
 
spy											&  \RRR{68} \\ 
svg.path									&  \RRR{69} \\ 
%through										&  \RRR{71} \\ 
topaths										&  \RRR{70} \\ 
trees graph drawing library					& 
\\ \hline
\end{tabular}  


%\newpage
%%
%% \tableofcontents
%\renewcommand{\bibname}{Sources}
%
\label{sources}
%\input{bib}

\newpage

\begin{thebibliography}{99}
\bibitem{pgfmanual} pgfmanual.pdf  	\hspace{1cm}	version 3.0.1a \hspace{1cm} 	1161 pages 	\hspace{1cm}	\DGB
\bibitem{pgfplots} pgfplots.pdf 	\hspace{1cm}	version 1.80 \hspace{1cm} 	439 pages 	\hspace{1cm}	\DGB
\bibitem{tikstab} tkz-tab-screen.pdf 	\hspace{1cm}	version 1.1c \hspace{1cm} 	83 pages 	\hspace{1cm}	\DFR
\bibitem{tikzpeople} tikzpeople.pdf 	\hspace{1cm}	 \hspace{1cm} 	19 pages 	\hspace{1cm}	\DGB

\bibitem{tikzducks} tikzducks-doc.pdf 	\hspace{1cm}	version 0.6  \hspace{1cm} 	28 pages 	\hspace{1cm}	\DGB

\bibitem{tikzsymbols} tikzsymbols.pdf 	\hspace{1cm}	version sept 2017  \hspace{1cm} 	15 pages 	\hspace{1cm}	\DGB

\bibitem{animate} animate.pdf 	\hspace{1cm}	 \hspace{1cm} 	26 pages 	\hspace{1cm}	\DGB

\bibitem{optics} tikz-optics.pdf	\hspace{1cm}	version 0.2.2  \hspace{1cm} 	39 pages 	\hspace{1cm}	\DFR
\end{thebibliography}


%
%\newpage 
%
%%\printindex 
  
\end{document}:'
\end{verbatim}
bi dobili barvno verzijo dokumenta.

\subsection{Lastni paketi}

Če definiramo veliko novih okolij in ukazov, potem bo preambula dokumenta 
postala kar dolga. V takem primeru je dobro narediti  nov 
\LaTeX{}ov paket, ki vsebuje vse definicije novih okolij in ukazov. 
V dokumentu potem uporabimo ukaz \ci{usepackage}, da naložimo paket in
s tem v dokumentu omogočimo nove ukaze in okolja.
\begin{figure}[!htbp]
\begin{lined}{\textwidth}
\begin{verbatim}
% Demo Package by Tobias Oetiker and Bor Plestenjak
\ProvidesPackage{demopack}
\newcommand{\xvec}[1] {x_1,\ldots,x_n}
\newcommand{\lvec}[1] {#1_1,\ldots,#1_n}
\newenvironment{kralj}
  {\rule{1ex}{1ex}   \hspace{\stretch{1}}}
  {\hspace{\stretch{1}}  \rule{1ex}{1ex}}
\end{verbatim}
\end{lined}
\caption{Zgled paketa.} \label{package}
\end{figure}

Pisanje paketa v glavnem pomeni kopiranje vsebine preambule v ločeno datoteko 
s končnico \texttt{.sty}. Poleg tega je še poseben ukaz 
\begin{lscommand}
\ci{ProvidesPackage}\verb|{|\emph{package name}\verb|}|
\end{lscommand}
\noindent ki ga je potrebno uporabiti povsem na začetku datoteke s paketom. 
Ukaz \verb|\ProvidesPackage| sporoči \LaTeX{}u ime paketa in javi resno 
napako v primeru, ko želimo paket naložiti dvakrat. Slika~\ref{package}
prikazuje majhen paket z ukazi, ki smo jih definirali v prejšnjih zgledih.


\section{Pisave in velikosti črk}

\subsection{Ukazi za spreminjanje pisave}
\index{pisava}\index{velikost pisave} \LaTeX{} izbere pisavo in velikost črk glede na 
logično strukturo dokumenta (razdelki, opombe, \ldots). V določenih primerih pa 
bi radi ročno spremenili pisavo in velikosti črk. To lahko naredimo z ukazi, 
navedenimi v tabelah~\ref{fonts} in~\ref{sizes}. Dejanska velikost vsake pisave
je oblikovalski problem in je odvisna od razreda dokumenta in uporabljenih opcij.
Tabela~\ref{tab:pointsizes} prikazuje absolutno velikost črk pri uporabljenih navedenih ukazih 
v standardnih razredih dokumentov.

\begin{example}
{\small The small and 
\textbf{bold} Romans ruled}
{\Large all of great big 
\textit{Italy}.}
\end{example}

Pomembna lastnost \LaTeXe{} je, da so vsi atributi pisave neodvisni. To pomen, da lahko 
vključimo ukaze za spreminjanje velikosti ali tipa pisave, pa se bodo še vedno ohranili
atributi za krepko ali poševno pisavo, ki smo jih vključili prej.

V \emph{matematičnem načinu} lahko uporabljamo ukaze za spreminjanje pisave tako, da 
gremo začasno ven iz \emph{matematičnega načina} in vnesemo normalen tekst. Če želimo pri
stavljenje formul uporabljati drugačno pisavo, potem za to obstajajo posebno ukazi. Našteti so
v tabeli~\ref{mathfonts}.

\begin{table}[!bp]
\caption{Pisave.} \label{fonts}
\begin{lined}{12cm}
%
% Alan suggested not to tell about the other form of the command
% eg \verb|\sffamily| or \verb|\bfseries|. This seems a good thing to me.
%
\begin{tabular}{@{}rl@{\qquad}rl@{}}
\ci{textrm}\verb|{...}|        &      \textrm{\wi{pokončna pisava}}&
\ci{textsf}\verb|{...}|        &      \textsf{\wi{gladka pisava}}\\
\ci{texttt}\verb|{...}|        &      \texttt{pisalni stroj}\\[6pt]
\ci{textmd}\verb|{...}|        &      \textmd{srednja pisava}&
\ci{textbf}\verb|{...}|        &      \textbf{\wi{krepka pisava}}\\[6pt]
\ci{textup}\verb|{...}|        &       \textup{\wi{pokončna pisava}}&
\ci{textit}\verb|{...}|        &       \textit{\wi{kurzivna pisava}}\\
\ci{textsl}\verb|{...}|        &       \textsl{\wi{nagnjena pisava}}&
\ci{textsc}\verb|{...}|        &       \textsc{\wi{velike male črke}}\\[6pt]
\ci{emph}\verb|{...}|          &            \emph{poudarjena pisava} &
\ci{textnormal}\verb|{...}|    &    \textnormal{običajna pisava} font
\end{tabular}

\bigskip
\end{lined}
\end{table}


\begin{table}[!bp]
\index{velikost črk}
\caption{Velikosti črk.} \label{sizes}
\begin{lined}{12cm}
\begin{tabular}{@{}ll}
\ci{tiny}      & \tiny        drobna pisava \\
\ci{scriptsize}   & \scriptsize  velikost indeksov\\
\ci{footnotesize} & \footnotesize  velikost opomb pod črto \\
\ci{small}        &  \small            majhna pisava \\
\ci{normalsize}   &  \normalsize  normalna velikost\\
\ci{large}        &  \large       veliki znaki
\end{tabular}%
\qquad\begin{tabular}{ll@{}}
\ci{Large}        &  \Large       veliki znaki \\[5pt]
\ci{LARGE}        &  \LARGE       zelo veliki znaki \\[5pt]
\ci{huge}         &  \huge        ogromni znaki \\[5pt]
\ci{Huge}         &  \Huge        največji znaki
\end{tabular}

\bigskip
\end{lined}
\end{table}

\begin{table}[!tbp]
\caption{Absolutna velikost pisave v standardnih razredih.}\label{tab:pointsizes}
\label{tab:sizes}
\begin{lined}{12cm}
\begin{tabular}{lrrr}
\multicolumn{1}{c}{size} &
\multicolumn{1}{c}{10pt (privzeto) } &
           \multicolumn{1}{c}{11pt opcija}  &
           \multicolumn{1}{c}{12pt opcija}\\
\verb|\tiny|       & 5pt  & 6pt & 6pt\\
\verb|\scriptsize| & 7pt  & 8pt & 8pt\\
\verb|\footnotesize| & 8pt & 9pt & 10pt \\
\verb|\small|        & 9pt & 10pt & 11pt \\
\verb|\normalsize| & 10pt & 11pt & 12pt \\
\verb|\large|      & 12pt & 12pt & 14pt \\
\verb|\Large|      & 14pt & 14pt & 17pt \\
\verb|\LARGE|      & 17pt & 17pt & 20pt\\
\verb|\huge|       & 20pt & 20pt & 25pt\\
\verb|\Huge|       & 25pt & 25pt & 25pt\\
\end{tabular}

\bigskip
\end{lined}
\end{table}


\begin{table}[!bp]
\caption{Matematične pisave.} \label{mathfonts}
\begin{lined}{\textwidth}
\begin{tabular}{@{}lll@{}}
\textit{Ukaz}&\textit{Zgled}&    \textit{Rezultat}\\[6pt]
\ci{mathcal}\verb|{...}|&    \verb|$\mathcal{B}=c$|&     $\mathcal{B}=c$\\
\ci{mathrm}\verb|{...}|&     \verb|$\mathrm{K}_2$|&      $\mathrm{K}_2$\\
\ci{mathbf}\verb|{...}|&     \verb|$\sum x=\mathbf{v}$|& $\sum x=\mathbf{v}$\\
\ci{mathsf}\verb|{...}|&     \verb|$\mathsf{G\times R}$|&        $\mathsf{G\times R}$\\
\ci{mathtt}\verb|{...}|&     \verb|$\mathtt{L}(b,c)$|&   $\mathtt{L}(b,c)$\\
\ci{mathnormal}\verb|{...}|& \verb|$\mathnormal{R_{19}}\neq R_{19}$|&
$\mathnormal{R_{19}}\neq R_{19}$\\
\ci{mathit}\verb|{...}|&     \verb|$\mathit{ffi}\neq ffi$|& $\mathit{ffi}\neq ffi$
\end{tabular}

\bigskip
\end{lined}
\end{table}

V povezavi z ukazi za spreminjanje velikosti črk imajo velik pomen
\wi{zaviti oklepaji}. Z njimi gradimo \emph{skupine}.  Skupine
omejujejo območje delovanja \LaTeX{}ovega ukaza.\index{združevanje}

\begin{example}
He likes {\LARGE large and 
{\small small} letters}. 
\end{example}
 
Ukazi za spreminjanje velikosti pisave spremenijo tudi razmike med vrsticami,
toda le, če se odstavek konča znotraj območja delovanja ukaza za velikost pisave. 
Desni zaviti oklepaj \verb|}| zato ne sme nastopati prezgodaj v tekstu. Primerjajte položaj ukaza
\ci{par}{} v naslednjih dveh primerih. \footnote{\texttt{\bs{}par}
je ekvivalentno prazni vrstici}


\begin{example}
{\Large Don't read this! It is not
true. You can believe me!\par}
\end{example}

\begin{example}
{\Large This is not true either.
But remember I am a liar.}\par
\end{example}

Če želimo, da ukaz za spremenjeno velikost pisave deluje za celotni odstavek 
ali celo za večji kos teksta, potem je priporočljivo uporabljati okolja 
za spreminjanje velikosti pisave.
\begin{example}
\begin{Large} 
This is not true.
But then again, what is these
days \ldots
\end{Large}
\end{example}
\noindent To nas lahko reši pred štetjem velikega števila zavitih oklepajev.


\subsection{Nevarnost na vidiku}

Kot smo omenili že na začetku tega poglavja, je nevarno 
razmetati eksplicitne ukaze za spreminjanje oblike pisave vsepovsod po tekstu,
saj je to v nasprotju z osnovno idejo \LaTeX{}a, ki pravi, da je potrebno ločiti oznake za logični in vizualni 
del dokumenta. To pomeni, da če uporabljamo isti ukaz za spreminjanje pisave na več mestih z namenom, 
da poudarimo določen podatek, potem je bolje za to definirati nov ukaz preko
\verb|\newcommand| in tako logično povezati vrsto podatka, ki ga poudarjamo 
s spreminjanjem pisave.

\begin{example}
\newcommand{\oops}[1]{\textbf{#1}}
Do not \oops{enter} this room,
it's occupied by a \oops{machine}
of unknown origin and purpose.
\end{example}

Ta pristop ima to prednost, da če se kdaj kasneje odločimo, da bomo za nevarnost uporabili 
drugačno pisavo kot pa \verb|\textbf|, potem to spremenimo na enem mestu. Sicer bi morali 
v celotnem dokumentu poiskati vse pojavitve ukaza \verb|\textbf|, potem pa bi se morali pri vsakem posamezno 
še odločiti, ali je ukaz \verb|\textbf| uporabljen zaradi nevarnosti in moramo pisavo zato zamenjati ali pa 
ukaz \verb|\textbf| pomeni kaj drugega in ga pustimo pri miru.

\subsection{Nasvet}

Za konec našega izleta v deželo pisav in velikosti črk še kratek nasvet:\nopagebreak

\begin{quote}
  \underline{\textbf{Pomnite\Huge!}} \textit{Čim}
  \textsf{V\textbf{\LARGE E} \texttt{Č}} pisav \Huge uporabljate
  \footnotesize \textbf{v} vašem \small \texttt{dokumentu},
  \large \textit{tem} \normalsize lažje \textsc{berljiv} in
  \textsl{\textsf{lepši} pos\large t\Large a\LARGE n\huge e}.
\end{quote}

\section{Presledki}
 
\subsection{Razmik med vrsticami}

\index{razmik med vrsticami} Če želimo uporabljati v dokumentu večje razmike 
med vrsticami, 
potem lahko to spremenimo z ukazom
\begin{lscommand}
\ci{linespread}\verb|{|\emph{factor}\verb|}|
\end{lscommand}
\noindent v preambuli dokumenta.
Vrednost \verb|\linespread{1.3}| ustreza ">ena in polovičnemu"<,
vrednost \verb|\linespread{1.6}| pa ">dvojnemu"< razmiku med vrsticami. Normalno
vrstice niso razmaknjene, zato je privzeta vrednost~1.\index{dvojni razmik med vrsticami}
 
Bodite pozorni na to, da je rezultat, ki ga dobimo z uporabo ukaza
\ci{linespread} dokaj drastičen in ni primeren za objavljena dela.
Če imate dober razlog za spreminjanje privzetih razmikov med vrsticami,
zato raje uporabljajte ukaz:
\begin{lscommand}
\verb|\setlength{\baselineskip}{1.5\baselineskip}|
\end{lscommand}

\begin{example}
{\setlength{\baselineskip}%
           {1.5\baselineskip}
This paragraph is typeset with
the baseline skip set to 1.5 of
what it was before. Note the par
command at the end of the
paragraph.\par}

This paragraph has a clear
purpose, it shows that after the
curly brace has been closed,
everything is back to normal.
\end{example}

\subsection{Oblikovanje odstavka}\label{parsp}

V \LaTeX{}u imamo dva parametra, ki vplivata na obliko odstavka.
Z vključitvijo definicije kot npr.
\begin{code}
\ci{setlength}\verb|{|\ci{parindent}\verb|}{0pt}| \\
\verb|\setlength{|\ci{parskip}\verb|}{1ex plus 0.5ex minus 0.2ex}|
\end{code}
v preambulo dokumenta lahko spremenimo obliko odstavkov. Prvi ukaz nastavi
zamik prve vrstice v odstavku na 0 (brez zamika), drugi ukaz pa nastavi navpični razmik med
odstavki. \index{razmik med odstavki}\index{zamik odstavka}

Vrednosti navedeni za \texttt{plus} in \texttt{minus} povesta \TeX{}u, 
za koliko lahko maksimalno skrči oziroma razširi predpisani razmik med odstavkoma,
da se bodo odstavki lepo poravnali na stran.

V Evropi so odstavki pogosto ločeni z določenim razmikom, začetne vrstice pa niso zamaknjene.
Toda pazite, saj to vpliva tudi na kazalo. Vrstice v kazalu so sedaj ločene z večjimi razmiki kot sicer.
Da se izognemo temu, je bolje dva zgornja ukaza iz preambule prestaviti na neko mesto za 
\verb|\tableofcontents| ali pa jih sploh ne uporabljati, saj večina knjig uporablja začetni zamik in ne 
navpični razmik za ločevanje odstavkov.

Če želimo zamakniti odstavek, ki ni zamaknjen, to naredimo z ukazom
\begin{lscommand}
\ci{indent}
\end{lscommand}
\noindent na samem začetku odstavka.\footnote{Če želite zamakniti prvi odstavek v vsakem razdelku, potem 
uporabite paket \pai{indentfirst} iz svežnja `tools'.} To bo očitno imelo učinek le v primeru,
ko vrednost \verb|\parindent| ni nastavljena na $0$.

Če želimo nezamaknjen odstavek, potem uporabimo 
\begin{lscommand}
\ci{noindent}
\end{lscommand}
\noindent kot prvi ukaz v odstavku. To pride v poštev, kadar začnemo besedilo kar s tekstom in 
ne z ukazom za definiranje poglavja, razdelka, ipd.

\subsection{Vodoravni razmiki in zapolnjevalci}

\label{sec:hspace}
\LaTeX{} avtomatično določi presledek med besedami in stavki. Če želimo dodati
vodoravni razmik, uporabimo ukaz \index{vodoravni!presledek}
\begin{lscommand}
\ci{hspace}\verb|{|\emph{dolžina}\verb|}|
\end{lscommand}

Če naj se ta presledek obdrži tudi v primeru, ko pade na začetek ali na konec vrstice,
uporabimo \verb|\hspace*| namesto \verb|\hspace|. Argument
\emph{dolžina} je v enostavni obliki enak številu in merski enoti. Najpomembnejše 
merske enote so naštete v  tabeli~\ref{units}. 
\index{merska enota}\index{dolžina}

\begin{example}
Ta\hspace{1.5cm}razmik ima 
dolžino 1.5 cm.
\end{example}
\suppressfloats
\begin{table}[tbp]
\caption{\TeX{} Merske enote.} \label{units}\index{merske enote}
\begin{lined}{9.5cm} 
\begin{tabular}{@{}ll@{}}
\texttt{mm} &  milimeter $\approx 1/25$~inch \quad \demowidth{1mm} \\
\texttt{cm} & centimeter = 10~mm  \quad \demowidth{1cm}                     \\
\texttt{in} & palec (inč) $=$ 25.4~mm \quad \demowidth{1in}\\
\texttt{pt} & točka (pika) $\approx 1/72$~inča $\approx \frac{1}{3}$~mm  \quad\demowidth{1pt}\\
\texttt{em} & približna širina `M' v trenutni pisavi \quad \demowidth{1em}\\
\texttt{ex} & približna višina `x' v trenutni pisavi \quad \demowidth{1ex}
\end{tabular}

\bigskip
\end{lined}
\end{table}

\label{cmd:stretch} 
Ukaz
\begin{lscommand}
\ci{stretch}\verb|{|\emph{n}\verb|}|
\end{lscommand} 
\noindent naredi poseben raztegljiv presledek. Razteza se dokler ne zapolni ves
preostali prostor na vrstici. Če uporabimo dva ukaza 
\verb|\hspace{\stretch{|\emph{n}\verb|}}| v isti vrstici, potem se bosta razširila glede na faktor širjenja.

\begin{example}
x\hspace{\stretch{1}}
x\hspace{\stretch{3}}x
\end{example}

Ko uporabljamo vodoravne presledke med tekstom, je smiselno velikost 
presledkov relativno prilagoditi velikosti izbrane pisave. To lahko
dosežemo z merskima enotama \texttt{em} in
\texttt{ex}:

\begin{example}
{\Large{}velik\hspace{1em}y}\\
{\tiny{}drobni\hspace{1em}y}
\end{example}

Ukaz 
\begin{lscommand}
\ci{hfill}
\end{lscommand}
\noindent je okrajšava za \verb|\hspace{\fill}|. Tu je \verb|\fill| posebna raztegljiva  dolžina, ki se lahko od 0
raztegne do maksimalne možne širine. Ukaza
\begin{lscommand}
\ci{dotfill}\quad in\quad \ci{hrulefill}
\end{lscommand} 
\noindent delujeta tako kot \verb|\hfill|, le da vmesni prostor zapolnita s pikami oziroma z vodoravno črto.

\begin{example}
Začetek \dotfill\ Konec\\
Levo\ \hrulefill\ Sredina\ 
  \hrulefill\ Desno\\
X\ \hfill\hfill Malo na desno 
  \ \hfill\ X
\end{example}

\subsection{Navpični presledki}

Razmik med odstavki, razdelki, podrazdelki, \ldots\ je v \LaTeX{}u avtomatično določen.
Kadar je potrebno, lahko dodatni navpični razmik \emph{med dvema odstavkoma} vstavimo z ukazom:
\begin{lscommand}
\ci{vspace}\verb|{|\emph{length}\verb|}|
\end{lscommand}

Ta ukaz naj bi bil normalno uporabljen med dvema praznima vrsticama.
Če želimo zadržati prostor na vrhu ali na dnu strani, potem lahko uporabljamo
ukaz \verb|\vspace*| namesto \verb|\vspace|.
\index{navpični presledek}

Ukaz \verb|\stretch| v povezavi z ukazom \verb|\pagebreak| lahko uporabimo za to, 
da tekst vstavimo na zadnjo vrstico strani ali pa da tekst navpično postavimo na 
sredo strani.
\begin{code}
\begin{verbatim}
Nekaj teksta \ldots

\vspace{\stretch{1}}
To gre na zadnjo vrstico strani.\pagebreak
\end{verbatim}
\end{code}

Dodatni razmik med dvema vrsticama v \emph{istem} odstavku je 
določen z ukazom
\begin{lscommand}
\ci{\bs}\verb|[|\emph{dolžina}\verb|]|
\end{lscommand}

Z ukazom \ci{bigskip} in \ci{smallskip} lahko naredimo navpične razmike že vnaprej definiranih velikosti in nam
tako ni potrebno skrbeti za točna števila.


\section{Oblika strani}

\begin{figure}[!hp]
\begin{center}
\makeatletter\@mylayout\makeatother
\end{center}
\vspace*{1.8cm}
\caption{Parametri oblike strani.}
\label{fig:layout}
\cih{footskip}
\cih{headheight}
\cih{headsep}
\cih{marginparpush}
\cih{marginparsep}
\cih{marginparwidth}
\cih{oddsidemargin}
\cih{paperheight}
\cih{paperwidth}
\cih{textheight}
\cih{textwidth}
\cih{topmargin}
\end{figure}
\index{oblika strani}
\LaTeXe{} nam omogoča da v ukazu \verb|\documentclass| podamo \wi{velikost strani}. 
Velikost \wii{robovi}{robov} za tekst se potem določi avtomatično.
V nekaterih primerih lahko nismo zadovoljni s privzetimi vrednostmi in te vrednosti se 
seveda da ročno spremeniti.
%no idea why this is needed here ...
\thispagestyle{fancyplain}
Slika~\ref{fig:layout} prikazuje vse parametre, ki se jih da spremeniti.
Narejena je bila s paketom \pai{layout} iz svežnja ´tools'%
\footnote{\texttt{CTAN:/tex-archive/macros/latex/required/tools}}.

\textbf{POČAKAJTE!} \ldots preden začnete razmišljati v stilu 
">Naredimo to ozko stran malce  širšo"<, si vzemite nekaj sekund za razmislek. 
Kakor velja za večino stvari v \LaTeX{}u, je tudi tu dober razlog za to, da je oblika 
strani takšna kot je 

Seveda, če jo primerjamo s stranjo, narejeno s programom MS Word, zgleda grozno ozka. 
Toda poglejte v vašo priljubljeno knjigo\footnote{Tu mislim na pravo natisnjeno knjigo, ki jo je izdala priznana založba.} 
in preštejte število znakov v povprečni vrstici teksta. Opazili boste, da v vrstici ni več kot 
66 znakov. Isto ponovite na vaši \LaTeX{}ovi strani. Opazili boste, da je tudi tu okrog 66 znakov
na stran.  Izkušnje kažejo, da branje postane težje takoj, ko vrstica vsebuje
več znakov. To je zaradi tega, ker je oči težko premikati od konca ene vrstice 
na začetek naslednje. To je med drugim tudi en izmed razlogov, da je besedilo
v časopisu v več stolpcih.

Če torej povečate širino telesa s tekstom, mislite tudi na to, 
da s tem otežujete življenje vašim bralcem. Naj bo sedaj konec opozoril 
in si poglejmo, kako lahko vseeno spremenimo obliko strani.
 
Za spreminjanje parametrov ima \LaTeX{} na voljo dva ukaza. Ponavadi jih uporabljamo v 
preambuli dokumenta.

Prvi ukaz določi fiksno vrednost izbranemu parametru:
\begin{lscommand}
\ci{setlength}\verb|{|\emph{parameter}\verb|}{|\emph{dolžina}\verb|}|
\end{lscommand}

Drugi ukaz doda dolžino izbranemu parametru: 
\begin{lscommand}
\ci{addtolength}\verb|{|\emph{parameter}\verb|}{|\emph{dožina}\verb|}|
\end{lscommand} 

Drugi ukaz je bolj praktičen od \ci{setlength}, saj 
lahko obliko spreminjamo relativno glede na obstoječo obliko.
Če želimo širino teksta povečati za en centimeter, v preambulo dodamo
naslednje ukaze:
\begin{code}
\verb|\addtolength{\hoffset}{-0.5cm}|\\
\verb|\addtolength{\textwidth}{1cm}|
\end{code}

Tu se mogoče splača pogledati paket \pai{calc}, ki nam omogoča
da v argumentih ukaza \ci{setlength} in drugih, kjer vnašamo numerične vrednosti, uporabljamo aritmetične operacije.

%%%%%%%%%%%%%%%%%%%%%%%%%%%%%%%%%%%%%%%%%%%%%%%%%%%%%%%%%%%%%%%%


\section{Še več zabave z dolžinami}

Kadarkoli se da, se izogibam uporabi absolutnih dolžin v 
\LaTeX{}ovih dokumentih. Raje uporabljam za osnovne mere širino ali višino
elementov na strani. Za širino slike je to npr.~\verb|\textwidth|, če želimo, da slika zapolni celo stran.

Naslednji trije ukazi določajo širino, višino in globino tekstovnega niza.

\begin{lscommand}
\ci{settoheight}\verb|{|\emph{lscommand}\verb|}{|\emph{tekst}\verb|}|\\
\ci{settodepth}\verb|{|\emph{lscommand}\verb|}{|\emph{tekst}\verb|}|\\
\ci{settowidth}\verb|{|\emph{lscommand}\verb|}{|\emph{tekst}\verb|}|
\end{lscommand}

\noindent Naslednji zgled prikazuje možno uporabo teh ukazov.

\begin{example}
\flushleft
\newenvironment{vardesc}[1]{%
  \settowidth{\parindent}{#1:\ }
  \makebox[0pt][r]{#1:\ }}{}

\begin{displaymath}
a^2+b^2=c^2
\end{displaymath}

\begin{vardesc}{Where}$a$, 
$b$ -- are adjunct to the right 
angle of a right-angled triangle.  

$c$ -- is the hypotenuse of 
the triangle and feels lonely.

$d$ -- finally does not show up 
here at all. Isn't that puzzling?
\end{vardesc}
\end{example}

\section{Škatle}
\LaTeX{} sestavlja svoje strani z zlaganjem škatel. Kot prvo je vsaka črka
majhna škatla, ta škatla pa se zloži z ostalimi črkami v škatlo za besedo. Beseda se 
zloži z ostalimi besedami, toda tokrat s posebnim vmesnim vezivom,
ki je raztegljiv, kar omogoča, da se vrsta besed tako skrči oziroma raztegne, da zapolnjuje
eno vrstico na strani.

Priznam, da je to zelo poenostavljen pogled na to, kar se v resnici dogaja,
toda bistvo je v tem, da \TeX{} res deluje s škatlami in vmesnim vezivom.
Škatle so lahko ne samo črke, pač pa lahko v škatlo vstavimo praktično karkoli, 
vključno z drugimi škatlami. Vsako škatlo potem \LaTeX{} obravnava tako, kot da gre
za posamezno črko.

V prejšnjih poglavjih smo se že srečali s škatlami, le omenjali jih nismo eksplicitno. 
Tako npr.~okolje \ei{tabular} in ukaz \ci{includegraphics} naredita škatlo. To pomeni, da lahko
zlahka dve tabeli ali sliki postavimo drugo zraven druge. Poskrbeti moramo le za to,
da njuna kombinacija ni širša od širine teksta.

Poljuben odstavek lahko vložimo v škatlo ali z ukazom
  \begin{lscommand}  \ci{parbox}\verb|[|\emph{položaj}\verb|]{|\emph{širina}\verb|}{|\emph{tekst}\verb|}|
  \end{lscommand}
ali pa v okolju
  \begin{lscommand} \verb|\begin{|\ei{minipage}\verb|}[|\emph{položaj}\verb|]{|\emph{širina}\verb|}| tekst
  \verb|\end{|\ei{minipage}\verb|}|
\end{lscommand}
Vrednost parametra \texttt{položaj} je ena izmed črk 
\texttt{c, t} ali \texttt{b} ki pove, kako se škatla navpično poravna glede na osnovnico besedila v 
tekoči vrstici. Pri tem \texttt{t} pomeni poravnavo zgornjih robov, \texttt{c} sredinsko poravnavo in 
$\texttt{b}$ poravnana spodnja robova. Z argumentom 
\texttt{širina} podamo širino škatle. Glavna razlika med 
\ei{minipage} in \ei{parbox} je, da znotraj \ei{parbox} ne moremo uporabljati vseh ukazov in okolij,
medtem ko je v okolju \ei{minipage} možno skoraj vse.

Medtem, ko \ci{parbox} vstavi celoten odstavek tako, da prelamlja vrstice, 
obstaja tudi razred ukazov za škatle, ki delujejo le na vodoravno poravnanih objektih.
Enega že poznamo. To je ukaz \ci{mbox}, ki vrsto škatel sestavi v eno, ukaz pa 
ponavadi uporabljamo zato, da preprečimo \LaTeX{}u, da bi prelomil vrstico med dvema besedama. 
Ker lahko škatle vstavljamo v nove škatle, nam ti ukazi za vodoravno 
sestavljanje škatel omogočajo zelo veliko možnosti.

V ukazu
\begin{lscommand}
\ci{makebox}\verb|[|\emph{širina}\verb|][|\emph{položaj}\verb|]{|\emph{tekst}\verb|}|
\end{lscommand}
argument \emph{širina} definira širino škatle, kot je vidna od zunaj.\footnote{To pomeni, da je škatla lahko
  manjša kot pa je širina objektov, ki jo sestavljajo. Širino lahko postavimo celo na 0pt, s čimer dosežemo,
  da se tekst izpiše, a nima nobenega vpliva na sosednje škatle.} Poleg numeričnih vrednosti za dolžino lahko
  v argumentu \emph{širina} uporabljamo tudi ukaze \ci{width}, \ci{height}, \ci{depth} in
  \ci{totalheight}. Njihove vrednosti so odvisne od dimenzij stavljenega besedila v parametru \emph{tekst}. 
  Parameter \emph{položaj} ima 
za vrednost eno črko: \textbf{c} pomeni sredinsko poravnavo, \textbf{l} levo poravnavo,
  \textbf{r} desno poravnavo in \textbf{s} obojestransko poravnavo teksta znotraj škatle.

Ukaz \ci{framebox} deluje tako kot \ci{makebox}, le da dodatno nariše še okvir okrog teksta.

Naslednji zgledi prikazujejo, kaj vse se da narediti z ukazoma 
\ci{makebox} in \ci{framebox}.

\begin{example}
\makebox[\textwidth]{%
    s r e d i n s k o}\par
\makebox[\textwidth][s]{%
    o b o j e s t r a n s k o}\par
\framebox[1.1\width]{Sedaj sem 
    pa uokvirjen!} \par
\framebox[0.8\width][r]{Smola, 
    jaz sem pa preširok!} \par
\framebox[1cm][l]{Nič hudega,
    tudi jaz sem} 
Kdor prebere to ni osel!
\end{example}

Sedaj poznamo vodoravne škatle in naslednji korak je, da se lotimo navpičnih škatel.\footnote{Popoln
nadzor lahko dobimo le s hkratno vodoravno in navpično kontrolo \ldots}
Tudi tu ni težav za \LaTeX{}. Z ukazom

\begin{lscommand}
\ci{raisebox}\verb|{|\emph{dvig}\verb|}[|\emph{globina}\verb|][|\emph{višina}\verb|]{|\emph{tekst}\verb|}|
\end{lscommand}

\noindent lahko določimo navpične lastnosti škatle. V prvih treh parametrih lahko uporabljamo tudi 
\ci{width}, \ci{height}, \ci{depth} in \ci{totalheight}, da velikost določimo glede na 
velikost teksta v argumentu \emph{tekst}.

\begin{example}
\raisebox{0pt}[0pt][0pt]{\Large%
\textbf{Aaaa\raisebox{-0.3ex}{a}%
\raisebox{-0.7ex}{aa}%
\raisebox{-1.2ex}{r}%
\raisebox{-2.2ex}{g}%
\raisebox{-4.5ex}{h}}}
se je drl, toda tudi naslednji v 
vrsti ni opazil, da se mu je 
dogodilo nekaj groznega.
\end{example}


\section{Poljubne črte}
\label{sec:rule}

Nekaj strani nazaj ste mogoče opazili ukaz
\begin{lscommand}
\ci{rule}\verb|[|\emph{dvig}\verb|]{|\emph{dolžina}\verb|}{|\emph{višina}\verb|}|
\end{lscommand}
ki nariše črno črto z dano \emph{dolžino} in debelino \emph{višina} v višini
\emph{dvig} nad osnovnico tekoče vrstice. Parameter \emph{dvig} je lahko
tudi negativen.

\newpage
\begin{example}
\rule{3mm}{.1pt}%
\rule[-1mm]{5mm}{1cm}%
\rule{3mm}{.1pt}%
\rule[1mm]{1cm}{5mm}%
\rule{3mm}{.1pt}
\end{example}

\noindent Ta ukaz uporabljamo za risanje navpičnih in vodoravnih črt. 
Črta na naslovnici je bila tako npr. narejena z ukazom \ci{rule}.

Poseben primer črte je takšna, ki nima širine, a ima določeno višino. V 
tiskarstvu se to imenuje \wi{prečnik}. Uporabimo ga za to, da poskrbimo,
da za določeni objekt na strani podamo minimalno višino. 
Uporabljamo ga lahko tudi v okolju \texttt{tabular}, s čimer poskrbimo,
da imajo vse vrstice določeno minimalno višino.

\begin{example}
\begin{tabular}{|c|}
\hline
\rule{1pt}{4ex}S črto \ldots\\
\hline
\rule{0pt}{4ex}Z nevidnim 
prečnikom \ldots \\
\hline
\end{tabular}
\end{example}

\bigskip
{\flushright Konec.\par}

%

% Local Variables:
% TeX-master: "lshort2e"
% mode: latex
% mode: flyspell
% End:
