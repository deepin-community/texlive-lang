%%%%%%%%%%%%%%%%%%%%%%%%%%%%%%%%%%%%%%%%%%%%%%%%%%%%%%%%%%%%%%%%
% Contents: Math typesetting with LaTeX
% $Id: math.tex,v 1.2 2003/03/19 20:57:46 oetiker Exp $
%%%%%%%%%%%%%%%%%%%%%%%%%%%%%%%%%%%%%%%%%%%%%%%%%%%%%%%%%%%%%%%%%
 
\chapter{Stavljenje matematičnih formul}

\begin{intro}
  Sedaj ste pripravljeni! V tem poglavju se bomo lotili najmočnejšega orožja v
  \TeX{}u: stavljenja matematičnih tekstov. Naj vas opozorim, da se v poglavju
  snovi le na grobo dotaknemo. Stvari, ki so omenjene, bodo 
  zadoščale večini ljudi, če pa med njimi ne najdete rešitve
  za vaš problem stavljenja matematičnega teksta, ne obupajte.
  Obstaja velika verjetnost, da je vaš problem obdelan v 
  paketu \AmS-\LaTeX{}.%
  \footnote{Ameriško matematično društvo (\emph{American Mathematical Society})
  je pripravilo močno nadgradnjo \LaTeX{}a. Številni primeri 
  v tem poglavju so bili narejeni s pomočjo te razširitve.
  V zadnjem času je vključena v vse distribucije 
  \TeX{}a. Če jo vseeno pogrešate, jo lahko naložite na \CTANref|macros/latex/required/amslatex|.}
\end{intro}
 
\section{Uvod}

\LaTeX{} ima poseben način za stavljenje \wii{matematični izrazi}{matematičnih izrazov}.
Matematični tekst lahko 
znotraj odstavka vnesemo v t.i.~\emph{vrstičnem načinu} med \ci{(}
in \ci{)}, \index{$@\texttt{\$}} %$
med \texttt{\$} in \texttt{\$} ali med %}
\verb|\begin{|\ei{math}\verb|}| in \verb|\end{math}|.\index{formule}
Vsi trije načini so ekvivalentni.
\begin{example}
Vsota $a$ na kvadrat in $b$ na 
kvadrat je $c$ na kvadrat. Oziroma, 
če zapišemo bolj matematično:
\(c^{2}=a^{2}+b^{2}\).
\end{example}
\begin{example}
\TeX{} se izgovarja kot
$\tau\epsilon\chi$.\\[6pt]
100~m$^{3}$ vode.\\[6pt]
To prihaja iz mojega $\heartsuit$.
\end{example}

Večje matematične enačbe oziroma formule lahko ločimo od preostanka
odstavka tako, da jih 
\emph{prikažemo} namesto, da jih pišemo med tekstom. To naredimo tako,
da izraz pišemo v t.i. \emph{prikaznem načinu} med
\ci{[} in \ci{]}, med \verb|$$| in \verb|$$| ali pa med 
\verb|\begin{|\ei{displaymath}\verb|}| in
  \verb|\end{displaymath}|. Vsi trije načini so ekvivalentni. 
\begin{example}
Vsota $a$ na kvadrat in $b$ na 
kvadrat je $c$ na kvadrat. Oziroma, 
če zapišemo bolj matematično:
\begin{displaymath}
c^{2}=a^{2}+b^{2}
\end{displaymath}
Pišemo lahko tudi $$a+b=c.$$
\end{example}

Če želimo, da \LaTeX{} oštevilči enačbe, uporabimo okolje \ei{equation}.
Potem lahko z \ci{label} poimenujemo enačbo in se nato
sklicujemo nanjo drugje v dokumentu z uporabo 
ukaza \ci{ref} ali \ci{eqref} iz paketa \pai{amsmath}, ki avtomatično
da oznako med oklepaje:
\begin{example}
\begin{equation} \label{eq:eps}
\epsilon > 0
\end{equation}
Iz (\ref{eq:eps}), sledi 
\ldots{}S pomočjo \eqref{eq:eps}
lahko naredimo isto.
\end{example}

Bodite pozorni na to, da se izraz stavi drugače v prikaznem kot v 
vrstičnem načinu:
\begin{example}
$\lim_{n \to \infty} 
\sum_{k=1}^n \frac{1}{k^2} 
= \frac{\pi^2}{6}$
\end{example}
\begin{example}
\begin{displaymath}
\lim_{n \to \infty} 
\sum_{k=1}^n \frac{1}{k^2}
= \frac{\pi^2}{6}
\end{displaymath}
\end{example}

Obstajajo razlike med \emph{matematičnim načinom} in \emph{tekstovnim načinom}. 
V matematičnem načinu tako npr. velja: 

\begin{enumerate}

\item Večina presledkov in prelomov vrstic nima nobenega pomena. Vsi presledki sledijo ali 
iz matematičnih izrazov ali pa je zanje potrebno uporabiti posebne ukaze, kot so npr.
\ci{,}, \ci{quad} ali \ci{qquad}.
 
\item Prazne vrstice niso dovoljene. Formula ne more biti sestavljena iz več odstavkov.

\item Vsaka črka se obravnava kot ime spremenljivke in se tako tudi stavi v ustrezni pisavi.
Če želimo v matematičnem načinu v formuli pisati normalen tekst (normalna pokončna 
pisava z normalnimi presledki), potem je potrebno tak tekst vnesti s pomočjo ukaza
\verb|\textrm{...}|  (poglejte tudi razdelek \ref{sec:fontsz} na strani \pageref{sec:fontsz}).

\item Če želimo kot spremenljivke uporabljati šumnike oz.~druge črke z akcenti, potem
  moramo v matematičnem načinu uporabljati matematične akcente. Namesto 
  \verb|\v{c}| moramo tako pisati \verb|\check{c}|.
\end{enumerate}
\begin{example}
\begin{equation}
\forall x \in \mathbf{R}:
\qquad x^{2} \geq 0
\end{equation}
\end{example}
\begin{example}
\begin{equation}
x^{2} \geq 0\qquad
\textrm{za vsak }x\in\mathbf{R}
\end{equation}
\end{example}
\begin{example}
\begin{equation}
a+b+c+\check{c}+d=e
\end{equation}
\end{example}

%
% Add AMSSYB Package ... Blackboard bold .... R for realnumbers
%
Matematiki so lahko zelo sitni glede simbolov, ki naj se jih uporablja.
Lep primer sta simbola za realna in kompleksna števila. Dogovor je,
da naj se uporabljajo krepki simboli dobljeni z ukazom 
\ci{mathbb} iz paketa \pai{amsfonts} ali \pai{amssymb}.
\ifx\mathbb\undefined\else
Predzadnji primer se tako spremeni v 
\begin{example}
\begin{displaymath}
x^{2} \geq 0\qquad
\textrm{za vsak }x\in\mathbb{R}
\end{displaymath}
\end{example}
\fi


\section{Združevanje v matematičnem načinu}

Ukazi v \LaTeX{}u, ki potrebujejo argument, pričakujejo, da je argument
med zavitimi oklepaji, sicer pa kot argument vzamejo prvo naslednjo črko.
To velja tudi v matematičnem načinu, kjer večina ukazov deluje le na naslednjem znaku.
Če želimo, da ukaz deluje na več znakih skupaj, jih združimo tako, da jih damo 
med zavite oklepaje: \verb|{...}|.
\begin{example}
\begin{equation}
a^x+y \neq a^{x+y}
\end{equation}
\end{example}

\section{Osnovni gradniki matematičnih formul}

V tem razdelku bomo opisali najpomembnejše ukaze za stavljenje matematičnih 
izrazov. Za podroben seznam vseh ukazov si poglejte razdelek~\ref{symbols} na 
strani~\pageref{symbols}.

\textbf{Male \wi{grške črke}} vnašamo kot \verb|\alpha|,
 \verb|\beta|, \verb|\gamma|, \ldots, velike grške črke pa kot
\verb|\Gamma|, \verb|\Delta|, \ldots\footnote{Pri velikih črkah imamo le tiste, 
  ki se razlikujejo od latinskih, zato v \LaTeXe{} ni velikega Alpha, saj je to kar veliki A. 
  Stvari se bodo spremenile, ko bo pripravljeno novo kodiranje matematičnih znakov.} 
\begin{example}
$\lambda,\xi,\pi,\mu,\Phi,\Omega$
\end{example}
%\enlargethispage{\baselineskip}
%\pagebreak[4]

\textbf{Potence in indekse} vnašamo s pomočjo znakov \index{potenca}\index{indeks}
\verb|^|\index{^@\verb"|^"|} in \verb|_|\index{_@\verb"|_"|}.
\begin{example}
$a_{1}$ \qquad $x^{2}$ \qquad
$e^{-\alpha t}$ \qquad
$a^{3}_{ij}$\\
$e^{x^2} \neq {e^x}^2$
\end{example}

\textbf{Kvadratni koren}\index{koren} vnesemo kot \ci{sqrt}, $n$-ti koren 
pa z ukazom \verb|\sqrt[|$n$\verb|]|. Velikost znaka za koren avtomatično določi 
\LaTeX. Če potrebujemo le znak za koren, uporabimo \verb|\surd|.
\begin{example}
$\sqrt{x}$ \qquad 
$\sqrt{ x^{2}+\sqrt{y} }$ 
\qquad $\sqrt[3]{2}$\\[3pt]
$\surd[x^2 + y^2]$
\end{example}

Ukaza \ci{overline} in \ci{underline} naredita \textbf{vodoravno črto} nad oziroma pod izrazom.
\index{vodoravna!črta}
\begin{example}
$\overline{m+n}$ in 
$\underline{a+b}$
\end{example}

Ukaza \ci{overbrace} in \ci{underbrace} naredita \textbf{vodoravni zaviti oklepaj}, ki združuje elemente izraza,
nad oziroma pod izrazom. Zaviti oklepaj lahko po želji dodatno opremimo z indekom.
\index{vodoravni!zaviti oklepaj}
\begin{example}
$\underbrace{ a+b+\cdots+z }_{26}$ 
in 
$\overbrace{ 1+1+\cdots+1 }^{17}$
\end{example}

\index{matematični!akcenti} Če želimo znakom dodati matematične akcente, kot 
sta npr.~puščica za vektorje in tilda, potem uporabimo ukaze, ki
so podani v tabeli~\ref{mathacc} na strani \pageref{mathacc}. Strešice in tilde, ki se raztezajo čez več znakov, 
dobimo z ukazoma \ci{widehat} in \ci{widetilde}. Znak \verb|'|\index{'@\verb"|'"|} uporabljamo za
\wi{odvod} oz. za spremenljivke ">s črtico"<.
% a dash is --
\begin{example}
\begin{displaymath}
y=x^{2}\qquad y'=2x\qquad y''=2
\end{displaymath}
\end{example}

\textbf{Vektorje}\index{vektor} običajno pišemo tako, da nad izrazom pišemo \index{puščica}puščico.
Za to imamo na voljo ukaz \ci{vec}. Ukaza \ci{overrightarrow} in
\ci{overleftarrow} prideta v poštev za daljše izraze, kot je npr. vektor, ki gre od točke $A$ do $B$.
\begin{example}
\begin{displaymath}
\vec a\quad\overrightarrow{AB}
\end{displaymath}
\end{example}


Pri množenju ponavadi ne pišemo pike med izrazoma. V nekaterih primerih pa je 
to vseeno potrebno, da se bralec lažje znajde. Tedaj uporabimo ukaz \ci{cdot}
\begin{example}
\begin{displaymath}
v = {\sigma}_1 \cdot {\sigma}_2
    {\tau}_1 \cdot {\tau}_2
\end{displaymath}
\end{example}


Imena funkcij, kot so logaritem, sinus, \ldots, ponavadi pišemo v pokončni pisavi in ne poševno kot 
spremenljivke. Zato ima \LaTeX{} naslednje ukaze za večino najpomembnejših matematičnih funkcij:
\index{matematične!funkcije}\vspace{1em}

\begin{tabular}{lllllll}
\ci{arccos} &  \ci{cos}  &  \ci{csc} &  \ci{exp} &  \ci{ker}    & \ci{limsup} & \ci{min} \\
\ci{arcsin} &  \ci{cosh} &  \ci{deg} &  \ci{gcd} &  \ci{lg}     & \ci{ln}     & \ci{Pr}  \\
\ci{arctan} &  \ci{cot}  &  \ci{det} &  \ci{hom} &  \ci{lim}    & \ci{log}    & \ci{sec} \\
\ci{arg}    &  \ci{coth} &  \ci{dim} &  \ci{inf} &  \ci{liminf} & \ci{max}    & \ci{sin} \\
\ci{sinh} & \ci{sup} & \ci{tan} & \ci{tanh}\\
\end{tabular}

\begin{example}
\[\lim_{x \rightarrow 0}
\frac{\sin x}{x}=1\]
\end{example}

Za \index{modulska funkcija}modulsko funkcijo imamo dva ukaza : \ci{bmod} za binarni operator
">$a \bmod b$"< in \ci{pmod}
za izraze kot npr. ">$x\equiv a \pmod{b}$."<

\textbf{Ulomke} pišemo z ukazom \ci{frac}\verb|{...}{...}|.
Pogosto v tekstovnem načinu za ulomke uporabimo kar obliko $1/2$ namesto 
$\frac{1}{2}$, saj zgleda lepše.
\begin{example}
$1\frac{1}{2}$~ura
\begin{displaymath}
\frac{ x^{2} }{ k+1 }\qquad
x^{ \frac{2}{k+1} }\qquad
x^{ 1/2 }
\end{displaymath}
\end{example}

Za binomske koeficiente in podobne strukture imamo na voljo ukaza
\verb|{... |\ci{choose}\verb| ...}| in \verb|{... |\ci{atop}\verb| ...}|. 
Drugi ukaz vrne podoben rezultat kot prvi, le 
brez oklepajev.\footnote{Uporaba teh starih izrazov je prepovedana v paketu 
\pai{amsmath}, kjer sta nadomeščena z \ci{binom} in \ci{genfrac}.}

\begin{example}
\begin{displaymath}
{n \choose k}\qquad {x \atop y+2}
\end{displaymath}
\end{example}

Pri binarnih relacijah je pomembno znati postaviti izraze drug na drugega. Ukaz
\ci{stackrel} postavi podani prvi argument v velikosti enaki velikosti potenc na
drugi argument, ki je v normalni velikosti.
\begin{example}
\begin{displaymath}
\int f_N(x) \stackrel{!}{=} 1
\end{displaymath}
\end{example}

Znak za \textbf{\wi{integral}} dobimo z \ci{int}, za 
\textbf{\wii{vsota}{vsoto}} s \ci{sum} in za \textbf{\wi{produkt}}
s \ci{prod}. Zgornjo in spodnjo mejo podamo z~\verb|^| in~\verb|_|, tako 
kot potence in indekse.\footnote{\AmS-\LaTeX{} pozna tudi zgornje in spodnje indekse v več vrsticah.}
\begin{example}
\begin{displaymath}
\sum_{i=1}^{n} \qquad
\int_{0}^{\frac{\pi}{2}} \qquad
\prod_\epsilon
\end{displaymath}
\end{example}

Za še boljši nadzor nad postavitvijo indeksov v kompleksnih 
izrazih inamo v paketu \pai{amsmath} na voljo še dodatni orodji:
ukaz \ci{substack} in okolje \ei{subarray}:
\begin{example}
\begin{displaymath}
\sum_{\substack{0<i<n \\ 1<j<m}}
   P(i,j) =
\sum_{\begin{subarray}{l}
         i\in I\\
         1<j<m
      \end{subarray}}     Q(i,j)
\end{displaymath}
\end{example}

\medskip

Za \textbf{\wii{oklepaji}{oklepaje}} in druge razmejitvene simbole imamo v 
\TeX{}u vse vrste simbolov (npr.~$[\;\langle\;\|\;\updownarrow$).
Okrogle in oglate oklepaje dobimo z ustreznimi tipkami, zavite z \verb|\{|, 
za ostale pa uporabimo posebne ukaze (npr.~\verb|\updownarrow|). Za seznam vseh 
možnosti poglejte tabelo~\ref{tab:delimiters} na strani \pageref{tab:delimiters}.
\begin{example}
\begin{displaymath}
{a,b,c}\neq\{a,b,c\}
\end{displaymath}
\end{example}

Če pred začetni oklepaj postavimo ukaz \ci{left}, pred zadnjega pa \ci{right}, bo \TeX{} avtomatično 
določil velikost oklepajev. Vedno mora vsak \ci{left} imeti tudi svoj \ci{right}, velikost pa bo pravilna le,
če bosta oba uporabljena v isti vrstici. Če želimo imeti oklepaj le na eni strani, potem na drugi 
strani namesto znaka za oklepaj vstavimo piko, ki pomeni nevidni oklepaj. Če npr. želimo oklepaj le na 
levi, potem na desni uporabimo '\ci{right.}'.
\begin{example}
\begin{displaymath}
1 + \left( \frac{1}{ 1-x^{2} }
    \right) ^3
\end{displaymath}
\end{example}

V nekaterih primerih moramo sami ročno določiti velikost oklepajev\index{velikost!oklepaj}, kar lahko
naredimo z ukazi \ci{big}, \ci{Big}, \ci{bigg} in 
\ci{Bigg}, ki jih uporabimo pred večino oklepajev.\footnote{Ti ukazi
  ne delujejo pravilno, če spreminjamo velikost pisave ali če uporabljamo 
  velikost \texttt{11pt} ali \texttt{12pt}. To se da popraviti z uporabo paketa \pai{exscale} ali \pai{amsmath}.}
\begin{example}
$\Big( (x+1) (x-1) \Big) ^{2}$\\
$\big(\Big(\bigg(\Bigg($\quad
$\big\}\Big\}\bigg\}\Bigg\}$\quad
$\big\|\Big\|\bigg\|\Bigg\|$
\end{example}

Za vnos \textbf{\wii{tri pike}{treh pik}} v formulo imamo na voljo več ukazov.
Ukaz \ci{ldots} izpiše pike na dnu, \ci{cdots} da pike na sredo, poleg tega pa imamo
še ukaza \ci{vdots} za \wi{navpične pike} in \ci{ddots} za \wi{diagonalne pike}.\index{vertikalne!pike}\index{vodoravne!pike} 
Več primerov lahko najdete v razdelku~\ref{sec:vert}.
\begin{example}
\begin{displaymath}
x_{1},\ldots,x_{n} \qquad
x_{1}+\cdots+x_{n}
\end{displaymath}
\end{example}
 
\section{Presledki v matematičnem načinu}

\index{matematični presledki} Če presledki v enačbah, ki jih izbere \TeX{}
niso zadovoljivi, jih lahko popravimo s pomočjo posebnih ukazov. 
Tako imamo na voljo nekaj ukazov za majhne presledke: \ci{,} za
$\frac{3}{18}\:\textrm{quad}$ (\demowidth{0.166em}), \ci{:} za $\frac{4}{18}\:
\textrm{quad}$ (\demowidth{0.222em}) in \ci{;} za $\frac{5}{18}\:
\textrm{quad}$ (\demowidth{0.277em}).  Standardni ukaz za presledek 
\verb*.\ . naredi srednje velik presledek, ukaza \ci{quad} 
(\demowidth{1em}) in \ci{qquad} (\demowidth{2em}) pa velika presledka. 
Velikost \ci{quad} ustreza širini znaka `M' v trenutni pisavi.  Ukaz \verb|\!|\cih{"!} naredi 
negativni presledek s širino $-\frac{3}{18}\:\textrm{quad}$ (\demowidth{0.166em}).
\begin{example}
\newcommand{\ud}{\mathrm{d}}
\begin{displaymath}
\int\!\!\!\int_{D} g(x,y)
  \, \ud x\, \ud y 
\end{displaymath}
namesto
\begin{displaymath}
\int\int_{D} g(x,y)\ud x \ud y
\end{displaymath}
\end{example}
Bodite pozorni na to, da je znak za diferencial `d' zapisan v pokončni pisavi.

V \AmS-\LaTeX{}u imamo na voljo dodatne ukaze za fino nastavitev presledkov med 
večkratnimi znaki za integriranje, to so ukazi
\ci{iint}, \ci{iiint}, \ci{iiiint} in \ci{idotsint}.
Če naložimo paket \pai{amsmath}, lahko zgornji primer sestavimo tudi na naslednji način:
\begin{example}
\newcommand{\ud}{\mathrm{d}}
\begin{displaymath}
\iint_{D} \, \ud x \, \ud y
\end{displaymath}
\end{example}

Za več podrobnosti poglejte dokument testmath.tex (vsebovan je v paketu 
\AmS-\LaTeX) ali poglavje 8 v ">The LaTeX Companion"<.

\section{Navpično poravnavanje}
\label{sec:vert}

Za sestavljanje \textbf{matematičnih razpredelnic} uporabljamo okolje \ei{array}. Deluje podobno
kot okolje \texttt{tabular}. Za prelom vrstice uporabljamo ukaz \verb|\\|.
\begin{example}
\begin{displaymath}
\mathbf{X} =
\left( \begin{array}{ccc}
x_{11} & x_{12} & \ldots \\
x_{21} & x_{22} & \ldots \\
\vdots & \vdots & \ddots
\end{array} \right)
\end{displaymath}
\end{example}

Okolje \ei{array} lahko uporabljamo tudi za sestavljanje izrazov,
ki imajo na eni strani en velik oklepaj, če na drugi strani uporabimo prazni oklepaj, kot 
npr. `\verb|\right.|'.
\begin{example}
\begin{displaymath}
y = \left\{ \begin{array}{ll}
 a & \textrm{če $d>c$}\\
 b+x & \textrm{zjutraj}\\
 l & \textrm{čez cel dan}
  \end{array} \right.
\end{displaymath}
\end{example}


Kot v okolju \verb|tabular| lahko tudi v okolju \ei{array}
rišemo navpične in vododravne črte, ki npr.~ločujejo elemente matrike:
\begin{example}
\begin{displaymath}
\left(\begin{array}{c|c}
 1 & 2 \\
\hline
3 & 4
\end{array}\right)
\end{displaymath}
\end{example}




Za formule, ki se raztezajo čez več vrstic oz.~za \index{sistemi enačb}sisteme enačb
uporabljamo okolji \ei{eqnarray} in \verb|eqnarray*| namesto
\texttt{equation}. V \texttt{eqnarray} se vsaka vrstica avtomatično oštevilči, pri 
\verb|eqnarray*| pa se nobena vrstica ne oštevilči.


Okolji \texttt{eqnarray} in \verb|eqnarray*| delujeta kot 
razpredelnica s tremi stolpci oblike \verb|{rcl}|, kjer se srednji stolpec
uporablja za enačaj,  neenačaj, ali pa katerikoli drugi znak, po katerem želimo poravnati vrstice.
Ukaz \verb|\\| pomeni prehod v novo vrstico.
\begin{example}
\begin{eqnarray}
f(x) & = & \cos x     \\
f'(x) & = & -\sin x   \\
\int_{0}^{x} f(y)dy &
 = & \sin x
\end{eqnarray}
\end{example}
Opazimo lahko, da je presledek tako na levi kot tudi na desni strani 
enačaja precej velik. To lahko zmanjšamo s
\verb|\setlength\arraycolsep{2pt}|, kot je to prikazano v naslednjem zgledu.


\index{dolge enačbe} \textbf{Dolge enačbe} se avtomatično ne delijo lepo. Avtor mora sam povedati, kje 
naj se začne nova vrstica in kolikšen naj bo začetni zamik. V ta namen
se najpogosteje
uporabljata naslednji dve metodi.
\begin{example}
{\setlength\arraycolsep{2pt}
\begin{eqnarray}
\sin x & = & x -\frac{x^{3}}{3!}
     +\frac{x^{5}}{5!}-{}
                    \nonumber\\
 & & {}-\frac{x^{7}}{7!}+{}\cdots
\end{eqnarray}}
\end{example}
\begin{example}
\begin{eqnarray}
\lefteqn{ \cos x = 1
     -\frac{x^{2}}{2!} +{} }
                    \nonumber\\
 & & {}+\frac{x^{4}}{4!}
     -\frac{x^{6}}{6!}+{}\cdots
\end{eqnarray}
\end{example}

%\enlargethispage{\baselineskip}
\noindent Ukaz \ci{nonumber} pove \LaTeX{}u, da naj ne oštevilči enačbe. Argument ukaza 
\ci{lefteqn} \LaTeX{} izpiše, vendar ga obravnava, kot da ima širino 0. Prva vrstica v zadnjem primeru je 
tako sestavljena le iz levega stolpca, ker pa ima za \LaTeX{} širino 0, se naslednja vrstica začne že pred koncem
zgornje vrstice.

Če z navedenimi metodami še vedno ne uspemo pravilno navpično poravnati
enačb, potem v paketu \pai{amsmath} obstajajo še 
močnejši ukazi (poglejte okolji \verb|split| in \verb|align|).


\section{Fantomski objekti}

Fantomskih objektov ne vidimo, kljub temu pa zavzemajo prostor. Z njimi lahko v \LaTeX{}u izvedemo
zanimive trike.

Ko navpično poravnavamo tekst, ki vsebuje \verb|^| in \verb|_|, je \LaTeX{} dostikrat 
celo preveč uslužen. S pomočjo ukaza \ci{phantom} lahko rezerviramo prostor za znake, ki se v končnem 
rezultatu ne pokažejo. Najbolje, da pogledamo kar naslednja zgleda.
\begin{example}
\begin{displaymath}
{}^{12}_{\phantom{1}6}\textrm{C}
\qquad \textrm{ali} \qquad
{}^{12}_{6}\textrm{C}
\end{displaymath}
\end{example}
\begin{example}
\begin{displaymath} 
\Gamma_{ij}^{\phantom{ij}k}
\qquad \textrm{ali} \qquad
\Gamma_{ij}^{k}
\end{displaymath}  
\end{example}

\section{Velikost pisave v matematičnem načinu}\label{sec:fontsz}

\index{velikost matematične pisave} V matematičnem načinu \TeX{} določi velikost pisave 
glede na kontekst. Potence in indeksi se npr. izpišejo z manjšo velikostjo. 
Če želite del enačbe izpisati pokončno, potem ne uporabite ukaza 
\verb|\textrm|, saj v tem primeru spreminjanje velikosti pisave ne deluje,
saj \verb|\textrm| začasno prestavi v tekstovni način. Pravilno je 
uporabiti \verb|\mathrm|, saj se tu velikost pisave avtomatično prilagaja. Paziti pa je potrebno,
da \ci{mathrm} deluje dobro le na kratkih izrazih. 
Presledki in znaki z akcenti ne delujejo.\footnote{Če vključimo paket \AmS-\LaTeX{}, potem ukaz \ci{textrm} deluje s prilagajanjem velikosti pisave.}
\begin{example}
\begin{equation}
2^{\textrm{nd}} \quad 
2^{\mathrm{nd}}
\end{equation}
\end{example}


Kljub temu je v \LaTeX{}u včasih potrebno ročno določiti pravilno velikost 
pisave. V matematičnem načinu imamo za to na voljo naslednje štiri ukaze:
\begin{flushleft}
\ci{displaystyle}~($\displaystyle 123$),
 \ci{textstyle}~($\textstyle 123$), 
\ci{scriptstyle}~($\scriptstyle 123$) in
\ci{scriptscriptstyle}~($\scriptscriptstyle 123$).
\end{flushleft}

Spreminjanje velikosti vpliva tudi na to, kako se izpisujejo meje.
\begin{example}
\begin{displaymath}
\mathop{\mathrm{corr}}(X,Y)= 
 \frac{\displaystyle 
   \sum_{i=1}^n(x_i-\overline x)
   (y_i-\overline y)} 
  {\displaystyle\biggl[
 \sum_{i=1}^n(x_i-\overline x)^2
\sum_{i=1}^n(y_i-\overline y)^2
\biggr]^{1/2}}
\end{displaymath}    
\end{example}
% This is not a math accent, and no maths book would be set this way.
% mathop gets the spacing right.

\noindent To je eden izmed primerov, kjer potrebujemo večji 
oklepaj od tistega, ki ga dobimo iz \verb|\left[  \right]|.


\section{Izreki, trditve, \ldots}

Ko pišemo matematični tekst, potrebujemo način za stavljenje 
 lem, definicij, izrekov, aksiomov in podobnih struktur. 
 \LaTeX{} podpira to z ukazom
\begin{lscommand}
\ci{newtheorem}\verb|{|\emph{ime}\verb|}[|\emph{stevec}\verb|]{|%
         \emph{naslov}\verb|}[|\emph{section}\verb|]|
\end{lscommand}
Argument \emph{ime} je kratka ključna beseda, s katero povemo \LaTeX{}u za kakšno matematično trditev gre.
V argumentu \emph{naslov} navedemo dejansko ime trditve, ki se izpiše v prevedenem dokumentu.

Argumenti v oglatih oklepajih so neobvezni. Z njimi lahko določimo način oštevilčenja trditev. 
Argument \emph{stevec} vsebuje ime trditve, po kateri naj se številči trditev. 
Izberemo lahko številčenje po že predhodno definirani trditvi ali pa vnesemo kar isto 
ime kot v parametru \emph{ime}. 
Nove trditve se potem številčijo v istem zaporedju. Argument \emph{section} je ime logične strukture
znotraj katere številčimo trditve.

Če damo ukaz \ci{newtheorem} v preambulo dokumenta, lahko potem 
znotraj dokumenta uporabljamo okolje
\verb|\begin{ime} ... \end{ime}|
\begin{code}
\verb|\begin{|\emph{ime}\verb|}[|\emph{tekst}\verb|]|\\
To je moj zanimiv izrek.\\
\verb|\end{|\emph{ime}\verb|}|     
\end{code}

V paketu \pai{amsthm} je na voljo ukaz \ci{newtheoremstyle}\verb|{|\emph{style}\verb|}|,
s katerim lahko definiramo vrsto matematične trditve z izbiro
ene izmed treh vnaprej pripravljenih oblik: \texttt{definition} (krepak naslov, 
pokončna pisava),
\texttt{plain} (krepak naslov, poševna pisava) ali 
\texttt{remark} (poševen naslov, pokončna pisava).

Dovolj teorije. Naslednji zgledi bodo razjasnili vse dvome 
in dokončno dokazali, da je okolje \verb|\newtheorem|
prezapleteno, da bi ga lahko razumeli.

% actually define things
\theoremstyle{definition} \newtheorem{izrek}{Izrek}
\theoremstyle{plain}      \newtheorem{posl}[izrek]{Posledica}
\theoremstyle{remark}     \newtheorem*{opomba}{Opomba}

Najprej definiramo nove matematične izjave:

\begin{verbatim}
\theoremstyle{definition} \newtheorem{izrek}{Izrek}
\theoremstyle{plain}      \newtheorem{posl}[izrek]{Posledica}
\theoremstyle{remark}     \newtheorem*{opomba}{Opomba}
\end{verbatim}

\begin{example}
\begin{izrek}[Pitagora] 
\label{izrek:Pit}
V pravokotnem trikotniku velja 
$c^2=a^2+b^2$.
\end{izrek}
\begin{posl}
Če velja $c^2\ne a^2+b^2$, potem 
trikotnik ni pravokoten
(poglej izrek~\ref{izrek:Pit}).
\end{posl}
\begin{opomba}To je bil moj 
najnovejši izrek o \ldots
\end{opomba}
\end{example}

\newcounter{law}
Trditev ">Posledica"< uporablja isti števec kot trditev ">Izrek"<,
zato dobi številko iz istega zaporedja. 
Neobvezni argument v oglatih oklepajih uporabimo za to, da določimo naslov trditve ali
kaj podobnega (npr. avtorja in referenco).
\begin{example}
\flushleft
\newtheorem{mur}{Murphy}[section]
\begin{mur}
Če se da nekaj narediti na več
načinov in eden izmed njih vodi
v katastrofo, potem bo nekdo 
izbral prav ta način.
\end{mur}
\end{example}

Trditev ">Murphy"< dobi številko, ki je vezana na številko trenutnega razdelka.
Namesto tega lahko uporabimo tudi kakšno drugo logično enoto, npr. 
poglavje ali podrazdelek.

V paketu \pai{amsthm} je na voljo tudi okolje \ei{proof} za dokaze.

\begin{example}
\begin{proof}
 Očitno, uporabimo
\[E=mc^2.\]
\end{proof}
\end{example}

Z ukazom \ci{qedhere} lahko premaknemo znak za konec dokaza
v situacijah, ko bi sicer stal na koncu prazne vrstice.

\begin{example}
\begin{proof}
 Očitno, uporabimo
\[E=mc^2. \qedhere\]
\end{proof}
\end{example}

\section{Krepki simboli}
\index{krepki simboli}

V \LaTeX{}u ni najbolj enostavno dobiti krepkih simbolov; to je najbrž namerno zaradi tega, ker 
jih začetniki preveč uporabljajo. Ukaz za spremembo pisave
\verb|\mathbf| sicer naredi krepke črke, toda to so pokončne črke, matematični simboli pa so ponavadi 
poševni. Obstaja ukaz \ci{boldmath}, toda tega \emph{lahko uporabljamo le zunaj matematičnega načina}.
Deluje pa tudi za simbole.
\begin{example}
\begin{displaymath}
\mu, M \qquad \mathbf{M} \qquad
\mbox{\boldmath $\mu, M$}
\end{displaymath}
\end{example}

\noindent
V zgornjem primeru je krepka tudi vejica, kar ni ravno to, kar bi radi.

Paket \pai{amsbsy} (ki se avtomatično vključi z \pai{amsmath}) kakor tudi paket
\pai{bm} olajšata zadevo z ukazom \ci{boldsymbol}.
\ifx\boldsymbol\undefined\else
\begin{example}
\begin{displaymath}
\mu, M \qquad
\boldsymbol{\mu}, \boldsymbol{M}
\end{displaymath}
\end{example}
\fi

% Local Variables:
% TeX-master: "lshort2e"
% mode: latex
% mode: flyspell
% End:
