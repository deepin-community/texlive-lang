%%%%%%%%%%%%%%%%%%%%%%%%%%%%%%%%%%%%%%%%%%%%%%%%%%%%%%%%%%%%%%%%%
% Contents: TeX and LaTeX and AMS symbols for Maths
% $Id: lssym.tex,v 1.1.1.1 2002/02/26 10:04:21 oetiker Exp $
%%%%%%%%%%%%%%%%%%%%%%%%%%%%%%%%%%%%%%%%%%%%%%%%%%%%%%%%%%%%%%%%%
\section{Danh sách các kí hiệu toán học}\label{symbols}

Các bảng sau đây trình bày tất cả các kí hiệu thông thường có thể sử dụng trong \emph{chế độ soạn thảo toán học}.

Để sử dụng các kí hiệu được liệt kê ở bảng~\ref{AMSD}--\ref{AMSNBR}\footnote{các bảng sau được trích từ \texttt{symbols.tex} được soạn bởi David~Carlisle và sau đó được thay đổi nhiều theo sự gợi ý của Josef~Tkadlec.}, thì bạn cần phải đưa gói \pai{amssymb} vào tài liệu ở phần tựa đề của tài liệu và các font chữ AMS dành cho toán học phải được cài sẵn trên máy. Nếu gói AMS và các font chữ chưa được cài đặt thì bạn có thể tải về ở địa chỉ \texttt{CTAN:/tex-archive/macros/latex/required/amslatex}. Bạn cũng có thể tải về một danh sách chi tiết hơn về các kí hiệu tại địa chỉ \texttt{CTAN:info/symbols/comprehensive}.

\begin{table}[!htb]
\caption{Các dấu trọng âm trong chế độ soạn thảo toán học.}
\label{mathacc}
\begin{symbols}{*4{cl}}
\W{\hat}{a}     & \W{\check}{a} & \W{\tilde}{a} & \W{\acute}{a} \\
\W{\grave}{a} & \W{\dot}{a} & \W{\ddot}{a}    & \W{\breve}{a} \\
\W{\bar}{a} &\W{\vec}{a} &\W{\widehat}{A}&\W{\widetilde}{A}\\
\end{symbols}
\end{table}

\begin{table}[!htb]
\caption{Các chữ cái Hy Lạp viết thường.}
\begin{symbols}{*4{cl}}
 \X{\alpha}     & \X{\theta}     & \X{o}          & \X{\upsilon}  \\
 \X{\beta}      & \X{\vartheta}  & \X{\pi}        & \X{\phi}      \\
 \X{\gamma}     & \X{\iota}      & \X{\varpi}     & \X{\varphi}   \\
 \X{\delta}     & \X{\kappa}     & \X{\rho}       & \X{\chi}      \\
 \X{\epsilon}   & \X{\lambda}    & \X{\varrho}    & \X{\psi}      \\
 \X{\varepsilon}& \X{\mu}        & \X{\sigma}     & \X{\omega}    \\
 \X{\zeta}      & \X{\nu}        & \X{\varsigma}  & &             \\
 \X{\eta}       & \X{\xi}        & \X{\tau}
\end{symbols}
\end{table}

\begin{table}[!tb]
\caption{Các chữ cái Hy Lạp viết hoa.}
\begin{symbols}{*4{cl}}
 \X{\Gamma}     & \X{\Lambda}    & \X{\Sigma}     & \X{\Psi}      \\
 \X{\Delta}     & \X{\Xi}        & \X{\Upsilon}   & \X{\Omega}    \\
 \X{\Theta}     & \X{\Pi}        & \X{\Phi}
\end{symbols}
\end{table}
\clearpage

\begin{table}[!htb]
\caption{Quan hệ hai ngôi.}
\bigskip
Bạn có thể có được các kí hiệu ngược lại tương ứng với các kí hiệu
ở đây bằng cách thêm vào tiền tố \ci{not} trước lệnh tương ứng.
\begin{symbols}{*3{cl}}
 \X{<}           & \X{>}           & \X{=}          \\
 \X{\leq}or \verb|\le|   & \X{\geq}or \verb|\ge|   & \X{\equiv}     \\
 \X{\ll}         & \X{\gg}         & \X{\doteq}     \\
 \X{\prec}       & \X{\succ}       & \X{\sim}       \\
 \X{\preceq}     & \X{\succeq}     & \X{\simeq}     \\
 \X{\subset}     & \X{\supset}     & \X{\approx}    \\
 \X{\subseteq}   & \X{\supseteq}   & \X{\cong}      \\
 \X{\sqsubset}$^a$ & \X{\sqsupset}$^a$ & \X{\Join}$^a$    \\
 \X{\sqsubseteq} & \X{\sqsupseteq} & \X{\bowtie}    \\
 \X{\in}         & \X{\ni}, \verb|\owns|  & \X{\propto}    \\
 \X{\vdash}      & \X{\dashv}      & \X{\models}    \\
 \X{\mid}        & \X{\parallel}   & \X{\perp}      \\
 \X{\smile}      & \X{\frown}      & \X{\asymp}     \\
 \X{:}           & \X{\notin}      & \X{\neq}or \verb|\ne|
\end{symbols}
\centerline{\footnotesize $^a$Sử dụng gói \textsf{latexsym} để sử
dụng các kí hiệu này}
\end{table}

\begin{table}[!htb]
\caption{Các toán tử hai ngôi.}
\begin{symbols}{*3{cl}}
 \X{+}              & \X{-}              & &                 \\
 \X{\pm}            & \X{\mp}            & \X{\triangleleft} \\
 \X{\cdot}          & \X{\div}           & \X{\triangleright}\\
 \X{\times}         & \X{\setminus}      & \X{\star}         \\
 \X{\cup}           & \X{\cap}           & \X{\ast}          \\
 \X{\sqcup}         & \X{\sqcap}         & \X{\circ}         \\
 \X{\vee}, \verb|\lor|     & \X{\wedge}, \verb|\land|  & \X{\bullet}       \\
 \X{\oplus}         & \X{\ominus}        & \X{\diamond}      \\
 \X{\odot}          & \X{\oslash}        & \X{\uplus}        \\
 \X{\otimes}        & \X{\bigcirc}       & \X{\amalg}        \\
 \X{\bigtriangleup} &\X{\bigtriangledown}& \X{\dagger}       \\
 \X{\lhd}$^a$         & \X{\rhd}$^a$         & \X{\ddagger}      \\
 \X{\unlhd}$^a$       & \X{\unrhd}$^a$       & \X{\wr}
\end{symbols}

\end{table}

\begin{table}[!tbp]
\caption{Các toán tử lớn.}
\begin{symbols}{*4{cl}}
 \X{\sum}      & \X{\bigcup}   & \X{\bigvee}   & \X{\bigoplus}\\
 \X{\prod}     & \X{\bigcap}   & \X{\bigwedge} &\X{\bigotimes}\\
 \X{\coprod}   & \X{\bigsqcup} & &             & \X{\bigodot} \\
 \X{\int}      & \X{\oint}     & &             & \X{\biguplus}
\end{symbols}

\end{table}


\begin{table}[!tbp]
\caption{Các dấu mũi tên.}
\begin{symbols}{*3{cl}}
 \X{\leftarrow}or \verb|\gets|& \X{\longleftarrow} & \X{\uparrow}          \\
 \X{\rightarrow}or \verb|\to|& \X{\longrightarrow} & \X{\downarrow}        \\
 \X{\leftrightarrow}    & \X{\longleftrightarrow}& \X{\updownarrow}      \\
 \X{\Leftarrow}         & \X{\Longleftarrow}     & \X{\Uparrow}          \\
 \X{\Rightarrow}        & \X{\Longrightarrow}    & \X{\Downarrow}        \\
 \X{\Leftrightarrow}    & \X{\Longleftrightarrow}& \X{\Updownarrow}      \\
 \X{\mapsto}            & \X{\longmapsto}        & \X{\nearrow}          \\
 \X{\hookleftarrow}     & \X{\hookrightarrow}    & \X{\searrow}          \\
 \X{\leftharpoonup}     & \X{\rightharpoonup}    & \X{\swarrow}          \\
 \X{\leftharpoondown}   & \X{\rightharpoondown}  & \X{\nwarrow}          \\
 \X{\rightleftharpoons} & \X{\iff}(bigger spaces)& \X{\leadsto}$^a$

\end{symbols}
\centerline{\footnotesize $^a$Sử dụng gói \textsf{latexsym} để sử
dụng các kí hiệu này}
\end{table}

\begin{table}[!tbp]
\caption{Các dấu ngoặc.}\label{tab:delimiters}
\begin{symbols}{*4{cl}}
 \X{(}            & \X{)}            & \X{\uparrow} & \X{\Uparrow}    \\
 \X{[}or \verb|\lbrack|   & \X{]}or \verb|\rbrack|  & \X{\downarrow}   & \X{\Downarrow}  \\
 \X{\{}or \verb|\lbrace|  & \X{\}}or \verb|\rbrace|  & \X{\updownarrow} & \X{\Updownarrow}\\
 \X{\langle}      & \X{\rangle}  & \X{|}or \verb|\vert| &\X{\|}or \verb|\Vert|\\
 \X{\lfloor}      & \X{\rfloor}      & \X{\lceil}       & \X{\rceil}      \\
 \X{/}            & \X{\backslash}   & &. (cả hai đều trống)
\end{symbols}
\end{table}

\begin{table}[!tbp]
\caption{Các dấu ngoặc lớn.}
\begin{symbols}{*4{cl}}
 \Y{\lgroup}      & \Y{\rgroup}      & \Y{\lmoustache}  & \Y{\rmoustache} \\
 \Y{\arrowvert}   & \Y{\Arrowvert}   & \Y{\bracevert}
\end{symbols}
\end{table}


\begin{table}[!tbp]
\caption{Các kí hiệu khác.}
\begin{symbols}{*4{cl}}
 \X{\dots}       & \X{\cdots}      & \X{\vdots}      & \X{\ddots}     \\
 \X{\hbar}       & \X{\imath}      & \X{\jmath}      & \X{\ell}       \\
 \X{\Re}         & \X{\Im}         & \X{\aleph}      & \X{\wp}        \\
 \X{\forall}     & \X{\exists}     & \X{\mho}$^a$      & \X{\partial}   \\
 \X{'}           & \X{\prime}      & \X{\emptyset}   & \X{\infty}     \\
 \X{\nabla}      & \X{\triangle}   & \X{\Box}$^a$     & \X{\Diamond}$^a$ \\
 \X{\bot}        & \X{\top}        & \X{\angle}      & \X{\surd}      \\
\X{\diamondsuit} & \X{\heartsuit}  & \X{\clubsuit}   & \X{\spadesuit} \\
 \X{\neg}or \verb|\lnot| & \X{\flat}       & \X{\natural}    & \X{\sharp}

\end{symbols}
\centerline{\footnotesize $^a$Sử dụng gói \textsf{latexsym} để sử
dụng các kí hiệu này.}
\end{table}

\begin{table}[!tbp]
\caption{Các kí hiệu thông thường.}
\bigskip
These symbols can also be used in text mode.
\begin{symbols}{*4{cl}}
 \SC{\dag}  &  \SC{\S}  &  \SC{\copyright} &  \SC{\textregistered}  \\
 \SC{\ddag} &  \SC{\P}  &  \SC{\pounds}    &  \SC{\%}               \\
\end{symbols}
\end{table}

%
%
% If the AMS Stuff is not available, we drop out right here :-)
%

\begin{table}[!tbp]
\caption{Các dấu ngoặc theo AMS.}\label{AMSD}
\bigskip
\begin{symbols}{*4{cl}}
\X{\ulcorner}&\X{\urcorner}&\X{\llcorner}&\X{\lrcorner}\\
\X{\lvert}&\X{\rvert}&\X{\lVert}&\X{\rVert}
\end{symbols}
\end{table}

\begin{table}[!tbp]
\caption{Chữ cái Hy Lạp và Do Thái theo AMS.}
\begin{symbols}{*5{cl}}
\X{\digamma}     &\X{\varkappa} & \X{\beth}& \X{\daleth}     &\X{\gimel}
\end{symbols}
\end{table}

\begin{table}[!tbp]
\caption{Quan hệ hai ngôi theo AMS.}
\begin{symbols}{*3{cl}}
 \X{\lessdot}           & \X{\gtrdot}            & \X{\doteqdot}or \verb|\Doteq| \\
 \X{\leqslant}          & \X{\geqslant}          & \X{\risingdotseq}     \\
 \X{\eqslantless}       & \X{\eqslantgtr}        & \X{\fallingdotseq}    \\
 \X{\leqq}              & \X{\geqq}              & \X{\eqcirc}           \\
 \X{\lll}or \verb|\llless|      & \X{\ggg}or \verb|\gggtr| & \X{\circeq}  \\
 \X{\lesssim}           & \X{\gtrsim}            & \X{\triangleq}        \\
 \X{\lessapprox}        & \X{\gtrapprox}         & \X{\bumpeq}           \\
 \X{\lessgtr}           & \X{\gtrless}           & \X{\Bumpeq}           \\
 \X{\lesseqgtr}         & \X{\gtreqless}         & \X{\thicksim}         \\
 \X{\lesseqqgtr}        & \X{\gtreqqless}        & \X{\thickapprox}      \\
 \X{\preccurlyeq}       & \X{\succcurlyeq}       & \X{\approxeq}         \\
 \X{\curlyeqprec}       & \X{\curlyeqsucc}       & \X{\backsim}          \\
 \X{\precsim}           & \X{\succsim}           & \X{\backsimeq}        \\
 \X{\precapprox}        & \X{\succapprox}        & \X{\vDash}            \\
 \X{\subseteqq}         & \X{\supseteqq}         & \X{\Vdash}            \\
 \X{\Subset}            & \X{\Supset}            & \X{\Vvdash}           \\
 \X{\sqsubset}          & \X{\sqsupset}          & \X{\backepsilon}      \\
 \X{\therefore}         & \X{\because}           & \X{\varpropto}        \\
 \X{\shortmid}          & \X{\shortparallel}     & \X{\between}          \\
 \X{\smallsmile}        & \X{\smallfrown}        & \X{\pitchfork}        \\
 \X{\vartriangleleft}   & \X{\vartriangleright}  & \X{\blacktriangleleft}\\
 \X{\trianglelefteq}    & \X{\trianglerighteq}   &\X{\blacktriangleright}
\end{symbols}
\end{table}

\begin{table}[!tbp]
\caption{Các dấu mũi tên theo AMS.}
\begin{symbols}{*3{cl}}
 \X{\dashleftarrow}      & \X{\dashrightarrow}     & \X{\multimap}          \\
 \X{\leftleftarrows}     & \X{\rightrightarrows}   & \X{\upuparrows}        \\
 \X{\leftrightarrows}    & \X{\rightleftarrows}    & \X{\downdownarrows}    \\
 \X{\Lleftarrow}         & \X{\Rrightarrow}        & \X{\upharpoonleft}     \\
 \X{\twoheadleftarrow}   & \X{\twoheadrightarrow}  & \X{\upharpoonright}    \\
 \X{\leftarrowtail}      & \X{\rightarrowtail}     & \X{\downharpoonleft}   \\
 \X{\leftrightharpoons}  & \X{\rightleftharpoons}  & \X{\downharpoonright}  \\
 \X{\Lsh}                & \X{\Rsh}                & \X{\rightsquigarrow}   \\
 \X{\looparrowleft}      & \X{\looparrowright}     &\X{\leftrightsquigarrow}\\
 \X{\curvearrowleft}     & \X{\curvearrowright}    & &                      \\
 \X{\circlearrowleft}    & \X{\circlearrowright}   & &
\end{symbols}
\end{table}

\begin{table}[!tbp]
\caption{Quan hệ phủ định hai ngôi và các dấu mũi tên theo
AMS.}\label{AMSNBR}
\begin{symbols}{*3{cl}}
 \X{\nless}           & \X{\ngtr}            & \X{\varsubsetneqq}  \\
 \X{\lneq}            & \X{\gneq}            & \X{\varsupsetneqq}  \\
 \X{\nleq}            & \X{\ngeq}            & \X{\nsubseteqq}     \\
 \X{\nleqslant}       & \X{\ngeqslant}       & \X{\nsupseteqq}     \\
 \X{\lneqq}           & \X{\gneqq}           & \X{\nmid}           \\
 \X{\lvertneqq}       & \X{\gvertneqq}       & \X{\nparallel}      \\
 \X{\nleqq}           & \X{\ngeqq}           & \X{\nshortmid}      \\
 \X{\lnsim}           & \X{\gnsim}           & \X{\nshortparallel} \\
 \X{\lnapprox}        & \X{\gnapprox}        & \X{\nsim}           \\
 \X{\nprec}           & \X{\nsucc}           & \X{\ncong}          \\
 \X{\npreceq}         & \X{\nsucceq}         & \X{\nvdash}         \\
 \X{\precneqq}        & \X{\succneqq}        & \X{\nvDash}         \\
 \X{\precnsim}        & \X{\succnsim}        & \X{\nVdash}         \\
 \X{\precnapprox}     & \X{\succnapprox}     & \X{\nVDash}         \\
 \X{\subsetneq}       & \X{\supsetneq}       & \X{\ntriangleleft}  \\
 \X{\varsubsetneq}    & \X{\varsupsetneq}    & \X{\ntriangleright} \\
 \X{\nsubseteq}       & \X{\nsupseteq}       & \X{\ntrianglelefteq}\\
 \X{\subsetneqq}      & \X{\supsetneqq}      &\X{\ntrianglerighteq}\\[0.5ex]
 \X{\nleftarrow}      & \X{\nrightarrow}     & \X{\nleftrightarrow}\\
 \X{\nLeftarrow}      & \X{\nRightarrow}     & \X{\nLeftrightarrow}

\end{symbols}
\end{table}

\begin{table}[!tbp]
\caption{Các toán tử nhị phận theo AMS.}
\begin{symbols}{*3{cl}}
 \X{\dotplus}        & \X{\centerdot}      & \X{\intercal}      \\
 \X{\ltimes}         & \X{\rtimes}         & \X{\divideontimes} \\
 \X{\Cup}or \verb|\doublecup|& \X{\Cap}or \verb|\doublecap|& \X{\smallsetminus} \\
 \X{\veebar}         & \X{\barwedge}       & \X{\doublebarwedge}\\
 \X{\boxplus}        & \X{\boxminus}       & \X{\circleddash}   \\
 \X{\boxtimes}       & \X{\boxdot}         & \X{\circledcirc}   \\
 \X{\leftthreetimes} & \X{\rightthreetimes}& \X{\circledast}    \\
 \X{\curlyvee}       & \X{\curlywedge}		 &
\end{symbols}
\end{table}

\begin{table}[!tbp]
\caption{Các kí hiệu khác theo AMS.}
\begin{symbols}{*3{cl}}
 \X{\hbar}             & \X{\hslash}           & \X{\Bbbk}            \\
 \X{\square}           & \X{\blacksquare}      & \X{\circledS}        \\
 \X{\vartriangle}      & \X{\blacktriangle}    & \X{\complement}      \\
 \X{\triangledown}     &\X{\blacktriangledown} & \X{\Game}            \\
 \X{\lozenge}          & \X{\blacklozenge}     & \X{\bigstar}         \\
 \X{\angle}            & \X{\measuredangle}    & \X{\sphericalangle}  \\
 \X{\diagup}           & \X{\diagdown}         & \X{\backprime}       \\
 \X{\nexists}          & \X{\Finv}             & \X{\varnothing}      \\
 \X{\eth}              & \X{\mho}							 &
\end{symbols}
\end{table}



\begin{table}[!tbp]
\caption{Các kiểu chữ cái trong toán.}
\begin{symbols}{@{}*3l@{}}
Ví dụ& Lệnh &Gói lệnh cần dùng\\
\hline
\rule{0pt}{1.05em}$\mathrm{ABCdef}$
        & \verb|\mathrm{ABCdef}|
        &       \\
$\mathit{ABCdef}$
        & \verb|\mathit{ABCdef}|
        &       \\
$\mathnormal{ABCdef}$
        & \verb|\mathnormal{ABCdef}|
        &       \\
$\mathcal{ABC}$
        & \verb|\mathcal{ABC}|
        & \pai{euscript} với tuỳ chọn \texttt{mathcal} \\
$\mathscr{ABC}$
        &\verb|\mathscr{ABC}|
        &\pai{mathrsfs}\\
$\mathfrak{ABCdef}$
        & \verb|\mathfrak{ABCdef}|
        &\pai{eufrak}                \\
$\mathbb{ABC}$
        & \verb|\mathbb{ABC}|
        &\pai{amsfonts} hay \textsf{amssymb}        \\
\end{symbols}
\end{table}

\endinput

%
% Local Variables:
% TeX-master: "lshort2e"
% mode: latex
% coding: utf-8
% End: