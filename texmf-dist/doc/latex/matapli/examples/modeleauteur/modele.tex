\documentclass{matapli}

% pour la compilation avec PDFlatex
\usepackage[utf8]{inputenc}
\usepackage[T1]{fontenc}

% pour les exemples
\usepackage{lipsum}

% on ajoute notre fichier de bibliographie
\addbibresource{chap1.bib}

\begin{document}

% titre de l'article
\titre[court={Titre court}]{Titre long de la contribution}

% on renseigne les auteurs et autrices. Ici avec tous les champs possibles. L'encart en fin d'article n'est généré que si 
% le champs minibio est renseigné
\author[
affiliation = {CNRS, Laboratoire de l'université de France},
minibio = {Georges \bsc{Felepin} est Ingénieur de Recherche au CNRS. Sa discrétion n'a d'égal que l'ampleur de ses travaux.},
photo = portrait.png,
email = georges@felepin.fr,
webpage = www.felepin.fr/
]
{Felepin,Georges}

% un auteur de type collectif (et non individu)
\author[
type=collectif,
]
{Centre National de la Recherche Scientifique}

% table des matières propre à l'article
\articletableofcontents

% sectionnement de l'article en parties si besoin 
% (au dessus de section)
\partie{Un soustitre pour les différentes parties}

\section{Première section}

\lipsum[1-4]
Citation, voir~\cite{Knuth1984}.

\subsection{Sous section}
\lipsum[5]

\[\int_0^1 f(x)\mathrm{d}x=F(1)-F(0).\]

\section{Quelques environnements de la classe}

\lipsum[1]

\begin{bloc}
  \lipsum[10]
\end{bloc}

\begin{Important}
\lipsum[8]
\end{Important}

\section{Les interviews}

\MatapliQuestion[Maxime]{Que se passe-t-il ?}
\MatapliReponse{Laurent}{Rien.}

\MatapliQuestion{Question sans nom pour la poser ?}
\MatapliReponse{Laurent}{\cite{TeXMetafont}}

\section{Les maths}

\begin{theorem}{Test}{test}
  Voici mon Théorème, classique, mais efficace.
\end{theorem}


\begin{proof}
  On y fait référence~\ref{th:test} pour en écrire la preuve.
\end{proof}

\begin{definition}{Test}{test}
  Ma super définition
\end{definition}

\begin{lemma}
  Un petit lemme.
\end{lemma}

\begin{corollary}
  Un corrolaire.
\end{corollary}

\begin{remark}
  Une remarque.
\end{remark}


% on imprime la bibliographie avec l'option pour que celle-ci ne crée pas un autre article
\printbibliography[heading=subbibintoc]

% commande obligatoire pour générer la composition des auteurs et autrices
\printauthors
\end{document}
