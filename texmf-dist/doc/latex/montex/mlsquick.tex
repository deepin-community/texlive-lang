%%%%%%%%%%%%%%%%%%%%%%%%%%%%%%%%%%%%%%%%%%%%%%%%%%%%%%%%%%%%%%%%
%        File: mlsquick.tex
%      Author: Oliver Corff
%        Date: \VersionDate November 1st, 2001
%     Version: \VersionRelease
%   Copyright: Ulaanbaatar, Beijing, Berlin
%
% Description: MonTeX -- Mongolian for LaTeX2e
%              Implementation Level \ImplementationLevel
%              System Documentation
%
%%%%%%%%%%%%%%%%%%%%%%%%%%%%%%%%%%%%%%%%%%%%%%%%%%%%%%%%%%%%%%%%
% ----------------- identification ends here -------------------
%
\documentclass[11pt,a4paper]{article}
\usepackage{longtable}
\IfFileExists{ctib}{%
	\usepackage{ctib}
}{}
\usepackage[latin1]{mls}
\input{mtdocmac.tex}
%\usepackage{pslatex}
\usepackage{hyperref}
\begin{document}
\title{\MonTeX\\A Quick Guide\\(\emph{Draft})}
\author{Oliver Corff}
\maketitle
\tableofcontents
\section{General Settings}
In order to access the commands of \MonTeX\ the package must be
loaded in the document preamble by saying

\begin{verbatim}
\usepackage[<language options>,<encoding options>]{mls}
\end{verbatim}

The options include choices for the basic document language and
input encodings.

\subsubsection{Document Language}

The document language can be set with one of
	\verb"bicig",
	\verb"bithe",
	\verb"buryat",
	\verb"english",
	\verb"russian" or
	\verb"xalx"
like in
\begin{verbatim}
\usepackage[xalx]{mls}
\end{verbatim}
which issues all captions and the date in Modern Mongolian.

The options \refcmda{bicig} and \refcmda{bithe} are introduced 
in part~\ref{BicigandBithe}, ``Full Vertical Text Pages''.

The options \cmda{buryat}, \cmda{russian} and \cmda{xalx}
	produce captions in Buryat, Russian and Modern Mongolian.

\begin{center}
	\begin{tabular}{ll}
	    \textbf{Buryat}	&\BuryatToday\\\label{BuryatToday}
	    \textbf{Xalx}	&\XalxToday\\\label{XalxToday}
	    \textbf{Russian}	&\RussianToday\\\label{RussianToday}
	\end{tabular}
\end{center}


The option
\cmda{english}, at least as a \verb"\usepackage" option, is
essentially a do-nothing: it sets captions to English (which is
the default of this package anyway).
\section{Cyrillic Text -- \xalx{Kirill "us"ag}}

\subsection{Cyrillic Text in Transliteration (\LMC) Mode%
	\label{section:CyrillicTransliterationMode}}

\MonTeX\ provides two basic modes of operation: in
\begin{itemize}\label{SetDocumentEncoding}
	\item Transliteration Mode (intimately linked to the \LMC\
		encoding) all incoming text is regarded as
		transliterated Cyrillic. This allows users to
		compose Cyrillic documents on pure ASCII machines.
		In contrast, the
	\item Immediate Mode does nothing and waits for explicit
		Cyrillic characters in the input in order to generate
		Cyrillic output.
\end{itemize}
Two commands are used to switch between these modes:
\begin{quote}
\begin{verbatim}
\SetDocumentEncodingLMC
\SetDocumentEncodingNeutral
\end{verbatim}
\end{quote}

The first command switches to Transliteration Mode, the second
command deactivates the transliteration and thus, by definition,
activates Immediate Mode.

In the \LMC\ encoding, most Cyrillic characters are mapped directly to
a single Latin character but for some characters there is a text
command which became necessary since there are more Cyrillic than
Latin characters. For convenience, a few ligatures were defined, too.
Details are given in table~\ref{cyralpha}.

\begin{table}
\begin{center}
\begin{tabular}{|r|cc|cc|ll|}
\hline
%\multicolumn{7}{|c|}{Cyrillic Alphabet Input Methods} \\\hline
   &\multicolumn{2}{|c|}{Cyrillic Letter}&\multicolumn{2}{|c|}{\LMC\
Input}&\multicolumn{2}{|c|}{Generic Command}\\\hline
 1 &\mnr A &\mnr a &\verb"A" &\verb"a" &\verb"\CYRA" &\verb"\cyra" \\\hline
 
 2 &\mnr B &\mnr b &\verb"B" &\verb"b" &\verb"\CYRB" &\verb"\cyrb" \\\hline
 
 3 &\mnr W &\mnr w &\verb"W" &\verb"w" &\verb"\CYRV" &\verb"\cyrw" \\\hline
 
 4 &\mnr G &\mnr g &\verb"G" &\verb"g" &\verb"\CYRG" &\verb"\cyrg" \\\hline
 
 5 &\mnr D &\mnr d &\verb"D" &\verb"d" &\verb"\CYRD" &\verb"\cyrd" \\\hline
 
 6 &\mnr E &\mnr e &\verb"E" &\verb"e" &\verb"\CYRE"&\verb"\cyre" \\\hline
 
 7 &\CYRYO &\cyryo &\texttt{\"E}/\verb'"E'&\texttt{\"e}/\verb'"e'&%
 			\verb"\CYRYO"&\verb"\cyryo"\rule{0mm}{2.25ex}\\ 
   &       &       &\{\verb"\"\}\verb"YO"&\{\verb"\"\}\verb"yo"&  & \\\hline

 8 &\mnr J &\mnr j &\verb"J" &\verb"j" &\verb"\CYRZH" &\verb"\cyrzh" \\\hline
 
 9 &\mnr Z &\mnr z &\verb"Z" &\verb"z" &\verb"\CYRZ" &\verb"\cyrz" \\\hline

10 &\mnr I &\mnr i &\verb"I" &\verb"i" &\verb"\CYRI" &\verb"\cyri" \\\hline

11 &\CYRISHRT &\cyrishrt &\texttt{\"I}/\verb'"I'&\texttt{\"i}/\verb'"i'&%
			\verb"\CYRISHRT"&\verb"\cyrishrt"\rule{0mm}{2.25ex} \\
   &       &       &\{\verb"\"\}\verb"YI"&\{\verb"\"\}\verb"yi"&  & \\\hline

12 &\mnr K &\mnr k &\verb"K" &\verb"k" &\verb"\CYRK" &\verb"\cyrk" \\\hline

13 &\mnr L &\mnr l &\verb"L" &\verb"l" &\verb"\CYRL" &\verb"\cyrl" \\\hline

14 &\mnr M &\mnr m &\verb"M" &\verb"m" &\verb"\CYRM" &\verb"\cyrm" \\\hline

15 &\mnr N &\mnr n &\verb"N" &\verb"n" &\verb"\CYRN" &\verb"\cyrn" \\\hline

16 &\mnr O &\mnr o &\verb"O" &\verb"o" &\verb"\CYRO" &\verb"\cyro" \\\hline

17 &\CYROTLD &\cyrotld &\texttt{\"O}/\verb'"O'&\texttt{\"o}/\verb'"o'&%
			\verb"\CYROTLD"&\verb"\cyrotld" \rule{0mm}{2.25ex}\\\hline

18 &\mnr P &\mnr p &\verb"P" &\verb"p" &\verb"\CYRP" &\verb"\cyrp" \\\hline

19 &\mnr R &\mnr r &\verb"R" &\verb"r" &\verb"\CYRR" &\verb"\cyrr" \\\hline

20 &\mnr S &\mnr s &\verb"S" &\verb"s" &\verb"\CYRS" &\verb"\cyrs" \\\hline

21 &\mnr T &\mnr t &\verb"T" &\verb"t" &\verb"\CYRT" &\verb"\cyrt" \\\hline

22 &\mnr U &\mnr u &\verb"U" &\verb"u" &\verb"\CYRU" &\verb"\cyru" \\\hline

23 &\mnr "U &\mnr "u &\texttt{\"U}/\verb'"U'&\texttt{\"u}/\verb'"u'&%
			\verb"\CYRY"&\verb"\cyry" \rule{0mm}{2.25ex}\\\hline

24 &\mnr F &\mnr f &\verb"F" &\verb"f" &\verb"\CYRF" &\verb"\cyrf" \\\hline

25 &\mnr X &\mnr x &\verb"X" &\verb"x" &\verb"\CYRH" &\verb"\cyrh" \\\hline

26 &\mnr H &\mnr h &\verb"H" &\verb"h" &\verb"\CYRHSHA" &\verb"cyrhsha" \\\hline

27 &\mnr C &\mnr c &\verb"C" &\verb"c" &\verb"\CYRC" &\verb"\cyrc" \\\hline

28 &\mnr Q &\mnr q &\verb"Q" &\verb"q" &\verb"\CYRCH" &\verb"\cyrch" \\
   &       &       &\verb"\Ch"&\verb"\ch"&           &             \\\hline

29 &\mnr\Sh&\mnr\sh&\verb"\Sh"&\verb"\sh"&\verb"\CYRSH"&\verb"\cyrsh"\\
   &       &       &          &\verb"sh" &             &             \\\hline

30 &\mnr\Sc&\mnr\sc&\verb"\Sc"&\verb"\sc"&\verb"\CYRSHCH"&\verb"\cyrshch"\\
   &       &       &\verb"\Qh"&\verb"\qh"&             &             \\\hline

31 &\mnr \CYRHRDSN &\mnr \cyrsftsn &\verb"\Y" &\verb"\y" &%
			\verb"\CYRHRDSN" &\verb"\cyrhrdsn" \\\hline

32 &\mnr Y &\mnr y &\verb"Y" &\verb"y" &\verb"\CYRERY" &\verb"\cyrery" \\\hline

33 &\mnr \CYRSFTSN &\mnr \cyrsftsn &\verb"\I" &\verb"\i" &%
			\verb"\CYRSFTSN" &\verb"\cyrsftsn" \\\hline

34 &\CYREREV &\cyrerev &\texttt{\"A}/\verb'"A'&\texttt{\"a}/\verb'"a'&%
			\verb"\CYREREV"&\verb"\cyrerev" \rule{0mm}{2.25ex}\\\hline

35 &\mnr YU&\mnr yu&\{\verb"\"\}\verb"YU"&\{\verb"\"\}\verb"yu"&%
			\verb"\CYRYU"&\verb"\cyryu"\\\hline

36 &\mnr YA&\mnr ya&\{\verb"\"\}\verb"YA"&\{\verb"\"\}\verb"ya"&%
			\verb"\CYRYA"&\verb"\cyrya"\\\hline
\end{tabular}
\caption{Cyrillic Alphabet Input Methods\label{cyralpha}}
\end{center}
\end{table}

Front vowels can be entered directly using the encoding slot of a
valid and active input encoding, or they can be expressed via an
abbreviated \verb'"'\emph{v} notation where \emph{v} stands for any
desired vowel. In the \LMC\ encoding used by \MonTeX, \verb'"' is not
an active character; selecting the proper letter is done by ligature
statements in the Metafont sources.
 
Some letters can be entered with or without a preceding \verb"\",
like \cyryu\ and \cyrya. Both \verb"\yu" and \verb"yu" will produce
a \cyryu. While \verb"yu" is interpreted as a ligature, \verb"\yu"
allows for the character \cyryu\ to be combined with accents.
Accents are not commonly used in Mongolian since there are precise
rules for word stress. This feature is taken from the \textsf{OT2} encoding 
and is included mainly for the sake of completeness, convenience and
compatibility\footnote{The magic triple-C!}.

Here now a sample of Mongolian text:
\exa
	{\mnr<<Xalxyn gurwan "ond"or>> 
	x"am"a"an aldarshsan, Z"u"un xyazgaaryg
	toxinuulax sa"id N.~Dugarjaw ardyn
	xuw\i sgalyn b"u"ur "ax"an "ue"as
	xamgi"in "agz"agt"a"i am\i\ d"u"is"an
	alband tomilogdox c"ar"ag da"iny olon
	quxal daalgawryg xiq"a"ang"u"il"an
	biel"u"ulj yawsan t"u"uxt"a"i x"un.}
\exb
\begin{verbatim}
	{\mnr<<Xalxyn gurwan "ond"or>> 
	x"am"a"an aldarshsan, Z"u"un xyazgaaryg
	toxinuulax sa"id N.~Dugarjaw ardyn
	xuw\i sgalyn b"u"ur "ax"an "ue"as
	xamgi"in "agz"agt"a"i am\i\ d"u"is"an
	alband tomilogdox c"ar"ag da"iny olon
	quxal daalgawryg xiq"a"ang"u"il"an
	biel"u"ulj yawsan t"u"uxt"a"i x"un.}
\end{verbatim}
\exc

\subsection{Shorthands for embedding words in a different
		typeface}\label{typefacecapsules}

Sometimes it may be necessary to give short portions of text not
only in a different encoding (for which the \cmd{lat}\verb:{...}:
and \cmd{mnr}\verb:{...}: commands are
useful) but it may also be necessary to switch the typeface
temporarily. Usually capsules using \verb'\text'\emph{xx} do the
work if only the typeface is concerned, and building nested commands
like \verb'\textsf{\lat{...}}' is cumbersome if these changes have
to be applied very often. \MonTeX\ provides an abbreviated style
following the rule 
\begin{quote}
	\texttt{[k|l]}\emph{two letter font style code}\verb'{...}'
\end{quote}
where the font style code is one of
	\verb'rm',
	\verb'bf',
	\verb'it',
	\verb'sl',
	\verb'sf',
	\verb'sc' and
	\verb'tt',
like \verb'\ksl{...}', \verb'\lsc{...}', etc.


\subsection{Shorthands for writing transliterated texts}

\MonTeX\ provides shortcuts for writing certain accented symbols
used in conventional transliterating of Mongolian by 
haceks, the nasal and the gamma. These shortcuts are essentially
mnemonics replacing the somewhat more tedious accent notation (see
table~\ref{shortcuts}).

\begin{table}
\begin{center}\begin{tabular}{ll|ll}
%\hline
Letter	& Input 	& Letter	& Input \\
	&		&		&\\
\hline
	&		&		&\\
\ch 	& \verb"\ch"	& \Ch		& \verb"\Ch" \\
\jh 	& \verb"\jh"	& \Jh		& \verb"\Jh" \\
\sh 	& \verb"\sh"	& \Sh		& \verb"\Sh" \\
\zh 	& \verb"\zh"	& \Zh		& \verb"\Zh" \\
\ng 	& \verb"\ng"	& \Ng		& \verb"\Ng" \\
\g	& \verb"\g"	& \G		& \verb"\G" \\
%\hline
\end{tabular}\end{center}
\caption{Shortcuts for Mongolian Transliteration Symbols\label{shortcuts}}
\end{table}

It must be observed that these commands are by default dependent on
the environment they are used in. \verb"\Sh" yields a \Sh\ when used
in a Latin environment but results in a \mnr\Sh\rnm\ when used in a
Cyrillic context\footnote{The authors wish to thank J.~Knappen for
resolving one instability in the original code for these letters.}:

\exa
	\emph{\Sh agdar} and \emph{\Ch adraa}
	are transliterations for
	{\mnr\Sh agdar} and {\mnr\Ch adraa}.
\exb
	\begin{verbatim}
	\emph{\Sh agdar} and \emph{\Ch adraa}
	are transliterations for
	{\mnr\Sh agdar} and {\mnr\Ch adraa}.
	\end{verbatim}
\exc


\section{Uighur Mongolian and Manju Input}

A comprehensive table of the Mongolian alphabet and its MLS
transliteration, the input conventions of the MLS transliteration in
\MonTeX\ and the Simplified Transliteration is given in
table~\ref{table:bcgcagan}.

\newcommand{\bcgcagan}[4]{%
	\mbosoo{#1}	& \texttt{#2}	& \texttt{#3}	& \texttt{#4}	%\\
	}

\begin{table}[h]
\begin{center}
\begin{tabular}{cccc|cccc}
%\hline
Uighur&\multicolumn{2}{c}{MLS} &Simplified
				&Uighur&\multicolumn{2}{c}{MLS}&Simplified\\
Script&Transl.& Input	 &Input     &Script&Transl.& Input&Input\\
\hline
\bcgcagan{a}{a}{a}{a}		& \bcgcagan{s}{s}{s}{s} \\
\bcgcagan{E}{\"a}{\"a, E}{e}	& \bcgcagan{S}{sh}{S}{sh}\\
\bcgcagan{e}{e}{e}{v}		& \bcgcagan{t}{t}{t}{t}	 \\
\bcgcagan{i}{i}{i}{i}		& \bcgcagan{d}{d}{d}{d, t}\\
\bcgcagan{o}{o}{o}{u}		& \bcgcagan{l}{l}{l}{l}	\\
\bcgcagan{u}{u}{u}{u}		& \bcgcagan{m}{m}{m}{m}	\\
\bcgcagan{O}{\"o}{\"o, O}{ui, u}& \bcgcagan{c}{c}{c}{c}	\\
\bcgcagan{U}{\"u}{\"u, U}{ui, u}& \bcgcagan{z}{z}{z}{z}	\\
\bcgcagan{n}{n}{n}{n}		& \bcgcagan{y}{y}{y}{y}	\\
\bcgcagan{|ng}{*ng}{ng}{ng}	& \bcgcagan{r}{r}{r}{r}	\\
\bcgcagan{x}{x}{x}{x}		& \bcgcagan{v}{v}{v}{v}	\\
\bcgcagan{G}{\g}{G}{g}		& \bcgcagan{h}{h}{h}{h}	\\
\bcgcagan{k}{k}{k}{k}		& \bcgcagan{j}{j}{j}{j}	\\
\bcgcagan{g}{g}{g}{g, k}	& \bcgcagan{K}{K}{K}{K}	\\
\bcgcagan{b}{b}{b}{b}		& \bcgcagan{Q}{[--]}{Q}{q}\\
\bcgcagan{p}{p}{p}{p}		& \bcgcagan{C}{C}{C}{C}	\\
\bcgcagan{f}{f}{f}{f}		& \bcgcagan{Z}{Z}{Z}{Z}	\\
%\hline
\end{tabular}
\end{center}
\caption{Mongolian Script Transliterations\label{table:bcgcagan}}
\end{table}

The possible combinations of Mongolian writing input methods
and display commands are listed in table~\ref{table:Combinations}.
The columns stand for each possible input encoding, the rows contain
the display command types. Each table cell at the contains the command
that is available for a given combination of input method and
command.

\newcommand{\ComparisonTable}[4]{%
	#1 &%			% Command Type
	#2 &%			% MLS Command
	#3 &%			% Simplified Command
	#4 \\%			% Manju Command
}
\begin{table}[h]
\begin{center}
\begin{tabular}{p{2cm}|p{3.25cm}|p{3.25cm}|p{3.25cm}}
Command		& \multicolumn{2}{c|}{Mongolian}& Manju	\\
Type		& MLS		& Simplified	&	\\
\hline
\ComparisonTable{Document Encoding}
		{only available as font encoding \LMS, not as
		document encoding}
		{\texttt{LMO}}
		{\texttt{LMA}}
\hline
\ComparisonTable{Horizontal Capsules}
		{\refcmd{bcg}}
		{\refcmd{bicig}}
		{\refcmd{bithe}}
\hline
\ComparisonTable{Horizontal Paragraphs}
		{not available}
		{\refcmda{bicigtext}}
		{\refcmda{bithetext}}
\hline
\ComparisonTable{Vertical Capsules}
		{\refcmd{mbosoo}}
		{\refcmd{mobosoo}}
		{\refcmd{mabosoo}}
\hline
\ComparisonTable{Vertical Paragraph Boxes}
		{not available}
		{\refcmd{mobox}}
		{\refcmd{mabox}}
\hline
\ComparisonTable{Vertical Pages}
		{not available}
		{\refcmda{bicigpage}}
		{\refcmda{bithepage}}
%\hline
\end{tabular}
\caption{Mongolian Input and Display Commands\label{table:Combinations}}
\end{center}
\end{table}

While the input method for the majority of characters matches the
transliteration conventions, some letters require a slightly
different treatment:
\begin{enumerate}
	\item	Although the diphtong \mobosoo{*aii*} is usually
		rendered as \textit{ayi}, it must be entered
		as \texttt{aii} in order to produce the desired
		effect.
	
	\item	The back vowels \emph{o} and \emph{u} are both rendered
		as \texttt{u}.

	\item	The front vowels \emph{\"o} and \emph{\"u} are both
		rendered as \texttt{ui} in first syllables and as
		\texttt{u} in later syllables.

	\item	Since \mobosoo{t} means both \emph{t} and \emph{d},
		it is necessary to spell this letter as \texttt{t}
		in the beginning of words, and \texttt{d} in the
		middle of words, regardless of the actual meaning.

	\item	The four consonants \emph{\g}, \emph{g}, \emph{x}
		and \emph{k} are constrained with regard to the
		following vowels. The Simplified Transliteration
		renders these as \texttt{g} (before \emph{a}
		and \emph{u} only), \texttt{g} (before \emph{a}
		and \emph{u} only), \texttt{x} and \texttt{k}.
\end{enumerate}


As it was demonstrated in subsection~\ref{section:CyrillicTransliterationMode},
it is technically possible to choose between an automatic document encoding
and the neutral mode. In the case of Uighur Mongolian, the mode of choice
activates the Simplified Transliteration Mode and is called with 
\begin{quote}
\begin{verbatim}
\SetDocumentEncodingBicig
\end{verbatim}
\end{quote}


With
\verb"\SetDocumentEncodingBicig"\label{cmd:SetDocumentEncodingBicig} set,
it is possible to switch to the Simplified Transliteration Mode anywhere
in the document, not only in the preamble.

\textit{Caveat:} Since switching to Uighur Mongolian text
requires a lot of settings to be effected at the same time, there
are high-level commands available (see below all kinds of  Mongolian
and Manju Display Commands) which do all the work, including the definition of the
document encoding. Thus, while \verb|\SetDocumentEncodingBicig|
is indeed classified as a user-level command, it is certainly not
necessary for everyday work.


\subsection{Character Variants}

With the assistance of special, non-printing characters like the
Form Variant Selectors, the appearance of certain characters can be
modified in order to display typographical and orthographical
variants. Notably, the \emph{n} will loose its dot before vowels,
as will \emph{\g}. Let's assume the word ``place'' is written in an
old book as \bicig{g'azar}. It should be understood that this is a
variant of \bicig{gazar} and should be spelled \emph{\g'azar}, not
\emph{xazar}. With vowels, the Form Variant Selectors can change the
shape that is usually required by graphical context. At present,
only the first of two Form Variant Selectors actually does
something, the exact behaviour of the second Form Variant Selector
waits to be implemented.

The following short example shows a concrete application of this
method. It renders the six syllable mantra \emph{om ma ni padme hum}
(tib.  {\tib \om, ma nxi pa\V{de}{ma} \hung}) as it is
displayed on a huge bronze incense burner in front of the Gandan
Monastery in Ulaanbaatar:

\exa
	\mobox{3cm}{\noindent\sffamily
	\om	uva\\
	\  	ma'=a\\
	\  	n'i\\
	\  	badmi'\\
	\om	huu}
\exb
	\begin{verbatim}
	\mobox{3cm}{\noindent\sffamily
	\om	uva\\
	\ 	ma'=a\\
	\ 	n'i\\
	\ 	badmi'\\
	\om	huu}
	\end{verbatim}
\exc
	
\subsection{Special Characters\label{section:SpecialMLSCharacters}}

For the correct operation of retransliterating systems processing
Mongolian script additional symbols are needed. These include
Form Variant Selectors (\textsf{FVS}), the Vowel Separator, and
other symbols like the Mongolian Positional Indicator. As can be
seen from its usage in table~\ref{table:bcgcagan}, entering \verb|*ng|
tells the system to consider this \emph{ng} to be in non-initial
position.\footnote{Unfortunately, though it is now commonly
agreed in the scientific community that these symbols are needed,
their definition is still in a state of flux, and thus the symbols
given here are presented on a preliminary basis.}

Besides these symbols, table~\ref{table:SpecialMLSCharacters} includes
also some useful punctuation marks etc.\ as they are used in
Mongolian Script.

\begin{table}
\begin{center}
\begin{tabular}{c|l|l}
%\hline
 Symbol		& Name			& Input		\\
 \hline
  \bosoo{\glyphbcg{!}}	& Exclamation Mark	& \verb|!|	\\
  \bosoo{\glyphbcg{?}}	& Question Mark		& \verb|?|	\\
  \bosoo{\glyphbcg{!?}}	& Exclamation Question Mark& \verb|!?|	\\
  \bosoo{\glyphbcg{?!}}	& Question Exclamation Mark& \verb|?!|	\\
  \bosoo{\glyphbcg{*}}	& Mong. Positional Indicator& \verb|*|	\\
  \bosoo{\glyphbcg{\char32}}	& Mongolian Space	& \verb*|-|	\\
  \bosoo{\glyphbcg{(}}	& Opening Bracket	& \verb|(|	\\
  \bosoo{\glyphbcg{)}}	& Closing Bracket	& \verb|)|	\\
  \bosoo{\glyphbcg{<}}	& Opening Angle Bracket	& \verb|<|	\\
  \bosoo{\glyphbcg{>}}	& Closing Angle Bracket	& \verb|>|	\\
  \bosoo{\glyphbcg{<<}}	& Opening Guillemot	& \verb|<<|	\\
  \bosoo{\glyphbcg{>>}}	& Closing Guillemot	& \verb|>>|	\\
% \bosoo{\glyphbcg{\{}}	& Opening Parenthesis	& \verb|{|	\\
% \bosoo{\glyphbcg{\}}}	& Closing Parenthesis	& \verb|}|	\\
  \bosoo{\glyphbcg{'}}	& Form Variant Selector 1& \verb|'|	\\
  \bosoo{\glyphbcg{"}}	& Form Variant Selector 2& \verb|"|	\\
  \bosoo{\glyphbcg{\char43}}& Mong. Vowel Separator	& \verb|=|	\\
  \bosoo{\glyphbcg{|}}	& Mongolian Nuruu	& \verb'|'	\\
  \bosoo{\glyphbcg{.}}	& Period		& \verb|.|	\\
  \bosoo{\glyphbcg{,}}	& Comma			& \verb|,|	\\
  \bosoo{\glyphbcg{:}}	& Colon			& \verb|:|	\\
  \bosoo{\glyphbcg{;}}	& D\"orw\"oljin		& \verb|;|	\\
  \bosoo{\glyphbcg{..}}	& Ellipsis		& \verb|..|	\\
  \bosoo{\glyphbcg{0}}	& Digit zero		& \verb|0|	\\
  \bosoo{\glyphbcg{1}}	& Digit one		& \verb|1|	\\
  \bosoo{\glyphbcg{2}}	& Digit two		& \verb|2|	\\
  \bosoo{\glyphbcg{3}}	& Digit three		& \verb|3|	\\
  \bosoo{\glyphbcg{4}}	& Digit four		& \verb|4|	\\
  \bosoo{\glyphbcg{5}}	& Digit five		& \verb|5|	\\
  \bosoo{\glyphbcg{6}}	& Digit six		& \verb|6|	\\
  \bosoo{\glyphbcg{7}}	& Digit seven		& \verb|7|	\\
  \bosoo{\glyphbcg{8}}	& Digit eight		& \verb|8|	\\
  \bosoo{\glyphbcg{9}}	& Digit nine		& \verb|9|	\\
% \hline
\end{tabular}
\end{center}
\caption{Mongolian Script Special Symbols and Punctuation
	Marks\label{table:SpecialMLSCharacters}}
\end{table}

\subsection{Manju Input}

Manju documents can be compiled with the \refcmda{bithe} option
to the \verb|\usepackage| command, which will create complete
documents in Manju. Anywhere in the document, it is possible to
switch to Manju input (transliteration mode)
with
\verb"\SetDocumentEncodingBithe"\label{cmd:SetDocumentEncodingBithe} which
internally activates the \LMA\label{a:LMA} encoding.

\textit{Caveat:} Since switching to Manju text
requires a lot of settings to be effected at the same time, there
are high-level commands available (see below) which do all the work, including
the definition of the document encoding. Thus, while
\verb|\SetDocumentEncodingBithe| is indeed classified as a
user-level command, it is certainly not necessary for everyday work.


\subsection{Basic Character Set and Romanization}

Given by dictionary order, the system provides a basic
character set as shown in table~\ref{table:ManjuBasicChars}.

\newcommand{\MaEntry}[3]{\mabosoo{#1}& #2 & #3 }
\begin{table}
\begin{center}
\begin{tabular}{ccc|ccc|ccc}
Manju&Input&Latin&Manju&Input&Latin&Manju&Input&Latin\\
\hline
\MaEntry{a}{a}{a}	& \MaEntry{h}{h}{h}	& \MaEntry{c}{c}{c}	\\
\MaEntry{e}{e}{e}	& \MaEntry{b}{b}{b}	& \MaEntry{j}{j}{j}	\\
\MaEntry{i}{i}{i}	& \MaEntry{p}{p}{p}	& \MaEntry{y}{y}{y}	\\
\MaEntry{o*}{o}{o}	& \MaEntry{s}{s}{s}	& \MaEntry{k'}{k'}{k'}	\\
\MaEntry{u*}{u}{u}	& \MaEntry{s'}{s'}{\v s}	& \MaEntry{g'}{g'}{g'}	\\
\MaEntry{v}{v}{\={u}}	& \MaEntry{t}{t}{t}	& \MaEntry{h'}{h'}{h'}	\\
\MaEntry{n}{n}{n}	& \MaEntry{d}{d}{d}	& \MaEntry{r}{r}{r}	\\
\MaEntry{k}{k}{k}	& \MaEntry{l}{l}{l}	& \MaEntry{f}{f}{f}	\\
\MaEntry{g}{g}{g}	& \MaEntry{m}{m}{m}	& \MaEntry{w}{w}{w}	\\
\end{tabular}
\caption{Manju Basic Character Set\label{table:ManjuBasicChars}}
\end{center}
\end{table}

While the input method for the majority of characters matches the
transliteration conventions, some letters require a slightly
different treatment:
\begin{enumerate}
	\item Although the diphtong \mabosoo{*aii*} is
		usually rendered as \textit{ai}, it must be entered
		as \texttt{aii} in order to produce the desired
		effect.
	\item The vowel which is conventionally rendered as \textit{\^u}
		or \textit{\=u} \mabosoo{v} can be entered as \texttt{v}
		or as \verb|\={u}| due to the fact that a character
		\textit{\^u} is not readily available on most systems.
	\item The consonant \textit{\v s} \mabosoo{s'} can be entered as
		\texttt{s'} or as \verb|\v{s}|, but not as *\texttt{sh}
		as to avoid undesired mergers of \textit{s} and \textit{h}
		like in \textit{ishun} \mabosoo{ishun} which should not be
		*\textit{i\v{s}un} \mabosoo{is'un}!
\end{enumerate}


\subsection{Small Portions of Mongolian and Manju in Running Text}

For displaying short Mongolian snippets in running text
use 
\begin{itemize}
	\item [MLS Romanization] \cmd{bcg}\verb|{...}|.
	\item [Simplified Transliteration] \cmd{bicig}\verb|{...}|.
\end{itemize}

For displaying short Manju snippets in running text
use \cmd{bithe}\verb|{...}|.

\exa
	This is \bicig{munggul bicik}. 

	That is \bithe{manju bithe}.
\exb
	\begin{verbatim}
	This is \bicig{munggul bicik}.
	
	That is \bithe{manju bithe}.
	\end{verbatim}
\exc

\subsection{Horizontal Paragraphs of Mongolian or Manju Text}

If one needs more than a few words of Mongolian or Manju but does
not want to change the line orientation, then the environments
\cmda{bicigtext} for Mongolian (which should be entered in
Mongolian Simplified Transliteration) and \cmda{bithetext} for Manju are
useful.

\exa
	\begin{bicigtext}
	uindur gegen zanabazar.

	17..18 d'ugar zagun-u munggul-un
	neiigem, ulus tuiru, shasin-u uiiles-tu,
	ilangguy=a uralig-un kuikzil-du uncukui
	ekurge kuiicedgeksen uindur gegen
	zanabazar, cingkis xagan-u aldan
	urug-un izagur surbulzidan abadai
	saiin nuyan xan-u kuiu tuisiyedu xan
	gumbudurzi-yin ger-tu 1635 un-du
	tuiruksen.%
	\end{bicigtext}
\exb
	{\mdoublehyphenon
	\begin{verbatim}
	\begin{bicigtext}
	uindur gegen zanabazar.

	17..18 d'ugar zagun-u munggul-un
	neiigem, ulus tuiru, shasin-u
	uiiles-tu, ilangguy=a uralig-un
	kuikzil-du uncukui ekurge
	kuiicedgeksen uindur gegen
	zanabazar, cingkis xagan-u
	aldan urug-un izagur surbulzidan
	abadai saiin nuyan xan-u kuiu
	tuisiyedu xan gumbudurzi-yin
	ger-tu 1635 un-du tuiruksen.
	\end{bicigtext}
	\end{verbatim}}
\exc


\exa
	\begin{bithetext}
	han-i araha sunja
	hacin-i hergen kamciha
	manju gisun-i buleku
	bithe. abkai so\v{s}ohon.
	emu hacin. nadan meyen.%
	\end{bithetext}
\exb
	\begin{verbatim}
	\begin{bithetext}
	han-i araha sunja
	hacin-i hergen kamciha
	manju gisun-i buleku
	bithe. abkai so\v{s}ohon.
	emu hacin. nadan meyen.%
	\end{bithetext}
	\end{verbatim}
\exc

\subsection{Vertical Capsules}

Individual Mongolian and Manju words can be placed vertically
anywhere in otherwise horizontal text like in
the keyword entry of dictionaries.\footnote{Famous dictionaries with
	a mixture of vertical and horizontal printing are I.~J.~Schmidt's
	Mongolian-Russian-German dictionary (1835) and F.~Lessing's
	Mongolian-English dictionary (1960).}.\marginpar{%
\mbosoo{mongGol}\mbosoo{bicig}

		\vspace{2.54mm}

		\raggedright\small
		without PostScript support Mongolian text enclosed in
		vertical capsules will be printed \emph{horizontally}!}
The capsule containing the
Mongolian or Manju word will automatically request sufficient space
so that ugly overlaps with neighbouring lines will not happen.

For presenting text given in broad (or MLS) transliteration, use the command
\cmd{mbosoo}\verb|{...}|; when writing in Mongolian Simplified
Transliteration, use \cmd{mobosoo}\verb|{...}|; likewise for Manju, use
\cmd{mabosoo}\verb|{...}|. All these commands are derived from a
command \cmd{bosoo}\verb|{...}| which places text in vertical
capsules but leaves the contents untouched as far as the encoding is
concerned.

\exa
	This is \bosoo{vertical}
		\bosoo{text}. 
	This is \mbosoo{mongGol}
		\mbosoo{bicig}, 
	this is \mobosoo{munggul}
		\mobosoo{bicik}, 
	that is \mabosoo{manju}
		\mabosoo{bithe}.
\exb
	\begin{verbatim}
	This is \bosoo{vertical}
		\bosoo{text}. 
	This is \mbosoo{mongGol}
		\mbosoo{bicig}, 
	this is \mobosoo{munggul}
		\mobosoo{bicik}, 
	that is \mabosoo{manju}
		\mabosoo{bithe}.
	\end{verbatim}
\exc


\subsection{Vertical Text Boxes}

For presenting individual paragraphs of Mongolian or Manju text in
vertical manner in an otherwise horizontal text, there are the box
commands \cmd{mobox}\verb|{...}{...}| for Mongolian%
\footnote{Mongolian input \emph{must} be coded in Mongolian Simplified
	Transliteration; MLS input won't work.}
and
\cmd{mabox}\verb|{...}{...}|
for Manju. These boxes take two arguments. The first argument
indicates the \textit{vertical depth} of the box, or its line
length. The second argument contains the desired text.

\exa
	\mobox{7.5cm}{%
%	uindur gegen zanabazar.
%
	17..18 d'ugar zagun-u munggul-un
	neiigem, ulus tuiru, shasin-u uiiles-tu,
	ilangguy=a uralig-un kuikzil-du uncukui
	ekurge kuiicedgeksen uindur gegen
	zanabazar, cingkis xagan-u aldan
	urug-un izagur surbulzidan abadai
	saiin nuyan xan-u kuiu tuisiyedu xan
	gumbudurzi-yin ger-tu 1635 un-du
	tuiruksen.%
	}
\exb
	%\vskip-9cm
	{\mdoublehyphenon
	\begin{verbatim}
	\mobox{7.5cm}{%
	uindur gegen zanabazar.

	17..18 d'ugar zagun-u munggul-un
	neiigem, ulus tuiru, shasin-u
	uiiles-tu, ilangguy=a uralig-un
	kuikzil-du uncukui ekurge
	kuiicedgeksen uindur gegen
	zanabazar, cingkis xagan-u
	aldan urug-un izagur surbulzidan
	abadai saiin nuyan xan-u kuiu
	tuisiyedu xan gumbudurzi-yin
	ger-tu 1635 un-du tuiruksen.%
	}
	\end{verbatim}}
\exc


\exa
	\mabox{3.75cm}{%
	\noindent\raggedleft han-i araha sunja
	hacin-i hergen kamciha
	manju gisun-i buleku
	bithe. abkai so\v{s}ohon.
	emu hacin. nadan meyen.%
	}
\exb
	%\vskip-5.25cm
	\begin{verbatim}
	\mabox{3.75cm}{%
	\raggedleft han-i araha sunja
	hacin-i hergen kamciha
	manju gisun-i buleku
	bithe. abkai so\v{s}ohon.
	emu hacin. nadan meyen.%
	}
	\end{verbatim}
\exc

\subsection{Full Vertical Text Pages\label{BicigandBithe}}

If you need several pages of Mongolian output, enclose your text
in an evironment \cmda{bicigpage}, and use \cmda{bithepage}
likewise for Manju texts. Note that Mongolian must be entered in
Simplified Transliteration.

Finally, if you want the whole document and its basic language to be
Classical, or Uighur Mongolian, say \verb|\usepackage[bicig,...]{mls}|.
Likewise, complete Manju documents are produced with 
\verb|\usepackage[bithe,...]{mls}|.

If you start a document with a \verb|\usepackage[bicig]{mls}|
declaration you can still switch back to Latin by issuing an
\verb|\end{bicigpage}| command.

Likewise, if you start a document with a \verb|\usepackage[bithe]{mls}|
declaration you can still switch back to Latin by issuing an
\verb|\end{bithepage}| command.

The following snippet of Mongolian text is presented in full
page mode on the next pages, first in Simplified Transliteration form, then
in Uighur form; in order to achieve this result the text had to be
included in the environment \texttt{bicigpage}.

\noindent
\begin{figure}
\begin{verbatim}
\begin{bicigpage}
uindur gegen zanabazar.

17||18 d'ugar zagun-u munggul-un neiigem, ulus tuiru, shasin-u
uiiles-tu, ilangguy=a uralig-un kuikzil-du uncugui ekurge
kuiicedgeksen uindur gegen zanabazar, cingkis xagan-u aldan
urug-un izagur surbulzidan abadai saiin nuyan xan-u kuiu
tuisiyedu xan gumbudurzi-yin ger-tu 1635 un-du tuiruksen.
badu muingke dayan xagan-u 6-d'aki uiy=e-yin kuimun. gurban
nasudai-d'agan num ungsizu enedkek gazar tuibed kele-yi xar=a
ayandagan surcu, keuked axui cag-aca erdem num-un duiri-tei
bulugsan zanabazar 15 nasu-tai-dagan baragun zuu (lhasa)
uruzu tabudugar dalai lam=a-d'u shabilan saguzu, ulamar
zebCundamba-yin xubilgan tudurazei. uran barimalci, zirugaci,
kele sinzigeci, uran barilgaci, kuin uxagandan zanabazar ulan
zagun zil-un daiin tululdugan-d'u nerbekden suliduzu, zugsunggi
baiidal-d'u urugsan dumdadu zagun-u munggul-un suyul uralig-i
serkun manduxu-d'u yeke xubi nemekuri urugulugsan yum. tekun-u
abiyas bilig nuiri yeke kuidelmuri-ber munggul-un uralig nigen
uiy=e tanigdasi uigei uindurlik-tu kuiruksen azei. xarin 1654
un-d'u neiislel kuiriyen-u tulg=a-yin cilagu-yi tabilcagsan
zanabazar-un uran barilg=a-yin buidugel-ece uinudur-i uizeksen
zuiil barug uigei ni xaramsaldai. zanabazar uindesun-u bicig
uisuk-i kuikzikulku-d'u beyecilen urulcazu, suyungbu uisuk-i
zukiyazu ene uiy=e suyungbu ni man-u tusagar tugdanil-un belge
temdek bulugsagar baiin=a.  tere-ber <<cag-i tukinagulugci>>
gedek silukleksen zukiyal-d'agan arad tuimen-u-ben engke
amugulang, saiin saiixan-i imagda kuisen muirugedezu yabudag
sedkil-un-iien uige-i ilerkeiileksen baiidag. uindur gegen
duirsuleku uralig-un xubi-d'u uirun=e-yin sunggudag-ud-tai
eng zergeceku buidugel-tei kuimun abacu basa xari ulus-un
buzar bacir arg=a-d'u abdagdan yabugsan nigen.
...
... more text ...
              ...
\end{bicigpage}
\end{verbatim}
\caption{Input Example of a Mongolian text}
\end{figure}

%%%%%%%%%%%%%%%%%%%%%%%%%%%%%%%%%%%%%%%%%%%%%%%%%%%%%%%%%%%%%%%%
\begin{bicigpage}
	uindur gegen zanabazar.

	17||18 d'ugar zagun-u munggul-un neiigem, ulus tuiru,
	shasin-u uiiles-tu, ilangguy=a uralig-un kuikzil-du
	uncugui ekurge kuiicedgeksen uindur gegen zanabazar,
	cingkis xagan-u aldan urug-un izagur surbulzidan abadai
	saiin nuyan xan-u kuiu tuisiyedu xan gumbudurzi-yin
	ger-tu 1635 un-du tuiruksen.  badu muingke dayan xagan-u
	6-d'aki uiy=e-yin kuimun. gurban nasudai-d'agan num
	ungsizu enedkek gazar tuibed kele-yi xar=a ayandagan
	surcu, keuked axui cag-aca erdem num-un duiri-tei
	bulugsan zanabazar 15 nasu-tai-dagan baragun zuu
	(lhasa) uruzu tabudugar dalai lam=a-d'u shabilan
	saguzu, ulamar zebCundamba-yin xubilgan tudurazei. uran
	barimalci, zirugaci, kele sinzigeci, uran barilgaci,
	kuin uxagandan zanabazar ulan zagun zil-un daiin
	tululdugan-d'u nerbekden suliduzu, zugsunggi
	baiidal-d'u urugsan dumdadu zagun-u munggul-un suyul
	uralig-i serkun manduxu-d'u yeke xubi nemekuri
	urugulugsan yum. tekun-u abiyas bilig nuiri yeke
	kuidelmuri-ber munggul-un uralig nigen uiy=e tanigdasi
	uigei uindurlik-tu kuiruksen azei. xarin 1654 un-d'u
	neiislel kuiriyen-u tulg=a-yin cilagu-yi tabilcagsan
	zanabazar-un uran barilg=a-yin buidugel-ece uinudur-i
	uizeksen zuiil barug uigei ni xaramsaldai. zanabazar
	uindesun-u bicig uisuk-i kuikzikulku-d'u beyecilen
	urulcazu, suyungbu uisuk-i zukiyazu ene uiy=e suyungbu
	ni man-u tusagar tugdanil-un belge temdek bulugsagar
	baiin=a.  tere-ber <<cag-i tukinagulugci>> gedek
	silukleksen zukiyal-d'agan arad tuimen-u-ben engke
	amugulang, saiin saiixan-i imagda kuisen muirugedezu
	yabudag sedkil-un-iien uige-i ilerkeiileksen
	baiidag. uindur gegen duirsuleku uralig-un xubi-d'u
	uirun=e-yin sunggudag-ud-tai eng zergeceku buidugel-tei
	kuimun abacu basa xari ulus-un buzar bacir arg=a-d'u
	abdagdan yabugsan nigen.

	munggul-d'u urcigulxu uxagan yeke delgerezu baiigsan ni
	man-u erden ba dumdadu uiy=e-yin suyul-un nigen uncalig
	azei. erden-u enedkek-un kuin uxagan-u iragu naiirag,
	kele bicik-un sudulul, anagaxu uxagan, uralaxu uxagan
	zerge tabun uxagan-u zukiyal-i bagdagagsan buikude 334
	budi <<ganzuur>>, <<danzuur>>-i num-un mergen bagsi
	kuinggaudsar terikudei 64 erdemden lama urcigulun
	neiideleksen baiin=a. 400 zil-un terdege urcigulg=a-yin
	iimu eke kuiriyeleng munggul-d'u azillazu baiigsan-i
	tuisugelen buduxu-d'u baxadai. munggulcud erden-ece
	inagsi daguu xugur-tai buizik nagadum-tai xurdun
	kuiluk murid-tai. er=e-yin gurban nagadum-i erkimelen
	kuikzilduzu, ide xabu-ban bulgazu ireksen baiin=a.
	munggul-un zirgalang ni buizik, xurim bile. xudala-i
	xagan-d'u erkumzileged xurxunag-un sagalagar mudun-u
	duur=a xabirg=a gazar-i xalcaradal=a, ebuduk gazar-i
	uilduredel=e debkecen buiziklezu xurimlaba gesen uige
	<<niguca tubciyan>>-d'u bui. munggul arad-un medelge
	uxagan erde-ece inagsi mal azu axui, udun urun,
	gazar zuii, anagaxu uxagan, baiigali, neiigem-un
	ulan salburi-bar kuikzizu irebe.  <<aldan tubci>>,
	<<erdeni-yin tubci>>, <<bulur tuli>>, <<subud erike>>
	medu teuke-yin ulan arban zukiyal gargazei.

	manzu nar <<munggul uyun>>-i muikugeku-yi kedui-ber
	uruldubacu uyun bilikdu, cecen celmek, erdem uxagandan
	tuduran garugsagar baiiba.  19-d'uger zagun bul
	iragu naiiragci dangzirabzai. yeke zukiyalci inzinasi
	dangzigvangzil nar-un amidurazu, buidugezu baiigsan
	uiy=e bile.
\end{bicigpage}

\subsection{Pure Uighur Mongolian and Manju Documents}

Writing a complete document in Mongolian or Manju is as simple and
straightforward as writing a document in English or Xalx Mongolian.

The example file, \texttt{zanabazr.tex}
(shipped together with this documentation and located in the
directory \texttt{../examples/}) demonstrates how a pure
Mongolian Bicig document can be created.

\begin{figure}[h]
\begin{verbatim}
\documentclass{article}
\usepackage[bicig]{mls}
\begin{document}
uindur gegen zanabazar.

17||18 d'ugar zagun-u munggul-un neiigem, ulus tuiru,
shasin-u uiiles-tu, ilangguy=a uralig-un kuikzil-du
...
... more text ...
              ...
\end{document}
\end{verbatim}
\end{figure}

The concept is the same for Manju documents: instead of \cmda{bicig}
one would use the \verb|\usepackage[...]{mls}| option \cmda{bithe}
and enter Manju text.

\subsection{Font Selection Commands}

There are two distinct styles of Mongolian script: one style
is typically used for modern print, whereas the other style
appears in old block prints and stone inscriptions.

Since there is no proper equivalent between Latin and Mongolian
typographical features, a somewhat arbitrary assignment was made
to the effect that the block print style can be activated by
setting the font family sans serif with \cmd{sffamily}. In 
contrast, setting the roman default family with \cmd{rmfamily}
switches back to the modern style.

\exa
	\mobox{2cm}{%
	\parindent=0pt
	\par
	munggul\\
	\sffamily munggul\\
	\rmfamily munggul
	}
\exb
	\begin{verbatim}
	\mobox{2cm}{\noindent
	munggul\\
	\sffamily munggul\\
	\rmfamily munggul}
	\end{verbatim}
\exc

\end{document}
