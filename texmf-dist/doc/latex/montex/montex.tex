%%%%%%%%%%%%%%%%%%%%%%%%%%%%%%%%%%%%%%%%%%%%%%%%%%%%%%%%%%%%%%%%
%        File: montex.tex
%      Author: Oliver Corff
%        Date: \VersionDate July 1st, 2002
%     Version: \VersionRelease
%   Copyright: Ulaanbaatar, Beijing, Berlin
%
% Description: MonTeX -- Mongolian for LaTeX2e
%              Implementation Level \ImplementationLevel
%              System Documentation
%
%%%%%%%%%%%%%%%%%%%%%%%%%%%%%%%%%%%%%%%%%%%%%%%%%%%%%%%%%%%%%%%%
% ----------------- identification ends here -------------------
%
\documentclass[11pt,a4paper]{report}
\usepackage{longtable}
%\usepackage{pslatex} % Buggy for Cyrillic and Special Characters
\IfFileExists{ctib}{%
	\usepackage{ctib}
}{}
\usepackage[latin1]{mls}
\input{mtdocmac.tex}
\usepackage{hyperref}
%
% If emtex goofs with printer-related memory overflow
% when attempting to print this document then set the
% following number to "1", recompile and increase the
% number step by step until all examples are printed.
% The maximum value is 6.
%
% It is safe to set the FontSamples value to 4 on
% emtex systems.
%
\newcounter{FontSamples}
\setcounter{FontSamples}{6} % <--- Modify this number!
%
% Make sure your installation includes ec sources
% for Dunhill and Funny typefaces otherwise LaTeX
% will complain about font files not found.
%
%\nonstopmode
%%
%% Meaningful text of document montex.tex begins here
%%
\title{%
	\ifthenelse{\value{PostScriptAvailable}=1}
	{\mabosoo{manju}\mabosoo{bithe,}\quad
	 \mbosoo{mongGol}\mbosoo{bicig tai}\mbosoo{lateh}\\[0.75cm]
	}{%
	 \textit{manju bithe, mong\g ol bicig-t�i lateh}\protect\footnote{%
	 	Instead of the transliterated text, the same text
		should appear in vertical Mongolian writing. Since
		\MonTeX\ did not find PostScript support on your
		system, it decided to print this text horizontally
		and in transliteration. It is suggested that you
		update your \LaTeX\ system by adding PostScript
		support (see section~\ref{PostScriptSupport}).}\\}
	\xalx{Kirill "Us"agt"a"i La\TeX}\\\bigskip
	\MonTeX\\
	Mongolian and Manju\\
	for\\\LaTeXe\\\bigskip
	Implementation Level \ImplementationLevel\\
	v.~\VersionRelease}

\author{Oliver Corff\protect\thanks{The author received
			a DFG (Deutsche Forschungsgemeinschaft)
			postdoctoral grant (Habilitationsstipendium) while
			developing the first version of \MonTeX\ and
			expresses his sincere gratitude.}
	\and
	Dorjpalam Dorj\protect\thanks{Institute for Language and
			Literature, Mongolian Academy of Sciences,
			Ulaanbaatar.}}

\date{\VersionDate}

\begin{document}
%\onecolumn
%\layout
\maketitle
\begin{abstract}
	\MonTeX\ is now released in Implementation Level
	\ImplementationLevel\
	implying that there is not only Cyrillic Support for
	Modern Mongolian
	(Xalx dialect), Buryat and Russian; this version also includes
	comprehensive support for Mongolian Script (also known as
	Uighur Script) and Manju. All writings can be mixed freely
	within the same document, and within the same page.

	The present release (\VersionRelease) is still very early;
	expect bugs and irregular behaviour. Especially the Mongolian
	full page mode still shows occasional quirks, depending on
	the document class loaded.
\end{abstract}

\tableofcontents
\listoftables
\listoffigures

%%%%%%%%%%%%%%%%%%%%%%%%%%%%%%%%%%%%%%%%%%%%%%%%%%%%%%%%%%%%%%%%
%%%%%%%%%%%%%%%%%%%%%%%%%%%%%%%%%%%%%%%%%%%%%%%%%%%%%%%%%%%%%%%%
\part{\MonTeX: System Overview}


%%%%%%%%%%%%%%%%%%%%%%%%%%%%%%%%%%%%%%%%%%%%%%%%%%%%%%%%%%%%%%%%
\chapter{How to read and use the \MonTeX\ documentation}

According to your specific needs, you can read various parts
of this documentation separately.
\begin{description}
	\item	[First-time] users interested in generating
		\emph{Cyrillic-only} documents can confine
		themselves to the sections beginning on
		part~\ref{GeneralSettingsCyrillicInput},
		page~\pageref{GeneralSettingsCyrillicInput}.

	\item	[First-time] users primarily interested in creating
		\emph{Mongol Bicig} and \emph{Manju} documents
		or text insertions in these languages can
		directly jump to section~\ref{part:BicigandBithe},
		page~\pageref{part:BicigandBithe}.

	\item	[Advanced] users interested in improved
		\emph{Mongolian} and \emph{Manju} display functions
		should directly go to 
		section~\ref{chapter:DisplayCommands},
		page~\pageref{chapter:DisplayCommands}.

	\item	[Advanced] users already familiar with the majority
		of \MonTeX\ functions can refer to the Alphabetic
		Command Reference beginning on
		page~\pageref{CommandReference}.
\end{description}

\textbf{Nota Bene:} Please note that \MonTeX\ includes the complete
functionality of Manju\TeX\ which is hereby declared obsolete.
Manju\TeX\ will not longer be supported.


%%%%%%%%%%%%%%%%%%%%%%%%%%%%%%%%%%%%%%%%%%%%%%%%%%%%%%%%%%%%%%%%
\chapter{System Features}

\section{Scope of \MonTeX}

\MonTeX\ is a package which offers support for writing documents in
Mongolian, Manju, Buryat and Russian.  Mongolian can be represented
in traditional Uighur script (also known as Classical or Traditional
Script) and Cyrillic. Manju resembles the Traditional Mongolian script
(from which it is derived) but uses a rich choice of diacritics in
order to eliminate numerous ambiguities of the Mongolian script
ancestor. Modern Buryat, like Mongolian in its present form, is written
with a Cyrillic alphabet, but both Mongolian (35 letters) and Buryat
(36 letters) use more letters than Russian (33 letters).

\subsection{Mongolian}

The word \emph{Mongolian} is actually an umbrella term for
several languages rather than the precise name of a single language.
Things become more complicated when names of ethnic groups, languages
and writing systems are mixed.
\begin{description}
	\item [Xalx] or Khalkha is the name of the Mongolian
		nationality residing in Mongolia proper. Their
		dialect forms the basis of Mongolian written with
		Cyrillic letters. Throughout this text, \emph{Modern
		Mongolian} is used as a synonym.
	\item [Buryat] 
		is the name of the Mongolian nationality
		residing in Buryatia, north of Mongolia, east of
		Lake Baikal, being a part of the Russian Federation.
		The Buryat call themselves \emph{Buryaad} while Xalx
		Mongolians call them \emph{Buriad}. The English name
		follows the Russian orthography.
		Linguistically, Xalx and Buryat Mongol are fairly
		close languages; Buryat has a slightly different
		sound system in which the phoneme /s/ partially
		shifted to /h/; the modern Buryat Cyrillic alphabet
		(virtually identical with the Cyrillic alphabet used
		for writing Modern Mongolian) has one additional
		letter (H/h, \xalx{H/h}) for marking the difference to /s/.
		\enlargethispage*{1ex}
	\item [Bicig] (literally \emph{script} in Mongolian) denotes
		text written in the traditional Mongolian script
		which is also referred to as Uighur. Throughout this
		document, the term \emph{Bicig} will be used on an
		equal footing with \emph{Classical} and
		\emph{Traditional} Mongolian. The latter term is
		used in the name of the Unicode/ISO10646 character
		plane U1800 which contains Mongolian, Manju, Sibe
		and sets of special characters called Ali Gali or
		Galig. In order to identify Mongolian script related
		commands distinct for Mongolian and Manju, the
		Mongolian commands have the name root \texttt{bicig}
		whereas the Manju commands have the name root
		\texttt{bithe}.
\end{description}

Xalx Mongolian, or Modern Colloquial Mongolian, is about as
different from the form written in Classical script as modern
English in phonetical spelling (assume it be written in Shavian
letters) from the highly historical orthography of Standard English.
Beyond these differences, Mongolian written in Classical Script
usually preserves a substantial amount of historical grammatical
features which make it look a bit like Elizabethan English.


\subsection{Manju}

Manju is a Tungusic language closely related to Mongolian. Though
Manju is virtually not spoken anymore, it has been the official
language during 300 years of Manju government in Qing Dynasty China.
Vast amounts of official documents survive, as well as some of the
finest multilingual dictionaries ever compiled, e.\,g. the
Pentaglot, or Mirror in Five Languages, a dictionary with 18671
entries in five languages (Manju, Tibetan, Mongolian, Uighur and
Chinese). Manju writing is derived from Uighur Mongolian by adding
diacritics in the form of dots and circles.

\section{\MonTeX\ Implementation Levels}

During several years of developing \MonTeX, the desired capabilities
of a software package serving the described scope were classified and
implemented along four Implementation Levels which have the following,
well-defined properties:
\begin{center}
\begin{tabular}{rl}
Implementation&	Features\\
	Level & \\\hline
I	&     Modern Mongolian in Cyrillic Script and Buryat\\
II	& I   and Mongolian script in LR mode horizontal\\
IIa	& II  and Mongolian script portions in LR mode vertical\\
III	& II  and Mongolian script text in horizontal RL mode\\
IV	& III and Mongolian script text printed vertically \\
\end{tabular}
\end{center}

Implementation Level I is good for producing documents in Modern
Mongolian; Implementation Level II adds support for words and lines of
Classical Mongolian embedded in other (Cyrillic and Latin) scripts which
is essentially useful for dictionaries etc.; Implementation Level IIa
allows single words to be placed in vertical capsules; Implementation
Level III allows the composition of purely Classical Mongolian documents
while Implementation Level IV finally allows the combination of both
scripts in freely assignable quantities and locations within the document.

Mongolian linguistic culture provides a perplexing richness of
writing systems of varying regional, historical and socio-political
importance. Developing a Mongolian system which covers \emph{all}
documented writing systems is tantamount to writing a Mongolian
Babel system and cannot be done in a truly elegant manner with
respect to the current \LaTeXe\ limitations.

% Full ISO 10646/Unicode character set support is one desideratum, the
% free switching of all sorts of writing directions in the document is
% another cornerstone of success. 

\section{Requirements and Limitations}

In order to run \MonTeX\ a recent version of \LaTeXe\ is necessary.
\MonTeX\ relies on the NFSS font selection mechanism and the
ligature capacities of Metafont. This package has not been tested
under \LaTeX2.09 and will most certainly not function satisfactorily
under that environment. Depending on the implementation level,
further software support becomes necessary since not all features
can be realized smoothly in \LaTeXe\ alone.

\begin{center}
\begin{tabular}{r|l}
Implementation Level & Requirements \\\hline
I	& A working \LaTeXe\ system\\
II	& A working \LaTeXe\ system\\
IIa	& like II, plus PostScript support\\
III	& like II, plus functional \texttt{TeX--XeT} system\\
IV	& like III, plus PostScript support \\
\end{tabular}
\end{center}

The e\TeX\ (available for DOS and UNIX based computers alike) system
provides full Right-to-Left writing support; e\TeX\ and e\LaTeXe\ are
part of all modern \TeX\ implementations for the majority of
operating systems\footnote{%
	It is also possible to build an e\TeX\ system from scratch
	using the \texttt{web2c} (or teTeX) sources, replacing
	\texttt{tex.web} with \texttt{tex--xet.web} and \texttt{tex.ch}
	with \texttt{tex--xet.ch}.}.

Post\-Script support is a standard feature of most UNIX installations
and is also supplied with most of the available \TeX\ for Windows
distributions%
	\footnote{The authors used Linux (Red Hat 4.2 through 7.1) systems
		for the developing work; on the same hardware, PostScript
		under Windows\emph{xx} is significantly slower
		than under Linux; this holds true for document and
		font compilation as well.}.

A word of warning is necessary here. \MonTeX\ is not a small,
convenient system which can be used without any effort. Much like
its very foundation \LaTeX\ it requires some willingness to study
a few (and indeed simple) rules; occasionally one or the other old
habit has to be overcome. The reward is text typeset in excellent
quality so that scholarly achievement no longer disappears in badly
typeset documents.


\section{PostScript Support\label{PostScriptSupport}}

PostScript is used for creating vertical capsules of text within
horizontal text for \MonTeX\ implementation levels IIa as well as
for complete pages with implementation level IV. This requires the
presence of the \texttt{rotating} package for \LaTeX\ which itself
relies on the \texttt{graphics} package. The \texttt{rotating} and
\texttt{graphics} packages come with teTeX but do not come with
emtex. They can be found at CTAN.

Besides the above-mentioned packages it is necessary that the
generated \texttt{.dvi} files can be processed further, e.\,g. by
\texttt{dvips} which generates a PostScript file out of a
\texttt{.dvi} file. If there is no PostScript printer at your site,
PostScript emulation is necessary which is usually provided by
GhostScript and GhostView. Implementations of these systems are
available for a large number of operating systems and can also be
found at CTAN. Linux, a free UNIX system, comes which GhostScript,
and the \texttt{winemtex} distribution of \LaTeX\ includes GhostScript
as well.
A sample command sequence to produce and preview a document with these
utilities can be found in illustration~\ref{figure:RunExampleTeTeX}.

\begin{figure}
\begin{verbatim}
$ elatex montex.tex     # Compile document
$ dvips montex.dvi      # Create PostScript out of DVI
$ gv montex.ps          # Preview document
\end{verbatim}
\caption{PostScript Compilation and Preview Cycle}\label{figure:RunExampleTeTeX}
\end{figure}

Without PostScript support, only implementation level II can be
realized (instead of IIa and IV): Mongolian script can be printed
horizontally but not vertically. It must be noted here that most DVI
viewers are \emph{not} capable of presenting vertical text
correctly; the conversion step from DVI to PostScript is virtually
always necessary.


\section{PDF Support\label{PDFScriptSupport}}

With the arrival of PDF\TeX\ it is possible to generate a PDF
(Adobe's Portable Document Format, that is) directly from the
\texttt{.tex} sources without going through the \texttt{.dvi}
stage. All systems offering \texttt{pdfelatex} can be used to
compile a \MonTeX\ Implementation Level IV document. provided the
necessary PostScript is installed. PDF is the recommended form of
output on systems without PostScript views and printers (like,
unfortunately, most of the Windows\textit{xx} world). PDF documents
reproduce everything \MonTeX\ generates as is, and with Type1 fonts
for Classical Mongolian, the on-screen display of Classical
Mongolian and Manju material is fast and pleasant.

A sample command sequence to produce and preview a document with PDF
output can be found in illustration~\ref{figure:RunExamplePDF}.

\begin{figure}
\begin{verbatim}
$ pdfelatex montex.tex     # Compile document
$ acroread montex.pdf      # View PDF Document with Acrobat Reader
\end{verbatim}
\caption{PDF Compilation and Preview Cycle}\label{figure:RunExamplePDF}
\end{figure}

It is recommended to users with menu-driven environments (WinEdt,
TeXnicCenter, TeXshell etc.) to set the compilation commands to the
effect that \texttt{pdfelatex} is invoked as the default compilation
engine, and the Acrobat Reader is invoked as the default viewer.
Please consult the software documentation of these products for the
necessary steps and procedures.
%%%%%%%%%%%%%%%%%%%%%%%%%%%%%%%%%%%%%%%%%%%%%%%%%%%%%%%%%%%%%%%%
\chapter{Acknowledgements}

The authors wish to thank the creators of \TeX\ and \LaTeX\ as well
as the designers of the existing fonts for their generosity of
providing the world with such inspiring pieces of software. The packages
from which pieces of code originated by inspiration or blunt copy
are far too numerous; the Russian captions were taken from the file
\texttt{russian.sty} (as were the English captions), most of the
Cyrillic letters were produced with fonts by Nana Glonti and Alexander
Samarin; additional letters were taken from J.~Knappen's font files.
Special thanks go to David Carlisle who offered the solution for a
serious problem with the ligature mechanism in \TeX. During the
development of Implementation Level IV, important suggestions came
from David Kastrup, Robin Fairbairns, Dan Luecking, e.\,a. Intensive
communications about Cyrillic fonts and integrating \MonTeX\ with
the LH fonts took place with Vladimir Volovich, and other problems
were discussed with Werner Lemberg.

Among the friends and colleagues in Mongolia and Germany who offered
information, support and encouragement the authors wish to name
B.~Nerguy, Urgamal, M.~Balk, Q.~\"Anxzayaa and K.~Maezono (without
implication of any particular order or precedence). They contributed
test data as well as their ideas for encoding, font shapes, user
interfaces, and, last but not least, were patient alpha testers
who helped the authors with numerous problem reports.

Many of the improvements between version 0.1 and the present version
are not actually improvements; they are simply eliminations of
partially awful bugs as well as ugly hacks (rather than
\emph{code}) and aim to make this package simply usable (if not
useful).


\section{Sources of Code and Inspiration}

Some Cyrillic packages have been available for a few years. All
Cyrillic packages available for \TeX/\LaTeX\ stem from one of
two lines of ancestry:
\begin{itemize}
	\item Fonts developed at the University of Washington
	\item Fonts by Nana Glonti and Alexander Samarin
\end{itemize}
The two lines differ substantially in scope of characters and
printing quality. The University of Washington series in \textsf{OT2}
encoding has a broad support for East European languages, but the
praise for printing quality is given to the characters designed by
Nana Glonti and Alexander Samarin. The Glonti/Samarin line of
characters has undergone numerous minor modifications, not so much
in glyph shapes but basically in determining encoding slots.
Fortunately, the fonts are set up in a way that allows for
convenient redefining of individual code positions.

Only in 1999, a comprehensive set of Cyrillic glyphs in various
encodings called LH was finally implemented as standard Cyrillic
support for \LaTeXe, but at that time it was decided that for the
time being \MonTeX\ will continue to offer its own Cyrillic font
set, for which there are mainly three reasons:
\begin{enumerate}
	\item	The \MonTeX\ set has a seven-bit basis and allows
		for the all-Latin, all-ASCII communication of
		Mongolian texts, while Mongolian hyphenation
		is active---a matching LH encoding is not yet
		established;

	\item	\MonTeX\ offers a Mongolian currency sign \Togrog\
		which is not yet included in the LH fonts;

	\item	Mongolia, on of the prime markets for \MonTeX,
		continues to use partially outdated \LaTeXe\
		installations.
		
\end{enumerate}

Neither original line of Cyrillic characters offers the additional
characters necessary for writing non-Slavic languages like
Mongolian; already in the beginning of the 1990s, J\"org Knappen
filled the gap and designed additional letters which were intended
to be used with Bashkir, a Turkic language. In fact, most of the
letter forms employed there can also be used in other non-Slavic
languages used throughout Central Asia since these letters are not
specific to Bashkir. Some of J.~Knappen's letter forms (accidentally
mostly those which are not necessary for writing Modern Mongolian)
do need some refinement, and are then immediately suitable for a
range of languages including Kasakh, Tuvinian etc. In the present
stage of the system, only those letters used in Mongolian and Buryat
are incorporated from J.~Knappen's files.

\begin{sloppypar}
After discussing the typeface issues with Mongolian specialists, the
Glonti/Samarin letter forms were chosen for their superior appearance in
volume text. The fonts had to be renamed; failing to do so would
have resulted in unpermissible ambiguity.
\end{sloppypar}

One feature of the traditional Cyrillic font packages for \TeX\ (besides
their lacking support for non-Slavic languages) is the intimate relationship
between input encoding and output encoding. The first step in
building Mongolian support was to separate these two spheres as
numerous Mongolian encodings exist which should all be supported by
the Mongolian package. A new encoding was then defined (\LMC\ ---
Local Mongolian Cyrillic) which is a close approximation of a
transliteration based on Latin1 encoding, notably with front vowels
\"a, \"o, \"u (\xalx{"a, "o, "u}) and \"i\ (\xalx{"i}) in matching
positions. 

The encoding is completely detached from the existing Cyrillic
codepages of which there are too many; in addition it should be
possible to produce Mongolian documents in 7-bit environments so as
to assure maximum document portability.

An additional ligature table for Metafont was then supplied which
takes care of most of the two-letter combinations necessary for
entering Cyrillic since the Cyrillic alphabet has more letters
(36 in the present version) than the Latin alphabet which prohibited
any 1:1-mapping scheme.

The used transliteration is very closely modelled after the MLS
system yet provides enough transparency for accepting alternative
spellings in some cases.


%%%%%%%%%%%%%%%%%%%%%%%%%%%%%%%%%%%%%%%%%%%%%%%%%%%%%%%%%%%%%%%%
\chapter{Input and Output Encodings}

\section{The Need for Encodings}

Any Mongolian text system has to deal with the issues of how to store,
transmit, process and represent the following entities:
\begin{itemize}
	\item	Normal Latin letters, numbers and punctuation marks:
	a, b, c, etc.;
	\item	Cyrillic letters, including those not present in
		basic Cyrillic but needed by Mongolian: \xalx{a, b,
		w, \"o, \"u};
	\item	Special symbols like the Mongolian Currency sign: \Togrog;
	\item	Classical Mongolian script;
	\item	Special symbols used in Latin scripts for purposes
		of transliterating Mongolian scripts: \"a, \"o, \"u,
		\g\ e.\,a.
\end{itemize}

All these sets of symbols, alphabets and characters have their own
unique properties, especially when it comes to non-Latin writings
like Mongolian or Tibetan.

Unfortunately, prior to the arrival of Unicode, all computer systems
based on 8-bit encodings (with a maximum character set of 256
characters) can only represent subsets of the above-mentioned
entities as basic characters. All computer systems with 8-bit
character encodings must either switch between several character
sets (or code pages) or use non-standard commands to invoke
individual character entities.

It is important to understand that the issue of how to enter all
these characters is more or less completely detached from the issue
of how to represent these characters on screen or in a document.
Misleadingly, the usage of Latin characters in modern computers
seems to suggest that there is a simple, 1:1 relationship (or
mapping) between input and output, but for a number of languages,
including Mongolian, this is simply not the case. While, due to the
origin and history of computers, simple-minded systems do not make
any difference between the two realms, \TeX\ separates these two
domains clearly, allowing for the amazing flexibility \TeX\ shows
when treating languages and writing systems.

It must also be understood that even though Unicode allows for the
unambiguous representation of the characters and symbols of the
world's major languages, it does not define any output conventions,
and thus, input and output domains should still be treated as
separate areas.

\section{Input Encodings}

\MonTeX\ is flexible enough to deal with several kinds of input
encodings including code pages with Cyrillic letters and Unicode.
Input encodings are declared as an option to the main package in the
document preample. E.\,g., a user working on an IBM compatible DOS
platform is likely to specify the option \cmda{mls}:

\begin{verbatim}
\usepackage[mls]{mls}
\end{verbatim}

\subsection{7-bit ASCII and Mongolian}

Basically it is possible to use \MonTeX\ without anything else but
the plain 7-bit ASCII Latin character set since internal and external
mechanisms are available which can render transliterated texts (both
Cyrillic and Traditional Mongolian) into their appropriate script
presentations.

\subsection{The MLS Codepage}

The MLS codepage was the ancestor of all comprehensive,
IBM-compatible Mongolian systems which intended to cover both
Cyrillic and Classical Mongolian. Developed in the early 1990s, the
MLS system tried to offer full Mongolian support for existing
hardware and software as it was available then. While modern
technological developments have confined the original approach to
history, it is preserved here for preserving backward compatibility.

The MLS codepage is compatible with the IBM 437 codepage as far as
the front vowels are concerned but features additional Cyrillic
letters and Classical Mongolian.

\subsection{8-bit Encodings}

Most available 8-bit input encodings support either front vowels
or Cyrillic letters or Classical Mongolian but usually not several
of them at the same time.

If a local environment supports Cyrillic and Script codepages
then texts can be composed using these codepages. Table
\ref{inputenc} shows which codepages are supported. Those codepage
names which are followed by a `(+)' are supplied by \MonTeX\ whereas
the other codepage declarations are recognized and passed through to
the system assuming that the appropriate table exists. The column
``Front Vowels'' indicates whether the vowels \"o and \"u (and their
Mongolian counterparts \xalx{"o, "u}) are available in that particular
codepage. \MonTeX\ recognizes both numbers and numbers preceded by
\texttt{cp}, like \texttt{1250} and \texttt{cp1250} as names of
codepages which are known by their number.

\begin{table}
\begin{center}
\begin{tabular}{l|c|c|c|c|c|c|c|c|c|c}
Enc. Option&\multicolumn{7}{|c|}{Latin Transliteration Symbols}%
			&Cyrillic&\multicolumn{2}{|c}{Front Vowels}\\\hline
\rule{0mm}{2.25ex}%
	&\"A/\"a&\"O/\"o&\"U/\"u&\"E/\"e&\"I/\"i
					&\v C/\v c&\v S/\v s
						& &\xalx{"O/"o}&\xalx{"U/"u}\\
	\hline
\texttt{mls} (+) &+	&+	&+	&\"e&\"\i &- &- &+	&+	&+ \\
\texttt{ncc} (+) &-	&-	&-	&-  &- &- &- &+	&+	&+ \\
\texttt{mos} (+) &-	&-	&-	&-  &- &- &- &+	&+	&+ \\
\texttt{mnk} (+) &-	&-	&-	&-  &- &- &- &+	&+	&+ \\
\texttt{dbk} (+) &-	&-	&-	&-  &- &- &- &+	&+	&+ \\
\texttt{ctt} (+) &-	&-	&-	&-  &- &- &- &+	&+	&+ \\
\texttt{ibmrus}(+)&-	&-	&-	&-  &- &- &- &+	&-	&- \\
\texttt{koi} (+) &-	&-	&-	&-  &- &- &- &+	&-	&- \\
\texttt{437}	&+	&+	&+	&\"e&\"\i&-&-&-	&-	&- \\
\texttt{437de}	&+	&+	&+	&\"e&\"\i&-&-&-	&-	&- \\
\texttt{850}	&+	&+	&+	&+  &+ &- &- &-	&-	&- \\
\texttt{852}	&+	&+	&+	&+  &- &- &- &-	&-	&- \\
\texttt{865}	&+	&+	&+	&\"e&\"\i&-&-&-	&-	&- \\
\texttt{1250}	&\"A	&+	&\"u	&+  &- &- &- &-	&-	&- \\
\texttt{1252}	&+	&+	&+	&+  &+ &- &- &-	&-	&- \\
\texttt{applemac}&+	&+	&+	&+  &+ &- &- &-	&-	&- \\
\texttt{mac}	&+	&+	&+	&+  &+ &- &- &-	&-	&- \\
\texttt{ansinew}&+	&+	&+	&+  &+ &- &+ &-	&-	&- \\
\texttt{ascii}	&-	&-	&-	&-  &- &- &- &-	&-	&- \\
\texttt{atari}	&+	&+	&+	&+  &\"I&-&- &-	&-	&- \\
\texttt{decmulti}&+	&+	&+	&+  &+ &- &- &-	&-	&- \\
\texttt{isolatin}&+	&+	&+	&+  &\"I&-&- &-	&-	&- \\
\texttt{latin1}	&+	&+	&+	&+  &+ &- &- &-	&-	&- \\
\texttt{latin2}	&+	&+	&+	&+  &- &+ &+ &-	&-	&- \\
\texttt{latin3}	&+	&+	&+	&+  &+ &- &- &-	&-	&- \\
\texttt{latin5}	&+	&+	&+	&+  &+ &- &- &-	&-	&- \\
\texttt{next}	&+	&+	&+	&+  &+ &- &- &-	&-	&- \\
\texttt{pc850}	&+	&+	&+	&+  &\"I&-&- &-	&-	&- \\
\texttt{roman8}	&+	&+	&+	&+  &\"I&-&- &-	&-	&- \\
\end{tabular}
\caption{\MonTeX\ Input Encodings}\label{inputenc}
\end{center}
\end{table}

%
\subsection{utf-8 Unicode}

In summer of 2002, a new input encoding was made available for
existing \LaTeXe\ installations which allows the processing of
utf8-encoded Unicode material. This package can be invoked
with the option \cmda{utf8}:

\begin{verbatim}
\usepackage[utf8]{mls}
\end{verbatim}

There are some caveats, however. The relevant code is still under
development, and at present, \MonTeX\ only deals with the Mongolian,
Manju and Sibe subsets of the Traditional Mongolian Character Plane
beginning at U1800; a Todo character set remains to be implemented,
and some of the more arcane special characters present in Unicode
are as yet unavailable in \MonTeX. The resulting constraints do not
affect the work with contemporary text material and are only felt
when dealing with frequently bilingual, mostly Tibetan and Sanskrit,
religious texts of earlier centuries. Consult chapter~\ref{Unicode}
on page~\pageref{Unicode} and table~\ref{table:UnicodeMongolian}
for details.

%%%%%%%%%%%%%%%%%%%%%%%%%%%%%%%%%%%%%%%%%%%%%%%%%%%%%%%%%%%%%%%%
\section{Output Encodings}

Several output encodings are defined for \MonTeX:

\begin{description}
	\item [\LMC] Local Mongolian Cyrillic: This encoding was
		defined in order to avoid collisions with existing
		Cyrillic encodings for \TeX\ and \LaTeX. \LMC\ is
		a 7-bit encoding which implies that most of its 
		characters are addressed in the range of ordinary
		ASCII characters; when this encoding is active, all
		text typed in ASCII Latin characters will
		automatically appear in Cyrillic. Unlike some other
		available 7-bit encodings (like WN Cyrillic) it
		provides characters used in Mongolian.
	\item [\LMA] Local Manju: Manju in Ligature Mode. Any text
		typed in romanized Manju is automatically converted
		into Manju characters. \LMA\ acts thus like a
		typical 7-bit encoding.
	\item [\LMO] Local Mongolian: Similar to Manju in Ligature
		Mode, Mongolian in Ligature Mode is typed in a
		special romanized form and is then automatically
		converted into Uighur Mongolian characters. \LMO, too,
		acts thus like a typical 7-bit encoding.
	\item [\LMS] Local Mongolian Script: The system's original
		encoding for the Mongolian script. Mongolian is
		represented by a Latin transliteration the letters
		of which are essentially treated as future Mongolian
		canonical code positions. Once Mongolian Unicode will
		be available, the Latin transliteration can be
		seamlessly replaced by Mongolian canonical characters.
		The arrangement of code positions in this encoding
		does not reflect Unicode but follows the MLS
		system's keyboard support.
	\item [\LMT] Local Mongolian-Tibetan: This encoding is reserved
		to ensure access to the characters in the future \Zanabazar\
		package: \Soyombo\footnote{%
			It is possible to use the Soyombo package
			available since 1996 as long as \Zanabazar\
			is not available.}
			 and \XD. It is designed to comprise
		Tibetan as well, and Sirlin's Tibetan fonts can be 
		directly used with this encoding.
	\item [\LMX] Local Mongolian \XD: This encoding is used for
		the \XD\ Script (available on CTAN) but is not
		frozen yet. Individual code positions are still
		subject to change.
	\item [\LMU] Local Mongolian Superset (\textsf{U} stands for
		`Umbrella'', ``Unknown'', or whatever you like to
		pick): This encoding is used to access all glyphs 
		of the \texttt{bxg} glyph container, but is not
		frozen yet. Individual code positions are still
		subject to change.
\end{description}


\section{\MonTeX\ and Recent \TeX\ Trends}

As soon as the LH Cyrillic fonts support the Mongolian currency sign,
\MonTeX\ will switch to this font set. At the moment the
private encoding \LMC\ is favoured over LH; future implementations
of \MonTeX\ will provide a smooth transition for the user: documents
developed with older versions of \MonTeX\ will be upward compatible.

The \texttt{babel} package will, perhaps, also be supported in due
course; at the moment, \texttt{babel} support is lacking mainly due
to font encoding questions and a private RL setup. At present,
\MonTeX\ is \emph{not} built with \texttt{babel} compatibility in mind.
It must be seen as a stand-alone extension similar to
\texttt{german.sty} or the \textsf{CJK} package.

The future belongs to 16-bit character sets; the first \TeX\
development supporting larger character sets is $\Omega$mega of
which experimental versions exist. One of the great features of
$\Omega$mega is the capability to process canonical input encodings
in order to generate glyph variants for document presentation. These
so-called translation processes are far more powerful than anything
Metafont can offer via ligatures, and they are the only feasible way
to avoid external preprocessors or internal retransliteration engines
coded in \TeX\ needed to process Mongolian script.\footnote{%
	The retransliteration engine provided with the \LMS\
	encoding of \MonTeX\ has
	a rather `combined' approach; basic letter forms are selected
	in the retransliteration section while typical ligatures
	are composed with the ligature tables of Metafont. The authors
	express their sincerest gratitude to David Carlisle who
	contributed the missing link between characters in the output 
	list and \TeX/Metafont's ligature mechanism.}
Prof.~Lagally's Arab\TeX\ is the only \LaTeX\ package known to the
authors where an extensive retransliteration engine is realized
as pure \TeX\ code; it is an impressive piece of work defying any
simple-minded imitation. So far, $\Omega$mega translation processes
exist for Tibetan and Arabic (paragons of complex relations between
original script and any attempted romanization).
%%%%%%%%%%%%%%%%%%%%%%%%%%%%%%%%%%%%%%%%%%%%%%%%%%%%%%%%%%%%%%%%
%%%%%%%%%%%%%%%%%%%%%%%%%%%%%%%%%%%%%%%%%%%%%%%%%%%%%%%%%%%%%%%%


%%%%%%%%%%%%%%%%%%%%%%%%%%%%%%%%%%%%%%%%%%%%%%%%%%%%%%%%%%%%%%%%
\chapter{Installation}

Before this latest version of \MonTeX\ is installed please make sure
that old installations of \MonTeX\ \emph{and} Manju\TeX\ are purged
from disk as there are file name conflicts between earlier and
recent versions of this software. In addition, Manju\TeX\ \emph{is
not required} any more as its functionality is now completely
covered by \MonTeX.

\section{Hyphenation Patterns}

\MonTeX\ provides hyphenation rules for Modern Mongolian (Xalx).
Hyphenation patterns for English are activated with English as
selected language; hyphenation patterns for Russian exist at CTAN
but they are unfortunately not suited for \MonTeX\ withour prior
work. Hyphenation patterns for Buryat have not been developed yet.

Due to the very nature of \TeX, hyphenation patterns for a given
language cannot easily be loaded at run-time but must be compiled into
a so-called format file which gets loaded by \TeX\ whenever the
command \texttt{latex} is executed. A format file is usually
created when a new \TeX\ or \LaTeXe\ system is installed, but creating
a new format can be done at any later time again. A special variant
of \TeX\ called \texttt{initex} is used for this purpose.
The procedure sounds more intimidating than it actually is.
Since there are many different types of \TeX\ installations, the 
procedure is somewhat system-dependent. There is detailed on-line
documentation available for performing this task, either in form of
a text file for emtex, or in form of a FAQ file which can be
displayed using the command \texttt{texconfig faq} on teTeX systems.

%%%%%%%%%%%%%%%%%%%%%%%%%%%%%%%%%%%%%%%%%%%%%%%%%%%%%%%%%%%%%%%%
%%%%%%%%%%%%%%%%%%%%%%%%%%%%%%%%%%%%%%%%%%%%%%%%%%%%%%%%%%%%%%%%
\part	[General Settings and Cyrillic Input]
	{User Commands I\\General Settings\\Cyrillic Input%
	\label{GeneralSettingsCyrillicInput}}


%%%%%%%%%%%%%%%%%%%%%%%%%%%%%%%%%%%%%%%%%%%%%%%%%%%%%%%%%%%%%%%%
\chapter{Introduction}

With regard to the substantial differences between Latin-like scripts
(including Cyrillic) and Mongolian scripts, the user documentation
of \MonTeX\ is divided into two parts. This part deals with general
settings, like language choices and input encoding definitions,
whereas the commands specific to Mongolian and Manju are dealt with
in part~III, ``Mongol Bicig and Manju Bithe''. An alphabetic command
reference covering \emph{all} commands is presented in part~IV.

\section{General Settings}

In order to access the commands of \MonTeX\ the package must be
loaded in the document preamble by saying

\begin{verbatim}
\usepackage[<language options>,<encoding options>]{mls}
\end{verbatim}

The options include choices for the basic document language and
input encodings.

\subsection{Document Language}

The document language can be set with one of
	\verb"bicig",
	\verb"bithe",
	\verb"buryat",
	\verb"english",
	\verb"russian" or
	\verb"xalx"
like in
\begin{verbatim}
\usepackage[xalx]{mls}
\end{verbatim}
which issues all captions and the date in Modern Mongolian.

The options \refcmda{bicig} and \refcmda{bithe} are discussed extensively 
in part~\ref{part:BicigandBithe}, ``Mongol Bicig and Manju Bithe''.

The options \cmda{buryat} (see table~\ref{Buryatcaptions}),
	\cmda{russian} (see table~\ref{Russiancaptions})
	and
	\cmda{xalx} (see table~\ref{Xalxcaptions}) produce 
	captions in Buryat, Russian and Modern Mongolian.

The option
\cmda{english}, at least as a \verb"\usepackage" option, is
essentially a do-nothing: it sets captions to English (which is
the default of this package anyway).

\begin{table}[h]
	{\captionsburyat
	\CaptionsList{Buryat}}
\end{table}

\begin{table}[h]
	{\captionsrussian
	\CaptionsList{Russian}}
\end{table}

\begin{table}[h]
	{\captionsxalx
	\CaptionsList{Xalx}}
\end{table}

The date form follows \TeX\ conventions and is thus a mixture of
numbers and words. Thus for \cmd{today} (\today) we get% 
	\footnote{The actual date at compilation time is
used for the examples.} what is shown in
table~\ref{caption:BXRDates}. The Uighur Mongolian and Manju dates
are presented in section~\ref{section:UMMDates},
page~\pageref{section:UMMDates}.

\begin{figure}[h]
\begin{center}
	\begin{tabular}{ll}
	    \textbf{Buryat}	&\BuryatToday\\\label{BuryatToday}
	    \textbf{Xalx}	&\XalxToday\\\label{XalxToday}
	    \textbf{Russian}	&\RussianToday\\\label{RussianToday}
	\end{tabular}
\caption{Dates in Buryat, Xalx and Russian}\label{caption:BXRDates}
\end{center}
\end{figure}

The language specifiers
\verb"buryat",
\verb"english",
\verb"russian" and
\verb"xalx" can
also be used anywhere in the document as arguments to the
\verb"\selectlanguage" command.
Instead of stating an argument to \verb"\usepackage[...]{mls}" it is
possible to say in your document
\begin{quote}
	\verb"\selectlanguange{xalx}"
\end{quote}
which would set captions to Xalx Mongolian.




%%%%%%%%%%%%%%%%%%%%%%%%%%%%%%%%%%%%%%%%%%%%%%%%%%%%%%%%%%%%%%%%
\chapter{Cyrillic Text -- \xalx{Kirill "us"ag}}

\section{Cyrillic Text in Transliteration (\LMC) Mode%
	\label{section:CyrillicTransliterationMode}}

\MonTeX\ provides two basic modes of operation: in
\begin{itemize}\label{SetDocumentEncoding}
	\item Transliteration Mode (intimately linked to the \LMC\
		encoding) all incoming text is regarded as
		transliterated Cyrillic. This allows users to
		compose Cyrillic documents on pure ASCII machines.
		In contrast, the
	\item Immediate Mode does nothing and waits for explicit
		Cyrillic characters in the input in order to generate
		Cyrillic output.
\end{itemize}
Two commands are used to switch between these modes:
\begin{quote}
\begin{verbatim}
\SetDocumentEncodingLMC
\SetDocumentEncodingNeutral
\end{verbatim}
\end{quote}

The first command switches to Transliteration Mode, the second
command deactivates the transliteration and thus, by definition,
activates Immediate Mode.

In the \LMC\ encoding, most Cyrillic characters are mapped directly to
a single Latin character but for some characters there is a text
command which became necessary since there are more Cyrillic than
Latin characters. For convenience, a few ligatures were defined, too.
Details are given in table~\ref{cyralpha}.

\begin{table}
\begin{center}
\begin{tabular}{|r|cc|cc|ll|}
\hline
%\multicolumn{7}{|c|}{Cyrillic Alphabet Input Methods} \\\hline
   &\multicolumn{2}{|c|}{Cyrillic Letter}&\multicolumn{2}{|c|}{\LMC\
Input}&\multicolumn{2}{|c|}{Generic Command}\\\hline
 1 &\mnr A &\mnr a &\verb"A" &\verb"a" &\verb"\CYRA" &\verb"\cyra" \\\hline
 
 2 &\mnr B &\mnr b &\verb"B" &\verb"b" &\verb"\CYRB" &\verb"\cyrb" \\\hline
 
 3 &\mnr W &\mnr w &\verb"W" &\verb"w" &\verb"\CYRV" &\verb"\cyrw" \\\hline
 
 4 &\mnr G &\mnr g &\verb"G" &\verb"g" &\verb"\CYRG" &\verb"\cyrg" \\\hline
 
 5 &\mnr D &\mnr d &\verb"D" &\verb"d" &\verb"\CYRD" &\verb"\cyrd" \\\hline
 
 6 &\mnr E &\mnr e &\verb"E" &\verb"e" &\verb"\CYRE"&\verb"\cyre" \\\hline
 
 7 &\CYRYO &\cyryo &\texttt{\"E}/\verb'"E'&\texttt{\"e}/\verb'"e'&%
 			\verb"\CYRYO"&\verb"\cyryo"\rule{0mm}{2.25ex}\\ 
   &       &       &\{\verb"\"\}\verb"YO"&\{\verb"\"\}\verb"yo"&  & \\\hline

 8 &\mnr J &\mnr j &\verb"J" &\verb"j" &\verb"\CYRZH" &\verb"\cyrzh" \\\hline
 
 9 &\mnr Z &\mnr z &\verb"Z" &\verb"z" &\verb"\CYRZ" &\verb"\cyrz" \\\hline

10 &\mnr I &\mnr i &\verb"I" &\verb"i" &\verb"\CYRI" &\verb"\cyri" \\\hline

11 &\CYRISHRT &\cyrishrt &\texttt{\"I}/\verb'"I'&\texttt{\"i}/\verb'"i'&%
			\verb"\CYRISHRT"&\verb"\cyrishrt"\rule{0mm}{2.25ex} \\
   &       &       &\{\verb"\"\}\verb"YI"&\{\verb"\"\}\verb"yi"&  & \\\hline

12 &\mnr K &\mnr k &\verb"K" &\verb"k" &\verb"\CYRK" &\verb"\cyrk" \\\hline

13 &\mnr L &\mnr l &\verb"L" &\verb"l" &\verb"\CYRL" &\verb"\cyrl" \\\hline

14 &\mnr M &\mnr m &\verb"M" &\verb"m" &\verb"\CYRM" &\verb"\cyrm" \\\hline

15 &\mnr N &\mnr n &\verb"N" &\verb"n" &\verb"\CYRN" &\verb"\cyrn" \\\hline

16 &\mnr O &\mnr o &\verb"O" &\verb"o" &\verb"\CYRO" &\verb"\cyro" \\\hline

17 &\CYROTLD &\cyrotld &\texttt{\"O}/\verb'"O'&\texttt{\"o}/\verb'"o'&%
			\verb"\CYROTLD"&\verb"\cyrotld" \rule{0mm}{2.25ex}\\\hline

18 &\mnr P &\mnr p &\verb"P" &\verb"p" &\verb"\CYRP" &\verb"\cyrp" \\\hline

19 &\mnr R &\mnr r &\verb"R" &\verb"r" &\verb"\CYRR" &\verb"\cyrr" \\\hline

20 &\mnr S &\mnr s &\verb"S" &\verb"s" &\verb"\CYRS" &\verb"\cyrs" \\\hline

21 &\mnr T &\mnr t &\verb"T" &\verb"t" &\verb"\CYRT" &\verb"\cyrt" \\\hline

22 &\mnr U &\mnr u &\verb"U" &\verb"u" &\verb"\CYRU" &\verb"\cyru" \\\hline

23 &\mnr "U &\mnr "u &\texttt{\"U}/\verb'"U'&\texttt{\"u}/\verb'"u'&%
			\verb"\CYRY"&\verb"\cyry" \rule{0mm}{2.25ex}\\\hline

24 &\mnr F &\mnr f &\verb"F" &\verb"f" &\verb"\CYRF" &\verb"\cyrf" \\\hline

25 &\mnr X &\mnr x &\verb"X" &\verb"x" &\verb"\CYRH" &\verb"\cyrh" \\\hline

26 &\mnr H &\mnr h &\verb"H" &\verb"h" &\verb"\CYRHSHA" &\verb"cyrhsha" \\\hline

27 &\mnr C &\mnr c &\verb"C" &\verb"c" &\verb"\CYRC" &\verb"\cyrc" \\\hline

28 &\mnr Q &\mnr q &\verb"Q" &\verb"q" &\verb"\CYRCH" &\verb"\cyrch" \\
   &       &       &\verb"\Ch"&\verb"\ch"&           &             \\\hline

29 &\mnr\Sh&\mnr\sh&\verb"\Sh"&\verb"\sh"&\verb"\CYRSH"&\verb"\cyrsh"\\
   &       &       &          &\verb"sh" &             &             \\\hline

30 &\mnr\Sc&\mnr\sc&\verb"\Sc"&\verb"\sc"&\verb"\CYRSHCH"&\verb"\cyrshch"\\
   &       &       &\verb"\Qh"&\verb"\qh"&             &             \\\hline

31 &\mnr \CYRHRDSN &\mnr \cyrsftsn &\verb"\Y" &\verb"\y" &%
			\verb"\CYRHRDSN" &\verb"\cyrhrdsn" \\\hline

32 &\mnr Y &\mnr y &\verb"Y" &\verb"y" &\verb"\CYRERY" &\verb"\cyrery" \\\hline

33 &\mnr \CYRSFTSN &\mnr \cyrsftsn &\verb"\I" &\verb"\i" &%
			\verb"\CYRSFTSN" &\verb"\cyrsftsn" \\\hline

34 &\CYREREV &\cyrerev &\texttt{\"A}/\verb'"A'&\texttt{\"a}/\verb'"a'&%
			\verb"\CYREREV"&\verb"\cyrerev" \rule{0mm}{2.25ex}\\\hline

35 &\mnr YU&\mnr yu&\{\verb"\"\}\verb"YU"&\{\verb"\"\}\verb"yu"&%
			\verb"\CYRYU"&\verb"\cyryu"\\\hline

36 &\mnr YA&\mnr ya&\{\verb"\"\}\verb"YA"&\{\verb"\"\}\verb"ya"&%
			\verb"\CYRYA"&\verb"\cyrya"\\\hline
\end{tabular}
\caption{Cyrillic Alphabet Input Methods}\label{cyralpha}
\end{center}
\end{table}

Front vowels can be entered directly using the encoding slot of a
valid and active input encoding, or they can be expressed via an
abbreviated \verb'"'\emph{v} notation where \emph{v} stands for any
desired vowel. In the \LMC\ encoding used by \MonTeX, \verb'"' is not
an active character; selecting the proper letter is done by ligature
statements in the Metafont sources.
 
Some letters can be entered with or without a preceding \verb"\",
like \cyryu\ and \cyrya. Both \verb"\yu" and \verb"yu" will produce
a \cyryu. While \verb"yu" is interpreted as a ligature, \verb"\yu"
allows for the character \cyryu\ to be combined with accents.
Accents are not commonly used in Mongolian since there are precise
rules for word stress. This feature is taken from the \textsf{OT2} encoding 
and is included mainly for the sake of completeness, convenience and
compatibility\footnote{The magic triple-C!}.

Here now a sample of Mongolian text:
\begin{figure}[h]
\exa
	{\mnr<<Xalxyn gurwan "ond"or>> 
	x"am"a"an aldarshsan, Z"u"un xyazgaaryg
	toxinuulax sa"id N.~Dugarjaw ardyn
	xuw\i sgalyn b"u"ur "ax"an "ue"as
	xamgi"in "agz"agt"a"i am\i\ d"u"is"an
	alband tomilogdox c"ar"ag da"iny olon
	quxal daalgawryg xiq"a"ang"u"il"an
	biel"u"ulj yawsan t"u"uxt"a"i x"un.}%
\exb
\begin{verbatim}
	{\mnr<<Xalxyn gurwan "ond"or>> 
	x"am"a"an aldarshsan, Z"u"un xyazgaaryg
	toxinuulax sa"id N.~Dugarjaw ardyn
	xuw\i sgalyn b"u"ur "ax"an "ue"as
	xamgi"in "agz"agt"a"i am\i\ d"u"is"an
	alband tomilogdox c"ar"ag da"iny olon
	quxal daalgawryg xiq"a"ang"u"il"an
	biel"u"ulj yawsan t"u"uxt"a"i x"un.}%
\end{verbatim}%
\exc
\caption{Romanized Cyrillic Input Example}\label{figure:CyrInputExample}
\end{figure}

In order to make the document you are reading at the moment truly
portable, the somewhat more clumsy \verb'"'\emph{v} notation was
used in this example; if your environments supports an 8-bit
codepage (what it usually does), all front vowels can be entered
as \texttt{\"a}, \texttt{\"o} and \texttt{\"u} etc. using the
slots of those vowels in the particular active codepage.\footnote{%
	Looking at the source code of this document the astute
	reader will discover that all front vowels are indeed 
	produced using the \tttt{ "a} (etc.) notation; thus
	the document source can be viewn and manipulated on any
	7-bit ASCII platform; it can also safely be transmitted
	via e-mail.}



\section{Entering Cyrillic Text in Immediate Mode}

For freely combining Latin and Cyrillic characters without
using any explicite commands it is necessary that the codepage
in use supports some Cyrillic encoding. It should be noted,
however that these documents are not easily portable between
different platforms anymore since they need recoding; some
of the Cyrillic codepages are defective in one or the other
way thus individual characters can get lost.

\begin{sloppypar}
The user simply specifies the desired input encoding as a
\verb:\usepackage[:\emph{<encoding>}\verb:]{mls}: option, and
\MonTeX\ takes care of the rest.
It is a feature and not a bug that input encoding and document
language are chosen independently. It is well possible that a
user working on a computer with default Mongolian codepage
wants to create a document in Russian, English or any other
language yet wants to include Mongolian fragments in her text
without explicitely issuing any command.
\end{sloppypar}

In case a need arises for switching from Transliteration Mode
to Immediate Mode the command \label{cmd:SetDocumentEncodingNeutral}
can be issued anywhere in the preamble or the document itself;
like \verb"\SetDocumentEncodingLMC"\label{cmd:SetDocumentEncodingLMC} it affects the Cyrillic
transliteration only and leaves the document language
in its chosen state.


\section{Entering Cyrillic Characters by Name}

Outside the Cyrillic environments, individual Cyrillic characters
can be entered by using the commands beginning with
\verb"\cyr"\textit{x} from the two right columns of table
\ref{cyralpha} where \textit{x} stands for the letter name.
This command works in any encoding.

\section{Entering Special Cyrillic Characters\label{section:SpecialCyrCharacters}}

A few special characters are available, notably the guillemots
frequently used for quoting text, the currency symbol, the
ordinal number symbol and the currency sign. See
table~\ref{table:SpecialCyrCharacters}.

\begin{table}[h]
\begin{center}
\begin{tabular}{cll}
	Symbol	&	Command			& Alternative\\
	\mnr\lgu&	\verb|\lgu|		& \verb|<<|\\
	\mnr\rgu&	\verb|\rgu|		& \verb|>>|\\
	\No	&	\verb|\No|		& \\
	\Togrog	&	\verb|\Togrog|		& \\
	\togrog	&	\verb|\togrog|		& \\
\end{tabular}
\caption{\MonTeX\ Special Cyrillic Characters}\label{table:SpecialCyrCharacters}
\end{center}
\end{table}

The command producing the guillemots (\verb|\lgu|, \verb|\rgu|)
\emph{only} works in a Cyrillic environment --- it is not a generic
command.

There are actually two versions of the \cmd{togrog} command. While
\cmd{Togrog} produces a sans serif \Togrog\ (considered standard)
with any font selected it is also possible to print
	serif (\MyTogrog),
	italic (\textit{\MyTogrog})
and typewriter (\texttt{\MyTogrog}) versions of this
symbol.\footnote{The currency symbol is not limited to these three
	typefaces; all typefaces can be selected.}
For achieving this result the commands \cmd{MyTogrog} and \cmd{mytogrog} 
are available. Unlike the standard command they simply pick the
current font style of the surrounding letters for the currency
symbol.

\section{Running Text with Embedded Words in Different Encodings}

Independently of the document language it is possible to produce
portions of Cyrillic text within Latin text and vice versa.
The two commands \cmd{mnr} and \cmd{rnm} switch from ordinary
Latin text to transliterated Cyrillic text and back to Latin text.
The command stands for \emph{m}ongolian \emph{n}ew \emph{r}omanization
and its reversal (which can, by accident, also be read as
\emph{r}eturn to \emph{n}or\emph{m}al). They can be used as stream
commands or for initializing groups:\label{mnrnm}

\exa
	\mnr mongol x"al ba \rnm english
	text with a {\mnr mongol} word
	inserted 
\exb
	\begin{verbatim}
	\mnr mongol x"al ba \rnm english
	text with a {\mnr mongol} word
	inserted 
	\end{verbatim}
\exc

For enhanced convenience, portions of text can also be encapsulated
into the commands \cmd{xalx}\verb"{...}" for Cyrillic text and
\cmd{lat}\verb"{...}" for neutral (i.~e. Latin) texts.\label{capsules}

The commands \verb"\mnr", \verb"\rnm", \verb"\xalx{...}" and
\verb"\lat{...}" do \emph{not} switch the default encoding; this 
shows up when a construct like \verb:\lat{\verb|article|}: is placed
in Transliteration Mode; the result will be \texttt{\xalx{article}}
rather than \texttt{article}; in order to generate the desired form,
the mode switching commands must be used.


\section{Font Selection Commands}

The Cyrillic fonts are set up in a manner which allows for seamless
switching between Roman and Cyrillic typefaces. The font switching
commands used for modifying typefaces (by \verb"\text..") are
completely transparent to the encoding; no precaution whatsoever has
to be taken.  Most of the
typefaces supplied with the traditional \textsf{OT1} encoding are
also available for \MonTeX; Dunhill and Funny Roman are included.\footnote{%
	A complete overview of the NFSS classification of the
	Computer Modern fonts can be found in The \LaTeX\ Companion,
	by Michel Goossens, Frank Mittelbach and Alexander Samarin,
	Addison-Wesley 1994, p.~181.}
\MonTeX\ offers the following font families as shown in
table~\ref{MonTeXFontFamilies}:
\begin{table}
\begin{center}
\begin{tabular}{p{3cm}|l}
\verb'\fontfamily{...}' Parameter & Family Description \\
\hline
	cmr  & Computer Modern Serif\\
	cmss & Computer Modern Sans Serif\\
	cmtt & computer Modern Typewriter\\
	cmvtt& Computer Modern Variable Width Typewriter\\
	cmfr & Computer Modern Funny\\
	cmfib& Computer Modern Fibonacci\\
	cmdh & Computer Modern Dunhill\\
	cmssq& Computer Modern Sans Serif Quotation Style 8pt\\
\end{tabular}
\end{center}
\caption{Font Families Supported by \MonTeX}\label{MonTeXFontFamilies}
\end{table}

The word ``Roman'' was avoided since in \MonTeX\ these families
also cover matching typefaces in Cyrillic script. The first three
families have support for combinations of different weights and
shapes (e.\,g. bold and italic) whereas the other series usually
only offer an italic variant. The Sans Serif Quotation Style 8pt
typeface is not by default installed in standard \LaTeX\
distributions hence it cannot be guaranteed that switching to and
from Cyrillic letters maintains the typeface. The fonts (upright and
slanted) can be accessed via the \verb|\fontfamily{cmssq}| command
but are not shown in table~\ref{table:typeface}. See
table~\ref{table:typeface} for a therefore incomplete list of
available typeface examples.

\begin{table}
\begin{center}
\begin{tabular}{|p{0.75cm}p{7cm}|p{1.5cm}p{1.5cm}|}
\hline
\multicolumn{2}{|c|}{Family and Command Example}&
		\multicolumn{2}{c|}{Typeface Examples} \\
\hline
\multicolumn{2}{|c|}{Computer Modern Serif}&& \\
cmr	& (default)		&        {\Tw}&       {\WtSh}\\
	& \tttt{ textbf\{...\}}& \textbf{\Tw}&\textbf{\WtSh}\\
	& \tttt{ textsl\{...\}}& \textsl{\Tw}&\textsl{\WtSh}\\
	& \tttt{ textsc\{...\}}& \textsc{\Tw}&\textsc{\WtSh}\\
	& \tttt{ textit\{...\}}& \textit{\Tw}&\textit{\WtSh}\\
	& \tttt{ fontseries\{bx\}\char92 textit\{...\}}&
		\fs{bx}\textit{\Tw}&\fs{bx}\textit{\WtSh}\\
\hline
\end{tabular}

\ifthenelse{\value{FontSamples}>1}{%
\begin{tabular}{|p{0.75cm}p{7cm}|p{1.5cm}p{1.5cm}|}
\multicolumn{2}{|c|}{Computer Modern Typewriter}&& \\
cmtt	& \tttt{texttt\{...}\}& \texttt{\Tw}&\texttt{\WtSh}\\
	& \tttt{texttt\{\char92textit\{...\}\}}&
		\texttt{\textit{\Tw}}&\texttt{\textit{\WtSh}}\\
	& \tttt{texttt\{\char92textsl\{...\}\}}&
		\texttt{\textsl{\Tw}}&\texttt{\textsl{\WtSh}}\\
	& \tttt{texttt\{\char92textsc\{...\}\}}&
		 \texttt{\textsc{\Tw}}&\texttt{\textsc{\WtSh}}\\
\hline
\end{tabular}}{\rule{0mm}{4ex}Computer Modern Typewriter example suspended\dots}

\ifthenelse{\value{FontSamples}>2}{%
\begin{tabular}{|p{0.75cm}p{7cm}|p{1.5cm}p{1.5cm}|}
\multicolumn{2}{|c|}{Computer Modern Variable Width Typewriter}&& \\
cmvtt	& \tttt{ fontfamily\{cmvtt\}...}&
		\ff{cmvtt}\Tw&\ff{cmvtt}\Wt\\
	& \tttt{ fontfamily\{cmvtt\}\char92 textit\{...\}}&
		\ff{cmvtt}\textit{\Tw}&\ff{cmvtt}\textit{\Wt}\\
\hline
\end{tabular}}{\rule{0mm}{4ex}Computer Modern Variable Width
	Typewriter example suspended\dots}

\ifthenelse{\value{FontSamples}>3}{%
\begin{tabular}{|p{0.75cm}p{7cm}|p{1.5cm}p{1.5cm}|}
\multicolumn{2}{|c|}{Computer Modern Sans}&& \\
cmss	& \tttt{ textsf\{...\}}	&
		\textsf{\Tw}&\textsf{\WtSh}\\	
	& \tttt{ textsf\{\char92 textsl\{...\}\}}&
		\textsf{\textsl{\Tw}}&\textsf{\textsl{\WtSh}}\\	
	& \tttt{ textsf\{\char92 fontseries\{bx\}...\}}&
		\textsf{\fs{bx}\Tw}&\textsf{\fs{bx}\WtSh}\\
	& \tttt{ textsf\{\char92 fontseries\{sbc\}...\}}&
		\textsf{\fs{sbc}\Tw}&\textsf{\fs{sbc}\WtSh}\\
\hline
\end{tabular}}{\rule{0mm}{4ex}Computer Modern Sans example suspended\dots}

\ifthenelse{\value{FontSamples}>4}{%
\begin{tabular}{|p{0.75cm}p{7cm}|p{1.5cm}p{1.5cm}|}
\multicolumn{2}{|c|}{Computer Modern Funny}&& \\
cmfr	& \tttt{ fontfamily\{cmfr\}...}&
		\ff{cmfr}\Tw&\ff{cmfr}\Wt\\
\hline
\end{tabular}}{\rule{0mm}{4ex}Computer Modern Funny example suspended\dots%
	For printing the Modern Funny example
	the user is required to edit this file
	(\texttt{montex.tex}) at line 26 and set the value
	of \texttt{fontSamples} to ``5''.}

\ifthenelse{\value{FontSamples}>5}{%
\begin{tabular}{|p{0.75cm}p{7cm}|p{1.5cm}p{1.5cm}|}
\multicolumn{2}{|c|}{Computer Modern Dunhill}&& \\
cmdh	& \tttt{ fontfamily\{cmdh\}...}&
		\ff{cmdh}\TW&\ff{cmdh}\WT\\
\hline
\end{tabular}}{\rule{0mm}{4ex}Computer Modern Dunhill example suspended\dots%
	For printing the Modern Dunhill example
	the user is required to edit this file
	(\texttt{montex.tex}) at line 26 and set the value
	of \texttt{fontSamples} to ``6''.}

\caption{Typeface Consistency for Cyrillic and Latin}\label{table:typeface}
\end{center}
\end{table}

Besides these transparent commands for scalable fonts \MonTeX\ also
offers two inch-high variants of bold Computer Modern Sans typefaces
for Latin and Cyrillic: \cmd{cminch} and \cmd{kminch}.
These commands bypass the NFSS font setup and should only be used for
book titles etc. The command sequence \verb|{\cminch AB} {\kminch AB}|
produces the output shown in figure~\ref{cminchkminch}.

\begin{figure}
\vspace{5mm}
\begin{center}
{\cminch AB} {\kminch AB}
\end{center}
\vspace{5mm}
\caption{\texttt{inch} Font Examples}\label{cminchkminch}
\end{figure}

\section{Shorthands for Embedding Words in a Different
		Typeface}\label{typefacecapsules}

Sometimes it may be necessary to give short portions of text not
only in a different encoding (for which the \refcmd{lat}\verb:{...}:
and \refcmd{mnr}\verb:{...}: commands are
useful) but it may also be necessary to switch the typeface
temporarily. Usually capsules using \verb'\text'\emph{xx} do the
work if only the typeface is concerned, and building nested commands
like \verb'\textsf{\lat{...}}' is cumbersome if these changes have
to be applied very often. \MonTeX\ provides an abbreviated style
following the rule 
\begin{quote}
	\texttt{[k|l]}\emph{two letter font style code}\verb'{...}'
\end{quote}
where the font style code is one of
	\verb'rm',
	\verb'bf',
	\verb'it',
	\verb'sl',
	\verb'sf',
	\verb'sc' and
	\verb'tt',
like \verb'\ksl{...}', \verb'\lsc{...}', etc.


\section{Shorthands for Writing Transliterated Texts}

\MonTeX\ provides shortcuts for writing certain accented symbols
used in conventional transliterating of Mongolian by 
haceks, the nasal and the gamma. These shortcuts are essentially
mnemonics replacing the somewhat more tedious accent notation (see
table~\ref{table:shortcuts}).

\begin{table}
\begin{center}\begin{tabular}{ll|ll}
%\hline
Letter	& Input 	& Letter	& Input \\
	&		&		&\\
\hline
	&		&		&\\
\ch 	& \verb"\ch"	& \Ch		& \verb"\Ch" \\
\jh 	& \verb"\jh"	& \Jh		& \verb"\Jh" \\
\sh 	& \verb"\sh"	& \Sh		& \verb"\Sh" \\
\zh 	& \verb"\zh"	& \Zh		& \verb"\Zh" \\
\ng 	& \verb"\ng"	& \Ng		& \verb"\Ng" \\
\g	& \verb"\g"	& \G		& \verb"\G" \\
%\hline
\end{tabular}\end{center}
\caption{Shortcuts for Mongolian Transliteration Symbols}\label{table:shortcuts}
\end{table}

It must be observed that these commands are by default dependent on
the environment they are used in. \verb"\Sh" yields a \Sh\ when used
in a Latin environment but results in a \mnr\Sh\rnm\ when used in a
Cyrillic context\footnote{The authors wish to thank J.~Knappen for
resolving one instability in the original code for these letters.}:

\exa
	\emph{\Sh agdar} and \emph{\Ch adraa}
	are transliterations for
	{\mnr\Sh agdar} and {\mnr\Ch adraa}.
\exb
	\begin{verbatim}
	\emph{\Sh agdar} and \emph{\Ch adraa}
	are transliterations for
	{\mnr\Sh agdar} and {\mnr\Ch adraa}.
	\end{verbatim}
\exc


\section{Gamma Typeface}

If modern Greek is supported by your \LaTeXe\ installation then the
shape of the gamma will match the neighbouring typeface as closely
as possible%
\ifthenelse{\value{GreekGammaAvailable}=1}{
	\space as can be seen from table~\ref{table:typeface}%
	}%
	{};
\cmd{g} otherwise, the selection of gamma shapes and styles is limited
to the gamma math typeface supplied by standard \TeX\ installations.

%%  \ifthenelse{\value{GreekGammaAvailable}=1}{
%%  \begin{table}
%%  \begin{center}
%%  \begin{tabular}{rl}
%%  		& \ {\Sh a\g dur}\\
%%  \end{tabular}
%%  \end{center}
%%  \caption{Available Gamma Typefaces\label{greekgamma}}
%%  \end{table}}{}%
%%  

\section{Oirat Double Accents}

All accented characters which are contained in the \textsf{T1} encoding
or can be generated out of these via accents can be produced. This comes
in conveniently for transliterating Oirat texts which need vowels with
double diacritics, like \rule{0mm}{2.5ex}\={\"a} which can be entered as
any combination of two nested accent commands (like \verb:\={\"a}:)
or one accent command and a vowel with diacritics (provided an 8-bit
input codepage is available).


\section{Numbering by Cyrillic Letters}

Analogous to the \verb:\Alpha: command which provides an alphabetical
counter in English, \MonTeX\ features counters for Buryat, Modern
Mongolian, and Russian.
\begin{description}
	\item [Buryat]\label{Uzeg}
	The counter for Buryat is invoked with
	\cmd{Uzeg}\verb"{"\emph{n}\verb"}" or
	\cmd{uzeg}\verb"{"\emph{n}\verb"}" and is valid for $1\le n\le32$.

	\def\NR #1{$^{#1}$\Uzeg{#1}/\uzeg{#1}}
	\begin{tabular}{cccccc}
	\NR{1} &\NR{2} &\NR{3} &\NR{4} & \NR{5}&\NR{6} \\
	\NR{7} &\NR{8} &\NR{9} &\NR{10}&\NR{11}&\NR{12}\\
	\NR{13}&\NR{14}&\NR{15}&\NR{16}&\NR{17}&\NR{18}\\
	\NR{19}&\NR{20}&\NR{21}&\NR{22}&\NR{23}&\NR{24}\\
	\NR{25}&\NR{26}&\NR{27}&\NR{28}&\NR{29}&\NR{30}\\
	\NR{31}&\NR{32}\\
	\end{tabular}

	\item [Modern, or Xalx Mongolian]\label{Useg}
	The counter for Modern Mongolian is invoked with
	\cmd{Useg}\verb"{"\emph{n}\verb"}" or
	\cmd{useg}\verb"{"\emph{n}\verb"}" and is valid for $1\le n\le31$.

	\def\NR #1{$^{#1}$\Useg{#1}/\useg{#1}}
	\begin{tabular}{cccccc}
	\NR{1} &\NR{2} &\NR{3} &\NR{4} & \NR{5}&\NR{6} \\
	\NR{7} &\NR{8} &\NR{9} &\NR{10}&\NR{11}&\NR{12}\\
	\NR{13}&\NR{14}&\NR{15}&\NR{16}&\NR{17}&\NR{18}\\
	\NR{19}&\NR{20}&\NR{21}&\NR{22}&\NR{23}&\NR{24}\\
	\NR{25}&\NR{26}&\NR{27}&\NR{28}&\NR{29}&\NR{30}\\
	\NR{31}\\
	\end{tabular}

	\item [Russian]\label{Asbuk}
	The counter for Russian is invoked with
	\cmd{Asbuk}\verb"{"\emph{n}\verb"}" or
	\cmd{asbuk}\verb"{"\emph{n}\verb"}" and is valid for $1\le n\le28$.

	\def\NR #1{$^{#1}$\Asbuk{#1}/\asbuk{#1}}
	\begin{tabular}{cccccc}
	\NR{1} &\NR{2} &\NR{3} &\NR{4} & \NR{5}&\NR{6} \\
	\NR{7} &\NR{8} &\NR{9} &\NR{10}&\NR{11}&\NR{12}\\
	\NR{13}&\NR{14}&\NR{15}&\NR{16}&\NR{17}&\NR{18}\\
	\NR{19}&\NR{20}&\NR{21}&\NR{22}&\NR{23}&\NR{24}\\
	\NR{25}&\NR{26}&\NR{27}&\NR{28}\\
	\end{tabular}
\end{description}

%%%%%%%%%%%%%%%%%%%%%%%%%%%%%%%%%%%%%%%%%%%%%%%%%%%%%%%%%%%%%%%%
%%%%%%%%%%%%%%%%%%%%%%%%%%%%%%%%%%%%%%%%%%%%%%%%%%%%%%%%%%%%%%%%
\part	[Mongol Bicig and Manju Bithe]%
	{Mongol Bicig and Manju Bithe\\
	\mobosoo{munggul}\mbosoo{bicik,}\ \mabosoo{manju}\mabosoo{bithe.}%
	\label{part:BicigandBithe}}


%%%%%%%%%%%%%%%%%%%%%%%%%%%%%%%%%%%%%%%%%%%%%%%%%%%%%%%%%%%%%%%%
\chapter{Introduction}

This part describes in detail all aspects of typesetting Mongolian
and Manju with \MonTeX. The following sections cover the various
input methods for these languages, the commands for presenting small
snippets, big portions and whole documents composed in Mongolian and
Manju, as well as the relationship between input notations and
script-related commands.

\section{Mongolian and Manju Script Fundamentals}

Mongolian Script, or \emph{bicig}, is a writing with an intriguing
and complex relationship between the canonical letters of
the alphabet and their presentations in context. Virtually any canonical
letter can assume several shapes. As a rule of thumb, there are
three or four basic shapes: the letter in isolated form, the letter
in initial, medial and final position of a word. Only a few letters
stay the same, and in rare cases there are up to ten possibilities
for representing a single letter.

On the other hand, some letters share the same shape in different
contexts; one so-called \emph{glyph} can represent more than one
letter, sometimes three or four different letters.

The Manju writing, or \emph{bithe} system is a close relative of the
Mongolian system; the basical letter shapes are the same. Yet for Manju,
a set of diacritics (\emph{dots und circles}) was designed to the effect
that all the ambiguities of Mongolian are eliminated.

Decomposing the writing system and using glyphs as the atoms of
writing is one of several conceivable methods of writing Mongolian
script.

In \MonTeX, Mongolian script can be entered in three ways, either
by writing transliterated Mongolian in one of two different
romanization systems, by an approximated symbol for every glyph or
by generic name. There are certain constraints concerning the
possible combinations of Mongolian input methods and Mongolian
writing display commands. Since Manju has only one input method,
these constraints do not apply to Manju. The possible combinations
are listed in table~\ref{table:Combinations}.

A complete guide to
the principles of glyph analysis can be found in the MLS Report
by one of the authors.\footnote{Oliver Corff: MLS Report. UNU/IIST
Report No.~8, Macau 1993}

Due to technical constraints of \MonTeX, there is an intimate
relationship between various script-related commands and Mongolian
input methods.


\section{General Settings\label{section:UMMDates}}

As for Modern (Xalx) Mongolian, Buryat and Russian documents, it is
possible to set the document language to Uighur Mongolian or Manju
with a language option:

\begin{verbatim}
\usepackage[<language options>,<encoding options>]{mls}
\end{verbatim}

The two language options are \cmda{bicig} for Uighur Mongolian 
and \cmda{bithe} for Manju documents. Among other things, they
set the document encoding, the captions and the date in either
Uighur Mongolian or Manju.

The date form follows \TeX\ conventions and is thus a mixture of
numbers and words. Thus for \cmd{today} (\today) we get% 
	\footnote{The actual date at compilation time is
used for the examples.} what is shown in
table~\ref{caption:UMMDates}. 
\enlargethispage*{1ex}

\begin{figure}[h]
\begin{center}
	\begin{tabular}{cc}
	    \textbf{Mongolian}%		
	    \mobosoo{\BicigToday}
	    \label{cmd:BicigToday}	&\textbf{Manju}
	    				 \mabosoo{\BitheToday}
	    				 \label{cmd:BitheToday}\\
	\end{tabular}
\caption{Dates in Uighur Mongolian and Manju}\label{caption:UMMDates}
\end{center}
\end{figure}

The document language option \verb'bicig' can only be used with the
Mongolian input method named ``Simplified Transliteration'' (see the
following chapter and table~\ref{table:Combinations}).

%%%%%%%%%%%%%%%%%%%%%%%%%%%%%%%%%%%%%%%%%%%%%%%%%%%%%%%%%%%%%%%%
\chapter{Mongolian Input}

It has been mentioned before that each Mongolian input method
provided by \MonTeX\ has a slightly different application scope.
The next sections cover
\begin{itemize}
	\item	\emph{Simplified Transliteration Mode} which is
		the mode of choice for bulk text due to its availability
		as document encoding; the associated font encoding
		is labelled \LMO\label{a:LMO} and internally activated
		by the command \refcmd{SetDocumentEncodingBicig}.

	\item	\emph{MLS Transliteration Mode} which is most
		suitable for short portions of text, like dictionary
		entries, quotations, etc.

	\item	\emph{Immediate Mode} is the mode accepting
		Mongolian characters encoded in the MLS codepage.
		Together with this mode, the input encoding
		\refcmda{mls} should be specified, ideally combined
		with the command \refcmd{SetDocumentEncodingNeutral}.

	\item	\emph{Glyph Input} which is useful mainly for
		rendering individual words in unorthodox or
		incorrect spellings, e.\,g. for reproducing
		idiosyncrasies found in old books.
\end{itemize}

A comprehensive table of the Mongolian alphabet and its MLS
transliteration, the input conventions of the MLS transliteration in
\MonTeX\ and the Simplified Transliteration is given in
table~\ref{table:bcgcagan}.

\newcommand{\bcgcagan}[4]{%
	\mbosoo{#1}	& \texttt{#2}	& \texttt{#3}	& \texttt{#4}	%\\
	}

\begin{table}[h]
\begin{center}
\begin{tabular}{cccc|cccc}
%\hline
Uighur&\multicolumn{2}{c}{MLS} &Simplified
				&Uighur&\multicolumn{2}{c}{MLS}&Simplified\\
Script&Transl.& Input	 &Input     &Script&Transl.& Input&Input\\
\hline
\bcgcagan{a}{a}{a}{a}		& \bcgcagan{s}{s}{s}{s} \\
\bcgcagan{E}{\"a}{\"a, E}{e}	& \bcgcagan{S}{sh}{S}{sh}\\
\bcgcagan{e}{e}{e}{v}		& \bcgcagan{t}{t}{t}{t}	 \\
\bcgcagan{i}{i}{i}{i}		& \bcgcagan{d}{d}{d}{d, t}\\
\bcgcagan{o}{o}{o}{u}		& \bcgcagan{l}{l}{l}{l}	\\
\bcgcagan{u}{u}{u}{u}		& \bcgcagan{m}{m}{m}{m}	\\
\bcgcagan{O}{\"o}{\"o, O}{ui, u}& \bcgcagan{c}{c}{c}{c}	\\
\bcgcagan{U}{\"u}{\"u, U}{ui, u}& \bcgcagan{z}{z}{z}{z}	\\
\bcgcagan{n}{n}{n}{n}		& \bcgcagan{y}{y}{y}{y}	\\
\bcgcagan{|ng}{*ng}{ng}{ng}	& \bcgcagan{r}{r}{r}{r}	\\
\bcgcagan{x}{x}{x}{x}		& \bcgcagan{v}{v}{v}{v}	\\
\bcgcagan{G}{\g}{G}{g}		& \bcgcagan{h}{h}{h}{h}	\\
\bcgcagan{k}{k}{k}{k}		& \bcgcagan{j}{j}{j}{j}	\\
\bcgcagan{g}{g}{g}{g, k}	& \bcgcagan{K}{K}{K}{K}	\\
\bcgcagan{b}{b}{b}{b}		& \bcgcagan{Q}{[--]}{Q}{q}\\
\bcgcagan{p}{p}{p}{p}		& \bcgcagan{C}{C}{C}{C}	\\
\bcgcagan{f}{f}{f}{f}		& \bcgcagan{Z}{Z}{Z}{Z}	\\
%\hline
\end{tabular}
\end{center}
\caption{Mongolian Script Transliterations}\label{table:bcgcagan}
\end{table}

The possible combinations of Mongolian writing input methods
and display commands are listed in table~\ref{table:Combinations}.
The columns stand for each possible input encoding, the rows contain
the display command types. Each table cell at the contains the command
that is available for a given combination of input method and
command.

\newcommand{\ComparisonTable}[4]{%
	#1 &%			% Command Type
	#2 &%			% MLS Command
	#3 &%			% Simplified Command
	#4 \\%			% Manju Command
}
\begin{table}[h]
\begin{center}
\begin{tabular}{p{2cm}|p{3.25cm}|p{3.25cm}|p{3.25cm}}
Command		& \multicolumn{2}{c|}{Mongolian}& Manju	\\
Type		& MLS		& Simplified	&	\\
\hline
\ComparisonTable{Document Encoding}
		{only available as font encoding \LMS, not as
		document encoding}
		{\refcmda{LMO}}
		{\refcmda{LMA}}
\hline
\ComparisonTable{Horizontal Capsules}
		{\refcmd{bcg}}
		{\refcmd{bicig}}
		{\refcmd{bithe}}
\hline
\ComparisonTable{Horizontal Paragraphs}
		{not available}
		{\refcmda{bicigtext}}
		{\refcmda{bithetext}}
\hline
\ComparisonTable{Vertical Capsules}
		{\refcmd{mbosoo}}
		{\refcmd{mobosoo}}
		{\refcmd{mabosoo}}
\hline
\ComparisonTable{Vertical Paragraph Boxes}
		{not available}
		{\refcmd{mobox}}
		{\refcmd{mabox}}
\hline
\ComparisonTable{Vertical Pages}
		{not available}
		{\refcmda{bicigpage}}
		{\refcmda{bithepage}}
%\hline
\end{tabular}
\caption{Mongolian Input and Display Commands}\label{table:Combinations}
\end{center}
\end{table}


\section{Simplified Transliteration Mode}

The broad romanization of the Mongolian script as realized in the
MLS system focuses on lexical properties (the \emph{information layer})
rather than graphical properties (the \emph{presentation layer}).
The obvious advantage of such a method is the possibility to store
and transmit Mongolian language information in and between systems
without devices for displaying Mongolian writing.

With the ambiguities of the Mongolian script (the consonants \emph{t/d},
the vowels \emph{a/e}, and many other ambiguous shapes give vivid
evidence hereof) it is however possible to enter misleading or wrong
romanizations which lead to a desired yet semantically misleading
\emph{display} of Mongolian in which case the underlying information
is not suitable for further processing.

Another aspect is the retrieval of information from e.\,g. library
catalogues when only the display of potentially unknown words like
in book titles is available. It must be possible to enter Mongolian
script into an information processing system without knowing at
every moment which underlying letter generates a given shape. This
implies that glyph analysis does not decompose complex glyph shapes
into atoms if the shape transformation is purely dictated by 
graphical rather than linguistical context.

Notwithstanding this fact, an obvious \emph{n} appearing as \suul\
following a vowel should be entered as \texttt{n} while an \emph{a}
following a consonant which also appears as \suul\ should certainly
entered as vowel, not as consonant. Anything going deeper in glyph
analysis can only be considered as atomic coding which may be highly
useful in special cases but renders the input process more than
cumbersome in general cases.

The Mongolian Simplified Transliteration proposed here is
based on principles laid out by Dr.\,Michael Balk of the Deutsche
Staatsbibliothek, Stiftung Preussischer Kulturbesitz, Berlin.
During its development, various proposals were discussed at
DIN, MNISM and ISO standardization meetings during 1994 to 1997.

The most important principle of this simplified input method is the
consequent elimination of ambiguities in the relation between
romanized input (as performed on an ordinary computer keyboard) and
its Mongolian script target. If the Mongolian language provides
several readings for certain vowels, then only one vowel is
available in the simplified method; if alternating consonants (like
\emph{k/g} can swap shapes, then each input letter is associated with
one and only one output shape. Furthermore, the \emph{input
alphabet} (speaking in terms of computer theory) is limited to the
basic Latin alphabet. It uses only
\texttt{a b c d e f g h i j k l m n p q r s t u v x y z C K Z}
and the following characters with special meanings:
\texttt{- = ' "}. The first symbol is used to separate grammatical
endings from preceding words, the second separates floating vowels
from word stems, and the third and fourth character act as Variant
Selectors. At present, the second Variant Selector is not yet
assigned.

%\enlargethispage*{4ex}
Unlike a purely atomic rendering, the resulting romanization as
shown in table~\ref{table:bcgcagan} is
easy to learn, much easier to read than atomic code, and yet
acceptably close to conventional Mongolian transliterations, as can
be seen from table~\ref{table:SimplifiedExamples}. Each row contains
one or more instances of every letter listed in the first column.
From left to right, these are the beginning of a word, the middle of 
a word and the end of a word. Every single cell features three
elements: the \emph{example} in MLS romanization appears in italics; 
the correct Simplified Transliteration \texttt{input} appears in
typewriter style, and the word in Uighur Mongolian letters appears
in the right half of the cell.

\newcommand{\mytextit}{\bgroup\mdoublehyphenon\moretextit}%
\newcommand{\moretextit}[1]{\textit{#1}\egroup}%
%
\newcommand{\rr}{\phantom{;}}
%
\newcommand{\T}[7]{%
	%\hline
	#1	&%				% Zeichen / useg
	\mytextit{#2}\newline\texttt{#3}&\mobosoo{#3\rr}&%% Anfang/exen
	\mytextit{#4}\newline\texttt{#5}&\mobosoo{#5\rr}&%% Mitte /dund
	\mytextit{#6}\newline\texttt{#7}&\mobosoo{#7\rr} %% Ende  /adag
	\\%
	}

{%\mdoublehyphenon
\begin{longtable}{c|p{2.0cm}l|p{2.0cm}l|p{2.0cm}l}
%\hline
Letter &\multicolumn{2}{c|}{Beginning}
		&\multicolumn{2}{c|}{Middle}
			&\multicolumn{2}{c}{End}\\
\hline
% % % % % % % % % % % % % % % % % % % % % % % % % % % % % % % % 
\T{a}					% Zeichen : useg
	{arad}		{arad}		% Anfang / exen
	{ba\g atur}	{bagadur} 	% Mitte  / dund
	{la}		{la}		% Ende   / adag
% % % % % % % % % % % % % % % % % % % % % % % % % % % % % % % % 
\T{}					% Zeichen : useg
	{}		{}		% Anfang / exen
	{}		{}		% Mitte  / dund
	{sana\g =a}	{sanag=a}	% Ende   / adag
%%%%%%%%%%%%%%%%%%%%%%%%%%%%%%%%%%%%%%%%%%%%%%%%%%%%%%%%%%%%%%%%
\hline
\T{�}					% Zeichen : useg
	{�rk�}		{erke}		% Anfang  : exen
	{c�c�g}		{cecek}		% Mitte   : dund
	{s�k�}		{suike}		% Ende    : adag
%%%%%%%%%%%%%%%%%%%%%%%%%%%%%%%%%%%%%%%%%%%%%%%%%%%%%%%%%%%%%%%%
\hline
\T{e}					% Zeichen : useg
	{eKs}		{evKs}		% Anfang  : exen
	{geologi}	{kvuluki}	% Mitte   : dund
	{}		{}		% Ende    : adag
%%%%%%%%%%%%%%%%%%%%%%%%%%%%%%%%%%%%%%%%%%%%%%%%%%%%%%%%%%%%%%%%
\hline
\T{i}					% Zeichen : useg
	{iza\g ur}	{izagur}	% Anfang  : exen
	{minu}		{minu}		% Mitte   : dund
	{bandi}		{bandi}		% Ende    : adag
%%%%%%%%%%%%%%%%%%%%%%%%%%%%%%%%%%%%%%%%%%%%%%%%%%%%%%%%%%%%%%%%
\hline
\T{o}					% Zeichen : useg
	{olan}		{ulan}		% Anfang  : exen
	{a\g ul=a}	{agul=a}	% Mitte   : dund
	{}		{}		% Ende    : adag
% % % % % % % % % % % % % % % % % % % % % % % % % % % % % % % % 
\T{u}					% Zeichen : useg
	{ulus}		{ulus}		% Anfang  : exen
	{}		{}		% Mitte   : dund
	{\g arxu}	{garxu}		% Ende    : adag
%%%%%%%%%%%%%%%%%%%%%%%%%%%%%%%%%%%%%%%%%%%%%%%%%%%%%%%%%%%%%%%%
\hline
\T{�}					% Zeichen : useg
	{�nd�r}		{uindur}	% Anfang  : exen
	{c�m=�}		{cuim=e}	% Mitte   : dund
	{}		{}		% Ende    : adag
% % % % % % % % % % % % % % % % % % % % % % % % % % % % % % % % 
\T{�}					% Zeichen : useg
	{�s�g}		{uisuk}		% Anfang  : exen
	{}		{}		% Mitte   : dund
	{}		{}		% Ende    : adag
%%%%%%%%%%%%%%%%%%%%%%%%%%%%%%%%%%%%%%%%%%%%%%%%%%%%%%%%%%%%%%%%
\hline
\T{n + \{V\}}				% Zeichen : useg
	{nam}		{nam}		% Anfang  : exen
	{onol}		{unul}		% Mitte   : dund
	{bayin=a}	{baiin=a}	% Ende    : adag
% % % % % % % % % % % % % % % % % % % % % % % % % % % % % % % % 
\T{n + \{C\}}				% Zeichen : useg
	{}		{}		% Anfang  : exen
	{bandi}		{bandi}		% Mitte   : dund
	{}		{}		% Ende    : adag
%%%%%%%%%%%%%%%%%%%%%%%%%%%%%%%%%%%%%%%%%%%%%%%%%%%%%%%%%%%%%%%%
\T{n' + \{V\}}					% Zeichen : useg
	{n'am}		{n'am}		% Anfang  : exen
	{on'ol}		{un'ul}		% Mitte   : dund
	{bayin'=a}	{baiin'=a}	% Ende    : adag
% % % % % % % % % % % % % % % % % % % % % % % % % % % % % % % % 
\T{n' + \{C\}}				% Zeichen : useg
	{}		{}		% Anfang  : exen
	{KoNTor}	{Kun'tur}	% Mitte   : dund
	{ban'di}	{ban'di}	% Ende    : adag
%%%%%%%%%%%%%%%%%%%%%%%%%%%%%%%%%%%%%%%%%%%%%%%%%%%%%%%%%%%%%%%%
\hline
\T{ng}					% Zeichen : useg
	{}	  	{}		% Anfang  : exen
	{mong\g ol}  	{munggul}	% Mitte   : dund
	{vang}	  	{vang}		% Ende    : adag
%%%%%%%%%%%%%%%%%%%%%%%%%%%%%%%%%%%%%%%%%%%%%%%%%%%%%%%%%%%%%%%%
\hline
\T{x}					% Zeichen : useg
	{xota}	  	{xuda}		% Anfang  : exen
	{abxu}	  	{abxu}		% Mitte   : dund
	{mix=a}	  	{mix=a}		% Ende    : adag
%%%%%%%%%%%%%%%%%%%%%%%%%%%%%%%%%%%%%%%%%%%%%%%%%%%%%%%%%%%%%%%%
\hline
\T{\g}					% Zeichen : useg
	{\g azar}  	{gazar}		% Anfang  : exen
	{ba\g atur}  	{bagadur}	% Mitte   : dund
	{tu\g}	  	{tug}		% Ende    : adag
%%%%%%%%%%%%%%%%%%%%%%%%%%%%%%%%%%%%%%%%%%%%%%%%%%%%%%%%%%%%%%%%
\hline
\T{\g'}					% Zeichen : useg
	{\g'azar}  	{g'azar}	% Anfang  : exen
	{ba\g'atur}  	{bag'adur}	% Mitte   : dund
	{}	  	{}		% Ende    : adag
%%%%%%%%%%%%%%%%%%%%%%%%%%%%%%%%%%%%%%%%%%%%%%%%%%%%%%%%%%%%%%%%
\hline
\T{k}					% Zeichen : useg
	{k�r�g}	  	{kerek}		% Anfang  : exen
	{�rkil�k�}  	{erkileku}	% Mitte   : dund
	{}	  	{}		% Ende    : adag
% % % % % % % % % % % % % % % % % % % % % % % % % % % % % % % % 
\hline
\T{g}					% Zeichen : useg
	{g�r}	  	{ger}		% Anfang  : exen
	{�g�i}	  	{uigei}		% Mitte   : dund
	{bicig}	  	{bicik}		% Ende    : adag
% % % % % % % % % % % % % % % % % % % % % % % % % % % % % % % % 
\hline
\T{b}					% Zeichen : useg
	{ba\g =a}  	{bag=a}		% Anfang  : exen
	{d�bt�r}  	{tebder}	% Mitte   : dund
	{�b}	  	{eb}		% Ende    : adag
% % % % % % % % % % % % % % % % % % % % % % % % % % % % % % % % 
\hline
\T{p}					% Zeichen : useg
	{pangsa}  	{pangsa}	% Anfang  : exen
	{}	  	{}		% Mitte   : dund
	{}	  	{}		% Ende    : adag
% % % % % % % % % % % % % % % % % % % % % % % % % % % % % % % % 
\hline
\T{f}					% Zeichen : useg
	{feodal}  	{fvudal}	% Anfang  : exen
	{Cifr}	  	{Cifr}		% Mitte   : dund
	{}	  	{}		% Ende    : adag
% % % % % % % % % % % % % % % % % % % % % % % % % % % % % % % % 
\hline
\T{s}					% Zeichen : useg
	{saxal}	  	{saxal}		% Anfang  : exen
	{basa}	  	{basa}		% Mitte   : dund
	{nas}	  	{nas}		% Ende    : adag
% % % % % % % % % % % % % % % % % % % % % % % % % % % % % % % % 
\hline
\T{sh}					% Zeichen : useg
	{sha\g dur}  	{shagdur}	% Anfang  : exen
	{}	  	{}		% Mitte   : dund
	{}	  	{}		% Ende    : adag
% % % % % % % % % % % % % % % % % % % % % % % % % % % % % % % % 
\hline
\T{t}					% Zeichen : useg
	{tomu}	  	{tumu}		% Anfang  : exen
	{ba\g atur}  	{bagadur}	% Mitte   : dund
	{}	  	{}		% Ende    : adag
%%%%%%%%%%%%%%%%%%%%%%%%%%%%%%%%%%%%%%%%%%%%%%%%%%%%%%%%%%%%%%%%
\hline
\T{d}					% Zeichen : useg
	{dumdadu}  	{dumdadu}	% Anfang  : exen
	{odu}	  	{udu}		% Mitte   : dund
	{arad}	  	{arad}		% Ende    : adag
% % % % % % % % % % % % % % % % % % % % % % % % % % % % % % % % 
\T{}					% Zeichen : useg
	{}	  	{}		% Anfang  : exen
	{s�dgil}  	{sedkil}	% Mitte   : dund
	{�D}	  	{ed'}		% Ende    : adag
%%%%%%%%%%%%%%%%%%%%%%%%%%%%%%%%%%%%%%%%%%%%%%%%%%%%%%%%%%%%%%%%
\hline
\T{l}					% Zeichen : useg
	{la}		{la}		% Anfang  : exen
	{aldar}		{aldar}		% Mitte   : dund
	{onul}		{unul}		% Ende    : adag
% % % % % % % % % % % % % % % % % % % % % % % % % % % % % % % % 
\T{}					% Zeichen : useg
	{}		{}		% Anfang  : exen
	{blam=a}	{blam=a}	% Mitte   : dund
	{}		{}		% Ende    : adag
%%%%%%%%%%%%%%%%%%%%%%%%%%%%%%%%%%%%%%%%%%%%%%%%%%%%%%%%%%%%%%%%
\hline
\T{m}					% Zeichen : useg
	{mong\g ol}	{munggul}	% Anfang  : exen
	{nomin}		{numin}		% Mitte   : dund
	{nom}		{num}		% Ende    : adag
%%%%%%%%%%%%%%%%%%%%%%%%%%%%%%%%%%%%%%%%%%%%%%%%%%%%%%%%%%%%%%%%
\hline
\T{c}					% Zeichen : useg
	{ca\g an}  	{cagan}		% Anfang  : exen
	{�c�n}	  	{ecen}		% Mitte   : dund
	{}	  	{}		% Ende    : adag
%%%%%%%%%%%%%%%%%%%%%%%%%%%%%%%%%%%%%%%%%%%%%%%%%%%%%%%%%%%%%%%%
\hline
\T{z}					% Zeichen : useg
	{zam}	  	{zam}		% Anfang  : exen
	{\g azar}  	{gazar}		% Mitte   : dund
	{}	  	{}		% Ende    : adag
%%%%%%%%%%%%%%%%%%%%%%%%%%%%%%%%%%%%%%%%%%%%%%%%%%%%%%%%%%%%%%%%
\hline
\T{y}					% Zeichen : useg
	{yondan}  	{yundan}	% Anfang  : exen
	{bayar}	  	{bayar}		% Mitte   : dund
	{xoriy=a}  	{xuriy=a}	% Ende    : adag
%%%%%%%%%%%%%%%%%%%%%%%%%%%%%%%%%%%%%%%%%%%%%%%%%%%%%%%%%%%%%%%%
\hline
\T{r}					% Zeichen : useg
	{rashan}  	{rashan}	% Anfang  : exen
	{oros}	  	{urus}		% Mitte   : dund
	{bolor}	  	{bulur}		% Ende    : adag
%%%%%%%%%%%%%%%%%%%%%%%%%%%%%%%%%%%%%%%%%%%%%%%%%%%%%%%%%%%%%%%%
\hline
\T{v}					% Zeichen : useg
	{vang}	  	{vang}		% Anfang  : exen
	{}	  	{}		% Mitte   : dund
	{}	  	{}		% Ende    : adag
%%%%%%%%%%%%%%%%%%%%%%%%%%%%%%%%%%%%%%%%%%%%%%%%%%%%%%%%%%%%%%%%
\hline
\T{h}					% Zeichen : useg
	{heze}	  	{hvzv}		% Anfang  : exen
	{lhas}	  	{lhas}		% Mitte   : dund
	{}	  	{}		% Ende    : adag
%%%%%%%%%%%%%%%%%%%%%%%%%%%%%%%%%%%%%%%%%%%%%%%%%%%%%%%%%%%%%%%%
\hline
\T{j}					% Zeichen : useg
	{j}	  	{j}		% Anfang  : exen
	{}	  	{}		% Mitte   : dund
	{}	  	{}		% Ende    : adag
%%%%%%%%%%%%%%%%%%%%%%%%%%%%%%%%%%%%%%%%%%%%%%%%%%%%%%%%%%%%%%%%
\hline
\T{K}					% Zeichen : useg
	{KoNTor}  	{Kun'tur}	% Anfang  : exen
	{}	  	{}		% Mitte   : dund
	{}	  	{}		% Ende    : adag
%%%%%%%%%%%%%%%%%%%%%%%%%%%%%%%%%%%%%%%%%%%%%%%%%%%%%%%%%%%%%%%%
\hline
\T{gh}					% Zeichen : useg
	{ghombo}  	{qumbu}		% Anfang  : exen
	{}	  	{}		% Mitte   : dund
	{}	  	{}		% Ende    : adag
%%%%%%%%%%%%%%%%%%%%%%%%%%%%%%%%%%%%%%%%%%%%%%%%%%%%%%%%%%%%%%%%
\hline
\T{C}					% Zeichen : useg
	{Cifr}	  	{Cifr}		% Anfang  : exen
	{}	  	{}		% Mitte   : dund
	{sTan'C}  	{stan'C}	% Ende    : adag
%%%%%%%%%%%%%%%%%%%%%%%%%%%%%%%%%%%%%%%%%%%%%%%%%%%%%%%%%%%%%%%%
\hline
\T{Z}					% Zeichen : useg
	{Zambu}	  	{Zambu}		% Anfang  : exen
	{aZi}	  	{aZi}		% Mitte   : dund
	{}	  	{}		% Ende    : adag
% % % % % % % % % % % % % % % % % % % % % % % % % % % % % % % % 
%\hline
\caption{Mongolian Simplified Transliteration by Example%
	}\label{table:SimplifiedExamples}\\
\end{longtable}}

While the input method for the majority of characters matches the
transliteration conventions, some letters require a slightly
different treatment:
\begin{enumerate}
	\item	Although the diphtong \mobosoo{*aii*} is usually
		rendered as \textit{ayi}, it must be entered
		as \texttt{aii} in order to produce the desired
		effect.
	
	\item	The back vowels \emph{o} and \emph{u} are both rendered
		as \texttt{u}.

	\item	The front vowels \emph{\"o} and \emph{\"u} are both
		rendered as \texttt{ui} in first syllables and as
		\texttt{u} in later syllables.

	\item	Since \mobosoo{t} means both \emph{t} and \emph{d},
		it is necessary to spell this letter as \texttt{t}
		in the beginning of words, and \texttt{d} in the
		middle of words, regardless of the actual meaning.

	\item	The four consonants \emph{\g}, \emph{g}, \emph{x}
		and \emph{k} are constrained with regard to the
		following vowels. The Simplified Transliteration
		renders these as \texttt{g} (before \emph{a}
		and \emph{u} only), \texttt{g} (before \emph{a}
		and \emph{u} only), \texttt{x} and \texttt{k}.
\end{enumerate}


As it was demonstrated in section~\ref{section:CyrillicTransliterationMode},
it is technically possible to choose between an automatic document encoding
and the neutral mode. In the case of Uighur Mongolian, the mode of choice
activates the Simplified Transliteration Mode and is called with 
\begin{quote}
\begin{verbatim}
\SetDocumentEncodingBicig
\end{verbatim}
\end{quote}


With
\verb"\SetDocumentEncodingBicig"\label{cmd:SetDocumentEncodingBicig} set,
it is possible to switch to the Simplified Transliteration Mode anywhere
in the document, not only in the preamble.

\textit{Caveat:} Since switching to Uighur Mongolian text
requires a lot of settings to be effected at the same time, there
are high-level commands available (see below,
chapter~\ref{chapter:DisplayCommands}: Mongolian and Manju Display
Commands) which do all the work, including the definition of the
document encoding. Thus, while \verb|\SetDocumentEncodingBicig|
is indeed classified as a user-level command, it is certainly not
necessary for everyday work.


\subsection{Character Variants}

With the assistance of special, non-printing characters like the
Form Variant Selectors, the appearance of certain characters can be
modified in order to display typographical and orthographical
variants. Notably, the \emph{n} will loose its dot before vowels,
as will \emph{\g}. Let's assume the word ``place'' is written in an
old book as \bicig{g'azar}. It should be understood that this is a
variant of \bicig{gazar} and should be spelled \emph{\g'azar}, not
\emph{xazar}. With vowels, the Form Variant Selectors can change the
shape that is usually required by graphical context. At present,
only the first of two Form Variant Selectors actually does
something, the exact behaviour of the second Form Variant Selector
waits to be implemented.

The following short example shows a concrete application of this
method. It renders the six syllable mantra \emph{om ma ni padme hum}
\ifx\tib\undefined\relax
	\else(tib. {\tib \om, ma nxi pa\V{de}{ma} \hung.})
\fi also featuring the
special syllable \cmd{om} as it is displayed on a huge bronze
incense burner in front of the Gandan Monastery in Ulaanbaatar:

\begin{figure}[h]
\exa
	\mobox{3cm}{\noindent\sffamily
	\om	uva\\
	\  	ma'=a\\
	\  	n'i\\
	\  	badmi'\\
	\om	huu}
\exb
	\begin{verbatim}
	\mobox{3cm}{\noindent\sffamily
	\om	uva\\
	\ 	ma'=a\\
	\ 	n'i\\
	\ 	badmi'\\
	\om	huu}
	\end{verbatim}
\exc
\caption{Mongolian Character Variants Example}\label{figure:MongCharVarExample}
\end{figure}

\section{MLS Transliteration Mode}

\begin{sloppypar}
In Transliteration Mode (activated with the commands \cmd{bcg}\verb|{...}|
or \cmd{mbosoo}\verb|{...}|) Mongolian text portions can be entered using
a transliteration which is a rough approximation to the MLS system.
The major difference is that only pure Latin alphabetical symbols
can be used for virtually all letters. Front vowels are either
entered via the traditional vowels with diacritics (\emph{\"a, \"o, \"u})
or can be entered with capitalized versions of the normal vowels.
Capitalized letters have to be used for entering 
{\g} and {\sh} which are entered as \textit{G} and \textit{S}.
Special variants for certain letters can be selected with
Form Variant Selectors.
%Examples are given in section~\ref{bicig-examples}.%
\footnote{%
	The suggested solution has the advantage that it
	can be used on computers featuring codepages without umlaut 
	symbols as most of the Cyrillic code pages are `defective'
	in this point.}
\end{sloppypar}

The available Mongolian characters (\emph{ca\g an tolu\g ai}) are shown
in table~\ref{table:bcgcagan}.\footnote{The alphabetical arrangement follows
large that given on p.~17 of N.~Poppe's \textit{Grammar of Written 
Mongolian}, Wiesbaden 1954, 1964, 1974 (third printing). Letters not
given there are appended to Poppe's list.}


\section{Immediate Mode}

For freely combining Mongolian Script with other characters without
using any explicite commands it is necessary that the codepage in use
supports Mongolian Script glyphs; currently this is the MLS codepage.
The MLS input encoding is specified like \verb:\usepackage[mls]{mls}:.
As with Cyrillic codepages it should be noted that these documents are
not easily portable between different platforms anymore since they cannot
be recoded at ease. See table~\ref{table:glyphnames} for a list
of available symbols.


\section{Glyphs by Symbol}

Without MLS codepage support, Mongolian words can also be entered
using the \cmd{glyphbcg}\verb"{...}" command in running text. Within
these groups, Mongolian Script glyphs are entered in the form of
approximated symbols; sometimes these symbols reflect the underlying
canonical letter, sometimes functional equivalents (for punctuation
marks etc.) are chosen; sometimes there is no evident relation between
glyph and input symbol simply because a free slot within the ASCII range
$c\ge32\le127$ was chosen. Please consult table~\ref{table:glyphchars}
of available glyphs and their input equivalents.


\section{Glyphs by Name}

Without any preparations on the side of the text environment it is
possible to enter individual Mongolian glyphs by name in a way
similar for that of entering Cyrillic characters; the Mongolian
glyph names can be found in table~\ref{table:glyphnames}. Thus,
\verb|\shilbe| produces a \shilbe. A number in the
MLS column indicates the encoding position of the MLS codepage; a
missing number in this column indicates that the glyph is part of
extended \MonTeX\ glyph set without being part of the original MLS.
%\newpage

\newcommand{\glyphtabledata}{%
    \bge{"C2}{\titem}		{@}&\bge{"EB}{\matgarshilbe}	{v}\\
    \bge{"C3}{\shud}		{a}&\bge{"EC}{\bituushilbe}	{h}\\
    \bge{"C5}{\secondaryshud}	{A}&\bge{"ED}{\secondaryqagt}	{K}\\
    \bge{"C6}{\shilbe}		{i}&\bge{"EE}{\qagt}		{k}\\
    \bge{"C7}{\gedes}		{o}&\bge{"EF}{\secnumtdelbenqix}{P}\\
    \bge{"CF}{\secondarygedes}	{O}&\bge{"F0}{\numtdelbenqix}	{p}\\
    \bge{"D0}{\cegteishud}	{n}&\bge{"F1}{\secsertenqixtnum}{F}\\
    \bge{"D1}{\lewer}		{l}&\bge{"F2}{\sertenqixtnum}	{f}\\
    \bge{"D2}{\suuliinlewer}	{L}&\bge{"F3}{\zadgaizardigt}	{Z}\\
    \bge{"D3}{\tertiarylewer}	{Q}&\bge{"F4}{\bituuzardigt}	{C}\\
    \bge{"D4}{\mewer}		{m}&\bge{"F5}{\malgaitaititem}	{j}\\
    \bge{"D5}{\suuliinmewer}	{M}&\bge{"F6}{\suul}		{e}\\
    \bge{"D6}{\xewteeqix}	{x}&\bge{"F7}{\orxic}		{E}\\
    \bge{"D7}{\dawxarcegtxewteeqix}{X}&\bge{"F8}{\biodoisuul}	{Y}\\
    \bge{"D8}{\halfnum}		{g}&\bge{"F9}{\bagodoisuul}	{G}\\
    \bge{"DB}{\num}		{I}&\bge{"FA}{\nceg}		{-}\\
    \bge{"DC}{\halfnumtgedes}	{B}&\bge{"FB}{\gceg}		{=}\\
    \bge{"DD}{\numtaigedes}	{b}&\bge{"FC}{\ceg}		{,}\\
    \bge{"DE}{\buruuxarsangedes}{t}&\bge{"FD}{\dorwoljin}	{;}\\
    \bge{"DF}{\gedesteishilbe}	{d}&\bge{   }{ - }		{V}\\
    \bge{"E0}{\erweeljinshilbe}	{r}&\bge{   }{ - }		{u}\\
    \bge{"E3}{\secerweeljin}	{R}&\bge{   }{ - }		{T}\\
    \bge{"E4}{\bosooshilbe}	{z}&\bge{   }{ - }		{U}\\
    \bge{"E5}{\etgershilbe}	{y}&\bge{   }{ - }		{W}\\
    \bge{"E6}{\zawj}		{s}&\bge{   }{ - }		{w}\\
    \bge{"E8}{\suuliinzawj}	{S}&\bge{   }{ - }		{ml}\\
    \bge{"E9}{\dawxarcegtzawj}	{q}&\bge{   }{ - }		{ll}\\
    \bge{"EA}{\sereeewer}	{c}& \\}

\newcommand{\bge}[3]{%
	\bosoo{\glyphbcg{#3}}			% Glyph
		&\texttt{\string#2}		% Generic Name
			&\texttt{#3}		% Input Letter or Symbol
%				& \texttt{#1}	% MLS Code Position
	}

\begin{center}
\begin{longtable}{clc|clc}
%\hline
Glyph&Generic	&Input	&Glyph&Generic	&Input	\\
     &Name	&Char.	&     &Name	&Char.	\\
\hline
    \glyphtabledata
%\hline
\caption{MLS Named Basic Glyphs}\label{table:glyphnames}
\end{longtable}
\end{center}

\clearpage

\renewcommand{\bge}[3]{%
	\bosoo{\glyphbcg{#3}}			% Glyph
		&\texttt{\string#2}		% Generic Name
%			&\texttt{#3}		% Input Letter or Symbol
				& \texttt{#1}	% MLS Code Position
	}

\begin{center}
\begin{longtable}{clc|clc}
%\hline
Glyph&Generic	&MLS	&Glyph&Generic	&MLS\\
     &Name	&Code	&     &Name	&Code\\
\hline
	\glyphtabledata
%\hline
\caption{MLS Basic Glyph Positions}\label{table:glyphchars}
\end{longtable}
\end{center}

\section{Special Characters\label{section:SpecialMLSCharacters}}

For the correct operation of retransliterating systems processing
Mongolian script additional symbols are needed. These include
Form Variant Selectors (\textsf{FVS}), the Vowel Separator, and
other symbols like the Mongolian Positional Indicator. As can be
seen from its usage in table~\ref{table:bcgcagan}, entering \verb|*ng|
tells the system to consider this \emph{ng} to be in non-initial
position.\footnote{Unfortunately, though it is now commonly
agreed in the scientific community that these symbols are needed,
their definition is still in a state of flux, and thus the symbols
given here are presented on a preliminary basis.}

Besides these symbols, table~\ref{table:SpecialMLSCharacters} includes
also some useful punctuation marks etc.\ as they are used in
Mongolian Script.

\begin{table}
\begin{center}
\begin{tabular}{c|l|l}
%\hline
 Symbol		& Name			& Input		\\
 \hline
  \bosoo{\glyphbcg{!}}	& Exclamation Mark	& \verb|!|	\\
  \bosoo{\glyphbcg{?}}	& Question Mark		& \verb|?|	\\
  \bosoo{\glyphbcg{!?}}	& Exclamation Question Mark& \verb|!?|	\\
  \bosoo{\glyphbcg{?!}}	& Question Exclamation Mark& \verb|?!|	\\
  \bosoo{\glyphbcg{*}}	& Mong. Positional Indicator& \verb|*|	\\
  \bosoo{\glyphbcg{\char32}}	& Mongolian Space	& \verb*|-|	\\
  \bosoo{\glyphbcg{(}}	& Opening Bracket	& \verb|(|	\\
  \bosoo{\glyphbcg{)}}	& Closing Bracket	& \verb|)|	\\
  \bosoo{\glyphbcg{<}}	& Opening Angle Bracket	& \verb|<|	\\
  \bosoo{\glyphbcg{>}}	& Closing Angle Bracket	& \verb|>|	\\
  \bosoo{\glyphbcg{<<}}	& Opening Guillemot	& \verb|<<|	\\
  \bosoo{\glyphbcg{>>}}	& Closing Guillemot	& \verb|>>|	\\
% \bosoo{\glyphbcg{\{}}	& Opening Parenthesis	& \verb|{|	\\
% \bosoo{\glyphbcg{\}}}	& Closing Parenthesis	& \verb|}|	\\
  \bosoo{\glyphbcg{'}}	& Form Variant Selector 1& \verb|'|	\\
  \bosoo{\glyphbcg{"}}	& Form Variant Selector 2& \verb|"|	\\
  \bosoo{\glyphbcg{\char43}}& Mong. Vowel Separator	& \verb|=|	\\
  \bosoo{\glyphbcg{|}}	& Mongolian Nuruu	& \verb'|'	\\
  \bosoo{\glyphbcg{.}}	& Period		& \verb|.|	\\
  \bosoo{\glyphbcg{,}}	& Comma			& \verb|,|	\\
  \bosoo{\glyphbcg{:}}	& Colon			& \verb|:|	\\
  \bosoo{\glyphbcg{;}}	& D\"orw\"oljin		& \verb|;|	\\
  \bosoo{\glyphbcg{..}}	& Ellipsis		& \verb|..|	\\
  \bosoo{\glyphbcg{0}}	& Digit zero		& \verb|0|	\\
  \bosoo{\glyphbcg{1}}	& Digit one		& \verb|1|	\\
  \bosoo{\glyphbcg{2}}	& Digit two		& \verb|2|	\\
  \bosoo{\glyphbcg{3}}	& Digit three		& \verb|3|	\\
  \bosoo{\glyphbcg{4}}	& Digit four		& \verb|4|	\\
  \bosoo{\glyphbcg{5}}	& Digit five		& \verb|5|	\\
  \bosoo{\glyphbcg{6}}	& Digit six		& \verb|6|	\\
  \bosoo{\glyphbcg{7}}	& Digit seven		& \verb|7|	\\
  \bosoo{\glyphbcg{8}}	& Digit eight		& \verb|8|	\\
  \bosoo{\glyphbcg{9}}	& Digit nine		& \verb|9|	\\
% \hline
\end{tabular}
\end{center}
\caption{Mongolian Script Special Symbols and Punctuation
	Marks}\label{table:SpecialMLSCharacters}
\end{table}

\section{Displaying Transliterations}

For huge word lists and similar material it is convenient to enter
the transliteration only once and use it as input both for the
Mongolian retransliteration engine and the presentation of the
transliteration. A construct like

\exa
	\vspace{8mm}
	\newcommand{\Keyword}[1]{#1 \bcg{#1}}
	
	\Keyword{anda} / \emph{Looks nice.}

	\Keyword{SaGdur} / \emph{Not as nice.}
\exb
	\begin{verbatim}
	\newcommand{\Keyword}[1]{#1 \bcg{#1}}
	
	\Keyword{anda} / \emph{Looks nice.}

	\Keyword{SaGdur} / \emph{Not as nice.}
	\end{verbatim}
\exc

is helpful as long as no capitalized single-letter entity is used.
Capitalized entities look less pleasing in conventional texts; for
these purposes, the command \verb|\PrettyMLS| is provided which
takes input with single-letters entities and converts it to a
more traditional representation.\label{cmd:PrettyMLS}


\exa
	\vspace{5mm}
	\newcommand{\Keyword}[1]{%
		\PrettyMLS{#1} \bcg{#1}}
	\Keyword{anda} / \emph{Good.}\par
	\Keyword{SaGdur} / \emph{Good again.}
\exb
	\begin{verbatim}
	\newcommand{\Keyword}[1]{%
		\PrettyMLS{#1} \bcg{#1}}
	\Keyword{anda} / \emph{Good.}\par
	\Keyword{SaGdur} / \emph{Good again.}
	\end{verbatim}
\exc

Two additional flags, \verb|\ShowSpecialMLStrue|  and
\verb|\ShowSpecialMLSfalse|, can be used to activate canonical
identifiers instead of the conventional notation for the special
characters of table~\ref{table:SpecialMLSCharacters}.\label{ShowSpecialMLS}

\exa
	\PrettyMLS{SaGdur blam=a}

	\vspace{8mm}
	\ShowSpecialMLStrue
	\PrettyMLS{SaGdur blam=a}

	\vspace{8mm}
	\ShowSpecialMLSfalse
	\PrettyMLS{SaGdur blam=a}
\exb
	\begin{verbatim}
	\PrettyMLS{SaGdur blam=a}

	\ShowSpecialMLStrue
	\PrettyMLS{SaGdur blam=a}

	\ShowSpecialMLSfalse
	\PrettyMLS{SaGdur blam=a}
	\end{verbatim}
\exc

The complete set of characters covered by \verb|\PrettyMLS| is shown
in table~\ref{table:PrettyMLS}.

\newcommand{\PrettyCodes}[1]{%
	{#1}	& \ShowSpecialMLStrue\PrettyMLS{#1}
		& \ShowSpecialMLSfalse\PrettyMLS{#1}\\
	}

\begin{table}[h]
\begin{center}
\begin{tabular}{|c|c|c|}
\hline
\MonTeX	&\multicolumn{2}{|c|}{\texttt{\char92ShowSpecialMLS}}\\
Input	&true	& false\\
\hline
\PrettyCodes{E}
\PrettyCodes{O}
\PrettyCodes{U}
\PrettyCodes{G}
\PrettyCodes{S}
\PrettyCodes{-}
\PrettyCodes{=}
\PrettyCodes{'}
\PrettyCodes{"}
\PrettyCodes{*}
\hline
\end{tabular}
\end{center}
\caption{MLS transliteration restauration}\label{table:PrettyMLS}
\end{table}

%%  \section{Mongolian Script Input Examples}\label{bicig-examples}
%%  
%%  \textbf{Nota bene}: The following examples are all given in horizontal 
%%  mode. It is visible that grammatical endings are always separated by
%%  \verb|-| whereas final vowels are separated from the stem by \verb|=|.
%%  Form Variant Selectors are used for differentiating between various
%%  forms of the same letter in similar environments; \verb|Ed'| \bcg{Ed'}
%%  and \verb|sayid| \bcg{sayid} will show you the difference. Without
%%  the Form Variant Selector, *\verb|Ed| and \verb|on| will be the same:
%%  \bcg{on}.
%%  
%%  \exa
%%  	\raggedright
%%  	\bcg{mongGol bicig. ulaGanbaGatur
%%  	xota bol mongGol ulus-un nEyislEl.
%%  	Ed'-Un zasaG, sayid. EngkEbayar.
%%  
%%  	1999 on.
%%  
%%  	40 tOgOrOg 20 mOnggU. bUri.
%%  
%%  	blam=a badm=a.
%%  
%%  	utasu: 00976,11,321654}
%%  \exb
%%  	\begin{verbatim}
%%  	\raggedright
%%  	\bcg{mongGol bicig. ulaGanbaGatur
%%  	xota bol mongGol ulus-un nEyislEl.
%%  	Ed'-Un zasaG, sayid. EngkEbayar.
%%  
%%  	1999 on.
%%  
%%  	40 tOgOrOg 20 mOnggU. bUri.
%%  
%%  	blam=a badm=a.
%%  
%%  	utasu: 00976,1,321654}
%%  	\end{verbatim}
%%  \exc
%%  
%%  \exa
%%  	\mobox{8cm}{%
%%  		\noindent munggul bicik. ulaganbagadur
%%  		xuda bul munggul ulus-un neyislel.
%%  		ed'-un zasag, sayid. engkebayar.
%%  
%%  		1999 un.
%%  
%%  		40 tuikurug 20 muingku. buiri.
%%  
%%  		blam=a badm=a.
%%  
%%  		udasu: 00976,1,321654
%%  	}
%%  \exb
%%  	%\vskip-8cm
%%  	\begin{verbatim}
%%  	\mobox{8cm}{%
%%  		\noindent munggul bicik. ulaganbagadur
%%  		xuda bul munggul ulus-un neyislel.
%%  		ed'-un zasag, sayid. engkebayar.
%%  
%%  		1999 on.
%%  
%%  		40 tuigurug 20 muingku. buiri.
%%  
%%  		blam=a badm=a.
%%  
%%  		udasu: 00976,1,321654
%%  	}
%%  	\end{verbatim}
%%  \exc
%%  

%%%%%%%%%%%%%%%%%%%%%%%%%%%%%%%%%%%%%%%%%%%%%%%%%%%%%%%%%%%%%%%%
\chapter{Manju Input}

Manju documents can be compiled with the \refcmda{bithe} option
to the \verb|\usepackage| command, which will create complete
documents in Manju. Anywhere in the document, it is possible to
switch to Manju input (transliteration mode)
with
\verb"\SetDocumentEncodingBithe"\label{cmd:SetDocumentEncodingBithe} which
internally activates the \LMA\label{a:LMA} encoding.

\textit{Caveat:} Since switching to Manju text
requires a lot of settings to be effected at the same time, there
are high-level commands available (see below,
chapter~\ref{chapter:DisplayCommands}) which do all the work, including
the definition of the document encoding. Thus, while
\verb|\SetDocumentEncodingBithe| is indeed classified as a
user-level command, it is certainly not necessary for everyday work.


\section{Basic Character Set and Romanization}

Given by dictionary order, the system provides a basic
character set as shown in table~\ref{table:ManjuBasicChars}.

\newcommand{\MaEntry}[3]{\mabosoo{#1}& #2 & #3 }
\begin{table}
\begin{center}
\begin{tabular}{ccc|ccc|ccc}
Manju&Input&Latin&Manju&Input&Latin&Manju&Input&Latin\\
\hline
\MaEntry{a}{a}{a}	& \MaEntry{h}{h}{h}	& \MaEntry{c}{c}{c}	\\
\MaEntry{e}{e}{e}	& \MaEntry{b}{b}{b}	& \MaEntry{j}{j}{j}	\\
\MaEntry{i}{i}{i}	& \MaEntry{p}{p}{p}	& \MaEntry{y}{y}{y}	\\
\MaEntry{o*}{o}{o}	& \MaEntry{s}{s}{s}	& \MaEntry{k'}{k'}{k'}	\\
\MaEntry{u*}{u}{u}	& \MaEntry{s'}{s'}{\v s}	& \MaEntry{g'}{g'}{g'}	\\
\MaEntry{v}{v}{\={u}}	& \MaEntry{t}{t}{t}	& \MaEntry{h'}{h'}{h'}	\\
\MaEntry{n}{n}{n}	& \MaEntry{d}{d}{d}	& \MaEntry{r}{r}{r}	\\
\MaEntry{k}{k}{k}	& \MaEntry{l}{l}{l}	& \MaEntry{f}{f}{f}	\\
\MaEntry{g}{g}{g}	& \MaEntry{m}{m}{m}	& \MaEntry{w}{w}{w}	\\
\end{tabular}
\caption{Manju Basic Character Set}\label{table:ManjuBasicChars}
\end{center}
\end{table}

While the input method for the majority of characters matches the
transliteration conventions, some letters require a slightly
different treatment:
\begin{enumerate}
	\item Although the diphtong \mabosoo{*aii*} is
		usually rendered as \textit{ai}, it must be entered
		as \texttt{aii} in order to produce the desired
		effect.
	\item The vowel which is conventionally rendered as \textit{\^u}
		or \textit{\=u} \mabosoo{v} can be entered as \texttt{v}
		or as \verb|\={u}| due to the fact that a character
		\textit{\^u} is not readily available on most systems.
	\item The consonant \textit{\v s} \mabosoo{s'} can be entered as
		\texttt{s'} or as \verb|\v{s}|, but not as *\texttt{sh}
		as to avoid undesired mergers of \textit{s} and \textit{h}
		like in \textit{ishun} \mabosoo{ishun} which should not be
		*\textit{i\v{s}un} \mabosoo{is'un}!
\end{enumerate}

\section{Extended Character Set}
The following special characters listed in major dictionaries are
provided:
\begin{center}
\begin{tabular}{ccc}
Manju	& Input &Latin\\
\MaEntry{sy}{sy}{sy}	\\
\MaEntry{cy}{cy}{cy}	\\
\MaEntry{j'}{j'}{jy}	\\
\MaEntry{dz}{dz}{dz}	\\
\MaEntry{tsh}{tsh}{tsh}	\\
\MaEntry{tshy}{tshy}{tshy}	\\
\MaEntry{zr}{zr}{zr}	\\
\end{tabular}
\end{center}

Please note that due to internal limitations of the retransliteration
engine, \textit{jy} \mabosoo{j'} has to be entered as \texttt{j'}.

\section{Tibetan Transliteration Character Set}
Besides these characters, an additional small set of special characters
is provided for rendering Tibetan and Uighur transliterations:

\begin{center}
\begin{tabular}{ccc}
Manju	& Input &Latin	\\
\MaEntry{z}{z}{z}		\\
\MaEntry{zh}{zh}{zh}	\\
\MaEntry{ts}{ts}{ts}	\\
\MaEntry{ng'}{ng'}{ng'}	\\
\MaEntry{l'}{l'}{l'}		\\
\MaEntry{p'}{p'}{p'}		\\
\MaEntry{t'}{t'}{t'}		\\
\end{tabular}
\end{center}


\newcommand{\MT}[2]{{\tib #1} \textit{#1} \mabosoo{#2}}

\ifx\tib\undefined
	\def\tib{\ttfamily}
\fi

\newcommand{\MTibetan}[8]{%
	\tib #1 & \mabosoo{#2}&
		\tib #3 & \mabosoo{#4}&
			\tib #5 & \mabosoo{#6}&
				\tib #7 & \mabosoo{#8}\\
	\tt #1 & \tt #2 &
		\tt #3 & \tt #4 &
			\tt #5 & \tt #6 &
				\tt #7 & \tt #8\\
		}

\begin{table}[h]
\begin{center}
\begin{tabular}{cc|cc|cc|cc}
\MTibetan{ka}{g'a}		{kha}{k'a}	{ga}{ga}	{nga}{ng'a}
\hline
\MTibetan{ca}{jiya}		{cha}{cia}	{ja}{ja}	{nya}{niya}
\hline
\MTibetan{ta}{t'a}		{tha}{ta}	{da}{da}	{na}{na}
\hline
\MTibetan{pa}{ba}		{pha}{pa}	{ba}{wa}	{ma}{ma}
\hline
\MTibetan{tsa}{tsa}		{tsha}{tsha}	{dza}{dza}	{wa}{wa}
\hline
\MTibetan{zha}{zha}		{za}{za}	{'}{ea}		{ya}{ya}
\hline
\MTibetan{ra}{ra}		{la}{la}	{sha}{s'a}	{sa}{sa}
\hline
\MTibetan{ha}{h|a}		{a}{a}		{}{}		{}{}
\end{tabular}
\caption{Tibetan Transliteration Character Set}\label{table:ManTibTrans}
\end{center}
\end{table}

This allows to spell out the Tibetan alphabet in Manju writing, as
used in the Pentaglot dictionary for Tibetan
(see table~\ref{table:ManTibTrans}) and Uighur transliterations. The
following rules apply:
\begin{enumerate}
	\item \MT{nga}{ng'a} (ma. \textit{ng'a}) is used for Tibetan
		initials and subscripts while
		finals are expressed as \mabosoo{*ng} (ma.  \textit{*ng});
	\item While \MT{ha}{h|a} is used for Tibetan initial
		{\tib ha},
		a different form is taken for subscripted
		\textit{ha}, as in \MT{lha}{l'|a} (ma.  \textit{l'a}).
\end{enumerate}

\subsection{Special Characters}

\enlargethispage*{4ex}
Manju shares with Mongolian the complete set of numbers and
punctuation marks as well as a few special characters used for
influencing the presentation of the writing. See also
section~\ref{section:SpecialMLSCharacters}.

Provided a word should end with a non-final glyph shape then the
Environment Marker \mabosoo{**} is used which is entered as an
asterisque \verb-*-. This is helpful for writing abbreviated words
or marking non-final vowels, like \mabosoo{o*} which is entered as
\verb-o*-.

Whenever the plethora of diacritics used in Manju writing causes
ugly clashes between adjacent letters, then the `backbone' (mong.
\textit{nirugu}), entered as \verb'|', can be used to stretch the
distance between clashing letter elements, like in \mabosoo{h|a}
which should be entered \verb-h|a- rather than \verb-ha- resulting
in \mabosoo{ha}.


%%%%%%%%%%%%%%%%%%%%%%%%%%%%%%%%%%%%%%%%%%%%%%%%%%%%%%%%%%%%%%%%
\chapter{Display Commands\label{chapter:DisplayCommands}}

Depending on the size of the Mongolian or Manju material to be
displayed, the user can choose between various commands and
environments which have a similar structure for both Mongolian and
Manju.

\section{Small Portions of Mongolian and Manju in Running Text}

For displaying short Mongolian snippets in running text
use \cmd{bicig}\verb|{...}|.

For displaying short Manju snippets in running text
use \cmd{bithe}\verb|{...}|.

\exa
	This is \bicig{munggul bicik}. 

	That is \bithe{manju bithe}.
\exb
	\begin{verbatim}
	This is \bicig{munggul bicik}.
	
	That is \bithe{manju bithe}.
	\end{verbatim}
\exc

\section{Horizontal Paragraphs of Mongolian or Manju Text}

If one needs more than a few words of Mongolian or Manju but does
not want to change the line orientation, then the environments
\cmda{bicigtext} for Mongolian (which should be entered in
Mongolian Simplified Transliteration) and \cmda{bithetext} for Manju are
useful.

\exa
	\begin{bicigtext}
	uindur gegen zanabazar.

	17..18 d'ugar zagun-u munggul-un
	neiigem, ulus tuiru, shasin-u uiiles-tu,
	ilangguy=a uralig-un kuikzil-du uncukui
	ekurge kuiicedgeksen uindur gegen
	zanabazar, cinggis xagan-u aldan
	urug-un izagur surbulzidan abadai
	saiin nuyan xan-u kuiu tuisiyedu xan
	gumbudurzi-yin ger-tu 1635 un-du
	tuiruksen.%
	\end{bicigtext}
\exb
	{\mdoublehyphenon
	\begin{verbatim}
	\begin{bicigtext}
	uindur gegen zanabazar.

	17..18 d'ugar zagun-u munggul-un
	neiigem, ulus tuiru, shasin-u
	uiiles-tu, ilangguy=a uralig-un
	kuikzil-du uncukui ekurge
	kuiicedgeksen uindur gegen
	zanabazar, cinggis xagan-u
	aldan urug-un izagur surbulzidan
	abadai saiin nuyan xan-u kuiu
	tuisiyedu xan gumbudurzi-yin
	ger-tu 1635 un-du tuiruksen.
	\end{bicigtext}
	\end{verbatim}}
\exc


\exa
	\begin{bithetext}
	han-i araha sunja
	hacin-i hergen kamciha
	manju gisun-i buleku
	bithe. abkai so\v{s}ohon.
	emu hacin. nadan meyen.%
	\end{bithetext}
\exb
	\begin{verbatim}
	\begin{bithetext}
	han-i araha sunja
	hacin-i hergen kamciha
	manju gisun-i buleku
	bithe. abkai so\v{s}ohon.
	emu hacin. nadan meyen.%
	\end{bithetext}
	\end{verbatim}
\exc

\section{Vertical Capsules}

%%  \section{Text in Vertical Capsules\label{VerticalCapsules}}\label{bosoo}
%%  
%%  With PostScript support, it is possible to create vertical capsules
%%  containing short portions of text (typically one or two words);
%%  the commands \verb"\bosoo{...}" does not change the current encoding
%%  and can be used for non-Mongolian words whereas \verb"\mbosoo{...}"
%%  sets the encoding temporarily to \LMS. Accepting the same input as
%%  \verb"\bcg{...}", writing \verb"\mbosoo" is simply a shortcut for writing
%%  \verb"\bosoo{\bcg{...}}" and facilitates the printing of individual
%%  Mongolian words in vertical appearance.
%%  
%%  
%%  capsules containing one word or a short series of words,
%%  See chapter \ref{VerticalCapsules}
%%  for the necessary commands. Vertical capsules like these \mbosoo{mongGol}
%%  \mbosoo{bicig} (\textit{mong\g ol bicig}) can appear anywhere in your
%%  text; line spacing etc. will adapt automatically. It is also
%%  possible to create margin notes with Mongolian Script.
%%  
%%  
Individual Mongolian and Manju words can be placed vertically
anywhere in otherwise horizontal text like in
the keyword entry of dictionaries.\footnote{Famous dictionaries with
	a mixture of vertical and horizontal printing are I.~J.~Schmidt's
	Mongolian-Russian-German dictionary (1835) and F.~Lessing's
	Mongolian-English dictionary (1960).}.
The capsule containing the
Mongolian or Manju word will automatically request sufficient space
so that ugly overlaps with neighbouring lines will not happen.

For presenting text given in broad (or MLS) transliteration, use the command
\cmd{mbosoo}\verb|{...}|; when writing in Mongolian Simplified
Transliteration, use \cmd{mobosoo}\verb|{...}|; likewise for Manju, use
\cmd{mabosoo}\verb|{...}|. All these commands are derived from a
command \cmd{bosoo}\verb|{...}|
				which places text in vertical
capsules but leaves the contents untouched as far as the encoding is
concerned.\marginpar{%
\mbosoo{mongGol}\mbosoo{bicig}

		\vspace{2.54mm}

		\raggedright\small
		without PostScript support Mongolian text enclosed in
		vertical capsules will be printed \emph{horizontally}!}

\begin{figure}
\exa
	This is \bosoo{vertical}
		\bosoo{text}. 
	This is \mbosoo{mongGol}
		\mbosoo{bicig}, 
	this is \mobosoo{munggul}
		\mobosoo{bicik}, 
	that is \mabosoo{manju}
		\mabosoo{bithe}.
\exb
	\begin{verbatim}
	This is \bosoo{vertical}
		\bosoo{text}. 
	This is \mbosoo{mongGol}
		\mbosoo{bicig}, 
	this is \mobosoo{munggul}
		\mobosoo{bicik}, 
	that is \mabosoo{manju}
		\mabosoo{bithe}.
	\end{verbatim}
\exc
\caption{Vertical Text Capsules}\label{figure:VerticalTextCapsules}
\end{figure}


\section{Vertical Text Boxes}

For presenting individual paragraphs of Mongolian or Manju text in
vertical manner in an otherwise horizontal text, there are the box
commands \cmd{mobox}\verb|{...}{...}| for Mongolian%
\footnote{Mongolian input \emph{must} be coded in Mongolian Simplified
	Transliteration; MLS input won't work.}
and
\cmd{mabox}\verb|{...}{...}|
for Manju. These boxes take two arguments. The first argument
indicates the \textit{vertical depth} of the box, or its line
length. The second argument contains the desired text. An example
is shown in figure~\ref{figure:VerticalTextBox} for Mongolian, and
below for Manju.

\begin{figure}[h]
\exa
	\mobox{7.5cm}{%
%	uindur gegen zanabazar.
%
	17..18 d'ugar zagun-u munggul-un
	neiigem, ulus tuiru, shasin-u uiiles-tu,
	ilangguy=a uralig-un kuikzil-du uncukui
	ekurge kuiicedgeksen uindur gegen
	zanabazar, cinggis xagan-u aldan
	urug-un izagur surbulzidan abadai
	saiin nuyan xan-u kuiu tuisiyedu xan
	gumbudurzi-yin ger-tu 1635 un-du
	tuiruksen.%
	}
\exb
	%\vskip-9cm
	{\mdoublehyphenon
	\begin{verbatim}
	\mobox{7.5cm}{%
	17..18 d'ugar zagun-u munggul-un
	neiigem, ulus tuiru, shasin-u
	uiiles-tu, ilangguy=a uralig-un
	kuikzil-du uncukui ekurge
	kuiicedgeksen uindur gegen
	zanabazar, cinggis xagan-u
	aldan urug-un izagur surbulzidan
	abadai saiin nuyan xan-u kuiu
	tuisiyedu xan gumbudurzi-yin
	ger-tu 1635 un-du tuiruksen.%
	}
	\end{verbatim}}
\exc
\caption{A Vertical Text Box}\label{figure:VerticalTextBox}
\end{figure}


%\begin{figure}[h]
\exa
	\mabox{3.75cm}{%
	\noindent\raggedleft
	han-i araha sunja
	hacin-i hergen kamciha
	manju gisun-i buleku
	bithe. abkai so\v{s}ohon.
	emu hacin. nadan meyen.%
	}
\exb
	%\vskip-5.25cm
	\begin{verbatim}
	\mabox{3.75cm}{%
	\noindent\raggedleft
	han-i araha sunja
	hacin-i hergen kamciha
	manju gisun-i buleku
	bithe. abkai so\v{s}ohon.
	emu hacin. nadan meyen.%
	}
	\end{verbatim}
\exc
%\caption{A Manju Vertical Text Box\label{example:VerticalTextBox}}
%\end{figure}


\section{Full Vertical Text Pages}

If you need several pages of Mongolian output, enclose your text
in an environment \cmda{bicigpage}, and use \cmda{bithepage}
likewise for Manju texts. Note that Mongolian must be entered in
Simplified Transliteration.

Finally, if you want the whole document and its basic language to be
Classical, or Uighur Mongolian, say \verb|\usepackage[bicig,...]{mls}|.
Likewise, complete Manju documents are produced with 
\verb|\usepackage[bithe,...]{mls}|.

If you start a document with a \verb|\usepackage[bicig]{mls}|
declaration you can still switch back to Latin by issuing an
\verb|\end{bicigpage}| command.

Likewise, if you start a document with a \verb|\usepackage[bithe]{mls}|
declaration you can still switch back to Latin by issuing an
\verb|\end{bithepage}| command.

The following snippet of Mongolian text is presented in full
page mode on the next pages, first in Simplified Transliteration form, then
in Uighur form; in order to achieve this result the text had to be
included in the environment \texttt{bicigpage}.

\noindent
\begin{figure}
\begin{verbatim}
\begin{bicigpage}
uindur gegen zanabazar.

17||18 d'ugar zagun-u munggul-un neiigem, ulus tuiru, shasin-u
uiiles-tu, ilangguy=a uralig-un kuikzil-du uncugui ekurge
kuiicedgeksen uindur gegen zanabazar, cinggis xagan-u aldan
urug-un izagur surbulzidan abadai saiin nuyan xan-u kuiu
tuisiyedu xan gumbudurzi-yin ger-tu 1635 un-du tuiruksen.
badum�ngke daiyan xagan-u 6-d'aki uiy=e-yin kuimun. gurban
nasudai-d'agan num ungsizu enedkek gazar tuibed kele-yi xar=a
ayandagan surcu, keuked axui cag-aca erdem num-un duiri-tei
bulugsan zanabazar 15 nasu-tai-dagan baragun zuu (lhasa)
uruzu tabudugar dalai lam=a-d'u shabilan saguzu, ulamar
zebCundamba-yin xubilgan tudurazei. uran barimalci, zirugaci,
kele sinzigeci, uran barilgaci, kuin uxagandan zanabazar ulan
zagun zil-un daiin tululdugan-d'u nerbekden suliduzu, zugsunggi
baiidal-d'u urugsan dumdadu zagun-u munggul-un suyul uralig-i
serkun manduxu-d'u yeke xubi nemekuri urugulugsan yum. tekun-u
abiyas bilig nuiri yeke kuidelmuri-ber munggul-un uralig nigen
uiy=e tanigdasi uigei uindurlik-tu kuiruksen azei. xarin 1654
un-d'u neiislel kuiriyen-u tulg=a-yin cilagu-yi tabilcagsan
zanabazar-un uran barilg=a-yin buidugel-ece uinudur-i uizeksen
zuiil barug uigei ni xaramsaldai. zanabazar uindesun-u bicig
uisuk-i kuikzikulku-d'u beyecilen urulcazu, suyungbu uisuk-i
zukiyazu ene uiy=e suyungbu ni man-u tusagar tugdanil-un belge
temdek bulugsagar baiin=a.  tere-ber <<cag-i tukinagulugci>>
gedek silukleksen zukiyal-d'agan arad tuimen-u-ben engke
amugulang, saiin saiixan-i imagda kuisen muirugedezu yabudag
sedkil-un-iien uige-i ilerkeiileksen baiidag. uindur gegen
duirsuleku uralig-un xubi-d'u uirun=e-yin sunggudag-ud-tai
eng zergeceku buidugel-tei kuimun abacu basa xari ulus-un
buzar bacir arg=a-d'u abdagdan yabugsan nigen.
...
... more text ...
              ...
\end{bicigpage}
\end{verbatim}
\caption{Input Example of a Mongolian Text}\label{figure:InputMongolianExample}
\end{figure}

%%%%%%%%%%%%%%%%%%%%%%%%%%%%%%%%%%%%%%%%%%%%%%%%%%%%%%%%%%%%%%%%
\begin{bicigpage}
	uindur gegen zanabazar.

	17||18 d'ugar zagun-u munggul-un neiigem, ulus tuiru,
	shasin-u uiiles-tu, ilangguy=a uralig-un kuikzil-du
	uncugui ekurge kuiicedgeksen uindur gegen zanabazar,
	cinggis xagan-u aldan urug-un izagur surbulzidan abadai
	saiin nuyan xan-u kuiu tuisiyedu xan gumbudurzi-yin
	ger-tu 1635 un-du tuiruksen.  badum�ngke daiyan xagan-u
	6-d'aki uiy=e-yin kuimun. gurban nasudai-d'agan num
	ungsizu enedkek gazar tuibed kele-yi xar=a ayandagan
	surcu, keuked axui cag-aca erdem num-un duiri-tei
	bulugsan zanabazar 15 nasu-tai-dagan baragun zuu
	(lhasa) uruzu tabudugar dalai lam=a-d'u shabilan
	saguzu, ulamar zebCundamba-yin xubilgan tudurazei. uran
	barimalci, zirugaci, kele sinzigeci, uran barilgaci,
	kuin uxagandan zanabazar ulan zagun zil-un daiin
	tululdugan-d'u nerbekden suliduzu, zugsunggi
	baiidal-d'u urugsan dumdadu zagun-u munggul-un suyul
	uralig-i serkun manduxu-d'u yeke xubi nemekuri
	urugulugsan yum. tekun-u abiyas bilig nuiri yeke
	kuidelmuri-ber munggul-un uralig nigen uiy=e tanigdasi
	uigei uindurlik-tu kuiruksen azei. xarin 1654 un-d'u
	neiislel kuiriyen-u tulg=a-yin cilagu-yi tabilcagsan
	zanabazar-un uran barilg=a-yin buidugel-ece uinudur-i
	uizeksen zuiil barug uigei ni xaramsaldai. zanabazar
	uindesun-u bicig uisuk-i kuikzikulku-d'u beyecilen
	urulcazu, suyungbu uisuk-i zukiyazu ene uiy=e suyungbu
	ni man-u tusagar tugdanil-un belge temdek bulugsagar
	baiin=a.  tere-ber <<cag-i tukinagulugci>> gedek
	silukleksen zukiyal-d'agan arad tuimen-u-ben engke
	amugulang, saiin saiixan-i imagda kuisen muirugedezu
	yabudag sedkil-un-iien uige-i ilerkeiileksen
	baiidag. uindur gegen duirsuleku uralig-un xubi-d'u
	uirun=e-yin sunggudag-ud-tai eng zergeceku buidugel-tei
	kuimun abacu basa xari ulus-un buzar bacir arg=a-d'u
	abdagdan yabugsan nigen.

	munggul-d'u urcigulxu uxagan yeke delgerezu baiigsan ni
	man-u erden ba dumdadu uiy=e-yin suyul-un nigen uncalig
	azei. erden-u enedkek-un kuin uxagan-u iragu naiirag,
	kele bicik-un sudulul, anagaxu uxagan, uralaxu uxagan
	zerge tabun uxagan-u zukiyal-i bagdagagsan buikude 334
	budi <<ganzuur>>, <<danzuur>>-i num-un mergen bagsi
	kuinggaudsar terikudei 64 erdemden lama urcigulun
	neiideleksen baiin=a. 400 zil-un terdege urcigulg=a-yin
	iimu eke kuiriyeleng munggul-d'u azillazu baiigsan-i
	tuisugelen buduxu-d'u baxadai. munggulcud erden-ece
	inagsi daguu xugur-tai buizik nagadum-tai xurdun
	kuiluk murid-tai. er=e-yin gurban nagadum-i erkimelen
	kuikzilduzu, ide xabu-ban bulgazu ireksen baiin=a.
	munggul-un zirgalang ni buizik, xurim bile. xudala-i
	xagan-d'u erkumzileged xurxunag-un sagalagar mudun-u
	duur=a xabirg=a gazar-i xalcaradal=a, ebuduk gazar-i
	uilduredel=e debkecen buiziklezu xurimlaba gesen uige
	<<niguca tubciyan>>-d'u bui. munggul arad-un medelge
	uxagan erde-ece inagsi mal azu axui, udun urun,
	gazar zuii, anagaxu uxagan, baiigali, neiigem-un
	ulan salburi-bar kuikzizu irebe.  <<aldan tubci>>,
	<<erdeni-yin tubci>>, <<bulur tuli>>, <<subud erike>>
	medu teuke-yin ulan arban zukiyal gargazei.

	manzu nar <<munggul uyun>>-i muikugeku-yi kedui-ber
	uruldubacu uyun bilikdu, cecen celmek, erdem uxagandan
	tuduran garugsagar baiiba.  19-d'uger zagun bul
	iragu naiiragci dangzirabzai. yeke zukiyalci inzinasi
	dangzigvangzil nar-un amidurazu, buidugezu baiigsan
	uiy=e bile.
\end{bicigpage}

\section{Pure Uighur Mongolian and Manju Documents}

Writing a complete document in Mongolian or Manju is as simple and
straightforward as writing a document in English or Xalx Mongolian.

The example file, \texttt{zanabazr.tex}
(shipped together with this documentation and located in the
directory \texttt{../examples/}) demonstrates how a pure
Mongolian Bicig document can be created.

\begin{figure}[h]
\begin{verbatim}
\documentclass{article}
\usepackage[bicig]{mls}
\begin{document}
uindur gegen zanabazar.

17||18 d'ugar zagun-u munggul-un neiigem, ulus tuiru,
shasin-u uiiles-tu, ilangguy=a uralig-un kuikzil-du
...
... more text ...
              ...
\end{document}
\end{verbatim}
\end{figure}

The concept is the same for Manju documents: instead of \verb|bicig|
one would use the \verb|\usepackage[...]{mls}| option \verb|bithe|
and enter Manju text.

\section{Font Selection Commands}

There are two distinct styles of Mongolian script: one style
is typically used for modern print, whereas the other style
appears in old block prints and stone inscriptions.

Since there is no proper correspondance between Latin and Mongolian
typographical features, a somewhat arbitrary assignment was made
to the effect that the block print style can be activated by
setting the font family sans serif with \cmd{sffamily}. In 
contrast, setting the roman default family with \cmd{rmfamily}
switches back to the modern style.

\begin{figure}
\exa
	\mobox{2cm}{\noindent
	munggul\\
	\sffamily munggul\\
	\rmfamily munggul
	}
\exb
	\begin{verbatim}
	\mobox{2cm}{\noindent
	munggul\\
	\sffamily munggul\\
	\rmfamily munggul}
	\end{verbatim}
\exc
\caption{Mongolian Font Styles}\label{figure:MongolianFontStyles}
\end{figure}

The same two commands can be applied to Manju, too. In this context
it makes sense to assign, e.\,g., \verb|\sffamily| to Mongolian and
\verb|\rmfamily| (which is the default anyway) to Manju. At one
glance one can tell which writing represents which language.

\textbf{Nota Bene}: The MLS-related Mongolian display commands
are internally limited to the sans serif, or block print style,
so that there is always a clear visual distinction possible which
input mode was chosen.
%%%%%%%%%%%%%%%%%%%%%%%%%%%%%%%%%%%%%%%%%%%%%%%%%%%%%%%%%%%%%%%%
%%%%%%%%%%%%%%%%%%%%%%%%%%%%%%%%%%%%%%%%%%%%%%%%%%%%%%%%%%%%%%%%

\chapter{\MonTeX\ Software Internals}

\section{\MonTeX\ System Layout}

\MonTeX\ consists of many files each performing dedicated functions.
These files are listed here in systematical order.


\subsection{Main Package}

The main package is \texttt{mls.sty}. The RL capabilities are
provided by \texttt{rlbicig.sty}.

\subsection{Hyphenation Patterns}
{\parindent=0pt
\begin{verbatim}
texinput/mnhyphen.tex   # Modern Mongolian
texinput/mnhyphex.tex   # Modern Mongolian Exceptions
\end{verbatim}}

\subsection{Transliteration Engines}
{\parindent=0pt
\begin{verbatim}
texinput/mlstrans.tex   # Main Transliteration Engine
texinput/mlsgalig.tex   # Latin Presentation Engine
\end{verbatim}}


\subsection{Input Encodings}
{\parindent=0pt
\begin{verbatim}
texinput/cpctt.def
texinput/cpdbk.def
texinput/cpibmrus.def
texinput/cpkoi.def
texinput/cpmls.def
texinput/cpmnk.def
texinput/cpmos.def
texinput/cpncc.def
\end{verbatim}}


\subsection{Output or Font Encodings}
{\parindent=0pt
\begin{verbatim}
texinput/lmaenc.def     # Manju
texinput/lmcenc.def     # Cyrillic
texinput/lmoenc.def     # Mongolian (Simplified Input)
texinput/lmsenc.def     # Mongolian Script (deprecated)
texinput/lmuenc.def     # Traditional Mongolian Glyph Container
\end{verbatim}}

\subsection{Caption Translations}
{\parindent=0pt
\begin{verbatim}
texinput/bicig.def      # Mongolian
texinput/bithe.def      # Manju
texinput/buryat.def     # Buryat
texinput/english.def    # English
texinput/kazakh.def     # Kazakh, implementation pending
texinput/russian.def    # Russian
texinput/xalx.def       # Modern Mongolian
\end{verbatim}}

\subsection{Font Definitions}
{\parindent=0pt
\begin{verbatim}
texinput/lmabthhs.fd    # Manju horizontal 'steel'
texinput/lmabthhw.fd    # Manju horizontal 'wood' 
texinput/lmabthvs.fd    # Manju vertical 'steel' 
texinput/lmabthvw.fd    # Manju vertical 'wood'  
texinput/lmccmdh.fd     # Cyrillic Dunhill
texinput/lmccmfib.fd    # Cyrillic Fibonacci
texinput/lmccmfr.fd     # 
texinput/lmccmiss.fd    # 
texinput/lmccmr.fd      # Cyrillic CM Roman
texinput/lmccmss.fd     # Cyrillic CM Sans Serif
texinput/lmccmssq.fd    # Cyrillic CM Sans Serif Quotes
texinput/lmccmtt.fd     # Cyrillic CM TeleType
texinput/lmccmvtt.fd    # Cyrillic CM Variable TeleType
texinput/lmclcmss.fd    # 
texinput/lmobcghs.fd    # Mongolian horizontal 'steel'
texinput/lmobcghw.fd    # Mongolian horizontal 'wood' 
texinput/lmobcgvs.fd    # Mongolian vertical 'steel' 
texinput/lmobcgvw.fd    # Mongolian vertical 'wood'  
texinput/lmsbcgh.fd     # 
texinput/lmsbcgv.fd     # 
texinput/lmubxghs.fd    # Glyph container horizontal 'steel'
texinput/lmubxghw.fd    # Glyph container horizontal 'wood' 
texinput/lmubxgvs.fd    # Glyph container vertical 'steel' 
texinput/lmubxgvw.fd    # Glyph container vertical 'wood'  
\end{verbatim}}


\subsection{Miscellae}
{\parindent=0pt
\begin{verbatim}
texinput/mtdocmac.tex   # Macro collection for this document
texinput/TODO           # The Author's To Do List
\end{verbatim}}

\section{\MonTeX\ Mongolian Font Layout}

\begin{sloppypar}
Mongolian and Manju fonts are generated from common sources in
\texttt{mfinput/bcgbase}. Mongolian-specific material is kept in
\texttt{mfinput/bicig}, Manju-specific material is kept in
\texttt{mfinput/bithe}. All Mongolian fonts can be used for RL
\emph{and} for LR typesetting. Individual font names are best
described by the following regular expression:

	$$ (bcg|bth|bxg)[hv][sw][mb] $$

Here, \emph{bcg} stands for Mongolian, \emph{bth} for Manju and
\emph{bxg} for the generic Mongolian glyph container. The next
letter indicates whether the material is to be typeset
\emph{h}orizontally or \emph{v}ertically. The next letter indicates
the typeface: \emph{s}teel or \emph{w}. The last letter indicates a
\emph{m}edium or \emph{b}old font.
\end{sloppypar}

\section{\texttt{bxg}: A Generic Mongolian Glyph Container}

Besides dedicated fonts for Mongolian and Manju, the \MonTeX\
font system offers a generic glyph container which is accessible
through the \LMU\ encoding. The name of this glyph container is
\texttt{bxg}, and all glyphs (the superset of Mongolian and Manju)
are available in both font families (block print and modern print
styles) of the \LMO\ and \LMA\ encodings. Please note that at
present there is no working ligature mechanism associated with
\texttt{bxg}; hence it cannot be used for general-purpose text at
the moment.

In the future, the \texttt{bxg} generic glyph container will
manage the Unicode interface.

\section{Unicode Mongolian and \MonTeX\label{Unicode}}

In the present version, a first attempt was made to provide Unicode 
compatibility. Please note that at this stage the Unicode of
\MonTeX\ is purely experimental!

Unicode-encoded Traditional Mongolian is located at
plane \texttt{U+1800} and contains canonical characters for
Mongolian, Sibe, Manju and Todo. There is also a rich collection of
Ali Gali (or Galig) characters used for transliterating Sanskrit,
Tibetan and other languages into Mongolian, Manju etc.

\MonTeX\ covers a subset of Unicode Traditional Mongolian which is
sufficient to typeset modern Mongolian and Manju texts as well as a
choice of Tibetan words transliterated in Manju (as in the Pentaglot
dictionary, e.\,g.).

At the moment, language-specific groups of Unicode characters are
mirrored into the related encodings. It is therefore necessary to
tag the desired language with the \texttt{SetDocumentEncoding<...>}
command in order to achieve the appropriate ligature behaviour.

The availability of individual Unicode Mongolian characters and
their canonical names are shown in table~\ref{table:UnicodeMongolian}.

The astute observer will note several discrepancies between the
official Unicode standard documentation and this particular,
\emph{experimental} implementation:

\begin{enumerate}
	\item	Canonical letter shapes differ from those shown in
		the standard documentation. In context, however, the
		characters behave as they should.

	\item	There is not yet a third MVS in \MonTeX.

	\item	The Mongolian front vowels are not yet treated
		properly.

	\item	The complete Todo range of characters is missing.

	\item	Most of the Mongolian Ali Gali (Galig) characters
		are missing; there are, however some Manju Galig
		characters.

	\item	Unicode decided to choose a special space to
		separate morpheme boundaries; this character is
		\emph{not} part of the Traditional Mongolian plane
		(sic!). Also, this character is defined as a
		non-breaking space, which contrasts with the
		understanding of the \MonTeX\ authors.
\end{enumerate}

This list of differences between Unicode Traditional Mongolian and
\MonTeX\ Mongolian and Manju is incomplete.

%
% Key: 	0 no language
%	1 available in LMO
%	2 available in LMA
%	3 = 1 + 2 -> available in LMO and LMA
%	4 available in Todo
%	8 available in Symbols
%
\newcommand{\UM}[5]{%
	% 1. Unicode Code Position
	\texttt{U+#1}&
	% 2. Availability in MonTeX; key see above
	\ifcase#2 	% 0
	        \or\mobosoo{\csname #4\endcsname\noboundary}% 1
	        \or\mabosoo{\csname #4\endcsname\noboundary}% 2
	        \or\mobosoo{\csname #4\endcsname\noboundary}% 3
	\else \fi &
	% 3. Unicode Name
	\texttt{\small\ifnum#2=0(\fi#3\ifnum#2=0)\fi}\newline
	% 4. MonTeX Name
	\texttt{\small\char92 #4}&
	% 5. MonTeX Encoding
	\ifnum#2=0 (n.\,a.)\else #5\fi\\
}
\noindent
\begin{longtable}{rcp{8cm}l}
 Code	&Character&Unicode Name\newline\MonTeX\ Name& \MonTeX\ Encoding\\
\hline
\UM{1800}{0}{MONGOLIAN BIRGA}
	{textmongolianbirga}{}
\UM{1801}{3}{MONGOLIAN ELLIPSIS}
	{textmongolianellipsis}{LMO, LMA}
\UM{1802}{1}{MONGOLIAN COMMA}
	{textmongoliancomma}{LMO}
\UM{1803}{1}{MONGOLIAN FULL STOP}
	{textmongolianfullstop}{LMO}
\UM{1804}{3}{MONGOLIAN COLON}
	{textmongoliancolon}{LMO, LMA}
\UM{1805}{3}{MONGOLIAN FOUR DOTS}
	{textmongolianfourdots}{LMO, LMA}
\UM{1806}{0}{MONGOLIAN TODO SOFT HYPHEN}
	{textmongoliantodosofthyphen}{}
\UM{1807}{0}{MONGOLIAN SIBE SYLLABLE BOUNDARY MARKER}
	{textmongoliansibesyllableboundarymarker}{}
\UM{1808}{2}{MONGOLIAN MANCHU COMMA}
	{textmongolianmanchucomma}{LMA}
\UM{1809}{2}{MONGOLIAN MANCHU FULL STOP}
	{textmongolianmanchufullstop}{LMA}
\UM{180A}{3}{MONGOLIAN NIRUGU}
	{textmongoliannirugu}{LMO, LMA}
\UM{180B}{3}{MONGOLIAN FREE VARIATION SELECTOR ONE}
	{textmongolianfreevariationselectorone}{LMO, LMA}
\UM{180C}{3}{MONGOLIAN FREE VARIATION SELECTOR TWO}
	{textmongolianfreevariationselectortwo}{LMO, LMA}
\UM{180D}{0}{MONGOLIAN FREE VARIATION SELECTOR THREE}
	{textmongolianfreevariationselectorthree}{}
\UM{180E}{3}{MONGOLIAN VOWEL SEPARATOR}
	{textmongolianvowelseparator}{LMO, LMA}
%%%%%%%%%%%%%%%%%%%%%%%%%%%%%%%%%%%%%%%%%%%%%%%%%%%%%%%%%%%%%%%%
\UM{1810}{3}{MONGOLIAN DIGIT ZERO}
	{textmongolianzero}{LMO, LMA}
\UM{1811}{3}{MONGOLIAN DIGIT ONE}
	{textmongolianone}{LMO, LMA}
\UM{1812}{3}{MONGOLIAN DIGIT TWO}
	{textmongoliantwo}{LMO, LMA}
\UM{1813}{3}{MONGOLIAN DIGIT THREE}
	{textmongolianthree}{LMO, LMA}
\UM{1814}{3}{MONGOLIAN DIGIT FOUR}
	{textmongolianfour}{LMO, LMA}
\UM{1815}{3}{MONGOLIAN DIGIT FIVE}
	{textmongolianfive}{LMO, LMA}
\UM{1816}{3}{MONGOLIAN DIGIT SIX}
	{textmongoliansix}{LMO, LMA}
\UM{1817}{3}{MONGOLIAN DIGIT SEVEN}
	{textmongolianseven}{LMO, LMA}
\UM{1818}{3}{MONGOLIAN DIGIT EIGHT}
	{textmongolianeight}{LMO, LMA}
\UM{1819}{3}{MONGOLIAN DIGIT NINE}
	{textmongoliannine}{LMO, LMA}
%%%%%%%%%%%%%%%%%%%%%%%%%%%%%%%%%%%%%%%%%%%%%%%%%%%%%%%%%%%%%%%%
\UM{1820}{3}{MONGOLIAN LETTER A}
	{textmongoliana}{LMO, LMA}
\UM{1821}{1}{MONGOLIAN LETTER E}
	{textmongoliane}{LMO}
\UM{1822}{3}{MONGOLIAN LETTER I}
	{textmongoliani}{LMO, LMA}
\UM{1823}{3}{MONGOLIAN LETTER O}
	{textmongoliano}{LMO, LMA}
\UM{1824}{1}{MONGOLIAN LETTER U}
	{textmongolianu}{LMO}
\UM{1825}{1}{MONGOLIAN LETTER OE}
	{textmongolianoe}{LMO}
\UM{1826}{1}{MONGOLIAN LETTER UE}
	{textmongolianue}{LMO}
\UM{1827}{1}{MONGOLIAN LETTER EE}
	{textmongolianee}{LMO}
\UM{1828}{3}{MONGOLIAN LETTER NA}
	{textmongolianna}{LMO, LMA}
\UM{1829}{3}{MONGOLIAN LETTER ANG}
	{textmongolianang}{LMO, LMA}
\UM{182A}{3}{MONGOLIAN LETTER BA}
	{textmongolianba}{LMO, LMA}
\UM{182B}{1}{MONGOLIAN LETTER PA}
	{textmongolianpa}{LMO}
\UM{182C}{1}{MONGOLIAN LETTER QA}
	{textmongolianqa}{LMO}
\UM{182D}{1}{MONGOLIAN LETTER GA}
	{textmongolianga}{LMO}
\UM{182E}{3}{MONGOLIAN LETTER MA}
	{textmongolianma}{LMO, LMA}
\UM{182F}{3}{MONGOLIAN LETTER LA}
	{textmongolianla}{LMO, LMA}
\UM{1830}{3}{MONGOLIAN LETTER SA}
	{textmongoliansa}{LMO, LMA}
\UM{1831}{1}{MONGOLIAN LETTER SHA}
	{textmongoliansha}{LMO}
\UM{1832}{1}{MONGOLIAN LETTER TA}
	{textmongolianta}{LMO}
\UM{1833}{1}{MONGOLIAN LETTER DA}
	{textmongolianda}{LMO}
\UM{1834}{3}{MONGOLIAN LETTER CHA}
	{textmongoliancha}{LMO, LMA}
\UM{1835}{3}{MONGOLIAN LETTER JA}
	{textmongolianja}{LMO, LMA}
\UM{1836}{3}{MONGOLIAN LETTER YA}
	{textmongolianya}{LMO, LMA}
\UM{1837}{1}{MONGOLIAN LETTER RA}
	{textmongolianra}{LMO}
\UM{1838}{1}{MONGOLIAN LETTER WA}
	{textmongolianwa}{LMO}
\UM{1839}{1}{MONGOLIAN LETTER FA}
	{textmongolianfa}{LMO}
\UM{183A}{1}{MONGOLIAN LETTER KA}
	{textmongolianka}{LMO}
\UM{183B}{1}{MONGOLIAN LETTER KHA}
	{textmongoliankha}{LMO}
\UM{183C}{1}{MONGOLIAN LETTER TSA}
	{textmongoliantsa}{LMO}
\UM{183D}{1}{MONGOLIAN LETTER ZA}
	{textmongolianza}{LMO}
\UM{183E}{1}{MONGOLIAN LETTER HAA}
	{textmongolianhaa}{LMO}
\UM{183F}{1}{MONGOLIAN LETTER ZRA}
	{textmongolianzra}{LMO}
\UM{1840}{1}{MONGOLIAN LETTER LHA}
	{textmongolianlha}{LMO}
\UM{1841}{1}{MONGOLIAN LETTER ZHI}
	{textmongolianzhi}{LMO}
\UM{1842}{1}{MONGOLIAN LETTER CHI}
	{textmongolianchi}{LMO}
%%%%%%%%%%%%%%%%%%%%%%%%%%%%%%%%%%%%%%%%%%%%%%%%%%%%%%%%%%%%%%%%
\UM{1843}{0}{MONGOLIAN LETTER TODO LONG VOWEL SIGN}
	{textmongoliantodolongvowelsign}{}
\UM{1844}{0}{MONGOLIAN LETTER TODO E}
	{textmongoliantodoe}{}
\UM{1845}{0}{MONGOLIAN LETTER TODO I}
	{textmongoliantodoi}{}
\UM{1846}{0}{MONGOLIAN LETTER TODO O}
	{textmongoliantodoo}{}
\UM{1847}{0}{MONGOLIAN LETTER TODO U}
	{textmongoliantodou}{}
\UM{1848}{0}{MONGOLIAN LETTER TODO OE}
	{textmongoliantodooe}{}
\UM{1849}{0}{MONGOLIAN LETTER TODO UE}
	{textmongoliantodoue}{}
\UM{184A}{0}{MONGOLIAN LETTER TODO ANG}
	{textmongoliantodoang}{}
\UM{184B}{0}{MONGOLIAN LETTER TODO BA}
	{textmongoliantodoba}{}
\UM{184C}{0}{MONGOLIAN LETTER TODO PA}
	{textmongoliantodopa}{}
\UM{184D}{0}{MONGOLIAN LETTER TODO QA}
	{textmongoliantodoqa}{}
\UM{184E}{0}{MONGOLIAN LETTER TODO GA}
	{textmongoliantodoga}{}
\UM{184F}{0}{MONGOLIAN LETTER TODO MA}
	{textmongoliantodoma}{}
\UM{1850}{0}{MONGOLIAN LETTER TODO TA}
	{textmongoliantodota}{}
\UM{1851}{0}{MONGOLIAN LETTER TODO DA}
	{textmongoliantododa}{}
\UM{1852}{0}{MONGOLIAN LETTER TODO CHA}
	{textmongoliantodocha}{}
\UM{1853}{0}{MONGOLIAN LETTER TODO JA}
	{textmongoliantodoja}{}
\UM{1854}{0}{MONGOLIAN LETTER TODO TSA}
	{textmongoliantodotsa}{}
\UM{1855}{0}{MONGOLIAN LETTER TODO YA}
	{textmongoliantodoya}{}
\UM{1856}{0}{MONGOLIAN LETTER TODO WA}
	{textmongoliantodowa}{}
\UM{1857}{0}{MONGOLIAN LETTER TODO KA}
	{textmongoliantodoka}{}
\UM{1858}{0}{MONGOLIAN LETTER TODO GAA}
	{textmongoliantodogaa}{}
\UM{1859}{0}{MONGOLIAN LETTER TODO HAA}
	{textmongoliantodohaa}{}
\UM{185A}{0}{MONGOLIAN LETTER TODO JIA}
	{textmongoliantodojia}{}
\UM{185B}{0}{MONGOLIAN LETTER TODO NIA}
	{textmongoliantodonia}{}
\UM{185C}{0}{MONGOLIAN LETTER TODO DZA}
	{textmongoliantododza}{}
%%%%%%%%%%%%%%%%%%%%%%%%%%%%%%%%%%%%%%%%%%%%%%%%%%%%%%%%%%%%%%%%
\UM{185D}{2}{MONGOLIAN LETTER SIBE E}
	{textmongoliansibee}{LMA}
\UM{185E}{0}{MONGOLIAN LETTER SIBE I}
	{textmongoliansibei}{}
\UM{185F}{0}{MONGOLIAN LETTER SIBE IY}
	{textmongoliansibeiy}{}
\UM{1860}{2}{MONGOLIAN LETTER SIBE UE}
	{textmongoliansibeue}{LMA}
\UM{1861}{2}{MONGOLIAN LETTER SIBE U}
	{textmongoliansibeu}{LMA}
\UM{1862}{2}{MONGOLIAN LETTER SIBE ANG}
	{textmongoliansibeang}{LMA}
\UM{1863}{0}{MONGOLIAN LETTER SIBE KA}
	{textmongoliansibeka}{}
\UM{1864}{2}{MONGOLIAN LETTER SIBE GA}
	{textmongoliansibega}{LMA}
\UM{1865}{2}{MONGOLIAN LETTER SIBE HA}
	{textmongoliansibeha}{LMA}
\UM{1866}{2}{MONGOLIAN LETTER SIBE PA}
	{textmongoliansibepa}{LMA}
\UM{1867}{2}{MONGOLIAN LETTER SIBE SHA}
	{textmongoliansibesha}{LMA}
\UM{1868}{2}{MONGOLIAN LETTER SIBE TA}
	{textmongoliansibeta}{LMA}
\UM{1869}{2}{MONGOLIAN LETTER SIBE DA}
	{textmongoliansibeda}{LMA}
\UM{186A}{0}{MONGOLIAN LETTER SIBE JA}
	{textmongoliansibeja}{}
\UM{186B}{2}{MONGOLIAN LETTER SIBE FA}
	{textmongoliansibefa}{LMA}
\UM{186C}{2}{MONGOLIAN LETTER SIBE GAA}
	{textmongoliansibegaa}{LMA}
\UM{186D}{2}{MONGOLIAN LETTER SIBE HAA}
	{textmongoliansibehaa}{LMA}
\UM{186E}{2}{MONGOLIAN LETTER SIBE TSA}
	{textmongoliansibetsa}{LMA}
\UM{186F}{2}{MONGOLIAN LETTER SIBE ZA}
	{textmongoliansibeza}{LMA}
\UM{1870}{2}{MONGOLIAN LETTER SIBE RAA}
	{textmongoliansiberaa}{LMA}
\UM{1871}{2}{MONGOLIAN LETTER SIBE CHA}
	{textmongoliansibecha}{LMA}
\UM{1872}{0}{MONGOLIAN LETTER SIBE ZHA}
	{textmongoliansibezha}{}
\UM{1873}{0}{MONGOLIAN LETTER MANCHU I}
	{textmongolianmanchui}{}
\UM{1874}{2}{MONGOLIAN LETTER MANCHU KA}
	{textmongolianmanchuka}{LMA}
\UM{1875}{2}{MONGOLIAN LETTER MANCHU RA}
	{textmongolianmanchura}{LMA}
\UM{1876}{2}{MONGOLIAN LETTER MANCHU FA}
	{textmongolianmanchufa}{LMA}
\UM{1877}{2}{MONGOLIAN LETTER MANCHU ZHA}
	{textmongolianmanchuzha}{LMA}
%%%%%%%%%%%%%%%%%%%%%%%%%%%%%%%%%%%%%%%%%%%%%%%%%%%%%%%%%%%%%%%%
\UM{1880}{1}{MONGOLIAN LETTER ALI GALI ANUSVARA ONE}
	{textmongolianaligalianusvaraone}{LMO}
\UM{1881}{0}{MONGOLIAN LETTER ALI GALI VISARGA ONE}
	{textmongolianaligalivisargaone}{}
\UM{1882}{0}{MONGOLIAN LETTER ALI GALI DAMARU}
	{textmongolianaligalidamaru}{}
\UM{1883}{0}{MONGOLIAN LETTER ALI GALI UBADAMA}
	{textmongolianaligaliubadama}{}
\UM{1884}{0}{MONGOLIAN LETTER ALI GALI INVERTED UBADAMA}
	{textmongolianaligaliinvertedubadama}{}
\UM{1885}{0}{MONGOLIAN LETTER ALI GALI BALUDA}
	{textmongolianaligalibaluda}{}
\UM{1886}{0}{MONGOLIAN LETTER ALI GALI THREE BALUDA}
	{textmongolianaligalithreebaluda}{}
\UM{1887}{0}{MONGOLIAN LETTER ALI GALI A}
	{textmongolianaligalia}{}
\UM{1888}{0}{MONGOLIAN LETTER ALI GALI I}
	{textmongolianaligalii}{}
\UM{1889}{0}{MONGOLIAN LETTER ALI GALI KA}
	{textmongolianaligalika}{}
\UM{188A}{0}{MONGOLIAN LETTER ALI GALI NGA}
	{textmongolianaligalinga}{}
\UM{188B}{0}{MONGOLIAN LETTER ALI GALI CA}
	{textmongolianaligalica}{}
\UM{188C}{0}{MONGOLIAN LETTER ALI GALI TTA}
	{textmongolianaligalitta}{}
\UM{188D}{0}{MONGOLIAN LETTER ALI GALI TTHA}
	{textmongolianaligalittha}{}
\UM{188E}{0}{MONGOLIAN LETTER ALI GALI DDA}
	{textmongolianaligalidda}{}
\UM{188F}{0}{MONGOLIAN LETTER ALI GALI NNA}
	{textmongolianaligalinna}{}
\UM{1890}{0}{MONGOLIAN LETTER ALI GALI TA}
	{textmongolianaligalita}{}
\UM{1891}{0}{MONGOLIAN LETTER ALI GALI DA}
	{textmongolianaligalida}{}
\UM{1892}{0}{MONGOLIAN LETTER ALI GALI PA}
	{textmongolianaligalipa}{}
\UM{1893}{0}{MONGOLIAN LETTER ALI GALI PHA}
	{textmongolianaligalipha}{}
\UM{1894}{0}{MONGOLIAN LETTER ALI GALI SSA}
	{textmongolianaligalissa}{}
\UM{1895}{0}{MONGOLIAN LETTER ALI GALI ZHA}
	{textmongolianaligalizha}{}
\UM{1896}{0}{MONGOLIAN LETTER ALI GALI ZA}
	{textmongolianaligaliza}{}
\UM{1897}{0}{MONGOLIAN LETTER ALI GALI AH}
	{textmongolianaligaliah}{}
\UM{1898}{0}{MONGOLIAN LETTER TODO ALI GALI TA}
	{textmongoliantodoaligalita}{}
\UM{1899}{0}{MONGOLIAN LETTER TODO ALI GALI ZHA}
	{textmongoliantodoaligalizha}{}
\UM{189A}{0}{MONGOLIAN LETTER MANCHU ALI GALI GHA}
	{textmongolianmanchualigaligha}{}
\UM{189B}{2}{MONGOLIAN LETTER MANCHU ALI GALI NGA}
	{textmongolianmanchualigalinga}{LMA}
\UM{189C}{2}{MONGOLIAN LETTER MANCHU ALI GALI CA}
	{textmongolianmanchualigalica}{LMA}
\UM{189D}{0}{MONGOLIAN LETTER MANCHU ALI GALI JHA}
	{textmongolianmanchualigalijha}{}
\UM{189E}{0}{MONGOLIAN LETTER MANCHU ALI GALI TTA}
	{textmongolianmanchualigalitta}{}
\UM{189F}{0}{MONGOLIAN LETTER MANCHU ALI GALI DDHA}
	{textmongolianmanchualigaliddha}{}
\UM{18A0}{0}{MONGOLIAN LETTER MANCHU ALI GALI TA}
	{textmongolianmanchualigalita}{}
\UM{18A1}{0}{MONGOLIAN LETTER MANCHU ALI GALI DHA}
	{textmongolianmanchualigalidha}{}
\UM{18A2}{0}{MONGOLIAN LETTER MANCHU ALI GALI SSA}
	{textmongolianmanchualigalissa}{}
\UM{18A3}{0}{MONGOLIAN LETTER MANCHU ALI GALI CYA}
	{textmongolianmanchualigalicya}{}
\UM{18A4}{2}{MONGOLIAN LETTER MANCHU ALI GALI ZHA}
	{textmongolianmanchualigalizha}{LMA}
\UM{18A5}{2}{MONGOLIAN LETTER MANCHU ALI GALI ZA}
	{textmongolianmanchualigaliza}{LMA}
\UM{18A6}{0}{MONGOLIAN LETTER ALI GALI HALF U}
	{textmongolianaligalihalfu}{}
\UM{18A7}{0}{MONGOLIAN LETTER ALI GALI HALF YA}
	{textmongolianaligalihalfya}{}
\UM{18A8}{0}{MONGOLIAN LETTER MANCHU ALI GALI BHA}
	{textmongolianmanchualigalibha}{}
\UM{18A9}{0}{MONGOLIAN LETTER ALI GALI DAGALGA}
	{textmongolianaligalidagalga}{}
% % % % % % % % % % % % % % % % % % % % % % % % % % % % % % % % 
%\hline
\caption{Unicode Mongolian Code Positions and Associated Commands%
	}\label{table:UnicodeMongolian}\\
\end{longtable}
%%%%%%%%%%%%%%%%%%%%%%%%%%%%%%%%%%%%%%%%%%%%%%%%%%%%%%%%%%%%%%%%

\chapter{External Support Software}

\section{MLS Software}

With \MonTeX, it is still possible to process documents generated
with the MLS software package. The MLS converter produces Cyrillic
and Mongolian Script texts out of transliterations using 
the MLS codepage. Documents encoded in MLS can be directly
processed, no further conversion is necessary.

\section{Simplified Transliteration Converter}

The directory \texttt{../source/} contains a small MLS to Simplified
Transliteration converter written in Perl. This file can be used
directly if Perl exists on your system. Perl is available under a
Public Licence for a huge variety of platforms. Consult CPAN
(\texttt{www.cpan.org}) for information and downloads.

%%%%%%%%%%%%%%%%%%%%%%%%%%%%%%%%%%%%%%%%%%%%%%%%%%%%%%%%%%%%%%%%
%%%%%%%%%%%%%%%%%%%%%%%%%%%%%%%%%%%%%%%%%%%%%%%%%%%%%%%%%%%%%%%%


\chapter{Shortcomings, Bugs and Desiderata}

\section{Hyphenation Patterns}

The Mongolian hyphenation patterns delivered with \MonTeX\ are still
under development, so please expect occasional hyphenation errors.
It must be also noted that for proper hyphenation of critical words
\sh\ should be entered as \verb|\sh|, not as \verb|sh| since the
first is a character command processed by \LaTeXe\ while the latter
is a ligature processed by Metafont. If a wrong hyphenation is
spotted please check first whether the word in question contains
ligature statements (\verb|sh|, \verb|ya|, \verb|yu| etc.) which
should then replaced by the proper character commands. It is usually
sufficient to add a leading back slash and include the entity in
braces: \verb|{\ya}| is as good as \verb*|\ya |.

Hyphenation patterns for Russian exist but are still to be
re-encoded in \LMC\ encoding; Buryat hyphenation rules are still to be
defined.

\section{Retransliteration Problems}

Apart from being incomplete as far as some arcane writing variants
are concerned, the MLS (Broad Romanization) retransliteration engine
provided with \MonTeX\ has two serious shortcomings. Firstly, the
input can only consist of letters, punctuation marks and numbers.
Any \TeX\ or \LaTeX\ command (including \verb|\"a| for \emph{\"a}
and friends) makes the retransliteration engine fail. Secondly,
for large quantities of text, the retransliteration system is
agonizingly slow.

The Simplified Transliteration is incorporated into a fontencoding,
\LMO, which can be selected as default encoding. This allows for
complete freedom of all \LaTeX\ commands but requires an initial
amount of practise.

\section{Missing Caption Definitions}

The translated captions provided with \MonTeX\ are not completely
translated at the moment. Notably \tttt{ccname} and
\tttt{headtoname} are missing in Mongolian and Buryat,
mainly due to grammatical reasons. This will be fixed in later versions.


\section{Page Headers and Text Encodings}

In rare cases it is possible that a \verb|\section| text appearing in
a header or footer which is supposed to be typeset in Cyrillic letters
is output in Latin letters. This happens if the text on that very page
contains encoding selection commands which happen to fall near the
page boundary. The only remedy is to enclose the argument text in an
additional \verb|{\mnr ...}| command (or vice versa for Latin).


\section{The \texttt{kminch} Font}

The Cyrillic typefaces of \MonTeX\ are completed by inch-high sans
serif capital letters good for book titles etc. Unfortunately, they
cannot be used orthogonally with the other fonts in \Tone\ (Latin
characters) and \LMC\ (Cyrillic characters)
encoding since their definition is based on \TeX\ primitives rather
than \LaTeXe's NFSS font selection scheme.

%%  \section{The \emph{gamma} and its Shape}
%%  
%%  The original \TeX\ environment only provides an italic gamma
%%  which is intended for the math mode. This gamma is used for
%%  transliterating Mongolian; it does not perfectly blend into roman
%%  text. The \texttt{LGR} Greek font provides a gamma which blends better
%%  with roman text. If \texttt{LGR} Greek is installed on your system then
%%  \MonTeX\ will automatically pick the appropriate fonts.
%%  
%%  \begin{quote}
%%  \begin{verbatim}
%%  \documentclass[11pt,a4paper]{article}
%%  %\usepackage[T3,LMS,LMC,T1]{fontenc}	% <- Uncomment these lines
%%  \usepackage[latin1]{mls}		%    if you use the TIPA
%%  %\usepackage[noenc]{tipa}		% <- package. Check line 42
%%  [...]
%%  %
%%  % You obtain a nicer `gamma' by uncommenting the following line
%%  % but if you do so make sure the TIPA package gets included.
%%  % You should then uncomment line no. 18 of this file.
%%  %
%%  %\renewcommand{\g}{\textipa{\textgamma}}
%%  
%%  \end{verbatim}
%%  \end{quote}
%%  
%%  
%%  \section{Error Message \texttt{option clash}}
%%  
%%  The interaction of \MonTeX\ with other language support packages for
%%  \LaTeXe\ has not been thoroughly tested and problems may well arise
%%  from using conflicting commands, character presentations etc. One known
%%  constellation causing concern is the \texttt{tipa} package. This
%%  package calls \texttt{fontenc} with its own options with the
%%  potential of producing an option clash. In this case follow the
%%  instructions given in the \texttt{tipa} manual and start the
%%  \texttt{tipa} package with the
%%  \texttt{noenc} option. The following excerpt from a document
%%  preamble is known to work:
%%  \begin{quote}
%%  \begin{verbatim}
%%  \usepackage[T3,LMS,LMC,T1]{fontenc}
%%  \usepackage[latin1]{mls}
%%  \usepackage[noenc]{tipa}
%%  \end{verbatim}
%%  \end{quote}
%%  
%%  
\section{Problems with PostScript Fonts}

\begin{sloppypar}
Any attempt to compile this document with \texttt{pslatex} or
declare \tttt{usepackage\{pslatex\}} in the preamble works for the
bulk of the document but reduces the Cyrillic typefaces to Roman
only (see tables~\ref{table:SpecialCyrCharacters} and
~\ref{table:typeface}) and eliminates some of the
transliteration symbols (see table~\ref{table:shortcuts}). A solution
has not yet been defined.
\end{sloppypar}

\section{Error Message regarding \tttt{selectlanguage}}

There seem to be differences in the nature of installed \LaTeXe\
platforms; emtex shows a behaviour different from teTeX with regard
to pre-loaded language options. On teTeX systems, no error message
concerning the redefinition of the \texttt{selectlanguage} command
occurs, on emtex systems such a message may occur if no other
language support packages are loaded. This error message can be
safely ignored but the author hopes to find a solution later.


\section{Printer Memory Overflow}

Depending on the printing system it may happen that a Printer Memory
Overflow message is generated when attempting to print this text. So
far, this happened only on emtex systems running on plain
\textsf{DOS}. This is an exceptional situation caused by the very
high number of fonts used for this document. The error message never
occurred on systems with PostScript postprocessing of the DVI file.

It is very simple and straightforward to reduce the number of
typeface examples of this document. Near the beginning of the source
file of this very text, the lines

\begin{quote}
\begin{verbatim}
% If emtex goofs with (printer) memory overflow
% when attempting to print this document then
% set the following number to "1", recompile and
% increase the number step by step until all
% examples are printed. The maximum value is 6.

\newcounter{FontSamples}
\setcounter{FontSamples}{6} % <--- Modify this number!
\end{verbatim}
\end{quote}

can be found. It is now possible to increase the number of printed
typeface samples step by step until either memory saturation is
reached or the system manages to print all fonts. In addition, it
should be noted that printing this documentation for the first time
may take some time until all fonts are computed.

\section{Error Reports}

Time is a most precious resource and one of the main reasons why the
authors decided \emph{not} to support other environments than
\LaTeXe. If \MonTeX\ does not work for you because you use a \LaTeX%
2.09 installation, or expect to find a working system for plain \TeX\
support, then the author cannot assist you beyond the advice to
install the most recent version of \LaTeXe.

If you find a bug or think a feature is missing which you'd like to
see included then your comments are most welcome. One of the authors
can be reached by e-mail (\texttt{corff@zedat.fu-berlin.de}), and
available updates will appear in Infosystem Mongolei
(\texttt{http://userpage.fu-berlin.de/\~{}corff}). Please check the MLS
directory for available releases and patches.


\section{Outlook and Desiderata}

Unfortunately, some code positions in the Metafont sources of
\MonTeX\ haven't been frozen yet. In addition, the authors are
not happy yet with some of the interaction performed by certain
glyph combinations. This will have to be refined definitely!
Last but not least, some of the font metrics will undergo further
tuning which all implies that documents containing Mongolian or
Manju text should be recompiled once a new version of this software
is issued.


With $\Omega$mega lurking around, \MonTeX\ should actually be obsolete
work. A unified encoding comprising all Mongolian writings has been
integrated into Unicode 3.0 and ISO 10646. The author needed a quick
solution for ongoing lexicographical work (the Pentaglot database,
that is) and will merge Unicode support with the existing Mon\TeX\
system later. At a future point, there will also be full-featured
$\Omega$mega support.

Anyway, whatever the mistakes and the shortcomings are that have 
crept into this system, I can only kindly ask you to blame me.

\vspace{1cm}
\hfill\parbox{6cm}{\kit{Migj"ad Janra"isig Burxny m"almi"i
				n"a"as"an o"in "olzi"it"a"i "od"or
				biqiw.}}

\vspace{1cm}
\hfill\parbox{6cm}{\it Now go forth and create beautiful Manju text!\\
			Oliver Corff, Shenyang, April 1st, 2001}


%%%%%%%%%%%%%%%%%%%%%%%%%%%%%%%%%%%%%%%%%%%%%%%%%%%%%%%%%%%%%%%%
%%%%%%%%%%%%%%%%%%%%%%%%%%%%%%%%%%%%%%%%%%%%%%%%%%%%%%%%%%%%%%%%
\part{Commands in Alphabetical Order\label{CommandReference}}

\chapter{Alphabetical Command Reference}

All user level commands available in \MonTeX\ are given here in
alphabetical order. Every entry in the following list has up to
seven sections which are only present if necessary:
\begin{description}
	\item [Synopsis] shows the usage of the command;
	\item [Function] states its purpose and function;
	\item [Limitations] in functional range, allowed
			input etc. are stated here;
	\item [Comments] and additional information about
			purpose and nature of the command;
	\item [Related commands] in the command reference;
	\item [See page] of the main text;
	\item [Example] shows a typical application. If several
		related commands have the same usage and
		command syntax, then only one example is
		given which is typically found at the first
		place a command is mentioned. One example
		is the command for numbering by letters:
		The commands \verb|\Asbuk|, \verb|\Useg| and
		\verb|\Uzeg| are similar, and an example is only
		given under \verb|\Asbuk|.
\end{description}

%%%%%%%%%%%%%%%%%%%%%%%%%%%%%%%%%%%%%%%%%%%%%%%%%%%%%%%%%%%%%%%%%%%%%%%%
\MyCommand{Asbuk}		% Command Name
	{\{\textit{<number>}\}}	% Synopsis, number of arguments
	{Provides counting by upper case 
	 Cyrillic letters, Russian style.}		% Function
	{+}			% Example
	{\emph{<number>} must be between 1 and 28.}	% Limitations
	{-}			% Comments
	{\char92asbuk \char92Useg \char92useg
	 \char92Uzeg \char92uzeg}	% See also
	{cmd:Asbuk}			% See page

\exa
	Position 25 is
	\Uzeg{25}) in Buryat,
	\Useg{25}) in Xalx Mongolian and
	\Asbuk{25}) in Russian.
\exb
	\begin{verbatim}
	Position 25 is
	\Uzeg{25}) in Buryat,
	\Useg{25}) in Xalx Mongolian and
	\Asbuk{25}) in Russian.
	\end{verbatim}
\exc
%%%%%%%%%%%%%%%%%%%%%%%%%%%%%%%%%%%%%%%%%%%%%%%%%%%%%%%%%%%%%%%%%%%%%%%%
\MyCommand{asbuk}		% Command Name
	{\{\textit{<number>}\}}	% Synopsis, number of arguments
	{Provides counting by lower case 
	 Cyrillic letters, Russian style.}		% Function
	{-}			% Example
	{\emph{<number>} must be between 1 and 28.}	% Limitations
	{-}			% Comments
	{\char92Asbuk \char92Useg \char92useg %
	 \char92Uzeg \char92uzeg}	% See also
	{cmd:asbuk}			% See page

%%%%%%%%%%%%%%%%%%%%%%%%%%%%%%%%%%%%%%%%%%%%%%%%%%%%%%%%%%%%%%%%%%%%%%%%
\MyCommand{bcg}			% Command Name
	{\{\emph{<text>}\}}	% Synopsis, number of arguments
	{Generates Classical Mongolian out of
	 \MonTeX-ified MLS transliteration.}	% Function
	{+}			% Example
	{\emph{<text>} can only consist of unexpandable
	 characters; any \TeX\ or \LaTeXe\ command sequence
	 (even those for dotted vowels like \tttt{{}"{}a})
	 make the system derail.}		% Limitations
	{-}			% Comments
	{\char92glyphbcg \char92PrettyMLS}	% See also
	{cmd:bcg}		% See page

\exa
	\emph{mong\g ol bicig}
	is \bcg{mongGol bicig}.
\exb
	\begin{verbatim}
	\emph{mong\g ol bicig}
	is \bcg{mongGol bicig}.
	\end{verbatim}
\exc


%%  \MyCommand{blr}			% Command Name
%%  	{-}			% Synopsis, number of arguments
%%  	{-}			% Function
%%  	{-}			% Example
%%  	{-}			% Limitations
%%  	{Implemented, but not yet functional}	% Comments
%%  	{-}			% See also
%%  	{-}			% See page
%%  

%%%%%%%%%%%%%%%%%%%%%%%%%%%%%%%%%%%%%%%%%%%%%%%%%%%%%%%%%%%%%%%%%%%%%%%%
\MyCommand{bicig}			% Command Name
	{\{\emph{<text>}\}}	% Synopsis, number of arguments
	{Generates Classical Mongolian out of
	 Simplified Transliteration.}	% Function
	{+}			% Example
	{-}			% Limitations
	{-}			% Comments
	{\char92bcg \char92bithe}% See also
	{cmd:bicig}		% See page

\exa
	\emph{munggul bicik}
	is \bicig{munggul bicik}.
\exb
	\begin{verbatim}
	\emph{munggul bicik}
	is \bicig{munggul bicik}.
	\end{verbatim}
\exc

%%%%%%%%%%%%%%%%%%%%%%%%%%%%%%%%%%%%%%%%%%%%%%%%%%%%%%%%%%%%%%%%%%%%%%%%
\MyEnvironment{bicig}		% Command Name
	{-}			% Synopsis, number of arguments
	{Sets document language to Uighur, or Bicig Mongolian.}	% Function
	{-}			% Example
	{Cooperates well only with Simplified Transliteration as
	 its underlying encoding is \LMO.}			% Limitations
	{-}			% Comments
	{bithe buryat english russian xalx}		% See also
	{a:bicig}		% See page
%%%%%%%%%%%%%%%%%%%%%%%%%%%%%%%%%%%%%%%%%%%%%%%%%%%%%%%%%%%%%%%%%%%%%%%%
\MyEnvironment{bicigpage}	% Command Name
	{-}			% Synopsis, number of arguments
	{Similar to \texttt{bithepage}, it
	 provides full pages of vertical
	 Mongolian text.}	% Function
	{-}			% Example
	{Like all commands of the vertical output family, this
	 command requires PostScript support for proper vertical
	 display. In addition, a functional e-\LaTeX\ environment
	 is mandatory.
	 
	 Mongolian must be entered in Simplified Transliteration.%
	 	}		% Limitations
	{-}			% Comments
	{bithepage bicigtext bithetext}	% See also
	{a:bicigpage}		% See page
%%%%%%%%%%%%%%%%%%%%%%%%%%%%%%%%%%%%%%%%%%%%%%%%%%%%%%%%%%%%%%%%%%%%%%%%
\MyEnvironment{bicigtext}	% Command Name
	{-}			% Synopsis, number of arguments
	{Similar to \texttt{bicigpage}, it provides full paragraphs
	 of Uighur Mongolian text, but in horizontal line orientation.%
	 	}		% Function
	{-}			% Example
	{Mongolian must be entered in Simplified Transliteration,
	and a functional e-\LaTeX\ environment is mandatory.%
	 	}		% Limitations
	{-}			% Comments
	{bicigpage bithepage bithetext}	% See also
	{a:bicigtext}		% See page
%%%%%%%%%%%%%%%%%%%%%%%%%%%%%%%%%%%%%%%%%%%%%%%%%%%%%%%%%%%%%%%%%%%%%%%%
\MyCommand{BicigToday}		% Command Name
	{-}			% Synopsis, number of arguments
	{Provides the date in Uighur Mongolian.} % Function
	{-}			% Example
	{-}			% Limitations
	{Internal command. Authors should
	use \tttt{today}
	 which is redefined automatically
	 by the \texttt{bicig} option
	 when calling the \texttt{mls} package.} % Comments
	{\char92BitheToday \char92BuryatToday
	 \char92RussianToday \char92XalxToday}% See also
	{cmd:today}		% See page

%%%%%%%%%%%%%%%%%%%%%%%%%%%%%%%%%%%%%%%%%%%%%%%%%%%%%%%%%%%%%%%%%%%%%%%%
\MyCommand{bithe}			% Command Name
	{\{\emph{<text>}\}}	% Synopsis, number of arguments
	{Generates Manju out of
	 transliterated material.}	% Function
	{+}			% Example
	{-}			% Limitations
	{-}			% Comments
	{\char92bicig}% See also
	{cmd:bithe}		% See page

\exa
	\emph{manju bithe}
	is \bithe{manju bithe}.
\exb
	\begin{verbatim}
	\emph{manju bithe}
	is \bithe{manju bithe}.
	\end{verbatim}
\exc

%%%%%%%%%%%%%%%%%%%%%%%%%%%%%%%%%%%%%%%%%%%%%%%%%%%%%%%%%%%%%%%%%%%%%%%%
\MyEnvironment{bithe}		% Command Name
	{-}			% Synopsis, number of arguments
	{Sets document language to Manju.}	% Function
	{-}			% Example
	{-}			% Limitations
	{-}			% Comments
	{bicig buryat english russian xalx}	% See also
	{a:bithe}		% See page
%%%%%%%%%%%%%%%%%%%%%%%%%%%%%%%%%%%%%%%%%%%%%%%%%%%%%%%%%%%%%%%%%%%%%%%%
\MyEnvironment{bithepage}		% Command Name
	{-}			% Synopsis, number of arguments
	{Similar to \texttt{bicigpage}, it
	 provides full pages of vertical
	 Manju text.}		% Function
	{-}			% Example
	{Like all commands of the vertical output family, this
	 command requires PostScript support for proper vertical
	 display. In addition, a functional e-\LaTeX\ environment
	 is mandatory.}		% Limitations
	{-}			% Comments
	{bicigpage bicigtext bithetext}	% See also
	{a:bithepage}		% See page
%%%%%%%%%%%%%%%%%%%%%%%%%%%%%%%%%%%%%%%%%%%%%%%%%%%%%%%%%%%%%%%%%%%%%%%%
\MyEnvironment{bithetext}		% Command Name
	{-}			% Synopsis, number of arguments
	{Similar to \texttt{bithepage}, it
	 provides full pages of Manju text, but in horizontal
	 line orientation.}	% Function
	{-}			% Example
	{A functional e-\LaTeX\ environment is mandatory.%
	 	}		% Limitations
	{-}			% Comments
	{bicigpage bithepage bicigtext}	% See also
	{a:bithetext}		% See page

%%%%%%%%%%%%%%%%%%%%%%%%%%%%%%%%%%%%%%%%%%%%%%%%%%%%%%%%%%%%%%%%%%%%%%%%
\MyCommand{BitheToday}		% Command Name
	{-}			% Synopsis, number of arguments
	{Provides the date in Manju.}	% Function
	{+}			% Example
	{-}			% Limitations
	{Internal command. Authors should
	use \tttt{today}
	 which is redefined automatically
	 by the \texttt{bithe} option
	 when calling the \texttt{mls} package.} % Comments
	{\char92BicigToday \char92RussianToday \char92XalxToday}% See also
	{cmd:today}		% See page

	\marginpar{\mabosoo{\BitheToday}}
\exa
\exb
	\begin{verbatim}
	\marginpar{\mabosoo{\BitheToday}}
	\end{verbatim}
\exc
%%%%%%%%%%%%%%%%%%%%%%%%%%%%%%%%%%%%%%%%%%%%%%%%%%%%%%%%%%%%%%%%%%%%%%%%
\MyCommand{bosoo}		% Command Name
	{\{\emph{<text>}\}}	% Synopsis, number of arguments
	{Prints text in vertical capsules.}			% Function
	{+}			% Example
	{PostScript support is required for presenting
	 the output. The \texttt{rotating} package must
	 be installed. If \MonTeX\ cannot find \texttt{rotating}
	 encapsulated material will be printed horizontally.}% Limitations
	{Line spacing etc. adjust automatically.
	 Useful for dictionaries etc.}	% Comments
	{\char92mabosoo \char92mbosoo \char92mobosoo}		% See also
	{cmd:bosoo}			% See page

\exa
	A \bosoo{vertical} word.
\exb
	\begin{verbatim}
	A \bosoo{vertical} word.
	\end{verbatim}
\exc
%%%%%%%%%%%%%%%%%%%%%%%%%%%%%%%%%%%%%%%%%%%%%%%%%%%%%%%%%%%%%%%%%%%%%%%%
\MyEnvironment{buryat}		% Command Name
	{}	% Synopsis, number of arguments
	{Sets document language to Buryat.}	% Function
	{-}			% Example
	{-}			% Limitations
	{-}			% Comments
	{bicig bithe english russian xalx}		% See also
	{a:buryat}		% See page

%%%%%%%%%%%%%%%%%%%%%%%%%%%%%%%%%%%%%%%%%%%%%%%%%%%%%%%%%%%%%%%%%%%%%%%%
\MyCommand{BuryatToday}		% Command Name
	{-}			% Synopsis, number of arguments
	{Provides the date in Buryat.}	% Function
	{+}			% Example
	{-}			% Limitations
	{Internal command. Authors should
	use \tttt{today}
	 which is redefined automatically
	 by the \texttt{buryat} option
	 when calling the \texttt{mls} package.} % Comments
	{\char92BicigToday \char92BitheToday
	 \char92RussianToday \char92XalxToday}% See also
	{cmd:today}		% See page

\exa
	\BuryatToday\par
	\RussianToday\par
	\XalxToday\par
\exb
	\begin{verbatim}
	\BuryatToday\par
	\RussianToday\par
	\XalxToday\par
	\end{verbatim}
\exc

%%%%%%%%%%%%%%%%%%%%%%%%%%%%%%%%%%%%%%%%%%%%%%%%%%%%%%%%%%%%%%%%%%%%%%%%
\MyCommand{ch}			% Command Name
	{-}			% Synopsis, number of arguments
	{Creates a \emph{ch} which is used
	 for Mongolian transliterations.}	% Function
	{+}			% Example
	{-}			% Limitations
	{-}			% Comments
	{\char92g \char92sh}	% See also
	{table:shortcuts}	% See page

\exa
	\emph{\Sh agdar} and \emph{\Ch adraa}
	are transliterations for
	{\mnr\Sh agdar} and {\mnr\Ch adraa}.
\exb
	\begin{verbatim}
	\emph{\Sh agdar} and \emph{\Ch adraa}
	are transliterations for
	{\mnr\Sh agdar} and {\mnr\Ch adraa}.
	\end{verbatim}
\exc
%%%%%%%%%%%%%%%%%%%%%%%%%%%%%%%%%%%%%%%%%%%%%%%%%%%%%%%%%%%%%%%%%%%%%%%%
\MyCommand{cminch}		% Command Name
	{-}			% Synopsis, number of arguments
	{Produces inch-high bold sans serif
	 latin letters for book titles etc.}	% Function
	{-}			% Example
	{Only capital letters and numbers available.}	% Limitations
	{This command bypasses the NFSS font setup,
	 hence deprecated since the font provided
	 by this command does not orthogonally follow
	 with the font changes of the main document.}	% Comments
	{\char92kminch}		% See also
	{cmd:cminch}		% See page

%%%%%%%%%%%%%%%%%%%%%%%%%%%%%%%%%%%%%%%%%%%%%%%%%%%%%%%%%%%%%%%%%%%%%%%%
\MyCommand{CYR}%		% Command Name
	{\{\emph{<letter name>}\}}	% Synopsis, number of arguments
	{Allows writing of Cyrillic letters in non-Cyrillic
	 environments without changing the document language.}% Function
	{-}			% Example
	{\emph{letter name} must be one of
		A, B, V, G, D, E,	% Limitations
		YO, ZH, Z, I, ISHRT, K, L,
		M, N, O, OTLD, P, R, S,
		T, U, Y, F, H, HSHA,
		C, CH, SH, SHCH, HRDSN,
		ERY, SFTSN, EREV, YU or YA.}
	{This set of letter names provides compatibility 
	 with the forthcoming T2 Cyrillic encoding designed
	 to be the future \LaTeXe{} standard encoding for
	 the extended Cyrillic alphabets.}	% Comments
	{\char92cyr}				% See also
	{cyralpha}		% See page

%%%%%%%%%%%%%%%%%%%%%%%%%%%%%%%%%%%%%%%%%%%%%%%%%%%%%%%%%%%%%%%%%%%%%%%%
\MyCommand{cyr}%		% Command Name
	{\{\emph{<letter name>}\}}	% Synopsis, number of arguments
	{Allows writing of Cyrillic letters in non-Cyrillic
	 environments without changing the document language.}% Function
	{+}			% Example
	{\emph{letter name} must be one of
		a, b, v, G, D, e,	% Limitations
		yo, zh, z, i, ishrt, k, l,
		m, n, o, otld, p, r, s,
		t, u, y, f, h, hsha,
		c, ch, sh, shch, hrdsn,
		erevy, hrdsn, erev, yu or ya.}
	{-}				% Comments
	{\char92CYR}			% See also
	{cyralpha}			% See page

\exa
	\CYRM\cyro\cyrn\cyrg\cyro\cyrl
\exb
	\begin{verbatim}
	\CYRM\cyro\cyrn\cyrg\cyro\cyrl
	\end{verbatim}
\exc


%%  \MyCommand{CyrLHStyle}%		% Command Name
%%  	{\texttt{(true|false)}}	% Synopsis, number of arguments
%%  	{Allows writing of Cyrillic letters in non-Cyrillic
%%  	 environments without changing the document language.}% Function
%%  	{-}			% Example
%%  	{Only the letter names
%%  	\texttt{w e j ij oe ue h x q qh xat y zln e}
%%  		are affected.}		% Limitations
%%  	{The default is \texttt{false}.}	% Comments
%%  	{\char92CYR \char92cyr \char92CyrLHStyle}	% See also
%%  	{cyrlhstyle}		% See page
%%  
%%%%%%%%%%%%%%%%%%%%%%%%%%%%%%%%%%%%%%%%%%%%%%%%%%%%%%%%%%%%%%%%%%%%%%%%
\MyCommand{g}			% Command Name
	{-}			% Synopsis, number of arguments
	{Creates a \emph{gamma} which is used for
	 Mongolian transliterations.}	% Function
	{+}			% Example
	{Only a limited number of typefaces
	 is available in standard \MonTeX.}	% Limitations
	{More \emph{gamma} shapes are provided by
	the Modern Greek package which is loaded
	automatically by \MonTeX\ if available.}% Comments
	{\char92ch \char92sh}	% See also
	{table:shortcuts}	% See page

\exa
	mong\g ol-un \g azar nutu\g
\exb
	\begin{verbatim}
	mong\g ol-un \g azar nutu\g
	\end{verbatim}
\exc

%%%%%%%%%%%%%%%%%%%%%%%%%%%%%%%%%%%%%%%%%%%%%%%%%%%%%%%%%%%%%%%%%%%%%%%%
\MyCommand{glyphbcg}		% Command Name
	{\{\emph{<text>}\}}	% Synopsis, number of arguments
	{Accepts MLS glyph transliteration as input
	 for Mongolian.}	% Function
	{+}			% Example
	{-}			% Comments
	{Inconvenient for anything longer than five glyphs.}% Limitations
	{\char92bcg}		% See also
	{cmd:glyphbcg}		% See page

\exa
	\glyphbcg{@moaNnnoL @aoloS}
\exb
	\begin{verbatim}
	\glyphbcg{@moaNnnoL @aoloS}
	\end{verbatim}
\exc

%%%%%%%%%%%%%%%%%%%%%%%%%%%%%%%%%%%%%%%%%%%%%%%%%%%%%%%%%%%%%%%%%%%%%%%%
\MyCommand{ImplementationLevel}	% Command Name
	{-}			% Synopsis, number of arguments
	{Shows the Implentation
		Level of \MonTeX.}	% Function
	{+}			% Example
	{-}			% Limitations
	{Only for administrative purposes.}	% Comments
	{\char92Version(Date|Kirill|Mongol|Release)}% See also
	{-}			% See page
\exa
	This is \MonTeX\
	\ImplementationLevel
\exb
	\begin{verbatim}
	This is \MonTeX\
	\ImplementationLevel
	\end{verbatim}
\exc

%%%%%%%%%%%%%%%%%%%%%%%%%%%%%%%%%%%%%%%%%%%%%%%%%%%%%%%%%%%%%%%%%%%%%%%%
\MyCommand{kbf}			% Command Name
	{\{\emph{<text>}\}}	% Synopsis, number of arguments
	{Cyrillic boldface capsule.}	% Function
	{+}			% Example
	{-}			% Limitations
	{-}			% Comments
	{\char92k(it|rm|sc|sf|sl|tt)
	 \char92l(bf|it|rm|sc|sf|sl|tt)}% See also
	{typefacecapsules}	% See page

\exa
	This is
	\kbf{kirill b\"ud\"u\"un}
	writing.
\exb
	\begin{verbatim}
	This is
	\kbf{kirill b\"ud\"u\"un}
	writing.
	\end{verbatim}
\exc

%%%%%%%%%%%%%%%%%%%%%%%%%%%%%%%%%%%%%%%%%%%%%%%%%%%%%%%%%%%%%%%%%%%%%%%%
\MyCommand{kit}			% Command Name
	{\{\emph{<text>}\}}	% Synopsis, number of arguments
	{Cyrillic italic capsule.}	% Function
	{+}			% Example
	{-}			% Limitations
	{-}			% Comments
	{\char92k(bf|rm|sc|sf|sl|tt)
	 \char92l(bf|it|rm|sc|sf|sl|tt)}% See also
	{typefacecapsules}	% See page

\exa
	This is
	\kit{kirill biqm\"al}
	writing.
\exb
	\begin{verbatim}
	This is
	\kit{kirill biqm\"al}
	writing.
	\end{verbatim}
\exc

%%%%%%%%%%%%%%%%%%%%%%%%%%%%%%%%%%%%%%%%%%%%%%%%%%%%%%%%%%%%%%%%%%%%%%%%
\MyCommand{kminch}		% Command Name
	{-}			% Synopsis, number of arguments
	{Produces inch-high bold sans serif
	 cyrillic letters for book titles etc.}	% Function
	{-}			% Example
	{Only capital letters and numbers available.}	% Limitations
	{This command bypasses the NFSS font setup,
	 hence deprecated since the font provided
	 by this command does not orthogonally follow
	 with the font changes of the main document.}	% Comments
	{\char92 cminch}		% See also
	{cmd:kminch}		% See page

%%%%%%%%%%%%%%%%%%%%%%%%%%%%%%%%%%%%%%%%%%%%%%%%%%%%%%%%%%%%%%%%%%%%%%%%
\MyCommand{krm}			% Command Name
	{\{\emph{<text>}\}}	% Synopsis, number of arguments
	{Cyrillic <<roman>> capsule.}	% Function
	{+}			% Example
	{-}			% Limitations
	{-}			% Comments
	{\char92k(bf|it|sc|sf|sl|tt)
	 \char92l(bf|it|rm|sc|sf|sl|tt)}% See also
	{typefacecapsules}	% See page

\exa
	This is
	\krm{kirill shuluun}
	writing.
\exb
	\begin{verbatim}
	This is
	\krm{kirill shuluun}
	writing.
	\end{verbatim}
\exc

%%%%%%%%%%%%%%%%%%%%%%%%%%%%%%%%%%%%%%%%%%%%%%%%%%%%%%%%%%%%%%%%%%%%%%%%
\MyCommand{ksc}			% Command Name
	{\{\emph{<text>}\}}	% Synopsis, number of arguments
	{Cyrillic small caps capsule.}	% Function
	{+}			% Example
	{-}			% Limitations
	{-}			% Comments
	{\char92k(bf|it|rm|sf|sl|tt)
	 \char92l(bf|it|rm|sc|sf|sl|tt)}% See also
	{typefacecapsules}	% See page

\exa
	This is
	\ksc{kirill jijig tom \"usgi\"in}
	writing.
\exb
	\begin{verbatim}
	This is
	\ksc{kirill jijig tom \"usgi\"in}
	writing.
	\end{verbatim}
\exc

%%%%%%%%%%%%%%%%%%%%%%%%%%%%%%%%%%%%%%%%%%%%%%%%%%%%%%%%%%%%%%%%%%%%%%%%
\MyCommand{ksf}			% Command Name
	{\{\emph{<text>}\}}	% Synopsis, number of arguments
	{Cyrillic sans serif capsule.}	% Function
	{+}			% Example
	{-}			% Limitations
	{-}			% Comments
	{\char92k(bf|it|rm|sc|sl|tt)
	 \char92l(bf|it|rm|sc|sf|sl|tt)}% See also
	{typefacecapsules}	% See page

\exa
	This is
	\ksf{kirill ogtolson}
	writing.
\exb
	\begin{verbatim}
	This is
	\ksf{kirill ogtolson}
	writing.
	\end{verbatim}
\exc

%%%%%%%%%%%%%%%%%%%%%%%%%%%%%%%%%%%%%%%%%%%%%%%%%%%%%%%%%%%%%%%%%%%%%%%%
\MyCommand{ksl}			% Command Name
	{\{\emph{<text>}\}}	% Synopsis, number of arguments
	{Cyrillic slanted capsule.}	% Function
	{+}			% Example
	{-}			% Limitations
	{-}			% Comments
	{\char92k(bf|it|rm|sc|sf|tt)
	 \char92l(bf|it|rm|sc|sf|sl|tt)}% See also
	{typefacecapsules}	% See page

\exa
	This is
	\ksl{kirill naluu}
	writing.
\exb
	\begin{verbatim}
	This is
	\ksl{kirill naluu}
	writing.
	\end{verbatim}
\exc

%%%%%%%%%%%%%%%%%%%%%%%%%%%%%%%%%%%%%%%%%%%%%%%%%%%%%%%%%%%%%%%%%%%%%%%%
\MyCommand{ktt}			% Command Name
	{\{\emph{<text>}\}}	% Synopsis, number of arguments
	{Cyrillic typewriter capsule.}	% Function
	{+}			% Example
	{-}			% Limitations
	{-}			% Comments
	{\char92k(bf|it|rm|sc|sf|sl)
	 \char92l(bf|it|rm|sc|sf|sl|tt)}% See also
	{typefacecapsules}	% See page


\exa
	This is
	\ktt{kirill biqgi\"in mashiny}
	writing.
\exb
	\begin{verbatim}
	This is
	\ktt{kirill biqgi\"in mashiny}
	writing.
	\end{verbatim}
\exc
%%%%%%%%%%%%%%%%%%%%%%%%%%%%%%%%%%%%%%%%%%%%%%%%%%%%%%%%%%%%%%%%%%%%%%%%
\MyCommand{lat}			% Command Name
	{\{\emph{<text>}\}}	% Synopsis, number of arguments
	{Latin capsule.}	% Function
	{-}			% Example
	{-}			% Limitations
	{-}			% Comments
	{\char92xalx}		% See also
	{capsules}		% See page

\exa
	{\mnr Mongol ba \lat{English}}
\exb
	\begin{verbatim}
	{\mnr Mongol ba \lat{English}}
	\end{verbatim}
\exc

%%%%%%%%%%%%%%%%%%%%%%%%%%%%%%%%%%%%%%%%%%%%%%%%%%%%%%%%%%%%%%%%%%%%%%%%
\MyCommand{lbf}			% Command Name
	{\{\emph{<text>}\}}	% Synopsis, number of arguments
	{Latin boldface capsule.} % Function
	{+}			% Example
	{-}			% Limitations
	{-}			% Comments
	{\char92k(bf|it|rm|sc|sf|sl|tt)
	 \char92l(it|rm|sc|sf|sl|tt)}% See also
	{typefacecapsules}		% See page

\exa
	{\mnr \"An\"a bol
	\lbf{latin boldface}
	shrift.}
\exb
	\begin{verbatim}
	{\mnr \"An\"a bol
	\lbf{latin boldface}
	shrift.}
	\end{verbatim}
\exc

%%%%%%%%%%%%%%%%%%%%%%%%%%%%%%%%%%%%%%%%%%%%%%%%%%%%%%%%%%%%%%%%%%%%%%%%
\MyCommand{lit}			% Command Name
	{\{\emph{<text>}\}}	% Synopsis, number of arguments
	{Latin italic capsule.} % Function
	{+}			% Example
	{-}			% Limitations
	{-}			% Comments
	{\char92k(bf|it|rm|sc|sf|sl|tt)
	 \char92l(bf|rm|sc|sf|sl|tt)}% See also
	{typefacecapsules}		% See page

\exa
	{\mnr \"An\"a bol
	\lit{latin italic}
	shrift.}
\exb
	\begin{verbatim}
	{\mnr \"An\"a bol
	\lit{latin italic}
	shrift.}
	\end{verbatim}
\exc

%%%%%%%%%%%%%%%%%%%%%%%%%%%%%%%%%%%%%%%%%%%%%%%%%%%%%%%%%%%%%%%%%%%%%%%%
\MyCommand{lrm}			% Command Name
	{\{\emph{<text>}\}}	% Synopsis, number of arguments
	{Latin roman capsule.}	% Function
	{+}			% Example
	{-}			% Limitations
	{-}			% Comments
	{\char92k(bf|it|rm|sc|sf|sl|tt)
	 \char92l(bf|it|sc|sf|sl|tt)}% See also
	{typefacecapsules}	% See page

\exa
	{\mnr \"An\"a bol
	\lrm{latin roman}
	shrift.}
\exb
	\begin{verbatim}
	{\mnr \"An\"a bol
	\lrm{latin roman}
	shrift.}
	\end{verbatim}
\exc

%%%%%%%%%%%%%%%%%%%%%%%%%%%%%%%%%%%%%%%%%%%%%%%%%%%%%%%%%%%%%%%%%%%%%%%%
\MyCommand{lsc}			% Command Name
	{\{\emph{<text>}\}}	% Synopsis, number of arguments
	{Latin small caps capsule.}	% Function
	{+}			% Example
	{-}			% Limitations
	{-}			% Comments
	{\char92k(bf|it|rm|sc|sf|sl|tt)
	 \char92l(bf|it|rm|sf|sl|tt)}% See also
	{typefacecapsules}	% See page

\exa
	{\mnr \"An\"a bol
	\lsc{latin small caps}
	shrift.}
\exb
	\begin{verbatim}
	{\mnr \"An\"a bol
	\lsc{latin small caps}
	shrift.}
	\end{verbatim}
\exc

%%%%%%%%%%%%%%%%%%%%%%%%%%%%%%%%%%%%%%%%%%%%%%%%%%%%%%%%%%%%%%%%%%%%%%%%
\MyCommand{lsf}			% Command Name
	{\{\emph{<text>}\}}	% Synopsis, number of arguments
	{Latin sans serif capsule.}	% Function
	{+}			% Example
	{-}			% Limitations
	{-}			% Comments
	{\char92k(bf|it|rm|sc|sf|sl|tt)
	 \char92l(bf|it|rm|sc|sl|tt)}% See also
	{typefacecapsules}	% See page

\exa
	{\mnr \"An\"a bol
	\lsf{latin sans serif}
	shrift.}
\exb
	\begin{verbatim}
	{\mnr \"An\"a bol
	\lsf{latin sans serif}
	shrift.}
	\end{verbatim}
\exc

%%%%%%%%%%%%%%%%%%%%%%%%%%%%%%%%%%%%%%%%%%%%%%%%%%%%%%%%%%%%%%%%%%%%%%%%
\MyCommand{lsl}			% Command Name
	{\{\emph{<text>}\}}	% Synopsis, number of arguments
	{Latin slanted capsule.}% Function
	{+}			% Example
	{-}			% Limitations
	{-}			% Comments
	{\char92k(bf|it|rm|sc|sf|sl|tt)
	 \char92l(bf|it|rm|sc|sf|tt)}% See also
	{typefacecapsules}	% See page

\exa
	{\mnr \"An\"a bol
	\lsl{latin slanted}
	shrift.}
\exb
	\begin{verbatim}
	{\mnr \"An\"a bol
	\lsl{latin slanted}
	shrift.}
	\end{verbatim}
\exc

%%%%%%%%%%%%%%%%%%%%%%%%%%%%%%%%%%%%%%%%%%%%%%%%%%%%%%%%%%%%%%%%%%%%%%%%
\MyCommand{ltt}			% Command Name
	{\{\emph{<text>}\}}	% Synopsis, number of arguments
	{Latin typewriter capsule.}	% Function
	{+}			% Example
	{-}			% Limitations
	{-}			% Comments
	{\char92k(bf|it|rm|sc|sf|sl|tt)
	 \char92l(bf|it|rm|sc|sf|sl)}% See also
	{typefacecapsules}	% See page

\exa
	{\mnr \"An\"a bol
	\ltt{latin typewriter}
	shrift.}
\exb
	\begin{verbatim}
	{\mnr \"An\"a bol
	\ltt{latin typewriter}
	shrift.}
	\end{verbatim}
\exc


%%%%    \MyCommand{mbc}			% Command Name
%%%%    	{-}			% Synopsis, number of arguments
%%%%    	{-}			% Function
%%%%    	{-}			% Example
%%%%    	{-}			% Limitations
%%%%    	{-}			% Comments
%%%%    	{bcg}			% See also
%%%%    	{-}			% See page
%%%%    
%%%%    \exa
%%%%    	\ImplementationLevel
%%%%    \exb
%%%%    	\begin{verbatim}
%%%%    	\ImplementationLevel
%%%%    	\end{verbatim}
%%%%    \exc
%%%%    
%%%%    
%%%%%%%%%%%%%%%%%%%%%%%%%%%%%%%%%%%%%%%%%%%%%%%%%%%%%%%%%%%%%%%%%%%%%%%%
\MyCommand{mabosoo}		% Command Name
	{\{\emph{<text>}\}}	% Synopsis, number of arguments
	{Similar to \char92mobosoo, it
	 provides vertical capsules of
	 text, but \emph{<text>} is 
	 treated as Manju.}	% Function
	{+}			% Example
	{Like all commands of the \tttt{bosoo} family, this
	 command requires PostScript support for proper vertical
	 display.}		% Limitations
	{-}			% Comments
	{\char92bosoo \char92mbosoo \char92mobosoo}	% See also
	{cmd:mabosoo}		% See page

\exa
	\emph{manju} \mabosoo{manju}
	writing \mabosoo{bithe}
	looks beautiful indeed.
\exb
	\begin{verbatim}
	\emph{manju} \mabosoo{manju}
	writing \mabosoo{bithe}
	looks beautiful indeed.
	\end{verbatim}
\exc

%%%%%%%%%%%%%%%%%%%%%%%%%%%%%%%%%%%%%%%%%%%%%%%%%%%%%%%%%%%%%%%%%%%%%%%%
\MyCommand{mabox}		% Command Name
	{\{\emph{<vertical length>}\}\{\emph{<text>}\}}	% Synopsis, number of arguments
	{Similar to \char92mobox, it
	 provides boxes of vertical
	 text, but \emph{<text>} is 
	 treated as Manju.}	% Function
	{+}			% Example
	{Like all commands of the \tttt{box} family, this
	 command requires PostScript support for proper vertical
	 display.}		% Limitations
	{-}			% Comments
	{\char92mobox}		% See also
	{cmd:mabox}		% See page

\exa
	\mabox{1.5cm}{%
	\noindent manju\\bithe.
	}
\exb
	\begin{verbatim}
	\mabox{1.5cm}{%
	\noindent manju\\bithe.
	}
	\end{verbatim}
\exc

%%%%%%%%%%%%%%%%%%%%%%%%%%%%%%%%%%%%%%%%%%%%%%%%%%%%%%%%%%%%%%%%%%%%%%%%
\MyCommand{mbosoo}		% Command Name
	{\{\emph{<text>}\}}	% Synopsis, number of arguments
	{Similar to \char92bosoo, it
	 provides vertical capsules of
	 text, but \emph{<text>} is
	 converted to Mongolian.}	% Function
	{+}			% Example
	{Like \tttt{bosoo}, this
	 command requires PostScript support.
	 Like \tttt{bcg}, the input
	 text may only contain letters, transliteration
	 symbols and numbers but no \TeX\ commands.
	 The command is internally defined as
	 \tttt{bosoo{\char92bcg\{...\}}}.}	% Limitations
	{-}			% Comments
	{\char92bosoo \char92mabosoo \char92mobosoo}		% See also
	{cmd:mbosoo}		% See page

\exa
	\emph{mong\g ol}
	\mbosoo{mongGol}
	writing \mbosoo{bicig}
	looks beautiful indeed.
\exb
	\begin{verbatim}
	\emph{mong\g ol}
	\mbosoo{mongGol}
	writing \mbosoo{bicig}
	looks beautiful indeed.
	\end{verbatim}
\exc
%%%%%%%%%%%%%%%%%%%%%%%%%%%%%%%%%%%%%%%%%%%%%%%%%%%%%%%%%%%%%%%%%%%%%%%%
\MyCommand{mobosoo}		% Command Name
	{\{\emph{<text>}\}}	% Synopsis, number of arguments
	{Similar to \char92mbosoo, it
	 provides vertical capsules of
	 text, but \emph{<text>} is
	 converted to Mongolian using
	 the Simplified Transliteration.}	% Function
	{+}			% Example
	{Like all commands of the \tttt{bosoo} family, this
	 command requires PostScript support for proper vertical
	 output.}	% Limitations
	{-}			% Comments
	{\char92bosoo\char92mabosoo  \char92mbosoo }		% See also
	{cmd:mobosoo}		% See page

\exa
	\emph{mong\g ol}
	\mobosoo{munggul}
	writing \mobosoo{bicik}
	looks beautiful indeed.
\exb
	\begin{verbatim}
	\emph{mong\g ol}
	\mobosoo{munggul}
	writing \mobosoo{bicik}
	looks beautiful indeed.
	\end{verbatim}
\exc
%%%%%%%%%%%%%%%%%%%%%%%%%%%%%%%%%%%%%%%%%%%%%%%%%%%%%%%%%%%%%%%%%%%%%%%%
\MyCommand{mobox}		% Command Name
	{\{\emph{<vertical length>}\}\{\emph{<text>}\}}	% Synopsis, number of arguments
	{Similar to \char92mabox, it
	 provides boxes of vertical
	 text, but \emph{<text>} is 
	 treated as Mongolian.}	% Function
	{+}			% Example
	{Mongolian must be entered in Simplified Transliteration.
	 Currently, LMS input is not accepted.
	
	 Like all commands of the \tttt{box} family, this
	 command requires PostScript support for proper vertical
	 display.}		% Limitations
	{-}			% Comments
	{\char92mobox}		% See also
	{cmd:mobox}		% See page

\exa
	\mobox{2cm}{munggul\\bicik}
\exb
	\begin{verbatim}
	\mabox{2cm}{munggul\\bicik}
	\end{verbatim}
\exc

%%%%%%%%%%%%%%%%%%%%%%%%%%%%%%%%%%%%%%%%%%%%%%%%%%%%%%%%%%%%%%%%%%%%%%%%
\MyCommand{mnr}			% Command Name
	{-}			% Synopsis, number of arguments
	{Switches the current stream to Xalx
	transliteration of Latin characters.}	% Function
	{+}			% Example
	{-}			% Limitations
	{\tttt{mnr} can be understood as
	 Mongolian New Romanization.}	% Comments
	{\char92rnm}		% See also
	{mnrnm}			% See page

\exa
	First \mnr kirill,
	\rnm then latin.
\exb
	\begin{verbatim}
	First \mnr kirill,
	\rnm then latin.
	\end{verbatim}
\exc
%%%%%%%%%%%%%%%%%%%%%%%%%%%%%%%%%%%%%%%%%%%%%%%%%%%%%%%%%%%%%%%%%%%%%%%%
\MyCommand{MonTeX}		% Command Name
	{-}			% Synopsis, number of arguments
	{Produces the \MonTeX{-} logo.}	% Function
	{+}			% Example
	{-}			% Limitations
	{-}			% Comments
	{-}			% See also
	{-}			% See page

\exa
	You are using \MonTeX,
	a \LaTeXe\ package
	providing Mongolian.
\exb
	\begin{verbatim}
	You are using \MonTeX,
	a \LaTeXe\ package
	providing Mongolian.
	\end{verbatim}
\exc

%%%%%%%%%%%%%%%%%%%%%%%%%%%%%%%%%%%%%%%%%%%%%%%%%%%%%%%%%%%%%%%%%%%%%%%%
\MyCommand{MyTogrog}		% Command Name
	{-}			% Synopsis, number of arguments
	{Provides the Mongolian currency denominator.}	% Function
	{+}			% Example
	{-}			% Limitations
	{Matches the typeface of the environment.}	% Comments
	{\char92mytogrog \char92Togrog \char92togrog}	% See also
	{cmd:MyTogrog}	% See page

\exa
	\kit{\"Un\"a 200 \MyTogrog}.
\exb
	\begin{verbatim}
	\kit{\"Un\"a 200 \MyTogrog}.
	\end{verbatim}
\exc

%%%%%%%%%%%%%%%%%%%%%%%%%%%%%%%%%%%%%%%%%%%%%%%%%%%%%%%%%%%%%%%%%%%%%%%%
\MyCommand{mytogrog}		% Command Name
	{-}			% Synopsis, number of arguments
	{Provides the Mongolian currency denominator,
	 lower case variant (not considered standard).}	% Function
	{-}			% Example
	{-}			% Limitations
	{Matches the typeface of the environment.}	% Comments
	{\char92MyTogrog \char92Togrog \char92togrog}	% See also
	{cmd:mytogrog}	% See page

%%%%%%%%%%%%%%%%%%%%%%%%%%%%%%%%%%%%%%%%%%%%%%%%%%%%%%%%%%%%%%%%%%%%%%%%
\MyCommand{PrettyMLS}		% Command Name
	{\{\emph{<text>}\}}	% Synopsis, number of arguments
	{Replaces some of the Mongolian
	 transliteration shorthands with
	 nicer output.}		% Function
	{+}			% Example
	{-}			% Limitations
	{-}			% Comments
	{\char92glyphbcg \char92ShowSpecialMLS}	% See also
	{table:PrettyMLS}		% See page

\exa
	\emph{SaGdur} \bcg{SaGdur} is
	\emph{\PrettyMLS{SaGdur}}.
\exb
	\begin{verbatim}
	\emph{SaGdur} \cbg{SaGdur}
	\emph{\PrettyMLS{SaGdur}}.
	\end{verbatim}
\exc

%%%%%%%%%%%%%%%%%%%%%%%%%%%%%%%%%%%%%%%%%%%%%%%%%%%%%%%%%%%%%%%%%%%%%%%%
\MyCommand{om}			% Command Name
	{-}			% Synopsis, number of arguments
	{Used for Tibetan transliterations.}% Function
	{+}			% Example
	{-}			% Limitations
	{-}			% Comments
	{-}			% See also
	{mnrnm}			% See page


\exa
	\mobox{3cm}{\noindent\sffamily
	\om	uva\\
	\  	ma'=a\\
	\  	n'i\\
	\  	badmi'\\
	\om	huu}
\exb
	\begin{verbatim}
	\mobox{3cm}{\noindent\sffamily
	\om	uva\\
	\ 	ma'=a\\
	\ 	n'i\\
	\ 	badmi'\\
	\om	huu}
	\end{verbatim}
\exc
	
%%%%%%%%%%%%%%%%%%%%%%%%%%%%%%%%%%%%%%%%%%%%%%%%%%%%%%%%%%%%%%%%%%%%%%%%
\MyCommand{rmfamily}		% Command Name
	{-}			% Synopsis, number of arguments
	{Sets normal Mongolian or Manju font family.}% Function
	{+}			% Example
	{Works only for \LMA\ and \LMO\ encodings.}		% Limitations
	{There is no good equivalent between \emph{Roman} and
	 Mongolian typographical styles.}			% Comments
	{-}			% See also
	{cmd:rmfamily}			% See page


\exa
	\mobox{2cm}{\noindent
	munggul\\
	\sffamily munggul\\
	\rmfamily munggul
	}
\exb
	\begin{verbatim}
	\mobox{2cm}{\noindent
	munggul\\
	\sffamily munggul\\
	\rmfamily munggul}
	\end{verbatim}
\exc

	
%%%%%%%%%%%%%%%%%%%%%%%%%%%%%%%%%%%%%%%%%%%%%%%%%%%%%%%%%%%%%%%%%%%%%%%%
\MyCommand{rnm}			% Command Name
	{-}			% Synopsis, number of arguments
	{Disables transliteration of Latin
	 characters to Xalx in the current stream.}% Function
	{+}			% Example
	{-}			% Limitations
	{\tttt{rnm} can be understood as
	 Return to NorMal.}	% Comments
	{\char92mnr}		% See also
	{mnrnm}			% See page

\exa
	\mnr Odoo kirill
	daraa \rnm latin
\exb
	\begin{verbatim}
	\mnr Odoo kirill
	daraa \rnm latin
	\end{verbatim}
\exc
%%%%%%%%%%%%%%%%%%%%%%%%%%%%%%%%%%%%%%%%%%%%%%%%%%%%%%%%%%%%%%%%%%%%%%%%
\MyEnvironment{russian}		% Command Name
	{}	% Synopsis, number of arguments
	{Sets document language to Russian.}	% Function
	{-}			% Example
	{-}			% Limitations
	{-}			% Comments
	{bicig bithe buryat english xalx}		% See also
	{a:russian}		% See page

%%%%%%%%%%%%%%%%%%%%%%%%%%%%%%%%%%%%%%%%%%%%%%%%%%%%%%%%%%%%%%%%%%%%%%%%
\MyCommand{RussianToday}	% Command Name
	{-}			% Synopsis, number of arguments
	{Provides the date in Russian.}			% Function
	{-}			% Example
	{-}			% Limitations
	{Internal command. Authors should
	use \tttt{today}
	 which is redefined automatically
	 by the \texttt{russian} option
	 when calling the \texttt{mls} package.} % Comments
	{\char92BuryatToday \char92XalxToday}% See also
	{cmd:today}			% See page

%%%%%%%%%%%%%%%%%%%%%%%%%%%%%%%%%%%%%%%%%%%%%%%%%%%%%%%%%%%%%%%%%%%%%%%%
\MyCommand{SetDocumentEncodingBicig}	% Command Name
	{-}			% Synopsis, number of arguments
	{Sets the document encoding to Classical
	 Mongolian, also known as Uighur.}		% Function
	{-}			% Example
	{-}			% Limitations
	{The romanization used for this encoding is a simplified
	 system with an emphasis on graphical, not phonetical
	 properties of the Uighur writing system.}	% Comments
	{\char92 SetDocumentEncodingBithe}	% See also
	{cmd:SetDocumentEncodingBicig}	% See page

%%%%%%%%%%%%%%%%%%%%%%%%%%%%%%%%%%%%%%%%%%%%%%%%%%%%%%%%%%%%%%%%%%%%%%%%
\MyCommand{SetDocumentEncodingBithe}	% Command Name
	{-}			% Synopsis, number of arguments
	{Sets the document encoding
	 to Classical Manju.}		% Function
	{-}			% Example
	{-}			% Limitations
	{The romanization used for this encoding is, with a
	 few simple exceptions, a close match of Hauer's
	 system which is the \emph{de facto} standard.}	% Comments
	{\char92SetDocumentEncodingBicig}	% See also
	{cmd:SetDocumentEncodingBithe}	% See page

%%%%%%%%%%%%%%%%%%%%%%%%%%%%%%%%%%%%%%%%%%%%%%%%%%%%%%%%%%%%%%%%%%%%%%%%
\MyCommand{SetDocumentEncodingLMC}	% Command Name
	{-}			% Synopsis, number of arguments
	{Sets the document encoding
	 to Modern Mongolian (Xalx in Cyrillic writing).}% Function
	{+}			% Example
	{-}			% Limitations
	{Used for writing Mongolian texts on
	 Latin-only platforms.}			% Comments
	{\char92SetDocumentEncodingNeutral}	% See also
	{cmd:SetDocumentEncodingLMC}		% See page

\exa
	\SetDocumentEncodingLMC
	Kirill \"us\"ag, mongol x\"al\\
	\SetDocumentEncodingNeutral
	Latin \"us\"ag, mongol x\"al
\exb
	\begin{verbatim}
	\SetDocumentEncodingLMC
	Kirill \"us\"ag, mongol x\"al\\
	\SetDocumentEncodingNeutral
	Latin \"us\"ag, mongol x\"al
	\end{verbatim}
\exc

%%%%%%%%%%%%%%%%%%%%%%%%%%%%%%%%%%%%%%%%%%%%%%%%%%%%%%%%%%%%%%%%%%%%%%%%
\MyCommand{SetDocumentEncodingNeutral}	% Command Name
	{-}			% Synopsis, number of arguments
	{Resets the document encoding so
	 that Latin appears as Latin again
	 and is not anymore converted to
	 Cyrillic automatically.}	% Function
	{-}			% Example
	{-}			% Limitations
	{-}			% Comments
	{\char92SetDocumentEncodingLMC}% See also
	{cmd:SetDocumentEncodingNeutral}% See page

%%%%%%%%%%%%%%%%%%%%%%%%%%%%%%%%%%%%%%%%%%%%%%%%%%%%%%%%%%%%%%%%%%%%%%%%
\MyCommand{sffamily}		% Command Name
	{-}			% Synopsis, number of arguments
	{Sets Block Print Style Mongolian or Manju font family.}% Function
	{-}			% Example
	{Works only for \LMA\ and \LMO\ encodings.}		% Limitations
	{There is no good equivalent between \emph{Roman} and
	 Mongolian typographical styles.}			% Comments
	{-}			% See also
	{cmd:sffamily}			% See page


%%%%%%%%%%%%%%%%%%%%%%%%%%%%%%%%%%%%%%%%%%%%%%%%%%%%%%%%%%%%%%%%%%%%%%%%
\MyCommand{sh}			% Command Name
	{-}			% Synopsis, number of arguments
	{Creates a \emph{\sh} which is used
	 for Mongolian transliterations.}	% Function
	{-}			% Example
	{-}			% Limitations
	{-}			% Comments
	{\char92ch \char92g}	% See also
	{table:shortcuts}	% See page

%%%%%%%%%%%%%%%%%%%%%%%%%%%%%%%%%%%%%%%%%%%%%%%%%%%%%%%%%%%%%%%%%%%%%%%%
\MyCommand{ShowSpecialMLS}	% Command Name
	{\texttt{(true|false)}}	% Synopsis, number of arguments
	{Controls the behaviour of \tttt{PrettyMLS}
	 and either reveals or hides FVS and other
	 codes for input of \texttt{-'*} etc.}		% Function
	{-}			% Example
	{This function accepts only character tokens as input,
	 no \TeX\ commands.}	% Limitations
	{-}			% Comments
	{\char92bcg \char92glyphbcg \char92PrettyMLS}	% See also
	{cmd:PrettyMLS}		% See page

%%%%%%%%%%%%%%%%%%%%%%%%%%%%%%%%%%%%%%%%%%%%%%%%%%%%%%%%%%%%%%%%%%%%%%%%
\MyCommand{Togrog}		% Command Name
	{-}			% Synopsis, number of arguments
	{Provides the Mongolian currency denominator.}	% Function
	{+}			% Example
	{-}			% Limitations
	{Never changes the typeface. If you want to
	 match \tttt{Togrog} with the environment
	 use \tttt{MyTogrog} instead.}		% Comments
	{\char92togrog \char92MyTogrog \char92mytogrog}	% See also
	{cmd:Togrog}	% See page

\exa
	\xalx{\"Un\"a 200 \Togrog}.
\exb
	\begin{verbatim}
	\xalx{\"Un\"a 200 \Togrog}.
	\end{verbatim}
\exc

%%%%%%%%%%%%%%%%%%%%%%%%%%%%%%%%%%%%%%%%%%%%%%%%%%%%%%%%%%%%%%%%%%%%%%%%
\MyCommand{togrog}		% Command Name
	{-}			% Synopsis, number of arguments
	{Provides the Mongolian currency denominator,
	 lower case variant (not considered standard).}	% Function
	{-}			% Example
	{-}			% Limitations
	{Never changes the typeface. If you want to
	 match \char92Togrog with the environment
	 use \char92MyTogrog instead.}			% Comments
	{\char92Togrog \char92MyTogrog \char92mytogrog}	% See also
	{cmd:togrog}	% See page

%%%%%%%%%%%%%%%%%%%%%%%%%%%%%%%%%%%%%%%%%%%%%%%%%%%%%%%%%%%%%%%%%%%%%%%%
\MyCommand{Useg}		% Command Name
	{\{\textit{<number>}\}}	% Synopsis, number of arguments
	{Provides counting by upper case 
	 Cyrillic letters, Xalx Mongolian style.}	% Function
	{-}			% Example
	{\emph{<number>} must be between 1 and 31.}	% Limitations
	{-}			% Comments
	{\char92Asbuk \char92asbuk \char92useg
		\char92Uzeg \char92uzeg}% See also
	{cmd:Useg}			% See page
%%%%%%%%%%%%%%%%%%%%%%%%%%%%%%%%%%%%%%%%%%%%%%%%%%%%%%%%%%%%%%%%%%%%%%%%
\MyCommand{useg}		% Command Name
	{\{\textit{<number>}\}}	% Synopsis, number of arguments
	{Provides counting by lower case 
	 Cyrillic letters, Xalx Mongolian style.}	% Function
	{-}			% Example
	{\emph{<number>} must be between 1 and 31.}	% Limitations
	{-}			% Comments
	{\char92Asbuk \char92asbuk \char92Useg
		\char92useg \char92Uzeg}	% See also
	{cmd:useg}			% See page
%%%%%%%%%%%%%%%%%%%%%%%%%%%%%%%%%%%%%%%%%%%%%%%%%%%%%%%%%%%%%%%%%%%%%%%%
\MyCommand{Uzeg}		% Command Name
	{\{\textit{<number>}\}}	% Synopsis, number of arguments
	{Provides counting by upper case 
	 Cyrillic letters, Buryat style.}		% Function
	{-}			% Example
	{\emph{<number>} must be between 1 and 32.}	% Limitations
	{-}			% Comments
	{\char92Asbuk \char92asbuk \char92Useg
		\char92useg \char92Uzeg}	% See also
	{cmd:Uzeg}			% See page
%%%%%%%%%%%%%%%%%%%%%%%%%%%%%%%%%%%%%%%%%%%%%%%%%%%%%%%%%%%%%%%%%%%%%%%%
\MyCommand{uzeg}		% Command Name
	{\{\textit{<number>}\}}	% Synopsis, number of arguments
	{Provides counting by lower case 
	 Cyrillic letters, Buryat style.}		% Function
	{-}			% Example
	{\emph{<number>} must be between 1 and 32.}	% Limitations
	{-}			% Comments
	{\char92Asbuk \char92asbuk \char92Useg
		\char92useg \char92Uzeg}% See also
	{cmd:uzeg}			% See page
%%%%%%%%%%%%%%%%%%%%%%%%%%%%%%%%%%%%%%%%%%%%%%%%%%%%%%%%%%%%%%%%%%%%%%%%
\MyCommand{VersionDate}	% Command Name
	{-}			% Synopsis, number of arguments
	{Provides the release date of
	 of the current version}	% Function
	{+}			% Example
	{-}			% Limitations
	{Only for administrative purposes.}	% Comments
	{\char92Version(Kirill|Mongol|Release)
	 \char92ImplementationLevel}		% See also
	{-}			% See page

\exa
	This version was officially
	released \VersionDate.
\exb
	\begin{verbatim}
	This version was officially
	released \VersionDate.
	\end{verbatim}
\exc

%%%%%%%%%%%%%%%%%%%%%%%%%%%%%%%%%%%%%%%%%%%%%%%%%%%%%%%%%%%%%%%%%%%%%%%%
\MyCommand{VersionKirill}	% Command Name
	{-}			% Synopsis, number of arguments
	{Provides the version number of the
	 \MonTeX\ code related to Cyrillic.}	% Function
	{+}			% Example
	{-}			% Limitations
	{Only for administrative purposes.}	% Comments
	{\char92Version(Date|Mongol|Release)
	 \char92ImplementationLevel}		% See also
	{-}			% See page

\exa
	Cyrillic version: \VersionKirill
\exb
	\begin{verbatim}
	Cyrillic version: \VersionKirill
	\end{verbatim}
\exc

%%%%%%%%%%%%%%%%%%%%%%%%%%%%%%%%%%%%%%%%%%%%%%%%%%%%%%%%%%%%%%%%%%%%%%%%
\MyCommand{VersionMongol}	% Command Name
	{-}			% Synopsis, number of arguments
	{Provides the version number of the
	 \MonTeX\ code related to Mongolian.}	% Function
	{+}			% Example
	{-}			% Limitations
	{Only for administrative purposes.}	% Comments
	{\char92Version(Date|Kirill|Release)
	 \char92ImplementationLevel}		% See also
	{-}			% See page

\exa
	Mongolian version: \VersionMongol
\exb
	\begin{verbatim}
	Mongolian version: \VersionMongol
	\end{verbatim}
\exc

%%%%%%%%%%%%%%%%%%%%%%%%%%%%%%%%%%%%%%%%%%%%%%%%%%%%%%%%%%%%%%%%%%%%%%%%
\MyCommand{VersionRelease}	% Command Name
	{-}			% Synopsis, number of arguments
	{Comprehensive version information.}	% Function
	{+}			% Example
	{-}			% Limitations
	{Only for administrative purposes.}	% Comments
	{\char92Version(Date|Kirill|Mongol)
	 \char92ImplementationLevel}		% See also
	{-}			% See page

\exa
	This is \MonTeX\ \VersionRelease
\exb
	\begin{verbatim}
	This is \MonTeX\ \VersionRelease
	\end{verbatim}
\exc
%%%%%%%%%%%%%%%%%%%%%%%%%%%%%%%%%%%%%%%%%%%%%%%%%%%%%%%%%%%%%%%%%%%%%%%%
\MyEnvironment{xalx}		% Command Name
	{}	% Synopsis, number of arguments
	{Sets document language to Xalx, or Modern Mongolian.}	% Function
	{-}			% Example
	{-}			% Limitations
	{-}			% Comments
	{bicig bithe buryat english russian}		% See also
	{a:xalx}		% See page
%%%%%%%%%%%%%%%%%%%%%%%%%%%%%%%%%%%%%%%%%%%%%%%%%%%%%%%%%%%%%%%%%%%%%%%%
\MyCommand{xalx}		% Command Name
	{\{\emph{<text>}\}}	% Synopsis, number of arguments
	{Creates capsules with Modern Mongolian transliteration
	 for including Xalx words in other languages.}	% Function
	{+}			% Example
	{-}			% Limitations
	{-}			% Comments
	{\char92lat}		% See also
	{cmd:xalx}		% See page

\exa
	English and \xalx{mongol}
\exb
	\begin{verbatim}
	English and \xalx{mongol}
	\end{verbatim}
\exc
%%%%%%%%%%%%%%%%%%%%%%%%%%%%%%%%%%%%%%%%%%%%%%%%%%%%%%%%%%%%%%%%%%%%%%%%
\MyCommand{XalxToday}		% Command Name
	{-}			% Synopsis, number of arguments
	{Provides the date in Xalx Mongolian.}	% Function
	{-}			% Example
	{-}			% Limitations
	{Internal command. Authors should
	 use \tttt{today} which is redefined
	 automatically by the \texttt{xalx} option
	 when calling the \texttt{mls} package.} % Comments
	{\char92BuryatToday \char92RussianToday}% See also
	{cmd:today}		% See page

\end{document}
