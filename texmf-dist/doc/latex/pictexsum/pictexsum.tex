% Copyright (c) 1990 Angus Duggan
% All rights reserved.
% 
% Redistribution and use in source and binary forms, with or without
% modification, are permitted provided that the following conditions are met:
% 
%     Redistributions of source code must retain the above copyright
%     notice, this list of conditions and the following disclaimer.
% 
%     Redistributions in binary form must reproduce the above copyright
%     notice, this list of conditions and the following disclaimer in the
%     documentation and/or other materials provided with the distribution.
% 
% THIS SOFTWARE IS PROVIDED BY THE COPYRIGHT HOLDERS AND CONTRIBUTORS "AS
% IS" AND ANY EXPRESS OR IMPLIED WARRANTIES, INCLUDING, BUT NOT LIMITED
% TO, THE IMPLIED WARRANTIES OF MERCHANTABILITY AND FITNESS FOR A
% PARTICULAR PURPOSE ARE DISCLAIMED. IN NO EVENT SHALL THE COPYRIGHT
% HOLDER OR CONTRIBUTORS BE LIABLE FOR ANY DIRECT, INDIRECT, INCIDENTAL,
% SPECIAL, EXEMPLARY, OR CONSEQUENTIAL DAMAGES (INCLUDING, BUT NOT LIMITED
% TO, PROCUREMENT OF SUBSTITUTE GOODS OR SERVICES; LOSS OF USE, DATA, OR
% PROFITS; OR BUSINESS INTERRUPTION) HOWEVER CAUSED AND ON ANY THEORY OF
% LIABILITY, WHETHER IN CONTRACT, STRICT LIABILITY, OR TORT (INCLUDING
% NEGLIGENCE OR OTHERWISE) ARISING IN ANY WAY OUT OF THE USE OF THIS
% SOFTWARE, EVEN IF ADVISED OF THE POSSIBILITY OF SUCH DAMAGE.
 
\documentstyle[useful,a4mod]{article}

\author{Compiled by\\[6pt]
Angus Duggan
\\[6pt]
{\tt tex@ed.lfcs}}

\title{\PiCTeX{} command summary}

\date{22\textsuperscript{\rm nd} January 1990}

% reset page margins
\setleftmargin{1in}{0in}
\setrightmargin{1in}{0in}
%\setheadmargin{1in}{\mainsize}{2\mainsize}
%\setfootmargin{1in}{\mainsize}{2\mainsize}
\setmarginnote{0in}{0in}

% reset float fractions
\setcounter{topnumber}{3}
\renewcommand{\topfraction}{.5}
\setcounter{bottomnumber}{3}
\renewcommand{\bottomfraction}{.5}
\setcounter{totalnumber}{5}
\renewcommand{\textfraction}{.25}

% Don't display overfull boxes...
\overfullrule=0pt
% Don't complain about long lines unless they are over 3pt too big...
\hfuzz=3pt

% names of sysops
\newcommand{\syspers}{Carol Anstruther}
\newcommand{\texpers}{Angus Duggan}

% font for warnings, notes, etc.
\newfont{\uit}{cmu10}
% font for thin vertical bar
\newfont{\symfont}{cmsy10}

% abbreviations
\newcommand{\unix}{{\sc unix}}
\newcommand{\dvilaser}{{\sc DviLaser}}
\newcommand{\tex}{{tex}}
\newcommand{\Tex}{{Tex}}

\newcommand{\rvb}{{\symfont\char'152}}

\renewcommand{\emph}[1]{{\em#1\/}}
\newcommand{\slan}[1]{{\sl#1\/}}
\newcommand{\type}[1]{{\tt#1}}

% indented \tt paragraphs, for commands
\newenvironment{command}%
  {\begin{list}{}{\tt}\item[]}%
  {\end{list}}
% indented verbatim paragraphs, for commands
\newverbatim{verbcommand}%
  {\begin{list}{}{\parsep0pt\itemsep0pt}\item[]}%
  {\end{list}}
% small indented verbatim for examples
\newverbatim{example}%
  {\begin{list}{}{\small\parsep0pt\itemsep0pt}\item[]}%
  {\end{list}}
% list for syntax decls.
\newenvironment{syntax}%
  {\begin{labelled}}%
  {\end{labelled}}

\pagestyle{headings}
%\raggedbottom

%\makeindex

\begin{document}

\maketitle

This article contains a summary of the \PiCTeX{} commands. This is intended as
a reminder for users who have read the \PiCTeX{} manual (which is available
from the LFCS library). The following conventions should be observed when
using \PiCTeX{} commands:
\begin{itemize}
\item
At least one blank must be present for each blank in the command prototypes
below.
\item
Quantities in \type{<>}'s must be specified as explicit dimensions, or in
terms of \TeX's dimension registers.
\item
\emph{coord}, \emph{xcoord}, \emph{ycoord}, \emph{x}, and \emph{y}, with or
without subscripts or superscripts denote coordinates with respect to the
current coordinate system. In particular, they are dimensionless quantities.
Values must be expressed in fixed point notation, with at most 5 digits to the
right of the decimal point.
\item
Parts of a command enclosed in [\,]'s may be omitted.
\end{itemize}

\section{Commands}
The \PiCTeX{} drawing commands are:
\begin{syntax}
\item[{\tt\linestack[l]{\bsl accountingoff\\
\bsl accountingon}}]
These commands suspend and resume \PiCTeX's updating procedure for the minimum
size box enclosing the picture. They should only be used when \PiCTeX{} has
been notified of
the minimum size box already (e.g.~by executing a \verb+\setplotarea+).
\item[{\tt\bsl arrow <$\ell$> [$\beta$,$\gamma$]
{\rm[}<\emph{xshift},\emph{yshift}>{\rm]} from \emph{xcoord}$_s$ \emph{ycoord}$_s$ to \emph{xcoord}$_e$
\emph{ycoord}$_e$}]
This command draws an arrow, where (\emph{xcoord}$_s$,\emph{ycoord}$_s$) is the start of the
line on which the arrow lies, (\emph{xcoord}$_e$,\emph{ycoord}$_e$) is the end of the line
on which the arrow lies, $\ell$ is the length of the arrowhead, $\beta\ell$ is
the width of the arrowhead at half its length, and $\gamma\ell$ is the width
of the arrowhead at its full length. The arrowhead will be open, and will be
drawn with a smooth curve through the width points and the end of the line.
The arrowhead will curve in if $2\beta < \gamma$, and curve out if
$2\beta > \gamma$.
\type{<\emph{xshift},\emph{yshift}>} has the same effect as in the \verb+\put+ command.
\item[{\tt\linestack[l]{\bsl axis {\rm[}bottom{\rm]} {\rm[}top{\rm]} {\rm[}left{\rm]} {\rm[}right{\rm]}\\
\quad{\rm[}shiftedto y=\emph{ycoord}{\rm]} {\rm[}shiftedto x=\emph{xcoord}{\rm]}\\
\quad{\rm[}visible{\rm]} {\rm[}invisible{\rm]}\\
\quad{\rm[}label \{\emph{axis label}\}{\rm]}\\
\quad{\rm[}ticks{\rm]}\\
\quad\quad{\rm[}out{\rm]} {\rm[}in{\rm]}\\
\quad\quad{\rm[}long{\rm]} {\rm[}short{\rm]} {\rm[}length <\emph{length}>{\rm]}\\
\quad\quad{\rm[}width <\emph{width}>{\rm]}\\
\quad\quad{\rm[}butnotacross{\rm]} {\rm[}andacross{\rm]}\\
\quad\quad{\rm[}unlabeled{\rm]} {\rm[}numbered{\rm]} {\rm[}withvalues \emph{value}$_1$ \emph{value}$_2$
\ldots\ /{\rm]}\\
\quad\quad{\rm[}unlogged{\rm]} {\rm[}logged{\rm]}\\
\quad\quad{\rm[}quantity $q${\rm]} {\rm[}from \emph{coord}$_s$ to \emph{coord}$_e$ by
\emph{dcoord}{\rm]}\\
\quad\quad\quad{\rm[}at \emph{coord}$_1$ \emph{coord}$_2$ \ldots\ /{\rm]}\\
\quad/}}]
This command draws an axis along the \type{bottom}, \type{top}, \type{left},
or \type{right} edge of the current plot area (one of these keywords must be
specified).
The \type{shiftedto} option causes a bottom or top axis to be
drawn at the specified $y$-coordinate, and a left or right axis to be drawn at
the specified $x$-coordinate.
The keyword \type{invisible} suppresses the drawing of the axis, but not of
tick marks, labels, etc. \type{visible} is the default.
The text specified by \emph{axis label} is centred with respect to the
appropriate edge of the plot area.
\type{ticks} causes tick marks to be drawn on the axis:
\begin{itemize}
\item
Ticks normally point \type{out} from the plot area; \type{in} makes them point
into the plot area.
\item
Ticks are normally \type{long}, but can be made \type{short}, or given an
arbitrary \emph{length} with the \type{length} option.
\item
The \emph{width} of the ticks can be set with the \type{width} option.
\item
Ticks can be extended across the whole plot area with the \type{andacross}
option, making grid lines. The default is \type{butnotacross}, which stops the
ticks from extending across the plot area.
\item
Ticks are normally \type{unlabeled}. If the \type{numbered} option is used,
the \type{at} or \type{from} options below assign numeric values to them.
Arbitrary tick labels can be specified by the \type{withvalues} option; the
labels \emph{value}$_1$, \emph{value}$_2$, \ldots{} are assigned to subsequent ticks until
the list of values is exhausted or an \type{unlabeled} or \type{numbered}
keyword is encountered. Values must be separated by at least one blank, and at
least one blank must precede the `\type{/}' that terminates the list. If a
value contains a blank or `\type{/}', enclose the entire value in
\type{\{\}}'s.
\item
The option \type{quantity $q$} draws $q$ ticks equally spaced from left to
right, or from bottom to top. The first and last ticks are at the ends of the
axis.
\item
The \type{from} option draws ticks at the indicated coordinates. \emph{coord}$_s$,
\emph{coord}$_e$, and \emph{dcoord} must be fixed point numbers, with the same number of
digits to the right of the decimal point (if any), and \emph{dcoord} must be
positive. If the \type{numbered} option is in effect, the coordinate of the
tick is used as the tick label.
\item
The \type{at} option draws ticks at the specified coordinates. As with the
\type{from} option, the coordinates must be fixed point numbers, which are
used as tick labels if \type{numbered} is in effect. The list of coordinates
must be terminated by `\type{/}'.
\item
The \type{logged} option applies only to the positioning subsequent ticks
specified by the \type{at} or \type{from} options. Ticks are placed at the
$\log_{10}$'s of the specified locations; the original unlogged numbers are used
as labels if \type{numbered} is in effect. \type{unlogged} is the default.
\end{itemize}
\item[\tt\bsl beginpicture]
This command is used to start \PiCTeX{} pictures.
\item[{\tt\bsl betweenarrows \{\emph{text}\} {\rm[}[{\rm[}$o_x${\rm]}{\rm[}$o_y${\rm]}]{\rm]}
{\rm[}<\emph{xshift},\emph{yshift}>{\rm]} from \emph{xcoord}$_s$ \emph{ycoord}$_s$ to \emph{xcoord}$_e$
\emph{ycoord}$_e$}]
\begin{sloppypar}
This command centres \emph{text} between a pair of arrows pointing outwards.
\type{<\emph{xshift},\emph{yshift}>} and \type{[{\rm[}$o_x${\rm]}{\rm[}$o_y${\rm]}]}
have the same effect as in the \verb+\put+ command.
(\emph{xcoord}$_s$,\emph{ycoord}$_s$) and (\emph{xcoord}$_e$,\emph{ycoord}$_e$) are the start and end
coordinates of the arrow pair.
Either \emph{xcoord}$_s$ and \emph{xcoord}$_e$ should be the
same, or  \emph{ycoord}$_s$ and \emph{ycoord}$_e$ should be the same.
\end{sloppypar}
\item[\tt\bsl circulararc $\theta$ degrees from \emph{xcoord}$_s$ \emph{ycoord}$_s$ center
at \emph{xcoord}$_c$ \emph{ycoord}$_c$]
This command draws an arc of a circle with a centre at  (\emph{xcoord}$_c$,
\emph{ycoord}$_c$); the arc starts from (\emph{xcoord}$_s$,\emph{ycoord}$_s$) and extends
anticlockwise through $\theta$ degrees. $\theta$ can have any real value
between -360 and 360.
\item[\tt\bsl Divide <\emph{dividend}> by <\emph{divisor}> forming <\emph{quotient}>]
This is \PiCTeX's division command for dimensions. \emph{dividend} and \emph{divisor}
may be explicit dimensions or dimension registers; \emph{quotient} must be a
dimension register. 
\item[\tt\bsl dontsavelinesandcurves]
This command stops \PiCTeX{} from saving plot locations to a file (see
\verb+\savelinesandcurves+).
\item[\tt\bsl ellipticalarc axes ratio $\xi$:$\eta$ $\theta$ degrees from
\emph{xcoord}$_s$ \emph{ycoord}$_s$ center at \emph{xcoord}$_c$ \emph{ycoord}$_c$]
This command draws an arc of an ellipse with a centre at  (\emph{xcoord}$_c$,
\emph{ycoord}$_c$); the arc starts from (\emph{xcoord}$_s$,\emph{ycoord}$_s$) and extends
anticlockwise through $\theta$ degrees. $\xi$ and $\eta$ are numbers
proportional to the lengths of the horizontal and vertical axes of the ellipse.
\item[\tt\bsl endpicture]
This command terminates a \PiCTeX{} picture.
\item[\tt\bsl endpicturesave <\emph{xreg},\emph{yreg}>]
This command is used to terminate sub-pictures, saving the left
edge and baseline in \emph{xreg} and \emph{yreg}.
If the subpicture is then \verb+\put+ at (\emph{xcoord},\emph{ycoord}) with the options
`\type{[Bl] <\emph{xreg},\emph{yreg}>}', the reference point of the sub-picture will be
at (\emph{xcoord},\emph{ycoord}).
\item[\tt\bsl findlength \{\emph{curve commands}\}]
\PiCTeX{} executes the curve drawing commands specified and puts the length
into the dimension register \verb+\totalarclength+. This can be used as the
$\lambda$ argument to \verb+\setdotsnear+ and \verb+\setdashesnear+.
\item[{\tt\bsl frame {\rm[}<\emph{separation}>{\rm]} \{\emph{text}\}}]
This command frames \emph{text}, with an optional border of
\emph{separation}.
This command has its normal \LaTeX{} meaning outside of \PiCTeX{} pictures,
but \verb+\pictexframe+ can be used outside of \PiCTeX{}
pictures to get the same effect as \PiCTeX's \verb+\frame+.
\item[\tt\bsl grid \{$c$\} \{$r$\}]
This command partitions the the plot area in to $c$ columns and $r$ rows.
\item[\tt\bsl gridlines]
This command sets the default for the \type{andacross}/\type{butnotacross}
option of the \type{axis} command to be \type{andacross}.
\item[{\tt\linestack[l]{\bsl hshade $y_0$ $x_0^{(l)}$ $x_0^{(r)}$ \ldots\
{\rm[}<$\epsilon_{l;i}$,$\epsilon_{r;i}$,$\epsilon_{d;i}$,$\epsilon_{u;i}$>{\rm]}
$y_i$ $x_i^{(l)}$ $x_i^{(r)}$ \ldots\ /\\
\bsl hshade $y_0$ $x_0^{(l)}$ $x_0^{(r)}$ \ldots\
{\rm[}<$\epsilon_{l;i}$,$\epsilon_{r;i}$,$\epsilon_{d;i}$,$\epsilon_{u;i}$>{\rm]}
$y_{2i-1}$ $x_{2i-1}^{(l)}$ $x_{2i-1}^{(r)}$ $y_{2i}$ $x_{2i}^{(l)}$
$x_{2i}^{(r)}$ \ldots\ /}}]
This command shades a region with piecewise linear/quadratic left and right
boundaries. Sub-regions are defined by the coordinates ($x_n^{(l)}$,$y_n$),
($x_n^{(r)}$,$y_n$), ($x_{n+1}^{(r)}$,$y_{n+1}$) and
($x_{n+1}^{(l)}$,$y_{n+1}$).
The relations $y_n<y_{n+1}$ and $x_n^{(l)}\leq x_n^{(r)}$
should hold. For the duration of the shading the optional edge effect field
`\type{<$\epsilon_{l;i}$,$\epsilon_{r;i}$,$\epsilon_{d;i}$,$\epsilon_{u;i}$>}'
overrides the specifications made by \verb+\setshadesymbol+. The second form
should be used when \verb+\setquadratic+ is in effect.
\item[\tt\bsl inboundscheckoff]
This command disables checking whether plot symbols are outside the current
plot area.
\item[\tt\bsl inboundscheckon]
This command enables checking whether plot symbols are outside the current
plot area.
\item[\tt\bsl invisibleaxes]
This command sets the default for the \type{visible}/\type{invisible}
option of the \type{axis} command to be \type{invisible}.
\item[{\tt\linestack[l]{\bsl lines {\rm[}[$o$]{\rm]} \{\emph{line}$_1$\bsl cr
\emph{line}$_2$\bsl cr \ldots\ \}\\
\bsl Lines {\rm[}[$o$]{\rm]} \{\emph{line}$_1$\bsl cr \emph{line}$_2$\bsl cr \ldots\ \}}}]
These commands produce stacks of lines, spaced normally. The lines will be
left justified if $o$ is `\type{l}', right justified if $o$ is `\type{r}', and
centred otherwise. \verb+\Lines+ is similar to \verb+\lines+, except the
baseline of the stack is the baseline of the top line instead of the baseline
of the bottom line.
\item[\tt\bsl loggedticks]
This command sets the default for the \type{logged}/\type{unlogged}
option of the \type{axis} command to be \type{logged}.
\item[{\tt\linestack[l]{\bsl multiput\{\emph{text}\}
{\rm[}[{\rm[}$o_x${\rm]}{\rm[}$o_y${\rm]}]{\rm]}
{\rm[}<\emph{xshift},\emph{yshift}>{\rm]} at "\emph{file name}"\\
\bsl multiput \{\emph{text}\}
{\rm[}[{\rm[}$o_x${\rm]}{\rm[}$o_y${\rm]}]{\rm]}
{\rm[}<\emph{xshift},\emph{yshift}>{\rm]} at \ldots\ \emph{xcoord} \emph{ycoord} \ldots\ *$n$
\emph{dxcoord} \emph{dycoord} \ldots\ /}}]
This command is used to \verb+\put+ copies of the same text at multiple
locations. The text will by put at each (\emph{xcoord},\emph{ycoord}), and at
each ($xcoord+i\cdot dxcoord$,$ycoord+i\cdot dycoord$) for $i$ from 1 to $n$.
\item[\tt\bsl nogridlines]
This command sets the default for the \type{andacross}/\type{butnotacross}
option of the \type{axis} command to be \type{butnotacross}.
\item[\tt\bsl normalgraphs]
This command resets the default \type{axis} options and values for the
axis parameters.
\item[\tt\bsl placehypotenuse for <$\xi$> and <$\eta$> in <$\zeta$>]
This command calculates Euclidean distance $\zeta=\sqrt{\xi^2+\eta^2}$. $\xi$
and $\eta$
may be explicit dimensions or dimension registers; $\zeta$ must be a
dimension register. 
\item[\tt\bsl placevalueinpts of <\emph{dimension register}> in
\{\emph{control sequence}\}]
This command puts the value of \emph{dimension register}, in points, into
\emph{control sequence}.
\item[{\tt\linestack[l]{\bsl plot "\emph{file name}"\\
\bsl plot \emph{xcoord}$_0$ \emph{ycoord}$_0$ \emph{xcoord}$_1$ \emph{ycoord}$_1$ \emph{xcoord}$_2$ \emph{ycoord}$_2$
\ldots\ /}}]
This commands plots the points given (or points from a file, if the first form
is used), in the current \emph{interpolation mode}. The interpolation modes
are selected by the commands \verb+\setbars+, \verb+\sethistograms+,
\verb+\setlinear+ and \verb+\setquadratic+.
\item[\tt\bsl plotheading \{\emph{heading}\}]
This command places \emph{heading} centred above the plot area.
\item[{\tt\bsl put \{\emph{text}\} 
{\rm[}[{\rm[}$o_x${\rm]}{\rm[}$o_y${\rm]}]{\rm]}
{\rm[}<\emph{xshift},\emph{yshift}>{\rm]} at \emph{xcoord} \emph{ycoord}}]
This commands places \emph{text} with its enclosing box centred about
(\emph{xcoord},\emph{ycoord}). If $o_x$ is `\type{r}' or `\type{l}' the right or left
edge of the box will be aligned on \emph{xcoord}. If $o_y$ is `\type{t}',
`\type{b}' or `\type{B}', the top, bottom, or baseline  will be aligned on
\emph{ycoord}. If \type{<\emph{xshift},\emph{yshift}>} is specified, the object will be
shifted \emph{xshift} right and \emph{yshift} up from where it would otherwise go.
\item[{\tt\bsl putbar {\rm[}<\emph{xshift},\emph{yshift}>{\rm]} breadth <$\beta$> from
\emph{xcoord}$_s$ \emph{ycoord}$_s$ to \emph{xcoord}$_e$ \emph{ycoord}$_e$}]
This command draws a rectangle which has (\emph{xcoord}$_s$,\emph{ycoord}$_s$) and
(\emph{xcoord}$_e$,\emph{ycoord}$_e$) as the mid-points of opposite sides of length $\beta$.
Either \emph{xcoord}$_s$ and \emph{xcoord}$_e$ should be the
same, or  \emph{ycoord}$_s$ and \emph{ycoord}$_e$ should be the same.
\type{<\emph{xshift},\emph{yshift}>} has the same effect as in the \verb+\put+ command.
\item[{\tt\bsl putrectangle {\rm[}<\emph{xshift},\emph{yshift}>{\rm]} corners at
\emph{xcoord}$_s$ \emph{ycoord}$_s$ and \emph{xcoord}$_e$ \emph{ycoord}$_e$}]
This command draws a rectangle with opposite corners at the points
(\emph{xcoord}$_s$,\emph{ycoord}$_s$) and (\emph{xcoord}$_e$,\emph{ycoord}$_e$).
\item[{\tt\bsl putrule {\rm[}<\emph{xshift},\emph{yshift}>{\rm]} from
\emph{xcoord}$_s$ \emph{ycoord}$_s$ to \emph{xcoord}$_e$ \emph{ycoord}$_e$}]
This command draws a rule from the point
(\emph{xcoord}$_s$,\emph{ycoord}$_s$) to the point (\emph{xcoord}$_e$,\emph{ycoord}$_e$), with breadth
\verb+\linethickness+.
Either \emph{xcoord}$_s$ and \emph{xcoord}$_e$ should be the
same, or  \emph{ycoord}$_s$ and \emph{ycoord}$_e$ should be the same.
\type{<\emph{xshift},\emph{yshift}>} has the same effect as in the \verb+\put+ command.
\item[\tt\bsl rectangle <$w$> <$h$>]
This command draws a rectangle of width $w$ and height $h$, with its baseline
on its bottom edge.
\item[\tt\bsl replot "\emph{file name}"]
This command replots lines and curves which were saved to a file by
\verb+\savelinesandcurves+.
\item[\tt\bsl savelinesandcurves on "\emph{file name}"]
This command writes out the locations at which it places plot symbols while
plotting lines (not rules) and curves.
\item[{\tt\linestack[l]{\bsl setbars {\rm[}<\emph{xshift},\emph{yshift}>{\rm]} breadth
<$\beta$> at $z$ = \emph{zcoord}\\
\quad{\rm[}baselabels ({\rm[}[{\rm[}$o_x${\rm]}{\rm[}$o_y${\rm]}]{\rm]}
{\rm[}<\emph{xshift},\emph{yshift}>{\rm]}){\rm]}\\
\quad{\rm[}endlabels ({\rm[}[{\rm[}$o_x${\rm]}{\rm[}$o_y${\rm]}]{\rm]}
{\rm[}<\emph{xshift},\emph{yshift}>{\rm]}){\rm]}}}]
This command sets the interpolation mode to bar plotting mode. If $z$ is
`\type{x}', the bars start from $x=zcoord$ and extend horizontally, and if $z$
is `\type{y}', the bars start from $y=zcoord$ and extend vertically.
\type{<\emph{xshift},\emph{yshift}>} has the same effect as in the \verb+\put+ command.
Labels can be attached to the bases of the bars with the \type{baselabels}
option. Each coordinate specification in the \verb+\plot+ command should be
followed by the appropriate label, enclosed in quotation marks. The
orientation and shifts may be used to adjust the label position.
Labels can similarly be attached to the ends of the bars with the
\type{endlabels}
option.
\item[\tt\bsl setcoordinatemode]
This command cancels \verb+\setdimensionmode+.
\item[{\tt\bsl setcoordinatesystem {\rm[}units <\emph{xunit},\emph{yunit}>{\rm]} {\rm[}
point at \emph{xcoord} \emph{ycoord}{\rm]}}]
This command redefines the coordinate system in use. \emph{xunit} is the size of
one unit on the $x$-axis, \emph{yunit} is the size of one unit on the $y$-axis. The
\type{point} option sets the reference point for the coordinate system. The
reference points of all of the coordinate systems in a picture are aligned by
\PiCTeX.
\item[{\tt\bsl setdashes {\rm[}<$\ell$>{\rm]}}]
This command resets the line pattern to be dashes of length $\ell$ followed by
gaps of length $\ell$.
\item[\tt\bsl setdashesnear <$\ell$> for <$\lambda$>]
This command sets the line pattern to be dashes of about length $\ell$, so
that a line of length $\lambda$ starts and ends with a complete dash.
\item[\tt\bsl setdashpattern <$d_1$,$g_1$,$d_2$,$g_2$,\ldots>]
This command resets the line pattern to be a dash of length $d_1$ followed by
a gap of length $g_1$, followed by a dash of length $d_2$, followed by a gap
of length $g_2$, etc.
\item[\tt\bsl setdimensionmode]
This command sets dimension mode; each location in this mode should be
specified by the absolute distance horizontally and vertically from the
origin, as dimensions.
\item[{\tt\bsl setdots {\rm[}<$\ell$>{\rm]}}]
This command resets the line pattern to be dots spaced distance $\ell$ apart.
\item[\tt\bsl setdotsnear <$\ell$> for <$\lambda$>]
This command sets the line pattern to be dots spaced about distance $\ell$
apart, so
that a line of length $\lambda$ starts and ends with a dot.
\item[\tt\bsl sethistograms]
This command sets the interpolation mode to histogram mode. In this mode,
\verb+\plot+ plots histograms composed of rectangles with corners at
(\emph{xcoord}$_0$,\emph{ycoord}$_0$) and (\emph{xcoord}$_1$,\emph{ycoord}$_1$), (\emph{xcoord}$_1$,\emph{ycoord}$_0$)
and (\emph{xcoord}$_2$,\emph{ycoord}$_2$), etc.
\item[\tt\bsl setlinear]
This command sets the interpolation mode to linear mode. In this mode,
\verb+\plot+ draws straight lines between coordinates.
\item[\tt\bsl setplotarea x from \emph{xcoord}$_l$ to \emph{xcoord}$_r$, y from \emph{ycoord}$_b$
to \emph{ycoord}$_t$]
\begin{sloppypar}
This command sets the current plot area to be a rectangle from
(\emph{xcoord}$_l$,\emph{ycoord}$_b$) to (\emph{xcoord}$_r$,\emph{ycoord}$_t$).
\end{sloppypar}
\item[{\tt\bsl setplotsymbol (\{\emph{plot symbol}\}
{\rm[}[{\rm[}$o_x${\rm]}{\rm[}$o_y${\rm]}]{\rm]}
{\rm[}<\emph{xshift},\emph{yshift}>{\rm]})}]
This command sets the symbol which is used to make lines and curves to be
\emph{plot symbol}.
\type{<\emph{xshift},\emph{yshift}>} and \type{[{\rm[}$o_x${\rm]}{\rm[}$o_y${\rm]}]}
have the same effect as in the \verb+\put+ command.
\item[\tt\bsl setquadratic]
This command sets the interpolation mode to be quadratic mode. In this mode,
quadratic arcs are drawn through the \verb+\plot+ coordinates.
\item[{\tt\bsl setshadegrid {\rm[}span <$s$>{\rm]} {\rm[}point at \emph{xcoord}
\emph{ycoord}{\rm]}}]
This command resets the anchor point of the grid used for shading to be
(\emph{xcoord},\emph{ycoord}), and the size of the grid to be $s$.
\item[{\tt\bsl setshadesymbol
{\rm[}<$\epsilon_{l}$,$\epsilon_{r}$,$\epsilon_{d}$,$\epsilon_{u}$>{\rm]}
(\{\emph{shade symbol}\}
{\rm[}[{\rm[}$o_x${\rm]}{\rm[}$o_y${\rm]}]{\rm]}
{\rm[}<\emph{xshift},\emph{yshift}>{\rm]})}]
This command resets the symbol used to shade areas to be \emph{shade symbol}.
The optional `edge effects' field 
\type{<$\epsilon_{l}$,$\epsilon_{r}$,$\epsilon_{d}$,$\epsilon_{u}$>}
specifies the distances from the left, right, bottom and top edges within
which the shade symbol will \emph{not} be placed. 0pt may be specified by
`\type{z}'.
\type{<\emph{xshift},\emph{yshift}>} and \type{[{\rm[}$o_x${\rm]}{\rm[}$o_y${\rm]}]}
have the same effect as in the \verb+\put+ command.
\item[\tt\bsl setsolid]
This command restores the line pattern to draw solid lines.
\item[\tt\bsl shaderectanglesoff]
This command cancels \verb+\shaderectangleson+.
\item[\tt\bsl shaderectangleson]
This command causes all rectangles plotted by \PiCTeX{} to be shaded
automatically.
\item[{\tt\bsl stack {\rm[}[$o$]{\rm]}
{\rm[}<\emph{leading}>{\rm]}\{\emph{list}\}}]
This command stacks textual items vertically. \emph{list} is a list of items
to be stacked, from top to bottom, separated by commas. Items are left
justified if $o$ is `\type{l}', right justified if $o$ is `\type{r}', and
centred otherwise. \emph{leading} is the distance separating the enclosing
boxes of the items in the stack. The baseline of the stack is the baseline of
the bottom item.
\item[{\tt\bsl startrotation {\rm[}by $\cos(\theta)$ $\sin(\theta)${\rm]}
{\rm[}about $x_p$ $y_p${\rm]}}]
This command causes \PiCTeX{} to rotate lines, curves, shading patterns, and
\verb+\put+ coordinates by $\theta$ degrees anticlockwise around the point
($x_p$,$y_p$). The rotation lasts until a \verb+\stoprotation+, or until the
enclosing group ends.
\item[\tt\bsl stoprotation]
This command cancels any rotation in effect.
\item[\tt\bsl ticksin]
This command sets the default for the \type{in}/\type{out}
option of the \type{axis} command to be \type{in}.
\item[\tt\bsl ticksout]
This command sets the default for the \type{in}/\type{out}
option of the \type{axis} command to be \type{out}.
\item[\tt\bsl unloggedticks]
This command sets the default for the \type{logged}/\type{unlogged}
option of the \type{axis} command to be \type{unlogged}.
\item[\tt\bsl visibleaxes]
This command sets the default for the \type{visible}/\type{invisible}
option of the \type{axis} command to be \type{visible}.
\item[{\tt\linestack[l]{\bsl vshade $x_0$ $y_0^{(b)}$ $y_0^{(t)}$ \ldots\
{\rm[}<$\epsilon_{l;i}$,$\epsilon_{r;i}$,$\epsilon_{d;i}$,$\epsilon_{u;i}$>{\rm]}
$x_i$ $y_i^{(b)}$ $y_i^{(t)}$ \ldots\ /\\
\bsl vshade $x_0$ $y_0^{(b)}$ $y_0^{(t)}$ \ldots\
{\rm[}<$\epsilon_{l;i}$,$\epsilon_{r;i}$,$\epsilon_{d;i}$,$\epsilon_{u;i}$>{\rm]}
$x_{2i-1}$ $y_{2i-1}^{(b)}$ $y_{2i-1}^{(t)}$ $x_{2i}$ $y_{2i}^{(b)}$
$y_{2i}^{(t)}$ \ldots\ /}}]
This command shades a region with piecewise linear/quadratic bottom and top
boundaries. Sub-regions are defined by the coordinates ($x_n$,$y_n^{(b)}$),
($x_n$,$y_n^{(t)}$), ($x_{n+1}$,$y_{n+1}^{(t)}$) and
($x_{n+1}$,$y_{n+1}^{(b)}$).
The relations $x_n<x_{n+1}$ and $y_n^{(b)}\leq y_n^{(t)}$
should hold. For the duration of the shading the optional edge effect field
`\type{<$\epsilon_{l;i}$,$\epsilon_{r;i}$,$\epsilon_{d;i}$,$\epsilon_{u;i}$>}'
overrides the specifications made by \verb+\setshadesymbol+. The second form
should be used when \verb+\setquadratic+ is in effect.
\item[\tt\bsl writesavefile \{\emph{message}\}]
\begin{sloppypar}
This command writes out the text \emph{message} on the file specified by the
most recent \verb+\savelinesandcurves+ command.
\end{sloppypar}
\item[\tt\bsl Xdistance\{\emph{xcoord}\}]
This command is used to get the horizontal distance from the point \emph{xcoord} in
the current coordinate system to the origin.
\item[\tt\bsl Ydistance\{\emph{ycoord}\}]
This command is used to get the vertical distance from the point \emph{ycoord} in
the current coordinate system to the origin.
\end{syntax}

\section{Parameters}
The parameters which can be altered to change \PiCTeX's behaviour are:
\begin{syntax}
\item[\tt\bsl headingtoplotskip]
This is the distance between the baseline of the heading and the top of the
plot area, or the top of the top axis structure.
\item[\tt\bsl linethickness]
This parameter is the default thickness of axes, tick marks, and grid lines.
This control sequence has its normal \LaTeX{} meaning outside of \PiCTeX{}
pictures,
but \verb+\pictexlinethickness+ can be used outside of \PiCTeX{}
pictures to get the same effect as \PiCTeX's \verb+\linethickness+.
\item[\tt\bsl longticklength]
This is the default length of long ticks.
\item[\tt\bsl plotsymbolspacing]
This parameter defines the distance between plotted symbols in lines and
curves.
\item[\tt\bsl shortticklength]
This is the default length of short ticks.
\item[\tt\bsl stackleading]
This is the default space put between items in a \type{stack}.
\item[\tt\bsl tickstovaluesleading]
This parameter defines the distance separating the ticks and the box enclosing
the tick values.
\item[\tt\bsl totalarclength]
This register in general contains the length of the last line or curve.
\item[\tt\bsl valuestolabelleading]
This is the distance separating the box enclosing the tick values and the box
enclosing the axis label.
\end{syntax}

\section{Miscellaneous}
A couple of extra commands are provided by \PiCTeX{} for formatting names:
\begin{syntax}
\item[\tt\bsl PiC]
This command produces `\PiC'.
\item[\tt\bsl PiCTeX]
This command produces `\PiCTeX'.
\end{syntax}



\end{document}
