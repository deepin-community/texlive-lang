%# -*- coding: utf-8 -*-
\ifx\epTeXinputencoding\undefined\else % defined in e-pTeX (> TL2016)
  \epTeXinputencoding utf8    % ensure utf-8 encoding for platex
\fi

\documentclass[a4paper]{jsarticle}% supports \verb in \footnote
\usepackage[nohyperref]{doc}% doc v3
\usepackage{plarray}
\GetFileInfo{plarray.sty}
\title{Package \textsf{plarray} \fileversion}
\author{Hironobu Yamashita}
\date{\filedate}
\begin{document}

\maketitle

This package \textsf{plarray} provides small patches to \textsf{array}
package (\textsf{latex-tools} bundle) for use with Japanese
p\LaTeX/up\LaTeX.
By loading \textsf{plarray} in addition to/instead of \textsf{array},
you can avoid conflicts between (u)p\LaTeX\ kernel and \textsf{array}%
\footnote{Below is recent changes introduced to
p\LaTeX/up\LaTeX\ kernels by Japanese \TeX\ Development Community,
which cannot cope with \textsf{array} package:
\begin{itemize}
\item Remove extra \verb+\xkanjiskip+ around tabular environment
\item Inhibit JFM (Japanese \TeX\ font metric) glue
      around tabular cell components
\end{itemize}
The package \textsf{plarray} re-adds these changes to \textsf{array}.}.

If you are going to use both \textsf{plext} and \textsf{array} packages
on p\LaTeX/up\LaTeX, the easiest way is to use \textsf{plextarray};
please refer to its document.

This package is part of \textsf{platex-tools} bundle:
\begin{verbatim}
  https://github.com/aminophen/platex-tools
\end{verbatim}

\bigskip

この\textsf{plarray}パッケージは、\textsf{latex-tools}バンドルの
\textsf{array}パッケージをp\LaTeX/up\LaTeX で使うときのwrapperとして
機能します。さらに、\textsf{array}パッケージによって上書きされて
無効化されてしまうp\LaTeX/up\LaTeX カーネルの修正点を再有効化する
働きもします(具体的には、「tabular環境前後に\verb+\xkanjiskip+由来の
余分なアキが入る現象」と、「表のセル内の要素が始め括弧類で始まる場合や
終わり括弧類で終わる場合にJFMグルーが不自然に入る現象」に対処します)。

\bigskip
\begin{minipage}{0.45\linewidth}
\begin{verbatim}
  \documentclass{jarticle} % if you are using pLaTeX,
  \usepackage{plarray}     % load this!
  \begin{document}
  これは表の例です
  \begin{tabular}{|>{$}c<{$}|c|}\hline
    \exp(x) & 指数関数 \\ \hline
    \log(x) & 対数関数 \\ \hline
  \end{tabular}%
  終わり
  \end{document}
\end{verbatim}
\end{minipage}
\begin{minipage}{0.5\linewidth}
  これは表の例です
  \begin{tabular}{|>{$}c<{$}|c|}\hline
    \exp(x) & 指数関数 \\ \hline
    \log(x) & 対数関数 \\ \hline
  \end{tabular}%
  終わり
\end{minipage}
\bigskip

p\LaTeX/up\LaTeX で、\textsf{array}パッケージに加えてさらに
\textsf{plext}パッケージも使いたい場合は、
\textsf{plextarray}パッケージを読み込むのが簡単です。

\end{document}
