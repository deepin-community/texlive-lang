%# -*- coding: utf-8 -*-
\ifx\epTeXinputencoding\undefined\else % defined in e-pTeX (> TL2016)
  \epTeXinputencoding utf8    % ensure utf-8 encoding for platex
\fi

\documentclass[a4paper]{jsarticle}
\usepackage[nohyperref]{doc}% doc v3
\usepackage{colortbl}
\usepackage{plextcolortbl}
\GetFileInfo{plextcolortbl.sty}
\title{Package \textsf{plextcolortbl} \fileversion}
\author{Hironobu Yamashita}
\date{\filedate}
\begin{document}

\maketitle

The package \textsf{plextcolortbl} provides a tiny patch to
make \textsf{colortbl} compatible with \textsf{plext}.

\bigskip

\textsf{plextcolortbl}パッケージは、David Carlisle氏による
\textsf{colortbl}パッケージと、p\LaTeX の拡張パッケージである
\textsf{plext}パッケージを同時に使えるようにするものです。

\section{使いかた}

\textsf{plext}と\textsf{colortbl}を共存させたいときに、プリアンブルに
\verb+\usepackage{plextcolortbl}+と書きます。

以下に例を示します。

\bigskip
\begin{minipage}{0.5\linewidth}
\begin{verbatim}
  \documentclass{jsarticle}
  %\usepackage{plext}
  %\usepackage{colortbl}
  \usepackage{plextcolortbl}
  \begin{document}
  \begin{tabular*}<t>{3cm}{%
  @{\extracolsep{\fill}}
  >{\columncolor{green}[0pt][20mm]}l
  >{\columncolor{yellow}[5mm][0pt]}l
  @{}}
  one & いち \\
  two & に
  \end{tabular*}
  \end{document}
\end{verbatim}
\end{minipage}
\begin{minipage}{0.4\linewidth}
  \begin{tabular*}<t>{3cm}{%
  @{\extracolsep{\fill}}
  >{\columncolor{green}[0pt][20mm]}l
  >{\columncolor{yellow}[5mm][0pt]}l
  @{}}
  one & いち \\
  two & に
  \end{tabular*}
\end{minipage}

\end{document}
