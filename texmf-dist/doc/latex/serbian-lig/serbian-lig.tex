%% serbian-lig.tex
%% Copyright 2011 Zoran Filipovi�
%
% This work may be distributed and/or modified under the
% conditions of the LaTeX Project Public License, either version 1.3
% of this license or (at your option) any later version.
% The latest version of this license is in
% http://www.latex-project.org/lppl.txt
% and version 1.3 or later is part of all distributions of LaTeX
% version 2005/12/01 or later.
%
% This work has the LPPL maintenance status `maintained'.
%
% The Current Maintainer of this work is Zoran Filipovi�
%

\documentclass[a4paper,12pt]{article}
\usepackage[T1]{fontenc}
\usepackage[cp1250]{inputenc}
\usepackage{textcomp}

\usepackage{serbian-lig} 

\usepackage{xspace}
\usepackage{soul}
\usepackage{pifont}
\usepackage{tipa}

\title{The \textsf{serbian-lig} package\protect\footnote{this package is under LPPL licence,
       version 1.3c}}
\author{Zoran T. Filipovi\'{c} \\ Jurija Gagarina 263/6 \\ 11070 New Belgrade, Serbia}

\begin{document}
\frenchspacing
\maketitle

\begin{abstract}
This package provide disable ligatures for serbian languages. 
In serbian languages, in latin scripts, exist only \verb|fi| and \verb|fl| ligatures, 
so this package provide disable ligatures for this type of ligatures. 
\end{abstract}

\section{Introduction}

This package is activate by typing \verb|\usepacake{serbian-lig}| in preamble for your 
document, and provide disable ligature for words which contains \fbox{fi} \textit{\&} 
\fbox{fl} ligatures in serbian language in latin scripts. The list of words with ligatures
you may see in document \verb|lig-list.pdf| and same words in \verb|serbian-lig.sty|
package. 

\section{Example}

\noindent 
If you tupe: \verb|Kada je oficir sreo prefinjenu damu najfinijeg manira| \\
\verb|po�eo je da vadi fleke.|

\noindent
This produce: Kada je oficir sreo prefinjenu damu najfinijeg manira po\v{c}eo je da 
vadi fleke.

\noindent
If you tupe: \verb|Kada je \oficir sreo \prefinjenu damu \najfinijeg manira| \\
\verb|po�eo je da vadi \fleke.|

\noindent
This produce: Kada je \oficir sreo \prefinjenu damu \najfinijeg manira po�eo je da 
vadi \fleke{.}
\begin{center}
\ding{167} \qquad \ding{167}  \qquad \ding{167}
\end{center}
\centerline{So, happy \TeX ing in serbian language.}

\end{document}