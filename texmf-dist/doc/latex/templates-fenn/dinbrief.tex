%
% dinbrief.tex
%
% This file is published and distributed under the latest version of
% the LPPL. By Juergen Fenn.
% You can reach me at http://www.juergenfenn.de/ .
%
% dinbrief.tex provides a commented template for writing a letter with
% dinbrief according to the German DIN standards. 
%
% This file is only available in German. If you would prefer an
% internationalised version please write me an e-mail.
%
\documentclass[11pt]{dinbrief}

\usepackage{german}
\usepackage[T1]{fontenc}
\usepackage[latin1]{inputenc}
% \usepackage{mathptmx} % Fuer Times auf 12pt umstellen!
\usepackage{mathpazo} % Fuer Palatino auf 11pt umstellen!
\usepackage{url}

\address{frei gestalteter Briefkopf}
\signature{(Name)}

\backaddress{Absender Name. Stra�e 1. 12345 Ort.}
\nowindowrules                 %% keine Rahmen um das Adre�fenster

\def\enclname{Anlage}
\newenvironment{myquote}{\begin{quote}\footnotesize}{\end{quote}}
\sloppy %                                  ja, trotz allem...
\begin{document}
%\setaddressheight {3cm}       %% H�he des Adressfensters ohne Absender
\begin{letter} {HerrnFrauFirma\\ \\Stra�e\\9999 Stadt}
%\postremark {Einschreiben mit R�ckschein}
%\handling{Eilt sehr!}
\centeraddress                 %% Vertikeles Zentrieren der Anschrift
% \setreflinetop{120mm}        %% Oberkante Bezugszeichenzeile = Briefanfang
% \yourmail{Empf�nger & Co. }  %% Ihr Zeichen
% \sign{Meine Firma}           %% Unser Zeichen
% \phone{12345}{ 67890}        %% Unsere Durchwahl

\date{\today}

\subject{}
\opening{Sehr geehrte Damen und Herren,}


%\pagebreak                    %% Seitenumbruch bei Bedarf

\closing{Mit freundlichen Gr��en,}

%\ps{}                                  %% PS:
%\encl{}                                %% Anlagen
%\cc{}                                  %% Verteiler
\end{letter}

\end{document}
%
% end of file