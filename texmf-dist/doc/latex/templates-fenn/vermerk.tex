%
% vermerk.tex
%
% This file is published and distributed under the latest version of
% the LPPL. By Juergen Fenn.
% You can reach me at http://www.juergenfenn.de/ .
%
% vermerk.tex provides a commented template for writing a general form
% for taking down notes on events in the office based on dinbrief.
%
% This file is only available in German. If you would prefer an
% internationalised version please write me an e-mail.
%
\documentclass[a4paper,german,11pt]{article}

\renewcommand{\familydefault}{\sfdefault}
\usepackage{babel,textcomp}
\usepackage[T1]{fontenc}
\usepackage[latin1]{inputenc}

\usepackage{fancyhdr}% Code by Peter Flynn from ctt, sligtly modified
\usepackage[headheight=16pt]{geometry}
\pagestyle{fancy}
\lhead{\vbox to0pt{\vrule height128mm width0pt\llap{---\hspace{2cm}}\vss}}
\rhead{}
\cfoot{}
\renewcommand{\headrulewidth}{0pt}

\begin{document}

\noindent \rule{\textwidth}{0.4pt}
\begin{center}\huge\textbf{Vermerk}\end{center}
\noindent \rule{\textwidth}{0.4pt}

\begin{description}
\item[Datum/Uhrzeit:]
\item[Kontakt zu:] 
\item Firma/Beh�rde:
\item Abteilung:
\item Ansprechpartner:
\item Stra�e:
\item Ort:
\item Telefon:
\item Telefax:
\item E-Mail/WWW:
\item[Vorgang:] 
\end{description}

\end{document}
%
% end of file