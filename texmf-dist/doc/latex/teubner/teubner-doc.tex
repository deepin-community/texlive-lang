% !TEX encoding = UTF-8 Unicode
% !TEX TS-program = pdflatex
\documentclass[12pt,a4paper]{article}
%\usepackage{trace}
\usepackage[utf8]{inputenc}
\usepackage[T1]{fontenc}
%\usepackage{lmodern}
\usepackage{txfonts}
%\usepackage{pxfonts}
\usepackage{mflogo}
\usepackage[greek,english]{babel}
\def\workingdate{2020/11/01}% see two lines ahead
\usepackage[boldLipsian,10pt,GlyphNames]% <--------  teubner options
           {teubner}[\workingdate]% <--- older versions don't work well
\usepackage[scaled=0.90]{helvet}
\ifFamily{phv}{cmss}
\makeatletter
	\def\GetFileInfo#1{%
	  \def\filename{#1}%
	  \def\@tempb##1 ##2 ##3\relax##4\relax{%
	    \def\filedate{##1}%
	    \def\fileversion{##2}%
	    \def\fileinfo{##3}}%
	  \edef\@tempa{\csname ver@#1\endcsname}%
	  \expandafter\@tempb\@tempa\relax? ? \relax\relax}
	\GetFileInfo{teubner.sty}
  \ifnum\expandafter\@parse@version\filedate//00\@nil<%
        \expandafter\@parse@version\workingdate//00\@nil
     \def\@tbnhlp{Probably your typesetting will fail; 
     in any case\MessageBreak
     expect a lot of errors and warnings!\@ehc}%
     \@latex@error{This file must be processed with\MessageBreak
     a teubner package\MessageBreak 
     with a file date not older than \workingdate;\MessageBreak 
     your teubner package has a date mark of \filedate.
     }\@tbnhlp     
  \fi
\makeatother
\usepackage{mflogo}
\usepackage{afterpage}
\usepackage{graphicx}
\usepackage{microtype}
\usepackage{multicol}
\title{\texttt{teubner}%
\thanks{This paper documents \texttt{teubner} version \fileversion\ of \filedate.}\\
An extension for Greek philology\\
User manual}
\author{Claudio Beccari\thanks{\texttt{claudio dot beccari at gmail dot com}}}
\date{October 2022}% <---- this is the date of the teubner-doc document!
\def\comando#1{\expandafter\texttt\expandafter{\string#1}}
\def\cs#1{\texttt{\char92#1}}
\def\ARG#1{\texttt{\char123}\meta{#1}\texttt{\char125}}
\def\meta#1{$\langle${\normalfont\itshape#1}$\rangle$}
\let\marg\ARG\let\Arg\meta
\def\Marg#1{\texttt{\char123#1\char125}}\let\brArg\Marg
\def\oarg#1{\texttt{[\meta{#1}]}}
\def\Oarg#1{\texttt{[#1]}}
\def\opt#1{\texttt{[}\meta{#1}\texttt{]}}
\def\g#1 {\textDidot{#1}}
\def\l#1 {\textLipsias{#1}}
\renewcommand*{\textfraction}{0}
\renewcommand*{\topfraction}{1.}
\renewcommand*{\bottomfraction}{.5}
\def\acro#1{\expandafter\textsc\expandafter{\MakeLowercase{#1}}}

\def\thrule{\par\hrule\medskip}
\def\bhrule{\par\medskip\hrule}

\newcommand*\class[1]{\textsf{\slshape#1}}\let\amb\class
\newcommand*\pack[1]{\textsf{\slshape#1}}
\def\babel{\pack{babel}}
\newcommand*\file[1]{\texttt{#1}}

\newcommand\pdfLaTeX{pdf\/\LaTeX}
\newcommand\XeLaTeX{X\kern-0.14em\raisebox{-0.7ex}{\rotatebox[origin=c]{180}{E}}\kern-0.075em\LaTeX}
\newcommand\LuaLaTeX{Lua\/\LaTeX}

\advance\textheight80pt \advance\topmargin-40pt

\providecommand\setfontsize{}
\DeclareRobustCommand*\setfontsize[2][1.2]{%
\linespread{#1}\fontsize{#2}{#2}\selectfont}

\listfiles

\begin{document}\errorcontextlines=20


\maketitle
{\columnseprule=0.4pt
\begin{multicols}{2}
\tableofcontents
\end{multicols}}

\section{Introduction}
This package \texttt{teubner.sty} is a module that extends the \textsf{greek} option of the \babel\ package intended to typeset classical Greek with a philological approach. This version 4.x cannot yet typeset the critical apparatus as the philologists are used to, but may be this work will continue and include also that facility. Apparently it is not incompatible with \textsf{eledmac}, but it may be considered a complement.

{\tolerance=9999
This document does not substitute the official package documentation \texttt{teubner.dtx} and its typeset version \texttt{teubner.pdf}; it extends the information contained into those files.\par}

This package is supposed to work with my CB fonts available on the
Comprehensive \TeX\ Archive Network (\textsc{ctan}); one of the actions of this package consists in adding to the default “italic” Greek shape another one called “Lispiakos” in Greece; this name derives from the high quality of the fonts used in the printers' shops in the city of Lipsia in the past 100~years or so; one of the printer shops that continues printing books for philologists (since 1849) is the B.G.~Teubner Verlags\-gesell\-schaft, that publishes the collection called “Bibliotheca Scriptorum Graecorum et Latinorum Teubneriana”. The name given to this package is in homage to that printing company and to its high quality tradition in printing Greek texts.

This package is generally loaded by default with any complete distribution of the \TeX\ system; in basic distributions it might not be there, but in general it is sufficient to use the distribution facilities for loading it; with MiKTeX, for example one uses the MiKTeX settings and the Package tab. With \TeX\-live it is sufficient to operate with the program \textsf{tlmgr}; and so on.

The problem, if there is one, is to have the full collection of the CB fonts; in partial distributions the scalable PostScript CB fonts are in one size, 10\,pt, and it is necessary to use the \emph{10pt} option in order to have the various enlarged or reduced sizes. For professional work, if the user has a partial \TeX\ system installation, it is necessary to load the \texttt{cbgreek-full} font collection and be sure that the map files are duly upgraded. In case, read your distribution instructions to see how to control and, possibly, how to provide for this upgrade. Please, notice that this very document has been typeset with the \emph{10pt} option in force, so that if you want to typeset it again from source, you need the minimum amount of vector fonts needed by \textsf{pdflatex} to do its job.

In any case this \texttt{teubner} package makes the CB font collection (full or reduced) directly compatible by construction with both the CM fonts (OT1 and T1 encoded; the latter ones are often referred to as the EC fonts) and with the LM fonts; it also contains adequate hooks in order to make them compatible and usable with other font collections; this very file has been successfully typeset using the CM and LM fonts, and also resorting to the Times and Palatino eXtended collections requested with the packages \texttt{txfonts}\footnote{Antonis Tsolomitis wrote package \textsf{txfontsb} for using a different Greek font collection together with the TX fonts so that the Latin and Greek glyph styles directly match each other; see the \textsf{txfontsb} package documentation for more information.} and \texttt{pxfonts} respectively.

This additional documentation will start by briefly recalling some peculiarities of the CB fonts and their mapping to the Latin keyboard; afterwards it will list the new commands and their syntax.

\section{Usage}
The \pack{teubner} package is loaded in the usual way, but there are some simple rules to follow:
\begin{quote}
\texttt{\string\usepackage}\opt{options}\Marg{teubner}
\end{quote}


Since \pack{teubner} is a module that adds to the |greek.ldf| support for \pack{babel} package the functionalities needed by the Hellenists, it must be loaded after the latter. If you load it before and/or if you load it after, but you invoked the \texttt{babel} package without specifying the \meta{greek} language (with or without the \texttt{ancient} Greek language attribute\footnote{It is sufficient to specify the language \texttt{greek} to \pack{babel}, and \pack{teubner} provides to set the ancient attribute. If you specify also the \texttt{ancient} attribute to \texttt{babel}, this sends a message informing you that this attribute has been specified a second time.}) \texttt{teubner} refuses to completely load and outputs a message very clear on this subject, so that you know what you should do in order to use the facilities offered by this module. 

\noindent\textbf{Warning}: Some people like the Lipsian font shape and want to load it also for writing mathematics with it. No problem, but they shouldn't do this with \pack{teubner}, which is useful only to typeset Greek text. For using other alphabets in mathematics there are more suitable ways that rely on the commands described in the \texttt{fntguide.pdf} distributed with every \TeX\ system in \texttt{\$TEXMFDIST\slash doc\slash latex\slash base/}.

\section{Package options}

The  options available to \pack{teubner} are collected in table~\ref{t:opzioni} with their meaning and usefulness.
\begin{table}
\thrule
\centering\def\R{\rule[-1ex]{0pt}{3ex}}\let \i\itshape
\begin{tabular}{lp{.7\textwidth}}
Option	& Meaning\R	\\\hline
\i or\R		& obsolete option kept for backwards compatibility\\
\i 10pt		& for using only one  real font size and geometrically enlarging or reducing the other sizes from the original single 10\,pt one\\
\i boldLipsian& to be selected in order to use a moderately blacker Lipsian font\\
\i NoGlyphNames& to be selected in order to avoid the definition of accented glyphs (default)\\
\i GlyphNames & to be selected in order to have available also the set of macros that directly address the accented glyphs\\[1ex]
\hline
\end{tabular}
\caption{Options for the \pack{teubner} package}\label{t:opzioni}
\bhrule
\end{table}

You should never need (and therefore never use) the \textit{or} option for two reasons: $(a)$ glyph name macros are not needed for the first letter of any word since there are no kerning problems with a preceding letter; $(b)$ glyph name macros are insensitive to uppercasing.

\noindent\textbf{Warning}: You might need to use the \textit{10pt} option because your Greek font collection is the minimal one. But if you are using \texttt{teubner}, you are likely to seek the best possible typeset text; therefore you should download and install the full collection of the Greek CB fonts. This full collection is generally already available with any complete distribution of the \TeX\ system.

If you are using Latin fonts different from the CM, CM-Super, EC, or LM collections, you have to control what you get with or without the option \textit{boldLipsian}; generally speaking, this option is best used with darker normal fonts; I have tried the Times, the Palatino, and the Fourier ones, and actually this option is desirable. This might not be true with other vector fonts.

With the inclusion of the extended accent macros in the \pack{teubner} package (see below), the named accented glyphs should not be needed anymore; these accented glyph name macros are not defined by default, but for backwards compatibility they are available if the \emph{GlyphNames} option is specified; with the 2013 new Greek support for the \babel\ package, the extended accent macros are defined in a more efficient way and you should not need them any more; therefore, unless you specify this option, you avoid overloading the internal \TeX\ memory areas, thus leaving extra space for more useful packages. I suggest you to let the \emph{default} option \textit{NoGlyphNames} act as its name implies, and make use of the extended accent macros, should the need arise (see below). Moreover, if your keyboard has facilities for entering polytonic Greek, or if you install a keyboard driver that allows you to do so, you can directly input Greek text with the Greek alphabet, without using the transliteration provided by the Greek support for \babel.

\section{The CB Greek fonts}
The CB Greek fonts (full collection) come in all shapes, sizes and series as the European Modern fonts that conform with the T1 encoding introduced after the Cork Conference of the \TeX\ Users Group Society in 1991\footnote{If you want to or you have to use the single 10pt size fonts, you certainly produce smaller PDF files, at the expense of a slightly poorer typographical quality.}. The CB fonts conform to the encoding that is still being called LGR, since up to now there is no
established encoding name for the Greek alphabet among the \TeX\ users, not yet, at least. For several reasons I suggest to use the Latin Modern fonts, instead of the CM or EC (CM-Super) ones, because the former are piecewise continuously scalable, while the latter offer only fixed font sizes. Recently also the CM-Super fonts may be continuously scaled as the Latin Modern ones, but it is necessary to use a specific extra package without any option.

The regular shape has capital letters with serifs that are in the
same style as the Latin capital ones, while the lower case letters
derive from the design by Didot and are very common in all texts.
This shape comes also in boldface, together with the corresponding
oblique (or slanted) versions. The CB fonts contain
also the upright and slanted, medium and boldface small caps
alphabets.

The “italic” shape was designed in order to imitate the Olga font designed so as to have a contrasting style compared with the slanted Didot shape, in order to play the same role as the italic letters play with the Latin roman ones. The Olga alphabets come in medium and boldface series, and in oblique and upright shapes.

The CB fonts are completed with the sans serif fonts, the
monospaced typewriter fonts and the fonts for slides, besides an
outline family that shows the regular shapes and series just with
their contours; there is also a shape with serifed lower case letters.

The CB Lipsian fonts imitate the beautiful shapes used in Lipsia; they come in medium, bold, and extra-bold series, without an upright version, and they are meant to replace the corresponding Olga shapes. Their `simple bold' series is good for mixing with PostScript fonts, whose medium series is slightly blacker than the corresponding EC and LM fonts normally used with \LaTeX; this is easily achieved by passing the \emph{boldLipsian} option when invoking this \pack{teubner} file. Notice that the Lipsian font produces an alternative to the ordinary ‘italic’ Olga shape, that is not any more the default `italic' shape as it used to be with previous versions of this package; now, in the same document, you can use both shapes and produce both versions: 
\foreignlanguage{greek}{\textit{Baq'ulides} 
\textlatin{and} 
\textli{Baq'ulides}}. 
With the availability of the upright Olga font and the serifed lowercase alphabet, you can use also the \verb|\textui| and the \verb|\textrs| commands so as to obtain \foreignlanguage{greek}{\textui{Baq'ulides} \textlatin{and} \textrs{Baq'ulides}}; also in extended boldface: \foreignlanguage{greek}{\bfseries\textui{Baq'ulides} \textlatin{\mdseries and} \textrs{Baq'ulides}}.

But if you really want to permanently change the “italic” Greek font shape to the Lipsian one, without keeping the alternative, then in your preamble add the following statements \emph{after} you have loaded the \texttt{teubner} package:
\begin{verbatim}
\addto\extrasgreek{\def\itdefault{li}}%
\addto\noextrasgreek{\def\itdefault{it}}%
\end{verbatim}
With these settings, \verb|\textit| and \verb|\textli| become equivalent and both use the Lipsian shape while typesetting Greek text. Reverting to Latin script the \verb|\textli| text command uses the italic shape by default and there is no need to add anything else to the 
\verb|\noextrasgreek| macro.


\subsection*{Technical information}
Typesetting documents with different scripts sets forth some problems. The \emph{greek} option to the \babel\ package, besides setting up the typographical rules for Greek, as it does for any other language, provides the script change; it defines also the \verb|\textlatin| macro in order to typeset something with the Latin script while the default script is Greek, as well as the \verb|\textgreek| macro to typeset something with the Greek script when typesetting with a Latin one. Of course the usual \verb|\selectlanguage| and \verb|\foreinglanguage| commands, as well as the \texttt{otherlanguage} environment, provide for a global change of the typesetting characteristics or an environment where the settings are reset to Greek.

With standard \babel\ the CB fonts used to work only in conjunction with the CM fonts with either the OT1 (real CM fonts) or T1 (EC fonts) encoding. Since the 2008 distribution of standard \babel, the CB fonts work fine also with the Latin Modern fonts. This \pack{teubner} module tries to work seamlessly also with other font families, but it is not that simple. Some technical explanations are necessary.

When \LaTeX\ needs to use a specific font in a certain encoding and belonging to a  particular family, series and shape, available in such and such sizes, it gets this information by reading a \emph{font description file}; this file's name is composed with the encoding and the family names glued together and has the extension \texttt{.fd}; for OT1 encoded Computer Modern CM regular (serifed) fonts this file would be \texttt{ot1cmr.fd}. Any other package that is requested for using different fonts defines possibly a different encoding and certainly different family names. For using the Times eXtended TX fonts with T1 encoding, the \texttt{txfonts.sty} package defines the family name \texttt{txr} (for serifed fonts) so that \LaTeX\ reads the font description file \texttt{t1txr.fd} that contains the relevant information for all the series, shapes, and sizes available.

The CB fonts are encoded according to the LGR encoding but have the same family names as the CM ones; since 2008, also the family names of the Latin Modern LM collection are recognised; therefore the relevant font description files for the regular\footnote{The regular lower case Greek alphabet does not have serifs; serifs are present in the font shape called `serifed' that corresponds to the\texttt{\string\textrs} font command.}  families are \texttt{lgrcmr.fd} and \texttt{lgrlmr.fd} respectively. The two collections of description files are not equivalent with one another, and they are not equivalent to the OT1 or T1 encoded CM or LM fonts, in the sense that the series and shapes available for these sets of fonts are not identical, even if most of them are.

In this way with CM and LM fonts the script switching for the same (existing) series and shapes amounts to switching the encoding name. At the same time there might exist some font switching commands that refer to a series or shape that does not exist in the other script families; one important example in our case is the Lipsian shape that is available only with the CB fonts. There is no problem in declaring the Lipsian shape switching commands that behave in a proper way together with the CM and LM fonts, but even if I did my best for working with other font families, I am not 100\% confident that my macros restore correctly the other font characteristics when declaring a different series or shape.

Going more technical, the default family settings are stored into the three macros 
\verb|\rmdefault|, \verb|\sfdefault|, \verb|\ttdefault|; the series and the shape symbols are stored in similar macros; these default macros are accessed every time a font characteristic switching declaration or command is used. In the background the \verb|\selectfont| macro is executed and during the whole process the \emph{current} font characteristic macros are updated. Such macros are \verb|\f@encoding|, \verb|\f@family|, \verb|\f@series|, and \verb|\f@shape|; therefore when just one characteristic is modified, the new value is stored in the relevant current value macro and used in order to create the association with the actual font to be used.

If the font switching macros are used within a group, upon exiting that group the previous values are restored; but if declarations are used without a delimiting scope, there is no simple (universal) way to use another ready-made declaration that resets valid font characteristics. 

An example where the necessity of delimiting scopes is shown: suppose we are typesetting with T1 encoded LM roman medium normal (upright) fonts; then the current font characteristics codes are stored in the relevant macros as \texttt{T1}, \texttt{lmr}, \texttt{m}, \texttt{n}. We switch to Greek with a language setting declaration, and the current encoding is changed to \texttt{LGR}, but the other characteristics remain the same, therefore we would be typesetting with LGR encoded, CB Didot (upright) medium normal font. While typesetting in Greek we switch to Lipsian shape with a declaration such as \verb|\lishape|\footnote{Notice that the \texttt{\string\lishape} declaration and the corresponding text command \texttt{\string\textli} are defined in such a way that they switch to the Lipsian shape only when the Greek encoding LGR is in force; with other encodings they behave as the corresponding italic font commands. In spite of this the example being carried on is valid, because the \texttt{\string\lishape} \emph{declaration} is supposed to be issued while the Greek encoding was in force.}, and this declaration changes only the current shape characteristic, so that the four ones would be \texttt{LGR}, \texttt{lmr}, \texttt{m}, \texttt{li}. Up to this point everything runs smoothly because every characteristic that was set is present in the specific font family in use. At this point we revert to typesetting with the Latin script by means of a language switching declaration; the only change that takes place is on the encoding and the four characteristics would become \texttt{T1}, \texttt{lmr}, \texttt{m}, \texttt{li}, but\dots\ The Latin Modern fonts do not contain a Lipsian shape, therefore there is no actual font that meets the requirements and \LaTeX\ selects the \emph{error font}, the T1 encoded LM roman medium normal one, that in this case is the correct one, and apparently the font switching process achieved the correct result. This is only apparent: remember that the \emph{error font} was selected, not the right font. If the Latin family and the Greek family hadn't been the same, there would have been other difficulties and the error font would have been selected; this is why, if the user does not pay attention, when the default font should be, say, the TX font, after a switch to Greek, upon reverting to the Latin script the wrong font series or shape might be used. In the next paragraphs some indications are given in order to overcome this feature.

All these technicalities are really too technical, but it's necessary to have some clues in order to find out why sometimes the font switching commands don't work as expected. In some cases the font description file might provide a smart substitution for missing fonts, but it is not always the case.

In this package I tried to forecast most situations, but I am not sure I coped with every font characteristic combination.

In particular I hooked the language changing declarations with suitable default family names; for example, when using the Times or the Palatino eXtended TX or PX  fonts,  three new family description files are created so as to connect the LGR encoding and the names of the above scalable fonts to the corresponding CB fonts\footnote{I chose the CM Greek families, instead of the LM ones, because only the former are described by means of macros that cope with the \emph{10pt} option to the \texttt{teubner} package; just in case\dots\ The actual used fonts are the same in any case, except possibly for the visual sizes; the CM fonts come in fixed sizes, while the LM fonts are continuously scalable by enlarging or reducing a smaller number of base visual sizes.}. 

If you use different scalable fonts you can specify yourself the font associations you want to use; simply, after loading a package that sets other default font family or families, open the package file and take notice of the new family names; for each \meta{latin} family (serifed, sans serif, monospace) create a connection with the corresponding  \meta{greek} family by means of the following command:
\begin{quote}
\texttt{\string\ifFamily}\marg{latin}\marg{greek}
\end{quote}
right after the \verb|\usepackage| command with which you call that package.

Example: suppose you want to use the regular Fourier fonts to replace the default roman normal serifed fonts; then you should load the package \texttt{fourier} \emph{after} the \texttt{teubner} module and specify:
\begin{verbatim}
\usepackage{fourier}
\Lipsiantrue
\ifFamily{futs}{cmr}
\end{verbatim}
You can see that the family name \texttt{futs}, corresponding to the regular Fourier font family, has been matched to the CB Greek \texttt{cmr} font family; the name \texttt{futs} has been deduced by reading the \texttt{fourier.sty} file from which one can see that the family declaration for the regular Fourier font family is \texttt{futs}. This sort of coding does not take place with all fonts: if the Iwona fonts had to be chosen, for example, then the font family name would coincide with the font name, making it difficult to distinguish between the \emph{font name} and the \emph{font family name}.

Notice the \verb|\Lipsiantrue| command before issuing the matching command \verb|\ifFamily}|. This command is optional and is used only if the composer wishes to use the Lipsian fonts; in this case the \texttt{li} shape is also defined, so that the \verb|\textli| and 
\verb|\lishape| commands do not issue any error message. At the end of the execution of the \verb|\ifFamily| command, the boolean switch is automatically reset to \verb|\Lipsianfalse|.

\noindent\textbf{Warning}: According to several maintainers of the language support packages, the procedure used with \verb|\ifFamily|, that produces in the local working directory a real, permanent \texttt{.fd} file, is not the best approach to create such pairing of fonts. G\"unter Milde, for example, produced a \pack{substitutefontfamily} package that defines the command \comando\subtitutefontfamily\ which creates a similar association, but does not write anything to the disk; simply the created association declarations are stored in the \TeX\ working memory, so that when the typesetting job is over, no traces remain on disk; this certainly avoids clogging the disk with the same files scattered all over in different directories that were the working ones for different typesetting jobs.

\noindent\textbf{A small caveat}: when you issue for the first time the \verb|\ifFamily| command you might not see the expected result (while with Milde's package you immediately obtain the expected result), and your Greek text might be typeset with the default “Greek error font”. But the second time you typeset your document the expected result is obtained with the correct fonts. This is because with the very first run a new font description file is generated, and this file will be available only in subsequent typesetting runs.

\noindent\textbf{Another warning}: this \pack{teubner} package sometimes uses the “bold not extended” series, typical with the Lipsian fonts, so as to better match their medium blackness with darker fonts such as, for example, the Times extend ones. Sometimes the Lipsian shape is in force, but the metric symbols (see below) are also used, but such fonts lack the Lipsian shape. It might happen that you use some math symbols; the math symbols with \pdfLaTeX\ do not have anything similar to the Lipsian shape.In such cases the typesetting program issues a warning and at the end of the job you might read in the \texttt{.log} file a message such as \emph{Some font shapes were not available, defaults substituted.} No worry; the message is correct, but no errors are actually visible in the output PDF file. See below what you can do if you find such warning messages as “annoying” ones.

Nevertheless this does not imply that the correct fonts are used if the font switching macros are used without scoping groups or environments. Sometimes, when you use declarations instead of commands, it might be necessary to issue an apparently redundant \verb|\rmfamily| or \verb|\selectfont| command in order to re-establish the correct defaults.

In order to insert \emph{short texts} in Greek, either in Didot upright or in Lipsian inclined shape, the \emph{text} commands \verb|\textDidot| and \verb|\textLipsias| may be used, as well as a redefined 
\verb|\textlatin| \emph{text} command for typesetting a short Latin script text while typesetting in Greek; these macros should already select the correct encoding, family, series and shape in most circumstances.

The \emph{text commands}, contrary to the corresponding \emph{text declarations}, typeset their argument within a group, so that the font characteristics are also correctly restored after the command execution is completed. 

Therefore we suggest you to either use the text commands or to use the declarations as environment names (without the initial backslash), so that they provide the necessary group delimiters; it is syntactically correct and useful to input something such as:
\begin{quote}
\verb|\begin{Lipsiakostext}|\\
\meta{Greek text to be typeset with the Lipsian font}\\
\verb|\end{Lipsiakostext}|
\end{quote}

Some authors have defined themselves the environments \amb{GRD} and \amb{GRL} to typeset longer texts with Didot or, respectively, Lipsian fonts and such definitions are included in the \pack{teubner} package; see the documentation of the code in the \file{teubner.pdf} file.

Such environments have been tested with the Latin Modern, extended Times and extended Palatino fonts and with the standard classes besides the \class{memoir} and \class{scrbook} ones. No kludges were necessary with the Latin Modern fonts and the standard classes; some kludges were necessary in order to get the expected results with the other tested  fonts and classes; such kludges are described in the above code documentation file. They are not complicated, but users that use different fonts and/or different classes might need to apply similar kludges to their software. 


\section{Font installation}
In order to use the Greek CB fonts and the facilities provided by this
package, you need to install them, if they are not already installed by default when you install your preferred \TeX\ system \emph{complete} distribution. You can freely download those fonts from \acro{ctan}, where you can find both the PostScript scalable ones and the driver files for generating their bitmapped versions with \MF; since nowadays it's very unlikely that \LaTeX\ users limit themselves to a final DVI file, but typeset their documents in PDF (or PS formats), the pixel files are very unlikely needed to produce their final documents; the \MF\ pixel files just allow the DVI previewer (should one be used) to run \MF\ in the background in order to produce the necessary pixel files so as to display on the screen the typeset documents\footnote{I mean documents that have been typeset with \textsf{latex}, not with \textsf{pdflatex}.}. These days there are other solutions to preview directly the typeset output file in PDF format and to use the previewer for direct and inverse search, so that the DVI format becomes really necessary only in very special and rare occasions. In the future it's very likely that the choice among such PDF previewers is much greater than today: there are certainly the multi\-platform shell editors \textsf{TeXworks}, \textsf{TeXstudio}, and \textsf{Texmaker} that incorporate a PDF previewer capable of direct and inverse search; for modern Windows platforms there is the \textsf{SumatraPDF} previewer that may be configured to work with various shell editors in order to perform in the same way. For Mac platforms, besides the default shell editor and previewer \textsf{TeXShop}, and the multi-platform program \textsf{TeXworks}, the Mac specific \textsf{TeXnicle} and \textsf{Texpad}, the user can proceed with \textsf{Aquamacs} as a shell editor and \textsf{Skim} as PDF previewer that work together so as to allow direct and inverse search. 
 
\section{\TeX\ font metric files}
The \texttt{teubner} package requires the updated \TeX\ font metric files \texttt{.tfm} for all sizes and series of the Lipsian fonts.  With your file system explorer go to the folder \texttt{\$TEXMFDIST/fonts/tfm/public/cbfonts/} and read the date of, say, \texttt{grml1000.tfm}; if this date precedes the year 2010, then this \texttt{.tfm} file is “old” and must be replaced, and you should update your Greek CB font collection (may be you should update your whole \TeX\ system installation).

The updated \texttt{.tfm} files add some ligature and kerning information that is missing from the “old” ones. The Type~1 \texttt{pfb} font files have not been modified at all.
 
 
\section{Greek text and Latin keys}

In order to input Greek text with a Latin keyboard\footnote{Although tables~\ref{t:corrispondenza} and~\ref{t:accenti} display only Latin ASCII characters in the second line, there are some national keyboards where some of these ASCII symbols can't be typeset by striking a single, possibly shifted, keyboard key or a simple key combination; an example is the Italian keyboard, where both the `back tick' and the `tilde' can be input only in one of these two ways: $(a)$ open the Character Map accessory [some shell editors can open their internal character map, where to select the desired characters] and select the required glyph(s), or $(b)$ while pressing the \fbox{\rule{0pt}{1.1ex}\smash{\ttfamily Alt}} key, input the numerical glyph code from the numeric pad. Very uncomfortable!} some simple and mostly obvious key substitutions are performed according to the correspondence shown in table~\ref{t:corrispondenza}.

\begin{table}[bt]\tabcolsep4pt
\thrule
\centering
\begin{tabular}{*{25}c}
\g a & \g b &\g g &\g d &\g e &\g z &\g h &\g j &\g i &\g k &\g l &\g m &\g n &\g x &\g o &\g p &\g r &\g c &\g sv &\g t &\g u &\g f &\g q &\g y &\g w \\
   a &    b &   g &   d &   e &   z &   h &   j &   i &   k &   l &   m &   n &   x &   o &   p &   r &   c &   s  &   t &   u &   f &   q &   y &   w
\end{tabular}
\caption{Keyboard correspondence between Latin and Greek
letters}\label{t:corrispondenza}
\bhrule
\end{table}

Notice that there is the possibility of inputting \texttt{c} in
order to get the final sigma \textDidot{c}, but the CB fonts are
conceived with the non-Greek user in mind, so that it is even
possible to input \texttt{s} at the end of words, because the
whole software is smart enough to detect the word boundary and to
use the correct shape of the letter sigma within or at the word
end. This mechanism is so “sticky” that it becomes difficult to
type an isolated initial or middle sigma; the CB
fonts contain an invisible character, \texttt{v}, that may be
used for several purposes, one of which is to hide the word
boundary after a sigma; therefore if you type \texttt{sv}, you get
\textDidot{sv} without any effort.

The invisible character \texttt{v} may be used also as a support
for (apparently) isolated accents, especially when macros have to
be used; if you type \verb"\M{v}" you get
\enspace\textDidot{\M{v}}\enspace, while if you omit the invisible
\texttt{v} you get \enspace\textDidot{\M{}}\enspace.

In LICR (\LaTeX\ Internal Character Representation), that is being used by the modern (2013) Greek support for \babel, the \texttt{v} character is represented by the macro \comando\textcompwordmark\ whose name recalls a compound word separator; actually it may play also this role; but it is a special character, not simply an invisible strut; it has the same category code as the other alphabetic characters, it has a very small width but its height is equal to the other lower case letters without ascenders; it's a real character even if in practice it is invisible; in particular it does not impeach hyphenation as an equally sized normal strut would do. But its usefulness is $(a)$ to hide the word boundary; $(b)$ to support an accent that would drop to the base line if this character was absent; $(c)$ to break ligatures in those rare cases where it's necessary to do so.

Accents, spirits and dieresis may be input \emph{before} each letter
(prefix notation) without using any particular control sequence, that is resorting to the ligature mechanism  that is part of the font characteristics; but since with certain fonts this mechanism may break kernings, it's better to use the proper accent macros;
the correspondence between the Latin symbols and the Greek
diacritical marks is shown in table~\ref{t:accenti}; all “upper”
diacritical marks must be prefixed (in any order), while the iota
subscript must be postfixed. Therefore if you input \verb">'a|", you
get~\textDidot{>'a|}.

\begin{table}\let\t\ttfamily
\thrule
\centering
\begin{tabular}{l*{7}c}
Greek diacritics  &\g >   & \g < &\g "v   &\g ' &\g ` &\g \~v     &\g |  \\
Latin keys  &\t>    &  \t< &\t"     & \t' &\t ` &\t\string~{}    &\t|  \\
extended accent macros
        &\t\string\>&\t\string\<&\t\string\"&\t\string\'&\t\string\`&\t\string\~& 
\end{tabular}
\caption{Correspondence between the Latin keyboard symbols and
Greek diacritical marks; extended accent macros are also shown; notice that a couple of “high” diacritical marks may be joined in one macro (in any order) to produce the same result as with two separate macros; in other words, \texttt{\string\>\string\'}, \texttt{\string\'\string\>}, \texttt{\string\>'} and \texttt{\string\'>} are equivalent with one another. For what concerns \texttt{\string\~}, see remarks below table~\ref{t:accentmacros}}\label{t:accenti}
\bhrule
\end{table}

Macrons and breves are just single glyphs and do not appear in combination with any letter, due to the limitation of 256 glyphs per font; but they may be input by means of commands \verb|\M| and \verb|\B| respectively when typesetting in Greek in order to use them as single diacritics; for more than one diacritic superimposed to one another, when macrons and breves are involved, some other commands are available as shown in table~\ref{t:accentmacros}. 
f
The new (2013) Greek support for \babel\ contains now the LICR encoded accent macros that allow to use any encoding input method, including direct input of Greek characters. The internal representation of accent macros resorts to the same symbols shown in table~\ref{t:accenti}; simply they are prefixed by a backslash and/or the accent sequence is prefixed with one backslash; therefore \textDidot{>'a|} may be obtained with \verb+>'a|+, or \verb+\>\'a|+, or  \verb+\>'a|+. The last form is to be  preferred with some shapes as discussed in the following paragraphs. But the LICR (\LaTeX\ Internal Character Representation) is much more than this; it separates the input encoding from the output font encoding, so that direct literal greek input is possible (of course with the \opt{utf8} encoding) and the output is useful for both LGR encoded 8-bit fonts and the UNICODE encoded OpentType fonts; this way \pack{babel} and the Greek support for \babel, provided by the new \file{greek.ldf}, may be used also with \XeLaTeX\ and \LuaLaTeX. Moreover now (October 2014) file \file{greek.ldf} can select the ancient Greek hyphenation when typesetting with \pdfLaTeX\footnote{This is why with this release of the \pack{teubner} package, the default language attribute is set to \texttt{ancient}.}; with \pack{polyglossia} this possibility was available from the very beginning of the Greek support by this package.
The 2020 version of the \file{greek.ldf} file introduces some new coding incompatible with the extension provided by this module (up to version~48) to get Greek Milesian and Attic numerals. From \pack{teubner} version~5.0, the LICR codes used by the new \file{greek.ldf} file are extended in oder to get Greek numerals that use the ancient glyphs, and therefore some macros are redefined. On the occasion of this update, the necessary macros for the Greek numerals have been redefined by means of the \LaTeX\,3 language through the interface macros previously defined by the \pack{xparse} package and now (2020) integral part of the \LaTeX\ kernel; see below for further details.

Together with the macros for inserting such symbols, a complete set is available for inserting any combination of diacritical marks over or under any letter, not only vowels: see table~\ref{t:accentmacros}. 

Of course the results may not be comparable with the ones one can obtain with the regular ligature mechanism or by using the LICR macros; the advantage of the redefined accent macros is twofold: (a) it is connected to the possibility of inserting macrons and breves and/or to set the various combinations over or under \emph{any} letter, even if it is a consonant; (b) for all accent vowel combinations that have a specific glyph in the font, the actual accented symbol is used so that kernings and ligatures are maintained; this result is achieved also by using the extended accent macros shown in the third line of table~\ref{t:accenti}; as shown elsewhere there is a noticeable difference between {\Lipsiakostext a>ut'os} and \textLipsias{a\s{u}t\'os} or \textLipsias{a\>ut\'os} or \textLipsias{a\>ut'os}. In this example the first word is typed in as \verb+a>ut'os+, while the other words may be typed in as \verb+a\s{u}t\'os+, or \verb*+a\us t\oa s+, or \verb+a\>ut'os+, or even in mixed form \verb*+a\us t'os+, thanks to the fact that there is no kerning between `tau' and `omicron with or without acute'.

The attentive readers might have noticed that some accent macros use the control symbols that are available for other standard \LaTeX\ environments or even defined in the \LaTeX\ kernel. Accent macros \cs{`}, \cs{'} and \cs{=} are already taken care by the \LaTeX\ kernel that uses a particular trick to use them as control macros in the \amb{tabbing} environment; therefore there should not be any conflict. 

On the opposite some conflicts may arise with \cs{<} and \cs{>}, as reported by David Kastrup. We have examined these conflict occurrences and we noticed that actually \cs{>} is defined as a `math wide space' command; an alias \cs{:} is defined immediately after the definition, and the \LaTeX\ handbook by Leslie Lamport does not even mention \cs{>}; therefore \cs{>} may be considered a leftover from the beginnings of \LaTeX, and it would be surprising if anybody used it; nevertheless\dots\ The control symbol \cs{<} has been reported as being used as a macro with a delimited argument \cs{<}\meta{arg}\texttt{>} in the documentation of package \pack{calc}. We immagine that it is very unlikely that this \pack{teubner} module functionalities might be used in another package documentation, but\dots

These conflicts are avoided if Günter Milde package \pack{textalpha} is loaded; we did not find out any conflict in any testing run we made, but such conflicts may arise; therefore we decided to have \pack{teubner} load the \pack{textalpha} package at \verb|\begin{document}| execution time, so as to avoid any possible conflicts.


\begin{table}[tb]\thrule
\makebox[\textwidth]{\def\V{\rule{0pt}{2.5ex}}%
\begin{tabular}{cl|cl}
Example     & \multicolumn1c{Syntax}            & Example    & \multicolumn1c{Syntax}        \\[0.7ex]
\hline
\g \`{a}    & \comando\G\ARG{letter}\ap{1}\V   & \g\ut{aw}   & \comando\ut\ARG{letters}  \\
\g \'{a}    & \comando\A\ARG{letter}\ap{1}     & \g\Ab{a}    & \comando\Ab\ARG{letter}  \\
\g \~{a}    & \comando\C\ARG{letter}\ap{1}     & \g\Gb{a}    & \comando\Gb\ARG{letter}  \\
\g \"{i}    & \comando\D\ARG{letter}\ap{1}     & \g\Arb{a}   & \comando\Arb\ARG{letter} \\
\g \B{a}    & \comando\B\ARG{letter}\ap{4}     & \g\Grb{a}   & \comando\Grb\ARG{letter} \\
\g \U{ai}   & \comando\U\ARG{diphthong}        & \g\Asb{a}   & \comando\Asb\ARG{letter} \\
\g \M{a}    & \comando\M\ARG{letter}\ap{4}     & \g\Gsb{a}   & \comando\Gsb\ARG{letter} \\
\g \>a      & \comando\>\ARG{letter}\ap{3}     & \g\Am{a}    & \comando\Am\ARG{letter}  \\
\g \<a      & \comando\<\ARG{letter}\ap{3}     & \g\Gm{a}    & \comando\Gm\ARG{letter}  \\
\g \r{a}    & \comando\r\ARG{letter}           & \g\Cm{a}    & \comando\Cm\ARG{letter}  \\
\g \s{a}    & \comando\s\ARG{letter}           & \g\Arm{a}   & \comando\Arm\ARG{letter} \\
\g\Ad{i}    & \comando\Ad\ARG{letter}          & \g\Grm{a}   & \comando\Grm\ARG{letter} \\
\g\Gd{i}    & \comando\Gd\ARG{letter}          & \g\Crm{a}   & \comando\Crm\ARG{letter} \\
\g\Cd{i}    & \comando\Cd\ARG{letter}          & \g\Asm{a}   & \comando\Asm\ARG{letter} \\
\g\Ar{a}    & \comando\Ar\ARG{letter}          & \g\Gsm{a}   & \comando\Gsm\ARG{letter} \\
\g\Gr{a}    & \comando\Gr\ARG{letter}          & \g\Csm{a}   & \comando\Csm\ARG{letter} \\
\g\Cr{a}    & \comando\Cr\ARG{letter}          & \g\Sm{a}    & \comando\Sm\ARG{letter}  \\
\g\As{a}    & \comando\As\ARG{letter}          & \g\Rm{a}    & \comando\Rm\ARG{letter}  \\
\g\Sb{a}    & \comando\Sb\ARG{letter}          & \g\Rb{a}    & \comando\Rb\ARG{letter}  \\
\g\Gs{a}    & \comando\Gs\ARG{letter}          & \g\iS{a}    & \comando\iS\ARG{letter}  \\
\g\Cs{a}    & \comando\Cs\ARG{letter}          & \g\d{p}     & \comando\d\ARG{letter}   \\
\g\c{i}     & \comando\c\ARG{letter}           & \g\bd{i}    & \comando\bd\ARG{letter}  \\
\semiv{u}   & \comando\semiv\ARG{letter}\ap{2} & \ring{a}    & \comando\ring\ARG{letter}\ap{2} \\[0.7ex]
\hline
\end{tabular}}
\caption{Accent macros}\label{t:accentmacros}
\vskip\baselineskip
REMARKS\\[3pt]\footnotesize
\mbox{\quad}\ap{1}These four commands superimpose the corresponding accent over the \meta{letter} belonging to the Greek alphabet; the corresponding commands \comando\`, \comando\', \comando\~, and \comando\" play the same rôle also with non Greek letters; this allows the stacking of several accents on the same base \meta{letter}. \\
\mbox{\quad}\ap{2}Most commands may be used also with latin letters.\\
\mbox{\quad}\ap{3}While typesetting in Greek within a \texttt{tabbing} environment,
both commands \comando\> and \comando\< play a different r\^ole as tab shifters;
in this case either \comando\s\ and \comando\r\ are used for the spirit macros, or 
the standard ligature mechanism is used, especially if the spirit diacritics 
fall on the very first vowel of a word. Suggestion; never use the \texttt{tabbing} environment; there are more effective ways to align chunks of text, for example the \texttt{tabular} environment.\\
\mbox{\quad}\ap{4}In \pack{teubner} versions preceding 4.0 macros for typesetting macrons and breves in Greek had the same name as the corresponding macros used when typesetting with Latin script, that is \comando\=\ and \comando\u. With the new Greek support for the \pack{babel} package that was implemented during year 2013, the Latin macros were not functional any more in the proper way, and were updated in \pack{teubner} version 4.0 with new names.
\bhrule
\end{table}

\section{Ligatures}
It should be clear from the previous section that the ligature mechanism is the one that offers good results with most accented vowels, while speeding up the keying-in of the text to be typeset; nevertheless there are situations where you might be unsatisfied.

For example compare \textLipsias{a>ut'os} with \textLipsias{a\>ut\'os}. The small spacing difference, if any, between tau and the accented omicron is not noticeable, but the spacing difference between alpha and the marked upsilon is remarkable. Where does that difference come from? It comes from the fact that the smooth spirit marker inhibits kerning between the previous alpha and the resulting ligature from the spirit marker and the upsilon\footnote{The CB fonts may be used also with monotoniko spelling; in this case \textLipsias{A"ULOS} comes out well even with ligatures \texttt{A"ULOS}, because a special kludge has been devised; with the extended accent macros it would not be too boring to typeset \texttt{A\string\"ULOS} and do away with that kludge; without it and without such extended macros that word would be set as \textLipsias{Av"ULOS} with a much larger space between the capital alpha and the marked capital upsilon.}. 
In other words, by inputting \verb"a>ut'os", as it is suggested in the previous section, the spirit marker and the acute accent inhibit the kerning mechanism with the previous letter. In most instances the lack of such kerning is hardly noticeable, but in others it strikes your attention.

For this reason two mechanisms are implemented in this package.
\begin{enumerate}\itemsep=0pt\parsep=0pt
\item  The extended accent macros created by Günter Milde are now the default setting in the Greek support for \babel, and
\item The character names have been defined with macros that access directly their own glyphs.
\end{enumerate}
The first solution has been described in the previous section and the extended accent  macros are shown in table~\ref{t:accenti}. At the expense of one slash, these macros create a chain of commands that deeply exploit the \LaTeXe\ kernel commands and allow to fetch directly each accented character; notice that there is no need to treat the \verb+|+ sign this way, because its postfixed position does not break the kerning/ligature mechanism. The spirit macros \verb+\>+ and \verb+\<+ work also with the consonant `r' to produce \textLipsias{\>r, \<r}. The extended accent macros work correctly also with capital letters (where, in all caps, accents should not be typeset, except the dieresis, while in normal initial capitalising the upper diacritics must be written at the left of the initial letter); for example\footnote{In the capitalised word the spirit and accent at the left of A imply a hiatus between the A and the \textDidot{u}; in the all caps word this hiatus is marked with the “dialytika” (dieresis) over the \textDidot{U}. This is an unusual example, but it  shows also why the \texttt{\string\uppercase} and the \texttt{\string\MakeUppercase} require some attention in polytonic and ancient Greek.} \textDidot{\>'Aulos, A\"ULOS} was typeset with \verb+\>'Aulos, A\"ULOS+. Notice also that the initial capital vowel has a spirit and possibly an accent, or is not preceded by anything, since the possible spirit with or without accent falls on the second element of a descending diphthong: \textDidot{a\>ut'os}, \textDidot{A\>ut'os}, and \textDidot{AUTOS}; \textDidot{e\>'inai}, \textDidot{E\>'inai}, \textDidot{EINAI}. If you really want to show an example of how accents should \emph{not} be used in all caps words, you can type \textDidot{A\>UT\A{O}S} but you have to use the macros of 
table~\ref{t:accentmacros} as such: \verb+A\>UT\A{O}S+. 

Another solution is available if the \emph{GlyphNames} option is specified to the \texttt{teubner} package: a set of macros has been defined such that it is possible to input the accented characters directly, without resorting to the ligature mechanism. Such macros have a common structure; they are formed with the letters that make up the complex glyph in a certain order, precisely every macro is made
up as follows
\begin{enumerate}\itemsep=0pt\parsep=0pt
\item the first character, obviously, is the backslash character \texttt{\char92}.
\item The next character is the name of the base character, one of the vowels a, e, h, i, o, u, w, or the consonant r, or one of the capitalised vowels I~or~U.
\item The next optional character is the code for dieresis, or smooth, or rough spirit, with one of the letters d, s, r.
\item The next character is the code for the circumflex, acute, or grave accent with one of the letters c, a, or g.
\item The last optional character indicates iota subscript with the presence of an i.
\item There are no glyph names for upper case letters, since they should never be marked with diacritics, except for the diaeresis over \textDidot{I} and~\textDidot{U} and for these glyphs adequate names are provided.
\end{enumerate}

This means that, for instance, \comando\asai\ stands for \textDidot{\asai}. For your convenience such macros are collected in table~\ref{t:compoundvowels}.
Of course one can always resort to the accent--vowel combination as exemplified at the end of the previous section; the above example \textDidot{\asai} may be obtained also with \verb+\>'a|+ or 
\verb+\As{a}|+.%
\footnote{Postfixed markings do not pose any problem with
kernings and ligatures; this is why the postfixed ligature for the iota
subscript may still be used also when the accent--vowel combinations are used.}

\begin{table}[tb]\thrule\let\c\comando\let\l\textLipsias
\def\V{\rule{0pt}{2.5ex}}\def\D{\rule[-1.2ex]{0pt}{2ex}}
\makebox[\textwidth]{%
\begin{tabular}{*6{lc}}
\c\aa   &\l\aa  &\c\ag  &\l\ag  &\c\ac  &\l\ac  &\c\ai  &\l\ai  &\c\ar  &\l\ar  &\c\as  &\l\as  \\
\c\asa  &\l\asa &\c\asg &\l\asg &\c\asc &\l\asc &\c\asi &\l\asi &\c\aai &\l\aai \\
\c\ara  &\l\ara &\c\arg &\l\arg &\c\arc &\l\arc &\c\ari &\l\ari &\c\agi &\l\agi &\c\aci &\l\aci \\
\c\arai &\l\arai&\c\argi&\l\argi&\c\arci&\l\arci&\c\asai&\l\asai&\c\asgi&\l\asgi&\c\asci&\l\asci\D\\
\hline
\c\ha   &\l\ha  &\c\hg  &\l\hg  &\c\hc  &\l\hc  &\c\hi  &\l\hi  &\c\hr  &\l\hr  &\c\hs  &\l\hs  \\
\c\hsa  &\l\hsa &\c\hsg &\l\hsg &\c\hsc &\l\hsc &\c\hsi &\l\hsi &\c\hai &\l\hai \\
\c\hra  &\l\hra &\c\hrg &\l\hrg &\c\hrc &\l\hrc &\c\hri &\l\hri &\c\hgi &\l\hgi &\c\hci &\l\hci \\
\c\hrai &\l\hrai&\c\hrgi&\l\hrgi&\c\hrci&\l\hrci&\c\hsai&\l\hsai&\c\hsgi&\l\hsgi&\c\hsci&\l\hsci\D\\
\hline
\c\wa   &\l\wa  &\c\wg  &\l\wg  &\c\wc  &\l\wc  &\c\wi  &\l\wi  &\c\wr  &\l\wr  &\c\ws  &\l\ws  \\
\c\wsa  &\l\wsa &\c\wsg &\l\wsg &\c\wsc &\l\wsc &\c\wsi &\l\wsi &\c\wai &\l\wai \\
\c\wra  &\l\wra &\c\wrg &\l\wrg &\c\wrc &\l\wrc &\c\wri &\l\wri &\c\wgi &\l\wgi &\c\wci &\l\wci \\
\c\wrai &\l\wrai&\c\wrgi&\l\wrgi&\c\wrci&\l\wrci&\c\wsai&\l\wsai&\c\wsgi&\l\wsgi&\c\wsci&\l\wsci\D\\
\hline
\c\ia   &\l\ia  &\c\ig  &\l\ig  &\c\ic  &\l\ic  &\c\ir  &\l\ir  &\c\is  &\l\is  \\
\c\isa  &\l\isa &\c\isg &\l\isg &\c\isc &\l\isc &\c\ida &\l\ida &\c\idg &\l\idg \\
\c\ira  &\l\ira &\c\irg &\l\irg &\c\irc &\l\irc &\c\idc &\l\idc &\c\id  &\l\id  &\c\Id  &\l\Id  \\
\hline
\c\ua   &\l\ua  &\c\ug  &\l\ug  &\c\uc  &\l\uc  &\c\ur  &\l\ur  &\c\us  &\l\us  \\
\c\usa  &\l\usa &\c\usg &\l\usg &\c\usc &\l\usc &\c\uda &\l\uda &\c\udg &\l\udg \\
\c\ura  &\l\ura &\c\urg &\l\urg &\c\urc &\l\urc &\c\udc &\l\udc &\c\ud  &\l\ud &\c\Ud   &\l\Ud \\
\hline
\c\ea   &\l\ea  &\c\eg  &\l\eg  &\c\er  &\l\er  &\c\es  &\l\es  \\
\c\esa  &\l\esa &\c\esg &\l\esg &\c\era &\l\era &\c\erg &\l\erg \\
\hline
\c\oa   &\l\oa  &\c{\`o}\ap{1,3}  &\l{\`o}  &\c{\<o}\ap{2,3}&\l{\<o}  &\c\os  &\l\os  & \V\\
\c\osa  &\l\osa &\c\osg &\l\osg &\c\ora &\l\ora &\c\org &\l\org \\
\hline
\c\rs   &\l\rs  &\c\rr  &\l\rr  &\D       \\
\hline
\end{tabular}%
}
\caption{Glyph name macros}\label{t:compoundvowels}
\bigskip
REMARKS\\[3pt]\footnotesize
\mbox{\quad}\ap{1}Some users remarked that with previous \pack{teubner} versions  \verb|\og| macro conflicted with the homonymous command available when typesetting French texts with the \texttt{french} option to \babel\ in force; for this reason it's necessary to use  \verb|\`o| in its place.\par
\mbox{\quad}\ap{2}As mentioned in the body of this text the  command
\comando{\or} is incompatible with the primitive \TeX\ command with the same name, therefore it is necessary to use  \verb|\<o| in its place.\par
\mbox{\quad}\ap{3}In both cases  macros are unnecessary, specifically for \verb|\<o|, since an initial accented letter has no preceding letter whith whom some kerning might be necessary; therefore at the start of a word use \verb|'o| or \verb|<o|, and use \verb|\'o| in a word internal position, should some kerning be necessary; actually the shape of the omicron seldom requires any kerning.
\bhrule
\end{table}

What I suggest is to typeset your paper with the regular accent
vowel ligatures and to substitute them in the final revision with
the extended accent macros or the glyph name macros only in those instances where the lack of kerning is disturbing. The extended accent macros should set forth less conflicts with other packages and should be the first choice when cleaning up the final revision. The glyph names will be kept in future releases of this package just for backwards compatibility. Up to now these glyph name macros are disabled by default but, if the option \emph{GlyphNames} is specified in the call statement to the \texttt{teubner.sty} package, they are turned on and become available.



\section{Upper case initials and capitalised text}
As we have seen in the previous sections the ligature and kerning mechanisms are strictly connected and the extended accent macros may be useful in solving certain situations. The suggestion at the end of the previous section holds true, but some remarks must be underlined in order to use at its best the coexistence of both methods for using the accented glyphs.

The remarks concern the use of capital letters and, to a certain extent also the caps and small caps font shape. As mentioned before in Greek typography words with capital initials starting with a vowel require their diacritics in front and at the left of the initial capital vowel (for example:\textLipsias{>Aristot'elhs}) provided the vowel is not the first element of a Greek diphthong; in the latter case the diacritics go on top of the diphthong lower case second element (for example: \textLipsias{A>isq'ulos}). On the opposite, all-caps words, such as in book titles or, with certain \LaTeX\ classes, in the headings, are typeset without any diacritic, except the diaeresis.

I have already shown the unusual example of the word \textLipsias{>'aulos} where the first two vowels do not form a diphthong, but a hiatus; and this is why the diacritics fall on top of the alpha, not on the second element of the diphthong, for the very reason that there is no diphthong. In order to stress this unusual situation the word is most often spelled as \textLipsias{>'a\"ulos}, where the dieresis is sort of redundant, because the diacritics over the alpha already denote the hiatus. Similarly \textLipsias{>a\"upn'ia} is spelled with the redundant dieresis, although the soft spirit on the alpha already marks the hiatus. When these words require a capital initial, they become \textLipsias{>'A\"ulos} and \textLipsias{>A\"upn'ia}. 

Notice that an initial single vowel may receive only a spirit marker with or without an accompanying accent, never a dieresis or a single accent\footnote{Of course we are talking of the ancient spelling, since this package \texttt{teubner} sets this spelling as the default one; nobody forbids to spell in monotonic even when the ancient spelling is assumed, but in some rare instances there might be some inconsistencies.}. I tried hard to set up all the ligatures and extended accent macros so as to do the right things when capitalising or uppercasing, and if you follow the suggestions given in this section you should not meet any particular inconvenience.

Assume these words have to go in an all-caps header that is made up (behind the scenes) by making use of the \verb|\MakeUppercase| command; they have to be spelled \textLipsias{A\"ULOS} and \textLipsias{A\"UPNIA}, even if in the sectioning command argument they were spelled in lowercase with all the necessary diacritics. 

This is where the ligature and/or the extended accent macros may show their different behaviour. With the \texttt{teubner} generated secondary \texttt{LGRaccents-glyphs\discretionary{}{.}{.}def} file, that contains all the extended accent macros, such macros do not disappear in uppercasing, i.e.\ in transforming the mixed case argument of the 
\verb|\MakeUppercase| command into an all-caps letter string. On the opposite the ligature sequences completely loose any reference to diacritics.


In other words we have:
\begin{flushleft}
1) \verb|\MakeUppercase{>'aulos}| yields \textLipsias{\MakeUppercase{>'aulos}};\\
2) \verb|\MakeUppercase{>'a"ulos}| yields \textLipsias{\MakeUppercase{>'a"ulos}};\\
3) \verb|\MakeUppercase{>'a\"ulos}| yields \textLipsias{\MakeUppercase{>'a\"ulos}};\\
4) \verb|\MakeUppercase{\>'a\"ulos}| yields \textLipsias{\MakeUppercase{\>'a\"ulos}};\\
5) \verb|\MakeUppercase{\>\'a"ulos}| yields \textLipsias{\MakeUppercase{\>\'a\"ulos}};\\
6) \verb*|\MakeUppercase{\asa\ud los}| yields \textLipsias{\MakeUppercase{\asa\ud los}}\\
7) \verb|\MakeUppercase{e>uzw"'ia}| yields \textLipsias{\MakeUppercase{e>uzw"'ia}};\\
8) \verb|\MakeUppercase{e>uzw'"ia}| yields \textLipsias{\MakeUppercase{e>uzw'"ia}};\\
9) \verb|\MakeUppercase{e>uzw\"\'ia}| yields \textLipsias{\MakeUppercase{e>uzw\"\'ia}};\\
10) \verb|\MakeUppercase{e>uzw\"ia}| yields \textLipsias{\MakeUppercase{e>uzw\"ia}}.%\\
\end{flushleft}
%
and it's easily seen that: 1) corresponds to a non redundant lowercase correct spelling but misses the required dieresis in upper case; 2) is correct but it relies on a special kludge on which it's better not to rely on, because in future releases of the fonts it may be eliminated; meanwhile it is usable; 3) is correct; 4) is correct because \verb|\MakeUppercase| does not act on macros, but \verb|\>'| \emph{is} a macro, precisely an accent macro, that acts on the following letter before uppercasing, and the uppercase of an alpha with diacritics si simply the capital alpha; 5) is correct, but relies in the same kludge as case 2, so the same warnings apply; case 6) displays the result of using the accented glyph names which are not subject to capitalisation since they are given by means of macros; therefore they can never be used for words that are possibly subject to all-caps transformation. 

Let's examine another case where the lower case word has both the dieresis and an accent over the same vowel: \textLipsias{e>uzw"'ia}. Case 7) shows the effect of uppercasing when using ligatures, and the result is not correct because the dieresis does not fall over the capital iota, but between the capital omega end the capital iota; 8) by simply inverting the sequence of the dieresis and the acute accent a correct result is obtained, but kerning problems might take place because of the invisible presence of the lower cased acute accent; notice that the same result would be obtained if instead of the dieresis-vowel ligature the extended dieresis macro had been used; case 9) displays the situation when both extended accent macros are used so that the result does not suffer of any kerning problem; finally, case 10) displays a correct upper case result but the lower case counterpart would miss the acute accent.

When using small caps or caps and small caps, in other words, when using the \verb|\scshape| declaration or the \verb|\textsc| command you should pay attention to other details; for example:
\begin{flushleft}
11)  \verb|\textDidot{\scshape >Arqim'hdhs}| yields \textDidot{\scshape>Arqim'hdhs}\\
12)  \verb|\textDidot{\scshape \>Arqim'hdhs}| yields \textDidot{\scshape\>Arqim'hdhs}\\
13)  \verb|\textDidot{\scshape \>Arqim\'hdhs}| yields \textDidot{\scshape\>Arqim\'hdhs}\\
14)  \verb|\textDidot{\scshape >Arqim\'hdhs}| yields \textDidot{\scshape>Arqim\'hdhs}\\
15)  \verb|\textDidot{\scshape \>Arqim\ha dhs}| yields \textDidot{\scshape\>Arqim\ha dhs}\\
16)  \verb|\textDidot{\scshape \>{v}Arqim\'hdhs}| yields \textDidot{\scshape\>{v}Arqim\'hdhs}\\
\end{flushleft}

%>>>>>>>>>

With caps and small caps a spirit, or a spirit-accent combination prefixed to a capital letter does not produce any mark; this is the usual modern Greek habit of avoiding accents with capital letters and the \verb|\scshape| specification to Greek fonts excludes all the ligatures and kernings with such signs; the same holds true also with small caps. The extended accent macros explicitly avoid any mark over capital letters, but even in front they are “silenced” by the very characteristics of the specific font shape. With small caps there is a similar situation, and  the extended accent macros produce the expected result. But accents are apparently used in French also when proper names are typeset in caps and small caps, contrary to Greek, so that in French the spelling of case 16 is correct; for the initial spirit, therefore it is necessary to resort to the invisible character obtained with the Latin letter `v', that is uppercased to itself and is suitable for supporting any diacritic mark at the proper height. Julien Browaeys pointed out this specific typesetting French tradition; I thank him very much for his feedback on this point.

Conclusion: when writing the input code for a sectioning command, the argument of which has to be transformed to upper case, use \verb|\"| (in the proper sequence with other accents) for the internal diereses, as in cases 3), 4), 8), and 9) above. With the extended accent macros there are really few problems; the remaining ones take place in complete uppercasing remain with French traditions in caps and small caps that can be easily handled through the use of the \texttt{v} special invisible glyph. If you want to avoid also this possibility, then either avoid classes that typeset their headings in all-caps, or use any external package that defines sectioning commands with two optional arguments, one for the heading contents and the other for the table of contents; or use equivalent tricks: for example the \texttt{memoir} class has a specific command \verb|\nouppercaseheads| that eliminates any heading uppercasing.

\section{Other Greek symbols}
Other Greek symbols may be obtained with ligatures or explicit
commands; table~\ref{t:gsymbols} contains such ligatures and
symbols; notice that some of these are specific additions
introduced with this module.

\begin{table}\thrule\let\c\comando\let\l\textDidot
\makebox[\textwidth]{%
\begin{tabular}{lclclc}
\c{'{}'}      &\l{''}     &\c{((}     &\l{((}     &\c{))}     &\l{))} \\
\c\GEodq    &\l\GEodq   &\c\GEcdq   &\l\GEcdq   &\c:        &\l:    \\
\c\GEoq     &\l\GEoq    &\c\GEcq    &\l\GEcq    &\c?        &\l?    \\
\c\ENodq    &\l\ENodq   &\c\ENcdq   &\l\ENcdq   &\c;        &\l;    \\
\c\stigma   &\l\stigma  &\c\varstigma&\l\varstigma&\c\Stigma&\l\Stigma\\
\c\coppa    &\l\coppa   &\c\koppa &\l\koppa     &\c\Koppa   &\l\Koppa\\
\c\sampi    &\l\sampi   &\c\Sampi   &\l\Sampi   &\c\permill &\l\permill\\
\c\textdigamma  &\l\textdigamma &\c\Digamma &\l\Digamma &\c\euro    &\l\euro\\
\c\f        &\l\f       &\c\F       &\l\F       &\c\shwa    &\l\shwa\\
\end{tabular}%
}
\caption{Greek and other symbols}\label{t:gsymbols}
\bhrule
\end{table}

Please notice the necessity of using the ligature \comando{'{}'} for producing the simple apostrophe, which, by the way, in Greek typography  must always be followed by a space. The single tick mark \comando'\ produces an acute accent, not an apostrophe, this is why it is necessary to use the double tick mark ligature.  Actually also a double quote mark \verb+"+ followed by a space produces an apostrophe followed by a space.

\section{Milesian and Attic numerals}

The Milesian numerals should not worry anybody, because they are seldom used as
isolated symbols; the \textsf{greek} or \textsf{greek.ancient} language option or the \textsf{ancient} attribute of the Greek language with the \babel\ package offer the commands \comando\greeknumeral\ and
\comando\Greeknumeral, that convert common Arabic positive numbers in the
Milesian counterparts within a Greek section of the document; but the 2020 version of the \file{greek.ldf} file has been modified in such a way that it performs very well with modern Greek, but is missing some features when the real ancient Greek has to be typeset. Since its first version \pack{teubner} had to redefine the above macros so as to use the ancient Milesian symbols \textqoppa, \textvarstigma, \textsampi, with the possibility of using \textdigamma\ in place of \textqoppa.

Therefore version~5.0 of \pack{teubner} redefines those two \comando\greeknumeral\ and \comando\Greeknumeral\ in order to restore the ancient features for the philologist benefit.\footnote{Notice, though, that this document is typeset with the \texttt{txfonts} and the Greek \texttt{txfontsb} fonts; the latter have only the raised digamma glyph which is not suited for “ancient” Greek numerals; in this display example we switched to the usual Latin Modern fonts.} 
\begin{quote}\fontfamily{lmr}\selectfont\obeylines
if you type \texttt{\comando\greeknumeral\{1996\}} you get %
\greeknumeral{1996}
if you type \texttt{\comando\Greeknumeral\{1996\}} you get %
\Greeknumeral{1996}
if you type \texttt{\comando\greeknumeral*{1996}} you get %
\greeknumeral*{1996}
if you type \texttt{\comando\Greeknumeral*{1996}} you get %
\Greeknumeral*{1996}
if you type \texttt{\comando\greeknumeral\{123456\}} you get %
\greeknumeral{123456}
This was typeset on \begin{otherlanguage}{greek}\grtoday, \end{otherlanguage} i.e. \the\day\ \ifcase\month\or
        January\or February\or March\or April\or May\or June\or
        July\or August\or September\or October\or November\or
        December\fi\ \the\year.
\end{quote}

The \textsf{teubner} package offers also the possibility of typesetting the Attic numerals, without the need of loading Apostolos Syropoulos' \textsf{athnum} package; the functionality is the same, although the code is different; in order to avoid clashes, the \textsf{teubner} command for transforming Arabic numerals into Attic ones is \comando\AtticNumeral. As for the original macro, the maximum value that can be transformed is 99\,999, while, of course, no vanishing or negative numbers can be transformed:
\begin{quote}\obeylines
if you type \texttt{\comando\AtticNumeral\{2015\}} you get \textLipsias{\AtticNumeral{2015}}
if you type \texttt{\comando\AtticNumeral\{1999\}} you get \textLipsias{\AtticNumeral{1999}}
if you type \texttt{\comando\AtticNumeral\{55555\}} you get \textLipsias{\AtticNumeral{55555}}
\end{quote}


\begin{table}[!p]\thrule
\let\c\comando
\resizebox*{\textwidth}{!}{\footnotesize
\begin{tabular}{cl|cl}
Example                 & Syntax                        &Example                & Syntax                        \\[3pt]
\l{Baq\'ulides}         & (declaration)                 &abcde                  & (declaration)                 \\
\l{Baq\'ulides}         & \c\textLipsias\ARG{text}      &\l{\lesp{abg}}         &\c\lesp\ARG{text}              \\
\g{Baq\'ulides}         &\c\textDidot\ARG{text}         &\LitNil                &\c\LitNil                      \\
text                    & \c\textlatin\ARG{text}        &\cap{g}                &\c{\cap}\ARG{letter}\ap{($a$)} \\
\l\frapar{Baq\'ulides}  & \c\frapar\ARG{text}           &\Coronis               &\c\Coronis                     \\
(                       & \c\lpar                       &\lmqi                  &\c\lmqi                        \\
)                       &\c\rpar                        &\rmqi                  &\c\rmqi                        \\
\qmark                  &\c\qmark                       &\l{\mqi{abg}}          &\c\mqi\ARG{text}               \\
\l{\Dots[3]}            &\c\Dots\opt{number}            &\lmqs                  &\c\lmqs                        \\
\l{\DOTS[3]}            &\c\DOTS\opt{number}            &\rmqs                  &\c\rmqs                        \\
\l{\Dashes[3]}          &\c\Dashes\opt{number}          &\l{\mqs{abg}}          &\c\mqs\ARG{text}               \\
\l{\DASHES[3]}          &\c\DASHES\opt{number}          &\l{\zeugma{abg}}       &\c\zeugma\ARG{text}            \\
\l\slzeugma{ba}         &\c\slzeugm\ARG{two letters}    &\l\rszeugma{ab}        &\c\rszeugma\ARG{two letters}   \\
\ap{foo}                &\c\ap\ARG{text}                &\l{\siniz{abg}}        &\c\siniz\ARG{text}             \\
\sinafia                &\c\sinafia                     &\paragr                &\c\paragr                      \\
\l{\:}                  &\c\:                           &\dparagr               &\c\dparagr                     \\
\l\;                    &\c\;                           &\FinisCarmen           &\c\FinisCarmen                 \\
\l\?                    &\c\?                           &\crux                  &\c\crux                        \\
\Lipsiakostext\mutpers  &\c\antilabe                    &\l\apici{abg}          &\c\apici\ARG{text}             \\
\|                      &\c\|                           &\apex                  &\c\apex                        \\
\dBar                   &\c\dBar                        &\responsio             &\c\responsio                   \\
\tBar                   &\c\tBar                        &\Int                   &\c\Int                         \\
\lbrk                   &\c\lbrk                        &\l{\star a}            &\c\star                        \\
\rbrk                   &\c\rbrk                        &\l{\dstar a}           &\c\dstar                       \\
\l{\ladd{abg}}          &\c\ladd\ARG{text}              &\l{\tstar a}           &\c\tstar                       \\
\l{\lladd{abg}}         &\c\lladd\ARG{text}             &\l{\|\|\quad\|\,\|}    &\c\,                           \\
\l{\Ladd{abg}}          &\c\Ladd\ARG{text}              &\l{\|\|\quad\|\!\|}    &\c\!                           \\
\l{\LLadd{abg}}         &\c\LLadd\ARG{text}             &\OSN{0123456789}       &\c\OSN\ARG{digits}             \\
\l{\nexus{abgdezh}}         &\c\nexus\ARG{text}             &\kclick                &\c\kclick\\
\Utie{AB}               &\c\Utie\ARG{2 letters}         &\l{a\h b}              &\c\h                           \\
\l{a\yod b}             &\c\yod                         &\l{a\shwa b}           &\c\shwa                        \\
\l{a\q b}               &\c\q                           &\l{A\F B}              &\c\F                           \\
\l{a\f b}               &\c\f                           &\semiv{i}              &\c\semiv\ARG{letter}           \\
\hv                    &\c\skewstack\ARG{base}\ARG{apex}&\md{e}                 &\c\md\ARG{letter}              \\
\Ud{e}                  &\c\Ud\ARG{letter}              &\mO{e}                 &\c\mO\ARG{letter}              \\
\UO{e}                  &\c\UO\ARG{letter}              &\Open{e}               &\c\Open\ARG{letter}            \\
\nasal{e}               &\c\nasal\ARG{letter}           &\cut{d}                &\c\cut\ARG{b\|d\|g}            \\
\dracma                 &\c\dracma                      &\denarius              &\c\denarius                    \\
\stater                 &\c\stater                      &\etos                  &\c\etos                        \\
\hemiobelion            &\c\hemiobelion                 &\tetartemorion         &\c\tetartemorion               \\
\splus                  &\c\splus                       &\stimes                &\c\stimes                  
\end{tabular}%
}
\bhrule\smallskip

{\footnotesize\mbox{\ap{($a$)}\enspace}Ansten M\o rch Klev spotted a conflict 
with the homonymous math command; this is now resolved. Thank you Ansten.\par}
\caption{Extended commands}\label{t:extcomandi}
\end{table}

\section{New commands}
This package introduces many new commands for typesetting Greek in a philological way. Most such commands are collected in table~\ref{t:extcomandi}.

A short remark on the command \comando\ap: this useful command inserts \emph{anything} as a superscript of anything else; it works both in text mode and in math mode\footnote{Numerical superscripts or apices do not require math mode; numerical footnote labels are automatically inserted by \LaTeX's \comando\footnote\ command; non numerical footnote labels are easily inserted with \LaTeX's \comando\footnotemark\ and \comando\footnotetext\ commands with their optional arguments.}. In particular while typesetting a philological text in different languages and with different alphabets, \comando\ap\ typesets the superscript with the current language and alphabet; if any change is required, the \comando\ap's argument can contain any language or alphabet specific declaration. You can typeset something such as \textLipsias{Baq\'ulides\,\ap{\textlatin{a}}} by switching language and alphabet as required; the specific declarations and the commands contained in table~\ref{t:extcomandi} come handy also in these cases.

Please, read the \file{teubner.pdf} file for what concerns command \cs{LLadd}; it accepts two \emph{optional} kerning values in order to adapt the internal spaces in accordance with the nature of the \meta{text} argument; with capital letters and/or sloping fonts, such kerning values may come handy for a perfect final result. Nevertheless a lot of attention was used to cope with such situations so as to avoid the above described corrections.
Its new syntax is the following:
\begin{quote}
\cs{LLadd}\oarg{left kerning}\marg{text}\oarg{right kerning}
\end{quote}
where both kerning values imply a \emph{reduction} of space if they are specified as positive values; therefore if you specify negative values, the space is increased. Using of font size units, such as \texttt{ex} or \texttt{em} is recommended, since xuch units assure the automatic correct value even when changing font size.


\section{Metrics}
Philological writings often require the description of metrics;
for this purpose a new font has been developed that contains most
of the frequent metric signs; the corresponding macros have been
defined so as to set the metric glyphs as if they were text; but,
most important, a new definition command has been introduced so
as to enable to declare new control sequences to represent
complete metric feet or even complete verse metrics.

The metric glyph names are collected in table~\ref{t:metrics},
while the declaration command is described hereafter.

The syntax for that definition command is similar to that of
\comando{\newcommand};
\[
\hbox{\comando\newmetrics\ARG{name}\ARG{definition}}
\]
where \Arg{name} is a control sequence name made up of letters
(as usual with \LaTeX) with the exception that it may start with
one of the digits 2, or 3, or 4. Of course the \Arg{definition}
must reflect the replication by 2, or 3, or 4 times; moreover if the
\Arg{name} starts with a digit, when it is used
\emph{it must be followed by a space}. Some examples follow:
\begin{quote}\let\c\comando\obeylines%
\c\newmetrics\brArg{\c\iam}\brArg{\c\barbrevis\c\longa\c\brevis\c\longa}
\c\newmetrics\brArg{\c{\2iam}}\brArg{\c\iam\c\iam}
\c\newmetrics\brArg{\c{\4MACRO}}\brArg{\c\longa\c\longa\c\longa\c\longa}
\end{quote}
The above definitions produce the following results (notice the space
before the colon):
\begin{quote}\let\c\comando\obeylines%
\verb*|\iam|: \iam
\verb*|\2iam |: \iam\iam
\verb*|\4MACRO |: \4MACRO \
\end{quote}
The definitions may contain also some symbols collected in
table~\ref{t:extcomandi}, such as \dBar, for example, and other
symbols from the other tables.

\afterpage{\clearpage
\begin{table}[!tb]\thrule\let\c\comando\small
\tabcolsep=4pt
\makebox[\textwidth]{%
\begin{tabular}{llll}
Command         & Metric symbol     & Command       & Metric symbol     \\[3pt]
\c\longa        & \longa            &\c\brevis      & \brevis           \\
\c\bbrevis      & \bbrevis          &\c\barbrevis   & \barbrevis        \\
\c\ubarbrevis   & \ubarbrevis       &\c\ubarbbrevis & \ubarbbrevis      \\
\c\ubarsbrevis  & \ubarsbrevis      &\c\coronainv   & \coronainv        \\
\c\corona       & \corona           & \c\ElemInd    & \ElemInd          \\
\c\catal        & \catal            &\c\ipercatal   & \ipercatal        \\
\c\anceps       & \anceps           &\c\banceps     & \banceps          \\
\c\ancepsdbrevis& \ancepsdbrevis    &\c\hiatus\ap{1}& \hiatus           \\
\c\iam\ap{2}    & \iam              &\c\chor        & \chor             \\
\c\enopl        & \enopl            &\c{\4MACRO}    & \4MACRO \         \\
\c\aeolchorsor  & \aeolchorsor      &\c\hexam       & \hexam            \\
\c{\2tr}        & \2tr \            &\c\pentam      & \pentam           \\
\c\ubrevislonga & \ubrevislonga     &\c\aeolicbii   & \aeolicbii        \\
\c\aeolicbiii   & \aeolicbiii       &\c\aeolicbiv   & \aeolicbiv\ap{3}  \\
\end{tabular}%
}
\caption{Metric symbols}\label{t:metrics}
\bigskip
\footnotesize 
REMARKS\\[3pt]
\ap{1}\quad A similar command \c\Hiatus\ produces the same
visible result as \c\hiatus, except for the fact that it does not
occupy horizontal space; it is useful in the definitions of full
verse metrics where a hiatus needs to be inserted between two
consecutive metric symbols; for example: \longa\Hiatus\longa\,.\\
\ap{2}\quad This module predefines some examples
of metric feet and complete verses.\\
\ap{3}\quad Sometimes it might be convenient to use a shortcut
for inserting the Aeolic bases
by inputting \texttt{I} or
\texttt{II} or
\texttt{III}, while the \texttt{\string\metricsfont} declaration is in force, in order to get
\aeolicbii\ or \aeolicbiii\ or \aeolicbiv.
\bhrule
\end{table}}

Another important metric command is the following:
\[
\hbox{\comando\metricstack\ARG{base}\ARG{superscript}}
\]
which is meant for superimposing some superscript (generally a
number) over some metric symbol, which may be a single symbol or a
metric foot, such as~\smash{\metricstack{\longa\brevis\brevis}{48}}; since 
the superscript gets printed in math mode, the
superscript hiatus \hiatus\ may be obtained with \comando\Hiatus\
when it falls between two metric symbols, but must be well
described as a math roman element when it is superscripted over
something else; similarly any other superscript which is not a
math symbol must be suitably set as a math roman object. 

The environment for setting metric sequences grouped with braces
is described in the next section, since it is generally used within
the composition of verses.

\section{Poetry environments}
In order to set poetry it is always possible to use the
standard \LaTeX\ \texttt{verse} environment; nevertheless such
simple environment is not suited for philological purposes, except
perhaps for very short citations. This module contains
three new environments with various levels of complexity. Due to
their relative complexity an example will be given for each one
with both the input code and the corresponding result. All three
environments require that any language change be declared before
their opening statement, otherwise the language change lasts only
to the end of the verse. It's worth noting that if you feel uncomfortable with Italian names for verses, you can use the Latin aliases, \texttt{versus}, \texttt{Versus}, and \texttt{VERSUS}.

\begin{description}
\item[\texttt{versi}] This environment does not actually set each
verse on a separate line; it rather resembles an in-line list; it
resorts to a command \comando\verso\ that inserts a small vertical
separator with a progressive number over it. Both the environment
opening and the command \comando\verso\ accept arguments
according to the following syntax:
    \begin{quote}\let\c\comando\obeylines
    \c\begin\brArg{versi}\ARG{label}
    \Arg{verses}
    \c\end\brArg{versi}
    \null\
    \c\verso\opt{number}
    \end{quote}
where \Arg{label} is a short text (let's say not more than 15
characters) indicating for example the poem title and the stanza
number; the whole set of verses will be typeset with a left margin
wide enough to contain \Arg{label}; the optional argument
\Arg{number} indicates the starting value for the verse
enumeration; the default value is~1, but if it is specified, it is
required only with the first occurrence of \comando\verso\ or when
the enumeration is restarted. In this environment the standard
\LaTeX\ command \comando{\\} behaves normally as in regular text.
\begin{verbatim}
\begin{versi}{Meropis fr. 3}
>'enj'' <o m`en e\ladd{>isplh} \verso[68] j`un
Mer'opwn k'ien. <h \ladd{d`e dia} \verso pr`o\\
a>iqem\hc i sj~htos \ladd{>'elassen.}
\verso <`o d'' >ex'equt''; o>u
g`ar \ladd{<omo~iai}\\
\ladd{>a} \verso j'anatai jnhta~isi bol\ladd{a`i kat`a}
\verso ga~ian >'asin.\\
prh\lladd{m}n\ladd{~hs d\dots} \verso thse. m'elas d`e
perie.\ladd{\dots}\verso rw
\end{versi}
\end{verbatim}

\begin{Lipsiakostext}
\begin{versus}{\textlatin{Meropis fr. 3}}
>'enj'' <o m`en e\ladd{>isplh} \verso[68] j`un
Mer'opwn k'ien. <h \ladd{d`e dia} \verso pr`o\\
a>iqem\hc i sj~htos \ladd{>'elassen.} \verso <`o d'' >ex'equt'';
o>u
g`ar \ladd{<omo~iai}\\
\ladd{>a} \verso j'anatai jnhta~isi bol\ladd{a`i kat`a}
\verso ga~ian >'asin.\\
prh\lladd{m}n\ladd{~hs d\dots} \verso thse. m'elas d`e perie.\ladd{\dots}\verso
rw
\end{versus}
\end{Lipsiakostext}

Since each verse in this environment is not on a single line, unless it's deliberately specified, this environment may be used also for prose whose sentences are numbered as, for example, biblical versicles:
\begin{Lipsiakostext}
\begin{versi}{\textlatin{Mt: 6,8--6,13}}
\verso[8]\textDidot{m`h o>~un <omoiwj~hte a\>uto~is; o~>iden g`ar <o pat`hr <um~wn <~wn qre'ian >'eqete pr`o to~u <um~as a>it~hsai a\>ut'on. \verso o<'utws o>~un prose'uqesje
<ume~is;}\\[1ex]
P'ater <hm~wn <o >en to~is o>urano~is;\\
<agisj'htw t`o >'onom'a sou; \\
\verso
>elj'etw <h basile'ia sou;\\
genhj'htw t`o j'elhm'a sou,\\
<ws >en o>uran~w| ka`i >ep`i g~hs;\\
\verso
t`on >'arton <hm~wn t`on >epio'usion d`os <hm~in s'hmeron;\\
\verso 
ka`i >'afes <hm~in t`a >ofeil'hmata <hm~wn,\\
<ws ka`i <hme~is >af'hramen to~is >ofeil'etais <hm~wn;\\
\verso
ka`i m`h e>isen'egkhis <hm~as e>is peirasm'on,\\
>all`a <r~usai <hm~as >ap`o to~u ponhro~u,\\
>'oti sou >'estin <h basile'ia ka`i <h d'unamis ka`i <h d'oxa\\
>'eis to`us a>i~wnas; >am'hn.
\end{versi}
\end{Lipsiakostext}

\item[\texttt{Versi}] This environment is very similar to the
standard \LaTeX\ environment \texttt{verse}; the difference is
that \texttt{Versi} automatically enumerates the verses
(displaying only verse numbers that are multiples of 5) with a
number in the left margin. The syntax is as follows:
    \begin{quote}\let\c\comando\obeylines
    \c\begin\brArg{Versi}\opt{number}
    \Arg{verses}
    \c\end\brArg{Versi}
    \end{quote}
where \Arg{number} is the starting value of the verse enumeration;
of course each verse is separated from the next one with the usual
command \comando{\\}, which has been redefined so that it just
divides the verses and provides to the possible display of the
verse number; it accepts the optional information that the
standard \LaTeX\ command usually accepts, both the asterisk and
the vertical space amount.



\begin{verbatim}
\begin{Versi}[45]
ta; pr'osje qeir~wn b'ian\\
\paragr de\ladd{'i}xomen;
t`a d" >epi'onta da\ladd{'imo}n srine~i.---\\
t'os" e>'ipen >ar'etaikmos <'hrws;\\
t{\rbrk}'afon d`e naub'atai\\
f{\rbrk}wt`os <uper'afanon\\[1ex]
j{\rbrk}'arsos; <Al'iou te gambr~wi q'olwsen >~htor
\end{Versi}
\end{verbatim}

\begin{Lipsiakostext}
\begin{Versi}[45]
ta; pr'osje qeir~wn b'ian\\
\paragr de\ladd{'i}xomen;
t`a d" >epi'onta da\ladd{'imo}n srine~i.---\\
t'os" e>'ipen >ar'etaikmos <'hrws;\\
t{\rbrk}'afon d`e naub'atai\\
f{\rbrk}wt`os <uper'afanon\\[1ex]
j{\rbrk}'arsos; <Al'iou te gambr~wi q'olwsen >~htor
\end{Versi}
\end{Lipsiakostext} 

\item[\texttt{VERSI}] This third poetry environment behaves
similarly to \texttt{Versi} but it displays a double verse
enumeration in the left margin. The principal verse enumeration
is displayed when the value is a multiple of 5; the second
enumeration, just to the left of the verses, may be turned on and
off; when the secondary enumeration is on, the verses are flush
left, while when it is off the verses are suitably indented. The
turning on and off of the secondary enumeration is achieved by
means of the commands \comando\SubVerso\ and \comando\NoSubVerso;
the syntax is as follows:
    \begin{quote}\let\c\comando\obeylines
    \c\begin\brArg{VERSI}\opt{outer number}
    \Arg{verses}
    \c\end\brArg{VERSI}
    \null{}
    \c\SubVerso\opt{inner number}
    \c\NoSubVerso{}
    \end{quote}
where \Arg{outer number} is the starting value of the primary
verse enumeration, while \Arg{inner number} is the starting value
of the secondary enumeration. The commands \comando\SubVerso\ and
\comando\NoSubVerso\ must be input at the very beginning of the
verse they should be applicable to. The command \comando{\\} behaves
as in \LaTeX, and accepts the usual optional arguments.

With the environments \texttt{Versi} and \texttt{VERSI}, when typesetting in two
column format, you have the possibility of specifying \comando\BreakVersitrue\
(and of course \comando\BreakVersifalse) for allowing (or disallowing) line
breaks of verses; broken verses are continued on the next line with a
generous indentation so as to recognize them as belonging to the same verse;
the verse counter is not incremented when breaking verses across lines.


\begin{verbatim}
\begin{VERSI}[40]
k'elomai pol\ua stonon\\
\SubVerso[18]
>er'uken <'ub\apex rin; o>u g`ar >`an j'eloi-\\
\NoSubVerso
m'' >'ambroton >erann`on >Ao\lbrk ~us\\
\SubVerso
>ide~in f'aos, >epe'i tin'' >h"ij\siniz{'e\lbrk w}n\\
s`u dam'aseias >a'ekon-\\
\NoSubVerso
ta; pr'osje qeir~wn b'ian\\
\SubVerso
\paragr de\ladd{'i}xomen;
t`a d'' >epi'onta da\ladd{'imw}n krine~i.\GEcdq\\
\SubVerso[1]
t'os'' e>~ipen >ar'etaiqmos <'hrws;\\
t\rbrk 'afon d`e na\ua batai\\
f\rbrk wt`os <uper'afanon\\
j\rbrk 'arsos;\Dots[4]
\end{VERSI}
\end{verbatim}

\Lipsiakostext
\begin{VERSI}[40]
k'elomai pol\ua stonon\\
\SubVerso[18]
>er'uken <'ub\apex rin; o>u g`ar >`an j'eloi-\\
\NoSubVerso
m'' >'ambroton >erann`on >Ao\lbrk ~us\\
\SubVerso
>ide~in f'aos, >epe'i tin'' >h"ij\siniz{'e\lbrk w}n\\
s`u dam'aseias >a'ekon-\\
\NoSubVerso
ta; pr'osje qeir~wn b'ian\\
\SubVerso
\coronis de\ladd{'i}xomen; t`a d'' >epi'onta da\ladd{'imw}n krine~i.\GEcdq\\
\SubVerso[1]
t'os'' e>~ipen >ar'etaiqmos <'hrws;\\
t\rbrk 'afon d`e na\ua batai\\
f\rbrk wt`os <uper'afanon\\
j\rbrk 'arsos;\Dots[4]
\end{VERSI}%}
\NoLipsiakostext

\item[\texttt{bracedmetrics}] This environment is different
from the preceding ones, although it always deals with verses.
Its purpose is to set the verse metric lines grouped with a right
brace, so as to show the variants of a certain metric scheme.

In order to align the metric variants and in order to place the
right brace in the proper place it is necessary to fix  specific
lengths in terms of a unit that is compatible with the metric
symbols; therefore the syntax of such spacing command and of the
environment itself is the following
    \begin{quote}\let\c\comando\obeylines
    \c\begin\brArg{bracedmetrics}\ARG{length}
    \Arg{metric lines}
    \c\end\brArg{bracedmetrics}
    \null
    \c\verseskip\ARG{number}
    \end{quote}
where \Arg{number} specifies the number of metric symbols the
\comando\verseskip\ should be equivalent to. Approximately the
\comando\verseskip\ will be as long as a sequence of \Arg{number}
long syllables; the \Arg{length} specified as the width of the
environment should equal the longest metric line contained in the
block, and should be specified by means of the
\comando\verseskip\ command with its argument; but since the
metric symbols are not all of the same length, it is wise to
count the symbols of the longest metric line and to add a couple
of units; after producing the first draft it is possible to review
the number specified as the argument of \comando\verseskip. Of
course the same \comando\verseskip\ command may be used to align
the various fragments of metric lines within the environment.
Examine the following example of input code:
\begin{verbatim}
\begin{verse}
    \brevis\svert\longa\brevis\brevis\longa
        \brevis\brevis\longa\svert\longa\\
    \begin{bracedmetrics}{\verseskip{13}}
        \Hfill \brevis\svert\longa\brevis\longa
            \svert\longa\\
        \longa\brevis\brevis\longa\brevis\brevis
            \zeugma{\longa\svert\longa}\\
        \Hfill\longa\brevis\longa\svert\longa
            \verseskip{2}\\
        \Hfill\longa\brevis\longa\dBar
    \end{bracedmetrics}\\
    \begin{bracedmetrics}{\longa\brevis\brevis\longa
        \brevis\brevis\longa\svert\longa}
        \longa\brevis\brevis\longa\brevis\brevis
            \longa\svert\longa\\
        \longa\brevis\brevis\longa\brevis\brevis\longa
    \end{bracedmetrics}\\
\verseskip{7}\brevis\brevis\longa\svert\ubarbrevis\tBar
\end{verse}
\end{verbatim}
which produces:
\begin{verse}
    \brevis\svert\longa\brevis\brevis\longa
        \brevis\brevis\longa\svert\longa\\
    \begin{bracedmetrics}{\verseskip{13}}
        \Hfill \brevis\svert\longa\brevis\longa\svert\longa\\
        \longa\brevis\brevis\longa\brevis\brevis
            \zeugma{\longa\svert\longa}\\
        \Hfill\longa\brevis\longa\svert\longa\verseskip{2}\\
        \Hfill\longa\brevis\longa\dBar
    \end{bracedmetrics}\\
    \begin{bracedmetrics}{\longa\brevis\brevis\longa
        \brevis\brevis\longa\svert\longa}
        \longa\brevis\brevis\longa\brevis\brevis\longa\svert\longa\\
        \longa\brevis\brevis\longa\brevis\brevis\longa
    \end{bracedmetrics}\\
\verseskip{7}\brevis\brevis\longa\svert\ubarbrevis\tBar
\end{verse}%


\end{description}

\section*{Acknowledgements}
This project was  initially carried on with the help of Mr Paolo Ciacchi when he was writing his ancient Greek philology master thesis. After he got his master's degree at the University of Trieste, I continued by myself, but I remain really indebted to Mr~Ciacchi.

\end{document}
