%%http://www.ctan.org/tex-archive/macros/latex/contrib/arsclassica
\documentclass[10pt,a4paper,twoside,openright,titlepage,fleqn,%
               headinclude,,footinclude,BCOR5mm,%
               numbers=noenddot,cleardoublepage=empty,%
               tablecaptionabove]{scrreprt}
               
\usepackage[italian,ngerman]{babel}
%\usepackage[applemac]{inputenc}
\usepackage[utf8]{inputenc}
%\usepackage[latin1]{inputenc}
\usepackage[T1]{fontenc}
\usepackage[final]{microtype}
\usepackage{amsmath,amssymb,amsthm}
\usepackage{varioref}
\usepackage[style=philosophy-modern,hyperref,backref,square,natbib,ibidtracker=false]{biblatex}
\usepackage[tight,italian]{minitoc}
\usepackage{wrapfig}
\usepackage{chngpage}
\usepackage{calc}
\usepackage{mflogo}
\usepackage{caption,listings,graphicx,subfig}
\usepackage{multicol}
\usepackage{makeidx}
\usepackage{xspace}
\usepackage{mparhack}
\usepackage{fixltx2e}
\usepackage{relsize}
\usepackage{lipsum}
\usepackage[eulerchapternumbers,subfig,beramono,eulermath,pdfspacing,listings]{classicthesis}
\usepackage{guit}
\usepackage{arsclassica}
%\input{arsclassica-preamble}
\newcommand{\myName}{Lorenzo Pantieri\xspace}
\newcommand{\myTitle}{Customizing ClassicThesis with the ArsClassica package\xspace}
\newcommand{\mySubTitle}{User Manual\xspace}
\newcommand{\myLocation}{Cesena\xspace}
\newcommand{\myGroup}{Italian \TeX{} and \LaTeX{} User Group\xspace}
\newcommand{\myUrl}{\url{http://www.guit.sssup.it/}\xspace}
\newcommand{\myTime}{2011, June\xspace}

% ********************************************************************
% hyperref
% ******************************************************************** 
\hypersetup{%
    colorlinks=true, linktocpage=true, pdfstartpage=1, pdfstartview=FitV,%
    breaklinks=true, pdfpagemode=UseNone, pageanchor=true, pdfpagemode=UseOutlines,%
    plainpages=false, bookmarksnumbered, bookmarksopen=true, bookmarksopenlevel=1,%
    hypertexnames=true, pdfhighlight=/O,%
    urlcolor=webbrown, linkcolor=RoyalBlue, citecolor=RoyalBlue, pagecolor=RoyalBlue,
    breaklinks,%
% uncomment the following line if you want to have black links (e.g., for printing)
% urlcolor=Black, linkcolor=Black, citecolor=Black, pagecolor=Black,%
    pdftitle={\myTitle},%
    pdfauthor={\textcopyright\ \myName},%
    pdfsubject={},%
    pdfkeywords={},%
    pdfcreator={pdfLaTeX},%
    pdfproducer={LaTeX con hyperref e ClassicThesis}%
}

\hypersetup{citecolor=webgreen}
\hypersetup{hyperfootnotes=false,pdfpagelabels}

\newcommand{\mail}[1]{\href{mailto:#1}{\texttt{#1}}}

% ********************************************************************
% caption
% ********************************************************************
\captionsetup{format=hang,font=small}
\captionsetup[table]{skip=\medskipamount} 

% ********************************************************************
% makeidx, multicol
% ********************************************************************
\let\orgtheindex\theindex
\let\orgendtheindex\endtheindex
\def\theindex{%
	\def\twocolumn{\begin{multicols}{2}}%
	\def\onecolumn{}%
	\clearpage
	\orgtheindex
}
\def\endtheindex{%
	\end{multicols}%
	\orgendtheindex
}

\makeindex

% ********************************************************************
% babel
% ********************************************************************


% ********************************************************************
% listings
% ********************************************************************

\definecolor{lightergray}{gray}{0.99}

\lstset{language=[LaTeX]Tex,
    keywordstyle=\color{RoyalBlue},
    basicstyle=\normalfont\ttfamily,
    commentstyle=\color{Emerald}\ttfamily,
    stringstyle=\rmfamily,
    numbers=none,
    numberstyle=\scriptsize,
    stepnumber=5,
    numbersep=8pt,
    showstringspaces=false,
    breaklines=true,
    frameround=ftff,
    frame=lines,
    backgroundcolor=\color{lightergray}
} 

\lstset{	morekeywords=%
        {RequirePackage,newboolean,DeclareOption,setboolean,%
        ProcessOptions,PackageError,ifthenelse,boolean,%
        chapterNumber,sodef,textls,allcapsspacing,%
        MakeTextLowercase,orgtheindex,endtheindex,%
        @ifpackageloaded,undefined,sfdefault,%
        DeclareRobustCommand,spacedallcaps,%
        microtypesetup,MakeTextUppercase,lowsmallcapsspacing,%
        lowsmallcapsspacing,spacedlowsmallcaps,
        spacedlowsmallcaps,lehead,headmark,color,%
        headfont,partname,thepart,titleformat,part,
        titlerule,chapter,thechapter,thesection,%
        subsection,thesubsection,thesubsubsection,%
        paragraph,theparagraph,descriptionlabel,titlespacing,%
        graffito,lineskiplimit,finalhyphendemerits,%
        colorbox,captionsetup,labelitemi,%
        myincludegraphics,hypersetup,setlength,%
        definecolor,lsstyle,textssc,subsubsection,%
        graffito@setup,includegraphics,ifdefined,%
        myTitle,textcopyright,myName,lstset,lstnewenvironment,%
        setkeys,lst@BeginAlsoWriteFile,contentsname,%
        toc@heading,@ppljLaTeX,z@,check@mathfonts,%
        sf@size,ptctitle,mtctitle,stctitle,lst@intname,%
        @empty,math@fontsfalse,@ppljscTeX,@iwonaTeX,%
        @iwonascLaTeX,@ctTeX,tw@,ct@sc,@ctTeX,f@family,%
        f@shape,ct@sc,ctLaTeX,ctLaTeXe,@twoe,@sctwoe,%
        texorpdfstring,m@th,ctTeX,@mkboth,ProvidesPackage,%
        theindex,PackageInfo,PackageWarningNoLine,%
        mtifont,mtcindent,@iwonaLaTeX,@ppljTeX,@iwonascTeX,%
        rohead,orgendtheindex,@ppljscLaTeX,%
        @ifclassloaded,toc@headingbkORrp,backreftwosep,%
        backrefalt,backreflastsep,areaset,pnumfont,%
        arsincludegraphics,ExecuteOptions,PackageWarning,%
        MessageBreak,ars@@includegraphics,ifcld@backref,rofoot,formatchapter},
        commentstyle=\color{Emerald}\ttfamily,%
        frame=lines}

\lstset{basicstyle=\normalfont\ttfamily}
\lstset{flexiblecolumns=true}
\lstset{moredelim={[is][\normalfont\itshape]{/*}{*/}}}

\DeclareRobustCommand*{\pacchetto}[1]{{\normalfont\ttfamily#1}%
\index{Pacchetto!#1@\texttt{#1}}%
\index{#1@\texttt{#1}}}

\DeclareRobustCommand*{\classicthesis}{ClassicThesis}


\DeclareRobustCommand*{\bibtex}{\textsc{Bib}\TeX%
\index{bibtex@\textsc{Bib}\protect\TeX}%
}

\DeclareRobustCommand*{\amseuler}{\protect\AmS{} Euler%
\index{AmS Euler@\protect\AmS~Euler}%
\index{Font!AmS Euler@\protect\AmS~Euler}}

\lstset{basicstyle=\normalfont\ttfamily}
\lstset{flexiblecolumns=false}
\lstset{moredelim={[is][\ttfamily]{!?}{?!}}} 
\lstset{escapeinside={£*}{*£}}
\lstset{firstnumber=last}

\lstnewenvironment{code}% 
{\setkeys{lst}{columns=fullflexible,keepspaces=true}%
\lstset{basicstyle=\small\ttfamily}%
}{}

\lstset{extendedchars} 
\lstnewenvironment{sidebyside}{% 
    \global\let\lst@intname\@empty 
    \setbox\z@=\hbox\bgroup 
    \setkeys{lst}{columns=fullflexible,% 
    linewidth=0.45\linewidth,keepspaces=true,%
    breaklines=true,% 
    breakindent=0pt,%
    boxpos=t,%
    basicstyle=\small\ttfamily
}% 
    \lst@BeginAlsoWriteFile{\jobname.tmp}% 
}{% 
    \lst@EndWriteFile\egroup 
        \begin{center}% 
            \begin{minipage}{0.45\linewidth}% 
                \hbox to\linewidth{\box\z@\hss} 
            \end{minipage}% 
            \qquad 
            \begin{minipage}{0.45\linewidth}%
            \setkeys{lst}{frame=none}% 
                \leavevmode \catcode`\^^M=5\relax 
                \small\input{\jobname.tmp}% 
            \end{minipage}% 
        \end{center}% 
} 


\newcommand{\omissis}{[\dots\negthinspace]}

\graphicspath{{Graphics/}}

\newcommand{\meta}[1]{$\langle${\normalfont\itshape#1}$\rangle$}
\lstset{escapeinside={£*}{*£}}

\lstset{moredelim={[is][\ttfamily]{!?}{?!}}}

\DeclareRobustCommand*{\miktex}{MiK\TeX%
\index{miktex@MiK\protect\TeX}%
}

\DeclareRobustCommand*{\metafont}{\MF%
\index{METAFONT@\protect\MF}%
}

\DeclareRobustCommand*{\metapost}{\MP%
\index{METAPOST@\protect\MP}%
}

\DeclareRobustCommand*{\texlive}{\TeX{}~Live%
\index{texlive@\protect\TeX{}~Live}%
}

%%%%%%%%%%%%
% biblatex %
%%%%%%%%%%%%

\bibliography{Bibliography}

\renewcommand{\nameyeardelim}{, }

\defbibheading{bibliography}{%
\cleardoublepage
\manualmark
\phantomsection
\addcontentsline{toc}{chapter}{\tocEntry{\bibname}}
\chapter*{\bibname\markboth{\spacedlowsmallcaps{\bibname}}
{\spacedlowsmallcaps{\bibname}}}}     

  \DeclareCiteCommand{\citeyearpar}[\mkbibparens] 
  {\boolfalse{citetracker}% 
   \boolfalse{pagetracker}% 
   \usebibmacro{prenote}} 
  {\printtext[bibhyperref]{\printfield{year}}} 
  {\multicitedelim} 
  {\usebibmacro{postnote}} 

\makeatletter 
  \DeclareCiteCommand{\citetalias} 
  {\usebibmacro{prenote}} 
  {\usebibmacro{citeindex}% 
   \bibhyperref{\@citealias{\thefield{entrykey}}}} 
  {\multicitedelim} 
  {\usebibmacro{postnote}} 
\makeatother 


%%%%%%%%%%%%%%%%%%
% other commands %
%%%%%%%%%%%%%%%%%%

\newcommand{\ita}[1]{% 
  \begin{otherlanguage*}{italian}#1\end{otherlanguage*}}
  
\DeclareRobustCommand*{\pkgname}[1]{{\normalfont\sffamily#1}%
\index{Package!#1@\textsf{#1}}%
\index{#1@\textsf{#1}}}

\DeclareRobustCommand*{\envname}[1]{{\normalfont\ttfamily#1}%
\index{Environment!#1@\texttt{#1}}%
\index{#1@\texttt{#1}}}

\DeclareRobustCommand*{\optname}[1]{{\normalfont\ttfamily#1}%
\index{Option!#1@\texttt{#1}}%
\index{#1@\texttt{#1}}}

\DeclareRobustCommand*{\clsname}[1]{{\normalfont\sffamily#1}%
\index{Class!#1@\textsf{#1}}%
\index{#1@\textsf{#1}}}

\DeclareRobustCommand*{\cmdname}[1]{\mbox{\lstinline!\\#1!}%
\index{#1@\texttt{\hspace*{-1.2ex}\textbackslash#1}}}

\DeclareRobustCommand*{\classicthesis}{Classic\-Thesis}

\DeclareRobustCommand*{\arsclassica}{{\normalfont\sffamily ArsClassica}}

\DeclareRobustCommand*{\miktex}{MiK\TeX%
\index{miktex@MiK\protect\TeX}}

\DeclareRobustCommand*{\texlive}{\TeX{}~Live%
\index{texlive@\protect\TeX{}~Live}}

\begin{document}
\pagenumbering{roman}
\pagestyle{plain}
%******************************************************************
% Frontmatter
%******************************************************************
%\input{FrontBackMatter/Titlepage}
\begin{titlepage}
\pdfbookmark{Titlepage}{Titlepage}
\changetext{}{}{}{((\paperwidth  - \textwidth) / 2) - \oddsidemargin - \hoffset - 1in}{}
\null\vfill
\begin{center}
\large
\sffamily

\bigskip

{\Large\spacedlowsmallcaps{\myName}} \\

\bigskip

{\huge\spacedlowsmallcaps{Anpassen von ClassicThesis mit dem ArsClassica-Paket} \\
}

\bigskip
    
\vspace{1cm}
    
{\large\spacedlowsmallcaps{Übersetzung: Iona Gessinger (FSU
Jena),\newline Juli 2011}} \\

\bigskip

\vspace{9cm}

\begin{tabular} {cc}
\parbox{0.3\textwidth}{\includegraphics[width=2.5cm]{Birds}}
&
\parbox{0.7\textwidth}{{\Large\spacedlowsmallcaps{Benutzerhandbuch}} \\ 
{\normalsize Italienische \TeX{}- und \LaTeX{}-Anwendergruppe \\
\myUrl \\
Juni 2011}}
\end{tabular}
\end{center}
\vfill
\end{titlepage}

%\input{FrontBackMatter/Titleback}
\thispagestyle{empty}

\hfill

\vfill

\noindent\myName:
\textit{Anpassen von ClassicThesis mit dem ArsClassica Paket,} Benutzerhandbuch,
\textcopyright\ Juni 2011.

\medskip
\noindent{\spacedlowsmallcaps{Website}}: \\
\url{http://www.lorenzopantieri.net/}

\medskip
\noindent{\spacedlowsmallcaps{E-Mail}}: \\
\mail{lorenzo.pantieri@iperbole.bologna.it}

\vspace{1cm}
\hrule
\bigskip

\noindent Die Titelseite zeigt einen Stich von Maurits Cornelis Escher
namens \emph{Flächenfüllung mit Vögeln} (das Bild stammt von der Website \url{http://www.mcescher.com/}).
\clearpage
%\input{FrontBackMatter/Abstract+Sommario}
\pdfbookmark{Abstract}{Abstract}
\begingroup
\let\clearpage\relax
\let\cleardoublepage\relax
\let\cleardoublepage\relax

\chapter*{Inhaltsangabe}
Das Paket verändert einige typographische Aspekte des \classicthesis{} Stils von Andr\'e Miede. Es 
befähigt zur Nachahmung der graphischen Gestaltung des Handbuchs \emph{Die Kunst mit \LaTeX{} zu 
schreiben} (auf  Italienisch)~\citep{pantieri:art}. Den Tipp für die
ursprüngliche Veränderung von \classicthesis{} habe ich von Daniel Gottschlag erhalten. Das Paket
wurde für die italienische \TeX{} und \LaTeX{}-Anwendergruppe geschrieben 
(\GuIT, \url{http://www.guit.sssup.it/}). 

\vfill

\selectlanguage{italian}
\pdfbookmark[1]{Sommario}{Sommario}
\chapter*{Sommario}
Il pacchetto modifica alcuni aspetti tipografici dello stile \classicthesis, di Andr\'e Miede. 
Permette di riprodurre la veste grafica della guida \emph{L'arte di scrivere con 
\LaTeX}~\citep{pantieri:art}. Lo spunto per l'originale rielaborazione di \classicthesis{} mi 
\`e stato offerto da Daniel Gottschlag. Il pacchetto \`e stato scritto per il Gruppo Utilizzatori
Italiani di \TeX{} e \LaTeX{} (\GuIT, \url{http://www.guit.sssup.it/}).

\selectlanguage{ngerman}

\endgroup			

\vfill

%\input{FrontBackMatter/Acknowledgements}
\pdfbookmark{Acknowledgements}{Acknowledgements}

\begingroup
\let\clearpage\relax
\let\cleardoublepage\relax
\let\cleardoublepage\relax

\chapter*{Danksagung}
Ich möchte zunächst den Mitgliedern der italienischen \TeX{}- und
\LaTeX{}-Anwendergruppe (\GuIT*, \url{http://www.guit.sssup.it/}),
insbesondere\\ Prof. Enrico Gregorio und Andrea Tonelli, für ihre außerordentlich wertvolle Hilfe beim 
Schreiben dieser Arbeit, die detaillierten Erklärungen, die Geduld und
Präzision ihrer Vorschläge, die gelieferten Lösungen, sowie für ihre Kompetenz und Freundlichkeit 
danken.
Mein Dank gilt ebenfalls all den Menschen, die mit mir im \GuIT*~Forum diskutiert, mir viele wertvolle 
Beobachtungen mitgeteilt und gute Ratschläge gegeben haben.

Schließlich danke ich Andr\'e Miede für seinen wunderbaren ClassicThesis
Stil und Daniel Gottschlag, der mir den Tipp für diese ursprüngliche Nachbearbeitung des Stils gegeben 
hat.
\endgroup
\pagestyle{scrheadings} 
\clearpage
%\input{FrontBackMatter/Contents}
\phantomsection
\pdfbookmark{\contentsname}{tableofcontents}
\setcounter{tocdepth}{2}
\begingroup 
    \let\clearpage\relax
    \let\cleardoublepage\relax
    \let\cleardoublepage\relax

\dominitoc\tableofcontents
\endgroup
\markboth{\spacedlowsmallcaps{\contentsname}}{\spacedlowsmallcaps{\contentsname}} 

\begingroup 
    \let\clearpage\relax
    \let\cleardoublepage\relax
    \let\cleardoublepage\relax
\endgroup
\cleardoublepage
%******************************************************************
% Mainmatter
%******************************************************************
\pagenumbering{arabic}
%\input{Chapters/Fundamentals}
%************************************************
\chapter{Grundlagen}
\label{chp:fundamentals}
\minitoc\mtcskip
%************************************************

Dieses Kapitel führt die (wirklich einfachen) Grundbegriffe des Pakets \arsclassica{} ein und 
präsentiert seine wesentlichen Ideen und Besonderheiten.


\section{Einleitung}

Das \graffito{Die Kunst mit \LaTeX~zu schreiben} \arsclassica{} Paket verändert einige typographische
Aspekte des \classicthesis{} Stils von Andr\'e
Miede~\citep{miede:classicthesis,pantieri:classicthesis}. Es befähigt zur
Nachahmung der graphischen Gestaltung meines Handbuchs \emph{Die Kunst mit
\LaTeX{}zu schreiben} ~\citep{pantieri:art} (auf Italienisch) und dieses Dokuments. Den Tipp für die 
ursprüngliche Nachbearbeitung von \classicthesis{} habe ich von Daniel Gottschlag erhalten. 


\section{Anwendung des Pakets}

Das Paket ist darauf ausgelegt mit einer \emph{vollständig} installierten Version von \miktex{} oder
\texlive{} ausgeführt zu werden und verwendet frei erhältlichen Zeichensatz.

Die Installation von \arsclassica{} ist sehr einfach. Laden Sie die Datei \texttt{arsclassica.zip}  
\href{http://www.ctan.org/tex-archi\-ve/macros/latex/contrib/arsclassica/}
(\url{http://www.ctan.org/tex-archive/macros/latex/contrib/arsclassica/})
von {\texttt{CTAN}} herunter ; entpacken Sie die Datei und 
installieren Sie  \texttt{arsclassica.sty} auf die übliche Weise.  

Das \graffito{ArsClassica benötigt die Version 3.0 von ClassicThesis} Paket arbeitet mit den 
\clsname{KOMA-Script}-Klassen (\clsname{scrreprt}, \clsname{scrbook} und \clsname{scrartcl}) und 
benötigt das \emph{auf die neueste Version (3.0) aktualisierte}
\pkgname{classicthesis}-Paket. Es muss nach letzterem geladen werden; und zwar auf folgende Weise:
\newpage
\begin{code}
\documentclass[(...)]{scrreprt} % or scrbook or scrartcl

\usepackage[(...)]{classicthesis}
\usepackage{arsclassica}

\begin{document}
...
\end{document}
\end{code}

Dieses Dokument wurde zum Beispiel mit folgendem Quelltext erstellt: 
\begin{code}
\documentclass[10pt,a4paper,twoside,openright,titlepage,fleqn,%
               headinclude,,footinclude,BCOR5mm,%
               numbers=noenddot,cleardoublepage=empty,%
               tablecaptionabove]{scrreprt}

\usepackage{(...)}
\usepackage[eulerchapternumbers,subfig,beramono,%
            eulermath,pdfspacing]{classicthesis}
\usepackage{arsclassica}

\begin{document}
...
\end{document}
\end{code}

Es ist ratsam, aber nicht zwingend erforderlich, die Optionen \optname{beramono}, 
\optname{eulerchapternumbers} und \optname{eulermath} zusammen mit \arsclassica{} zu verwenden.


\section{Optionen des Pakets}

Für Referenzen \graffito{Referenzen mit \texttt{backref} in Italienisch} stehen fünf Optionen
zur Verfügung:
\optname{english} (default), \optname{french}, \optname{german},
\optname{spanish} und \optname{italian},
\begin{code}
\usepackage[(...{language})]{arsclassica}
\end{code}
die die bibliographischen Angaben endgültig haben können (generiert durch
das Paket \pkgname{backref}, automatisch geladen aus
\pkgname{classicthesis-ldpkg}); auch die Beschriftungen der Minikapitel
(wenn das Paket \pkgname{minitoc} geladen wurde) bzw. in Englisch, Französisch,
Deutsch, Spanisch und Deutsch.

\section{Der Stil}

Der \graffito{Die Unterschiede zwischen ArsClassica und ClassicThesis} mit \arsclassica{} erzielte 
typographische Stil unterscheidet sich in den folgenden Punkten von \classicthesis{}:
\begin{itemize}
\item Einsatz des Iwona\index{Iwona} Zeichensatzes von Janusz M.~Nowacki
für die Titel der Dokumentabschnitte (chapters, sections, subsections,
sub-subsections, paragraphs, subparagraphs), für die Markierungszeichen von
description-Listen, für die Überschriften und die Markierungszeichen der Beschriftungen 
(\classicthesis{} verwendet keine serifenlosen Zeichensätze); 
\item maßgefertigte Kapitelnummern;
\item halb-transparente Überschriften; die Überschriften sind durch eine kleine Linie von den 
Seitenzahlen getrennt;
\item Beschriftungen mit Markierungszeichen in Fettschrift (\classicthesis{} verwendet keine 
Fettschrift-Zeichensätze);
\item itemize-Listen mit halb-transparenten Markierungen;
\item "`double square"' Text, für Dokumente, die im A4-Format geschrieben wurden und den 
Palatino-Zeichensatz verwenden.
\end{itemize}

Das \graffito{Es ist empfehlenswert, die Einstellungen von ArsClassica nicht zu verändern.} 
\arsclassica{}-Paket ist dafür konzipiert, dem Benutzer einen verwendungsbereiten typographischen 
Stil zur Verfügung zu stellen: Daher hat es keine Ladeoptionen und ist \emph{in keiner Weise} 
konfigurierbar 
oder anpassbar. Wenn Sie die Voreinstellungen verändern, laufen Sie Gefahr, das Gleichgewicht des 
Stils zu zerstören. Deswegen ist es \emph{sehr empfehlenswert}, die Einstellungen unverändert zu 
lassen.

Eines der Prinzipien von \LaTeX{} ist es, dem Benutzer die Möglichkeit zu geben, sich voll auf die 
Struktur und den Inhalt seines Dokuments zu konzentrieren, ohne sich um typographische Fragen kümmern 
zu müssen. Dieser Aspekt sollte immer berücksichtigt werden: Wenn ein Benutzer einen Stil verwendet, 
der von anderen geschrieben wurde, akzeptiert er damit alle typographischen Einstellungen, die der 
Autor des Stils für ihn ausgewählt hat und muss sich somit nicht selbst mit Typographie beschäftigen, 
um das Layout seiner Publikationen festzulegen. Das trifft auch für \arsclassica{} zu: Wenn Sie die 
Einstellungen 
verändern, wirken Sie dieser Philosophie entgegen und müssen sich demzufolge gründlich mit Typographie
beschäftigen, um akzeptable Ergebnisse zu erzielen.

Daher ist der mit \arsclassica{} erhaltene Stil \emph{weder} konfigurierbar \emph{noch} anpassbar. Der 
typographische Stil ist etwas sehr Persönliches: Wenn Sie das Paket mögen
und Ihnen die Tatsache gefällt, dass Sie sich nicht mit dem Problem der
Stildefinition beschäftigen müssen, werden Sie \arsclassica{} zufrieden nutzen; wenn Sie 
allerdings andere Ansprüche haben oder mit dem Layout des Pakets nicht zufrieden sind, dann sollten 
Sie andere Klassen und Pakete ausprobieren oder sogar einen eigenen Stil entwickeln.

\begin{figure}[h!]
\centering
\subfloat[][Bild ohne Hintergrund.]{\includegraphics[width=0.45\columnwidth]{GuITlogo}}\qquad
\subfloat[][Bild mit Hintergrund.\label{fig:background}]
{\arsincludegraphics[width=0.45\columnwidth]{GuITlogo}}
\caption[Graphics with coloured background.]{Graphiken mit farbigem Hintergrund.}
\end{figure}

\section{Neue Befehle}

\subsection{Besondere Logos}
Das \graffito{Die \textbackslash ctLaTeX, \textbackslash ctLaTeXe und \textbackslash ctTeX Befehle} 
Paket bietet die Befehle \cmdname{ctLaTeX}, \cmdname{ctLaTeXe} und \cmdname{ctTeX}, mit denen die 
Logos \LaTeX{}, \LaTeXe{} und \TeX{} jeweils korrekt in Iwona\index{Iwona} darstellbar sind.


\subsection{Graphiken mit farbigem Hintergrund}

Der \graffito{Der \textbackslash arsincludegraphics Befehl}
\cmdname{arsincludegraphics}-Befehl erlaubt es, Graphiken mit "`Alice
Blue"'-farbenem Hintergrund, wie in Abbildung~\vref{fig:background}, darzustellen; er funktioniert 
als \cmdname{includepraphics}-Befehl. Das gelingt natürlich nur mit Graphiken, die einen transparenten
Hintergrund besitzen, so wie \textsc{pdf}-Dokumente und einige
\textsc{png}-Dokumente.


\section{Beispiele}

Lorem\graffito{Anmerkung: Der Inhalt dieses Kapitels ist ein Blindtext. Es handelt sich um keine echte
Sprache.} ipsum dolor sit amet, consectetuer adipiscing elit. Ut purus elit, vestibulum ut, placerat ac,
adipiscing vitae, felis. Curabitur dictum gravida mauris. Nam arcu libero, nonummy eget, consectetuer id, vulputate a, magna. Donec vehicula augue eu neque.


\begin{figure}[h!]
\centering
\subfloat[Asia personas duo.]
{\includegraphics[width=.45\columnwidth]{Example_1}} \quad
\subfloat[Pan ma signo.]
{\label{fig:example-b}%
\includegraphics[width=.45\columnwidth]{Example_2}} \\
\subfloat[Methodicamente o uno.]
{\includegraphics[width=.45\columnwidth]{Example_3}} \quad
\subfloat[Titulo debitas.]
{\includegraphics[width=.45\columnwidth]{Example_4}}
\caption[Tu duo titulo debitas latente]{Tu duo titulo debitas
latente.}\label{fig:example}
\end{figure}

\subsection*{Eine Subsection}
\lipsum[2]

\subsubsection*{Eine Subsubsection}
\lipsum[3]

\begin{table}[h!]
\caption[Lorem ipsum dolor]{Lorem ipsum dolor sit amet, consectetuer adipiscing elit. 
Curabitur dictum gravida mauris.}
\centering
\begin{tabular}{cc}
\toprule
$p$ & $\lnot p$ \\ 
\midrule
V   & F \\ 
F   & V \\
\bottomrule 
\end{tabular}
\end{table}

\paragraph{Ein Paragraph} Lorem ipsum dolor sit amet, consectetuer adipiscing elit. Ut purus elit, 
vestibulum ut, placerat ac, adipiscing vitae, felis. Curabitur dictum gravida mauris. Nam arcu libero,
nonummy eget, consectetuer id, vulputate a, magna.

\bigskip

\lipsum[2]

\begin{description}
\item[Mane] Lorem\graffito{Die Markierungen der description-Listen sind in Iwona\index{Iwona} 
gesetzt.} ipsum dolor sit amet, consectetuer adipiscing elit. 
\item[Tekel] Ut purus elit, vestibulum ut, placerat ac, adipiscing vitae, felis. Curabitur dictum gravida mauris.
\item[Fares] Nam arcu libero, nonummy eget, consectetuer 
id, vulputate a, magna.
\end{description}

\lipsum[1]

%\input{Chapters/Code}
%************************************************
\chapter{Der Quelltext}
\label{chp:code}
%************************************************
 
\lstset{numbers=left,
    numberstyle=\scriptsize,
    stepnumber=1,
    numbersep=8pt
}    

Ankündigung des Pakets und Anforderung der notwendigen Pakete:
\begin{lstlisting}[firstnumber=1]
\NeedsTeXFormat{LaTeX2e}
\ProvidesPackage{arsclassica}[2011/06/29 v3.0 Customizing ClassicThesis (LP)]
\RequirePackage{classicthesis}
\end{lstlisting}

Gebrauch der Iwona\index{Iwona} als Fettschrift:
\begin{lstlisting}
\renewcommand{\sfdefault}{iwona}
\end{lstlisting}

Angepasste Kapitelnummern:
\begin{lstlisting}
\let\chapterNumber\undefined 
\ifthenelse{\boolean{@eulerchapternumbers}}
{\newfont{\chapterNumber}{eurb10 scaled 5000}}% 
{\newfont{\chapterNumber}{pplr9d scaled 5000}}
\end{lstlisting}


Kleine Beschriftungen in Fettschrift:
\begin{lstlisting}
\ifthenelse{\boolean{@minionprospacing}}%
{%
  \DeclareRobustCommand{\spacedallcaps}[1]{\sffamily%
  \textssc{\MakeTextUppercase{#1}}}%
  \DeclareRobustCommand{\spacedlowsmallcaps}[1]%
  {\sffamily\textssc{\MakeTextLowercase{#1}}}%
}{%
  \ifthenelse{\boolean{@pdfspacing}}%
  {%
    \microtypesetup{expansion=false}%
    \DeclareRobustCommand{\spacedallcaps}[1]%
    {\sffamily\textls[160]{\MakeTextUppercase{#1}}}%
    \DeclareRobustCommand{\spacedlowsmallcaps}[1]%
    {\sffamily\textls[80]{\scshape\MakeTextLowercase{#1}}}%
  }{%
    \RequirePackage{soul} 
    \sodef\allcapsspacing{\sffamily\upshape}%
    {0.15em}{0.65em}{0.6em}%
    \sodef\lowsmallcapsspacing{\sffamily\scshape}%
    {0.075em}{0.5em}{0.6em}%   
    \DeclareRobustCommand{\spacedallcaps}[1]%
    {\MakeTextUppercase{\allcapsspacing{#1}}}%   
	\DeclareRobustCommand{\spacedlowsmallcaps}[1]%
	{\MakeTextLowercase{\textsc%
	   {\lowsmallcapsspacing{#1}}}}%
  }%
}
\end{lstlisting}

Halb-transparente Überschriften und Seitenzahlen bei der Iwona.
\begin{lstlisting}
\renewcommand{\sectionmark}[1]{\markright{\textsc%
{\MakeTextLowercase{\thesection}} \spacedlowsmallcaps{#1}}}
\lehead{\mbox{\llap{\small\thepage\kern1em\color{halfgray}%
\vline}%
\color{halfgray}\hspace{0.5em}\headmark\hfil}} 
\rohead{\mbox{\hfil{\color{halfgray}%
\headmark\hspace{0.5em}}%
\rlap{\small{\color{halfgray}\vline}\kern1em\thepage}}}
\renewcommand{\headfont}{\normalfont\sffamily}
\renewcommand{\pnumfont}{\small\sffamily}
\end{lstlisting}

Einsatz des Iwona\index{Iwona}-Zeichensatzes für die Titel der Dokumentabschnitte (chapters, sections, 
subsections, sub-subsections, paragraphs, subparagraphs) und für die Markierungszeichen von 
description-Listen:
\begin{lstlisting}
\RequirePackage{titlesec}
		% parts
		\ifthenelse{\boolean{@parts}}%
		{%
    \titleformat{\part}[display]
        {\normalfont\centering\large}%
        {\thispagestyle{empty}\partname~\thepart}{1em}%
        {\color{Maroon}\spacedallcaps}
    }{\relax}
    % chapters
    \ifthenelse{\boolean{@linedheaders}}%
    {%
    \titleformat{\chapter}[display]%             
        {\relax}{\raggedleft{\color{halfgray}%
        \chapterNumber\thechapter} \\ }{0pt}%
        {\titlerule\vspace*{.9\baselineskip}\raggedright%
        \spacedallcaps}%
        [\normalsize\vspace*{.8\baselineskip}\titlerule]%
    }{%  
    \titleformat{\chapter}[block]%
        {\normalfont\Large\sffamily}%
        {{\color{halfgray}\chapterNumber\thechapter%
        \hspace{10pt}\vline}  }{10pt}%
        {\spacedallcaps}}
    % sections
    \titleformat{\section} 
    	  {\normalfont\Large\sffamily}{\textsc%
	  {\MakeTextLowercase{\thesection}}}%
         {1em}{\spacedlowsmallcaps}
    % subsections
    \titleformat{\subsection}
        {\normalfont\sffamily}{\textsc{\MakeTextLowercase%
        {\thesubsection}}}{1em}{\normalsize}
    % subsubsections
    \titleformat{\subsubsection}
        {\normalfont\sffamily\itshape}{\textsc%
        {\MakeTextLowercase{\thesubsubsection}}}%
        {1em}{\normalsize\itshape}        
    % paragraphs
    \titleformat{\paragraph}[runin]
        {\normalfont\normalsize\sffamily}{\textsc%
        {\MakeTextLowercase{\theparagraph}}}%
        {0pt}{\spacedlowsmallcaps}    
    % descriptionlabels
    \renewcommand{\descriptionlabel}[1]{\hspace*{\labelsep}%
    \bfseries\spacedlowsmallcaps{#1}}
    \titlespacing*{\chapter}{0pt}{1\baselineskip}%
    {2\baselineskip}
    \titlespacing*{\section}{0pt}{2\baselineskip}%
    {.8\baselineskip}[\marginparsep]
    \titlespacing*{\subsection}{0pt}{1.5\baselineskip}%
    {.8\baselineskip}[\marginparsep]
    \titlespacing*{\paragraph}{0pt}{1\baselineskip}%
    {1\baselineskip}

    \newcommand\formatchapter[1]{% 
    \vbox to \ht\strutbox{ 
    \setbox0=\hbox{\chapterNumber\thechapter\hspace{10pt}\vline\ } 
    \advance\hsize-\wd0 \advance\hsize-10pt\raggedright 
    \spacedallcaps{#1}\vss}} 
    \titleformat{\chapter}[block] 
       {\normalfont\Large\sffamily} 
       {\textcolor{halfgray}{\chapterNumber\thechapter} 
       \hspace{10pt}\vline\ }{10pt} 
    {\formatchapter}    

    \rofoot[\mbox{\makebox[0pt][l]{\kern1em\thepage}}]{}
\end{lstlisting}


Itemize-Listen mit halb-transparenten Markierungen:
\begin{lstlisting}
\renewcommand\labelitemi{\color{halfgray}$\bullet$} 
\end{lstlisting}

Einstellung der Beschriftungen:
\begin{lstlisting}
\captionsetup{format=hang,font=small,labelfont={sf,bf}}
\captionsetup[table]{skip=\medskipamount}
\end{lstlisting}

"`Double square"' Text (wie in Version 2.3 von ClassicThesis) für
Dokumente, die im A4-Format geschrieben wurden und den Palatino Zeichensatz verwenden.
\begin{lstlisting}
\ifthenelse{\boolean{@a5paper}}%
{\relax}%
{% A4
  \ifthenelse{\boolean{@minionpro}}%
  {\relax}%
  {% Palatino or other
    \PackageInfo{classicthesis}{A4 paper, Palatino or other}
    \areaset[5mm]{312pt}{699pt}
    % 624 + 33 head + 42 head \the\footskip
    \setlength{\marginparwidth}{7em}%
    \setlength{\marginparsep}{2em}%
  }%
}
\end{lstlisting}

Der \textbackslash arsincludegraphics-Befehl ermöglicht die Verwendung von Graphiken mit einem 
Hintergrund der Farbe "`Alice Blue"'. In den vorhergehenden Versionen des
ArsClassica-Pakets nannte sich der \textbackslash arsincludepraphics-Befehl 
\textbackslash myincludegraphics: Der alte Name wird (vorerst) noch
unterstützt, ist aber ungern gesehen:
\begin{lstlisting}
\definecolor{aliceblue}{RGB}{240,248,255}

\let\ars@@includegraphics\includegraphics
\newcommand{\arsincludegraphics}[2][]{%
  \begingroup\setlength{\fboxsep}{0pt}%
   \colorbox{aliceblue}{\ars@@includegraphics[#1]{#2}}%
  \endgroup}
\end{lstlisting}


Einstellungen für \pkgname{hyperref}:

\begin{lstlisting}
\hypersetup{%
    colorlinks=true, linktocpage=true, pdfstartpage=1, 
    pdfstartview=FitV, breaklinks=true, pdfpagemode=UseNone, 
    pageanchor=true, pdfpagemode=UseOutlines,%
    plainpages=false, bookmarksnumbered,
    bookmarksopen=true,%
    bookmarksopenlevel=1,%
    hypertexnames=true, pdfhighlight=/O,%
    urlcolor=webbrown, linkcolor=RoyalBlue, 
    citecolor=webgreen,%
    hyperfootnotes=false,pdfpagelabels,
    pdfsubject={},%
    pdfkeywords={},%
    pdfcreator={pdfLaTeX},%
    pdfproducer={LaTeX con hyperref e ClassicThesis}%
}
\end{lstlisting}



Einige kleine Anpassungen für den Fall, dass das \pkgname{minitoc}-Paket
verwendet wird:
\begin{lstlisting}
\@ifpackageloaded{minitoc} 
{% 
      \MakeLowercase{\gdef\noexpand\ptctitle{\ptctitle}} 
      \MakeLowercase{\gdef\noexpand\mtctitle{\mtctitle}} 
      \MakeLowercase{\gdef\noexpand\stctitle{\stctitle}} 
      \setlength{\mtcindent}{0pt} 
      \renewcommand{\mtifont}{\normalsize\sffamily 
         \scshape\lsstyle} 
} 
{}
\end{lstlisting}

Festlegung der
{\ttfamily\textbackslash\color{RoyalBlue}{ctLaTeX}},
{\ttfamily\textbackslash\color{RoyalBlue}{ctLaTeXe}} und
{\ttfamily\textbackslash\color{RoyalBlue}{ctTeX}} 
Befehle, mit denen die Logos \LaTeX, \LaTeXe{} und \TeX{} jeweils korrekt
in der Iwona darstellbar sind:\index{Iwona}
\begin{lstlisting}
\def\@ppljLaTeX{{\upshape 
   \sbox\z@{\check@mathfonts\fontsize\sf@size\z@%
   \math@fontsfalse\selectfont A}% 
   \sbox\tw@ T% 
   L\kern-.55\wd\z@ 
   \vbox to\ht\tw@{\copy\z@\vss}% 
   \kern-.25\wd0 
        \@ctTeX}} 
\def\@ppljTeX{{\upshape T\kern -.08em \lower .3ex\hbox{E}%
\kern -.08em X}} 

\def\@ppljscLaTeX{{\upshape\scshape 
   \sbox\z@{\check@mathfonts\fontsize\sf@size\z@%
   \math@fontsfalse\selectfont a}% 
   \sbox\tw@ t% 
   l\kern-.6\wd\z@ 
   \vbox to\ht\tw@{\copy\z@\vss}% 
   \kern-.25\wd0 
        \@ctTeX}} 
\def\@ppljscTeX{{\upshape\scshape t\kern -.085em
\lower .25ex\hbox{e}\kern -.085em x}} 

\def\@iwonaLaTeX{{\upshape 
   \sbox\z@{\check@mathfonts\fontsize\sf@size\z@%
   \math@fontsfalse\selectfont A}% 
   \sbox\tw@ T% 
   L\kern-.5\wd\z@ 
   \vbox to\ht\tw@{\copy\z@\vss}% 
   \kern-.2\wd0 
        \@ctTeX}} 
\def\@iwonaTeX{{\upshape T\kern -.12em \lower .3ex\hbox{E}%
   \kern -.12em X}} 

\def\@iwonascLaTeX{{\upshape\scshape 
   \sbox\z@{\check@mathfonts\fontsize\sf@size\z@%
   \math@fontsfalse%
   \selectfont a}% 
   \sbox\tw@ t% 
   l\kern-.5\wd\z@ 
   \vbox to\ht\tw@{\copy\z@\vss}% 
   \kern-.2\wd0 
        \@ctTeX}} 
\def\@iwonascTeX{{\upshape\scshape t\kern -.1em
   \lower .25ex\hbox{e}\kern -.1em x}} 

\def\ct@sc{sc} 
\def\@ctTeX{\csname @\f@family\ifx\f@shape\ct@sc sc%
\fi TeX\endcsname} 

\DeclareRobustCommand\ctLaTeX{% 
  \texorpdfstring{\textls[1]{\csname @\f@family\ifx%
  \f@shape\ct@sc sc\fi LaTeX\endcsname}}{LaTeX}} 
\DeclareRobustCommand\ctLaTeXe{% 
  \texorpdfstring{\textls[1]{\ctLaTeX\csname @\ifx%
  \f@shape\ct@sc sc\fi twoe\endcsname}}{LaTeX2e}} 

\def\@twoe{\kern.1em$\m@th2_{\textstyle\varepsilon}$} 
\def\@sctwoe{\kern.15em$\m@th{\scriptscriptstyle2}%
_\varepsilon$}

\DeclareRobustCommand\ctTeX{% 
  \texorpdfstring{\textls[1]{\@ctTeX}}{TeX}}

\def\toc@headingbkORrp{% 
  \def\toc@heading{% 
    \chapter*{\contentsname}% 
    \@mkboth{\spacedlowsmallcaps{\contentsname}} 
      {\spacedlowsmallcaps{\contentsname}}}} 
\@ifclassloaded{scrreprt}{\toc@headingbkORrp}{} 
\@ifclassloaded{scrbook}{\toc@headingbkORrp}{}
\end{lstlisting}
% *****************************************************************
% Backmatter
%******************************************************************
\clearpage
% \input{FrontBackMatter/Bibliography}
\nocite{*}
\printbibliography

\clearpage
%\input{FrontBackMatter/Index}
\manualmark
\markboth{\spacedlowsmallcaps{\indexname}}{\spacedlowsmallcaps{\indexname}}
\phantomsection
\begingroup 
    \let\clearpage\relax
    \let\cleardoublepage\relax
    \let\cleardoublepage\relax
\pagestyle{scrheadings} 
\addcontentsline{toc}{chapter}{\tocEntry{\indexname}}
\printindex
\endgroup
\end{document}
