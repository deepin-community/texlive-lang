%% xelatex 
\listfiles
%LaTeX Project Public License (LPPL).
\documentclass{ltxdockit}[2011/03/25]
\usepackage[a4paper,margin=0.75in]{geometry}
\usepackage{btxdockit}
\usepackage{fontspec}
%\setmainfont{LinLibertine_R.otf}
\defaultfontfeatures{Ligatures=TeX,Numbers=OldStyle}
\setmainfont[
BoldFont          = LinLibertine_RZ.otf,
ItalicFont        = LinLibertine_RI.otf,
BoldItalicFont    = LinLibertine_RZI.otf,
Ligatures=TeX]{LinLibertine_R.otf}
\usepackage{selinput}

\usepackage[american, german,ngerman]{babel} 

\usepackage[strict]{csquotes}
\usepackage{tabularx}
\usepackage{longtable}
\usepackage{booktabs}
\usepackage{shortvrb}
\usepackage{pifont}
\usepackage{microtype}
\usepackage{typearea}
\usepackage{mdframed}
\usepackage{enumitem}
\usepackage{calc}
\areaset[current]{370pt}{700pt}
\lstset{
    basicstyle=\ttfamily,
    keepspaces=true,
    upquote=true,
    frame=single,
    breaklines=true,
    postbreak=\raisebox{0ex}[0ex][0ex]{\ensuremath{\color{red}\hookrightarrow\space}}
}
\KOMAoptions{numbers=noenddot}
\addtokomafont{title}{\sffamily}
\addtokomafont{paragraph}{\spotcolor}
\addtokomafont{section}{\spotcolor}
\addtokomafont{subsection}{\spotcolor}
\addtokomafont{subsubsection}{\spotcolor}
\addtokomafont{descriptionlabel}{\spotcolor}
\setkomafont{caption}{\bfseries\sffamily\spotcolor}
\setkomafont{captionlabel}{\bfseries\sffamily\spotcolor}
\pretocmd{\cmd}{\sloppy}{}{}
\pretocmd{\bibfield}{\sloppy}{}{}
\pretocmd{\bibtype}{\sloppy}{}{}
\makeatletter
\patchcmd{\paragraph}
  {3.25ex \@plus1ex \@minus.2ex}{-3.25ex\@plus -1ex \@minus -.2ex}{}{}
\patchcmd{\paragraph}{-1em}{1.5ex \@plus .2ex}{}{}
\makeatother

\newrobustcmd*{\CSdelim}{%
  \textcolor{spot}{\margnotefont Kontext sensitiv}}
\newrobustcmd*{\CSdelimMark}{%
  \leavevmode\marginpar{\CSdelim}}

\MakeAutoQuote{«}{»}
\MakeAutoQuote*{<}{>}
\MakeShortVerb{\|}

\newcommand*{\biber}{Biber\xspace}
\newcommand*{\biblatex}{BibLaTeX\xspace}
\newcommand*{\biblatexml}{BibLaTeXML\xspace}
\newcommand*{\ris}{RIS\xspace}
\newcommand*{\biblatexhome}{http://sourceforge.net/projects/biblatex/}
\newcommand*{\biblatexctan}{http://mirror.ctan.org/macros/latex/contrib/biblatex/}

\usepackage{listings}

\MakeShortVerb{\|}

\newcommand*{\tabrefe}{\refs{Tabelle}{Tabelle}}

\titlepage{% 
title={\begin{tabular}{c} Das \sty{biblatex} Paket \\
Das Benutzerhandbuch
\end{tabular}},
subtitle={Programmierbares Bibliografieren und Zitieren}, 
url={\biblatexhome}, 
author={\centering 
Philip Kime, Moritz Wemheuer\\ Philipp Lehman
\footnote{In die deutsche Sprache übersetzt von Christine Römer
(Christine\_Roemer@t-online.de) und den
Studierenden Stephan Wedekind, Lisa Glaser, Maximilian Walter, Franziska Schade,
Luise Modersohn, Silvia Müller, Susanne Berghoff, Maiko Sauerteig, Lukas
Klimmasch, Peer Aramillo Irizar. Anpassung an die Version 3.15 durch Christine 
Römer. License: \LaTeX{} Project Public License (LPPL).
Aufgrund der häufigen Änderungen in letzter Zeit
wurde nur die Übersetzung des Nutzerhandbuchs aktualisiert. Zum Nachschlagen im
Autorenhandbuch muss in die englische Originalversion geschaut werden.}
},
email={},
revision={3.15b},
date={28. May 2021}}

\hypersetup{% 
pdftitle={The \biblatex Package}, pdfsubject={Programmable
Bibliographies and Citations}, pdfauthor={Philipp Lehman, Philip Kime}, pdfkeywords={tex,
latex, bibtex, bibliography, references, citation}}

% tables

\newcolumntype{H}{>{\sffamily\bfseries\spotcolor}l}
\newcolumntype{L}{>{\raggedright\let\\=\tabularnewline}p}
\newcolumntype{R}{>{\raggedleft\let\\=\tabularnewline}p}
\newcolumntype{C}{>{\centering\let\\=\tabularnewline}p}
\newcolumntype{V}{>{\raggedright\let\\=\tabularnewline\ttfamily}p}

\newcommand*{\sorttablesetup}{%
  \tablesetup
  \ttfamily
  \def\new{\makebox[1.25em][r]{\ensuremath\rightarrow}\,}%
  \def\alt{\par\makebox[1.25em][r]{\ensuremath\hookrightarrow}\,}%
  \def\note##1{\textrm{##1}}}

\newcommand{\tickmarkyes}{\Pisymbol{psy}{183}}
\newcommand{\tickmarkno}{\textendash}
\providecommand*{\textln}[1]{#1}
\providecommand*{\lnstyle}{}

\setcounter{secnumdepth}{5}

\makeatletter

\newenvironment{nameparts}
  {\trivlist\item
   \tabular{@{}ll@{}}}
  {\endtabular\endtrivlist}

\newenvironment{namedelims}
  {\trivlist\item
   \tabularx{\textwidth}{@{}c@{=}l>{\raggedright\let\\=\tabularnewline}X@{}}}
  {\endtabularx\endtrivlist}

\newenvironment{namesample}
  {\def\delim##1##2{\@delim{##1}{\normalfont\tiny\bfseries##2}}%
   \def\@delim##1##2{{%
     \setbox\@tempboxa\hbox{##1}%
     \@tempdima=\wd\@tempboxa
     \wd\@tempboxa=\z@
     \box\@tempboxa
     \begingroup\spotcolor
     \setbox\@tempboxa\hb@xt@\@tempdima{\hss##2\hss}%
     \vrule\lower1.25ex\box\@tempboxa
     \endgroup}}%
   \ttfamily\trivlist
   \setlength\itemsep{0.5\baselineskip}}
  {\endtrivlist}

\makeatother

\newrobustcmd*{\Deprecated}{%
  \textcolor{spot}{\margnotefont Deprecated}}
\newrobustcmd*{\DeprecatedMark}{%
  \leavevmode\marginpar{\Deprecated}}
\newrobustcmd*{\BiberOnly}{%
  \textcolor{spot}{\margnotefont Biber only}}
\newrobustcmd*{\BiberOnlyMark}{%
  \leavevmode\marginpar{\BiberOnly}}
\newrobustcmd*{\BibTeXOnly}{%
  \textcolor{spot}{\margnotefont BibTeX only}}
\newrobustcmd*{\BibTeXOnlyMark}{%
  \leavevmode\marginpar{\BibTeXOnly}}
\newrobustcmd*{\LF}{%
  \textcolor{spot}{\margnotefont Label field}}
\newrobustcmd*{\LFMark}{%
  \leavevmode\marginpar{\LF}}

% following snippet is based on code by Michael Ummels (TeX Stack Exchange)
% <http://tex.stackexchange.com/a/13073/8305>
\makeatletter
  \newcommand\fnurl@[1]{\footnote{\url@{#1}}}
  \DeclareRobustCommand{\fnurl}{\hyper@normalise\fnurl@}
\makeatother

\hyphenation{%
  star-red
  bib-lio-gra-phy
  white-space
}

\begin{document}

\printtitlepage
\tableofcontents
\listoftables

\newpage
\section{Einleitung} \label{int}

Dieses Dokument ist ein systematisches Referenzhandbuch für das Paket
\biblatex. Sehen Sie sich auch die Beispieldokumente an, die mit diesem Paket
mitgeliefert wurden, um einen ersten Eindruck zu bekommen.
\fnurl{\biblatexctan/doc/examples} Für einen schnellen Einstieg, lesen Sie
\secref{int:abt, bib:typ, bib:fld, bib:use, use:opt, use:xbx, use:bib, use:cit,
use:use}.

\subsection{Über \biblatex} \label{int:abt}

Dieses Paket bietet erweiterte bibliografische Möglichkeiten für die Verwendung
von \latex zusammen mit \bibtex. Das Paket ist eine komplette Neuimplementierung der
bibliografischen Einrichtungen, welche von \latex zur Verfügung gestellt
werden. Das \biblatex-Paket arbeitet mit dem  \enquote{Backend}(Program) \biber,
das standardmäßig verwendet wird, das
die Dateien im  \bibtex-Format verarbeitet und alle Sortierungen, Etikettierungen
(und noch viel mehr) vornimmt. Die Formatierung der Bibliografie wird vollständig von 
\TeX-Makros gesteuert.
Gute Kenntnisse in \latex sollten ausreichend sein, um eine neue
Bibliografie und Zitatvorlage zu erstellen. Dieses Paket unterstützt auch
unterteilte Bibliografien, mehrere Bibliografien innerhalb eines Dokuments und
separate Listen von bibliografischen Kürzeln. Bibliografien können in Teile
unterteilt und\slash oder nach Themen segmentiert werden. Genau wie die
Bibliografiestile, können alle Zitatbefehle frei definiert werden. Es stehen Merkmale wie anpassbare Sortierung, mehrere Bibliografien mit unterschiedlicher Sortierung, anpassbare Etiketten und dynamische Datenänderung zur Verfügung. Bitte beachten Sie den Abschnitt \secref{int:pre:bibercompat} mit Informationen zu der \biber-\biblatex-Versionskompatibilität. 
Bitte beachten Sie die \tabrefe{bib:fld:tab1} für eine Liste der Sprachen, 
die derzeit von diesem Paket unterstützt werden. 

\subsection{Lizenz}

Copyright \textcopyright\ 2006--2012 Philipp Lehman, 2012--2017 Philip Kime, 
Audrey Boruvka, Joseph Wright, 2018--Philip Kime und Moritz Wemheuer. Es ist erlaubt, zu
kopieren, zu verteilen und \slash oder die Software zu modifizieren, unter
den Bedingungen der \lppl, Version
1.3.\fnurl{http://www.latex-project.org/lppl.txt}


\subsection{Rückmeldungen} \label{int:feb}

Bitte benutzen Sie die  \biblatex-Projektseite auf GitHub, um Fehler zu
melden und um Featureanfragen zu
senden.\fnurl{http://github.com/plk/biblatex} 
Bevor Sie eine Featureanfrage senden, stellen Sie
bitte sicher, das Sie dieses Benutzerhandbuch gründlich gelesen haben. Wenn Sie
keinen Bug melden wollen oder eine Featureanfage stellen, sondern einfach Hilfe
benötigen, können Sie Ihre Frage in der \texttt{comp.text.tex} Newsgroup oder
bei \tex-\latex Stack Exchange stellen.\fnurl{http://tex.stackexchange.com/questions/tagged/biblatex} 

\subsection{Danksagungen}

Das Paket wurde ursprünglich von Philipp Lehman geschrieben und ein Großteil seines exzellenten Orignalcodes bleibt im Kern bestehen. Philip Kime übernahm das Paket 2012 mit Moritz Wemheuer, der ab 2017 regelmäßig wertvolle Beiträge leistete. Die Hauptautoren möchten die wertvolle Hilfe von Audrey Borurvka und Josef Wright hervorheben, die 2012 und in den folgenden Jahren beim Eigentümerwechsel geholfen haben.

Die Sprachmodule dieses Pakets beinhalten die Beiträge von (in der Reihenfolge
der Einreichung):
Augusto Ritter Stoffel, Mateus Araújo (Brazilianisch);
Sebastià Vila-Marta (Katalanisch);
Ivo Pletikosić (Kroatisch);
Michal Hoftich (Tschechisch);
Jonas Nyrup (Dänisch);
Johannes Wilm (Danish\slash Norwegisch);
Alexander van Loon, Pieter Belmans, Hendrik Maryns (Niederländisch);
Hannu Väisänen, Janne Kujanpää (Finnisch);
Denis Bitouzé (Französisch);
Apostolos Syropoulos, Prokopis (Griechisch);
Baldur Kristinsson (Isländisch);
Enrico Gregorio, Andrea Marchitelli (Italienisch);
Håkon Malmedal (Norwegisch);
Anastasia Kandulina, Yuriy Chernyshov (Polnisch);
José Carlos Santos (Portugiesisch);
Oleg Domanov (Russisch);
Tea Tušar and Bogdan Filipič (Slovenisch);
Ignacio Fernández Galván (Spanisch);
Per Starbäck, Carl-Gustav Werner, Filip Åsblom (Schwedisch);
Martin Vrábel, Dávid Lupták (Slovakisch),
Benson Muite (Estisch);

\subsection{Voraussetzungen} \label{int:pre}

Dieser Abschnitt gibt einen Überblick über alle nötigen Mittel, die für dieses
Paket erforderlich sind und diskutiert Kompatibilitätsprobleme. 

\subsubsection{Anforderungen} \label{int:pre:req}

Die Mittel, die in diesem Abschnitt aufgeführt sind, sind unbedingt
erforderlich, damit \biblatex funktionieren kann. Das Paket wird nicht
funktionieren, wenn sie nicht verfügbar sind. 

\begin{marglist}

\item[\etex]
Das \biblatex-Paket erfordert \etex. \tex-Distributionen enthalten seit
geraumer Zeit \etex-Binärdateien, die bekannten Distributionen nutzen diese
heutzutage  standardmäßig. Das \biblatex-Paketes überprüft, ob es unter
\etex funktioniert. Wenn Sie Ihre Dokumente wie gewohnt kompilieren, stehen die
Chancen gut, dass es funktioniert. Wenn Sie eine Fehlermeldung erhalten,
versuchen Sie Ihr Dokument mit \bin{elatex} statt \bin{latex} bzw.
\bin{pdfelatex} statt mit \bin{pdflatex} zu kompilieren. 

\item[\biber]
\biber ist das Backend für \biblatex zur Übertragung von Daten aus Quelldateien in den \latex-Code. \biber kommt mit TeXLive und ist auch erhaltbar in SourceForge.\fnurl{http://biblatex-biber.sourceforge.net/}. \biber verwendet die "`\texttt{btparse} C library"' für
die \bibtex-Format-Datei-Analyse, die mit mit den \bibtex-Parsing-Regeln aber vereinbar sein sollte und auch auf die Korrektur der häufigsten Probleme gerichtet ist.  Zu den Einzelheiten siehe die Hilfeseite zum "`Perl \texttt{Text::BibTeX} Modul"'\fnurl{http://search.cpan.org/~ambs/Text-BibTeX}.

\item[etoolbox] 
Dieses \latex-Paket, das automatisch geladen wird, bietet
generische Programmierung, welche zur Einrichtungen von \biblatex
erforderlich ist. Es ist über \acr{CTAN}
\fnurl{http://ctan.org/pkg/kvoptions}
erhältlich.

\item[kvoptions]
Dieses \latex-Paket, das auch automatisch geladen wird, wird für interne Optionen
verwendet. Es ist im "`\sty{oberdiek} package bundle"' von 
\acr{CTAN}.\fnurl{http://www.ctan.org/pkg/kvoptions}

\item[logreq] Dieses \latex-Paket, das auch automatisch geladen wird, stellt ein
Programm für das Schreiben von maschinenlesbaren Nachrichten für eine
zusätzliche Log-Datei bereit. Es ist erhältlich über
\acr{CTAN}.\fnurl{http://ctan.org/pkg/logreq/} 

\item[xstring]
Dieses \latex-Paket, das auch automatisch geladen wird, stellt erweiterte
Stringverarbeitungsmakros bereit. Es ist erhältlich über 
\acr{CTAN}.\fnurl{http://ctan.org/pkg/xstring/}

\end{marglist}

Abgesehen von den oben genannten Mitteln, erfordert \biblatex auch die
Standard-\latex-Pakete \sty{keyval} und \sty{ifthen} sowie das \sty{url}-Paket.
Diese Pakete sind in allen gängigen \tex-Distributionen enthalten und werden
automatisch geladen. 

\subsubsection{Empfohlene Pakete} \label{int:pre:rec}

Die Pakete in diesem Abschnitt sind nicht erforderlich für die
Funktionsfähigkeit von \biblatex, aber sie bieten zusätzliche Funktionen
oder Verbesserungen zu vorhandenen Features. Die Reihenfolge, wie man die Pakete
lädt, ist unerheblich. 

\begin{marglist}

\item[babel/polyglossia]
Die \sty{babel} und \sty{polyglossia} Pakete bieten die Kernarchitektur für
mehrsprachigen Schriftsatz.  Wenn Sie in einer anderen Sprache als in
amerikanischen Englisch schreiben, wird die Nutzung dieser Pakete empfohlen.
Wenn \biblatex geladen wird, wird \sty{babel} oder \sty{polyglossia} automatisch erkannt.

\item[csquotes] 
Wenn dieses Paket verfügbar ist, wird \biblatex seinen
Sprachteil nutzen, um bestimmte Titel in Anführungszeichen zu setzen. Wenn
nicht, nutzt \biblatex passende Zitierweisen für das amerikanische
Englisch. Beim Schreiben in einer anderen Sprache, wird das Laden von
\sty{csquotes}
empfohlen.\fnurl{http://ctan.org/pkg/csquotes/}


\item[xpatch]
Das \sty{xpatch}-Paket erweitert die "`patching Befehle"' von \sty{etoolbox} für
\biblatex-Bibliografie-Makros, Treiber und 
Formatierungsanweisungen.\fnurl{http://ctan.org/pkg/xpatch/}

\end{marglist}

\subsubsection{Kompatible Klassen und Pakete} \label{int:pre:cmp}

Das Paket \biblatex bietet Kompatibilitätscodes für Klassen und Pakete,
welche in diesem Abschnitt aufgeführt sind. 

\begin{marglist}

\item[hyperref] Das \sty{hyperref}-Paket verwandelt Zitierungen in Hyperlinks.
Siehe \opt{hyperref}- und \opt{backref}-Paketoptionen in \secref{use:opt:pre:gen}
für weitere Details. Bei der Verwendung des \sty{hyperref}-Pakets, wird empfohlen,
vorher \sty{biblatex} zu laden. 

\item[showkeys] Das \sty{showkeys}-Paket zeigt die internen Schlüssel, unter
anderem von Zitaten im Text und Begriffen in der Bibliografie. Die Reihenfolge,
wie man die Pakete lädt, ist unerheblich. 

\item[memoir] Bei Verwendung der \sty{memoir}-Klasse sind die
Standard-Bibliografie-Überschriften so angepasst, dass sie gut mit dem
Standard-Layout dieser Klasse zusammenpassen. Siehe \secref{use:cav:mem} für
weitere Verwendungshinweise. 

\item[\acr{KOMA}-Script] Bei Verwendung der \sty{scrartcl}-, \sty{scrbook}- oder
\sty{scrreprt}-Klassen, die Standard-Bibliografie-Überschriften werden so
angepasst, dass sie gut mit dem Standard-Layout dieser Klassen zusammenpassen.
Siehe \secref{use:cav:scr} für die weitere Verwendungshinweise. 

\end{marglist}

\subsubsection{Inkompatible Pakete } \label{int:pre:inc}

Die in diesem Abschnitt aufgeführten Pakete sind nicht kompatibel mit \biblatex. Seit
der Reimplementierung der Bibliografieausstattung von \latex hat \biblatex
Probleme mit allen Paketen, die die selben Ausstattungen modifiziert haben. Dies
ist nicht spezifisch von \sty{biblatex}. Einige der unten aufgeführten Pakete
sind aus dem selben Grund miteinander inkompatibel. 

\begin{marglist}

% FIXME: complete the list
%
% amsrefs
% apacite
% drftcite
% footbib
% multibbl
% overcite
% bibtopicprefix

\item[babelbib]
Das \sty{babelbib}-Paket bietet Unterstützung für mehrsprachige
Bibliografien. Dies ist ein Standardfeature von \biblatex. Verwenden Sie
das Feld \bibfield{langid} und die Paketoption \opt{autolang} für ähnliche
Funktionen. Beachten Sie, dass \biblatex automatisch die Sprache
entsprechend dem Hauptdokument anpasst, wenn \sty{babel} oder \sty{polyglossia}
geladen ist. Sie
brauchen nur die oben genannten Funktionen, falls Sie die Sprachen auf einer
pro-Eintrag-Grundlage innerhalb der Bibliografie ändern wollen. Siehe
\secref{bib:fld:spc, use:opt:pre:gen} für Details. Sehen Sie auch \secref{use:lng}. 

\item[backref] 
Das \sty{backref}-Paket schafft Rückverweise in die
Bibliografie. Siehe die Paketoptionen \opt{hyperref} und \opt{backref} in
\secref{use:opt:pre:gen} für vergleichbare Funktionalität.

\item[bibtopic] 
Das \sty{bibtopic}-Paket bietet Unterstützung für Bibliografien
gegliedert nach Themen, Typ oder anderen Kriterien. Für  Bibliografien,
unterteilt nach Themen, finden Sie in der Kategorie Feature (in
\secref{use:bib:cat}) und die entsprechenden Filter in \secref{use:bib:bib}.
Alternativ können Sie das Feld \bibfield{keywords} in Verbindung mit den
\opt{keyword} und \opt{notkeyword} Filter für vergleichbarer Funktionalität
verwenden, vgl. \secref{bib:fld:spc, use:bib:bib} für weitere Einzelheiten. Für
Bibliografien unterteilt nach Typ, nutzen sie den \opt{type} und \opt{nottype}
Filtern. Siehe auch \secref{use:use:div} für Beispiele.

\item[bibunits]
Das \sty{bibunits}-Paket bietet Unterstützung für mehrere
Teilbibliografien (z.\,B. pro Kapitel). Siehe \sty{chapterbib}. 

\item[chapterbib] 
Das \sty{chapterbib}-Paket bietet Unterstützung für mehrere
Teilbibliografien. Verwenden Sie die \env{refsection}-Umgebung und den
\opt{section}-Filter für vergleichbare Funktionalität. Alternativ können Sie
auch die \env{refsegment}-Umgebung und den \opt{segment}-Filter verwenden. Siehe
\secref{use:bib:sec, use:bib:seg, use:bib:bib} für weitere Einzelheiten. Siehe
auch \secref{use:use:mlt} für Beispiele. 

\item[cite] 
Das \sty{cite}-Paket sortiert automatisch die numerischen Zitate und
kann eine Liste von komprimierten fortlaufenden Nummern auf einen Bereich
erstellen. Es macht auch die Interpunktion, welche in Zitaten verwendet wird,
konfigurierbar. Für sortierte und komprimierte numerische Zitate finden Sie in
der \opt{sortcites}-Paketoption in \secref{use:opt:pre:gen} und den
\texttt{numeric-comp}-Zitierstil in \secref{use:xbx:cbx}. Für
konfigurierbare Interpunktion siehe \secref{use:fmt}. 

\item[citeref] 
Ein weiteres Paket für die Erstellung für Rückverweise in die
Bibliografie. Sehen Sie auch \sty{backref}. 

\item[inlinebib] 
Das \sty{inlinebib}-Paket ist für traditionelle Zitate in
Fußnoten konzipiert. Für vergleichbare Funktionalität finden Sie den
ausführlichen Zitierstil in \secref{use:xbx:cbx}. 

\item[jurabib] 
Ursprünglich für Zitierungen im Jurastudium und (meist deutsche)
gerichtliche Dokumente entworfen, bietet das \sty{jurabib}-Paket auch Funktionen
an für Anwender in den Geisteswissenschaften. Im Hinblick auf die Eigenschaften
gibt es einige Ähnlichkeiten zwischen \sty{jurabib} und \sty{biblatex}, aber die
Ansätze der beiden Pakete sind recht verschieden. Seit die beiden Pakete
\sty{jurabib} und \sty{biblatex} voll ausgestattet sind, ist die Liste der
Gemeinsamkeiten und Unterschiede zu lang, um hier dargestellt zu werden. 

\item[mcite]
Das \sty{mcite}-Paket bietet Unterstützung für gruppierte Zitate, d.\,h. es
können mehrere Elemente als eine einzige Referenz zitiert und als einzelner
Eintrag in der Bibliografie aufgelistet werden. Die Zitatgruppen sind so
definiert, wie die Elemente zitiert wurden. Dies funktioniert nur mit
unsortierten Bibliografien. Das Paket \sty{biblatex} unterstützt auch
gruppierte Zitate, welche in diesem Handbuch <Eintragstypen/entry sets> oder
<Referenztypen/reference sets> heißen. Siehe
\secref{use:use:set,use:bib:set,use:cit:mct} für Einzelheiten. 

\item[mciteplus] 
Eine deutlich verbesserte Neuimplementierung des
\sty{mcite}-Paketes, die Gruppierung in sortierten Bibliografien unterstützt.
Siehe \sty{mcite}. 

\item[multibib] 
Das \sty{multibib}-Paket bietet Unterstützung für
Bibliografien, welche thematisch oder nach anderen Kriterien unterteilt sind.
Siehe \sty{bibtopic}.

\item[natbib] 
Das \sty{natbib}-Paket unterstützt numerische und
Autor"=Jahr"=Zitat-Systeme unter Einbeziehung von Sortier-und
Kompressions-Codes, welche Sie im \sty{cite}-Paket finden. Es bietet auch
zusätzliche Zitierbefehle und mehrere Konfigurationsoptionen. Sehen Sie die
\texttt{numeric} und author"=year-Zitierstile und ihre Varianten in
\secref{use:xbx:cbx}, die \opt{sortcites}-Paket-Option in
\secref{use:opt:pre:gen}, die Zitierbefehle in \secref{use:cit} und die
Einstellungen in den \secref{use:bib:hdg, use:bib:nts, use:fmt} für
vergleichbare Funktionalität. Siehe auch \secref{use:cit:nat}. 

\item[splitbib] 
Das \sty{splitbib}-Paket bietet Unterstützung für
Bibliografien, welche thematisch unterteilt sind. Siehe \sty{bibtopic}.

\item[titlesec]
Das \sty{titlesec}-Paket redefiniert Befehle der "`user-level-Document-Division"', 
solche wie \cmd{chapter} oder \cmd{section}. Dieser Ansatz ist nicht kompatibel
mit dem Wechsel interner Befehle, angewendet bei \biblatex s \texttt{refsection} und \texttt{refsegment}-Optioneneinstellungen, beschrieben in \secref{use:opt:pre:gen}.

\item[ucs]
Das \sty{ucs}-Paket bietet Unterstützung für \acr{UTF-8}-kodierte Eingaben.
Entweder verwenden sie von \sty{inputenc} das Standard-\file{utf8}-Modul oder
eine unicode-fähige Engine wie \xetex oder \luatex. 

\end{marglist}

\subsubsection{Kombatibilitätsmatrix \biber}
\label{int:pre:bibercompat}

Die \biber-Versionen sind eng verbunden mit den \biblatex-Versionen. Sie benötigen die richtige Kombination der beiden. \biber wird Sie im Kombilierungsprozess warnen, wenn
Informationen auf Inkompatibilitäten mit der \biblatex-Version auftreten. 
\tabrefe{tab:int:pre:bibercompat} zeigt eine Kompatibilitätsmatrix für die
aktuellen Versionen.

\begin{table}
\tablesetup\centering
\begin{tabular}{cc}
\toprule
\sffamily\bfseries\spotcolor Biberversion
  & \sffamily\bfseries\spotcolor \biblatex version\\
\midrule
2.13 & 3.13\\
2.12 & 3.12\\
2.11 & 3.11\\
2.10 & 3.10\\
2.9 & 3.9\\
2.8 & 3.8\\
2.7 & 3.7\\
2.6 & 3.5\\
2.5 & 3.4\\
2.4 & 3.3\\
2.3 & 3.2\\
2.2 & 3.1\\
2.1 & 3.0\\
2.0 & 3.0\\
1.9 & 2.9\\
1.8 & 2.8\\
1.7 & 2.7\\
1.6 & 2.6\\
1.5 & 2.5\\
1.4 & 2.4\\
1.3 & 2.3\\
1.2 & 2.1, 2.2\\
1.1 & 2.1\\
1.0 & 2.0\\
0.9.9 & 1.7x\\
0.9.8 & 1.7x\\
0.9.7 & 1.7x\\
0.9.6 & 1.7x\\
0.9.5 & 1.6x\\
0.9.4 & 1.5x\\
0.9.3 & 1.5x\\
0.9.2 & 1.4x\\
0.9.1 & 1.4x\\
0.9 & 1.4x\\
\bottomrule
\end{tabular}
\caption{\biber/\biblatex\ Kompatibilitätsmatrix}
\label{tab:int:pre:bibercompat}
\end{table}

\section{Anleitung zur Datenbasis} \label{bib} 

Dieses Kapitel beschreibt das Standarddatenmodell, welches in der Datei \file{blx-dm.def}, die Teil von \path{biblatex} ist, definiert wurde. Das Datenmodell wurde definiert unter Benutzung der Makros, die im Kapitel § 4.5.4 (engl. Gesamtversion) %\secref{aut:ctm:dm} 
dokumentiert sind. Es ist möglich, das Datenmodell
neu zu definieren, das \biblatex und \biber verwendet, so kann die Datenquelle neue Eintragstypen und Felder (was Stilunterstützung verlangt) enthalten. Die Spezifikation des Datenmodells
ermöglicht auch Beschränkungen, so dass Datenquellen so definiert werden, dass sie gegen das Datenmodell validiert werden (nehmend \biber's \path{--validate-Datenmodel}-Option). 
Nutzer, die das Datenmodell anpassen möchten, müssen dafür in die \file{blx-dm.def}
schauen und § 4.5.4 (eng. Gesamtversion)
%\secref{aut:ctm:dm} 
lesen.

\subsection{Eingabetypen} \label{bib:typ} 

Dieser Abschnitt gibt einen Überblick
über die Eingabetypen und deren Felder, welche von \sty{biblatex} standardmäßig 
unterstützt werden. 

\subsubsection{Grundtypen} \label{bib:typ:blx}

Die folgende Liste enthält die Felder, die von jeweiligen Eingabetyp unterstützt
werden. Beachten Sie, dass die Zuordnung der Felder zu einem Eingabetyp im
Ermessen des Biblografiestils liegt. Die folgende Liste verfolgt daher zwei
Ziele. Sie zeigt die Felder, welche von den Standardstilen unterstützt werden, die
in diesem Paket enthalten sind und sie dienen auch als Vorbild für eigene
Designs. Beachten Sie, dass die <erforderlichen> Felder nicht in allen Fällen
unbedingt erforderlich sind, siehe \secref{bib:use:key} für weitere
Einzelheiten. Die Felder <optional> sind optional in einem technischen Sinne.
Bibliografische Formatierungsregeln erfordern in der Regel mehr als nur die
<erforderlich>-Felder. Die Standardstile unterliegen in der Regel keinen
formalen Gültigkeitskontrollen, haben aber jedoch eigene Designs, ISBNs und 
andere Spezialfelder wie \bibfield{gender}. Aber diese Muster sind nur nutzbar mit
der "`\biber \path{--validate_datamodel}"'-Option. Generische
Felder wie \bibfield{abstract} und \bibfield{annotation} oder \bibfield{label}
und \bibfield{shorthand} sind nicht in der Liste unten aufgeführt, weil sie
unabhängig vom Eingabetyp sind. Die speziellen Felder, die in
\secref{bib:fld:spc} erklärt werden und auch unabhängig vom Eingabetyp sind,
sind ebenfalls nicht in der Liste enthalten.  Die Spezifikationen des Standarddatenmodells
sehen Sie in der Datei \file{blx-dm.def}, die kommt mit \biblatex für eine vollständige
Spezifikation.

Die in diesem Unterabschnitt erwähnte <alias>-Relation ist die mit 
\cmd{DeclareBibliographyAlias} definierte <soft alias>. Dies bedeutet,
dass alias den selben Bibliografietreiber wie der Typ, für den er einen
Alias erstellt, seine typspezifische Formatierung wird jedoch weiterhin
unabhängig vom Alias-Typ behandelt.

\begin{typelist}

\typeitem{article}

Ein Artikel in einem Journal, einer Zeitschrift, einer Zeitung oder einem
anderem Periodikum, welches einen eigenen Abschnitt mit einem eigenen Titel
erhält. Der Titel des Periodikums ist im \bibfield{journaltitle}-Feld enthalten.
Wenn die Ausgabe neben dem Haupttitel des Periodikums einen eigenen Titel hat,
ist dieser im \bibfield{issuetitle}-Feld einzugeben. Beachten Sie, dass
\bibfield{editor} und verwandte Felder zum Journal gehören, während
\bibfield{translator} und verwandte Felder, zu \bibtype{article} gehören. 

Pflichtfelder: \texttt{author, title, journaltitle, year/date}\\ 
Wahlfelder:
\texttt{translator, annotator, commentator, subtitle, titleaddon, editor, editora, editorb, editorc, journalsubtitle, issuetitle, issuesubtitle, language, origlanguage, series, volume, number, eid, issue, month, pages, version, note, issn, addendum, pubstate, doi, eprint, eprintclass, eprinttype, url, urldate}

\typeitem{book} 

Ein Einzelstück, ein Buch mit einem oder mehreren Autoren, wo die Autoren alle
zusammen gearbeitet haben. Beachten Sie, dass dieser Eingabetyp auch die
Funktion des \bibtype{inbook}-Typs des herkömmlichen \bibtex umfasst , siehe
\secref{bib:use:inb} für weitere Einzelheiten. 

Pflichtfelder: \texttt{author, title, year/date}.

Wahlfelder: \texttt{editor, editora, editorb, editorc,
translator, annotator,\\ commentator, introduction, foreword, afterword, subtitle,
titleaddon,\\
maintitle, mainsubtitle, maintitleaddon, language, origlanguage,
volume, part, edition, volumes, series, number, note, publisher, location, isbn,
chapter, pages, pagetotal, addendum, pubstate, doi, eprint, eprintclass,
eprinttype, url, urldate}

\typeitem{mvbook}

Ein mehrbändiges Buch. Für rückwärtskompatible mehrbändige Bücher gilt der
Support des Eintragstypes \bibtype{book}. Jedoch ist es ratsam, den
Eintragstyp \bibtype{mvbook} zu nehmen.

Pflichtfelder: \texttt{author, title, year/date}\\
Wahlfelder: \texttt{editor, editora, editorb, editorc, translator,
annotator, commentator, introduction,
foreword,\\ afterword, subtitle, titleaddon, language, origlanguage, edition, volumes, series, number, 
note, publisher, location, isbn, pagetotal, addendum,\\ pubstate, doi, eprint, eprintclass, eprinttype, 
url, urldate}


\typeitem{inbook} 

Ein Teil eines Buches, der eine in sich geschlossene Einheit
mit seinem eigenen Titel bildet. Beachten Sie, dass das Profil dieses
Eingabetyps verschieden ist vom Standard-\bibtex, siehe \secref{bib:use:inb}. 

Pflichtfelder: \texttt{author, title, booktitle, year/date}\\ 
Wahlfelder: \texttt{bookauthor, editor,
editora, editorb, editorc, translator, annotator, commentator, introduction,
foreword, afterword, subtitle,\\ titleaddon, maintitle, mainsubtitle,
maintitleaddon, booksubtitle, book\-titleaddon, language, origlanguage, volume,
part, edition, volumes, series, number, note, publisher, location, isbn,
chapter, pages, addendum, pubstate, doi, eprint, eprintclass, eprinttype, url,
urldate}

\typeitem{bookinbook} 

Ähnlich wie \bibtype{inbook}, aber bestimmt für Artikel,
welche ursprünglich als eigenständiges Buch veröffentlicht werden sollten. Ein
typisches Beispiel sind Bücher, welche in einem Sammelwerk eines Autors abgedruckt
sind. 


\typeitem{suppbook}

Zusätzliches Material in einem \bibtype{book}. Diese Typ ist eng mit dem
\bibtype{inbook}-Eingabetyp verwandt. Während \bibtype{inbook} in erster Linie
für einen Teil eines Buches mit eigenem Titel bestimmt ist (d.\,h. einen
einzelnen Aufsatz in einer Sammlung von Essays von demselben Autor), ist diese
Art vorgesehenen für Elemente wie Vorreden, Einleitungen, Vorworte, Nachworte,
etc., welche oft nur eine allgemeine Bezeichnung haben. "`Style Guides"' erfordern
solche Elemente, um andere \bibtype{inbook}-Elemente zu formatiert. Die
Standardstile behandeln diese Eingabetypen ähnlich wie \bibtype{inbook}. 

\typeitem{booklet}

Eine buchähnliche Arbeit ohne formalen Verlag oder eine Sponsoring-Institution.
Verwenden Sie das Feld \bibfield{howpublished} für zusätzliche
Veröffentlichungsinformationen, wenn zutreffend. Das Feld \bibfield{type} kann
auch sinnvoll sein. 

Pflichtfelder: \texttt{author/editor, title, year/date}\\ Wahlfelder:
\texttt{subtitle, titleaddon,
language, howpublished, type, note, lo\-cation, chapter, pages, pagetotal,
addendum, pubstate, doi, eprint,\\ eprintclass, eprinttype, url, urldate}

\typeitem{collection} 

Ein Buch mit mehreren, in sich geschlossenen Beiträgen,
die verschiedene Autoren haben und ihre eigenen Titel. Die Arbeit als Ganzes hat
keinen allgemeinen Autor, aber es gibt in der Regel einen Herausgeber. 

Pflichtfelder: \texttt{editor, title, year/date}\\ Wahlfelder: \texttt{editora, editorb, 
editorc,
translator, annotator, commentator, introduction, foreword, afterword, subtitle,
titleaddon, maintitle, main\-subtitle, maintitleaddon, language, origlanguage,
volume, part, edition,\\ volumes, series, number, note, publisher, location, isbn,
chapter, pages, pagetotal, addendum, pubstate, doi, eprint, eprintclass,
eprinttype, url, urldate}

\typeitem{mvcollection}

Eine mehrbändige Kollektion (\bibtype{collection}). Rückwärtskompatible
mehrbändige "`Collections"' werden unterstützt vom Eintragtyp
\bibtype{collection}. Trotzdem ist es sinnvoll, den Eintragtyp
\bibtype{mvcollection} zu nehmen.

Pflichtfelder: \texttt{editor, title, year/date}.\\
Wahlfelder: \texttt{editora, editorb, editorc, translator, annotator,
commentator, introduction, foreword, afterword, subtitle, titleaddon,
language, origlanguage, edition, volumes, series, number, note, publisher,
location, isbn, pagetotal, addendum, pubstate, doi, eprint, eprintclass,
eprinttype, url, urldate}.


\typeitem{incollection} 

Ein Beitrag zu einer Sammlung, die eine in sich
geschlossene Einheit bildet, mit einem deutlichen Autor und Titel. Der
\bibfield{author} bezieht sich auf den \bibfield{title}, der \bibfield{editor}
auf den \bibfield{booktitle}, d.\,h., den Titel der Sammlung. 

Pflichtfelder: \texttt{author, title, booktitle, year/date}\\
Wahlfelder: \texttt{editor, editora, editorb,
editorc, translator, annotator, commentator, introduction, foreword, afterword,
subtitle, titleaddon, maintitle, main\-subtitle, maintitleaddon, booksubtitle,
booktitleaddon, language, orig\-language, volume, part, edition, volumes, series,
number, note, publisher,\\ location, isbn, chapter, pages, addendum, pubstate,
doi, eprint, eprint\-class, eprinttype, url, urldate}

\typeitem{suppcollection}

Zusätzliches Material in \bibtype{collection}. Dieser Typ ist ähnlich wie
\bibtype{collection}, aber bezieht sich auf den \bibtype{collection}-Eingabetyp.
Der Standardstil behandelt diesen Eingabetyp ähnlich wie \bibtype{incollection}. 

\typeitem{dataset}

Ein Datensatz oder eine ähnliche Sammlung von (meistens) Rohdaten.

Pflichtfelder: \texttt{author/editor, title, year/date}

Wahlfelder: \texttt{subtitle, titleaddon, language, edition, type, series, number, version, note, organization, publisher, location, addendum, pubstate, doi, eprint, eprintclass, eprinttype, url, urldate}


\typeitem{manual} 

Technische oder sonstige Unterlagen, nicht unbedingt in
gedruckter Form. Der \bibfield{author} oder \bibfield{editor} sind im Sinne des
\secref{bib:use:key} weggelassen worden. 

Pflichtfelder: \texttt{author/editor, title, year/date}\\ Wahlfelder:
\texttt{subtitle, titleaddon,
language, edition, type, series, number, version, note, organization, publisher,
location, isbn, chapter, pages, pagetotal, addendum, pubstate, doi, eprint,
eprintclass, eprinttype, url, urldate}

\typeitem{misc}

Ein Typ für Einträge, die in keine andere Kategorie passen.
Verwenden Sie das Feld \bibfield{howpublished} für zusätzliche
Veröffentlichungsinformationen, wenn zutreffend. Das Feld \bibfield{type} kann
auch sinnvoll sein. Der \bibfield{author}, \bibfield{editor} und \bibfield{year}
sind im Sinne des \secref{bib:use:key} weggelassen worden. 

Pflichtfelder: \texttt{author/editor, title, year/date}\\ Wahlfelder:
\texttt{subtitle, titleaddon,
language, howpublished, type, version, note, organization, location, date,
month, year, addendum, pubstate,\\ doi, eprint, eprintclass, eprinttype, url,
urldate}

\typeitem{online} 

Eine Online-Quelle. \bibfield{author}, \bibfield{editor} und
\bibfield{year} sind im Sinne von \secref{bib:use:key} weggelassen worden.
Dieser Eingabetyp ist für reine Onlinequellen wie Websites gedacht. Beachten
Sie, dass alle Eingabetypen das \bibfield{url}-Feld unterstützen. Zum Beispiel,
wenn Sie einen Artikel aus einer Zeitschrift, die online verfügbar ist,
zitieren, verwenden Sie den \bibtype{article}-Typ und sein \bibfield{url}-Feld. 

Pflichtfelder: \texttt{author/editor, title, year/date, url}\\ Wahlfelder:
\texttt{subtitle, titleaddon,
language, version, note, organization, date, month, year, addendum, pubstate,
urldate}

\typeitem{patent} 

Ein Patent oder eine Patentanfrage. Die Anzahl oder
Eintragsnummer ist im Feld \bibfield{number} angegeben. Verwenden Sie das Feld
\bibfield{type}, um den Typ zu spezifizieren und das Feld \bibfield{location},
um den Umfang des Patentes anzugeben, falls dieser vom Standardmäßigen abweicht.
Beachten Sie, dass das \bibfield{location}-Feld eine Schlüsselliste für diesen
Eingabetyp ist, siehe \secref{bib:fld:typ} für weitere Einzelheiten. 

Pflichtfelder: \texttt{author, title, number, year/date}\\ Wahlfelder:
\texttt{holder, subtitle,
titleaddon, type, version, location, note, date, month, year, addendum,
pubstate, doi, eprint, eprintclass, eprint\-type, url, urldate}

\typeitem{periodical} 

Eine ganze Ausgabe einer Zeitschrift, ähnlich einer
speziellen Ausgabe einer Zeitschrift. Der Titel der Zeitschrift ist im Feld
\bibfield{title} gegeben. Wenn die Ausgabe zusätzlich zu dem Haupttitel der
Zeitschrift einen eigenen Titel hat, wird dieser im Feld \bibfield{issuetitle}
angegeben. Der \bibfield{editor} wurde im Sinne von \secref{bib:use:key}
weggelassen. 

Pflichtfelder: \texttt{editor, title, year/date}\\ Wahlfelder: \texttt{editora, editorb, 
editorc, subtitle,
issuetitle, issuesubtitle, language, series, volume, number, issue, date, month,
year, note, issn, addendum, pubstate, doi, eprint, eprintclass, eprinttype, url,
urldate}

\typeitem{suppperiodical}

Zusätzliches Material in einem \bibtype{periodical}. Dieser Typ ist ähnlich wie
\bibtype{suppbook}, gehört aber zum \bibtype{periodical}-Eingabetyp. Die Rolle
dieses Eintragstyps kann besser verstanden werden, wenn man bedenkt, dass der
\bibtype{article}-Typ auch \bibtype{inperiodical} genannt werden könnte. Dieser
Typ kann nützlich sein, wenn es um Dinge wie regelmäßige Kolumnen, Nachrufe,
Leserbriefe, etc. geht, die nur eine allgemeine Bezeichnung haben. Stile
erfordern solche Elemente, da sie anders als Artikel formatiert sind. Der
Standardstil behandelt diesen Eingabetyp ähnlich wie \bibtype{article}. 

\typeitem{proceedings} 

Bericht von einer Konferenz. Dieser Typ ist ähnlich wie
\bibtype{collection}. Er unterstützt ein optionales Feld
\bibfield{organization}, welches das Sponsorinstitut beinhaltet. Der
\bibfield{editor} wurde ist im Sinne von \secref{bib:use:key} weggelassen. 

Pflichtfelder: \texttt{editor, title, year/date}.
\\ Wahlfelder: \texttt{subtitle, titleaddon, 
maintitle,
mainsubtitle, maintitleaddon, eventtitle, eventdate, venue, language, volume,
part, volumes, series, number, note, organization, publisher, location, month,
isbn, chapter, pages, pagetotal, addendum, pubstate, doi, eprint, eprintclass,
eprint\-type, url, urldate}.

\typeitem{mvproceedings}

Ein mehrbändiger Bericht von einer Konferenz 
(\bibtype{proceedings}). Rückwärtskompatible
mehrbändige Proceedings werden unterstützt vom Eintragstyp
\bibtype{proceedings}. Trotzdem ist es sinnvoll, vom Eintragstyp
\bibtype{mvproceedings} Gebrauch zu machen.

Pflichtfelder: \texttt{editor, title, year/date}.\\
Wahlfelder: \texttt{subtitle, titleaddon, eventtitle, eventdate, venue,
language, volumes, series, number, note, organization, publisher, location,
month, isbn, pagetotal, addendum, pubstate, doi, eprint, eprintclass,
eprinttype, url, urldate}.


\typeitem{inproceedings}

Ein Artikel in einem Konferenzband. Dieser Typ ist ähnlich wie
\bibtype{incollection}. Er unterstützt ein optionales Feld
\bibfield{organization}. 

Pflichtfelder: \texttt{author, title, booktitle, year/date}\\
Wahlfelder: \texttt{editor, subtitle,
titleaddon, maintitle, mainsubtitle, maintitleaddon, booksubtitle,
booktitleaddon, eventtitle, eventdate, venue, language,\\ volume, part, volumes,
series, number, note, organization, publisher,\\ location, month, isbn, chapter,
pages, addendum, pubstate, doi, eprint, eprintclass, eprinttype, url, urldate}

\typeitem{reference}

Ein Nachschlagewerk wie ein Lexikon und Wörterbuch. Dies ist eine spezifischere
Variante des allgemeinen \bibtype{collection}-Eintragstyps. Der Standardstil
behandelt diesen Eingabetyp ähnlich wie \bibtype{collection}.

\typeitem{mvreference}

Ein mehrbändiges Nachschlagwerk (\bibtype{reference}). Die Standardstile
behandeln diesen Eintragstyp als Alias von \bibtype{mvcollection}.
Rückwärtskompatible
mehrbändige References werden unterstützt vom Eintragstyp
\bibtype{reference}. Trotzdem ist es sinnvoll, den Eintragstyp
\bibtype{mvreference} zu nehmen.


\typeitem{inreference}

Ein Artikel in einem Nachschlagewerk. Dies ist eine spezifischere Variante des
allgemeinen \bibtype{incollection}-Eintragstyps. Der Standardstil behandelt
diesen Eingabetyp ähnlich wie \bibtype{incollection}.

\typeitem{report}

Ein technischer Bericht, Forschungsbericht oder ein Weißbuch, wurde von einer
Universität oder einer anderen Institution veröffentlicht. Verwenden Sie das
Feld \bibfield{type}, um die Art des Berichts angeben. Das Sponsorinstitut wird
im Feld \bibfield{institution} angegeben. 

Pflichtfelder: \texttt{author, title, type, institution, year/date}\\
Wahlfelder: \texttt{subtitle,
titleaddon, language, number, version, note, loca\-tion, month, isrn, chapter,
pages, pagetotal, addendum, pubstate, doi, eprint, eprintclass, eprinttype, url,
urldate}

\typeitem{set}

Eine Eintragsammlung. Dieser Eintrag Typ ist etwas Besonderes, siehe
\secref{use:use:set} für Details. 

\typeitem{thesis}

Eine Doktorarbeit, die für eine Bildungseinrichtung geschrieben wurde, als
Voraussetzung für einen wissenschaftlichen Grad. Verwenden Sie das Feld
\bibfield{type}, um die Art der Arbeit angeben. 

Pflichtfelder: \texttt{author, title, type, institution, year/date}\\
Wahlfelder: \texttt{subtitle,
titleaddon, language, note, location, month, chapter, pages, pagetotal,
addendum, pubstate, doi, eprint, eprintclass, eprint\-type, url, urldate}

\typeitem{unpublished}

Eine Arbeit mit einem Autor und einen Titel, die bisher nicht offiziell, ein
Manuskript oder das Skript eines Vortrags, veröffentlicht wurde. Verwenden Sie
die Felder \bibfield{howpublished} und \bibfield{note}, um zusätzliche Angaben
zu machen. 

Pflichtfelder: \texttt{author, title, year/date}\\ Wahlfelder: \texttt{subtitle, 
titleaddon, language,
howpublished, note, location, date, month, year, addendum, pubstate, url,
urldate}

\typeitem{xdata}

Dieser Eintragstyp ist speziell. \bibtype{xdata}-Einträge 
enthalten Daten,  die von anderen Einträgen mit dem Feld \bibfield{xdata} 
geerbt werden können.  Einträge diesen Typs dienen nur als Datencontainer; 
sie können nicht in der Bibliografie zitiert oder eingefügt werden. 
Sehen Sie in  \secref{use:use:xdat} weitere Einzelheiten.

\typeitem{custom[a--f]}

Benutzerdefinierte Typen für spezielle Bibliografiestile. Diese werden nicht
von den Standardstilen verwendet. 

\end{typelist}

\subsubsection{Typ-Aliasnamen} \label{bib:typ:als}

Die Eintragsarten, die in diesem Abschnitt aufgeführt werden, sind für
Abwärtskompatibilität mit herkömmlichen \bibtex-Stiles bereitgestellt. Diese
Aliase werden von \bibtex aufgelöst, wenn die Daten exportiert werden. Die
Bibliografiestile verweisen mit ihren Einträgen auf den Alias und nicht auf den
Aliasnamen. Alle unbekannten Eintragsarten sind in der Regel als \bibtype{misc}
exportiert. 

\begin{typelist}

\typeitem{conference}
Ein älteres Alias \bibtype{inproceedings}.

\typeitem{electronic}
Ein Alias \bibtype{online}.

\typeitem{mastersthesis}
Ähnlich wie \bibtype{thesis}, außer dass das Feld \bibfield{type} optional ist
und standardmäßig auf den lokalisierten Begriff <Master's thesis> gesetzt ist.
Sie können immer noch das Feld \bibfield{type} nutzen, um dies zu überschreiben.

\typeitem{phdthesis}
Ähnlich wie \bibtype{thesis}, außer dass das Feld \bibfield{type} optional ist
und standardmäßig auf den lokalisierten Begriff <PhD thesis> gesetzt ist. Sie
können immer noch das Feld \bibfield{type} nutzen, um dies zu überschreiben.

\typeitem{techreport}
Ähnlich wie \bibtype{report} mit der Ausnahme, dass das Feld \bibfield{type}
optional ist und standardmäßig auf den lokalisierten Begriff <technical report>
gesetzt ist. Sie können immer noch das Feld \bibfield{type} zu überschreiben. 

\typeitem{www}
Ein Alias für \bibtype{online}, für \sty{jurabib}-Kompatibilität bereitgestellt. 

\end{typelist}

\subsubsection{Nicht unterstützte Typen} \label{bib:typ:ctm}

Die Typen in diesem Abschnitt sind ähnlich wie die benutzerdefinierte Typen
\bibtype{customa[a--f]} bis \bibtype{customf}, d.\,h., die Standardbibliografiestile bieten keine Unterstützung für diese Typen. Bei der Verwendung von
Standardstilen werden sie als \bibtype{misc}-Einträge behandelt.

\begin{typelist}

\typeitem{artwork}

Werke der bildenden Kunst wie Gemälde, Skulpturen und Installationen. 

\typeitem{audio}

Audio-Aufnahmen, in der Regel auf Audio-\acr{CD}, \acr{DVD}, Audio-Kassette oder
ähnlichen Medien. Siehe auch \bibtype{music}.

\typeitem{bibnote}

Ein spezieller Eingabetyp, der nicht wie andere in \file{bib}-Dateien verwendet
wird. Er ist vorgesehenen für Pakete von Drittanbietern wie \sty{notes2bib},
welche Anmerkungen zu der Bibliografie hinzufügen. Die Anmerkungen sollen in
das Feld \bibfield{note}. Es wird darauf hingewiesen, dass der
\bibtype{bibnote}-Typ in keinem Zusammenhang zum \cmd{defbibnote}-Befehl steht.
\cmd{defbibnote} ist für das Hinzufügen von Kommentaren zu Beginn oder am Ende
der Bibliografie, während der \bibtype{bibnote}-Typ für Pakete gedacht ist, die
Anmerkungen zu den Bibliografie-Einträgen hinzufügen. 

\typeitem{commentary}

Kommentare, die einen anderen Status als normale Bücher haben, solche wie 
Rechtskommentare.

\typeitem{image}

Bilder, Gemälde, Fotografien und ähnlichen Medien. 

\typeitem{jurisdiction}

Gerichtsurteile, gerichtliche Aufnahmen und ähnliche Dinge. 

\typeitem{legislation}

Gesetze, Gesetzesvorlage, Legislativvorschläge und ähnliche Dinge. 

\typeitem{legal}

Juristische Dokumente wie Verträge.

\typeitem{letter}

Persönliche Korrespondenz wie Briefe, E-Mails, Memoiren, etc. 

\typeitem{movie}

Bewegte Bilder. Siehe auch \bibtype{video}.

\typeitem{music}

Musikalische Aufnahmen. Dies ist eine speziellere Variante von \bibtype{audio}.

\typeitem{performance}

Musik- und Theateraufführungen sowie andere Werke der darstellenden Künste.
Dieser Typ bezieht sich speziell auf die Veranstaltung als auf eine Aufnahme, eine
Partitur oder eine gedruckte Aufführung. 

\typeitem{review}

Kritik zu anderen Arbeiten. Dies ist eine spezielle Variante des
\bibtype{article}-Types. Die Standardstile behandeln diesen Eintrag wie
\bibtype{article}.

\typeitem{software}

Computersoftware. 

\typeitem{standard}

Nationale und internationale Standards durch ein zuständiges Gremium wie
die
Internationale Organisation für Standardisierung. 

\typeitem{video}

Audiovisuelle Aufzeichnungen, in der Regel auf \acr{DVD}, \acr{VHS}-Kassette
oder ähnlichen Medien. Siehe auch \bibtype{movie}. 

\end{typelist}

\subsection{Eingabefelder} \label{bib:fld}

Dieser Abschnitt gibt einen Überblick über die Eingabefelder, die vom
\biblatex-Standarddatenmodell unterstützt werden. Sehen Sie in \secref{bib:fld:typ} 
für eine
Einführung in die Datentypen, die von diesem Paket unterstützt werden und
\secref{bib:fld:dat, bib:fld:spc} für die konkrete Felderauflistung.

\subsubsection{Datentypen} \label{bib:fld:typ}

In Datenquellen wie einer \file{bib}-Datei werden alle bibliografischen Daten in
Feldern angegeben.
Einige dieser Felder, zum Beispiel \bibfield{author} und \bibfield{editor},
können eine Liste von Elementen enthalten. Diese Listenstruktur wird in \bibtex
mit Hilfe des Stichwortes <|and|> erreicht, welches zur Trennung der einzelnen
Elemente in der Liste gebraucht wird. Das \biblatex-Paket beinhaltet drei
verschiedene Datenarten, um bibliografische Daten zu handhaben: Namenslisten,
Wörterlisten und Felder. Es gibt auch mehrere Listen und Feld-Subtypen. Dieser
Abschnitt gibt einen Überblick über die Datentypen, welche durch dieses Paket
unterstützt werden. Siehe \secref{bib:fld:dat, bib:fld:spc} für Informationen
über die Umwandlung von \bibtex-Feldern in \biblatex-Datentypen. 

\begin{description}

\item[Namenslisten] werden analysiert und in die einzelnen Positionenen und das
Bindewort \texttt{and} gespalten. Jedes Element in der Liste wird dann in vier
Namensbestandteile zerlegt: Vorname, Namenspräfix (von, van, of, da, de, della,
\dots), Nachnamen und Namenssuffix (Junior, Senior, \dots). Namentliche Listen
können in der \file{bib}-Datei mit dem Schlüsselwort <\texttt{and others}>
gekürzt werden. Typische Beispiele von Namenslisten sind \bibfield{author} und
\bibfield{editor}.

Namenslistenfelder werden nach dem  \cmd{ifuse*}-Test 
in dem Standardmodell automatisch erstellt (\secref{aut:aux:tst}). Es wird auch 
automatisch eine \opt{ifuse*}-Option geschaffen, die die Kennzeichnung und die Namenssortierung steuert (\secref{use:opt:bib:hyb}). \biber unterstützt
eine anpassbare Liste der Namensteile, aber aktuell ist diese von den gleichen Elementen bestimmt wie bei \bibtex:

\begin{itemize}
\item Family name (Familienname) (auch erkenbar als <letzter> Teil)
\item Given name (Vorname) (auch erkenbar als <erster> Teil)
\item Namenpräfix (auch erkenbar als <von> Teil)
\item Namensuffix (auch erkenbar als <Jr> Teil)
\end{itemize}

Die unterstützten Namensteile der Liste werden als eine konstante Liste in dem 
Standarddatenmodell mit dem Befehl \cmd{DeclareDatamodelConstant} definiert
(\ref{aut:ctm:dm}). Allerdings, es ist nicht ausreichend, einfach in diese Liste andere Namensteile zur Unterstützung hinzuzufügen. In der Regel sind sie hart codiert in den Bibliografietreibern und in die Backendverarbeitung. Diese Konstante
wird so viel wie möglich verwendet und ist als Grundlage für künftige 
Verallgemeinerungen und Erweiterungen vorgesehen.

\item[Wörterlisten] werden analysiert und in die einzelnen Positionen und das
Bindewort \texttt{and} aufgespalten, aber nicht weiter zerlegt. Wörtliche Listen
können in der \file{bib}-Datei mit dem Stichwort <\texttt{and others}> gekürzt
werden. Es gibt zwei Unterarten: 

\begin{description}

\item[Wörterlisten] im engeren Sinne werden, wie oben beschrieben, behandelt. Die
einzelnen Begriffe werden einfach ausgegeben. Typische Beispiele für solche
wörtliche Listen sind \bibfield{publisher} und \bibfield{location}. 

\item[Schlüssellisten] sind eine Variante der Wörterlisten, die druckfähige
Daten oder Lokalisierungsschlüssel enthalten können. Für jedes Element in der
Liste, wird ein Test durchgeführt, um festzustellen, ob es ein bekannter
Lokalisationsschlüssel ist (die standardmäßigen Lokalisationsschlüssel sind in
\secref{aut:lng:key} aufgeführt). Wenn ja, wird die lokalisierte Zeichenfolge
dargestellt. Wenn nicht, wird das Element ausgegeben. Ein typisches Beispiel für
eine Schlüsselliste ist das Element \bibfield{language}. 

\end{description} 
\end{description}

\begin{description}

\item[Felder]  werden in der Regel als Ganzes ausgegeben. Es gibt mehrere
Subtypen: 

\begin[align=left]{description}
Wortwörtliche Felder werden ausgegeben, wie sie sind. Typische Beispiele für
Wortfelder sind \bibfield{title} und \bibfield{note}.

\item[Range fields] Bereichsfelder bestehen aus einem oder mehreren Bereichen, in
denen alle Gedankenstriche normiert sind durch den Befehl 
 \cmd{bibrangedash} ersetzt werden. Ein Bereich ist etwas, dem gegebenenfalls ein oder mehrere Bindestriche folgen, bei einigen keiner (bspw. \texttt{5--7}). 
Eine beliebige
Anzahl von aufeinander folgenden Bindestrichen ergibt nun nur einen einzigen
Bindestrich. Ein typisches Beispiel für einen Felderbereich ist das Feld
\bibfield{pages}. Beachten Sie auch den \cmd{bibrangessep}-Befehl, der verwendet werden kann,
das Trennzeichen zwischen mehreren Bereichen anzupassen. Bereichsfelder werden übersprungen
und es wird eine Warnung ausgegeben, wenn sie nicht aus einem oder mehreren Bereichen bestehen.
Sie können "`unordentliche"' Bereichsfelder mit \cmd{DeclareSourcemap} normalisieren, bevor sie analysiert werden (sehen Sie § 4.5.3 (engl. Version)). %\secref{aut:ctm:map}). 

\item[Integer fields] beinhalten unformatiert Zahlen, die in Ordnungszahlen
oder Zeichenfolgen umgewandelt werden können, falls sie ausgegeben werden. Ein
typisches Beispiel ist das Feld \bibfield{extrayear} oder \bibfield{volume}. Solche Felder werden als ganze zahlen sortiert. \biber unternimmt einen (sehr seriösen) Aufwand, um nichtarabische
Darstellungen (römische Ziffern zum Beispiel) auf ganze Zahlen für Sortierzwecke abzubilden.

\item[Datepart fields] beinhalten unformatierte ganze Zahlen, die zu Ordnungszahlen oder Strings umgewandelt werden können. Ein typisches Beispiel ist das Feld \bibfield{month}.
Für jedes Feld X vom Datentyp \bibfield{date} im Datenmodell werden Datenteil-Felder automatisch
erstellt mit den den folgenden Namen:

\bibfield{$<$datetype$>$year}, \bibfield{$<$datetype$>$endyear}, \bibfield{$<$datetype$>$month}, \bibfield{$<$datetype$>$endmonth}, \bibfield{$<$datetype$>$day}, \bibfield{$<$datetype$>$endday}, \bibfield{$<$datetype$>$hour}, \bibfield{$<$datetype$>$endhour}, \bibfield{$<$datetype$>$minute}, \bibfield{$<$datetype$>$endminute}, \bibfield{$<$datetype$>$second}, \bibfield{$<$datetype$>$endsecond}, \bibfield{$<$datetype$>$timezone}, \bibfield{$<$datetype$>$endtimezone}.

\item[Date fields] beinhalten eine Datumsangabe im Format~\texttt{yyyy-mm-ddThh:nn[+
|-][hh[:nn]|Z]} oder einen Datumsbereich im Format
\texttt{yyyy-mm-ddThh:nn[+|-][hh[:nn]|Z]/yyyy-mm-ddThh:nn[+|-][hh[:nn]|Z]} und andere
Formate, erlaubt durch die EDTF-Stufe 1, sehen Sie \secref{bib:use:dat}. 
Datumsfelder sind in der Hinsicht speziell, dass sie analysiert und in ihre Komponenten
aufgespalten werden. Die \bibfield{datepart}-Komponenten (sehen Sie oben) werden automatisch definiert und erfasst, wenn ein Feld vom Datentyp \bibfield{date} im Datenmodell definiert ist.
Ein typisches Beispiel ist \bibfield{date}.

\item[Verbatim fields] sind im wörtlichen Modus und können Sonderzeichen enthalten.
Typische Beispiele für solche Felder sind \bibfield{url} und \bibfield{doi}.

\item[URI fields] werden im Verbatim-Modus verarbeitet und enthalten Sonderzeichen.
Sie sind auch "`URL-escaped"', wenn sie nicht aussehen, wie sie es ohnehin schon tun.
Das typische Beispiel eines uri-Felds ist \bibfield{url}.

\item[Separated value fields] Eine getrennte Liste von wörtlichen Werten.
Beispiele sind die Felder \bibfield{keywords} und \bibfield{options}.
Der Trenner kann konfiguriert sein mit jedem regulären Perl-Ausdruck
über die \opt{xsvsep}-Option, standardmäßig die üblichen \bibtex
Kommas (optional von Leerzeichen umgeben).

\item[Pattern fields] Wörtliches Felder, die mit einem bestimmten Muster übereinstimmen müssen. Ein Beispiel ist das Feld aus \secref{bib:fld:spc}.

\item[Key fields] Schlüsselfelder enthalten druckbare Daten oder Lokalisierungsschlüssel.
Es wird ein Test durchgeführt, um festzustellen, ob der Wert des Feldes ein
bekannter Lokalisierungsschlüssel ist (die standardmäßig definierten
Lokalisierungsschlüssel werden in \secref{aut:lng:key} aufgeführt). Wenn ja,
wird die lokalisierte Zeichenfolge gedruckt. Wenn ja, wird die lokalisierte Zeichenfolge 
ausgegeben. Wenn nicht, wird der Wert
ausgegeben. Ein typisches Beispiel ist das Feld \bibfield{type}.

\item[Code fields] Unterdrücken \tex-Code.

\end{description} 
\end{description}


\subsubsection{Datenfelder} \label{bib:fld:dat}

Die Felder in diesem Abschnitt sind diejenigen, die regulär druckfähige Daten
enthalten. Der Name auf der linken Seite ist der Name des Feldes, der von
\biblatex und dessen Backend verwendet wird. Der \biblatex-Datentyp ist
rechts neben dem Namen angegeben. Siehe \secref{bib:fld:typ} für Erläuterungen zu
der verschiedenen Datentypen.

Einige Felder werden als <Label-Felder> markiert; dies meint,
dass sie oft als Abkürzungsetiketten beim Drucken von Bibliografielisten, im Sinne von 
Abschnitt \secref{use:bib:biblist}, verwendet werden. \biblatex erstellt automatisch die Unterstützung von Makros für solche Felder. Sehen Sie \secref{use:bib:biblist}.

\begin{fieldlist}

\fielditem{abstract}{literal}

Dieses Feld ist für die Aufnahme von Zusammenfassungen in eine \file{bib}-Datei,
welche durch einen speziellen Bibliografiestil ausgegeben wird. Es wird nicht
von allen Standardbibliografie-Designs verwendet.

\fielditem{addendum}{literal}

Verschiedene bibliografische Daten, die am Ende des Eintrags gedruckt werden.
Dies ist ähnlich dem \bibfield{note}-Feld, außer dass es am Ende des
Bibliografieeintrages gedruckt wird.

\listitem{afterword}{Name}

Der/Die Autor/en von einem Nachwort zu einer Arbeit. Wenn der Autor des Nachwort
mit dem \bibfield{editor} und\slash oder \bibfield{translator} übereinstimmt,
verketten die Standard-Stile diese Felder in der Bibliografie automatisch.
Siehe auch \bibfield{introduction} und \bibfield{foreword}.

\fielditem{annotation}{literal}

Dieses Feld kann nützlich sein bei der Umsetzung eines Stils für kommentierte
Bibliografien. Es wird nicht von allen Standardbibliografie-Designs
verwendet. Beachten Sie, dass dieses Feld nichts mit \bibfield{annotator} zu tun
hat. Der \bibfield{annotator} ist der Autor von Anmerkungen, welche Teile der
Arbeit zitieren.

\listitem{annotator}{Name}

Der/Die Autor/en der Anmerkungen zum Werk. Wenn der Kommentator mit dem
\bibfield{editor} und\slash oder \bibfield{translator} übereinstimmt, verketten
die Standardstile diese Felder in der Bibliografie automatisch. Siehe auch
\bibfield{commentator}.

\listitem{author}{Name}

Der/Die Autor/en des \bibfield{title}.

\fielditem{authortype}{Schlüssel}

Die Art des Autors. Dieses Feld wirkt sich auf die Zeichenfolge (sofern
vorhanden) aus, um den  Autor einzuführen. Wird nicht von den 
Standardibliografie-Designs verwendet.

\listitem{bookauthor}{Name}

Der/Die Autor/en des \bibfield{booktitle}.

\fielditem{bookpagination}{Schlüssel}

Wenn die Arbeit als Teil einer anderen veröffentlicht wird, ist dies das
Seitennummerierungsschemas der Arbeit, d.\,h. \bibfield{bookpagination} gehört
zu \bibfield{pagination} wie \bibfield{booktitle} zu \bibfield{title}. Der Wert
dieses Feldes beeinflusst die Formatierung der Felder \bibfield{pages} und
\bibfield{pagetotal}. Der Schlüssel sollte im Singular angegeben werden.
Mögliche Schlüssel sind \texttt{page}, \texttt{column}, \texttt{line},
\texttt{verse}, \texttt{section} und \texttt{paragraph}. 
Siehe auch \bibfield{pagination} sowie \secref{bib:use:pag}.

\fielditem{booksubtitle}{literal}

Der Untertitel, der zu dem \bibfield{booktitle} gehört. Wenn das
\bibfield{subtitle}-Feld auf eine Arbeit verweist, die Teil einer größeren
Veröffentlichung ist, ist ein möglicher Untertitel der Hauptarbeit in diesem
Feld anzugeben. Siehe auch \bibfield{subtitle}.

\fielditem{booktitle}{literal}

Wenn das Feld \bibfield{title} den Titel einer Arbeit enthält, die Teil einer
größeren Publikation ist, wird der Titel des Hauptwerkes in diesem Feld
angegeben. Siehe auch \bibfield{title}.

\fielditem{booktitleaddon}{literal}

Ein Anhang zum \bibfield{booktitle}, der in einer anderen Schrift dargestellt
wird.

\fielditem{chapter}{literal}

Ein Kapitel oder Abschnitt oder eine andere Einheit eines Werkes.

\listitem{commentator}{Name}

Der/Die Autor/en eines Kommentars über die Arbeit. Beachte, dass dieses Feld für
kommentierte Ausgaben, die einen Kommentator zusätzlich zu dem Autor haben,
bestimmt ist. Wenn die Arbeit ein eigenständiger Kommentar ist, sollte der
Kommentator im Feld \bibfield{author} angegeben werden. Wenn der Kommentator mit
dem \bibfield{editor} und\slash oder \bibfield{translator} übereinstimmt,
verketten die Standard-Stile diese Felder in der Bibliografie automatisch.
Siehe auch \bibfield{annotator}.

\fielditem{date}{Datum}

Zeitpunkt der Publikation. Siehe auch \bibfield{month} und \bibfield{year},
sowie \secref{bib:use:dat}.

\fielditem{doi}{verbatim}

Der digitale Objektbezeichner des Werkes.

\fielditem{edition}{Integer oder literal}

Die Ausgabe einer gedruckten Publikation. Dies muss eine ganze Zahl und nicht
eine Ordnungszahl sein. Nicht |edition={First}| oder |edition={1st}|, sondern
|edition={1}|. Die Bibliografie-Stile wandeln diese in Ordnungszahlen um. Es
ist auch möglich, die Ausgabe als Zeichenkette anzugeben, zum Beispiel «Dritte,
überarbeitete und erweiterte Auflage».

\listitem{editor}{Name}

Der/Die Editor/en von \bibfield{title}, \bibfield{booktitle} oder
\bibfield{maintitle}, je nach Eingabetyp Typ. Verwenden Sie das Feld
\bibfield{editortype}, um anzugeben, wenn es verschieden von <\texttt{editor}>
ist. Siehe \secref{bib:use:edr} für weitere Hinweise.

\listitem{editora}{Name}

Ein zweiter Herausgeber, der eine andere redaktionelle Rolle innehat, wie die
Erstellung, Schwärzen, etc. Nutzen Sie das \bibfield{editoratype}-Feld, um dies
zu spezifizieren. Siehe \secref{bib:use:edr} für weitere Hinweise.

\listitem{editorb}{Name}

Ein weiterer sekundärer Herausgeber, der eine weitere redaktionelle Rolle
innehat. Verwenden Sie das Feld \bibfield{editorbtype}, um dies anzugeben. Siehe
\secref{bib:use:edr} für weitere Hinweise.

\listitem{editorc}{Name}

Ein weiterer sekundärer Herausgeber, der eine weitere redaktionelle Rolle
innehat. Verwenden Sie das Feld \bibfield{editorctype}, um dies anzugeben. Siehe
\secref{bib:use:edr} für weitere Hinweise.

\fielditem{editortype}{Schlüssel}

Die Art der redaktionellen Rolle, die der \bibfield{editor} hat. Rollen, die
standardmäßig unterstützt werden, sind \texttt{editor}, \texttt{compiler},
\texttt{founder}, \texttt{continuator}, \texttt{redactor},
\texttt{colla\-borator}. Die Rolle <\texttt{editor}> ist die Standardeinstellung.
In diesem Fall kann das Feld weggelassen werden. Siehe \secref{bib:use:edr} für
weitere Hinweise.

\fielditem{editoratype}{Schlüssel}

Ähnlich wie \bibfield{editortype}, aber auf \bibfield{editora} bezogen. Siehe
\secref{bib:use:edr} für weitere Hinweise.

\fielditem{editorbtype}{Schlüssel}

Ähnlich wie \bibfield{editortype}, aber auf \bibfield{editorb} bezogen. Siehe
\secref{bib:use:edr} für weitere Hinweise.

\fielditem{editorctype}{Schlüssel}

Ähnlich wie \bibfield{editortype}, aber die \bibfield{editorc} bezogen. Siehe
\secref{bib:use:edr} für weitere Hinweise.

\fielditem{eid}{literal}

Die elektronische Kennzeichnung eines \bibtype{article}.

\fielditem{entrysubtype}{literal}

Dieses Feld, das nicht von den Standardstilen verwendet wird, kann für das Spezifizieren eines Sybtyps eines Eintragstyps genommen werden. Dies kann für Bibliografiestile, die
feinkörnigere Gruppe von Eintragstypen unterstützen, verwendet werden.

\fielditem{eprint}{verbatim}

Die elektronische Kennzeichnung einer Online-Publikation. Diese ist in etwa vergleichbar
mit \acr{doi}, jedoch spezifisch für ein bestimmtes Archiv, Repositorie, Service
oder System. Sehen Sie \secref{use:use:epr} zu weiteren Details. Sehen Sie auch zu den
Feldern \bibfield{eprinttype} und \bibfield{eprintclass}.

\fielditem{eprintclass}{literal}

Weitere Informationen, die durch
\bibfield{eprinttype} gegeben sind. Diese könnten ein Teil eines Archivs, ein
Dateipfad, oder eine Klassifizierung etc. sein. Siehe \secref{use:use:epr} für
Details. Siehe auch \bibfield{eprint} und \bibfield{eprinttype}.

\fielditem{eprinttype}{literal}

Die Art der \bibfield{eprint}-Kennung, d.\,h., der Name des Archivs, einer Quelle,
Services  oder des System, zu dem \bibfield{eprint} gehört. Siehe
\secref{use:use:epr} für Details. Siehe auch \bibfield{eprint} und
\bibfield{eprintclass}.

\fielditem{eventdate}{Datum}

Das Datum der Konferenz, eines Symposiums oder einer anderen Veranstaltung, die
in \bibtype{proceedings} und \bibtype{inproceedings} eingetragen ist. Dieses
Feld kann auch für die benutzerdefinierte Typen, die in \secref{bib:typ:ctm}
aufgelistet sind, nützlich sein. Siehe auch \bibfield{eventtitle} und
\bibfield{venue}, sowie auch \secref{bib:use:dat}.

\fielditem{eventtitle}{literal}

Der Titel der Konferenz, eines Symposiums oder einer anderen Veranstaltung, die
in \bibtype{proceedings} und \bibtype{inproceedings} eingetragen ist. Dieses
Feld kann auch für die benutzerdefinierten Typen, die in \secref{bib:typ:ctm}
aufgelistet sind, nützlich sein. Beachten Sie, dass dieses Feld den
schlichten/nüchtern Titel der Veranstaltung enthält. Dinge wie «Verfahren der
Fünften XYZ-Konferenz» werden in \bibfield{titleaddon} oder
\bibfield{booktitleaddon} eingetragen. Siehe auch \bibfield{eventdate} und
\bibfield{venue}.

\fielditem{file}{literal}

Ein lokaler Link zu einer \acr{pdf}- oder anderen Version des Werkes. Wird nicht
von den Standardbibliografien verwendet.

\listitem{foreword}{Name} 

Der/Die Autor/en eines Vorwortes zum Werk. Wenn der
Autor des Vorwortes mit dem \bibfield{editor} und\slash oder
\bibfield{translator} übereinstimmt, verketten die Standardstile diese Felder
in der Bibliografie automatisch. Siehe auch \bibfield{introduction} und
\bibfield{afterword}.

\listitem{holder}{Name}

Der/Die Inhaber eines Patents ( \bibtype{patent}), falls verschieden von \bibfield{author}.
Die kooperierenden Inhaber müssen nicht zusätzlich in Klammern gesetzt werden,
siehe \secref{bib:use:inc}  für weitere Einzelheiten. Diese Liste kann auch für
die benutzerdefinierte Typen aus \secref{bib:typ:ctm} nützlich sein.

\fielditem{howpublished}{literal}

Eine veröffentlichte Bekanntmachung für eine außergewöhnliche Publikation, die
in keine der üblichen Kategorien passt.

\fielditem{indextitle}{literal} 

Ein Titel für eine Indizierung anstelle des
regulären \bibfield{title}-Feldes. Dieses Feld kann nützlich sein, wenn Sie
möchten, das ein Eintrag wie «Eine Einführung in die \dots» als «Einführung in
die \dots, Eine» indiziert wird. Stilautoren sollten beachten, dass \biblatex
automatisch den Wert von \bibfield{title} in \bibfield{indextitle} kopiert, wenn
letzteres undefiniert ist.

\listitem{institution}{literal}

Der Name der Universität oder einer anderen Institution, je nach Art des
Eintrags. Herkömmlich benutzt \bibtex dafür das Feldes \bibfield{school},
welches als Alias unterstützt wird. Siehe auch \secref{bib:fld:als,
bib:use:and}.

\listitem{introduction}{Name}

Der/Die Autor/en der Einführung in die Arbeit. Wenn der Autor der Einführung mit
\bibfield{editor} und\slash oder \bibfield{translator} übereinstimmt, verketten
die Standardstile diese Felder in der Bibliografie automatisch. Siehe auch
\bibfield{foreword} und \bibfield{afterword}.

\fielditem{isan}{literal}

Die International Standard Audiovisual Number eines audiovisuellen Werkes. Wird
nicht von den Standardbibliografiestilen verwendet.

\fielditem{isbn}{literal}

Die International Standard Book Number eines Buches.

\fielditem{ismn}{literal}

Die International Standard Music Number für gedruckte Musik wie Notenbücher.
Wird nicht von den Standardbibliografiestilen  verwendet.

\fielditem{isrn}{literal}

Die International Standard Technical Report Number eines technischen Berichts.

\fielditem{issn}{literal}

Die International Standard Serial Number einer Zeitschrift.

\fielditem{issue}{literal}

Die Ausgabe einer Zeitschrift. Dieses Feld ist für Zeitschriften, deren einzelne
Ausgaben durch Bezeichnung wie <Frühling> oder <Sommer> anstatt des Monats oder
einer Nummer identifiziert werden. Da die Verwendung von \bibfield{issue}
ähnlich zu \bibfield{month} und \bibfield{number} ist, kann dieses Feld auch für
Doppelausgaben und andere Sonderfälle sinnvoll sein. Siehe auch
\bibfield{month}, \bibfield{number} und \secref{bib:use:iss}.

\fielditem{issuesubtitle}{literal}

Der Untertitel einer bestimmten Ausgabe einer Zeitschrift oder einer anderen
Zeitung.

\fielditem{issuetitle}{literal}

Der Titel einer bestimmten Ausgabe einer Zeitschrift oder einer anderen Zeitung.

\fielditem{iswc}{literal}

Der Internationale Standard Work Code eines musikalischen Werkes. Wird  nicht
von den Standradbibliografiestilen  verwendet.

\fielditem{journalsubtitle}{literal}

Der Untertitel eines Journals, einer Zeitung oder einer anderen Zeitschrift.

\fielditem{journaltitle}{literal}

Der Name eines Journals, einer Zeitung oder einer anderen Zeitschrift.

\fielditem{label}{literal}

Eine Bezeichnung, die bei den Zitierstilen als Ersatz für standardmäßige Labels
verwendet wird, wenn alle erforderlichen Daten zur Generierung des
standardmäßigen Labels fehlen. Zum Beispiel, wenn eine Autor-Jahr-Zitierweise
generiert wird, der Autor oder das Jahr fehlt, kann es in \bibfield{label}
angegeben werden. Siehe \secref{bib:use:key} für weitere Einzelheiten. Beachten
Sie, dass im Gegensatz zu \bibfield{shorthand}, \bibfield{label} als Absicherung
verwendet wird. Siehe auch \bibfield{shorthand}.

\listitem{language}{key}

Die Sprache/n des Werkes. Sprachen können wörtlich oder als
Lokalisierungsschlüssel angegeben werden. Wenn Lokalisierungsschlüssel verwendet
werden, ist das Präfix \texttt{lang} wegzulassen. Siehe auch
\bibfield{origlanguage} und \bibfield{hyphenation} in \secref{bib:fld:spc}.

\fielditem{library}{literal}

Dieses Feld kann nützlich sein, um Informationen über beispielsweise den Namen
einer Bibliothek und eine Signatur anzugeben. Dies kann durch einen speziellen
Bibliografiestil erreicht werden, falls gewünscht. Wird nicht von den
Standardbibliografiestilen  verwendet.

\listitem{location}{literal}

Der/Die Ort/e der Veröffentlichung, d.\,h. der Standort des \bibfield{publisher}
oder der \bibfield{institution}, je nach Art des Eingabetyps. Herkömmlich
benutzt \bibtex das Feld \bibfield{address}, welches ähnlich genutzt wird. Siehe
auch \secref{bib:fld:als, bib:use:and}. Mit \bibtype{patent}-Einträgen zeigt
diese Liste den Umfang eines Patents und wird als ein Schlüsselliste behandelt.
Diese Liste kann auch für die benutzerdefinierte Typen aus \secref{bib:typ:ctm}
nützlich sein.

\fielditem{mainsubtitle}{literal}

Der Untertitel, der zum \bibfield{maintitle} gehört. Siehe auch
\bibfield{subtitle}.

\fielditem{maintitle}{literal}

Die Haupttitel eines mehrbändigen Buches, wie bei \emph{Gesammelte Werke}.
Wenn \bibfield{title} oder \bibfield{booktitle} den Titel eines einzigen Bandes,
welches Teil eines mehrbändigen Buches ist, ist der Titel der gesamten Arbeit in
diesem Feld enthalten.

\fielditem{maintitleaddon}{literal}

Ein Anhang zu \bibfield{maintitle}, der in einer anderen Schrift ausgegeben
wird.

\fielditem{month}{integer}

Der Monat der Veröffentlichung. Dies muss eine ganze Zahl und nicht eine
Ordnungszahl oder eine Zeichenkette sein. Nicht |month={January}|, sondern
|month={1}|. Die Bibliografie-Stile wandelt dies, falls erforderlich, in eine
Zeichenkette oder Ordnungszahl, der jeweiligen Sprache entsprechend, um. Siehe
auch \bibfield{date} sowie \secref{bib:use:iss, bib:use:dat}.

\fielditem{nameaddon}{literal}

Ein Zusatz direkt hinter dem Namen des Autors, der in der Bibliografie
angegeben wird. Wird nicht von den Standardbibliografiestilen  verwendet.
Dieses Feld kann nützlich sein, um ein Aliasnamen oder ein Pseudonym
hinzuzufügen (oder der richtige Namen, wenn das Pseudonym genutzt wird, um sich
auf den Verfasser zu beziehen).

\fielditem{note}{literal}

Verschiedene bibliografische Daten, die nicht in ein anderes Feld passen.
\bibfield{note}  wird verwendet, um bibliografische Daten in einem freien
Format aufzuzeichnen. Fakten über die Veröffentlichung, wie «Nachdruck der
Ausgabe London 1831» sind typische Kandidaten für dieses Feldes. Siehe auch
\bibfield{addendum}.

\fielditem{number}{literal}

Die Nummer einer Zeitschrift oder die Ausgabe\slash Nummer eines Buches aus
einer \bibfield{series}. Siehe auch \bibfield{issue} sowie \secref{bib:use:ser,
bib:use:iss}. Mit \bibtype{patent} ist dies die Anzahl oder
Eintragsummer eines Patents oder von Patentrechteanfragen.

\listitem{organization}{literal}

Die Organisation, die ein \bibtype{manual} oder eine \bibtype{online}-Quelle
veröffentlichte oder eine Konferenz sponserte. Siehe auch \secref{bib:use:and}.

\fielditem{origdate}{date}

Wenn das Werk eine Übersetzung, eine Neuauflage oder etwas Ähnliches ist, dann
das Datum der Veröffentlichung der Originalausgabe. Wird nicht von den
Standardbibliographiestilen verwendet. Siehe auch \bibfield{date}.

\fielditem{origlanguage}{key}

Wenn das Werk eine Übersetzung ist, dann die Sprache des Originals. Siehe auch
\bibfield{language}.

\listitem{origlocation}{literal}

Wenn das Werk eine Übersetzung, eine Neuauflage oder etwas Ähnliches ist, dann
\bibfield{location} der Originalausgabe. Wird nicht von den
Standardbibliografiestilen verwendet. Siehe auch \bibfield{location} und
\secref{bib:use:and}.

\listitem{origpublisher}{literal}

Wenn das Werk eine Übersetzung, eine Neuauflage oder etwas Ähnliches ist, der
\bibfield{publisher} der Originalausgabe. Wird nicht von den
Standardbibliografiestilen verwendet. Siehe auch \bibfield{publisher} und
\secref{bib:use:and}.

\fielditem{origtitle}{literal}

Wenn das Werk eine Übersetzung ist, der \bibfield{title} des Originals. Wird
nicht von den Standardbibliografiestilen verwendet. Siehe auch
\bibfield{title}.

\fielditem{pages}{Bereich}

Eine oder mehrere Seitenzahlen oder Seitenbereiche. Wenn das Werk als Teil eines
anderen veröffentlicht wurde, wie ein Artikel in einer Zeitschrift oder einer
Sammlung, enthält dieses Feld den entsprechenden Seitenbereich der anderen
Arbeit. Es kann auch verwendet werden, um den Bezug auf einen bestimmten Teil
eines Werkes zu begrenzen (beispielsweise ein Kapitel in einem Buch).

\fielditem{pagetotal}{literal}

Die Gesamtzahl der Seiten des Werkes.

\fielditem{pagination}{key}

Die Seitennummerierung der Arbeit. Der Wert dieses Feldes legt die Formatierung
des \prm{postnote}-Argumentes für einen Zitatbefehl fest. Der Schlüssel sollte
im Singular angegeben werden. Mögliche Schlüssel können sein: \texttt{page},
\texttt{column}, \texttt{line}, \texttt{verse}, \texttt{section} und
\texttt{paragraph}. Siehe auch \bibfield{bookpagination} sowie \secref{bib:use:pag,
use:cav:pag}.

\fielditem{part}{literal}

Die Nummer einer Teilausgabe. Dieses Feld gilt nur für Bücher und nicht für
Fachzeitschriften. Es kann verwendet werden, wenn ein logischer Band aus zwei
oder mehreren physischen besteht. In diesem Fall wird die Nummer des logischen
Bandes in \bibfield{volume} geschrieben und die Nummer des Teils in
\bibfield{part}. Siehe auch \bibfield{volume}.

\listitem{publisher}{literal}

Der/Die Name/n des/der Herausgeber. Siehe auch \secref{bib:use:and}.

\fielditem{pubstate}{key}

Der Stand der Veröffentlichung der Arbeiten, z.\,B. <im Druck> oder
<eingereicht> (bei einer Zeitschrift). Bekannte Stadien sind \texttt{inpress}
und \texttt{submitted}.

\fielditem{reprinttitle}{literal}

Der Titel einer Neuauflage des Werkes. Wird nicht von den Standardstilen
verwendet.

\fielditem{series}{literal}

Der Name einer Publikationsreihe, wie «Studien in \dots», oder die Nummer einer
Zeitschriftenreihe. Bücher in einer Publikationsreihe sind in der Regel
nummeriert. Die Nummer oder Ausgabe eines Buches in einer Serie, wird in
\bibfield{number} eingetragen. Beachten Sie, dass auch \bibtype{article} die
Eingabe von \bibfield{series} nutzt, aber es in besonderer Weise handhaben.
Siehe \secref{bib:use:ser} für weitere Einzelheiten.

\listitem{shortauthor}{name}{name\LFMark}

Der/Die Autor/en der Arbeit in abgekürzter Form. Dieses Feld soll vor allem für
Abkürzungen von kooperierenden Autoren genutzt werden, siehe
\secref{bib:use:inc} für weitere Einzelheiten.

\listitem{shorteditor}{name}{name\LFMark}

Der/Die Herausgeber des Werkes, in abgekürzter Form. Dieses Feld soll vor allem
zur Abkürzung von kooperierenden Herausgebern genutzt werden, siehe
\secref{bib:use:inc} für weitere Einzelheiten.

\fielditem{shorthand}{literal\LFMark}

Eine besondere Bezeichnung, die bei der Zitierweise anstelle der üblichen
verwendet wird. Wenn
definiert, überschreibt es das standardmäßige Label.
Sehen Sie auch \bibfield{label}.

\fielditem{shorthandintro}{literal}

Der ausführliche Zitierstil, der mit diesem Paket geliefert wird, benutzt
Sätze,
wie «fortan zitiert als [Kurzform]», um Abkürzungen im ersten Zitat einzuführen.
Wenn der \bibfield{shorthandintro}-Bereich definiert ist, überschreibt es den
Standardsatz. Beachten Sie, dass die alternativen Sätze Abkürzungen enthalten
müssen.

\fielditem{shortjournal}{literal\LFMark}

Eine kurze Version oder eine Abkürzung von \bibfield{journaltitle}. Wird nicht
von den Standardbibliografiestilen verwendet.

\fielditem{shortseries}{literal\LFMark}

Eine kurze Version oder eine Abkürzung von \bibfield{series}. Wird nicht von den
Standardbibliografiestilen verwendet.

\fielditem{shorttitle}{literal\LFMark}

Der Titel in einer verkürzten Form. Dieses Feld ist in der Regel nicht in der
Bibliografie enthalten. Es ist für Zitierungen im Autor"=Titel"=Format
vorgesehen. Falls vorhanden, verwendet der Autor"=Titel"=Zitatstil dieses Feld
anstatt \bibfield{title}.

\fielditem{subtitle}{literal}

Der Untertitel des Werks.

\fielditem{title}{literal}

Der Titel des Werks.

\fielditem{titleaddon}{literal}

Ein Anhang zum \bibfield{title}, der in einer anderen Schrift ausgegeben wird.

\listitem{translator}{name}

Der/Die Übersetzer von \bibfield{title} oder \bibfield{booktitle}, je nach Art
des Eintrags. Wenn der Übersetzer mit dem \bibfield{editor} übereinstimmt,
verketten die Standardstile diese Felder in der Bibliografie automatisch.

\fielditem{type}{key}

Die Art von \bibfield{manual}, \bibfield{patent}, \bibfield{report}, oder
\bibfield{thesis}. Dieses Feld kann auch für die benutzerdefinierte Typen aus
\secref{bib:typ:ctm} nützlich sein.

\fielditem{url}{uri}

Die \acr{URL} einer Online-Publikation. Wenn es kein URL-Escapezeichen (kein <\%> Zeichen),
mit \biber\ ist, wird es als URI-Escape gemäß RFC 3987 sein. Das heisst, sogar 
Unicode-Zeichen werden korrekt maskiert.

\fielditem{urldate}{Datum}

Der Zugriffszeitpunkt auf die Adresse, die im \bibfield{url}-Feld angegeben ist.
Siehe auch \secref{bib:use:dat}.

\fielditem{venue}{literal}

Der Veranstaltungsort einer Konferenz, eines Symposiums oder eine andere
Veranstaltung in \bibtype{proceedings}- und \bibtype{inproceedings}"=Einträgen.
Dieses Feld kann auch für die benutzerdefinierte Typen aus \secref{bib:typ:ctm}
nützlich sein. Beachten Sie, dass \bibfield{location} den Ort der
Veröffentlichung enthält. Es entspricht daher der \bibfield{publisher}- und
\bibfield{institution}-Liste. Der Veranstaltungsort der Veranstaltung wird in
\bibfield{venue} eingetragen. Siehe auch \bibfield{eventdate} und
\bibfield{eventtitle}.

\fielditem{version}{literal}

Die Revisionsnummer einer Software, eines Handbuches, etc.

\fielditem{volume}{integer}

Das Nummer eines mehrbändigen Buches oder eine Zeitschrift. Es wird eine ganze Zahl erwartet, nicht unbedingt in arabischen Ziffern, da \biber automatisch von römischen Ziffern oder arabischen Zahlen
intern auf ganze Zahlen für Sortierzwecke geht. Siehe auch \bibfield{part}.

\fielditem{volumes}{integer}

Die Gesamtzahl der Bände eines mehrbändigen Werkes. Je nach Art des Eintrags
bezieht sich dieses Feld auf \bibfield{title} oder \bibfield{maintitle}.
Es wird eine ganze Zahl erwartet, nicht unbedingt in arabischen Ziffern, da \biber automatisch von römischen Ziffern oder arabischen Zahlen
intern auf ganze Zahlen für Sortierzwecke geht.

\fielditem{year}{literal}

Das Jahr der Veröffentlichung. Es ist besser das \bibfield{date} zu verwenden, da dieses auch mit
"`plain years"' kompatibel ist. Siehe auch \secref{bib:use:dat}.

\end{fieldlist}

\subsubsection{Spezielle Felder} \label{bib:fld:spc}

Die Felder in diesem Abschnitt werden nicht ausgegeben, sondern dienen einem
anderen Zweck. Sie gelten für alle Arten von Einträgen im Standarddatenmodell.

\begin{fieldlist}

\fielditem{crossref}{entry key / Eingabeschlüssel}

Dieses Feld enthält einen Eingabeschlüssel für die \bibtex="Querverweisfunktion.
Es wird intern von \bibtex verwendet. Abhängige
"`Kind"'einträge mit einem \bibfield{crossref}-Feld erben alle Daten von den
Elterneinträgen, die im \bibfield{crossref}-Feld angegeben sind. Wenn die Anzahl
der Kindeinträge auf einen bestimmten Elterneintrag verweist, wird der
Elterneintrag automatisch in die Bibliografie hinzugefügt, auch wenn er nicht
explizit zitiert wurde. Man kann dies über die  Kommandozeile
steuern.\footnote{siehe die \opt{mincrossrefs} package option in
\secref{use:opt:pre:gen}.} Stilautoren sollten beachten, dass das
\bibfield{crossref}-Feld des Kindeintrags auf der \sty{biblatex}-Ebene definiert
wird, abhängig von der Verfügbarkeit der Elterneinträge. Wenn der Elterneintrag
verfügbar ist, wird das \bibfield{crossref}-Feld des Kindeintrages definiert.
Wenn nicht, erben sie die Daten aus den Elterneinträgen, aber ihr
\bibfield{crossref} Feld bleibt undefiniert. Ob der Elterneintrag implizit oder
explizit in die Bibliografie aufgenommen wurde, spielt keine Rolle. Siehe auch
\bibfield{xref} in this section as well as \secref{bib:cav:ref}.

\fielditem{entryset}{separated values}

Dieses Feld ist speziell für Eintragssätze.
Siehe \secref{use:use:set} für Details. Dieses feld wird für die Backend-Verarbeitung 
gebraucht und erscheint nicht in der \path{.bbl}. 

\fielditem{execute}{code}

Ein spezielles Feld, welches beliebige \tex-Codes enthält, die ausgeführt werden
können, wenn auf die Daten von den entsprechenden Einträgen zugegriffen wird.
Dies kann nützlich sein, um spezielle Fälle zu behandeln. Dieses Feld ist
vergleichbar mit den Schnittstellen \cmd{AtEveryBibitem}, \cmd{AtEveryLositem}
und \cmd{AtEveryCitekey} aus \secref{aut:fmt:hok}, außer dass es auf einer
pro-Eintrag Grundlage in der \file{bib}-Datei definierbar ist. Jeder Code dieses
Felds wird direkt automatisch nach den Hookes ausgeführt.

\fielditem{gender}{Pattern matching one of: \opt{sf}, \opt{sm}, \opt{sn}, \opt{pf}, \opt{pm}, \opt{pn},
\opt{pp}}

Das Geschlecht des Autors oder das Geschlecht des Herausgebers, wenn es keinen
Autor gibt. Die folgenden Kennungen werden unterstützt: \opt{sf} (femininer
Singular, ein einzelner weiblicher Name), \opt{sm} (maskuliner Singular, ein
einzelner männlicher Name), \opt{sn} (Neutrum Singular, ein einzelner neutraler
Name),  \opt{pf} (femininer Plural, mehrere weibliche Namen), \opt{pm}
(maskuliner Plural, mehrere männliche Namen), \opt{pn} (Neutrum Plural, mehrere
neutrale Namen), \opt{pp} (Plural, mehrere Namen unterschiedlichen Geschlechts).
Diese Information ist nur für spezielle Bibliografie- und Zitierstile
erforderlich und auch nur in bestimmten Sprachen. Zum Beispiel kann ein
Zitatstil sich wiederholender Autorennamen mit einem Begriff wie <idem>
ersetzen. Wenn das lateinische Wort verwendet wird, so wie es im englischen und
französischem Brauch ist, muss das Geschlecht nicht angegeben werden. In
deutschen Veröffentlichungen werden jedoch solche Schlüsselbegriffe in der Regel
in Deutsch gegeben und sind in diesem Fall geschlechtsspezifisch.


\begin{table} \tablesetup \centering
\begin{tabularx}{\textwidth}{@{}p{100pt}@{}p{100pt}@{}p{110pt}@{}} %\toprule
\multicolumn{1}{@{}H}{Sprache} & \multicolumn{1}{@{}H}{Region/Dialekt} &
\multicolumn{1}{@{}H}{Babel Identifiers} \\
\cmidrule(r){1-1}\cmidrule(r){2-2}\cmidrule{3-3} 
Bulgarisch    & Bulgarien       & \opt{bulgarian} \\
Katalanisch      & Spanien, Frankreich, Andorra, Italien & \opt{catalan} \\
Kroatisch     & Kroatien, Bosnien und Herzegovina, Serbien & \opt{croatian} \\
Tschechisch        & Tschechien & \opt{czech} \\
Dänisch & Dänemark        & \opt{danish} \\ 
Niederländisch 	& Niederlande	    & \opt{dutch} \\
Englisch			  & USA            & \opt{american},
\opt{USenglish}, 	\opt{english} \\ & Großbritanien & 
\opt{british}, \opt{UKenglish} \\ 
 & Kanada         & \opt{canadian} \\ & Australien      &
\opt{australian} \\ & Neuseeland    & \opt{newzealand} \\ 
Estisch     & Estland        & \opt{estonian} \\
Finnisch   & Finnland
& \opt{finnish} \\ Französisch& Frankreich, Kanada & \opt{french},
\opt{francais}, \opt{canadien} \\
Deutsch   & Deutschland        & \opt{german} \\
          & Österreich         & \opt{austrian} \\
Deutsch (neu) & Deutschland    & \opt{ngerman} \\
              & Österreich     & \opt{naustrian} \\
Griechisch& Griechenland         & \opt{greek} \\ 
Ungarisch    & Ungarn        & \opt{magyar}, \opt{hungarian} \\
Isländisch    & Island        & \opt{icelandic} \\
Italienisch & Italien & \opt{italian} \\
Lettisch      & Lettland         & \opt{latvian} \\
Norwegisch (Bokmål)  & Norwegen  & \opt{norsk} \\
Norwegisch (Nynorsk) & Norwegen  & \opt{nynorsk} \\
Polnisch       & Polen         & \opt{polish} \\
Portugiesisch& Brasilien	   & \opt{brazil} \\ 
             & Portugal	   & \opt{portuges}, \opt{portuges} \\
Russisch      & Russland         & \opt{russian} \\
Slowakisch    & Slowakei       & \opt{slovak} \\
Slowenisch    & Slovenien      & \opt{slovene} \\
Spanisch   & Spanien          & \opt{spanish} \\ 
Schwedisch & Schweden         & \opt{swedish} \\ 
Ukrainisch    & Ukraine        & \opt{ukrainian} \\
\bottomrule 
\end{tabularx}
\caption{Unterstütze Sprachen} \label{bib:fld:tab1} 
\end{table}

\fielditem{langid}{identifier}

Die Sprachen-id des Bibliografieeintrags. Die Alias-\bibfield{hyphenation} wurde für Rückwärtskompatibilität entwickelt. Der Identifizierer muss ein Sprachenname aus den
Paketen \sty{babel}/\sty{polyglossia} sein. Diese Informationen können verwendet werden für das Umschalten der Trennmuster und zu lokalisierten Strings in der Bibliografie. Beachten Sie, dass Ländernamen Kasus-sensitiv sind. Die Sprachen, die derzeit von diesem Paket unterstützt werden, sind aufgeführt in \tabref{bib:fld:tab1}. 
Beachten Sie, dass \sty{babel} die Kennung \opt{english} als alias für 
\opt{british} oder \opt{american} behandelt, abhängig von der \sty{babel}-Version. 
Das \biblatex-Paket behandelt sie immer alias für \opt{american}. Es ist 
empfehlenswert, die Sprachkennungen  \opt{american} und \opt{british} (\sty{babel}) 
oder eine spezifische Sprachenoption zu verwenden, eine spezifische Sprachvariante
(\sty{polyglossia} anzugeben in \bibfield{langidopts} field), um mögliche Verwirrungen zu vermeiden. Vergleichen Sie \bibfield{language} in \secref{bib:fld:dat}.

\fielditem{langidopts}{literal}

Für \sty{polyglossia}-Verwender: Es ermöglicht pro Eintrag spezifische Sprachoptionen.
Der wörtliche Wert für dieses Feld ist anzupassen dem \sty{polyglossia}-Sprachenwechsel, wenn die Paketoption \opt{autolang=langname} genommen wird. 
Beispielsweise die Felder:

\begin{lstlisting}[style=bibtex]{}
langid         = {english},
langidopts     = {variant=british},
\end{lstlisting}
%
würden einbinden den Bibliografieeintrag in:

\begin{ltxexample}
\english[variant=british]
...
\endenglish
\end{ltxexample}
%

\fielditem{ids}{separated list of entrykeys}

Zitierungsschlüssel für "`aliase"' für den Hauptzierungsschlüssel.
Ein Eintrag kann durch eines seiner Aliase und \biblatex zitiert werden, als
ob man den priären Zitatschlüssel benutzt hätte. Dies geschieht, um Anwendern zu helfen, die ihre Zitierungsschlüssel ändern, jedoch ältere Dokumente haben, die ältere Schlüssel für denselben Eintrag verwenden. Dieses Feld wird bei der Backendverarbeitung gebraucht und erscheint nicht in \path{.bbl}.

\fielditem{indexsorttitle}{literal}

Der Titel wird verwendet, um den Index zu sortieren. Im Gegensatz zu \bibfield{indextitle}, dieses Feld wird nur zur Sortierung verwendet.
Der gedruckte Titel vom Index ist aus \bibfield{indextitle} oder aus dem Feld  \bibfield{title}. Dieses Feld kann nützlich sein, wenn der Titel Sonderzeichen
oder Befehle enthält, die die Sortierung des Index stören. Beachten Sie folgendes Beispiel:

\begin{lstlisting}[style=bibtex]{}
title          = {The \LaTeX\ Companion},
indextitle     = {\LaTeX\ Companion, The},
indexsorttitle = {LATEX Companion},
\end{lstlisting}
%
Stilautoren sollten beachten, dass \biblatex den Wert automatisch kopiert, entweder das  \bibfield{indextitle}- oder das \bibfield{title}-Feld zu  \bibfield{indexsorttitle}, wenn dieses Feld nicht definiert ist. 

\fielditem{keywords}{separated values}

Eine getrennte Liste von Schlüsselwörtern. Diese Schlüsselwörter sind für die Bibliografiefilter bestimmt (sehen Sie \secref{use:bib:bib, use:use:div}), sie werden in der Regel nicht gedruckt. Beachten Sie, das der Standardseparator (Komma)
Lehrräume um den Separator ignoriert.

\fielditem{options}{separated \keyval options}

Eine getrennte Liste von Eitragsoptionen in \keyval-Notation. Dieses Feld
wird verwendet, um Optionen pro Eintragsbasis zu setzen. Sehen Sie für Details \secref{use:opt:bib} an. Beachten Sie, dass Zitierungen und Bibliografiestile
zusätzliche Eintragsoptionen hinzufügen können.

\fielditem{presort}{string}

Ein spezielles Feld, um die Sortierreihenfolge der Bibliografie zu modifizieren.
Dieses Feld ist der erste Eintrag für die Sortierroutine, wenn die Bibliografie
sortiert wird, daher kann es verwendet werden, um Einträge in Gruppen zu ordnen.
Dies kann nützlich sein, wenn Subbibliografien mit Bibliografiefiltern eingerichtet werden. Bitte sehen Sie für weitere Details  \secref{use:srt}. Beachten Sie auch \secref{aut:ctm:srt}. Dieses Feld wird von der Backendverarbeitung benutzt
und erscheint nicht \path{.bbl}.

\fielditem{related}{separated values}

Zitierungsschlüssel für andere Einträge, die eine Beziehung zu diesen Einträgen haben. Diese Beziehungen werden jeweils über das Feld \bibfield{relatedtype} angegeben. Für weitere Details sehen Sie bitte  \secref{use:rel} an. 

\fielditem{relatedoptions}{separated values}

"`Per"=type options"', um einen Bezugseintrag vorzunehmen.
Beachten Sie, dies setzt nicht die Optionen auf den bezugseintra selbst,
nur \opt{dataonly} klont, was als Datenquelle von dem Elterneintrag verwendet wird.

\fielditem{relatedtype}{identifier}

Ein Identifizierer, der die Art der Beziehung für die Schlüsselliste im Feld \bibfield{related} auflistet. Der Identifizier ist eine lokalisierte
Bibliografiezeichenfolge, gedruckt vor den Daten der Bezugseintragsliste.
Es wird auch verwendet, um Art-spezifische Formatierungsanweisungen
und Bibliografiemakros für Bezugseinträge zu erkennen. Für weitere Details sehen Sie bitte \secref{use:rel} an.

\fielditem{relatedstring}{literal}

Ein Feld, das verwendet wird, um Bibliografiezeichenfolgen, die in dem Feld
\bibfield{relatedtype} angegeben werden, zu überschreiben. Für weitere 
Details sehen Sie bitte \secref{use:rel} an. 

\fielditem{sortkey}{literal}

Ein Feld, um die Sortierungsordnung der Bibliografie zu ändern. Stellen Sie sich dieses Feld als Mastersortierungsschlüssel vor.
Falls vorhanden, verwendet \biblatex dieses Feld während des Sortierens und ignoriert
alles andere, mit der Ausnahme des Feldes \bibfield{presort}. Für weitere 
Details sehen Sie bitte  \secref{use:srt} an. Dieses Feld wird von der
den Backendverarbeitung benötigt und erscheint nicht in \path{.bbl}.

\listitem{sortname}{name}

Ein Name oder eine Liste von Namen, verwendet, um die Sortierreihenfolge der Bibliografie zu modifizieren. Falls vorhanden, wird dieses Feld anstelle von   \bibfield{author} oder \bibfield{editor} für die Bibliografiesortierung verwendet. 
Für weitere Details sehen Sie bitte  \secref{use:srt} an. Dieses Feld wird von der
den Backendverarbeitung benötigt und erscheint nicht in \path{.bbl}.

\fielditem{sortshorthand}{literal}

Ähnlich mit \bibfield{sortkey}, aber benutzt für die Abkürzungsliste. 
Wenn vorhanden, benutzt \biblatex dieses Feld anstelle von 
\bibfield{shorthand} für die Sortierung der Abkürzungsliste. Dies ist nützlich, wenn
das Feld \bibfield{shorthand}  Formatierungsbefehle enthält, solche wie \cmd{emph} oder \cmd{textbf}. Dieses Feld wird von der
den Backendverarbeitung benötigt und erscheint nicht in \path{.bbl}. 

\fielditem{sorttitle}{literal}

Ein Feld, um die Sortierungsordnung der Bibliografie zu modifizieren.
Falls vorhanden, wird dieses Feld anstelle des Feldes \bibfield{title} bei der
Bibliografiesortierung verwendet. Das Feld \bibfield{sorttitle} kann praktisch sein, wenn man einen Eintrag mit einem Titel hat wie «An Introduction to\dots» und 
möchte diesen in der alphabetischen Sortierung unter  <I> und nicht unter  <A> eingeordnet haben. In 
diesem Fall könnte «Introduction to\dots» in das \bibfield{sorttitle} eingesetzt werden. Für weitere Details sehen Sie bitte \secref{use:srt} an.
Dieses Feld wird von der
den Backendverarbeitung benötigt und erscheint nicht in \path{.bbl}. 

\fielditem{sortyear}{literal}

Ein Feld, um die Sortierungsordnung der Bibliografie zu modifizieren.
Falls vorhanden, wird dieses Feld anstelle des Feldes \bibfield{year} für die Bibliografiesortierung verwendet.  Für weitere Details sehen Sie bitte
\secref{use:srt} an. Dieses Feld wird von der
den Backendverarbeitung benötigt und erscheint nicht in \path{.bbl}. 

\fielditem{xdata}{separated list of entrykeys}

Dieses Feld übernimmt Daten von einem oder mehreren \bibtype{xdata}-Einträgen. 
Von der Konzeption her steht das \bibfield{xdata}-Feld im Zusammenhang mit  \bibfield{crossref} und \bibfield{xref}: \bibfield{crossref} etabliert eine
logische Eltern/Kind-Relation und erbt Daten; \bibfield{xref} etabliert eine,
logische Eltern/Kind-Relation ohne Daten zu erben; \bibfield{xdata} erbt Daten
ohne eine Relation aufzubauen. Der Wert von  \bibfield{xdata} kann ein einzelner
Eitragsschlüssel oder eine getrennte Schlüsselliste sein.
Für weitere Details sehen Sie bitte \secref{use:use:xdat} an. 
Dieses Feld wird von der
den Backendverarbeitung benötigt und erscheint nicht in \path{.bbl}.

\fielditem{xref}{entry key}

Dieses Feld hat einen alternativen "`cross"=referencing"'-Mechanismus. 
Er unterscheidet sich von dem Feld \bibfield{crossref} dahingehend, dass 
der Kindeintrag
keine Daten von Elterneintrag erbt, die im Feld \bibfield{xref} spezifiziert sind.
Wenn die Anzahl der Kindeinträge auf einen spezifischen Elterneintrag verweist, 
wird eine bestimmte Schwelle gefunden; der Elterneintrag wird der Bibliografie automatisch hinzugefügt, auch wenn er nicht ausdrücklich zitiert wird. Die Schwelle ist einstellbar mit der \opt{mincrossrefs}-Paketoption aus \secref{use:opt:pre:gen}.
Stilautoren sollten beachten, ob oder nicht die \bibfield{xref}-Felder
der Kindeinträge auf der untergeordneten Ebene definiert werden, auf der
\biblatex-Ebene ist es abhängig von der Verfügbarkeit des Elerneintrags. Wenn der
Elterneintrag vorhanden ist, werden die Felder \bibfield{xref}der Kindeinträge
definiert werden. Wenn nicht, bleiben ihre \bibfield{xref}-Felder undefiniert. Ob der
Elterneintrag eingefügt wird in die Bibliografie wegen der Schwelle oder explizit, weil er zitiert wurde, spielt keine Rolle. Sehen Sie auch zu dem 
\bibfield{crossref}-Feld in diesem Abschnitt an sowie \secref{bib:cav:ref}.

\end{fieldlist}

\subsubsection{Benutzerdefinierte Felder} \label{bib:fld:ctm}

Die Felder in diesem Abschnitt sind für spezielle Bibliografiestile bestimmt.
Sie werden nicht in den Standardbibliografiestilen verwendet.

\begin{fieldlist}

\listitem{name{[a--c]}}{Name}

Benutzerdefinierte Listen für spezielle Bibliografiestile. Wird nicht von den
Standardbibliografiestilen verwendet.

\fielditem{name{[a--c]}type}{Schlüssel}

Ähnlich wie \bibfield{authortype} und \bibfield{editortype}, aber bezogen auf
\bibfield{name{[a--c]}}. Wird nicht von den Standardbibliografiestilen
verwendet.

\listitem{list{[a--f]}}{literal}

Benutzerdefinierte Listen für spezielle Bibliografiestile. Wird nicht von den
Standardbibliografiestilen verwendet.

\fielditem{user{[a--f]}}{literal}

Benutzerdefinierte Felder für spezielle Bibliografiestile. Wird nicht von den
Standardbibliografiestilen verwendet.

\fielditem{verb{[a--c]}}{literal}

Ähnlich wie die benutzerdefinierten Felder oben, außer dass es sich um wörtlich
Felder handelt. Wird nicht von den Standardbibliografiestilen verwendet.

\end{fieldlist}

\subsubsection{Aliasfeldnamen} \label{bib:fld:als}

Die Aliase in diesem Abschnitt sind für die Abwärtskompatibilität mit
herkömmlichen \bibtex und anderen Anwendungen, die darauf basieren. Beachten
Sie, dass diese Aliase sofort aufgelöst werden, sobald die \file{bib}-Datei
verarbeitet wird. Alle Bibliografie- und Zitierstile müssen die Namen der
Felder nutzen, auf die sie verweisen und nicht den Alias. In \file{bib}-Dateien
können Sie entweder den Alias oder den Namen des Feldes benutzen, jedoch nicht
beide gleichzeitig.

\begin{fieldlist}

\listitem{address}{literal}

Ein Alias für \bibfield{location}, für  \bibtex-Kompatibilität bereitgestellt.
Herkömmlich verwendet  \bibtex den leicht irreführenden Feldnamen
\bibfield{address} für den Ort der Veröffentlichung, d.\,h. den Ort des
Verlages, während \sty{biblatex} den generische Feldnamen \bibfield{location}
nutzt. Siehe \secref{bib:fld:dat,bib:use:and}.

\fielditem{annote}{literal}

Ein Alias für \bibfield{annotation}, für \sty{jurabib}-Kompatibilität
bereitgestellt. Siehe \secref{bib:fld:dat}.

\fielditem{archiveprefix}{literal}

Ein Alias für \bibfield{eprinttype} für arXiv-Kompatibilität bereitgestellt.
Siehe \secref{bib:fld:dat,use:use:epr}. 

\fielditem{journal}{literal}

Ein Alias für \bibfield{journaltitle}, für \bibtex-Kompatibilität
bereitgestellt. Siehe \secref{bib:fld:dat}.

\fielditem{key}{literal}

Ein Alias für \bibfield{sortkey}, für \bibtex-Kompatibilität bereitgestellt.
Siehe \secref{bib:fld:spc}.

\fielditem{pdf}{verbatim}

Ein Alias für \bibfield{file}, für JabRef-Kompatibilität bereitgestellt. Siehe
\secref{bib:fld:dat}.

\fielditem{primaryclass}{literal}

Ein Alias für \bibfield{eprintclass} für arXiv-Kompatibilität bereitgestellt.
Siehe \secref{bib:fld:dat,use:use:epr}. 

\listitem{school}{literal}

Ein Alias für \bibfield{institution}, für \bibtex-Kompatibilität bereitgestellt.
\bibfield{institution} wird herkömmlich von \bibtex für technische Berichte
verwendet, während \bibfield{school} die Institution einer Diplomarbeit enthält.
Das Paket \biblatex verwendet das generische Feld \bibfield{institution} in
beiden Fällen. Siehe \secref{bib:fld:dat,bib:use:and}.

\end{fieldlist}

\subsection{Verwendungshinweise} \label{bib:use}

Die Eingabetypen und Felder, die durch dieses Paket unterstützt werden, sollten
zum größten Teil intuitiv für jedermann nutzbar sein, der mit \bibtex vertraut
ist. Doch abgesehen von den zusätzlichen Typen und Feldern, die von diesem Paket
zu Verfügung gestellt werden, werden einige der alten Bekannten in einer Art und
Weise gehandhabt, die einer Erklärung benötigt.  Dieses Paket enthält einige
Kompatibilitätskodes für \file{bib}-Dateien, die mit einem herkömmlichen
\bibtex-Stil erzeugt wurden. Leider ist es nicht möglich, alle älteren Dateien
automatisch in den Griff zu bekommen, da das \biblatex-Datenmodell etwas
anders als das herkömmliche \bibtex ist. Daher benötigen solche
\file{bib}-Dateien höchstwahrscheinlich eine Bearbeitung, um richtig mit diesem
Paket funktionieren zu können. Zusammenfassend sind die folgenden Elemente von
den herkömmlichen \bibtex-Stilen verschieden:

\begin{itemize} 
\setlength{\itemsep}{0pt} 
\item Der Eingabetyp \bibtype{inbook}.
Siehe \secref{bib:typ:blx, bib:use:inb} für weitere Einzelheiten.  
\item Die
Felder \bibfield{institution}, \bibfield{organization} und \bibfield{publisher}
sowie die Aliase \bibfield{address} und \bibfield{school}. Siehe
\secref{bib:fld:dat, bib:fld:als, bib:use:and} für weitere Details.  
\item Der
Umgang mit bestimmten Titelarten. Siehe \secref{bib:use:ttl} für weitere
Einzelheiten. 
\item Das Feld \bibfield{series}. Siehe \secref{bib:fld:dat,
bib:use:ser} für Einzelheiten.  
\item Die Felder \bibfield{year} und
\bibfield{month}. Siehe \secref{bib:fld:dat, bib:use:dat, bib:use:iss} für
Einzelheiten.  
\item Das Feld \bibfield{edition}. Siehe \secref{bib:fld:dat} für
Einzelheiten.  
\item Das Feld \bibfield{key}. Siehe \secref{bib:use:key} für
Einzelheiten.  
\end{itemize}

Die Nutzer des \sty{jurabib}-Paketes sollten beachten, dass
\bibfield{shortauthor} wie eine Namensliste von \biblatex behandelt wird,
siehe \secref{bib:use:inc} für weitere Einzelheiten.

\subsubsection{Der Eingabetyp \bibtype{inbook}} \label{bib:use:inb}

Verwenden Sie \bibtype{inbook} nur für einen in sich geschlossenen Teil eines
Buches mit eigenem Titel. Es bezieht sich auf \bibtype{book} genau wie
\bibtype{incollection} auf \bibtype{collection}. Siehe \secref{bib:use:ttl} für
Beispiele. Wenn Sie auf ein Kapitel oder einen Abschnitt eines Buches verweisen
möchten, nutzen Sie einfach \bibfield{book} und fügen Sie ein
\bibfield{chapter}- und\slash oder \bibfield{pages}-Feld hinzu. Ob eine
Bibliografie überhaupt Verweise auf Kapitel oder Abschnitte enthalten sollte,
ist umstritten, da ein Kapitel keine bibliografische Einheit darstellt.


\subsubsection{Fehlende und weggelassene Daten} \label{bib:use:key}

Die Felder, die als <erforderlich> in \secref{bib:typ:blx} angegeben wurden,
sind nicht unbedingt in allen Fällen erforderlich. Die mitgelieferten
Bibliografiestile dieses Paketes können mit wenig auskommen, wie
\bibfield{title} ausreicht für die meisten Typen. Ein Buch, das anonym
veröffentlicht wurde, eine Zeitschrift ohne einen expliziten Herausgeber oder
ein Softwarehandbuch ohne einen expliziten Autor sollten  kein Problem
darstellen, so weit es die Bibliografie betrifft. Zitierstile können jedoch
andere Anforderungen haben. Zum Beispiel erfordert ein Autor-Jahr-Zitat
offensichtlich einen \bibfield{author}\slash \bibfield{editor} und ein
\bibfield{year}.

Sie können in der Regel mit dem \bibfield{label}-Feld einen Ersatz für die
fehlenden Daten für Zitierungen bereitstellen. Wie das \bibfield{label}-Feld
verwendet wird, hängt von dem Zitierstil ab. Der Autor-Jahr-Zitierstil, der in
diesem Paket enthalten ist, nutzt das \bibfield{label}-Feld als Absicherung,
wenn entweder \bibfield{author}\slash \bibfield{editor} oder \bibfield{year}
fehlen. Auf der anderen Seite nutzen die numerischen Designs dies nicht bei
allem, da das numerische Schema unabhängig von den verfügbaren Daten ist. Das
Autor-Titel-Design ignoriert es weitgehend, da der bloße \bibfield{title} in der
Regel für eine einheitliche Zitierform ausreichend ist, ein Titel wird
erwartungsgemäß in jedem Fall zur Verfügung stehen. Das \bibfield{label}-Feld
kann auch verwendet werden, um die nicht-numerischen Teile des automatisch
generierten \bibfield{labelalpha} Feldes (alphabetische Zitierform)  zu
überschreiben.  Siehe § 4.2.4 (engl. Version) %\secref{aut:bbx:fld} 
für weitere Einzelheiten.

Beachten Sie, dass herkömmliche \bibtex-Stile ein Feld \bibfield{key} für die
Alphabetisierung besitzen, falls sowohl \bibfield{author} und \bibfield{editor}
fehlen. Das Paket \biblatex behandelt \bibfield{key} ähnlich wie
\bibfield{sortkey}. Darüber hinaus bietet es eine sehr feine
Sortierungskontrolle, vgl. \secref{bib:fld:spc, use:srt} für Details. Das Paket
\sty{natbib} verwendet \bibfield{key} als ein Absicherungslabel für Zitate.
Verwenden Sie stattdessen das \bibfield{label}-Feld.

\subsubsection{Kooperierende Autoren und Herausgeber} \label{bib:use:inc}

Kooperierende Autoren und Herausgeber werden jeweils in \bibfield{author}- oder
\bibfield{editor}-Feld angeführt. Beachten Sie, dass diese in einem Extrapaar
geschweifter Klammern gesetzt werden müssen, um zu verhindern, dass sie \bibtex
als Personennamen in ihre Bestandteile zerlegt. Verwenden Sie
\bibfield{shortauthor}, wenn Sie eine gekürzte Form  des Namens oder eine
Abkürzung für den Einsatz in Zitaten haben wollen.

\begin{lstlisting}[style=bibtex]{}
author       = {<<{National Aeronautics and Space Administration}>>}, 
shortauthor  = {NASA}, 
\end{lstlisting}
%
In den Standardzitierstilen wird der Kurzname in allen Zitaten ausgegeben,
während der volle Name in der Bibliografie ausgegeben wird. Für kooperierende
Herausgeber verwenden Sie die entsprechenden Felder \sty{editor} und
\sty{shorteditor}. Da alle dieser Felder nun als Namenslisten behandelt werden,
ist es möglich, persönliche Namen und kooperierende Namen gemischt aufzuführen,
vorausgesetzt, sie sind in Klammern gesetzt.

\begin{lstlisting}[style=bibtex]{} 
editor       = {<<{National Aeronautics and Space Administration}>>
		and Doe, John}, 
shorteditor  = {NASA and Doe, John},
\end{lstlisting}
%
Benutzer, die von \sty{jurabib} zu \sty{biblatex} gewechselt haben, sollten
beachten, dass \bibfield{shortauthor} als Namensliste behandelt wird.

\subsubsection{Wörtliche Listen} \label{bib:use:and}

Die Felder \bibfield{institution}, \bibfield{organization}, \bibfield{publisher}
und \bibfield{location} sind wörtliche Listen  im Sinne des \secref{bib:fld}. Dies
gilt auch für \bibfield{origlocation}, \bibfield{origpublisher} und die
Aliasfelder \bibfield{address} und \bibfield{school}. Alle diese Felder
enthalten eine Liste von Elemente, die durch <|and|> getrennt werden. Wenn sie
ein wörtliches <|and|> enthalten, muss es in Klammern gesetzt werden.

\begin{lstlisting}[style=bibtex]{} 
publisher    = {William Reid <<{and}>> Company},
institution  = {Office of Information Management <<{and}>> Communications}, 
organization = {American Society for Photogrammetry <<{and}>>
		Remote Sensing and American Congress on Surveying 
		<<{and}>> Mapping},
\end{lstlisting}
%
Beachten Sie den Unterschied zwischen einem wörtlichen <|{and}|> und dem
Listentrennzeichen <|{and}|> in den obigen Beispielen. Sie können auch den
gesamten Namen in Klammern setzen:

\begin{lstlisting}[style=bibtex]{} 
publisher    = {<<{William Reid and Company}>>}, 
institution  = {<<{Office of Information Management and Communications}>>}, 
organization = {<<{American Society for Photogrammetry and Remote Sensing}>> 
		and <<{American Congress on Surveying and Mapping}>>},
\end{lstlisting}
%
Ältere Dateien, die noch nicht für den Einsatz mit \biblatex aktualisiert
wurden, funktionieren trotzdem, wenn diese Felder kein wörtliches <and>
enthalten. Beachten Sie jedoch, dass Sie die zusätzlichen Merkmale für wörtliche
Listen in diesem Fall vermissen, wie die konfigurierbare Formatierung und
automatische Abkürzung.

\subsubsection{Titel} \label{bib:use:ttl}

Die folgenden Beispiele veranschaulichen, wie man mit den unterschiedlichen
Arten von Titeln umgehen sollte. Beginnen wir mit einem fünfbändigen Werk, das
als Ganzes angesehen wird:

\begin{lstlisting}[style=bibtex]{} 
@Book{works, 
author     = {Shakespeare, William}, 
title      = {Collected Works}, 
volumes    = {5}, 
...
\end{lstlisting}
%
Die einzelnen Bände eines mehrbändigen Werkes haben in der Regel einen eigenen
Titel. Angenommen, der vierte Band von \emph{Collected Works} umfasst
Shakespeares Sonette und wir beziehen uns nur auf dieses Band:

\begin{lstlisting}[style=bibtex]{} 
@Book{sonnets,
author     = {Shakespeare, William}, 
maintitle  = {Collected Works}, 
title      = {Sonntte},
volume     = {4}, 
...  
\end{lstlisting}
%
Wenn die einzelnen Bände keinen eigenen Titel haben, gibt man den Haupttitel in
\bibfield{title} an und fügt eine Bandnummer hinzu:

\begin{lstlisting}[style=bibtex]{} 
@Book{sonnets, 
author     = {Shakespeare, William}, 
title      = {Collected Works}, 
volume     = {4}, 
...
\end{lstlisting}
%
Im nächsten Beispiel werden wir uns auf einen Teil eines Bandes beziehen, der
aber eine in sich geschlossene Arbeit mit eigenem Titel darstellt. Der jeweilige
Band hat auch einen Titel, und es gibt immer noch den Haupttitel der gesamten
Ausgabe:

\begin{lstlisting}[style=bibtex]{}
@InBook{lear, 
author     = {Shakespeare, William}, 
bookauthor = {Shakespeare, William}, 
maintitle  = {Gesammelte Werke},
booktitle  = {Tragedies}, 
title      = {King Lear}, 
volume     = {1}, 
pages      = {53-159}, ...  
\end{lstlisting}
%
Angenommen, der erste Band der \emph{Collected Works}  enthält ein
nachgedrucktes Essay eines bekannten Gelehrten. Dies ist nicht die übliche
Einleitung des Herausgebers, sondern eine in sich geschlossene Arbeit.
\emph{Collected Works} hat ebenso einen eigenen Herausgeber:

\begin{lstlisting}[style=bibtex]{} 
@InBook{stage, 
author     = {Expert, Edward},
title      = {Shakespeare and the Elizabethan Stage}, 
bookauthor = {Shakespeare, William}, 
editor     = {Bookmaker, Bernard}, 
maintitle  = {Collected Works},
booktitle  = {Tragedies}, 
volume     = {1}, 
pages      = {7-49}, ...
\end{lstlisting}
%
Siehe \secref{bib:use:ser} für weitere Beispiele.

\subsubsection{Herausgeberfunktionen} \label{bib:use:edr}

Die Art der redaktionellen Tätigkeit eines Herausgebers ist in einem der
\bibfield{editor}-Felder aufgeführt (z.\,B. \bibfield{editor},
\bibfield{editora}, \bibfield{editorb}, \bibfield{editorc}) und sie können in dem
entsprechenden  \bibfield{editor...type} Feld spezifiziert werden. Die folgenden
Funktionen werden standardmäßig unterstützt. Die Funktion <\texttt{editor}> ist
die Standardeinstellung. In diesem Fall wird \bibfield{editortype} weggelassen.

\begin{marglist} 
\setlength{\itemsep}{0pt} 
\item[editor] Der Hauptherausgeber.
Dies ist die allgemeine redaktionelle Rolle und der Standardwert.
\item[compiler] Ähnlich wie \texttt{editor}, wird aber verwendet, wenn die
Aufgabe des Herausgebers hauptsächlich das Zusammenstellen ist.  
\item[founder] Der Gründer und Herausgeber einer Zeitschrift oder eines umfassenden
Publikationsprojektes, wie eine <Gesammelte Werke>-Edition oder eine lang
andauernde Kolumne.  
\item[continuator] Ein Redakteur, der die Arbeit des
Gründers (\texttt{founder}) fortgesetzt, aber später durch den aktuellen
Herausgeber (\texttt{editor}) ersetzt wurde.  
\item[redactor] Ein zweiter
Herausgeber, dessen Aufgabe es ist, die Arbeit zu schwärzen.
\item[collaborator] Eine weiterer Herausgeber oder ein Berater des Herausgebers.
\end{marglist}
%
Zum Beispiel, wenn die Aufgabe des Herausgebers in der Erstellung besteht,
können Sie dies im entsprechenden \bibfield{editortype}-Feld angeben:

\begin{lstlisting}[style=bibtex]{} 
@Collection{..., 
editor      = {Editor, Edward}, 
editortype  = {compiler}, 
...  }
\end{lstlisting}
%
Es kann auch weitere Herausgeber neben dem Hauptherausgeber geben:

\begin{lstlisting}[style=bibtex]{} 
@Book{..., 
author      = {...}, 
editor      = {Editor, Edward}, 
editora     = {Redactor, Randolph}, 
editoratype = {redactor},
editorb     = {Consultant, Conrad}, 
editorbtype = {collaborator}, 
... }
\end{lstlisting}
%
Zeitschriften oder lang andauernde Veröffentlichungsprojekte können mehrere
Generationen von Herausgebern enthalten. Zum Beispiel kann es einen Gründer
zusätzlich zu dem derzeitigen Herausgeber geben:

\begin{lstlisting}[style=bibtex]{} 
@Book{..., 
author      = {...}, 
editor      = {Editor, Edward}, 
editora     = {Founder, Frederic}, 
editoratype = {founder},
...  }
\end{lstlisting}
%
Beachten Sie, dass nur der \bibfield{editor} in Zitaten angegeben wird und bei
der Sortierung der Bibliografie. Wenn ein Eintrag typischerweise mit dem
Gründer zitiert wird (und entsprechend in der Bibliografie sortiert), kommt der
Gründer in das Feld \bibfield{editor} und der aktuelle Editor kommt in eines der
\bibfield{editor...}-Felder:

\begin{lstlisting}[style=bibtex]{} 
@Collection{..., 
editor      = {Founder, Frederic}, 
editortype  = {founder}, 
editora     = {Editor, Edward}, 
...  }
\end{lstlisting}
%
Sie können weitere Funktionen  durch Initialisierung und Definition eines neuen
Lokalisierungsschlüssels hinzufügen, dessen Name dem Bezeichner im
\bibfield{editor...type}-Feld entspricht. Siehe %\secref{use:lng,aut:lng:cmd} für
§§ 3.9 und in engl. Version 4.9.1 weitere Einzelheiten.

\subsubsection{Veröffentlichungen und Zeitschriftenserien} \label{bib:use:ser}

Das \bibfield{series}-Feld wird von herkömmlichen \bibtex-Stilen sowohl für den
Haupttitel eines mehrbändigen Werkes, als auch für eine Zeitungsserie genutzt,
d.\,h. eine lose Abfolge von Bücher vom gleichen Verlag, die sich mit dem
gleichen Thema befassen oder zum gleichen Feld der Forschung gehören. Dies kann
mehrdeutig sein. Dieses Paket stellt ein \bibfield{maintitle}-Feld für
mehrbändige Werke zur Verfügung und bietet \bibfield{series} nur für
Zeitungsserien. Die Ausgabe oder Nummer des Buches in der Reihe steht in diesem
Fall in \bibfield{number}:

\begin{lstlisting}[style=bibtex]{} 
@Book{..., 
author        = {Expert, Edward},
title         = {Shakespeare and the Elizabethan Age}, 
series        = {Studies in English Literature and Drama}, 
number        = {57}, 
... }
\end{lstlisting}
%
Der \bibtype{article}-Eingabetyp nutzt auch  \bibfield{series}, aber auf eine
andere Art und Weise. Zuerst wird ein Test durchgeführt, um festzustellen, ob
der Wert des Feldes eine ganze Zahl ist. Wenn ja, wird es als eine Ordnungszahl
ausgegeben. Wenn nicht, wird ein weiterer Test durchgeführt, um festzustellen,
ob es ein Lokalisierungsschlüssel ist. Wenn ja, wird der lokalisierten String
ausgegeben. Wenn nicht, wird der Wert ausgegeben. Betrachten Sie das folgende
Beispiel für eine Zeitschrift, die in nummerierter Reihenfolge erschienen ist:

\begin{lstlisting}[style=bibtex]{} 
@Article{..., 
journal         = {Journal Name}, 
series          = {3}, 
volume          = {15}, 
number          = {7},
year            = {1995}, 
... }
\end{lstlisting}
%
Dieser Eintrag wird als «\emph{Journal Name}. 3rd ser. 15.7 (1995)» ausgegeben.
Einige Zeitschriften verwenden Bezeichnungen wie <alte Serie> und <neuen Serie>
anstelle einer Nummer. Solche Bezeichnungen werden in \bibfield{series}
angegeben, entweder als Zeichenkette oder als Lokalisierungsschlüssel.
Betrachten Sie das folgende Beispiel, welches die Verwendung des
Lokalisierungsschlüssels \texttt{newseries} zeigt:

\begin{lstlisting}[style=bibtex]{} 
@Article{..., 
journal         = {Journal Name}, 
series          = {newseries}, 
volume          = {9}, 
year            = {1998}, 
... }
\end{lstlisting}
%
Dieser Eintrag wird als «\emph{Journal Name}. New ser. 9 (1998)» ausgegeben.
Siehe § 4.9. (engl. Version) %\secref{aut:lng:key} 
für eine Liste der standardmäßig definierten Lokalisierungsschlüssel.

\subsubsection{Datumsspezifikationen} \label{bib:use:dat}

\begin{table} 
\tablesetup 
\begin{tabularx}{\columnwidth}{@{}>{\ttfamily}llX@{}}
\toprule 
\multicolumn{1}{@{}H}{Datumspezifikation} &
\multicolumn{2}{H}{Formatierte Daten (Beispiele)} \\ 
\cmidrule(l){2-3} &
\multicolumn{1}{H}{Kurz/12-Stundenformat} & \multicolumn{1}{H}{Lang/24-Stundenformat} \\
\cmidrule{1-1}\cmidrule(l){2-2}\cmidrule(l){3-3} 
1850			& 1850  & 1850 \\ 
1997/			& 1997--			& 1997-- \\
1997/ \ldots		& 1997--			& 1997-- \\
\ldots\ 1997		& 1997--			& 1997-- \\
1967-02			& 02/1967			& Februar 1967 \\
2009-01-31		& 31/01/2009			& 31ter Januar 2009 \\
1988/1992		& 1988--1992			& 1988--1992 \\
2002-01/2002-02		& 01/2002--02/2002	& Januar 2002--Februar 2002 \\
1995-03-30/1995-04-05	& 30/03/1995--05/04/1995	& 30ter März
1995--5ter April 1995 \\
2004-04-05T14:34:00 & 05/04/2004 2:34 PM & 5ter April 2004 14:34:00\\
\bottomrule 
\end{tabularx}

\caption{Datums- und Zeitspezifikationen} \label{bib:use:tab1} 
\end{table}

Datumsfelder, wie die Daten des Standarddatenmodells \bibfield{date},
\bibfield{origdate}, \bibfield{eventdate} und
\bibfield{urldate}, halten sich an \acr{ISO8601-2} (Extended Date/Time Format Stufe 1) In Ergänzung zu den leeren Datumsmarkierungen \acr{ISO86-2} können sie auch
einen offenen End-/Start-Datumsbereich angeben, indem Sie das Bereichstrennzeichen angeben und das End-/Sartdatum weglassen (wie \texttt{YYYY/}, \texttt{/YYYY}). Siehe %\tabref{bib:use:tab1}
Tabelle 3, hier finden Sie Beispiele für gültige 
Datumsspezifikationen und die von \biblatex\ automatisch generierten formatierten Daten. Das formatierte Datum ist sprachspezifisch und wird automatisch angepasst.
Wenn ein Eitrag kei \bibfield{date}-Feld enthält, berücksichtigt \biblatex auch die \bibfield{year}- und \bibfield{month}-Felder, um die Abwärtskompatibilität mit
dem traditionellem \bibtex zu gewährleisten. Dies wird jedoch nicht empfohlen als
explizite \bibfield{year}- und \bibfield{month}-Felder, da sie nicht geparst sind
für Datums-Metainformationsmarker oder -Zeiten und sie werden unverändert verwendet.
Stilautoren sollten beachten, dass Datumsfelder wie \bibfield{date} oder
\bibfield{origdate} nur in der Datei \file{bib} verfügbar sind. Alle Daten werden
analysiert und in ihre Komponenten zerlegt, während die Datei \file{bib}
verarbeitet wird. Die Datums- und Zeitkomponenten werden in den Stilen über die
in § 4.2.4.3 %\secref{aut:bbx:fld:dat} 
beschriebenen Spezialfelder zur Verfügung gestellt.
Weitere Informationen finden Sie in diesem Abschnitt und in Tabelle~10 %\tabref{aut:bbx:fld:tab1}
und auf Seite~169.%\pageref{aut:bbx:fld:tab1}.

Erweitertes Datum/Zeit-Format) Spezikationsebenen 0 and 1. \acr{EDTF} ist eine strengere Teilmenge der etwas chaotischen, erlaubten Formate von \acr{ISO8601v2004} und ist besser geeignet für bibliografische Daten. Neben dem \acr{EDTF} leeren Datumsbereichmarker können 
Sie können einen unbefristeten Zeitraum angeben, indem der Bereichstrenner
angeben und das Enddatum weggelassen wird (z.\,B. \texttt{yyyy/}). Sehen Sie
\tabrefe{bib:use:tab1} für einige Beispiele für gültige Datenspezifikationen und
die formatierten Daten, die automatisch von \biblatex generiert werden. Das
formatierte Datum ist sprachspezifisch und wird automatisch angepasst. Wenn kein
\bibfield{date} im Eingabetyp ist, nutzt \biblatex auch die Felder
\bibfield{year} und \bibfield{month} für die Abwärtskompatibilität des
herkömmlichen \bibtex.  Stilautoren sollten beachten, dass \bibfield{date} oder
\bibfield{origdate} nur in der bib-Datei vorhanden sind. Alle Daten werden
analysiert und in ihre Bestandteile zerlegt, sobald  die \file{bib}-Datei
verarbeitet wird. Die Datumskomponenten stehen für die Stile zur Verfügung
%\secref{aut:bbx:fld:dat}. 
§ 4.2.3. (eng. V.) Siehe diesen Abschnitt und dort Tabelle 10. % \tabrefe{aut:bbx:fld:tab1}
%auf Seite~\pageref{aut:bbx:fld:tab1} 
für weitere Informationen.

%Datumsfeldnamen \emph{müssen} mit der Zeichenfolge <date> enden, wie bei den %Standarddatumsfeldern. Denken Sie daran, beim Hinzufügen neuer Datenfelder mit dem 
%Datenmodell (sehen Sie § 4.5.4 (eng. V.) %\secref{aut:ctm:dm}). 
%\biblatex wird alle Datumsfelder  nach dem Lesen des Datumsmodells überprüfen und %wird mit einer Fehlerausgabe enden, wenn es ein Datumsfeld findet, das nicht dieser %Namenskonvention entspricht.

\acr{ISO8601-2} unterstützt Daten vor der verbreiteten Zeit (BCE/BC) mittels eines negativen Datumsformats und unterstützt <ungefähre> (circa) und unsichere Daten. Solche Datenformate setzen interne Marker, auf die getestet werden kann, damit
geeignete lokalisierte Marker (wie \opt{circa} oder \opt{beforecommonera} eingefügt werden können. Ebenfalls unterstützt werden auch
<nicht spezifizierte> Daten ((\acr{ISO8601-2} 4.3), die automatisch expandieren in die entsprechenden Datenbereiche durch das Feld \bibfield{$<$datetype$>$dateunspecified}, die Details der
Granularität von unspezifizierten Daten. Stile können diese Informationen benutzen, um solche Termine entsprechend zu formatieren, aber die Standardstile tun dies nicht. Sehen Sie Tabelle~4 %\tabref{bib:use:tab3} 
auf Seite~\pageref{bib:use:tab3} für die erlaubten 
\acr{ISO8601-2} <unspezifizierten> Formate, deren Programmerweiterungen und
\bibfield{$<$datetype$>$dateunspecified}-Werte (sehen Sie § 4.2.4.1 (engl. Version)). %\secref{aut:bbx:fld:gen}).

\begin{table}
\tablesetup
\begin{tabularx}{\textwidth}{@{}>{\ttfamily}llX@{}}
\toprule
\multicolumn{1}{@{}H}{Datenspezifikation} &
\multicolumn{1}{H}{Erweitertes Angebot} &
\multicolumn{1}{H}{Metainformation} \\
\cmidrule{1-1}\cmidrule(l){2-2}\cmidrule(l){3-3}
199X       & 1990/1999             & yearindecade \\
19XX       & 1900/1999             & yearincentury \\
1999-XX    & 1999-01/1999-12       & monthinyear \\
1999-01-XX & 1999-01-01/1999-01-31 & dayinmonth \\
1999-XX-XX & 1999-01-01/1999-12-31 & dayinyear \\
\bottomrule
\end{tabularx}
\caption{ISO8601-2 4.3 Unspezifisches Datenparsing}
\label{bib:use:tab3}
\end{table}

\tabrefe{bib:use:tab2} zeigt Formate, die entsprechenden Prüfungen und Formatierung.
Sehen Sie Daten-meta-informationstest %in \secref{aut:aux:tst} 
und Lokalisierungsstrings.
%in \secref{aut:lng:key:dt}. 
Sehen Sie auch die Beispieldatei \file{96-dates.tex}, die
komplette Beispiele für die Tests und Lokalisierungsstrings verwendet.

Die Ausgabe von <circa>, Unsicherheit und Zeitinformationen  in Standardstilen (oder
benutzerdefinierte Stilen brauchen nicht die internen \cmd{mkdaterange*}-Makros) wird durch die Paketoptionen \opt{datecirca}, \opt{dateuncertain}, \opt{dateera} und \opt{dateeraauto} gesteuert (sehen Sie \secref{use:opt:pre:gen}). Sehen Sie
\tabrefe{bib:use:tab2} auf Seite~\pageref{bib:use:tab2} für Beispiele, die diese Optionen Übernehmen und alle nutzen.

\begin{table}
\tablesetup
\begin{tabularx}{\textwidth}{@{}>{\ttfamily}llX@{}}
\toprule
\multicolumn{1}{@{}H}{Datumsspezification} &
\multicolumn{2}{H}{Formatierte Datums (Beispiele)} \\
\cmidrule(l){2-3}
&
\multicolumn{1}{H}{Output-Format} &
\multicolumn{1}{H}{Output-Format-Stichworte} \\
\cmidrule{1-1}\cmidrule(l){2-2}\cmidrule(l){3-3}
0000        & 1 BC            & \kvopt{dateera}{christian} druckt \opt{beforechrist} localisation\\
-0876			  & 877 BCE			     & \kvopt{dateera}{secular} druckt \opt{beforecommonera} Localisation-string\\
-0877/-0866 & 878 BC--867 BC & nehmen \cmd{ifdateera} Test und \opt{beforechrist} Localisation-string\\
0768 & 0768 CE & nehmen \opt{dateeraauto} setzt auf <1000>  und \opt{commonera} Localisation-string\\
-0343-02 & 344-02 BCE & \\
0343-02-03 & 343-02-03 CE & mit \opt{dateeraauto=400} \\
0343-02-03 & 343-02-02 CE & mit \opt{dateeraauto=400} und \opt{julian} \\
1723\textasciitilde & circa 1723 & nehmen \cmd{ifdatecirca} Test\\
1723? & 1723? & nehmen \cmd{ifdateuncertain} Test\\
1723?\textasciitilde & circa 1723? & nehmen \cmd{ifdateuncertain} und \cmd{ifdatecirca} Tests\\
2004-22 & 2004 & auch, \bibfield{season} gesetzt auf den Localisation-string <Sommer>\\
2004-24 & 2004 & auch, \bibfield{season} gesetzt auf den Localisation-string <Winter>\\
\bottomrule
\end{tabularx}
\caption{Verbesserte Datumspezifikationen}
\label{bib:use:tab2}
\end{table}

\subsubsection{Jahr, Monat und Datum}
\label{bib:use:yearordate}

Die Felder \bibfield{year} und \bibfield{month} werden weiterhin von \biblatex
unerstützt, aber alle Datumsfunktionen (Tages- und Uhrzeitgenauigkeit, Bereiche,
\ldots) können nur mit dem Feld \bibfield{date} verwendet werden. Es wird 
daher empfohlen,
das Feld \bibfield{date} gegenüber den Felder \bibfield{year} und \bibfield{month}
vorzuziehen; es sei denn, die Abwärtskompatibilität der \file{bib}-Datei mit dem
klassischem \bibtex fordert dies.

\subsubsection{Monats- und Zeitschriftenausgaben} \label{bib:use:iss} 

Das Feld
\bibfield{month} ist ein Integerfeld. Die Bibliografiestile wandeln, falls
erforderlich, den Monat zu einer sprachabhängigen Zeichenfolge um. Aus Gründen
der Abwärtskompatibilität sollten Sie die folgenden Abkürzungen mit drei
Buchstaben verwenden: \texttt{jan}, \texttt{feb}, \texttt{mar}, \texttt{apr},
\texttt{may}, \texttt{jun}, \texttt{jul}, \texttt{aug}, \texttt{sep},
\texttt{oct}, \texttt{nov}, \texttt{dec}. Beachten Sie, dass diese Abkürzungen
\bibtex-Zeichenketten sind und ohne Klammern oder Anführungszeichen angegeben
werden. Also nicht |month={jan}| oder |month="jan"|, sondern |month=jan|. Es ist
nicht möglich, einen Monat in der Form |month={8/9}| zu spezifizieren. Verwenden
Sie stattdessen \bibfield{date}, um Zeiträume anzugeben. Quartalszeitschriften
sind in der Regel mit <Frühling> oder <Sommer> beschriftet, welche in
\bibfield{issue} angegeben werden sollte. Die Platzierung des
\bibfield{issue}-Feldes in \bibtype{article} Einträgen ähnelt und überschreibt
das Feld \bibfield{month}.

\subsubsection{Zeitschriftennummern und Ausgabe} \label{bib:use:issnum}

Die Wörter <number> (Nummer) und <issue> (Ausgabe) werden von Zeitschriften häufig synonym verwendet, um auf die Unterteilung eines \bibfield{volume}-Feldes zu verweisen. Die Tatsache, dass das Datenmodell von \biblatex Felder mit beiden
Namen enthält, kann manchmal zur Verwirrung darüber führen, welches Feld verwendet werden soll. In erster Linie sollte das Wort, das die Zeitschrift für die Unterteilung einer \bibfield{volume}-Feld verwendet, von untergeordneter Bedeutung sein. Entscheidend ist die Rolle im Datenmodell. Als Faustregel gilt, dass das \bibfield{number}-Feld in den meisten Fällen das richtige Feld ist. In den
Standardstilen ändert das Feld \bibfield{number} das Feld \bibfield{volume},
während das Feld \bibfield{issue} das Datum (Jahr) des Eintrags ändert. Numerische Bezeichner und Kurzbezeichner, die nicht unbedingt (vollständig) numerisch sind, wie
<A>, <S1>, <C2>, <Suppl.\ 3>, <4es> würden in das Feld \bibfield{number} aufgenommen,weil sie üblicherweise das Feld \bibfield{volume} ändern. Die Ausgabe von -- besonders langen -- nicht numerischen Eingaben für das \bibfield{number}-Feld sollten überprüft werden, da sie bei einigen Stilen möglicherweise seltsam aussehen. Das Feld \bibfield{issue} kann für Bezeichnungen wie <Spring>, <Winter> oder <Michaelmas term> verwendet werden, wenn dies üblicherweise für die Bezugnahme auf die Zeitschrift verwendet wird.


\subsubsection{Seitennummerierung} \label{bib:use:pag}

Bei der Angabe einer Seite oder eines Seitenbereiches, entweder im Feld
\bibfield{pages} oder im \prm{postnote}-Argument eines Zitierbefehls, ist es
zweckmäßig, dass \biblatex-Präfixe wie <p.> oder <pp.> automatisch
hinzugefügt werden und dies tut dieses Paket in der Tat standardmäßig. Jedoch
verwenden einige Werke ein anderes Seitennummerierungsschema bzw. werden nicht
durch Seitenangaben, sondern durch Vers- oder Zeilennummer zitiert. Hier kommen
\bibfield{pagination} und \bibfield{bookpagination} ins Spiel. Als Beispiel
betrachten wir den folgenden Eintrag:

\begin{lstlisting}[style=bibtex]{} 
@InBook{key, 
title          = {...},
pagination     = {verse},
booktitle      = {...}, 
bookpagination = {page}, 
pages	       = {53--65}, 
... }
\end{lstlisting}
%
Das \bibfield{bookpagination}-Feld beeinflusst die Formatierung der \bibfield{pages} und
\bibfield{pagetotal}, die Liste der Referenzen. Seit \texttt{page} der Standard
ist, wird dieses Feld im obigen Beispiel weggelassen. In diesem Fall wird der
Seitenbereich als <pp.~53--65> formatiert. Angenommen, dass es bei einem Zitat
aus einer Arbeit üblich ist, Verszahlen anstelle von Seitenzahlen in Zitaten zu
verwenden, dann wird dies durch \bibfield{pagination} widergespiegelt, welches
die Formatierung des \prm{postnote}-Arguments für jeden Zitierbefehl
beeinflusst. Mit einem Zitat wie |\cite[17]{key}|, wird \prm{postnote} als
<v.~17> formatiert. Die Einstellung von \bibfield{pagination} würde <\S~17>
ergeben. Siehe \secref{use:cav:pag} für weitere Verwendungshinweise.

Die Felder \bibfield{pagination} und \bibfield{bookpagination} sind Schlüsselfelder. Dieses
Paket wird versuchen, ihren Wert als Lokalisierungsschlüssel zu verwenden,
vorausgesetzt, dass der Schlüssel definiert wurde. Benutzen Sie immer die
Singularform des Schlüssels in \file{bib}-Dateien, der Plural wird automatisch
gebildet. Die Schlüssel \texttt{page}, \texttt{column}, \texttt{line},
\texttt{verse}, \texttt{section}, und \texttt{paragraph} sind vordefiniert, mit
\texttt{page} als Standardwert. Die Zeichenfolge <\texttt{none}> hat eine
besondere Bedeutung, wenn man \bibfield{pagination} oder
\bibfield{bookpagination} Felder verwendet. Es unterdrückt das Präfix für den
jeweiligen Eintrag. Wenn es keine vordefinierten Lokalisierungsschlüssel für die
Seitennummerierung gibt, können Sie sie einfach hinzufügen. Vergleiche die
Befehle\\ \cmd{NewBibliographyString} und \cmd{DefineBibliographyStrings} in
\secref{use:lng}. Sie müssen zwei Lokalisierungsstrings für jedes zusätzliche
Seitennummerierungsschema definieren: Die Singularform (der
Lokalisierungsschlüssel entspricht dem Wert des Feldes \bibfield{pagination})
und die Pluralform (deren Lokalisierungsschlüssel muss den Singular plus den
Buchstaben <\texttt{s}> enthalten). Sehen Sie die vordefinierten Schlüssel und
Beispiele in § 4.9.2 (engl. V.). %\secref{aut:lng:key}.

\subsection{Hinweise und Warnungen} \label{bib:cav}

Dieser Abschnitt enthält einige zusätzliche Hinweise zu der
Datenschnittstellen des Pakets. Er geht auch auf einige häufig auftretende
Probleme ein.

\subsubsection{Querverweise} \label{bib:cav:ref}

\biber-Merkmale verfügen über einen hochgradig anpassbaren Querverweismechanismus mit flexiblen Datenvererbungsregeln. Das Duplizieren bestimmter Felder in den Eltern-Eintrag oder das Hinzufügen von leeren Feldern in den Kind-Eintrag ist nicht mehr erforderlich. Einträge werden auf natürliche Weise spezifiziert:

\begin{lstlisting}[style=bibtex]{}
@Book{book,
  author	= {Author},
  title		= {Booktitle},
  subtitle	= {Booksubtitle},
  publisher	= {Publisher},
  location	= {Location},
  date		= {1995},
}
@InBook{inbook,
  crossref	= {book},
  title		= {Title},
  pages		= {5--25},
}
\end{lstlisting}
%
Das \bibfield{title}-Feld von den Eltern wird
auf das \bibfield{booktitle}-Feld des Kindes übertragen,
wobei der \bibfield{subtitle} erhält \bibfield{booksubtitle}. Das \bibfield{author} 
der Erltern erhält \bibfield{bookauthor} des Kinds und da das Kind ein
 \bibfield{author}-Feld nicht erhält, ist es auch zu dublzieren wie der \bibfield{author}
 des Kindes. Nach der Datenvererbung sieht der Kindeintrag ähnlich zu dem folgenden aus:

\begin{lstlisting}[style=bibtex]{}
author	  	= {Author},
bookauthor	= {Author},
title		= {Title},
booktitle	= {Booktitle},
booksubtitle	= {Booksubtitle},
publisher	= {Publisher},
location	= {Location},
date		= {1995},
pages		= {5--25},
\end{lstlisting}
%
Im Anhang (engl. Version) finden Sie eine Liste von standardmäßig eingerichteten Zuordnungsregeln.
Beachten Sie, dass dies alles anpassbar ist. Sehen Sie in § 4.5.11 (engl. Version) %\secref{aut:ctm:ref}, wie
\biber's cross"=referencing-Mechanismus konfiguriert ist. Sehen Sie auch \secref{bib:fld:spc}.

\paragraph{Das \bibfield{xref}-Feld} \label{bib:cav:ref:ref}

Zusätzlich zum  \bibfield{crossref}-Feld unterstützt  \biblatex einen
einfachen Querverweis-Mechanismus mit dem  \bibfield{xref}-Feld. Dieses ist
günstig, wenn eine Relation zwischen einer Eltern-Kind-Relation bei zwei
verbundenen Einträgen etabliert werden soll, die es jedoch vorzieht, ihre
Unabhängigkeit zu erhalten, so weit die Angaben konzentriert sind. Das
\bibfield{xref}-Feld unterscheidet sich vom \bibfield{crossref} dahingehend,
dass der Kindeintrag nicht alle Daten des Elterneintrags erben wird, die in
\bibfield{xref} angegeben sind. Wenn auf den Elterneintrag durch eine bestimmte
Anzahl an Kindeinträgen verwiesen wird, wird \biblatex diese automatisch in
die Bibliografie einfügen. Der Schwellenwert wird durch die
\opt{minxrefs}-Paketoption aus \secref{use:opt:pre:gen} gesteuert. Siehe dazu auch.
\secref{bib:fld:spc}.
 
\subsubsection{Sortier- und Kodierungsprobleme} \label{bib:cav:enc}


\biber kann mit Ascii, 8-Bit-Kodierungen und Latin\,1 und \acr{UTF-8} umgehen. Es
verfügt über echte Unicodunterstützung und ist in der Lage, im laufenden Betrieb
eine \file{bib}-Datei zu kodieren. Für die Sortierung verwendet \biber eine
Perl-Implementierung des Unicode Collation Algorithm (\acr{UCA}), wie in Unicode
Technical Standard \#10 dargestellt.\fnurl{http://unicode.org/reports/tr10/}
Sortierungsüberschneidungen auf der Basis von Unicode Common Locale Data
Repository (\acr{CLDR}) werden hinzugefügt.\fnurl{http://cldr.unicode.org/}

Unicode-Unterstützung bedeutet wesentlich mehr als \utf-Eingabe. Unicode ist
eine komplexe Standardabdeckung, die mehr als seine bekannten Teile enthält, die
Unicode- Zeichenkodierung und Transportkodierungen wie \utf. Außerdem
standardisiert es Aspekte wie Zeichensortierung, die für die Groß- und
Kleinschreibung basierte Sortierung erforderlich ist. Beispielsweise kann Biber
durch Verwendung des Unicode Collation Algorithm, den Buchstaben 'ß' ohne manuellen Eingriff handhaben. Alles was Sie
tun müssen, um eine lokalisierte Sortierung zu erhalten, ist das lokale
Schema anzugeben:

\begin{ltxexample}
\usepackage[sortlocale=de]{biblatex}
\end{ltxexample}
%
oder wenn Sie "`german"' als Hauptdokumentsprache mit Babel oder Polyglossia nehmen:

\begin{ltxexample}
\usepackage[sortlocale=auto]{biblatex}
\end{ltxexample}
%
\biblatex wird dann  Babel/Polyglossia die Hauptdokumentensprache wie die lokale übergeben, so dass \biber eine geeignete lokalen Standards abbildet. \biber wird nicht versuchen, lokale Informationen aus seiner Umgebung zu erhalten, wie dies Dokumentenverarbeitung mit Abhängigkeit von etwas, das nicht im Dokument ist, macht. 
Das ist gegen den Geist der Reproduzierbarkeit von \tex. Dies macht auch Sinn, da
Babel/Polyglossia in der Tat die entsprechende Dokumentumgebung sind.
Beachten Sie, dass dies auch mit
8-Bit-Kodierungen, wie Latin\,9 funktioniert, d.\,h., dass Sie auch einen
Vorteil aus Unicode-basierter Sortierung ziehen, wenn Sie nicht mit
\utf-Eingaben arbeiten. Siehe \secref{bib:cav:enc:enc}, wie Eingabe- und
Datenkodierungen richtig angeben werden.

\paragraph{Besondere Kodierungen} \label{bib:cav:enc:enc}

Bei Verwendung einer Nicht-Ascii-Kodierung in der \file{bib}-Datei, ist es
wichtig zu verstehen, was \biblatex für Sie tun kann und was einen
manuellen Eingriff erfordert. Das Paket kümmert sich um die \latex-Seite, d.\,h.
es wird sichergestellt, dass die Daten aus der \file{bbl}-Datei richtig
importiert werden, sofern Sie die \opt{bibencoding}-Paketoption richtig
eingestellt haben. All dies wird 
automatisch gehandhabt und es sind keine weiteren Schritte, abgesehen von der
Einstellung \opt{bibencoding}, erforderlich. Hier sind ein paar typische
Einsatzmöglichkeiten zusammen mit den relevanten Zeilen aus der
Dokumentpräambel:

\begin{itemize} 
\setlength{\itemsep}{0pt}

\item Ascii-Notation sowohl in der \file{tex}- und \file{bib}-Datei mit \pdftex
oder herkömmlichen \tex: 

\begin{ltxexample} 
\usepackage{biblatex} 
\end{ltxexample}

\item Latin\,1-Kodierung (\acr{ISO}-8859-1) in \file{tex}-Dateien, Ascii-Zeichen
in \file{bib}-Dateien mit \pdftex oder herkömmlichen \tex:

\begin{ltxexample} 
\usepackage[latin1]{inputenc}
\usepackage[bibencoding=ascii]{biblatex} 
\end{ltxexample}

\item Latin\,9-Kodierung (\acr{ISO}-8859-15) in \file{tex}- und \file{bib}-Dateien mit \pdftex oder traditionell:

\begin{ltxexample} 
\usepackage[latin9]{inputenc}
\usepackage[bibencoding=auto]{biblatex} 
\end{ltxexample}
%
Seit \kvopt{bibencoding}{auto} die Standardeinstellung ist, wird die Option
weggelassen. Das folgende Setup hat den gleiche Effekt:

\begin{ltxexample} 
\usepackage[latin9]{inputenc}
\usepackage{biblatex} 
\end{ltxexample}

\item \acr{UTF-8}-Kodierung in \file{tex}-Dateien, Latin\,1 (\acr{ISO}-8859-1)
in \file{bib}-Dateien mit \pdftex oder herkömmlichen \tex: 

\begin{ltxexample} 
\usepackage[utf8]{inputenc}
\usepackage[bibencoding=latin1]{biblatex} 
\end{ltxexample}

Das Gleiche mit \xetex oder \luatex im  herkömmlichen \acr{UTF-8} Modus:

\begin{ltxexample} 
\usepackage[bibencoding=latin1]{biblatex}
\end{ltxexample}

\end{itemize}

\biber kann mit der Ascii-Notation, 8-Bit-Kodierungen, wie Latin\,1 und
\acr{UTF-8}, umgehen. Es ist auch in der Lage, im Betrieb die \file{bib}-Datei zu
kodieren (ersetzt das begrenzte Kodierfeauture von \biblatex auf der
Makroebene). Dies geschieht automatisch, falls erforderlich, vorausgesetzt, dass
Sie die Kodierung der \file{bib}-Dateien korrekt angegeben haben. Zusätzlich zu
den oben beschriebenen Szenarien kann Biber auch mit den folgenden Fällen
umgehen:

\begin{itemize}

\item Klarer \acr{UTF-8}-Arbeitsablauf, d.\,h., \acr{UTF-8} Kodierung sowohl in
der \file{tex}- und in der \file{bib}-Datei mit \pdftex oder herkömmlichen \tex:

\begin{ltxexample} 
\usepackage[utf8]{inputenc}
\usepackage[bibencoding=auto]{biblatex} 
\end{ltxexample}
%
Seit \kvopt{bibencoding}{auto} die Standardeinstellung ist, wird diese Option
weggelassen:

\begin{ltxexample} 
\usepackage[utf8]{inputenc}
\usepackage{biblatex} 
\end{ltxexample}

Das Gleiche mit \xetex oder \luatex im herkömmlichen \acr{UTF-8}-Modus:

\begin{ltxexample} 
\usepackage{biblatex} 
\end{ltxexample}

\item Es ist sogar möglich, eine 8-Bit kodierte \file{tex}-Datei mit \acr{UTF-8}
in einer \file{bib}-Datei zu kombinieren, sofern alle Zeichen in der
\file{bib}-Datei auch von der 8-Bit-Codierung abgedeckt sind:

\begin{ltxexample} 
\usepackage[latin1]{inputenc}
\usepackage[bibencoding=utf8]{biblatex} 
\end{ltxexample} 

\end{itemize}

Einige Workarounds sind erforderlich bei der Verwendung des herkömmlichen \tex
oder \pdftex mit der \utf-Kodierung, weil der Stil  \sty{inputenc} mit
\file{utf8}-Modulen nicht alles von Unicode abdeckt. Vergröbert gesagt, es deckt
nur den westeuropäischen Unicodeteil ab. Beim Laden von \sty{inputenc} mit
\file{utf8}-Möglichkeiten kann \biblatex \biber normalerweise anweisen, die
\file{bib}-Datei zu "`reencoden"'. Dies kann in der \sty{inputenc}-Datei zu
Fehlern führen, wenn einige der Eigenschaften in der \file{bib}-Datei außerhalb
des Unicodebereichs sind, der von \sty{inputenc} supportet wird.

\begin{itemize}

\item Wenn Sie von diesem Problem betroffen sind, versuchen Sie es mit
der\\
\opt{safeinputenc}-Option.

\begin{ltxexample} 
\usepackage[utf8]{inputenc}
\usepackage[safeinputenc]{biblatex} 
\end{ltxexample}
%
Wenn diese Option aktiviert ist, ignoriert \sty{inputenc} die \opt{utf8}-Option
und verwendet Ascii. Biber wird dann versuchen, die \file{bib}-Daten in die
Ascii-Notation zu konvertieren. Beispielsweise wird es \texttt{\k{S}} in |\k{S}|
konvertieren. Diese Option ähnelt dem Setzen von \kvopt{texencoding}{ascii}, sie
wird aber erst in einem speziellen Szenario wirksam (\sty{inputenc}\slash
\sty{inputenx} mit \utf). Dieser Workaround nutzt die Tatsache, dass sowohl
Unicode als auch die \utf-Transport-Kodierung abwärtskompatibel zu Ascii sind.

\end{itemize}

Diese Lösung kann als Workaround akzeptabel sein, wenn die Daten in der
\file{bib}-Datei hauptsächlich in Ascii kodiert sind. Meist sind es aber nur
wenige Strings, wie einige Autorennamen, die Probleme verursachen. Man sollte
jedoch im Hinterkopf behalten, dass das traditionelle \tex oder \pdftex Unicode
nicht unterstützt. Es kann helfen, wenn  okkasionelle, seltsame Eigenschaften
von \sty{inputenc} nicht unterstützt werden, diese mit \tex zu verarbeiten, so
wenn Sie einen Akzentbefehl benötigen (z.\,B. |\d{S}| anstelle von \texttt{\d{S}}).
Wenn Sie vollständige Unicodeunterstützung benötigen, sollten Sie zu  \xetex
oder \luatex wechseln.

Typische Fehlerangaben, wenn \sty{inputenc} nicht umgehen kann mit UTF-8 Zeichen, sind:

\begin{verbatim}
! Package inputenc Error: Unicode char <char> (U+<codepoint>)
(inputenc)                not set up for use with LaTeX.
\end{verbatim}
%
aber auch weniger offensichtliche Dinge wie:

\begin{verbatim}
! Argument of \UTFviii@three@octets has an extra }.
\end{verbatim}

\section{Benutzerhandbuch} \label{use}

Dieser Teil des Handbuches gibt einen Überblick über die
Bedienoberfläche des \biblatex-Pakets. Dabei deckt er alles ab, was Sie
über die dem Paket beiliegenden Standardstile des \biblatex-Pakets wissen
müssen. In jedem Fall sollte diese Anleitung als erstes gelesen werden. Möchten
Sie eigene Literaturverweisregeln oder Bibliografiestile erstellen, können sie
mit der anschließenden Anleitung für Autoren (in der englischsprachigen Originalversion)
fortfahren.

\subsection{Paketoptionen} \label{use:opt} 

Alle Paketoptionen werden als \keyval-Notationen dargestellt. Dabei kann der
Befehl \texttt{true} für alle Booleanschen Schlüssel weggelassen werden. Das
heißt, dass man zum Beispiel bei \opt{sortcites} den Wert weglassen kann, was
dann \kvopt{sortcites}{true} entspricht.

\subsubsection{Load-time-Optionen} \label{use:opt:ldt} 

Die folgenden Optionen müssen gesetzt sein, während \biblatex geladen wird.
Dabei werden sie in die optionalen Argumente von \cmd{usepackage} geschrieben.

\begin{optionlist}

\optitem[biber]{backend}{\opt{bibtex}, \opt{bibtex8}, \opt{bibtexu}, \opt{biber}}

Einzelheiten der Datenbasenbackends. Die folgende Backends werden unterstützt. 

\begin{valuelist}

\item[biber] \biber, das Standardbackend von \biblatex, unterstützend Ascii, 8-bit Encodings, \utf, on-the-fly-Reencoding, lokale"=spezifische Sortierung und viele andere
Dinge. Lokale"=specifische Sortierung, Fall"=sensitive Sortierung und 
aufwärts \slash vorrangige Prozesse werden mit den Optionen \opt{sortlocale}, \opt{sortcase} and \opt{sortupper} gesteuert.

\item[bibtex] Ererbtes \bibtex. Traditionelles \bibtex unterstützt nur Ascii-Enco\-ding.  Die Sortierung ist immer Fall"=insensitive.

\item[bibtex8] \bin{bibtex8}, die 8-bit-Implementation von \bibtex, unterstützt
Ascii- und 8-bit-Encodings, wie beispielsweise Latin~1. 

\end{valuelist}

Sehen Sie in \secref{use:bibtex} nach Details für das Nehmen von \bibtex als Backend. 

\valitem[numeric]{style}{file} 

Lädt den Bibliogragrafiestil \path{file.bbx} sowie die Literaturverweisregel
\path{file.cbx}. Einen Überblick zu den Standardstilen finden Sie in
\secref{use:xbx}.

\valitem[numeric]{bibstyle}{file}

Lädt den Bibliografiestil \path{file.bbx}. Einen Überblick zu den
Standard-Bibliografiestilen finden Sie in \secref{use:xbx:bbx}.

\valitem[numeric]{citestyle}{file}

Lädt die Literaturverweisregel \path{file.cbx}. Einen Überblick zu den
Standard-Literatur\-verweisregel finden Sie in \secref{use:xbx:cbx}.

\boolitem[false]{natbib}

Lädt das Kompatibilitätsmodul, welches Pseudonyme für die Befehle der
Literaturverweisregeln des \sty{natbib}-Pakets zur Verfügung stellt. Für weitere
Details siehe \secref{use:cit:nat}.

\boolitem[false]{mcite}

Lädt ein Literaturverweisregelmodul, das Befehle zur Verfügung stellt, die
ähnlich wie bei \sty{mcite}\slash\sty{mciteplus} funktionieren. Für Details
siehe \secref{use:cit:mct}.

\end{optionlist}

\subsubsection{Präambeloptionen} \label{use:opt:pre}

\paragraph{Allgemeines} \label{use:opt:pre:gen} \hfill \\ 

Die nachfolgenden Optionen
können Argumente zu \cmd{usepackage} sein oder aber auch in der
Konfigurationsdatei oder der Präambel stehen. Die Standardwerte, die unten
aufgelistet sind, sind auch die Standardwerte für das Paket. Beachten Sie, dass
die Bibliografie- und die Literaturverweisregelstile auch die
Standardeinstellungen beim Laden verändern können. Für Details schauen Sie in
\secref{use:xbx}.

\begin{optionlist}

\optitem[nty]{sorting}{\opt{nty}, \opt{nyt}, \opt{nyvt}, \opt{anyt},
\opt{anyvt}, \opt{ynt}, \opt{ydnt}, \opt{none}, \opt{debug}, \prm{name}}

Die Sortierung in der Bibliografie: Wenn nicht anders beschrieben, dann sind die
Einträge aufsteigend geordnet. Die folgenden Möglichkeiten sind standardmäßig
benutzbar.

\begin{valuelist} 
\item[nty] Sortiert nach Name, Titel, Jahr.  
\item[nyt]
Sortiert nach Name, Jahr, Titel.  
\item[nyvt] Sortiert nach Name, Jahr, Ausgabe,
Titel.  
\item[anyt] Sortiert nach dem alphabetischem Etikett, Name, Jahr, Titel.
\item[anyvt] Sortiert nach dem alphabetischem Etikett, Name, Jahr, Ausgabe,
Titel.  
\item[ynt] Sortiert nach Jahr, Name, Titel.  
\item[ydnt] Sortiert nach
Jahr (absteigend), Name, Titel.  
\item[none] Nicht sortieren. Alle Einträge
werden in der Reihenfolge der Literaturverweise angegeben. 
\item[debug]
Sortiert nach dem Eintragungsschlüssel. Nur für die Fehlersuche bestimmt.
\item[\prm{name}]  Benutze \prm{name} wie in
\cmd{DeclareSortingScheme} definiert (§ 4.5.6; engl. Version) %\secref{aut:ctm:srt}) 
\end{valuelist}

Egal welches <alphabetisches> Sortierschema man benutzt, es macht nur Sinn in
Verbindung mit einem Bibliografiestil, der die entsprechenden Etiketten
ausgibt. Beachten Sie, dass einige Bibliografiestile diese Paketoptionen mit
einem Wert starten, der vom Paketstandard abweicht (\opt{nty}). Für weitere
Details schauen Sie in \secref{use:xbx:bbx}. Für eine genaue Erklärung, der oben
aufgelisteten Sortieroptionen sowie zum Thema Sortierprozess, schauen sie
bitte in \secref{use:srt}. Um vordefinierte Schemen zu adaptieren oder neue zu
definieren, schauen Sie in \\ § 4.5.6 (engl. Version). %\secref{aut:ctm:srt}.

\boolitem[true]{sortcase} 

Einerlei ob die Bibliografie zu sortieren ist und die Kürzelliste "`case"=sensitively"' ist.

\boolitem[true]{sortupper} 

Diese Option entspricht dem \biber-Kommandozeilen-Befehl |--sortupper|.  Ist die Option aktiviert, wird die
Bibliografie nach <Großschreibung vor Kleinschreibung> sortiert.
Deaktiviert, bedeutet genau anders herum.

\optitem{sortlocale}{\opt{auto}, \prm{locale}}[\BiberOnly]

Diese Option setzt die lokale Sortierung auf lokal. Jedes Sortierungsschema
erbt dies lokal, wenn es keine Spezifizierung der \prm{locale}-Option für \cmd{printbibliography} gibt. Fordert diese \opt{auto}, dass für das Hauptdokument
die Babel/Polyglossia-Sprachenerkennung eingestellt wird, dann nimmt das Paket 
diese und ansonsten \texttt{en\_US}. \biber wird die 
Babel/Polyglossia-Sprachidentifikatoren 
in vernünftige lokale Identifikatoren abbilden (sehen Sie dazu die
\biber-Dokumentation). Sie können daher die Spezifikation entweder mit
einem üblichen lokalen Identifikator wie \texttt{de\_DE\_phonebook}, \texttt{es\_ES} 
oder mit dem unterstützten Babel/Polyglossia-Sprachidentifikatoren vornehmen,
wenn es für Sie in Ordnung ist, dass \biber dies tut.

\boolitem[true]{related}

Unabhängig davon, ob Informationen von verbundenen Einträgen zu nehmen sind oder nicht. Sehen Sie weiter in \secref{use:rel}.

\boolitem[false]{sortcites} 

Unabhängig davon, ob Zitate zu sortieren sind oder nicht, wenn mehrere Eintragschlüssel 
zu einem Zitierbefehl passen.
Ist diese Option aktiviert, werden die Literaturverweise nach
nach der Reihenfolge der Bibliografie sortiert (sehen Sie \secref{use:bib:context}).
Dieser Befehl funktioniert mit allen Literaturverweisstilen.

\intitem[3]{maxnames} 

Eine Grenze, die alle Namenslisten (\bibfield{author}, \bibfield{editor}, usw.)
beeinflusst. Sollte eine Liste diese Grenze überschreiten, das heißt, wenn sie
mehr als \prm{integer}-Namen beinhaltet, wird sie automatisch abgeschnitten, je
nach dem wie die Einstellungen der \opt{minnames} sind. \opt{maxnames} ist
die Masteroption, die sowohl \opt{maxbibnames} als auch \opt{maxcitenames} setzt.

\intitem[1]{minnames} 

Eine Begrenzung, die alle Namenslisten (\bibfield{author}, \bibfield{editor},
usw.) beeinflusst. Sollte eine Liste mehr als \prm{maxnames} beinhalten, wird
sie automatisch auf \prm{minnames}. Der \prm{minnames}-Wert muss
natürlich kleiner oder gleich \prm{maxnames} sein. \opt{minnames} ist die
Masteroption, die sowohl \opt{minbibnames} als auch \opt{mincitenames} setzt.

\intitem[\prm{maxnames}]{maxbibnames}

Ähnlich wie \opt{maxnames}, jedoch nur die Bibliografie beeinflussend.

\intitem[\prm{minnames}]{minbibnames} 

Ähnlich wie \opt{minnames}, jedoch nur die Bibliografie beeinflussend.

\intitem[\prm{maxnames}]{maxcitenames} 

Ähnlich wie  \opt{maxnames}, jedoch nur die Zitierungen im Dokument
beeinflussend.

\intitem[\prm{minnames}]{mincitenames} 

Ähnlich wie  \opt{minnames}, jedoch nur die Zitierungen im Dokument
beeinflussend.

\intitem[3]{maxitems} 

Ähnlich wie \opt{maxnames}, aber beeinflusst alle Literaturlisten
(\bibfield{publisher}, \bibfield{location}, usw.).

\intitem[1]{minitems} 

Ähnlich zu \opt{minnames}, aber beeinflusst alle Literaturlisten
(\bibfield{publisher}, \bibfield{location}, usw.).

\optitem{autocite}{\opt{plain}, \opt{inline}, \opt{footnote}, \opt{superscript},
\opt{...}} 

Diese Option kontrolliert das Verhalten des \cmd{autocite}-Befehls, wie er in
\secref{use:cit:aut} behandelt wird. Die Option \opt{plain} sorgt dafür, dass sich
\cmd{autocite} wie \cmd{cite} verhält, \opt{footnote} verhält sich dann wie
\cmd{footcite} und \opt{superscript} wie \cmd{supercite}. Die Optionen
\opt{plain}, \opt{inline} und \opt{footnote} sind immer verfügbar, die
\opt{superscript}-Option wird nur von den numerischen Literaturverweisstilen,
welche diesem Paket beiliegen, zur Verfügung gestellt. Der Literaturverweisstil
kann auch zusätzliche Optionen definieren. Die Standardeinstellungen der Optionen
hängen vom gewählten Literaturverweisstil ab, beachten Sie dazu
\secref{use:xbx:cbx}.

\boolitem[true]{autopunct} 

Diese Option kontrolliert, ob die Literaturverweisbefehle nach Satzzeichen
vorwärts gerichtet suchen sollen. Für Details beachten Sie § 3.8  %\secref{use:cit} und
\cmd{DeclareAutoPunctuation} in § 4.7.5 %\secref{aut:pct:cfg}.

\optitem[autobib]{language}{\opt{autobib}, \opt{autocite}, \opt{auto}, \prm{language}}

Diese Option kontrolliert die Unterstützung von Mehrsprachigkeit. Auf \opt{autobib}, \opt{autocite} oder \opt{auto} gestellt, wird \biblatex 
versuchen, die Hauptsprache des Dokuments aus
\sty{babel}/\sty{polyglossia} zu benutzen (und automatisch Englisch nehmen, 
sollte \sty{babel}/\sty{polyglossia}
nicht vorhanden sein). Es ist auch möglich, die
Dokumentsprache manuell auszuwählen. In diesem Fall wird \bibfield{langid}
 die Eintrage überschreiben und Sie sollten noch eine Umgebung für die Sprachumschaltung mit der Option \opt{autolang} wählen, um festzulegen, wie der Schalter auf die manuell gewählte Sprache umgangen wird.
Bitte beachten Sie \tabrefe{use:opt:tab1} für
eine Liste der unterstützten Sprachen und ihren entsprechenden Identifikator.
\opt{autobib} schaltet die Sprache für jeden Eintrag in der Bibliografie an, das
Nehmen des Feldes \bibfield{langid} und der Sprachenumgebung spezifiziert
die \opt{autolang}-Option. \opt{autocite} schaltet die Sprache für jede
Zitierung an, durch das Nehmen des \bibfield{langid}-Felds und der
Sprachenumgebung.  
\opt{auto} ist eine Abkürzung,
um sowohl \opt{autobib} und \opt{autocite} zu setzen. 

\boolitem[true]{clearlang} 

Wenn diese Option aktiviert ist, wird \biblatex das \bibfield{language}-Feld
von allen Einträgen, dessen Sprache mit der \sty{babel}-Sprache übereinstimmt,
leeren (oder die Sprache, die durch die \opt{language}-Option genauer
spezifiziert ist), um sich wiederholende Spracheinträge zu vermeiden. Die
Sprachkarten, die dafür benötigt werden, werden vom
\cmd{DeclareRedundantLanguages}-Befehl aus § 4.9.1 
(engl. Version) %\secref{aut:lng:cmd}
zur Verfügung gestellt.

\optitem[none]{autolang}{\opt{none}, \opt{hyphen}, \opt{other}, \opt{other*}, \opt{langname}}

Diese Option kontrolliert die \sty{babel}-Sprachumgebung\footnote{\sty{polyglossia} verwendet die \sty{babel}-Sprachumgebungen auch und
so verwendet diese Option beide, die \sty{babel}- und die \sty{polyglossia}-Sprachumgebungen.}, wenn  das \sty{babel}/\sty{polyglossia}-Paket geladen ist und ein
Bibliografieeintrag enthält ein \bibfield{langid}-Feld (beachten Sie \secref{bib:fld:spc}). Berücksichtigen Sie, \biblatex asst sich automatisch der Dokumentsprache an, wenn   \sty{babel}/\sty{polyglossia} geladen wurde. In mehrsprachigen Dolumenten wird es sich auch kontinuierlich anpassen, soweit es Zitate und die Bibliografiestandardsprache der aktuellen Sprache betrifft. Die Wirkung der Spracheinstellung hängt von der Sprachumgebung ab, die durch diese Option ausgewählt
wird. Die Wahlmöglichkeiten sind:

\begin{valuelist}

\item[none] Schaltet die Funktion aus, d.\,h. es benutzt gar keine Sprachumgebung.

\item[hyphen] Fügt den Eintrag in eine \env{hyphenrules}-Umgebung. Dies wird das
Verhalten der Silbentrennung für die gewählte Sprache in das
\env{hyphenation}-Feld des Eintrags laden, so vorhanden.

\item[other] Fügt den Eintrag in eine \env{otherlanguage}-Umgebung. Diese wird
das Verhalten der Silbentrennung für die gewählte Sprache laden, alle
zusätzlichen Definitionen, welche \sty{babel}/\sty{polyglossia} und \biblatex für die
entsprechende Sprache zur Verfügung stellen, werden aktiviert, und
Schlüsselelemente wie <editor> und <volume> werden übersetzt. Diese extra
Definitionen beinhalten Lokalisationen der Datenformatierung, Ordinale und
ähnliche Dinge.

\item[other*] Fügt den Eintrag in eine \env{otherlanguage*}-Umgebung. Bitte
beachten Sie, dass \biblatex \env{otherlanguage*} behandelt wie
\env{otherlanguage}, aber andere Pakete könnten hier eine Unterscheidung machen.

\item[langname]
\sty{polyglossia} nur. Setzen Sie den Eitrag in eine Umgebung \env{<languagename>}. 
Der Vorteil dieses Optionenwertes für \sty{polyglossia}-Nutzer ist, dass dieser Kenntnis nimmt vom \bibfield{langidopts}-Feld, sodass sie pro Sprache Optionen in den Eintrag hinzufügen können (wie eine Sprachvariante auszuwählen). Wenn \sty{babel} genommen wird, 
hat diese Option den selben Wert wie der Optionenwert \opt{other}.

\end{valuelist}

\optitem[none]{block}{\opt{none}, \opt{space}, \opt{par}, \opt{nbpar}, \opt{ragged}}

Diese Option kontrolliert den extra Raum zwischen den einzelnen Einträgen,
d.\,h. die Abstände zwischen den Bibliografieblöcken. Dabei stehen folgende
Möglichkeiten zur Verfügung:

\begin{valuelist}

\item[none] Füge nichts hinzu.

\item[space] Füge zusätzlichen horizontalen Raum zwischen den Einträgen hinzu.
Das hat einen ähnlichen Effekt wie das Standardverhalten der \latex-
Dokument-Klassen.

\item[par] Beginne einen neuen Paragraphen für jeden Eintrag. Das hat einen
ähnlichen Effekt wie die \opt{openbib}-Option des Standardverhaltens der
\latex-Dokument-Klassen.

\item[nbpar] Ähnlich der \opt{par}-Option, aber verbietet Seitensprünge zwischen
den Übergängen und innerhalb der Einträge.

\item[ragged] Fügt einen kleinen Strafraum ein, um Zeilenumbrüche an Blockgrenzen 
zu fördern und die Bibliografie rechts orientiert (Flattersatz) zu setzen.

\end{valuelist} 

Der Befehl \cmd{newblockpunct} kann auch verändert werden, um verschiedene
Resultate zu erzielen, sehen Sie \secref{use:fmt:fmt}. Beachten Sie auch § 4.7.1 (engl. Version) % \secref{aut:pct:new} 
für weitere Informationen dazu.

\optitem[foot+end]{notetype}{\opt{foot+end}, \opt{footonly}, \opt{endonly}}

Diese Option kontrolliert das Verhalten von \cmd{mkbibfootnote},
\cmd{mkbibendnote} und ähnliche Wrapper aus § 4.10.4 
(engl. Version) %\secref{aut:fmt:ich}. 
Die mög\-li\-chen Varianten sind:

\begin{valuelist} 
\item[foot+end] Unterstützt beides, Fuß- und Endnoten, d.\,h.
\cmd{mkbibfootnote} erstellt Fußnoten und \cmd{mkbibendnote} Endnoten.
\item[footonly] Erzwingt Fußnoten, d.\,h. auch \cmd{mkbibendnote} erstellt
Fußnoten.  
\item[endonly] Erzwingt Endnoten, d.\,h. auch \cmd{mkbibfootnote}
erstellt Endnoten.  \end{valuelist}

\optitem[auto]{hyperref}{\opt{true}, \opt{false}, \opt{auto}, \opt{manual}} 

Ermöglicht Literaturverweise und vice versa die Endreferenzen, zu anklickbaren
Links zu machen. Benötigt das \sty{hyperref}-Paket. Außerdem muss der gewählte
Literaturverweisstil Hyperlinks unterstützen, die Standardstile aus diesem Paket
tun dies bereits. Der Befehl \kvopt{hyperref}{auto} erkennt automatisch, ob das
\sty{hyperref}-Paket geladen wurde. Dies ist die Standardeinstellung. \kvopt{hyperref}{false} deaktiviert explizit Links, selbst wenn \sty{hyperref}
geladen ist. \kvopt{hyperref}{true} aktiviert Links, wenn \sty{hyperref} geladen
ist. Es kann keine expliziten Links aktivieren, wenn \sty{hyperref} nicht
geladen ist. Daher funktioniert es wie \kvopt{haperref}{auto}, außer es wird eine Warnung ausgegeben, wenn \sty{hyperref} nicht geladen ist. \kvopt{hyperref}{manuel} bietet die vollständige manuelle Kontrolle über die \sty{hyperref}-Interaktionen. Sie sollte von den Paketautoren nur unter ganz besonderen Umständen benötigt werden. Mit der Einstellung \kvopt{hyperref}{manual}
ist man selbst für sie Unterstützung von \sty{hyperref} mit \cmd{BiblatexManualHyperrefOn} oder \cmd{BiblatexManualHyperrefOff} verantwortlich,
man muss es selbst aktivieren bzw. deaktivieren. Einer der beiden Befehle
muss genau einmal eingegeben werden. \cmd{BiblatexManualHyperrefOn} kann nur
ausgeführt werden, wenn \sty{hyperref} geladen wurde.

\boolitem[false]{backref} 

Entscheidet, ob die Endreferenzen in die Bibliografie geschrieben werden
sollen. Die Endreferenzen sind eine Liste von Seitenzahlen, die angeben, auf welchen
Seiten der entsprechende Bibliografieeintrag zitiert wird. Sollten
\env{refsection}-Umgebungen im Dokument vorhanden sein, werden die Endreferenzen
an die gleiche Stelle gesetzt wie die Referenz-Einträge. Um genau zu sein,
kontrolliert diese Option nur, ob das \biblatex-Paket die Daten sammeln
soll, die es braucht, um solche Endreferenzen zu erzeugen. Natürlich muss auch
der gewählte Bibliografiestil diese Möglichkeit unterstützen. Alle Standardstile, die
mit diesem Paket versendet werden, tun dies.

\optitem[three]{backrefstyle}{\opt{none}, \opt{three}, \opt{two}, \opt{two+},
\opt{three+}, \opt{all+}} 

Diese Option kontrolliert, wie die Sequenz der fortlaufenden Seiten in der Liste
der Endreferenzen formatiert werden. Dabei sind folgende Stile möglich:

\begin{valuelist}

\item[none] Dieses Feature deaktivieren, d.\,h. die Seitenliste nicht kürzen.

\item[three] Jegliche Sequenz von drei oder mehr aufeinander folgenden Seiten zu
einem Bereich kürzen, z.\,B. die Liste <1, 2, 11, 12, 13, 21, 22, 23, 24> wird
zu <1, 2, 11--13, 21--24>.

\item[two] Jegliche Sequenz von zwei oder mehr aufeinander folgenden Seiten zu
einem Bereich kürzen, z.\,B. die obere Liste zu <1--2, 11--13, 21--24>..

\item[two+] Ähnlich dem Konzept von \opt{two}, aber die Sequenz von zwei genau
aufeinander folgenden Seiten wird ausgegeben und benutzt die Startseite und den
lokalen Folge \texttt{sequens}, z.\,B. wird die obere Liste zu <1\,sq., 11--13,
21--24>.

\item[three+] Ähnliche dem Konzept von \opt{two+}, aber die Sequenz von drei
genau aufeinanderfolgenden Seiten wird ausgegeben und benutzt die Startseite und
den lokalen Folge \texttt{sequens}, z.B. wird die obere Liste zu <1\,sq.,
11\,sqq., 21--24>.

\item[all+] Ähnlich dem Konzept von \opt{three+}, aber die Sequenz aufeinander
folgender Seiten wird als offener, endloser Bereich ausgegeben, z.\,B. wird die
obere Liste zu <1\,sq., 11\,sqq., 21\,sqq.> zusammengefasst.

\end{valuelist} 

Alle Stile unterstützten sowohl arabische als auch römische Zahlen. Um
Verwechslungen bei den Listen zu vermeiden, werden verschiedene Sets von Ziffern
beim Erstellen der Bereiche nicht vermischt, z.\,B. wird die Liste <iii, iv, v,
6, 7, 8> als <iii--v, 6--8> zusammengefasst.

\optitem[setonly]{backrefsetstyle}{\opt{setonly}, \opt{memonly}, \opt{setormem},
\opt{setandmem}, \opt{memandset}, \opt{setplusmem}} 

Diese Option kontrolliert, wie die Endreferenzen mit den \bibtype{set}-Einträgen
und ihren Mitgliedern umgehen werden. Dabei sind die folgenden Optionen
verfügbar:

\begin{valuelist}

\item[setonly] Alle Endreferenzen werden zu \bibtype{set}-Einträgen hinzugefügt.
Die \bibfield{pageref}-Liste der eingesetzten Mitglieder bleibt leer.

\item[memonly] Referenzen der eingesetzten Mitglieder werden dem entsprechenden
Mitglied hinzugefügt. Referenzen der \bibtype{set}-Einträge werden zu allen
Mitgliedern hinzugefügt. Die \bibfield{pageref}-Liste der \bibtype{set}-Einträge
bleibt leer.

\item[setormem] Referenzen des \bibtype{set}-Eintrags werden dem
\bibtype{set}-Eintrag hinzugefügt. Referenzen zu eingesetzten Mitgliedern werden
dem entsprechenden Eintrag hinzugefügt.

\item[setandmem] Referenzen der \bibtype{set}-Einträge werden dem
\bibtype{set}-Eintrag hinzugefügt. Referenzen zu eingesetzten Mitgliedern werden
dem \bibtype{set}-Eintrag und dem entsprechenden Mitglied hinzugefügt.

\item[memandset] Referenzen der \bibtype{set}-Einträge werden dem
\bibtype{set}-Eintrag und allen Mitgliedern hinzugefügt. Referenzen der
gesetzten Mitgliedern werden dem entsprechenden Mitglied hinzugefügt.

\item[setplusmem] Referenzen der \bibtype{set}-Einträge werden dem \bibtype{set}
Eintrag und allen Mitgliedern hinzugefügt. Referenzen der eingesetzten
Mitglieder werden dem entsprechenden Mitglied und dem \bibtype{set}-Eintrag
hinzugefügt.

\end{valuelist}

\optitem[false]{indexing}{\opt{true}, \opt{false}, \opt{cite}, \opt{bib}}

Diese Option kontrolliert die Katalogisierung in Literaturverweisen und der
Bibliografie. Genauer gesagt, beeinflusst sie der \cmd{ifciteindex}- und
\cmd{ifbibindex}-Befehl aus § 4.6.2 (engl. Version) %\secref{aut:aux:tst}. 
Die Option kann für alles,
eine Pro-Typ- oder eine Pro-Eintrag-Basis gesetzt werden. Dabei stehen folgende
Möglichkeiten zur Verfügung:

\begin{valuelist} 
\item[true] Aktiviert Katalogisierung für alles.  
\item[false] Deaktiviert Katalogisierung für alles.  
\item[cite] Aktiviert Katalogisierung
nur für Literaturverweise.  
\item[bib] Aktiviert Katalogisierung nur für die
Bibliografie.  
\end{valuelist} 
%
Dieses Feature muss vom gewählten Literaturverweisstil unterstützt werden. Alle
Standardstile, die diesem Paket beiliegen, unterstützten die Erstellung von
Katalogen sowohl für Literaturverweise als auch für die Bibliografie. Beachten
Sie, dass Sie immer mit \cmd{makeindex} das Erstellen eines Katalogs aktivieren
müssen, um ihn angezeigt zu bekommen.

\boolitem[false]{loadfiles}

Diese Option kontrolliert, ob externe Dateien durch den \cmd{printfile}-Befehl
angefordert werden. Für Details schauen Sie auch in \secref{use:use:prf} und
\cmd{printfile} in § 4.4.1 (engl. Version) %\secref{aut:bib:dat}. 
Beachten Sie, dass dieses Feature
standardmäßig deaktiviert ist, um Leistungseinbrüche zu verhindern.

\optitem[none]{refsection}{\opt{none}, \opt{part}, \opt{chapter}, \opt{section},
\opt{subsection}}

Diese Option startet automatisch eine Referenzsektion an einer
Dokumentenaufspaltung, wie zum Beispiel bei einem Kapitel oder einer Sektion.
Das entspricht dem \cmd{newrefsection}-Befehl. Für Details schauen Sie in
\secref{use:bib:sec}. Dabei stehen folgende Möglichkeiten für die
Dokumentenaufspaltung zur Verfügung:

\begin{valuelist} 
\item[none] Deaktiviert dieses Feature.  
\item[part] Beginnt eine Referenzsektion an jedem \cmd{part} Befehl.  
\item[chapter] Beginnt eine
Referenzsektion an jedem \cmd{chapter} Befehl.	
\item[chapter+] Beginnt eine Referenzsektion bei jedem  \cmd{chapter} 
    und jeder höheren Ebene der Untergliederung, d.\,h- \cmd{part}.
\item[section] Beginnt eine neue
Referenzsektion an jedem \cmd{section} Befehl. 
\item[section+] Beginnt eine Referenzsektion bei jeder \cmd{section} und jeder höheren Ebene der Untergliederung, d.\,h. \cmd{part} und \cmd{chapter} (falls verfügbar).
\item[subsection] Beginnt eine
neue Referenzsektion an jedem \cmd{subsection} Befehl. 
\item[subsection+] Beginnt eine Referenzsektion bei jeder\cmd{subsection} 
und jeder höheren Ebene der Untergliederung, d.\,h. \cmd{part}, \cmd{chapter} 
    (falls verfügbar) und \cmd{section}.

\end{valuelist}
%
Die Sternchenversionen dieser Kommandos werden keine neue Referenzsektion
beginnen.

\optitem[none]{refsegment}{\opt{none}, \opt{part}, \opt{chapter}, \opt{chapter+}, \opt{section}, \opt{section+}, \opt{subsection}, \opt{subsection+}}

Ähnlich der \opt{refsection}-Option, aber startet dabei ein neues
Referenzsegment. Dies entspricht einem \cmd{newrefsegment}-Befehl. Beachten Sie
auch \secref{use:bib:seg} für Details. Sollten Sie beide Optionen benutzen,
beachten Sie, dass Sie diese Option nur unter einer niedrigeren
Dokumentebene als der \opt{refsection} verwenden können, und dass
verschachtelten Referenzsegmente nur lokal auf die eingekapselten
Referentensektion angewandt werden.

\optitem[none]{citereset}{\opt{none}, \opt{part}, \opt{chapter}, \opt{chapter+}, \opt{section}, \opt{section+}, \opt{subsection}, \opt{subsection+}}
\optitem[none]{citereset}{\opt{none}, \opt{part}, \opt{chapter}, \opt{chapter+}, \opt{section}, \opt{section+}, \opt{subsection}, \opt{subsection+}}

Diese Option führt automatisch den \cmd{citereset}-Befehl von
\secref{use:cit:msc} an einer Dokumentenaufspaltung, wie einem Kapitel oder
einer Sektion, aus. Dabei stehen die folgenden Möglichkeiten für die
Dokumentenaufteilung zur Verfügung:

\begin{valuelist} 
\item[none] Deaktivieren des Features.  
\item[part] Führt ein
Reset an jedem \cmd{part} Befehl aus.  
\item[chapter] Führt ein Reset an jedem
\cmd{chapter} Befehl aus.
\item[chapter+] Führt ein Reset an jedem \cmd{chapter} und \cmd{part} Befehl aus.
\item[section] Führt ein Reset an jedem \cmd{section}
Befehl aus. 
\item[section+] Führt ein Reset an jedem \cmd{section}, \prm{chapter} (wenn
    von der Klasse unterstützt) und \cmd{part} Befehl aus.
\item[subsection] Führt ein Reset an jedem \cmd{subsection} Befehl
aus. 
\item[subsection+] Führt ein Reset an jedem \cmd{subsection}, \cmd{section}, \prm{chapter} (wenn von der Klasse unterstützt) und \cmd{part} Befehl aus.
\end{valuelist}
%

\boolitem[true]{abbreviate} 

Entscheidet, ob lange oder verkürzte Abfolgen in den Literaturverweisen und der
Bibliografie benutzt werden. Diese Option betrifft die lokalisierten Module.
Sollte sie aktiviert sein, werden Schlüsseleinträge wie <editor> verkürzt. Wenn
nicht, dann werden sie ausgeschrieben. Diese Option kann auch per-Typ oder 
per-Eintrag festgelegt werden.

\optitem[comp]{date}{\opt{year}, \opt{short}, \opt{long}, \opt{terse}, \opt{comp}, \opt{ymd}, \opt{iso}}

Diese Option kontrolliert das Basisformat der ausgegebenen Datenspezifikationen.
Dabei stehen folgende Möglichkeiten zur Verfügung:

\begin{valuelist} 
\item[year] Benutzt nur Jahr, beispielsweise:\par
2010\par
2010--2012\par
\item[short] Benutze das kurze, wortkarge Format zur
Abgrenzung: \par 01/01/2010\par 21/01/2010--30/01/2010\par
01/21/2010--01/30/2010 
\item[long] Benutze das lange, wortkarge Format zur
Abgrenzung: \par 1st January 2010\par 21st January 2010--30th January 2010\par
January 21, 2010--January 30, 2010\par 
\item[terse] Benutze das kurze Format mit
kompakter Abgrenzung: \par 21--30/01/2010\par 01/21--01/30/2010 
\item[comp]
Benutze das lange Format mit kompakter Abgrenzung: \par 21st--30th January
2010\par January 21--30, 2010\par 
\item[iso] Nimmt das ISO8601 Extended Format (\texttt{yyyy-mm-dd}), 
    beispielsweise:\par
2010-01-01\par
2010-01-21/2010-01-30
\item[ymd] Ein Jahr-Monat-Tag-Format, das modifiziert sein kann, mit anderen
Optionen als das strikte EDTF, beispielsweise:\par
2010-1-1\par
2010-1-21/2010-1-30
\end{valuelist} 
%
Beachten Sie, dass das \opt{iso}-Format erzwungen wird von \kvopt{dateera}{astronomical}, \kvopt{datezeros}{true}, \kvopt{timezeros}{true}, \kvopt{seconds}{true}, \kvopt{$<$datetype$>$time}{24h} und \kvopt{julian}{false}. \opt{ymd} ist ein ETFT-artiges Format, das aber verschiedene Optionen wechseln kann, die die strikte \opt{iso}-Option nicht gestattet.

Wie in den Beispielen oben ersichtlich, ist das tatsächliche Format von der
Sprache abhängig. Beachten Sie, dass der Monatsname in allen langen Formaten auf
die \opt{abbreviate}-Paketoption reagiert. Die vorangestellten Nullen für Monate und Tage in allen
kurzen Formaten können einzeln durch die \opt{datezeroes}-Paketoption
kontrolliert werden. Die vorangestellten Nullen für Stunden, Minuten und Sekunden in allen kurzen Formaten können separat mit der Paketoption \opt{timezeros}
gesteuert werden. Beim Ausdrucken von times, wird das Ausgeben von Sekunden und Zeitzonen über die Optionen \opt{seconds} und \opt{timezones} gesteuert.

Die Optionen \opt{julian} und \opt{gregorianstart}  kann man zur Steuerung benutzen, wenn Julian-Kalenderdaten verwendet werden.

\optitem[year]{datelabel}{\opt{year}, \opt{short}, \opt{long}, \opt{terse}, \opt{comp}, 
\opt{ymd}, \opt{edtf}}

Ähnlich der \opt{date}-Option, kontrolliert aber das Format des date-Felds, ausgewählt mit \cmd{DeclareLabeldate}.

\optitem[comp]{$<$datetype$>$date}{\opt{year}, \opt{short}, \opt{long}, \opt{terse}, \opt{comp}, \opt{ymd}, \opt{edtf}}

Ähnlich der \opt{date}-Option, kontrolliert aber das Format des 
\bibfield{$<$datetype$>$date}-Felds im Datenmodell.

\optitem{alldates}{\opt{year}, \opt{short}, \opt{long}, \opt{terse}, \opt{comp}, \opt{edtf}}

Setzt die Option für alle Daten des Datenmodells auf den gleichen Wert. Die Datenfelder
des Standarddatenmodells sind \bibfield{date}, \bibfield{origdate}, \bibfield{eventdate}
und \bibfield{urldate}.

\boolitem[false]{julian}

Diese Option kontrolliert die Daten vor dem angegebenen Datum, die
\opt{gregorianstart}-Option wird automatisch in den Julianischen Kalender umwandeln. Daten, so geändert, gehen zurück zu <true> für \cmd{ifdatejulian} und \cmd{if$<$datetype$>$datejulian} testet (sehen Sie %\secref{aut:aux:tst}). 
4.6. (eng. Vers.).
Bitte beachten Sie, dass Daten, die aus gerade einem Jahr wie  <1565> bestehen,
niemals in ein Datum des Julianischen Kalenders umgewandelt werden,  da ein Datum
ohne einen Monat und Tag eine mehrdeutige Repräsentation im Julianischen Kalender hat\footnote{Dies ist möglicherweise zutreffend für Daten ohne Zeitangaben, aber dies
ist für bibliografische Arbeiten nicht relevant.}. Zum Beispiel, in dem Fall <1565>, 
ist das Julianische Jahr <1564>, bis nach <10. Januar 1565> des Gregorianischen Kalenders, wenn das Julianische Jahr <1565> wird.

\valitem{gregorianstart}{YYYY-MM-DD}

Diese Option kontrolliert das Datum vor den Tagen die in den Julianischen Kalender umgewandelt werden. Es ist ein strenger Formatstring, 4-stelliges Jahr, 
2-stelliger Monat und Tag, separariert von einem einzelnen Bindestrichzeichen
(jedes gültige Unicodezeichen mit <Strich>-Eigenschaft). Der Standard ist
'1582-10-15', das Datum der Veranlassung des Standard-Gregorianischen Kalenders. 
Diese Option veranlast nichts, wenn die Option \opt{julian} auf <true> gesetzt ist.

\boolitem[true]{datezeros} 

Diese Option steuert, ob bei \texttt{short} und \texttt{terse} die Daten
mit vorangestellten Nullen ausgeben werden.

\boolitem[false]{timezones}

Diese Option steuert, ob die Zeitzonen ausgegeben werdn, wenn die Zeiten ausgegeben werden.

\boolitem[false]{seconds}

Diese Option steuert, ob Sekunden ausgegeben werden, wenn die Zeiten ausgegeben werden.

\boolitem[true]{dateabbrev}

 Diese Option kontrolliert, ob die Daten bei \texttt{long} und \texttt{com} mit
 langen oder abgeschnittenen Monatsnamen ausgegeben werden. Diese Option ist
 ähnlich der allgemeinen \opt{abbreviate}-Option, aber spezifiziert das Format
 des Datums.

\boolitem[false]{datecirca}

Diese Option steuert, ob <circa> zur Ausgabe der Informationen zu den Daten kommt.
Wenn sie auf \opt{true} gesetzt ist, werden Daten mit dem \cmd{datecircaprint}-Makro 
erweitert (\secref{use:fmt:fmt}).

\boolitem[false]{dateuncertain}

Diese Option steuert, ob zur Ausgabe Unsicherheitsinformationen zu den Terminen 
kommen. Wenn sie auf \opt{true} gestzt ist, werden die Daten der Erweiterung
durch das \cmd{dateuncertainprint}-Makro folgen (\secref{use:fmt:fmt}).

\optitem[astronomical]{dateera}{\opt{astronomical}, \opt{secular}, \opt{christian}}

Diese Option steuert, wie Datumsepocheninformationen ausgegeben werden. 
<astronomical> verwendet \cmd{dateeraprintpre} zum Ausgeben der Epochen Information \emph{vor} Start/Enddatums. <sekular> und <christlisch> nehmen \cmd{dateeraprint} 
zum Ausgeben der Epocheninformation \emph{nach} den Start/Ende/Datums. Standardmäßig  führt <astronomical> zu einem Minuszeichen vor BCE/BC-Daten und <secular>/<christian> führt zu den relevanten Lokalisationsstrings wie <BCE> oder <BC> nachr BCE/BC-Daten. 
Sehen sie dazu die entsprechenden Kommentare in \secref{use:fmt:fmt} und 
zu den Lokalisationsstrings in \secref{aut:lng:key:dt}.

\intitem[0]{dateeraauto}

Diese Option setzt das astronomische Jahr, bei denen Epochenlokalisierungsstrings
automatisch hinzugefügt werden. Diese Option tut nichts, ohne dass \opt{dateera} auf <secular> oder <christian> gesetzt wird.

\optitem[24h]{time}{\opt{12h}, \opt{24h}, \opt{24hcomp}}

Diese Option steuert das Basisformat der auszugenenden Zeitspzifikationen.
Die folgenden Möglichkeiten stehen zur Verfügung:

\begin{valuelist}
\item[24h] 24-Stundenformat, beispielsweise:\par
14:03:23\par
14:3:23\par
14:03:23+05:00\par
14:03:23Z\par
14:21:23--14:23:45\par
14:23:23--14:23:45\par
\item[24hcomp] 24-Stundenformat mit komprimierten Bereichen, beispielsweise:\par
14:21--23 (Stunden sind die gleichen)\par
14:23:23--45 (hStunde und Minute sind die gleichen)\par
\item[12h] 12-Stundenformat mit (lokalisiert) AM/PM-Markern, beispielsweise:\par
2:34 PM\par
2:34 PM--3:50 PM\par
\end{valuelist}
%
Wie in den obigen Beispielen ersichtlich ist das tatsächliche Zeitformat sprachspezifisch. Beachten Sie, dass die AM/PM-Zeichenfolge ist eine Reaktion auf  die \opt{abbreviate}-Paketoption, wenn dies ein Differenz im spezifischen Ort ergibt. 
Die führenden Nullen im 24-Stundenformat können separat mit der 
\opt{timezeros}-Paketotion gesteuert werden. Der Separator zwischen den Zeitkomponenten (\cmd{bibtimesep} und \cmd{bibtzminsep}) und zwischen der Zeit und den beliebigen
Zeitzonen (\cmd{bibtimezonesep}) sind auch sprachspezifisch und anpassbar, sehen Sie \secref{use:fmt:lng}. Es gibt globale Paketoptionen, die bestimmen welche Sekunden
und Zeitzonen ausgegeben werden (\opt{seconds} und \opt{timezones}, sehen Sie jeweils \secref{use:opt:pre:gen}). Zeitzonen, falls vorhanden, sind entweder <Z> oder
eine eine positive oder negative Nummer. Kein Standardstil gibt Zeitinformationen aus.  Benutzerdefinierte Stile können die zeit ausgeben, wenn 
\cmd{print<datetype>time}-Befehle genutzt werden, sehen Sie § 4.4.1 (engl. Version).%\secref{aut:bib:dat}.

\optitem[24h]{labeltime}{\opt{12h}, \opt{24h}, \opt{24hcomp}}

Ähnlich der \opt{time}-Option,  steuert aber das Format der Zeitteilfelder, ausgewählt
aus dem Feld mit \cmd{DeclareLabeldate}.

\optitem[24h]{$<$datetype$>$time}{\opt{12h}, \opt{24h}, \opt{24hcomp}}

Ähnlich der \opt{time}-Option, steuert aber das Format der Zeitteilfelder, ausgewählt
aus dem Feld \bibfield{$<$datetype$>$date} im Datenfeld.

\optitem{alltimes}{\opt{12h}, \opt{24h}, \opt{24hcomp}}

Setzt die \opt{labeltime} und die \opt{$<$datetype$>$time}-Option für alle Zeiten im Datenmodell mit den gleichen Werten. Die Datenfelder unterstützenden Zeitteile
im Standarddatenmodel sind \bibfield{date}, \bibfield{origdate}, \bibfield{eventdate} und \bibfield{urldate}.

\boolitem[false]{dateusetime}

Gibt an, ob jede Zeitkomponente von einem Datumsfeld nach den Zeitkomponeneten
auszugeben ist. Der Separator zwischen dem Datum und Zeitkomponenten ist \cmd{bibdatetimesep} aus \secref{use:fmt:lng}. Diese Option tut nichts, wenn ein
kompaktes Datumsformat benutzt wird (sehen Sie \secref{use:opt:pre:gen}), da dies sehr verwirrend wäre. 

\boolitem[false]{labeldateusetime}

Ähnlich der \opt{dateusetime}-Option, steuert aber, ob Zeitkomponenten für das Feld
mit \cmd{DeclareLabeldate} zum Ausgeben ausgewählt werden..

\boolitem[false]{$<$datetype$>$dateusetime}

Ähnlich der \opt{dateusetime}-Option, steuert aber, ob die Zeitkomponenten für das
\bibfield{$<$datetype$>$date}-Feld im Datenmodell ausgegeben werden.

\boolitem[false]{alldatesusetime}

Setzt die \opt{labeldateusetime} und die \opt{$<$datetype$>$dateusetime}-Option 
für alle \bibfield{$<$datetype$>$date}-Felder im Datenmodell.


\boolitem[false]{defernumbers} 

Im Gegensatz zum Standard-\latex werden durch dieses Paket die nummerierten
Etiketten normalerweise an die komplette Liste der Referenzen am Anfang eines
Dokuments angehangen. Sollte diese Option aktiviert sein, werden die nummerierten
Etiketten (also die \bibfield{labelnumber} wie in %\secref{aut:bbx:fld}
§ 4.2.4 (engl. Vers.)
behandelt) zugeordnet, wenn ein Bibliografieintrag das erste Mal, in jeglicher
Bibliografie, ausgegeben wird. Für weitere Erklärungen beachten Sie auch
\secref{use:cav:lab}.
Diese Option erfordert zwei \latex-Durchläufe, nachdem die Daten exportiert wurden in die
\file{bbl}-Datei mit dem Backend (zusätzlich zu allen anderen Läufen durch 
Seitenumbrüche etc. erforderlich). Eine wichtige Sache ist zu beachten, wenn Sie den
Wert für diese
Option in ihrem Dokument ändern (oder den Wert von Optionen, welche von dieser abhängen, solche wie zu dem \cmd{printbibliography}-Makro, sehen Sie \secref{use:bib:bib}), dann ist es wahrscheinlich, dass Sie die zu ihrer aktuellen Datei gehörige \file{aux}-Datei löschen müssen und \latex erneut durchlaufen lassen müssen, um die korrekte Nummerierung zu erhalten. Sehen Sie auch § 4.1 (engl. Version). %\secref{aut:int}.

\boolitem[false]{punctfont}

Diese Option aktiviert einen alternativen Mechanismus, der benutzt wird, wenn
eine Einheit von Satzzeichen nach einem Feld, welches in einer anderen
Schriftart steht (z.\,B. ein Titel der kursiv geschrieben ist) genutzt wird. Für
Details beachten Sie \cmd{setpunctfong} in § 4.7.1 (engl. Version). %\secref{aut:pct:new}.

\optitem[abs]{arxiv}{\opt{abs}, \opt{ps}, \opt{pdf}, \opt{format}}

Wählt einen Pfad für arXiv-Verknüpfungen. Wenn Hyperlinks-Unterstützung
aktiviert ist, kontrolliert diese Option, auf welche Version des Dokuments die
arXiv-\bibfield{eprint}-Verknüpfung zeigt. Dabei stehen die folgenden Möglichkeiten zur Verfügung:

\begin{valuelist} 
\item[abs] Verlinkt den Entwurf der Seite.  
\item[ps] Verlinkt die PostScript Version.  
\item[pdf] Verlinkt die \pdf Version.  
\item[format] Verlinkt zur Formatauswahl-Seite.  
\end{valuelist} 
%
Für Details zum Support von arXiv und elektronische Veröffentlichungsinformationen
schauen Sie in § 3.13.7%\secref{use:use:epr}.

\optitem[auto]{texencoding}{\opt{auto}, \prm{encoding}}

Spezifiziert die Verschlüsselung der \file{tex}-Datei. Diese Option wirkt sich 
auf den 
Datentransfer vom Backend zu \biblatex aus. Bei der Verwendung von \biber entspricht
diese der \biber|--output_encoding|-Option von Biber. Zur Verfügung stehen 
folgende Wahlmöglichkeiten:

\begin{valuelist}
\item[auto] Versuchen Sie selbst die Eingabekodierung herauszufinden.
Wenn das\\
\sty{inputenc}\slash \sty{inputenx}\slash \sty{luainputenc}-Paket verfügbar
ist, erhält \biblatex die wichtigste Kodierung von diesem Paket. Wenn
nicht, nimmt es \utf-Kodierung, wenn \xetex oder \luatex erkannt wurden, an und
ansonsten dann Ascii.

\item[\prm{encoding}] Spezifiziert das \prm{encoding} explizit. Dies trifft für den seltenen Fall 
zu, in dem die automatische Erkennung fehlschlägt oder Sie aus einem
speziellen Grund eine bestimmte Kodierung erzwingen möchten.

\end{valuelist}
%
Beachten Sie, dass das Einstellen von \kvopt{texencoding}{\prm{encoding}}
sich auch auf die
\opt{bibencoding}-Option bei \kvopt{bibencoding}{auto} auswirkt.


\optitem[auto]{bibencoding}{\opt{auto}, \prm{encoding}} 

Spezifiziert die Verschlüsselung der \file{bib}-Datei. Bei der Verwendung von
Biber entspricht diese der |--input_encoding|-Option von Biber. Zur Verfügung stehen 
folgende Wahlmöglichkeiten:

\begin{valuelist}

\item[auto] Benutzen Sie diese Option, wenn der Arbeitsablauf transparent sein
soll, d.\,h., wenn die Verschlüsselung der \file{bib}-Datei identisch zur
Verschlüsselung der \file{tex}-Datei ist. 

\item[\prm{encoding}] Wenn das Encoding des \file{bib}-Files sich
unterscheidet von dem des \file{tex}-Files , geben Sie \prm{encoding} explizit an. 

\end{valuelist} 
%
\biblatex setzt normalerweise voraus, dass die \file{tex}- und
\file{bib}-Dateien die gleiche Kodierung haben (\kvopt{bibencoding}{auto}). 

\boolitem[false]{safeinputenc}

Wenn diese Option aktiviert ist, wird \biblatex automatisch \kvopt{texencoding}{ascii}
forciert, wenn das  \sty{inputenc}\slash \sty{inputenx}-Paket geladen wurde und
die Eingabekodierung \utf ist, d.\,h. es ignoriert alle \utf-basierten Makros
und nimmt nur Ascii. Biber wird dann versuchen, alle Nicht-Ascii-Daten in dem
Verzeichnis \file{bib}-Datei in Ascii-Kodierung umzuwandeln. Beispielsweise wird
es  \texttt{\d{S}} in |\d{S}| konvertieren. Siehe \secref{bib:cav:enc:enc} für
die Erläuterung, warum Sie vielleicht dies Option aktivieren.

\boolitem[true]{bibwarn} 

Im Normalfall wird \biblatex automatisch Probleme mit den Daten in der \file{bib}-Datei , diejenigen die Backends betreffen, wie \latex-Probleme übermitteln. Um diese abzustellen,
benutzen Sie diese Option.

\intitem[2]{mincrossrefs} 

Setzt das Minimum von Querverweisen auf \prm{integer} wenn ein Backend-Lauf
angefordert wird. \footnote{Sollte ein Eintrag, welcher von anderen Einträgen
als Querverweis angeben wird, in der \file{bib}-Datei auf einen Grenzwert
treffen, wird er in die Bibliografie aufgenommen, selbst wenn er nicht zitiert
wurde. Das ist eine Standardeigenschaft von \bibtex und nicht \sty{biblatex}
spezifisch. Beachten Sie die Beschreibung des \bibfield{crossref}-Felds in
\secref{bib:fld:spc} für weitere Informationen.} Diese Option wirkt sich auch auf den Umgang mit dem Feld \bibfield{xref} aus. Beachten Sie die Feld-Beschreibung in
\secref{bib:fld:spc} und \secref{bib:cav:ref} für Details.

\intitem[2]{minxrefs}

Wie \opt{mincrossrefs}, aber für  \bibfield{xref}-Felder.

\end{optionlist}

\paragraph{Stilspezifik} \label{use:opt:pre:bbx} 

Die folgenden Optionen werden von den Standardstilen (entgegen den Kernpaketen)
zur Verfügung gestellt. Technisch gesehen sind sie Präambeloptionen, wie die in
\secref{use:opt:pre:gen}.

\begin{optionlist}

\boolitem[true]{isbn} 

Diese Option kontrolliert, ob die Felder \bibfield{isbn}\slash
\bibfield{issn}\slash \bibfield{isrn} ausgegeben werden.

\boolitem[true]{url} 

Diese Option kontrolliert, ob das \bibfield{url}-Feld und die Zugangsdaten
ausgegeben werden. Die Option beeinflusst nur die Einträge, dessen
\bibfield{url}-Informationen optional sind. Das \bibfield{url}-Feld des
\bibtype{online} Eintrags wird immer mit ausgegeben.

\boolitem[true]{doi} Diese Option kontrolliert, ob das \bibfield{doi} mit
ausgegeben wird.

\boolitem[true]{eprint} Diese Option kontrolliert, ob die
\bibfield{eprint}-Information mit ausgegeben wird.

\boolitem[true]{related}

Gibt an, ob Informationen aus verwandten Einträgen verwendet werden sollen oder nicht. Siehe \secref{use:rel}.


\end{optionlist}

\subparagraph{\texttt{alphabetic}/\texttt{numeric}} Darüber hinaus,
Sile der \texttt{alphabetic}- und \texttt{numeric}-Familien unterstützen die Option \opt{subentry} global, per-Typ- and per-Eintrag-Bereich.

\begin{optionlist}

\boolitem[false]{subentry}

Diese Option wirkt sich auf das Behandeln von Zitaten für Gruppenmitglieder
und die Anzeige von Gruppen in der Bibliografie aus. Wenn diese Option aktiviert
ist, erhalten Zitate zu einzelnen Gruppenmitgliedern einen zusätzlichen Buchstaben,
der das Mitglied identifiziert. Dieser Buchstabe wird auch in der Bibliografie gedruckt. Wenn diese Option deaktiviert ist, wird ein Zitat für das Mitglied 
einer Gruppe nur als Zitat für das gesamte Set angezeigt, und die Bibliografieeinträge, in denen die Mitglieder aufgeführt sind, enthalten keinen zusätzlichen Buchstaben. 

Angenommen \texttt{key1} und \texttt{key2} sind Mitglieder des Sets 
\texttt{set1}. Wenn \opt{subentry} in einem numerischen Stil auf\texttt{true}
gesetzt wird, wirde ein Zitat mit \texttt{key1} angezeigt als <[1a]> und ein
Zitat mit \texttt{key2} als <[1b]>, da die gesamte Gruppe \texttt{set1} zitiert
wird als <[1]>. Außerdem werden <(a)> und <(b)> vor den Einträgen der Gruppenmitglieder im Bibliografieeintrag hinzugefügt. Wenn \opt{subentry}
auf \texttt{false} gesetzt ist, werden Zitate für alle drei Schlüssel als 
<[1]> angezeigt, kein zusätzlicher Buchstabe wird dann in der Bibliografie ausgedruckt.

\end{optionlist}

\subparagraph{\texttt{authortitle}/\texttt{authoryear}} Alle Bibliografiestile
der Familien \texttt{authoryear} und \texttt{authortitle} sowie alle Bibliografiestile der \texttt{verbose}-Familie -- deren Bibliografiestile
auf  der von \texttt{authortitle} basieren -- unterstützen die Option
\opt{dashed} im globalen Bereich.

\begin{optionlist}

\boolitem[true]{dashed}

Diese Option steuert, ob wiederkehrende Autoren-Schrägstriche \slash -Editorlisten in der Bibliografie durch einen Bindestrich ersetzt werden (\cmd{bibnamdeash}, 
siehe \secref{use:fmt:fmt}). Wenn diese Option aktiviert ist, werden nachfolgende
Erwähnungen derselben Namensliste am Anfang eines Eintrags durch einen Bindestrich ersetzt, sofern der Eintarg nicht der erste auf der aktuellen Seite ist.
Wenn die Option deaktiviert ist, werden Namenslisten niemals durch einen Bindestrich ersetzt. 
\end{optionlist}

\subparagraph{\texttt{authoryear}} Bibliografiestile der \texttt{authoryear}-Familie unterstützen die Option \opt{mergedate} global, per-Type und per-Eintragsbereich.

\begin{optionlist}

\optitem[true]{mergedate}{\opt{false}, \opt{minimum}, \opt{basic}, \opt{compact}, \opt{maximum}, \opt{true}}

Diese Option steuert, ob und wie die Datumsangabe im Eintrag mit der Datumsbezeichnung zusammengeführt wird, die direkt nach der Liste des Autors
\slash Editors angezeigt wird.

\begin{valuelist}
\item[false] Trennt die angegeben Datumsangabe (angegeben mit \opt{date}) strikt von der Datumsangabe, die festgelegt wurde mit \opt{labeldate}. Das Datum wird immer zweimal angezeigt.
\item[minimum] Lässt die Datumsangabe weg, wenn sie \emph{genau} -- einschließend  die \bibfield{extradate}-Information -- mit der Ausgabe der Datumsbezeichnung übereinstimmt. 
\item[basic] Mit der  \opt{minimum}-Option, aber die Datumsangabe wird 
auch weggelassen, wenn sie nur durch das Fehlen des Zeichens 
\bibfield{extradate} von der Datumsangabe abweicht.
\item[compact] Fügt alle Datumsspezifikationen mit dem Datumsetikett zusammen.
Das Datumsformat dieses zusammengeführten Datumsetiketts wird von \opt{date}
und nicht von \opt{labeldate} gesteuert, selbst wenn es an der Position des Datumsetiketts gedruckt wird. Das \bibfield{issue}-Feld wird nicht zusammmengeführt. \item[maximum] Wie \opt{compact}, aber, wenn vorhanden, wird das
\bibfield{issue}-Feld auch in die Datumsbezeichnung am Anfang des Eintrags verschoben. 
\item[true] Ein Alias für die  \opt{compact}-Option.
\end{valuelist}

Ausführlichere Beispiele für diese Option sind in den Stilbeispielen zu finden.\end{optionlist}

\subparagraph{<ibid> styles} Zitierstile mit der <ibid.>-Funktion, namentlich \texttt{authortitle-ibid}, \texttt{author\\ title-icomp}, \texttt{author\\ year-ibid}, \texttt{authoryear-icomp}, \texttt{ver\\ bose-ibid}, \texttt{verbose-inote}, \texttt{verbose-trad1}, \texttt{verbose-trad2} und \texttt{verbose-trad3} stellen die \opt{ibidpage}-Option zur Verfügung.

\begin{optionlist}

\boolitem[false]{ibidpage}

Ob \emph{ibidem} ohne Seitenverweis <gleiche Arbeit> oder <gleiche Arbeit + 
gleiche Seite> bedeutet. Bei der Einstellung auf \texttt{true} wird eine Seitenbereichspostnote in einem  \emph{ibidem}-Zitat unterdrückt, wenn sich das letzte Zitat auf denselben Seitenbereich bezieht. Mit \texttt{ibidpage=false}
wird die Postnote nicht weggelassen. Zitate zu anderen Seitenbereichen als den vorherigen erzeugten immer die Seitenbereiche mit beiden Einstellungen. 
\end{optionlist}

\subparagraph{\texttt{verbose}} Alle Zitierstile der \texttt{verbose}-Familie stellen die globale \opt{citepages}-Option zur Verfügung.

\begin{optionlist}

\optitem[permit]{citepages}{\opt{permit}, \opt{suppress}, \opt{omit}, \opt{separate}}

Diese Option steuert die Ausgabe des \bibfield{page}\slash\bibfield{pagetotal}-Felds bei einem vollständigen Zitat in Verbindung mit einer Postnote, die einen Seitenbereich enthält., Diese Option kann verwendet werden, um Verweise auf zwei Seitenbereiche in vollständigen Zitaten, wie den folgenden, zu unterdrücken. 
    
\begin{quote}
Author. \enquote{Title.} In: \emph{Book,} pp.\,100--150, p.\,125.
\end{quote}

Hier ist <p.\,125> das \bibfield{postnote}-Argument und <pp.\,100--150> 
ist der Wert des \bibfield{pages}-Felds.

\begin{valuelist}
\item[permit] Erlaubt das Duplizieren von Seitenspezifikationen, d.\,h. das
Drucken von \bibfield{page}\slash\bibfield{pagetotal} und \bibfield{postnote}.
\item[suppress] Unterdrückt bedingungslos die \bibfield{pages}\slash \bibfield{pagetotal}-Felder in Zitaten, unabhängig von \bibfield{postnote}.
\item[omit] Unterdrückt \bibfield{pages}\slash \bibfield{pagetotal}, wenn
das \bibfield{postnote}-Feld einen Seitenbereich enthält. Sie werden weiterhin gedruckt, wenn kein \bibfield{postnote}-Feld vorhanden ist oder wenn das  \bibfield{postnote}-Feld keine Zahl oder kein Bereich ist. 
\item[separate] Trennt Felder \bibfield{pages}\slash \bibfield{pagetotal} von  \bibfield{postnote}, wenn Letzteres einen Seitenbereich enthält. Die Zeichenfolge \texttt{thiscite}  wird hinzugefügt, um die beiden Seitenbereiche zu trennen. 
\end{valuelist}

\end{optionlist}

\subparagraph{\texttt{verbose-trad}} Die Zitierstile der \texttt{verbose-trad}-Familie unterstützen die globale Option \opt{strict}.

\begin{optionlist}

\boolitem[false]{strict}

Mit dieser Option können sie die wissenschaftlichen Abkürzungen <ibid.> und 
<op.~cit.> beschränken, um Mehrdeutigkeiten zu vermeiden. Wenn die Option
auf \texttt{true} gesetzt ist, werden diese Zeichen nur verwendet, wenn das entsprechende Werk in derselben oder einer vorherigen Fußnote zitiert wurde.
\end{optionlist}

\subparagraph{\texttt{reading}} Der  \texttt{reading}-Stil unterstützt eine Reihe zusätzlicher Optionen, aber diese sind nicht von allgemeinem Interesse und sind in dem Stilbeistil zu finden.

\paragraph{Intern} \label{use:opt:pre:int} 

Die Standardeinstellungen der folgenden Prämbeloptionen werden von der
Bibliografie und den Literaturverweisstilen kontrolliert. Abgesehen von den
\opt{pagetracker}- und \opt{firstinits}-Optionen, je nach dem welche Sie
bevorzugen, gibt es keinen Grund diese explizit zu setzen.

\begin{optionlist}

\optitem[false]{pagetracker}{\opt{true}, \opt{false}, \opt{page}, \opt{spread}}

Diese Option kontrolliert die Seitensuche, welche von \cmd{ifsamepage}- und
\cmd{iffirstonpage}-Tests aus § 4.6.2 (engl. Version) %\secref{aut:aux:tst} 
benutzt wird. Dabei stehen folgende Möglichkeiten zur Verfügung.

\begin{valuelist} 
\item[true] Aktiviert den automatischen Suchmodus. Das ist
dasselbe, wie \opt{spread}, wenn \latex im zweiseitigen Modus ist und bei
anderen Fällen wie \opt{page}.
\item[false] Deaktiviert die Suche.  
\item[page] Aktiviert die Suche im Seiten-Modus. In diesem Modus funktioniert die Suche auf
einer "`pro-Seite"'-Basis.	
\item[spread] Aktiviert die Suche im Spread-Modus. In diesem Modus funktioniert die Suche
auf einer "`Pro-Spread"'-Basis.	
\end{valuelist} 

Beachten Sie, dass diese Suche in allen Umläufen deaktiviert ist, schauen Sie in
§ 4.11.5 (engl. Version). %\secref{aut:cav:flt}.

\optitem[false]{citecounter}{\opt{true}, \opt{false}, \opt{context}}

Diese Option kontrolliert den Zitierungsschalter, der nötig ist für
\cnt{citecounter} aus § 4.6.2 (engl. Version). % \secref{aut:aux:tst}. 
Dabei stehen folgende Möglichkeiten zur Verfügung.

\begin{valuelist} 
\item[true] Aktiviert den Zitierungsschalter auf globaler
Ebene.  
\item[false] Deaktiviert den Zitierungsschalter. 
\item[context]
Aktiviert den Zitierungsschalter im Kontext-sensitivem-Modus.  In diesem Modus
werden Literaturverweise in Fußnoten und im Text unabhängig von einander
gesucht. 
\end{valuelist}

\optitem[false]{citetracker}{\opt{true}, \opt{false}, \opt{context},
\opt{strict}, \opt{constrict}}

Diese Option kontrolliert die Literaturverweissuche, welche erforderlich ist
für den \cmd{ifciteseen} und den \cmd{ifentryseen} Tests aus
§ 4.6.2 (engl. Version). %\secref{aut:aux:tst}. 
Dabei stehen folgende Möglichkeiten zur Verfügung:

\begin{valuelist} 
\item[true] Aktiviert den Sucher auf globaler Ebene.
\item[false] Deaktivert den Sucher.  
\item[context] Aktiviert den Sucher im
Kontext-sensitivem-Modus. In diesem Modus werden Literaturverweise in Fußnoten
und im Text unabhängig von einander gesucht.  
\item[strict] Aktiviert den Sucher
im Genaugkeitsmodus. In diesem Modus wird jeder Eintrag nur vom Sucher beachtet,
wenn in einem allein stehendem Literaturverweis auftaucht, d.\,h. wenn ein
einzelner Eintragsschlüssel im Literaturverweismodus angehangen wurde.
\item[constrict] Dieser Modus kombiniert die Eigenschaften von \opt{context} und
\opt{strict}.  
\end{valuelist} 
Beachten Sie, dass dieser Sucher in allen
Umläufen deaktiviert ist, schauen sie auch in § 4.11.5 (engl. Version). %\secref{aut:cav:flt}.

\optitem[false]{ibidtracker}{\opt{true}, \opt{false}, \opt{context},
\opt{strict}, \opt{constrict}} 

Diese Option kontrolliert den <ibidem>-Sucher, welcher vom \cmd{ifciteibid}-Test
beansprucht wird. Dabei stehen folgende Möglichkeiten zur Verfügung:

\begin{valuelist} 
\item[true] Aktiviert Sucher im globalen Modus.  
\item[false] Deaktiviert Sucher.  
\item[context] Aktiviert den Sucher im
Kontext-Sensitivem-Modus. In diesem Modus werden Literaturverweise in Fußnoten
und im Text unabhängig voneinander gesucht. 
\item[strict] Aktiviert den Sucher
im genauen Modus. In diesem Modus werden potentiell mehrdeutige Verweise
unterdrückt. Ein Verweis ist mehrdeutig, wenn entweder die aktuelle Zitierung
(eine der <ibidem>-Referie\-rungen) aus einer Liste von Referenzen
besteht.\footnote{Angenommen, das erste Zitat ist «Jones, \emph{Title};
Williams, \emph{Title}» und das folgende ein «ibidem».  Aus technischer Sicht
ist es ziemlich klar, dass <ididem> sich auf <Williams> bezieht, denn dies ist
die letzte Referenz, die vom vorherigen Zitatbefehl verarbeitet wurde. Für einen
menschlichen Leser ist dies jedoch nicht offensichtlich, weil sich <ibidem> auf
beide Titel beziehen kann. Den strikten Modus sollte man bei mehrdeutigen
Verweisen vermeiden.} 
\item[constrict] Dieser Modus kombiniert Eigenschaften von
\opt{context} und \opt{strict}. Darüber hinaus versucht der Weg der
Fußnote,
Zahlen und potentielle mehrdeutige Verweise in Fußnoten zu erkennen, in einer
strengeren Weise als die \opt{strict}-Option. Zusätzlich zu den Bedingungen, die
die \opt{strict}-Option auferlegt, wird ein Verweis in einer Fußnote nur als
eindeutig in Betracht gezogen, wenn das aktuelle Zitat und die vorherige Nennung in
der gleichen Fußnote oder in unmittelbar aufeinander folgenden Fußnoten
auftreten.  
\end{valuelist}
%

Beachten Sie, dass diese Suche in allen Umläufen (floats) deaktiviert ist,
schauen Sie in § 4.11.5 (engl. Version). %\secref{aut:cav:flt}.

\optitem[false]{opcittracker}{\opt{true}, \opt{false}, \opt{context},
\opt{strict}, \opt{constrict}}

Diese Option kontrolliert den <opcit>-Sucher, was durch den \cmd{ifopcit}-Test
aus § 4.6.2 (engl. Version). %\secref{aut:aux:tst} 
erforderlich ist. Diese Funktion ist ähnlich dem
<ibidem>-Sucher, außer dass die Zitate-Suche auf der author/editor-Basis
erfolgt, d.\,h. \cmd{ifopcit} erbringt \texttt{true}, wenn sie in der gleichen
Weise erfolgt wie beim letzten Mal bei diesem "`author\slash editor"'. Die
Möglichkeiten sind:

\begin{valuelist} 
\item[true] Aktiviert den Sucher im globalen Modus.
\item[false] Deaktiviert den Sucher.  
\item[context] Aktiviert den Sucher im
Kontext-sensitivem-Modus. In diesem Modus werden Literaturverweise in den
Fußnoten und im Text separat gesucht.  
\item[strict] Aktiviert den Sucher im
globalen Modus. In diesem Modus werden potentiell ambige Referenzen
aufgehoben. Siehe zu Einzelheiten \\
\kvopt{ibidtracker}{strict}.  
\item[constrict]
Dieser Modus kombiniert die Eigenschaften von \opt{context} und \opt{strict}.
Siehe zu Einzelheiten die Ausführungen zu \kvopt{ibidtracker}{constrict}.
\end{valuelist}

Vermerke, dass dieser Tracker in allen Floats inaktiviert ist, sehen Sie
§ 4.11.5 (engl. Version). %\secref{aut:cav:flt}.

\optitem[false]{loccittracker}{\opt{true}, \opt{false}, \opt{context}, \opt{strict}, \opt{constrict}}

Diese Option kontrolliert den <loccit>-Sucher, welcher vom \cmd{ifloccit}-Test
aus § 4.6.2 (engl. version). %\secref{aut:aux:tst} 
benötigt wird. Diese Funktion ist ähnlich der
<opcit>-Suche, außerdem wird auch geprüft, ob die \prm{postnote}-Argumente
übereinstimmen, d.\,h. \cmd{ifloccit} erbringt \texttt{true}, wenn das
Zitat sich auf die gleiche Seite bezieht, wie das Zitat davor. Dabei stehen 
folgende Möglichkeiten zur Verfügung:

\begin{valuelist} 
\item[true] Aktiviert den Sucher im globalen Modus.
\item[false] Deaktiviert den Sucher.  
\item[context] Aktiviert den Sucher im
Kontext-sensitivem-Modus. In diesem Modus werden Literaturverweise in den
Fußnoten und im Text separat gesucht. 
\item[strict] Aktiviert den Sucher im
globalen Modus. In diesem Modus werden werden potentiell ambige Referenzen
aufgehoben. Siehe zu Einzelheiten \kvopt{ibidtracker}{strict}. In Ergänzung
dazu, dieser Modus überprüft auch das \prm{postnote}-Argument und ist
numerisch (basierend auf \cmd{ifnumerals} aus %\secref{aut:aux:tst}). 
    § 4.6.2 (engl. Vers.).
\item[constrict]
Dieser Modus kombiniert die Eigenschaften von \opt{context} und \opt{strict}.
Siehe zu Einzelheiten die Ausführungen zu \kvopt{ibidtracker}{constrict}. In Ergänzung
dazu, dieser Modus überprüft auch das \prm{postnote}-Argument und ist
numerisch (basierend auf \cmd{ifnumerals} aus § 4.6.2 (engl. Version). %\secref{aut:aux:tst}). 

Vermerke, dass dieser Tracker in allen Floats inaktiviert ist, sehen Sie
§ 4.11.5 (engl. Version). %\secref{aut:cav:flt}.
\end{valuelist}


\optitem[false]{idemtracker}{\opt{true}, \opt{false}, \opt{context},
\opt{strict}, \opt{constrict}}

Diese Option kontrolliert den <idem>-Tracker, der erforderlich ist für den
\cmd{ifciteidem} Test aus § 4.6.2 (engl. Version). %\secref{aut:aux:tst}.
Folgende Möglichkeiten bestehen:

\begin{valuelist} 
\item[true] Aktiviert den Sucher im globalen Modus.
\item[false] Deaktiviert den Sucher.  
\item[context] Aktiviert den Sucher im
Kontext-sensitivem-Modus. In diesem Modus werden Literaturverweise in den
Fußnoten und im Text separat gesucht.  
\item[strict] Dies ist ein Alias for \texttt{true}, entwickelt nur, um konsistent
zu sein mit den anderen Trackern. Seit dem Austausch von <idem> bekommt man
Mehrdeutigkeit nicht in dergleichen Weise wie bei <ibidem> or <op.~cit.>, der
\texttt{strikte} tracking-Modus wird da nicht angewendet.
\item[constrict] Dieser Modus ist ähnlich mit dem von \opt{context}, mit einer
zusätzlichen Bedingung: Eine Referenz in eine Fußnote wird nur
als eindeutig anerkannt, wenn das aktuelle Zitat und die vorherige Zitierung in der
selben Fußnote oder in unmittelbar aufeinander folgenden Fußnoten benannt
wird.
 \end{valuelist}

Vermerken Sie, dass dieser Tracker in allen Floats inaktiviert ist, sehen Sie
§ 4.11.5 (engl. Version). %\secref{aut:cav:flt}.

\boolitem[true]{parentracker}

Diese Option kontrolliert die Einschübe der Tracker, welche die Suche von ineinander 
verschachtelten Klammern und Einschüben vorantreibt. Dies wird von \cmd{parentext} und 
\cmd{brackettext} genommen
aus \secref{use:cit:txt} sowie \cmd{mkbibparens} und \cmd{mkbibbrackets} aus
§ 4.10.4 (e. V.)%\secref{aut:fmt:ich} 
und \cmd{bibopenparen}, \cmd{bibcloseparen},
\cmd{bibopenbracket}, \cmd{bibclosebracket} (auch in § 4.10.4 (e. V.)).
%\secref{aut:fmt:ich}).

\intitem[3]{maxparens}

Gibt das Maximum erlaubter geschachtelter Ebenen von
Einschüben und Klammern an. Sollten Einschübe und Klammern tiefer als dieser
Wert verschachtelt sein, wird \biblatex Fehler verursachen.

\boolitem[false]{$<$namepart$>$inits}

Wenn diese Funktion aktiviert ist, werden alle $<$namepart$>$-Namenteile als Initiale erscheinen. Die Option beeinflußt die \cmd{if$<$namepart$>$inits}-Tests aus %\secref{aut:aux:tst}. 
§ 4.6.2. (engl. Vers.)
Die gültigen Namenteile sind im Datenmodel mit dem  \cmd{DeclareDatamodelConstant}-Befehl definiert %(\secref{aut:bbx:drv}).
§ 4.2.3 (engl. Vers.).
Für den angegebenen Namen lautet die Option beispielsweise \opt{giveninits}.
Diese Option kann auch pro Typ, pro Eintrag, pro Namensliste und pro Name festgelegt werden.

Wenn \opt{giveninits} auf \opt{true} gesetzt ist, werden in den Standardnamenformaten nur die Initalen des Vornamens und nicht der vollständige Vorname gerendert. Die Standardstile verwenden nur den Test  \cmd{ifgiveninits} und reagieren daher nur auf die Option \opt{giveninits}. Das Festlegen der Option
für einen anderen Namensteil als \texttt{given/angegeben} hat keine Auswirkungen auf die Standardformate.

Beachten Sie, dass die Sortierung und die Namenseindeutigkeit von dieser Option nicht automatisch 
betroffen sind. Sie müssen explizit über \cmd {DeclareSortingNamekeyTemplate} und die Option 
\opt{Uniquename} (bzw. \cmd {DeclareUniquenameTemplate}) angefordert werden. Eine Warnung wird ausgegeben, wenn \opt {Giveninits} zusammen mit \opt{Uniquename} verwendet wird, der Befehl auf einen der 
\opt{full}-Werte gesetzt ist und \opt{Uniquename} automatisch auf den entsprechenden \opt{init}-Wert 
gesetzt wird.

\boolitem[false]{terseinits} 

Diese Option kontrolliert das Format von
Initialen, die von \biblatex generiert werden. Standardmäßig fügt
\biblatex einen Punkt nach Initialen hinzu. Ist diese Option aktiviert,
wird ein gedrängtes Format ohne Punkte und Leerraum benutzt. Zum Beispiel werden
die Initiale von Donald Ervin Knuth standardmäßig als <D.~E.> ausgegeben und
als <DE> wenn aktiviert.Die Option arbeitet 
mit der Redefinierung einiger Makros, die das Format der Initiale kontrollieren.
Genaueres in \secref{use:cav:nam}.

\boolitem[false]{labelalpha}

Ob die speziellen Felder \bibfield{labelalpha} und \bibfield{extraalpha} zur Verfügung gestellt werden, für Details beachten Sie § 4.2.4 (e. V.). %\secref{aut:bbx:fld}.  
Diese Option kann diese Option auch auf
Per-Typ-Basis gesetzt werden. Siehe auch \opt{maxalphanames} und \opt{minalphanames}.
Tabelle \ref{use:opt:tab1} summiert die verschiedenen \opt{extra*}-Begriffsklärungszähler
und was sie tun.

\intitem[3]{maxalphanames}

Ähnlich zur \opt{maxnames}-Option, aber notwendig für das Format des
\bibfield{labelalpha}-Feldes.

\intitem[1]{minalphanames}

Ähnlich zur \opt{minnames}-Option, aber notwendig für das Format des
\bibfield{labelalpha}-Feldes.

\boolitem[false]{labelnumber}

Ob das spezielle Feld \bibfield{labelnumber} zur
Verfügung gestellt wird, für Details beachten Sie § 4.2.4 (e. V.).
%\secref{aut:bbx:fld}.
Diese Option ebenfalls auf Per-Typ-Basis setzbar.

\boolitem[false]{labeltitle}

Unabhängig davon ob das spezielle Feld \bibfield{extratitle} zur Verfügung steht, 
sehen Sie in § 4.2.4 (e. V.) %\secref{aut:bbx:fld} 
für weitere Details nach. Beachten Sie, dass das Spezialfeld
\bibfield{labeltitle} immer zur Verfügung steht und diese Option nicht steuert, ob \bibfield{labeltitle} verwendet wird, um \bibfield{extratitle}-Informationen generiert werden. Diese Option ist auch auf einer pro-Typ-Basis einstellbar. Tabelle 
\ref{use:opt:tab1} fasst die verschiedenen \opt{extra*}-Begriffserklärungszähler
zusammen und was sie beabsichtigen.

\boolitem[false]{labeltitleyear}

Unabhängig davon ob das spezielle Feld \bibfield{extratitle} zur Verfügung steht, sehen Sie
in § 4.2.4 (e. V.). %\secref{aut:bbx:fld} 
für weitere Details nach. Beachten Sie, dass das Spezialfeld \bibfield{labeltitle}  immer zur Verfügung steht und diese Option nicht steuert, ob \bibfield{labeltitle} verwendet wird, um \bibfield{extratitleyear}-Informationen generiert werden. Diese Option ist auch auf einer pro-Typ-Basis einstellbar. Tabelle  
\ref{use:opt:tab1} fasst die verschiedenen \opt{extra*}-Begriffserklärungszähler
zusammen und was sie beabsichtigen.

\boolitem[false]{labeltitleyear}

Unabhängig davon, ob das spezielle Feld \bibfield{extratitleyear} zur Verfügung steht, 
sehen Sie %\secref{aut:bbx:fld}
§ 4.2 (engl. V.) für Details. Beachten Sie, dass das Spezialfeld
\bibfield{labeltitle} immer zur Verfügung gestellt ist und diese Option steuert
diese nicht. \bibfield{labeltitle} wird verwendet, um eine
\bibfield{extratitleyear}-Information zu erzeugen. Diese Option ist auch auf einer
pro-Typ-Basis einstellbar. Tabelle \ref{use:opt:tab1} fasst die verschiedenen
\opt{extra*}-Eindeutigkeitszähler zusammen und was sie beabsichtigen.

\boolitem[false]{labeldateparts}

Unabhängig davon, ob diese Speziealfelder zur Verfügung stehen
\bibfield{labelyear}, \bibfield{labelmonth}, \bibfield{labelday}, \bibfield{labelendyear}, \bibfield{labelendmonth}, \bibfield{labelendday}, \bibfield{labelhour}, \bibfield{labelendhour}, \bibfield{labelminute}, \bibfield{labelendminute}, \bibfield{labelsecond}, \bibfield{labelendsecond}, \bibfield{labelseason}, \bibfield{labelendseason}, \bibfield{labeltimezone}, \bibfield{labelendtimeone} and \bibfield{extrayear}, sehen Sie § 4.2.4 (e. V.)
%\secref{aut:bbx:fld} 
für Details. 
Diese Option ist auch einstellbar auf eine pro-Typ-Basis. Tabelle \ref{use:opt:tab1} fasst die verschiedenen
\opt{extra*}-Eindeutigkeitszähler zusammen und was sie beabsichtigen.

\begin{table}
\footnotesize
\ttfamily
\tablesetup
\begin{tabularx}{\textwidth}{XXX}
\toprule
\multicolumn{1}{@{}H}{Option} &
\multicolumn{1}{@{}H}{Test} &
\multicolumn{1}{@{}H}{Tracks} \\
\cmidrule(r){1-1}\cmidrule(r){2-2}\cmidrule(r){3-3}
singletitle & \cmd{ifsingletitle} & Labelname\\
uniquetitle & \cmd{ifuniquetitle} & Labeltitel\\
uniquebaretitle & \cmd{ifuniquebaretitle} & Labeltitel, wenn der Labelname Null ist\\
uniquework  & \cmd{ifuniquework}  & Labelname+Labeltitel\\
\bottomrule
\end{tabularx}
\caption{Arbeit der Vereindeutigungsoptionen}
\label{use:opt:wu}
\end{table}

\boolitem[false]{singletitle}

Ob oder nicht die erforderlichen Daten bereitstehen für den \cmd{ifsingletitle}-Test, 
sehen Sie in § 4.6.2 (e. V.)
%\secref{aut:aux:tst}  
nach weiteren Details. Diese Option ist auch einstellbar auf einer pro-Typ-Basis.

\boolitem[false]{uniquetitle}

Ob oder nicht die erforderlichen Daten bereitstehen für den \cmd{ifuniquetitle}-Test, 
sehen Sie § 4.6.2 (e. V.) %\secref{aut:aux:tst} 
für Details. Sehen Sie \tabrefe{use:opt:wu} für 
Details, auf das, was die Daten für diesen Test bestimmt. 
Diese Option ist auch einstellbar auf einer pro-Typ-Basis.

\boolitem[false]{uniquebaretitle}

Ob oder nicht die erforderlichen Daten bereitstehen für den
\cmd{ifuniquebaretitle}-Test, sehen Sie § 4.6.2 (e. V.) %\secref{aut:aux:tst} 
für Details. 
Sehen Sie \tabrefe{use:opt:wu} für Details,  auf das, was die Daten für diesen 
Test bestimmt.   
Diese Option ist auch einstellbar auf einer pro-Typ-Basis.

\boolitem[false]{uniquework}

Ob oder nicht die erforderlichen Daten bereitstehen für den \cmd{ifuniquework}-Test, 
sehen Sie § 4.6.2 (e. V.) %\secref{aut:aux:tst} 
für Details. Sehen Sie \tabrefe{use:opt:wu} für
Details auf das, was die Daten für diesen 
Test bestimmt. Diese Option ist auch einstellbar auf einer pro-Typ-Basis.  

\boolitem[false]{uniqueprimaryauthor}

Ob oder nicht die erforderlichen Daten bereitstehen für den
\cmd{ifuniqueprimaryauthor}-Test, sehen Sie § 4.6.2 (e. V.) %\secref{aut:aux:tst} 
für Details.

\optitem[false]{uniquename}{\opt{true}, \opt{false}, \opt{init}, \opt{full},
\opt{allinit}, \opt{allfull}, \opt{mininit}, \opt{minfull}}

Ob ein \cnt{uniquname}-Zähler gesetzt werden soll, für Details schauen Sie in
§ 4.6.2 (e.~V.) %\secref{aut:aux:tst} 
nach. Dieses Feature kann individuelle Namen in der
\bibfield{labelname}-Liste disambiguieren. Diese Option ist auch auch auf pro-Typ-Basis 
setzbar.
Möglich sind: 

\begin{valuelist} 
\item[true] Ein Alias für \opt{full}.  
\item[false] Abschalten dieses Features.  
\item[init] Disambiguieren von Namen mit Initialen.  
\item[full] Disambiguieren von Namen mit Initialen oder vollem Namen, nach
Erfordernis.  
\item[allinit] Ähnlich wie \opt{init}, disambiguiert jedoch alle Namen in
der \bibfield{labelname}-Liste mit \opt{maxnames}\slash \opt{minnames}\slash 
\opt{uniquelist}.
\item[allfull] Ähnlich wie \opt{full}, disambiguiert jedoch alle Namen in
der \bibfield{labelname}-Liste mit \opt{maxnames}\slash \opt{minnames}\slash
\opt{uniquelist}.  
\item[mininit] Eine Variante von \texttt{init}, die aber disambiguiert
Namen in den Listen mit identischen Familiennamen. 
\item[minfull] Eine Variante von \texttt{full}, die aber disambiguiert
Namen in den Listen mit identischen Familiennamen.  
\end{valuelist}
%
Beachten Sie, dass sich die \opt{uniquename}-Option auf
\opt{uniquelist}, den \cmd{ifsingletitle}-Test und das
\bibfield{extrayear}-Feld auswirkt. Siehe § 4.11.4 (e.~V.) %\secref{aut:cav:amb} 
für weitere Details und praktische Beispiele.


\optitem[false]{uniquelist}{\opt{true}, \opt{false}, \opt{minyear}}

Ob die \cnt{uniquelist}-Zähler aktualisiert werden oder nicht, sehen Sie 
§ 4.6.2 (e.~V.) %\secref{aut:aux:tst}
zu Details. Dieses Feature disambiguiert die \bibfield{labelname}-Liste,
wenn es eine Ambiguität bekommen hat nach einem \opt{maxnames}\slash
\opt{minnames}-Abbruch. Im Wesentlichen überschreibt es
\opt{maxnames}\slash \opt{minnames} auf der pro-Feld-Basis. Diese Option
ist auch auch auf pro-Typ-Basis setzbar. Möglich sind: 

\begin{valuelist} 
\item[true] Disambiguieren der \bibfield{labelname}-Liste.
\item[false] Abschalten dieses Features.  
\item[minyear] Disambiguieren der \bibfield{labelname}-Liste, nur wenn die
verkürzte Liste identisch ist mit einer mit dem gleichen
\bibfield{labelyear}. Dieser Modus eignet sich gut für author-year-Stile
und erfordert \kvopt{labelyear}{true}.  
\end{valuelist}
%
Beachten Sie, dass die \opt{uniquelist}-Option sich auch auswirkt auf den 
\cmd{ifsingletitle}-Test und das \bibfield{extrayear}-Feld. Sehen Sie auch
§ 4.11.4 (e.~V.) %\secref{aut:cav:amb} 
für weitere Details und praktische Beispiele. 

\boolitem[false]{nohashothers}

Standardmäßig führen Namenslisten, die mit <et al> abgeschnitten werden --
 entweder explizit von <and others> in der Datenquelle oder den Optionen \opt{uniquelist} und \opt{min/maxnames} -- zu unterschiedlichen Namenslisten-Hashs  (und daher unterschiedlichen \opt{extraname} and \opt{extradate} Werten) und unterschiedlichen Sortierungen. Mit dieser Option kann dieses Verhalten
 optimiert werden. Bei der Einstellung \prm{true} ignoriert \biber 
 <et al>-Kürzungen, um Namenslisten-Hashs zu generieren. Erwägen sie:

\begin{lstlisting}{}
Jones 1972
Jones/and others 1972
Smith 2000
Smith/Vogel/Beast/Tremble 2000
\end{lstlisting}
%
Mit \kvopt{maxnames}{3}, \kvopt{minnames}{1}, \kvopt{nohashothers}{false}, entsteht das Resultat:

\begin{lstlisting}{}
  Jones 1972
  Jones et al 1972
  Smith 2000
  Smith et al 2000
\end{lstlisting}
%
Wohingegen mit \kvopt{maxnames}{3}, \kvopt{minnames}{1}, \kvopt{nohashothers}{true}, 
das Resultat entsteht:

\begin{lstlisting}{}
  Jones 1972a
  Jones et al 1972b
  Smith 2000a
  Smith et al 2000b
\end{lstlisting}

Falls gewünscht, könnte dies weiter vereinfacht werden mit:

\begin{ltxexample}
  \DefineBibliographyStrings{english}{andothers={}}
\end{ltxexample}
%
Um zu bekommen:

\begin{lstlisting}{}
  Jones 1972a
  Jones 1972b
  Smith 2000a
  Smith 2000b
\end{lstlisting}
%
Beachten sie, dass die \opt{nohashothers}-Option die  \bibfield{extradate}-  und
\bibfield{extraname}-Felder beeinflussen wird.

Diese Option ist einstellbar per Typ, per Eintrag und per Namenslistenbasis.

\boolitem[false]{nosortothers}

Diese Option hat einen Bezug zu der Option \opt{nohashothers}, sie gilt jedoch für die Sortierung, -- die sichtbare Namensliste (die der Wert von \opt{minsortnames} ist) wird
mit der Sortierung bestimmt, sie ignoriert jegliches Abschneiden. Dies bedeutet, dass mit
\kvopt{nosortothers}{true}, der Name aufgelistet wird:

\begin{lstlisting}{}
Jones, Smith
Jones, Smith et al
\end{lstlisting}
%
Es wird genau das Gleiche sortieren. Die Standardeinstellung von \opt{nosortothers}
wird immer in der im Beispiel gezeigten Reihenfolge sortieren, d.\,h. abgeschnittene
Namenslisten werden immer standardmäßig nach Namenslisten sortiert, die mit dem 
Kürzungspunkt identisch sind.
Diese Option ist einstellbar per Typ, per Eintrag und per Namenslistenbasis.

\begin{ltxexample}
  \DefineBibliographyStrings{english}{andothers={}}
\end{ltxexample}
%
Man erhält:

\begin{lstlisting}{}
  Jones 1972a
  Jones 1972b
  Smith 2000a
  Smith 2000b
\end{lstlisting}
%
Beachten sie, dass die \opt{nohashothers}-Option will die \bibfield{extradate}- und \bibfield{extraname}-Felder beeinflussen wird.

Diese Option ist einstellbar per Typ, per Eintrag und per Namenslistenbasis.

\boolitem[false]{nosortothers}

Diese Option hat einen Bezug zu \opt{nohashothers}, sie gilt aber für die
Sortierung, -- die sichtbare Namensliste (die der Wert ist für \opt{minsortnames}),
die mit der Sortierung bestimmt wird, ignoriert jegliches Abschneiden.
Die bedeutet, dass mit \kvopt{nosortothers}{true}, der Name Folgendes auflistet:

\begin{lstlisting}{}
Jones, Smith
Jones, Smith et al
\end{lstlisting}
%
Es wird genau das Gleiche sortiert. Die Standardeinstellung von \opt{nosortothers}
wird immer in der im Beispiel gezeigten Reihenfolge sortieren, d.\,h. 
abgeschnittene Namenslisten werden standardmäßig immer als Namenslisten soertiert,
die mit dem Kürzungspunkt identisch sind.
Diese Option kann auch 
 per Type, per Eintrag und per Namensliste eingestellt werden.


\end{optionlist}

\begin{table}
\footnotesize
\ttfamily
\tablesetup
\begin{tabularx}{\textwidth}{XXXX}
\toprule
\multicolumn{1}{@{}H}{Option} &
\multicolumn{1}{@{}H}{Aktivierte Feld(er)} &
\multicolumn{1}{@{}H}{Aktivierter Counter} &
\multicolumn{1}{@{}H}{Counter-Tracker} \\
\cmidrule(r){1-1}\cmidrule(r){2-2}\cmidrule(r){3-3}\cmidrule{4-4}
labelalpha     & labelalpha       & extraalpha     &  label\\
labeldateparts & labelyear        & extrayear      &  labelname+\\
               & labelmonth       &                &  labelyear\\
               & labelday         &                &  \\
               & labelendyear     &                &  \\
               & labelendmonth    &                &  \\
               & labelendday      &                &  \\
               & labelhour        &                &  \\
               & labelminute      &                &  \\
               & labelsecond      &                &  \\
               & labelendhour     &                &  \\
               & labelendminute   &                &  \\
               & labelendsecond   &                &  \\
               & labelseason      &                &  \\
               & labelendseason   &                &  \\
               & labeltimezone    &                &  \\
               & labelendtimezone &                &  \\
labeltitle     & \rmfamily{---}   & extratitle     &  labelname+labeltitle\\
labeltitleyear & \rmfamily{---}   & extratitleyear &  labeltitle+labelyear\\
\bottomrule
\end{tabularx}
\caption{Disambiguationscounter}
\label{use:opt:tab1}
\end{table}

\subsubsection{Eintragsoptionen} \label{use:opt:bib}

Eintragsoptionen sind Paketoptionen, welche bestimmen, wie eine Bibliografie Dateneinträge behandeln wird. Sie können in verschiedenen Bereichen gesetzt werden, die im Folgenden definiert werden.

\paragraph{Prämbel/Typ/Eintragoptionen} \label{use:opt:bib:hyb} 

Die folgenden Optionen sind alle auf pro-Eintrag-Basis oder pro-Eintrag in das \bibfield{options}-Feld setzbar.
Zusätzlich dazu können sie auch als optionales Argument in \cmd{usepackage}
so wie in die Konfigurationsdatei und die Dokumentpräambel geschrieben werden.
Das ist nützlich, wenn Sie das standardmäßige Verhalten global ändern wollen.

\begin{optionlist}

\boolitem[true]{useauthor} 

Ob der \bibfield{author} in Etiketten benutzt und
beim  Sortieren beachtet wird. Dies kann nützlich sein, wenn ein Eintrag ein
\bibfield{author}-Feld aufweist, aber normalerweise nicht beim Autor
zitiert wird, aus welchen Gründen auch immer. Wenn \kvopt{useauthor}{false}
gesetzt wird, heißt das nicht, dass \bibfield{author} ignoriert wird. Es
bedeutet nur, dass \bibfield{author} bei den Etiketten nicht benutzt und beim
Sortieren ignoriert wird. Der Eintrag wird durch \bibfield{editor} oder
\bibfield{editor} alphabetisch sortiert. Benutzt man die Standardstile, wird
hier \bibfield{author} hinter dem Titel ausgegeben. Für weitere Informationen
schauen Sie in \secref{use:srt}.  Diese Option ist auch
auf pro-Typ-Basis setzbar.

\boolitem[true]{useeditor} 

Ob \bibfield{editor} einen fehlenden
\bibfield{author} in den Etiketten und während des Sortierens ersetzen soll.
Dies kann nützlich sein, wenn ein Eintrag ein \bibfield{editor}-Feld
beinhaltet, aber nicht mit dem Editor zitiert wird. Setzt man
\kvopt{useeditor}{false} heißt das nicht, dass \bibfield{editor} komplett
ignoriert wird. Es heißt nur, dass \bibfield{editor} \bibfield{author} in den
Etiketten und beim Sortieren nicht ersetzt. Der Eintrag wird dann nach
\bibfield{title} alphabetisiert. Benutzt man die Standardstile, so wird hier
\bibfield{editor} nach dem Titel ausgegeben. Für weitere Informationen schauen
Sie in \secref{use:srt}.
Diese Option ist diese Option auch auf pro-Typ Basis setzbar.

\boolitem[false]{usetranslator}

Ob \bibfield{translator} fehlende
\bibfield{author}\slash \bibfield{editor} bei den Etiketten oder während des
Sortierens ersetzen soll. Setzt man \kvopt{usetranslator}{true} heißt das nicht,
dass \bibfield{translator} \bibfield{author}\slash \bibfield{editor}
überschreibt. Es bedeutet nur, dass \bibfield{translator} als Rückgriff dient,
sollten \bibfield{author}\slash \bibfield{editor} fehlen oder \opt{useauthor}
und \opt{useeditor} auf \texttt{false} gesetzt sind. Mit anderen Worten, wenn
Sie ein Buch lieber mit dem Übersetzer zitieren wollen, als mit dem Autor, dann müssen
Sie folgende Optionen setzen: Diese Option ist auch auf
pro-Typ-Basis setzbar.

\begin{lstlisting}[style=bibtex]{}
@Book{..., 
options    = {useauthor=false,usetranslator=true},
author     = {...}, translator = {...},
translator = {...},
...
\end{lstlisting} 

Mit den Standardstilen wird \bibfield{translator} standardmäßig
nach dem Titel ausgegeben. Für mehr Informationen beachten Sie auch \secref{use:srt}.

\boolitem[true]{use$<$name$>$}

Wie mit \opt{useauthor}, \opt{useeditor} und \opt{usetranslator}, alle Namenslisten, mit dem Datenmodell definiert, haben eine Option, die ihr Verhalten bei der Sortierung und Kennzeichnung automatisch steuert. Global pro-Type und pro-Eintrag genannt,
<use$<$name$>$> wird automatisch erzeugt.

\boolitem[false]{useprefix} 

Ob im Standardnamenmodell ein Präfix als Teil eines Namens (von, van, of, da, de, della, usw.) berücksichtigt wird, ist dann, 

\begin{itemize}
\item wenn der  Nachname in den Literaturverweisen ausgegeben wird; 
\item bei Sortierung;
\item bei Erzeugung bestimmter Arten von Etiketten;
\item bei Erzeugung von Namen mit einzigartigen Informationen;
\item bei Formatierungsaspekten in der Bibliografie.
\end{itemize}

Beispielsweise, wenn diese Option aktiviert ist, dann geht mit \biblatex der Nachname mit dem Präfix voran -- Ludwig van Beethoven würde zitiert werden als «van Beethoven» und alphabetisiert werden als «Van Beethoven, Ludwig». Ist diese Option jedoch
deaktiviert (der Standard), wird er als «Beethoven» zitiert und als
«Van Beethoven, Ludwig» alphabetisiert.  Diese Option kann auch auf
pro-Typ-Scopus gesetzt werden.  Mit \biblatexml-Datenquellen und dem erweiterten
\bibtex-Namensformat (unterstützt von \biber), ist dies auch einstellbar auf pro-Namensliste und pro-Namen-Scopus.

\optitem{indexing}{\opt{true}, \opt{false}, \opt{cite}, \opt{bib}} 

Die \opt{indexing}-Option kann auch auf pro-Typ- oder pro-Eintrag gesetzt werden. 
Für Details beachten Sie \secref{use:opt:pre:gen}.

\end{optionlist}

\paragraph{Typ/Eintragoptionen} \label{use:opt:bib:ded}

Folgende Optionen sind auf einer pro-Type-Basis oder einem pro-Eintrag in dem
\bibfield{options}-Eintrag einstellbar. Sie sind nicht global verfügbar.

\begin{optionlist}

\boolitem[false]{skipbib}

Wenn diese Option aktiviert ist, wird der Eintrag aus
der Bibliografie ausgeschlossen, kann aber immer noch zitiert werden.
Diese Option ist auch auf pro-Typ-Basis setzbar.

\boolitem[false]{skipbiblist}

Wenn diese Option aktiviert wird, wird der Eintrag davon ausgeschlossen, auch aus der Bibliografieliste. Er sit immer noch in der Bibliografie enthalten und kann auch durch eine Abkürzung etc. zitiert werden. Diese Option ist einstellbar auf einer pro-Type-Basis.

\boolitem[false]{skiplab} 

Ist diese Option aktiviert, dann wird \biblatex
den Einträgen keine Etiketten zuweisen. Normalerweise wird diese Option nicht
benötigt. Benutzen Sie sie mit Vorsicht! Wenn sie aktiviert ist, kann
\biblatex nicht garantieren, dass Literaturverweise für die entsprechenden
Einträge eindeutig sind für den jeweiligen Eintrag und Zitatstil, die die Etiketten benötigen, 
könnten dabei scheitern, validierte Literaturverweise für die Einträge zu erstellen.
Diese Option ist auch auf pro-Typ-Basis setzbar.

\boolitem[false]{dataonly} 

Diese Option zu setzen entspricht den \kvopt{uniquename}{false}, \kvopt{uniquelist}{false, }\opt{skipbib}, \opt{skipbiblist}, and \opt{skiplab}-Optionen. Normalerweise wird diese Option nicht benötigt. Benutzen Sie sie mit Vorsicht.
Diese Option ist auch auf pro-Typ-Basis setzbar.

\end{optionlist}

\paragraph{Eintrag-only-Optionen}
\label{use:opt:bib:entry}

Die folgenden Optionen sind nur per"=entry in das \bibfield{options}-Feld 
eintragbar.  Sie sind nicht global oder per"=type verfügbar.

\begin{optionlist}

\valitem{labelnamefield}{fieldname}

Besonders ist das Feld zunächst zu prüfen, wenn Sie einen 
\bibfield{labelname}-Kandidaten suchen. Es wird im Wesentlichen 
mit \cmd{DeclareLabelname} nur für diesen Eintrag erstellt.

\valitem{labeltitlefield}{fieldname}

Besonders ist das Feld zunächst zu prüfen, wenn Sie einen 
\bibfield{labeltitle}-Kandidaten suchen. Es wird im Wesentlichen 
mit \cmd{DeclareLabeltitle} nur für diesen Eintrag erstellt.

\end{optionlist}

\subsubsection{Vererbungsoptionen} 

Die folgenden Vererbungsoptionen sind global nur als optionales Argument von 
\cmd{documentclass} nutzbar oder lokal als optionales Argument von \cmd{usepackage}:

\begin{optionlist}

\legitem{openbib}\DeprecatedMark  
Diese Option wird für rückwärtige Kompatibilität mit der
Standard-\LaTeX-document\-Klasse zur Verfügung gestellt. \opt{openbib} entspricht
dabei \kvopt{block}{par}.

\end{optionlist}

\subsection{Globale~Anwenderanpas\-sung} \label{use:cfg} 

Unabhängig von der Möglichkeit neue Literaturverweis- und
Bibliografiestile zu erstellen, gibt es zahlreiche Möglichkeiten, die dem Paket
beiliegenden Stile zu verändern. Dabei werden die manuellen Veränderungen
normalerweise in die Präambel geschrieben, aber es gibt auch eine
Konfigurationsdatei für dauerhafte Anpassungen. Diese Datei kann auch dazu
benutzt werden, die Paketoptionen auf einen Wert zu setzen, der sich vom
Standardwert unterscheidet.

\subsubsection{Konfigurationsdatei} \label{use:cfg:cfg} 

Ist diese Option (Configuration File)
aktiviert, wird dieses Paket die Konfigurationsdatei \path{biblatex.cfg}
laden. Die Datei wird am Ende des Pakets gelesen, gleich nachdem die
Literaturverweis- und die Bibliografiestile geladen wurden.

\subsubsection{Umgebungspaketoptionen} \label{use:cfg:opt} 

Die "`load-time"'-Paketoption in \secref{use:opt:ldt} muss im optionalen Argument zu
\cmd{usepackage} stehen. Die Paketoptionen aus \secref{use:opt:pre} können auch
in der Präambel stehen. Die Optionen werden mit dem folgendem Befehl ausgelöst:

\begin{ltxsyntax}

\cmditem{ExecuteBibliographyOptions}[entry type, \dots]{key=value, \dots} 

Dieser Befehl kann auch in der Konfigurationsdatei benutzt werden, um die
Standardeinstellungen der Paketoption zu verändern. Einige Optionen sind auch
auf pro-Typ-Basis setzbar. In diesem Fall spezifiziert das optionale \prm{entry
type}-Argument den Eintragstyp. Das \prm{entry type}-Argument kann auch eine
durch Kommata unterschiedene Liste von Werten sein.

\end{ltxsyntax}

\subsection{Standardstile} \label{use:xbx}

Dieses Kapitel bietet eine kurze
Beschreibung aller Bibliografie- und Literaturverweisstile, die dem
\biblatex-Paket beiliegen. Wenn Sie ihre eigenen Stile schreiben wollen, 
schauen Sie in %\secref{aut} 
§ 4 (englische Version) nach.

\subsubsection{Zitierstile} \label{use:xbx:cbx} 

Die Literaturverweisstile, die diesem Paket beiliegen, implementieren einige, 
verbreitete Zitierstile. Alle
Standardstile sorgen für das \bibfield{shorthand}-Feld und
unterstützen Hyperlinks und das Erstellen von Indexen.

\begin{marglist}

\item[numeric] Dieser Stil beinhaltet ein numerisches Zitierschema, ähnlich dem
der bibliografischen Standardeinrichtungen von \latex. Er sollte in Verbindung
mit den Bibliografiestilen, die für die Ausgabe der entsprechenden Etiketten in
der Bibliografie zuständig sind, eingefügt werden. Er ist für einen
Literaturverweis im Text bestimmt. Dieser Stil wird die folgenden Paketoptionen
beim Laden setzen: \kvopt{autocite}{inline}, \kvopt{labelnumber}{true}. Dieser
Stil stellt auch eine zusätzliche Präambeloption, namens \opt{subentry}, zur
Verfügung, welche die Abfertigung der Eintragssätze beeinflussen. Ist diese
Option deaktiviert, werden Literaturverweise zu einem Mitglied dieses Satzes auf
den ganzen Satz verweisen. Ist er aktiviert, dann unterstützt der Stil Verweise
wie «[5c]» die nur auf einen Untereintrag in einem Satz (das dritte Beispiel)
zeigen. Für Details beachten Sie die Stilbeispiele.


\item[numeric-comp] Eine kompakte Variante des \texttt{numeric}-Stils,
welcher
eine Liste von mehr als zwei aufeinanderfolgende Nummern in einer Reihe ausgibt.
Dieser Stil ist ähnlich dem \sty{cite}-Paket und der \opt{sort\&compress}-Option
vom \sty{natbib}-Paket im nummeralen Modus. Zum Beispiel wird anstatt «[8, 3, 1,
7, 2]» dank diesem Stil «[1--3, 7, 8]» ausgegeben. Er ist für Literaturverweise
im Text bestimmt. Dieser Stil wird die folgenden Paketoptionen beim Laden
setzen:
\kvopt{autocite}{inline}, \kvopt{sortcites}{true}, \kvopt{labelnumber}{true}.

Er stellt auch eine \opt{subentry}-Option zur Verfügung.

\item[numeric-verb] Eine wortreiche Variante der \texttt{numeric}-Stils. Der
Unterschied beeinflusst die Handhabung einer Liste von Zitierungen und ist nur
offensichtlich, wenn mehrere Eingabeschlüssel an einen einzelnen
Zitierungsbefehl abgeben werden. Zum Beispiel, anstelle von «[2, 5, 6]» würde
dieser Stil «[2]; [5]; [6]» ausgeben. Er ist für im-Text-Zitierungen vorgesehen.
Der Stil setzt die folgenden Paketoptionen zur Ladezeit auf:

\kvopt{autocite}{inline}, \kvopt{labelnumber}{true}.

Er unterstützt auch die \opt{subentry}-Option.     

\item[alphabetic] Dieser Stil führt ein alphabetisches Zitierschema aus, ähnlich 
dem
\path{alpha.bst}-Stil vom üblichen \bibtex. Die alphabetischen Labels ähneln
ansatzweise dem kompakten Autor"=Jahr-Stil, aber die Art, wie sie eingesetzt
werden, ist dem numerischen Zitierschema ähnlich. Zum Beispiel, anstelle von
«Jones 1995» würde dieser Stil das Label «[Jon95]» benutzen. «Jones and Williams
1986» würde erstellt werden als «[JW86]». Dieser Stil sollte im Zusammenhang mit
einen alphabetischen Bibliografiestil eingesetzt werden, der die entsprechenden
Labels in die Bibliografie ausgibt. Er ist für im-Text-Zitierung vorgesehen.
Der Stil setzt die folgenden Paketoptionen zur Ladezeit auf:

\kvopt{autocite}{inline}, \kvopt{labelalpha}{true}.        

\item[alphabetic-verb] Eine wortreiche Variante des \texttt{alphabetic}-Stils. 
Der Unterschied beeinflusst die Handhabung einer Liste von Zitierungen und ist nur
offensichtlich, wenn mehrere Eingabeschlüssel an einen einzelnen
Zitierungsbefehl abgegeben werden. Zum Beispiel, anstelle von «[Doe92; Doe95;
Jon98]» würde dieser Stil «[Doe92]; [Doe95]; [Jon98]» ausgeben. Er ist für
im-Text-Zitierungen vorgesehen. Der Stil setzt die folgenden Paketoptionen zur
Ladezeit auf:

\kvopt{autocite}{inline}, \kvopt{labelalpha}{true}.   

\item[authoryear] Der Stil führt ein Autor"=Jahr-Zitierschema aus. Wenn die 
Bibliografie zwei
oder mehr Arbeiten eines Autors enthält, die alle im selben Jahr veröffentlicht
wurden, wird ein Buchstabe am Jahr angehängt. Zum Beispiel würde dieser Stil
Zitierungen wie «Doe 1995a; Doe 1995b; Jones 1998» ausgeben. Dieser Stil sollte
in Zusammenhang mit einen Autor"=Jahr-Bibliografiestil eingesetzt werden, der
die entsprechenden Label in der Bibliografie ausgibt. Er ist vorrangig für
im-Text-Zitierungen vorgesehen, aber er kann auch für Zitierungen verwendet
werden, die in Fußnoten angegeben werden. Der Stil setzt die folgenden
Paketoptionen, zur Ladezeit auf: 

\kvopt{autocite}{inline}, \kvopt{labeldateparts}{true}, \kvopt{uniquename}{full}, \kvopt{uniquelist}{true}.

\item[authoryear-comp] Eine kompakte Variante des \texttt{authoryear} Stils, 
die den Autor nur einmal
ausgibt, wenn nachfolgende Referenzen, die an einen einzelnen Zitierbefehl
weitergegeben wurden, den selben Autor benutzen. Wenn sie ebenfalls dasselbe
Jahr benutzen, wird das Jahr auch nur einmal ausgegeben. Zum Beispiel, anstelle
von «Doe 1995b; Doe 1992; Jones 1998; Doe 1995a» würde dieser Stil «Doe 1992,
1995a,b; Jones 1998» ausgeben. Er ist vorrangig für im-Text-Zitierungen
vorgesehen, aber er kann auch für Zitierungen benutzt werden, die in Fußnoten
angegeben werden. Der Stil setzt die folgenden Paketoptionen zur Ladezeit auf:

 \kvopt{autocite}{inline}, \kvopt{sortcites}{true}, \kvopt{labeldateparts}{true}, \kvopt{uniquename}{full}, \kvopt{uniquelist}{true}.
      

\item[authoryear-ibid] Eine Variante des \texttt{authoryear}-Stils, die
wiederholte Zitierungen 
mit Abkürzungen \emph{ebenda} ersetzt, außer wenn die Zitierung die erste auf der
aktuellen Seite oder Doppelseite ist oder wenn \emph{ebenda} zweideutig im Sinne
der Paketoption \kvopt{ibidtracker}{constrict} sein würde. Der Stil setzt die
folgenden Paketoptionen zur Ladezeit auf: 

 \kvopt{ibidtracker}{constrict}. The style will set the following package options at load time: \kvopt{autocite}{inline}, \kvopt{labeldateparts}{true}, \kvopt{uniquename}{full}, \kvopt{uniquelist}{true}, \kvopt{ibidtracker}{constrict}, \kvopt{pagetracker}{true}.

Dieser Stil
unterstützt auch eine zusätzliche Präambeloption, mit dem Namen \opt{ibidpage}.
Für Details sehen Sie die Stilbeispiele an.

\item[authoryear-icomp] Ein Still der \texttt{authoryear-comp} 
und \texttt{authoryear-ibid} kombiniert.
Der Stil setzt die folgenden Paketoptionen zur Ladezeit auf:

kvopt{autocite}{inline}, \kvopt{labeldateparts}{true}, \kvopt{uniquename}{full}, \kvopt{uniquelist}{true}, \kvopt{ibidtracker}{constrict}, \kvopt{pagetracker}{true}, \kvopt{sortcites}{true}.


Dieser Stil unterstützt auch eine zusätzliche
Präambeloption mit dem Namen \opt{ibidpage}. Für Details siehe die
Stilbeispiele. 

\item[authortitle] Dieser Stil führt ein einfaches
Autor"=Titel-Zitierungsschema aus. Er verwendet
das \bibfield{shorttitle}-Feld, wenn es vorhanden ist. Er ist für Zitierungen
vorgesehen, die in Fußnoten angegeben werden. Der Stil setzt die folgenden
Paketoptionen, zur Ladezeit auf: 

\kvopt{autocite}{footnote}, \kvopt{uniquename}{full}, \kvopt{uniquelist}{true}.

\item[authortitle-comp] Eine kompakte Variante des \texttt{authortitle}-Stils,
der den Autor nur einmal
ausgibt, wenn nachfolgende Referenzen, die an einen einzelnen Zitierbefehl
weitergegeben wurden, den selben Autor benutzen. Zum Beispiel, anstelle von
«Doe, \emph{Erster Titel}; Doe, \emph{Zweiter Titel}» würde dieser Stil «Doe,
\emph{Erster Titel}, \emph{Zweiter Titel}» ausgeben. Er ist für Zitierungen
vorgesehen, die in Fußnoten angegeben werden. Der Stil setzt die folgenden
Paketoptionen zur Ladezeit auf:

\kvopt{autocite}{footnote}, \kvopt{sortcites}{true}, \kvopt{uniquename}{full}, \kvopt{uniquelist}{true}.

\item[authortitle-ibid] Eine Variante des \texttt{authortitle}-Stils, 
der wiederholte Zitierungen mit
Abkürzungen ersetzt, außer wenn die Zitierungen die erste auf der aktuellen
Seite oder Doppelseite ist, oder wenn \emph{ebenda} zweideutig im Sinne der
Paketoption-\kvopt{ibidtracker}{constrict} sein würde. Er ist für Zitierungen
vorgesehen, die in Fußnoten angegeben werden. Der Stil setzt die folgenden
Paketoptionen, zur Ladezeit auf: 

\kvopt{autocite}{footnote}, \kvopt{uniquename}{full}, \kvopt{uniquelist}{true}, \kvopt{ibidtracker}{constrict}, \kvopt{pagetracker}{true}

Dieser Stil
unterstützt eine zusätzliche Präambeloption mit dem Namen \opt{ibidpage}. Für
Details siehe die Stilbeispiele. 

\item[authortitle-icomp] Ein Stil der die Eigenschaften von 
\texttt{authortitle-comp} und
\texttt{authortitle-ibid} kombiniert. Der Stil setzt die folgenden
Paketoptionen zur Ladezeit auf: 

\kvopt{autocite}{footnote}, \kvopt{uniquename}{full}, \kvopt{uniquelist}{true}, \kvopt{ibidtracker}{constrict}, \kvopt{pagetracker}{true}, \kvopt{sortcites}{true}. 

Dieser Stil unterstützt eine zusätzliche
Präambeloption mit dem Namen \opt{ibidpage}. Für Details siehe die
Stilbeispiele.

\item[authortitle-terse] Eine prägnante Variante des 
\texttt{authortitle}-Stils, der nur den Titel
ausgibt, wenn die Bibliografie mehr als zwei Arbeiten des entsprechenden
Autors\slash Editors beinhaltet. Dieser Stil verwendet das
\bibfield{shorttitle}-Feld, wenn es verfügbar ist. Er ist sowohl für
im-Text-Zitierungen als auch für
Zitierungen, die in Fußnoten angegeben werden, geeignet. Der Stil setzt die
folgenden Paketoptionen zur Ladezeit auf: 

\kvopt{autocite}{inline}, \kvopt{singletitle}{true}, \kvopt{uniquename}{full}, \kvopt{uniquelist}{true}.

\item[authortitle-tcomp] Ein Stil, der die Eigenschaften von 
\texttt{authortitle-comp} und
\texttt{authortitle-terse} kombiniert. Dieser Stil verwendet das
\bibfield{shorttitle}-Feld, wenn es verfügbar ist. Er ist sowohl für
im-Text-Zitierungen als auch für Zitierungen, die in Fußnoten angegeben werden,
geeignet. Der Stil setzt die folgenden Paketoptionen zur Ladezeit auf:

\kvopt{autocite}{inline}, \kvopt{sortcites}{true}, \kvopt{singletitle}{true}, \kvopt{uniquename}{full}, \kvopt{uniquelist}{true}.

\item[authortitle-ticomp] Ein Stil der die Eigenschaften von 
\texttt{authortitle-icomp} und
\texttt{authortitle-terse} kombiniert. Anders formuliert: eine Variante des
\texttt{authortitle-tcomp} Stils mit einer \emph{ebenda} Eigenschaft. Dieser
Stil ist sowohl für im-Text-Zitierungen als auch für Zitierungen, die in
Fußnoten angegeben werden, geeignet. Er setzt die folgenden Paketoptionen zur
Ladezeit auf: 

\kvopt{autocite}{inline}, \kvopt{ibidtracker}{constrict}, \kvopt{pagetracker}{true}, \kvopt{sortcites}{true}, \kvopt{singletitle}{true}, \kvopt{uniquename}{full}, \kvopt{uniquelist}{true}. This style also provides an additional preamble option called \opt{ibidpage}.


Dieser Stil unterstützt eine zusätzliche
Präambeloption, mit dem Namen \opt{ibidpage}. Für Details sehen Sie das
Stilbeispiel.     

\item[verbose] Ein "`wortreicher"' Zitierungsstil, der eine komplette 
Zitierung ausgibt, ähnlich
mit einem Bibliografieeintrag, wenn ein Eintrag das erste Mal zitiert wird und
eine kurze Zitierung danach zitiert wird. Wenn verfügbar, wird das
\bibfield{shorttitle}-Feld in allen kurzen Zitierungen benutzt. Wenn das
\bibfield{shorthand}-Feld definiert ist, wird die "`Kurzschrift"' bei der ersten
Zitierung eingeführt und danach als die kurze Zitierung verwendet. Dieser Stil
kann ohne ein Liste von Referenzen und Abkürzungen benutzt werden, da alle
bibliografischen Daten bei der ersten Zitierung bereitgestellt werden. Es ist
für Zitierungen vorgesehen, die in Fußnoten angegeben werden. Der Stil setzt die
folgenden Paketoptionen zur Ladezeit auf: 


\kvopt{autocite}{footnote}, \kvopt{citetracker}{context}. 

Dieser Stil unterstützt eine zusätzliche
Präambeloption, mit dem Namen \opt{citepages}. Für Details sehen Sie das
Stilbeispiel.  

\item[verbose-ibid] Eine Variante des \texttt{verbose}-Stils, der die 
wiederholten Zitierungen mit
der Abkürzung \emph{ebenda} austauscht, außer wenn die Zitierung die erste auf
der aktuellen Seite oder Doppelseite ist, oder wenn \emph{ebenda} zweideutig im
Sinne der Paketoption \kvopt{ibidtracker}{strict} sein würde. Dieser Stil ist
für Zitierungen vorgesehen, die in Fußnoten angegeben werden. Der Stil setzt die
folgenden Paketoptionen zur Ladezeit auf: 

\kvopt{autocite}{footnote}, \kvopt{citetracker}{context}, \kvopt{ibidtracker}{constrict}, \kvopt{pagetracker}{true}.


Dieser Stil unterstützt auch zusätzliche
Präambeloptionen mit den Namen \opt{ibidpage} und \opt{citepages}. Für Details
sehen Sie das Stilbeispiel.  

\item[verbose-note] Dieser Stil ist ähnlich mit dem \texttt{verbose}-Stil.
Bei diesem wird eine
komplette Zitierung ausgegeben ähnlich einem Bibliografieeintrag, wenn ein
Eintrag das erste Mal zitiert wird und eine kurze Zitierung danach zitiert wird.
Im Unterschied zu dem \texttt{verbose}-Stil ist die kurze Zitierung ein Zeiger
auf die Fußnote mit der kompletten Zitierung. Wenn die Bibliografie mehr als
zwei Arbeiten vom entsprechenden Autor\slash Editor beinhaltet, schließt der
Zeiger auch den Titel ein. Wenn verfügbar, wird das \bibfield{shorttitle}-Feld
in allen kurzen Zitierungen benutzt. Wenn das \bibfield{shorthand}-Feld
definiert ist, wird es wie beim \texttt{verbose}-Stil gehandhabt. Dieser Stil
kann ohne eine Liste von Referenzen und Abkürzungen benutzt werden, da alle
bibliografischen Daten bei der ersten Zitierung bereitgestellt werden. Er ist
ausschließlich für Zitierungen vorgesehen, die in Fußnoten angegeben werden. Der
Stil setzt die folgenden Paketoptionen zur Ladezeit auf:

\kvopt{autocite}{footnote}, \kvopt{citetracker}{context},
\kvopt{singletitle}{true}. 

Dieser Stil unterstützt auch zusätzliche
Präambeloptionen mit den Namen \opt{pageref} und \opt{citepages}. Für Details
sehen Sie das Stilbeispiel. 

\item[verbose-inote] Eine Variante des \texttt{verbose"=note}, der 
wiederholte Zitierungen mit der
Abkürzung \emph{ebenda} austauscht, außer wenn die Zitierung die erste auf der
aktuellen Seite oder Doppelseite ist, oder wenn \emph{ebenda} zweideutig im
Sinne der Paketoption \kvopt{ibidtracker}{strict} sein würde. Dieser Stil ist
ausschließlich für Zitierungen vorgesehen, die in Fußnoten angegeben
werden. Er setzt die folgenden Paketoptionen zur Ladezeit auf:

\kvopt{autocite}{footnote}, \kvopt{citetracker}{context}, \kvopt{ibidtracker}{constrict}, \kvopt{singletitle}{true}, \kvopt{pagetracker}{true}.

Dieser Stil unterstützt auch zusätzliche
Präambeloptionen mit den Namen \opt{ibidpage}, \opt{pageref} und
\opt{citepages}. Für Details sehen Sie das Stilbeispiel.    

\item[verbose-trad1] Dieser Stil führt ein übliches Zitierschema aus.
Er ist ähnlich dem
\texttt{verbose}-Stil. Bei diesem wird eine komplette Zitierung ausgegeben
ähnlich einem Bibliografieeintrag, wenn ein Eintrag das erste Mal zitiert
wird und eine kurze Zitierung danach zitiert wird. Ansonsten, benutzt er die
wissenschaftlichen Abkürzungen \emph{ebenda}, \emph{idem}, \emph{op.~cit.}
und \emph{loc.~cit.} um wiederkehrenden Autoren, Titel und Seitennummern in
wiederholten Zitierungen in einer spezieller Weise zu ersetzen. Wenn das
\bibfield{shorthand}-Feld definiert ist, wird die Abkürzung bei der ersten
Zitierung eingeführt und danach als die kurze Zitierung verwendet. Dieser Stil
kann ohne eine Liste von Referenzen und Kurzschriften benutzt werden, da alle
bibliografischen Daten bei der ersten Zitierung bereitgestellt werden. Er ist
für für Zitierungen vorgesehen, die in Fußnoten angegeben werden. Der Stil setzt
die folgenden Paketoptionen zur Ladezeit auf: 

\kvopt{autocite}{footnote}, \kvopt{citetracker}{context}, \kvopt{ibidtracker}{constrict}, \kvopt{idemtracker}{constrict}, \kvopt{opcittracker}{context}, \kvopt{loccittracker}{context}.

Dieser Stil unterstützt auch zusätzliche
Präambeloptionen mit den Namen \opt{ibidpage}, \opt{strict} und
\opt{citepages}. Für Details sehen Sie die Stilbeispiele.      

\item[verbose-trad2] Ein anderes übliches Zitierungsschema. Es ist ähnlich
dem \texttt{verbose}-Stil,
aber benutzt wissenschaftliche Abkürzungen wie \emph{ebenda} und \emph{idem} bei
wiederholten Zitierungen. Im Unterschied zum \texttt{verbose-trad1}-Stil ist
die Logik von der \emph{op.~cit.} Abkürzung in diesem Stil anders und
\emph{loc.~cit.} wird überhaupt nicht benutzt. Es ist sogar dem
\texttt{verbose-ibid} und \texttt{verbose-inote} ähnlicher als den
\texttt{verbose-trad1}. Der Stil setzt die folgenden Paketoptionen zur
Ladezeit auf:

\kvopt{autocite}{footnote}, \kvopt{citetracker}{context}, \kvopt{ibidtracker}{constrict}, \kvopt{idemtracker}{constrict}. 

Dieser Stil
unterstützt auch zusätzliche Präambeloptionen mit den Namen \opt{ibidpage},
\opt{strict} und \opt{citepages}. Für Details sehen Sie das Stilbeispiel.        

\item[verbose-trad3] Noch ein übliches Zitierungsschema. Es ist dem 
\texttt{verbose-trad2} ähnlich,
benutzt aber die wissenschaftlichen Abkürzungen, \emph{ebenda} und
\emph{op.~cit.}, in leicht abgewandelter Weise. Der Stil setzt die folgenden
Paketoptionen zur Ladezeit auf: 

\kvopt{autocite}{footnote}, \kvopt{citetracker}{context}, \kvopt{ibidtracker}{constrict}, \kvopt{loccittracker}{constrict}.

Dieser Stil unterstützt auch zusätzliche
Präambeloptionen mit den Namen \opt{strict} und \opt{citepages}. Für Details
sehen Sie das Stilbeispiel.        
 
 
\item[reading] Ein Zitierungsstil, der zum Bibliografiestil gehört mit dem
gleichen Namen. Er lädt lediglich den \texttt{authortitle}-Stil. 

\end{marglist}

Die folgenden Zitierstile sind Stile für besondere Zwecke. Sie sind nicht für
die Endfassung von einem Dokument vorgesehen:

\begin{marglist}

\item[draft] Ein Konzeptstil, der die Eintragsschlüssel in Zitierungen benutzt.
Der Stil setzt die folgenden Paketoptionen zur Ladezeit, auf:
\kvopt{autocite}{plain}. 

\item[debug] Dieser Stil gibt eher den Eintragsschlüssel aus als irgendeine Art
von Label. Er ist nur für die Fehlersuche vorgesehen und setzt die folgenden
Paketoptionen zur Ladezeit auf: \kvopt{autocite}{plain}.  

\end{marglist}

\subsubsection{Bibliografiestile} \label{use:xbx:bbx}

Alle Bibliografiestile, die mit diesem Paket kommen, benutzen dasselbe
grundlegende Format für die verschieden Bibliografieeinträge. Sie unterscheiden
sich nur in der Art des Labels, das in der Bibliografie ausgegeben wird und im
gesamten Formatieren der Liste von Referenzen. Es gibt einen übereinstimmenden
Bibliografiestil für jeden Zitierungsstil. Beachten Sie, dass manche
Bibliografiestile im Folgenden nicht erwähnt werden, da sie einfach einen
allgemeineren Stil laden. Zum Beispiel, der Bibliografiestil
\texttt{authortitle-comp} wird den \texttt{authortitle}-Stil laden.  

\begin{marglist}

\item[numeric] Dieser Stil gibt eine numerische Bezeichnung aus, ähnlich
den bibliografischen Standardfunktionen von \latex. Er ist für die Verwendung
im Zusammenhang mit einem numerischen Zitierungsstil vorgesehen. Beachten
Sie, dass
das \bibfield{shorthand}-Feld die vorgegebenen Labels aufhebt. Der Stil setzt
die folgenden Paketoptionen zur Ladezeit auf: \kvopt{labelnumber}{true}.

Dieser Stil unterstützt auch eine zusätzliche Präambeloption mit dem Namen
\opt{subentry}, die das Formatieren von Eintragssets beeinflusst. Wenn
diese Option aktiv ist, werden alle Mitglieder eines Sets mit einem Buchstaben
markiert, der in Zitierungen  benutzt werden kann, um lieber auf ein Setmitglied
Bezug zu nehmen als auf das ganze Set. Für Details sehen Sie das Stilbeispiel. 

\item[alphabetic] Dieser Stil gibt alphabetisches Labels aus, ähnlich dem
\path{alpha.bst}-Stil des herkömmlichen \bibtex. Er ist für die Verwendung
im Zusammenhang mit einem alphabetischen Zitierstil vorgesehen.
Beachten Sie, dass
das \bibfield{shorthand}-Feld die vorgegebenen Labels aufhebt. Der Stil setzt die
folgenden Paketoptionen zur Ladezeit, auf: \kvopt{labelalpha}{true},
\kvopt{sorting}{anyt}.  

\item[authoryear] Dieser Stil unterscheidet sich von den anderen Stilen darin,
dass der Erscheinungstermin nicht gegen Ende des Eintrages ausgegeben wird,
sondern eher nach dem Autor\slash Editor. Es ist für die Verwendung im
Zusammenhang mit einen Autor"=Jahr-Zitierstil vorgesehen. Wiederkehrende
Autoren- und Editornamen werden durch einen Bindestrich ersetzt, außer wenn der
Eintrag der erste auf der aktuellen Seite oder Doppelseite ist. Dieser Stil
unterstützt eine zusätzliche Präambeloption mit dem Namen \opt{dashed}, die
diese Eigenschaft kontrolliert. Er unterstützt auch eine Präambeloption mit
dem Namen \opt{mergedate}. Für Details sehen Sie das Stilbeispiel. Der Stil setzt die
folgenden Paketoptionen zur Ladezeit, auf: 

\kvopt{labeldateparts}{true}, \kvopt{sorting}{nyt}, \kvopt{pagetracker}{true}, \kvopt{mergedate}{true}.

\item[authortitle] Dieser Stil gibt überhaupt keine Labels aus. Er ist für die
Verwendung im Zusammenhang mit einen Autor"=Titel-Stil vorgesehen.
Wiederkehrende Autoren- und Editorenamen werden durch einen Bindestrich ersetzt,
außer wenn der Eintrag der erste auf der aktuellen Seite oder Doppelseite ist.
Dieser unterstützt auch eine Präambeloption mit dem Namen \opt{dashed}, die
diese Eigenschaft kontrolliert. Für Details sehen Sie das Stilbeispiel. Der Stil
setzt die folgenden Paketoptionen zur Ladezeit, auf: \kvopt{pagetracker}{true}.

\item[verbose] Dieser Stil ist dem \texttt{authortitle}-Stil ähnlich. Er
unterstützt auch eine Präambeloption mit dem Namen \opt{dashed}. Für Details
sehen Sie das Stilbeispiel. Der Stil setzt die folgenden Paketoptionen zur
Ladezeit auf: \kvopt{pagetracker}{true}.  

\item[reading] Dieser besondere Bibliografiestil ist für persönliche
Leselisten, kommentierte Bibliografien und Anwendungen vorgesehen. Es umfasst
wahlweise die Felder\\
\bibfield{annotation}, \bibfield{abstract},
\bibfield{library} und \bibfield{file} in der Bibliografie. Wenn gewünscht,
fügt er verschiedene Arten von kurzen Headers zu der Bibliografie. Dieser Stil
unterstützt auch die Präambeloptionen 

\opt{entryhead}, \opt{entrykey}, \opt{annotation}, \opt{abstract}, \opt{library}
und \opt{file}, die
kontrollieren, ob die entsprechenden Begriffe in der Bibliografie ausgegeben
werden. Für Details sehen Sie das Stilbeispiel. Siehe auch \secref{use:use:prf}. Der
Stil setzt die folgenden Paketoptionen zur Ladezeit auf:

\kvopt{loadfiles}{true}, \kvopt{entryhead}{true}, \kvopt{entrykey}{true}, \kvopt{annotation}{true}, \kvopt{abstract}{true}, \kvopt{library}{true}, 
\kvopt{file}{true}.

\end{marglist}

Die folgenden Bibliografiestile sind Stile für besondere Zwecke. Sie sind nicht
für die Endfassung von einem Dokument vorgesehen:

\begin{marglist}

\item[draft] Dieser Konzeptstil umfasst die Eintragsschlüssel in der
Bibliografie. Die Bibliografie wird nach den Eintragsschlüsseln sortiert. Der
Stil setzt die folgende Paketoption zur Ladezeit auf:

\kvopt{sorting}{debug}. 

\item[debug] Dieser Stil gibt alle bibliographischen Daten in tabellarischer Form
aus. Er ist nur für die Fehlersuche vorgesehen und setzt die folgende
Paketoption zur Ladezeit auf: \kvopt{sorting}{debug}. 

\end{marglist}

\subsection{Verwandte Einträge}
\label{use:rel}

Fast alle Bibliografiestile fordern von den Autoren bestimmte Beziehungen 
zwischen den Einträgen anzugeben: «Reprint of», «Reprinted in» etc. Es ist nicht
möglich Datenfelder für all diese Relationen zu schaffen.  \biblatex stellt
einen allgemeinen Mechanismus dafür bereit, der die Eintragsfelder \bibfield{related}, \bibfield{relatedtype} und \bibfield{relatedstring} 
verwendet. Ein verwandter Eintrag muss nicht zitiert werden und erscheint nicht selbst in der Bibliografie (es sei denn, er wird selbst unabhängig zitiert) 
als ein Clon, der von dem verwandten Eintrag genommen wird wie eine 
Datenquelle.  Das \bibfield{relatedtype}-Feld sollte eine spezifische
Lokalisierungszeichenfolge angeben, die geschrieben werden soll, bevor die
die Information aus den verwanden Einträgen geschrieben wird, beispielsweise   «Orig. Pub. wie». Das \bibfield{relatedstring}-Feld kann verwendet werden, um
via \bibfield{relatedtype} die Zeichenfolge zu überschreiben. Einige Beispiele:

\begin{lstlisting}[style=bibtex]{}
@Book{key1,
  ...
  related     = {key2},
  relatedtype = {reprintof},
  ...
}

@Book{key2,
  ...
}
\end{lstlisting}
%
Hier haben wir die Spezifik, dass der Eintrag \texttt{key1} eine Neuauflage
des Eintrags \texttt{key2} ist. In dem Bibliografietreiber für
\texttt{Book}-Einträge, wenn \cmd{usebibmacro\{related\}}, wird der Eintrag  \texttt{key1} genannt:

\begin{itemize}
\item Wenn der Lokalisierungsstring «\texttt{reprintof}» definiert wird, wird er
im \texttt{relatedstring:reprintof}-Format gedruckt. Wenn diese Formatierungsrichtlinie nicht definiert ist, wird die Zeichenfolge im   \texttt{relatedstring:default}-Format gedruckt.
\item Wenn das \texttt{related:reprintof}-Makro definiert ist, wird es verwendet, um die Informationen im Eintrag \texttt{key2} zu formatieren, ansonsten wird das  \texttt{related:default}-Makro verwendet. 
\item Wenn das \texttt{related:reprintof}-Format definiert ist, wird es verwendet,
um die lokalisierungszeichenfolge und die Daten zu formatieren.
Wenn dieses Format nicht definiert ist, dann wird stattdessen das 
\texttt{related}-Format verwendet.
\end{itemize}
%
Es wird auch eine Kaskadierung und/oder kreisförmige Beziehung unterstützt: 

\begin{lstlisting}[style=bibtex]{}
@Book{key1,
  ...
  related     = {key2},
  relatedtype = {reprintof},
  ...
}

@Book{key2,
  ...
  related     = {key3},
  relatedtype = {translationof},
  ...
}

@Book{key3,
  ...
  related     = {key2},
  relatedtype = {translatedas},
  ...
}
\end{lstlisting}
%
Mehrere Beziehungen zum selben Eintrag sind möglich:
\begin{lstlisting}[style=bibtex]{}
@MVBook{key1,
  ...
  related     = {key2,key3},
  relatedtype = {multivolume},
  ...
}

@Book{key2,
  ...
}

@Book{key3,
  ...
}
\end{lstlisting}
%
Beachten Sie die Reihenfolge der Schlüssel in den Listen aufeinander bezogener 
Einträge; dies ist wichtig. Die Daten von mehreren  aufeinander bezogenen Einträgen werden in der Reihenfolge der aufgeführten Schlüssel in diesem Feld gedruckt.
Sehen Sie %\secref{aut:ctm:rel} 
§ 4.5.1 (engl. Vers.) für weitere Details zu den Mechanismen hinter diesen Merkmalen. Sie können diese Funktionen deaktivieren, indem Sie die Paketoption \opt{related} aus \secref{use:opt:pre:gen} nehmen.

Sie können die \bibfield{relatedoptions}-Option auf die Optionen der verwandten
Eintragsdatenklone setzen. Die ist nützlich, wenn Sie die \opt{dataonly}-Option
überschreiben müssen, die standardmäßig auf alle verwandten Eintragsklone eingerichtet ist. Beispielsweise, wenn Sie von den Namen einige in den verwandten Klonen in ihrem Dokument aussetzen wollen, Sie können die namen in den anderen Einträgen vereindeutigt haben, aber normalerweise wird dies nicht über verwandte Klone geschehen, die  pro"=Eintrag-\opt{dataonly}-option wird gesetzt und
dies wiederum setzt \kvopt{uniquename}{false} und \kvopt{uniquelist}{false}. In solch einem Fall können Sie \bibfield{relatedoptions} setzen, nur \opt{skiplab, skipbib, skiplos/skipbiblist}.

\subsection{Sortierungsoptionen} \label{use:srt}

Dieses Paket unterstützt verschiedene Sortierungsschemata für
Bibliografien.
Das Sortierungsschema wird  mit der Paketoption \opt{sorting} aus
\secref{use:opt:pre:gen} ausgewählt. Abgesehen von den normalen
Datenfeldern gibt es auch
besondere Felder, die benutzt werden können, um das Sortieren der
Bibliografie zu verbessern. C.1 und C.2 (e. V.) %\Apxref{apx:srt:a1, apx:srt:a2} 
geben einen Entwurf
des alphabetischen Schema an, das von \biblatex unterstützt wird.
Chronologische Sortierungsschemata sind in C.3 %\apxref{apx:srt:chr} 
aufgelistet. Einige Erklärungen bezüglich dieser Schemata sind sortiert.        

Der erste Begriff, der im Sortierungsprozess berücksichtigt wird, ist immer das
\bibfield{presort}-Feld des Eintrages. Wenn dieses Feld nicht definiert ist,
wird \biblatex den vorgegebenen Wert <\texttt{mm}> als eine vorsortierte
Reihung benutzen. Der nächste Begriff, der berücksichtigt wird, ist das
\bibfield{sortkey}-Feld. Wenn dieses Feld nicht definiert ist, dient es als
Hauptsortierungsschlüssel. Abgesehen vom \bibfield{presort}-Feld, werden keine
weiteren Daten in diesem Fall berücksichtigt. Wenn das \bibfield{sortkey}-Feld
nicht definiert ist, wird das Sortieren mit den Namen fortgesetzt. Das Paket
wird versuchen, die \bibfield{sortname}-, \bibfield{author}-,
\bibfield{editor}- und \bibfield{translator}-Felder in dieser Reihenfolge zu benutzen.
Welche Felder
berücksichtigt werden, hängt von der Konfiguration der \opt{useauthor}-,
\opt{useeditor}- und \opt{usetranslator}-Optionen ab. Wenn alle drei aktiv sind,
wird das \bibfield{sortname}-Feld ebenfalls ignoriert. Beachten Sie, dass alle
Namenfelder auf die Konfiguration der allgemeinen \opt{maxnames}- und
\opt{minnames}-Optionen reagieren. Wenn kein Feld verfügbar ist, entweder weil
alle undefiniert sind oder weil \opt{useauthor}, \opt{useeditor} und
\opt{usetranslator} deaktiviert sind, wird \biblatex auf die
\bibfield{sorttitle}- und \bibfield{title}-Felder als letzten Ausweg
zurückgreifen. Die restlichen Begriffe sind in verschiedener Reihenfolge: Das
\bibfield{sortyear}-Feld, wenn definiert, oder ansonsten die ersten vier Zahlen
von dem \bibfield{year}-Feld; das \bibfield{sorttitle}-Feld, wenn definiert oder
ansonsten das \bibfield{title}-Feld; das \bibfield{volume}-Feld, welches zu vier
Zahlen mit führenden Nullen ausgefüllt wird oder ansonsten die Reihung
\texttt{0000}. Beachten Sie, dass die Sortierungsschemata, die in C.3 % \apxref{apx:srt:a2}
dargestellt sind, einen zusätzlichen Begriff enthalten: \bibfield{labelalpha} ist das
Label, benutzt vom <alphabetischen> Bibliografiestil. Genau genommen ist die
Reihung, die für das Sortieren benutzt wird, \bibfield{labelalpha}~+
\bibfield{extraalpha}. Die Sortierungsschemata im Appendix (engl. V.) %\apxref{apx:srt:a2} sind
vorgesehen, nur in Zusammenhang mit alphabetischen Stilen benutzt zu werden.                            

Die chronologischen Sortierungsschemata, aufgeführt im Appendix (engl. V.)
%\apxref{apx:srt:chr},
benutzen auch die \bibfield{presort}- und \bibfield{sortkey}-Felder, wenn
definiert. Der nächste Begriff, der betrachtet wird, ist das
\bibfield{sortyear}- oder das \bibfield{year}-Feld, abhängig von der Verfügbarkeit.
Das \opt{ynt}-Schema zieht die ersten vier arabischen Zahlen aus den Feld. Wenn beide 
Felder nicht definiert sind, wird die Reihung \texttt{9999} als Rückgriffswert
benutzt. Das bedeutet, dass alle Einträge ohne Jahr zum Ende der Liste gerückt
werden. Das \opt{ydnt} ist im Konzept ähnlich, sortiert aber Jahr in
absteigender Reihenfolge. Wie auch beim \opt{ynt}-Schema wird die Reihung
\texttt{9999} als Rückgriffswert benutzt. Die restlichen Begriffe sind ähnlich
den alphabetischen Sortierungsschemata, die oben behandelt wurden. Beachten
Sie, dass
das \opt{ydnt}-Sortierungsschema die Daten nur in absteigender Reihenfolge
sortieren wird. Alle anderen Begriffe  werden, wie gewohnt, in aufsteigender
Reihenfolge sortiert.             

Das Benutzen von besonderen Feldern wie \bibfield{sortkey}, \bibfield{sortname} oder
\bibfield{sorttitle} wird normalerweise nicht benötigt. Das \biblatex-Paket
ist durchaus fähig, die erwünschte Sortierreihenfolge durch das Benutzen der Daten,
die in regulären Feldern eines Eintrags gefunden wurden, zu erstellen. Man wird
sie nur gebrauchen, wenn man manuell die Sortierreihenfolge der Bibliografie
ändern will oder wenn irgendwelche Daten, die für das Sortieren benötigt werden,
fehlen. Bitte verweisen Sie auf die Feldbeschreibung in \secref{bib:fld:spc} für
Details bei möglicher Verwendungen der besonderen Felder. 

\subsection{Daten"=Annotationen}\label{use:annote}

Idealerweise sollten in einer Bibliografiedatendatei keine Formatierungsinformationen
sein, jedoch scheint es manchmal so, dass es eine fragwürdige Praxis ist, dies als
einzigen Weg für die gewünschten Ergebnisse anzusehen. Daten-Annotationen sind ein Weg
dies anzugehen, indem Nutzer semantische Informationen anfügen (und nicht als
typografische Markups), Informationen in einer Bibliografiedatenquelle, so dass
die Information genutzt werden kann für das Markup durch einen Stil. 
Beispielsweise, wenn man in einem Werk bei einem bestimmten Namen markieren möchte,
ob es sich um einen Studentenautor (angezeigt durch ein hochgestelltes Sternchen)
oder einen korrespondierenden Autor (angezeigt durch Fettdruck) handelt, dann könnte 
man es folgendermaßen versuchen: 
\
\begin{lstlisting}[style=bibtex]{}
@MISC{Article1,
  AUTHOR = {Last1\textsuperscript{*}, First1 and \textbf{Last2}, \textbf{First2} and Last3, First3}
}
\end{lstlisting}
%
Damit gibt es mehrere Probleme. Zum einen werden damit \bibtex s fragile 
Namensparsing-Routinen durchbrochen und wahrscheinlich dieses gar nicht kompilieren. 
Zum anderen ist es nicht nur das vermischen von Daten mit Markup, 
es wird so ein hartcodierter Vorgang: Diese Daten können nicht leicht geteilt und mit anderen Stilen verwendet werden. Dagegen ist es möglich, diese Formatierung mit 
\biblatex-Einbauten in einen Stil oder in einem Dokument zu erreichen, dies ist eine komplexes und unzuverlässiges Verfahren, das wir nicht vielen Benutzer für die Benutzung wünschen. 

Um diese Probleme zu lösen, hat \biblatex eine allgemeine Datenannotationseinrichtungh,
die ermöglicht, mit einer Komma"=separierten Liste von Datenanmerkungen in Datenfeldern
aufzulisten, Einträge in Datenfeldlisten (wie Namen) und sogar Teile von spezifischen Einträgen, wie Teile von Namen (Vorname, Familienname etc.). Es gibt Makros,
entwickelt zur Überprüfung von Annotationen, die in Formatierungsanweisungen verwendet
werden können.

Es gibt drei «scopes» für Daten-Annotationen, angeordnet nach zunehmender Spezifik:

\begin{itemize}
\item \opt{field}--angelegt für "`top-level"'-Felder in einem Datenquell-Eintrag. 
\item \opt{item}--angelegt für Einträge eines Listenfelds in einem Datenquell-Eintrag.  
\item \opt{part}--angelegt für Teile von Einträgen eines Listenfelds in einem 
Datenquell-Eintrag.
\end{itemize}
%
Daten-Annotationen werden supportet für \bibtex- und \biblatexml-Datenquellen.

\begin{lstlisting}[style=bibtex]{}
@MISC{ann1,
  AUTHOR           = {Last1, First1 and Last2, First2 and Last3, First3},
  AUTHOR+an        = {1:family=student;2=corresponding},
  TITLE            = {The Title},
  TITLE+an:default = {=titleannotation},
  TITLE+an:french  = {="Le titre"},
  TITLE+an:german  = {="Der Titel"}}
\end{lstlisting}
%
Das Feldnamen-Suffix \texttt{+an} ist benutzerdefinierbar\footnote{Beachten sie die Option \biber's \opt{--annotation-marker}.} Ein Suffix, das ein Datenfeld als eine Annotation des nicht angehängten 
Felds markiert. Für dasselbe Feld können mehrere Anmerkungen bereitgestellt werden, da alle 
Anmerkungen benannt sind. Nach dem Annotationsmarker befindet sich der optional benannte Anmerkungsmarker  \footnote{Beachten sie die \biber Option \opt{--named-annotation-marker}.} und 
ein optionaler Annotationsname. Der Annotationname ist <default>, wenn er nicht spzifiziert wurde;
im obigen Beispiel sind die beiden Folgenden gleichwertig:

\begin{lstlisting}[style=bibtex]{}
TITLE+an         = {=titleannotation},
TITLE+an:default = {=titleannotation},
\end{lstlisting}
%
Das Format des Annotationsfeldes in der  \bibtex -Datenquelle sieht wie folgt aus:

\begin{lstlisting}
<annotationspecs> ::= <annotationspec> [ ";" <annotationspec> ]
<annotationspec>  ::= [ <itemcount> [ ":" <part> ] ] "=" <annotations>
<annotations>     ::= <annotation> [ "," <annotation> ]
<annotation>      ::= ["] (string) ["]
\end{lstlisting}
%
Das heißt, eine oder mehrere Spezifikationen sind durch Semikolons getrennt. 
Jed Angabe hat ein Gleichheitszeichen, gefolgt von einer Komma"=separierten Liste von 
Annotationsschlüsselwörtern oder einer Zeichenfolge in doppelten Anführungszeichen 
(eine <literal> Annotation, siehe unten ). Um einen spezifischen Eintrag in einer Liste mit 
einer Annotation zu versehen, geben sie die Nummer des Listenelements vor dem Gleichheitszeichen ein
(Listen beginnen mit der 1). Wenn sie einen bestimmten Teil des Listenelements mit Annotationen
versehen möchten, geben sie seinen Namen nach der Nummer des Listenelements ein, der ein Doppelpunkt voramgestellt ist. Namen der Namensteile werden im Datenmodell definiert, sehen sie § 4.2.3 (e.~V.) %\secref{aut:bbx:drv}. 
Einige weitere Beispiele:

\begin{lstlisting}[style=bibtex]{}
AUTHOR      = {Last1, First1 and Last2, First2 and Last3, First3},
AUTHOR+an   = {3:given=annotation1, annotation2},
TITLE       = {A title},
TITLE+an    = {=a title annotation, another title annotation},
LANGUAGE    = {english and french},
LANGUAGE+an = {1=annotation3; 2=annotation4}
}
\end{lstlisting}
%
Annotationen an Daten anzuhängen ist in \biblatexml-Datenquellen einfach, wenn sie durch einfache XML-Attribute festgelegt sind. Bezugnehmend auf das obige Beispiel, haben wir: 

\begin{lstlisting}[language=xml]{}
<bltx:entries xmlns:bltx="http://biblatex-biber.sourceforge.net/biblatexml">
  <bltx:entry id="test" entrytype="misc">
    <bltx:names type="author">
      <bltx:name>
        <bltx:namepart type="given" initial="F">First1</bltx:namepart>
        <bltx:namepart type="family" initial="L">Last1</bltx:namepart>
      </bltx:name>
      <bltx:name>
        <bltx:namepart type="given" initial="F">First2</bltx:namepart>
        <bltx:namepart type="family" initial="L">Last2</bltx:namepart>
      </bltx:name>
      <bltx:name>
        <bltx:namepart type="given" initial="F">First3</bltx:namepart>
        <bltx:namepart type="family" initial="L">Last3</bltx:namepart>
      </bltx:name>
    </bltx:names>
    </bltx:annotation field="author" item="1" part="family">student</bltx:annotation>
    </bltx:annotation field="author" item="2">corresponding</bltx:annotation>
  </bltx:entry>
</bltx:entries>
\end{lstlisting}
%
Um beim Formatieren von Bibliografiedaten auf die Annotationinformationen zuzugreifen, werden Makros bereitgestellt, die den drei Annotationsbereichen entsprechen:

\begin{ltxsyntax}

\cmditem{iffieldannotation}{annotation}{true}{false}

Wählt \prm{true}, wenn das aktuelle Datenfeld eine Annotation hat \prm{annotation} und 
false andernfalls.

\cmditem{ifitemannotation}{annotation}{true}{false}

Wählt \prm{true}, wenn das aktuelle Item in dem aktuellen Datenfeld eine 
Annotation hat \prm{annotation} und false andernfalls.  

\cmditem{ifpartannotation}{part}{annotation}{true}{false}

Wählt \prm{true}, wenn der Teilnamen \prm{part} des aktuellen Items in dem 
aktuellen Datenfeld eine 
Annotation hat \prm{annotation} und false andernfalls.

\end{ltxsyntax}
%
Diese Makros sind an den gleichen Stellen wie \cmd{currentfield}, \cmd{currentlist} 
und \cmd{currentname} (sehen Sie § 4.4.2 (e.~V.), %\secref{aut:bib:fmt}), 
das heißt innerhalb der
Formatierungsanweisungen. Sie bestimmen automatisch den Namen des aktuellen 
Datenfelds, das verarbeitet wird, und auch den aktuellen \opt{listcount}-Wert, der den aktuellen Eintrag in Listenfeldern bestimmt.  Teile, sowas wie Namensteile, müssen explizit benannt werden. Ein Beispiel dafür, wie die Annotationsinformation zu verwenden ist,
wird, um das eingangs im Kapitel präsentierte Problem zu lösen: Dies könnte verwendet werden bei den Richtlinien der Namensformatierung, für das Setzen von Sternchen nach allen Namen, die als  «Student» annotiert werden sollen:

\begin{lstlisting}[style=latex]{}
  \ifpartannotation{family}{student}
    {\textsuperscript{*}}
    {}%
\end{lstlisting}
%
Um  Familiennamen aus Namenslisteneinträgen mit Fettdruck als «korrespondierend» 
zu markieren:

\begin{lstlisting}[style=latex]{}
\renewcommand*{\mkbibnamegiven}[1]{%
  \ifitemannotation{corresponding}
    {\textbf{#1}}
    {#1}}

\renewcommand*{\mkbibnamefamily}[1]{%
  \ifitemannotation{corresponding}
    {\textbf{#1}}
    {#1}}
\end{lstlisting}

\subsubsection{Wörtliche\newline Annotations}

Wenn die Annotation eine in doppelte Anführungszeichen eingeschlossene Zeichenfolge ist, ist die 
Annotation ein <literal> Annotation. In diesem Fall kann die Annotation abgerufen und als 
Zeichenfolge und nicht als Metainformation zur Bestimmung der Formatierung verwendet werden.
Dies ist nützlich, um bestimmte Anmerkungen an daten anzuhängen, die unverändert gedruckt werden sollen.
Zum Beispiel:

\begin{lstlisting}[style=bibtex]{}
AUTHOR = {{American Educational Research Association} and {American Psychological Association}
            and {National Council on Measurement in Education}},
AUTHOR+an = {1:family="AERA"; 2:family="APA"; 3:family="NCME"}
}
\end{lstlisting}
%
Solche Annotationen sind keine Schlüssel, deren Vorhandensein getestet werden kann, sondern es
sind wörtliche Informationen, angehängt an den Daten. Die Werte werden von den folgenden
Makros abgerufen.

\begin{ltxsyntax}

\cmditem{getfieldannotation}[field][annotationname]

Ruft alle wörtlichen Annotationen für das Feld \prm{field} ab. Wenn \prm{annotationname} 
nicht angegeben wird, wird die Annotation <default> angenommen (dies ist der Name
für Annotationen, die ohne einen expliziten Namen definiert wurden). Wenn \prm{field}
    nicht angegeben wird, wird das aktuelle Datenfeld angenommen, dass durch \cmd{currentfield}, \cmd{currentlist} oder \cmd{currentname} angegen wird (sehen sie §~4.4.2 (engl. Vers.). % \secref{aut:bib:fmt}).
Die ist natürlich nur möglich, wenn diese Befehle innerhalb innerhalb von Formatierungsanweisungen definiert sind.

\cmditem{getitemannotation}[field][annotationname][item]

Ruft alle wörtlichen Annotationen für das Feld \prm{field} ab. Wenn \prm{annotationname} 
nicht angegeben wird, wird die Annotation <default> angenommen (dies ist der Name
für Annotationen, die ohne einen expliziten Namen definiert wurden). Das optionale
Argument \prm{field} kann abgeleitet werden, wenn nicht wie mit \cmd{getfieldannotation}
angegeben. Wenn \prm{item} nicht angegeben wird, wird die Eintragsnummer des aktuell verarbeiteten Elements angenommen.

\cmditem{getpartannotation}[field][annotationname][item]{part}

Ruft alle wörtlichen Annotationen für für den Teil \prm{part}ab. Wenn \prm{annotationname} 
nicht angegeben wird, wird die Annotation <default> angenommen (dies ist der Name
für Annotationen, die ohne expliziten Namen definiert wurden). Die beiden optionalen
Argumente \prm{field} und \prm{item} können abgeleitet werden, so wie \cmd{getitemannotation}. Der Parameter \prm{part} kann niemals abgeleitet werden und ist daher ein obligatorisches Argument.

Datenfelder sind speziell und agieren in einem Kontext, wo \cmd{currentfield} nicht
zugänglich ist. Daher gibt es einen vierten Befehl, um auf wörtliche Annotationen für Daten 
zugreifen zu können.

\cmditem{getdateannotation}[annotationname]{datetype}

Ruft alle wörtlichen Annotationen für das Datenfeld \prm{datetype} ab. Wenn \prm{annotationname} nicht angegeben wird, dann wird die annotation <default> genommen (ies ist der Name
für Annotationen, die ohne expliziten Namen definiert wurden). Das Argument \prm{datetype} ist
obligatorisch, da es in den meisten Kontexten nicht zugänglich ist, wo \cmd{getdateannotation} verwendet wird.

\end{ltxsyntax}
%
In Anbetracht des obigen Bibliografieeintrages können wir beispielsweise Folgendes
in die Präamble aufnehmen:

\begin{ltxexample}[style=latex]
\renewcommand*{\mkbibnamefamily}[1]{%
  #1\space\mkbibparens{\getpartannotation{family}}}
\end{ltxexample}
%
Um beim Formatieren so etwas in die Bibliografie aufzunehmen:

\begin{lstlisting}[style=bibtex]{}
  American Educational Research Association (AERA) and
  American Psychological Association (APA), and
  National Council on Measurement in Education (NCME)
}
\end{lstlisting}
%
Natürlich gibt es semantisch elegantere Möglichkeiten des Umgehens mit korporierenden
Autoren, ohne den Namensteil <family> zu verwenden (sehen sie~§ 4.2.3), % \secref{aut:bbx:drv}),
jedoch zeigt dieses Beispiel deutlich die Verwendung von wörtlichen Anmerkungen.

\subsubsection{Bibliografiebefehle} \label{use:bib}

\subsubsection{Ressourcen} \label{use:bib:res}

\begin{ltxsyntax}

\cmditem{addbibresource}[options]{resource}

Fügt eine \prm{resource}, wie beispielsweise eine \file{.bib}-Datei, zu der
vorgegebenen Ressourceliste. Dieser Befehl kann nur in der Präambel benutzt
werden. Er ersetzt den alten \cmd{bibliography}-Befehl. Beachten Sie, dass Dateien mit
ihren vollständigen Namen festgelegt werden müssen, einschließlich der
Erweiterungen. Mit \biber kann der Ressourcenname ein Muster im "`BSD-style glob"' Stile sein.
Dies ist nur sinnvoll, wenn Ressourcen auf Dateien mit einem absoluten oder relativen
Pfad verweisen und nicht funktionieren, wenn Datenressourcen in den Eingabe-/Ausgabeverzeichnis 
von \biber oder mit Ressourcen, die sich in \prm{kpsewhich} etc.befinden, gesucht wird.
Unter Windows / \biber wird in einen Windows-kompatiblen "`globbing"'-Modus gewechselt, in
dem auch Backslashes verwendet werden können, da Pfadtrennzeichen und Groß- und 
Kleinschreibung keine Rolle spielen.  Wenn die Ressourcen doppelte Einträge enthalten
(d.\,h. doppelt \bibfield{entrykey}s), hängt es vom Backend ab, was passiert. Beispielsweise
ignoriert \biber standardmäßig das weitere Auftreten von \bibfield{entrykey}, sofern nicht die
Optionen \opt{--noskipduplicates} verwendet werden.
Rufen Sie
\cmd{addbibresource}  mehrmals auf, um mehr Ressourcen hinzuzufügen, zum
Beispiel:       

\begin{lstlisting}[style=latex]{} 
\addbibresource{bibfile1.bib}
\addbibresource{bibfile2.bib}
\addbibresource{bibfiles/bibfile*.bib}
\addbibresource{bibfile-num?.bib}
\addbibresource{bibfile{1,2,3}.bib}
\addbibresource[location=remote]{http://www.citeulike.org/bibtex/group/9517}
\addbibresource[location=remote,label=lan]{ftp://192.168.1.57/~user/file.bib}
\end{lstlisting}
%
Da die \prm{resource}-Reihung in einer Wort für Wort ähnlichen Weise gelesen
wird, beinhaltet es möglicherweise beliebige Zeichen. Die einzige Einschränkung
ist, dass jede geschweifte Klammer ausgeglichen werden muss. Die folgenden
\prm{options} sind verfügbar:    

\begin{optionlist*}

\valitem{bibencoding}{bibencoding}

Diese Option kann benutzt werden, um die globale \opt{bibencoding}-Option für eine partikuläre \prm{resource} zu überschreiben.
    

\valitem{label}{identifier}

Weist  einer Ressource ein Label zu. Der \prm{identifier} kann an Stelle des
vollständigen Ressourcenamens im optionalen Argument von \env{refsection}
verwendet werden (siehe § \ref{use:bib:sec}).

\valitem[local]{location}{location}

Der Ort der Ressource. Die \prm{location} kann entweder \texttt{lokal} sein für
lokale Ressourcen oder \texttt{entfernt} für \acr{URL}s. Entfernte Ressourcen
benötigen \biber. Die Protokolle \acr{HTTP} und \acr{FTP} werden unterstützt. Die
entfernte \acr{URL} muss ein vollständiger berechtigter Pfad zu einer
\file{bib}-Datei sein oder eine \acr{URL}, die eine \file{bib} Datei zurückgibt.     

\valitem[file]{type}{type}

Der Typ einer Ressource. Derzeit der einzige unterstützte Typ ist
\texttt{file}. 

\valitem[bibtex]{datatype}{datatype}

Der Datentyp (Format) der Ressource. Die folgenden Formate werden derzeit
unterstützt: 

\begin{valuelist}

\item[bibtex] \bibtex-Format.

\item[biblatexml] Experimental XML-Format für 
\biblatex. Sehen Sie D. %\secref{apx:biblatexml}.

\end{valuelist}

\end{optionlist*}

\cmditem{addglobalbib}[options]{resource}

Dieser Befehl unterscheidet sich von \cmd{addbibresource}, indem die
\prm{resource} zu der allgemeinen Ressourceliste hinzugefügt ist. Der
Unterschied zwischen den vorgegebenen Ressourcen und den allgemeinen Ressourcen
ist nur relevant, wenn es Referenzenabschnitte im Dokument gibt und die
optionalen Argumente von \env{refsection} (\secref{use:bib:sec}) benutzt
werden, 
um alternative Ressourcen festzulegen, die die vorgegebene Ressourceliste
ersetzt. Alle allgemeinen Ressourcen werden zu allen Referentenabschnitten
hinzugefügt.

\cmditem{addsectionbib}[options]{resource}

Dieser Befehl unterscheidet sich von \cmd{addbibresource}, indem die
Ressourcen \prm{options} registriert werden, aber die \prm{Ressourcen} zu keiner
Ressourceliste hinzugefügt werden. Dies wird nur für Ressourcen benötigt,\\
die 1. ausschließlich in den optionalen Argumente von \env{refsection}
(\ref{use:bib:sec}) stehen und \\
2. Optionen benötigen, die  von den
vorgegebenen Einstellungen abweiche. In diesen Fall kommt \cmd{addsectionbib} zum
Einsatz, um die \prm{resource} vorher zu markieren für das Benutzen, beim
Festlegen der geeigneten \prm{Optionen} in der Präambel. Die \opt{label}-Option
kann nützlich sein, um kurze Namen den Ressourcen zuzuordnen.

\cmditem{bibliography}{bibfile, \dots}\DeprecatedMark

Der alte Befehl für das Hinzufügen von bibliografischen Ressourcen, er unterstützt
die umgekehrte Kompatibilität. Wie \cmd{addbibresource} kann dieser Befehl
nur in der Präambel benutzt werden und fügt Ressourcen zu der vorgegebenen
Ressourceliste hinzu. Sein Argument ist eine Komma"=getrennte Liste von
\file{bib}-Dateien. Die \file{.bib}-Erweiterung kann vom Dateinamen weggelassen
werden. Den Befehl mehrmals aufzurufen, mehrere Dateien hinzuzufügen, ist erlaubt.
Dieser Befehl wird missbilligt. Bitte überlegen Sie stattdessen, \cmd{addbibresource}
zu benutzen.  

\subsection{Die Bibliografie} \label{use:bib:bib}

\cmditem{printbibliography}[key=value, \dots]

Dieser Befehl gibt die Bibliografie aus. Er erfordert ein optionales Argument,
welches eine Liste von Optionen ist, die in \keyval-Notation angegeben ist. Die
folgenden Optionen sind verfügbar: 

\end{ltxsyntax}

\begin{optionlist*}

\valitem[bibliography/shorthands]{env}{name}

Das <high-level>-Layout der Bibliografie und die Liste der Kürzel wird
von der Nachbarschaft mit \cmd{defbibenvironment} kontrolliert. Diese Option
sucht eine Nachbarschaft aus. Der \prm{name} entspricht dem Identifier, der
benutzt wird, wenn die Nachbarschaft mit \cmd{defbibenvironment} definiert wird.
Automatisch benutzt der \cmd{printbibliography}-Befehl den Identifier
\texttt{bibliography}; \cmd{printshorthands} benutzt \texttt{shorthands}. Sehen Sie
auch \secref{use:bib:biblist, use:bib:hdg}.       

\valitem[bibliography/shorthands]{heading}{name}

Die Bibliografie und die Liste von Kürzeln haben typischerweise eine
Kapitel- oder eine Teilüberschrift. Diese Option wählt den Überschrift \prm{name}
aus, wie mit \cmd{defbibheading} definiert wird. Automatisch benutzt der
\cmd{printbibliography}-Befehl die Überschrift \texttt{bibliography};
\cmd{printshorthands} benutzt \texttt{shorthands}. Sehen Sie auch
\secref{use:bib:biblist,use:bib:hdg}.    

\valitem{title}{text}

Diese Option überschreibt den vorgegebenen Titel, der von der Überschritft
bereitgestellt wird, die mit der \opt{heading}-Option ausgewählt wurde, wenn sie
von der Überschriftdefinition unterstützt wird. Für Details sehen Sie
\secref{use:bib:hdg}.  

\valitem{prenote}{name}

Die Prenote (Vornotiz) ist ein beliebiges Stück Text, das nach der Überschrift ausgegeben
wird, aber von vor der Liste von Referenzen stammt. Diese Option wählt die Prenote
\prm{name} aus, wie in \cmd{defbibnote} definiert. Automatisch wird keine
Prenote ausgegeben. Die Notiz wird in der Standardtextschrift ausgegeben. Sie
wird nicht von \cmd{bibsetup} und \cmd{bibfont} beeinflusst, aber sie kann ihre
eigene Schriftdeklaration beinhalten. Für Details sehen Sie \secref{use:bib:nts}.   

\valitem{postnote}{name}

Die Postnote ist ein beliebiges Stück Text, das nach der Liste von Referenzen
ausgegeben wird. Diese Option wählt den Postnote-\prm{name} aus, wie er in
\cmd{defbibnote} definiert ist. Automatisch wird keine Postnote ausgegeben. Die
Notiz wird in der Standardtextschrift ausgegeben. Sie wird nicht von
\cmd{bibsetup} und \cmd{bibfont} beeinflusst, aber sie kann ihre eigene
Schriftdeklaration beinhalten. Für Details sehen Sie \secref{use:bib:nts}. 

\intitem[\normalfont\em current section]{section}

Gibt nur Einträge aus, die in der Referenzsektion \prm{integer} zitiert wurden.
Die Referenzsektionen sind nummeriert, beginnend bei~1. Alle Zitierungen, die
außerhalb einer \env{refsection}-Nachbarschaft angegeben werden, sind der
Section~0 zugeordnet. Für Details sehen Sie \secref{use:bib:sec} und für
Anwendungsbeispiele \secref{use:use:mlt}.

\intitem{segment}

Gibt nur Einträge aus, die im Referenzsegment \prm{integer} zitiert wurden. Die
Referenzsgmente sind nummeriert, beginnend bei~1. Alle Zitierungen, die außerhalb
einer \env{refsegment}-Nachbarschaft angegeben werden, sind dem Segment~0
zugeordnet. Für Details sehen Sie \secref{use:bib:seg} und für
Anwendungsbeispiele \secref{use:use:mlt}. 

Denken Sie daran, dass die Segmente in einem Abschnitt inbezug auf den Abschnitt lokal nummeriert werden, so dass das Segment, das Sie anfordern, das n-te Segment im gewünschten (oder gerade aktive einschließend) Abschnitt sein wird.

\valitem{type}{entrytype}

Gibt nur Einträge aus, deren Eintragtyp \prm{entrytype} ist. 

\valitem{nottype}{entrytype}

Gibt nur Einträge aus, deren Eintragtyp nicht \prm{entrytype} ist. Diese Option
kann mehrere Male benutzt werden.   

\valitem{subtype}{subtype}

Gibt nur Einträge aus, deren \bibfield{entrysubtype} definiert ist und
\prm{subtype} ist.

\valitem{notsubtype}{subtype}

Gibt nur Einträge aus, deren \bibfield{entrysubtype} undefiniert ist und nicht
\prm{subtype} ist. Diese Option kann mehrere Male benutzt werden.   

\valitem{keyword}{keyword}

Gibt nur Einträge aus, deren \bibfield{keywords} Feld ein \prm{keyword} mit
einschließt. Diese Option kann mehrere Male benutzt werden.  

\valitem{notkeyword}{keyword}

Gibt nur Einträge aus, deren \bibfield{keywords} kein \prm{keyword} mit
einschließt. Diese Option kann mehrere Male benutzt werden.   

\valitem{category}{category}

Gibt nur Einträge aus, die zur Kategorie \prm{category} zugeteilt sind. Diese
Option kann mehrere Male benutzt werden.  

\valitem{notcategory}{category}

Gibt nur Einträge aus, die nicht zur Kategorie \prm{category} zugeteilt sind.
Diese Option kann mehrere Male benutzt werden. 

\valitem{filter}{name}

Filtert die Einträge mit dem Filter \prm{name}, wie er mit \cmd{defbibfilter}
definiert ist. Für Details sehen Sie \secref{use:bib:flt}.  

\valitem{check}{name}

Filtert die Einträge mit dem Check \prm{name}, wie er mit \cmd{defbibcheck}
definiert ist. Für Details sehen Sie \secref{use:bib:flt}. 

\boolitem{resetnumbers}

Diese Option ist nur bei numerischen Zitierungen\slash Bibliografiestilen
anwendbar und verlangt, dass die \opt{defernumbers}-Option von
\secref{use:opt:pre:gen} allgemein aktiv ist. Wenn aktiv, wird sie die
numerischen Label zurücksetzen, die den Einträgen in der entsprechenden
Bibliografie zugeteilt werden, z.\,B.  wird die Nummerierung bei~1 neu
anfangen. Benutzen Sie diese Option mit Vorsicht, da \biblatex keine einzigartigen
Labels allgemein garantieren kann, wenn sie manuell zurückgesetzt werden.  

\boolitem{omitnumbers}

Diese Option ist nur bei numerischen Zitierungen\slash Bibliografiestilen
anwendbar und verlangt, dass die \opt{defernumbers}-Option von
\secref{use:opt:pre:gen} allgemein aktiv ist. Wenn aktiv, wird \biblatex
ein numerisches Label nicht zu  dem Eintrag in der entsprechenden Bibliografie
zuweisen. Dies ist nützlich, wenn eine numerischen Bibliografie mit einer oder
mehreren Subbiliografien, die ein anderes Schema (z.\,B. Autor-Titel oder
Autor-Jahr) benutzen, vermischt wird.  

\end{optionlist*}

\begin{ltxsyntax}

\cmditem{bibbysection}[key=value, \dots]

Dieser Befehl führt automatisch Schleifen über alle Referenzensektionen aus.
Dies ist äquivalent zu dem Übergeben eines \cmd{printbibliography}-Befehls zu
jeder Sektion, hat aber den zusätzlichen Vorteil vom automatischen Überspringen
von Sektionen ohne Referenzen. Beachten Sie, dass \cmd{bibbysection} für
Referenzen in der Sektion \texttt{1}zu suchen  beginnt. Er wird Referenzen, die
außerhalb der \env{refsection}-Nachbarschaft angegeben werden, ignorieren, da
sie zur Section~0 zugeordnet sind. Für Anwendungsbeispiele sehen Sie
\secref{use:use:mlt}. Die Optionen sind eine Teilmenge von denen, die
\cmd{printbibliography} unterstützt. Gültige Optionen sind \opt{env},
\opt{heading}, \opt{prenote}, \opt{postnote}, \opt{maxnames}, \opt{minnames},
\opt{maxitems}, \opt{minitems}.   

\cmditem{bibbysegment}[key=value, \dots]

Dieser Befehl führt automatisch Schleifen über alle Referenzensegmente aus.
Dies ist äquivalent, zu dem Übergeben eines \cmd{printbibliography}-Befehls zu
jedem Segment, hat aber den zusätzlichen Vorteil vom automatischen Überspringen
von Segmenten ohne Referenzen. Beachte, dass \cmd{bibbysegment} für
Referenzen in dem Segment \texttt{1} beginnt zu suchen. Er wird Referenzen, die
außerhalb der  \env{refsegment} Nachbarschaft angegeben werden, ignorieren, da
sie zum Segment~0 zugeordnet sind. Für Anwendungsbeispiele sehen Sie
\secref{use:use:mlt}. Die Optionen sind eine Teilmenge von denen, die
\cmd{printbibliography} unterstützt. Gültige Optionen sind \opt{env},
\opt{heading}, \opt{prenote}, \opt{postnote}, \opt{maxnames}, \opt{minnames},
\opt{maxitems}, \opt{minitems} und \opt{section}. 

\cmditem{bibbycategory}[key=value, \dots]

Dieser Befehl führt automatisch Schleifen über alle Bibliografiekategorien
aus. Dies ist äquivalent zu dem Übergeben eines \cmd{printbibliography}-Befehls
zu jeder Kategorie, hat aber den zusätzlichen Vorteil vom automatischen
Überspringen von leeren Kategorien. Die Kategorien werden in der Reihenfolge
bearbeitet, in der sie deklariert wurden. Für Anwendnungsbeispiele sehen Sie
\secref{use:use:mlt}. Die Optionen sind eine Teilmenge von denen, die
\cmd{printbibliography} unterstützt. Gültige Optionen sind \opt{env},
\opt{prenote}, \opt{postnote}, \opt{maxnames}, \opt{minnames}, \opt{maxitems},
\opt{minitems} und \opt{section}. Beachten Sie, dass \opt{heading} bei diesen Befehl
nicht verfügbar ist. Der Name der aktuellen Kategorie wird automatisch als
Titelname benutzt. Dies ist äquivalent mit dem Übergeben von
\texttt{heading=\prm{category}} zu \cmd{printbibliography} und besagt, dass es
eine passende Titeldefinition für jede Kategorie geben muss (sehen Sie
\secref{use:bib:context}).   

\cmditem{printbibheading}[key=value, \dots]

Dieser Befehl gibt einen Bibliografietitel aus, der mit \cmd{defbibheading}
definiert wurde. Er erfordert eine optionales Argument, das eine Liste von
Optionen ist, die in \keyval-Notation angegeben werden. Die Optionen sind eine
kleine Teilmenge von denen, die \cmd{printbibliography} unterstützt. Gültige
Optionen sind \opt{heading} und \opt{title}. Automatisch benutzt dieser Befehl
den Titel \texttt{bibliography}. Für Details sehen Sie \secref{use:bib:hdg}, Für
Anwendungsbeispiele auch \secref{use:use:mlt,use:use:div}.  

Um eine Bibliografie mit einem anderen Sortierschema als dem globalen auszugeben,
wählen Sie die Bibliografie-Wechsel-Befehle aus \secref{use:bib:context}.

\end{ltxsyntax}

\subsubsection{Bibliografielisten}
\label{use:bib:biblist}

\biblatex kann zusätzlich zum Ausgeben von normalen Bibliografien auch beliebige Listen von Informationen ausdrucken. Diese sind  abgeleitet aus den Bibliografiedaten, wie von der
Abkürzungskürzel für bestimmte Einträge oder einer Liste der Abkürzungen von Zeitschriftentiteln.

Eine Bibliografieliste unterscheidet sich von einer normalen Bibliografie darin, dass
die gleichen Bibliografietreiber alle Einträge nehmen für die Ausgabe, anstatt einen speziellen Treiber für jede Eintragsableitung, je nach Art des Eintrags, zu verwenden.   

\begin{ltxsyntax}

\cmditem{printbiblist}[key=value, \dots]{<biblistname>}

Dieser Befehl gibt eine Bibliiografielsite aus. Er nimmt ein optionales Argument,
das aus einer Optionenliste in \keyval-Notation stammt. Gültige Optionen sind alle
Optionen, die von \cmd{printbibliography} unterstützt werden (\secref{use:bib:bib}),
außer \opt{prefixnumbers}, \opt{resetnumbers} und \opt{omitnumbers}. Wenn es
irgendwelche \env{refsection}-Umgebungen im Dokument sind, wird die Bibliografieliste
für diese Umgebungen lokal sein; sehen Sie \secref{use:bib:sec} für Details. 
Standardmäßig verwendet dieser Befehl die Überschrift \texttt{biblist}. Sehen
Sie \secref{use:bib:hdg} für Details.

Der \prm{biblistname} ist ein obligatorisches Argument, welches die Bibliografie
benennt. Dieser Name wird zum Identifizieren verwendet:
\begin{itemize}
\item Der Standardbibliografietreiber wird verwendet, um die Eintragsliste auszugeben. 
\item Ein Standardfilter wird mit \cmd{DeclareBiblistFilter} deklariert (sehen
Sie § 4.5.7 (e. V.), %\secref{aut:ctm:bibfilt}), 
um die ausgefilterten Einträge an \biber zu geben.
\item Eine Standardprüfung wird mit \cmd{defbibcheck} deklariert (sehen Sie %\secref{use:bib:flt}),
    § 3.7.9 (engl. V.),   um die Listeneinträge nachzubearbeiten. 
\item Die Standard-bib-Umgebung zu nehmen.
\item Das Standardsortierschema für Namen zu nehmen.
\end{itemize}

Die beiden zusätzlichen Optionen können einige der durch das obligatorische Argument festgelegten
Standardeinstellungen ändern.

\begin{optionlist*}
\valitem[\prm{biblistname}]{driver}{driver}

Ändert den Bibliografietreiber, der zum Drucken der Listeneinträge verwendet wird.

\valitem[\prm{biblistname}]{biblistfilter}{biblistfilter}

Ändert den Filter für die Bibliografieliste mit dem die Einträge gefiltert werden. \prm{biblistfilter} 
muss ein gültiger Bibliografielistenfilter sein, der mit \cmd{DeclareBiblistFilter} definiert ist (sehen  %\secref{aut:ctm:bibfilt})
    Sie~§ 4.5.7 (engl. Vers.).
\end{optionlist*}

In Bezug auf die Sortierung der Liste: Der Standard der Sortierung ist, das 
benannte Sortierungsschema nach der Bibliografieliste zu verwenden  (falls es existiert)
und erst dann, wenn es zurückfällt auf das aktuelle Kontextsortierschema, dies ist nicht definiert (sehen Sie %\secref{use:bib:context}).
    § 3.7.10 engl. V.).

Die häufigste Bibliografieliste ist eine Küzelabkürzungsliste für bestimmte Einträge 
und so hat diese einen Komfort mit \cmd{printshorthands[\dots]} für die Abwärtskompatibilität, die wie folgt definiert ist:

\begin{ltxexample}
\printbiblist[...]{shorthand}
\end{ltxexample}

\biblatex bietet eine automatische Unterstützung für Datenquellenfelder, die im
Standarddatenmodel markiert sind mit <Label fields> (Sehen Sie
\secref{bib:fld:dat}). Solche Felder sind automatisch definiert für Sie:

\begin{itemize}
\item Eine Standard-bib-Umgebung (sehen Sie \secref{use:bib:hdg}).
\item Ein Bibliografielistenfilter (sehen Sie § 4.5.7 (e. V.). %\secref{aut:ctm:bibfilt}).
\item Einige unterstützte Formate und Längen (sehen Sie §4.10.5 und § 4.10.4).
%\secref{aut:fmt:ilc} and \secref{aut:fmt:ich})
\end{itemize}
%
Daher ist nur eine minimales Setup nötig, um Listen mit solchen Feldern auszugeben.
Zum Beispiel, um eine Liste von Abkürzungen für Zeitschriften auszugeben, können Sie die einfach in ihre Präambel setzen: 

\begin{ltxexample}
\DeclareBibliographyDriver{shortjournal}{%
  \printfield{journaltitle}}
\end{ltxexample}
%
Sie können Sie auch in ihr Dokument setzen, wenn Sie diese Liste ausgeben möchten:

\begin{ltxexample}
\printbiblist[title={Journal Shorthands}]{shortjournal}
\end{ltxexample}
%
Weil \bibfield{shortjournal} in dem Standarddatenmodel als ein <Label field>
definiert ist, gibt es diese Beispiele:
\begin{itemize}
\item Das Nehmen der automatisch erzeugten <shortjournal>-bib-Umgebung.
\item Das Nehmen der automatisch erzeugten <shortjournal>-Bibliografie-Filterliste, um nur
Einträge mit einem  \bibfield{shortjournal}-Feld in die \file{.bbl} zurückzufehren.
\item Das Nehmen der definierten <shortjournal>-Bibliografietreiber, um Einträge auszugeben. 
\item Das Nehmen der Standard-<biblist>-Überschriftenheading, jedoch überschreibend die Titel mit <Journal Shorthands>
\item Das Nehmen des aktuellen Bibliografie-Kontextsortierschemas, wenn kein Schema mit dem namen  \bibfield{shortjournal} existiert.
\end{itemize}
%
Öfters werden Sie automatisch abgeholt, wenn Sie ein Labelfeld der Liste sortieren wollen und es nach der Liste benannt wird, in diesem Fall können Sie einfach Folgendes tun:

\begin{ltxexample}
\DeclareSortingScheme{shortjournal}{
  \sort{
        \field{shortjournal}
  }
}
\end{ltxexample}

Natürlich können alle Standards mit Optionen zu \cmd{printbiblist} überschrieben werden und auch Definitionen von Umgebungen, Filtern etc. Auf diese Weise können
beliebige Typen von Bibliografielisten ausgegeben werden, die eine Vielzahl von
Informationen der Bibliografiedaten enthalten.
\end{ltxsyntax}

Bibliografielisten werden oft zum Ausgeben von verschiedenen Arten von Kürzeln
verwendet und dies kann zu doppelten Einträgen führen, wenn mehr als ein Eintrag
das gleiche Kürzel hat. Beipspielsweise würden einige Zeitschriftenartikel in der
gleichen Zeitschrift in der Liste der Zeitschriftenkürzel zu doppelten Einträgen 
führen. Sie können die Tatsache nutzen, dass solche Listen ein automatisches Einlesen
eines \cmd{bibcheck} der gleichen Namen vornimmt, wie die Liste definiert einen Check definiert, um Duplikate zu entfernen. Wenn Sie eine Liste definieren um die Zeitschriftenkürzel auszugeben, nehmen Sie das \bibfield{shortjournal}-Feld,
Sie könnten einen \cmd{bibcheck} folgendermaßen definieren:

\begin{ltxexample}
\defbibcheck{shortjournal}{%
   \iffieldundef{shortjournal}
     {\skipentry}
     {\iffieldundef{journal}
       {\skipentry}
       {\ifcsdef{\strfield{shortjournal}=\strfield{journal}}
         {\skipentry}
         {\savefieldcs{journal}{\strfield{shortjournal}=\strfield{journal}}}}}}
\end{ltxexample}

\subsubsection{Bibliografiesektionen} \label{use:bib:sec}

Die \env{refsection}-Nachbarschaft wird im Dokument benutzt, um eine
Referenzsektion zu markieren. Diese Nachbarschaft ist nützlich, wenn man
separate, unabhängige Bibliografien und Listen von Kurzschriften in jedem
Kapitel, den Sektionen oder in jedem anderen Teil des Dokuments haben will. Innerhalb
einer Referenzsektion sind alle zitierte Arbeiten zugeteilte Labels, die lokal
zu der Nachbarschaft sind. Eigentlich sind Referenzsektionen komplett unabhängig
von den Dokumentteilen, wie \cmd{chapter} und \cmd{section}, obwohl sie am
ehesten per Kapitel oder Sektion benutzt werden. Sehen Sie die
\opt{refsection}-Paketoptionen in \secref{use:opt:pre:gen} als einen Weg an, diese 
zu automatisieren.
Für Anwendungsbeispiele sehen Sie auch \secref{use:use:mlt}.   

\begin{ltxsyntax}

\envitem{refsection}[resource, \dots]

Das optionale Argument ist eine Komma"=geteilte-Liste von Ressourcen
spezifisch für die Referenzsektion. Wenn das Argument fehlt, wird die 
Referenzsektion die
vorgegebenen Ressourcelisten benutzten, wie sie mit \cmd{addbibresource} in der
Präambel festgelegt werden. Wenn das Argument bereitgestellt ist, ersetzt es die
vorgegebene Ressourceliste. Allgemeine Ressourcen, die mit \cmd{addglobalbib}
festgelegt werden, werden immer betrachtet. \env{refsection}-Nachbarschaften
dürfen nicht verschachtelt sein, aber man kann \env{refsegment}-Nachbarschaften
innerhalb einer \env{refsection} benutzen, um sie in Segmente aufzuteilen.
Es benutzt die \opt{section}-Option von \cmd{printbibliography}, um eine
Sektion auszuwählen, wenn die Bibliografie ausgegeben wird und die dazugehörige Option
von \cmd{printshorthands}, wenn die Liste von Kürzeln ausgegeben wird.
Bibliografiesektionen sind nummeriert, beginnend bei~\texttt{1}. Die Nummer der
aktuellen Sektion wird auch in die Kopiedatei geschrieben. Alle Zitierungen, die
außerhalb einer \env{refsection}-Nachbarschaft angegeben werden, sind der
Sektion~0 zugeteilt. Wenn \cmd{printbibliography} innerhalb einer
\env{refsection} benutzt wird, wird sie automatisch die aktuelle Sektion
auswählen. Die \opt{section}-Option wird in diesen Fall nicht benötigt. Dies
gilt auch für \cmd{printbiblist}.      

\cmditem{newrefsection}[resource, \dots]

Dieser Befehl ist der \env{refsection}-Nachbarschaft ähnlich, außer dass er eher
ein selbstständiger Befehl ist als eine Nachbarschaft. Er beendet automatisch
die bisherige Referenzsektion (gegebenfalls) und beginnt sofort eine neue.
Beachten Sie, dass die Referenzsektion, die beim letzten \cmd{newrefsection}-Befehl
im Dokument startet, genau bis zum Ende des Dokuments erweitert wird.
Man benutzt \cmd{endrefsection}, wenn man es früher beenden will.  

\end{ltxsyntax}


\subsubsection{Bibliografiesegmente} \label{use:bib:seg}

Die \env{refsegment}-Nachbarschaft wird im Dokument benutzt, um ein
Referenzsegment zu markieren. Diese Nachbarschaft ist nützlich, wenn man eine
allgemeine Bibliografie will, die in Kapitel, Sektionen oder jeden anderen Teil
des Dokuments unterteilt wird. Eigentlich sind Referenzsegmente komplett
unabhängig von den Dokumentteilen, wie etwa \cmd{chapter} und \cmd{section},
obwohl sie speziell für Kapitel oder Sektion benutzt werden. Sehen Sie die
\opt{refsegment}-Paketoptionen in \secref{use:opt:pre:gen} als einen Weg an, diese zu
automatisieren. Für Anwendungsbeispiele sehen Sie auch \secref{use:use:mlt}.  

\begin{ltxsyntax}

\envitem{refsegment}

Der Unterschied zwischen einer \env{refsection}- und einer
\env{refsegment}-Nachbarschaft ist, dass die erstere Labels erzeugt, die lokal zu der
Nachbarschaft sind, während die letztere ein Ziel für den \opt{segment}-Filter
von \cmd{printbibliography} unterstützt, ohne dass die Labels betroffen sind. Sie
werden eindeutig über das ganze Dokument hinweg sein. \env{refsegment}-Nachbarschaften 
dürfen nicht verschachtelt sein, aber man kann sie in Verbindung
mit \env{refsection} benutzen, um eine Referenzsektion in Segmente zu
unterteilen. In diesem Fall sind die Segmente lokal zu der einschließenden
\env{refsection}-Nachbarschaft. Benutzen Sie die \env{refsegment}-Option von
\cmd{printbibliography}, um ein Segment auszuwählen, wenn die Bibliografie
ausgegeben wird. Die Referenzsegmente sind nummeriert, beginnend bei~\texttt{1}
und die Nummer vom aktuellen Segment wird in die Kopiedatei geschrieben. Alle
Zitierungen, die außerhalb einer \env{refsegment}-Nachbarschaft angegeben
werden, sind der Sektion~0 zugeteilt. Im Gegensatz zu der \env{refsection}-Nachbarschaft 
wird das aktuelle Segment nicht automatisch ausgewählt, wenn
\cmd{printbibliography} innerhalb einer \env{refsegment}-Nachbarschaft benutzt
wird.     

\csitem{newrefsegment}

Dieser Befehl ist der \env{refsegment}-Nachbarschaft ähnlich, außer dass er
eher ein selbstständiger Befehl ist als eine Nachbarschaft. Er beendet
automatisch das bisherige Referenzsegment (gegebenenfalls) und beginnt sofort ein
neues. Beachten Sie, dass das Referenzsegment, das beim letzten
\cmd{newrefsegment}-Befehl startet, bis zum Ende des Dokuments erweitert
wird. Benutzen Sie \cmd{endrefsegment},  wenn man es früher beenden will.  

\end{ltxsyntax}

\subsubsection{Bibliografiekategorien} \label{use:bib:cat}

Bibliografiekategorien ermöglichen es, die Bibliografie in mehrere Teile, die
verschiedenen Themen oder verschiedenen Typen von Referenzen gewidmet sind, zu
teilen, zum Beispiel in primäre und sekundäre Quellen. Für Anwendungsbeispiele
sehen Sie \secref{use:use:div}.

\begin{ltxsyntax}

\cmditem{DeclareBibliographyCategory}{category}

Deklariert eine neue \prm{category}, die im Zusammenhang mit \cmd{addtocategory}
und den \opt{category}- und \opt{notcategory}-Filtern von \cmd{printbibliography}
benutzt wird. Dieser Befehl wird in der Präambel des Dokuments benutzt. 
  
\cmditem{addtocategory}{category}{key}

Ordnet einen \prm{key} einer \prm{category} zu und wird in Verbindung mit den
\opt{category}- und \opt{notcategory}-Filtern von \cmd{printbibliography}
verwendet. Dieser Befehl kann in die Präambel und in das Dokumentgerüst
geschrieben werden. Der \prm{key} kann entweder ein einzelner
Eingabeschlüssel sein oder eine durch Kommata getrennte Liste von
Schlüsseln. Die Verknüpfung ist global.

\end{ltxsyntax}

\subsubsection{Bibliografieüberschriften und Umgebungen} \label{use:bib:hdg}

\begin{ltxsyntax}

\cmditem{defbibenvironment}{name}{begin code}{end code}{item code}

Dieser Befehl definiert Bibliografie-Umgebungen. Der \prm{name} ist ein
Identifikator, der an die \opt{env}-Option von \cmd{printbibliography} und
\cmd{printshorthands} bei der Auswahl der Umgebung weitergegeben wird. Der
\prm{begin code} ist \latex-Code, welcher zu Beginn der Umgebung ausgeführt
wird; der \prm{end code} wird zum Schluß der Umgebung ausgeführt; der
\prm{item code} ist Code, der zu Beginn jeder Eingabe in die Bibliografie oder
der Liste der Kurzschriften ausgeführt wird. Hier ein Beispiel einer
Definition, basierend auf der Standard-\env{list}-Umgebung von \latex:

\begin{ltxexample}
\defbibenvironment{bibliography}
  {\list{}
     {\setlength{\leftmargin}{\bibhang}%
      \setlength{\itemindent}{-\leftmargin}%
      \setlength{\itemsep}{\bibitemsep}%
      \setlength{\parsep}{\bibparsep}}}
  {\endlist}
  {\item}
\end{ltxexample}
%
Wie man in dem obigen Beispiel sehen kann, ist die Verwendung von \\
\cmd{defbibenvironment} annäherungsweise ähnlich zu \cmd{newenvironment},
abgesehen davon, dass es hier ein zusätzliches vorgeschriebenes Argument für
den \prm{item code} gibt.

\cmditem{defbibheading}{name}[title]{code}

Dieser Befehl definiert Bibliografieüberschriften. Der \prm{name} ist ein
Identifikator, der  an die \opt{heading}-Option von \cmd{printbibliography} oder
\cmd{printbibheading} und \cmd{printshorthands} bei der Auswahl des Kopfes
weitergegeben wird. Der \prm{code} sollte \latex-Code sein, der, wenn
gewünscht, eine vollständige Überschrift generiert, inklusive Seitenkopf und
einem Eintrag in das Inhaltsverzeichnis. Wenn \\
\cmd{printbibliography} oder
\cmd{printshorthands} sich auf eine \opt{title}-Option berufen, wird der Titel
an die Kopfdefinition als |#1| weitergegeben. Wenn nicht, wird anstelle dessen
der Standardtitel, der durch das optionale \prm{title}-Argument spezifiziert
wird, als |#1| weitergegeben. Typischerweise sind \cmd{bibname}, \cmd{refname}
oder \cmd{losname} die \prm{title}-Argument (siehe § 4.9.2.1 (e. V.).
%\secref{aut:lng:key:bhd}).
Hier ein Beispiel einer einfachen Kopfdefinition: 

\begin{ltxexample}
\defbibheading{bibliography}[\bibname]{%
  \chapter*{#1}%
  \markboth{#1}{#1}}
\end{ltxexample}

\end{ltxsyntax}

Die folgenden Überschriften, welche für die Verwendung mit
\cmd{printbibliography} und \cmd{printbibheading} vorgesehen sind, sind
vordefiniert:

\begin{valuelist*}

\item[bibliography] Dies ist der Standardkopf, der von \cmd{printbibliografie}
verwendet wird, wenn die \opt{heading}-Option nicht festgelegt ist. Seine
Standarddefinition hängt von der Klasse des Dokuments ab (document class). Wenn
die Klasse einen \cmd{chapter}-Befehl bereitstellt, ist der Kopf ähnlich dem
Bibliografiekopf der Standard-\latex-\texttt{book}-Klasse, das heißt, er
verwendet \cmd{chapter*}, um einen unnummerierten Kapitelkopf zu erzeugen,
welcher nicht in das Inhaltsverzeichnis eingebunden wird. Wenn es keinen
\cmd{chapter}-Befehl gibt, ist er ähnlich dem Bibliografiekopf der
Standard-\latex-\texttt{article}-Klasse, das heißt, er verwendet \cmd{section*}, um
einen unnummerierten Teilabschnittskopf zu erzeugen, welcher nicht in das
Inhaltsverzeichnis eingebunden wird. Die Zeichenfolge, welche im Kopf verwendet
wird, hängt von der Dokumentklasse ab. In \texttt{book}-ähnlichen Klassen wird
der Lokalisierungsstring \texttt{bibliography} verwendet, in anderen Klassen ist
das \texttt{references} (siehe § 4.9.2 (e. V.). %\secref{aut:lng:key}). 
Für klassenspezifische
Anleitungen siehe ebenfalls \secref{use:cav:scr, use:cav:mem}.

\item[subbibliography] Ist ähnlich wie \texttt{bibliography}, nur eine
Sektionierungsebene darunter. Diese Kopfdefinition verwendet \cmd{section*}
anstelle von \cmd{chapter*} in \texttt{book}-ähnlichen Klassen und ansonsten
\cmd{subsection*} anstelle von \cmd{section*}.

\item[bibintoc] Ähnlich wie \texttt{bibliography} oben, aber fügt einen
Eintrag in das Inhaltsverzeichnis ein.

\item[subbibintoc] Ähnlich wie \texttt{subbibliography} oben, aber fügt einen
Eintrag in das Inhaltsverzeichnis ein.

\item[bibnumbered] Ähnlich wie \texttt{bibliography} oben, aber verwendet
\cmd{chapter} oder \cmd{section}, um einen nummerierten Kopf zu erzeugen, der
ebenfalls in das Inhaltsverzeichnis eingefügt wird.

\item[subbibnumbered] Ähnlich wie \texttt{subbibliography} oben, aber verwendet
\cmd{section} oder \cmd{subsection}, um einen nummerierten Kopf zu erzeugen, der
ebenfalls in das Inhaltsverzeichnis eingefügt wird.

\item[none]
Eine leere Kopfdefinition. Zu nehmen, um den Kopf abzustellen.

\end{valuelist*}

Die folgenden Kopfzeilen, bestimmt zur Verwendung mit \cmd{printbiblist},
sind vordefiniert:

\begin{valuelist*}

\item[biblist]
Das ist die Standardüberschrift von \cmd{printbiblist}, wenn die
\opt{heading}-Option nicht gegeben wird.  Sie ist ähnlich mit \texttt{bibliography}, außer dass die Lokalisierungstringa \texttt{shorthands} anstelle von  \texttt{bibliography} oder \texttt{references} verwendet wird (sehen Sie § 4.9.2 (e. V.),
% \secref{aut:lng:key}),
auch \secref{use:cav:scr, use:cav:mem} für klassenspezifische Hinweise.

\item[biblistintoc]
Ähnlich mit \texttt{biblist} von oben, aber ein Eintrag ins Inhaltsverzeichnis wird erstellt.

\item[biblistnumbered]
Ähnlich mit \texttt{biblist} von oben, aber genommen, um für \cmd{chapter} oder \cmd{section} eine nummerierte Überschrift, die auch ins Inhaltsverzeichnis gefügt wird,
zu schaffen.

\end{valuelist*}


\subsubsection{Bibliografieanmerkungen} \label{use:bib:nts}

\begin{ltxsyntax}

\cmditem{defbibnote}{name}{text}

Definiert die Bibliografienotiz \prm{name}, die über die \opt{prenote}- und
\opt{postnote}-Optionen von \cmd{printbibliography} und \cmd{printshorthands}
verwendet wird. Der \prm{text} kann jedes beliebige Stück Text sein, der
möglicherweise auch mehrere Paragraphen umfasst und Schriftsatzdeklarationen
enthält. Sehen Sie ebenfalls \secref{use:cav:act}.

\end{ltxsyntax}

\subsubsection{Bibliografiefilter und -kontrollen} \label{use:bib:flt}

\begin{ltxsyntax}

\cmditem{defbibfilter}{name}{expression}

Definiert den üblichen Bibliografie-Filter \prm{name}, der über die
\opt{filter}-Option von \cmd{printbibliography} verwendet wird. Die
\prm{expression} ist ein komplexer Test, der auf den logischen Operatoren
\texttt{and}, \texttt{or}, \texttt{not}, dem Gruppenseparator \texttt{(...)},
und den folgenden Prüfoptionen besteht:

\end{ltxsyntax}

\begin{optionlist*}

\valitem{segment}{integer}

Gibt alle Einträge aus, die im Referenzsegment \prm{integer} eingetragen sind.

\valitem{type}{entrytype}

Gibt alle Einträge aus, deren Eingabeart eine bestimmte \prm{entrytype} ist.

\valitem{subtype}{subtype}

Gibt alle Einträge aus, deren \bibfield{entrysubtype} eine bestimmte
\prm{subtype} ist.

\valitem{keyword}{keyword}

Gibt alle Einträge aus, deren \bibfield{keywords}-Feld ein \prm{keyword}
beinhaltet. Wenn das \prm{keyword} eine Leerstelle enthält, muss es in Klammern
gesetzt werden.

\valitem{category}{category}

Gibt alle Einträge mit \cmd{addtocategory} aus, die einer \prm{category}
zugeordnet sind.

\end{optionlist*}

Hier ein Beispiel für einen Filter:

\begin{ltxexample}[style=latex,keywords={and,or,not,type,keyword}]{}
\defbibfilter{example}{%
  ( type=book or type=inbook )
  and keyword=abc
  and not keyword={x y z}
}
\end{ltxexample}
%

Dieser Filter wird alle Einträge ausgeben, deren Eingabeart entweder
\bibtype{book} oder \bibtype{inbook} ist und deren \bibfield{keywords}-Feld das
Schlüsselwort <\texttt{abc}>, jedoch nicht <\texttt{x y z}> beinhaltet. Wie man
im obigen Beispiel sehen kann, werden alle Elemente durch Leerstellen
voneinander getrennt (Leerzeichen, Tabulatoren oder Zeilenenden). Es gibt keinen
Zeichenabstand um gleichwertige Symbole herum. Die logischen Operatoren werden
mit dem \cmd{ifboolexpr}-Befehl aus dem \sty{etoolbox}-Paket ausgewertet. Sehen Sie in die
\sty{etoolbox}-Anleitung für Einzelheiten bezüglich der Syntax. Die Syntax des
\cmd{ifthenelse}-Befehls aus dem \sty{ifthen}-Paket, welches in den älteren
Versionen von \sty{biblatex} Verwendung fand, wird nach wie vor unterstützt. Dies
ist der selbe Test unter Verwendung von \sty{ifthen}-ähnlicher Syntax:

\begin{ltxexample}[style=ifthen,morekeywords={\\type,\\keyword}]{}
\defbibfilter{example}{%
  \( \type{book} \or \type{inbook} \)
  \and \keyword{abc}
  \and \not \keyword{x y z}
}
\end{ltxexample}
%

Man beachte, dass normale Filter lokal bezüglich des Referenzabschnitts sind,
in welchem sie verwendet werden. Um einen anderen Abschnitt auszuwählen,
benutzt man den \texttt{section}-Filter von \cmd{printbibliography}. Das ist in
einem normalen Filter nicht möglich.

\begin{ltxsyntax}

\cmditem{defbibcheck}{name}{code}

Definiert den gebräuchlichen Bibliografiefilter \prm{name}, der über die
\opt{check}-Option von \cmd{printbibliography} verwendet wird. \cmd{defbibcheck}
ist vom Prinzip her ähnlich wie \cmd{defbibfilter}, nur auf einer, um einiges
niedrigeren Ebene. Anders als ein Ausdruck auf hoher Ebene, ist der \prm{code}
\latex-Code, ähnlich dem Code, der in Treiberdefinitionen verwendet wird,
welcher beliebige Tests durchführen kann, um zu entscheiden, ob ein Eintrag
gedruckt werden soll oder nicht. Die Bibliografiedaten des betreffenden
Eintrages sind vorhanden, wenn der \prm{code} ausgeführt wird. Das Verwenden
des Befehls \cmd{skipentry} führt dazu, dass der momentane Eintrag
übersprungen wird. Zum Beispiel: Der folgende Filter wird nur Einträge mit
einem \bibfield{abstract}-Feld ausgeben:

\begin{ltxexample}
\defbibcheck{<<abstract>>}{%
  \iffieldundef{abstract}{<<\skipentry>>}{}}
...
\printbibliography[<<check=abstract>>]
\end{ltxexample}
%

Der folgende Test wird alle Einträge beinhalten, die vor dem Jahr 2000
veröffentlicht wurden:

\begin{ltxexample}
\defbibcheck{recent}{%
  \iffieldint{year}
    {\ifnumless{\thefield{year}}{2000}
       {\skipentry}
       {}}
    {\skipentry}}
\end{ltxexample}
%

Siehe in das Autorenhandbuch, insbesondere §§ 4.6.2 und 4.6.3 (e. V.)
%\secref{aut:aux:tst,aut:aux:ife}, für
weitere Details.

\end{ltxsyntax}

\subsubsection{Referenzkontexte}
\label{use:bib:context}

Referenzen in einer Bibliografie sind zitiert und ausgegeben in einem <Kontext>.
Der Kontext bestimmt die Daten, die tatsächlich verwendet werden beim Zitieren
oder bibliografische Daten, die für einen Eintrag zur Verfügung stehen. Ein
Kontext besteht aus den folgenden Informationen (Das <Kontext>-Konzept ist als zukünftig
erweiterbar entworfen):

\begin{itemize}
 \item Ein Sortierschema,
 \item ein Schema für das Erstellen von Sortierschlüsseln für Namen,
 \item ein string-Präfix-Schema, das alphabetische oder numerische Labels verwendet. 
\end{itemize}
%
Der Bibliografiekontext-Punkt ist doppelt. Zum einen werden Kontexte verwendet,
um Optionen zu setzen, die eine ausgegebene Bibliografie beeinflussen. Zum anderen,
um die Datenausgabe durch Zitierbefehle zu beeinflussen. 
Die frühe Nutzung ist die häufigste, wenn man mehr als eine Bibliografieliste mit Unterschieden ausgeben muss,um Beispiel beim Sortieren. 

\begin{ltxexample}
\usepackage[sorting=nyt]{biblatex}
\begin{document}
\cite{one}
\cite{two}
\printbibliography
\newrefcontext[sorting=ydnt]
\printbibliography
\end{ltxexample}
%
Hier geben wir zwei Bibliografien aus, eine mit dem Standardsortierschema <nyt>
und eine mit dem <ydnt>-Sortierschema.

Um die zweite Art der Nutzung des Bibliografiekontextes zu demonstrieren,
müssen wir verstehen, dass die tatsächlichen Daten für einen Eintrag je nach 
Kontext variieren.
Am deutlichsten wird dies im Falle der \opt{extra*}-Felder wie \opt{extrayear},
die vom Backend entsprechend der Reihenfolge der Einträge \emph{nach} der 
Sortierung generiert werden, so dass sie in der erwarteten Folge <a, b, c> 
erscheinen. Dies zeigt deutlich, dass die \emph{Daten} eines Eintrags unterschiedlich
zum Sortierschema sein können. Wenn ein Dokument mehr als eine Bibliografieliste mit verschiedenen Sortiersystemen enthält, kann es passieren, dass die \file{.bbl}
Listen mit der gleichen Eintragssortierung aber unterschiedlichen Daten enthält
(einen anderen Wert für \bibfield{extrayear}, beispielsweise). Der Zweck der Bibliografiekontexte ist, die Dinge in einem Kontext zu kapseln, so dass 
\biblatex die korrekten Eintragsdaten verwenden kann. Ein Beipiel ist die Ausgabe
einer Bibliografieliste mit einer anderen Sortierreihenfolge für die globale
Sortierreihenfolge,  wo die \opt{extra*}-Felder unterschiedlich sind zwischen Sortierlisten mit dem gleichen Eintrag:

\begin{ltxexample}
\usepackage[sorting=nyt,style=authoryear]{biblatex}
\DeclareSortingScheme{yntd}{
  \sort{
    \field[strside=left,strwidth=4]{sortyear}
    \field[strside=left,strwidth=4]{year}
    \literal{9999}
  }
  \sort{
    \field{sortname}
    \field{author}
    \field{editor}
  }
  \sort[direction=descending]{
    \field{sorttitle}
    \field{title}
  }
}
\begin{document}
\cite{one}
\cite{two}
\printbibliography
\newrefcontext[sorting=yntd]
\cite{one}
\cite{two}
\printbibliography
\end{ltxexample}
%
Hier nun, die zweite Verwendung der Zitate, zusammen mit dem 
\cmd{printbibliography}-Befehl werden Daten aus dem Kontext des benutzerdefinierten
Sortierschema <yntd>  verwendet, das auch unterschiedlich zu den Daten im Zusammenhang 
mit dem Standardschema  <nyt>. Das heißt, die Zitierlabels (in einem authoryear-Stil,
der \opt{extrayear} nimmt) können Unterschiede \emph{für die genau gleichen Einträge}
zwischen verschiedenen Bibliografiekontexten sein und so können die Zitate selbst 
anders aussehen.

Standardmäßig werden Daten für eine Zitierung vom Referenzkontext der letzten Bibliografie,
in welcher sie ausgegeben wurden, geholt. Beispielsweise in das Fragment:

\begin{ltxexample}[style=latex]{}
\begin{document}

\cite{book, article, misc}

\printbibliography[type=book]

\newrefcontext[labelprefix=A]
\printbibliography[type=article]

\newrefcontext[sorting=ydnt, resetnumbers]
\printbibliography[type=misc]

\end{document}
\end{ltxexample}
%
Dieses Beispiel zeigt auch die Deklarierung und das Benutzen von benannten Referenzkontexten. Unter der Annahme, dass ihre Eingabetasten eingerichtete sind auf ihre Eintragstypen,
ist dies die Standardsituation fürs Zitate, die normalerweise den Erwartungen der Nutzer entsprechen:

\begin{itemize}
\item Das Zitat des Eintrags \bibfield{book} würde seine Daten aus dem globalen
Referenzkontext entnehmen, weil die letzte Bibliografie, in der es ausgegeben wurde,
der globale Referenzkontext war.
\item  Das Zitat des Eintrags \bibfield{article} würde seine Daten aus dem
Referenzkontext mit dem \opt{labelprefix=A} entnehmen und würde daher ein <A> Präfix haben, wenn es zitiert wird.
\item  Das Zitat des Eintrags \bibfield{misc}  würde seine Daten aus dem
Referenzkontext mit \opt{sorting=ydnt}.
\end{itemize}
%
In Fällen, in denen die Benutzer Einträge haben, die in mehreren Bibliografien in verschiedenen Formen oder mit potentiell unterschiedlichen Labels vorkommen (in einem numerischen
Schema mit verschiedenen \bibfield{labelprefix} Werten beispielsweise), kann es notwendig 
sein, \biblatex zu sagen, aus welchem Kontext man die Zitainformatioenen entnehmen möchte.
Wie oben gezeigt, kann dies durch Zitate, die explizit innerhalb der Referenzkontexte 
gesetzt wurden.  Dies kann in einem großen Dokument belastend sein und so gibt es 
eine spezifische Funktionalität für die programmatische Zuordnung der Zitate zu Referenzkontexten,
sehen Sie zu den \cmd{assignrefcontext*}-Makros unten. 

\begin{ltxsyntax}

%\envitem{refcontext}[key=value, \dots]
\cmditem{DeclareRefcontext}{name}{key=value, \dots}

Packt eine Bibliografiekontextumgebung ein. Die Optionen definieren die Kontextattribute. 
Alle Kontextattribute sind optional und standardmäßig bei den globalen Einstellungen 
nicht vorhanden. Die aktuelle Optionen sind:

\begin{optionlist*}

\valitem{sorting}{name}

Spezifizierung eines Sortierschemas, das zuvor mit \cmd{DeclareSortingScheme} definiert 
wurde. Dieses Schema wird genommen, um Daten für einen Eintrag in den Befehlen innerhalb des Kontexts zu bestimmen, abzurufen und/oder auszugeben.   

%\valitem{sortingnamekeyscheme}{name}
\valitem{sortingnamekeytemplatename}{name}

Spezifizierung einer Schlüsselvorlage für den Sortiernamen. die zuvor mit \cmd{DeclareSortingNamekeyTemplate} definiert wurde. Diese Vorlage wird verwendet, um Sortierschlüssel für Namen im Kontext zu erstellen. Der Vorlagenname kann auch (in aufsteigender Reihenfolge der Präferenz) pro"=Eintrag, pro"=Namenliste und per"=Name angegebn werden. Sehen Sie %\secref{apx:opt} 
    § E (engl. V.) für Information zum Setzen pro"=Option, pro"=Namensliste und pro"=Namensoptionen.

\valitem{uniquenametemplatename}{name}

    Spezifizierung einer Namen-Schlüsselschema-Sortierung, die zuvor mit \cmd{DeclareSortingNamekeyScheme} definiert wurde. Dieses Schema wird verwendet, Sortierungsschlüssel innerhalb des Kontextes zu konstruieren. Der Vorlagenname kann auch (in aufsteigender Reihenfolge der Präferenz) pro"=Eintrag, pro"=Namensliste und pro"=Name angegeben werden. Sehen Sie § E (eng. V.)
%\secref{apx:opt} 
für Information zum Setzen pro"=Option, pro"=Namensliste und pro"=Namensoptionen.


\valitem{labelalphanametemplatename}{name}

Spezifiziert eine Vorlage, die zuvor mit \cmd{DeclareLabelalphaNameTemplate} definiert
    wurde (sehen Sie %\secref{aut:ctm:lab})
§ 4.5.5 (engl. V.). Diese Vorlage wird verwendet, um Namensteile von alphabetischen Bezeichnungen für Namen innerhalb des Kontexts zu erstellen. Der Vorlagenname kann auch (in aufsteigender Reihenfolge der Präferenz) pro"= Eintrag, pro"= Namensliste und 
pro"= Name angegeben werden. Informationen zum Festlegen pro"= Option, pro"= Namenliste finden Sie unter % \secref {apx: opt}.
§ E.

\valitem{nametemplates}{name}

Ein praktische Metaoption, die \opt{sortingnamekeytemplate}, \opt{uniquenametemplate} und \opt{labelalphanametemplate} auf denselben Templatenamen seten kann. Diese Option kann
auch (in aufsteigender Reihenfolge) per"=entry, per"=name list und per"=name angegeben werden. Sehen sie  %\secref{apx:opt} 
§ E für Informationen zum Festlegen per"=option, per"=namelist and per"=name Optionen.

\valitem{labelprefix}{string}

Diese Option gilt nur für numerische Zitierungen\slash Bibliografiestile und erfordert, 
dass die \opt{defernumbers}-Option aus \secref{use:opt:pre:gen} global aktiviert wird.
Das Setzen dieser Option wird implicit  \opt{resetnumbers} für jede  
\cmd{printbibliography} im Skopus des Kontexts aktiviert. Die Option ordnet den \prm{string}
als ein Präfix zu allen Einträgen im Referenzkontext. Beispielsweise, wenn der \prm{string}  \texttt{A} ist, die numerischen Labels werden ausgeben werden als \texttt{[A1]}, \texttt{[A2]}, \texttt{[A3]}, etc. Dies ist nützlich, für unterteilte numerische Bibliografien, bei denen jede Subbibliografie ein anderes Präfix verwendet. Der \prm{string} ist für alle Stile im
\bibfield{labelprefix}-Feld aller betroffenen Einträge vorhanden.  Sehen Sie
%\secref{aut:bbx:fld:lab} 
    § 4.2.4.2 (engl. V.)    für weitere Details.

\end{optionlist*}
%
\envitem{refcontext}[key=value, \dots]{name}

Umschließt eine Referenzkontext-Umgebung. Die möglichen optionalen Argumente
\prm{key=value} sind für sowas wie \cmd{DeclareRefcontext} und überschreiben
Optionen, die für den benannten Refernzkontext \prm{name} angegeben wurden.
\prm{name} kann auch als \verb+{}+ oder durch das Weglassen der leeren Klammern weggelassen werden \footnote{Diese etwas merkwürdige Syntaxvariante ist das Ergebnis der Abwärtskompatibilität mit \biblatex $<$3.5}.

Die \opt{refcontext}-Umgebung kann nicht verschachtelt werden und \biblatex wird einen Fehler generieren, wenn Sie versuchen, dies zu tun.

\cmditem{newrefcontext}[key=value, \dots]

Dieser Befehl ist ähnlich zur \env{refcontext}-Umgebung, außer dass es ein alleinstehender Befehl ist und keine Umgebung. Er beendet automatisch den vorherigen Kontextabschnitt 
(falls vorhanden) und beginnt sofort eine neue. Beachten Sie, dass der Kontextabschnitt mit
dem letzten \cmd{newrefcontext}-Befehl gestartet wird, er wird ihn bis zum Dokumentende verlängern. 
\end{ltxsyntax}

%
Zu Beginn des Dokuments gibt es immer einen globalen Kontext, der globale Einstellungen
für die einzelnen Kontextattribute enthält. Jetzt ein zusammenfassendes Beispielfür Referenzkontexte mit verschiedenen Settings.

\begin{ltxexample}[style=latex]{}
\usepackage[sorting=nty]{biblatex}

\DeclareRefcontext{testrc}{sorting=nyt}

% Global reference context:
%   sorting=nty
%   sortingnamekeyscheme=global
%   labelprefix=

\begin{document}

\begin{refcontext}{testrc}
% reference context:
%   sorting=nyt
%   sortingnamekeyscheme=global
%   labelprefix=
\end{refcontext}

\begin{refcontext}[labelprefix=A]{testrc}
% reference context:
%   sorting=nyt
%   sortingnamekeyscheme=global
%   labelprefix=A
\end{refcontext}

\begin{refcontext}[sorting=ydnt,labelprefix=A]
% reference context:
%   sorting=ydnt
%   sortingnamekeyscheme=global
%   labelprefix=A
\end{refcontext}

\newrefcontext}[labelprefix=B]
% reference context:
%   sorting=nty
%   sortingnamekeyscheme=global
%   labelprefix=B
\endrefcontext

\newrefcontext}[sorting=ynt,labelprefix=C]{testrc}
% reference context:
%   sorting=ynt
%   sortingnamekeyscheme=global
%   labelprefix=C
\endrefcontext
\end{ltxexample}

\begin{ltxsyntax}

\cmditem{assignrefcontextkeyws}[key=value, \dots]{keyword1,keyword2, ...}
\cmditem{assignrefcontextkeyws*}[key=value, \dots]{keyword1,keyword2, ...}
\cmditem{assignrefcontextcats}[key=value, \dots]{category1, category2, ...}
\cmditem{assignrefcontextcats*}[key=value, \dots]{category1, category2, ...}
\cmditem{assignrefcontextentries}[key=value, \dots]{entrykey1, entrykey2, ...}
\cmditem{assignrefcontextentries*}[key=value, \dots]{entrykey1, entrykey2, ...}
\cmditem{assignrefcontextentries}[key=value, \dots]{*}
\cmditem{assignrefcontextentries*}[key=value, \dots]{*}

\end{ltxsyntax}

Diese Befehle setzen Zitate automatisch in die Refkontexte. Anstatt Zitierungsbefehlen in  \env{refcontext}-Umgebung einzusetzen, die fehleranfällig sein könnten und langweilig,
können Sie eine Komma-separierte Liste von  \prm{keywords}, \prm{categories} oder
\prm{entrykeys} registrieren, die erzeugen Einträge mit einem der angegebenen Schlüsselwörter,
Einträge in einer der in \secref{use:use:div} angegebenen Kategorien oder Einträge mit einer
der spezifizierten Zitiertasten in einem bestimmten Refkontext durch die angegebenen  \prm{refcontext key/values}, die genau analisiert werden als \env{refcontext}-Optionen.
Solche automatische refcontext-Zuordnungen sind spezifisch zur daktuellen Refsection.
Dies ist nützlich, wenn Sie mehrere Bibliografien haben, anders sortierend als bei der 
Benutzung von \opt{defernumbers}, dann muss \biblatex wissen, welche Bibliografie 
(die immer in einem bestimmt Refkontext ist) die Zitierung erzeugen soll, da die Zitate
auch unterschiedliche Nummern in verschiedenen Bibliografien haben können.
Sie können die gleichen Zitatschlüssel in diesen Befehlen angeben, sollten sich aber bewußt sein, das die Zuordnung in der Reihenfolge  \prm{keywords}, \prm{categories}, \prm{entrykeys} später Spezifikationen von früheren überschreiben wird. Ein Beispiel:

\begin{ltxexample}[style=latex]{}
\assignrefcontextentries[labelprefix=A]{key2}
\cite{key1}
\begin{refcontext}[labelprefix=B]
\cite{key2}
\end{refcontext}
\end{ltxexample}
%
Hier werden die Daten für die Zitierung \bibfield{key2} entnommen vom Refkontext \opt{labelprefix=A} und nicht von \opt{labelprefix=B} (was zu einem Label mit dem Präfix 
<A> und nicht <B> führt).
Die gesternte Version überschreibt nicht einen lokalen Refkontext und somit:

\begin{ltxexample}[style=latex]{}
\assignrefcontextentries*[labelprefix=A]{key2}
\cite{key1}
\begin{refcontext}[labelprefix=B]
\cite{key2}
\end{refcontext}
\end{ltxexample}
%
Die Daten für die Zitierung \bibfield{key2} werden aus dem Refkontext \opt{labelprefix=B}
entnommen. Beachten Sie, dass diese Befehle nur selten benötigt werden, dann wenn Sie
mehrere Bibliografien haben, in denen die gleichen Zitate sein können und  \biblatex\ 
standardmäßig nicht sagen kann, auf welche Bibliografie sich ein Zitat bezieht.
Sehen Sie die Beispieldatei \file{94-labelprefix.tex} für weitere Details an.


\subsubsection{Dynamische Eintragssätze} \label{use:bib:set}

Zusätzlich zu dem \bibtype{set}-Eingabetyp unterstützt \biblatex auch
dynamische Mengeneinträge, die auf der Basis von "`ref-document\slash
per-refsection"' definiert werden. Die folgenden Befehle definieren die Menge
\prm{key} und können in der Präambel oder im Dokument verwendet werden:

\begin{ltxsyntax}

\cmditem{defbibentryset}{key}{key1,key2,key3, \dots}

Der \prm{key} ist der Eingabeschlüssel der Menge (des "`sets"'), welcher wie jeder
andere Eingabeschlüssel in Bezug auf die Menge verwendet wird. Der \prm{key}
muss einzigartig sein und er darf nicht in Konflikt mit anderen
Eingabeschlüsseln geraten. Das zweite Argument ist eine durch Kommata getrennte
Liste von Eingabeschlüsseln, welche die Menge bilden. \cmd{defbibentryset}
beinhaltet ein Äquivalent eines \cmd{nocite}-Befehls, das heißt, alle Mengen,
welche deklariert wurden, werden auch in das Inhaltsverzeichnis geschrieben.
Wenn eine Menge mehr als einmal deklariert wurde, wird nur der erste Aufruf von
\cmd{defbibentryset} die Menge bestimmen. Darauffolgende Definitionen des selben
\prm{key} werden ignoriert und wie \cmd{nocite}\prm{key} behandelt. Dynamische
Mengeneinträge werden im Dokumentgerüst definiert und sind lokal bezüglich
der umschliessenden \env{refsection}-Umgebung, wenn vorhanden. Andernfalls
werden sie der Sektion~0 des Literaturverzeichnisses zugeordnet.
Diejenigen, die in
der Präambel bestimmt wurden, ebenfalls. 
Sehen Sie \secref{use:use:set}  für weitere Details.

\end{ltxsyntax}

\subsection{Zitierbefehle} \label{use:cit}

Alle Befehle für Literaturangaben beinhalten generell ein zwingendes und zwei
optionale Argumente. Die \prm{prenote} ist Text, der zu Beginn der
Literaturstelle ausgegeben wird. Dies sind in der Regel Notizen wie <siehe> oder
<vergleiche>. Die \prm{postnote} ist Text, der zum Schluss der Literaturstelle
ausgegeben wird. Das ist normalerweise eine Seitenzahl. Wenn man eine spezielle
Vorbemerkung, aber keine Nachbemerkung machen möchte, muss man das zweite
optionale Argument leer lassen, so wie bei |\cite[see][]{key}|. Das
\prm{key}-Argument für alle Befehle von Literaturstellen ist Pflicht. Dies ist
ein Eingabeschlüssel oder eine durch Kommata getrennte Liste von Schlüsseln
gemäss den Eingabeschlüsseln in der \sty{bib}-Datei. Zusammengefasst haben
alle unten aufgeführten grundlegenden Literaturbefehle im Wesentlichen folgende
Syntax:

\begin{ltxsyntax}

\cmditem*{command}[prenote][postnote]{keys}<punctuation>

Wenn die \opt{autopunct}-Paket Option aus \secref{use:opt:pre:gen} freigegeben
ist, wird sie nach jeder \prm{punctuation} suchen und sofort ihrem letzten
Argument folgen. Das ist nützlich, um störende Satzzeichen nach der
Literaturstelle zu vermeiden. Diese Dokumentation wurde mit
\cmd{DeclareAutoPunctuation} konfiguriert, siehe § 4.7.5 (e. V.)
%\secref{aut:pct:cfg} 
für Details.

\end{ltxsyntax}

\subsubsection{Standardbefehle} \label{use:cit:std}

Die folgenden Befehle werden durch den Stil für Literaturverweise definiert.
Solche Stile können jede beliebige Anzahl spezieller Befehle enthalten, aber
dies hier sind die Standardbefehle, die typischerweise Stile für die allgemeine
Verwendung bereitstellen.

\begin{ltxsyntax}

\cmditem{cite}[prenote][postnote]{key} 
\cmditem{Cite}[prenote][postnote]{key}

Dies sind die einfachen Befehle für Literaturangaben. Sie geben die
Literaturstelle ohne jeden Zusatz aus, wie etwa Einschübe. Die numerischen und
alphabetischen Formen schließen das Label in eckige Klammern ein, da die
Referenz andererseits doppeldeutig sein kann. \cmd{Cite} ist genauso wie
\cmd{cite}, schreibt aber das Präfix des ersten Namens des Literaturverweises
in Großbuchstaben, wenn die \opt{useprefix}-Option vorhanden ist,
vorausgesetzt,
es gibt ein Namenspräfix und die Art der Literaturangabe gibt überhaupt alle
Namen aus.

\cmditem{parencite}[prenote][postnote]{key}
\cmditem{Parencite}[prenote][postnote]{key}

Diese Befehle verwenden ein Format, welches \cmd{cite} gleicht, aber die
Literaturangabe komplett in runden Klammern  ausgibt. Die numerischen und
alphabetischen Stile verwenden stattdessen eckige Klammern. \cmd{Parencite} ist
wie \cmd{parencite}, schreibt aber das Präfix des ersten Namens des
Literaturverweises in Großbuchstaben, wenn die \opt{useprefix}-Option vorhanden
ist, vorausgesetzt es gibt ein Namenspräfix.

\cmditem{footcite}[prenote][postnote]{key}
\cmditem{footcitetext}[prenote][postnote]{key}

Dieser Befehl verwendet ein Format, welches ähnlich zu \cmd{cite} ist, packt
aber die komplette Quellenangabe in eine Fußnote und setzt am Ende einen Punkt.
In der Fußnote wird das Präfix des Vornamen automatisch in Großbuchstaben
geschrieben, wenn die \opt{useprefix}-Option vorhanden ist, vorausgesetzt es gibt
ein Namenspräfix und die Art der Literaturangabe gibt überhaupt alle Namen
aus. \cmd{footcitetext} unterscheidet sich von \cmd{footcite} dadurch, dass es
\cmd{footnotetext} anstelle von \cmd{footnote} verwendet.

\end{ltxsyntax}

\subsubsection{Stilspezifische Befehle} \label{use:cit:cbx}

Die folgenden zusätzlichen Befehle für Literaturangaben werden nur von einigen
Arten von Quellenangaben bereitgestellt, welche mit diesem Paket geladen werden.

\begin{ltxsyntax}

\cmditem{textcite}[prenote][postnote]{key}
\cmditem{Textcite}[prenote][postnote]{key}

Diese Zitierbefehle werden von allen Stildateien zur Verfügung gestellt, welche von diesem Paket geladen werden. Sie sind 
für die Verwendung im Textfluss vorgesehen, um das Subjekt des Satzes auszutauschen und
geben den Autor oder Editor aus, gefolgt von den in runden Klammern stehenden
anderen Kennzeichen der Literaturangabe. Abhängig von der Art der
Literaturangabe kann das Kennzeichen eine Nummer, das Erscheinungsjahr, eine
gekürzte Version des Titels oder Ähnliches sein. Die numerischen und alphabetischen
Stile verwenden stattdessen eckige Klammern. In den verbose-Stilen wird das Label mit einer Fußnote versehen. Nachgestellte Zeichensetzung wird zwischen den Autor- oder Herausgebernamen und die Fußnotenmarke bewegt. \cmd{Textcite} ist ähnlich wie
\cmd{textcite}, schreibt aber das Präfix des ersten Namens des
Literaturverweises in Großbuchstaben, wenn die \opt{useprefix}-Option vorhanden
ist, vorausgesetzt es gibt ein Namenspräfix.

\cmditem{smartcite}[prenote][postnote]{key}
\cmditem{Smartcite}[prenote][postnote]{key}

Wie \cmd{parencite} in einer Fußnote und wie \cmd{footcite} im Dokument.

\cmditem{cite*}[prenote][postnote]{key} 

Dieser Befehl wird von allen Autor-Jahr-
und Autor-Titel-Stilen bereitgestellt. Er ist ähnlich wie der normale
\cmd{cite}-Befehl, gibt aber lediglich jeweils Jahr oder Titel aus. 

\cmditem{parencite*}[prenote][postnote]{key}

Dieser Befehl wird von allen Autor-Jahr und Autor-Titel Stilen bereitgestellt.
Er ist ähnlich wie der normale \cmd{parencite}-Befehl, gibt aber lediglich
jeweils Jahr oder Titel aus. 

\cmditem{supercite}{key}

Dieser Befehl, welcher nur von den Numerik-Stilen unterstützt wird, gibt
numerische Literaturangaben als Indizes ohne Klammern aus. Er verwendet
\cmd{supercitedelim} anstelle von \cmd{multicitedelim} als Abgrenzung der
Angaben. Man beachte, dass jedes \prm{prenote}- und jedes
\prm{postnote}-Argument ignoriert wird. Sollten sie gegeben sein, wird
\cmd{supercite} sie streichen und eine Warnmeldung ausgeben. 

\end{ltxsyntax}

\subsubsection{Qualifizierte Listen von Literaturangaben} \label{use:cit:mlt}

Dieses Paket unterstützt eine Klasse von speziellen Befehlen für
Literaturangaben, genannt <multicite>-Befehle. Das Besondere daran ist, dass
ihre Argumente aus einer kompletten Liste von Literaturangaben bestehen, in
denen
jeder Eintrag einen vollständigen Literaturnachweis mit Vor- und\slash oder
Schlussbemerkung darstellt. Dies ist besonders nützlich, bei eingeschobenen
Literaturangaben und solchen, die in Fußnoten geschrieben werden. Es ist
ebenfalls möglich, eine Vor-\slash oder Schlussbemerkung in die vollständige
Liste einzufügen. Diese Befehle für multiple Literaturangaben werden zu Beginn
von hinten anstehenden Befehlen, so wie \cmd{parencite} und \cmd{footcite}
eingebaut. Dieser Zitierstil unterstützt eine genaue Bestimmung für multiple
Literaturangaben mit \cmd{DeclareMultiCiteCommand} (siehe § 4.3.1 (e. V.)).
%\secref{aut:cbx:cbx}).
Das folgende Beispiel illustriert die Syntax solcher  <multicite>-Befehle:

\begin{ltxexample}
\parencites[35]{key1}[88--120]{key2}[23]{key3}
\end{ltxexample}
%
Das Format der Argumente ist ähnlich wie die regulären Befehle für
Literaturangaben, nur dass lediglich ein Zitierbefehl gegeben ist. Wenn nur ein
optionales Argument für ein Argument der Liste vorhanden ist, wird es als
Schlussbemerkung ausgegeben. Möchte man eine spezifisch Vor-, jedoch keine
Schlussbemerkung, muss man das zweite optionale Argument der betreffenden
Beschreibung offen/leer lassen:

\begin{ltxexample}
\parencites[35]{key1}[chapter 2 in][]{key2}[23]{key3}
\end{ltxexample}
%
Zusätzlich dazu kann die komplette Liste auch eine Vor- und\slash oder eine
Schlussbemerkung enthalten. Die Syntax dieser globalen Bemerkungen unterscheidet
sich von anderen optionalen Argumenten darin, dass sie eher in geschweiften denn
in üblichen Klammern gesetzt wird:

\begin{ltxexample}
\parencites<<(>>and chapter 3<<)>>[35]{key1}[78]{key2}[23]{key3}
\parencites<<(>>Compare<<)()>>[35]{key1}[78]{key2}[23]{key3}
\parencites<<(>>See<<)(>>and the introduction<<)>>[35]{key1}[78]{key2}[23]{key3}
\end{ltxexample}
%
Man beachte, dass die Befehle für multiple Literaturangaben so lange nach
Argumenten suchen, bis sie auf ein Merkmal stoßen, welches nicht der Beginn
eines optionalen oder Pflichtarguments ist. Wenn einer linken geschweiften oder
eckigen Klammer ein Befehl für eine multiple Literaturangabe folgt, muss man
diese durch das Einfügen von \cmd{relax} oder einem Backslash, gefolgt von ein
Leerzeichen nach dem letzten gültigen Argument verschleiern. Das erzwingt das
Ende des Suchvorgangs. 

\begin{ltxexample}[style=latex,showspaces]{}
\parencites[35]{key1}[78]{key2}<<\relax>>[...]
\parencites[35]{key1}[78]{key2}<<\ >>{...}
\end{ltxexample}
%
Dieses Paket unterstützt standardmäßig die folgenden  Befehle, welche den
regulären Befehlen von \secref{use:cit:std, use:cit:cbx} entsprechen:

\begin{ltxsyntax}


\cmditem{cites}(multiprenote)(multipostnote)[prenote][postnote]{key}|...|[prenote][postnote]{key}
\cmditem{Cites}(multiprenote)(multipostnote)[prenote][postnote]{key}|...|[prenote][postnote]{key}

Ist die entsprechende Version multipler Literaturangaben von \cmd{cite} und
\cmd{Cite}.

\cmditem{parencites}(multiprenote)(multipostnote)[prenote][postnote]{key}|...|[prenote][postnote]{key}
\cmditem{Parencites}(multiprenote)(multipostnote)[prenote][postnote]{key}|...|[prenote][postnote]{key}

Ist die entsprechende Version multipler Literaturangaben von \cmd{parencite} und
\cmd{Parencite}.

\cmditem{footcites}(multiprenote)(multipostnote)[prenote][postnote]{key}|...|[prenote][postnote]{key}
\cmditem{footcitetexts}(multiprenote)(multipostnote)[prenote][postnote]{key}|...|[prenote][postnote]{key}

Ist die entsprechende Version multipler Literaturangaben von \cmd{footcite} und
\cmd{footcitetext}.

\cmditem{smartcites}(multiprenote)(multipostnote)[prenote][postnote]{key}|...|[prenote][postnote]{key}
\cmditem{Smartcites}(multiprenote)(multipostnote)[prenote][postnote]{key}|...|[prenote][postnote]{key}

Ist die entsprechende Version multipler Literaturangaben von \cmd{smartcite} und
\cmd{Smartcite}.

\cmditem{textcites}(multiprenote)(multipostnote)[prenote][postnote]{key}|...|[prenote][postnote]{key}
\cmditem{Textcites}(multiprenote)(multipostnote)[prenote][postnote]{key}|...|[prenote][postnote]{key}

Ist die entsprechende Version multipler Literaturangaben von \cmd{textcite} und
\cmd{Textcite}. 

\cmditem{supercites}(multiprenote)(multipostnote)[prenote][postnote]{key}|...|[prenote][postnote]{key}

Ist die entsprechende Version multipler Literaturangaben von \cmd{supercite}.
Dieser Befehl wird lediglich von numerischen Stilen unterstützt. 

\end{ltxsyntax}

\subsubsection{Stilunabhängige Befehle} \label{use:cit:aut}

Manchmal ist es nötig, Literaturangaben in der Quelldatei in einem Format
anzugeben, welches nicht  an einen speziellen Stil gebunden ist und in der
Präambel global modifiziert werden kann. Das Format dieser Literaturangaben
kann leicht durch das Laden eines anderen Stiles für Literaturangaben geändert
werden. Dennoch, wenn man Befehle wie \cmd{parencite} oder \cmd{footcite}
verwendet, ist die Art, wie die Literaturangaben in den Text eingebettet werden,
trotzdem gewissermaßen vorprogrammiert. Die Idee hinter dem
\cmd{autocite}-Befehl ist die, die Heraufsetzung übergeordneter
Literaturangaben zu unterstützen, welche es möglich macht, inline-Literaturangaben  
global mit solchen zu tauschen, die als Fußnote (oder Index)
angezeigt werden. Der \cmd{autocite}-Befehl wird zu Beginn von hinten
anstehenden Befehlen, so wie \cmd{parencite} und \cmd{footcite} eingebaut.
Dieser Zitierstil unterstützt eine Definition von \cmd{autocite} durch
\cmd{DeclareAutoCiteCommand} (sehen Sie \secref{aut:cbx:cbx}) Diese Definition kann
mithilfe der \cmd{autocite}-
Paketoption aus \secref{use:opt:pre:gen} aktiviert werden. Der Stil der
Literaturangabe von dieser Paketoption ist in einer Weise initialisiert,
die für
den Stil angemessen ist, sehen Sie  \secref{use:xbx:cbx}  für Details. Man beachte, dass es
gewisse Grenzen für die Heraufsetzung übergeordneter Literaturangaben gibt.
Zum Beispiel integrieren inline-Literaturangaben die Autor-Jahr-Übersicht so
eng mit dem Text, dass es virtuell unmöglich ist, diese in Fußnoten zu
ändern. Der \cmd{autocite}-Befehl ist nur verwendbar in Fällen, in denen man
normalerweise \cmd{parencite} oder \cmd{footcite} (oder \cmd{supercite} bei
einem numerischen Stil) verwendet. Die Literaturangaben sollten am Ende des
Satzes oder des Teilsatzes festgelegt werden, direkt vor dem letzten
abschließenden Satzzeichen und sie sollte nicht Teil des Satzes im
grammatikalischen Sinne sein (wie \cmd{textcite} zum Beispiel).

\begin{ltxsyntax}

\cmditem{autocite}[prenote][postnote]{key}
\cmditem{Autocite}[prenote][postnote]{key}

Im Gegensatz zu anderen Befehlen für Literaturverzeichnisse sucht der
\cmd{autocite}-Befehl nicht nur nachfolgend nach Satzzeichen, die dem letzten
Argument folgen, um doppelte Satzzeichen zu vermeiden, tatsächlich
verschiebt er sie sogar, wenn es erforderlich ist. Zum Beispiel wird bei
\kvopt{autocite}{footnote} ein mitgeführtes Satzzeichen so verschoben, dass die
Fußnotenmarkierung nach dem Zeichen ausgegeben wird. \cmd{Autocite} ist so
ähnlich wie \cmd{autocite}, schreibt aber das Präfix des ersten Namens des
Literaturverweises in Großbuchstaben, wenn die \opt{useprefix}-Option vorhanden ist,
vorausgesetzt es gibt ein Namenspräfix und die Art der Literaturangabe gibt
überhaupt alle Namen aus.

\cmditem*{autocite*}[prenote][postnote]{key}
\cmditem*{Autocite*}[prenote][postnote]{key}

Diese mit einem Stern versehene Variante von \cmd{autocite} verhält sich nicht
anders als die reguläre. Der Stern wird lediglich auf das Argument am Ende
übergeben. Wenn zum Beispiel \cmd{autocite} so konfiguriert werden soll,
dass es \cmd{parencite} verwendet, wird \cmd{autocite*} \cmd{parencite*}
ausgeführt. 

\cmditem{autocites}(multiprenote)(multipostnote)[prenote][postnote]{key}|...|[prenote][postnote]{key}
\cmditem{Autocites}(multiprenote)(multipostnote)[prenote][postnote]{key}|...|[prenote][postnote]{key}

Dies ist die Version für multiple Literaturangaben von \cmd{autocite}. Sie
findet und verschiebt ebenfalls Interpunktionen, wenn nötig. Man beachte,
dass es keine Version mit Stern gibt. \cmd{Autocite} ist genau wie \cmd{autocite},
schreibt aber das Präfix des ersten Namens des Literaturverweises in
Großbuchstaben, wenn die \opt{useprefix}-Option vorhanden ist, vorausgesetzt es
gibt ein Namenspräfix und die Art der Literaturangabe gibt überhaupt alle
Namen aus.

\end{ltxsyntax}

\subsubsection{Textbefehle} \label{use:cit:txt}

Die folgenden Befehle werden vom Kern von \biblatex bereitgestellt. Sie
sind dafür ausgelegt, im fließenden Text verwendet zu werden. Man beachte,
dass alle Textbefehle bei der Suche  nach Literaturangaben ausgeschlossen sind. 

\begin{ltxsyntax}

\cmditem{citeauthor}[prenote][postnote]{key}
\cmditem*{citeauthor*}[prenote][postnote]{key}
\cmditem{Citeauthor}[prenote][postnote]{key}
\cmditem*{Citeauthor*}[prenote][postnote]{key}

Diese Befehle geben die Autoren aus. Genauer gesagt, geben sie die
\bibfield{labelname}-Liste aus, die \bibfield{author},  \bibfield{editor}
oder \bibfield{translator} sein können. \cmd{Citeauthor} ist wie
\cmd{citeauthor}, schreibt aber das Präfix des ersten Namens des
Literaturverweises in Großbuchstaben, wenn die \opt{useprefix}-Option vorhanden
ist, vorausgesetzt es gibt ein Namenspräfix.
Die gesternte Variante erzwingt effektiv "`maxcitenames"' nur für diesen Befehl zu 1, 
so nur den ersten Namen in der Labelnamenliste  auszugeben (gefolgt möglicherweise
von einem «et al» String, wenn es mehr Namen sind). Dies ermöglicht mehr einen
natürlichen Textfluss, wenn man sich in einem Papier im Singular bezieht, wenn sonst 
\cmd{citeauthor} eine (natürlich im Plural) Namensliste erzeugen würde.

\cmditem{citetitle}[prenote][postnote]{key}
\cmditem*{citetitle*}[prenote][postnote]{key}

Dieser Befehl gibt den Titel aus. Er verwendet den gekürzten Titel des
Feldes \\
\bibfield{shorttitle}, wenn vorhanden. Andernfalls greift er auf das
\bibfield{title}-Feld mit dem vollen Titel zurück. Die Variante mit Stern gibt immer
den vollen Titel aus. 

\cmditem{citeyear}[prenote][postnote]{key}
\cmditem*{citeyear*}[prenote][postnote]{key}

Dieser Befehl gibt das Jahr aus (bibfield{year}-Feld oder die Jahreskomponente
von \bibfield{date}). Die gesternte Variante beinhaltet die 
\bibfield{extrayear}-Information, wenn vorhanden.

\cmditem{citedate}[prenote][postnote]{key}
\cmditem*{citedate*}[prenote][postnote]{key}

Dieser Befehl gibt das komplette Datum aus (\bibfield{date} oder
\bibfield{year}). Die gesternte Variante beinhaltet die
\bibfield{extrayear}-Information, wenn vorhanden.

\cmditem{citeurl}[prenote][postnote]{key}

Dieser Befehl gibt das \bibfield{url}-Feld aus.

\cmditem{parentext}{text}

Dieser Befehl setzt den \prm{Text} in kontextabhängige runde Klammern.

\cmditem{brackettext}{text}

Dieser Befehl setzt den \prm{Text} in kontextabhängige eckige Klammern.

\end{ltxsyntax}

\subsubsection{Spezielle Befehle} \label{use:cit:spc}

Die folgenden speziellen Befehle werden ebenfalls vom \biblatex-Kern
unterstützt.

\begin{ltxsyntax}

\cmditem{nocite}{key} 
\cmditem*{nocite}|\{*\}|

Dieser Befehl gleicht dem Standard-\latex-Befehl \cmd{nocite}. Er fügt den
\prm{key} in die Bibliografie ein, ohne den Literaturverweis auszugeben. Ist der
\prm{key} ein Stern, werden alle verfügbaren Einträge in der \file{bib}-Datei
in die Bibliografie eingefügt. Wie alle anderen Befehle für Literaturverweise
sind \cmd{nocite}-Befehle im Gerüst des Dokuments lokal zu der umgebenden
\env{refsection}-Umgebung, sofern diese vorhanden ist. Im Unterschied zum
Standard-\latex kann \cmd{nocite} auch in der Präambel verwendet werden. In
diesem Fall sind die Referenzen für die Sektion~0 bestimmt. 

\cmditem{fullcite}[prenote][postnote]{key}

Dieser Befehl verwendet die Bibliografietreiber für den betreffenden
Eingabetyp, um einen vollständigen Literaturverweis ähnlich zu dem
Bibliografieeintrag zu erstellen. Deshalb ist er eher dem Bibliografiestil, denn
dem Stil für Literaturverweise zuzuordnen.

\cmditem{footfullcite}[prenote][postnote]{key}

Ähnelt \cmd{fullcite}, setzt aber den kompletten Literaturverweis in eine
Fußnote und setzt am Ende einen Punkt.

\cmditem{volcite}[prenote]{volume}[page]{key}
\cmditem{Volcite}[prenote]{volume}[page]{key}

Diese Befehle sind ähnlich wie \cmd{cite} und \cmd{Cite}, sie sind aber für
Referenzen in mehrbändigen Arbeiten vorgesehen, welche mit der Nummer des
Bandes und der
Seitenzahl zitiert werden. Anstelle der \prm{postnote} verwenden sie ein
zwingendes \prm{volume}- und ein optionales \prm{page}-Argument. Seit sie
lediglich eine Vorbemerkung bilden und diese an den \cmd{cite} Befehl
weitergeben, welcher vom Stil der Literaturangabe als \prm{postnote}-Argument
unterstützt wird, sind diese Befehle stilunabhängig.
Das Format des Inhaltsanteils wird durch die Feldformatierungsdirektive \opt{volcitevolume}
kontrolliert; Das Format des Seite/Text-Teils wird durch die  Feldformatierungsdirektive  \opt{volcitepages} gesteuert (§ 4.10.4 (e. V.)).
%\secref{aut:fmt:ich}).
Das Trennzeichen zwischen
dem Inhaltsanteil und dem Seite/Text-Teil kann mit einer Neudefinition des Makros  \cmd{volcitedelim} geändert werden (§ 4.10.1 (e. V.)). %\secref{aut:fmt:fmt}).

\cmditem{volcites}(multiprenote)(multipostnote)[prenote]{volume}[page]{key}|\\...|[prenote]{volume}[page]{key}
\cmditem{Volcites}(multiprenote)(multipostnote)[prenote]{volume}[page]{key}|\\...|[prenote]{volume}[page]{key}

Die "`multicite"'-Version von \cmd{volcite} und \cmd{Volcite}.

\cmditem{pvolcite}[prenote]{volume}[page]{key}
\cmditem{Pvolcite}[prenote]{volume}[page]{key}

Gleicht \cmd{volcite}, basiert aber auf \cmd{parencite}.

\cmditem{pvolcites}(multiprenote)(multipostnote)[prenote]{volume}[page]{key}|\\...|[prenote]{volume}[page]{key}
\cmditem{Pvolcites}(multiprenote)(multipostnote)[prenote]{volume}[page]{key}|\\...|[prenote]{volume}[page]{key}

Die "`multicite"'-Version von \cmd{pvolcite} und \cmd{Pvolcite}.

\cmditem{fvolcite}[prenote]{volume}[page]{key}
\cmditem{ftvolcite}[prenote]{volume}[page]{key}

Gleicht \cmd{volcite}, basiert aber auf \cmd{footcite} beziehungsweise
\cmd{footcitetext}.

\cmditem{fvolcites}(multiprenote)(multipostnote)[prenote]{volume}[page]{key}|\\...|[prenote]{volume}[page]{key}
\cmditem{Fvolcites}(multiprenote)(multipostnote)[prenote]{volume}[page]{key}|\\...|[prenote]{volume}[page]{key}

Die "`multicite"'-Version von \cmd{fvolcite} und \cmd{Fvolcite}.

\cmditem{svolcite}[prenote]{volume}[page]{key}
\cmditem{Svolcite}[prenote]{volume}[page]{key}

Gleicht \cmd{volcite}, basiert aber auf \cmd{smartcite}.

\cmditem{svolcites}(multiprenote)(multipostnote)[prenote]{volume}[page]{key}|\\...|[prenote]{volume}[page]{key}
\cmditem{Svolcites}(multiprenote)(multipostnote)[prenote]{volume}[page]{key}|\\...|[prenote]{volume}[page]{key}

Die "`multicite"'-Version von \cmd{svolcite} und \cmd{Svolcite}.

\cmditem{tvolcite}[prenote]{volume}[page]{key}
\cmditem{Tvolcite}[prenote]{volume}[page]{key}

Gleicht \cmd{volcite}, basiert aber auf \cmd{textcite}.

\cmditem{tvolcites}(multiprenote)(multipostnote)[prenote]{volume}[page]{key}|\\...|[prenote]{volume}[page]{key}
\cmditem{Tvolcites}(multiprenote)(multipostnote)[prenote]{volume}[page]{key}|\\...|[prenote]{volume}[page]{key}

Die "`multicite"'-Version von \cmd{tvolcite} und \cmd{Tvolcite}.

\cmditem{avolcite}[prenote]{volume}[page]{key}
\cmditem{Avolcite}[prenote]{volume}[page]{key}

Gleicht \cmd{volcite}, basiert aber auf \cmd{autocite}.

\cmditem{avolcites}(multiprenote)(multipostnote)[prenote]{volume}[page]{key}|\\...|[prenote]{volume}[page]{key}
\cmditem{Avolcites}(multiprenote)(multipostnote)[prenote]{volume}[page]{key}|\\...|[prenote]{volume}[page]{key}

Die "`multicite"'-Version von \cmd{avolcite} und \cmd{Avolcite}.

\cmditem{notecite}[prenote][postnote]{key}
\cmditem{Notecite}[prenote][postnote]{key}

Diese Befehle geben das \prm{prenote}- und das \prm{postnote}-Argument aus,
jedoch nicht den Literaturverweis. Anstelle dessen wird ein \cmd{nocite}-Befehl
für jeden \prm{key} ausgegeben. Das kann für Autoren nützlich sein, die
implizite Literaturverweise in ihr Schriftstück mit  einbeziehen, jedoch nur
die Information dazu angeben und diese zuvor noch nicht im laufenden Text
erwähnt haben, die aber trotzdem den Vorteil der automatischen
\prm{postnote}-Formatierung und der impliziten \cmd{nocite}-Funktion ausnutzen
wollen. Dies ist ein generischer, stilabhängiger Befehl für Literaturverweise.
Spezielle Stile für Literaturangaben können elegantere Möglichkeiten für den
selben Zweck bereitstellen. Die großgeschriebene Version erzwingt die
Großschreibung (man beachte, dass dies nur dann verwendet werden kann, wenn die
Bemerkung mit einem Befehl beginnt, der sensitiv ist für die
\biblatex-Zeichensuche).

\cmditem{pnotecite}[prenote][postnote]{key}
\cmditem{Pnotecite}[prenote][postnote]{key}

Gleicht \cmd{notecite}, doch die Bemerkungen werden in runden Klammern
ausgegeben.

\cmditem{fnotecite}[prenote][postnote]{key}

Gleicht \cmd{notecite}, doch die Bemerkungen werden in einer Fußnote
ausgegeben.

\end{ltxsyntax}

\subsubsection{Befehle niederer Ebenen} \label{use:cit:low}

Die folgenden Befehle werden ebenfalls von dem \biblatex-Kern unterstützt.
Sie gewähren den Zugriff auf alle Listen und Felder auf einer niedrigeren Ebene.

\begin{ltxsyntax}

\cmditem{citename}[prenote][postnote]{key}[format]{name list}

Das \prm{format} ist eine Formatierungsdirektive, die durch
\cmd{DeclareNameFormat} definiert wird. Formatierungsdirektiven werden in
§ 4.4.2 (e. V.) 
%\secref{aut:bib:fmt} 
behandelt. Wenn das optionale Argument ausgelassen wird,
fällt dieser Befehl zurück auf das Format \texttt{citename}. Das letzte
Argument ist der Name des \prm{name list}, dessen Sinn in \secref{bib:fld}
erklärt wird.

\cmditem{citelist}[prenote][postnote]{key}[format]{literal list}

Das \prm{format} ist eine Formatierungsdirektive, die durch
\cmd{DeclareNameFormat} definiert wird. Formatierungsdirektiven werden in % \secref{aut:bib:fmt}
    § 4.4.2 (engl. V.) behandelt. Wenn das optionale Argument ausgelassen wird,
fällt dieser Befehl zurück auf das format \texttt{citelist}. Das letzte
Argument ist der Name des \prm{literal list}, dessen Sinn in \secref{bib:fld}
erklärt wird.

\cmditem{citefield}[prenote][postnote]{key}[format]{field}

Das \prm{format} ist eine Formatierungsdirektive, die durch
\cmd{DeclareNameFormat} definiert wird. Formatierungsdirektiven werden in %\secref{aut:bib:fmt}
§ 4.4.2    behandelt. Wenn das optionale Argument ausgelassen wird,
fällt dieser Befehl zurück auf das Format \texttt{citefield}. Das letzte
Argument ist der Name des \prm{field}, dessen Sinn in \secref{bib:fld}
erklärt wird.

\end{ltxsyntax}

\subsubsection{Sonstige Befehle} \label{use:cit:msc}

Die Befehle in diesem Abschnitt sind kleine Helfer, die mit Literaturverweisen
verwandt sind.

\begin{ltxsyntax}

\csitem{citereset}

Dieser Befehl setzt den Stil des Literaturstelle zurück auf Null. Das kann
nützlich sein, wenn der Stil sich wiederholende Literaturverweise durch
Abkürzungen, wie \emph{ibidem}, \emph{idem}, \emph{op. cit.} usw ersetzt und
man eine vollständige Literaturangabe zu Beginn eines neuen Kapitels,
Abschnitts oder einer anderen Position erzwingen möchte. Dieser Befehl führt
einen spezifischen Initialisierungsaufhänger aus, der durch
\cmd{InitializeCitationStyle} aus § 4.3.1 (e. V.)
%\secref{aut:cbx:cbx} 
definiert wird. Er setzt
ebenfalls die eingebauten Literaturstellensucher dieses Paketes zurück. Der
Reset wirkt sich aus auf die Tests \cmd{ifciteseen}, \cmd{ifentryseen},
\cmd{ifciteibid} und \cmd{ifciteidem}, welche in %\secref{aut:aux:tst} 
    § 4.6.2 (engl. V.)    behandelt
werden. Wenn der Befehl innerhalb einer \env{refsection}-Umgebung eingesetzt
wird, ist die Rückstellung lokal zur momentanen \env{refsection}-Umgebung.
Siehe ebenfalls beim \opt{citereset}-Paket in \secref{use:opt:pre:gen}.

\csitem{citereset*}

Ähnlich zu \cmd{citereset}, führt aber lediglich den
Initialisierungsaufhänger des Stils aus, ohne die eingebauten
Literaturstellensucher zurückzusetzen.

\csitem{mancite}

Man verwendet diesen Befehl, um eingebettete Literaturstellen manuell zu
markieren, wenn man automatisch generierte und manuell erstellte
Literaturangaben gemischt verwenden will. Das ist besonders nützlich, wenn der
Stil des Literaturverweises sich wiederholende Stellen durch Abkürzungen wie
\emph{ibidem} ersetzt, welche sonst nicht eindeutig oder irreführend sein
können.
\cmd{mancite} sollte immer im gleichen Kontext wie die manuell erstellten
Literaturverweise. , zum Beispiel, wenn der Verweis als Fußnote gegeben ist,
bezieht man \cmd{mancite} in die Fußnote mit ein. Der \cmd{mancite}-Befehl
führt einen stilspezifischen Resetaufhänger aus, der mit dem
\cmd{OnManualCitation}-Befehl aus § 4.3.1 (e. V.) 
%\secref{aut:cbx:cbx} 
definiert wird. Er setzt
ebenfalls die eingebetteten 'ibidem' und 'idem' Sucher des Paketes zurück. Der
Reset wirkt sich auf \cmd{ifciteibid}- und \cmd{ifciteidem}-Tests aus, die in
%\secref{aut:aux:tst} 
§ 4.6.2 (engl. V.) behandelt werden. 

\csitem{pno}

Dieser Befehl erzwingt ein einzelnes Seitenpräfix im \prm{postnote}-Argument
eines Befehls für Literaturstellen. Siehe \secref{use:cav:pag} für
weitergehende Details und Verwendungsanleitungen. Man beachte, dass dieser
Befehl nur lokal in Literaturstellen und Bibliografien vorhanden ist.

\csitem{ppno}

Gleicht \cmd{pno}, erzwingt aber ein abgegrenztes Präfix. Siehe
\secref{use:cav:pag} für weitergehende Details und Verwendungsanleitungen. Man
beachte, dass dieser Befehl nur lokal in Literaturstellen und Bibliografien
vorhanden ist.

\csitem{nopp}

Gleicht \cmd{pno}, unterdrückt aber alle Präfixe. Siehe \secref{use:cav:pag}
für weitergehende Details und Verwendungsanleitungen. Man beachte, dass dieser
Befehl nur lokal in Literaturstellen und Bibliografien vorhanden ist.

\csitem{psq}

Im Argument der \prm{postnote} eines Befehls für Literaturangaben bezeichnet
dieser Befehl einen Bereich von zwei Seiten, wo nur die Startseite gegeben ist.
Siehe \secref{use:cav:pag} für weitere Details und Verwendungsanleitungen.
Das ausgegebene
Suffix ist der Lokalisierungsstring \texttt{sequens}, siehe %\secref{aut:lng:key}.
§ 4.9.2 (engl. V.). Der eingebettete Abstand zwischen dem Suffix und der
Seitenzahl kann durch die Neudefinition des Makros \cmd{sqspace} modifiziert
werden. Die Grundeinstellung ist eine unveränderbare Leerstelle innerhalb des
Wortes. Man beachte, dass dieser Befehl nur lokal in Literaturstellen und
Bibliografien vorhanden ist.

\csitem{psqq}

Gleicht \cmd{psq}, gibt aber einen Seitenabschnitt mit offenem Ende aus. Siehe
\secref{use:cav:pag} für weitere Details und Anwendungsbeschreibungen. Das
ausgegebene Suffix ist der Lokalisierungsstring \texttt{sequentes},
sehen Sie %\secref{aut:lng:key}. 
§ 4.9.2 (engl. V.). Man beachte, dass dieser Befehl nur lokal in
Literaturstellen und Bibliografien vorhanden ist.

\cmditem{pnfmt}{text}

This command formats is argument \prm{text} in the same format as \bibfield{postnote}. The command can be used to format a page range while adding additional text in the postnote argument of a cite command.

\begin{ltxexample}
\autocite[\pnfmt{378-381, 383} and more]{sigfridsson}
\end{ltxexample}

\cmditem{RN}{integer}

Dieser Befehl gibt einen Ganzzahlwert als großgeschriebene römische Zahl aus.
Die Formatierung des Zahlenzeichens kann durch die Neudefinition des Makros
\cmd{RNfont} modifiziert werden.

\cmditem{Rn}{integer}

Gleicht \cmd{RN}, gibt aber eine kleingeschriebene römische Zahl aus. Die
Formatierung des Zahlenzeichens kann durch die Neudefinition des Makros
\cmd{Rnfont} modifiziert werden.

\end{ltxsyntax}

\subsubsection{\sty{natbib}-kompatible Befehle} \label{use:cit:nat}

Das \opt{natbib}-Paket lädt ein \sty{natbib}-kompatibles Modul. Dieses Modul
definiert Pseu\-donyme für die vom \sty{natbib}-Paket unterstützten Befehle
für
Literaturverweise. Dies beinhaltet Parallelbezeichnungen für die Kernbefehle
\cmd{citet} und \cmd{citep} ebenso, wie die Varianten \cmd{citealt} und
\cmd{citealp}. Die Version mit Stern dieser Befehle, welche die komplette
Autorenliste ausgibt, wird ebenfalls unterstützt. Der \cmd{cite}-Befehl,
welcher von \sty{natbib} auf besondere Weise behandelt wird, wird nicht speziell
behandelt. Die Textbefehle (\cmd{citeauthor}, \cmd{citeyear}, usw.) werden
ebenfalls unterstützt, ebenso wie alle Befehle, welche das Namenspräfix in
Großbuchstaben setzen (\cmd{Citet}, \cmd{Citep}, \cmd{Citeauthor}, etc.). 
Das Setzen von Ersatznamen durch \cmd{defcitealias}, \cmd{citetalias}
und \cmd{citepalias} ist
möglich. Man beachte, dass die Kompatibilitätsbefehle nicht durch das
Format der Literaturangaben des \sty{natbib}-Pakets emuliert werden. Sie wandeln
lediglich \sty{natbib}-Befehle in Pseudonyme der funktional äquivalenten
Möglichkeiten des \sty{biblatex}-Paketes um. Das Format der
Literaturverweise hängt vom Hauptstil der Literaturangaben ab. Trotzdem wird sich der
Kompatibilitätsstil \cmd{nameyeardelim} angleichen, um mit dem Standardstil des
\sty{natbib}-Pakets übereinzustimmen.

\subsubsection[\sty{mcite}-ähnliche Befehle für
Literaturverweise]{\sty{mcite}-artige Zitierungsbefehle} 
\label{use:cit:mct}

Die \opt{mcite}-Paketoption lädt ein spezielles Modul für Literaturangaben,
welches \sty{mcite}\slash \sty{mciteplus}-ähnliche Befehle für
Literaturverweise unterstützt. Genau genommen ist das, was die Module
bereitstellen, die Hülle für Befehle des Hauptstils für Literaturstellen. Zum
Beispiel der folgende Befehl:


\begin{ltxexample}
\mcite{key1,setA,*keyA1,*keyA2,*keyA3,key2,setB,*keyB1,*keyB2,*keyB3}
\end{ltxexample}
%
Er ist tatsächlich äquivalent zu:

\begin{ltxexample}
\defbibentryset{setA}{keyA1,keyA2,keyA3}%
\defbibentryset{setB}{keyB1,keyB2,keyB3}%
\cite{key1,setA,key2,setB}
\end{ltxexample}
%
Der \cmd{mcite}-Befehl arbeitet mit allen Stilen, seit es üblich ist, dass der
letzte Befehl von \cmd{cite} von vom Hauptstil für Literaturverweise
kontrolliert wird. Das \texttt{mcite}-Modul stellt Hüllen für die
Standardbefehle aus \secref{use:cit:std,use:cit:cbx} bereit. Siehe
\tabrefe{use:cit:mct:tab2} für einen Überblick. Vor- und Nachbemerkungen,
ebenso wie die mit Stern versehene Version aller Befehle, werden unterstützt.
Die Parameter werden an den hintenan stehenden Befehl weitergegeben. Zum
Beispiel:

\begin{ltxexample}
\mcite*[pre][post]{setA,*keyA1,*keyA2,*keyA3}
\end{ltxexample}
%
Er führt aus:

\begin{ltxexample}
\defbibentryset{setA}{keyA1,keyA2,keyA3}%
\cite*[pre][post]{setA}
\end{ltxexample}
%
Man beachte, dass das \texttt{mcite}-Modul kein Kompatibilitätsmodul ist. Es
stellt Befehle zur Verfügung, die zwar sehr ähnlich, doch nicht gleich in
Syntax und Funktion zum \sty{mcite}-Befehl sind. Wechselt man von
\sty{mcite}\slash\sty{mciteplus} zu \sty{biblatex}, müssen die Altdateien
(legacy files) aktualisiert werden. Mit \sty{mcite} ist das erste Element der
Gruppe von Literaturverweisen auch der Identifikator der Gruppe als Ganzes. Um
ein Beispiel aus der \sty{mcite}-Beschreibung zu übernehmen, es ist diese
Gruppe: 

\begin{table} 
\tablesetup
\begin{tabular}{@{}V{0.5\textwidth}@{}V{0.5\textwidth}@{}} \toprule
\multicolumn{1}{@{}H}{Standardbefehl} &
\multicolumn{1}{@{}H}{\sty{mcite}-artiger Befehl} \\ 
\cmidrule(r){1-1}\cmidrule{2-2} 
|\cite|		& |\mcite| \\
|\Cite|		& |\Mcite| \\
|\parencite|	& |\mparencite| \\
|\Parencite|	& |\Mparencite| \\
|\footcite|	& |\mfootcite| \\
|\footcitetext|	& |\mfootcitetext| \\
|\textcite|	& |\mtextcite| \\
|\Textcite|	& |\Mtextcite| \\
|\supercite|	& |\msupercite| \\
\bottomrule
\end{tabular}
\caption{\sty{mcite}-artige Befehle} \label{use:cit:mct:tab1} 
\end{table}

\begin{ltxexample}
\cite{<<glashow>>,*salam,*weinberg}
\end{ltxexample}
%
Sie besteht aus drei Einträgen und der Eingabeschlüssel des ersten fungiert auch
als Identifikator der gesamten Gruppe. Im Gegensatz dazu ist eine
\sty{biblatex}-Eingabe\-menge eine eigenständige Einheit. Dafür ist ein
eindeutig bestimmbarer Eingabeschlüssel notwendig, welcher der Menge zugewiesen
wird, wie sie definiert ist:

\begin{ltxexample}
\mcite{<<set1>>,*glashow,*salam,*weinberg}
\end{ltxexample}
%
Einmal definiert, kann die Eingabemenge wie jeder reguläre Eintrag in der
\file{bib}-Datei behandelt werden. Verwendet man einen der
\texttt{numeric}-Stile, welche mit \texttt{biblatex} geladen werden und die
\opt{subentry}-Option aktivieren, ist es auch möglich, auf Mengenelemente zu
verweisen. Sehen Sie \tabrefe{use:cit:mct:tab2} für einige Beispiele. Eine erneute
Angabe der originalen Definition ist nicht nötig, aber zulässig. Im Gegensatz
zu \sty{mciteplus} ist hier eine teilweise Neuangabe der ursprünglichen
Definition unzulässig. Man verwendet stattdessen den Eingabeschlüssel der
Menge. 

\begin{table} 
\tablesetup
\begin{tabular}{@{}V{0.5\textwidth}@{}V{0.1\textwidth}@{}p{0.4\textwidth}@{}}
\toprule 
\multicolumn{1}{@{}H}{Eingabe} & \multicolumn{1}{@{}H}{Ausgabe} &
\multicolumn{1}{@{}H}{Kommentierung} \\
\cmidrule(r){1-1}\cmidrule(r){2-2}\cmidrule{3-3}
|\mcite{set1,*glashow,*salam,*weinberg}|& [1]	& Bestimmen und zitieren
den Satz \\
|\mcite{set1}|				& [1]	& Nachfolgendes Zitieren
des Satzes
\\ |\cite{set1}|				& [1]	& Übliches
reguläres |\cite|-Arbeiten \\
 |\mcite{set1,*glashow,*salam,*weinberg}|& [1]	& Redundant, aber
permisible \\ |\mcite{glashow}|			& [1a]	& Zitieren
ein Satzmitglied \\ |\cite{weinberg}|			& [1c]	& Ebenso reguläres
|\cite|-Arbeiten \\ \bottomrule 
\end{tabular} \caption[\sty{mcite}-artige Syntax]
{\sty{mcite}-artige Syntax (Beispiel der Ausgabe mit
\kvopt{style}{numeric}- und \opt{subentry}-Option)} \label{use:cit:mct:tab2} 
\end{table}

\subsection{Lokalisierungsbefehle} \label{use:lng}

Das \biblatex-Paket unterstützt Übersetzungen für
Schlüsselbezeichnungen wie <edition> oder <volume> ebenso, wie Definitionen
für sprachspezifische Merkmale, wie das Datumsformat und Ordinale. Diese
Definitionen, welche automatisch geladen werden, können in der Präambel des
Dokuments oder in der Konfigurationsdatei mit den in diesem Abschnitt
vorgestellten Befehlen modifiziert oder erweitert werden.

\begin{ltxsyntax}

\cmditem{DefineBibliographyStrings}{language}{definitions}

Dieser Befehl wird verwendet, um Lokalisierungsstrings zu definieren. Die
\prm{language} muss der Name einer Sprache aus dem \sty{babel}-Paket sein, das
heißt, einer der Identifikatoren aus \tabrefe{bib:fld:tab1} auf Seite
\pageref{bib:fld:tab1}. Die \prm{definitions} sind \keyval-Paare, welche einem
Ausdruck einen Identifikator zuordnen:

\begin{ltxexample}
\DefineBibliographyStrings{american}{%
  bibliography  = {Bibliography},
  shorthands    = {Abbreviations},
  editor        = {editor},
  editors       = {editors},
}
\end{ltxexample}
%
Eine vollständige Liste aller Schlüssel, die von der Standardeinstellung
bereitgestellt werden, findet sich in % \secref{aut:lng:key}. 
§ 4.9.2 (engl. V.).    Man beachte,
dass alle Ausdrücke in Großbuchstaben geschrieben sein sollten, was sie
gewöhnlich auch sind, wenn sie innerhalb eines Satzes verwendet werden. Das
\biblatex-Paket schreibt das erste Wort automatisch groß, wenn das zu
Beginn eines Satzes erforderlich ist. Ausdrücke, die für den Gebrauch in
Überschriften vorgesehen sind, sollten auf eine Weise großgeschrieben werden,
wie es für einen Titel angemessen ist. Im Gegensatz zu\\
\cmd{DeclareBibliographyStrings} überschreibt \cmd{DefineBibliographyStrings}
sowohl die komplette, wie auch die abgekürzte Version des Strings. Sehen Sie
%\secref{aut:lng:cmd} 
§ 4.9.1 (engl. V.) für weitere Details. 

\cmditem{DefineBibliographyExtras}{language}{code}

Dieser Befehl wird verwendet, um sprachspezifische Merkmale wie das Datumsformat
und Ordinale zu adaptieren. Die \prm{language} muss der Name einer Sprache aus
dem \sty{babel}-Paket sein. Der \prm{code}, der beliebiger \latex-Code sein
kann, besteht normalerweise aus Neudefinitionen der Formatierungsbefehle aus
\secref{use:fmt:lng}.

\cmditem{UndefineBibliographyExtras}{language}{code}

Dieser Befehl wird verwendet, um die Originaldefinition aller durch\\
\cmd{DefineBibliographyExtras} modifizierten Befehle wieder herzustellen. Wenn
ein neu definierter Befehl aus \secref{use:fmt:lng} inbegriffen ist, muss seine
vorherige Definition nicht wieder hergestellt werden, seit diese Befehle sowieso
in allen Sprachmodulen eingebaut wurden.

\cmditem{DefineHyphenationExceptions}{language}{text}

Dies ist ein \latex-Kopfende eines \tex-\cmd{hyphenation}-Befehls, der Ausnahmen
für die Silbentrennung definiert. Die \prm{language} muss der Name einer
Sprache aus dem \sty{babel}-Paket sein. Der \prm{text} ist eine durch
Leerzeichen unterteilte Liste von Wörtern. Silbetrennungen werden durch einen
Trennstrich markiert:

\begin{ltxexample}
\DefineHyphenationExceptions{american}{%
  hy-phen-ation ex-cep-tion
}
\end{ltxexample}

\cmditem{NewBibliographyString}{key}

Dieser Befehl deklariert neue Lokalisierungsstrings, das heißt, er
initialisiert einen neuen \prm{key}, der in den \prm{definitions} von
\cmd{DefineBibliographyStrings} verwendet wird. Das \prm{key}-Argument kann
ebenfalls eine durch Kommata getrennte Liste von Schlüsselnamen sein. Die
Schlüssel, die in § 4.9.2 (e. V.) 
%\secref{aut:lng:key} 
aufgelistet werden, sind durch die
Standardeinstellung definiert.

\end{ltxsyntax}

\subsection{Formatierungsbefehle} \label{use:fmt}

Die Befehle und Möglichkeiten, die in diesem Abschnitt erläutert werden,
können zur Abstimmung der Formate von Literaturverweisen und Bibliografien
verwendet werden.

\begin{ltxsyntax}
\cmditem{ifentryseen}{entrykey}{true}{false}
\cmditem{ifentryinbib}{entrykey}{true}{false}
\cmditem{ifentrycategory}{entrykey}{category}{true}{false}
\cmditem{ifentrykeyword}{entrykey}{keyword}{true}{false}
\end{ltxsyntax}

\subsubsection{Generelle Befehle und Hooks} \label{use:fmt:fmt}

Die Befehle in diesem Abschnitt können mit \cmd{renewcommand} in der Dokumentpräambel
neu definiert werden. Diejenigen, die als  <Context Sensitive> im Rand markiert sind,
kann man auch mit \cmd{DeclareDelimFormat} individualisierne und ausgeben mit 
\cmd{printdelim} (\secref{use:fmt:csd}). Man beachte, dass alle Befehle, die mit \cmd{mk\dots}
beginnen, ein Argument benötigen. All diese Befehle sind in \path{biblatex.def}
definiert.

\begin{ltxsyntax}

\csitem{bibsetup}

Beliebiger Kode, der zu Beginn der Bibliografie ausgeführt wird, für Befehle
gedacht, die das Layout der Bibliografie betreffen.

\csitem{bibfont}

Beliebiger Kode, der die in der Bibliografie verwendete Schriftart bestimmt.
Dieser ist sehr ähnlich zu \cmd{bibsetup}, ist aber für wechselnde
Schriftarten gedacht. 

\csitem{citesetup}

Beliebiger Kode, der zu Beginn eines jeden Befehls für Literaturverweise
ausgeführt wird.

\csitem{newblockpunct}

Das Gliederungszeichen zwischen Bereichen unterschiedlicher Bedeutung, welches
in § 4.7.1 (e. V.) 
%\secref{aut:pct:new} 
erklärt wird. Die Standarddefinition wird von der
Paketoption \opt{block} kontrolliert (sehen Sie \secref{use:opt:pre:gen}). 

\csitem{newunitpunct}

Das Gliederungszeichen zwischen Sinneinheiten, welches in § 4.7.1 (e. V.)
%\secref{aut:pct:new}
erklärt wird. Das ist in der Regel ein Punkt oder ein Komma mit einem
Leerzeichen. Die Standarddefinition ist ein Punkt mit einem Leerzeichen.

\csitem{finentrypunct}

Die Interpunktion, die zum Schluss eines jeden Bibliografieeintrags gesetzt
wird, normalerweise ein Punkt.  Die Standarddefinition ist ein Punkt.

\csitem{entrysetpunct}
Die Interpunktion, die zwischen den Bibliografieuntereinträgen eines Eintragssatzes
ausgegeben wird. Die Standarddefinition ist ein Semikolon und ein Leerzeichen. 

\csitem{bibnamedelima}

Dieses Trennzeichen steuert den
Abstand zwischen den Elementen, die ein Namensteil sind. Es wird
automatisch nach dem Vornamen eingefügt, wenn das Element länger als drei
Zeichen lang ist und vor dem letzten Element ist. Die Standarddefinition
ist ein wortinterner Raum, penaltisiert vom Wert des
\cnt{highnamepenalty}-Zählers (\secref{use:fmt:len}). Sehen Sie bitte in
\secref{use:cav:nam} für weitere Details.

\csitem{bibnamedelimb}

Dieses Trennzeichen ist zwischen dem
eingesetzten Element, das einen Namensteil bildet, der \cmd{bibnamedelima}
nicht verwendet. Die Standarddefinition ist ein wortinterner Raum, penaltisiert vom Wert des
des \cnt{lownamepenalty}-Zählers (\secref{use:fmt:len}). Sehen Sie bitte in
\secref{use:cav:nam} für weitere Details.

\csitem{bibnamedelimc} 

Dieses Trennzeichen steuert den
Abstand zwischen Namensteilen. Es ist eingefügt zwischen dem Namenspräfix
und dem Nachnamen, wenn \kvopt{useprefix}{true}. Die Standarddefinition ist ein
wortinterner Raum, penaltisiert vom Wert des\\ \cnt{highnamepenalty}-Zählers
(\secref{use:fmt:len}). Sehen Sie bitte in
\secref{use:cav:nam} für weitere Details.

\csitem{bibnamedelimd}

Dieses Trennzeichen ist zwischen allen Namensteilen,
die \cmd{bibnamedelimc} nicht verwendet. Die Standarddefinition ist ein
wortinterner Raum, penaltisiert vom Wert des \cnt{lownamepenalty}-Zählers.
(\secref{use:fmt:len}). Sehen Sie bitte in
\secref{use:cav:nam} für weitere Details.

\csitem{bibnamedelimi}

Dieses Trennzeichen ersetzt
\cmd{bibnamedelima/b} nach Initialen. Beachten Sie, dass dies nur für
Initialen im  \file{bib}-File gegeben ist,  nicht für Initialen, die
automatisch von \sty{biblatex} generiert werden, die ihren eigenen Satz von
Trennzeichen benutzen.

\csitem{bibinitperiod}

Fügt die Interpunktion nach Initialen
ein, es sei denn \cmd{bibinithyphendelim} gilt. Die Standarddefinition ist
ein Punkt (\cmd{adddot}). Sehen Sie bitte in
\secref{use:cav:nam} für weitere Details.

\csitem{bibinitdelim}

Der Abstand zwischen mehreren
eingefügten Initialen, es sei denn\\ \cmd{bibinithyphendelim} gilt. Die
Standarddefinition ist ein nicht zerbrechlicher Interwort-Raum. Sehen Sie bitte in
\secref{use:cav:nam} für weitere Details.

\csitem{bibinithyphendelim}

Die Satzzeichen zwischen den 
Initials von Namensteilen mit Bindestrich, ersetzt \cmd{bibinitperiod} und
\cmd{bibinitdelim}. Die Standarddefinition ist ein Raum gefolgt von einem
nichtzerbrechlichen Bindestrich.  Sehen Sie bitte in
\secref{use:cav:nam} für weitere Details.

\csitem{bibindexnamedelima}

Ersetzt \cmd{bibnamedelima} im Index.

\csitem{bibindexnamedelimb}

Ersetzt \cmd{bibnamedelimb} im Index.

\csitem{bibindexnamedelimc}

Ersetzt \cmd{bibnamedelimc} im Index.

\csitem{bibindexnamedelimd}

Ersetzt \cmd{bibnamedelimd} im Index.

\csitem{bibindexnamedelimi}

Ersetzt \cmd{bibnamedelimi} im Index.

\csitem{bibindexinitperiod}

Ersetzt \cmd{bibinitperiod} im Index.

\csitem{bibindexinitdelim}

Ersetzt \cmd{bibinitdelim} im Index.

\csitem{bibindexinithyphendelim}

Ersetzt \cmd{bibinithyphendelim} im Index.

\csitem{revsdnamepunct}

Die Zeichensetzung zwischen Vor- und Nachnamen, wenn Teile ausgegeben werden, wenn der Name
umgekehrt wird. Hier ist ein Beispiel eines Namens mit dem Standardkomma als  \cmd{revsdnamedelim} zu sehen:

\begin{ltxexample}
Jones<<,>> Edward
\end{ltxexample}

Dieser Befehl sollte mit \cmd{bibnamedelimd} als "`reversed-name separator"'
in Formatierungsanweisungen für Namenslisten verwendet werden. Bitte beachten Sie,
\secref{use:cav:nam} für weitere Details.

\csitem{bibnamedash}

Der Strich, der als Ersatz für wiederkehrende Autoren oder
Herausgeber in der Bibliografie verwendet wird. Standardmäßig ist das ein <em>-
oder <en>-Strich, abhängig von der Größe der Referenzliste.

\csitem{labelnamepunct}\DeprecatedMark

Das Gliederungszeichen, welches nach dem Namen ausgegeben wird, um die
Bibliografie alphabetisch zu ordnen (\bibfield{author} oder \bibfield{editor},
falls das \bibfield{author}-Feld nicht definiert ist). Bei den Standardstilen
ersetzt dieses Trennzeichen an dieser Stelle den \cmd{newunitpunct}. Die
Standarddefinition ist \cmd{newunitpunct}, das heißt, es wird nichts anders
verwendet als reguläre Abschnittsinterpunktionen.
 This punctuation command is deprecated and has been superseded by the context-sensitive \cmd{nametitledelim} (see \secref{use:fmt:csd}). For backwards compatibility reasons, however, \cmd{nametitledelim} still defaults to \cmd{labelnamepunct} in the \texttt{bib} and \texttt{biblist} contexts. Style authors may want to consider replacing \cmd{labelnampunct} with \texttt{\textbackslash printdelim\{nametitledelim\}} and users may want to prefer modifying the context-sensitive \texttt{nametitledelim} with \cmd{DeclareDelimFormat} over redefining \cmd{labelnamepunct}.
 
\csitem{subtitlepunct}

Das Gliederungszeichen, welches zwischen den Feldern \bibfield{title} und
\bibfield{subtitle}, \bibfield{booktitle} und 
\bibfield{booksubtitle} so wie
\bibfield{maintitle} und \bibfield{mainsubtitle} gesetzt wird. Bei den
Standardstilen ersetzt dieses Trennzeichen an dieser Stelle \cmd{newunitpunct}.
Die Standarddefinition ist \cmd{newunitpunct}, das heißt, es wird nichts anders
verwendet als reguläre Abschnittsinterpunktionen. 

\csitem{intitlepunct}

Das Gliederungszeichen zwischen dem Wort «in» (aus) und dem darauffolgenden Titel in
Anführungszeichen so wie \bibtype{article}, \bibtype{inbook},
\bibtype{incollection}, usw. Die Standarddefinition ist ein Doppelpunkt mit
einem Leerzeichen (zum Beispiel «Artikel, in: \emph{Journal}»
 oder «Titel, in: \emph{Book}»). Man beachte, dass dies ein String ist und nicht nur eine
Interpunktion. Wenn man keinen Doppelpunkt nach "`aus"' haben möchte, fügt
\cmd{intitlepunct} trotzdem eine Leerstelle ein.

\csitem{bibpagespunct}

Das Gliederungszeichen, welches vor dem \bibfield{pages}-Feld eingefügt wird.
Die Standardeinstellung ist ein Komma mit einem Leerzeichen.

\csitem{bibpagerefpunct}

Das Gliederungszeichen, welches vor dem \bibfield{pageref}-Feld ausgegeben wird.
Die Standardeinstellung ist ein Leerzeichen. 

\csitem{multinamedelim}\CSdelimMark

Das Trennzeichen, welches zwischen multiplen Begriffen in einer Namensliste, wie
\bibfield{author} oder \bibfield{editor}, wenn es mehr als zwei Namen in der
Liste gibt. Die Standard\-einstellung ist ein Komma, gefolgt von einem
Leerzeichen. Siehe \cmd{finalnamedelim} für ein Beispiel.\footnote{Man beachte,
dass \cmd{multinamedelim} keinesfalls verwendet werden kann, wenn es nur zwei
Namen in der Liste gibt. In diesem Fall verwenden die Standardstile
\cmd{finalnamedelim}.}

\csitem{finalnamedelim}

Das Trennzeichen, welches anstelle von \cmd{multinamedelim} vor dem letzten
Namen in einer Namensliste eingefügt wird. Die Standardeinstellung ist der
Lokalisierungsterm <and>, begrenzt durch Leerzeichen. Hier ein Beispiel:

\begin{ltxexample}
Michel Goossens<<,>> Frank Mittelbach <<and>> Alexander Samarin
Edward Jones <<and>> Joe Williams
\end{ltxexample}
%
Das Komma im ersten Beispiel kommt von \cmd{multinamedelim}, während der String
"`and"' in beiden Beispielen von \cmd{finalnamedelim} kommt. Sehen Sie
ebenfalls in  \secref{use:fmt:lng}.

\csitem{revsdnamedelim}\CSdelimMark
Ein zusätzliches Trennzeichen, welches nach dem Vornamen in einer Namensliste
ausgegeben wird, wenn der Vorname gegeben ist. Die Standardvariante ist ein
offener String, das bedeutet, es wird kein zusätzliches Trennzeichen
ausgegeben. Hier ein Beispiel, welches eine Namensliste mit einem Komma als
\cmd{revsdnamedelim} zeigt:

\begin{ltxexample}
Jones, Edward<<, and>> Joe Williams
\end{ltxexample}

In diesem Beispiel kommt das Komma nach <Edward> von \cmd{revsdnamedelim},
während die Zeichenfolge <und> das \cmd{finalnamedelim} ist, 
das zusätzlich zu dem Ersteren gedruckt wird.

\csitem{andothersdelim}\CSdelimMark
Das Trennzeichen, welches vor dem Lokalisation"string <\texttt{andothers}>
ausgegeben wird, wenn eine Namensliste wie \bibfield{author} oder
\bibfield{editor} beschränkt wird. Die Standardversion ist ein Leerzeichen. 

\csitem{multilistdelim}\CSdelimMark

Das Trennzeichen, welches zwischen multiplen Einträgen in einer Wortliste, wie
\bibfield{publisher} oder \bibfield{location}, wenn es mehr als zwei Einträge
in dieser Liste gibt. Die Standardversion ist ein Komma mit einem Leerzeichen.
Sehen Sie \cmd{multinamedelim} für weitere Erklärungen.

\csitem{finallistdelim}\CSdelimMark

Das Trennzeichen, welches anstelle von \cmd{multilistdelim} vor den letzten
Eintrag in einer Wortliste gesetzt wird. Die Standardversion ist der
Lokalisationsstring <and>, der durch Leerzeichen abgegrenzt wird. Sehen Sie
\cmd{finalnamedelim} für weitere Erklärungen.

\csitem{andmoredelim}\CSdelimMark

Das Trennzeichen, welches vor dem Lokalisierungsstring <\texttt{andmore}>
ausgegeben wird, wenn eine Wortliste wie \bibfield{publisher} oder
\bibfield{location} beschränkt wird. Die Standardversion ist ein Leerzeichen.

\csitem{multicitedelim}

Das Trennzeichen, welches zwischen Literaturverweisen gesetzt wird, wenn
multiple Eingabeschlüssel an einen einzelnen Befehl für Literaturangaben
weitergegeben wird. Die Standardversion ist ein Semikolon mit einem Leerzeichen.

\csitem{supercitedelim}

Gleicht \cmd{multicitedelim}, wird aber nur vom \cmd{supercite}-Befehl
verwendet. Die Standardversion ist ein Komma.

\csitem{compcitedelim}

Gleicht \cmd{multicitedelim}, wird aber von mehreren Stilen für
Literaturangaben zum <komprimieren> von multiplen Literaturverweisen verwendet.

\csitem{textcitedelim}
Ähnlich zu \cmd{multicitedelim}, aber mit \cmd{textcite} und zugehörigen Befehlen
zu nehmen (\secref{use:cit:cbx}). Der Standardwert ist ein Komma und ein 
Wortzwischenraum. Die Standardstile modifizieren diese vorläufige definition, um
sicherzustellen, dass das Trennzeichen vor der letzten Zitierung der lokalisierte
Term <and> ist, getrennt von Wortzwischenräumen. Sehen Sie dazu auch \cmd{finalandcomma}
und \cmd{finalandsemicolon} in \secref{use:fmt:lng}

\csitem{nametitledelim}\CSdelimMark
Das Trennzeichen zwischen Autor\slash Herausgeber und dem Titel aus Autor-Titel
und einigen anderen "`wortreichen"' Stilen für Literaturangaben. Die
Standarddefinition ist ein Komma mit einem Leerzeichen. 
Das Trennzeichen zwischen dem Autor\slash Editor und dem Titel nach dem Autortitel
und einigen ausführlichen Zitierstilen sowie in der Bibliografie gedruckt wird.
In Autor-Jahr-Bibliografiestilen wird dieses Trennzeichen nach dem Autor\slash 
Editor und dem Jahr uvor dem Titel platziert. Die Standarddefinition in Bibliografien ist der jetzt veraltete \cmd{labelnamepunct} und ist ansonsten ein Komma plus
Zwischenwort-Leerzeichen.

\csitem{namelabeldelim}\CSdelimMark
Das Trennzeichen, welches zwischen dem Nemen\slash Titel und dem Label von alphabetischen und numerischen Zitierstilen gesetzt wird. Die Standardversion ist
ein Leerzeichen.  

\csitem{nonameyeardelim}\CSdelimMark
Das Trennzeichen, welches zwischen dem Ersatz für den Labelnamen gesetzt wird, wenn er nicht vorhanden ist (in der Regel Label oder Titel in Standardstilen) und das Jahr in
den Autor-Jahr-Zitierstilen. Dies wird nur verwendet, wenn kein Label da ist, jedoch ein Labelname existiert, \cmd{nameyeardelim} wird verwendet. Die Standardversion ist
ein Leerzeichen. 

\csitem{authortypedelim}\CSdelimMark
Das Trennzeichen, das gesetzt wird zwischen dem Autor und dem \texttt{authortype}. 
Der Standardfall ist ein Komma gefolgt von einem Leerraum.

\csitem{editortypedelim}\CSdelimMark

Das Trennzeichen, das gesetzt wird zwischen dem Editor und \texttt{editor} oder \texttt{editortype}-Zeichenfolge. Der Standardfall ist ein Komma gefolgt von einem Leerraum.

\csitem{translatortypedelim}\CSdelimMark
Das Trennzeichen, das gesetzt wird zwischen dem Übersetzer und der 
\texttt{translator}Zeichenfolge. Der Standardfall ist ein Komma gefolgt von einem Leerraum.

\csitem{labelalphaothers}
Ein String, der an den nicht numerischen Teil des
\bibfield{labelalpha}-Feldes
angehängt wird (das heißt, der Feldanteil des Labels für Literaturverweise,
welches von den alphabetischen Stilen für Literaturstellen verwendet wird),
wenn die Abzahl von Autoren\slash Herausgebern \opt{maxnames}-Grenze übersteigt
oder die \bibfield{author}\slash \bibfield{editor}-Liste in der \file{bib}-Datei
durch das Schlüsselwort <\texttt{and others}> beschränkt wurde. Das ist
typischerweise eine einzelne Eigenschaft, wie ein Plussymbol oder ein Stern. Die
Standardversion ist ein Plussymbol. Dieser Befehl kann auch durch einen leeren
String neu definiert werden, um dieses Merkmal auszuschalten. In jedem Fall
muss es in der Präambel neu definiert werden.

\csitem{sortalphaothers}
Gleicht \cmd{labelalphaothers}, wird aber im Sortierprozess
verwendet. Es in einem anderen Zusammenhang zu verwenden ist angebracht, wenn
das Letztere einen Formatierungsbefehl enthält, zum Beispiel:

\begin{ltxexample}
\renewcommand*{\labelalphaothers}{\textbf{+}}
\renewcommand*{\sortalphaothers}{+}
\end{ltxexample}
%
Wenn \cmd{sortalphaothers} nicht neu definiert wird, wird er zu
\cmd{labelalphaothers}.

\csitem{volcitedelim}
Das Trennzeichen zwischen dem Inhaltsteil und dem Seiten/Textteil von
\cmd{volcite} und zugehörigen Befehlen (\secref{use:cit:spc}).

\cmditem{mkvolcitenote}{volume}{pages}

Dieses Makro formatiert die Argumente \prm{volume} und \prm{pages} von \cmd{volcite} und 
zugehörigen Befehlen (\secref{use:cit:spc}), wenn sie an den zugrunde liegenden Zitierbefehl übergeben werden.

\csitem{prenotedelim}\CSdelimMark
Das Trennzeichen, welches nach dem \prm{prenote}-Argument eines
Literaturstellen-Befehls ausgegeben wird. Sehen Sie \secref{use:cit} für Details.
Die Voreinstellung ist ein Leerzeichen.

\csitem{postnotedelim}\CSdelimMark
Das Trennzeichen, welches nach dem \prm{postnote}-Argument eines
Literaturstellen-Befehls ausgegeben wird. Sehen Sie \secref{use:cit} für Details.

\csitem{extpostnotedelim}\CSdelimMark
Das Trennzeichen, das zwischen dem Zitat und dem Argument \prm{postnote}eines Zitierbefehls in Klammern ausgegeben wird, wenn die "`postnote"' außerhalb der Zitierklammern steht. In den
Standardstilen tritt dies auf, wenn das Zitat das "`shorthand"'-Feld verwendet. In \secref{use:cit} finden sie weitere Details. Standadwert ist Wortzwischenraum.

\csitem{multiprenotedelim}\CSdelimMark
Das Trennzeichen, das nach dem \prm{multiprenote}-Argument eines Zitierbefehls ausgegen wird. In \secref{use:cit} mehr Details. Der Standard ist \cs{prenotedelim}.

\csitem{multipostnotedelim}\CSdelimMark
Das Trennzeichenter, das vor dem \prm{multipostnote}-Argument eines Zitierbefehls ausgegeben
wird. In \secref{use:cit} mehr Details. Der Standard ist \cs{postnotedelim}.

\cmditem{mkbibname<namepart>}{text}
Dieser Befehl, der ein Argument nimmt, wird verwendet, um den Namensteil
<namepart> von Namenslistenfeldern zu formatiere. Das Standarddatenmodell definiert die
Namensteile <family>, <given>, <prefix> und <suffix> und deshalb sind die folgenden
Makros automatisch definiert:

\begin{ltxexample}
\makebibnamefamily
\makebibnamegiven
\makebibnameprefix
\makebibnamesuffix
\end{ltxexample}
%
Für die Rückwärtskompatibilität mit den ererbten \bibtex-Namensteilen sind die folgenden auch definiert:

\begin{ltxexample}
\makebibnamelast
\makebibnamefirst
\makebibnameaffix
\end{ltxexample}

\csitem{mkbibcompletenamefamily}{text}
Dieser Befehl, der ein Argument nimmt, wird für das Format des vollständigen Namens in der \texttt{family}-Formatreihenfolge verwendet.

\csitem{mkbibcompletenamefamilygiven}{text}
Dieser Befehl, der ein Argument nimmt, wird für das Format des vollständigen Namens in der \texttt{family-given}-Formatreihenfolge verwendet.

\csitem{mkbibcompletenamegivenfamily}{text}
Dieser Befehl, der ein Argument nimmt, wird für das Format des vollständigen Namens in der \texttt{given-family}-Formatreihenfolge verwendet.

\cmditem{mkbibcompletename}{text}
Der Anfangswert aller Standardformatiershooks \texttt{mkbibcompletename<formatorder>}.

\csitem{datecircadelim}\CSdelimMark

Wenn die Datenformatierung mit der globalen Option \opt{datecirca} aktiviert ist,
wird das Trennzeichen nach dem lokalen <circa>-Term ausgegeben. Standards beim Wortzwischenraum.

\csitem{dateeradelim}\CSdelimMark

Wenn die Datenformatierung mit der globalen Option
\opt{dateera} aktiviert ist, wird das Trennzeichen vor dem lokalen era-Term ausgegeben. 
Standards beim Wortzwischenraum.

\csitem{dateuncertainprint}
Gibt Datenunsicherheitsinformationen aus, wenn die globale Option \opt{dateuncertain}
aktiviert ist. Standardmäßig gibt es die sprachenspezifische Zeichenfolge \cmd{bibdateuncertain} aus (\secref{use:fmt:lng}).

\csitem{datecircaprint}
Gibt Datenunsicherheitsinformationen aus, wenn die globale Option \opt{datecirca}
aktiviert ist. Standardmäßig gibt es den <circa> lockalen Term aus 
%(\secref{aut:lng:key:dt}) 
und den \opt{datecircadelim} Trenner.

\csitem{enddateuncertainprint}
Druckt Unsicherheiten beim Datumsfeld, wenn die globale Option \opt{dateuncertain} aktiviert ist und der Test \cmd{ifenddateuncertain} stimmt. Standardmäßig wird die sprachenspezifische Zeichenfolge \cmd{bibdateuncertain} ausgegeben (\secref{use:fmt:lng}).

\csitem{enddatecircaprint}
Druckt circa Dateninformationen, wenn die globale Option \opt{datecirca} aktiviert ist und
der \cmd{ifenddatecirca}-Test stimmt. Standardmäßig wird das <circa> als lokalisierter Term (%\secref{aut:lng:key:dt}
§ 4.9.2.21 engl. V.) und das \opt{datecircadelim}-Trennzeichen ausgegeben.

\csitem{datecircaprintiso}
Druckt das \acr{ISO8601-2}-Format der circa-Dateninformation, wenn die globale Option \opt{datecirca} aktiviert ist und der \cmd{ifdatecirca}-Test stimmt. Gibt \cmd{textasciitilde}
aus.

\csitem{enddatecircaprintiso}
Druckt das \acr{ISO8601-2}-Format der circa-Dateninformatin, wenn die globale Option \opt{datecirca} aktiviert ist und der \cmd{ifenddatecirca}-Test stimmt. Gibt \cmd{textasciitilde} aus.

\csitem{dateeraprint}{yearfield}
gibt die Ära-Information aus, wenn die globale Option \opt{dateera} auf <secular> oder <christian> gesetzt ist. Standardmäßig wird der \opt{dateeradelim}-Trenner und der
der entsprechend loklisierte Ärebegriff ausgegeben (%\secref{aut:lng:key:dt}
§ 4.9.2.21 eng. V.). Wenn die \opt{dateeraauto}-Option gesetzt ist, dann wird das \prm{yearfield} getestet (das ist
der Name eines Jahrfeldes, sowas wie <year>, <origyear>, <endeventyear> etc.), um zu sehen,
ob sein Wert früher als die \opt{dateeraauto}-Schwelle ist und wenn ja, dann wird die 
BCE/CE-Lokaliserung geschlossen sein. Die Standardeinstellung für
\opt{dateeraauto} ist 0 
und so sind nur BCE/BC-Localisationsstrings Ausgabekandidaten. Ist ermittelt, ob
der Beginn oder das Ende eine Ärainformation auszugeben ist, wird der 
\prm{yearfield}-Name übergeben.

\csitem{dateeraprintpre}
Gibt Datumsangaben aus, wenn die globale Option \opt{dateera} auf <astronomical> gesetzt ist. 
Standardmäßig druckt dies \opt{bibdataeraprefix}. Es wird anhand des übergebenen Namens \prm{yearfield} erkannt, ob die Informationen zu Beginn oder Ende des Jahres ausgegeben werden sollen.

\csitem{relatedpunct}
Der Separator zwischen dem  \bibfield{relatedtype}-Bibliografie-Lokalisierungsstring und den Daten aus dem ersten diesbezüglichen Eintrag.
Hier ist ein Beispiel, mit einem \cmd{relatedpunct} in einen Gedankenstrich gesetzt:

\begin{ltxexample}
A. Smith. Title. 2000, (Orig. pub. as<<->>Origtitel)
\end{ltxexample}

\csitem{relateddelim}
Der Separator zwischen den Daten von mehreren verwandten Einträgen. Die Standarddefinition
ist ein optionaler Punkt plus Zeilenumbruch. Hier ist ein Beispiel, in dem die Bände
A-E verbundene Einträge von 5 Bänden der Hauparbeit sind:

\begin{ltxexample}
Donald E. Knuth. Computers & Typesetting. 5 vols. Reading, Mass.: Addison-
Wesley, 1984-1986.
Vol. A: The TEXbook. 1984.
Vol. B: TEX: The Program. 1986.
Vol. C: The METAFONTbook. By. 1986.
Vol. D: METAFONT: The Program. 1986.
Vol. E: Computer Modern Typefaces. 1986.
\end{ltxexample}

\csitem{relateddelim$<$relatedtype$>$}
Ein Trennzeichen zwischen den Daten mehrer verwandter Einträge vom Typ <relatedtype>. 
Es gibt dazu keine Standardeinstellung. Wenn ein solches typspezifisches Trennzeichen nicht vorhanden ist, wird \cmd{relateddelim} verwendet.

\csitem{begrelateddelim}
Ein generisches Trennzeichen vor dem Block verwandter Einträge. Die Standarddefinition \cmd{newunitpunct}.

\csitem{begrelateddelim$<$relatedtype$>$}
Ein Trennzeichen zwischen dem Block verwandter Einträge vom Typ <relatedtype>. Es gibt dafür
keine Standardeinstellung. Wenn ein solches typspezifisches Trennzeichen nicht vorhanden
ist, wird \cmd{relateddelim} verwendet.

\end{ltxsyntax}

\subsubsection{Kontext-sensitive Trennzeichen}
\label{use:fmt:csd}

Die Trennzeichen, beschrieben in \secref{use:fmt:fmt}, sind global definiert. Das
heißt, unabhängig davon, wo Sie diese verwenden, sie geben die gleiche Sache aus. 
Dies ist für Begrenzer nicht unbedingt wünschenswert, die Sie für verschiedene Dinge in unterschiedlichen Kontexten ausgeben möchten. Hier bedeutet <context>, Dinge
wie like <inside a text citation> oder <inside a bibliography item>. Aus diesem Grund 
bietet \biblatex eine anspruchsvollere Begrenzerspezifikation und eine Benutzeroberfläche
neben dem Standard,
basierend auf den normalen Makros, mit \cmd{newcommand} definiert.

\begin{ltxsyntax}

\cmditem{DeclareDelimFormat}[context, \dots]{name, \dots}{code}
\cmditem*{DeclareDelimFormat}*[context, \dots]{name, \dots}{code}

Deklariert die Begrenzermakros in Komma-separierten Namenslisten (\prm{names}) mit dem
"`replacement test"' \prm{code}. Wenn die optionale Komma-separierte Kontextliste  (\prm{contexts}) gegeben ist, deklarieren Sie die Namen (\prm{names}) nur für diese Kontexte.  \prm{names} ohne diese \prm{contexts} verhalten sich genauso wie die globalen Begrenzungsdefinitionen, mit \cmd{newcommand} gegeben -- nur ein einfaches Mokro
mit einem Ersatz/replacement, das als \cmd{name} verwendet werden kann. Sie können jedoch auch die auf diese Weise erzeugten Begrenzermakros aufrufen, indem Sie 
\cmd{printdelim} nehmen, welches kontextbewusst ist. Die gesternte Version löscht alle
kontext(\prm{context})spezischen Deklarationen für alle Namen (\prm{names}) zuerst. 

\cmditem{DeclareDelimAlias}{alias}{delim}
\cmditem*{DeclareDelimAlias}*[alias context, \dots]{alias}[delim context]{delim}

Deklariert \prm{alias} als alias für das Trennzeichen \prm{delim}. Die Zuweisung wird für
alle vorhandenen Kontexte des Ziels \prm{delim} durchgeführt.
Die markierte (starred) Version alias nur füe spezifische Kontexte zu. Das erste 
optionale Argument \prm{alias context} enthält eine Liste von Kontexten, für die die Zuweisung ausgeführt werden soll. Wenn es leer ist oder fehlt, wird der globale/leere Kontext
angenommen. Das zweite optionale Argument \prm{delim context} gibt den Kontext 
des Zieltrennzeichens an. Dieses Argument ist möglicherweise keine Liste, sonder kann nur 
einen Kontext enthalten. Wenn es fehlt, wird der \prm{alias context} angenommen 
(wenn \prm{alias context} ein Kontext ist, wird die Zuweisung für jedes Listenelement 
separat ausgeführt), wenn es leer ist, wird der globale Kontext verwendet.

\begin{ltxexample}[style=latex]
\DeclareDelimAlias*[bib,biblist]{finalnamedelim}[]{multinamedelim}
\end{ltxexample}
%
andererseits
\begin{ltxexample}[style=latex]
\DeclareDelimAlias*[bib,biblist]{finalnamedelim}{multinamedelim}
\end{ltxexample}
%
definiert \cmd{finalnamedelim} im Kontext \opt{bib} eines alias von \cmd{multinamedelim} im \opt{bib}-Kontext und definiert \cmd{finalnamedelim} im \opt{biblist}-Kontext als alias von \cmd{multinamedelim} in \opt{biblist}.

\cmditem{printdelim}[context]{name}

Gibt ein Trennzeichen mit dem Namen \prm{name} aus, lokal einen optionalen Kontext  (\prm{context}) zuerst festlegend. Ohne optional \prm{context}, \cmd{printdelim} nimmt
es den aktuell aktiven Begrenzerkontext.

Trennzeichenkontexte sind einfach ein String, der Wert des internen Makros
\cmd{blx@delimcontext}, welcher manuell mit dem Befehl \cmd{delimcontext}
eingestellt werden kann.

\cmditem{delimcontext}{context}

Setzt den Begrenzerkontext ein (\prm{context}). Diese Einstellung ist nicht global,
so dass Trennzeichenkontexte verschachtelt werden können, nehmend dafür die üblichen
 \latex-Grupppen-Verfahren.

 \cmditem{DeclareDelimcontextAlias}{alias}{name}

Das kontext-sensitive Trennsystem schafft Begrenzerkontexte, die auf dem Namen
der Zitierbefehle basieren (<parencite>, <textcite> etc.), übergebend dem 
\cmd{DeclareCiteCommand}. In bestimmten Fällen, in denen geschachtelte Definitionen 
von Zitierbefehlen, wobei der Befehl \cmd{DeclareCiteCommand} sich selbst aufruft
(sehen Sie die Definition vo \cmd{textcite} in \sty{authoryear-icomp an}
beispielsweise). Der Begrenzerkontext ist dann in der Regel inkorrekt und die
Begrenzerspezifikation  funktioniert nicht. Beispielsweise, die Definition von
\cmd{textcite} ist tatsächlich definiert und verwendet \cmd{cbx@textcite}, und so 
ist der Kontext automatisch auf \opt{cbx@textcite} gesetzt, wenn die Zitierung ausgegeben wird. Begrenzerdefinitionen erwarten den Kontext \opt{textcite} zu sehen, deshalb
funktionieren sie nicht. Deshalb wird dieser Befehl für Stilautoren zur Verfügung gestellt,
für Aliase der Kontext \prm{alias} zum Kontext \prm{name}. Zum Beispiel:

\begin{ltxexample}[style=latex]{}
\DeclareDelimcontextAlias{cbx@textcite}{textcite}
\end{ltxexample}
% 
Das (dies ist die Standardeinstellung) stellt es sicher, wenn innerhalb des Zitierbefehls
\cmd{cbx@textcite} der Kontext tatsächlich, wie erwartet, \opt{textcite} ist.

\end{ltxsyntax}
%
\biblatex\ hat mehrere Standardkontexte, welche automatisch an verschiedenen Stellen
festgelegt sind:

\begin{description}
\item[none] Zu Dokumentbeginn.
\item[bib] Innerhalb einer Bibliografie, begonnen mit 
\cmd{printbibliography} oder in einem \cmd{usedriver}. 
\item[biblist]  Innerhalb eine Bibliografieliste, begonnen mit \cmd{printbiblist}.
\item[<citecommand>] Innerhalb eines Zitierbefehls \cmd{citecommand}, festgelegt mit
einem  \cmd{DeclareCiteCommand}.
\end{description}

Beispielsweise sind die Standards für \cmd{nametitledelim}:

\begin{ltxexample}[style=latex]{}
\DeclareDelimFormat{nametitledelim}{\addcomma\space}
\DeclareDelimFormat[textcite]{nametitledelim}{\addspace}
\end{ltxexample}
%
Dies bedeutet, dass \cmd{nametitledelim} global als  <\cmd{addcomma}\cmd{space}> definiert ist, so wie per Standardtrennzeicheninterface. Jedoch kann zusätzlich das Trennzeichen mit  \cmd{printdelim} ausgegeben werden, das das Gleiche wie  \cmd{nametitledelim} ausgeben 
würde, außer in einem \cmd{textcite}, indem es \cmd{addspace} ausgeben würde, das mehr
geeignet ist, für die Textkompilierung. Falls gewünscht, kann ein Zusammenhang mit dem optionalen Argument  mit \cmd{printdelim} erzwungen werden, so

\begin{ltxexample}[style=latex]{}
\printdelim[textcite]{nametitledelim}
\end{ltxexample}
%
Würde \cmd{addspace} unabhängig vom Kontext des \cmd{printdelim} ausgeben. Kontexte
sind nur willkürliche Zeichenfolgen und so können Sie welche jederzeit einführen, nehmend \cmd{delimcontext}. Wenn \cmd{printdelim} für den Begrenzer (\prm{name}) keinen besonderen 
Wert im aktuellen Kontext findet, ist es einfach \cmd{name} auszugeben. Dies bedeutet,
dass Stilautoren \cmd{printdelim} nehmen können und Benutzer können erwarten, mit \cmd{renewcommand} Trennzeichen definieren zu können, mit einer Einschränkung -- eine solche Definition wird keine Trennzeichen ändern, die sind festgelegt.

\begin{ltxexample}[style=latex]{}
\DeclareDelimFormat{delima}{X}
\DeclareDelimFormat[textcite]{delima}{Y}
\renewcommand*{delima}{Z}
\end{ltxexample}
%
Hier, \cmd{delima} druckt immer <Z>. \verb+\printdelim{delima}+ in jedem Kontext, anders 
als  <textcite>, das auch \cmd{delima} und hence <Z> ausgibt, aber in einem
<textcite>-Kontext <Y> ausgibt. Sehen Sie die \file{04-delimiters.tex}-Beispieldatei, die
mit \biblatex\ arbeitet, für weitere Informationen.


\subsubsection{Sprachspezifische Befehle} \label{use:fmt:lng}

Die Befehle in diesem Abschnitt sind sprachspezifisch. Wenn man sie
neu definiert, muss man die neue Definition in einen
\cmd{DeclareBibliographyExtras}-Befehl einbetten (in eine \file{.lbx}-Datei) oder einen \cmd{DefineBibliographyExtras}-Befehl (Nutzerdokumente), sehen Sie \secref{use:lng} für
Details. Man beachte, dass alle Befehle, die mit \cmd{mk\dots} beginnen, ein
oder mehr Argumente benötigen.

\begin{ltxsyntax}

\csitem{bibrangedash}

Der sprachspezifische Bindestrich, der für die Abgrenzung von Zahlen verwendet
wird. Standards sind zu \cmd{textendash}.

\csitem{bibrangessep}

Der sprachspezifische Separator, der zwischen mehreren Bereichen verwendet wird. 
Standard: Ein Komma folgt einem Leerraum.

\csitem{bibdatedash}

Der sprachspezifische Bindestrich, der für die Abgrenzung von Datumsangaben
verwendet wird. Standards sind zu \cmd{hyphen}.

\csitem{bibdaterangesep}

Der sprachspezifische Separator, verwendbar bei Datumsbereichen. Standard zu
\cmd{textendash} für alle Datumsformate, außer \opt{ymd}, die standardmäßig sind mit 
einem \cmd{slash}. Die Datumsformatoption \opt{edtf} ist hartcodiert auf \cmd{slash},
da dies ein standardkonformes Format ist.

\csitem{mkbibdatelong}

Nimmt die Namen der drei Felder als Argumente, die sich auf die drei
Datumskomponenten beziehen (in der Reihenfolge Jahr\slash Monat\slash Tag) und
seine Wirkung dafür verwendet, das Datum im sprachspezifischen ausgeschrieben
Format auszugeben.

\csitem{mkbibdateshort}

Gleicht \cmd{mkbibdatelong}, verwendet aber die kurze Version des
sprachspezifischen Datumsformats.

\csitem{mkbibtimezone}

Modifiziert einen Zeitzonenstring, als ein einzelnes Argument übergeben. Standardmäßig
ändert diess <Z> auf den Wert von \cmd{bibtimezone}.

\csitem{bibdateuncertain}

Die sprachspezifischen Marker zur Verwendung nach unsicheren Daten, wenn die globale
Option \opt{dateuncertain} aktiviert ist. Standardmäßig ein Leerraum, gefolgt von einem Fragezeichen.

\csitem{bibdateeraprefix}

Ein sprachspezifischer Marker, der als ein Präfix zu Beginn von  BCE/BC-Daten, in einem
Datumsbereich, wenn die Option \opt{dateera} auf <astronomical> gesetzt ist. Standardmäßig auf \cmd{textminus}, wenn definiert und \cmd{textendash}.

\csitem{bibdateeraendprefix}

Ein sprachspezifischer Marker, der als ein Präfix am Ende von BCE/BC Daten, in einem
Datumsbereich, wenn die Option \opt{dateera} auf <astronomical> gesetzt ist. Standardmäßig ein kleiner Leerraum (thin space), gefolgt von \cmd{bibdateeraprefix} wenn \cmd{bibdaterangesep} auf eine Bindestrich (dash) und auf \cmd{bibdateeraprefix} gesetzt ist.  Dies ist ein separates Makro, so dass Sie mehr Raum vor einen negativen
Datumsmarker hinzufügen können, beispielsweise folgt ein Bindestrich einem Datumsbereichmarker, so kann dies ein wenig seltsam aussehen.

\csitem{bibtimesep}

Der sprachspezifische Marker, der Zeitkomponenten trennt. Der Standardwert ist ein Doppelpunkt.

\csitem{bibtimezonesep}

Der sprachspezifische Marker, der eine optionale zeitkomponente von einer Zeit trennt.
Ist standardmäßig leer.

\csitem{bibtzminsep}

Der sprachspezifische Marker, der die Stunden- und Minutenkomponente der Offsetzeitzonen trennt. Der Standardwert ist ein \cmd{bibtimesep}.

\csitem{bibdatetimesep}

Der sprachspezifische Separator, der zwischen Datums- und Zeitkomponenten ausgegeben wird,
ausgegeben wird, wenn Zeitkomponenten mit Datumskomponenten ausgegeben werden (sehen Sie \opt{$<$datetype$>$dateusetime}-Option in \secref{use:opt:pre:gen} an).  Defaults zu einem
Leerraum für nicht-EDTF-Ausgabeformate und 'T' für das EDTF-Ausgabeformat.

\csitem{finalandcomma}

Gibt die Kommata, die vor dem finalen <and> eingefügt wurden, in einer Liste
aus, wenn möglich in der jeweiligen Sprache. Hier ein Beispiel:

\begin{ltxexample}
Michel Goossens, Frank Mittelbach<<,>> and Alexander Samarin
\end{ltxexample}
%
\cmd{finalandcomma} ist das Komma vor dem Wort <and>. Siehe ebenfalls
\cmd{multinamedelim}, \cmd{finalnamedelim} und \cmd{revsdnamedelim} in
\secref{use:fmt:fmt}.

\csitem{finalandsemicolon}

Gibt das Semikolon entgültig aus, vor dem entgültigen <and> in einer Liste von Listen, ggf. 
in der jeweiligen Sprache. Hier ist ein Beispiel:

\begin{ltxexample}
Goossens, Mittelbach, and Samarin; Bertram and Wenworth<<;>> and Knuth
\end{ltxexample}
%
\cmd{finalandsemicolon} ist das Semikolon vor dem Wort <and>. Sehen Sie auch \cmd{textcitedelim} in \secref{use:fmt:fmt}.

\cmditem{mkbibordinal}{integer}

Dieser Befehl, welcher einen ganzzahligen Wert als Argument beinhaltet, gibt
eine normale Zahl aus.

\cmditem{mkbibmascord}{integer}

Gleicht \cmd{mkbibordinal}, gibt aber eine männliche Ordnungszahl aus, wenn
dies in der jeweiligen Sprache anwendbar ist.

\cmditem{mkbibfemord}{integer}

Gleicht \cmd{mkbibordinal}, gibt aber eine weibliche Ordnungszahl aus, wenn dies
in der jeweiligen Sprache anwendbar ist.

\cmditem{mkbibordedition}{integer}

Gleicht \cmd{mkbibordinal}, ist aber für die Verwendung mit dem Begriff
<Ausgabe> vorgesehen.

\cmditem{mkbibordseries}{integer}

Ähnlich mit \cmd{mkbibordinal}, ist aber für die Verwendung mit dem Begriff <series>.

\end{ltxsyntax}

\subsubsection{Längen und Zähler} \label{use:fmt:len}

Die Längenregister und Zähler in diesem Abschnitt können gegebenenfalls in
der Präambel mit \cmd{setlength} und \cmd{setcounter} geändert werden.

\begin{ltxsyntax}

\lenitem{bibhang}

Die hängende Einrückung der Bibliografie, wenn angemessen. Diese Länge wird
während der Ladezeit auf \cmd{parindent} initialisiert. 

\lenitem{biblabelsep}

Der Zeichenabstand zwischen Einträgen und ihren zugehörigen Labels in der
Bibliografie. Es findet nur bei Bibliografiestilen Anwendung, die Kennsätze
ausgeben, so wie die \texttt{numeric}- und \texttt{alphabetic}-Stile. Diese
Länge wird durch die Wirkung von \cmd{labelsep} während der Ladezeit doppelt
initialisiert.

\lenitem{bibitemsep}

Der Zeilenabstand zwischen den einzelnen Einträgen in der Bibliografie. Diese
Länge wird während der Ladezeit auf \cmd{itemsep} initialisiert. Man beachte,
dass \len{bibitemsep}, \len{bibnamesep} und \len{bibinitsep} kumulativ sind.
Sollten sie sich überschneiden, wird der Befehl mit der größten Wirkung
ausgeführt.

\lenitem{bibnamesep}

Der Zeilenabstand, der zwischen zwei Einträgen der Bibliografie eingefügt
wird, wann immer ein Eintrag mit einem Namen beginnt, der anders ist als der des
vorherigen Eintrags. Die Voreinstellung ist null. Setzt man die Länge auf einen
positiven Wert, der größer ist als \len{bibitemsep}, wird die Bibliografie
durch die Autoren-\slash Herausgebernamen gruppiert. Man beachte, dass
\len{bibitemsep}, \len{bibnamesep} und \len{bibinitsep} kumulativ sind. Sollten
sie sich überschneiden, wird der Befehl mit der größten Wirkung ausgeführt.

\lenitem{bibinitsep}

Der Zeilenabstand, der zwischen zwei Einträgen der Bibliografie eingefügt
wird, wann immer ein Eintrag mit einem Buchstaben beginnt, der anders ist als
der des vorherigen Eintrags. Die Voreinstellung ist null. Setzt man die Länge
auf einen positiven Wert, der größer ist als \len{bibitemsep}, wird die
Bibliografie alphabetisch gruppiert. Man beachte, dass \len{bibitemsep},
\len{bibnamesep} und \len{bibinitsep} kumulativ sind. Sollten sie sich
überschneiden, wird der Befehl mit der größten Wirkung ausgeführt.

\lenitem{bibparsep}

Der Zeilenabstand zwischen Paragrafen innerhalb eines Eintrags in der
Bibliografie. Der Standardwert ist null.

\cntitem{abbrvpenalty}

Dieser Zähler, der von Lokalisierungsmodulen verwendet wird, beinhaltet die
Strafe, die in kurzen oder gekürzten Strings verwendet wird. Zum Beispiel ist
ein Zeilenumbruch bei Ausdrücken wie «et al.» oder «ed. by» ungünstig, es
sollte aber dennoch möglich sein, überfüllte Zellen zu vermeiden. Dieser
Zähler wird während der Ladezeit auf \cmd{hyphenpenalty} initialisiert. Diese
Idee zwingt \tex dazu, den gesamten Ausdruck als ein einzelnes, durch
Bindestrich trennbares Wort anzusehen, soweit es von einem Zeilenumbruch
betroffen ist. Wenn man solche Zeilenumbrüche nicht mag, kann man einen höheren
Wert verwenden. Wenn es einem egal ist, setzt man den Zähler auf null. Wenn man
es unbedingt unterdrücken möchte, setzt man ihn auf <unendlich> (10\,000 oder
höher).\footnote{Die Standardwerte, die \cnt{abbrvpenalty},
\cnt{lownamepenalty} und \cnt{highnamepenalty} zugewiesen werden, sind
absichtlich sehr niedrig, um überfüllte Zellen zu vermeiden. Das bedeutet,
dass man kaum eine Auswirkung von Zeilenumbrüchen feststellen kann, wenn der Text im
Blocksatz gesetzt ist. Wenn man diese Zähler auf 10\,000 setzt, um eventuelle
Trennstellen zu unterdrücken, wird man die Auswirkungen sehen können, kann
aber auch mit überfüllten Zellen konfrontiert werden. Man sollte bedenken,
dass Zeilenumbrüche in der Bibliografie oft komplizierter als im Text sind und
man nicht darauf zugreifen kann, einen Satz umzuformulieren. In manchen Fällen
kann es wünschenswert sein, die gesamte Bibliografie in \cmd{raggedright} zu
setzen, um suboptimale Zeilenumbrüche zu vermeiden. In diesem Fall kann auch
die kleinste Standardstrafe einen sichtbaren Unterschied zeigen.}

\cntitem{highnamepenalty}

Dieser Zähler beinhaltet eine Strafe, welche den Zeilenumbruch bei Namen
beeinflusst. In \secref{use:cav:nam,use:fmt:fmt} ist eine Erklärung zu finden.
Dieser Zähler wird während der Ladezeit auf \cmd{hyphenpenalty} initialisiert.
Wenn man solche Zeilenumbrüche nicht mag, kann man einen höheren Wert
verwenden. Wenn es einem egal ist, setzt man den Zähler auf null.Wenn man das
traditionelle \bibtex-Verhalten bevorzugt (keine Zeilenumbrüche bei
\cnt{highnamepenalty}-Trennstellen), setzt man ihn auf "`unendlich"' (10\,000 oder
höher).

\cntitem{lownamepenalty}

Gleicht \cnt{highnamepenalty}. Erklärungen finden sich in
\secref{use:cav:nam,use:fmt:fmt}. Dieser Zähler wird während der Ladezeit zur
Hälfte auf \cmd{hyphenpenalty} initialisiert. Wenn man solche Zeilenumbrüche
nicht mag, kann man einen höheren Wert verwenden. Wenn es einem egal ist, setzt
man den Zähler auf null. 

Wenn dieser Zähler auf einen Wert größer als Null gesetzt wird, lässt
\biblatex Zeilenumbrüche nach Zahlen in allen Zeichenfolgen zu, die mit dem Befehl 
\cmd{url} des {url}-Paket formatiert wurden. Dies wirkt sich auf \acr{url}s und 
\acr{doi}s in der Bibliografie aus. Die Umbruchpunkte werden mit dem Wert dieses Zählers
betraft (penalized). Wenn \acr{url}s und/oder \acr{doi}s in der Bibliografie in den Rand gelangen, versuchen sie, diesen Zähler auf einen Wert größer größer als Null, aber kleiner als  10000 zu setzen (normalerweise ist ein Wert wie 9000 gewünscht). Wenn sie den Zähler auf Null setzen, wird diese Funktion deaktiviert. Dies ist die Standardeonstellung.

\cntitem{biburlucpenalty}

Ähnlich wie \cnt{biburlnumpenalty}, außer dass nach allen Großbuchstaben ein Umbruchpunkt
hinzugefügt wird.

\cntitem{biburllcpenalty}

Ähnlich wie \cnt{biburlnumpenalty}, außer dass nach allen Kleinbuchstaben ein Umbruchpunkt
hinzugefügt wird.

\cntitem{biburlbigbreakpenalty}

Die \sty{biblatex}-Version des \sty{url}'s \len{UrlBigBreakPenalty}. Der Standardwert
ist \texttt{100}.

\cntitem{biburlbreakpenalty}

Die \sty{biblatex}-Version des \sty{url}'s \len{UrlBreakPenalty}. Der Standardwert ist \texttt{200}.

\lenitem{biburlbigskip}

Die \sty{biblatex}-Version des \len{Urlmuskip}. Der Wert enthält einen zusätzlichen
(dehnbaren) Platz, der um zerbrechliche Zeichen im Befehl \cmd{url} des Pakets \sty{url} 
eingefügt wird. Der Standardwert ist \texttt{0mu plus 3mu}.

\lenitem{biburlnumskip}

Das zusätzliche Leerzeichen, das nach Zahlen in Zeichenfolgen eingefügt wird, die mit dem Befehl \cmd{url} des  \sty{url}-Pakets formatiert wurden. Dies wirkt sich auf \acr{url}s und \acr{doi}s in der Bibliografie aus. Wenn \acr{url}s und/oder \acr{doi}s in der Bibliografie im
Rand angezeigt werden, kann es hilfreich sein, diese Länge so einzustellen, dass ein kleiner dehnbarere Speicherpltz hinzugefügt wird, beispielsweise \texttt{0mu plus 1mu}. Die
Standardeinstellung ist \texttt{0mu}. Dieser Wert wird nur verwendet, wenn
\cnt{biburlnumpenalty} auf einen anderen Wert als Null gesetzt ist.

\lenitem{biburlucskip}

Ähnlich wie \cnt{biburlnumskip}, außer dass nach Großbuchstaben Leerzeichen hinzugefügt werden.

\end{ltxsyntax}

\subsubsection{Mehrzweckbefehle} \label{use:fmt:aux}

Die Befehle dieses Abschnitts sind multifunktionale Befehle, die generell
verfügbar sind, nicht nur für Zitierungen und die Bibliografie.

\begin{ltxsyntax}

\csitem{bibellipsis}

Ein elliptisches Symbol mit Klammern: <[\dots\unkern]>.

\csitem{noligature}

Deaktiviert Ligaturen an dieser Stelle und fügt etwas Leerraum ein. Mit
diesem Befehl können Standardligaturen wie <fi> and <fl> aufgebrochen
werden. Es ist mit \verb+"|+ vergleichbar, einer Kurzform, die von den
\sty{babel}/\sty{polyglossia}-Paketen für einige Sprachmodule zur 
Verfügung gestellt wird.

\csitem{hyphenate}

Ein abhängiger Trennstrich. Im Gegensatz zum Standardbefehl \cmd{-}, der
die Silbentrennung für den Rest des Wortes ermöglicht. Es ist vergleichbar
mit der Kurzform \verb|"-|, die von den
\sty{babel}/\sty{polyglossia}-Pakete für einige Sprachmodule zur 
Verfügung gestellt wird.

\csitem{hyphen}

Ein expliziter, zerbrechlicher Bindestrich in Komposita. Im Kontrast zu
einem literalen <\texttt{-}> ermöglicht dieser Befehl die Silbentrennung im
Wortrest. Es ist vergleichbar mit der Kurzform \verb|"=|,  die von den
\sty{babel}/\sty{polyglossia}-Paketen für einige Sprachmodule zur Verfügung gestellt wird.

\csitem{nbhyphen}

Ein expliziter, nicht-zerbrechlicher Bindestrich in Komposita. Im Kontrast zu
einem literalen <\texttt{-}> erlaubt dieser Befehl keine Zeilenumbrüche am
Bindestrich, jedoch noch die Silbentrennung im Wortrest.  Es ist vergleichbar mit der 
Kurzform \verb|"~|,  die von den
\sty{babel}/\sty{polyglossia}-Paketen für einige Sprachmodule zur Verfügung gestellt wird.

\csitem{nohyphenation}

Ein allgemeiner Schalter, der die Silbentrennung lokal unterdrückt. Er
sollte in der Regel auf eine Gruppe beschränkt werden.

\cmditem{textnohyphenation}{text}

Ähnlich zu \cmd{nohyphenation}, jedoch beschränkt auf das \prm{text}-Argument.

\cmditem{mknumalph}{integer}

Nimmt eine ganze Zahl aus dem Bereich 1--702 als Argument und wandelt sie in
einen String, wie die folgenden, um:
1=a, \textellipsis, 26=z, 27=aa, \textellipsis, 702=zz. Dies ist zur
Verwendung für Formatierungsrichtlinien in den \bibfield{extrayear}- und
\bibfield{extraalpha}-Feldern.

\cmditem{mkbibacro}{text}

Allgemeiner Befehl, der ein Akronym  für die \emph{small
caps}-Variante der aktuellen Schrift nimmt, falls verfügbar, und wie anderweilig.
Die Abkürzung sollte in Großbuchstaben eingegeben werden.

\cmditem{autocap}{character}

Automatische Konvertierung von \prm{character} in seine Großform, wenn der
Satzzeichen-Tracker von \sty{biblatex} einen lokalen Groß-String
nehmen  würde . Dieser Befehl ist robust und nützlich für abhängige Großschreibung
von bestimmten Zeichenketten in einem Eintrag. Beachten Sie, dass das 
\prm{character}-Argument ein einzelnes Zeichen ist, das in Kleinbuchstaben
einzugeben ist. Zum Beispiel:

\begin{ltxexample}
\autocap{s}pecial issue
\end{ltxexample}
%
Es wird <Special issue> oder <special issue>, als Passendes. Wenn die
Zeichenfolge mit Großschreibung startet, wird ein flektiertes Zeichen in
Ascii-Notation ausgegeben, einschließend den zugehörigen Akzentbefehl für das
\prm{character}-Argument, wie nachfolgend:

\begin{ltxexample}
\autocap{\'e}dition sp\'eciale
\end{ltxexample}
%
Dies wird <Édition spéciale> oder <édition spéciale> liefern. Wenn die Zeichenfolge
mit einem Befehl am beginn aktiviert wird, dann gibt es sowas wie \cmd{ae} oder
\cmd{oe} aus, setzen Sie einfach den Befehl in das \prm{character}-Argument:

\begin{ltxexample}
\autocap{\oe}uvres
\end{ltxexample}
%
Daraus ergibt sich '\OE uvres' oder '\oe uvres'.

\end{ltxsyntax}

\subsection[Anmerkungen zu Sprachen]{Sprachspezifische Anmerkungen} \label{use:loc}

Die Möglichkeiten, die in diesem Kapitel diskutiert werden, sind spezifisch
für bestimmte Lokalisierungsmodule.

\subsubsection{Bulgarisch}
\label{use:loc:bul}

Wie das griechische Lokalisierungsmodul benötigt auch das bulgarische Modul \utf-Unterstützung. 
Es funktioniert nicht mit anderen Kodierungen.


\subsubsection{Amerikanisches Englisch} \label{use:loc:us}

Das amerikanische  Lokalisierungsmodul verwendet \cmd{uspunctuation} aus
%\secref{aut:pct:cfg}
§ 4.7.5 (engl. V.), um eine <American-style>-Interpunktion zu ermöglichen.
Wenn diese Funktion aktiviert ist, werden alle Kommata und Punkte 
innerhalb von Anführungszeichen nach \\
\cmd{mkbibquote} verschoben. Wenn Sie diese Funktion
deaktivieren möchten, verwenden Sie \cmd{stdpunctuation} wie folgt:

\begin{ltxexample}
\DefineBibliographyExtras{american}{%
  \stdpunctuation
}
\end{ltxexample}
%
Standardmäßig ist die <American punctuation>-Funktion durch das
\texttt{american} Lokali\-sation-Modul aktiviert. Der obige Kode ist nur
erforderlich, wenn Sie die amerikanische Lokalisierung ohne die
amerikanische Zeichensetzung möchten. Ansonsten würde es redundant zu
anderen Sprachen sein.

Es ist äußerst ratsam, zu spezifizieren \texttt{american}, \texttt{british},
\texttt{australian}, etc., anstatt nur \texttt{english}, wenn Sie das 
\sty{babel}-Paket laden, um etwaige Verwechslungen zu vermeiden. Ältere
Versionen des \sty{babel}-Pakets verwenden die Option \opt{english} als
Alias für die Option \opt{british};
die neueren behandeln es als Alias für \opt{american}. Das \biblatex-Paket
behandelt im Wesentlichen \texttt{english} als Alias für \opt{american},
außer bei der obigen Funktion, die nur aktiviert wird, wenn
\texttt{american} angefordert wird.

\subsubsection{Spanisch} \label{use:loc:esp}

Die Handhabung von dem Wort <und> ist im Spanischen komplizierter als in anderen
Sprachen, welche von diesem Paket unterstützt werden, weil es ein <y> oder ein
<e> sein kann, je nachdem wie das nächste Wort beginnt. Hierfür nutzt das
spanische Lokalisierungsmodul nicht den Lokalisierungsstring <\texttt{and}>
sondern ein spezielles internes <smart and> Kommando. Das Verhalten dieses
Kommandos wird von dem \cnt{smartand} Zähler kontrolliert.

\begin{ltxsyntax}

\cntitem{smartand}

Dieser Zähler kontrolliert das Verhalten des internen <smart and> Kommandos.
Wenn er auf 1 steht, druckt er <y> oder <e>, vom Kontext abhängig. Wenn er auf
2 steht, druckt er immer ein <y>. Wenn er auf 3 steht, druckt er immer ein <e>.
Wenn er auf 0 steht, dann ist die Funktion <smart and> ausgeschaltet. Dieser
Zähler wird beim Start mit 0 initialisiert und kann in der Präambel geändert
werden. Man beachte, dass die Änderung des Zählers auf einen positiven Wert
impliziert, dass das Spanische Lokalisierungsmodul \cmd{finalnamedelim} und
\cmd{finallistdelim} ignoriert.

\csitem{forceE}

Man benutzt dieses Kommando in \file{bib} Dateien, falls \biblatex das
<und> vor einem bestimmten Wort falsch druckt. Wie der Name schon sagt, wird ein
<e> erzwungen. Dieses Kommando muss in einer bestimmten Weise benutzt
werden, um \bibtex nicht zu verwirren. Hier ist ein Beispiel:

\begin{lstlisting}[style=bibtex]{} 
author = {Edward Jones and Eoin Maguire},
author = {Edward Jones and <<{\forceE{E}}>>oin Maguire}, 
\end{lstlisting}
%
Man beachte, dass der erste Buchstabe der betreffenden Wortkomponente durch ein
Argument für \cmd{forceE} gegeben ist und das das gesamte Konstrukt in ein
zusätzliches Paar geschweifte Klammern gefasst ist.

\csitem{forceY}

Analog zu \cmd{forceE}, erzwingt aber <y>

\end{ltxsyntax}

\subsubsection{Griechisch} \label{use:loc:grk}

Das griechische Lokalisierungsmodul benötigt den \utf Support. Es arbeitet mit
keiner anderen Kodierung. Im Allgemeinen ist das \biblatex-Paket kompatibel
mit dem \sty{inputenc}-Paket und \xelatex. Das \sty{ucs}-Paket funktioniert
nicht. Nachdem das \sty{inputenc} Standard-\file{utf8}-Modul keine Bildzeichen
Funktion für das Griechische bereitstellt, bleibt für griechische Nutzer nur
noch \xelatex. Man beachte, dass man eventuell zusätzliche Pakete für
griechische Schriften laden muss. Als Daumenregel gilt, eine Einstellung, welche
für normale griechische Dokumente funktioniert, sollte auch mit \biblatex
funktionieren. Allerdings gibt es eine fundamentale Einschränkung. Zum jetzigen
Zeitpunkt unterstützt \biblatex nicht das Wechseln zwischen Schriften.
Griechische Titel in der Bibliografie sollten gut funktionieren, vorausgesetzt
man nutzt \biber als Backend, aber englische und andere Titel in der
Bibliografie könnten als griechische Buchstaben dargestellt werden. Falls man
Bibliografien mit mehreren Schriften braucht, ist \xelatex die einzige
vernünftige Wahl.

\subsubsection{Russisch}
\label{use:loc:rus}

Wie das griechische Lokalisierungsmodul benötigt das Russischmodul auch 
\utf-Unterstützung. Es wird nicht mit einer anderen Kodierung arbeiten. 

\subsubsection{Ungarisch}
\label{use:loc:hun}

Das ungarische Lokalisierungsmodul muss bestimmte Feldformate neu definieren, um grammatisch korrekte Wortreihenfolge zu erhalten. Dies bedeutet, dass diese feldformate immer überschrieben werden, wenn die ungarische Lokalisierung aktiv ist, unabhängig davon, ob sie in der Präambel oder durch einen benutzerdefinierten Stil definiert wurden. Beachten sie daher, dass die Verwendung des ungarischen Lokalisierungsmoduls dazu führen kann, dass die Bibliografieausgabe von dem Format abweicht, das durch die geladenen Stil- und Präambeldefinitionen vorgegeben ist. Änderungen an diesem Verhalten müssen mit  \cmd{DefineBibliographyExtras} vorgenommen werden. Insbesondere wird \cmd{mkpageprefix} neu definiert, um die Zeichenfolge <page> oder <pages> als Suffix nach der Seitenzahl gemäß den ungarischen Konventionen auszugeben, und alle Feldformate mit Seiten, Kapiteln und Bänden werden so geändert, dass Zahlen als Ordnungszahlen gedruckt werden. Das ungarische Lokalisierungsmodul gibt eine Warnung aus, um sie an diese Änderung zu erinnern, wenn ein Dokument geladen wird. Die Warnung macht auch deutlich, wie sie sie zum Schweigen bringen können. 

\subsubsection{Lettisch}
\label{use:loc:lav}

Das lettische Lokalisierungsmodul muss wie das ungarische Sprachmodul bestimmte Feldformate neu definieren, um die grammatisch korrekte Wortreihenfolge zu erhalten. Dies bedeutet, dass diese Feldformate immer dann überschrieben werden, wenn die lettische Lokalisierung aktiv ist, unabhängig davon, ob sie in der Präambel oder durch einen benutzerdefinierten Stil definiert wurden. 
Beachten sie daher, dass die Verwendung des lettischen Lokalisierungsmoduls dazu führen kann, dass die Bibliografieausgabe von dem Format dazu führen kann, dass die Bibliografieausgabe von dem Format abweicht, das durch die geladenen Stil- und Präambeldefinitionen vorgegeben ist. Änderungen an diesem Verhalten müssen mit  \cmd{DefineBibliographyExtras} vorgenommen werden. Insbesondere wird \cmd{mkpageprefix} neu definiert, um die Zeichenfolge <page> oder <pages> als Suffix nach der Seitenzahl gemäß den ungarischen Konventionen auszugeben, und alle Feldformate mit Seiten, Kapiteln und Bänden werden so geändert, dass Zahlen als Ordnungszahlen gedruckt werden. Das ungarische Lokalisierungsmodul gibt eine Warnung aus, um sie an diese Änderung zu erinnern, wenn ein Dokument geladen wird. Die Warnung macht auch deutlich, wie sie sie zum Schweigen bringen können. 

\subsection{Anwendungshinweise} \label{use:use}

Der folgende Abschnitt gibt einen grundlegenden Überblick über das
\biblatex-Paket und zeigt einige typische Anwendungsszenarien.

\subsubsection{Überblick} \label{use:use:int}

Die Benutzung des \biblatex-Pakets ist etwas anders als die Benutzung
traditioneller \bibtex -Stile und ähnlicher Pakete. Bevor wir zu einigen
spezifischen Anwendungsproblemen kommen, schauen wir uns dafür zunächst die
Struktur eines typischen Dokuments an:

\begin{ltxexample} 
\documentclass{...} 
\usepackage[...]{biblatex}
<<\addbibresource>>{<<bibfile.bib>>} 
\begin{document}
<<\cite>>{...}
...
<<\printbibliography>> 
\end{document} 
\end{ltxexample}
%
In dem traditionellen \bibtex dient das \cmd{bibliography}-Kommando zwei
Zwecken. Es markiert den Ort der Bibliografie und spezifiziert den/die
\file{bib}-Datei(en). Die Dateiendung wird weggelassen. Mit \biblatex
werden Quellen in der Präambel mit \cmd{addbibresource} durch die Nutzung des
vollen Namens mit \file{.bib}-Endung spezifiziert. Die Bibliografie wird mit
dem \cmd{printbibliographie}-Kommando gedruckt, welches eventuell mehrere Male
benutzt wird (sehen Sie \secref{use:bib} für Details). Der Dokumentkörper kann
mehrere Zitierkommandos enthalten (\secref{use:cit}). Die Verarbeitung dieses
Beispiels erfordert eine bestimmte Prozedur, die nachfolgend gezeigt wird.
Vorausgesetzt unser Beispiel heißt \path{example.tex} und unsere
Bibliografiedaten sind in \path{bibfile.bib}, dann ist die Prozedur folgende:

%\paragraph{\biber}

\begin{enumerate}

\item Laufenlassen \bin{latex} von \path{example.tex}. Falls die Datei irgendwelche Zitate
enthält, enthält \biblatex die betreffenden Daten von jedem Zitat, \biblatex wird die
entsprechenden Daten von \biber anfordern, indem Befehle in die auxiliar-Datei (Hilfsdatei) \path{example.bcf} eingeschrieben werden.

\item Laufenlassen von \bin{biber} auf dem Pfad \path{example.bcf}. \biber wird die Daten von  \path{bibfile.bib} holen und sie in die auxiliary-Datei \path{example.bbl} in einem Format
, das von \biblatex verarbeitet werden kann, schreiben.

\item Laufenlassen \bin{latex} von \path{example.tex}. \biblatex wird die Daten von the data aus \path{example.bbl} lesen und alle Zitate sowie die Bibliografie ausgeben.
\end{enumerate}

\subsubsection{Hilfsdateien} \label{use:use:aux}


%\paragraph{\biber}

Das \biblatex-Paket verwendet nur eine auxiliar-\file{bcf}-Datei. Auch wenn Zitierungsbefehle via \cmd{include} eingebracht werden, Sie müssen nur \biber mit der Haupt-\file{bcf}-Datei ausführen. Alle Informationen, die \biber braucht, sind in der \file{bcf}-Datei, 
einschließlich von Informationen über alle "`refsections"', wenn mehrere 
\env{refsection}-Umgebungen sind (sehen Sie \secref{use:use:mlt}).

\subsubsection{Mehrere Bibliografien} \label{use:use:mlt}

In einer Sammlung von Artikeln von verschiedenen Autoren, wie ein
Konferenzprotokoll zum Beispiel, ist es üblicher eine Bibliografie für jeden
Artikel zu haben, statt eine globale für das ganze Buch. Im Beispiel unten wird
jeder Artikel durch ein separates \cmd{chapter} mit seiner eigenen Bibliografie
dargestellt. 

\begin{ltxexample}
\documentclass{...}
\usepackage{biblatex}
\addbibresource{...}
\begin{document}
\chapter{...}
<<\begin{refsection}>>
...
<<\printbibliography[heading=subbibliography]>>
<<\end{refsection}>>
\chapter{...}
<<\begin{refsection}>>
...
<<\printbibliography[heading=subbibliography]>>
<<\end{refsection}>>
\end{document}
\end{ltxexample}
%
Falls \cmd{printbibliographie} in einer \env{refsection}-Umgebung benutzt wird,
beschränkt es automatisch den Bereich der Referenzliste auf die betreffende
\env{refsection}-Umgebung. Für eine wachsende Bibliografie, welche in Kapitel
unterteilt ist, jedoch am Ende des Buches gedruckt wird, nutzt man die
\opt{section}-Option von \cmd{printbibliography}, um einen Referenzabschnitt zu
wählen, wie man im nächsten Beispiel sieht. 

\begin{ltxexample}
\documentclass{...}
\usepackage{biblatex}
<<\defbibheading>>{<<subbibliography>>}{%
  \section*{References for Chapter \ref{<<refsection:\therefsection>>}}}
\addbibresource{...}
\begin{document}
\chapter{...}
<<\begin{refsection}>>
...
<<\end{refsection}>>
\chapter{...}
<<\begin{refsection}>>
...
<<\end{refsection}>>
\printbibheading
<<\printbibliography>>[<<section=1>>,<<heading=subbibliography>>]
<<\printbibliography>>[<<section=2>>,<<heading=subbibliography>>]
\end{document}
\end{ltxexample}
%
Man beachte im Beispiel oben die Definition des Bibliografietitels. Das ist die
Definition, welche auf die Untertitel in der Bibliografie aufpasst. Der
Haupttitel wird in diesem Fall durch ein einfaches \cmd{chapter}-Kommando
erzeugt. Das \biblatex-Paket setzt automatisch eine Beschriftung an den
Anfang einer jeden \env{refsection}-Umgebung, indem es das
Standard-\cmd{label}-Kommando nutzt. Die Bezeichnung, die genutzt wird, ist die 
Zeichenkette\texttt{refsection:} gefolgt von der Nummer der betreffenden 
\env{refsection}-Umgebung. Die Nummer des aktuellen Abschnitts ist durch den \cnt{refsection}-Zähler verügbar. Wenn man die \opt{section} Option von \cmd{printbibliography}
nutzt, ist der Zähler auch lokal gesetzt. Das heißt, dass man den Zähler für
Titeldefinitionen nutzen kann, um Untertitel, wie «Referenz für Kapitel 3», wie
oben gezeigt, zu drucken. Man kann außerdem den Titel des betreffenden Kapitels
als Untertitel nutzen indem man das \sty{nameref}-Paket lädt und \cmd{nameref}
statt \cmd{ref} nutzt:

\begin{ltxexample}
\usepackage{<<nameref>>}
\defbibheading{subbibliography}{%
  \section*{<<\nameref{refsection:\therefsection}>>}}
\end{ltxexample}
%
Da ein \cmd{printbibliography}-Befehl für jeden Teil der unterteilten
Bibliografie lästig ist, bietet \biblatex ein Kürzel. Der
\cmd{bibbysection}-Befehl durchläuft automatisch alle Referenzabschnitte. Das
ist das äquivalent für ein \\ \cmd{printbibliography}-Kommando für jeden
Abschnitt, jedoch hat es den zusätzlichen Vorteil des automatischen
Überspringens von Abschnitten ohne Referenzen. In dem Beispiel oben würde die
Bibliografie dann folgendermaßen erzeugt werden:

\begin{ltxexample}
\printbibheading
<<\bibbysection[heading=subbibliography]>>
\end{ltxexample}
%
Wenn man ein Format ohne eine wachsende Bibliografie nutzt, welche in Kapitel (oder
irgendeine andere Dokumentunterteilung) gegliedert ist, ist es vermutlich
angebrachter \env{refsegment}- statt \env{refsection}-Umgebungen zu nutzen. Der
Unterschied ist, dass die \env{refsection} Umgebung lokal für die Umgebung die
Beschriftungen erzeugt, während \env{refsegment} die Beschriftungserzeugung
nicht beeinflusst, daher sind sie somit einzigartig über das gesamte Dokument.
Das nächste Beispiel könnte ebenso in \secref{use:use:div}
genutzt werden, da es optisch eine globale Bibliografie,  in mehrere
Segmente  unterteilt, erzeugt.

\begin{ltxexample}
\documentclass{...}
\usepackage{biblatex}
<<\defbibheading>>{<<subbibliography>>}{%
  \section*{References for Chapter \ref{<<refsegment:\therefsection\therefsegment>>}}}
\addbibresource{...}
\begin{document}
\chapter{...}
<<\begin{refsegment}>>
...
<<\end{refsegment}>>
\chapter{...}
<<\begin{refsegment}>>
...
<<\end{refsegment}>>
\printbibheading
<<\printbibliography>>[<<segment=1>>,<<heading=subbibliography>>]
<<\printbibliography>>[<<segment=2>>,<<heading=subbibliography>>]
\end{document}
\end{ltxexample}
%

Die Nutzung von \env{refsegment} ist ähnlich zu \env{refsection} und es gibt
ebenso eine entsprechende \opt{segment}-Option für \cmd{printbibliographie}.
Das \sty{biblatex}-Paket setzt automatisch eine Bezeichnung an den Beginn von
jeder \env{refsegment}-Umgebung, indem es die Zeichenkette \texttt{refsegment:}
gefolgt von der Nummer der entsprechenden \env{refsegment}-Umgebung als
Bezeichner nutzt. Es gib einen angepassten \cnt{refsegment}-Zähler, welcher
für die Titeldefinitionen, wie oben gezeigt, genutzt werden kann. Wie bei den
Referenzabschnitten gibt es auch hier ein Kurzkommando, welches automatisch
über alle Referenzsegment läuft:

\begin{ltxexample}
\printbibheading
<<\bibbysegment[heading=subbibliography]>>
\end{ltxexample}
%
Das ist äquivalent zu je einem \cmd{printbibliography} Kommando für jedes
Segment in der genwärtigen \env{refsection}.

\subsubsection{Unterteilte Bibliografien} \label{use:use:div}

Es ist sehr üblich, eine Bibliografie nach bestimmten Kriterien zu unterteilen.
Zum Beispiel: Man will vielleicht gedruckte und online Quellen separat drucken
oder eine Bibliografie in Primär- und Sekundärliteratur teilen. Der erstere
Fall ist einfach, weil man den Eintragstypen als Kriterium für die
\opt{type}- und \opt{nottype}-Filter von \cmd{printbibliography} nutzen kann. Das nächste
Beispiel demonstriert außerdem, wie man angepasste Untertitel für zwei Teile
der Bibliografie erzeugt.

\begin{ltxexample}
\documentclass{...}
\usepackage{biblatex}
\addbibresource{...}
\begin{document}
...
\printbibheading
\printbibliography[<<nottype=online>>,heading=subbibliography,
                   <<title={Printed Sources}>>]
\printbibliography[<<type=online>>,heading=subbibliography,
                   <<title={Online Sources}>>]

\end{document}
\end{ltxexample}
%
Man kann auch mehr als zwei Unterteilungen nutzen:

\begin{ltxexample}
\printbibliography[<<type=article>>,...]
\printbibliography[<<type=book>>,...]
\printbibliography[<<nottype=article>>,<<nottype=book>>,...]
\end{ltxexample}
%
Es ist sogar möglich, eine Kette von verschiedenen Filtertypen zu geben:

\begin{lstlisting}[style=latex]{}
\printbibliography[<<section=2>>,<<type=book>>,<<keyword=abc>>,<<notkeyword=xyz>>]
\end{lstlisting}
%
Das würde alle zitierten Arbeiten in Referenzabschnitt~2 ausgeben, dessen
Eintragstyp \bibtype{book} ist und dessen \bibfield{keywords}-Feld das
Schlüsselwort <abc> aber nicht <xyz> beinhaltet. Wenn man Bibliografiefilter
in Verbindung mit einem numerischen Stil verwendet, sehen Sie \secref{use:cav:lab}.
Falls man einen komplexen Filter mit bedingten Ausdrücken benötigt, nutzt man
die \opt{filter}-Option in Verbindung mit einem spezifizierten Filter, definiert
durch \cmd{defbibfilter}. Sehen Sie \secref{use:bib:flt} für Details über
spezifizierte Filter.

\begin{ltxexample} 
\documentclass{...} 
\usepackage{biblatex}
\addbibresource{...} 
\begin{document} 
...  
\printbibheading
\printbibliography[<<keyword=primary>>,heading=subbibliography,%
                   <<title={Primary Sources}>>]
\printbibliography[<<keyword=secondary>>,heading=subbibliography,%
                   <<title={Secondary Sources}>>] 
\end{document} 
\end{ltxexample}
%
Eine Unterteilung der Bibliografie in Primär- und Sekundärliteratur ist mit
einem \bibfield{keyword}-Filter möglich, wie im Beispiel oben gezeigt. In
diesem Fall, mit nur zwei Unterbereichen, wäre es ausreichend, ein Schlüsselwort als
Filterkriterium zu nutzen:

\begin{ltxexample}
\printbibliography[<<keyword=primary>>,...]
\printbibliography[<<notkeyword=primary>>,...]
\end{ltxexample}
%
\begin{lstlisting}[style=latex]{} 
\printbibliography[<<keyword=primary>>,...]
\printbibliography[<<notkeyword=primary>>,...] 
\end{lstlisting}
%

Da \biblatex keine Möglichkeit hat, zu wissen, ob ein Eintrag in der
Bibliografie Primär- oder Sekundärliteratur sein soll, muss man die
Bibliografiefilter mit den benötigten Daten unterstützen, indem man ein
\bibfield{keywords}-Feld für jeden Eintrag in die \file{bib}-Datei hinzufügt.
Diese Schlüsselwörter können dann als Ziele für den \opt{keyword}- und
\opt{nokeyword}-Filter benutzt werden, wie oben gezeigt. Es kann eine gute Idee
sein, solche Schlüsselwörter während des Aufbaus einer \file{bib} Datei
hinzuzufügen.  

\begin{lstlisting}[style=bibtex]{} 
@Book{key, 
<<keywords>> = {<<primary>>,some,other,keywords}, 
... 
\end{lstlisting}
%
Ein alternativer Weg, um die Referenzliste zu unterteilen, sind
Bibliografiekategorien. Sie unterscheiden sich von dem schlüsselwortbasierten
Denkansatz, der im Beispiel oben gezeigt ist, darin, dass sie auf dem
Dokumentlevel arbeiten und keine Änderungen in der \file{bib}-Datei benötigen.

\begin{ltxexample} 
\documentclass{...} 
\usepackage{biblatex}
<<\DeclareBibliographyCategory>>{<<primary>>}
<<\DeclareBibliographyCategory>>{<<secondary>>}
<<\addtocategory>>{<<primary>>}{key1,key3,key6}
<<\addtocategory>>{<<secondary>>}{key2,key4,key5} 
\addbibresource{...}
\begin{document}
...  
\printbibheading
\printbibliography[<<category=primary>>,heading=subbibliography,%
<<title={Primary Sources}>>]
\printbibliography[<<category=secondary>>,heading=subbibliography,%
<<title={Secondary Sources}>>] 
\end{document} 
\end{ltxexample}
%
In diesem Fall wäre es ausreichend, nur eine Kategorie zu nutzen:

\begin{ltxexample}
\printbibliography[<<category=primary>>,...]
\printbibliography[<<notcategory=primary>>,...]
\end{ltxexample}
%
\begin{lstlisting}[style=latex]{} 
\printbibliography[<<category=primary>>,...]
\printbibliography[<<notcategory=primary>>,...] 
\end{lstlisting}
%
Es ist eine gute Idee, alle benutzten Kategorien in der Bibliografie explizit zu
deklarieren, weil es einen \cmd{bibbycategory}-Befehl gibt, welcher automatisch
über alle Kategorien läuft. Er ist äquivalent zu je einem
\cmd{printbibliographie}-Befehl für jede Kategorie, in der Reihenfolge, in der
sie deklariert wurden.

\begin{ltxexample} 
\documentclass{...} 
\usepackage{biblatex}
<<\DeclareBibliographyCategory>>{<<primary>>}
<<\DeclareBibliographyCategory>>{<<secondary>>}
\addtocategory{primary}{key1,key3,key6}
\addtocategory{secondary}{key2,key4,key5}
<<\defbibheading>>{<<primary>>}{\section*{Primary Sources}}
<<\defbibheading>>{<<secondary>>}{\section*{Secondary Sources}}
\addbibresource{...} 
\begin{document}
...  
\printbibheading <<\bibbycategory>>
\end{document} 
\end{ltxexample}
%
Die Behandlung der Titel ist in diesen Fall unterschiedlich von
\cmd{bibbysection} und \cmd{bibbysegment}. \cmd{bibbycategory} nutzt den Namen
der aktuellen Kategorie als Titelname: Das ist äquivalent zur Weitergabe von
\texttt{heading=\prm{category}} zu \\
\cmd{printbibliography} und impliziert, dass
man einen angepassten Titel für jede Kategorie braucht.

\subsubsection{Eintragssätze} \label{use:use:set}

Ein Eintragssatz ist eine Gruppe von Einträgen, die als Einzelreferenzen
zitiert wurden und als einzelne Einheit in der Bibliografie gelistet sind. Das
\biblatex-Paket unterstützt zwei Typen von Eintragssätzen. Statische
Eintragssätze sind in der \file{bib}-Datei definiert, wie jeder andere Eintrag.
Dynamische Eintragssätze sind mit \cmd{defbibentryset} (\secref{use:bib:set})
auf einer \emph{per-document\slash per-refsection} Basis in der Dokumentpräambel oder
im Dokumentenkörper definiert. Dieser Abschnitt beschäftigt sich mit der
Definition von Eintragssätzen; Stilautoren sollten sich auch
%\secref{aut:cav:set} 
§ 4.11.1 (engl. V.) für mehr Informationen ansehen.
Bitte beachten Sie, dass Eintragssätze nur für Stile sinnvoll sind, die auf Einträge mit Bezeichnungen wie \texttt{numeric} und \texttt{alphabetic} 
verweisen. Stile, die über Namen, Titel usw. auf Einträge verweisen (\texttt {authoryear}, 
\texttt{authortitle}, \texttt {verbose} usw.), verwenden selten Mengen und unterstützen sie standardmäßig nicht, wenn sie direkt zitiert werden. Benutzerdefinierte Stile können sich natürlich dafür entscheiden, eine Art von Unterstützung für festgelegte Zitate auf eine von ihnen gewählte Weise zu implementieren.

\paragraph{Statische Eintragssätze}

Statische Eintragssätze sind in der \file{bib}-Datei definiert,
wie jeder andere Eintrag. Man definiert einen
Eintragssatz einfach, indem man einen Eintrag vom Typ \bibtype{set} hinzufügt.
Der Eintrag hat ein \bibfield{entryset}-Feld, welches die Mitglieder des Satzes
als durch Kommas geteilte Liste von Eintragsschlüsseln definiert:

\begin{lstlisting}[style=bibtex]{} 
<<@Set>>{<<set1>>, 
<<entryset>> = {<<key1,key2,key3>>},
} 
\end{lstlisting}
%
Einträge können Teil von einem Satz in einem Dokument\slash Refsektion und
alleinstehende Referenzen in einem anderen sein, je nachdem, ob es den
\bibtype{set}-Eintrag gibt oder nicht. Falls der \bibtype{set}-Eintrag zitiert
wird, werden die Satzmitglieder automatisch gruppiert. Falls nicht, werden sie
wie jeder andere reguläre Eintrag gehandhabt.

\paragraph{Dynamische Eintragssätze}

Dynamische Eintragssätze werden ähnlich gebildet wie statische und
funktionieren auch ähnlich. Der Hauptunterschied ist, dass sie in der
Dokumentpräambel oder während des Schreibens im Dokumentenkörper durch Nutzung
des \cmd{defbibentryset}-Kommandos aus \secref{use:bib:set} definiert werden:

\begin{lstlisting}[style=bibtex]{} 
\defbibentryset{set1}{key1,key2,key3}
\end{lstlisting}
%
Dynamische Eintragssätze im Dokumentenkörper sind lokal bezüglich der
umgebenden \env{refsection}-Umgebung, falls es eine gibt. Anderenfalls zählen
sie zum Referenzabschnitt~0. Die, die in der Präambel definiert wurden,
gehören zum Referenzabschnitt~0. 

\subsubsection{Datencontainer}
\label{use:use:xdat}

Der \bibtype{xdata}-Eintragstyp dient als Datencontainer, der eines oder mehrere Felder
hält. Diese Felder können durch andere Einträge über das \bibfield{xdata}-Feld vererbt werden. \bibtype{xdata}-Einträge können in der Bibliografie nicht zitiert oder hinzugefügt werden. Dieser Datenvererbungsmechanismus ist für feste Feldkombinationen wie  \bibfield{publisher}\slash \bibfield{location} und  für andere häufig verwendete Daten nützlich:

\begin{lstlisting}[style=bibtex]{}
<<@XData>>{<<hup>>,
  publisher  = {Harvard University Press},
  location   = {Cambridge, Mass.},
}
@Book{...,
  author     = {...},
  title	     = {...},
  date	     = {...},
  <<xdata>>      = {<<hup>>},
}
\end{lstlisting}
%
Mit Hilfe einer getrennten Liste von Schlüsseln in seinem \bibfield{xdata}-Feld
kann ein Eintrag Daten aus mehreren \bibtype{xdata}-Einträgen erben. Mehrstufige \bibtype{xdata}-Einträge werden ebenfalls unterstützt, d.\,h. ein 
\bibtype{xdata}-Eintrag kann auf einen oder mehrere \bibtype{xdata}-Einträge
referieren:

\begin{lstlisting}[style=bibtex]{}
@XData{macmillan:name,
  publisher  = {Macmillan},
}
@XData{macmillan:place,
  location   = {New York and London},
}
@XData{macmillan,
  xdata      = {macmillan:name,macmillan:place},
}
@Book{...,
  author     = {...},
  title	     = {...},
  date	     = {...},
  xdata	     = {macmillan},
}
\end{lstlisting}
%
Es kann auf detaillierte \bibtype{xdata}-Eintragstypen verwiesen werden. Es ist nicht
erforderlich, nur auf ganze Felder zu referenzieren. Beispielsweise:

\begin{lstlisting}[style=bibtex]{}
@XData{someauthors,
  author     = {John Smith and Brian Brown}
}
@XData{somelocations,
  location   = {Location1 and Location2}
}
@XData{somenotes,
  note   = {A note}
}
@Book{...,
  author     = {Alan Drudge and xdata=someauthors-author-2},
  editor     = {xdata=someauthors-author},
  location   = {xdata=somelocations-location-1 and Location3},
  note       = {xdata=somenotes-note}
}
\end{lstlisting}
%
Das Format der granularen \bibtype{xdata}-Referenz lautet wie folgt:

\begin{namesample}
\item~\delim{x}{1}data\delim{=}{2}\delim{$<$}{3}key$>$\delim{-}{4}\delim{$<$}{5}field$>$\delim{-}{6}\delim{$<$}{7}index$>$
\end{namesample}

\begin{enumerate}
  \item Der Wert der \biber-Option \opt{--xdatamarker} (standardmäßig \glq \texttt{xdata}\grq).
  \item Der Wert der \biber-Option \opt{--xnamesep} (standardmäßig \glq \texttt{=}\grq).
  \item Ein gültiger Eintragsschlüssel eines \bibtype{xdata}-Eintrags.
  \item Der Wert der \biber-Option \opt{--xdatasep} (standardmäßig \glq \texttt{-}\grq).
  \item Ein gültiges Eintragsfeld der Quelle des \bibtype {xdata}-Eintrags.
  \item (Optional) Der Wert der \biber-Option \opt{--xdatasep} (standardmäßig \glq \texttt{-}\grq).
  \item (Optional) Ein gültiger 1-basierter Index in einem a list/name-Feld dem Quellenfeld des
      \bibtype{xdata}-Eintrags.
\end{enumerate}
%
Es gibt \opt{--output-*}-Varianten der oben genannten Optionen für die Ausgabe im \biber-Tool,
so dass diese Trennzeichen und Markierungen programmgesteuert geändert werden können. Im obigen 
Beispiel würde sich  \bibtype{book} wie folgt auflösen:

\begin{lstlisting}[style=bibtex]{}
@Book{...,
  author     = {Alan Drudge and Brian Brown},
  editor     = {John Smith},
  location   = {Location1 and Location3},
  note       = {A note}
}
\end{lstlisting}
%
Dinge, die mit granular \bibtype{xdata}-Referenzen zu beachten sind:

\begin{itemize}
  \item Verweise dürfen nur auf \bibtype{xdata}-Felder verweisen. Andernfalls wird eine Warnung
      generiert und die Referenz wird nicht aufgelöst.
\item Verweise dürfen nur auf \bibtype{xdata}-Felder vom gleichen Typ verweisen (list/name und datatype) wie das Referenzierungsfeld. Andernfalls wird eine Warnung generiert und die Referenz wird nicht aufgelöst.
 \item Verweise auf Felder des Datentyps 'date' sind nicht möglich. RVerweise zu den gerbten 
Felder \bibfield{year} und \bibfield{month} sind möglich.
 \item Verweise auf fehlende Einträge, Felder oder Listen/Namen-Indizies erzeugen eine Warnung
     und der Verweis wird nicht aufgelöst.
 \item Wenn ein Index für einen Verweis auf ein Listen-/Namensfeld fehlt, wird 1 angenommen. 
\end{itemize}


Sehen Sie auch \secref{bib:typ:blx,bib:fld:spc}.

\subsubsection[Elektronische Publikationen]{Informationen zu elektronischen Publikationen} 
\label{use:use:epr}

Das \biblatex-Paket unterstützt drei Felder für elektronische
Herausgabeinformationen: \bibfield{eprint}, \bibfield{eprinttype} und
\bibfield{eprintclass}. Das \bibfield{eprint}-Feld ist ein wortwörtliches Feld,
analog zu \bibfield{doi}, welches den Bezeichner einer Einheit enthält. Das
\bibfield{eprinttype}-Feld beinhaltet den Quellennamen, z.\,B. den Namen von der
Seite oder des elektronischen Archivs. Das optionale
\bibfield{eprintclass}-Feld
ist für zusätzliche Informationen speziell zu der Quelle, die durch das
\bibfield{eprinttype}-Feld angegeben wird, vorgesehen. Das könnte ein
Abschnitt, ein Pfad, Klassifikationsinformationen, etc sein. Falls ein
\bibfield{eprinttype}-Feld verfügbar ist, wird es den Standardstil als
einen
wortgetreuen Bezeichner benutzen. In dem folgenden Beispiel würde «Resource:
identifier» statt dem allgemeinen «eprint: identifier» gedruckt werden:

\begin{lstlisting}[style=bibtex]{} 
<<eprint>>     = {<<identifier>>},
<<eprinttype>> = {<<Resource>>}, 
\end{lstlisting}
%
Die Standardstile besitzen einen geeigneten Support für einige Onlinearchive.
Für arXiv-Referenzen setze man den Bezeichner in das \bibfield{eprint}-Feld und
die Zeichenkette \texttt{arxiv} in das \bibfield{eprinttype}-Feld:

\begin{lstlisting}[style=bibtex]{} 
<<eprint>>     = {<<math/0307200v3>>},
<<eprinttype>> = {<<arxiv>>}, 
\end{lstlisting}
%
Für Dokumente, welche nach dem neuen Bezeichnerschema (April 2007 und später)
benutzt werden, fügt man eine einfache Klassifikation in das
\bibfield{eprintclas}-Feld hinzu:

\begin{lstlisting}[style=bibtex]{} 
eprint      = {1008.2849v1}, 
eprinttype  = {arxiv}, 
<<eprintclass>> = {<<cs.DS>>}, 
\end{lstlisting}
%
Es gibt zwei Aliase, welche die Integration von arXiv-Einträgen erleichtern.
\bibfield{archiveprefix} wird als Alias für \bibfield{eprinttype} behandelt;
\bibfield{primaryclass} ist ein Pseudonym für \bibfield{eprintclass}. Falls
Hyperlinks aktiviert sind, wird der Eprint-Bezeichner in einen Link zu
\nolinkurl{arxiv.org} transformiert. Man betrachte für weitere Details die
Paketoptionen \opt{arxiv} in \secref{use:opt:pre:gen}.

Für \acr{JSTOR}-Referenzen schreibe man die feste \acr{JSTOR}-Nummer in das
\bibfield{eprint}-Feld und die Zeichenkette \texttt{jstor} in das
\bibfield{eprinttype}-Feld:

\begin{lstlisting}[style=bibtex]{} 
<<eprint>>     = {<<number>>}, 
<<eprinttype>> = {<<jstor>>}, 
\end{lstlisting}
%
Wenn man das Exportfeature von \acr{JSTOR} benutzt, um Zitate in das
\bibtex-Format zu exportieren, dann nutzt \acr{JSTOR} das \bibfield{url}-Feld
standardmäßig (wo die \prm{number} einzigartig ist und ein fester Bezeichner
ist):

\begin{lstlisting}[style=bibtex]{} 
url = {http://www.jstor.org/stable/<<number>>}, 
\end{lstlisting}
%
Dies wird, wie erwartet, funktionieren, jedoch tendieren vollständige \acr{URL}s
dazu, die Bibliografie zu verstopfen. Mit den \bibfield{eprint}-Feldern werden die
Standardstile das besser lesbare «\acr{JSTOR}: \prm{number}»-Format nutzen,
welches auch Hyperlinks unterstützt. Die \prm{number} wird ein anklickbarer
Link, falls der \sty{hyperref}-Support aktiviert ist.

Für PubMed-Referenzen schreibe man den festen PubMed-Bezeichner in das
\bibfield{eprint}-Feld und die Zeichenkette \texttt{pubmed} in das
\bibfield{eprinttype}-Feld. Das sieht dann folgendermaßen aus:

\begin{lstlisting}[style=bibtex]{} 
url = {http://www.ncbi.nlm.nih.gov/pubmed/<<pmid>>}, 
\end{lstlisting}
%

Es wird:

\begin{lstlisting}[style=bibtex]{} 
<<eprint>>     = {<<pmid>>}, 
<<eprinttype>> = {<<pubmed>>}, 
\end{lstlisting}
%
Und die Standardstile drucken «\acr{PMID}: \prm{pmid}» statt der langen
\acr{URL}. Falls der hyperref-Support aktiviert ist, wird \prm{pmid} ein
anklickbarer Link zu PubMed werden.

Für Handles (\acr{HDL}s) schreibe man den Handle in das \bibfield{eprint}-Feld
und die Zeichenkette \texttt{hdl} in das \bibfield{eprinttype}-Feld:

\begin{lstlisting}[style=bibtex]{} 
<<eprint>>     = {<<handle>>}, 
<<eprinttype>> = {<<hdl>>}, 
\end{lstlisting}
%
Für Google"=Books"=Referenzen schreibe man den Bezeichner von Google in das
\bibfield{eprint}-Feld und den String \texttt{googlebooks} in das
\bibfield{eprinttype}-Feld. Dies sieht zum Beispiel folgendermaßen aus:

\begin{lstlisting}[style=bibtex]{} 
url = {http://books.google.com/books?id=<<XXu4AkRVBBoC>>}, 
\end{lstlisting}
%

Es würde zu Folgendem werden:

\begin{lstlisting}[style=bibtex]{} 
<<eprint>>     = {<<XXu4AkRVBBoC>>},
<<eprinttype>> = {<<googlebooks>>}, 
\end{lstlisting}
%
Und die Standardstile würden «Google Books: |XXu4AkRVBBoC|» statt der vollen
\acr{URL} drucken. Falls der hyperref-Support aktiviert ist, wird der Bezeichner
zu einem anklickbaren Link zu Google Books.\footnote{Man beachte, dass die
Google Books \acr{id} eher ein <interner> Wert zu sein scheint. Zum jetzigen
Zeitpunkt scheint es keinen Weg zu geben, um nach einer \acr{id} in Google Books
zu suchen. Man könnte die Nutzung von \bibfield{url} in diesem Fall
bevorzugen.}

Man beachte, dass \bibfield{eprint} ein wörtliches Feld ist. Es gibt immer den
Bezeichner in der unmodifizierten Form. Zum Beispiel ist es unnötig ein |_|
durch ein |\_| zu ersetzen. Siehe dazu auch %\secref{aut:cav:epr}
Kap. 4.12.2 (e. V.), um zu sehen,
wie man einen geeigneten Support für andere Eprint-Quellen hinzufügt.

\subsubsection{Externe Inhaltsangaben und Anmerkungen} \label{use:use:prf}

Stile, die die Felder \bibfield{abstract} und\slash oder \bibfield{annotation}
ausgeben, können einen alternativen Weg unterstützen, um Inhaltsangaben oder
Anmerkungen zu der Bibliografie hinzuzufügen. Statt den Text in der
\file{bib}-Datei einzubinden, kann er ebenso in einer externen \latex-Datei gespeichert
werden. Zum Beispiel, statt zu sagen:

\begin{ltxexample}[style=bibtex]
@Article{key1,
  ...
  abstract	  = {This is an abstract of entry `key1'.}
}
\end{ltxexample}
%
Man kann in der \file{bib}-Datei eine Datei namens\path{bibabstract-key1.tex} schaffen
und das Abstrakt in diese Datei setzen:

\begin{ltxexample}
This is an abstract of entry `key1'.
\endinput
\end{ltxexample}
%
Der Name der externen Datei muss vor dem Eintragsschlüssel das Präfix
\texttt{bibabstract-} beziehungsweise \texttt{bibannotation-} haben. Man kann
diese Präfixe ändern, indem man \cmd{bibabstractprefix} und
\cmd{bibannotationsprefix} umdefiniert. Man beachte, dass dieses Feature
explizit aktiviert werden muss, indem man die Paketoption \opt{loadfiles}
aus
\secref{use:opt:pre:gen} setzt. Die Option ist standardmäßig aus
Performanzgründen deaktiviert. Man beachte außerdem, dass jedes
\bibfield{abstract}- und \bibfield{annotation}-Feld in der \file{bib}-Datei den
Vorzug vor externen Dateien besitzt. Die Nutzung von externen Dateien wird
dringend empfohlen, falls man lange Inhaltsangaben oder viele Anmerkungen hat,
da dies den Speicherbedarf signifikant erhöht. Es ist außerdem angenehmer
einen Text in einer zugehörigen \latex-Datei zu editieren. Stilautoren
sollten sich § 4.11.3 (e. V.)
%\secref{aut:cav:prf} 
für weitere Informationen anschauen.

\subsection{Hinweise und Warnungen} \label{use:cav}

Dieser Abschnitt bietet zusätzliche Nutzungshinweise und zeigt einige übliche
Probleme und potentielle Missverständnisse.

\subsubsection{Nutzung mit \acr{KOMA}-Script-Klassen} \label{use:cav:scr}

Wenn man \biblatex in Verbindung mit der \sty{scrbook}-,
\sty{scrreprt}- oder \sty{scrartcl}-Klasse benutzt, sind die Titel
\texttt{bibliography} und \texttt{shorthands} von \secref{use:bib:hdg}
ansprechbar für die bibliografiebezogenen Optionen dieser
Klassen.\footnote{Dies bezieht sich auf die traditionelle Syntax dieser Optionen
(\opt{bibtotoc} und \opt{bibtotocnumbered}) und ebenso auf die \keyval-Syntax,
die in \acr{KOMA}-Script 3.x hinzugefügt wurde, z.\,B. auf
\kvopt{bibliography}{nottotoc}, \kvopt{bibliography}{totoc} und
\kvopt{bibliography}{totocnumbered}. Die globalen \kvopt{toc}{bibliography} und
\kvopt{toc}{bibliographynumbered} Optionen, und ebenso ihre Aliase, werden
ebenso entdeckt. In jedem Fall müssen die Optionen global in das optionale
Argument von \cmd{documentclass} gesetzt werden.} Man kann die Standardtitel
überschreiben, indem man die \opt{heading}-Option von \cmd{printbibliography},
\cmd{printbibheading} und \cmd{printshorthands} benutzt. Siehe
\secref{use:bib:bib, use:bib:biblist, use:bib:hdg} für Details. Alle Standardtitel
werden in der Ladezeit derart adaptiert, dass sie sich mit dem Verhalten der
Klassen vermischen. Falls eine der oberen Klassen entdeckt wird, stellt
\biblatex folgende zusätzliche Tests bereit, welche nützlich für
nutzerspezifische Titeldefinitionen sein können:  

\begin{ltxsyntax}

\cmditem{ifkomabibtotoc}{true}{false}

Wird zu \prm{true}, falls die Klasse die Bibliografie zu dem Inhaltsverzeichnis
hinzufügen würde und zu \prm{false} anderenfalls.

\cmditem{ifkomabibtotocnumbered}{true}{false}

Wird zu \prm{true}, falls die Klasse die Bibliografie zu dem Inhaltsverzeichnis
als nummerierten Abschnitt hinzufügen würde, anderenfalls wird es zu
\prm{false}. Falls dieser Test \prm{true} bringt, wird \cmd{ifkomabibtotoc}
ebenso immer \prm{true} zurückgeben, aber nicht anderes herum.

\end{ltxsyntax}

\subsubsection{Nutzung mit der Memoir-Klasse} \label{use:cav:mem}

Wenn man \biblatex mit der \sty{memoir}-Klasse benutzt, haben die meisten
Möglichkeiten für die Adaption der Bibliografie keinen Effekt. Man benutze
stattdessen die dementsprechenden Hilfen von dem Paket (\secref{use:bib:bib,
use:bib:hdg, use:bib:nts}). Statt \cmd{bibsection} von \sty{memoir}
umzudefinieren, nutze man die \opt{heading}-Option von \\
\cmd{printbibliography} und \cmd{defbibheading} (\secref{use:bib:bib, use:bib:hdg}). 
Statt \cmd{prebibhook} und \cmd{postbibhook} nutze man die \opt{prenote}- und
\opt{postnote}-Option von \\
\cmd{printbibliography} und \cmd{defbibnote}
(\secref{use:bib:bib, use:bib:nts}). Alle Standardtitel werden beim Laden derart
adaptiert, dass sie sich mit dem standardmäßigen Layout dieser Klasse
vermischen. Die Defaulttitel \texttt{bibliography} und \texttt{shorthands}
(\secref{use:bib:hdg}) sind ebenfalls ansprechbar für \cmd{bibintoc} und
\cmd{nobibtoc} Wechsel von \sty{memoir}. Die Länge des Registers
\len{bibitemsep} wird durch \biblatex auf die gleiche Weise wie
\sty{memoir} genutzt (\secref{use:fmt:len}). Dieser Abschnitt stellt ebenfalls
einige zusätzliche Längenregister vor, welche \cmd{biblistextra} von
\sty{memoir} entsprechen. \cmd{setbiblabel} bildet nicht auf eine einzelne
Einheit des \biblatex-Pakets ab, da der Stil aller Beschriftungen von dem
Bibliografiestil kontrolliert wird. Sehen Sie \secref{aut:bbx:env} in dem
Autorenabschnitt dieser Anleitung für Details. Falls die \sty{memoir}-Klasse
entdeckt wurde, stellt \sty{biblatex} ebenso folgende zusätzlichen Tests
bereit, welche nützlich für selbstdefinierte Titel sein könnten:

\begin{ltxsyntax}

\cmditem{ifmemoirbibintoc}{true}{false}

Wird zu \prm{true} oder \prm{false}, abhängig von den \cmd{bibintoc}- und
\cmd{nobibintoc}-Schaltern von \sty{memoir}. Dies ist ein \latex -rontend für
den \cmd{ifnobibintoc}-Test von \sty{memoir}. Man beachte, dass die Logik des
Test umgekehrt ist.

\end{ltxsyntax}

\subsubsection{Seitenzahlen in Zitaten} \label{use:cav:pag}

Falls das \prm{postnote}-Argument für den Zitierbefehl eine Seitenzahl
oder ein
Seitenbereich ist, wird \biblatex ihm automatisch ein <p.> oder <pp.>
voranstellen. Das funktioniert in den typischen Fällen zuverlässig, jedoch ist
manchmal manuelles Eingreifen nötig. In diesem Fall ist es wichtig zu
verstehen,
wie ein Argument ein Detail behandelt. Zuerst checkt \biblatex, ob die
Postnote eine arabische oder römische Zahl ist (case insensitive). Falls dieser
Test erfolgreich ist, wird die Postnote als einzelne Seite oder andere Nummer
betrachten, welche mit <p.> oder einem anderen String gepräfixt wird, abhängig
von dem \sty{pagination}-Feld (sehen Sie \secref{bib:use:pag}). Falls es
fehlschlägt, wird ein zweiter Test durchgeführt, um herauszufinden, ob die
Postnote ein Bereich ist oder eine Liste von arabischen oder römischen Zahlen.
Falls der Test erfolgreich ist, wird die Postnote mit <pp.> oder einem String in
der Pluralform gepräfixt. Falls dies ebenfalls fehlschlägt, wird die Postnote
gedruckt, wie sie ist. Man beachte, dass beide Tests \prm{postnote} erweitern.
Alle Kommandos, die in diesem Argument benutzt werden, müssen hierfür robust
sein oder mit \cmd{protect} gepräfixt sein. Hier sind ein paar Beispiele für
\prm{postnote}-Argumente, die korrekt als einzelne Nummer, einen Bereich von
Nummern oder einer Liste von Nummern erkannt werden:

\begin{ltxexample}
\cite[25]{key}
\cite[vii]{key}
\cite[XIV]{key}
\cite[34--38]{key}
\cite[iv--x]{key}
\cite[185/86]{key}
\cite[XI \& XV]{key}
\cite[3, 5, 7]{key}
\cite[vii--x; 5, 7]{key}
\end{ltxexample}
%
In einigen anderen Fällen können die Tests falsche Ergebnisse geben und man
muss auf die Hilfskommandos \cmd{pno}, \cmd{ppno} und \cmd{nopp} von
\secref{use:cit:msc} ausweichen. Zum Beispiel, angenommen eine Arbeit ist durch
ein besonderes Nummerierungsschema, bestehend aus Nummern und Buchstaben,
zitiert. In diesem Schema würde der String <|27a|> bedeuten: <page~7, part~a>.
Da dieser String nicht wie eine Nummer oder ein Bereich für \biblatex
aussieht, muss man das Präfix für eine einzelne Zahl manuell erzwingen:

\begin{ltxexample}
\cite[\pno~27a]{key}
\end{ltxexample}
%
Es gibt ebenso einen \cmd{ppno}-Befehl, welcher ein Bereichspräfix erzwingt,
und ebenso einen \cmd{nopp}-Befehl, welcher Präfixe unterbindet:

\begin{ltxexample}
\cite[\ppno~27a--28c]{key}
\cite[\nopp 25]{key}
\end{ltxexample}
%
Diese Befehle können überall im \prm{postnote}-Argument benutzt werden; 
auch mehrmals. Zum Beispiel, wenn man mit einem Band und
Seitenummer zitiert, möchte man vielleicht das Präfix am Anfang des Postnotes
unterbrechen und es in der Mitte des Strings hinzufügen:

\begin{ltxexample}
\cite[VII, \pno~5]{key}
\cite[VII, \pno~3, \ppno~40--45]{key}
\cite[see][\ppno~37--46, in particular \pno~40]{key}
\end{ltxexample}
%
Es gibt zwei Hilfskommandos für Suffixe, wie <die folgende(n) Seite(n)>. Statt
ein solches Suffix wörtlich einzufügen (was \cmd{ppno} benötigen würde, um
ein Präfix zu erzwingen),

\begin{ltxexample}
\cite[\ppno~27~sq.]{key}
\cite[\ppno~55~sqq.]{key}
\end{ltxexample}
%
benutzt man die Hilfskommandos \cmd{psq} und \cmd{psqq}. Man beachte, dass kein
Leerzeichen zwischen Nummer und Kommando ist. Dieses Leerzeichen wird
automatisch eingefügt und kann durch Umdefinierung des Makros \cmd{sqspace}
modifiziert werden.

\begin{ltxexample}
\cite[27\psq]{key}
\cite[55\psqq]{key}
\end{ltxexample}
%
Da die Postnote ohne Präfix gedruckt wird, falls es ein Zeichen enthält,
welches keine arabische oder römische Zahl ist, kann man das Präfix auch
manuell eintippen:

\begin{ltxexample}
\cite[p.~5]{key}
\end{ltxexample}
%
Es ist möglich, das Präfix über eine pro"=Eintrag Basis zu unterdrücken, indem
man das \bibfield{pagination}-Feld von einem Eintrag auf <\texttt{none}> setzt,
sehen Sie \secref{bib:use:pag} für Details. Falls man keine Präfixe will oder es
bevorzugt, diese manuell einzutippen, kann man ebenso den kompletten Mechanismus
in der Dokumentenpräambel oder in der Konfigurationsdatei wie folgt
deaktivieren:

\begin{ltxexample}
\DeclareFieldFormat{postnote}{#1}
\end{ltxexample}
%

Das \prm{postnote}-Argument wird als Feld behandelt und die Formatierung dieses
Feldes wird von einer Feldformatierungsrichtlinie kontrolliert, welche frei
umdefiniert werden kann. Die obere Definitionen druckt die Postnote, so wie sie
ist. Sehen Sie §§ 4.3.2 und 4.4.2 (e. V.)
%\secref{aut:cbx:fld, aut:bib:fmt} 
in der Autorenanleitung für weitere Details.

\subsubsection{Namenteile und Namenzwischenräume} \label{use:cav:nam}

Das \biblatex-Paket bietet den Nutzern und Stilautoren eine sehr gute
"`gekörnte"' Kontrolle der Namenzwischenräume und des Zeilenumbruchs.
Die folgenden diskutierten
Befehle sind in \secref{use:fmt:fmt,aut:fmt:fmt} dokumentiert. Dieses
Kapitel beabsichtigt, einen Überblick über ihre Zusammensetzung zu geben.
Ein Hinweis zur Terminologie: Ein Name
\emph{part} ist ein grundlegender Teil des Namens, beispielsweise der erste
oder letzte Name (Vorname oder Familienname). Jeder Namenteil kann ein
einzelner Name oder aus mehreren Teilen zusammengesetzt sein.
Beispielsweise, der Namenteil \enquote*{first name} (Vorname) kann aus
einem ersten und zweiten Vornamen bestehen. Letztere sind Namen-\emph{elements} in
diesem Abschnitt. Betrachten wir zuerst einen einfachen Namen: \enquote{John Edward Doe}.
Dieser Name ist aus folgenden Teilen zusammengesetzt:

\begin{nameparts} 
First	& John Edward \\ 
Prefix	& --- \\ 
Last	& Doe \\ 
Suffix  & --- \\ 
\end{nameparts}
%
Der Abstand und der Zeilenumbruch wird von vier Makros kontrolliert:

\begin{namedelims} 
a & \cmd{bibnamedelima} & Vom Backend nach dem ersten Element jedes
Namensteils eingesetzt, wenn dieses Element weniger als drei Zeichen lang
ist und vor dem letzten Namenselement aller Namensteile.  \\ 
b & \cmd{bibnamedelimb} & Vom Backend zwischen alle Namenteilelemente
eingefügt, wo \cmd{bibnamedelima} nicht anwendbar ist. \\ 
c & \cmd{bibnamedelimc} & Eingefügt von einer \emph{Formatting directive}
zwischen dem Namenspräfix und dem Nachnamen, wenn \kvopt{useprefix}{true}. 
Wenn \kvopt{useprefix}{false}, dann ist \cmd{bibnamedelimd} zu nehmen. \\
d & \cmd{bibnamedelimd} &  Eingefügt von einer \emph{Formatting directive}
zwischen Namensteilen, wo \cmd{bibnamedelimc}  nicht anwendbar ist. \\ 
i & \cmd{bibnamedelimi} & Ersetzt \cmd{bibnamedelima/b} nach Initialen
(großen Anfangsbuchstaben).\\
p & \cmd{revsdnamepunct} & Inserted by a formatting directive after the family name when the name parts are reversed.
\end{namedelims}
%
Nun, wie Trennzeichen verwendet werden:

\begin{namesample}
\item John\delim{~}{a}Edward\delim{ }{d}Doe
\item Doe\delim{,}{p}\delim{ }{d}John\delim{~}{a}Edward
\end{namesample}
%
Initialen erhalten in der \file{bib}-Datei ein spezielles Trennzeichen:

\begin{namesample}
\item J.\delim{~}{i}Edward\delim{ }{d}Doe
\end{namesample}
%
Nun wollen wir komplexere Namen betrachten:
\enquote{Charles-Jean Étienne Gustave
Nicolas de La Vallée Poussin}. 
Diese Namen setzt sich aus folgenden Teilen zusammen:

\begin{nameparts}
Given	& Charles-Jean Étienne Gustave Nicolas \\
Prefix	& de \\
Family	& La Vallée Poussin \\
Suffix	& --- \\
\end{nameparts}
%
Die Trennzeichen:

\begin{namesample} 
\item Charles-Jean\delim{~}{b}Étienne\delim{~}{b}Gustave\delim{~}{a}Nicolas\delim{
}{d}% 
de\delim{ }{c}% 
La\delim{~}{a}Vallée\delim{~}{a}Poussin 
\end{namesample}
%
Beachten Sie, dass \cmd{bibnamedelima/b/i} durch das Backend eingefügt wurde.
Das Backend behandelt die Namenteile und kümmert sich um die Trennzeichen
zwischen den Elementen, dies baut einen Namenteil um, jeden
Namenteil individuell behandelnd. Im Gegensatz dazu werden die
Trennzeichen zwischen den Teilen eines kompletten Namens 
(\cmd{bibnamedelimc/d}) entsprechend den \emph{name formatting directives}
zu einem späteren Zeitpunkt der Verarbeitungskette hinzugefügt.
Abstandserzeugung und Zeichensetzung von Initialen werden auch vom Backend
behandelt und kann durch eine Neudefinition der folgenden drei Makros
angepasst werden:

\begin{namedelims} 
a & \cmd{bibinitperiod} & Eingefügt vom Backend nach Initialen.\\ 
b & \cmd{bibinitdelim} & Eingefügt vom Backend zwischen mehreren Initialen.\\ 
c & \cmd{bibinithyphendelim} & Eingefügt vom Backend zwischen den Initialen
eines Bindestrich-Namens, ersetzend \cmd{bibinitperiod} und
\cmd{bibinitdelim}.\\ 
\end{namedelims}
%
So werden sie eingesetzt:

\begin{namesample} 
\item J\delim{.}{a}\delim{~}{b}E\delim{.}{a} Doe 
\item K\delim{.-}{c}H\delim{.}{a} Mustermann 
\end{namesample}

\subsubsection{Bibliografiefilter und Zitatbeschriftungen} \label{use:cav:lab}

Die Zitatbeschriftungen, die mit diesem Paket generiert werden, werden in die
Liste der Referenzen übertragen, bevor sie von irgendeinem Bibliografiefilter
geteilt wird. Sie sind garantiert einzigartig bezüglich des gesamten Dokuments
(oder einer \env{refsection}-Umgebung), egal wie viele Bibliografiefilter
genutzt werden. Wenn man ein nummerisches Zitateschema benutzt, wird dies
meistens zu einer diskontinuierlichen Nummerierung in geteilten Bibliografien
führen. Man nutze \opt{defernumbers}-Pakete, um dieses Problem zu umgehen. Falls
diese Option aktiviert ist, werden nummerische Beschriftungen übertragen, wenn
das erste mal ein Eintrag in irgendeiner Bibliografie ausgegeben wird.

\subsubsection{Aktive Zeichen in Bibliografietiteln} \label{use:cav:act}

Pakete, welche aktive Zeichen benutzen, wie \sty{babel}, \sty{csquotes} oder
\sty{underscore}, machen sie normalerweise nicht aktiv bis der Körper des
Dokuments beginnt, um Interferenzen mit anderen Paketen zu verhindern. Ein
typisches Beispiel eines solchen aktiven Zeichens ist das Ascii zitieren|"|,
welches in verschiedenen Sprachmodulen des \sty{babel}-Pakets benutzt wird.
Falls Kürzel wie |"<| und |"a| in einem Argument für \cmd{defbibheading}
benutzt werden und die Titel in der Dokumentenpräambel definiert sind, ist die
nichtaktive Form des Zeichens in der Titeldefinition gesichert. Wenn der Titel
ein Schriftsatz ist, funktionieren sie nicht als Befehle, sondern werden
einfach wörtlich gedruckt. Die direkteste Lösung besteht darin
\cmd{defbibheading} hinter |\begin{document}| zu verschieben. Alternativ kann
man die \cmd{shorthandon} und \cmd{shorthandoff} Befehle von \sty{babel}
nutzen, um die Kürzel temporär in der Präambel zu aktivieren. Das oben bezieht
sich ebenfalls auf Bibliografieanmerkungen und das \cmd{defbibnote}-Befehle.

\subsubsection{Gruppierung in Referenzabschnitten und -segmenten}
\label{use:cav:grp}

Alle \latex-Umgebungen, umgeben von \cmd{begin} und \cmd{end}, formen eine
Gruppe. Dies könnte unerwünschte Nebeneffekte haben, falls die Umgebung etwas
beinhaltet, dass nicht dazu gedacht ist, in einer Gruppe benutzt zu werden.
Dieses Problem ist nicht spezifisch für \env{refsection}- und 
\env{refsegment}-Umgebungen, allerdings gilt es offensichtlich auch für sie. 
Da diese Umgebungen
normalerweise einen größeren Anteil des Dokuments umschließen als eine
typische \env{itemize}- oder ähnliche Umgebung, rufen sie mit höherer
Wahrscheinlichkeit Probleme bezüglich Gruppierungen hervor. Falls man
Fehlfunktionen beobachtet, nachdem man \env{refsection}-Umgebungen zu einem
Dokument hinzufügte (zum Beispiel, falls irgendetwas in der Umgebung <gefangen>
zu sein scheint), probiere man stattdessen folgende Syntax:

\begin{ltxexample}
\chapter{...}
<<\refsection>>
...
<<\endrefsection>>
\end{ltxexample}
%
Dies wird keine Gruppe erzeugen, aber funktioniert sonst normal. Soweit
\biblatex betroffen ist, ist es egal, welche Syntax man benutzt. Die
alternative Syntax wird ebenso von der \env{refsegment}-Umgebung unterstützt.
Man beachte, dass die Befehle \cmd{newrefsection} und \cmd{newrefsegment}
keine Gruppe erzeugen. Sehen Sie \secref{use:bib:sec, use:bib:seg} für Details.

\subsection{Anwenden des alternativen \bibtex-Back\-ends}
\label{use:bibtex}

Um alle hier beschriebenen Funktionen zu nutzen,  muss \biblatex mit dem 
\biber-Programm als Backend benutzt werden. Tatsächlich setzt die Dokumentation dieses voraus. Doch für eine \emph{begrenzte} Anzahl von Anwendungsfällen ist es auch möglich,
das schon lange etablierte \bibtex-Programm zu nehmen, entweder das 7-bit \texttt{bibtex} oder 
8-bit \texttt{bibtex8} als das unterstützende Backend. Dies arbeitet in Vielem wie \biber, nur
mit der Einschränkung dass \bibtex viel mehr als Backend beschränkt ist.

Das Verwenden von \bibtex als Backend erfordert, dass die Option \opt{backend=bibtex}
oder \opt{backend=bibtex8} zum Laden gegeben wird. Das \biblatex-Paket wird dann die entsprechenden
Daten für die Verwendung mit \bibtex in die Hilfsdatei(en) schreiben und 
eine spezielle Datendatei (automatisch eingeschlossen zum Lesen von \bibtex).  Das
\bibtex(8)-Programm  sollte dann für jede Hilfsdatei ausgeführt werden:
\biblatex wird alle erforderlichen Dateien in der log-Datei aufführen.

Schlüsseleinschränkungen vom \bibtex-Backend sind:
\begin{itemize}
\item Die Sortierung ist global und limitiert von Ascii.
\item Keine erneute Codierung ist möglich und somit müssen die Datenbankeinträge
  in der LICR-Form sein, um zuverlässig arbeiten zu können.
\item Das Datenmodell ist fest.
\item Querverweisen ist begrenzter und die Eintragssätze müssen in die \path{.bib}-Datei geschrieben werden. 
\item Feste Speicherkapazität (nehmend die \verb|--wolfgang|-Option mit
  \verb|bibtex8| ist streng empfohlen, um die Wahrscheinlichkeit eine Debatte dazu hier zu minimieren).
 
\end{itemize}

\end{document}


\newpage

\section{Versionsgeschichte} \label{apx:log}

Diese Versionsgeschichte ist eine Liste von Änderungen, die für den Nutzer des
Pakets von Bedeutung sind. Änderungen, die eher technischer Natur sind und für
den Nutzer des Pakets nicht relevant sind und das Verhalten des Pakets nicht
ändern, werden nicht aufgeführt. Wenn ein Eintrag der Versionsgeschichte ein
Feature als \emph{improved} oder \emph{extended} bekannt gibt, so bedeutet dies,
dass eine Modifikation die Syntax und das Verhalten des Pakets nicht
beeinflusst, oder das es für ältere Versionen kompatibel ist. Einträge, die
als \emph{modified}, \emph{renamed}, oder \emph{removed} deklariert sind,
verlangen besondere Aufmerksamkeit. Diese bedeuten, dass eine Modifikation
Änderungen in bereits existierenden Stilen oder Dokumenten mit sich zieht. Die
Zahlen an der rechten Seite stehen für die relevante Stelle dieser
Dokumentation.

\begin{changelog}
\begin{release}{3.14}{2019-12-01}
\item Added new mapping verbs for citation sources\see{aut:ctm:map}
\item Added documentation for new \biber granular \bibtype{xdata} functionality\see{use:use:xdat}
\item Enhanced \sty{polyglossia} support
\end{release}
\begin{release}{3.13a}{2019-08-31}
\item Bugfix release
\end{release}
\begin{release}{3.13}{2019-08-17}
\item Added new \bibtype{dataset} entry type\see{bib:typ:blx}
\item Promoted \bibtype{software} to regular entry type\see{bib:typ:blx}
\item Added \bibfield{entrykey} alias for entry keys in labels\see{aut:ctm:lab}
\item Added \opt{appendstrict} sourcemapping option\see{aut:ctm:map}
\item Added \opt{nohashothers} and \opt{nosortothers}\see{use:opt:pre:int}
\item Enhanced \cmd{addbibresource} with globbing\see{use:bib:res}
\item Added \cmd{DeclareBiblatexOption}\see{aut:bbx:bbx}
\item Expanded scope possibilities for several options\see{apx:opt}
\item Added \cmd{ifvolcite} test\see{aut:aux:tst}
\item Added special fields \bibfield{volcitevolume} and \bibfield{volcitepages}%
  \see{aut:cbx:fld}
\item Added \cmd{AtVolcite} hook\see{aut:fmt:hok}
\item Added \cmd{pnfmt} \see{use:cit:msc}
\item Added \cmd{mkbibcompletename} and \cmd{mkbibcompletename<formatorder>}\see{use:fmt:fmt}
\item Made \cmd{postnotedelim} and friends context sensitive\see{use:fmt:fmt}
\item Added \cmd{multipostnotedelim} and \cmd{multiprenotedelim}\see{use:fmt:fmt}
\item Added \cmd{thefirstlistitem} and friends\see{aut:aux:dat}
\item Added \prm{itempostpro} argument to \cmd{mkcomprange}, \cmd{mknormrange} and \cmd{mkfirstpage}\see{aut:aux:msc}
\item Added \len{biburlbigskip} and friends\see{use:fmt:len}
\item Added \cnt{biburlbigbreakpenalty} and \cnt{biburlbreakpenalty} and friends\see{use:fmt:len}
\item Added \cmd{DeclarePrintbibliographyDefaults}\see{use:bib:bib}
\item Added \bibfield{doi} to \bibtype{online}\see{bib:typ:blx}
\end{release}
\begin{release}{3.12}{2018-10-30}
\item Added literal and named annotation functionality\see{use:annote}
\item Added \cmd{ifnocite}\see{aut:aux:tst}
\item Added case-insensitive versions of matching operators\see{aut:ctm:map}
\item Added \bibfield{langid}s optional argument to \cmd{DeclareSortTranslit}\see{aut:ctm:srt}
\item Added \opt{noroman} option\see{use:opt:pre:int}
\item Changed \bibfield{sortyear} to an integer field\see{bib:fld:spc}
\item Added \bibfield{extraname}\see{aut:bbx:fld:lab}
\item Added \opt{bibencoding} option to \cmd{addbibresource}\see{use:bib:res}
\item Changed type of \bibfield{number} from integer to literal \see{bib:fld:dat}
\item Removed \opt{noerroretextools} option\see{int:pre:inc}
\item Added \opt{maxsortnames} and \opt{minsortnames}\see{use:opt:pre:gen}
\item Added \cmd{DeprecateFieldFormatWithReplacement} and friends\see{aut:bib:fmt}
\item Added list and name wrappers\see{aut:bib:fmt}
\item Added \cs{ifdateyearsequal}\see{aut:aux:tst}
\item Added <and higher> sectioning values for \opt{citereset}, \opt{refsection} and \opt{refsegment} options\see{use:opt:pre:gen}
\item Added Hungarian localisation\see{use:loc:hun}
\item Added \cmd{DeclareCitePunctuationPosition}\see{aut:cbx:cbx}
\end{release}
\begin{release}{3.11}{2018-02-20}
\item Added \opt{entrynocite} option to sourcemapping\see{aut:ctm:map}
\item Added \opt{driver} and \opt{biblistfilter} options to \cmd{printbiblist}\see{use:bib:biblist}
\item Added \cmd{mknormrange}\see{aut:aux:msc}
\item Added \cmd{ifdateannotation}\see{use:annote}
\item Extended \cmd{iffieldannotation} and friends\see{use:annote}
\item Changed \cmd{DeclareSourcemap} so that it can be used multiple times\see{aut:ctm:map}
\item Added Latvian localisation (Rihards Skuja)
\item Added \opt{locallabelwidth} option\see{use:opt:pre:gen}
\end{release}

\begin{release}{3.10}{2017-12-19}
\item Changed \opt{edtf} to \opt{iso}\see{use:opt:pre:gen}
\item Added \opt{noerroretextools} option\see{int:pre:inc}
\end{release}

\begin{release}{3.9}{2017-11-21}
\item Added \cmd{iffieldplusstringbibstring}\see{aut:aux:tst}
\item Fixed \cmd{mkpagetotal}\see{aut:aux:msc}
\end{release}
\begin{release}{3.8}{2017-11-04}
\item Added \kvopt{hyperref}{manual} option\see{use:opt:pre:gen}
\item Added field \bibfield{extradatescope}\see{aut:bbx:fld:lab}
\item Added \cmd{DeclareExtradate}\see{aut:ctm:fld}
\item Added \cmd{DeprecateFieldWithReplacement}, \cmd{DeprecateListWithReplacement} and \cmd{DeprecateNameWithReplacement}\see{aut:bib:dat}
\item Added \cmd{letbibmacro}\see{aut:aux:msc}
\item Renamed \opt{extrayear} to \opt{extradate}\see{aut:bbx:fld:lab}
\item Added \opt{sortsets} global option\see{use:opt:pre:gen}
\item Added \cmd{iflabelalphanametemplatename} and \cmd{uniquenametemplatename}\see{aut:aux:tst}
\item Renamed \cmd{ifsortingnamescheme} to \cmd{ifsortingnamekeytemplatename}\see{aut:aux:tst}
\item Renamed \opt{sortingnamekeyscheme} to \opt{sortingnamekeytemplate}\see{use:bib:context}
\item Renamed \cmd{DeclareSortingNamekeyScheme} to \cmd{DeclareSortingNamekeyTemplate}\see{aut:ctm:srt}
\item Renamed \cmd{DeclareSortingScheme} to \cmd{DeclareSortingTemplate}\see{aut:ctm:srt}
\item Changes to \cmd{DeclareUniquenameTemplate} and \cmd{DeclareLabelalphaNameTemplate} scopes\see{aut:cav:amb} and \see{aut:ctm:lab}
\item Added new \opt{disambiguation} option to \cmd{DeclareUniquenameTemplate}\see{aut:cav:amb}
\item Added new user-facing versions of some entry-querying commands\see{use:eq}
\item Changed \bibfield{origlanguage} to a list in line with \bibfield{language}\see{bib:fld:dat}
\item Deprecated \bibfield{childentrykey} and \bibfield{childentrytype}\see{aut:bbx:fld:gen}
\item Added \bibfield{bibnamehash} and name list specific variants\see{aut:bbx:fld:gen}
\item Added ALA-LC Russian romanisation transliteration support\see{aut:ctm:srt}
\item Added \bibfield{urlraw}\see{aut:bbx:fld:gen}
\item Added \cmd{AtUsedriver}\see{aut:fmt:hok}
\item Added Bulgarian localisation (Kaloyan Ganev)
\item \bibfield{sortyear} is now a literal, not an integer\see{bib:fld:spc}
\item Added \cmd{DeclareLanguageMappingSuffix}\see{aut:lng:cmd}
\item Changed default for \cmd{DeclarePrefChars}\see{aut:pct:cfg}
\item Added \cmd{authortypedelim}, \cmd{editortypedelim} and \cmd{translatortypedelim}\see{use:fmt:fmt}
\item Added \cmd{DeclareDelimAlias}\see{use:fmt:csd}
\item Added \opt{slovenian} as alias for \opt{slovene} due to Polyglossia
  name for the language\see{bib:fld:spc}
\item Added Ukrainian localisation (Sergiy M. Ponomarenko)
\end{release}
    

\begin{release}{3.7}{2016-12-08}
\item Corrected default for \cmd{bibdateeraprefix}\see{aut:fmt:lng}
\item Added \cmd{DeclareSortInclusion}\see{aut:ctm:srt}
\item Added \cmd{relateddelim$<$relatedtype$>$}\see{use:fmt:fmt}
\end{release}

\begin{release}{3.6}{2016-09-15}
\item Corrected some documentation and fixed a bug with labeldate
  localisation strings.
\end{release}

\begin{release}{3.5}{2016-09-10}
\item Added \cmd{ifuniquebaretitle} test\see{aut:aux:tst}
\item Documented \cmd{labelnamesource} and \cmd{labeltitlesource}\see{aut:bbx:fld:gen}
\item Added \cmd{bibdaterangesep}\see{use:fmt:lng}
\item Added \opt{refsection} option to \cmd{DeclareSourcemap}\see{aut:ctm:map}
\item Added \opt{suppress} option to inheritance specifications\see{aut:ctm:ref}
\item Added \cmd{ifuniquework}\see{aut:aux:tst}
\item Changed \cmd{DeclareStyleSourcemap} so that it can be used multiple times\see{aut:ctm:map}
\item Added \cmd{forcezerosy} and \cmd{forcezerosmdt}\see{aut:fmt:ich}
\item Changed \cmd{mkdatezeros} to \cmd{mkyearzeros}, \cmd{mkmonthszeros}
  and \cmd{mkdayzeros}\see{aut:fmt:ich}
\item Added \bibfield{namehash} and \bibfield{fullhash} for all name list fields\see{aut:bbx:fld:gen}
\item Generalised \opt{giveninits} option to all nameparts\see{use:opt:pre:int}
\item Added \opt{inits} option to \cmd{DeclareSortingNamekeyScheme}\see{aut:ctm:srt}
\item Added \cmd{DeclareLabelalphaNameTemplate}\see{aut:ctm:lab}
\item Added full \acr{EDTF} Levels 0 and 1 compliance for parsing and printing times\see{bib:use:dat}
\item Changed dates to be fully \acr{EDTF} Levels 0 and 1 compliant. Associated tests and localisation strings\see{bib:use:dat}
\item Added \opt{timezeros}\see{use:opt:pre:gen}
\item Added \opt{mktimezeros}\see{aut:fmt:ich}
\item Changed \opt{iso8601} to \opt{edtf}\see{use:opt:pre:gen}
\item Added \cmd{DeclareUniquenameTemplate}\see{aut:cav:amb}
\item Removed experimental RIS support
\item \opt{sortnamekeyscheme} and \opt{useprefix} can be now be set per-namelist and per-name for
  \bibtex datasources\see{aut:ctm:srt}
\item Added \cmd{DeclareDelimcontextAlias}\see{use:fmt:csd}
\item Added Estonian localisation (Benson Muite)
\item Reference contexts may now be named\see{use:bib:context}
\item Added \opt{notfield} step in Sourcemaps\see{aut:ctm:map}
\end{release}

\begin{release}{3.4}{2016-05-10}
\item Added \cmd{ifcrossrefsource} and \cmd{ifxrefsource}\see{aut:aux:tst}
\item Added data annotation feature\see{use:annote}
\item Added package option \opt{minxrefs}\see{use:opt:pre:gen}
\item Added \cmd{ifuniqueprimaryauthor} and associated global option\see{aut:aux:tst}
\item Added \cmd{DeprecateField}, \cmd{DeprecateList} and \cmd{DeprecateName}\see{aut:bib:dat}
\item Added \cmd{ifcaselang}\see{aut:aux:tst}
\item Added \cmd{DeclareSortTranslit}\see{aut:ctm:srt}
\item Added \opt{uniquetitle} test\see{aut:aux:tst}
\item Added \cmd{namelabeldelim}\see{use:fmt:fmt}
\item New starred variants of the \cmd{assignrefcontext*} macros\see{use:bib:context}
\item New context-sensitive delimiter interface\see{use:fmt:csd}
\item Moved \opt{prefixnumbers} option to \cmd{newrefcontext} and renamed to \opt{labelprefix}\see{use:bib:context}
\item Added \cmd{DeclareDatafieldSet}\see{aut:ctm:dsets}
\end{release}


\begin{release}{3.3}{2016-03-01}
\item Schema documentation for \biblatexml\see{apx:biblatexml}\BiberOnlyMark
\item Sourcemapping documentation and examples for \biblatexml\see{aut:ctm:map}\BiberOnlyMark
\item Changes for name formats to generalise available name parts\see{aut:bib:fmt}\BiberOnlyMark
\item \opt{useprefix} can now be specified per-namelist and per-name in \biblatexml datasources
\item New sourcemapping options for creating new entries dynamically and looping over map steps\see{aut:ctm:map}\BiberOnlyMark
\item Added \opt{noalphaothers} and enhanced name range selection in \cmd{DeclareLabelalphaTemplate}\see{aut:ctm:lab}
\item Added \cmd{DeclareDatamodelConstant}\see{aut:ctm:dm}\BiberOnlyMark
\item Renamed \opt{firstinits} and \opt{sortfirstinits}
\item Added \cmd{DeclareSortingNamekeyScheme}\see{aut:ctm:srt}
\item Removed messy experimental endnote and zoterordf support for \biber
\item Added \cmd{nonameyeardelim}\see{use:fmt:fmt}
\item Added \cmd{extpostnotedelim}\see{use:fmt:fmt}
\end{release}

\begin{release}{3.2}{2015-12-28}
\item Added \opt{pstrwidth} and \opt{pcompound} to \cmd{DeclareLabelalphaTemplate}\see{aut:ctm:lab}\BiberOnlyMark
\item Added \cmd{AtEachCitekey}\see{aut:fmt:hok}
\end{release}

\begin{release}{3.1}{2015-09}
\item Added \cmd{DeclareNolabel}\see{aut:ctm:lab}\BiberOnlyMark
\item Added \cmd{DeclareNolabelwidthcount}\see{aut:ctm:lab}\BiberOnlyMark
\end{release}

\begin{release}{3.0}{2015-04-20}
\item Improved Danish (Jonas Nyrup) and Spanish (ludenticus) translations
\item \bibfield{labelname} and \bibfield{labeltitle} are now resolved by \biblatex instead of \biber for more flexibility and future extensibility
\item New \cmd{entryclone} sourcemap verb for cloning entries during sourcemapping\see{aut:ctm:map}
\item New \cmd{pernottype} negated per-type sourcemap verb\see{aut:ctm:map}
\item New range calculation command \cmd{frangelen}\see{aut:aux:msc}
\item New bibliography context functionality\see{use:bib:context}
\item Name lists in the data model now automatically create internals for \cmd{ifuse$<$name$>$} tests and booleans\see{use:opt:bib:hyb} and \see{aut:aux:tst}
\end{release}


\begin{release}{2.9a}{2014-06-25}
\item \texttt{resetnumbers} now allows passing a number to reset to\see{use:bib:bib}
\end{release}

\begin{release}{2.9}{2014-02-25}
\item Generalised shorthands facility\see{use:bib:biblist}\BiberOnlyMark
\item Sorting locales can now be defined as part of a sorting scheme\see{aut:ctm:srt}\BiberOnlyMark
\item Added \bibfield{sortinithash}\see{aut:bbx:fld:gen}\BiberOnlyMark
\item Added Slovene localisation (Tea Tušar and Bogdan Filipič)
\item Added \cmd{mkbibitalic}\see{aut:fmt:ich}
\item Recommend \texttt{begentry} and \texttt{finentry} bibliography macros\see{aut:bbx:drv}
\end{release}
\begin{release}{2.8a}{2013-11-25}
\item Split option \opt{language=auto} into \opt{language=autocite} and \opt{language=autobib}\see{use:opt:pre:gen}\BiberOnlyMark
\end{release}

\begin{release}{2.8}{2013-10-21}
\item New \bibfield{langidopts}\see{bib:fld:spc}\BiberOnlyMark
\item \bibfield{hyphenation} field renamed to \bibfield{langid}\see{bib:fld:spc}
\item \sty{polyglossia} support
\item Renamed \opt{babel} option to \opt{autolang}\see{use:opt:pre:gen}
\item Corrected Dutch localisation
\item Added \opt{datelabel=year} option\see{use:opt:pre:gen}
\item Added \bibfield{datelabelsource} field\see{aut:bbx:fld:gen}
\end{release}

\begin{release}{2.7a}{2013-07-14}
\item Bugfix - respect maxnames and uniquelist in \cmd{finalandsemicolon}
\item Corrected French localisation
\end{release}

\begin{release}{2.7}{2013-07-07}
\item Added field \bibfield{eventtitleaddon} to default datamodel and styles\see{bib:fld:dat}
\item Added \cmd{ifentryinbib}, \cmd{iffirstcitekey} and \cmd{iflastcitekey}\see{aut:aux:tst}
\item Added \bibfield{postpunct} special field, documented \bibfield{multiprenote} and \bibfield{multipostnote} special fields\see{aut:cbx:fld}
\item Added \cmd{UseBibitemHook}, \cmd{AtEveryMultiCite}, \cmd{AtNextMultiCite}, \cmd{UseEveryCiteHook}, \cmd{UseEveryCitekeyHook}, \cmd{UseEveryMultiCiteHook}, \cmd{UseNextCiteHook}, \cmd{UseNextCitekeyHook}, \cmd{UseNextMultiCiteHook}, \cmd{DeferNextCitekeyHook}\see{aut:fmt:hok}
\item Fixed \cmd{textcite} and related commands in the numeric and verbose styles\see{use:cit:cbx}
\item Added multicite variants of \cmd{volcite} and related commands\see{use:cit:spc}
\item Added \cmd{finalandsemicolon}\see{use:fmt:lng}
\item Added citation delimiter \cmd{textcitedelim} for \cmd{textcite} and related commands to styles\see{aut:fmt:fmt}
\item Updated Russian localization (Oleg Domanov)
\item Fixed Brazilian and Finnish localization
\end{release}

\begin{release}{2.6}{2013-04-30}
\item Added \cmd{printunit}\see{aut:pct:new}
\item Added field \bibfield{clonesourcekey}\see{aut:bbx:fld:gen}\BiberOnlyMark
\item New options for \cmd{DeclareLabelalphaTemplate}\see{aut:ctm:lab}\BiberOnlyMark
\item Added \cmd{DeclareLabeldate} and retired \cmd{DeclareLabelyear}\see{aut:ctm:fld}\BiberOnlyMark
\item Added \texttt{nodate} localization string\see{aut:lng:key:msc}
\item Added \cmd{rangelen}\see{aut:aux:msc}
\item Added starred variants of \cmd{citeauthor} and \cmd{Citeauthor}\see{use:cit:txt}
\item Restored original \texttt{url} format. Added \texttt{urlfrom} localization key\see{aut:lng:key:lab}
\item Added \cmd{AtNextBibliography}\see{aut:fmt:hok}
\item Fixed related entry processing to allow nested and cyclic related entries
\item Added Croatian localization (Ivo Pletikosić)
\item Added Polish localization (Anastasia Kandulina, Yuriy Chernyshov)
\item Fixed Catalan localization
\item Added smart ``of'' for titles to Catalan and French localization
\item Misc bug fixes
\end{release}

\begin{release}{2.5}{2013-01-10}
\item Made \texttt{url} work as a localization string, defaulting to previously hard-coded value <\textsc{URL}>.
\item Changed some \biber\ option names to cohere with \biber\ 1.5.
\item New sourcemap step for conditionally removing entire entries\see{aut:ctm:map}\BiberOnlyMark
\item Updated Catalan localization (Sebastià Vila-Marta)
\end{release}

\begin{release}{2.4}{2012-11-28}
\item Added \bibfield{relatedoptions} field\see{aut:ctm:rel}\BiberOnlyMark
\item Added \cmd{DeclareStyleSourcemap}\see{aut:ctm:map}\BiberOnlyMark
\item Renamed \cmd{DeclareDefaultSourcemap} to \cmd{DeclareDriverSourcemap}\see{aut:ctm:map}\BiberOnlyMark
\item Documented \cmd{DeclareFieldInputHandler}, \cmd{DeclareListInputHandler} and \cmd{DeclareNameInputHandler}.
\item Added Czech localization (Michal Hoftich)
\item Updated Catalan localization (Sebastià Vila-Marta)
\end{release}

\begin{release}{2.3}{2012-11-01}
\item Better detection of situations which require a \biber\ or \LaTeX\ re-run
\item New append mode for \cmd{DeclareSourcemap} so that fields can be combined\see{aut:ctm:map}\BiberOnlyMark
\item Extended auxiliary indexing macros
\item Added support for plural localization strings with \bibfield{relatedtype}\see{aut:ctm:rel}\BiberOnlyMark
\item Added \cmd{csfield} and \cmd{usefield}\see{aut:aux:dat}
\item Added starred variant of \cmd{usebibmacro}\see{aut:aux:msc}
\item Added \cmd{ifbibmacroundef}, \cmd{iffieldformatundef}, \cmd{iflistformatundef}
  and \cmd{ifnameformatundef}\see{aut:aux:msc}
\item Added Catalan localization (Sebastià Vila-Marta)
\item Misc bug fixes
\end{release}

\begin{release}{2.2}{2012-08-17}
\item Misc bug fixes
\item Added \cmd{revsdnamepunct}\see{use:fmt:fmt}
\item Added \cmd{ifterseinits}\see{aut:aux:tst}
\end{release}

\begin{release}{2.1}{2012-08-01}
\item Misc bug fixes
\item Updated Norwegian localization (Håkon Malmedal)
\item Increased data model auto-loading possibilities\see{aut:ctm:dm}\BiberOnlyMark
\end{release}

\begin{release}{2.0}{2012-07-01}
\item Misc bug fixes
\item Generalised \opt{singletitle} test a little\see{aut:aux:tst}\BiberOnlyMark
\item Added new special field \bibfield{extratitleyear}\see{aut:bbx:fld}\BiberOnlyMark
\item Customisable data model\see{aut:ctm:dm}\BiberOnlyMark
\item Added \cmd{DeclareDefaultSourcemap}\see{aut:ctm:map}\BiberOnlyMark
\item Added \opt{labeltitle} option\see{use:opt:pre:int}\BiberOnlyMark
\item Added new special field \bibfield{extratitle}\see{aut:bbx:fld}\BiberOnlyMark
\item Made special field \bibfield{labeltitle} customisable\see{aut:bbx:fld}\BiberOnlyMark
\item Removed field \bibfield{reprinttitle}\see{use:rel}\BiberOnlyMark
\item Added related entry feature\see{use:rel}\BiberOnlyMark
\item Added \cmd{DeclareNoinit}\see{aut:ctm:noinit}\BiberOnlyMark
\item Added \cmd{DeclareNosort}\see{aut:ctm:nosort}\BiberOnlyMark
\item Added \opt{sorting} option for \cmd{printbibliography} and \cmd{printshorthands}\see{use:bib:bib}\BiberOnlyMark
\item Added \texttt{ids} field for citekey aliasing\see{bib:fld}\BiberOnlyMark
\item Added \opt{sortfirstinits} option\see{use:opt:pre:int}\BiberOnlyMark
\item Added data stream modification feature\see{aut:ctm:map}\BiberOnlyMark
\item Added customisable labels feature\see{aut:ctm:lab}\BiberOnlyMark
\item Added \cmd{citeyear*} and \cmd{citedate*}\see{use:cit:txt}
\end{release}

% \begin{release}{1.7}{2011-11-13}
% \item Added \texttt{xdata} containers\see{use:use:xdat}\BiberOnlyMark
% \item Added special entry type \bibfield{xdata}\see{bib:typ:blx}\BiberOnlyMark
% \item Added special field \bibfield{xdata}\see{bib:fld:spc}\BiberOnlyMark
% \item Support \opt{maxnames}/\opt{minnames} globally/per-type/per-entry\see{use:opt:pre:gen}\BiberOnlyMark
% \item Support \opt{maxbibnames}/\opt{minbibnames} globally/per-type/per-entry\see{use:opt:pre:gen}\BiberOnlyMark
% \item Support \opt{maxcitenames}/\opt{mincitenames} globally/per-type/per-entry\see{use:opt:pre:gen}\BiberOnlyMark
% \item Support \opt{maxitems}/\opt{minitems} globally/per-type/per-entry\see{use:opt:pre:gen}\BiberOnlyMark
% \item Support \opt{maxalphanames}/\opt{minalphanames} globally/per-type/per-entry\see{use:opt:pre:int}\BiberOnlyMark
% \item Support \opt{uniquelist} globally/per-type/per-entry\see{use:opt:pre:int}\BiberOnlyMark
% \item Support \opt{uniquename} globally/per-type/per-entry\see{use:opt:pre:int}\BiberOnlyMark
% \item Added \cmd{textcite} and \cmd{textcites} to all \texttt{verbose} citation styles\see{use:xbx:cbx}
% \item Added special field formats \texttt{date}, \texttt{urldate}, \texttt{origdate}, \texttt{eventdate}\see{aut:fmt:ich}
% \item Added \cmd{mkcomprange*}\see{aut:aux:msc}
% \item Added \cmd{mkfirstpage*}\see{aut:aux:msc}
% \item Added \cmd{volcitedelim}\see{aut:fmt:fmt}
% \item Added missing documentation for \cmd{ifentrytype}\see{aut:aux:tst}
% \item Added \cmd{mkbibneutord}\see{use:fmt:lng}
% \item Added counter \cnt{biburlnumpenalty}\see{aut:fmt:len}
% \item Added counter \cnt{biburlucpenalty}\see{aut:fmt:len}
% \item Added counter \cnt{biburllcpenalty}\see{aut:fmt:len}
% \item Added localization keys \texttt{book}, \texttt{part}, \texttt{issue}, \texttt{forthcoming}\see{aut:lng:key}
% \item Added some \texttt{lang...} and \texttt{from...} localization keys\see{aut:lng:key}
% \item Expanded documentation\see{apx:opt}
% \item Added support for Russian (Oleg Domanov)
% \item Updated support for Dutch (Pieter Belmans)
% \item Fixed compatibility issue with \sty{textcase} package
% \item Fixed some bugs
% \end{release}

% \begin{release}{1.6}{2011-07-29}
% \item Added special field \bibfield{sortshorthand}\see{bib:fld:spc}\BiberOnlyMark
% \item Revised options \opt{maxnames}/\opt{minnames}\see{use:opt:pre:gen}
% \item Options \opt{maxcitenames}/\opt{mincitenames} now supported by backend\see{use:opt:pre:gen}\BiberOnlyMark
% \item Options \opt{maxbibnames}/\opt{minbibnames} now supported by backend\see{use:opt:pre:gen}\BiberOnlyMark
% \item Added options \opt{maxalphanames}/\opt{minalphanames}\see{use:opt:pre:int}\BiberOnlyMark
% \item Removed local options \opt{maxnames}/\opt{minnames} from \cmd{printbibliography}\see{use:bib:bib}
% \item Removed local options \opt{maxitems}/\opt{minitems} from \cmd{printbibliography}\see{use:bib:bib}
% \item Removed local options \opt{maxnames}/\opt{minnames} from \cmd{bibbysection}\see{use:bib:bib}
% \item Removed local options \opt{maxitems}/\opt{minitems} from \cmd{bibbysection}\see{use:bib:bib}
% \item Removed local options \opt{maxnames}/\opt{minnames} from \cmd{bibbysegment}\see{use:bib:bib}
% \item Removed local options \opt{maxitems}/\opt{minitems} from \cmd{bibbysegment}\see{use:bib:bib}
% \item Removed local options \opt{maxnames}/\opt{minnames} from \cmd{bibbycategory}\see{use:bib:bib}
% \item Removed local options \opt{maxitems}/\opt{minitems} from \cmd{bibbycategory}\see{use:bib:bib}
% \item Removed local options \opt{maxnames}/\opt{minnames} from \cmd{printshorthands}\see{use:bib:biblist}
% \item Removed local options \opt{maxitems}/\opt{minitems} from \cmd{printshorthands}\see{use:bib:biblist}
% \item Added special field format \bibfield{volcitevolume}\see{use:cit:spc}
% \item Added special field format \bibfield{volcitepages}\see{use:cit:spc}
% \item Added special field \bibfield{hash}\see{aut:bbx:fld:gen}\BiberOnlyMark
% \item Added \cmd{mkcomprange}\see{aut:aux:msc}
% \item Added \cmd{mkfirstpage}\see{aut:aux:msc}
% \item Removed \cmd{mkpagefirst}\see{aut:aux:msc}
% \item Fixed some bugs
% \end{release}

% \begin{release}{1.5a}{2011-06-17}
% \item Fixed some bugs
% \end{release}

% \begin{release}{1.5}{2011-06-08}
% \item Added option \kvopt{uniquename}{mininit/minfull}\see{use:opt:pre:int}\BiberOnlyMark
% \item Added option \kvopt{uniquelist}{minyear}\see{use:opt:pre:int}\BiberOnlyMark
% \item Updated documentation of \cnt{uniquename} counter\see{aut:aux:tst}\BiberOnlyMark
% \item Updated documentation of \cnt{uniquelist} counter\see{aut:aux:tst}\BiberOnlyMark
% \item Expanded documentation for \opt{uniquename/uniquelist} options\see{aut:cav:amb}\BiberOnlyMark
% \item Added editorial role \texttt{reviser}\see{bib:use:edr}
% \item Added localization keys \texttt{reviser}, \texttt{revisers}, \texttt{byreviser}\see{aut:lng:key}
% \item Added bibliography heading \texttt{none}\see{use:bib:hdg}
% \item Fixed some \sty{memoir} compatibility issues
% \end{release}

% \begin{release}{1.4c}{2011-05-12}
% \item Fixed some bugs
% \end{release}

% \begin{release}{1.4b}{2011-04-12}
% \item Fixed some bugs
% \end{release}

% \begin{release}{1.4a}{2011-04-06}
% \item Enable \opt{uniquename} and \opt{uniquelist} in all \texttt{authortitle} styles\see{use:xbx:cbx}
% \item Enable \opt{uniquename} and \opt{uniquelist} in all \texttt{authoryear} styles\see{use:xbx:cbx}
% \item Fixed some bugs
% \end{release}

% \begin{release}{1.4}{2011-03-31}
% \item Added package option \opt{uniquelist}\see{use:opt:pre:int}\BiberOnlyMark
% \item Added special counter \cnt{uniquelist}\see{aut:aux:tst}\BiberOnlyMark
% \item Revised and improved package option \opt{uniquename}\see{use:opt:pre:int}\BiberOnlyMark
% \item Revised and improved special counter \cnt{uniquename}\see{aut:aux:tst}\BiberOnlyMark
% \item Added \cmd{bibnamedelimi}\see{use:fmt:fmt}\BiberOnlyMark
% \item Added \cmd{bibindexnamedelima}\see{use:fmt:fmt}
% \item Added \cmd{bibindexnamedelimb}\see{use:fmt:fmt}
% \item Added \cmd{bibindexnamedelimc}\see{use:fmt:fmt}
% \item Added \cmd{bibindexnamedelimd}\see{use:fmt:fmt}
% \item Added \cmd{bibindexnamedelimi}\see{use:fmt:fmt}
% \item Added \cmd{bibindexinitperiod}\see{use:fmt:fmt}
% \item Added \cmd{bibindexinitdelim}\see{use:fmt:fmt}
% \item Added \cmd{bibindexinithyphendelim}\see{use:fmt:fmt}
% \item Fixed conflict with some \acr{AMS} classes
% \end{release}

% \begin{release}{1.3a}{2011-03-18}
% \item Fixed some bugs
% \end{release}

% \begin{release}{1.3}{2011-03-14}
% \item Support \bibtype{thesis} with \bibfield{isbn}\see{bib:typ:blx}
% \item Updated \opt{terseinits} option\see{use:opt:pre:gen}
% \item Allow macros in argument to \cmd{addbibresource} and friends\see{use:bib:res}
% \item Allow macros in argument to \cmd{bibliography}\see{use:bib:res}
% \item Introducing experimental support for Zotero \acr{RDF}/\acr{XML}\see{use:bib:res}\BiberOnlyMark
% \item Introducing experimental support for EndNote \acr{XML}\see{use:bib:res}\BiberOnlyMark
% \item Added option \opt{citecounter}\see{use:opt:pre:int}
% \item Added \cnt{citecounter}\see{aut:aux:tst}
% \item Added \cmd{smartcite} and \cmd{Smartcite}\see{use:cit:cbx}
% \item Added \cmd{smartcites} and \cmd{Smartcites}\see{use:cit:mlt}
% \item Added \cmd{svolcite} and \cmd{Svolcite}\see{use:cit:spc}
% \item Added \cmd{bibnamedelima}\see{use:fmt:fmt}\BiberOnlyMark
% \item Added \cmd{bibnamedelimb}\see{use:fmt:fmt}\BiberOnlyMark
% \item Added \cmd{bibnamedelimc}\see{use:fmt:fmt}
% \item Added \cmd{bibnamedelimd}\see{use:fmt:fmt}
% \item Added \cmd{bibinitperiod}\see{use:fmt:fmt}\BiberOnlyMark
% \item Added \cmd{bibinitdelim}\see{use:fmt:fmt}\BiberOnlyMark
% \item Added \cmd{bibinithyphendelim}\see{use:fmt:fmt}\BiberOnlyMark
% \item Expanded documentation\see{use:cav:nam}
% \item Added \prm{position} parameter \texttt{f} to \cmd{DeclareAutoCiteCommand}\see{aut:cbx:cbx}
% \end{release}

% \begin{release}{1.2a}{2011-02-13}
% \item Fix in \cmd{mkbibmonth}\see{aut:fmt:ich}
% \end{release}

% \begin{release}{1.2}{2011-02-12}
% \item Added entry type \bibtype{mvbook}\see{bib:typ:blx}
% \item Added entry type \bibtype{mvcollection}\see{bib:typ:blx}
% \item Added entry type \bibtype{mvproceedings}\see{bib:typ:blx}
% \item Added entry type \bibtype{mvreference}\see{bib:typ:blx}
% \item Introducing remote resources\see{use:bib:res}\BiberOnlyMark
% \item Introducing experimental \acr{RIS} support\see{use:bib:res}\BiberOnlyMark
% \item Added \cmd{addbibresource}\see{use:bib:res}
% \item \cmd{bibliography} now deprecated\see{use:bib:res}
% \item \cmd{bibliography*} replaced by \cmd{addglobalbib}\see{use:bib:res}
% \item Added \cmd{addsectionbib}\see{use:bib:res}
% \item Updated and expanded documentation\see{bib:cav:ref}
% \item Introducing smart \bibfield{crossref} data inheritance\see{bib:cav:ref:bbr}\BiberOnlyMark
% \item Introducing \bibfield{crossref} configuration interface\see{aut:ctm:ref}\BiberOnlyMark
% \item Added \cmd{DefaultInheritance}\see{aut:ctm:ref}\BiberOnlyMark
% \item Added \cmd{DeclareDataInheritance}\see{aut:ctm:ref}\BiberOnlyMark
% \item Added \cmd{ResetDataInheritance}\see{aut:ctm:ref}\BiberOnlyMark
% \item Added \cmd{ifkeyword}\see{aut:aux:tst}
% \item Added \cmd{ifentrykeyword}\see{aut:aux:tst}
% \item Added \cmd{ifcategory}\see{aut:aux:tst}
% \item Added \cmd{ifentrycategory}\see{aut:aux:tst}
% \item Added \cmd{ifdriver}\see{aut:aux:tst}
% \item Added \cmd{forcsvfield}\see{aut:aux:msc}
% \item Extended \cmd{mkpageprefix}\see{aut:aux:msc}
% \item Extended \cmd{mkpagetotal}\see{aut:aux:msc}
% \item Extended \cmd{mkpagefirst}\see{aut:aux:msc}
% \item Added localization key \texttt{inpreparation}\see{aut:lng:key}
% \item Rearranged manual slightly, moving some tables to the appendix
% \end{release}

% \begin{release}{1.1b}{2011-02-04}
% \item Added option \opt{texencoding}\see{use:opt:pre:gen}\BiberOnlyMark
% \item Added option \opt{safeinputenc}\see{use:opt:pre:gen}\BiberOnlyMark
% \item Expanded documentation\see{bib:cav:enc:enc}
% \item Improved \opt{mergedate} option of bibliography style \texttt{authoryear}\see{use:xbx:bbx}
% \item Removed \opt{pass} option of \cmd{DeclareSortingScheme}\see{aut:ctm:srt}\BiberOnlyMark
% \item Fixed some bugs
% \end{release}

% \begin{release}{1.1a}{2011-01-08}
% \item Added unsupported entry type \bibtype{bibnote}\see{bib:typ:ctm}
% \item Added \cmd{bibliography*}\see{use:bib:res}
% \item Fixed some bugs
% \end{release}

% \begin{release}{1.1}{2011-01-05}
% \item Added option \opt{maxbibnames}\see{use:opt:pre:gen}
% \item Added option \opt{minbibnames}\see{use:opt:pre:gen}
% \item Added option \opt{maxcitenames}\see{use:opt:pre:gen}
% \item Added option \opt{mincitenames}\see{use:opt:pre:gen}
% \item Fixed \kvopt{idemtracker}{strict} and \kvopt{idemtracker}{constrict}\see{use:opt:pre:int}
% \item Added option \opt{mergedate} to bibliography style \texttt{authoryear}\see{use:xbx:bbx}
% \item Added support for \opt{prefixnumbers} to bibliography style \texttt{alphabetic}\see{use:xbx:bbx}
% \item Made option \bibfield{useprefix} settable on a per-type basis\see{use:opt:bib}\BiberOnlyMark
% \item Made option \bibfield{useauthor} settable on a per-type basis\see{use:opt:bib}\BiberOnlyMark
% \item Made option \bibfield{useeditor} settable on a per-type basis\see{use:opt:bib}\BiberOnlyMark
% \item Made option \opt{usetranslator} settable on a per-type basis\see{use:opt:bib}\BiberOnlyMark
% \item Made option \opt{skipbib} settable on a per-type basis\see{use:opt:bib}\BiberOnlyMark
% \item Made option \opt{skiplos} settable on a per-type basis\see{use:opt:bib}\BiberOnlyMark
% \item Made option \opt{skiplab} settable on a per-type basis\see{use:opt:bib}\BiberOnlyMark
% \item Made option \opt{dataonly} settable on a per-type basis\see{use:opt:bib}\BiberOnlyMark
% \item Made option \opt{labelalpha} settable on a per-type basis\see{use:opt:pre:int}\BiberOnlyMark
% \item Made option \opt{labelnumber} settable on a per-type basis\see{use:opt:pre:int}
% \item Made option \opt{labelyear} settable on a per-type basis\see{use:opt:pre:int}\BiberOnlyMark
% \item Made option \opt{singletitle} settable on a per-type basis\see{use:opt:pre:int}\BiberOnlyMark
% \item Made option \opt{uniquename} settable on a per-type basis\see{use:opt:pre:int}\BiberOnlyMark
% \item Made option \opt{indexing} settable on a per-type basis\see{use:opt:pre:gen}
% \item Made option \opt{indexing} settable on a per-entry basis\see{use:opt:pre:gen}
% \item Extended \cmd{ExecuteBibliographyOptions}\see{use:cfg:opt}
% \item Added \cmd{citedate}\see{use:cit:txt}
% \item Improved static entry sets\see{use:use:set}\BiberOnlyMark
% \item Introducing dynamic entry sets\see{use:use:set}\BiberOnlyMark
% \item Added \cmd{defbibentryset}\see{use:bib:set}\BiberOnlyMark
% \item Added option \opt{mcite}\see{use:opt:ldt}\BiberOnlyMark
% \item Added \sty{mcite}\slash\sty{mciteplus}-like commands\see{use:cit:mct}\BiberOnlyMark
% \item Added \cmd{sortalphaothers}\see{use:fmt:fmt}\BiberOnlyMark
% \item Extended \cmd{DeclareNameFormat}\see{aut:bib:fmt}
% \item Extended \cmd{DeclareIndexNameFormat}\see{aut:bib:fmt}
% \item Extended \cmd{DeclareListFormat}\see{aut:bib:fmt}
% \item Extended \cmd{DeclareIndexListFormat}\see{aut:bib:fmt}
% \item Extended \cmd{DeclareFieldFormat}\see{aut:bib:fmt}
% \item Extended \cmd{DeclareIndexFieldFormat}\see{aut:bib:fmt}
% \item Added \cmd{DeclareNameFormat*}\see{aut:bib:fmt}
% \item Added \cmd{DeclareIndexNameFormat*}\see{aut:bib:fmt}
% \item Added \cmd{DeclareListFormat*}\see{aut:bib:fmt}
% \item Added \cmd{DeclareIndexListFormat*}\see{aut:bib:fmt}
% \item Added \cmd{DeclareFieldFormat*}\see{aut:bib:fmt}
% \item Added \cmd{DeclareIndexFieldFormat*}\see{aut:bib:fmt}
% \item Introducing configurable sorting schemes\BiberOnlyMark
% \item Added \cmd{DeclareSortingScheme}\see{aut:ctm:srt}\BiberOnlyMark
% \item Added \cmd{DeclarePresort}\see{aut:ctm:srt}\BiberOnlyMark
% \item Added \cmd{DeclareSortExclusion}\see{aut:ctm:srt}\BiberOnlyMark
% \item Added \cmd{DeclareLabelname}\see{aut:ctm:fld}\BiberOnlyMark
% \item Added \cmd{DeclareLabelyear}\see{aut:ctm:fld}\BiberOnlyMark
% \item Improved special field \bibfield{labelname}\see{aut:bbx:fld}\BiberOnlyMark
% \item Improved special field \bibfield{labelyear}\see{aut:bbx:fld}\BiberOnlyMark
% \item Added \cmd{entrydata*}\see{aut:bib:dat}
% \item Added \cmd{RequireBiber}\see{aut:aux:msc}
% \item Added option \opt{check} to \cmd{printbibliography}\see{use:bib:bib}
% \item Added option \opt{check} to \cmd{printshorthands}\see{use:bib:biblist}
% \item Added \cmd{defbibcheck}\see{use:bib:flt}
% \item Updated support for Portuguese (José Carlos Santos)
% \item Fixed conflict with \sty{titletoc}
% \item Fixed some bugs
% \end{release}

% \begin{release}{1.0}{2010-11-19}
% \item First officially stable release
% \item Renamed option \kvopt{bibencoding}{inputenc} to \kvopt{bibencoding}{auto}\see{use:opt:pre:gen}
% \item Made \kvopt{bibencoding}{auto} the package default\see{use:opt:pre:gen}
% \item Added option \kvopt{backend}{bibtexu}\see{use:opt:pre:gen}
% \item Slightly updated documentation\see{bib:cav:enc}
% \item Updated support for Dutch (Alexander van Loon)
% \item Updated support for Italian (Andrea Marchitelli)
% \end{release}

%\begin{release}{0.9e}{2010-10-09}
%\item Updated and expanded manual\see{bib:cav:enc}
%\item Added option \opt{sortupper}\see{use:opt:pre:gen}
%\item Added option \opt{sortlocale}\see{use:opt:pre:gen}
%\item Added option \opt{backrefsetstyle}\see{use:opt:pre:gen}
%\item Added \cmd{bibpagerefpunct}\see{use:fmt:fmt}
%\item Added \cmd{backtrackertrue} and \cmd{backtrackerfalse}\see{aut:aux:msc}
%\item Disable back reference tracking in \acr{TOC}/\acr{LOT}/\acr{LOF}\see{aut:cav:flt}
%\item Improved back reference tracking for \bibtype{set} entries
%\item Fixed some bugs
%\end{release}
%
%\begin{release}{0.9d}{2010-09-03}
%\item Added workaround for \sty{hyperref} space factor issue
%\item Added workaround for \sty{xkeyval}'s flawed class option inheritance
%\item Added workaround for \sty{fancyvrb}'s flawed brute-force \cmd{VerbatimFootnotes}
%\item Removed option \kvopt{date}{none}\see{use:opt:pre:gen}
%\item Removed option \kvopt{urldate}{none}\see{use:opt:pre:gen}
%\item Removed option \kvopt{origdate}{none}\see{use:opt:pre:gen}
%\item Removed option \kvopt{eventdate}{none}\see{use:opt:pre:gen}
%\item Removed option \kvopt{alldates}{none}\see{use:opt:pre:gen}
%\item Added option \kvopt{date}{iso8601}\see{use:opt:pre:gen}
%\item Added option \kvopt{urldate}{iso8601}\see{use:opt:pre:gen}
%\item Added option \kvopt{origdate}{iso8601}\see{use:opt:pre:gen}
%\item Added option \kvopt{eventdate}{iso8601}\see{use:opt:pre:gen}
%\item Added option \kvopt{alldates}{iso8601}\see{use:opt:pre:gen}
%\end{release}
%
%\begin{release}{0.9c}{2010-08-29}
%\item Added field \bibfield{eprintclass}\see{bib:fld:dat}
%\item Added field alias \bibfield{archiveprefix}\see{bib:fld:als}
%\item Added field alias \bibfield{primaryclass}\see{bib:fld:als}
%\item Updated documentation\see{use:use:epr}
%\item Tweaked package option \kvopt{babel}{other*}\see{use:opt:pre:gen}
%\item Updated support for Brazilian (Mateus Araújo)
%\item Fixed some bugs
%\end{release}
%
%\begin{release}{0.9b}{2010-08-04}
%
%\item New dependency on \sty{logreq} package\see{int:pre:req}
%\item Improved separator masking in literal lists\see{bib:use:and}
%\item Added citation style \texttt{authortitle-ticomp}\see{use:xbx:cbx}
%\item Added option \opt{citepages} to all \texttt{verbose} citation styles\see{use:xbx:cbx}
%\item Added support for prefixes to all \texttt{numeric} citation styles\see{use:xbx:cbx}
%\item Added support for prefixes to all \texttt{numeric} bibliography styles\see{use:xbx:bbx}
%\item Renamed package option \opt{defernums} to \opt{defernumbers}\see{use:opt:pre:gen}
%\item Added option \opt{sortcase}\see{use:opt:pre:gen}
%\item Added option \opt{dateabbrev}\see{use:opt:pre:gen}
%\item Added option \kvopt{date}{none}\see{use:opt:pre:gen}
%\item Added option \kvopt{urldate}{none}\see{use:opt:pre:gen}
%\item Added option \kvopt{origdate}{none}\see{use:opt:pre:gen}
%\item Added option \kvopt{eventdate}{none}\see{use:opt:pre:gen}
%\item Added option \kvopt{alldates}{none}\see{use:opt:pre:gen}
%\item Added option \opt{clearlang}\see{use:opt:pre:gen}
%\item Added option \opt{prefixnumbers} to \cmd{printbibliography}\see{use:bib:bib}
%\item Added option \opt{resetnumbers} to \cmd{printbibliography}\see{use:bib:bib}
%\item Added option \opt{omitnumbers} to \cmd{printbibliography}\see{use:bib:bib}
%\item Added special field \bibfield{prefixnumber}\see{aut:bbx:fld}
%\item Added \cmd{DeclareRedundantLanguages}\see{aut:lng:cmd}
%\item Added support for handles (\acr{HDL}s)\see{use:use:epr}
%\item Extended \cmd{defbibfilter}\see{use:bib:flt}
%\item Added \cmd{nametitledelim}\see{use:fmt:fmt}
%\item Improved \cmd{newbibmacro}\see{aut:aux:msc}
%\item Improved \cmd{renewbibmacro}\see{aut:aux:msc}
%\item Added \cmd{biblstring}\see{aut:str}
%\item Added \cmd{bibsstring}\see{aut:str}
%\item Added \cmd{bibcplstring}\see{aut:str}
%\item Added \cmd{bibcpsstring}\see{aut:str}
%\item Added \cmd{bibuclstring}\see{aut:str}
%\item Added \cmd{bibucsstring}\see{aut:str}
%\item Added \cmd{biblclstring}\see{aut:str}
%\item Added \cmd{biblcsstring}\see{aut:str}
%\item Added \cmd{bibxlstring}\see{aut:str}
%\item Added \cmd{bibxsstring}\see{aut:str}
%\item Added \cmd{mkbibbold}\see{aut:fmt:ich}
%\item Modified and extended log messages\see{bib:cav:ide}
%\item Fixed some bugs
%
%\end{release}
%
%\begin{release}{0.9a}{2010-03-19}
%
%\item Modified citation style \texttt{numeric}\see{use:xbx:cbx}
%\item Modified citation style \texttt{numeric-comp}\see{use:xbx:cbx}
%\item Modified citation style \texttt{numeric-verb}\see{use:xbx:cbx}
%\item Modified citation style \texttt{alphabetic}\see{use:xbx:cbx}
%\item Modified citation style \texttt{alphabetic-verb}\see{use:xbx:cbx}
%\item Modified citation style \texttt{authoryear}\see{use:xbx:cbx}
%\item Modified citation style \texttt{authoryear-comp}\see{use:xbx:cbx}
%\item Modified citation style \texttt{authoryear-ibid}\see{use:xbx:cbx}
%\item Modified citation style \texttt{authoryear-icomp}\see{use:xbx:cbx}
%\item Modified citation style \texttt{authortitle}\see{use:xbx:cbx}
%\item Modified citation style \texttt{authortitle-comp}\see{use:xbx:cbx}
%\item Modified citation style \texttt{authortitle-ibid}\see{use:xbx:cbx}
%\item Modified citation style \texttt{authortitle-icomp}\see{use:xbx:cbx}
%\item Modified citation style \texttt{authortitle-terse}\see{use:xbx:cbx}
%\item Modified citation style \texttt{authortitle-tcomp}\see{use:xbx:cbx}
%\item Modified citation style \texttt{draft}\see{use:xbx:cbx}
%\item Modified citation style \texttt{debug}\see{use:xbx:cbx}
%\item Added option \opt{bibwarn}\see{use:opt:pre:gen}
%\item Added \cmd{printbibheading}\see{use:bib:bib}
%\item Added option \opt{env} to \cmd{printbibliography}\see{use:bib:bib}
%\item Added option \opt{env} to \cmd{printshorthands}\see{use:bib:biblist}
%\item Added \cmd{defbibenvironment}\see{use:bib:hdg}
%\item Removed \env{thebibliography}\see{aut:bbx:bbx}
%\item Removed \env{theshorthands}\see{aut:bbx:bbx}
%\item Removed \cmd{thebibitem}\see{aut:bbx:bbx}
%\item Removed \cmd{thelositem}\see{aut:bbx:bbx}
%\item Updated documentation\see{aut:bbx:bbx}
%\item Updated documentation\see{aut:bbx:env}
%\item Added \cmd{intitlepunct}\see{use:fmt:fmt}
%\item Added option \opt{parentracker}\see{use:opt:pre:int}
%\item Added option \opt{maxparens}\see{use:opt:pre:int}
%\item Added counter \cnt{parenlevel}\see{aut:aux:tst}
%\item Added \cmd{parentext}\see{use:cit:txt}
%\item Added \cmd{brackettext}\see{use:cit:txt}
%\item Improved \cmd{mkbibparens}\see{aut:fmt:ich}
%\item Improved \cmd{mkbibbrackets}\see{aut:fmt:ich}
%\item Added \cmd{bibopenparen} and \cmd{bibcloseparen}\see{aut:fmt:ich}
%\item Added \cmd{bibopenbracket} and \cmd{bibclosebracket}\see{aut:fmt:ich}
%\item Added special field \bibfield{childentrykey}\see{aut:bbx:fld}
%\item Added special field \bibfield{childentrytype}\see{aut:bbx:fld}
%\item Added \cmd{ifnatbibmode}\see{aut:aux:tst}
%\item Added missing documentation of \cmd{ifbibxstring}\see{aut:aux:tst}
%\item Added \cmd{providebibmacro}\see{aut:aux:msc}
%\item Added localization key \texttt{backrefpage}\see{aut:lng:key}
%\item Added localization key \texttt{backrefpages}\see{aut:lng:key}
%\item Slightly expanded documentation\see{bib:use:dat}
%\item Slightly expanded documentation\see{aut:bbx:fld:dat}
%\item Added support for Finnish (translations by Hannu Väisänen)
%\item Updated support for Greek (translations by Prokopis)
%
%\end{release}
%
%\begin{release}{0.9}{2010-02-14}
%
%\item Added entry type \bibtype{bookinbook}\see{bib:typ:blx}
%\item Support \bibfield{eventtitle}/\bibfield{eventdate}/\bibfield{venue} in \bibtype{proceedings}\see{bib:typ:blx}
%\item Support \bibfield{eventtitle}/\bibfield{eventdate}/\bibfield{venue} in \bibtype{inproceedings}\see{bib:typ:blx}
%\item Added support for multiple editorial roles\see{bib:use:edr}
%\item Added field \bibfield{editora}\see{bib:fld:dat}
%\item Added field \bibfield{editorb}\see{bib:fld:dat}
%\item Added field \bibfield{editorc}\see{bib:fld:dat}
%\item Added field \bibfield{editoratype}\see{bib:fld:dat}
%\item Added field \bibfield{editorbtype}\see{bib:fld:dat}
%\item Added field \bibfield{editorctype}\see{bib:fld:dat}
%\item Removed field \bibfield{redactor}\see{bib:fld:dat}
%\item Added field \bibfield{pubstate}\see{bib:fld:dat}
%\item Support \bibfield{pubstate} in all entry types\see{bib:typ:blx}
%\item Support full dates in all entry types\see{bib:typ:blx}
%\item Modified and extended date handling\see{bib:use:dat}
%\item Updated documentation\see{bib:use:iss}
%\item Removed field \bibfield{day}\see{bib:fld:dat}
%\item Modified data type of field \bibfield{year}\see{bib:fld:dat}
%\item Extended field \bibfield{date}\see{bib:fld:dat}
%\item Removed field \bibfield{origyear}\see{bib:fld:dat}
%\item Added field \bibfield{origdate}\see{bib:fld:dat}
%\item Added field \bibfield{eventdate}\see{bib:fld:dat}
%\item Removed fields \bibfield{urlday}/\bibfield{urlmonth}/\bibfield{urlyear}\see{bib:fld:dat}
%\item Updated documentation\see{bib:use:dat}
%\item Extended option \opt{date}\see{use:opt:pre:gen}
%\item Extended option \opt{urldate}\see{use:opt:pre:gen}
%\item Added option \opt{origdate}\see{use:opt:pre:gen}
%\item Added option \opt{eventdate}\see{use:opt:pre:gen}
%\item Added option \opt{alldates}\see{use:opt:pre:gen}
%\item Added option \opt{datezeros}\see{use:opt:pre:gen}
%\item Added option \opt{language}\see{use:opt:pre:gen}
%\item Added option \cnt{notetype}\see{use:opt:pre:gen}
%\item Added option \cnt{backrefstyle}\see{use:opt:pre:gen}
%\item Modified option \opt{indexing}\see{use:opt:pre:gen}
%\item Made option \kvopt{hyperref}{auto} the default\see{use:opt:pre:gen}
%\item Added option \kvopt{backend}{biber}\see{use:opt:pre:gen}
%\item Updated documentation\see{bib:cav:enc}
%\item Added option \opt{isbn}\see{use:opt:pre:bbx}
%\item Added option \opt{url}\see{use:opt:pre:bbx}
%\item Added option \opt{doi}\see{use:opt:pre:bbx}
%\item Added option \opt{eprint}\see{use:opt:pre:bbx}
%\item Improved citation style \texttt{authortitle-comp}\see{use:xbx:cbx}
%\item Improved citation style \texttt{authortitle-icomp}\see{use:xbx:cbx}
%\item Improved citation style \texttt{authortitle-tcomp}\see{use:xbx:cbx}
%\item Improved citation style \texttt{authoryear-comp}\see{use:xbx:cbx}
%\item Added citation style \texttt{authoryear-icomp}\see{use:xbx:cbx}
%\item Added citation style \texttt{verbose-trad3}\see{use:xbx:cbx}
%\item Improved bibliography style \texttt{authortitle}\see{use:xbx:bbx}
%\item Improved bibliography style \texttt{authoryear}\see{use:xbx:bbx}
%\item Improved bibliography style \texttt{verbose}\see{use:xbx:bbx}
%\item Added option \opt{title} to \cmd{printbibliography}\see{use:bib:bib}
%\item Added option \opt{title} to \cmd{printshorthands}\see{use:bib:biblist}
%\item Extended \cmd{defbibheading}\see{use:bib:hdg}
%\item Added options \opt{subtype}/\opt{notsubtype} to \cmd{printbibliography}\see{use:bib:biblist}
%\item Added options \opt{subtype}/\opt{notsubtype} to \cmd{printshorthands}\see{use:bib:biblist}
%\item Added test \opt{subtype} to \cmd{defbibfilter}\see{use:bib:flt}
%\item Added option \opt{segment} to \cmd{printshorthands}\see{use:bib:biblist}
%\item Added options \opt{type}/\opt{nottype} to \cmd{printshorthands}\see{use:bib:biblist}
%\item Added options \opt{keyword}/\opt{notkeyword} to \cmd{printshorthands}\see{use:bib:biblist}
%\item Added options \opt{category}/\opt{notcategory} to \cmd{printshorthands}\see{use:bib:biblist}
%\item Added option \opt{filter} to \cmd{printshorthands}\see{use:bib:biblist}
%\item Added \cmd{footcitetext}\see{use:cit:std}
%\item Added \cmd{footcitetexts}\see{use:cit:mlt}
%\item Added \cmd{ftvolcite}\see{use:cit:spc}
%\item Added \cmd{textcites} and \cmd{Textcites}\see{use:cit:mlt}
%\item Added \cmd{nohyphenation}\see{use:fmt:aux}
%\item Added \cmd{textnohyphenation}\see{use:fmt:aux}
%\item Added \cmd{mkpagefirst}\see{aut:aux:msc}
%\item Added \cmd{pagenote} support to \cmd{mkbibendnote}\see{aut:fmt:ich}
%\item Added \cmd{mkbibfootnotetext}\see{aut:fmt:ich}
%\item Added \cmd{mkbibendnotetext}\see{aut:fmt:ich}
%\item Added \cmd{bibfootnotewrapper}\see{aut:fmt:ich}
%\item Added \cmd{bibendnotewrapper}\see{aut:fmt:ich}
%\item Added \cmd{mkdatezeros}\see{aut:fmt:ich}
%\item Added \cmd{stripzeros}\see{aut:fmt:ich}
%\item Added support for \acr{jstor} links\see{use:use:epr}
%\item Added support for PubMed links\see{use:use:epr}
%\item Added support for Google Books links\see{use:use:epr}
%\item Improved \cmd{DeclareBibliographyDriver}\see{aut:bbx:bbx}
%\item Improved \cmd{DeclareBibliographyAlias}\see{aut:bbx:bbx}
%\item Added special fields \bibfield{day}/\bibfield{month}/\bibfield{year}\see{aut:bbx:fld}
%\item Added special fields \bibfield{endday}/\bibfield{endmonth}/\bibfield{endyear}\see{aut:bbx:fld}
%\item Added special fields \bibfield{origday}/\bibfield{origmonth}/\bibfield{origyear}\see{aut:bbx:fld}
%\item Added special fields \bibfield{origendday}/\bibfield{origendmonth}/\bibfield{origendyear}\see{aut:bbx:fld}
%\item Added special fields \bibfield{eventday}/\bibfield{eventmonth}/\bibfield{eventyear}\see{aut:bbx:fld}
%\item Added special fields \bibfield{eventendday}/\bibfield{eventendmonth}/\bibfield{eventendyear}\see{aut:bbx:fld}
%\item Added special fields \bibfield{urlday}/\bibfield{urlmonth}/\bibfield{urlyear}\see{aut:bbx:fld}
%\item Added special fields \bibfield{urlendday}/\bibfield{urlendmonth}/\bibfield{urlendyear}\see{aut:bbx:fld}
%\item Renamed special field \bibfield{labelyear} to \bibfield{extrayear}\see{aut:bbx:fld}
%\item Added new special field \bibfield{labelyear}\see{aut:bbx:fld}
%\item Renamed \cnt{maxlabelyear} to \cnt{maxextrayear}\see{aut:fmt:ilc}
%\item Renamed \cmd{bibdate} to \cmd{printdate}, modified \cmd{printdate}\see{aut:bib:dat}
%\item Added \cmd{printdateextra}\see{aut:bib:dat}
%\item Renamed \cmd{biburldate} to \cmd{printurldate}, modified \cmd{printurldate}\see{aut:bib:dat}
%\item Added \cmd{printorigdate}\see{aut:bib:dat}
%\item Added \cmd{printeventdate}\see{aut:bib:dat}
%\item Added \cmd{bibdatedash}\see{use:fmt:lng}
%\item Added \cmd{mkbibdatelong} and \cmd{mkbibdateshort}\see{use:fmt:lng}
%\item Removed \cmd{bibdatelong} and \cmd{bibdateshort}\see{use:fmt:lng}
%\item Removed \cmd{biburldatelong} and \cmd{biburldateshort}\see{use:fmt:lng}
%\item Added \cmd{ifciteindex}\see{aut:aux:tst}
%\item Added \cmd{ifbibindex}\see{aut:aux:tst}
%\item Added \cmd{iffieldint}\see{aut:aux:tst}
%\item Added \cmd{iffieldnum}\see{aut:aux:tst}
%\item Added \cmd{iffieldnums}\see{aut:aux:tst}
%\item Added \cmd{ifpages}\see{aut:aux:tst}
%\item Added \cmd{iffieldpages}\see{aut:aux:tst}
%\item Added \cmd{DeclarePageCommands} and \cmd{DeclarePageCommands*}\see{aut:aux:msc}
%\item Improved \cmd{NewBibliographyString}\see{aut:lng:cmd}
%\item Removed localization key \texttt{editor}\see{aut:lng:key}
%\item Removed localization key \texttt{editors}\see{aut:lng:key}
%\item Renamed localization key \texttt{typeeditor} to \texttt{editor}\see{aut:lng:key}
%\item Renamed localization key \texttt{typeeditors} to \texttt{editors}\see{aut:lng:key}
%\item Renamed localization key \texttt{typecompiler} to \texttt{compiler}\see{aut:lng:key}
%\item Renamed localization key \texttt{typecompilers} to \texttt{compilers}\see{aut:lng:key}
%\item Added localization key \texttt{founder}\see{aut:lng:key}
%\item Added localization key \texttt{founders}\see{aut:lng:key}
%\item Added localization key \texttt{continuator}\see{aut:lng:key}
%\item Added localization key \texttt{continuators}\see{aut:lng:key}
%\item Added localization key \texttt{collaborator}\see{aut:lng:key}
%\item Added localization key \texttt{collaborators}\see{aut:lng:key}
%\item Removed localization key \texttt{byauthor}\see{aut:lng:key}
%\item Renamed localization key \texttt{bytypeauthor} to \texttt{byauthor}\see{aut:lng:key}
%\item Removed localization key \texttt{byeditor}\see{aut:lng:key}
%\item Renamed localization key \texttt{bytypeeditor} to \texttt{byeditor}\see{aut:lng:key}
%\item Renamed localization key \texttt{bytypecompiler} to \texttt{bycompiler}\see{aut:lng:key}
%\item Added localization key \texttt{byfounder}\see{aut:lng:key}
%\item Added localization key \texttt{bycontinuator}\see{aut:lng:key}
%\item Added localization key \texttt{bycollaborator}\see{aut:lng:key}
%\item Added localization key \texttt{inpress}\see{aut:lng:key}
%\item Added localization key \texttt{submitted}\see{aut:lng:key}
%\item Added support for Dutch (translations by Alexander van Loon)
%\item Added support for Greek (translations by Apostolos Syropoulos)
%\item Added notes on Greek\see{use:loc:grk}
%
%\end{release}
%
%\begin{release}{0.8i}{2009-09-20}
%
%\item Fixed some bugs
%
%\end{release}
%
%\begin{release}{0.8h}{2009-08-10}
%
%\item Fixed some bugs
%
%\end{release}
%
%\begin{release}{0.8g}{2009-08-06}
%
%\item Fixed some bugs
%
%\end{release}
%
%\begin{release}{0.8f}{2009-07-25}
%
%\item Fixed some bugs
%
%\end{release}
%
%\begin{release}{0.8e}{2009-07-04}

%
%\item Added \cmd{mkbibordedition}\see{use:fmt:lng}
%\item Added \cmd{mkbibordseries}\see{use:fmt:lng}
%\item Added \cmd{mkbibendnote}\see{aut:fmt:ich}
%\item Added several localization keys related to \texttt{editor}\see{aut:lng:key}
%\item Added several localization keys related to \texttt{translator}\see{aut:lng:key}
%\item Added localization key \texttt{thiscite}\see{aut:lng:key}
%\item Removed several \texttt{country...} localization keys\see{aut:lng:key}
%\item Removed several \texttt{patent...} localization keys\see{aut:lng:key}
%\item Removed several \texttt{patreq...} localization keys\see{aut:lng:key}
%\item Updated and clarified documentation\see{aut:lng:key}
%\item Added support for Brazilian Portuguese (by Augusto Ritter Stoffel)
%\item Added preliminary support for Portuguese/Portugal\see{use:opt:pre:gen}
%\item Added revised Swedish translations (by Per Starbäck)\see{use:opt:pre:gen}
%\item Expanded documentation\see{aut:cav:nam}
%\item Improved concatenation of editorial and other roles
%\item Fixed some bugs
%
%\end{release}
%
%\begin{release}{0.8d}{2009-05-30}
%
%\item Removed package option \opt{bibtex8}\see{use:opt:pre:gen}
%\item Added package option \opt{backend}\see{use:opt:pre:gen}
%\item Slightly modified package option \bibfield{loccittracker}\see{use:opt:pre:int}
%\item Added \cmd{volcite} and \cmd{Volcite}\see{use:cit:spc}
%\item Added \cmd{pvolcite} and \cmd{Pvolcite}\see{use:cit:spc}
%\item Added \cmd{fvolcite}\see{use:cit:spc}
%\item Added \cmd{tvolcite} and \cmd{Tvolcite}\see{use:cit:spc}
%\item Added \cmd{avolcite} and \cmd{Avolcite}\see{use:cit:spc}
%\item Added \cmd{notecite} and \cmd{Notecite}\see{use:cit:spc}
%\item Added \cmd{Pnotecite} and \cmd{Pnotecite}\see{use:cit:spc}
%\item Added \cmd{fnotecite}\see{use:cit:spc}
%\item Added \cmd{addabthinspace}\see{aut:pct:spc}
%\item Disable citation and page trackers in \acr{TOC}/\acr{LOT}/\acr{LOF}\see{aut:cav:flt}
%\item Disable citation and page trackers in floats\see{aut:cav:flt}
%\item Improved on-demand loading of localization modules
%\item Fixed some bugs
%
%\end{release}
%
%\begin{release}{0.8c}{2009-01-10}
%
%\item Added <idem> tracker\see{use:opt:pre:int}
%\item Added package option \opt{idemtracker}\see{use:opt:pre:int}
%\item Added \cmd{ifciteidem}\see{aut:aux:tst}
%\item Added \cmd{ifentryseen}\see{aut:aux:tst}
%\item Improved citation style \texttt{verbose-trad1}\see{use:xbx:cbx}
%\item Improved citation style \texttt{verbose-trad2}\see{use:xbx:cbx}
%\item Renamed \len{bibitemextrasep} to \len{bibnamesep}\see{use:fmt:len}
%\item Slightly modified \len{bibnamesep}\see{use:fmt:len}
%\item Added \len{bibinitsep}\see{use:fmt:len}
%\item Increased default value of \cnt{highnamepenalty}\see{use:fmt:len}
%\item Increased default value of \cnt{lownamepenalty}\see{use:fmt:len}
%\item Updated documentation\see{use:loc:us}
%\item Added \cmd{uspunctuation}\see{aut:pct:cfg}
%\item Added \cmd{stdpunctuation}\see{aut:pct:cfg}
%\item Added \cmd{midsentence*}\see{aut:pct:ctr}
%\item Fixed some bugs
%
%\end{release}
%
%\begin{release}{0.8b}{2008-12-13}
%
%\item Added package/entry option \opt{usetranslator}\see{use:opt:bib}
%\item Added \cmd{ifusetranslator}\see{aut:aux:tst}
%\item Consider \bibfield{translator} when sorting\see{use:srt}
%\item Consider \bibfield{translator} when generating \bibfield{labelname}\see{aut:bbx:fld}
%\item Added field \bibfield{eventtitle}\see{bib:fld:dat}
%\item Support \bibfield{eventtitle} in \bibtype{proceedings} entries\see{bib:typ:blx}
%\item Support \bibfield{eventtitle} in \bibtype{inproceedings} entries\see{bib:typ:blx}
%\item Added unsupported entry type \bibtype{commentary}\see{bib:typ:ctm}
%\item Permit \cmd{NewBibliographyString} in \file{lbx} files\see{aut:lng:cmd}
%\item Improved behavior of \cmd{mkbibquote} in <American-punctuation> mode\see{aut:fmt:ich}
%\item Fixed some bugs
%
%\end{release}
%
%\begin{release}{0.8a}{2008-11-29}
%
%\item Updated documentation (important, please read)\see{int:feb}
%\item Added package option \kvopt{hyperref}{auto}\see{use:opt:pre:gen}
%\item Improved bibliography style \texttt{reading}\see{use:xbx:bbx}
%\item Updated \acr{KOMA}-Script support for version 3.x\see{use:cav:scr}
%\item Slightly modified special field \bibfield{fullhash}\see{aut:bbx:fld}
%\item Added documentation of \cmd{DeclareNumChars*}\see{aut:aux:msc}
%\item Added documentation of \cmd{DeclareRangeChars*}\see{aut:aux:msc}
%\item Added documentation of \cmd{DeclareRangeCommands*}\see{aut:aux:msc}
%\item Added \cmd{MakeSentenceCase}\see{aut:aux:msc}
%\item Added \cmd{DeclareCaseLangs}\see{aut:aux:msc}
%\item Support nested \cmd{mkbibquote} with American punctuation\see{aut:fmt:ich}
%\item Improved \cmd{ifpunctmark}\see{aut:pct:chk}
%\item Improved punctuation tracker\see{aut:pct:pct}
%\item Added \cmd{DeclarePunctuationPairs}\see{aut:pct:cfg}
%\item Added \cmd{DeclareLanguageMapping}\see{aut:lng:cmd}
%\item Added support for custom localization modules\see{aut:cav:lng}
%\item Added extended \pdf bookmarks to this manual
%\item Fixed various bugs
%
%\end{release}
%
%\begin{release}{0.8}{2008-10-02}
%
%\item Added \cmd{DefineHyphenationExceptions}\see{use:lng}
%\item Added \cmd{DeclareHyphenationExceptions}\see{aut:lng:cmd}
%\item Added \cmd{mkpagetotal}\see{aut:aux:msc}
%\item Improved \acr{KOMA}-Script support\see{use:cav:scr}
%\item Added \cmd{ifkomabibtotoc}\see{use:cav:scr}
%\item Added \cmd{ifkomabibtotocnumbered}\see{use:cav:scr}
%\item Added \cmd{ifmemoirbibintoc}\see{use:cav:mem}
%\item Updated documentation\see{use:bib:hdg}
%\item Updated documentation of \cmd{iffootnote}\see{aut:aux:tst}
%\item Added several new localization keys\see{aut:lng:key}
%\item Rearranged some localization keys (\texttt{section} vs. \texttt{paragraph})\see{aut:lng:key}
%\item Added unsupported entry type \bibtype{letter}\see{bib:typ:ctm}
%\item Added entry type \bibtype{suppbook}\see{bib:typ:blx}
%\item Added entry type \bibtype{suppcollection}\see{bib:typ:blx}
%\item Added entry type \bibtype{suppperiodical}\see{bib:typ:blx}
%\item Support \bibtype{reference} and \bibtype{inreference}\see{bib:typ:blx}
%\item Support \bibtype{review} as an alias\see{bib:typ:ctm}
%\item Added field \bibfield{origpublisher}\see{bib:fld:dat}
%\item Added field alias \bibfield{annote}\see{bib:fld:als}
%\item Expanded documentation\see{bib:cav:enc}
%\item Added \cmd{DeclareCapitalPunctuation}\see{aut:pct:cfg}
%\item Removed \cmd{EnableCapitalAfter} and \cmd{DisableCapitalAfter}\see{aut:pct:cfg}
%\item Added support for <American-style> punctuation\see{aut:pct:cfg}
%\item Added \cmd{DeclareQuotePunctuation}\see{aut:pct:cfg}
%\item Improved \cmd{mkbibquote}\see{aut:fmt:ich}
%\item Expanded documentation\see{use:loc:us}
%\item Improved all \texttt{numeric} citation styles\see{use:xbx:cbx}
%\item Improved \texttt{numeric} bibliography style\see{use:xbx:bbx}
%\item Added citation style \texttt{authoryear-ibid}\see{use:xbx:cbx}
%\item Improved all \texttt{authoryear} citation styles\see{use:xbx:cbx}
%\item Improved \texttt{authoryear} bibliography style\see{use:xbx:bbx}
%\item Added \opt{pageref} option to \texttt{verbose-note} style\see{use:xbx:cbx}
%\item Added \opt{pageref} option to \texttt{verbose-inote} style\see{use:xbx:cbx}
%\item Added citation style \texttt{reading}\see{use:xbx:cbx}
%\item Added bibliography style \texttt{reading}\see{use:xbx:bbx}
%\item Added citation style \texttt{draft}\see{use:xbx:cbx}
%\item Added bibliography style \texttt{draft}\see{use:xbx:bbx}
%\item Improved \sty{natbib} compatibility style\see{use:cit:nat}
%\item Added \cmd{ifcitation}\see{aut:aux:tst}
%\item Added \cmd{ifbibliography}\see{aut:aux:tst}
%\item Added \cmd{printfile}\see{aut:bib:dat}
%\item Added package option \opt{loadfiles}\see{use:opt:pre:gen}
%\item Added support for bibliographic data in external files\see{use:use:prf}
%\item Expanded documentation\see{aut:cav:prf}
%\item Modified field \bibfield{edition}\see{bib:fld:dat}
%\item Modified special field \bibfield{labelyear}\see{aut:bbx:fld}
%\item Modified special field \bibfield{labelalpha}\see{aut:bbx:fld}
%\item Added special field \bibfield{extraalpha}\see{aut:bbx:fld}
%\item Added counter \cnt{maxlabelyear}\see{aut:fmt:ilc}
%\item Added counter \cnt{maxextraalpha}\see{aut:fmt:ilc}
%\item Added \cmd{mknumalph}\see{use:fmt:aux}
%\item Added \cmd{mkbibacro}\see{use:fmt:aux}
%\item Added \cmd{autocap}\see{use:fmt:aux}
%\item Added package option \opt{firstinits}\see{use:opt:pre:gen}
%\item Added \cmd{iffirstinits}\see{aut:aux:tst}
%\item Added support for eprint data\see{use:use:epr}
%\item Added support for arXiv\see{use:use:epr}
%\item Expanded documentation \see{aut:cav:epr}
%\item Added field \bibfield{eprint}\see{bib:fld:dat}
%\item Added field \bibfield{eprinttype}\see{bib:fld:dat}
%\item Added eprint support to all standard entry types\see{bib:typ:blx}
%\item Added package option \opt{arxiv}\see{use:opt:pre:gen}
%\item Introduced concept of an entry set\see{use:use:set}
%\item Expanded documentation\see{aut:cav:set}
%\item Added entry type \bibtype{set}\see{bib:typ:blx}
%\item Added field \bibfield{entryset}\see{bib:fld:spc}
%\item Added special field \bibfield{entrysetcount}\see{aut:bbx:fld}
%\item Added \cmd{entrydata}\see{aut:bib:dat}
%\item Expanded documentation\see{aut:cav:mif}
%\item Added \cmd{entryset}\see{aut:bib:dat}
%\item Added \cmd{strfield}\see{aut:aux:dat}
%\item Improved \cmd{usedriver}\see{aut:aux:msc}
%\item Added \cmd{bibpagespunct}\see{use:fmt:fmt}
%\item Expanded documentation\see{aut:cav:pct}
%\item Added entry option \opt{skipbib}\see{use:opt:bib}
%\item Added entry option \opt{skiplos}\see{use:opt:bib}
%\item Added entry option \opt{skiplab}\see{use:opt:bib}
%\item Added entry option \opt{dataonly}\see{use:opt:bib}
%\item Modified special field \bibfield{namehash}\see{aut:bbx:fld}
%\item Added special field \bibfield{fullhash}\see{aut:bbx:fld}
%\item Added \cmd{DeclareNumChars}\see{aut:aux:msc}
%\item Added \cmd{DeclareRangeChars}\see{aut:aux:msc}
%\item Added \cmd{DeclareRangeCommands}\see{aut:aux:msc}
%\item Added support for Swedish (translations by Per Starbäck and others)
%\item Updated various localization files
%\item Various minor improvements throughout
%\item Fixed some bugs
%
%\end{release}
%
%\begin{release}{0.7}{2007-12-09}
%
%\item Expanded documentation\see{int:feb}
%\item New dependency on \sty{etoolbox} package\see{int:pre:req}
%\item Made \sty{url} a required package\see{int:pre:req}
%\item Modified package option \opt{sorting}\see{use:opt:pre:gen}
%\item Introduced concept of an entry option\see{use:opt:bib}
%\item Added option \bibfield{useauthor}\see{use:opt:bib}
%\item Added option \bibfield{useeditor}\see{use:opt:bib}
%\item Modified option \bibfield{useprefix}\see{use:opt:bib}
%\item Removed field \bibfield{useprefix}\see{bib:fld:spc}
%\item Added field \bibfield{options}\see{bib:fld:spc}
%\item Updated documentation\see{use:srt}
%\item Added citation style \texttt{authortitle-ibid}\see{use:xbx:cbx}
%\item Added citation style \texttt{authortitle-icomp}\see{use:xbx:cbx}
%\item Renamed citation style \texttt{authortitle-cterse} to \texttt{authortitle-tcomp}\see{use:xbx:cbx}
%\item Renamed citation style \texttt{authortitle-verb} to \texttt{verbose}\see{use:xbx:cbx}
%\item Renamed citation style \texttt{authortitle-cverb} to \texttt{verbose-ibid}\see{use:xbx:cbx}
%\item Added citation style \texttt{verbose-note}\see{use:xbx:cbx}
%\item Added citation style \texttt{verbose-inote}\see{use:xbx:cbx}
%\item Renamed citation style \texttt{authortitle-trad} to \texttt{verbose-trad1}\see{use:xbx:cbx}
%\item Removed citation style \texttt{authortitle-strad}\see{use:xbx:cbx}
%\item Added citation style \texttt{verbose-trad2}\see{use:xbx:cbx}
%\item Improved citation style \texttt{authoryear}\see{use:xbx:cbx}
%\item Improved citation style \texttt{authoryear-comp}\see{use:xbx:cbx}
%\item Improved citation style \texttt{authortitle-terse}\see{use:xbx:cbx}
%\item Improved citation style \texttt{authortitle-tcomp}\see{use:xbx:cbx}
%\item Improved all verbose citation styles\see{use:xbx:cbx}
%\item Expanded documentation\see{bib:fld:typ}
%\item Modified entry type \bibtype{article}\see{bib:typ:blx}
%\item Added entry type \bibtype{periodical}\see{bib:typ:blx}
%\item Added entry type \bibtype{patent}\see{bib:typ:blx}
%\item Extended entry types \bibfield{proceedings} and \bibfield{inproceedings}\see{bib:typ:blx}
%\item Extended entry type \bibfield{article}\see{bib:typ:blx}
%\item Extended entry type \bibfield{booklet}\see{bib:typ:blx}
%\item Extended entry type \bibfield{misc}\see{bib:typ:blx}
%\item Added entry type alias \bibtype{electronic}\see{bib:typ:als}
%\item Added new custom types\see{bib:typ:ctm}
%\item Support \bibfield{pagetotal} field where applicable\see{bib:typ:blx}
%\item Added field \bibfield{holder}\see{bib:fld:dat}
%\item Added field \bibfield{venue}\see{bib:fld:dat}
%\item Added field \bibfield{version}\see{bib:fld:dat}
%\item Added field \bibfield{journaltitle}\see{bib:fld:dat}
%\item Added field \bibfield{journalsubtitle}\see{bib:fld:dat}
%\item Added field \bibfield{issuetitle}\see{bib:fld:dat}
%\item Added field \bibfield{issuesubtitle}\see{bib:fld:dat}
%\item Removed field \bibfield{journal}\see{bib:fld:dat}
%\item Added field alias \bibfield{journal}\see{bib:fld:als}
%\item Added field \bibfield{shortjournal}\see{bib:fld:dat}
%\item Added field \bibfield{shortseries}\see{bib:fld:dat}
%\item Added field \bibfield{shorthandintro}\see{bib:fld:dat}
%\item Added field \bibfield{xref}\see{bib:fld:spc}
%\item Added field \bibfield{authortype}\see{bib:fld:dat}
%\item Added field \bibfield{editortype}\see{bib:fld:dat}
%\item Added field \bibfield{reprinttitle}\see{bib:fld:dat}
%\item Improved handling of field \bibfield{type}\see{bib:fld:dat}
%\item Improved handling of field \bibfield{series}\see{bib:fld:dat}
%\item Updated documentation\see{bib:use:ser}
%\item Renamed field \bibfield{id} to \bibfield{eid}\see{bib:fld:dat}
%\item Added field \bibfield{pagination}\see{bib:fld:dat}
%\item Added field \bibfield{bookpagination}\see{bib:fld:dat}
%\item Added special field \bibfield{sortinit}\see{aut:bbx:fld}
%\item Introduced concept of a multicite command\see{use:cit:mlt}
%\item Added \cmd{cites}\see{use:cit:mlt}
%\item Added \cmd{Cites}\see{use:cit:mlt}
%\item Added \cmd{parencites}\see{use:cit:mlt}
%\item Added \cmd{Parencites}\see{use:cit:mlt}
%\item Added \cmd{footcites}\see{use:cit:mlt}
%\item Added \cmd{supercites}\see{use:cit:mlt}
%\item Added \cmd{Autocite}\see{use:cit:aut}
%\item Added \cmd{autocites}\see{use:cit:aut}
%\item Added \cmd{Autocites}\see{use:cit:aut}
%\item Added \cmd{DeclareMultiCiteCommand}\see{aut:cbx:cbx}
%\item Added counter \cnt{multicitecount}\see{aut:fmt:ilc}
%\item Added counter \cnt{multicitetotal}\see{aut:fmt:ilc}
%\item Renamed \cmd{citefulltitle} to \cmd{citetitle*}\see{use:cit:txt}
%\item Added \cmd{cite*}\see{use:cit:cbx}
%\item Added \cmd{citeurl}\see{use:cit:txt}
%\item Added documentation of field \bibfield{nameaddon}\see{bib:fld:dat}
%\item Added field \bibfield{entrysubtype}\see{bib:fld:spc}
%\item Added field \bibfield{execute}\see{bib:fld:spc}
%\item Added custom fields \bibfield{verb{[a--c]}}\see{bib:fld:ctm}
%\item Added custom fields \bibfield{name{[a--c]}type}\see{bib:fld:ctm}
%\item Consider \bibfield{sorttitle} field when falling back to \bibfield{title}\see{use:srt}
%\item Removed package option \opt{labelctitle}\see{use:opt:pre:int}
%\item Removed field \opt{labelctitle}\see{aut:bbx:fld}
%\item Added package option \opt{singletitle}\see{use:opt:pre:int}
%\item Added \cmd{ifsingletitle}\see{aut:aux:tst}
%\item Added \cmd{ifuseauthor}\see{aut:aux:tst}
%\item Added \cmd{ifuseeditor}\see{aut:aux:tst}
%\item Added \cmd{ifopcit}\see{aut:aux:tst}
%\item Added \cmd{ifloccit}\see{aut:aux:tst}
%\item Added package option \opt{uniquename}\see{use:opt:pre:int}
%\item Added special counter \cnt{uniquename}\see{aut:aux:tst}
%\item Added package option \opt{natbib}\see{use:opt:ldt}
%\item Added compatibility commands for the \sty{natbib} package\see{use:cit:nat}
%\item Added package option \opt{defernums}\see{use:opt:pre:gen}
%\item Improved support for numeric labels\see{use:cav:lab}
%\item Added package option \opt{mincrossrefs}\see{use:opt:pre:gen}
%\item Added package option \opt{bibencoding}\see{use:opt:pre:gen}
%\item Expanded documentation\see{bib:cav:enc}
%\item Updated documentation\see{bib:cav:ide}
%\item Added package option \opt{citetracker}\see{use:opt:pre:int}
%\item Added package option \opt{ibidtracker}\see{use:opt:pre:int}
%\item Added package option \bibfield{opcittracker}\see{use:opt:pre:int}
%\item Added package option \bibfield{loccittracker}\see{use:opt:pre:int}
%\item Added \cmd{citetrackertrue} and \cmd{citetrackerfalse}\see{aut:aux:msc}
%\item Modified package option \opt{pagetracker}\see{use:opt:pre:int}
%\item Added \cmd{pagetrackertrue} and \cmd{pagetrackerfalse}\see{aut:aux:msc}
%\item Text commands now excluded from tracking\see{use:cit:txt}
%\item Updated documentation of \cmd{iffirstonpage}\see{aut:aux:tst}
%\item Updated documentation of \cmd{ifsamepage}\see{aut:aux:tst}
%\item Removed package option \opt{keywsort}\see{use:opt:pre:gen}
%\item Added package option \opt{refsection}\see{use:opt:pre:gen}
%\item Added package option \opt{refsegment}\see{use:opt:pre:gen}
%\item Added package option \opt{citereset}\see{use:opt:pre:gen}
%\item Added option \opt{section} to \cmd{bibbysegment}\see{use:bib:bib}
%\item Added option \opt{section} to \cmd{bibbycategory}\see{use:bib:bib}
%\item Added option \opt{section} to \cmd{printshorthands}\see{use:bib:biblist}
%\item Extended documentation of \env{refsection} environment\see{use:bib:sec}
%\item Added \cmd{newrefsection}\see{use:bib:sec}
%\item Added \cmd{newrefsegment}\see{use:bib:seg}
%\item Added heading definition \texttt{subbibliography}\see{use:bib:hdg}
%\item Added heading definition \texttt{subbibintoc}\see{use:bib:hdg}
%\item Added heading definition \texttt{subbibnumbered}\see{use:bib:hdg}
%\item Make all citation commands scan ahead for punctuation\see{use:cit}
%\item Updated documentation of \cmd{DeclareAutoPunctuation}\see{aut:pct:cfg}
%\item Removed \cmd{usecitecmd}\see{aut:cbx:cbx}
%\item Updated documentation of \opt{autocite} package option\see{use:opt:pre:gen}
%\item Updated documentation of \opt{autopunct} package option\see{use:opt:pre:gen}
%\item Added \cmd{citereset}\see{use:cit:msc}
%\item Added \cmd{citereset*}\see{use:cit:msc}
%\item Added \cmd{mancite}\see{use:cit:msc}
%\item Added \cmd{citesetup}\see{use:fmt:fmt}
%\item Added \cmd{compcitedelim}\see{use:fmt:fmt}
%\item Added \cmd{labelnamepunct}\see{use:fmt:fmt}
%\item Added \cmd{subtitlepunct}\see{use:fmt:fmt}
%\item Added \cmd{finallistdelim}\see{use:fmt:fmt}
%\item Added \cmd{andmoredelim}\see{use:fmt:fmt}
%\item Added \cmd{labelalphaothers}\see{use:fmt:fmt}
%\item Added \len{bibitemextrasep}\see{use:fmt:len}
%\item Renamed \cmd{blxauxprefix} to \cmd{blxauxsuffix}\see{use:use:aux}
%\item Added \cmd{DeclareBibliographyOption}\see{aut:bbx:bbx}
%\item Added \cmd{DeclareEntryOption}\see{aut:bbx:bbx}
%\item Renamed \cmd{InitializeBibliographyDrivers} to \cmd{InitializeBibliographyStyle}\see{aut:bbx:bbx}
%\item Added \cmd{InitializeCitationStyle}\see{aut:cbx:cbx}
%\item Added \cmd{OnManualCitation}\see{aut:cbx:cbx}
%\item Extended documentation of \cmd{DeclareCiteCommand}\see{aut:cbx:cbx}
%\item Modified \cmd{DeclareAutoCiteCommand}\see{aut:cbx:cbx}
%\item Improved \cmd{printtext}\see{aut:bib:dat}
%\item Improved \cmd{printfield}\see{aut:bib:dat}
%\item Improved \cmd{printlist}\see{aut:bib:dat}
%\item Improved \cmd{printnames}\see{aut:bib:dat}
%\item Improved \cmd{indexfield}\see{aut:bib:dat}
%\item Improved \cmd{indexlist}\see{aut:bib:dat}
%\item Improved \cmd{indexnames}\see{aut:bib:dat}
%\item Modified \cmd{DeclareFieldFormat}\see{aut:bib:fmt}
%\item Modified \cmd{DeclareListFormat}\see{aut:bib:fmt}
%\item Modified \cmd{DeclareNameFormat}\see{aut:bib:fmt}
%\item Modified \cmd{DeclareFieldAlias}\see{aut:bib:fmt}
%\item Modified \cmd{DeclareListAlias}\see{aut:bib:fmt}
%\item Modified \cmd{DeclareNameAlias}\see{aut:bib:fmt}
%\item Modified \cmd{DeclareIndexFieldFormat}\see{aut:bib:fmt}
%\item Modified \cmd{DeclareIndexListFormat}\see{aut:bib:fmt}
%\item Modified \cmd{DeclareIndexNameFormat}\see{aut:bib:fmt}
%\item Modified \cmd{DeclareIndexFieldAlias}\see{aut:bib:fmt}
%\item Modified \cmd{DeclareIndexListAlias}\see{aut:bib:fmt}
%\item Modified \cmd{DeclareIndexNameAlias}\see{aut:bib:fmt}
%\item Improved \cmd{iffirstonpage}\see{aut:aux:tst}
%\item Improved \cmd{ifciteseen}\see{aut:aux:tst}
%\item Improved \cmd{ifandothers}\see{aut:aux:tst}
%\item Added \cmd{ifinteger}\see{aut:aux:tst}
%\item Added \cmd{ifnumeral}\see{aut:aux:tst}
%\item Added \cmd{ifnumerals}\see{aut:aux:tst}
%\item Removed \cmd{ifpage}\see{aut:aux:tst}
%\item Removed \cmd{ifpages}\see{aut:aux:tst}
%\item Moved \cmd{ifblank} to \sty{etoolbox} package\see{aut:aux:tst}
%\item Removed \cmd{xifblank}\see{aut:aux:tst}
%\item Moved \cmd{docsvlist} to \sty{etoolbox} package\see{aut:aux:msc}
%\item Updated documentation of \cmd{docsvfield}\see{aut:aux:msc}
%\item Added \cmd{ifciteibid}\see{aut:aux:tst}
%\item Added \cmd{iffootnote}\see{aut:aux:tst}
%\item Added \cmd{iffieldxref}\see{aut:aux:tst}
%\item Added \cmd{iflistxref}\see{aut:aux:tst}
%\item Added \cmd{ifnamexref}\see{aut:aux:tst}
%\item Added \cmd{ifmoreitems}\see{aut:aux:tst}
%\item Added \cmd{ifbibstring}\see{aut:aux:tst}
%\item Added \cmd{iffieldbibstring}\see{aut:aux:tst}
%\item Added \cmd{mkpageprefix}\see{aut:aux:msc}
%\item Added \cmd{NumCheckSetup}\see{aut:aux:msc}
%\item Added \cmd{pno}\see{use:cit:msc}
%\item Added \cmd{ppno}\see{use:cit:msc}
%\item Added \cmd{nopp}\see{use:cit:msc}
%\item Added \cmd{ppspace}\see{aut:aux:msc}
%\item Added \cmd{psq}\see{use:cit:msc}
%\item Added \cmd{psqq}\see{use:cit:msc}
%\item Added \cmd{sqspace}\see{aut:aux:msc}
%\item Expanded documentation\see{bib:use:pag}
%\item Expanded documentation\see{use:cav:pag}
%\item Added \cmd{RN}\see{use:cit:msc}
%\item Added \cmd{Rn}\see{use:cit:msc}
%\item Added \cmd{RNfont}\see{use:cit:msc}
%\item Added \cmd{Rnfont}\see{use:cit:msc}
%\item Added package option \opt{punctfont}\see{use:opt:pre:gen}
%\item Added \cmd{setpunctfont}\see{aut:pct:new}
%\item Added \cmd{resetpunctfont}\see{aut:pct:new}
%\item Added \cmd{nopunct}\see{aut:pct:pct}
%\item Added \cmd{bibxstring}\see{aut:str}
%\item Added \cmd{mkbibemph}\see{aut:fmt:ich}
%\item Added \cmd{mkbibquote}\see{aut:fmt:ich}
%\item Added \cmd{mkbibfootnote}\see{aut:fmt:ich}
%\item Added \cmd{mkbibsuperscript}\see{aut:fmt:ich}
%\item Added \cmd{currentfield}\see{aut:fmt:ilc}
%\item Added \cmd{currentlist}\see{aut:fmt:ilc}
%\item Added \cmd{currentname}\see{aut:fmt:ilc}
%\item Added \cmd{AtNextCite}\see{aut:fmt:hok}
%\item Added \cmd{AtNextCitekey}\see{aut:fmt:hok}
%\item Added \cmd{AtDataInput}\see{aut:fmt:hok}
%\item Added several new localization keys\see{aut:lng:key}
%\item Added support for Norwegian (translations by Johannes Wilm)
%\item Added support for Danish (translations by Johannes Wilm)
%\item Expanded documentation\see{aut:cav:grp}
%\item Expanded documentation\see{aut:cav:mif}
%\item Numerous improvements under the hood
%\item Fixed some bugs
%
%\end{release}
%
%\begin{release}{0.6}{2007-01-06}
%
%\item Added package option \kvopt{sorting}{none}\see{use:opt:pre:gen}
%\item Renamed package option \kvopt{block}{penalty} to \kvopt{block}{ragged}\see{use:opt:pre:gen}
%\item Changed data type of \bibfield{origlanguage} back to field\see{bib:fld:dat}
%\item Support \bibfield{origlanguage} field if \bibfield{translator} is present\see{bib:typ:blx}
%\item Renamed field \bibfield{articleid} to \bibfield{id}\see{bib:fld:dat}
%\item Support \bibfield{id} field in \bibfield{article} entries\see{bib:typ:blx}
%\item Support \bibfield{series} field in \bibfield{article} entries\see{bib:typ:blx}
%\item Support \bibfield{doi} field\see{bib:typ:blx}
%\item Updated documentation of all entry types\see{bib:typ:blx}
%\item Updated documentation of field \bibfield{series}\see{bib:fld:dat}
%\item Added field \bibfield{redactor}\see{bib:fld:dat}
%\item Added field \bibfield{shortauthor}\see{bib:fld:dat}
%\item Added field \bibfield{shorteditor}\see{bib:fld:dat}
%\item Improved support for corporate authors and editors\see{bib:use:inc}
%\item Updated documentation of field \bibfield{labelname}\see{aut:bbx:fld}
%\item Added field alias \bibfield{key}\see{bib:fld:als}
%\item Added package option \opt{autocite}\see{use:opt:pre:gen}
%\item Added package option \opt{autopunct}\see{use:opt:pre:gen}
%\item Added \cmd{autocite}\see{use:cit:aut}
%\item Added \cmd{DeclareAutoCiteCommand}\see{aut:cbx:cbx}
%\item Added \cmd{DeclareAutoPunctuation}\see{aut:pct:cfg}
%\item Added option \opt{filter} to \cmd{printbibliography}\see{use:bib:bib}
%\item Added \cmd{defbibfilter}\see{use:bib:flt}
%\item Added package option \opt{maxitems}\see{use:opt:pre:gen}
%\item Added package option \opt{minitems}\see{use:opt:pre:gen}
%\item Added option \opt{maxitems} to \cmd{printbibliography}\see{use:bib:bib}
%\item Added option \opt{minitems} to \cmd{printbibliography}\see{use:bib:bib}
%\item Added option \opt{maxitems} to \cmd{bibbysection}\see{use:bib:bib}
%\item Added option \opt{minitems} to \cmd{bibbysection}\see{use:bib:bib}
%\item Added option \opt{maxitems} to \cmd{bibbysegment}\see{use:bib:bib}
%\item Added option \opt{minitems} to \cmd{bibbysegment}\see{use:bib:bib}
%\item Added option \opt{maxitems} to \cmd{bibbycategory}\see{use:bib:bib}
%\item Added option \opt{minitems} to \cmd{bibbycategory}\see{use:bib:bib}
%\item Added option \opt{maxitems} to \cmd{printshorthands}\see{use:bib:biblist}
%\item Added option \opt{minitems} to \cmd{printshorthands}\see{use:bib:biblist}
%\item Added counter \cnt{maxitems}\see{aut:fmt:ilc}
%\item Added counter \cnt{minitems}\see{aut:fmt:ilc}
%\item Added adapted headings for \sty{scrartcl}, \sty{scrbook}, \sty{scrreprt}\see{int:pre:cmp}
%\item Added adapted headings for \sty{memoir}\see{int:pre:cmp}
%\item Added \cmd{Cite}\see{use:cit:std}
%\item Added \cmd{Parencite}\see{use:cit:std}
%\item Added \cmd{Textcite}\see{use:cit:cbx}
%\item Added \cmd{parencite*}\see{use:cit:cbx}
%\item Added \cmd{supercite}\see{use:cit:cbx}
%\item Added \cmd{Citeauthor}\see{use:cit:txt}
%\item Added \cmd{nameyeardelim}\see{use:fmt:fmt}
%\item Added \cmd{multilistdelim}\see{use:fmt:fmt}
%\item Completed documenation\see{use:fmt:fmt}
%\item Completed documenation\see{aut:fmt:fmt}
%\item Added \cmd{usecitecmd}\see{aut:cbx:cbx}
%\item Added \cmd{hyphenate}\see{use:fmt:aux}
%\item Added \cmd{hyphen}\see{use:fmt:aux}
%\item Added \cmd{nbhyphen}\see{use:fmt:aux}
%\item Improved \cmd{ifsamepage}\see{aut:aux:tst}
%\item Removed \cmd{ifnameequalstr}\see{aut:aux:tst}
%\item Removed \cmd{iflistequalstr}\see{aut:aux:tst}
%\item Added \cmd{ifcapital}\see{aut:aux:tst}
%\item Added documentation of \cmd{MakeCapital}\see{aut:aux:msc}
%\item Added starred variant to \cmd{setunit}\see{aut:pct:new}
%\item Improved \cmd{ifterm}\see{aut:pct:chk}
%\item Straightened out documentation of \cmd{thelist}\see{aut:aux:dat}
%\item Straightened out documentation of \cmd{thename}\see{aut:aux:dat}
%\item Added \cmd{docsvfield}\see{aut:aux:msc}
%\item Added \cmd{docsvlist}\see{aut:aux:msc}
%\item Removed \cmd{CopyFieldFormat}\see{aut:bib:fmt}
%\item Removed \cmd{CopyIndexFieldFormat}\see{aut:bib:fmt}
%\item Removed \cmd{CopyListFormat}\see{aut:bib:fmt}
%\item Removed \cmd{CopyIndexListFormat}\see{aut:bib:fmt}
%\item Removed \cmd{CopyNameFormat}\see{aut:bib:fmt}
%\item Removed \cmd{CopyIndexNameFormat}\see{aut:bib:fmt}
%\item Added \cmd{savefieldformat}\see{aut:aux:msc}
%\item Added \cmd{restorefieldformat}\see{aut:aux:msc}
%\item Added \cmd{savelistformat}\see{aut:aux:msc}
%\item Added \cmd{restorelistformat}\see{aut:aux:msc}
%\item Added \cmd{savenameformat}\see{aut:aux:msc}
%\item Added \cmd{restorenameformat}\see{aut:aux:msc}
%\item Added \cmd{savebibmacro}\see{aut:aux:msc}
%\item Added \cmd{restorebibmacro}\see{aut:aux:msc}
%\item Added \cmd{savecommand}\see{aut:aux:msc}
%\item Added \cmd{restorecommand}\see{aut:aux:msc}
%\item Added documentation of \texttt{shorthands} driver\see{aut:bbx:bbx}
%\item Rearranged, renamed, and extended localization keys\see{aut:lng:key}
%\item Renamed counter \cnt{citecount} to \cnt{instcount}\see{aut:fmt:ilc}
%\item Added new counter \cnt{citecount}\see{aut:fmt:ilc}
%\item Added counter \cnt{citetotal}\see{aut:fmt:ilc}
%\item Rearranged and expanded documentation\see{bib:use}
%\item Expanded documentation\see{bib:cav}
%\item Expanded documentation\see{use:cav:scr}
%\item Expanded documentation\see{use:cav:mem}
%\item Completed support for Spanish\see{use:loc:esp}
%\item Added support for Italian (translations by Enrico Gregorio)
%\item Added language alias \opt{australian}\see{bib:fld:spc}
%\item Added language alias \opt{newzealand}\see{bib:fld:spc}
%\item Various minor improvements throughout
%
%\end{release}
%
%\begin{release}{0.5}{2006-11-12}
%
%\item Added \cmd{usedriver}\see{aut:aux:msc}
%\item Added package option \opt{pagetracker}\see{use:opt:pre:gen}
%\item Added \cmd{iffirstonpage}\see{aut:aux:tst}
%\item Added \cmd{ifsamepage}\see{aut:aux:tst}
%\item Corrected documentation of \cmd{ifciteseen}\see{aut:aux:tst}
%\item Added package option \opt{terseinits}\see{use:opt:pre:gen}
%\item Modified default value of package option \opt{maxnames}\see{use:opt:pre:gen}
%\item Renamed package option \opt{index} to \opt{indexing}\see{use:opt:pre:gen}
%\item Extended package option \opt{indexing}\see{use:opt:pre:gen}
%\item Removed package option \opt{citeindex}\see{use:opt:pre:gen}
%\item Removed package option \opt{bibindex}\see{use:opt:pre:gen}
%\item Added package option \opt{labelalpha}\see{use:opt:pre:int}
%\item Updated documentation of field \bibfield{labelalpha}\see{aut:bbx:fld}
%\item Added package option \opt{labelctitle}\see{use:opt:pre:int}
%\item Updated documentation of field \bibfield{labelctitle}\see{aut:bbx:fld}
%\item Added package option \opt{labelnumber}\see{use:opt:pre:int}
%\item Updated documentation of field \bibfield{labelnumber}\see{aut:bbx:fld}
%\item Added package option \opt{labelyear}\see{use:opt:pre:int}
%\item Updated documentation of field \bibfield{labelyear}\see{aut:bbx:fld}
%\item Added citation style \texttt{authortitle-verb}\see{use:xbx:cbx}
%\item Added citation style \texttt{authortitle-cverb}\see{use:xbx:cbx}
%\item Renamed citation style \texttt{traditional} to \texttt{authortitle-trad}\see{use:xbx:cbx}
%\item Improved citation style \texttt{authortitle-trad}\see{use:xbx:cbx}
%\item Added citation style \texttt{authortitle-strad}\see{use:xbx:cbx}
%\item Improved bibliography style \texttt{authoryear}\see{use:xbx:bbx}
%\item Improved bibliography style \texttt{authortitle}\see{use:xbx:bbx}
%\item Added option \opt{maxnames} to \cmd{printbibliography}\see{use:bib:bib}
%\item Added option \opt{minnames} to \cmd{printbibliography}\see{use:bib:bib}
%\item Added option \opt{maxnames} to \cmd{bibbysection}\see{use:bib:bib}
%\item Added option \opt{minnames} to \cmd{bibbysection}\see{use:bib:bib}
%\item Added option \opt{maxnames} to \cmd{bibbysegment}\see{use:bib:bib}
%\item Added option \opt{minnames} to \cmd{bibbysegment}\see{use:bib:bib}
%\item Added option \opt{maxnames} to \cmd{bibbycategory}\see{use:bib:bib}
%\item Added option \opt{minnames} to \cmd{bibbycategory}\see{use:bib:bib}
%\item Added option \opt{maxnames} to \cmd{printshorthands}\see{use:bib:biblist}
%\item Added option \opt{minnames} to \cmd{printshorthands}\see{use:bib:biblist}
%\item Renamed \env{bibsection} to \env{refsection} (conflict with \sty{memoir})\see{use:bib:sec}
%\item Renamed \env{bibsegment} to \env{refsegment} (consistency)\see{use:bib:seg}
%\item Extended \env{refsection} environment\see{use:bib:sec}
%\item Renamed \env{bibsection} counter to \env{refsection}\see{aut:fmt:ilc}
%\item Renamed \env{bibsegment} counter to \env{refsegment}\see{aut:fmt:ilc}
%\item Updated documentation\see{use:use:mlt}
%\item Added counter \cnt{citecount}\see{aut:fmt:ilc}
%\item Modified default definition of \cmd{blxauxprefix}\see{use:use:aux}
%\item Added \cmd{CopyFieldFormat}\see{aut:bib:fmt}
%\item Added \cmd{CopyIndexFieldFormat}\see{aut:bib:fmt}
%\item Added \cmd{CopyListFormat}\see{aut:bib:fmt}
%\item Added \cmd{CopyIndexListFormat}\see{aut:bib:fmt}
%\item Added \cmd{CopyNameFormat}\see{aut:bib:fmt}
%\item Added \cmd{CopyIndexNameFormat}\see{aut:bib:fmt}
%\item Added \cmd{clearfield}\see{aut:aux:dat}
%\item Added \cmd{clearlist}\see{aut:aux:dat}
%\item Added \cmd{clearname}\see{aut:aux:dat}
%\item Added \cmd{restorefield}\see{aut:aux:dat}
%\item Added \cmd{restorelist}\see{aut:aux:dat}
%\item Added \cmd{restorename}\see{aut:aux:dat}
%\item Renamed \cmd{bibhyperlink} to \cmd{bibhyperref}\see{aut:aux:msc}
%\item Added new command \cmd{bibhyperlink}\see{aut:aux:msc}
%\item Added \cmd{bibhypertarget}\see{aut:aux:msc}
%\item Renamed formatting directive \texttt{bibhyperlink} to \texttt{bibhyperref}\see{aut:fmt:ich}
%\item Added new formatting directive \texttt{bibhyperlink}\see{aut:fmt:ich}
%\item Added formatting directive \texttt{bibhypertarget}\see{aut:fmt:ich}
%\item Added \cmd{addlpthinspace}\see{aut:pct:spc}
%\item Added \cmd{addhpthinspace}\see{aut:pct:spc}
%\item Added field \bibfield{annotator}\see{bib:fld:dat}
%\item Added field \bibfield{commentator}\see{bib:fld:dat}
%\item Added field \bibfield{introduction}\see{bib:fld:dat}
%\item Added field \bibfield{foreword}\see{bib:fld:dat}
%\item Added field \bibfield{afterword}\see{bib:fld:dat}
%\item Updated documentation of field \bibfield{translator}\see{bib:fld:dat}
%\item Added field \bibfield{articleid}\see{bib:fld:dat}
%\item Added field \bibfield{doi}\see{bib:fld:dat}
%\item Added field \bibfield{file}\see{bib:fld:dat}
%\item Added field alias \bibfield{pdf}\see{bib:fld:als}
%\item Added field \bibfield{indextitle}\see{bib:fld:dat}
%\item Added field \bibfield{indexsorttitle}\see{bib:fld:spc}
%\item Changed data type of \bibfield{language}\see{bib:fld:dat}
%\item Changed data type of \bibfield{origlanguage}\see{bib:fld:dat}
%\item Updated documentation of entry type \bibfield{book}\see{bib:typ:blx}
%\item Updated documentation of entry type \bibfield{collection}\see{bib:typ:blx}
%\item Updated documentation of entry type \bibfield{inbook}\see{bib:typ:blx}
%\item Updated documentation of entry type \bibfield{incollection}\see{bib:typ:blx}
%\item Extended entry type \bibfield{misc}\see{bib:typ:blx}
%\item Added \cmd{UndefineBibliographyExtras}\see{use:lng}
%\item Added \cmd{UndeclareBibliographyExtras}\see{aut:lng:cmd}
%\item Added \cmd{finalandcomma}\see{use:fmt:lng}
%\item Added localization key \texttt{citedas}\see{aut:lng:key}
%\item Renamed localization key \texttt{editby} to \texttt{edited}\see{aut:lng:key}
%\item Renamed localization key \texttt{transby} to \texttt{translated}\see{aut:lng:key}
%\item Added localization key \texttt{annotated}\see{aut:lng:key}
%\item Added localization key \texttt{commented}\see{aut:lng:key}
%\item Added localization key \texttt{introduced}\see{aut:lng:key}
%\item Added localization key \texttt{foreworded}\see{aut:lng:key}
%\item Added localization key \texttt{afterworded}\see{aut:lng:key}
%\item Added localization key \texttt{commentary}\see{aut:lng:key}
%\item Added localization key \texttt{annotations}\see{aut:lng:key}
%\item Added localization key \texttt{introduction}\see{aut:lng:key}
%\item Added localization key \texttt{foreword}\see{aut:lng:key}
%\item Added localization key \texttt{afterword}\see{aut:lng:key}
%\item Added localization key \texttt{doneby}\see{aut:lng:key}
%\item Added localization key \texttt{itemby}\see{aut:lng:key}
%\item Added localization key \texttt{spanish}\see{aut:lng:key}
%\item Added localization key \texttt{latin}\see{aut:lng:key}
%\item Added localization key \texttt{greek}\see{aut:lng:key}
%\item Modified localization key \texttt{fromenglish}\see{aut:lng:key}
%\item Modified localization key \texttt{fromfrench}\see{aut:lng:key}
%\item Modified localization key \texttt{fromgerman}\see{aut:lng:key}
%\item Added localization key \texttt{fromspanish}\see{aut:lng:key}
%\item Added localization key \texttt{fromlatin}\see{aut:lng:key}
%\item Added localization key \texttt{fromgreek}\see{aut:lng:key}
%\item Expanded documentation\see{bib:use}
%\item Updated documentation\see{use:xbx:cbx}
%\item Updated documentation\see{use:xbx:bbx}
%\item Updated documentation\see{use:fmt:fmt}
%\item Updated documentation\see{aut:fmt:fmt}
%\item Updated and completed documentation\see{use:fmt:lng}
%\item Updated and completed documentation\see{aut:fmt:lng}
%\item Added support for Spanish (translations by Ignacio Fernández Galván)
%\item Various memory-related optimizations in \path{biblatex.bst}
%
%\end{release}
%
%\begin{release}{0.4}{2006-10-01}
%
%\item Added package option \opt{sortlos}\see{use:opt:pre:gen}
%\item Added package option \opt{bibtex8}\see{use:opt:pre:gen}
%\item Made \bibfield{pageref} field local to \env{refsection} environment\see{aut:bbx:fld}
%\item Renamed field \bibfield{labeltitle} to \bibfield{labelctitle}\see{aut:bbx:fld}
%\item Added new field \bibfield{labeltitle}\see{aut:bbx:fld}
%\item Added new field \bibfield{sortkey}\see{bib:fld:spc}
%\item Updated documentation\see{use:srt}
%\item Removed \cmd{iffieldtrue}\see{aut:aux:tst}
%\item Renamed counter \cnt{namepenalty} to \cnt{highnamepenalty}\see{use:fmt:len}
%\item Added counter \cnt{lownamepenalty}\see{use:fmt:len}
%\item Added documentation of \cmd{noligature}\see{use:fmt:aux}
%\item Added \cmd{addlowpenspace}\see{aut:pct:spc}
%\item Added \cmd{addhighpenspace}\see{aut:pct:spc}
%\item Added \cmd{addabbrvspace}\see{aut:pct:spc}
%\item Added \cmd{adddotspace}\see{aut:pct:spc}
%\item Added \cmd{addslash}\see{aut:pct:spc}
%\item Expanded documentation\see{use:cav}
%\item Various minor improvements throughout
%\item Fixed some bugs
%
%\end{release}
%
%\begin{release}{0.3}{2006-09-24}
%
%\item Renamed citation style \texttt{authortitle} to \texttt{authortitle-terse}\see{use:xbx:cbx}
%\item Renamed citation style \texttt{authortitle-comp} to \texttt{authortitle-cterse}\see{use:xbx:cbx}
%\item Renamed citation style \texttt{authortitle-verb} to \texttt{authortitle}\see{use:xbx:cbx}
%\item Added new citation style \texttt{authortitle-comp}\see{use:xbx:cbx}
%\item Citation style \texttt{traditional} now supports <loc.~cit.>\see{use:xbx:cbx}
%\item Added package option \opt{date}\see{use:opt:pre:gen}
%\item Added package option \opt{urldate}\see{use:opt:pre:gen}
%\item Introduced new data type: literal lists\see{bib:fld}
%\item Renamed \cmd{citename} to \cmd{citeauthor}\see{use:cit:txt}
%\item Renamed \cmd{citelist} to \cmd{citename}\see{use:cit:low}
%\item Added new \cmd{citelist} command\see{use:cit:low}
%\item Renamed \cmd{printlist} to \cmd{printnames}\see{aut:bib:dat}
%\item Added new \cmd{printlist} command\see{aut:bib:dat}
%\item Renamed \cmd{indexlist} to \cmd{indexnames}\see{aut:bib:dat}
%\item Added new \cmd{indexlist} command\see{aut:bib:dat}
%\item Renamed \cmd{DeclareListFormat} to \cmd{DeclareNameFormat}\see{aut:bib:fmt}
%\item Added new \cmd{DeclareListFormat} command\see{aut:bib:fmt}
%\item Renamed \cmd{DeclareListAlias} to \cmd{DeclareNameAlias}\see{aut:bib:fmt}
%\item Added new \cmd{DeclareListAlias} command\see{aut:bib:fmt}
%\item Renamed \cmd{DeclareIndexListFormat} to \cmd{DeclareIndexNameFormat}\see{aut:bib:fmt}
%\item Added new \cmd{DeclareIndexListFormat} command\see{aut:bib:fmt}
%\item Renamed \cmd{DeclareIndexListAlias} to \cmd{DeclareIndexNameAlias}\see{aut:bib:fmt}
%\item Added new \cmd{DeclareIndexListAlias} command\see{aut:bib:fmt}
%\item Renamed \cmd{biblist} to \cmd{thename}\see{aut:aux:dat}
%\item Added new \cmd{thelist} command\see{aut:aux:dat}
%\item Renamed \cmd{bibfield} to \cmd{thefield}\see{aut:aux:dat}
%\item Renamed \cmd{savelist} to \cmd{savename}\see{aut:aux:dat}
%\item Added new \cmd{savelist} command\see{aut:aux:dat}
%\item Renamed \cmd{savelistcs} to \cmd{savenamecs}\see{aut:aux:dat}
%\item Added new \cmd{savelistcs} command\see{aut:aux:dat}
%\item Renamed \cmd{iflistundef} to \cmd{ifnameundef}\see{aut:aux:tst}
%\item Added new \cmd{iflistundef} test\see{aut:aux:tst}
%\item Renamed \cmd{iflistsequal} to \cmd{ifnamesequal}\see{aut:aux:tst}
%\item Added new \cmd{iflistsequal} test\see{aut:aux:tst}
%\item Renamed \cmd{iflistequals} to \cmd{ifnameequals}\see{aut:aux:tst}
%\item Added new \cmd{iflistequals} test\see{aut:aux:tst}
%\item Renamed \cmd{iflistequalcs} to \cmd{ifnameequalcs}\see{aut:aux:tst}
%\item Added new \cmd{iflistequalcs} test\see{aut:aux:tst}
%\item Renamed \cmd{iflistequalstr} to \cmd{ifnameequalstr}\see{aut:aux:tst}
%\item Added new \cmd{iflistequalstr} test\see{aut:aux:tst}
%\item Renamed \cmd{ifcurrentlist} to \cmd{ifcurrentname}\see{aut:aux:tst}
%\item Added new \cmd{ifcurrentlist} test\see{aut:aux:tst}
%\item Entry type alias \bibtype{conference} now resolved by \bibtex\see{bib:typ:als}
%\item Entry type alias \bibtype{mastersthesis} now resolved by \bibtex\see{bib:typ:als}
%\item Entry type alias \bibtype{phdthesis} now resolved by \bibtex\see{bib:typ:als}
%\item Entry type alias \bibtype{techreport} now resolved by \bibtex\see{bib:typ:als}
%\item Entry type alias \bibtype{www} now resolved by \bibtex\see{bib:typ:als}
%\item Added new custom fields \bibfield{lista} through \bibfield{listf}\see{bib:fld:ctm}
%\item Changed data type of \bibfield{location}\see{bib:fld:dat}
%\item Changed data type of \bibfield{origlocation}\see{bib:fld:dat}
%\item Changed data type of \bibfield{publisher}\see{bib:fld:dat}
%\item Changed data type of \bibfield{institution}\see{bib:fld:dat}
%\item Changed data type of \bibfield{organization}\see{bib:fld:dat}
%\item Modified values of \bibfield{gender} field for \sty{jurabib} compatibility\see{bib:fld:spc}
%\item Modified and extended \texttt{idem\dots} keys for \sty{jurabib} compatibility\see{aut:lng:key}
%\item Improved \cmd{addtocategory}\see{use:bib:cat}
%\item Removed formatting command \cmd{mkshorthand}\see{use:fmt:fmt}
%\item Added field formatting directive \texttt{shorthandwidth}\see{aut:fmt:ich}
%\item Added documentation of \cmd{shorthandwidth}\see{aut:fmt:ilc}
%\item Removed formatting command \cmd{mklabelnumber}\see{use:fmt:fmt}
%\item Added field formatting directive \texttt{labelnumberwidth}\see{aut:fmt:ich}
%\item Added documentation of \cmd{labelnumberwidth}\see{aut:fmt:ilc}
%\item Removed formatting command \cmd{mklabelalpha}\see{use:fmt:fmt}
%\item Added field formatting directive \texttt{labelalphawidth}\see{aut:fmt:ich}
%\item Added documentation of \cmd{labelalphawidth}\see{aut:fmt:ilc}
%\item Renamed \cmd{bibitem} to \cmd{thebibitem}\see{aut:bbx:env}
%\item Renamed \cmd{lositem} to \cmd{thelositem}\see{aut:bbx:env}
%\item Modified \cmd{AtBeginBibliography}\see{aut:fmt:hok}
%\item Added \cmd{AtBeginShorthands}\see{aut:fmt:hok}
%\item Added \cmd{AtEveryLositem}\see{aut:fmt:hok}
%\item Extended \sty{showkeys} compatibility to list of shorthands\see{int:pre:cmp}
%\item Added compatibility code for the \sty{hyperref} package\see{int:pre:cmp}
%\item Added package option \opt{hyperref}\see{use:opt:pre:gen}
%\item Added package option \opt{backref}\see{use:opt:pre:gen}
%\item Added field \bibfield{pageref}\see{aut:bbx:fld}
%\item Added \cmd{ifhyperref}\see{aut:aux:msc}
%\item Added \cmd{bibhyperlink}\see{aut:aux:msc}
%\item Added field formatting directive \texttt{bibhyperlink}\see{aut:fmt:ich}
%\item Renamed \cmd{ifandothers} to \cmd{ifmorenames}\see{aut:aux:tst}
%\item Added new \cmd{ifandothers} test\see{aut:aux:tst}
%\item Removed field \bibfield{moreauthor}\see{aut:bbx:fld}
%\item Removed field \bibfield{morebookauthor}\see{aut:bbx:fld}
%\item Removed field \bibfield{moreeditor}\see{aut:bbx:fld}
%\item Removed field \bibfield{morelabelname}\see{aut:bbx:fld}
%\item Removed field \bibfield{moretranslator}\see{aut:bbx:fld}
%\item Removed field \bibfield{morenamea}\see{aut:bbx:fld}
%\item Removed field \bibfield{morenameb}\see{aut:bbx:fld}
%\item Removed field \bibfield{morenamec}\see{aut:bbx:fld}
%\item Updated documentation\see{aut:int}
%\item Updated documentation\see{aut:bbx:bbx}
%\item Updated documentation\see{aut:bbx:env}
%\item Updated documentation\see{aut:bbx:drv}
%\item Expanded documentation\see{aut:fmt}
%\item Modified internal \bibtex interface
%\item Fixed some typos in the manual
%\item Fixed some bugs
%
%\end{release}
%
%\begin{release}{0.2}{2006-09-06}
%
%\item Added bibliography categories\see{use:bib:cat}
%\item Added \cmd{DeclareBibliographyCategory}\see{use:bib:cat}
%\item Added \cmd{addtocategory}\see{use:bib:cat}
%\item Added \texttt{category} and \texttt{notcategory} filters\see{use:bib:bib}
%\item Added \cmd{bibbycategory}\see{use:bib:bib}
%\item Added usage examples for bibliography categories\see{use:use:div}
%\item Added documentation of configuration file\see{use:cfg:cfg}
%\item Added documentation of \cmd{ExecuteBibliographyOptions}\see{use:cfg:opt}
%\item Added documentation of \cmd{AtBeginBibliography}\see{aut:fmt:hok}
%\item Added \cmd{AtEveryBibitem}\see{aut:fmt:hok}
%\item Added \cmd{AtEveryCite}\see{aut:fmt:hok}
%\item Added \cmd{AtEveryCitekey}\see{aut:fmt:hok}
%\item Added optional argument to \cmd{printtext}\see{aut:bib:dat}
%\item Added \cmd{ifpage}\see{aut:aux:tst}
%\item Added \cmd{ifpages}\see{aut:aux:tst}
%\item Added field \bibfield{titleaddon}\see{bib:fld:dat}
%\item Added field \bibfield{booktitleaddon}\see{bib:fld:dat}
%\item Added field \bibfield{maintitleaddon}\see{bib:fld:dat}
%\item Added field \bibfield{library}\see{bib:fld:dat}
%\item Added field \bibfield{part}\see{bib:fld:dat}
%\item Added field \bibfield{origlocation}\see{bib:fld:dat}
%\item Added field \bibfield{origtitle}\see{bib:fld:dat}
%\item Added field \bibfield{origyear}\see{bib:fld:dat}
%\item Added field \bibfield{origlanguage}\see{bib:fld:dat}
%\item Modified profile of field \bibfield{language}\see{bib:fld:dat}
%\item Extended entry type \bibtype{book}\see{bib:typ:blx}
%\item Extended entry type \bibtype{inbook}\see{bib:typ:blx}
%\item Extended entry type \bibtype{collection}\see{bib:typ:blx}
%\item Extended entry type \bibtype{incollection}\see{bib:typ:blx}
%\item Extended entry type \bibtype{proceedings}\see{bib:typ:blx}
%\item Extended entry type \bibtype{inproceedings}\see{bib:typ:blx}
%\item Added entry type alias \bibtype{www}\see{bib:typ:als}
%\item Added compatibility code for the \sty{showkeys} package\see{int:pre:cmp}
%\item Support printable characters in \bibfield{keyword} and \texttt{notkeyword} filters\see{use:bib:bib}
%\item Support printable characters in \bibfield{keywords} field\see{bib:fld:spc}
%\item Ignore spaces after commas in \bibfield{keywords} field\see{bib:fld:spc}
%\item Internal rearrangement of all bibliography styles
%\item Fixed various bugs
%
%\end{release}
%
%\begin{release}{0.1}{2006-09-02}
%
%\item Initial public release
%
%\end{release}

\end{changelog}

\end{document}
%%% Local Variables:
%%% coding: utf-8
%%% eval: (auto-fill-mode -1)
%%% eval: (visual-line-mode)
%%% End:





