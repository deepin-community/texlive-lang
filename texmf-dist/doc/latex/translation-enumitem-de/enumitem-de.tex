\documentclass[a4paper,ngerman]{ltxguide}

\usepackage[english,ngerman]{babel}
\usepackage[T1]{fontenc}
\usepackage[utf8]{inputenc}
\usepackage{lmodern}
\usepackage[dvips]{color}
\usepackage{listings}
\usepackage{csquotes}
\usepackage[flushmargin]{footmisc}
\usepackage{url}
\usepackage{enumitem}

\makeatletter
\newenvironment{desc}
  {\if@nobreak
     \vskip-\lastskip
     \vspace*{-2.5ex}%
   \fi
   \decl}
  {\enddecl}
\makeatother

\newcommand\0{\unskip\enspace\fbox{\fontsize{4}{4}\selectfont NEU 3.0}}
\newcommand\3{\unskip\enspace\fbox{\fontsize{4}{4}\selectfont NEW 3.0}}

\usepackage{hyperref}
\parindent 0pt


\title{Anpassen von Listen\\ mit dem\\ Paket
\textsf{enumitem}\footnote{Version 3.3 \textbf{\"ubersetzt von Matthias Ludwig (FSU
Jena)}, auf die aktuelle Version 3.5.2 angepasst von Christine R\"omer.}}

\author{Javier Bezos}


\date{Version 3.5.2\\2011-09-28}

\IfFileExists{enumitem.sty}{\usepackage{enumitem}}{}

\addtolength{\topmargin}{-3pc}
\addtolength{\textwidth}{6pc}
\addtolength{\oddsidemargin}{-2pc}
\addtolength{\textheight}{7pc}

\raggedright
\parindent1.8em
\parskip0pt

\begin{document}

\maketitle

F\"ur Anregungen, um Kommentare und um Fehler zu melden, besuchen Sie 
\url{http://www.tex-tipografia.com/enumitem.html}.\\
(English is not my strong point, so contact me when you
find mistakes in the manual.) \\
Weitere Pakete des Autors: \textsf{gloss} (mit Jos\'e Luis D\'{\i}az), \textsf{accents, tensind, 
esindex, dotlessi, titlesec, titletoc}.


	\section{Einf\"uhrende Bemerkungen}
	
Als ich vor einigen Jahren begann \LaTeX{} zu nutzen, gab es zwei Punkte, die ich als ganz 
besonders \"argerlich empfand: Kopf- bzw. Fu\ss zeilen und Listen, denn es
war sehr kompliziert, sie zu modifizieren. Erstere lassen sich mittels meines
\textsf{titlesec}-Paket anpassen, aber f\"ur 
Letztere gab es bislang nur folgende M\"oglichkeiten:
	
\begin{itemize}
\item{\textsf{enumerate}, ein Paket, welches es nur erlaubt, das Label einer
Liste zu \"andern.}
\item{\textsf{mdwlist}, welches nach der zugeh\"origen Dokumentation lediglich einige eventuell 
n\"utzliche listenbezogene Befehle und Umgebungen zur Verf\"ugung stellt, aber keine in sich geschlossene
Methode ist, mit Listen zu arbeiten.}
\item{\textsf{paralist}, welches es erlaubt, Listen innerhalb eines Paragraphen anzuwenden (der 
eigentliche Zweck dieses Pakets) und einige sehr spezifische \"Anderungen und das optionale Argument
von \textsf{enumerate} zur Verf\"ugung stellt.}
\end{itemize}

Einer der gr\"o\ss ten Schwachpunkte des Standardpakets \textsf{list} ist, dass die Bedeutung seiner Parameter 
nicht immer klar ersichtlich ist. Um eine benutzerfreundlichere Bedienung zu erm\"oglichen, gab es zwei gangbare Wege: 
Entweder neue Listen definieren oder eine neue Syntax etablieren, die es einfacher macht, die Standardlisten anzupassen. 
F\"ur Marks habe ich in \textsf{titlesec} den ersten Ansatz gew\"ahlt, einfach weil ich keine befriedigende L\"osung 
f\"ur dieses Problem mit den \LaTeX -internen Makros gefunden habe, hier habe ich jedoch auf den zweiten Ansatz 
zur\"uckgegriffen.
 
Das Paket verwendet eine Art \glqq Vererbung\grqq. Das Verhalten von Listen l\"asst sich global einstellen, 
Parameter eines bestimmten Pakets wie z.\,B. \textsf{enumerate} k\"onnen von diesen Einstellungen aber abweichen, 
gleiches gilt auch f\"ur Parameter einer bestimmten Listenebene. Die jeweiligen Werte ergeben sich aus der Hierarchie.
	
	\section{Das Paket}
	
Dieses Paket ist daf\"ur gedacht, die Anpassung der drei Basislistenumgebungen \textsf{enumerate},\textsf{itemize} 
und \textsf{description} an eigene Bed\"urfnisse zu erleichtern. Es erweitert ihre Syntax dahingehend, dass sie ein 
optionales Argument erlauben, welches der Form \textsf{Schl\"ussel=Wert} entspricht.

	\begin{itemize}
		\item{Vertikale Abst\"ande}
			\begin{itemize}
				\item{\textsf{topsep}}
				\item{\textsf{partopsep}}
				\item{\textsf{parsep}}
				\item{\textsf{itemsep}}
			\end{itemize}
		\item{Horizontale Abst\"ande}
			\begin{itemize}
				\item{\textsf{leftmargin}}
				\item{\textsf{rightmargin}}
				\item{\textsf{listparindent}}
				\item{\textsf{labelwidth}}
				\item{\textsf{labelsep}}
				\item{\textsf{itemindent}}
			\end{itemize}
	\end{itemize}
	
	Zum Beispiel:

	\begin{verbatim}
	\begin{itemize}[itemsep=1ex,leftmargin=1cm]
	\end{verbatim}	
	
Die oben aufgef\"uhrten Schl\"ussel sind \"aquivalent zu den bekannten Listen-Parametern, weitere Informationen zu 
diesen finden sich in einem \LaTeX -Handbuch. Der n\"achste Abschnitt widmet sich den Erweiterungen, die 
\textsf{enumitem} bereitstellt.

	\section{Schl\"ussel}
	
Dieser Abschnitt beschreibt Schl\"ussel in angezeigten Listen. Die meisten von ihnen sind in 
Inline-Listen\footnote{Anm.\,d.\,\"U.: 
Bei Inline-Listen handelt es sich um Listen, die im Flie\ss text erscheinen, jedoch ist \glqq Flie\ss textliste\grqq{} 
kein konventionalisierter Ausdruck, weshalb hier der englische Begriff beibehalten wird.} verf\"ugbar, wo aber auch 
weitere Schl\"ussel verwendet werden k\"onnen (siehe \ref{s.inline}).

	\subsection{Label- und Querverweisformat} 
	
	\begin{desc}
	|label=<commands>|
	\end{desc}

Dieser Befehl setzt eine Markierung, die auf der aktuellen Ebene verwendet werden kann. Das Setzen einer gesternten 
Version von |\alph|, |\Alph|,|\arabic|,|\roman| und |\Roman| ohne Argument steht f\"ur den aktuellen Z\"ahler von 
\textsf{enumerate}.\footnote{Pr\"azise ausgedr\"uckt, der Asterisk ist das Argument, aber das k\"onnte sich \"andern. 
Deshalb sollte man sie einfach als gesternte Variante betrachten und die zugeh\"orige Syntax verwenden.} Demnach gibt
	
	\begin{verbatim}
		\begin{enumerate}[label=\emph{\alph*}]
	\end{verbatim}

\emph{a)},\,\emph{b)} usw. aus (dies ist der Standardstil in Spanisch und wurde ehemals auch von Chicago verwendet).	
Selbiges funktioniert auch mit |\value|	(vorrausgesetzt, das weiteste Label soll nicht verarbeitet werden oder 
|widest*| wird benutzt, siehe unten). Ein ausgefalleneres Beispiel (welches h\"asslich aussieht, aber demonstrieren 
soll, was theoretisch m\"oglich ist; ben\"otigt |color| und |pifont|):

	\begin{verbatim}
		\begin{enumerate}[label=\protect\fcolorbox{blue}{yellow}
		{\protect\ding{\value*}}, start=172]
	\end{verbatim}

Der Wert von |label| ist ein bewegliches Argument und empfindliche Befehle m\"ussen gesch\"utzt werden, was aber 
nicht f\"ur die Z\"ahler gilt. Deswegen ist der Gebrauch von |value| einigerma\ss en problematisch, da |\the| 
oder |\ifnum| einen konkreten Wert verlangen, was aber nicht der Fall ist, wenn |label| verwendet wird, um 
intern * mit dem Z\"ahlerargument zu ersetzen. Die in der Regel beste L\"osung f\"ur dieses Problem ist es, mit
Hilfe von |\AddEnumerateCounter| die Logik in einem neuen Z\"ahler einzukapseln.\footnote{Dies ist zugegebenerma\ss en 
ziemlich umst\"andlich. An einer besseren L\"osung wird gearbeitet.}
Wenn man die Art Label zu setzen bevorzugt, die im Paket \textsf{enumerate} verwendet wird, kann man 
\glqq Kurz-Label\grqq benutzen (siehe Abschnitt \ref{s.short}).

	\begin{desc}
	|label*=<commands>|
	\end{desc}

Dies funktioniert \"ahnlich wie |label|, nur das dessen Wert dem \"ubergeordneten Label hintenangestellt ist. So 
definiert beispielsweise der folgende Quelltext eine |legal|-Liste (1.,\,1.1,\,1.1.1 usw.):

	\begin{verbatim}
		\newlist{legal}{enumerate}{10}
		\setlist[legal]{label*=\arabic*.}
	\end{verbatim}

	\begin{desc}
	|ref=<commands>|
	\end{desc}
	
In der Default-Einstellung legt |label| auch die Form der Querverweise und
von \textsf{the...} fest, wobei Letzteres 
die Einstellungen der vorhergehenden Hierarchie-Ebenen \"uberschreibt, aber mit diesem Schl\"ussel kann man ein 
abweichendes Format definieren. Das folgende Beispiel zeigt, wie man die rechte Parenthese entfernen kann:
	
	\begin{verbatim}
		\begin{enumerate}[label=\emph{\alph*}),ref=\emph{\alph*}]
	\end{verbatim}

Sowohl bei |label| als auch bei |ref| k\"onnen die Z\"ahler wie gewohnt verwendet werden:

	\begin{verbatim}
		\begin{enumerate}[label=\theenumi.\arabic*.]
	\end{verbatim}
	
oder
	
	\begin{verbatim}
		\begin{enumerate}[label=\arabic{enumi}.\arabic*.] 
	\end{verbatim}
	
(vorrausgesetzt, man befindet sich auf der zweiten Gliederungsebene).
	
Allerdings ist zu beachten, dass sich die Verweise nicht aus den Labeln der einzelnen Gliederungsebenen zusammensetzen. 
M\"ochte man beispielsweise mit 1.\emph{a} auf einen Punkt in einer Liste verweisen, deren Gliederungsebenen mit 1) 
(erste Ebene) und \emph{a)} (zweite Ebene) beschriftet sind, muss man dies \"uber |ref| einstellen, beispielsweise 
mit |\ref{level1}.\ref{level2}| und den dazu passenden |ref|-Einstellungen. Jedoch stellte schon Robin Fairbairns in 
der \TeX{}-FAQ fest:
	
\begin{quote}
\dots{} [that] would be both tedious and error-prone. What is more, it 
would be undesirable, since you would be constructing a visual 
representation which is inflexible (you could not change all the 
references to elements of a list at one fell swoop).\footnote{Beides w\"are
m\"uhsam und fehleranf\"allig. Obendrein w\"are es auch nicht
w\"unschenswert, da man auf diesem Wege eine visuelle Repr\"asentation
erstellt, die nicht flexibel ist (man kann nicht alle Verweise auf Elemente
einer Liste mit einem gef\"uhlten Wisch ver\"andern).}
\end{quote}

Es wird empfohlen diesem Rat zu folgen, aber manchmal m\"ochte man etwas wie das Folgende:

	\begin{verbatim}
		...subitem \ref{level2} of item \ref{level1} ...
	\end{verbatim}

Der Wert von |ref| ist ein bewegliches Argument und empfindliche Befehle
m\"ussen gesch\"utzt werden, was aber \textit{nicht} f\"ur die Z\"ahler gilt.
	
	\begin{desc}
	|font=<commands>| \hspace{.5cm} |format=<commands>|
	\end{desc}
	
Dies setzt \glqq label font\grqq. Das ist n\"utzlich, wenn das Label mit |\item| oder in \textsf{description} 
ver\"andert wird. 
Der letzte Befehl in |<commands>| kann ein optionales Argument mit dem Label des Items  zu sich nehmen. In 
\textsf{description} sind die Klassen-Einstellungen aktiv, demnach kann es
n\"utzlich sein, mit |\normalfont| zu 
beginnen. Ein synonymer Befehl ist |format|.
	
	\begin{desc}
	|align=left| \hspace{.5cm} |align=right| \hspace{.5cm} |align=parleft| \3
	\end{desc}

Dies legt die Ausrichtung des Labels in Relation zu den Labelbox-Kanten fest. Drei Werte sind m\"oglich: |left|, 
der Default-Wert |right| und |parleft| (eine Parbox der Breite |\labelwidth| mit Text im linksb\"undigen Flattersatz). 
Die Parameter, die die Ausrichtung des Labels bestimmen, sollten richtig gesetzt werden, entweder per Hand oder 
bequemer mit der Einstellung * (siehe unten):
	
	\begin{verbatim}
		\begin{enumerate}[label=\Roman*., align=left, leftmargin=*]
	\end{verbatim}
	
(Wenn die Labelbox die urspr\"ungliche Breite haben soll, muss |left| verwendet werden.)
	
	\begin{desc}
	|\SetLabelAlign{<value>}{<commands>}| \3
	\end{desc}
	
Neue Ausrichtungen k\"onnen \"uber |\SetLabelAlign| definiert und
vorhandene ge\"andert werden; die vordefinierten Werte sind \"aquivalent zu:\footnote{Vor Version 3.0 war 
die linksb\"undige Ausrichtung fehlerhaft und Text und Label konnten sich \"uberlappen.}
	
	\begin{verbatim}
		\SetLabelAlign{right}{\hss\llap{#1}}
		\SetLabelAlign{left}{#1\hfil}
		\SetLabelAlign{parleft}{\strut\smash{\parbox[t]\labelwidth{\raggedright##1}}}
	\end{verbatim}
	
Wenn der letzte Punkt in der Definition ein Zeilenumbruch ist (typischerweise |\hfil|), wird er manchmal von 
\textsf{description} entfernt. Wenn man dies aus irgendwelchen Gr\"unden
verhindern will, muss am Ende lediglich \verb|\null| angef\"ugt werden.

Obwohl prim\"ar f\"ur die Ausrichtung bestimmt, sind diese Befehle anders
zu verwenden (wie das mitgelieferte \verb|parleft|). Beispielsweise mit dem
folgenden |align=right|  werden alle Labels hochgestellt gesetzt:

\begin{verbatim}
\SetLabelAlign{right}{\hss\llap{\textsuperscript{#1}}}
\end{verbatim}

(Ein neuer Name ist nt\"urlich auch m\"oglich.)

Wenn Sie m\"ochten, dass die internen Einstellungen f\"ur \texttt{align} und \texttt{font}
ignoriert werden, k\"onnen Sie die \textsf{enumitem}-Definition f\"ur
\verb|\makelabel| in \texttt{zuvor} \"uberschreiben:

\begin{verbatim}
\begin{description}[before={\renewcommand\makelabel[1]{\ref{##1}}}]
\end{verbatim}

(Alternativ k\"onnen Sie ein Makro definieren und \verb|\let| nehmen.)

	
	\subsection{Horizontale Ausrichtung von Labeln} \smallskip
	
	\begin{desc}
	|labelindent=<length>| \\
	|\labelindent|
	\end{desc}
	
Dieser Parameter sorgt in \textsf{enumitem} f\"ur einen Leerraum zwischen Liste/Text und dem linken Rand der Labelbox. 
Dies bedeutet eine gewisse Redundanz, da ein Parameter auf den anderen aufbaut, also auf Basis anderer Werte errechnet
werden muss, wie unten beschrieben. Es gibt eine neue Z\"ahlerl\"ange |\labelindent|, deren Defaultwert bei 0pt liegt. 
Die f\"unf Parameter verhalten sich untereinander wie folgt:

\[
\verb|\leftmargin|+\verb|\itemindent| = 
\verb|\labelindent|+\verb|\labelwidth|+\verb|\labelsep|
\]

\begin{desc}
|leftmargin=!|\qquad|itemindent=!|\qquad|labelsep=!|
\qquad|labelwidth=!|\qquad|labelindent=!|\3
\end{desc}

		
Legt fest, welcher Wert von den anderen errechnet wird. Dies geschieht, nachdem \emph{alle} Werte gelesen wurden. 
Explizit festgelegte Werte gehen mit den folgenden Einstellungen nicht verloren:
	
	\begin{verbatim}
		leftmargin=2em
		labelindent=1em,leftmargin=!
		labelindent=!
	\end{verbatim}
	
|leftmargin| betr\"agt 2em  und |\labelindent| ist der zu errechnende Wert. Die Default-Einstellung ist 
|\labelindent=!|. Es ist jedoch zu beachten, dass einige Schl\"ussel einen
anderen Wert setzen (|wide| und \textsf{description}-Stiles). Eine Einstellung mit den Werten |align=right| 
(der Default), |labelindent=!| und |labelwidth=!| verh\"alt sich in der
Praxis \"ahnlich.

\begin{desc}
|leftmargin=*|\qquad|itemindent=*|\qquad|labelsep=*|
\qquad|labelwidth=*|\qquad|labelindent=*|
\end{desc}

	
Wie vorher, nur dass |\labelwidth| auf die Breite des aktuellen Labels angepasst wird. Dabei wird der Default-Wert 
von \emph{0} bei |\arabic*|, \emph{viii} bei |\roman*|, \emph{m} bei
|\alph*| verwendet, die gro\ss geschriebenen Varianten funktionieren
\"ahnlich. (Diese Werte k\"onnen mittels 
|\widest| ver\"andert werden, siehe unten). Beispiele daf\"ur sind:
	
	\begin{verbatim}
		\begin{itemize}[label=\textbullet, leftmargin=*]
		\begin{enumerate}[label=\roman*), leftmargin=*, widest=iii]
		\begin{itemize}[label=\textbullet,leftmargin=2pc, labelsep=*]
		\begin{enumerate}[label=\arabic*., leftmargin=2\parindent,
					labelindent=\parindent, labelsep=*]
	\end{verbatim}

	
Am n\"utzlichsten sind dabei |labelsep=*| und |leftmargin=*|. Mit Ersterem
beginnt der Textk\"orper an einer fest definierten Stelle (namentlich |leftmargin|), mit Letzterem an einer variablen, 
basierend auf der Breite des Labels (aber innerhalb einer Liste nat\"urlich immer an der gleichen). In der Regel wird 
es am g\"unstigsten sein, |leftmargin=*| zu verwenden.

Leider definiert \LaTeX{} keinen Default-Wert f\"ur |labelsep|, der in
allen Listen angewendet werden k\"onnte, es wird 
einfach immer nur der momentane Wert genutzt. Mit \textsf{enumitem} lassen sich Default-Werte f\"ur jede Liste 
festlegen, wie unten beschrieben wird, und um sicherzugehen, dass man volle
Kontrolle \"uber |labelsep|hat, ben\"otigt man lediglich etwas wie:
	
	\begin{verbatim}
		\setlist{itemsep=.5em}
	\end{verbatim}
	
	|labelwidth=*| und |labelwidth=!| funktionieren analog.

\begin{desc}
|widest=<string>|\qquad|widest*=<integer>|\3\qquad|widest|
\end{desc}

Wenn man es w\"unscht, lassen sich diese Befehle in Verbindung mit den *-Werten verwenden. Sie \"uberschreiben den 
Default-Wert des weitesten ausgegebenen Z\"ahlers. Wenn Listen nicht sehr lang sind, ist ein Wert \emph{a} 
f\"ur |\alph| manchmal passender als der Default-Wert \emph{m}:
	
	\begin{verbatim}
		\begin{enumerate}[leftmargin=*,widest=a] % zweite Gliederungsebene
	\end{verbatim}
	
Wird kein Wert festgelegt, tritt wieder der Default-Wert in Kraft. Mit |widest*| wird die Zeichenkette unter Verwendung 
von |<integer>| als Z\"ahlerwert gebildet (beispielsweise sind
\verb|\roman|, \verb|widest=viii| und \verb|widest*=8| gleichbedeutend). 

Da |\value| keine Zeichenketten sondern nur Nummern verarbeiten kann, funktionieren |widest| und die *-Werte nicht. 
Es geht aber mit \verb|widest*|, da dessen Wert eine Nummer ist. 

	\subsection{Mehr zur horizontalen Ausrichtung}

 Da |\parindent| als solches innerhalb von Listen keine Anwendung findet, sondern nur intern f\"ur |\itemindent| oder 
 |\listparindent| gesetzt wird, gibt \textsf{enumitem}, wenn |\parindent| als Parameter verwendet wird, den globalen 
 Wert aus, also den Wert, den |\parindent| au\ss erhalb der Liste besitzt.
	
Der horizentale Leerraum des linken Rands in der momentanen Gliederungsebene verteilt sich wie 
folgt:\footnote{Zugegebenerma\ss en hat die Abbildung nicht die gew\"unschte Aussagekraft, aber dies soll sich in 
einer zuk\"unftigen Version \"andern.}

\begin{center}
\begin{tabular}{cc}
\fbox{\fbox{\strut \texttt{labelindent}}
  \fbox{\strut \texttt{labelwidth}}
  \fbox{\strut \texttt{labelsep} $-$ \texttt{itemintent}}}
&
\fbox{\strut\texttt{itemindent}}\\
\texttt{leftmargin}
\end{tabular}
\end{center}

\begin{desc}
	|labelsep*=<length>|\3
\end{desc}
	
Zu beachten ist, dass |labelsep| Teile von |leftmargin| und |itemindent|
umfasst, wenn Letzteres nicht 0 ist. Das ist oftmals verwirrend, deshalb wird hier ein weiterer Schl\"ussel 
eingef\"uhrt. Mit |labelsep*| wird der Wert vom linken Rand aus berechnet (es setzt |labelsep| und addiert
|itemindent| hinzu, zus\"atzlich werden auch sp\"atere \"Anderungen von |itemindent| ber\"ucksichtigt).

\begin{center}
\begin{tabular}{cc}
\fbox{\fbox{\strut \texttt{labelindent}}
  \fbox{\strut \texttt{labelwidth}}
  \fbox{\strut \texttt{labelsep*}}}
&
\fbox{\strut\texttt{itemindent}}\\
\texttt{leftmargin}
\end{tabular}
\end{center}

\begin{desc}
|labelindent*=<length>|\3
\end{desc}

Dies funktioniert wie |labelindent|, wird aber vom linken Rand der momentanen Liste berechnet und von dem 
der \"ubergeordneten Liste bzw. des Texts.

\subsection{Nummerierung, Unterbrechung und Wiederaufnahme} 
	
	\begin{desc}
	|start=<integer>|
	\end{desc}
	
Legt die Nummer des ersten Items fest.

	\begin{desc}
	|resume|
	\end{desc}

Der Z\"ahler der letzten |enumerate|-Liste wird fortgef\"uhrt und nicht wieder auf 1 gesetzt.
	
	\begin{verbatim}
		\begin{enumerate}
		\item First item.
		\item Second item.
		\end{enumerate}
		Text.
		\begin{enumerate}[resume]
		\item Third item
		\end{enumerate}
	\end{verbatim}
	
Dies funktioniert aber nur lokal. Globale Wiederaufnahme wird im folgenden Abschnitt
\"uber Serien erl\"autert.
	
	\begin{desc}
	|resume*|
	\end{desc}	
	
Wie |resume|, nur dass zus\"atzlich noch die Optionen der vorherigen Liste \"ubernommen werden. Diese Option muss auf 
das optionale Argument in einer Umgebung beschr\"ankt sein (denn nur da ist es sinnvoll). Es sollte sparsam eingesetzt 
werden -- verwendet man es oft, wird man wohl eher eine neue Liste
definieren (siehe Abschnitt \ref{s.clone}). Es 
ist m\"oglich weitere Schl\"ussel zu definieren, in diesem Fall werden die gespeicherten Optionen von denen der 
aktuellen Liste \"uberschrieben. Die Position von |resume*| spielt dabei keine Rolle. Zum Beispiel:
	
	\begin{verbatim}
	\begin{enumerate}[resume*,start=1] % oder [start=1,resume*]
	\end{verbatim}

Es nutzt die Schl\"ussel vom vorherigen \texttt{enumerate}, startet jedoch
den Z\"ahler. Wenn es eine Reihe einer bestimmten Liste mit
\texttt{resume*} fortsetzt, werden die Optionen aus der vorherigen Liste
\"ubernommen, mit Ausnahme von \texttt{start}.

	
	\subsection{Serien} 

\begin{desc}
|series=<series-name>|\3\\
|<series-name>|\qquad|resume*=<series-name>|
\qquad|resume=<series-name>|\3
\end{desc}
		
Eine neue M\"oglichkeit (3.0) Listen weiterzuf\"uhren, besteht in der Verwendung des Schl\"ussels \textsf{series}. 
Dabei fungiert eine Liste mit dem Schl\"ussel \textsf{series} als Startliste und ihre Einstellungen werden 
\emph{global} gespeichert, so dass die sp\"atere Wiederaufnahme mittels |resume|/|resume*| erfolgen kann. Alle 
Schl\"ussel verlangen einen Namen als Wert, der sich von allen bereits vergebenen Namen unterscheiden muss.
	
	\begin{itemize}
		\item{|resume=<Serien-Name>| f\"uhrt einfach die Nummerierung der Bezugsliste weiter}
		\item{|resume*=<Serie>| \"ubernimmt auch die Einstellungen der Startliste}
		\item{|<Serie>|, d.\,h. die Verwendung des Seriennamens als Schl\"ussel, ist synonym 
		zu |resume*=<Serie>|}
	\end{itemize}
	
Zum Beispiel gibt
	
	\begin{verbatim}
		\begin{enumerate}[label=\arabic*(a),leftmargin=1cm,series=lafter]
		\item A
		\item B
		\end{enumerate}
	\end{verbatim}
	
eine Nummerierung 1(a), 2(a) aus. Daran kann man mit

\begin{verbatim}
\begin{enumerate}[label=\arabic*(b),resume*=lafter]
                 % or [label=\arabic*(b),lafter]
\item A
\item B
\end{enumerate}
\end{verbatim}

anschlie\ss en, was zu 3(b), 4(b) f\"uhrt. (Man kann aber auch |start=1|
verwenden.)
	
Zu beachten ist, dass man noch weitere Argumente hinzuf\"ugen kann, die dann nach denen aus der Startliste 
ausgef\"uhrt werden und deshalb bevorzugt behandelt werden. So wird insbesondere auch |resume*| gegen\"uber 
einem |start=1| aus der Startliste bevorzugt.

Jedes Mal, wenn eine Serie gestartet wird, werden einige Befehle intern gespeichert, um Ressourcenverschwendung 
und die Verwendung des gleichen Namens f\"ur sich nicht \"uberlappende Serien zu vermeiden.

	\subsection{Penalties (Strafpunkte)} 

\begin{desc}
|beginpenalty=<integer>|\qquad
|midpenalty=<integer>|\qquad |endpenalty=<integer>|
\end{desc}

Setzt einen Penalty an den Beginn einer Liste, zwischen die Items oder ans Ende. Man sollte sich aber unbedingt in 
einem \TeX - oder \LaTeX{}-Handbuch erkundigen, wie sich Penalties auf Seitenbr\"uche auswirken. Wird eine neue Liste 
gestartet, werden, im Gegensatz zu anderen Parametern, deren Werte nicht zur\"uck auf Default gesetzt und finden 
folglich Anwendung in den T\"ochterlisten.
	
	\begin{desc}
	|before=<code>| \hspace{.5cm} |before*=<code>|
	\end{desc}	
	
F\"uhrt den Kode aus, bevor die Liste gestartet wird (oder pr\"aziser, im zweiten Argument der \textsf{list}-Umgebung, 
welches zu deren Definition genutzt wird). Die ungesternte Variante
bewirkt, dass der Kode ausgef\"uhrt wird und dabei jeden vorherigen Wert
\"uberschreibt, die gesternte f\"ugt den Kode zu dem bereits existierenden hinzu (unter Ber\"ucksichtigung der 
Befehlshierarchie, ohne Beeinflussung von \"ubergeordnetem Text oder den Listen). Es kann Regeln und Text enthalten, 
dies wurde aber noch nicht ausf\"uhrlich getestet. Alle Berechnungen werden zuerst beendet und man hat Zugriff auf die 
Listenparameter und kann diese ver\"andern. Folgendes Beispiel zeigt, wie
man beide R\"ander (rechts und links) auf das weiteste Label setzen kann:
	
	\begin{verbatim}
		\setlist{leftmargin=*,before=\setlength{\rightmargin}{\leftmargin}}
	\end{verbatim}
	
\begin{desc}
|after=<code>|\qquad|after*=<code>|
\end{desc}

Wie zuvor, nur kurz vor dem Ende der Liste.


	\subsection{\texttt{description}/Beschreibungsstile}
	
Hierbei handelt es sich um einen Schl\"ussel, der in |description| verf\"ugbar ist.
	
	\begin{desc}
	|style=<name>|
	\end{desc}
	
Dies legt den \textsf{description}-Stil fest. F\"ur |<name>| k\"onnen
folgende Stile eingesetzt werden:
	
\begin{itemize}
\item|standard|: Wie \textsf{description} in Standard-Klassen, wenn das Ergebnis mit anderen Klassen etwas anders 
ausfallen kann. Das Label wird eingerahmt. Setzt \verb|itemindent=!|.

\item |unboxed|: \"Ahnlich wie das Standard-\textsf{description}, nur ohne gerahmtes Label, um zu vermeiden, dass die 
Ausrichtung uneinheitlich wird oder dass sehr lange Labels nicht
umgebrochen werden. Setzt \verb|itemindent=!|.

\item |nextline|: Wenn das Label nicht in den Rand passt, wird der Text in
der n\"achsten Zeile weitergef\"uhrt. Andernfalls wird es in einem Kasten der Breite |\leftmargin|\,$-$\,|\labelsep| 
plaziert, d.\,h. der Textk\"orper der Items reicht niemals bis in den
linken Rand hinein. Setzt \verb|labelwidth=!|.

\item |sameline|: Wie bei |nextline|, nur dass, wenn das Label nicht in den Rand passt, der Text auf der gleichen 
Zeile weitergef\"uhrt wird. Synonym zu \verb|style=unboxed,labelwidth=!|.

\item |multiline|: Das Label erscheint in einer Parbox, deren Breite dem
linken Rand (|leftmargin|) entspricht. 
Falls n\"otig, reicht sie auch \"uber mehrere Zeilen. Synonym zu \verb|style=standard,align=parleft,labelwidth=!|. 
	
Drei Warnungen: (1) Die Mischung von gerahmten und umgerahmten Labels weist
kein wohldefiniertes Verhalten auf. 
(2) wenn Listen verschachtelt werden, sind zwar alle Kombinationen erlaubt, aber nicht alle ergeben einen Sinn und 
(3) verschachtelte Listen mit \verb|nextline| werden nicht unterst\"utzt
(zwar funktioniert es, aber das k\"onnte sich in Zukunft \"andern, da ihr momentanes Verhalten nicht dem entspricht, 
was man erwarten k\"onnte.)
\end{itemize}


	\subsection{Kompakte Listen} 
	
	\begin{desc}
	|noitemsep| \hspace{.5cm} |nolistsep|
	\end{desc}	
	
Der Schl\"ussel |noitemsep| entfernt den Leerraum zwischen Items und
Paragraphen (d.\,h. |itemsep=0pt| und |parsep=0pt|), w\"ahrend |nolistsep|
alle vertikalen Leerr\"aume entfernt.\footnote{Der Schl\"ussel
\texttt{nolistsep}, jetzt veraltet, f\"uhrte einen kleinen Leerraum
(\emph{thin stretch}) ein, was nicht das Gew\"unschte war.}
	
	
	\subsection{\glqq Weite\grqq{} Listen} 
	
\begin{desc}
\verb|wide|\3\\
\verb|wide=<parindent>|
\end{desc}
		
Mit diesem bequemen Schl\"ussel wird der linke Rand (|leftmargin|) auf Null gesetzt und das Label wird Teil des 
Texts, d.\,h. die Items sehen wie gew\"ohnliche Paragraphen
aus.\footnote{\texttt{fullwidth} verliert seinen Wert.} |labelsep| sorgt
dabei f\"ur den Abstand zwischen dem Label und dem ersten Wort. Der
Schl\"ussel ist \"aquivalent zu
	
	\begin{verbatim}
		align=left, leftmargin=0pt, labelindent=\parindent,
		listparindent=\parindent, labelwidth=0pt, itemindent=!
	\end{verbatim}
	
\"Uber |wide=<parindent>| l\"asst sich auf einmal ein anderer Wert setzen
(anstatt von |parindent|). Nat\"urlich k\"onnen diese Werte nach |wide|
auch au\ss er Kraft gesetzt werden, beispielsweise sei daran erinnert, dass
mit linksb\"undigen Etiketten der Text verschoben wird, wenn sie breiter
als  |labelwidth| sind. Sie k\"onnen |labelwidth=1.5em| f\"ur eine minimale
Breite setzen oder anstelle von |itemindent=!| |itemindent=*| bevorzugen.
Das setzt die minimale Breite in Bezug zum breitesten Label. Auf der
zweiten Stufe k\"onnen Sie |labelindent=2\parindent| vorziehen, und so
weiter und so fort. Vielleicht m\"ochten Sie es auch mit |noitemsep| oder
|nolistsep| kombinieren.
	
	\section{Inline-Listen} \label{s.inline}

\3

Inline-Listen sind \glqq horizontale\grqq Listen, die als normaler Text in
einem Absatz auftreten. Mit diesem Paket l\"onnen Sie Inline-Listen, wie im
Folgenden erl\"autert wird, mit \verb|\newlist|, die ihre eigen Labels und
Z\"ahler haben, kreieren. Jedoch sind dies in den meisten F\"allen
Inline-Versionen von Standardlisten, mit dem gleichen Labelschema. Es wird
ausreichen, die Paketoption \verb|inline| anzuwenden.

\begin{desc}
|inline| \qquad(package option)\\
\texttt{enumerate*}\qquad\texttt{itemize*}\qquad
\texttt{description*} \qquad(environments)
\end{desc}

Mit der Paket-Option |inline| werden drei Umgebungen f\"ur Inline-Listen definiert: |enumerate*|, |itemize*| und 
|description*|. Sie emulieren das Verhalten von |paralist| und |shortlst| dahingehend, dass Labels und Einstellungen
mit den angezeigten (d.\,h. \glqq normalen\grqq) Listen |enumerate|, |itemize| und |description| geteilt werden 
(man sollte im Kopf behalten, dass die Wiederaufnahme von Listen vom
Umgebungs-Namen abh\"angt und nicht von den 
Listen-Typen). Das gilt aber nur f\"ur diejenigen Listen, die mit |inline| erzeugt wurden, Inline-Listen, die wie 
unten beschrieben, mittels |\newlist| gebildet wurden, sind unabh\"angig und nutzen ihre eigenen Labels und Einstellungen.
Merken Sie sich, dass \verb|inline| nicht erforderlich ist, wenn die
Inline-Versionen der Standardlisten nicht ben\"otigt werden.

\begin{desc}
|itemjoin=<string>|\qquad|itemjoin*=<string>|
\qquad|afterlabel=<string>|
\end{desc}

	
Das Format wird mit den Schl\"usseln |itemjoin| (der Default ist ein Leerzeichen) und |afterlabel| (der Default 
ist |\nobreakspace|, also |~|) gesetzt. Ein zus\"atzlicher Schl\"ussel ist |itemjoin*|, welches, wenn es gesetzt wird, 
anstatt |itemjoin| vor dem letzten Item genutzt wird. Das ergibt mit

\begin{verbatim}
before=\unskip{: }, itemjoin={{; }}, itemjoin*={{, and }}
\end{verbatim}


die folgende Zeichensetzung zwischen den Items:

\begin{quote}
Blah blah: (a) Eins; (b) Zwei; (c) Drei und (d) Vier. Blah blah
\end{quote}


|itemjoin| wird im vertikalen Modus ignoriert (also bei |mode=unboxed|, gleich nach einem Zitat, einer angezeigten 
Liste und \"Ahnlichem).

\begin{desc}
|mode=unboxed|\qquad|mode=boxed|
\end{desc}

Die Items sind gerahmt, weshalb sie keine Gleitobjekte sein k\"onnen und verschachtelte Listen nicht erlaubt sind 
(man sollte sich daran erinnern, dass viele dargestellte Elemente als Listen definiert sind). Wenn man Gleitobjekte und 
Listen innerhalb von Inline-Listen verwenden m\"ochte, muss man einen alternativen \glqq Modus\grqq{} verwenden, 
welcher sich mittels |mode=unboxed| aktivieren l\"asst (der Default ist |mode=boxed|). Damit lassen sich Gleitobjekte 
frei verwenden, aber fehlplatzierte |\item|s werden nicht eingefangen und |itemjoin*| wird ignoriert (es wird auch eine 
Warnung in die Log-Datei geschrieben).

	\section{Globale Einstellungen}
	
Globale Ver\"anderung, die auf alle diese Listen angewendet werden, sind
ebenfalls m\"oglich.
	
	\begin{desc}
	|\setlist[enumerate,<levels>]{<format>}| \\
	|\setlist[itemize,<levels>]{<format>}| \\
	|\setlist[description,<levels>]{<format>}| \\
	|\setlist[<levels>]{<format>}|
	\end{desc}
	
Dabei ist |<level>| die Listenebene (eine oder mehrere) in \textsf{list} und die korrespondierende Ebene in 
\textsf{enumerate} und \textsf{itemize}.\footnote{\verb|\string\setenumerate|, 
\verb|\string\setitemize| und \verb|\string\setdescription| verlieren ihre G\"ultigkeit als Befehle.} 
Wird der optionale Parameter |<levels>| nicht spezifiziert, gilt das Format f\"ur alle Ebenen. Listen meint an dieser 
Stelle nicht alle Listen, sondern nur die drei, die von diesem Paket
verarbeitet werden, und diejenigen, die vom Paket umdefiniert oder \"uber |\newlist| definiert werden (siehe unten). 
Zum Beispiel:

\begin{verbatim}
\setlist{noitemsep}
\setlist[1]{\labelindent=\parindent} % << Usually a good idea
\setlist[itemize]{leftmargin=*}
\setlist[itemize,1]{label=$\triangleleft$}
\setlist[enumerate]{labelsep=*, leftmargin=1.5pc}
\setlist[enumerate,1]{label=\arabic*., ref=\arabic*}
\setlist[enumerate,2]{label=\emph{\alph*}),
                      ref=\theenumi.\emph{\alph*}}
\setlist[enumerate,3]{label=\roman*), ref=\theenumii.\roman*}
\setlist[description]{font=\sffamily\bfseries}
\end{verbatim}

Diese Einstellungen werden in folgender Reihenfolge gelesen: \textsf{list}, \textsf{list} auf der momentanen Ebene, 
\textsf{enumerate}/\textsf{itemize}/\textsf{description} und \textsf{enumerate}/\textsf{itemize}/\textsf{description} 
auf der momentanen Ebene. Erscheint ein Schl\"ussel mehrere Male mit unterschiedlichen Werten, wird der letzte, also 
der spezifischste, angewendet. Wird eine Serie oder eine Liste weitergef\"uhrt, finden die gespeicherten Schl\"ussel 
Anwendung. Als Letztes wird das optionale Argument (au\ss er |resume*|) gelesen.

\LaTeX{} stellt eine Reihe Makros zur Verf\"ugung, um viele dieser
Parameter zu \"andern, aber es ist konsistenter und manchmal flexibler, sie
mit diesem Paket zu setzen, auch wenn dies zu 
Lasten der Klarheit (und \textit{verbose}) geht.

Die Spezifikation der Liste kann Variablen und Z\"ahler enthalten, vorrausgesetzt, dass sie erweiterbar und die 
Z\"ahler \textsf{calc}-savvy sind. Wenn man das Paket l\"adt, kann man also etwas wie das Folgende schreiben: 

	\begin{verbatim}
	\newcount{toplist}
	\setcount{toplist}{1}
	\newcommand{\mylistname}{enumerate}
	\setlist[\mylistname,\value{toplist}+1]{labelsep=\itemindent+2em]
	\end{verbatim}

Damit lassen sich Listen in Schleifen definieren.
	
Momentan gibt es keine M\"oglichkeit, die Gr\"o\ss e des Fonts zu unterscheiden (|\normalsize|, |\small|\dots ).

	
	\section{\textsf{enumerate}-artige Label} \label{s.short}

	
	\begin{desc}
	|shortlabels|  (Paket-Option)
	\end{desc}
	
Mit der Paket-Option |shortlabels| kann eine Syntax wie bei \textsf{enumerate} verwendet werden, wo |A|, |a|, |I|, |i|
und |1| f\"ur |\Alph*|, |\alph*|, |\Roman*|, |\roman*| und |\arabic*| stehen. Die Hauptabsicht dahinter ist, dass dies 
als eine Art Kompatibilit\"atsmodus mit dem Paket \textsf{enumerate} dient, weshalb die folgende Regel Anwendung findet: 
Wird die allererste Option (auf jeder Ebene) nicht als g\"ultiger Schl\"ussel erkannt, dann wird davon ausgegangen, 
dass es sich um ein Label mit \textsf{enumerate}-artiger Syntax handelt. Zum Beispiel:
	
	\begin{verbatim}
	\begin{enumerate}[i), labelindent=\parindent]
	...
	\end{enumerate}
	\end{verbatim}
	
Auch wenn es vielleicht nicht allzu n\"utzlich ist, kann |label=| auch in
der \textsf{itemize}-Umgebung unter \"ahnlichen Bedingungen weggelassen werden:
	
	\begin{verbatim}
	\begin{itemize}[\textbullet]
	...
	\end{itemize}
	\end{verbatim}
	
	\begin{desc}
	|\SetEnumerateShortLabel{<key>}{<replacement>}|
	\end{desc}
	
Mit diesem Befehl k\"onnen neue Schl\"ussel definiert bzw. umdefiniert werden, was besonders n\"utzlich ist, um 
\textsf{enumerate} an spezifische typographische Regeln anzupassen oder dessen Funktionsumfang auch auf 
nicht-lateinische Schriften zu erweitern. |<replacement>| enth\"alt an dieser Stelle eine der gesternten Varianten 
der Z\"ahler. Zum Beispiel definiert
	
	\begin{verbatim}	
	\SetEnumerateShortLabel{i}{\textsc{\roman*}}
	\end{verbatim}
	
|i| in der Art um, dass Items mit kleinen r\"omischen Zahlen nummeriert werden. Der Schl\"ussel muss ein einzelnes 
Zeichen sein.

	\section{Klonen der Basisliste} \label{s.clone}
	
	\begin{desc}
	|\newlist{<name>}{<type>}{<max-depth>}| \\
	|\renewlist{<name>}{<type>}{<max-depth>}|
	\end{desc}
	
Die drei Listen k\"onnen geklont werden, um \glqq logische\grqq{} Umgebungen so zu definieren, dass sie sich 
wie diese verhalten. Um eine neue Liste zu definieren oder eine existierende umzudefinieren, benutzt man |\newlist| 
oder |\renewlist|. Dabei ist |<type>| entweder \textsf{enumerate}, \textsf{itemize} oder \textsf{description}.
	
	\begin{desc}
	|\setlist[<names>,<levels>]{<keys/values>}|
	|\setlist*[<names>,<levels>]{<keys/values>}|
	\end{desc}
	
Nachdem man eine Liste erstellt hat, kann man eine neue Liste mit |\setlist| setzen (eigentlich muss man sogar, 
zumindest das Label):
	
	\begin{verbatim}	
		\newlist{ingredients}{itemize}{1}
		\setlist[ingredients]{label=\textbullet}
		\newlist{steps}{enumerate}{2}
		\setlist[steps,1,2]{label=(\arabic*)}
	\end{verbatim}

Die Namen im optionalen Argument von |\setlist| besagen, auf welche Listen die Einstellungen angewendet werden und die
Nummer dr\"uckt aus, bis zu welcher Ebene. Mehrere Listen und/oder mehrere
Ebenen k\"onnen angegeben werden und alle Kombinationen sind gesetzt; z.\,B. setzt

	\begin{verbatim}
		\setlist[enumerate,itemize,2,3]{...}
	\end{verbatim}	
	
\noindent \textsf{enumerate}/2, \textsf{enumerate}/3, \textsf{itemize}/2 und \textsf{itemize}/3. Keine Nummer (oder 0) 
bedeutet \glqq alle Ebenen\grqq{} und kein Name \glqq alle Listen\grqq{}; kein optionales Argument bedeutet \glqq 
alle Listen auf allen Ebenen\grqq{}.
	
Die drei Inline-Listen haben die Typen \textsf{enumerate*}, \textsf{itemize*} und \textsf{description*}, die immer
verf\"ugbar sind, selbst ohne die Paket-Option |inline| (welche lediglich drei Umgebungen mit diesen Namen definiert).
	
Die gesternte Form \verb|\setlist*| f\"ugt die aktuellen Einstellungen zu den vorher gesetzten hinzu.
	
	\begin{desc}
	|\setlistdepth{<integer>}| \3
	\end{desc}
	
Standardm\"a\ss ig hat \LaTeX{} ein Limit von 5 verschachtelten Ebenen, aber wenn man Listen klont, kann dieser Wert zu 
klein sein, weshalb man einen neuen setzen m\"ochte. In Ebenen unterhalb der 5ten (oder der tiefsten von einer Klasse 
definierten) finden die Einstellungen der letzten Ebene Anwendung (d.\,h.
\verb|\@listvi|).

	
	\section{Mehr zu Z\"ahlern} 
	
	\subsection{Neue Z\"ahlerrepr\"sentation} 


	\begin{desc}
	|\AddEnumerateCounter{<LaTeX command>}{<internal command>}{<widest label>}|
	\end{desc}
	
Dieser Befehl \glqq registriert\grqq{} eine Repr\"asentation eines
Z\"ahlers, so dass \textsf{enumitem} ihn erkennen kann. Er ist haupts\"achlich f\"ur nicht-lateinische Schriften 
gedacht, aber auch f\"ur lateinische Schriften n\"utzlich. Zum Beispiel:
	
	\begin{verbatim}
		\makeatletter
		\def\ctext#1{\expandafter\@ctext\csname c@#1\endcsname}
		\def\@ctext#1{\ifcase#1\or First\or Second\or Third\or
		Fourth\or Fifth\or Sixth\fi}
		\makeatother
		\AddEnumerateCounter{\ctext}{\@ctext}{Second}
	\end{verbatim}
	
Die gesternte Variante erlaubt es, anstatt einer Zeichenkette eine Nummer
als weitestes Label anzugeben; z.\,B. wenn das weiteste Label jenes ist,
welches mit dem Wert 2:\3 korrespondiert: 

	\begin{verbatim}	
		\AddEnumerateCounter*{\ctext}{\@ctmoreext}{2}
	\end{verbatim}
	
Diese Variante sollte bevorzugt werden, wenn die Repr\"asentation keine
blo\ss e Zeichenkette sondern mit einem gewissen Stil versehen ist, z.\,B.
mit Kapit\"alchen. (Der Z\"ahlername kann ein @ enthalten, auch wenn es
kein Buchstabe ist.)
	
	\subsection{\textsf{enumerate} neu starten} 
	
	\begin{desc}
	|\restartlist{<list-name>}| \3
	\end{desc}
	
Momentan kann man mit 
	
	\begin{verbatim}
		\setlist[enumerate]{resume}
	\end{verbatim}
	
eine kontinuierliche Nummerierung innerhalb eines Dokuments erreichen.
Zus\"atzlich wurde eine neuer Befehl eingef\"ugt, um den Z\"ahler mitten im Dokument neu zu starten:
	
	\begin{verbatim}
		\restartlist{<list-name>}
	\end{verbatim}
	
Er basiert lediglich auf dem Listennamen, nicht dem Listentyp, was hei\ss t, dass |enumerate*|, wie es mit der 
Paket-Option |inline| definiert wurde, nicht gleichbedeutend mit |enumerate| ist, da die Namen verschieden sind.

	
	\section{Generische Schl\"ussel und Werte} 
	
	\begin{desc}
	|\SetEnumitemKey{<key>}{<replacement>}| \3
	\end{desc}
	
Mit diesem Befehle k\"onnen eigene Schl\"ussel (ohne Wert) erzeugt werden. Zum Beispiel:
	
	\begin{verbatim}
		\SetEnumitemKey{midsep}{topsep=3pt,partopsep=0pt}
	\end{verbatim}
	
Derartig erzeugte Schl\"ussel k\"onnen wie die anderen verwendet werden.
Ein weiteres Beispiel ist eine \texttt{multicolumn}-Liste, mit \textsf{multicol}:

\begin{verbatim}
\SetEnumitemKey{twocol}{
  itemsep=1\itemsep,
  parsep=1\parsep,
  before=\raggedcolumns\begin{multicols}{2},
  after=\end{multicols}}
\end{verbatim}

(Die Einstellungen f\"ur \texttt{itemsep} und \texttt{parsep} vernichten
die gedehnten und geschrumpften Teile. Nat\"urlich k\"onnen Sie sich
w\"unschen, eine neue Liste zu definieren.)

Ein Hinweis: Das Paket kann in der Zukunft neue Schl\"ussel einf\"uhren, so
ist \verb|\SetEnumitemKey| eine potenzielle Quelle f\"ur
Vorw\"artsinkompatibilit\"aten. Allerdings ist es sicher,
Nicht-Buchstaben-Zeichen zu nehmen, anders als mit Bindestrich oder Stern
im Schl\"usselnamen (beispielsweise \verb|:name| oder \verb|2_col|).


\begin{desc}
|\SetEnumitemValue{<key>}{<string-value>}{<replacement>}|\3
\end{desc}

Dieser Befehl stellt eine weitere Abstraktionsebene f\"ur die Schl\"ussel-Wert-Paare (|<key>=<value>| bereit. Mit 
ihm kann man logische Namen definieren, die dann zum aktuellen Wert \"ubertragen werden. Zum Beispiel kann mit
	
	\begin{verbatim}
		\SetEnumitemValue{label}{numeric}{\arabic*.}
		\SetEnumitemValue{leftmargin}{standard}{\parindent}
	\end{verbatim}
	
Folgendes gesetzt werden:
	
	\begin{verbatim}
		\begin{enumerate}[label=numeric,leftmargin=standard]
	\end{verbatim}
	
Damit l\"asst sich bis zum Ende offenhalten, was |label=numeric| bedeutet.

	\section{Paketoptionen}
	
Neben \verb|inline|, \verb|ignoredisplayed| und \verb|shortlabels| wird noch die folgende Option bereitgestellt:
	
	\begin{desc}
	|loadonly|
	\end{desc}
	
Mit dieser Option wird zwar |enumitem| geladen, die drei Listen aber nicht umdefiniert. Man kann aber trotzdem 
eigene Listen erstellen oder sogar bereits existierende umdefinieren.
		
	\section{Drei Vorlagen}
	
Es gibt gute Gr\"unde daf\"ur anzunehmen, dass drei Listenlayouts besonders
h\"aufig sind. Im Folgenden wenden wir die obigen Parameter auf diese an,
um sie umzudefinieren. (Beispiele finden sich unten.)
	
Die erste Vorlage richtet die Labels nach dem umgebenden |\parindent| aus,
w\"ahrend der Textk\"orper der Items von dem Label und einem fixierten
|labelsep| abh\"angt:
	
	\begin{verbatim}	
		labelindent=\parindent,
		leftmargin=*
	\end{verbatim}
	
Eine ziemlich h\"aufige Variante ist es, das Label am umgebenden Text auszurichten (es ist zu beachten, dass der 
Default von |labelindent| 0pt ist):
	
	\begin{verbatim}
		leftmargin=*	
	\end{verbatim}
	
Ersteres sieht auf der ersten Ebene besser aus, Zweiteres w\"are eventuell auf den untergeordneten Ebenen vorzuziehen. 
Dies kann sehr leicht mittels
	
	\begin{verbatim}
		\setlist{leftmargin=*}
		\setlist[1]{labelindent=\parindent} % Nur Ebene 1
	\end{verbatim}
	
gesetzt werden.
	
Die zweite Vorlage richtet den Textk\"orper der Items an dem umgebenden
|\parindent| aus. In diesem Fall:

	\begin{verbatim}
		leftmargin=\parindent		
	\end{verbatim}	
	
Die dritte Vorlage w\"urde schlie\ss lich das Label an |\parindent| und den
Textk\"orper der Items an |2\parindent| auszurichten:

	\begin{verbatim}
		labelindent=\parindent,
		leftmargin=2\parindent,
		itemsep=*
	\end{verbatim}	

Eine weitere Variante davon ist es, dass man das Label am umgebenden Text
und den Textk\"orper der Items an |\parindent| ausrichtet.

	\begin{verbatim}
		leftmargin=\parindent,
		itemsep=*
	\end{verbatim}	

Anzumerken ist, dass sich |\parindent| an dieser Stelle auf den globalen Wert bezieht, der auf die normalen 
Paragraphen angewendet wird.

	
	\section{Das trivlist-Problem}
	
\LaTeX{} verwendet eine vereinfachte Version von \textsf{list}, die \textsf{trivlist} genannt wird, um angezeigtes 
Material wie |center|, |verbatim|, |tabbing|, |theorem| usw. zu setzen, auch wenn es sich dabei konzeptuell gar nicht um 
Listen handelt. Ungl\"ucklicherweise greift \textsf{trivlist} auf die momentanen Listeneinstellungen zur\"uck, so dass 
es zu dem ungewollten Nebeneffekt kommen kann, dass Ver\"anderungen bei den
vertikalen Abst\"anden zwischen Listen manchmal zur Ver\"anderungen der
Abst\"ande in diesen Umgebungen f\"uhren. 
	
Dieses Paket modifiziert \textsf{trivlist} dahingehend, dass die Default-Einstellungen f\"ur die aktuelle Ebene 
(d.\,h. diejenigen, die durch die zugeh\"origen \textsf{clo}-Dateien bestimmt werden) wiederhergestellt werden. Im 
Standard-\LaTeX{} ist dies f\"ur gew\"ohnlich redundant, aber wenn man
Listen feinabstimmen m\"ochte, kann sich eine Nicht-wiederherstellung der Default-Werte als problematisch erweisen 
(besonders dann, wenn man die Option |nolistsep| verwenden m\"ochte).
	
Ein Minimum an Kontrolle der vertikalen Abst\"ande wird m\"oglich
durch\footnote{\verb|\string\setdisplayed| verliert seinen Wert.}

	\begin{verbatim}	
		\setlist[trivlist,<level>]{<keys/values>}
	\end{verbatim}
	
Aber \textsf{trivlist} selbst, welches eher selten direkt genutzt wird, erlaubt kein optionales Argument. Diese 
Funktion ist nicht als umfassender \textsf{trivlist}-Formatierer gedacht.
	
Wenn man es aus irgendwelchen Gr\"unden nicht m\"ochte, dass die Ver\"anderungen an \textsf{trivlist} in Kraft treten, 
kann dies \"uber die Option |ignoredisplayed| erreicht werden.

	
	\section{Beispiele}

%\expandafter\ifx\csname setenumerate\endcsname\relax

Installieren Sie bitte erst das Paket und setzen Sie das Dokument nochmal.

In diesen Beispielen ist |\setlist{noitemsep}| gesetzt:

\setlist{noitemsep}
\small

\newcommand{\newsample}{\vskip6pt\goodbreak\hrule height 1pt\vskip6pt}
\newcommand{\samplesep}{\vskip6pt\goodbreak\hrule\vskip6pt}
\newbox\vsep
\setbox\vsep\hbox{\vrule height 2ex depth 16ex width 1pt}
\dp\vsep0pt
\newcommand\showsep{\leavevmode\llap{\copy\vsep}}

\newsample

\begin{verbatim}
En un lugar de la Mancha, de cuyo nombre no quiero acordarme,
no ha mucho tiempo que viv\'{\i}a un hidalgo de los de
\begin{enumerate}[labelindent=\parindent,leftmargin=*]
  \item lanza en astillero,
  \item adarna antigua,
  \item roc\'{\i}n flaco, y
  \item galgo corredor.
\end{enumerate}
Una olla de algo m\'{a}s vaca que carnero, salpic\'{o}n las m\'{a}s
noches, duelos y quebrantos los s\'{a}bados...
\end{verbatim}

Diese Regel zeigt \verb|labelindent|. 

\samplesep

\showsep En un lugar de la Mancha, de cuyo nombre no quiero acordarme,
no ha mucho tiempo que viv\'{\i}a un hidalgo de los de
\begin{enumerate}[labelindent=\parindent,leftmargin=*]
\item lanza en astillero,
\item adarna antigua,
\item roc\'{\i}n flaco, y
\item galgo corredor.
\end{enumerate}
Una olla de algo m\'{a}s vaca que carnero, salpic\'{o}n las m\'{a}s
noches, duelos y quebrantos los s\'{a}bados...

\newsample

Mit |\begin{enumerate}[leftmargin=*] % labelindent=0pt by default|. 

Diese Regel zeigt \verb|labelindent|.

\samplesep

\noindent\showsep\hskip\parindent En un lugar de la Mancha, de cuyo nombre no quiero acordarme,
no ha mucho tiempo que viv\'{\i}a un hidalgo de los de
\begin{enumerate}[leftmargin=*]
\item lanza en astillero,
\item adarna antigua,
\item roc\'{\i}n flaco, y
\item galgo corredor.
\end{enumerate}
Una olla de algo m\'{a}s vaca que carnero, salpic\'{o}n las m\'{a}s
noches, duelos y quebrantos los s\'{a}bados...

\newsample

Mit |\begin{enumerate}[leftmargin=\parindent]|.

Diese Regel zeigt \verb|leftmargin|.

\samplesep

\showsep En un lugar de la Mancha, de cuyo nombre no quiero acordarme,
no ha mucho tiempo que viv\'{\i}a un hidalgo de los de
\begin{enumerate}[leftmargin=\parindent]
\item lanza en astillero,
\item adarna antigua,
\item roc\'{\i}n flaco, y
\item galgo corredor.
\end{enumerate}
Una olla de algo m\'{a}s vaca que carnero, salpic\'{o}n las m\'{a}s
noches, duelos y quebrantos los s\'{a}bados...

\newsample

Mit |\begin{enumerate}[labelindent=\parindent,|\allowbreak
| leftmargin=*,|\allowbreak| label=\Roman*.,|\allowbreak
| widest=IV,|\allowbreak| align=left]|.

Diese Regel zeigt \verb|labelindent|.

\samplesep

\showsep En un lugar de la Mancha, de cuyo nombre no quiero acordarme,
no ha mucho tiempo que viv\'{\i}a un hidalgo de los de
\begin{enumerate}[labelindent=\parindent, leftmargin=*,
                  label=\Roman*., widest=IV, align=left]
\item lanza en astillero,
\item adarna antigua,
\item roc\'{\i}n flaco, y
\item galgo corredor.
\end{enumerate}
Una olla de algo m\'{a}s vaca que carnero, salpic\'{o}n las m\'{a}s
noches, duelos y quebrantos los s\'{a}bados...

\newsample

Mit |\begin{enumerate}[label=\fbox{\arabic*}]|. Ein Verweis auf das erste Item ist \ref{i:first}

\samplesep

En un lugar de la Mancha, de cuyo nombre no quiero acordarme,
no ha mucho tiempo que viv\'{\i}a un hidalgo de los de
\begin{enumerate}[label=\fbox{\arabic*}]
\item \label{i:first}lanza en astillero,
\item adarna antigua,
\item roc\'{\i}n flaco, y
\item galgo corredor.
\end{enumerate}
Una olla de algo m\'{a}s vaca que carnero, salpic\'{o}n las m\'{a}s
noches, duelos y quebrantos los s\'{a}bados...

\newsample

Mit verschachtelten Listen.

\samplesep

\begin{verbatim}
En un lugar de la Mancha, de cuyo nombre no quiero acordarme,
no ha mucho tiempo que viv\'{\i}a un hidalgo de los de
\begin{enumerate}[label=(\alph*), labelindent=\parindent,
     leftmargin=*, start=12]
\item lanza en astillero,
\begin{enumerate}[label=(\alph{enumi}.\roman*), leftmargin=*, start=7]
\item adarna antigua,
\end{enumerate}
\item roc\'{\i}n flaco, y
\begin{enumerate}[label=(\alph{enumi}.\roman*), leftmargin=*, resume]
\item galgo corredor.
\end{enumerate}
\end{enumerate}
Una olla de algo m\'{a}s vaca que carnero, salpic\'{o}n las m\'{a}s
noches, duelos y quebrantos los s\'{a}bados...
\end{verbatim}

En un lugar de la Mancha, de cuyo nombre no quiero acordarme,
no ha mucho tiempo que viv\'{\i}a un hidalgo de los de
\begin{enumerate}[label=(\alph*), labelindent=\parindent,
     leftmargin=*, start=12]
\item lanza en astillero,
\begin{enumerate}[label=(\alph{enumi}.\roman*), leftmargin=*, start=7]
\item adarna antigua,
\end{enumerate}
\item roc\'{\i}n flaco, y
\begin{enumerate}[label=(\alph{enumi}.\roman*), leftmargin=*, resume]
\item galgo corredor.
\end{enumerate}
\end{enumerate}
Una olla de algo m\'{a}s vaca que carnero, salpic\'{o}n las m\'{a}s
noches, duelos y quebrantos los s\'{a}bados...

\newsample

\begin{verbatim}
En un lugar de la Mancha, de cuyo nombre no quiero acordarme,
no ha mucho tiempo que viv\'{\i}a un hidalgo de los de
\begin{description}[font=\sffamily\bfseries, leftmargin=3cm,
    style=nextline]
  \item[Lo primero que ten\'{\i}a el Quijote] lanza en astillero,
  \item[Lo segundo] adarna antigua,
  \item[Lo tercero] roc\'{\i}n flaco, y
  \item[Y por \'{u}ltimo, lo cuarto] galgo corredor.
\end{description}
Una olla de algo m\'{a}s vaca que carnero, salpic\'{o}n las m\'{a}s
noches, duelos y quebrantos los s\'{a}bados...
\end{verbatim}

\samplesep

En un lugar de la Mancha, de cuyo nombre no quiero acordarme,
no ha mucho tiempo que viv\'{\i}a un hidalgo de los de
\begin{description}[font=\sffamily\bfseries, leftmargin=3cm,
    style=nextline]
\item[Lo primero que ten\'{\i}a el Quijote] lanza en astillero,
\item[Lo segundo] adarna antigua,
\item[Lo tercero] roc\'{\i}n flaco, y
\item[Y por \'{u}ltimo, lo cuarto] galgo corredor.
\end{description}
Una olla de algo m\'{a}s vaca que carnero, salpic\'{o}n las m\'{a}s
noches, duelos y quebrantos los s\'{a}bados...

\newsample

Genauso, nur mit |sameline|.

\samplesep

En un lugar de la Mancha, de cuyo nombre no quiero acordarme,
no ha mucho tiempo que viv\'{\i}a un hidalgo de los de
\begin{description}[font=\sffamily\bfseries, leftmargin=3cm,
    style=sameline]
\item[Lo primero que ten\'{\i}a el Quijote] lanza en astillero,
\item[Lo segundo] adarna antigua,
\item[Lo tercero] roc\'{\i}n flaco, y
\item[Y por \'{u}ltimo, lo cuarto] galgo corredor.
\end{description}
Una olla de algo m\'{a}s vaca que carnero, salpic\'{o}n las m\'{a}s
noches, duelos y quebrantos los s\'{a}bados...

\newsample

Genauso, aber mit |multiline|. Es ist zu beachten, dass sich der Text
\"uberlappt, wenn der Textk\"orper f\"ur die Items zu kurz ist.

\samplesep

En un lugar de la Mancha, de cuyo nombre no quiero acordarme,
no ha mucho tiempo que viv\'{\i}a un hidalgo de los de
\begin{description}[font=\sffamily\bfseries, leftmargin=2cm,
    style=multiline]
\item[Lo primero que ten\'{\i}a el Quijote] lanza en astillero,
\item[Lo segundo] adarna antigua,
\item[Lo tercero] roc\'{\i}n flaco, y
\item[Y por \'{u}ltimo, lo cuar] galgo corredor.
\end{description}

Una olla de algo m\'{a}s vaca que carnero, salpic\'{o}n las m\'{a}s
noches, duelos y quebrantos los s\'{a}bados...

%\fi

\normalsize

	\section{Nachwort}

	\subsection{\LaTeX{}-Listen}

Wie bereits bekannt ist, gibt es in \LaTeX{} drei vordefinierte Listen: \textsf{enumerate}, \textsf{itemize} und 
\textsf{description}. Dabei handelt es sich um eine g\"angige Klassifikation, die so beispielsweise auch in HTML zu 
finden ist. Es existiert aber noch ein allgemeineres Modell, welches sich
in drei Felder einteilen l\"asst: Label, Titel
und K\"orper. So haben \textsf{enumerate} und \textsf{itemize} zwar Label (nummeriert und unnummeriert), aber keine 
Titel, w\"ahrend \textsf{description} Titel hat, aber keine Label. In diesem Modell kann man eine 
\textsf{description}-Liste haben, deren Eintr\"age mit Labels markiert sind, wie zum Beispiel (nat\"urlich ist diese 
einfache L\"osung nicht zufriedenstellend):

	\begin{verbatim}
		\newcommand\litem[1]{\item{\bfseries #1,\enspace}}
		\begin{itemize}[label=\textbullet]
		\litem{Lo primero que ten\'{\i}a el Quijote} lanza en astillero,
		... etc.
	\end{verbatim}

\newpage

\vskip6pt
\goodbreak
\hrule
\vskip6pt

\newcommand\litem[1]{\item{\bfseries #1,\enspace}}
En un lugar de la Mancha, de cuyo nombre no quiero acordarme,
no ha mucho tiempo que viv\'{\i}a un hidalgo de los de
\begin{itemize}[label=\textbullet]
\litem{Lo primero que ten\'{\i}a el Quijote} lanza en astillero,
\litem{Lo segundo} adarna antigua,
\litem{Lo tercero} roc\'{\i}n flaco, y
\litem{Y por \'{u}ltimo, lo cuarto} galgo corredor.
\end{itemize}

\vskip6pt
\goodbreak
\hrule
\vskip6pt

Diese Formatierung ist nicht selten und ein Werkzeug, um es zu definieren, ist in Arbeit und schon sehr weit 
fortgeschritten. Es ist in Version 3.0 noch nicht enthalten, da ich mir nicht sicher bin, ob dieses Paket der 
richtige Platz daf\"ur ist oder ob nicht doch \textsf{titesec} besser
geeignet w\"are; au\ss erdem arbeitet es noch nicht stabil genug.

	
	\subsection{Bekannte Probleme}
	
\begin{itemize}
\item Die Wiederaufnahme von Listen basiert auf Umgebungsnamen. Wenn eine
|\newenviroment| eine Liste enth\"alt, m\"ochte man vielleicht |\begin| und |\end| verwenden. Die Verwendung der 
korrespondieren Befehle ist kein Fehler, aber man ist selbst daf\"ur verantwortlich, dass das Resultat korrekt ist.

\item Es scheint keinen Weg zu geben, falsch geschriebene Namen in |\setlist| aufzufangen und der Compiler gibt die 
sinnlose Fehlermeldung \glqq Missing number, treated as zero\grqq{} aus.

\item Das Verhalten der gleichzeitigen Anwendung von gerahmten Labels
(einschlie\ss lich \textsf{enumerate} und \textsf{itemize}) und ungerahmten Labels ist nicht wohldefiniert. 
\"Ahnlich verh\"alt es sich mit der gleichzeitigen Weiterf\"uhrung von
Serien und Listen: Es ist zwar m\"oglich, aber ihr Verhalten ist ebenfalls nicht wohldefiniert.

\item (3.5.2) Eine Inkompatibilit\"at mit 2.x ist aufgetaucht -- wenn ein
optionales Argument genommen wurde, um den Wert von
\verb|\ref| zu \"ubertreffen, oder andere Makros erforderten expandierende
Makros -- dann wird ein Fehler ausgel\"ost. Eine schnelle L\"osung ist das
\verb|\makelabel| nach |\descriptionlabel| zu lassen.

\end{itemize}
	
	\subsection{Neuerungen in Version 3.0}
	
\begin{itemize}
\item  Inline-Listen mit Schl\"usseln, um zu bestimmen, wie die Items
verbunden werden (d.\,h. die Zeichensetzung zwischen ihnen). Es sind zwei
Modi erlaubt: \verb|boxed| und \verb|unboxed|.

\item \verb|\setlist| ist \textsf{calc}-savvy (z.\,B. zur Benutzung in Schleifen) und es lassen sich unterschiedliche 
Listen und Ebenen auf einmal festlegen.

\item Alle L\"angen, die mit Labels zu tun haben, k\"onnen den Wert
\verb|*| annehmen, nicht nur \verb|labelsep| und \verb|leftmargin|. Ihr Verhalten wurde vereinheitlicht und es gibt einen 
neuen Wert \verb|!|, mit dem das weiteste Label nicht verarbeitet wird.

\item Mit \verb|\restartlist{<list-name>}| k\"onnen Listenz\"ahler neu
gestartet werden (falls \verb|resume| verwendet wird).

\item \verb|resume*| kann zusammen mit anderen Schl\"usseln verwendet werden.

\item Listen k\"onnen global durch die Verwendung von Serien zusammengefasst werden, so dass sie wie eine einzige 
Liste behandelt werden. Um eine Serie zu starten, verwendet man einfach nur |series=<series-name>|, die Wiederaufnahme 
erfolgt durch |resume=<series-name>| oder |resume*=<series-name>|.

\item Das \glqq experimentelle\grqq{} \verb|fullwidth| wurde durch den
neuen Schl\"ussel \verb|wide| ersetzt.

\item |\SetLabelAlign| definiert neue Werte zur Ausrichtung.

\item Es lassen sich \glqq abstrakte\grqq{} Werte (z.\,B. \verb|label=numeric|) und neue Schl\"ussel definieren.
\end{itemize}

\begin{itemize}
\item (3.2) \verb|start| und \verb|widest*| sind \textsf{calc}-savvy.

\item (3.2) \verb|\value| kann zusammen mit \verb|widest*| verwendet werden.

\item (3.2) Eine interne Einschr\"ankungen in \verb|\arabic| und dergleichen wurden entfernt. Es ist flexibler zu Lasten 
einer \glqq entspannteren\grqq{} Fehlersuche.

\end{itemize}
	
	\subsection{Bug-Fixes}
	
\begin{itemize}
\item Gesternte Werte (z.\,B. \verb|leftmargin=*|) konnten nicht au\ss er Kraft gesetzt werden und neue Werte wurden
ignoriert.

\item \verb|nolistsep| wurde als erster von mehreren Schl\"usseln nicht immer erkannt und als Kurzlabel behandelt 
(d.\,h. \verb|nol\roman*stsep|).

\item \verb|labelwidth| funktionierte nicht immer (wenn es vorher ein
\verb|widest| und \verb|*| gab).

\item Mit \verb|align=right| konnten sich Label und nachfolgender Text \"uberlappen.

\item In \textsf{description} gab es Probleme mit der korrekten Listenebene.

\item Ab einem bestimmten Punkt (Version 2.x) h\"orte \verb|value*| auf, richtig zu funktionieren.

\item (3.1) Ungl\"ucklicherweise \glqq vernichtet\grqq{} |xkeyval|
|keyval|, so dass Letzteres in \textsf{enumitem} nachgebildet werden musste.

\item (3.3) Ein schwerwiegender Fehler wurde entfernt: Weder \textsf{itemize} noch \textsf{description} funktionierten 
in ihrer gesternten Variante.

\item (3.4) Ein schlechter Abstand wurde entfernt (fehl platziert
\verb|\unskip| vor dem ersten Element und ein falscher
\textsl{space}-Faktor zwischen Elementen).

\item (3.4) \verb|nolistsep| funktioniert nicht, wie vorgesehen, wurde vor
mehreren Jahren festgestellt, deshalb wurde ein neuer Schl\"ussel
\verb|nosep| zur Verf\"ugung gestellt.

\item (3.4) Das Problem von \verb|nolistsep| mit \verb|shortlabels| (siehe
oben) war nicht f\"ur alle F\"alle behoben. Hoffentlich ist es jetzt so.

\item (3.5.0) Es wurde ein Problem mit dem Abstandsfaktor zwischen Items
behoben.

\item (3.5.0) Es wurde ein Problem mit verschachtelten Boxen in
Inline-Listen beseitigt.

\item (3.5.1) \texttt{resume*} arbeitete nur einmal und die folgenden
verhielten sich nur wie \texttt{resume}.

\item (3.5.2) Beseitigte Fehler mit |\setlist*|, weil es nicht
funktionierte.

\end{itemize}

\end{document}
	
	\subsection{Danksagung}
	
Ich danke insbesonders Lars Madsen f\"ur seine Kommentare, Vorschl\"age und Fehlermeldungen. 
	
\end{document}
