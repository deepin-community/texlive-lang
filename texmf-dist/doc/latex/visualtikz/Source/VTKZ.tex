

  
\author{{\Huge Jean Pierre Casteleyn } \\ {\Huge IUT Génie Thermique et \'Energie } \\ {\Huge Dunkerque, France }}

\DeclareFixedFont{\RM}{T1}{ptm}{b}{n}{2cm}

\DeclareFixedFont{\RMM}{T1}{ptm}{b}{n}{1cm}

\title{ {\RM Visual TikZ} \\ \vspace{1cm} {\RMM Version 0.66} }



\date{
\begin{center}
\begin{animateinline}[loop,autoplay]{12}%
 \multiframe{24}{iAngle=0+15,icol=0+5}{\begin{tikzpicture}[rotate=90]
    \draw  (0,0) node[fill=white,circle] {\includegraphics[width=4cm]{LogoIUT}}  (0,0) circle (1);
  \end{tikzpicture}} 
\end{animateinline}% 
\end{center}
{\LARGE \TFRGB{mis à jour le \today}{Updated on \today} 
}
}


\maketitle



 \begin{animateinline}[autoplay,loop]{12}%
 \multiframe{24}{iAngle=0+15,icol=0+5}{\begin{tikzpicture}
 [scale=1.8] %
   \draw[line width=0pt] (-2,-2) rectangle(6,2); %
   \draw  (0,0) node[fill=white,circle,rotate=\iAngle] {\includegraphics[width=2cm]{LogoIUT}}  (0,0) circle (1);
    \draw (0,0) circle (1);
    \coordinate (abc) at (${sqrt(9-sin(\iAngle)*sin(\iAngle))+cos(\iAngle)}*(1,0)$) ;
    \coordinate (xyz) at (\iAngle:1);
    \draw[ultra thick] (0,0) --(xyz); 
    \draw[ultra thick] (xyz) -- (abc) ;
    \fill[color=blue!\icol] (abc)++(0.5,-1) rectangle (5,1) ;
    \draw[ultra thick] (abc) ++(0,-1) rectangle ++(.5,2) ;
    \draw[ultra thick]  (1.5,1) -- (5,1) -- (5,-1) -- (1.5,-1);
    \fill[red] (xyz) circle (4pt);
    \fill[red] (abc) circle (4pt); 
  \end{tikzpicture}}
 \end{animateinline} 


 
\newpage
 
 
\TFRGB{
\textbf{Objectifs }: 

\begin{itemize}
\item Avoir une image par  commande ou par paramètre.
\item Avoir un texte réduit au strict minimum.
\item Etre le plus complet possible au fil de mises à jour régulières.
\item Garder la même structure que visuel pstricks
\end{itemize} 
}
{\textbf{Objectives }: 

\begin{itemize}
\item One image per command or parameter.
\item the minimum amount of text possible.
\item the most complete possible update after update.
\item keep the same structure as VisualPSTricks
\end{itemize}}


\vspace{1cm}

\TFRGB{
\textbf{Remarques }: Le code donné est minimal et ne sert qu'à montrer les commandes concernées. Les effets sont parfois exagérés pour bien les mettre en évidence. Pour en savoir plus, vous pouvez voir la documentation. Pour se faire j'ai indiqué le numéro de \tikz[baseline=-1mm]  \draw node[draw,fill=red!20] {Section de pgfmanual} ;
}
{\textbf{Remarks }:
Minimal code is given to show the effect of a command or a parameter. The effects are sometime exaggerated for clarity   .To consult the documentation, I have given the number of the  \tikz[baseline=-1mm]  \draw node[draw,fill=red!20] {Section in pgfmanual} ;
}
\vspace{1cm}


\TFRGB{
\textbf{Vous pouvez me contacter à}
 \href{mailto:jpcdk@yahoo.fr}{mon e-mail personnel} pour

\begin{itemize}
\item me signaler les erreurs que vous avez constatés (merci d'indiquer la page où vous l'avez constaté)
\item me faire part de vos commentaires, suggestions \dots
\end{itemize}}
{
\textbf{You can contact me at }
 \href{mailto:jpcdk@yahoo.fr}{my personal email} to

\begin{itemize}
\item let me know the mistakes found (please indicate the page)
\item give me your commentaries, your suggestions \dots
\end{itemize}}

\vspace{1cm}
\TFRGB{
\textbf{Quoi de neuf ! } :

\begin{itemize}
\item Ajout de la library  chains \pageref{lib-chains}
\item Ajout de la library  through \pageref{lib-through}
\item Ajout de la library  turtle \pageref{lib-turtle}
\item Ajout de la library positioning \pageref{lib-pos}
\item Ajout du module tikzsymbols \pageref{symbol}
\item mise à jour du module tikzducks \pageref{ducks}
\item mise à jour des modules shape \pageref{formes}
\end{itemize}

}
{
\textbf{What's new } :
\begin{itemize}
\item chains library added \pageref{lib-chains}
\item through library added \pageref{lib-through}
\item turtle library added \pageref{lib-turtle}
\item positioning library added \pageref{lib-pos}
\item Tikzsymbols package added \pageref{symbol}
\item Tikzducks package updated \pageref{ducks}
\item shapes packages updated \pageref{formes}
\end{itemize}
}



\vspace{1cm}
\textbf{Licence } :


This work may be distributed and/or modified under the conditions of the LaTeX Project Public License, either version 1.3 of this license or (at your option) any later version.

 The latest version of this license is in  http://www.latex-project.org/lppl.txt and version 1.3 or later is part of all distributions of LaTeX
version 2005/12/01 or later.

This work has the LPPL maintenance status `maintained'.

The Current Maintainer of this work is M. Jean Pierre Casteleyn.

\vspace{2cm}
\textbf{\TFRGB{Merci à }{Thanks to}}:

Till Tantau  ,
Alain Matthes ,
Jim Diamond ,
Falk Rühl ,
Axel Kielhorn ,
Nils Fleischhacker ,
Michel Fruchart ,
Ben Vitecek
 
\newpage



\setcounter{tocdepth}{4}
 \tableofcontents
 \setcounter{tocdepth}{5}
\addtolength{\hoffset}{-1.5cm} 
\setlength{\parindent}{0pt}
\setlength{\topmargin}{0pt}
\setlength{\headsep}{0pt}

 \newpage


\SSCT{Chargement de TikZ}{Tikz loading}

\maboite{\BS{usepackage}\AC{tikz} }

\SSCT{Les figures de base}{Basic figures}



\begin{center}
\begin{tabular}{|c|c|c|} \hline
\BS{draw}  (0,0) {\color{red}-  -} (2,1) ; \RRR{14-2} & 
\BS{draw}  (0,0){\color{red} -|} (2,1) ;
  &
\BS{draw} (0,0) {\color{red} |-} (2,1) ; 
\\ \hline
\begin{tikzpicture}[blue,line width=2pt]
\draw[help lines] (0,0) grid (3,2); 
\draw  (0,0) -- (2,1) ; 
\end{tikzpicture}
&  
\begin{tikzpicture}[blue,line width=2pt]
\draw[help lines] (0,0) grid (3,2); 
\draw  (0,0) -| (2,1) ; 
\end{tikzpicture}
&
\begin{tikzpicture}[,blue,line width=2pt]
\draw[help lines] (0,0) grid (3,2); 
\draw  (0,0) |- (2,1) ; 
\end{tikzpicture}

\\ \hline 
\end{tabular} 
\end{center}
\bigskip
 
\noindent \begin{tabular}{|c|c|c|}\hline
 \multicolumn{3}{|c|}{\BS{draw} (0,2) . . \DDD{controls} (3,0)  .. (2,2); \RRR{ 14-3 }}
 \\  \hline 	 
  \begin{tikzpicture}[,blue,line width=2pt,fill=green]
  \draw[help lines] (-1,0) grid (3,3); 
\draw (0,2) .. controls (3,0)  .. (2,2);
  \end{tikzpicture}  
  &
  \begin{tikzpicture}[blue,line width=2pt,fill=green]
  \draw[help lines] (-1,0) grid (3,3); 
\fill (0,2) .. controls (3,0)  .. (2,2);
  \end{tikzpicture} 
  &
  \begin{tikzpicture}[blue,line width=2pt,fill=green]
  \draw[help lines] (-1,0) grid (3,3); 
\filldraw (0,2) .. controls (3,0)  .. (2,2);
  \end{tikzpicture} 
  \\ \hline 
\BSS{draw} 
&  
\BSS{fill} 
&  
\BSS{filldraw} 
\\ \hline
 \end{tabular}
 
 
 \bigskip

\noindent
 \begin{tabular}{|c|c|c|}\hline
 \multicolumn{3}{|c|}{\BS{draw} (0,2) . . \DDD{controls} (3,0) \DDD{and} (-1,0) .. (2,2); \RRR{14-3}}
 \\  \hline 	 
  \begin{tikzpicture}[blue,line width=2pt,fill=green]
  \draw[help lines] (-1,0) grid (3,3); 
\draw (0,2) .. controls (3,0) and (-1,0) .. (2,2);
  \end{tikzpicture}  
  &
  \begin{tikzpicture}[blue,line width=2pt,fill=green]
  \draw[help lines] (-1,0) grid (3,3); 
\fill (0,2) .. controls (3,0) and (-1,0) .. (2,2);
  \end{tikzpicture} 
  &
  \begin{tikzpicture}[blue,line width=2pt,fill=green]
  \draw[help lines] (-1,0) grid (3,3); 
\filldraw (0,2) .. controls (3,0) and (-1,0) .. (2,2);
  \end{tikzpicture} 
  \\ \hline 
\BSS{draw} 
&  
\BSS{fill} 
&  
\BSS{filldraw} 
\\ \hline
 \end{tabular}


\bigskip

\noindent 
\begin{tabular}{|c|c|c|} \hline
\multicolumn{3}{|c|}{\BS{draw} (0,0) \DDD{rectangle} (3,2); \RRR{14-4} }\\ 
\hline 
\begin{tikzpicture}[scale=.8,blue,line width=2pt]
\draw[help lines] (-1,-1) grid (4,3); 
\draw (0,0) rectangle (3,2);
\end{tikzpicture}
&  
\begin{tikzpicture}[scale=.8,blue,line width=2pt,fill=green]
\draw[help lines] (-1,-1) grid (4,3); 
\fill (0,0) rectangle (3,2);
\end{tikzpicture}
&  
\begin{tikzpicture}[scale=.8,blue,line width=2pt,fill=green]
\draw[help lines] (-1,-1) grid (4,3); 
\filldraw (0,0) rectangle (3,2);
\end{tikzpicture}
\\ \hline  
\BSS{draw} 
&  
\BSS{fill} 
&  
\BSS{filldraw} 
\\ \hline

\end{tabular} 

 
 \bigskip
  
\noindent \begin{tabular}{|c|c|c|}\hline
 \multicolumn{3}{|c|}{\BS{draw} (1,1) \DDD{circle} (1); \RRR{14-6}}\\ 
 \hline 
\begin{tikzpicture}[blue,line width=2pt,fill=green]
\draw[help lines] (0,0) grid (2,2); 
\draw (1,1) circle (1);	
\end{tikzpicture}  
&
\begin{tikzpicture}[blue,line width=2pt,fill=green]
\draw[help lines] (0,0) grid (2,2); 
\fill (1,1) circle (1);	
\end{tikzpicture} 
&
\begin{tikzpicture}[,blue,line width=2pt,fill=green]
\draw[help lines] (0,0) grid (2,2); 
\filldraw (1,1) circle (1);	
\end{tikzpicture} 
  \\ \hline 
  \BS{draw} 
  &  
  \BS{fill} 
  &  
  \BS{filldraw} 
  \\ \hline
\end{tabular} 


 \bigskip
  
\noindent \begin{tabular}{|c|c|c|}\hline
 \multicolumn{2}{|c|}{\BS{draw} (1,1) \DDD{circle} [\RDD{radius}=1cm]; } & 
 \BS{draw} (1,1)\DDD{ellipse} [\RDD{x radius}=2cm,\RDD{y radius}=1cm]
\\  \hline 
\begin{tikzpicture}[blue,line width=2pt,fill=green]
\draw[help lines] (0,0) grid (2,2); 
  \draw (1,1) circle [radius=1cm] ;	
  \end{tikzpicture}  
  &
  \begin{tikzpicture}[,blue,line width=2pt,fill=green]
  \draw[help lines] (-1,0) grid (3,2); 
  \draw (1,1) circle[x radius=2cm,y radius=1cm] ;	
  \end{tikzpicture} 
  &
   \begin{tikzpicture}[blue,line width=2pt,fill=green]
   \draw[help lines] (-1,0) grid (3,2); 
 \draw  (1,1) ellipse [x radius=2cm,y radius=1cm];	
   \end{tikzpicture}  
  \\ \hline 
\RDD{radius}=1cm & \RDD{x radius}=2cm,\RDD{y radius}=1cm
  \\ \hline 
\end{tabular} 

 \bigskip
  
\noindent \begin{tabular}{|c|c|}\hline
\BS{draw}  (1,1) circle {\color{red}(2 and 1)}; & \BS{draw}  (1,1) ellipse {\color{red}(2 and 1)};  
 \\ \hline 	
  \begin{tikzpicture}[scale=.8,blue,line width=2pt,fill=green]
 \draw[help lines] (-1,0) grid (3,2); 
\draw  (1,1) circle(2 and 1);
  \end{tikzpicture}  
  &	
  \begin{tikzpicture}[scale=.8,blue,line width=2pt,fill=green]
 \draw[help lines] (-1,0) grid (3,2); 
\draw  (1,1) ellipse (2 and 1);
  \end{tikzpicture}  
\\ \hline 
  \end{tabular}

\bigskip

\noindent \begin{tabular}{|c|c|c|}\hline 
\multicolumn{3}{|c|}{\BS{draw} (-2,0) \DDD{arc} (180:-45:2); \RRR{14-7} }\\ 
\hline  	
\begin{tikzpicture}[scale=.8,blue,line width=2pt,fill=green]
\draw[help lines] (-3,-2) grid (3,2);
\draw (-2,0) arc (180:-45:2); 	
\end{tikzpicture} 
&
\begin{tikzpicture}[scale=.8,blue,line width=2pt,fill=green]
\draw[help lines] (-3,-2) grid (3,2);
\fill (-2,0) arc (180:-45:2); 
\end{tikzpicture} 
&
\begin{tikzpicture}[scale=.8,blue,line width=2pt,fill=green]
\draw[help lines] (-3,-2) grid (3,2);
\filldraw (-2,0) arc (180:-45:2); 	
\end{tikzpicture}
\\ \hline 
\BS{draw} 
&  
\BS{fill} 
&  
\BS{filldraw} 
\\ \hline
 \end{tabular} 
 \bigskip
 
\noindent \begin{tabular}{|c|c||c|}  \hline 
\multicolumn{2}{|c||}{\BS{draw} (-2,0) arc [\RDD{start angle}=180, \RDD{end angle}=-45,\RDD{radius}=1] }
& \BS{draw} (-2,0)  \RDD{arc} (180:-45:2 and 1)
\\ \hline  	
  
\begin{tikzpicture}[scale=.8,blue,line width=2pt,fill=green]
\draw[help lines] (-3,-2) grid (3,2);
\draw (-2,0) arc [start angle=180, end angle=-45,radius=2]; 	
\end{tikzpicture}
 &  
\begin{tikzpicture}[scale=.8,blue,line width=2pt,fill=green]
\draw[help lines] (-3,-2) grid (3,2);
\draw (-2,0) arc [start angle=180, end angle=-45,x radius=2,y radius=1]; 	
\end{tikzpicture}
&
  \begin{tikzpicture}[scale=.8,blue,line width=2pt,fill=green]
\draw[help lines] (-3,-2) grid (3,2);
\draw (-2,0)  arc (180:-45:2 and 1); 
\end{tikzpicture}
 \\  \hline 
 \RDD{radius}=1 & \RDD{x radius}=1,\RDD{y radius}=.5 & \\ 
 \hline 
 \end{tabular} 

\bigskip
 
\noindent \begin{tabular}{|c|c|c|}\hline  
 \multicolumn{3}{|c|}{\BS{draw} (0,0) \DDD{parabola} (3,2);   \RRR{14-9} }
 \\ \hline 	
   \begin{tikzpicture}[blue,line width=2pt,fill=green]
   \draw[help lines] (0,0) grid (3,2); 
 \draw (0,0) parabola (3,2); 
   \end{tikzpicture}  
   &
   \begin{tikzpicture}[blue,line width=2pt,fill=green]
   \draw[help lines] (0,0) grid (3,2); 
 \fill (0,0) parabola (3,2); 
   \end{tikzpicture} 
   &
   \begin{tikzpicture}[blue,line width=2pt,fill=green]
   \draw[help lines] (0,0) grid (3,2); 
 \filldraw (0,0) parabola (3,2); 
   \end{tikzpicture} 
   \\ \hline 
\BS{draw} 
&  
\BS{fill} 
&  
\BS{filldraw} 
\\ \hline
   \end{tabular} ------
 
\noindent \begin{tabular}{|c|c|} \hline  
 \begin{tikzpicture}[blue,line width=2pt,fill=green]
 \draw[help lines] (0,0) grid (4,4); 
 \draw(0,1) parabola bend (1,0) (4,4); 
  \end{tikzpicture}
 & 
\begin{tikzpicture}[blue,line width=2pt,fill=green]
\draw[help lines] (0,0) grid (4,4); 
\draw (0,0) parabola[bend pos=0.25] (4,4);
 \end{tikzpicture} \\ 
 \hline 
  \BS{draw}(0,1) parabola \RDD{bend} (1,0) (4,4); & 
  \BS{draw}(0,0) parabola[\RDD{bend pos}=0.25] (4,4); 
  \\  \hline 
 \end{tabular} 
\bigskip

\noindent \begin{tabular}{|c|c|c|} \hline 
 \BS{draw}(0,1) parabola [\RDD{parabola height}=2cm] (3,0);  & \multicolumn{2}{|c|}{\BS{draw}(0,0) parabola[\RDD{bend at start}] (3,2);}\\ 
 \hline 
\begin{tikzpicture}[blue,line width=2pt,fill=green]
\draw[help lines] (0,0) grid (3,3); 
\draw (0,1) parabola[parabola height=2cm] (3,0);
 \end{tikzpicture}  & 
\begin{tikzpicture}[blue,line width=2pt,fill=green]
\draw[help lines] (0,0) grid (3,2); 
\draw (0,0) parabola[bend at start] (3,2);
 \end{tikzpicture}
&  
\begin{tikzpicture}[blue,line width=2pt,fill=green]
\draw[help lines] (0,0) grid (3,2); 
\draw (0,0) parabola[bend at end] (3,2);
 \end{tikzpicture}
\\ \hline &[\RDD{bend at start}]  & [\RDD{bend at end}] \\ 
\hline 
\end{tabular}  

\bigskip
 
\noindent \begin{tabular}{|c|c|c|}\hline 
 \multicolumn{3}{|c|}{\BS{draw} (0,0) \DDD{sin}  (1.57,2);  \RRR{14-10} }\\ 
 \hline 	
  \begin{tikzpicture}[scale=.8,blue,line width=2pt,fill=green]
  \draw[help lines] (0,0) grid (3,2); 
\draw  (0,0) sin (1.57,2); 
  \end{tikzpicture}  
  &
  \begin{tikzpicture}[scale=.8,blue,line width=2pt,fill=green]
  \draw[help lines] (0,0) grid (3,2); 
\fill  (0,0) sin (1.57,2); 
  \end{tikzpicture} 
  &
  \begin{tikzpicture}[scale=.8,blue,line width=2pt,fill=green]
  \draw[help lines] (0,0) grid (3,2); 
\filldraw (0,0) sin (1.57,2); 
  \end{tikzpicture} 
  \\ \hline 
 \BS{draw} 
 &  
 \BS{fill} 
 &  
 \BS{filldraw} 
 \\ \hline 	


  \begin{tikzpicture}[scale=.8,blue,line width=2pt,fill=green]
  \draw[help lines] (0,0) grid (3,2); 
\draw  (0,0) cos (1.57,2); 
  \end{tikzpicture}  
  &
  \begin{tikzpicture}[scale=.8,blue,line width=2pt,fill=green]
  \draw[help lines] (0,0) grid (3,2); 
\fill  (0,0) cos (1.57,2); 
  \end{tikzpicture} 
  &
  \begin{tikzpicture}[scale=.8,blue,line width=2pt,fill=green]
  \draw[help lines] (0,0) grid (3,2); 
\filldraw (0,0) cos (1.57,2); 
  \end{tikzpicture} 
  \\ \hline 
\multicolumn{3}{|c|}{\BS{draw} (0,0) \DDD{cos}  (1.57,2);  }\\ 
\hline 	 	
 \end{tabular} 

\bigskip
  
\begin{center}
\RRR{14-13}
\end{center}
  
\noindent \begin{tabular}{|c|c|c|} \hline  
\begin{tikzpicture}[blue,line width=2pt,fill=green]
\draw[help lines] (0,0) grid (3,2); 
\draw  (0,0) to (3,2); 
\end{tikzpicture} 
&  
\begin{tikzpicture}[blue,line width=2pt,fill=green]
\draw[help lines] (0,0) grid (3,2); 
\draw[out=0]  (0,0) to  (3,2); 
\end{tikzpicture}
&  
\begin{tikzpicture}[blue,line width=2pt,fill=green]
\draw[help lines] (0,0) grid (3,2); 
\draw[in=-90]  (0,0) to  (3,2); 
\end{tikzpicture}
\\ \hline  
\BS{draw}  (0,0) \DDD{to} (3,2);& \BS{draw}[\RDD{out}=0]  (0,0) to (3,2); & \BS{draw} [\RDD{in}=-90]  (0,0) to (3,2); 
\\  \hline 
\multicolumn{3}{|c|}{\TFRGB{voir}{see} section \ref{liaisons} page \pageref{liaisons}  }\\ 
\hline

\end{tabular} 

\bigskip

\noindent \begin{tabular}{|c|c|c|} \hline
\multicolumn{2}{|c|}{ \bf{\TFRGB{Dessin avec plot}{Drawing with plot} }}
\RRR{14-12} \RRR{22}
\\ \hline

\TFRGB{une liste de coordonnées}{list of coordinates} & \TFRGB{un fichier de coordonnées}{file of coordinates} & \TFRGB{une équation mathématique}{mathematical equation}
     \\ \hline
\begin{tikzpicture}[blue,line width=2pt,fill=green] 
\draw[help lines] (0,0) grid (5,3); 
\draw[blue,ultra thick] plot coordinates {(2,0) (3,1) (4,1) (5,2)};
\end{tikzpicture}
   &
\begin{tikzpicture}[blue,line width=2pt,fill=green] 
\draw[help lines] (0,0) grid (3,3);
\draw[blue,ultra thick] plot file {table.dat};
\end{tikzpicture}
   &
\begin{tikzpicture}[blue,line width=2pt,fill=green] 
\draw[help lines] (0,-1) grid (4,1); \draw[blue,domain=0:360,x=0.3,ultra thick] plot (\x,{sin(\x)});
 \end{tikzpicture}
  \\ \hline 
 plot coordinates  	&  plot file \AC{table.dat} &  plot (\BS{x},\AC{sin(\BS{x})}) \\
  \AC{(2,0) (3,1) (4,1) (5,2)} &&
  \\ \hline   
  \multicolumn{3}{|c|}{voir page \pageref{plot} }
 \\ \hline 

\end{tabular} 

\newpage 

\SSCT{Chemin}{Path and edge}

\SbSSCT{Notion de Chemin}{Path}
\begin{center}
\RRR{14}
\end{center}

\noindent \begin{tabular}{|c|c|} \hline 
\begin{tikzpicture}[scale=.8,blue,line width=2pt]
\draw[help lines] (0,0) grid (3,2); 
\draw  (0,0) -- (2,1) -- (3,0) ; 
\end{tikzpicture}
& 
\begin{tikzpicture}[scale=.8,blue,line width=2pt]
\draw[help lines] (0,0) grid (3,2); 
\draw  (0,0) -- (2,1) -- (3,0) -- cycle ; 
\end{tikzpicture}
\\ \hline
\BS{draw}  (0,0) - - (2,1) - - (3,0) ; 

& 
\BS{draw}  (0,0) -  - (2,1) -  - (3,0) -  - \RDD{cycle} ; 
\\ \hline 
\end{tabular} 

\bigskip
\noindent \begin{tabular}{|c|c|} \hline 
 \multicolumn{2}{|c|}{  \BS{draw} (0,0) - - (2,1) - - (3,3)  arc (135:-20:1)   .. controls (6,0) and (4,0) }\\
 \multicolumn{2}{|c|}{  .. (5,2) sin (6.57,0) cos (7.57,2) ;}
 \\ \hline
 
\begin{tikzpicture}[scale=.8,blue,line width=2pt]
\draw[help lines] (0,0) grid (8,4); 
\draw  (0,0) -- (2,1) -- (3,3)  arc (135:-20:1)   .. controls (6,0) and (4,0) .. (5,2) sin (6.57,0) cos (7.57,2) ;
\end{tikzpicture}
&  
\begin{tikzpicture}[scale=.8,blue,line width=2pt]
\draw[help lines] (0,0) grid (8,4); 
\filldraw  (0,0) -- (2,1) -- (3,3)  arc (135:-20:1)   .. controls (6,0) and (4,0) .. (5,2) sin (6.57,0) cos (7.57,2) ;
\end{tikzpicture}
\\ \hline 
\BS{draw} & \BS{filldraw}
\\ \hline  
\end{tabular} 





\bigskip 
\begin{center}
\RRR{14-5 } 
\end{center}

  
\noindent \begin{tabular}{|c|c|} \hline 
\begin{tikzpicture}[scale=.8,blue,line width=4mm]
\draw[help lines] (0,0) grid (3,2); 
\draw [rounded corners] (0,0) -- (2,1) -- (3,0) ; 
\end{tikzpicture}
& 
\begin{tikzpicture}[scale=.8,blue,line width=4mm]
\draw[help lines] (0,0) grid (3,2); 
\draw [sharp corners] (0,0) -- (2,1) -- (3,0)  ; 
\end{tikzpicture}
\\ \hline
\BS{draw} [\RDD{rounded corners}] (0,0) -- (2,1) -- (3,0) ; 

& 
\BS{draw} [\RDD{sharp corners}] (0,0) -  - (2,1) -  - (3,0) ; 
\\ 
\hline 
\end{tabular} 

\bigskip
\noindent \begin{tabular}{|c|c|} \hline  
\begin{tikzpicture}[scale=.8,blue,baseline=0pt,line width=2pt]
\draw[help lines] (0,0) grid (2,2); 
\draw   (0,0) -- (1,1.732)  -- (2,0)  -- cycle ;  
\end{tikzpicture}
&  
\BS{draw} [\RDD{rounded corners}=0.5cm]   (0,0) - - (1,1.732) - - (2,0)  - - cycle ; 
\\ \hline  
\begin{tikzpicture}[scale=.8,blue,baseline=0pt,line width=2pt]
\draw[help lines] (0,0) grid (2,2); 
\draw   (0,0) -- (1,1.732) [rounded corners=0.5cm]  -- (2,0)  -- cycle ;  
\end{tikzpicture}
&  
\BS{draw}  (0,0) - - (1,1.732) [\RDD{rounded corners}=0.5cm]  - - (2,0)  - - cycle ;  
\\ \hline 
\begin{tikzpicture}[scale=.8,blue,baseline=0pt,line width=2pt]
\draw[help lines] (0,0) grid (2,2); 
\draw  (0,0) -- (1,1.732) -- (2,0)[rounded corners=0.5cm] -- cycle ; 
\end{tikzpicture}
&  
\BS{draw} (0,0) - - (1,1.732) - - (2,0)[\RDD{rounded corners}=0.5cm] - - cycle ;
\\ \hline 
\begin{tikzpicture}[scale=.8,blue,baseline=0pt,line width=2pt]
\draw[help lines] (0,0) grid (2,2); 
\draw [rounded corners=0.5cm]  (0,0) -- (1,1.732)[sharp corners] -- (2,0) -- cycle ; 
\end{tikzpicture}
&  
\BS{draw} [\RDD{rounded corners}=0.5cm]  (0,0) - - (1,1.732)[\RDD{sharp corners}] - - (2,0) - - cycle ; 
\\ \hline 

\end{tabular} 

\bigskip
\begin{center}
\RRR{14-2-2}
\end{center}


\noindent \begin{tabular}{|c|c|} \hline 
\begin{tikzpicture}[scale=.8,blue,line width=2pt]
\draw[help lines] (0,0) grid (3,2); 
\draw  (0,0) -- (2,1) -| cycle ; 
\end{tikzpicture}
& 
\begin{tikzpicture}[scale=.8,blue,line width=2pt]
\draw[help lines] (0,0) grid (3,2); 
\draw (0,0) -- (2,1) |- cycle  ; 
\end{tikzpicture}
\\ \hline
\BS{draw}  (0,0) - - (2,1) \textbf{{\color{red}-|}} cycle  ; 
& 
\BS{draw} (0,0) - - (2,1) \textbf{{\color{red}|-}} cycle  ; 
\\ 
\hline 
\end{tabular} 

\bigskip

\noindent \begin{tabular}{|c|} \hline  
\BS{tikz} [{\color{red} c/.style}=\AC{\RDD{insert path}=\AC{circle[radius=3pt]}}] \\
\BS{draw}(0,0){\color{red}[c]} -- (1,2){\color{red}[c]} -- (3,1) {\color{red}[c]};
\\ \hline  
\begin{tikzpicture}[c/.style={insert path={circle[radius=3pt]}}]
\draw[help lines] (0,0) grid (4,3); 
\draw(0,0)[c] -- (1,2)[c]  -- (3,1) [c];
\end{tikzpicture}
\\ \hline 
\end{tabular} 


%TODO 2 commandes pour expert à voir plus tard

%\tikz \draw node [append after command={(xxx)--(1,1)},draw] (xxx){noeud};
%
%\tikz \fill  [append after command={[blue](0,0) rectangle (2,2)},draw] [red](1,1) rectangle (3,3);
%
%\tikz \draw node [prefix after command={(foo)--(1,1)},draw] (foo){foo};
%
%\tikz \fill  [prefix after command={[blue](0,0) rectangle (2,2)},draw] [red](1,1) rectangle (3,3);
\bigskip

\bf{\TFRGB{Coupure de chemin}{Path interrupted}} \RRR{14-1}

\bigskip

\noindent \begin{tabular}{|c|} \hline 
\BS{draw} (0.5,0.5) - -(2.5,0.5)   (0.5,1.5) - -(2.5,1.5);
\\ \hline 
\begin{tikzpicture}[blue,line width=2pt]
\draw[help lines] (0,0) grid (3,2); 
\draw (0.5,0.5) - -(2.5,0.5)   (0.5,1.5) - -(2.5,1.5);
\end{tikzpicture}
\\ \hline 
\end{tabular} 

\bigskip

\noindent \begin{tabular}{|c|} \hline  
\BS{draw} (0,0) - - (0,1) - - (1,1) (2,0) - - (2,1)
- - (3,1) - - (\RDD{current subpath start}); \\
\BS{fill}[red]  (\RDD{current subpath start}) circle (3pt);
\\ \hline 
\begin{tikzpicture}[blue,baseline=0pt,line width=2pt]
\draw[help lines] (0,0) grid (4,2); 
\draw (0,0) -- (0,1) -- (1,1) (2,0) -- (2,1)
-- (3,1) -- (current subpath start);
\fill[red]  (current subpath start) circle (3pt);
\end{tikzpicture}
\\ \hline 
\end{tabular} 

\SbSSCT{Chemins dans un chemin}{Pathes in a path : edge}

\begin{center}
 \RRR{17-12} 
\end{center}
 
 \bigskip
 
\begin{tabular}{|c|}  \hline  
\BS{draw} (0,0) - - (2,1) \RDD{edge}[dotted] (3,0) \RDD{edge}[red] (3,2)  - -(1,2) - -  (0,1) ;

\\ \hline  
\begin{tikzpicture}[blue,baseline=0pt,line width=2pt]
\draw[help lines] (0,0) grid (3,2);
\draw (0,0) - - (2,1) edge[dotted] (3,0) edge[red] (3,2)  --(1,2) --  (0,1) ;
\end{tikzpicture}
\\ \hline 
\end{tabular}  

 \bigskip
 
\begin{tabular}{|c|}  \hline  
\BS{draw} (0,0) - - (2,1) \RDD{edge}([red,\RDD{to path}=\AC{parabola (3,0)} ] () \\
\RDD{edge}[red,\RDD{to path}=\AC{arc(-90 : 90 : 0.5)}] ()   - -(1,2) - -  (0,1) ;

\\ \hline 
\begin{tikzpicture}[blue,baseline=0pt,line width=2pt]
\draw[help lines] (0,0) grid (3,2);
\draw (0,0) - - (2,1) edge[red,to path={parabola (3,0)}] ()  edge[red,to path={arc(-90 :90 :.5)}] ()   --(1,2) --  (0,1) ;
\end{tikzpicture}
\\ \hline 
\end{tabular} 

\newpage

\SSCT{Les paramètres disponibles}{Parameters}

\SbSSCT{\'Epaisseur de ligne}{Line width}
\begin{center}
\RRR{15-3-1}
\end{center}

\begin{tabular}{|c|c|c|c|} \hline 
 \multicolumn{4}{|c|}{ \BS{tikz} \BS{draw}[line width=.2cm] (0,0) - - (1,1);}
 \\ \hline
\tikz \draw[line width=.2cm,blue] (0,0) - - (1,1) ;
 &  
\tikz \draw[ultra thin,blue] (0,0) - - (1,1) ;
 &  
\tikz \draw [very thin,blue] (0,0) - - (1,1) ;
 &  
\tikz \draw [thin,blue] (0,0) - - (1,1) ; 
 \\ \hline  
[\RDD{line width}=.2cm] & [\RDD{ultra thin}] & [\RDD{very thin}] & [\RDD{thin}] \\ 
  					& (0.1pt) & (0.2pt) & (0.4pt) \\ \hline
\tikz \draw[semithick,blue] (0,0) - - (1,1) ;
 &  
\tikz \draw[thick,blue] (0,0) - - (1,1) ;
 &  
\tikz \draw [very thick,blue] (0,0) - - (1,1) ;
 &  
\tikz \draw [ultra thick,blue] (0,0) - - (1,1) ; 
 \\ \hline  
[\RDD{semithick}] & [\RDD{thick}] & [\RDD{very thick}] & [\RDD{ultra thick}] \\
  (0.6pt)	& (0.8pt) & (1.2pt) & (1.6pt) \\ \hline 
\end{tabular} 

\SbSSCT{Dimensions disponibles}{Dimensions available}

\begin{tabular}{|c|c|} \hline  
\begin{tikzpicture}[blue,line width=2pt,fill=green,baseline=.5cm]
\draw[use as bounding box][line width=0pt] (0,0)rectangle (4,1) ;
\draw[help lines] (0,0) grid (4,1); 
\draw[line width=10pt]  (2,0) to (2,1); 
  \end{tikzpicture}
& 
\BS{draw}[line width=10pt]  (2,0) to (2,1);  
\\ \hline 
\begin{tikzpicture}[blue,line width=2pt,fill=green,baseline=.5cm]
\draw[use as bounding box][line width=0pt] (0,0)rectangle (4,1) ;
\draw[help lines] (0,0) grid (4,1); 
\draw[line width=10bp]  (2,0) to (2,1); 
  \end{tikzpicture}  
&  
\BS{draw}[line width=10bp]  (2,0) to (2,1); 
\\ \hline  
\begin{tikzpicture}[blue,line width=2pt,fill=green,baseline=.5cm]
\draw[use as bounding box][line width=0pt] (0,0)rectangle (4,1) ;
\draw[help lines] (0,0) grid (4,1); 
\draw[line width=10mm]  (2,0) to (2,1); 
  \end{tikzpicture}
&  
\BS{draw}[line width=10mm]  (2,0) to (2,1);
\\ \hline  
\begin{tikzpicture}[blue,line width=2pt,fill=green,baseline=.5cm]
\draw[use as bounding box][line width=0pt] (0,0)rectangle (4,1) ;
\draw[help lines] (0,0) grid (4,1); 
\draw[line width=1cm]  (2,0) to (2,1); 
  \end{tikzpicture}
&  
\BS{draw}[line width=1cm]  (2,0) to (2,1);
\\ \hline  
\begin{tikzpicture}[blue,line width=2pt,fill=green,baseline=.5cm]
\draw[use as bounding box][line width=0pt] (0,0) rectangle (4,1) ;
\draw[help lines] (0,0) grid (4,1); 
\draw[line width=1in]  (2,0) to (2,1); 
\end{tikzpicture}
&  
\BS{draw}[line width=1in]  (2,0) to (2,1);
\\ \hline
\end{tabular} 

\bigskip


\begin{tabular}{|c|c|} \hline  
\begin{tikzpicture}[blue,line width=2pt,fill=green,baseline=.5cm]
\draw[use as bounding box][line width=0pt] (0,0) rectangle (4,1) ;
\draw[help lines] (0,0) grid (4,1); 
\draw[line width=1ex]  (0,0.5) to (4,.5);
\draw[green] (2,.5) node  {x};
\end{tikzpicture}
&  
\BS{draw}[line width=1ex]  (0,0.5) to (4,.5);
\\ \hline
\begin{tikzpicture}[blue,line width=2pt,fill=green,baseline=.5cm]
\draw[use as bounding box][line width=0pt] (0,0) rectangle (4,1) ;
\draw[help lines] (0,0) grid (4,1); 
\Huge \draw[line width=1ex] (0,0.5) to (4,.5);
\draw[green] (2,.5) node  {x};
\end{tikzpicture}
&  
\BS{Huge} \BS{draw}[line width=1ex]  (0,0.5) to (4,.5);
\\ \hline
\begin{tikzpicture}[blue,line width=2pt,fill=green,baseline=.5cm]
\draw[use as bounding box][line width=0pt] (0,0) rectangle (4,1) ;
\draw[help lines] (0,0) grid (4,1); 
 \draw[line width=1em]  (2,0) to (2,1);
\draw[green] (2,.5) node  {m};
\end{tikzpicture}
&  
\BS{draw}[line width=1em]  (2,0) to (2,1);
\\ \hline
\begin{tikzpicture}[blue,line width=2pt,fill=green,baseline=.5cm]
\draw[use as bounding box][line width=0pt] (0,0) rectangle (4,1) ;
\draw[help lines] (0,0) grid (4,1); 
\Huge \draw[line width=1em]  (2,0) to (2,1);
\draw[green] (2,.5) node  {m};
\end{tikzpicture}
&  
\BS{Huge} \BS{draw}[line width=1em]  (2,0) to (2,1);
\\ \hline
\end{tabular} 


\SbSSCT{Terminaisons de lignes}{Terminators}

\begin{tabular}{|c|c|c|} \hline  
\begin{tikzpicture}[blue,line width=.5cm] 
\draw[line cap=rect]  (0,0) - - (2,0);
\draw[red,line width=1pt](0,0) - - (2,0);
\end{tikzpicture}
&
\begin{tikzpicture}[blue,line width=.5cm] 
\draw[line cap=butt]  (0,0) - - (2,0);
\draw[red,line width=1pt](0,0) - - (2,0);
\end{tikzpicture}  
&
\begin{tikzpicture}[blue,line width=.5cm] 
\draw[line cap=round]  (0,0) - - (2,0);
\draw[red,line width=1pt](0,0) - - (2,0);
\end{tikzpicture}  \\ 
\hline [\RDD{line cap}=\RDDX{rect}{line cap}] & [\RDD{line cap}=\RDDX{butt}{line cap}] &  
[\RDD{line cap}=\RDDX{round}{line cap}]\\ 
\hline 
\end{tabular} 


\SbSSCT{Jonction de lignes}{Lines junction}


\begin{tabular}{|c|c|c|} \hline 

 \multicolumn{3}{|c|}{ \BS{draw}[\RDD{line join}=\RDDX{round}{line join}] (0,0)  - - (2,1) - - (0,2);}
 \\ \hline
\begin{tikzpicture}[blue,line width=.5cm] 
\draw[line join=round] (0,0)  -- (2,1) -- (0,2);
\draw[red,line width=1pt](0,0)  -- (2,1) -- (0,2);
\end{tikzpicture}
& 
\begin{tikzpicture}[,blue,line width=.5cm] 
\draw[line join=bevel] (0,0)  -- (2,1) -- (0,2);
\draw[red,line width=1pt](0,0)  -- (2,1) -- (0,2);
\end{tikzpicture}
&
\begin{tikzpicture}[blue,line width=.5cm]  
\draw[line join=miter] (0,0)  -- (2,1) -- (0,2);
\draw[red,line width=1pt](0,0)  -- (2,1) -- (0,2);
\end{tikzpicture}
\\ \hline  
[\RDD{line join}=\RDDX{round}{line join}] &  
[\RDD{line join}=\RDDX{bevel}{line join}] &  
[\RDD{line join}=\RDDX{miter}{line join}] 
\\ \hline 
\end{tabular} 
\bigskip

\begin{tabular}{|c|c|c|} \hline 
 \multicolumn{3}{|c|}{  \BS{draw}[\RDD{miter limit}=1]  (0,0) - - (2,1) - - (0,2);} \\
\multicolumn{3}{|c|}{   (\dft{} : miter limit=10) }
 \\ \hline 
\begin{tikzpicture}[blue,line width=.5cm] 
\draw[miter limit=1]  (0,0) -- (2,1) -- (0,2);
\draw[red,line width=1pt](0,0)  -- (2,1) -- (0,2);
\end{tikzpicture}
&  
\begin{tikzpicture}[blue,line width=.5cm] 
\draw[miter limit=2]  (0,0) -- (2,1) -- (0,2);
\draw[red,line width=1pt](0,0)  -- (2,1) -- (0,2);
\end{tikzpicture}
&  
\begin{tikzpicture}[blue,line width=.5cm] 
\draw[miter limit=3]  (0,0) -- (2,1) -- (0,2);
\draw[red,line width=1pt](0,0)  -- (2,1) -- (0,2);
\end{tikzpicture}
\\ \hline miter limit=1 & miter limit=2 & miter limit=3 \\ 
\hline 
\end{tabular} 

\SbSSCT{Styles de ligne}{Line styles}

\begin{center}
\RRR{15-3-2}
\end{center}

\begin{tabular}{|c|c|c|c|} \hline 
 \multicolumn{3}{|c|}{ \BS{tikz} \BS{draw}[\RDD{solid},line width=2mm] (0,0) - - (2,1);}
 \\ \hline
\tikz \draw[solid,line width=2mm,blue] (0,0) - - (2,1) ;
& &
 \\ \hline
 [\RDD{solid}] & &
\\ \hline 
   
\tikz \draw[dotted,line width=2mm,blue] (0,0) - - (2,1) ;
 &  
\tikz \draw [densely dotted,line width=2mm,blue] (0,0) - - (2,1) ;
 &  
\tikz \draw [loosely dotted,line width=2mm,blue] (0,0) - - (2,1) ;
 \\ \hline  
 [\RDD{dotted}] & [\RDD{densely dotted}] & [\RDD{loosely dotted}] 
\\ 	\hline

\tikz \draw[dashed,line width=2mm,blue] (0,0) - - (2,1) ;
 &  
\tikz \draw[densely dashed,line width=2mm,blue] (0,0) - - (2,1) ;
 &  
\tikz \draw [loosely dashed,line width=2mm,blue] (0,0) - - (2,1) ;
\\ 	\hline
[\RDD{dashed}] & [\RDD{densely dashed}] & [\RDD{loosely dashed}]
\\ \hline 
\tikz \draw [dash dot,line width=2mm,blue] (0,0) - - (2,1) ;
&
\tikz \draw [densely dash dot,line width=2mm,blue] (0,0) - - (2,1) ;
&
\tikz \draw [loosely dash dot,line width=2mm,blue] (0,0) - - (2,1) ;
\\ \hline  
[\RDD{dash dot}] & [\RDD{densely dash dot}] & [\RDD{loosely dash dot}] 
\\ \hline 
\tikz \draw [dash dot dot,line width=2mm,blue] (0,0) - - (2,1) ;
&
\tikz \draw [densely dash dot dot,line width=2mm,blue] (0,0) - - (2,1) ;
&
\tikz \draw [loosely dash dot dot,line width=2mm,blue] (0,0) - - (2,1) ;
\\ \hline  
[\RDD{dash dot dot}] & [\RDD{densely dash dot dot}] & [\RDD{loosely dash dot dot}]
\\ \hline
\end{tabular}

\bigskip
\begin{tabular}{|c|c|} \hline  
\begin{tikzpicture}[blue,line width=2pt,fill=green,baseline=.5cm]
\draw[help lines] (0,0) grid (6,1); 
\draw[dash pattern=on 1cm off .25cm on .25cm off .5cm,ultra thick,blue] (0,0.5) - - (6,.5) ; 
  \end{tikzpicture}
\\ \hline 
[\RDD{dash pattern}= \BDD{on} 1cm \BDD{off} 0.25cm \BDD{on} 0.25cm \BDD{off} 0.5cm] 
\\ \hline
\begin{tikzpicture}[blue,line width=2pt,fill=green,baseline=.5cm]
\draw[help lines] (0,0) grid (6,1); 

\draw[dash pattern=on 1cm off .25cm on .25cm off .5cm,dash phase=1cm,ultra thick,blue] (0,0.5) - - (6,.5) ;
  \end{tikzpicture}
\\ \hline  

[dash pattern=on 1cm off .25cm on .25cm off .5cm,\RDD{dash phase}=1cm] 
\\ \hline 
\end{tabular}

\bigskip

\begin{center}
\RRR{15-3-4}
\end{center}


\begin{tabular}{|c|c|c|c|} \hline 
 \multicolumn{4}{|c|}{ \BS{tikz} \BS{draw}[line width=.2cm,\RDD{double}] (0,0) - - (1,1);}
 \\ \hline

\tikz \draw[line width=.2cm,double,blue] (0,0) - - (1,1) ;
&
\tikz \draw[line width=.2cm,draw=blue,double=red] (0,0) - - (1,1) ;
&
\tikz \draw[line width=.2cm,double distance=.3cm] (0,0) - - (1,1) ;
&
\tikz \draw[line width=.2cm,double,double distance between line centers=.3cm] (0,0) - - (1,1) ;
\\ \hline 
\RDD{double} & draw=blue,double=red & \RDD{double distance}=.3cm & \RDD{double distance between line centers} \\
& & &  =.3cm
\\ \hline 
\end{tabular}

\bigskip


\begin{tabular}{|c|c|} \hline  
 \multicolumn{2}{|c|}{ \BS{Huge} = \BS{tikz} \BS{draw}[\RDD{double equal sign distance}] (0,0) - - (4,0);}
 \\ \hline
 \rule[-.3cm]{0pt}{1cm}
{\Huge  = \tikz[baseline=-.2cm] \draw[blue,double equal sign distance] (0,0) - - (4,0) ;}
&  
{\large  = \tikz[baseline=-.1cm]  \draw[blue,double equal sign distance] (0,0) - - (4,0) ;}
\\ \hline  
\BS{Huge}
&  
\BS{large}
\\ \hline 
\end{tabular} 

\SbSSCT{Remplissage en motifs}{Fillings}
\label{lib-patterns}

\begin{center}
\RRR{15-5-1} \RRR{60}
\end{center}

\maboite{\BS{usetikzlibrary}\AC{patterns}}
 

\begin{tabular}{|c|c|c|} \hline  
 \multicolumn{3}{|c|}{ \BS{draw}[\RDD{pattern}= \RDDX{dots}{pattern}] (0,0) - - (3,1);}
 \\ \hline

\begin{tikzpicture}
\draw[white] (0,0)--(0,1.2);
\draw[pattern=dots] (0,0) rectangle (3,1);
\end{tikzpicture}
&  
\begin{tikzpicture}
\draw[pattern=fivepointed stars] (0,0) rectangle (3,1);
\end{tikzpicture}
&  
\begin{tikzpicture}
\draw[pattern=sixpointed stars] (0,0) rectangle (3,1);
\end{tikzpicture}
\\ \hline  
\RDDX{dots}{pattern} & \RDDX{fivepointed stars}{pattern} & \RDDX{sixpointed stars}{pattern}   
\\ \hline

\begin{tikzpicture}
\draw[white] (0,0)--(0,1.2);
\draw[pattern=grid] (0,0) rectangle (3,1);
\end{tikzpicture}
&
\begin{tikzpicture}
\draw[pattern=horizontal lines] (0,0) rectangle (3,1);
\end{tikzpicture}
&  
\begin{tikzpicture}
\draw[pattern=vertical lines] (0,0) rectangle (3,1);
\end{tikzpicture}
\\ \hline  
\RDDX{grid}{pattern} & \RDDX{horizontal lines}{pattern} & \RDDX{vertical lines}{pattern} 
\\ \hline  

\begin{tikzpicture}
\draw[white] (0,0)--(0,1.2);
\draw[pattern=north east lines] (0,0) rectangle (3,1);
\end{tikzpicture}
&  
\begin{tikzpicture}
\draw[pattern=north west lines] (0,0) rectangle (3,1);
\end{tikzpicture}
&
\begin{tikzpicture}
\draw[pattern=crosshatch] (0,0) rectangle (3,1);
\end{tikzpicture}

\\ \hline
 \RDDX{north east lines}{pattern} & 
 \RDDX{north west lines}{pattern}  &  \RDDX{rosshatch}{pattern}
\\ \hline

 
\begin{tikzpicture}
\draw[white] (0,0)--(0,1.2);
\draw[pattern=crosshatch dots] (0,0) rectangle (3,1);
\end{tikzpicture}
&  
\begin{tikzpicture}
\draw[pattern=bricks] (0,0) rectangle (3,1);
\end{tikzpicture}
& 
\begin{tikzpicture}
\draw[pattern=checkerboard] (0,0) rectangle (3,1);
\end{tikzpicture}
 
\\ \hline
\RDDX{crosshatch dots}{pattern}  & \RDDX{bricks}{pattern} & \RDDX{checkerboard}{pattern}
\\ \hline
\end{tabular} 

\bigskip

\begin{tabular}{|c|} \hline  
\begin{tikzpicture}
\draw[pattern=fivepointed stars,pattern color=red] (0,0) rectangle (3,1);
\end{tikzpicture}
\\ \hline  
\BS{draw}[pattern=fivepointed stars,\RDD{pattern color}=red] (0,0) rectangle (3,1);
\\ \hline 
\end{tabular}

\bigskip

\begin{tabular}{|c|c|c|} \hline  
 \multicolumn{3}{|c|}{ \BS{draw}[pattern=\RDDX{checkerboard light gray}{pattern}] (0,0) - - ((3,2) ;}
 \\ \hline
\begin{tikzpicture}
\draw[pattern=checkerboard light gray] (0,0) rectangle (3,1);
\end{tikzpicture}
&
\begin{tikzpicture}
\draw[pattern=horizontal lines light gray] (0,0) rectangle (3,1);
\end{tikzpicture}
&
\begin{tikzpicture}
\draw[pattern=horizontal lines gray] (0,0) rectangle (3,1);
\end{tikzpicture} 
 \\ \hline
\RDDX{checkerboard light gray}{pattern} &  \RDDX{horizontal lines light gray}{pattern} & \RDDX{horizontal lines gray}{pattern}
 \\ \hline
\begin{tikzpicture}
\draw[pattern=horizontal lines dark gray] (0,0) rectangle (3,1);
\end{tikzpicture}
&
\begin{tikzpicture}
\draw[pattern=horizontal lines light blue] (0,0) rectangle (3,1);
\end{tikzpicture}
&
\begin{tikzpicture}
\draw[pattern=horizontal lines dark blue] (0,0) rectangle (3,1);
\end{tikzpicture}
 \\ \hline
\RDDX{horizontal lines dark gray}{pattern} &  \RDDX{horizontal lines light blue}{pattern} & \RDDX{horizontal lines dark blue}{pattern}
 \\ \hline
 \begin{tikzpicture}
 \draw[pattern=crosshatch dots gray] (0,0) rectangle (3,1);
 \end{tikzpicture}
 &
 \begin{tikzpicture}
 \draw[pattern=crosshatch dots light steel blue] (0,0) rectangle (3,1);
 \end{tikzpicture}
 &
  \\ \hline
\RDDX{crosshatch dots gray}{pattern} & \RDDX{crosshatch dots light steel blue}{pattern}
&  \\ \hline
\end{tabular} 

\SbSSCT{Règle de remplissage}{Filling rule}

\begin{center}
\RRR{15-5-2}
\end{center}


\begin{tabular}{|c|c|} \hline 
\multicolumn{2}{|c|}{ nonzero rule (\dft{}) }
\\ \hline 
\begin{tikzpicture}[scale=.8,blue,baseline=0pt,line width=2pt]
\filldraw[fill=green!20] (0,0) -- (0,3) -- (3,3) -- (3,0) -- cycle  (1,1) -- (1,2) -- (2,2) --(2,1) -- cycle; 
\end{tikzpicture}
&  
\begin{tikzpicture}[scale=.8,blue,baseline=0pt,line width=2pt]
\filldraw[fill=green!20] (0,0) -- (0,3) -- (3,3) -- (3,0) -- cycle  (1,1) -- (2,1) -- (2,2) --(1,2) -- cycle; 
\end{tikzpicture}
\\ \hline 
\BS{filldraw} [fill=green!20]  & \BS{filldraw} [fill=green!20] \\
(0,0) - - (0,3) - - (3,3) - - (3,0) - - cycle  &
(0,0) - - (0,3) - - (3,3) - - (3,0) - - cycle \\
(1,1) - - {\color{red}(1,2) - - (2,2) - -(2,1)} - - cycle ;  & 
(1,1) - - {\color{red}(2,1) - - (2,2) - -(1,2)} - - cycle; 
\\ \hline 
\end{tabular}


\begin{tabular}{|c|c||c|c|} \hline
\multicolumn{2}{|c|}{ even odd rule }
\\ \hline   
 \multicolumn{2}{|c||}{\BS[fill=[green] (0,0) - - (2,1) - - (1,2) circle (.5cm); } & 
  \multicolumn{2}{|c|}{\BS{filldraw}[fill=green] (0,0) -- (2,1) - - (1,2) circle (.5cm); }
 \\ \hline
\begin{tikzpicture}
\fill[green] (0,0) -- (2,1) -- (1,2) circle (.5cm);
\end{tikzpicture}
&  
\begin{tikzpicture}
\fill[even odd rule,green]  (0,0) -- (2,1) -- (1,2) circle (.5cm);
\end{tikzpicture}
&
\begin{tikzpicture}
\filldraw[fill=green] (0,0) -- (2,1) -- (1,2) circle (.5cm);
\end{tikzpicture}
&  
\begin{tikzpicture}
\filldraw[even odd rule,fill=green]  (0,0) -- (2,1) -- (1,2) circle (.5cm);
\end{tikzpicture}
\\ \hline 
[fill=green] & [\RDD{even odd rule},fill=green]  & [fill=green] & [\RDD{even odd rule},fill=green]
\\ \hline 
\end{tabular}

\SbSSCT{Remplissage à l'aide d'une image}{Filling with an image  }

\begin{center}
\RRR{15-6}
\end{center}


\begin{tabular}{|c|c|c|} \hline 
 \multicolumn{3}{|c|}{\BS{draw} [\RDD{path picture}=\{  \BS{node} at (path picture bounding box.center) }\\
 \multicolumn{3}{|c|}{ \AC{\BS{includegraphics}[height=3cm]\AC{tiger}\};}] (0,1) circle (1);}
\\   \hline 
\begin{tikzpicture}
\draw [path picture={
\node at (path picture bounding box.center){\includegraphics[height=3cm]{tiger}};}] (0,1) circle (1);
\end{tikzpicture}
&
\begin{tikzpicture}
\draw [path picture={
\node at (path picture bounding box.center){\includegraphics[height=3cm]{tiger}};}] (0,0) -- (-1,1) -- (0,2) -- (1,1) -- cycle;
\end{tikzpicture}
&
\begin{tikzpicture}
\draw [path picture={
\node at (path picture bounding box.center){\includegraphics[height=3cm]{tiger}};}] (1,0) parabola[parabola height=2cm] (3,0);
\end{tikzpicture}
\\ \hline 
(0,1) circle (1) & (0,0) - - (-1,1) - - (0,2) - - (1,1) - - cycle &  (1,0) parabola[parabola height=2cm] (3,0)\\ 
\hline
\end{tabular} 
\bigskip

\begin{tabular}{|c|c|c|c|c|} \hline 
 \multicolumn{5}{|c|}{\BS{draw} [path picture=\{  \BS{node} at (\RDD{path picture bounding box}.north) }\\
 \multicolumn{5}{|c|}{ \AC{\BS{includegraphics}[height=3cm]\AC{tiger}\};}] (0,1) circle (1);}
\\   \hline 
\begin{tikzpicture}
\draw [path picture={
\node at (path picture bounding box.north){\includegraphics[height=3cm]{tiger}};}] (0,1) circle (1);
\end{tikzpicture}
&
\begin{tikzpicture}
\draw [path picture={
\node at (path picture bounding box.south){\includegraphics[height=3cm]{tiger}};}] (0,1) circle (1);
\end{tikzpicture}
&
\begin{tikzpicture}
\draw [path picture={
\node at (path picture bounding box.east){\includegraphics[height=3cm]{tiger}};}] (0,1) circle (1);
\end{tikzpicture}
&
\begin{tikzpicture}
\draw [path picture={
\node at (path picture bounding box.west){\includegraphics[height=3cm]{tiger}};}] (0,1) circle (1);
\end{tikzpicture}
&
\begin{tikzpicture}
\draw [path picture={
\node at (path picture bounding box.south east){\includegraphics[height=3cm]{tiger}};}] (0,1) circle (1);
\end{tikzpicture}
\\   \hline 
north & south & east & west &south east
\\   \hline 
\end{tabular} 

\SbSSCT{Ombrage}{Shading}


\SbSbSSCT{Ombrages disponibles}{Shadings available}
\begin{center}
\RRR{15-7}
\end{center}

\begin{tabular}{|c|c|} \hline  
\begin{tikzpicture}
\draw[white] (0,0)--(0,1.2);
\shade (0,0) rectangle (3,1);
\end{tikzpicture}
&  
\begin{tikzpicture}
\shadedraw (0,0) rectangle (3,1);
\end{tikzpicture}
\\ \hline  
\BSS{shade} (0,0) rectangle (3,1); & \BSS{shadedraw} (0,0) rectangle (3,1);\\ 
\hline 
\end{tabular} 

\bigskip

\begin{tabular}{|c|c|c|} \hline 
 \multicolumn{3}{|c|}{\BS{shadedraw}[\RDD{shading}=\RDDX{axis}{shading}](0,0) rectangle (3,1); }
 \\ \hline
 
\begin{tikzpicture}
\draw[white] (0,0)--(0,1.2);
\shadedraw[shading=axis] (0,0) rectangle (3,1);
\end{tikzpicture}
&  
\begin{tikzpicture}
\shadedraw[shading=radial] (0,0) rectangle (3,1);
\end{tikzpicture}
&  
\begin{tikzpicture}
\shadedraw[shading=ball] (0,0) rectangle (3,1);
\end{tikzpicture}

\\ \hline  
\RDDX{axis}{shading}  & \RDDX{radial}{shading}  & \RDDX{ball}{shading}\\ 
\hline 
\end{tabular} 

\bigskip

\begin{tabular}{|c|c|c|} \hline  
\begin{tikzpicture}
\draw[white] (0,0)--(0,1.2);
\shadedraw[left color=red] (0,0) rectangle (3,1);
\end{tikzpicture}
&  
\begin{tikzpicture}
\shadedraw[right color=green] (0,0) rectangle (3,1);
\end{tikzpicture}
&  
\begin{tikzpicture}
\shadedraw[left color=red,right color=green] (0,0) rectangle (3,1);
\end{tikzpicture}
\\ \hline  
[\RDD{left color}=red] & [\RDD{right color}=green]  &  \RDD{left color}=red,\RDD{right color}=green \\ 
\hline 
\begin{tikzpicture}
\draw[white] (0,0)--(0,1.2);
\shadedraw[top color=red] (0,0) rectangle (3,1);
\end{tikzpicture}
&  
\begin{tikzpicture}
\shadedraw[bottom color=green] (0,0) rectangle (3,1);
\end{tikzpicture}
&  
\begin{tikzpicture}
\shadedraw[middle color=red] (0,0) rectangle (3,1);
\end{tikzpicture}
\\ \hline  
[\RDD{top color}=red] & [\RDD{bottom color}=green]  &  \RDD{middle color}=red \\ 
\hline 

\end{tabular} 

\bigskip

\begin{tabular}{|c|c|c|} \hline  
\begin{tikzpicture}
\draw[white] (0,0)--(0,1.2);
\shadedraw[shading angle=90] (0,0) rectangle (3,1);
\end{tikzpicture}
&  
\begin{tikzpicture}
\shadedraw[right color=green,shading angle=45] (0,0) rectangle (3,1);
\end{tikzpicture}
&  
\begin{tikzpicture}
\shadedraw[left color=red,shading angle=-45] (0,0) rectangle (3,1);
\end{tikzpicture}

\\ \hline  
 & \RDD{right color}=green  & left color=red \\ 
\RDD{shading angle}=90 & [\RDD{shading angle}=45]  & \RDD{shading angle}=-45 \\
\hline 
\end{tabular} 

\bigskip

\begin{tabular}{|c|c|c|} \hline  
\begin{tikzpicture}
\draw[white] (0,0)--(0,1.2);
\shadedraw[inner color=red] (0,0) rectangle (3,1);
\end{tikzpicture}
&  
\begin{tikzpicture}
\shadedraw[outer color=green] (0,0) rectangle (3,1);
\end{tikzpicture}
&  
\begin{tikzpicture}
\shadedraw[outer color=green,inner color=red] (0,0) rectangle (3,1);
\end{tikzpicture}

\\ \hline  
 \RDD{inner color}=red & \RDD{outer color}=green  & \RDD{inner color}=red \RDD{outer color}=green
 \\ \hline 
\end{tabular}

\newpage
\SbSbSSCT{Bibliothèque shadings}{Shading library}
\begin{center}
 \RRR{65}
\end{center}
 
 \maboite{\BS{usetikzlibrary}\AC{shadings}}
\label{lib-shadings}

\begin{tabular}{|c|c|c|c|c|} \hline
 \multicolumn{5}{|c|}{\BS{shadedraw}[\RDD{upper left}=red] (0,0) rectangle (2,2) ; }
 \\ \hline 
\begin{tikzpicture}
\shadedraw[upper left=red] (0,0) rectangle (2,2);
\end{tikzpicture} 
&  
\begin{tikzpicture}
\shadedraw[upper right=green] (0,0) rectangle (2,2);
\end{tikzpicture} 
&  
\begin{tikzpicture}
\shadedraw[lower left=blue] (0,0) rectangle (2,2);
\end{tikzpicture} 
&  
\begin{tikzpicture}
\shadedraw[lower right=yellow] (0,0) rectangle (2,2);
\end{tikzpicture} 
&
\begin{tikzpicture}
\shadedraw[upper left=red,upper right=green,
lower left=blue,lower right=yellow] (0,0) rectangle (2,2);
\end{tikzpicture} 
\\ \hline 
 \RDD{upper left}=red &  \RDD{upper right}=green &  \RDD{lower left}=blue  &  \RDD{lower right}=yellow & \\ 
\hline 
\end{tabular}  

\bigskip

\begin{tabular}{|c|c|c|} \hline 
 \multicolumn{3}{|c|}{\BS{shadedraw}[shading=\RDDX{color wheel}{shading}] (0,0) rectangle (2,2) ; }
 \\ \hline 
\begin{tikzpicture}
\shadedraw[shading=color wheel] (0,0) rectangle (2,2);
\end{tikzpicture} 
&
\begin{tikzpicture}
\shadedraw[shading=color wheel black center] (0,0) rectangle (2,2);
\end{tikzpicture}
&
\begin{tikzpicture}
\shadedraw[shading=color wheel white center] (0,0) rectangle (2,2);
\end{tikzpicture}
\\ \hline 
shading=\RDDX{color wheel}{shading} & shading=\RDDX{color wheel black center}{shading} & 
shading=\RDDX{color wheel white center}{shading}
\\ \hline 
\end{tabular} 

\bigskip

\begin{tabular}{|c|} \hline  
\begin{tikzpicture}
\shadedraw[shading=Mandelbrot set] (0,0) rectangle (2,2);
\end{tikzpicture}
\\\hline  
shading=\RDDX{Mandelbrot set}{shadingv}
\\ \hline 
\end{tabular} 





\newpage

\SbSSCT{Les extrémités}{Extremities}
\label{fleches}


\SbSbSSCT{Chargé automatiquement avec TikZ}{TikZ package}

\begin{tabular}{|c|c|c|c|} \hline 
 \multicolumn{4}{|c|}{ \BS{tikz} \BS{draw}[->,line width=.2cm,blue] (0,0) - - (1.5,1);}
 \\ \hline
\tikz \draw[->,line width=.2cm,blue] (0,0) - - (1.5,1) ;
 &  
\tikz \draw[->,line width=.2cm,blue] (0,0) - - (1.5,1) ;
 &  
\tikz \draw [<->,line width=.2cm,blue] (0,0) - - (1.5,1) ;
 &  
\tikz \draw [>->,line width=.2cm,blue] (0,0) - - (1.5,1) ; 
 \\ \hline  
[\FDD{->}] & [\FDD{<-}] & [\FDD{<->}] & [\FDD{>->}] \\ 
\hline
\tikz \draw[-to,line width=.2cm,blue] (0,0) - - (1.5,1) ;
 &  
\tikz \draw[-to reversed,line width=.2cm,blue] (0,0) - - (1.5,1) ;
 &  
\tikz \draw[-o,line width=.2cm,blue] (0,0) - - (1.5,1) ;
 &  
\tikz \draw [-|,line width=.2cm,blue] (0,0) - - (1.5,1) ; 
 \\ \hline  
[\FDD{-to}] & [\FDD{-to reversed}] & [\FDD{-o}] & [\FDD{-|}] \\
  \hline 
\tikz \draw[-latex,line width=.2cm,blue] (0,0) - - (1.5,1) ;
 &  
\tikz \draw[-latex reversed,line width=.2cm,blue] (0,0) - - (1.5,1) ;
 & 
\tikz \draw[-stealth,line width=.2cm,blue] (0,0) - - (1.5,1) ;
& 
\tikz \draw[-stealth reversed ,line width=.2cm,blue] (0,0) - - (1.5,1) ;

 \\ \hline  
[\FDD{-latex}] & [\FDD{-latex reversed}] & [\FDD{-stealth}] & [\FDD{-stealth reversed}] \\
  \hline 
\end{tabular}



\subsubsection{\og library arrow.meta \fg}

 \maboite{\BS{usetikzlibrary}\AC{arrows.meta}}
\label{lib-arrows.meta}


\begin{tabular}{|c|c|c|c|c|} \hline 
 \multicolumn{5}{|c|}{ \BS{tikz} \BS{draw}[ \FDD{-Arc Barb},line width=.2cm,blue ] (0,0) - - (1.5,1) ;}
 \\ \hline
\tikz \draw[-Arc Barb,line width=.2cm,blue] (0,0) - - (1.5,1) ;
 &  
\tikz \draw[-Bar,line width=.2cm,blue] (0,0) - - (1.5,1);
 &
 \tikz \draw[-Bracket,line width=.2cm,blue] (0,0) - - (1.5,1) ;
  &  
 \tikz \draw[-Hooks,line width=.2cm,blue] (0,0) - - (1.5,1) ;
 &
  \tikz \draw[-Stealth,line width=.2cm,blue] (0,0) - - (1.5,1) ;
 \\ \hline  
\FDD{-Arc Barb} & \FDD{-Bar} & \FDD{-Bracket} & \FDD{-Hooks} & \FDD{-Stealth} 
 \\ \hline  
\tikz \draw [-Parenthesis,line width=.2cm,blue] (0,0) - - (1.5,1) ;
 &  
\tikz \draw [-Straight Barb,line width=.2cm,blue] (0,0) - - (1.5,1) ;
 &
 \tikz \draw[-Tee Barb,line width=.2cm,blue] (0,0) - - (1.5,1) ;
  &  
 \tikz \draw[-Classical TikZ Rightarrow,line width=.2cm,blue] (0,0) - - (1.5,1);
 &
  \tikz \draw[-Square,line width=.2cm,blue] (0,0) - - (1.5,1) ;
 \\ \hline 
\FDD{-Parenthesis} & \FDD{-Straight Barb} &\FDD{-Tee Barb} & \FDD{-Classical TikZ Rightarrow} & \FDD{-Square}
 \\ \hline
 
 \tikz \draw[-Circle,line width=.2cm,blue] (0,0) - - (1.5,1) ;   

 &  
\tikz \draw [-Implies,double,line width=.2cm,blue] (0,0) - - (1.5,1) ;
 &
\tikz \draw [-Rectangle,line width=.2cm,blue] (0,0) - - (1.5,1) ;
  &  
\tikz \draw [-Computer Modern Rightarrow,line width=.2cm,blue] (0,0) - - (1.5,1) ;
&
 \tikz \draw[-Turned Square,line width=.2cm,blue] (0,0) - - (1.5,1) ;
 \\ \hline 
 \FDD{-Circle}  & \FDD{-Implies}, double & \FDD{-Rectangle}  & \FDD{-Computer Modern Rightarrow} &  \FDD{-Turned Square}
 \\ \hline
   & & & [\FDD{-To}] 
 \\ \hline   
\tikz \draw [-Diamond,line width=.2cm,blue] (0,0) - - (1.5,1) ;
 &  
\tikz \draw [-Ellipse,line width=.2cm,blue] (0,0) - - (1.5,1) ;
 &
 \tikz \draw[-Kite,line width=.2cm,blue] (0,0) - - (1.5,1) ;
  &  
 \tikz \draw[-Latex,line width=.2cm,blue] (0,0) - - (1.5,1) ;
 &
 \tikz \draw [-Triangle,line width=.2cm,blue] (0,0) - - (1.5,1) ;
 \\ \hline 
\FDD{-Diamond} & \FDD{-Ellipse} &\FDD{-Kite} & [\FDD{-Latex}]  &  \FDD{-Triangle}
 \\ \hline

\end{tabular}

\bigskip

\begin{tabular}{|c|c|c|c|c|} \hline 
 \multicolumn{5}{|c|}{ \BS{tikz} \BS{draw}[\FDD{-Butt Cap},line width=.2cm,blue] (0,0) - - (1.5,1) ;}
 \\ \hline
\tikz \draw [-Butt Cap,line width=.5cm,blue] (0,0) - - (1.5,1) ;
 &  
\tikz \draw [-Fast Round,line width=.5cm,blue] (0,0) - - (1.5,1) ;
 &
 \tikz \draw[-Fast Triangle,line width=.5cm,blue] (0,0) - - (1.5,1) ;
  &  
 \tikz \draw[-Round Cap,line width=.5cm,blue] (0,0) - - (1.5,1) ;
&
\tikz \draw[-Triangle Cap,line width=.5cm,blue] (0,0) - - (1.5,1) ; 
 \\ \hline 
\FDD{-Butt Cap} &\FDD{-Fast Round}  & \FDD{-Fast Triangle} & \FDD{-Round Cap}  & \FDD{-Triangle Cap}  
 \\ \hline
\end{tabular}

\bigskip

\begin{tabular}{|c|c|c|} \hline 
 \multicolumn{3}{|c|}{ \BS{tikz} \BS{draw}[{\color{red} Triangle-Circle},line width=.2cm,blue] (0,0) - - (3.5,1) ;}
 \\ \hline
\tikz \draw[Triangle-Circle,line width=.2cm,blue] (0,0) - - (3.5,1);
&
\tikz \draw[-{Circle[] Triangle[]},line width=.2cm,blue] (0,0) - - (3.5,1);
& 
\tikz \draw[-{Circle[] . Triangle[] Triangle[] },line width=.2cm,blue] (0,0) - - (3.5,1);
\\ \hline
{\color{red} Triangle-Circle }& {\color{red} \AC{Circle[] Triangle[]}} & {\color{red}\AC{Circle[] . Triangle[] Triangle[] } }
\\ \hline 
\end{tabular}
\bigskip


\begin{tabular}{|c|c|c|c|c|} \hline 
 \multicolumn{5}{|c|}{ \BS{tikz} \BS{draw}[\FDD{-Rays}],line width=.1cm,blue] (0,0) - - (1.5,1);}
 \\ \hline
\tikz \draw [-Rays,line width=.1cm,blue] (0,0) - -(1.5,1) ;
 &  
\tikz \draw [-{Rays[n=2]},line width=.1cm,blue] (0,0) - - (1.5,1) ;
 &
 \tikz \draw[-{Rays[n=3]},line width=.1cm,blue] (0,0) - - (1.5,1) ;
  &  
 \tikz \draw[-{Rays[n=4]},line width=.1cm,blue] (0,0) - - (1.5,1) ;
&
\tikz \draw[-{Rays[n=5]},line width=.1cm,blue] (0,0) - - (1.5,1) ; 
 \\ \hline 
Rays & \AC{Rays[n=2]} & \AC{Rays[n=3]} & \AC{Rays[n=4]} & \AC{Rays[n=5]}
 \\ \hline
 
\tikz \draw [-{Rays[n=6]},line width=.1cm,blue] (0,0) - - (1.5,1) ;
 &  
\tikz \draw [-{Rays[n=7]},line width=.1cm,blue] (0,0) - - (1.5,1) ;
 &
 \tikz \draw[-{Rays[n=8]},line width=.1cm,blue] (0,0) - - (1.5,1) ;
  &  
\tikz \draw[-{Rays[n=9]},line width=.1cm,blue] (0,0) - - (1.5,1) ;
&
\tikz \draw[-{Rays[n=10]},line width=.1cm,blue] (0,0) - - (1.5,1) ;
 \\ \hline 
\AC{Rays[n=6]} &\AC{Rays[n=7]} & \AC{Rays[n=8]}  & \AC{Rays[n=9]} &  \AC{Rays[n=10]}
 \\ \hline    
\end{tabular}




\bigskip

\Par{Paramètre sep}{Parameter sep}

\RRR{16-4-2}

\begin{tabular}{|c|c|c|c|c|c|} \hline 
 \multicolumn{6}{|c|}{ \BS{tikz} \BS{draw}[-\AC{Arc Barb[\RDD{sep}=.25cm]  Arc Barb[ ]},line width=.1cm,blue] (0,0) - - (1.5,1);}
 \\ \hline
\tikz \draw [-{Arc Barb[sep=.25cm] Arc Barb[]},line width=.1cm,blue] (0,0) - - (1.5,1) ;
&  
\tikz \draw [-{Bracket[sep=.25cm]  Bracket[]},line width=.1cm,blue] (0,0) - - (1.5,1) ;
 &
 \tikz \draw[-{Hooks[sep=.25cm] Hooks[]},line width=.1cm,blue] (0,0) - - (1.5,1) ;
  &  
 \tikz \draw[-{Parenthesis[sep=.25cm] Parenthesis[]},line width=.1cm,blue] (0,0) - - (1.5,1) ;
&  
\tikz \draw [-{Classical TikZ Rightarrow[sep=.25cm] Classical TikZ Rightarrow[]},line width=.1cm,blue] (0,0) - - (1.5,1) ;
& 
\tikz \draw[-{Rays[sep=.25cm] Rays[]},line width=.1cm,blue] (0,0) - - (1.5,1) ; 
\\ \hline 
Arc Barb & Bracket & Hooks & Parenthesis & Classical TikZ Rightarrow & Rays
 \\ \hline
 
\tikz \draw[-{Straight Barb[sep=.25cm] Straight Barb[]},line width=.1cm,blue] (0,0) - - (1.5,1) ;
& 
\tikz \draw [-{Tee Barb[sep=.25cm] Tee Barb[]},line width=.1cm,blue] (0,0) - - (1.5,1) ;

&  
\tikz \draw[-{Circle[sep=.25cm] Circle[]},line width=.1cm,blue] (0,0) - - (1.5,1) ;
&
\tikz \draw[-{Ellipse[sep=.25cm] Ellipse[]},line width=.1cm,blue] (0,0) - - (1.5,1) ;
 &
 \tikz \draw[-{Computer Modern Rightarrow[sep=.25cm] Computer Modern Rightarrow[]},line width=.1cm,blue] (0,0) - - (1.5,1) ;
&
\tikz \draw[-{Triangle[sep=.25cm] Triangle[]},line width=.1cm,blue] (0,0) - - (1.5,1) ;
 \\ \hline 
Straight Barb & Tee Barb  & Circle &  Ellipse & Computer Modern Rightarrow & Triangle
 \\ \hline 
 
\tikz \draw[-{Latex[sep=.25cm] Latex[]},line width=.1cm,blue] (0,0) - - (1.5,1) ; 
&
\tikz \draw[-{Kite[sep=.25cm] Kite[]},line width=.1cm,blue] (0,0) - - (1.5,1) ; 
&
\tikz \draw[-{Rectangle[sep=.25cm] Rectangle[]},line width=.1cm,blue] (0,0) - - (1.5,1) ; 
&
\tikz \draw[-{Square[sep=.25cm] Square[]},line width=.1cm,blue] (0,0) - - (1.5,1) ; 
&
\tikz \draw[-{Stealth[sep=.25cm] Stealth[]},line width=.1cm,blue] (0,0) - - (1.5,1) ; 
&
\tikz \draw[-{Turned Square[sep=.25cm] Turned Square[]},line width=.1cm,blue] (0,0) - - (1.5,1) ;
 \\ \hline
Latex  & Kite & Rectangle & Square  & Stealth & Turned Square
 \\ \hline 
\end{tabular}

\bigskip


\begin{tabular}{|c|c|c|c|c|c|} \hline 
 \multicolumn{6}{|c|}{ \BS{tikz} \BS{draw}[-\AC{Arc Barb[sep=.25cm] \tikz \fill[red] circle(2pt); Arc Barb[ ]},line width=.1cm,blue] (0,0) - - (1.5,1);}
 \\ \hline
\tikz \draw [-{Arc Barb[sep=.25cm]. Arc Barb[]},line width=.1cm,blue] (0,0) - - (1.5,1) ;
&  
\tikz \draw [-{Bracket[sep=.25cm] . Bracket[]},line width=.1cm,blue] (0,0) - - (1.5,1) ;
 &
 \tikz \draw[-{Hooks[sep=.25cm]. Hooks[]},line width=.1cm,blue] (0,0) - - (1.5,1) ;
  &  
 \tikz \draw[-{Parenthesis[sep=.25cm]. Parenthesis[]},line width=.1cm,blue] (0,0) - - (1.5,1) ;
&  
\tikz \draw [-{Classical TikZ Rightarrow[sep=.25cm]. Classical TikZ Rightarrow[]},line width=.1cm,blue] (0,0) - - (1.5,1) ;
& 
\tikz \draw[-{Rays[sep=.25cm]. Rays[]},line width=.1cm,blue] (0,0) - - (1.5,1) ; 
\\ \hline 
Arc Barb & Bracket & Hooks & Parenthesis & Classical TikZ Rightarrow & Rays
 \\ \hline
 
\tikz \draw[-{Straight Barb[sep=.25cm]. Straight Barb[]},line width=.1cm,blue] (0,0) - - (1.5,1) ;
& 
\tikz \draw [-{Tee Barb[sep=.25cm]. Tee Barb[]},line width=.1cm,blue] (0,0) - - (1.5,1) ;

&  
\tikz \draw[-{Circle[sep=.25cm]. Circle[]},line width=.1cm,blue] (0,0) - - (1.5,1) ;
&
\tikz \draw[-{Ellipse[sep=.25cm]. Ellipse[]},line width=.1cm,blue] (0,0) - - (1.5,1) ;
 &
 \tikz \draw[-{Computer Modern Rightarrow[sep=.25cm]. Computer Modern Rightarrow[]},line width=.1cm,blue] (0,0) - - (1.5,1) ;
&
\tikz \draw[-{Triangle[sep=.25cm]. Triangle[]},line width=.1cm,blue] (0,0) - - (1.5,1) ;
 \\ \hline 
Straight Barb & Tee Barb  & Circle &  Ellipse & Computer Modern Rightarrow & Triangle
 \\ \hline 
 
\tikz \draw[-{Latex[sep=.25cm]. Latex[]},line width=.1cm,blue] (0,0) - - (1.5,1) ; 
&
\tikz \draw[-{Kite[sep=.25cm]. Kite[]},line width=.1cm,blue] (0,0) - - (1.5,1) ; 
&
\tikz \draw[-{Rectangle[sep=.25cm]. Rectangle[]},line width=.1cm,blue] (0,0) - - (1.5,1) ; 
&
\tikz \draw[-{Square[sep=.25cm]. Square[]},line width=.1cm,blue] (0,0) - - (1.5,1) ; 
&
\tikz \draw[-{Stealth[sep=.25cm]. Stealth[]},line width=.1cm,blue] (0,0) - - (1.5,1) ; 
&
\tikz \draw[-{Turned Square[sep=.25cm]. Turned Square[]},line width=.1cm,blue] (0,0) - - (1.5,1) ;
 \\ \hline
Latex  & Kite & Rectangle & Square  & Stealth & Turned Square
 \\ \hline 
\end{tabular}






\newpage

\Par{Paramètre length}{Parameter length}

\RRR{16-3-1}

\begin{tabular}{|c|c|c|c|c|c|} \hline 
 \multicolumn{6}{|c|}{ \BS{tikz} \BS{draw}[-\AC{Arc Barb[\RDD{length}=1cm]},line width=.2cm,blue] (0,0) - - (1,1);}
 \\ \hline
 \begin{tikzpicture}[blue,line width=2pt,baseline=.5cm]
  \draw[help lines] (0.5,-2) grid (2,2); 
 \draw [-{Arc Barb[length=1cm]},line width=.2cm,blue] (0.5,0) - - (2,0) ; 
 \end{tikzpicture}
& 
\begin{tikzpicture}[blue,line width=2pt,baseline=.5cm]
\draw[help lines] (0.5,-2) grid (2,2); 
\draw [-{Hooks[length=1cm]},line width=.2cm,blue] (0.5,0) - - (2,0) ; 
\end{tikzpicture} 
&
\begin{tikzpicture}[blue,line width=2pt,baseline=.5cm]
\draw[help lines] (0.5,-2) grid (2,2); 
\draw[-{Straight Barb[length=1cm]},line width=.2cm,blue] (0.5,0) - -  (2,0) ; 
\end{tikzpicture} 
& 
\begin{tikzpicture}[blue,line width=2pt,baseline=.5cm]
\draw[help lines] (0.5,-2) grid (2,2); 
\draw[-{Tee Barb[length=1cm]},line width=.2cm,blue] (0.5,0) - -(2,0) ; 
\end{tikzpicture}
&
 \begin{tikzpicture}[blue,line width=2pt,baseline=.5cm]
 \draw[help lines] (0.5,-2) grid (2,2); 
\draw[-{Latex[length=1cm]},line width=.2cm,blue] (0.5,0) - - (2,0) ; 
 \end{tikzpicture} 
&
\begin{tikzpicture}[blue,line width=2pt,baseline=.5cm]
\draw[help lines] (0.5,-2) grid (2,2); 
\draw[-{Classical TikZ Rightarrow[length=1cm]},line width=.2cm,blue] (0.5,0) - -(2,0) ; 
\end{tikzpicture}
\\ \hline 
Arc Barb & Hooks & Straight Barb & Tee Barb & Latex & Classical TikZ Rightarrow
 \\ \hline

\begin{tikzpicture}[blue,line width=2pt,baseline=.5cm]
  \draw[help lines] (0.5,-2) grid (2,2); 
\draw [-{Straight Barb[length=1cm]},line width=.2cm,blue] (0.5,0) - - (2,0) ; 
  \end{tikzpicture} 
&
\begin{tikzpicture}[blue,line width=2pt,baseline=.5cm]
\draw[help lines] (0.5,-2) grid (2,2); 
\draw [-{Diamond[length=1cm]},line width=.2cm,blue] (0.5,0) - - (2,0) ; 
\end{tikzpicture}   
&
 \begin{tikzpicture}[blue,line width=2pt,baseline=.5cm]
 \draw[help lines] (0.5,-2) grid (2,2); 
 \draw[-{Ellipse[length=1cm]},line width=.2cm,blue] (0.5,0) - - (2,0) ; 
 \end{tikzpicture}
&
 \begin{tikzpicture}[blue,line width=2pt,baseline=.5cm]
 \draw[help lines] (0.5,-2) grid (2,2); 
\draw[-{Kite[length=1cm]},line width=.2cm,blue] (0.5,0) - - (2,0) ; 
 \end{tikzpicture}    

 &
 \begin{tikzpicture}[blue,line width=2pt,baseline=.5cm]
 \draw[help lines] (0.5,-2) grid (2,2); 
 \draw[-{Circle[length=1cm]},line width=.2cm,blue] (0.5,0) - - (2,0) ; 
 \end{tikzpicture} 
&
 \begin{tikzpicture}[blue,line width=2pt,baseline=.5cm]
 \draw[help lines] (0.5,-2) grid (2,2); 
 \draw[-{Computer Modern Rightarrow[length=1cm]},line width=.2cm,blue] (0.5,0) - - (2,0) ; 
 \end{tikzpicture}
\\ \hline 
Straight Barb & Diamond & Ellipse & Kite  & Circle & Computer Modern Rightarrow
 \\ \hline 
\end{tabular}
\bigskip

\begin{tabular}{|c|c|} \hline
 \multicolumn{2}{|c|}{ \BS{tikz} \BS{draw}[-\AC{Arc Barb[length={\color{green} 0cm} {\color{red} 10}]},line width={\color{blue}.1cm},blue] (0,0) - - (3,1);}
 \\ \hline  
 \begin{tikzpicture}[blue,line width=2pt,baseline=.5cm]
  \draw[help lines] (0,-2) grid (3,2); 
 \draw [-{Arc Barb[length=0cm 10]},line width=.1cm,blue] (0,0) - - (3,0) ; 
 \end{tikzpicture}
&
 \begin{tikzpicture}[blue,line width=2pt,baseline=.5cm]
  \draw[help lines] (0,-2) grid (3,2); 
 \draw [-{Arc Barb[length=.5cm 5]},line width=.1cm,blue] (0,0) - - (3,0) ; 
 \end{tikzpicture}
\\ \hline 
[length={\color{green} 0cm} {\color{red} 10}] & [length={\color{green}.5cm} {\color{red} 5 }]
\\ \hline 
{\color{green} 0cm} + {\color{red} 10} x {\color{blue}.1cm} = 1cm & {\color{green}.5cm} + {\color{red} 5 }x {\color{blue}.1cm} = 1cm
\\ \hline 
\end{tabular}

\bigskip

\begin{tabular}{|c|c|c|c|} \hline   
  \multicolumn{2}{|c|}{ \BS{tikz} \BS{draw}[-\AC{Arc Barb[length={\color{green} 0cm} {\color{red} 5 }]},line width={\color{blue}.1cm},blue,double,double distance = {\color{magenta}2 mm}] (0,0) - - (3,1);}
  \\ \hline  
 \begin{tikzpicture}[blue,line width=2pt,baseline=.5cm]
  \draw[help lines] (0,-2) grid (3,2); 
 \draw [-{Arc Barb[length=0cm 5 ]},line width=.1cm,blue,double,double distance =2mm] (0,0) - - (3,0) ; 
 \end{tikzpicture}
&  
 \begin{tikzpicture}[blue,line width=2pt,baseline=.5cm]
  \draw[help lines] (0,-2) grid (3,2); 
 \draw [-{Arc Barb[length=0cm 5 .6 ]},line width=.1cm,blue,double,double distance =2mm ] (0,0) - - (3,0) ; 
 \end{tikzpicture}
\\ \hline  
 [length={\color{green} 0cm}{\color{red} 5 } ] 
 &
 [length={\color{green} 0cm} {\color{red} 5 } {\color{orange} .6} ]
\\ \hline  
{\color{green} 0cm} + {\color{red} 5 } x ({\color{blue}.1cm} + {\color{magenta}2 mm} + {\color{blue}.1cm} ) = 2cm 
&
{\color{green} 0cm} + {\color{red} 5 } x (.6 x {\color{blue}.1cm}+ (1-{\color{orange} .6})({\color{blue}.1cm}+ {\color{magenta}2 mm}+{\color{blue}.1cm}) =  11 mm\\ 
\hline 
\end{tabular} 

 
   
\newpage
\Par{Paramètre width}{Parameter width}

\RRR{16-3-1}


\begin{tabular}{|c|c|c|c|c|} \hline 
 \multicolumn{5}{|c|}{ \BS{tikz} \BS{draw}[-\AC{Arc Barb[\RDD{width}=2cm]},line width=.2cm,blue] (0,0) - - (1,1);}
 \\ \hline
 \begin{tikzpicture}[blue,line width=2pt,baseline=.5cm]
  \draw[help lines] (0,-2) grid (2,2); 
 \draw [-{Arc Barb[width=2cm]},line width=.2cm,blue] (0,0) - - (2,0) ; 
 \end{tikzpicture}
& 
\begin{tikzpicture}[blue,line width=2pt,baseline=.5cm]
\draw[help lines] (0,-2) grid (2,2); 
\draw [-{Hooks[width=2cm]},line width=.2cm,blue] (0,0) - - (2,0) ; 
\end{tikzpicture} 
&
\begin{tikzpicture}[blue,line width=2pt,baseline=.5cm]
\draw[help lines] (0,-2) grid (2,2); 
\draw[-{Straight Barb[width=2cm]},line width=.2cm,blue] (0,0) - -  (2,0) ; 
\end{tikzpicture} 
& 
\begin{tikzpicture}[blue,line width=2pt,baseline=.5cm]
\draw[help lines] (0,-2) grid (2,2); 
\draw[-{Tee Barb[width=2cm]},line width=.2cm,blue] (0,0) - -(2,0) ; 
\end{tikzpicture} 
&
\begin{tikzpicture}[blue,line width=2pt,baseline=.5cm]
\draw[help lines] (0,-2) grid (2,2); 
\draw[-{Classical TikZ Rightarrow[width=2cm]},line width=.2cm,blue] (0,0) - -(2,0) ; 
\end{tikzpicture}
\\ \hline 
Arc Barb & Hooks & Straight Barb & Tee Barb &  Classical TikZ Rightarrow
 \\ \hline

\begin{tikzpicture}[blue,line width=2pt,baseline=.5cm]
  \draw[help lines] (0,-2) grid (2,2); 
\draw [-{Straight Barb[width=2cm]},line width=.2cm,blue] (0,0) - - (2,0) ; 
  \end{tikzpicture} 
&
\begin{tikzpicture}[blue,line width=2pt,baseline=.5cm]
\draw[help lines] (0,-2) grid (2,2); 
\draw [-{Diamond[width=2cm]},line width=.2cm,blue] (0,0) - - (2,0) ; 
\end{tikzpicture}   
&
 \begin{tikzpicture}[blue,line width=2pt,baseline=.5cm]
 \draw[help lines] (0,-2) grid (2,2); 
 \draw[-{Ellipse[width=2cm]},line width=.2cm,blue] (0,0) - - (2,0) ; 
 \end{tikzpicture}
&
 \begin{tikzpicture}[blue,line width=2pt,baseline=.5cm]
 \draw[help lines] (0,-2) grid (2,2); 
\draw[-{Kite[width=2cm]},line width=.2cm,blue] (0,0) - - (2,0) ; 
 \end{tikzpicture}    
&
 \begin{tikzpicture}[blue,line width=2pt,baseline=.5cm]
 \draw[help lines] (0,-2) grid (2,2); 
 \draw[-{Computer Modern Rightarrow[width=2cm]},line width=.2cm,blue] (0,0) - - (2,0) ; 
 \end{tikzpicture}
\\ \hline 
Straight Barb & Diamond & Ellipse & Kite &  Computer Modern Rightarrow
 \\ \hline    
\end{tabular}
 
\bigskip

\begin{tabular}{|c|c|} \hline
 \multicolumn{2}{|c|}{ \BS{tikz} \BS{draw}[-\AC{Arc Barb[width={\color{green} 0cm} {\color{red} 10}]},line width={\color{blue}.1cm},blue] (0,0) - - (3,1);}
 \\ \hline  
 \begin{tikzpicture}[blue,line width=2pt,baseline=.5cm]
  \draw[help lines,step=.5cm] (0,-1) grid (3,1); 
 \draw [-{Arc Barb[width=0cm 10]},line width=.1cm,blue] (0,0) - - (3,0) ; 
 \end{tikzpicture}
&
 \begin{tikzpicture}[blue,line width=2pt,baseline=.5cm]
  \draw[help lines,step=.5cm] (0,-1) grid (3,1); 
 \draw [-{Arc Barb[width=.5cm 5]},line width=.1cm,blue] (0,0) - - (3,0) ; 
 \end{tikzpicture}
\\ \hline 
[width={\color{green} 0cm} {\color{red} 10}] & [width={\color{green}.5cm} {\color{red} 5 }]
\\ \hline 
{\color{green} 0cm} + {\color{red} 10} x {\color{blue}.1cm} = 1cm & {\color{green}.5cm} + {\color{red} 5 }x {\color{blue}.1cm} = 1cm
\\ \hline 
\end{tabular}

\bigskip

\begin{tabular}{|c|c|c|c|} \hline   
  \multicolumn{2}{|c|}{ \BS{tikz} \BS{draw}[-\AC{Arc Barb[width={\color{green} 0cm} {\color{red} 5 }]},line width={\color{blue}.1cm},blue,double,double distance = {\color{magenta}2 mm}] (0,0) - - (3,1);}
  \\ \hline  
 \begin{tikzpicture}[blue,line width=2pt,baseline=.5cm]
  \draw[help lines,step=.5cm] (0,-1) grid (3,1); 
 \draw [-{Arc Barb[width=0cm 5 ]},line width=.1cm,blue,double,double distance =2mm] (0,0) - - (3,0) ; 
 \end{tikzpicture}
&  
 \begin{tikzpicture}[blue,line width=2pt,baseline=.5cm]
  \draw[help lines,step=.5cm] (0,-1) grid (3,1); 
 \draw [-{Arc Barb[width=0cm 5 .6 ]},line width=.1cm,blue,double,double distance =2mm ] (0,0) - - (3,0) ; 
 \end{tikzpicture}
\\ \hline  
 [width={\color{green} 0cm}{\color{red} 5 } ] 
 &
 [width={\color{green} 0cm} {\color{red} 5 } {\color{orange} .6} ]
\\ \hline  
{\color{green} 0cm} + {\color{red} 5 } x ({\color{blue}.1cm} + {\color{magenta}2 mm} + {\color{blue}.1cm} ) = 2cm 
&
{\color{green} 0cm} + {\color{red} 5 } x (.6 x {\color{blue}.1cm}+ (1-{\color{orange} .6})({\color{blue}.1cm}+ {\color{magenta}2 mm}+{\color{blue}=.1cm}) =  11 mm\\ 
\hline 
\end{tabular}  
 
\bigskip

\begin{tabular}{|c|c|} \hline
 \multicolumn{2}{|c|}{ \BS{tikz} \BS{draw}[-\AC{Arc Barb[length={\color{blue}1cm},width={\color{green} 0cm} {\color{red} 1.5}]},line width'=.1cm,blue] (0,0) - - (3,1);}
 \\ \hline  
 \begin{tikzpicture}[blue,line width=2pt,baseline=.5cm]
  \draw[help lines,step=.5cm] (0,-1) grid (3,1); 
 \draw [-{Arc Barb[length=1cm,width'=0cm 1.5]},line width=.1cm,blue] (0,0) - - (3,0) ; 
 \end{tikzpicture}
&
 \begin{tikzpicture}[blue,line width=2pt,baseline=.5cm]
  \draw[help lines,step=.5cm] (0,-1) grid (3,1); 
 \draw [-{Arc Barb[length=1cm,width'=.5cm .5]},line width=.1cm,blue] (0,0) - - (3,0) ; 
 \end{tikzpicture}
\\ \hline 
[width'={\color{green} 0cm} {\color{red} 1.5}] & [width'={\color{green}.5cm} {\color{red} .5 }]
\\ \hline 
{\color{green} 0cm} + {\color{red} 1.5} x {\color{blue} 1cm} = 1.5cm & 
{\color{green}.5cm} + {\color{red} .5 }x {\color{blue}1cm} = 1cm
\\ \hline 
\end{tabular}




\bigskip

\begin{tabular}{|c|c|c|c|} \hline   
  \multicolumn{2}{|c|}{ \BS{tikz} \BS{draw}[-\AC{Arc Barb[length={\color{blue}1cm},width'={\color{green} 0cm} {\color{red} 1.5 }]},line width=.1cm,blue,double,double distance = {\color{magenta}2 mm}] (0,0) - - (3,1);}
  \\ \hline  
 \begin{tikzpicture}[blue,line width=2pt,baseline=.5cm]
  \draw[help lines,step=.5cm] (0,-1) grid (3,1); 
 \draw [-{Arc Barb[length=1cm,width'=0cm 1.5 ]},line width=.1cm,blue,double,double distance =2mm] (0,0) - - (3,0) ; 
 \end{tikzpicture}
&  
 \begin{tikzpicture}[blue,line width=2pt,baseline=.5cm]
  \draw[help lines,step=.5cm] (0,-1) grid (3,1); 
 \draw [-{Arc Barb[length=1cm,width'=0cm 1.5 .6 ]},line width=.1cm,blue,double,double distance =2mm ] (0,0) - - (3,0) ; 
 \end{tikzpicture}
\\ \hline  
 [width'={\color{green} 0cm} {\color{red} 1.5 } ] 
 &
 [width'={\color{green} 0cm} {\color{red} 1.5 } {\color{orange} .6} ]
\\ \hline  
{\color{green} 0cm} + \color{red} 1.5  x {\color{blue}1cm}  = 1.5cm 
&
{\color{green} 0cm} + {\color{red} 1.5 } x (.6 x {\color{blue} 1cm}+ (1-{\color{orange} .6})({\color{blue} 1cm}+ {\color{magenta}2 mm}+{\color{blue} 1cm}) =  11 mm\\ 
\hline 
\end{tabular} 

% % A VOIR
%  
%\bigskip
%
% \begin{tikzpicture}[blue,line width=2pt,baseline=.5cm]
%  \draw[help lines,step=.5cm] (0,-1) grid (3,1); 
% \draw [-{Arc Barb[length=1cm,width'=0cm 2 0 ]},line width=.1cm,blue,double,double distance =2mm ] (0,0) - - (3,0) ; 
% \end{tikzpicture}
% \begin{tikzpicture}[blue,line width=2pt,baseline=.5cm]
%  \draw[help lines,step=.5cm] (0,-1) grid (3,1); 
% \draw [-{Arc Barb[length=1cm,width'=0cm 2 .5 ]},line width=.1cm,blue,double,double distance =2mm ] (0,0) - - (3,0) ; 
% \end{tikzpicture}
% \begin{tikzpicture}[blue,line width=2pt,baseline=.5cm]
%  \draw[help lines,step=.5cm] (0,-2) grid (3,2); 
% \draw [-{Arc Barb[length=1cm,width'=0cm 1 5 ]},line width=.1cm,blue,double,double distance =2mm ] (0,0) - - (3,0) ; 
% \end{tikzpicture}
 
 \bigskip
 \Par{Paramètre inset}{Parameter inset}
 
 \RRR{16-3-1}
 
 \begin{tabular}{|c|c|c|} \hline 
  \multicolumn{3}{|c|}{ \BS{tikz} \BS{draw}[-\AC{Tee Barb[\RDD{inset}=0pt]},line width=.2cm,blue] (0,0) - - (1,1);}
  \\ \hline
  \begin{tikzpicture}[blue,line width=2pt,baseline=.5cm]
   \draw[help lines] (0,-1) grid (2,1); 
  \draw [-{Tee Barb[inset=0pt]},line width=.2cm,blue] (0,0) - - (2,0) ; 
  \end{tikzpicture}
 &
  \begin{tikzpicture}[blue,line width=2pt,baseline=.5cm]
 \draw[help lines] (0,-1) grid (2,1); 
  \draw [-{Kite[inset=0pt]},line width=.2cm,blue] (0,0) - - (2,0) ; 
  \end{tikzpicture}
 &
  \begin{tikzpicture}[blue,line width=2pt,baseline=.5cm]
 \draw[help lines] (0,-1) grid (2,1);
  \draw [-{Stealth[inset=0pt]},line width=.2cm,blue] (0,0) - - (2,0) ; 
  \end{tikzpicture}
  \\ \hline
 Tee Barb[inset=0pt] & Kite[inset=0pt] & Stealth[inset=0pt]  
  \\ \hline 
   \begin{tikzpicture}[blue,line width=2pt,baseline=.5cm]
    \draw[help lines] (0,-1) grid (2,1); 
   \draw [-{Tee Barb[inset=1cm]},line width=.2cm,blue] (0,0) - - (2,0) ; 
   \end{tikzpicture}
 &
  \begin{tikzpicture}[blue,line width=2pt,baseline=.5cm]
   \draw[help lines] (0,-1) grid (2,1); 
  \draw [-{Kite[inset=1cm]},line width=.2cm,blue] (0,0) - - (2,0) ; 
  \end{tikzpicture}
 &
  \begin{tikzpicture}[blue,line width=2pt,baseline=.5cm]
   \draw[help lines] (0,-1) grid (2,1); 
  \draw [-{Stealth[inset=.5cm]},line width=.2cm,blue] (0,0) - - (2,0) ; 
  \end{tikzpicture}
 \\ \hline 
 Tee Barb[inset=1cm] & Kite[inset=1cm] & Stealth[inset=.5cm] 
  \\ \hline     
 \end{tabular}
 
 \bigskip
 
 \begin{tabular}{|c|c|c|c|} \hline 
  \multicolumn{4}{|c|}{ \BS{tikz} \BS{draw}[-\AC{Fast Round[inset=1cm]},line width=.2cm,blue] (0,0) - - (1,1);}
  \\ \hline
 \begin{tikzpicture}[blue,line width=2pt,baseline=.5cm]
 \draw[help lines] (0,-1) grid (3,1); 
 \draw [-{Fast Round[inset=1cm]},line width=.5cm,blue] (0,0) - - (3,0) ; 
 \end{tikzpicture}
 &
 \begin{tikzpicture}[blue,line width=2pt,baseline=.5cm]
 \draw[help lines] (0,-1) grid (3,1); 
 \draw [-{Fast Round[inset=2cm]},line width=.5cm,blue] (0,0) - - (3,0) ; 
 \end{tikzpicture}
 &
 \begin{tikzpicture}[blue,line width=2pt,baseline=.5cm]
 \draw[help lines] (0,-1) grid (3,1); 
 \draw [-{Fast Triangle[inset=1cm]},line width=.5cm,blue] (0,0) - - (3,0) ; 
 \end{tikzpicture}
 &
 \begin{tikzpicture}[blue,line width=2pt,baseline=.5cm]
 \draw[help lines] (0,-1) grid (3,1); 
 \draw [-{Fast Triangle[inset=2cm]},line width=.5cm,blue] (0,0) - - (3,0) ; 
 \end{tikzpicture}
 \\ \hline
 Fast Round[inset=1cm] & Fast Round[inset=2cm] &  Fast Triangle[inset=1cm] & Fast Triangle[inset=2cm]
 \\ \hline    
 \end{tabular}
 
 \bigskip
 
 \begin{tabular}{|c|c|c|c|} \hline
 \begin{tikzpicture}[blue,line width=2pt,baseline=.5cm]
 \draw[help lines] (0,-1) grid (3,1); 
 \draw [-{Kite[inset=1cm 1]},line width=.2cm,blue] (0,0) - - (3,0) ; 
 \end{tikzpicture}  
 %\tikz \draw[-{Kite[inset=1cm 1]},line width=.2cm,blue] (0,0) - - (3,1) ;
 &
 \begin{tikzpicture}[blue,line width=2pt,baseline=.5cm]
 \draw[help lines] (0,-1) grid (3,1); 
 \draw [-{Kite[inset=1cm 2]},line width=.2cm,blue] (0,0) - - (3,0) ; 
 \end{tikzpicture}   
 &
 \begin{tikzpicture}[blue,line width=2pt,baseline=.5cm]
 \draw[help lines] (0,-1) grid (3,1); 
 \draw [-{Kite[inset=1cm 4]},line width=.2cm,blue] (0,0) - - (3,0) ; 
 \end{tikzpicture}   
 &
 \begin{tikzpicture}[blue,line width=2pt,baseline=.5cm]
 \draw[help lines] (0,-1) grid (3,1); 
 \draw [-{Kite[inset=1cm .2]},line width=.2cm,blue] (0,0) - - (3,0) ; 
 \end{tikzpicture} 
 \\ \hline inset=1cm 1 & inset=1cm 2 &  inset=1cm 4 &  inset=1cm .2 \\ 
 \hline 
 \end{tabular} 
 
 
 \bigskip
 
 \begin{tabular}{|c|c|c|c|} \hline
 \begin{tikzpicture}[blue,line width=2pt,baseline=.5cm]
 \draw[help lines] (0,-1) grid (3,1); 
 \draw [-{Kite[inset=0cm 1]},line width=.2cm,blue] (0,0) - - (3,0) ; 
 \end{tikzpicture}  
 &
 \begin{tikzpicture}[blue,line width=2pt,baseline=.5cm]
 \draw[help lines] (0,-1) grid (3,1); 
 \draw [-{Kite[inset=0cm 2]},line width=.2cm,blue] (0,0) - - (3,0) ; 
 \end{tikzpicture}   
 &
 \begin{tikzpicture}[blue,line width=2pt,baseline=.5cm]
 \draw[help lines] (0,-1) grid (3,1); 
 \draw [-{Kite[inset=0cm 4]},line width=.2cm,blue] (0,0) - - (3,0) ; 
 \end{tikzpicture}   
 &
 \begin{tikzpicture}[blue,line width=2pt,baseline=.5cm]
 \draw[help lines] (0,-1) grid (3,1); 
 \draw [-{Kite[inset=1cm .2]},line width=.2cm,blue] (0,0) - - (3,0) ; 
 \end{tikzpicture} 
 \\ \hline inset=0cm 1 & inset=0cm 2 &  inset=0cm 4 &  inset=0cm .2 \\ 
 \hline 
 \end{tabular} 
 
 
 \bigskip
 
 \begin{tabular}{|c|c|c|c|} \hline
 \begin{tikzpicture}[blue,line width=2pt,baseline=.5cm]
 \draw[help lines] (0,-1) grid (3,1); 
 \draw[-{Kite[inset=1cm .2]},line width=.1cm,blue,double,double distance =2mm] (0,0) - - (3,0) ; 
 \end{tikzpicture}  
 &
 \begin{tikzpicture}[blue,line width=2pt,baseline=.5cm]
 \draw[help lines] (0,-1) grid (3,1); 
 \draw[-{Kite[inset=1cm .2 2]},line width=.1cm,blue,double,double distance =2mm] (0,0) - - (3,0) ; 
 \end{tikzpicture}   
 &
 \begin{tikzpicture}[blue,line width=2pt,baseline=.5cm]
 \draw[help lines] (0,-1) grid (3,1); 
 \draw[-{Kite[inset=1cm .2 10]},line width=.1cm,blue,double,double distance =2mm] (0,0) - - (3,0) ; 
 \end{tikzpicture}    
 &
 \begin{tikzpicture}[blue,line width=2pt,baseline=.5cm]
 \draw[help lines] (0,-1) grid (3,1); 
 \draw[-{Kite[inset=1cm 2 .5]},line width=.1cm,blue,double,double distance =2mm] (0,0) - - (3,0) ; 
 \end{tikzpicture} 
 \\ \hline
 inset=0cm .2 & inset=0cm .2 2 &  inset=0cm .2 10 &  inset=0cm 2 .5 \\ 
 \hline 
 \end{tabular} 
 \bigskip
 
 
 \begin{tabular}{|c|c|c|c|} \hline
 \begin{tikzpicture}[blue,line width=2pt,baseline=.5cm]
 \draw[help lines] (0,-1) grid (3,1); 
 \draw[-{Kite[inset=0cm .2]},line width=.1cm,blue,double,double distance =2mm] (0,0) - - (3,0) ; 
 \end{tikzpicture}  
 &
 \begin{tikzpicture}[blue,line width=2pt,baseline=.5cm]
 \draw[help lines] (0,-1) grid (3,1); 
 \draw[-{Kite[inset=0cm .2 2]},line width=.1cm,blue,double,double distance =2mm] (0,0) - - (3,0) ; 
 \end{tikzpicture}   
 &
 \begin{tikzpicture}[blue,line width=2pt,baseline=.5cm]
 \draw[help lines] (0,-1) grid (3,1); 
 \draw[-{Kite[inset=0cm .2 10]},line width=.1cm,blue,double,double distance =2mm] (0,0) - - (3,0) ; 
 \end{tikzpicture}    
 &
 \begin{tikzpicture}[blue,line width=2pt,baseline=.5cm]
 \draw[help lines] (0,-1) grid (3,1); 
 \draw[-{Kite[inset=0cm 2 .5]},line width=.1cm,blue,double,double distance =2mm] (0,0) - - (3,0) ; 
 \end{tikzpicture} 
 \\ \hline
 inset=0cm .2 & inset=0cm .2 2 &  inset=0cm .2 10 &  inset=0cm 2 .5 \\ 
 \hline 
 \end{tabular}
 
\bigskip

\Par{Paramètre angle}{Parameter angle}

\RRR{16-3-1}


\begin{tabular}{|c|c|c|c|c|} \hline 
 \multicolumn{5}{|c|}{ \BS{tikz} \BS{draw}[-\AC{Straight Barb[\RDD{angle}=60:.5cm 1]},line width=.2cm,blue] (0,0) - - (1,1);}
 \\ \hline
 
\tikz \draw[-{Straight Barb[angle=60:.5cm 1]},line width=1pt,blue] (0,0) - - (1,1) ;
&  
\tikz \draw[-{Straight Barb[angle=60:.5cm 5]},line width=1pt,blue] (0,0) - - (1,1) ;
&  
\tikz \draw[-{Straight Barb[angle=60:.5cm 20]},line width=1pt,blue] (0,0) - - (1,1) ;
&  
\tikz \draw[-{Straight Barb[angle=60:.5cm 5]},line width=3pt,blue] (0,0) - - (1,1) ;
&  
\tikz \draw[-{Straight Barb[angle=90:.5cm 5]},line width=3pt,blue] (0,0) - - (1,1) ;
\\ 
\hline 
[angle=60:.5cm 1] & [angle=60:.5cm 1] & [angle=60:.5cm 20] & [angle=60:.5cm 5] & [angle=90:.5cm 5] \\ 
\hline 
\end{tabular} 

\bigskip

\begin{tabular}{|c|c|c|c|c|} \hline 
 \multicolumn{5}{|c|}{ \BS{tikz} \BS{draw}[-\AC{Triangle[angle=60:.5cm 1]},line width=.2cm,blue] (0,0) - - (1,1);}
 \\ \hline
 
\tikz \draw[-{Triangle[angle=60:.5cm 1]},line width=1pt,blue] (0,0) - - (1,1) ;
&  
\tikz \draw[-{Triangle[angle=60:.5cm 5]},line width=1pt,blue] (0,0) - - (1,1) ;
&  
\tikz \draw[-{Triangle[angle=60:.5cm 20]},line width=1pt,blue] (0,0) - - (1,1) ;
&  
\tikz \draw[-{Triangle[angle=60:.5cm 5]},line width=3pt,blue] (0,0) - - (1,1) ;
&  
\tikz \draw[-{Triangle[angle=90:.5cm 5]},line width=3pt,blue] (0,0) - - (1,1) ;
\\ 
\hline 
[angle=60:.5cm 1] & [angle=60:.5cm 1] & [angle=60:.5cm 20] & [angle=60:.5cm 5] & [angle=90:.5cm 5] \\ 
\hline 
\end{tabular} 
 
\Par{Paramètre scale}{Parameter scale}
\RRR{16-3-2}


\begin{tabular}{|c|c|c|} \hline 
  \multicolumn{3}{|c|}{\BS{tikz} \BS{draw}[-\AC{Arc Barb[\RDD{scale}=4]},li ne width=.1cm,blue] (0,0) - - (3,0) ; }
  \\ \hline  
 \begin{tikzpicture}[blue,line width=2pt,baseline=.5cm]
  \draw[help lines,step=.5cm] (0,-1) grid (3,1); 
 \draw [-{Arc Barb[scale=4]},line width=.1cm,blue] (0,0) - - (3,0) ; 
 \end{tikzpicture}
&  
 \begin{tikzpicture}[blue,line width=2pt,baseline=.5cm]
  \draw[help lines,step=.5cm] (0,-1) grid (3,1); 
 \draw [-{Arc Barb[scale length=4]},line width=.1cm,blue] (0,0) - - (3,0) ; 
 \end{tikzpicture} 
& 
 \begin{tikzpicture}[blue,line width=2pt,baseline=.5cm]
  \draw[help lines,step=.5cm] (0,-1) grid (3,1); 
 \draw [-{Arc Barb[scale width=4]},line width=.1cm,blue] (0,0) - - (3,0) ; 
 \end{tikzpicture} 
\\ \hline  
\RDD{scale}=4 & \RDD{scale length}=4 & \RDD{scale width}=4 \\ 
\hline 
\end{tabular}   

\bigskip

\Par{Paramètre arc}{Parameter arc}
\RRR{16-3-3}

\begin{tabular}{|c|c|c|c|} \hline 
 \multicolumn{4}{|c|}{ \BS{tikz} \BS{draw}[-\AC{Arc Barb[\RDD{arc}=270]},line width=.2cm,blue] (0,0) - - (3,1);}
 \\ \hline
\tikz \draw [-{Arc Barb[arc=270]},line width=.2cm,blue] (0,0) - - (3,1) ;
 & 
 
 \tikz \draw [-{Arc Barb[arc=360]},line width=.2cm,blue] (0,0) - - (3,1) ;
  &  
\tikz \draw [-{Hooks[arc=270]},line width=.2cm,blue] (0,0) - - (3,1) ;
  &  
\tikz \draw [-{Hooks[arc=360]},line width=.2cm,blue] (0,0) - - (3,1) ;
 \\ \hline 
Arc Barb[arc=270] & Arc Barb[arc=360] & Hooks[arc=270] & Hooks[arc=360] 
 \\ \hline
\end{tabular}

\bigskip

\Par{Paramètre slant}{Parameter slant}
\RRR{16-3-4}

\begin{tabular}{|c|c|c|c|c|} \hline 
 \multicolumn{5}{|c|}{ \BS{tikz} \BS{draw}[-\AC{Arc Barb[\RDD{slant}=.3]},line width=.2cm,blue] (0,0) - - (1,1);}
 \\ \hline
\tikz \draw [-{Arc Barb[slant=0]},line width=.2cm,blue] (0,0) - - (1,1) ;
&
\tikz \draw [-{Arc Barb[slant=.3]},line width=.2cm,blue] (0,0) - - (1,1) ;
&
\tikz \draw [-{Arc Barb[slant=.5]},line width=.2cm,blue] (0,0) - - (1,1) ;
&
\tikz \draw [-{Arc Barb[slant=.8]},line width=.2cm,blue] (0,0) - - (1,1) ;
&
\tikz \draw [-{Arc Barb[slant=1
]},line width=.2cm,blue] (0,0) - - (1,1) ;
 \\ \hline  
slant=0 & slant=0.3 & slant=0.5 & slant=0.8 & slant=1 
 \\ \hline    
\end{tabular}

\bigskip

\begin{tabular}{|c|c|c|c|c|} \hline 
 \multicolumn{5}{|c|}{ \BS{tikz} \BS{draw}[-\AC{Arc Barb[slant=.5]},line width=.2cm,blue] (0,0) - - (1,1);}
 \\ \hline
\tikz \draw [-{Arc Barb[slant=.5]},line width=.2cm,blue] (0,0) - - (1,1) ;
 &
\tikz \draw[-{Bracket[slant=.5]},line width=.2cm,blue] (0,0) - - (1,1) ;
 &  
\tikz \draw [-{Hooks[slant=.5]},line width=.2cm,blue] (0,0) - - (1,1) ;
  &  
\tikz \draw[-{Parenthesis[slant=.5]},line width=.2cm,blue] (0,0) - - (1,1) ;
&
\tikz \draw[-{Classical TikZ Rightarrow[slant=.5]},line width=.2cm,blue] (0,0) - - (1,1) ; 
 \\ \hline 
Arc Barb & Bracket & Hooks & Parenthesis & Classical TikZ Rightarrow 
 \\ \hline
\tikz \draw [-{Straight Barb[slant=.5]},line width=.2cm,blue] (0,0) - - (1,1) ;
&  
\tikz \draw [-{Tee Barb[slant=.5]},line width=.2cm,blue] (0,0) - - (1,1) ;
&
\tikz \draw[-{Circle[slant=.5]},line width=.2cm,blue] (0,0) - - (1,1) ;
&  
\tikz \draw[-{Diamond[slant=.5]},line width=.2cm,blue] (0,0) - - (1,1) ;
&
\tikz \draw[-{Ellipse[slant=.5]},line width=.2cm,blue] (0,0) - - (1,1) ;
 \\ \hline 
Straight Barb & Tee Barb & Circle  & Diamond & Ellipse 
\\ \hline
 
\tikz \draw[-{Kite[slant=.5]},line width=.2cm,blue] (0,0) - - (1,1) ;
&
\tikz \draw[-{Latex[slant=.5]},line width=.2cm,blue] (0,0) - - (1,1) ;
&
\tikz \draw[-{Rectangle[slant=.5]},line width=.2cm,blue] (0,0) - - (1,1) ;
&
\tikz \draw[-{Square[slant=.5]},line width=.2cm,blue] (0,0) - - (1,1) ;
&
\tikz \draw[-{Stealth[slant=.5]},line width=.2cm,blue] (0,0) - - (1,1) ;
\\ \hline 
Kite & Latex & Rectangle & Square & Stealth 
\\ \hline

\tikz \draw[-{Turned Square[slant=.5]},line width=.2cm,blue] (0,0) - - (1,1) ;
&
\tikz \draw[-{Fast Round[slant=.5]},line width=.2cm,blue] (0,0) - - (1,1) ;
&
\tikz \draw[-{Fast Triangle[slant=.5]},line width=.2cm,blue] (0,0) - - (1,1) ;
&
\tikz \draw[-{Round Cap[slant=.5]},line width=.2cm,blue] (0,0) - - (1,1) ;
&
\tikz \draw[-{Triangle Cap[slant=.5]},line width=.2cm,blue] (0,0) - - (1,1) ;
\\ \hline 
Turned Square & Fast Round & Fast Triangle & Round Cap & Triangle Cap 
\\ \hline    
\end{tabular}

\Par{Paramètre reversed}{Parameter reversed}
\RRR{16-3-5}

\begin{tabular}{|c|c|c|c|} \hline 
 \multicolumn{4}{|c|}{ \BS{tikz} \BS{draw}[-\AC{Arc Barb[\RDD{reversed}},line width=.2cm,blue] (0,0) - - (2,1) ;}
 \\ \hline
\tikz \draw [-{Arc Barb[reversed]},line width=.2cm,blue] (0,0) - - (2,1) ;

 &
 \tikz \draw[-{Bracket[reversed]},line width=.2cm,blue] (0,0) - - (2,1) ;
  &  
 \tikz \draw [-{Hooks[reversed]},line width=.2cm,blue] (0,0) - - (2,1) ;

&
\tikz \draw[-{Classical TikZ Rightarrow[reversed]},line width=.2cm,blue] (0,0) - - (2,1) ; 
 \\ \hline 
Arc Barb & Bracket & Hooks  & Classical TikZ Rightarrow
 \\ \hline
\tikz \draw [-{Straight Barb[reversed]},line width=.2cm,blue] (0,0) - - (2,1) ;
 &  
\tikz \draw [-{Tee Barb[reversed]},line width=.2cm,blue] (0,0) - - (2,1) ;
  &  
\tikz \draw[-{Parenthesis[reversed]},line width=.2cm,blue] (0,0) - - (2,1) ;
 &
\tikz \draw[-{Computer Modern Rightarrow[reversed]},line width=.2cm,blue] (0,0) - - (2,1) ;
 \\ \hline 
Straight Barb & Tee Barb & Parenthesis &  Computer Modern Rightarrow
 \\ \hline
\end{tabular}


\bigskip


\begin{tabular}{|c|c|c|c|} \hline 
 \multicolumn{4}{|c|}{ \BS{tikz} \BS{draw}[-\AC{Fast Round[reversed]},line width=.5cm,blue] (0,0) - - (2,1);}
 \\ \hline 
 \tikz \draw[-{Fast Round[reversed]},line width=.5cm,blue] (0,0) - - (2,1) ;
  &  
\tikz \draw[-{Fast Triangle[reversed]},line width=.5cm,blue] (0,0) - - (2,1) ;
& 
\tikz \draw[-{Round Cap[reversed]},line width=.5cm,blue] (0,0) - - (2,1) ;
& 
\tikz \draw[-{Triangle Cap[reversed]},line width=.5cm,blue] (0,0) - - (2,1) ;
\\ \hline
 Fast Round  & Fast Triangle & Round Cap & Triangle Cap  
 \\ \hline    
\end{tabular}

\newpage

\Par{Paramètre left}{Parameter left}
\RRR{16-3-5}

\begin{tabular}{|c|c|c|c|c|c|} \hline 
 \multicolumn{6}{|c|}{ \BS{tikz} \BS{draw}[-\AC{Arc Barb[\RDD{left}]},line width=.2cm,blue] (0,0) - - (1.5,1);}
 \\ \hline
\tikz \draw [-{Arc Barb[left]},line width=.2cm,blue] (0,0) - - (1.5,1) ;
 &
 \tikz \draw[-{Bracket[left]},line width=.2cm,blue] (0,0) - - (1.5,1) ;
 &  
\tikz \draw [-{Hooks[left]},line width=.2cm,blue] (0,0) - - (1.5,1) ;

  &  
\tikz \draw[-{Parenthesis[left]},line width=.2cm,blue] (0,0) - - (1.5,1) ;
&
\tikz \draw[-{Classical TikZ Rightarrow[left]},line width=.2cm,blue] (0,0) - - (1.5,1) ;
&
\tikz \draw[-{Triangle[left]},line width=.2cm,blue] (0,0) - - (1.5,1) ; 
 \\ \hline 
Arc Barb & Bracket & Hooks & Parenthesis & Classical TikZ Rightarrow  & Triangle  
 \\ \hline

\tikz \draw [-{Straight Barb[left]},line width=.2cm,blue] (0,0) - - (1.5,1) ;
&  
\tikz \draw [-{Tee Barb[left]},line width=.2cm,blue] (0,0) - - (1.5,1) ;
&
\tikz \draw[-{Circle[left]},line width=.2cm,blue] (0,0) - - (1.5,1) ;
&  
\tikz \draw[-{Diamond[left]},line width=.2cm,blue] (0,0) - - (1.5,1) ;
&
\tikz \draw[-{Ellipse[left]},line width=.2cm,blue] (0,0) - - (1.5,1) ;
&
\tikz \draw[-{Turned Square[left]},line width=.2cm,blue] (0,0) - - (1.5,1) ;
\\ \hline 
Straight Barb & Tee Barb & Circle  & Diamond & Ellipse & Turned Square
 \\ \hline
 
\tikz \draw[-{Kite[left]},line width=.2cm,blue] (0,0) - - (1.5,1) ;  
&
\tikz \draw[-{Latex[left]},line width=.2cm,blue] (0,0) - - (1.5,1) ;  
& 
\tikz \draw[-{Rectangle[left]},line width=.2cm,blue] (0,0) - - (1.5,1) ;
&
\tikz \draw[-{Square[left]},line width=.2cm,blue] (0,0) - - (1.5,1) ;
&
\tikz \draw[-{Stealth[left]},line width=.2cm,blue] (0,0) - - (1.5,1) ;
&
\tikz \draw[-{Rays[left]},line width=.2cm,blue] (0,0) - - (1.5,1) ; 
\\ \hline
Kite & Latex & Rectangle & Square & Stealth & Rays
\\ \hline
\end{tabular}

\bigskip

\Par{Paramètre right}{Parameter right}
\RRR{16-3-5}


\begin{tabular}{|c|c|c|c|c|c|} \hline 
 \multicolumn{6}{|c|}{ \BS{tikz} \BS{draw}[-\AC{Arc Barb[\RDD{right}]},line width=.2cm,blue] (0,0) - - (1.5,1);}
 \\ \hline
\tikz \draw [-{Arc Barb[right]},line width=.2cm,blue] (0,0) - - (1.5,1) ;
 &
 \tikz \draw[-{Bracket[right]},line width=.2cm,blue] (0,0) - - (1.5,1) ;
 &  
\tikz \draw [-{Hooks[right]},line width=.2cm,blue] (0,0) - - (1.5,1) ;

  &  
\tikz \draw[-{Parenthesis[right]},line width=.2cm,blue] (0,0) - - (1.5,1) ;
&
\tikz \draw[-{Classical TikZ Rightarrow[right]},line width=.2cm,blue] (0,0) - - (1.5,1) ;
&
\tikz \draw[-{Triangle[right]},line width=.2cm,blue] (0,0) - - (1.5,1) ; 
 \\ \hline 
Arc Barb & Bracket & Hooks & Parenthesis & Classical TikZ Rightarrow  & Triangle  
 \\ \hline

\tikz \draw [-{Straight Barb[right]},line width=.2cm,blue] (0,0) - - (1.5,1) ;
&  
\tikz \draw [-{Tee Barb[right]},line width=.2cm,blue] (0,0) - - (1.5,1) ;
&
\tikz \draw[-{Circle[right]},line width=.2cm,blue] (0,0) - - (1.5,1) ;
&  
\tikz \draw[-{Diamond[right]},line width=.2cm,blue] (0,0) - - (1.5,1) ;
&
\tikz \draw[-{Ellipse[right]},line width=.2cm,blue] (0,0) - - (1.5,1) ;
&
\tikz \draw[-{Turned Square[right]},line width=.2cm,blue] (0,0) - - (1.5,1) ;
\\ \hline 
Straight Barb & Tee Barb & Circle  & Diamond & Ellipse & Turned Square
 \\ \hline
 
\tikz \draw[-{Kite[right]},line width=.2cm,blue] (0,0) - - (1.5,1) ;  
&
\tikz \draw[-{Latex[right]},line width=.2cm,blue] (0,0) - - (1.5,1) ;  
& 
\tikz \draw[-{Rectangle[right]},line width=.2cm,blue] (0,0) - - (1.5,1) ;
&
\tikz \draw[-{Square[right]},line width=.2cm,blue] (0,0) - - (1.5,1) ;
&
\tikz \draw[-{Stealth[right]},line width=.2cm,blue] (0,0) - - (1.5,1) ;
&
\tikz \draw[-{Rays[right]},line width=.2cm,blue] (0,0) - - (1.5,1) ; 
\\ \hline
Kite & Latex & Rectangle & Square & Stealth & Rays
\\ \hline
\end{tabular}

\Par{Paramètre harpoon}{Parameter harpoon}
\RRR{16-3-5}

\begin{tabular}{|c|c|c|c|c|c|c|} \hline 
 \multicolumn{7}{|c|}{ \BS{tikz} \BS{draw}[-\AC{Arc Barb[\RDD{harpoon}]},line width=.2cm,blue] (0,0) - - (1,1);}
 \\ \hline
\tikz \draw [-{Arc Barb[harpoon]},line width=.2cm,blue] (0,0) - - (1,1) ;
 &
\tikz \draw[-{Bracket[harpoon]},line width=.2cm,blue] (0,0) - - (1,1) ;
 &  
\tikz \draw [-{Hooks[harpoon]},line width=.2cm,blue] (0,0) - - (1,1) ;
  &  
\tikz \draw[-{Parenthesis[harpoon]},line width=.2cm,blue] (0,0) - - (1,1) ;
&
\tikz \draw[-{Classical TikZ Rightarrow[harpoon]},line width=.2cm,blue] (0,0) - - (1,1) ;
&
\tikz \draw [-{Straight Barb[harpoon]},line width=.2cm,blue] (0,0) - - (1,1) ;
 &  
\tikz \draw [-{Tee Barb[harpoon]},line width=.2cm,blue] (0,0) - - (1,1) ; 
 \\ \hline 
Arc Barb & Bracket& Hooks & Parenthesis & Classical TikZ Rightarrow & Straight Barb & Tee Barb
\\ \hline
\end{tabular}

\bigskip
\begin{tabular}{|c|c|c|c|c|c|c|} \hline 
 \multicolumn{7}{|c|}{ \BS{tikz} \BS{draw}[-\AC{Arc Barb[harpoon,\RDD{swap}]},line width=.2cm,blue] (0,0) - - (1,1);}
 \\ \hline
\tikz \draw [-{Arc Barb[harpoon,swap]},line width=.2cm,blue] (0,0) - - (1,1) ;
 &
\tikz \draw[-{Bracket[harpoon,swap]},line width=.2cm,blue] (0,0) - - (1,1) ;
 &  
\tikz \draw [-{Hooks[harpoon,swap]},line width=.2cm,blue] (0,0) - - (1,1) ;
  &  
\tikz \draw[-{Parenthesis[harpoon,swap]},line width=.2cm,blue] (0,0) - - (1,1) ;
&
\tikz \draw[-{Classical TikZ Rightarrow[harpoon,swap]},line width=.2cm,blue] (0,0) - - (1,1) ;
&
\tikz \draw [-{Straight Barb[harpoon,swap]},line width=.2cm,blue] (0,0) - - (1,1) ;
 &  
\tikz \draw [-{Tee Barb[harpoon,swap]},line width=.2cm,blue] (0,0) - - (1,1) ; 
 \\ \hline 
Arc Barb & Bracket& Hooks & Parenthesis & Classical TikZ Rightarrow & Straight Barb & Tee Barb
\\ \hline
\end{tabular}

\newpage

\Par{Paramètre color}{Parameter color}
\RRR{16-3-6}

\begin{tabular}{|c|c|c|} \hline
 \multicolumn{3}{|c|}{ \BS{tikz} \BS{draw}[-\AC{Arc Barb[\RDD{color}=red},line width=.2cm,blue] (0,0) - - (1,1);}
 \\ \hline  
\tikz \draw[-{Bracket[color=red]},line width=.2cm,blue] (0,0) - - (1,1) ;
&  
\tikz \draw[-{Bracket[color=green]},line width=.2cm,blue] (0,0) - - (1,1) ;
&  
\tikz \draw[-{Bracket[red]},line width=.2cm,blue] (0,0) - - (1,1) ;
\\ \hline 
Bracket[color=red] & Bracket[color=green] &Bracket[red]  \\ 
\hline 
\end{tabular} 


\bigskip



\begin{tabular}{|c|c|c|c|c|} \hline 
 \multicolumn{5}{|c|}{ \BS{tikz} \BS{draw}[-\AC{Arc Barb[\RDD{red}},line width=.2cm,blue] (0,0) - - (1,1);}
 \\ \hline
\tikz \draw [-{Arc Barb[red]},line width=.2cm,blue] (0,0) - - (1,1) ;
 &
\tikz \draw[-{Bracket[red]},line width=.2cm,blue] (0,0) - - (1,1) ;
 &  
\tikz \draw [-{Hooks[red]},line width=.2cm,blue] (0,0) - - (1,1) ;
&  
\tikz \draw[-{Parenthesis[red]},line width=.2cm,blue] (0,0) - - (1,1) ;
&
\tikz \draw[-{Classical TikZ Rightarrow[red]},line width=.2cm,blue] (0,0) - - (1,1) ; 
 \\ \hline 
Arc Barb & Bracket & Hooks & Parenthesis &  Classical TikZ Rightarrow
 \\ \hline
 
\tikz \draw [-{Straight Barb[red][open]},line width=.2cm,blue] (0,0) - - (1,1) ;
&  
\tikz \draw [-{Tee Barb[red]},line width=.2cm,blue] (0,0) - - (1,1) ;
 &
\tikz \draw[-{Circle[red]},line width=.2cm,blue] (0,0) - - (1,1) ;
  &  
\tikz \draw[-{Diamond[red]},line width=.2cm,blue] (0,0) - - (1,1) ;
&
\tikz \draw[-{Ellipse[red]},line width=.2cm,blue] (0,0) - - (1,1) ;
 \\ \hline 
Straight Barb & Tee Barb & Circle  & Diamond & Ellipse 
 \\ \hline 
 
\tikz \draw[-{Kite[red]},line width=.2cm,blue] (0,0) - - (1,1) ; 
&
\tikz \draw[-{Latex[red]},line width=.2cm,blue] (0,0) - - (1,1) ;
&
\tikz \draw[-{Rectangle[red]},line width=.2cm,blue] (0,0) - - (1,1) ;
& 
\tikz \draw[-{Square[red]},line width=.2cm,blue] (0,0) - - (1,1) ;
& 
\tikz \draw[-{Stealth[red]},line width=.2cm,blue] (0,0) - - (1,1) ;
 \\ \hline 
 Kite & Latex & Rectangle & Square & Stealth
 \\ \hline 
 
\tikz \draw[-{Triangle[red]},line width=.2cm,blue] (0,0) - - (1,1) ; 
&
\tikz \draw[-{Turned Square[red]},line width=.2cm,blue] (0,0) - - (1,1) ; 
&
\tikz \draw[-{Rays[red]},line width=.2cm,blue] (0,0) - - (1,1) ;
&  &  
 \\ \hline
Triangle & Turned Square & Rays & &
  \\ \hline     
\end{tabular}

\bigskip

\Par{Paramètre fill}{Parameter fill}
\RRR{16-3-6}

\begin{tabular}{|c|c|c|c|c|} \hline 
 \multicolumn{5}{|c|}{ \BS{tikz} \BS{draw}[-\AC{Circle[\RDD{fill}=red]},line width=.2cm,blue] (0,0) - - (1,1);}
 \\ \hline
\tikz \draw [-{Circle[fill=red]},line width=.2cm,blue] (0,0) - - (1,1) ;
 &  
\tikz \draw [-{Diamond[fill=red]},line width=.2cm,blue] (0,0) - - (1,1) ;
 &
 \tikz \draw[-{Ellipse[fill=red]},line width=.2cm,blue] (0,0) - - (1,1) ;
  &  
 \tikz \draw[-{Kite[fill=red]},line width=.2cm,blue] (0,0) - - (1,1) ;
&
\tikz \draw[-{Triangle[fill=red]},line width=.2cm,blue] (0,0) - - (1,1) ; 
 \\ \hline 
Circle & Diamond & Ellipse & Kite & Triangle 
 \\ \hline
 
\tikz \draw [-{Latex[fill=red]},line width=.2cm,blue] (0,0) - - (1,1) ;
 &  
\tikz \draw [-{Rectangle[fill=red]},line width=.2cm,blue] (0,0) - - (1,1) ;
 &
 \tikz \draw[-{Square[fill=red]},line width=.2cm,blue] (0,0) - - (1,1) ;
  &  
\tikz \draw[-{Stealth[fill=red]},line width=.2cm,blue] (0,0) - - (1,1) ;
&
\tikz \draw[-{Turned Square[fill=red]},line width=.2cm,blue] (0,0) - - (1,1) ;
 \\ \hline 
Latex & Rectangle & Square  & Stealth &  Turned Square
 \\ \hline    
\end{tabular}



\bigskip

\begin{tabular}{|c|c|c|c|c|} \hline 
 \multicolumn{5}{|c|}{ \BS{tikz} \BS{draw}[-\AC{Circle[\RDD{fill}=none]},line width=.2cm,blue] (0,0) - - (1,1);}
 \\ \hline
\tikz \draw [-{Circle[fill=none]},line width=.2cm,blue] (0,0) - - (1,1) ;
 &  
\tikz \draw [-{Diamond[fill=none]},line width=.2cm,blue] (0,0) - - (1,1) ;
 &
 \tikz \draw[-{Ellipse[fill=none]},line width=.2cm,blue] (0,0) - - (1,1) ;
  &  
 \tikz \draw[-{Kite[fill=none]},line width=.2cm,blue] (0,0) - - (1,1) ;
&
\tikz \draw[-{Triangle[fill=none]},line width=.2cm,blue] (0,0) - - (1,1) ; 
 \\ \hline 
Circle & Diamond & Ellipse & Kite & Triangle 
 \\ \hline
 
\tikz \draw [-{Latex[fill=none]},line width=.2cm,blue] (0,0) - - (1,1) ;
 &  
\tikz \draw [-{Rectangle[fill=none]},line width=.2cm,blue] (0,0) - - (1,1) ;
 &
 \tikz \draw[-{Square[fill=none]},line width=.2cm,blue] (0,0) - - (1,1) ;
  &  
\tikz \draw[-{Stealth[fill=none]},line width=.2cm,blue] (0,0) - - (1,1) ;
&
\tikz \draw[-{Turned Square[fill=none]},line width=.2cm,blue] (0,0) - - (1,1) ;
 \\ \hline 
Latex & Rectangle & Square  & Stealth &  Turned Square
 \\ \hline    
\end{tabular}

\newpage

\Par{Paramètre open}{Parameter open}
\RRR{16-3-6}

\begin{tabular}{|c|c|c|c|c|} \hline 
 \multicolumn{5}{|c|}{ \BS{tikz} \BS{draw}[-\AC{Circle[\RDD{open}]},line width=.2cm,blue] (0,0) - - (1.5,1) ;}
 \\ \hline
\tikz \draw [-{Circle[open]},line width=.2cm,blue] (0,0) - - (1.5,1) ;
 &  
\tikz \draw [-{Diamond[open]},line width=.2cm,blue] (0,0) - - (1.5,1) ;
 &
 \tikz \draw[-{Ellipse[open]},line width=.2cm,blue] (0,0) - - (1.5,1) ;
  &  
 \tikz \draw[-{Kite[open]},line width=.2cm,blue] (0,0) - - (1.5,1) ;
&
\tikz \draw[-{Triangle[open]},line width=.2cm,blue] (0,0) - - (1.5,1) ; 
 \\ \hline 
Circle & Diamond & Ellipse & Kite & Triangle 
 \\ \hline
 
\tikz \draw [-{Latex[open]},line width=.2cm,blue] (0,0) - - (1.5,1) ;
 &  
\tikz \draw [-{Rectangle[open]},line width=.2cm,blue] (0,0) - - (1.5,1) ;
 &
 \tikz \draw[-{Square[open]},line width=.2cm,blue] (0,0) - - (1.5,1) ;
  &  
\tikz \draw[-{Stealth[open]},line width=.2cm,blue] (0,0) - - (1.5,1) ;
&
\tikz \draw[-{Turned Square[open]},line width=.2cm,blue] (0,0) - - (1.5,1) ;
 \\ \hline 
Latex & Rectangle & Square  & Stealth &  Turned Square
 \\ \hline    
\end{tabular}

\bigskip

\Par{Paramètre line cap : round or butt}{Parameter line cap : round or butt}
\RRR{16-3-7}




\begin{tabular}{|c|c|c|c|c|c|c|c|} \hline 
 \multicolumn{8}{|c|}{ \BS{tikz} \BS{draw}[-\AC{Arc Barb[\RDD{line cap}=\RDDX{butt}{line cap}]},line width=.2cm,blue] (0,0) - - (1,1);}
 \\ \hline
\tikz \draw [-{Arc Barb[line cap=butt]},line width=.2cm,blue] (0,0) - - (1,1) ;
 &
\tikz \draw[-{Bracket[line cap=butt]},line width=.2cm,blue] (0,0) - - (1,1) ;
 &  
\tikz \draw [-{Hooks[line cap=butt]},line width=.2cm,blue] (0,0) - - (1,1) ;

  &  
\tikz \draw[-{Parenthesis[line cap=butt]},line width=.2cm,blue] (0,0) - - (1,1) ;
&
\tikz \draw[-{Ellipse[line cap=butt]},line width=.2cm,blue] (0,0) - - (1,1) ;
&
\tikz \draw[-{Rectangle[line cap=butt]},line width=.2cm,blue] (0,0) - - (1,1) ;
&
\tikz \draw[-{Square[line cap=butt]},line width=.2cm,blue] (0,0) - - (1,1) ;  
& 
\tikz \draw[-{Stealth[line cap=butt]},line width=.2cm,blue] (0,0) - - (1,1) ; 
 \\ \hline 
Arc Barb & Bracket & Hooks & Parenthesis & Ellipse 
& Rectangle & Square & Stealth 
\\ \hline
 
\tikz \draw [-{Straight Barb[line cap=butt]},line width=.2cm,blue] (0,0) - - (1,1) ;
 &  
\tikz \draw [-{Tee Barb[line cap=butt]},line width=.2cm,blue] (0,0) - - (1,1) ;
 &
 \tikz \draw[-{Diamond[line cap=butt]},line width=.2cm,blue] (0,0) - - (1,1) ;
 &  
\tikz \draw[-{Kite[line cap=butt]},line width=.2cm,blue] (0,0) - - (1,1) ;
&
\tikz \draw[-{Latex[line cap=butt]},line width=.2cm,blue] (0,0) - - (1,1) ;
 
&
\tikz \draw[-{Triangle[line cap=butt]},line width=.2cm,blue] (0,0) - - (1,1) ;  
&   
\tikz \draw[-{Turned Square[line cap=butt]},line width=.2cm,blue] (0,0) - - (1,1) ;  
& 
\tikz \draw[-{Rays[line cap=butt]},line width=.2cm,blue] (0,0) - - (1,1) ; 
\\ \hline 
Straight Barb & Tee Barb  & Diamond  & Kite  & Latex 
 & Triangle & Turned Square & Rays
 \\ \hline     
\end{tabular}

\bigskip

\begin{tabular}{|c|c|c|c|c|c|c|c|} \hline 
 \multicolumn{8}{|c|}{ \BS{tikz} \BS{draw}[-\AC{Arc Barb[\RDD{line cap}=\RDDX{round}{line cap}]},line width=.2cm,blue] (0,0) - - (1,1);}
 \\ \hline
\tikz \draw [-{Arc Barb[line cap=round]},line width=.2cm,blue] (0,0) - - (1,1) ;
 &
\tikz \draw[-{Bracket[line cap=round]},line width=.2cm,blue] (0,0) - - (1,1) ;
 &  
\tikz \draw [-{Hooks[line cap=round]},line width=.2cm,blue] (0,0) - - (1,1) ;

  &  
\tikz \draw[-{Parenthesis[line cap=round]},line width=.2cm,blue] (0,0) - - (1,1) ;
&
\tikz \draw[-{Ellipse[line cap=round]},line width=.2cm,blue] (0,0) - - (1,1) ;
&
\tikz \draw[-{Rectangle[line cap=round]},line width=.2cm,blue] (0,0) - - (1,1) ;
&
\tikz \draw[-{Square[line cap=round]},line width=.2cm,blue] (0,0) - - (1,1) ;  
& 
\tikz \draw[-{Stealth[line cap=round]},line width=.2cm,blue] (0,0) - - (1,1) ; 
 \\ \hline 
Arc Barb & Bracket & Hooks & Parenthesis & Ellipse 
& Rectangle & Square & Stealth 
\\ \hline
 
\tikz \draw [-{Straight Barb[line cap=round]},line width=.2cm,blue] (0,0) - - (1,1) ;
 &  
\tikz \draw [-{Tee Barb[line cap=round]},line width=.2cm,blue] (0,0) - - (1,1) ;
 &
 \tikz \draw[-{Diamond[line cap=round]},line width=.2cm,blue] (0,0) - - (1,1) ;
 &  
\tikz \draw[-{Kite[line cap=round]},line width=.2cm,blue] (0,0) - - (1,1) ;
&
\tikz \draw[-{Latex[line cap=round]},line width=.2cm,blue] (0,0) - - (1,1) ;
 
&
\tikz \draw[-{Triangle[line cap=round]},line width=.2cm,blue] (0,0) - - (1,1) ;  
&   
\tikz \draw[-{Turned Square[line cap=round]},line width=.2cm,blue] (0,0) - - (1,1) ;  
& 
\tikz \draw[-{Rays[line cap=round]},line width=.2cm,blue] (0,0) - - (1,1) ; 
\\ \hline 
Straight Barb & Tee Barb  & Diamond  & Kite  & Latex 
 & Triangle & Turned Square & Rays
 \\ \hline     
\end{tabular}

\bigskip

\Par{Paramètre line join : round or miter }{Parameter line join :  round or miter}
\RRR{16-3-7}




\begin{tabular}{|c|c|c|c|c|c|c|c|} \hline 
 \multicolumn{8}{|c|}{ \BS{tikz} \BS{draw}[-\AC{Arc Barb[\RDD{line join}=\RDDX{miter}{line join}]},line width=.2cm,blue] (0,0) - - (1,1);}
 \\ \hline
\tikz \draw [-{Arc Barb[line join=miter]},line width=.2cm,blue] (0,0) - - (1,1) ;
 &
\tikz \draw[-{Bracket[line join=miter]},line width=.2cm,blue] (0,0) - - (1,1) ;
 &  
\tikz \draw [-{Hooks[line join=miter]},line width=.2cm,blue] (0,0) - - (1,1) ;

  &  
\tikz \draw[-{Parenthesis[line join=miter]},line width=.2cm,blue] (0,0) - - (1,1) ;
&
\tikz \draw[-{Ellipse[line join=miter]},line width=.2cm,blue] (0,0) - - (1,1) ;
&
\tikz \draw[-{Rectangle[line join=miter]},line width=.2cm,blue] (0,0) - - (1,1) ;
&
\tikz \draw[-{Square[line join=miter]},line width=.2cm,blue] (0,0) - - (1,1) ;  
& 
\tikz \draw[-{Stealth[line join=miter]},line width=.2cm,blue] (0,0) - - (1,1) ; 
 \\ \hline 
Arc Barb & Bracket & Hooks & Parenthesis & Ellipse 
& Rectangle & Square & Stealth 
\\ \hline
 
\tikz \draw [-{Straight Barb[line cap=butt]},line width=.2cm,blue] (0,0) - - (1,1) ;
 &  
\tikz \draw [-{Tee Barb[line join=miter]},line width=.2cm,blue] (0,0) - - (1,1) ;
 &
 \tikz \draw[-{Diamond[line join=miter]},line width=.2cm,blue] (0,0) - - (1,1) ;
 &  
\tikz \draw[-{Kite[line join=miter]},line width=.2cm,blue] (0,0) - - (1,1) ;
&
\tikz \draw[-{Latex[line join=miter]},line width=.2cm,blue] (0,0) - - (1,1) ;
 
&
\tikz \draw[-{Triangle[line join=miter]},line width=.2cm,blue] (0,0) - - (1,1) ;  
&   
\tikz \draw[-{Turned Square[line join=miter]},line width=.2cm,blue] (0,0) - - (1,1) ;  
& 
\tikz \draw[-{Rays[line join=miter]},line width=.2cm,blue] (0,0) - - (1,1) ; 
\\ \hline 
Straight Barb & Tee Barb  & Diamond  & Kite  & Latex 
 & Triangle & Turned Square & Rays
 \\ \hline     
\end{tabular}

\bigskip

\begin{tabular}{|c|c|c|c|c|c|c|c|} \hline 
 \multicolumn{8}{|c|}{ \BS{tikz} \BS{draw}[-\AC{Arc Barb[\RDD{line cap}=\RDDX{round}{line cap}]},line width=.2cm,blue] (0,0) - - (1,1);}
 \\ \hline
\tikz \draw [-{Arc Barb[line join=round]},line width=.2cm,blue] (0,0) - - (1,1) ;
 &
\tikz \draw[-{Bracket[line join=round]},line width=.2cm,blue] (0,0) - - (1,1) ;
 &  
\tikz \draw [-{Hooks[line join=round]},line width=.2cm,blue] (0,0) - - (1,1) ;

  &  
\tikz \draw[-{Parenthesis[line join=round]},line width=.2cm,blue] (0,0) - - (1,1) ;
&
\tikz \draw[-{Ellipse[line join=round]},line width=.2cm,blue] (0,0) - - (1,1) ;
&
\tikz \draw[-{Rectangle[line join=round]},line width=.2cm,blue] (0,0) - - (1,1) ;
&
\tikz \draw[-{Square[line join=round]},line width=.2cm,blue] (0,0) - - (1,1) ;  
& 
\tikz \draw[-{Stealth[line join=round]},line width=.2cm,blue] (0,0) - - (1,1) ; 
 \\ \hline 
Arc Barb & Bracket & Hooks & Parenthesis & Ellipse 
& Rectangle & Square & Stealth 
\\ \hline
 
\tikz \draw [-{Straight Barb[line join=round]},line width=.2cm,blue] (0,0) - - (1,1) ;
 &  
\tikz \draw [-{Tee Barb[line join=round]},line width=.2cm,blue] (0,0) - - (1,1) ;
 &
 \tikz \draw[-{Diamond[line join=round]},line width=.2cm,blue] (0,0) - - (1,1) ;
 &  
\tikz \draw[-{Kite[line join=round]},line width=.2cm,blue] (0,0) - - (1,1) ;
&
\tikz \draw[-{Latex[line join=round]},line width=.2cm,blue] (0,0) - - (1,1) ;
 
&
\tikz \draw[-{Triangle[line join=round]},line width=.2cm,blue] (0,0) - - (1,1) ;  
&   
\tikz \draw[-{Turned Square[line join=round]},line width=.2cm,blue] (0,0) - - (1,1) ;  
& 
\tikz \draw[-{Rays[line join=round]},line width=.2cm,blue] (0,0) - - (1,1) ; 
\\ \hline 
Straight Barb & Tee Barb  & Diamond  & Kite  & Latex 
 & Triangle & Turned Square & Rays
 \\ \hline     
\end{tabular}

\Par{Paramètre round}{Parameter round}
\RRR{16-3-7}

\begin{tabular}{|c|c|c|c|c|c|c|c|} \hline 
 \multicolumn{8}{|c|}{ \BS{tikz} \BS{draw}[-\AC{Arc Barb[\RDD{round}]},line width=.2cm,blue] (0,0) - - (1,1);}
 \\ \hline
\tikz \draw [-{Arc Barb[round]},line width=.2cm,blue] (0,0) - - (1,1) ;
 &
\tikz \draw[-{Bracket[round]},line width=.2cm,blue] (0,0) - - (1,1) ;
 &  
\tikz \draw [-{Hooks[round]},line width=.2cm,blue] (0,0) - - (1,1) ;

  &  
\tikz \draw[-{Parenthesis[round]},line width=.2cm,blue] (0,0) - - (1,1) ;
&
\tikz \draw[-{Ellipse[round]},line width=.2cm,blue] (0,0) - - (1,1) ;
&
\tikz \draw[-{Rectangle[round]},line width=.2cm,blue] (0,0) - - (1,1) ;
&
\tikz \draw[-{Square[round]},line width=.2cm,blue] (0,0) - - (1,1) ;  
& 
\tikz \draw[-{Stealth[round]},line width=.2cm,blue] (0,0) - - (1,1) ; 
 \\ \hline 
Arc Barb & Bracket & Hooks & Parenthesis & Ellipse 
& Rectangle & Square & Stealth 
\\ \hline
 
\tikz \draw [-{Straight Barb[round]},line width=.2cm,blue] (0,0) - - (1,1) ;
 &  
\tikz \draw [-{Tee Barb[round]},line width=.2cm,blue] (0,0) - - (1,1) ;
 &
 \tikz \draw[-{Diamond[round]},line width=.2cm,blue] (0,0) - - (1,1) ;
 &  
\tikz \draw[-{Kite[round]},line width=.2cm,blue] (0,0) - - (1,1) ;
&
\tikz \draw[-{Latex[round]},line width=.2cm,blue] (0,0) - - (1,1) ;
 
&
\tikz \draw[-{Triangle[round]},line width=.2cm,blue] (0,0) - - (1,1) ;  
&   
\tikz \draw[-{Turned Square[round]},line width=.2cm,blue] (0,0) - - (1,1) ;  
& 
\tikz \draw[-{Rays[round]},line width=.2cm,blue] (0,0) - - (1,1) ; 
\\ \hline 
Straight Barb & Tee Barb  & Diamond  & Kite  & Latex 
 & Triangle & Turned Square & Rays
 \\ \hline     
\end{tabular}

\Par{Paramètre sharp}{Parameter sharp}
\RRR{16-3-7}

\begin{tabular}{|c|c||c|c|} \hline
 \multicolumn{4}{|c|}{ \BS{tikz} \BS{draw}[-\AC{Classical TikZ Rightarrow[\RDD{sharp}]},line width=.2cm,blue] (0,0) - - (2,0)) ;}
 \\ \hline
 \multicolumn{2}{|c||}{ -\AC{Classical TikZ Rightarrow[\RDD{sharp}]}} &
 \multicolumn{2}{|c|}{ -\AC{Computer Modern Rightarrow[\RDD{sharp}]}} 
 \\ \hline  
\tikz \draw[-{Classical TikZ Rightarrow[sharp]},line width=.5cm,blue] (0,0) - - (2,0) ;
&  
\tikz \draw[-{Classical TikZ Rightarrow[]},line width=.5cm,blue] (0,0) - - (2,0) ;
&  
\tikz \draw[-{Computer Modern Rightarrow[sharp]},line width=.5cm,blue] (0,0) - - (2,0) ;
&  
\tikz \draw[-{Computer Modern Rightarrow[]},line width=.5cm,blue] (0,0) - - (2,0) ;
\\ 
\hline \RDD{sharp} & \color{red}{[  ]} & \RDD{sharp} & \color{red}{[  ]} \\ 
\hline 
\end{tabular} 

\newpage

\Par{Paramètre line width}{Parameter line width}
\RRR{16-3-7}

\begin{tabular}{|c|c|c|c|} \hline 
 \multicolumn{4}{|c|}{ \BS{tikz} \BS{draw}[-\AC{Arc Barb[\RDD{line width}=.2cm]},line width=.4cm,blue] (0,0) - - (2,0);}
 \\ \hline
 \begin{tikzpicture}[blue,line width=2pt,baseline=.5cm]
  \draw[help lines] (0,-1.5) grid (2,1.5); 
 \draw [-{Arc Barb[line width=.2cm]},line width=.4cm,blue] (0,0) - - (2,0) ; 
 \end{tikzpicture}
& 
\begin{tikzpicture}[blue,line width=2pt,baseline=.5cm]
\draw[help lines] (0,-1.5) grid (2,1.5); 
\draw [-{Hooks[line width=.2cm]},line width=.4cm,blue] (0,0) - - (2,0) ; 
\end{tikzpicture} 

 
&
\begin{tikzpicture}[blue,line width=2pt,baseline=.5cm]
\draw[help lines] (0,-1.5) grid (2,1.5); 
\draw[-{Classical TikZ Rightarrow[line width=.2cm]},line width=.4cm,blue] (0,0) - -(2,0) ; 
\end{tikzpicture}
&
\begin{tikzpicture}[blue,line width=2pt,baseline=.5cm]
\draw[help lines] (0,-1.5) grid (2,1.5); 
\draw[-{Straight Barb[line width=.2cm]},line width=.4cm,blue] (0,0) - -  (2,0) ; 
\end{tikzpicture} 
\\ \hline 
Arc Barb & Hooks  &   Classical TikZ Rightarrow& Straight Barb
 \\ \hline

\begin{tikzpicture}[blue,line width=2pt,baseline=.5cm]
  \draw[help lines] (0,-1.5) grid (2,1.5); 
\draw [-{Straight Barb[line width=.2cm]},line width=.4cm,blue] (0,0) - - (2,0) ; 
  \end{tikzpicture} 
&
\begin{tikzpicture}[blue,line width=2pt,baseline=.5cm]
\draw[help lines] (0,-1.5) grid (2,1.5); 
\draw [-{Tee Barb[line width=.2cm]},line width=.4cm,blue] (0,0) - - (2,0) ; 
\end{tikzpicture}       
&
 \begin{tikzpicture}[blue,line width=2pt,baseline=.5cm]
 \draw[help lines] (0,-1.5) grid (2,1.5); 
 \draw[-{Computer Modern Rightarrow[line width=.2cm]},line width=.4cm,blue] (0,0) - - (2,0) ; 
 \end{tikzpicture}
 &
\\ \hline 
Straight Barb & Tee Bar &  Computer Modern Rightarrow &
 \\ \hline    
\end{tabular}
\bigskip

\begin{tabular}{|c|c|} \hline
 \multicolumn{2}{|c|}{ \BS{tikz} \BS{draw}[-\AC{Arc Barb[line width={\color{green} 0cm} {\color{red} 10}]},line width={\color{blue}.1cm},blue] (0,0) - - (3,1);}
 \\ \hline  
 \begin{tikzpicture}[blue,line width=2pt,baseline=.5cm]
  \draw[help lines] (0,-1.5) grid (3,1.5); 
 \draw [-{Arc Barb[line width=0cm .5]},line width=.4cm,blue] (0,0) - - (3,0) ; 
 \end{tikzpicture}
&
 \begin{tikzpicture}[blue,line width=2pt,baseline=.5cm]
  \draw[help lines] (0,-1.5) grid (3,1.5); 
 \draw [-{Arc Barb[line width=.1cm .5]},line width=.4cm,blue] (0,0) - - (3,0) ; 
 \end{tikzpicture}
\\ \hline 
[length={\color{green} 0cm} {\color{red} 10}] & [length={\color{green}.5cm} {\color{red} 5 }]
\\ \hline 
{\color{green} 0cm} + {\color{red} 10} x {\color{blue}.1cm} = 1cm & {\color{green}.5cm} + {\color{red} 5 }x {\color{blue}.1cm} = 1cm
\\ \hline 
\end{tabular}

\bigskip

\begin{tabular}{|c|c|c|c|} \hline   
  \multicolumn{2}{|c|}{ \BS{tikz} \BS{draw}[-\AC{Arc Barb[length={\color{green} 0cm} {\color{red} 5 }]},line width={\color{blue}.1cm},blue,double,double distance = {\color{magenta}2 mm}] (0,0) - - (3,1);}
  \\ \hline  
 \begin{tikzpicture}[blue,line width=2pt,baseline=.5cm]
  \draw[help lines] (0,-2) grid (3,2); 
 \draw [-{Arc Barb[line width=0cm 2 ]},line width=.4cm,blue,double,double distance =2mm] (0,0) - - (3,0) ; 
 \end{tikzpicture}
&  
 \begin{tikzpicture}[blue,line width=2pt,baseline=.5cm]
  \draw[help lines] (0,-2) grid (3,2); 
 \draw [-{Arc Barb[line width=0cm 2 .6 ]},line width=.4cm,blue,double,double distance =2mm ] (0,0) - - (3,0) ; 
 \end{tikzpicture}
\\ \hline  
 [length={\color{green} 0cm}{\color{red} 5 } ] 
 &
 [length={\color{green} 0cm} {\color{red} 5 } {\color{orange} .6} ]
\\ \hline  
{\color{green} 0cm} + {\color{red} 5 } x ({\color{blue}.1cm} + {\color{magenta}2 mm} + {\color{blue}.1cm} ) = 2cm 
&
{\color{green} 0cm} + {\color{red} 5 } x (.6 x {\color{blue}.1cm}+ (1-{\color{orange} .6})({\color{blue}.1cm}+ {\color{magenta}2 mm}+{\color{blue}.1cm}) =  11 mm\\ 
\hline 
\end{tabular} 

\newpage

\Par{Paramètre line width'}{Parameter line width'}
\RRR{16-3-7}

\begin{tabular}{|c|c|c|c|} \hline 
 \multicolumn{4}{|c|}{ \BS{tikz} \BS{draw}[-\AC{Arc Barb[\RDD{line width'}=.2cm]},line width=.4cm,blue] (0,0) - - (1,1);}
 \\ \hline
 \begin{tikzpicture}[blue,line width=2pt,baseline=.5cm]
  \draw[help lines] (0,-2) grid (2,2); 
 \draw [-{Arc Barb[line width'=.2cm]},line width=.4cm,blue] (0,0) - - (2,0) ; 
 \end{tikzpicture}
& 
\begin{tikzpicture}[blue,line width=2pt,baseline=.5cm]
\draw[help lines] (0,-2) grid (2,2); 
\draw [-{Hooks[line width'=.2cm]},line width=.4cm,blue] (0,0) - - (2,0) ; 
\end{tikzpicture} 
 
 
&
\begin{tikzpicture}[blue,line width=2pt,baseline=.5cm]
\draw[help lines] (0,-2) grid (2,2); 
\draw[-{Classical TikZ Rightarrow[line width'=.2cm]},line width=.4cm,blue] (0,0) - -(2,0) ; 
\end{tikzpicture}
&
\begin{tikzpicture}[blue,line width=2pt,baseline=.5cm]
\draw[help lines] (0,-2) grid (2,2); 
\draw[-{Straight Barb[line width'=.2cm]},line width=.4cm,blue] (0,0) - -  (2,0) ; 
\end{tikzpicture}
\\ \hline 
Arc Barb & Hooks  &   Classical TikZ Rightarrow & Straight Barb
 \\ \hline

\begin{tikzpicture}[blue,line width=2pt,baseline=.5cm]
  \draw[help lines] (0,-2) grid (2,2); 
\draw [-{Straight Barb[line width'=.2cm]},line width=.4cm,blue] (0,0) - - (2,0) ; 
  \end{tikzpicture} 
&
\begin{tikzpicture}[blue,line width=2pt,baseline=.5cm]
\draw[help lines] (0,-2) grid (2,2); 
\draw [-{Tee Barb[line width'=.2cm]},line width=.4cm,blue] (0,0) - - (2,0) ; 
\end{tikzpicture}   


    
&
 \begin{tikzpicture}[blue,line width=2pt,baseline=.5cm]
 \draw[help lines] (0,-2) grid (2,2); 
 \draw[-{Computer Modern Rightarrow[line width'=.2cm]},line width=.4cm,blue] (0,0) - - (2,0) ; 
 \end{tikzpicture}
 &
\\ \hline 
Straight Barb & Tee Bar  &  Computer Modern Rightarrow &
 \\ \hline    
\end{tabular}
\bigskip

\begin{tabular}{|c|c|} \hline
 \multicolumn{2}{|c|}{ \BS{tikz} \BS{draw}[-\AC{Arc Barb[line width={\color{green} 0cm} {\color{red} 10}]},line width'={\color{blue}.1cm},blue] (0,0) - - (3,1);}
 \\ \hline  
 \begin{tikzpicture}[blue,line width=2pt,baseline=.5cm]
  \draw[help lines] (0,-2) grid (3,2); 
 \draw [-{Arc Barb[line width'=0cm .5]},line width=.4cm,blue] (0,0) - - (3,0) ; 
 \end{tikzpicture}
&
 \begin{tikzpicture}[blue,line width=2pt,baseline=.5cm]
  \draw[help lines] (0,-2) grid (3,2); 
 \draw [-{Arc Barb[line width'=.1cm .5]},line width=.4cm,blue] (0,0) - - (3,0) ; 
 \end{tikzpicture}
\\ \hline 
[length={\color{green} 0cm} {\color{red} 10}] & [length={\color{green}.5cm} {\color{red} 5 }]
\\ \hline 
{\color{green} 0cm} + {\color{red} 10} x {\color{blue}.1cm} = 1cm & {\color{green}.5cm} + {\color{red} 5 }x {\color{blue}.1cm} = 1cm
\\ \hline 
\end{tabular}

\bigskip

\Par{Paramètre quick}{Parameter quick}
\RRR{16-3-8}

\begin{tabular}{|c|c|} \hline
 \multicolumn{2}{|c|}{ \BS{tikz} \BS{draw}[-\AC{Stealth[length=1cm,open,\RDD{quick}]}]
 (0,0) .. controls (1,-1) and (2,1) .. (3,1);}
 \\ \hline  
 \begin{tikzpicture}[blue,line width=2pt,baseline=.5cm]
  \draw[help lines] (0,-1) grid (3,2); 
\draw [-{Stealth[length=1cm,open,quick]}]
(0,0) .. controls (1,-1) and (2,1) .. (3,1);
 \end{tikzpicture}
 &
 \begin{tikzpicture}[blue,line width=2pt,baseline=.5cm]
  \draw[help lines] (0,-1) grid (3,2); 
\draw [-{Stealth[length=1cm,open]}]
(0,0) .. controls (1,-1) and (2,1) .. (3,1);
 \end{tikzpicture}
  \\ \hline
[-{Stealth[length=1cm,open,\RDD{quick}]}]
 & 
 [-{Stealth[length=1cm,open]}] 
   \\ \hline 
\end{tabular}

\newpage

\Par{Paramètre bending}{Parameter bending}
\RRR{16-3-8}

 \maboite{\BS{usetikzlibrary}\AC{bending}}
\label{lib-bending}


\begin{tabular}{|c|c|c|} \hline
 \multicolumn{3}{|c|}{ \BS{tikz} \BS{draw}[-\AC{Stealth[length=1cm,open,\RDD{flex}=0]}]
 (0,0) .. controls (1,-1) and (2,1) .. (3,1);}
 \\ \hline  
 \begin{tikzpicture}[blue,line width=2pt,baseline=.5cm]
  \draw[help lines] (0,-1) grid (3,2); 
\draw [-{Stealth[length=1cm,open,flex=0]}]
(0,0) .. controls (1,-1) and (2,1) .. (3,1);
 \end{tikzpicture}
 &
 \begin{tikzpicture}[blue,line width=2pt,baseline=.5cm]
  \draw[help lines] (0,-1) grid (3,2); 
\draw [-{Stealth[length=1cm,open,flex=.5]}]
(0,0) .. controls (1,-1) and (2,1) .. (3,1);
 \end{tikzpicture}
 &
 \begin{tikzpicture}[blue,line width=2pt,baseline=.5cm]
  \draw[help lines] (0,-1) grid (3,2); 
\draw [-{Stealth[length=1cm,open,flex=1]}]
(0,0) .. controls (1,-1) and (2,1) .. (3,1);
 \end{tikzpicture}
 \\ \hline  
flex=0 & flex=0.5 & flex=1
  \\ \hline  
\end{tabular}

\bigskip

\begin{tabular}{|c|c|c|} \hline
 \multicolumn{3}{|c|}{ \BS{tikz} \BS{draw}[-\AC{Stealth[length=1cm,open,\RDD{flex'}=0]}]
 (0,0) .. controls (1,-1) and (2,1) .. (3,1);}
 \\ \hline  
 \begin{tikzpicture}[blue,line width=2pt,baseline=.5cm]
  \draw[help lines] (0,-1) grid (3,2); 
\draw [-{Stealth[length=1cm,open,flex'=0]}]
(0,0) .. controls (1,-1) and (2,1) .. (3,1);
 \end{tikzpicture}
 &
 \begin{tikzpicture}[blue,line width=2pt,baseline=.5cm]
  \draw[help lines] (0,-1) grid (3,2); 
\draw [-{Stealth[length=1cm,open,flex'=.5]}]
(0,0) .. controls (1,-1) and (2,1) .. (3,1);
 \end{tikzpicture}
 &
 \begin{tikzpicture}[blue,line width=2pt,baseline=.5cm]
  \draw[help lines] (0,-1) grid (3,2); 
\draw [-{Stealth[length=1cm,open,flex'=1]}]
(0,0) .. controls (1,-1) and (2,1) .. (3,1);
 \end{tikzpicture}
 \\ \hline  
flex'=0 & flex'=0.5 & flex'=1
  \\ \hline  
\end{tabular}

\bigskip

\begin{tabular}{|c|c|} \hline
 \multicolumn{2}{|c|}{ \BS{tikz} \BS{draw}[-\AC{Stealth[length=1cm,open,\RDD{bend}]}]
 (0,0) .. controls (1,-1) and (2,1) .. (3,1);}
 \\ \hline  
 \begin{tikzpicture}[blue,line width=2pt,baseline=.5cm]
  \draw[help lines] (0,-1) grid (3,2); 
\draw [-{Stealth[length=1cm,open,bend]}]
(0,0) .. controls (1,-1) and (2,1) .. (3,1);
 \end{tikzpicture}
 &
 \begin{tikzpicture}[blue,line width=2pt,baseline=.5cm]
  \draw[help lines] (0,-1) grid (3,2); 
\draw [-{Stealth[length=1cm,open,bend]Stealth[length=1cm,open,bend,sep]}]
(0,0) .. controls (1,-1) and (2,1) .. (3,1);
 \end{tikzpicture} 
 \\ \hline  
[-\AC{Stealth[length=1cm,open,\RDD{bend}]}] & 
[-{Stealth[length=1cm,open,bend]Stealth[length=1cm,open,bend,sep]}] 
 \\ \hline   
\end{tabular}

\Par{Paramètre cap angle}{Parameter cap angle}
\RRR{16-5-4}

\begin{tabular}{|c|c|c|} \hline 
 \multicolumn{3}{|c|}{ \BS{tikz} \BS{draw}[-\AC{Fast Round[\RDD{cap angle}=60]},line width=.2cm,blue] (0,0) - - (3,1);}
 \\ \hline
\tikz \draw[-{Fast Round[cap angle=20]},line width=.5cm,blue] (0,0) - - (3,1) ;
&
\tikz \draw[-{Fast Round[cap angle=60]},line width=.5cm,blue] (0,0) - - (3,1) ;
&
\tikz \draw[-{Fast Round[cap angle=90]},line width=.5cm,blue] (0,0) - - (3,1) ;
\\ \hline 
Fast Round[cap angle=20] & Fast Round[cap angle=60] & Fast Round[cap angle=90]
\\ \hline 
\tikz \draw[-{Fast Triangle[cap angle=20]},line width=.5cm,blue] (0,0) - - (3,1) ;
&
\tikz \draw[-{Fast Triangle[cap angle=60]},line width=.5cm,blue] (0,0) - - (3,1) ;
&
\tikz \draw[-{Fast Triangle[cap angle=90]},line width=.5cm,blue] (0,0) - - (3,1) ;
\\ \hline
Fast Triangle[cap angle=20] & Fast Triangle[cap angle=60] & Fast Triangle[cap angle=90] 
\\ \hline    
\end{tabular}


\newpage

\SSCT{Insertion de petites images}{Small pictures}


\SbSSCT{Images créées}{Own small pictures}

\begin{center}
\RRR{14-19 }  \RRR{18 }
\end{center}

\bigskip

\tikzset{dfr/.pic={\filldraw[blue] (-2pt,0) rectangle (0,5pt) ; \filldraw[fill=white] (0,0) rectangle (2pt,5pt); \filldraw[fill=red] (2pt,0) rectangle (4pt,5pt); }}

\begin{tabular}{|c|c|}\hline 
\textbf{Création}  & \textbf{ Utilisation} \\ \hline 
\parbox{10cm}{\BS{tikzset}\AC{{\color{red}{dfr/.pic}}=\AC{\BS{filldraw}[blue] (-2pt,0) rectangle (0,5pt) ; \\ \BS{filldraw[fill=white] (0,0) rectangle (2pt,5pt);\\ \BS{filldraw}[fill=red] (2pt,0) rectangle (4pt,5pt); }}}}
&
\parbox{3cm}{\BS{tikz} \BSS{pic} \AC{dfr}; \\
\tikz \pic {dfr};} 
\\ \hline 
\end{tabular} 

\bigskip

\begin{tabular}{|c|c|c|} \hline 
\multicolumn{2}{|c|}{\textbf{\TFRGB{placement à une position}{Positioning }}}
\\ \hline 
\begin{tikzpicture}
\draw[help lines] (0,0) grid (2,2) ; 
\pic at (1,1) [pic type = dfr]; 
  \end{tikzpicture} 
&  
\begin{tikzpicture}
\draw[help lines] (0,0) grid (2,2) ; 
\pic at (1,1) {dfr}; 
\end{tikzpicture}
\\ \hline 
\BSS{pic} at (1,1) [\RDD{pic type} = dfr]; & 
\BSS{pic} at (1,1) \AC{dfr};  
\\ \hline   
\begin{tikzpicture}
\draw[help lines] (0,0) grid (2,2) ; 
\path (1,1) pic [pic type = dfr];
\end{tikzpicture} 
&  
\begin{tikzpicture}
\draw[help lines] (0,0) grid (2,2) ; 
\path (1,1) pic {dfr};
\end{tikzpicture} 
\\ \hline 
\BS{path} (1,1) \RDD{pic} [\RDD{pic type}= dfr]; &
\BS{path} (1,1) \RDD{pic} \AC{dfr};
\\ \hline
\begin{tikzpicture}
\draw[help lines] (0,0) grid (2,2) ; 
\pic [at={(1,1)},pic type = dfr];
\end{tikzpicture}
& 
\begin{tikzpicture}
\draw[help lines] (0,0) grid (2,2) ; 
\pic [at={(1,1)}] {dfr};
\end{tikzpicture}  
\\ \hline   
 \BSS{pic} [at=\AC{(1,1)}] [\RDD{pic type}= dfr]; &
 \BSS{pic} [at=\AC{(1,1)}] \AC{dfr}; 
\\ \hline
\end{tabular}

\bigskip

\begin{tabular}{|c|c|c|} \hline 
\multicolumn{3}{|c|}{\BS{pic}[scale=3] at (1,1) \AC{dfr}; }
\\ \hline 
\begin{tikzpicture}
\draw[help lines] (0,0) grid (2,2) ; 
\pic[scale=3] at (1,1) {dfr}; 
  \end{tikzpicture}
&  
\begin{tikzpicture}
\draw[help lines] (0,0) grid (2,2) ; 
\pic[scale=3,rotate=45] at (1,1) {dfr}; 
  \end{tikzpicture}
&  
\begin{tikzpicture}
\draw[help lines] (0,0) grid (2,2) ; 
\pic[scale=3,red] at (1,1) {dfr}; 
  \end{tikzpicture} 
\\ \hline 
[{\color{red}scale}=3] & [scale=3,{\color{red}rotate}=45] & [scale=3,{\color{red}red}] \\ 
\hline 
\end{tabular} 

\bigskip

\begin{tabular}{|c|c|} \hline  
\parbox{8cm}{\BS{tikz} [scale=4] {
\BS{pic} at (0,0) \AC{dfr}; \\
\BS{pic} at (.5,0) [\RDD{transform shape}] \AC{dfr};
}}
&  
\tikz [scale=4] {
\pic at (0,0) {dfr};
\pic at (.5,0) [transform shape] {dfr};
}
\\ \hline  
\end{tabular}  

\bigskip

\begin{tabular}{|c|} \hline
\textbf{\TFRGB{Placement sur un chemin}{On a path}}
\\ \hline    
\BS{tikz} \BS{draw}
(0,0) to [out=10,in=170]
pic [{\color{red} near start}] \AC{dfr}
pic \AC{dfr} \\
pic [{\color{red} sloped, near end}] \AC{dfr} (10,0);
\\ \hline  
\tikz \draw
(0,0) to [out=10,in=170]
pic [near start] {dfr}
pic {dfr}
pic [sloped, near end] {dfr} (10,0);
\\ \hline  
\BS{draw} (0,0) to [out=10,in=170] pic [pos=.3]  \\
\AC{\RDD{code}=\AC{\BS{draw} circle [radius=3mm];}} (10,0)  ;
\\ \hline  
\tikz \draw (0,0) to [out=10,in=170] 
pic [pos=.3] {code={\draw circle [radius=3mm];}} (10,0)  ;
\\ \hline 
\end{tabular} 


\tikzset{ my pic/.pic = {
\path [pic actions] (0,0) circle[radius=3mm];
\draw (-3mm,-3mm) rectangle (3mm,3mm);} }

\bigskip
\begin{tabular}{|c|c|c|c|c|} \hline
 \multicolumn{5}{|l|}{ Définition :} \\
 \multicolumn{5}{|l|}{ \BSS{tikzset}\{ my pic/.pic = \{ }\\
 \multicolumn{5}{|l|}{  \BS{path} [\RDD{pic actions}] (0,0) circle[radius=3mm];} \\
\multicolumn{5}{|l|}{ \BS{draw} (-3mm,-3mm) rectangle (3mm,3mm); \}  \} }
 \\ \hline 
 \multicolumn{5}{|c|}{ Utilisation : \hspace{2cm}\BS{pic} [red] \AC{my pic} }
 \\ \hline 
\tikz \pic [red] {my pic}; 
&
\tikz \pic [draw] {my pic};
&  
\tikz \pic [draw=red] {my pic};
&  
\tikz \pic [draw, shading=ball] {my pic};
&  
\tikz \pic [fill=red!50] {my pic};
\\ \hline  
[red] & [draw]  &  [draw=red]  &  [draw, shading=ball]  &   [fill=red!50] 
\\ \hline 
\end{tabular} 

\bigskip

\begin{tabular}{|c|} \hline  
\BS{tikz} \BS{pic} foreach \BS{x} in \AC{1,1.5,...,10} at (\BS{x},0) \AC{dfr};
\\ \hline  
\tikz \pic foreach \x in {1,1.5,...,10} at (\x,0) {dfr};
\\ \hline 
\end{tabular} 

\bigskip

\begin{tabular}{|c|c|} \hline 
 \multicolumn{2}{|l|}{ \BS{fill} [green]
 (0,0) - - (1,0)pic [\RDD{behind path},scale=3] \AC{dfr} -- (1,1) -- (0,1) -- cycle ;} 
 \\ \hline 
 
\tikz \fill [green]
(0,0) -- (1,0)pic [behind path,scale=3] {dfr} -- (1,1) -- (0,1) -- cycle ;
&  
\tikz \fill [green]
(0,0) -- (1,0)pic [scale=3] {dfr} -- (1,1) -- (0,1) -- cycle ;
\\ \hline 
[\RDD{behind path},scale=3] & [scale=3] \\ 
\hline 
\end{tabular} 

\bigskip

\begin{tabular}{|c|c|} \hline 
\parbox{10cm}{ 
\BSS{tikzset}\AC{
pics/{\color{purple}mon cercle}/.style = \AC{
\RDD{background code} = \AC{ \BS{fill} circle [radius={\color{blue}\#1}]; }
}
}\\
\BS{tikz} [fill=green]
\BS{draw}[line width=3pt]  (0,0) pic \AC{{\color{purple}mon cercle}={\color{blue}2mm}} - - (1,1) pic \AC{{\color{purple}mon cercle}={\color{blue}5mm}};}
&
\tikzset{
pics/mon cercle/.style = {
background code = { \fill circle [radius=#1]; }
}
}
\tikz[baseline=.5cm,fill=green]
\draw[line width=3pt]  (0,0) pic {mon cercle=2mm} -- (1,1) pic {mon cercle=5mm};
\\ \hline
\parbox{10cm}{ 
\BSS{tikzset}\AC{
pics/{\color{purple}mon cercle}/.style = \AC{
\RDD{foreground  code} = \AC{ \BS{fill} circle [radius={\color{blue}\#1}]; }
}
}\\
\BS{tikz} [fill=green]
\BS{draw}[line width=3pt]  (0,0) pic \AC{{\color{purple}mon cercle}={\color{blue}2mm}} - - (1,1) pic \AC{{\color{purple}mon cercle}={\color{blue}5mm}};} 
&
\tikzset{
pics/mon cercle/.style = {
foreground code = { \fill circle [radius=#1]; }
}
}
\tikz[baseline=.5cm,fill=green]
\draw[line width=3pt]  (0,0) pic {mon cercle=2mm} -- (1,1) pic {mon cercle=5mm};
\\ \hline
\end{tabular} 

\bigskip

\begin{tabular}{|c|c|}\hline 
\parbox{11cm}{ 
 \BS{fill} [green](-1,0) - - (1,0) \\ 
pic [pics/\RDD{background code}=\AC{\BS{fill}[blue] (0.5,0.5) circle (1cm );}\\
, pics/code={\BS{fill}[red] (-1,-.5) rectangle (0.5,0.5);} ]\\ \AC{} - - (1,2) - - (-1,2) - - cycle ;}
&  
\tikz[baseline=1cm] 
\fill [green] (-1,0) -- (1,0)  
pic [pics/background code={\fill[blue] (.5,.5) circle (1cm );}
, pics/code={\fill[red] (-1,-.5) rectangle (.5,.5);} ]  {} -- (1,2) -- (-1,2) -- cycle ;
\\ \hline
  
\parbox{11cm}{ 
 \BS{fill} [green] (-1,0) - - (1,0) \\ 
 pic [pics/\RDD{foreground code}={\BS{fill}[blue] (0.5,0.5) circle (1cm );}\\
,pics/code=\AC{\BS{fill}[red] (-1,-.5) rectangle (0.5,0.5);} ]\\ \AC{} - - (1,2) - - (-1,2) - - cycle ;}
&  
\tikz[baseline=1cm]  \fill [green]
(-1,0) -- (1,0)pic [pics/foreground code={\fill[blue] (.5,.5)circle (1cm );},pics/code={\fill[red] (-1,-.5) rectangle (.5,.5);} ] {} -- (1,2) -- (-1,2) -- cycle ;
\\ \hline  
\parbox{11cm}{ 
 \BS{fill} [green](-1,0) - - (1,0) \\ 
pic [pics/\RDD{background code}=\AC{\BS{fill}[blue] (0.5 , 0.5) circle (1cm );}\\
,pics/code=\AC{\BS{fill}[red] (-1 , -0.5) rectangle (0.5 , 0.5);},\RDD{behind path} ]\\ 
\AC{} - - (1,2) - - (-1,2) - - cycle ;}
&  
\tikz[baseline=1cm]  
\fill [green](-1,0) -- (1,0) 
pic [pics/background  code={\fill[blue] (.5,.5) circle (1cm );} , pics/code={\fill[red] (-1,-.5) rectangle (.5,.5);},behind path ] {} -- (1,2) -- (-1,2) -- cycle ;
\\ \hline 
 
\parbox{11cm}{ 
 \BS{fill} [green]
(-1,0) - - (1,0) \\
 pic [pics/\RDD{foreground code}=\AC{\BS{fill}[blue] (0.5 , 0.5) circle (1cm );}\\
, pics/code=\AC{\BS{fill}[red] (-1,-.5) rectangle (0.5 , 0.5);},\RDD{behind path} ]\\ 
\AC{} - - (1,2) - - (-1,2) - - cycle ;
}
&  
\tikz[baseline=1cm]  
\fill [green](-1,0) -- (1,0)
pic [pics/foreground code={\fill[blue] (.5,.5)circle (1cm );},pics/code={\fill[red] (-1,-.5) rectangle (.5,.5);},behind path ] 
{} -- (1,2) -- (-1,2) -- cycle ;
\\ \hline 

\end{tabular} 
   

\newpage


%TODO problème avec "$\alpha$"


\SbSSCT{Images prédéfinies : Marquage des angles}{Drawing angles}
\begin{center}
\RRR{39}
\end{center}

 \maboite{\BS{usetikzlibrary}\AC{angles} }
\label{lib-angles}


\begin{tabular}{|c|c|} \hline 
\multicolumn{2}{|c|}{\BS{tikz} \BS{draw} (2,0) coordinate (A) - - (0,0) coordinate (B) }\\
\multicolumn{2}{|c|}{ - - (1,1) coordinate (C)   \RDD{pic} [draw] \AC{angle};}  \\ \hline 
\tikz \draw (2,0) coordinate (A) -- (0,0) coordinate (B) -- (1,1) coordinate (C) pic [draw] {angle};
&  
\tikz \draw (2,0) coordinate (A) -- (0,0) coordinate (B) -- (1,1) coordinate (C) pic [fill] {angle};
\\ \hline  
pic [{\color{red}  draw}] \AC{angle}
&  
pic [{\color{red}  fill}] \AC{angle}
\\ \hline 
\end{tabular} 

\bigskip

\begin{tabular}{|c|c|} \hline 
\multicolumn{2}{|c|}{\BS{tikz} \BS{draw} (2,0) coordinate (X) - - (0,0) coordinate (Y) }\\
\multicolumn{2}{|c|}{ - - (1,1) coordinate (Z)  pic [draw] \AC{\RDD{angle}= X- -Y- -Z};}  \\ \hline 
\tikz \draw (2,0) coordinate (X) -- (0,0) coordinate (Y) -- (1,1) coordinate (Z) pic [draw] {angle= X--Y--Z};
&  
\tikz \draw (2,0) coordinate (X) -- (0,0) coordinate (Y) -- (1,1) coordinate (Z) pic [draw] {angle= Z--Y--X};
\\ \hline  
pic [{\color{red}  draw}] \AC{angle= X- -Y- -Z}
&  
pic [{\color{red}  fill}] \AC{angle = Z- -Y- -X}
\\ \hline
\multicolumn{2}{|c|}{\dft{} :  angle= A- -B- -C }
\\ \hline
\end{tabular} 

\bigskip


\begin{tabular}{|c|c|} \hline 
\multicolumn{2}{|c|}{\BS{tikz} \BS{draw} (2,0) coordinate (A) - - (0,0) coordinate (B) }\\
\multicolumn{2}{|c|}{ - - (1,1) coordinate (C)   pic [draw,->] \AC{angle};}  \\ \hline 
\tikz \draw (2,0) coordinate (A) -- (0,0) coordinate (B) -- (1,1) coordinate (C) pic [draw=red,->] {angle};
&  
\tikz \draw (2,0) coordinate (A) -- (0,0) coordinate (B) -- (1,1) coordinate (C) pic [fill=red!50] {angle};
\\ \hline  
pic [draw,{\color{red}  ->}] \AC{angle}
&  
pic [fill,{\color{red}  fill=red!50}] \AC{angle}
\\ \hline 
\end{tabular}

\bigskip

\begin{tabular}{|c|c|} \hline 
\multicolumn{2}{|c|}{\BS{tikz} \BS{draw} (2,0) coordinate (A) - - (0,0) coordinate (B) }\\
\multicolumn{2}{|c|}{ - - (1,1) coordinate (C)   pic [draw,\RDD{angle radius}=1cm] \AC{angle};}  \\ \hline 
\tikz \draw (2,0) coordinate (A) -- (0,0) coordinate (B) -- (1,1) coordinate (C) pic [draw,angle radius=1cm] {angle};
&  
\tikz \draw (2,0) coordinate (A) -- (0,0) coordinate (B) -- (1,1) coordinate (C) pic [fill,angle radius=1cm] {angle};
\\ \hline  
pic [draw,{\color{red}  angle radius=1cm}] \AC{angle}
&  
pic [fill,{\color{red}  angle radius=1cm}] \AC{angle}
\\ \hline 
\multicolumn{2}{|c|}{\dft{} :  angle radius=5mm }
\\ \hline
\end{tabular}

\bigskip

\maboite{\BS{usetikzlibrary}\AC{quotes}}
\label{quotes}

\begin{tabular}{|c|} \hline  
\BS{tikz} \BS{draw} (3,0) coordinate (A)
- - (0,1) coordinate (B)
- - (1,2) coordinate (C) \\
pic [draw,{\color{red}\verb|"|\$\BS{alpha}\$ \verb|"|}] \AC{angle};
\\ \hline  
\tikz \draw (3,0) coordinate (A)
-- (0,1) coordinate (B)
-- (1,2) coordinate (C)
pic [draw, "$\alpha$"] {angle};
\\ \hline 
\end{tabular} 

\bigskip

\begin{tabular}{|l|} \hline  
\BS{tikz} \BS{draw} (2,0) coordinate (A) \\ -  - (0,0) coordinate (B)
- - (1,2) coordinate (C)
\\ 
pic [draw, \verb|"| \$\BS{alpha}\$\verb|"|, {\color{red}angle eccentricity}=1]] \AC{angle};

\\ \hline 
\end{tabular} 
\begin{tabular}{|l|c|c|} \hline   
 \tikz \draw (2,0) coordinate (A)
-- (0,0) coordinate (B)
-- (1,2) coordinate (C)
pic ["$\alpha$", draw, angle eccentricity=1] {angle};
&
\tikz \draw (2,0) coordinate (A)
-- (0,0) coordinate (B)
-- (1,2) coordinate (C)
pic ["$\alpha$", draw, angle eccentricity=1.5] {angle};

\\ \hline 
\RDD{angle eccentricity}=1 & \RDD{angle eccentricity}=1.5
\\ \hline
 \multicolumn{2}{|c|}{ \dft{} : angle eccentricity= 0.6  } 
 \\ \hline
\end{tabular} 

\bigskip

 \begin{tabular}{|c|}  \hline  
 \BS{tikz} \{ \BS{draw} (2,0) coordinate (A)
 - - (0,0) coordinate (B)
 - - (1,2) coordinate (C) \\
 pic {\color{red}(xxx)} [draw,\verb|"|\$\BS{alpha}\$\verb|"|,angle radius= 1cm ] \AC{angle};\\
\BS{draw} {\color{red}(xxx)}circle [radius=5pt] ; \} 
 \\  \hline  
\tikz {
\draw (2,0) coordinate (A) -- (0,0) coordinate (B) -- (1,1) coordinate (C) pic (xxx) ["$\alpha$", draw,angle radius= 1cm ] {angle};
\draw (xxx) circle [radius=5pt];
} 
 \\  \hline 
 \end{tabular} 


\newpage

\SSCT{Les coordonnées }{Coordinates}
 

\SbSSCT{Quadrillage}{Grid}


\begin{tabular}{|c|}\hline 
\tikz \draw(0,0) grid (2,2); 
\\ \hline 
\BS{draw} (0,0) \RDD{grid} (2,2); \RRR{14-8}
\\ \hline 
\end{tabular} 


\bigskip
\begin{tabular}{|c|c|c|c|} \hline 
\multicolumn{4}{|c|}{ \BS{draw} (0,0) grid  [\RDD{step}=.75cm] (0,0) grid (3,3);   }\\ 
\hline  
\begin{tikzpicture}
\draw[dotted](0,0) grid (3,3); 
\draw[red] (0,0) grid [step=.75cm] (3,3);
\end{tikzpicture}
&  
\begin{tikzpicture}
\draw[dotted](0,0) grid (3,3); 
\draw[red] (0,0) grid [xstep=.75cm] (3,3);
\end{tikzpicture}
&  
\begin{tikzpicture}
\draw[dotted](0,0) grid (3,3); 
\draw[red] (0,0) grid [ystep=.75cm] (3,3);
\end{tikzpicture}
&
\begin{tikzpicture}
\draw[dotted](0,0) grid (3,3); 
\draw[red] (0,0) grid [step=(45:1)] (3,3);
\end{tikzpicture}
\\ \hline 
step=.75cm & x step=.75cm & ystep=.75cm  & step=(45:1)
\\ \hline 
\end{tabular} 

\bigskip

\begin{tabular}{|c|c|} \hline 
 
\BS{draw}[red] (0,0) grid [\RDD{rotate}=45] (3,3);
&  
\BS{draw}[\RDD{help lines}] (0,0) grid  (3,3);
\\ \hline  
\begin{tikzpicture}
\draw[dotted](0,0) grid (3,3); 
\draw[red] (0,0) grid [rotate=45] (3,3);
\end{tikzpicture}
& 
\tikz \draw[help lines] (0,0) grid (3,3); \\ 
\hline 
\end{tabular} 

\newpage 


\SbSSCT{Coordonnées}{Coordinates}
\begin{center}
\RRR{13-2-1}
\end{center}


\SbSbSSCT{Système de coordonnées \og canvas \fg}{Canvas coordinates}

\noindent


\tikzset{every picture/.style=blue,very thick,inner sep=0pt}

\begin{tabular}{|c|c|} \hline 
\TFRGB{Explicite}{explicit}  & \TFRGB{Implicite}{implicit}
\\ \hline
\begin{tikzpicture}
\draw[help lines] (0,0) grid (3,2);
\fill (canvas cs:x=2cm,y=1.5cm) circle (2pt);
\end{tikzpicture}
&
\begin{tikzpicture}
\draw[help lines] (0,0) grid (3,2);
\fill (2,1.5) circle (2pt);
\end{tikzpicture}

\\ \hline  
 \BS{fill} (\RDD{canvas cs}:\blll{x=2cm,y=1.5cm}) circle (2pt);
& \BS{fill} {\color{blue}(2cm,1.5cm)} circle (2pt);
\\ \hline 
\end{tabular} 


\SbSbSSCT{Système de coordonnées polaire \og canvas \fg}{Polar coordinates}

\noindent


\begin{tabular}{|c|c|c|} \hline
\TFRGB{Explicite}{explicit}  & \TFRGB{Implicite}{implicit}
\\ \hline
\begin{tikzpicture}
\draw[help lines] (0,0) grid (3,2);
\draw [dotted](0,2) arc (90 :0 :2);
\draw [dotted](0,0) --(2,2);
\fill (canvas polar cs:angle=45,radius=2cm) circle (2pt);
\end{tikzpicture}
&
\begin{tikzpicture}
\draw[help lines] (0,0) grid (3,2);
\draw [dotted](0,2) arc (90 :0 :2);
\draw [dotted](0,0) --(2,2);
\fill (45:2cm) circle (2pt);
\end{tikzpicture}
\\ \hline 
\BS{fill} (\RDD{canvas polar cs}:\RDD{angle}=45,\RDD{radius}=2cm) circle (2pt);
&
\BS{fill} {\color{blue}(45:2cm)} circle (2pt);
\\ \hline 
\end{tabular} 

\bigskip
\begin{tabular}{|c|} \hline  
\begin{tikzpicture}
\draw[help lines] (0,0) grid (3,2);
\draw [dotted](0,2) arc (90 :0 :3 and 2);
\draw [dotted](0,0) --(3,2);
\fill (canvas polar cs:angle=45,x radius=3cm,y radius=2cm) circle (2pt);
\end{tikzpicture}
\\ \hline  
\BS{fill} (canvas polar cs:angle=45,\RDD{x radius}=3cm,\RDD{y radius}=2cm) circle (2pt);
\\ \hline 
\end{tabular}


\SbSbSSCT{Système de coordonnées  xyz}{xyz coordinates}

\noindent


\begin{tabular}{|c|c|c|} \hline 
\begin{tikzpicture}[->]
\draw (0,0) -- (xyz cs:x=1);
\draw[red] (0,0) -- (xyz cs:y=1);
\draw[magenta] (0,0) -- (xyz cs:z=1);
\end{tikzpicture}
&
\begin{tikzpicture}[->]
\draw (0,0) -- (1,0,0);
\draw[red]  (0,0) -- (0,1,0);
\draw[magenta]  (0,0) -- (0,0,1);
\end{tikzpicture}
\\ \hline 
\BS{draw} (0,0) - - (\RDD{xyz cs}:x=1); & \BS{draw}  (0,0) - - (1,0,0); \\
\BS{draw}[red]  (0,0) - - (\RDD{xyz cs}:y=1); &  \BS{draw}[red] (0,0) - - (0,1,0); \\
\BS{draw}[magenta]  (0,0) - - (\RDD{xyz cs}:z=1); &  \BS{draw}[magenta]   (0,0) - - (0,0,1); 
\\ \hline 

\end{tabular} 

 
\newpage

\SbSbSSCT{Coordinate system xyz polar}{Coordinate system xyz polar}

\noindent

\begin{tabular}{|c|c|c|} \hline
\TFRGB{Explicite}{explicit}  & \TFRGB{Implicite}{implicit}
\\ \hline
\begin{tikzpicture}
\draw[help lines] (0,0) grid (3,2);
\draw [dotted](0,2) arc (90 :0 :2);
\draw [dotted](0,0) --(2,2);
\fill (xyz polar cs:angle=45,radius=2) circle (2pt);
\end{tikzpicture}
&
\begin{tikzpicture}
\draw[help lines] (0,0) grid (3,2);
\draw [dotted](0,2) arc (90 :0 :2);
\draw [dotted](0,0) --(2,2);
\fill (45:2) circle (2pt);
\end{tikzpicture}
\\ \hline 
\BS{fill} (\RDD{xyz polar cs}:\RDD{angle}=45,\RDD{radius}=2) circle (2pt);
&
\BS{fill} {\color{blue}(45:2cm)} circle (2pt);
\\ \hline 
\end{tabular} 

\bigskip
\begin{tabular}{|c|} \hline  
\begin{tikzpicture}
\draw[help lines] (0,0) grid (3,2);
\draw [dotted](0,2) arc (90 :0 :3 and 2);
\draw [dotted](0,0) --(3,2);
\fill (xyz polar cs:angle=45,x radius=3,y radius=2) circle (2pt);
\end{tikzpicture}
\\ \hline  
\BS{fill} (xyz polar cs:angle=45,\RDD{x radius}=3,\RDD{y radius}=2) circle (2pt);
\\ \hline 
\end{tabular} 

\bigskip

\begin{tabular}{|c|c|c|} \hline
\multicolumn{2}{|c|}{\BS{begin}\AC{tikzpicture}{\color{red}[x=1.5cm,y=1cm]} }
\\ \hline
\begin{tikzpicture}[x=1.5cm,y=1cm]
\draw[help lines] (0,0) grid (3,2);
\draw [dotted](0,2) arc (90 :0 :2);
\draw [dotted](0,0) --(2,2);
\fill (xyz polar cs:angle=45,radius=2) circle (2pt);
\end{tikzpicture}
&
\begin{tikzpicture}[x=1.5cm,y=1cm]
\draw[help lines] (0,0) grid (3,2);
\draw [dotted](0,2) arc (90 :0 :2);
\draw [dotted](0,0) --(2,2);
\fill (45:2) circle (2pt);
\end{tikzpicture}
\\ \hline 
\BS{fill} (\RDD{xyz polar cs}:\RDD{angle}=45,\RDD{radius}=2) circle (2pt);
&
\BS{fill} {\color{blue}(45:2cm)} circle (2pt);
\\ \hline 
\end{tabular} 
\bigskip

\begin{tabular}{|c|c|c|} \hline
\multicolumn{2}{|c|}{\BS{begin}\AC{tikzpicture}{\color{red}[x=\AC{(0cm,1cm)},y=\AC{(-1cm,0cm)}]} }
\\ \hline
\begin{tikzpicture}[x={(0cm,1cm)},y={(-1cm,0cm)}]
\draw[help lines] (0,0) grid (3,2);
\draw [dotted](0,2) arc (90 :0 :2);
\draw [dotted](0,0) --(2,2);
\fill (xyz polar cs:angle=45,radius=2) circle (2pt);
\end{tikzpicture}
&
\begin{tikzpicture}[x={(0cm,1cm)},y={(-1cm,0cm)}]
\draw[help lines] (0,0) grid (3,2);
\draw [dotted](0,2) arc (90 :0 :2);
\draw [dotted](0,0) --(2,2);
\fill (45:2) circle (2pt);
\end{tikzpicture}
\\ \hline 
\BS{fill} (\RDD{xyz polar cs}:\RDD{angle}=45,\RDD{radius}=2) circle (2pt);
&
\BS{fill} {\color{blue}(45:2cm)} circle (2pt);
\\ \hline 
\end{tabular} 

\SbSbSSCT{Coordonnées barycentriques}{Barycentric coordinates}

\begin{center}
\RRR{13-2-2}
\end{center}

\begin{tabular}{|c|c|c|} \hline
\multicolumn{3}{|c|}{  \BS{node} [circle,fill=red!20] at (\RDD{barycentric cs}:A=0.6,B=0.3 ) \AC{X};   }\\ 
\hline
\begin{tikzpicture}[scale=.6]
\draw[help lines] (0,0) grid (4,4);
\node[circle,fill=green!20,] (A) at (0,0) {A};
\node[circle,fill=green!20,] (B) at (4,0) {B};
\node[circle,fill=red!20] at (barycentric cs:A=0.3,B=0.3 ) {X};
\end{tikzpicture}
&
\begin{tikzpicture}[scale=.6]
\draw[help lines] (0,0) grid (4,4);
\node[circle,fill=green!20,] (A) at (0,0) {A};
\node[circle,fill=green!20,] (B) at (4,0) {B};
\node[circle,fill=green!20,] (C) at (4,4) {C};
\node[circle,fill=red!20] at (barycentric cs:A=0.4,B=0.4 ,C=.4) {X};
\end{tikzpicture}
&
\begin{tikzpicture}[scale=.6]
\draw[help lines] (0,0) grid (4,4);
\node[circle,fill=green!20,] (A) at (0,0) {A};
\node[circle,fill=green!20,] (B) at (4,0) {B};
\node[circle,fill=green!20,] (C) at (1,4) {C};
\node[circle,fill=green!20,] (D) at (4,4) {D};
\node[circle,fill=red!20] at (barycentric cs:A=0.5,B=0.5,C=.5,D=.5 ) {X};
\end{tikzpicture}
\\ \hline
A=0.3,B=0.3 & A=0.4,B=0.4 ,C=.4 & A=0.5,B=0.5,C=.5,D=.5 
\\ \hline
\begin{tikzpicture}[scale=.6]
\draw[help lines] (0,0) grid (4,4);
\node[circle,fill=green!20,] (A) at (0,0) {A};
\node[circle,fill=green!20,] (B) at (4,0) {B};
\node[circle,fill=red!20] at (barycentric cs:A=0.6,B=0.3 ) {X};
\end{tikzpicture}
&
\begin{tikzpicture}[scale=.6]
\draw[help lines] (0,0) grid (4,4);
\node[circle,fill=green!20,] (A) at (0,0) {A};
\node[circle,fill=green!20,] (B) at (4,0) {B};
\node[circle,fill=green!20,] (C) at (4,4) {C};
\node[circle,fill=red!20] at (barycentric cs:A=0.2,B=0.4 ,C=.6) {X};
\end{tikzpicture}
&
\begin{tikzpicture}[scale=.6]
\draw[help lines] (0,0) grid (4,4);
\node[circle,fill=green!20,] (A) at (0,0) {A};
\node[circle,fill=green!20,] (B) at (4,0) {B};
\node[circle,fill=green!20,] (C) at (1,4) {C};
\node[circle,fill=green!20,] (D) at (4,4) {D};
\node[circle,fill=red!20] at (barycentric cs:A=0.2,B=0.4,C=.6,D=.8 ) {X};
\end{tikzpicture}
\\ \hline
A=0.6,B=0.3 & A=0.2,B=0.4 ,C=.6 & A=0.2,B=0.4,C=.6,D=.8
\\ \hline
\end{tabular}

\SbSbSSCT{Coordonnées nominatives : n\oe ud}{Named coordinates: nodes}

\begin{center}
\RRR{13-2-3}
\end{center}

\begin{tabular}{|c|c|} \hline  
\begin{tikzpicture}[blue,very thick,baseline=1cm]
\draw[help lines] (0,0) grid (3,3);
\coordinate (centre) at (1.5,1.5) ;
\coordinate (A) at (.5,.5) ;
\coordinate (B) at (2.5,2.5) ;
\fill (centre) circle (3pt);
\draw[red] (A) rectangle (B) ;
\end{tikzpicture}
&  
\parbox[c]{8cm}{
\BSS{coordinate} {\color{blue}(centre)} at(1.5,1.5) ; \\
\BSS{coordinate} {\color{blue}(A)} at (.5,.5) ;\\
\BSS{coordinate} {\color{blue}(B)} at  (2.5,2.5) ;\\
\\
\BS{fill} {\color{blue}(centre)} circle (3pt);\\
\BS{draw}[red] {\color{blue}(A)} rectangle {\color{blue}(B)} ;\\
}
\\ \hline 
\end{tabular} 


\TFRGB{voir aussi}{see also} page \pageref{noeuds}


\SbSbSSCT{Coordonnées relatives à un noeud}{Coordinates relative to a node}

\noindent

\begin{tabular}{|c|c|c|c|} \hline
\multicolumn{4}{|l|}{  \BS{node} [draw,fill=green!20,] (A) at (1,1) \AC{\BS{huge}  noeud}; }\\ 
\multicolumn{4}{|l|}{  \BS{fill}[red] (\RDD{node cs}:\RDD{name}=A,\RDD{anchor}=south) circle (3pt);   }\\ 
\hline

\begin{tikzpicture}
\draw[help lines] (0,0) grid (2,2);
\node[draw,fill=green!20,] (A) at (1,1) {\huge noeud};
\fill[red] (node cs:name=A,anchor=south) circle (3pt);
\end{tikzpicture}
&
\begin{tikzpicture}
\draw[help lines] (0,0) grid (2,2);
\node[draw,fill=green!20,] (A) at (1,1) {\huge noeud};
\fill[red] (node cs:name=A,anchor=west) circle (3pt);
\end{tikzpicture}
&
\begin{tikzpicture}
\draw[help lines] (0,0) grid (2,2);
\node[draw,fill=green!20,] (A) at (1,1) {\huge noeud};
\fill[red] (node cs:name=A,anchor=north) circle (3pt);
\end{tikzpicture}
&
\begin{tikzpicture}
\draw[help lines] (0,0) grid (2,2);
\node[draw,fill=green!20,] (A) at (1,1) {\huge noeud};
\fill[red] (node cs:name=A,anchor=east) circle (3pt);
\end{tikzpicture}
\\ \hline
name=A,anchor=south & name=A,anchor=west & name=A,anchor=north & name=A,anchor=east
\\ \hline
\end{tabular}

\bigskip

\begin{tabular}{|c|c|c|c|} \hline
\multicolumn{4}{|l|}{  \BS{node} [draw,fill=green!20,] \blll{(A)} at (1,1) \AC{\BS{huge}  noeud}; }\\ 
\multicolumn{4}{|l|}{  \BS{fill}[red] (\blll{A}.south) circle (3pt);   }\\ 
\hline

\begin{tikzpicture}
\draw[help lines] (0,0) grid (2,2);
\node[draw,fill=green!20,] (A) at (1,1) {\huge noeud};
\fill[red] (A.south) circle (3pt);
\end{tikzpicture}
&
\begin{tikzpicture}
\draw[help lines] (0,0) grid (2,2);
\node[draw,fill=green!20,] (A) at (1,1) {\huge noeud};
\fill[red] (A.west) circle (3pt);
\end{tikzpicture}
&
\begin{tikzpicture}
\draw[help lines] (0,0) grid (2,2);
\node[draw,fill=green!20,] (A) at (1,1) {\huge noeud};
\fill[red] (A.north) circle (3pt);
\end{tikzpicture}
&
\begin{tikzpicture}
\draw[help lines] (0,0) grid (2,2);
\node[draw,fill=green!20,] (A) at (1,1) {\huge noeud};
\fill[red] (A.east) circle (3pt);
\end{tikzpicture}
\\ \hline
A.south & A.west & A.north & A.east
\\ \hline
\end{tabular}



\bigskip
\begin{tabular}{|c|c|c|c|} \hline
\multicolumn{4}{|c|}{  \BS{fill}[red] (node cs:\RDD{name}=A,\RDD{angle}=0) circle (3pt);  }\\ 
\hline

\begin{tikzpicture}
\draw[help lines] (0,0) grid (2,2);
\node[draw,fill=green!20,] (A) at (1,1) {\huge noeud};
\fill[red] (node cs:name=A,angle=0) circle (3pt);
\end{tikzpicture}
&
\begin{tikzpicture}
\draw[help lines] (0,0) grid (2,2);
\node[draw,fill=green!20,] (A) at (1,1) {\huge noeud};
\fill[red] (node cs:name=A,angle=-30) circle (3pt);
\end{tikzpicture}
&
\begin{tikzpicture}
\draw[help lines] (0,0) grid (2,2);
\node[draw,fill=green!20,] (A) at (1,1) {\huge noeud};
\fill[red] (node cs:name=A,angle=-90) circle (3pt);
\end{tikzpicture}
&
\begin{tikzpicture}
\draw[help lines] (0,0) grid (2,2);
\node[draw,fill=green!20,] (A) at (1,1) {\huge noeud};
\fill[red] (node cs:name=A,angle=-150) circle (3pt);
\end{tikzpicture}
\\ \hline
name=A,angle=0 & name=A,angle=-30 & nname=A,angle=-90 & name=A,angle=-150
\\ \hline
\end{tabular}

\bigskip


\begin{tabular}{|c|c|c|c|} \hline
\multicolumn{4}{|c|}{  \BS{fill}[red] (A.0) circle (3pt);  }\\ 
\hline

\begin{tikzpicture}
\draw[help lines] (0,0) grid (2,2);
\node[draw,fill=green!20,] (A) at (1,1) {\huge noeud};
\fill[red] (A.0) circle (3pt);
\end{tikzpicture}
&
\begin{tikzpicture}
\draw[help lines] (0,0) grid (2,2);
\node[draw,fill=green!20,] (A) at (1,1) {\huge noeud};
\fill[red] (A.-30) circle (3pt);
\end{tikzpicture}
&
\begin{tikzpicture}
\draw[help lines] (0,0) grid (2,2);
\node[draw,fill=green!20,] (A) at (1,1) {\huge noeud};
\fill[red] (A.-90) circle (3pt);
\end{tikzpicture}
&
\begin{tikzpicture}
\draw[help lines] (0,0) grid (2,2);
\node[draw,fill=green!20,] (A) at (1,1) {\huge noeud};
\fill[red] (A.-150) circle (3pt);
\end{tikzpicture}
\\ \hline
A.0 & A.-30 & A.-90 & A.-150
\\ \hline
\end{tabular}

\TFRGB{voir aussi}{see also} page \pageref{nomnoeud}


\newpage

\SbSbSSCT{Coordonnées relatives à deux points}{Coordinates relative to two points}
\begin{center}
\RRR{13-3-1}
\end{center}

\begin{tabular}{|c|c|} \hline
\multicolumn{2}{|c|}{  \BS{node} [circle,fill=red!20] at (1,1 {\color{red}|-} 3,3) \AC{X}   }\\ 
\hline
\begin{tikzpicture}
\draw[help lines] (0,0) grid (4,4);
\node[circle,fill=green!20,] (A) at (1,1) {A};
\node[circle,fill=green!20,] (B) at (3,3) {B};
\node[circle,fill=red!20] at (1,1 |- 3,3) {X};
\end{tikzpicture}
&
\begin{tikzpicture}
\draw[help lines] (0,0) grid (4,4);
\node[circle,fill=green!20,] (A) at (1,1) {A};
\node[circle,fill=green!20,] (B) at (3,3) {B};
\node[circle,fill=red!20] at (1,1 -| 3,3) {X};
\end{tikzpicture}
\\ \hline
at (1,1 {\color{red}|-} 3,3)
&
at (1,1 {\color{red}-|} 3,3)
\\ \hline
\end{tabular}



\SbSbSSCT{Coordonnée relative à une intersection}{Coordinates relative to an intersection}
\begin{center}
\RRR{13-3-2}
\end{center}

 \maboite{\BS{usetikzlibrary}\AC{intersections}}
\label{lib-intersections}


\begin{tabular}{|c|c|c|c|} \hline 
\multicolumn{4}{|l|}{  \BS{draw} [\RDD{name path}=XXX] (2,1) circle  (1cm);   }\\ 
\multicolumn{4}{|l|}{  \BS{draw} [\RDD{name path}=YYY] (0.5,0.5) rectangle +(3,1);   }\\ 
\multicolumn{4}{|l|}{ \BS{fill} [red,\RDD{ name intersections}=\AC{of=xxx and YYY}]
(\RDD{intersection}-1) circle (2pt)   }\\ 
\hline 
\begin{tikzpicture}[scale=.8]
\draw [help lines] grid (4,2);
\draw [name path=XXX] (2,1) circle  (1cm);
\draw [name path=YYY] (0.5,0.5) rectangle +(3,1);
\fill [red, name intersections={of=XXX and YYY}]
(intersection-1) circle (2pt)  ;
\end{tikzpicture}
& 
\begin{tikzpicture}[scale=.8]
\draw [help lines] grid (4,2);
\draw [name path=XXX] (2,1) circle  (1cm);
\draw [name path=YYY] (0.5,0.5) rectangle +(3,1);
\fill [red, name intersections={of=XXX and YYY}] (intersection-2) circle (2pt) ;
\end{tikzpicture} 
&  
\begin{tikzpicture}[scale=.8]
\draw [help lines] grid (4,2);
\draw [name path=XXX] (2,1) circle  (1cm);
\draw [name path=YYY] (0.5,0.5) rectangle +(3,1);
\fill [red, name intersections={of=XXX and YYY}] (intersection-3) circle (2pt) ;
\end{tikzpicture}
&  
\begin{tikzpicture}[scale=.8]
\draw [help lines] grid (4,2);
\draw [name path=XXX] (2,1) circle  (1cm);
\draw [name path=YYY] (0.5,0.5) rectangle +(3,1);
\fill [red, name intersections={of=XXX and YYY}] (intersection-4) circle (2pt) ;
\end{tikzpicture}
\\ 
\hline intersection-1 & intersection-2 &intersection-3  & intersection-4 \\ 
\hline 
\end{tabular} 

\bigskip

\begin{tabular}{|c|} \hline  
\BS{fill} [red, name intersections=\AC{of=XXX and YYY}] \\
(intersection-1) circle (2pt) {\color{red} node[black,above right] \AC{point a}} ;
\\ \hline  
\begin{tikzpicture}
\draw [help lines] grid (4,2);
\draw [name path=XXX] (2,1) circle  (1cm);
\draw [name path=YYY] (0.5,0.5) rectangle +(3,1);
\fill [red, name intersections={of=XXX and YYY}]
(intersection-1) circle (2pt) node[black,above right] {point a} ;
\end{tikzpicture} 
\\ \hline 
\end{tabular} 

\bigskip

\begin{tabular}{|c|} \hline 
\BS{fill} [red, name intersections=\AC{of=XXX and YYY, \RDD{name}=ZZZ}]; \\
\BS{draw} [red] (ZZZ-1) - - (ZZZ-3); \BS{draw} [green] (ZZZ-2) - - (ZZZ-4);
\\ \hline  
\begin{tikzpicture}
\draw [help lines] grid (4,2);
\draw [name path=XXX] (2,1) circle  (1cm);
\draw [name path=YYY] (0.5,0.5) rectangle +(3,1);
\fill [red, name intersections={of=XXX and YYY, name=ZZZ}];
\draw [red] (ZZZ-1) -- (ZZZ-3);
\draw [green] (ZZZ-2) -- (ZZZ-4);
\end{tikzpicture}
\\ \hline 
\end{tabular} 

\bigskip
\begin{tabular}{|c|} \hline  
\BS{fill} [red, name intersections=\AC{of=XXX and YYY , \RDD{by}=\AC{a,b,c,d}}]; \\
\BS{draw} [red] (a) - - (c); \hspace{1cm} \BS{draw} [green] (b) - - (d);
\\ \hline   
\begin{tikzpicture}
\draw [help lines] grid (4,2);
\draw [name path=XXX] (2,1) circle  (1cm);
\draw [name path=YYY] (0.5,0.5) rectangle +(3,1);
\fill [red, name intersections={of=XXX and YYY, by={a,b,c,d}}];
\draw [red] (a) -- (c);
\draw [green] (b) -- (d);
\end{tikzpicture}
\\ \hline 
\end{tabular} 

\bigskip

\begin{tabular}{|c|} \hline  
\BS{fill} [name intersections=\AC{of=XXX and YYY, name=i, \RDD{total}=\BS{t}}] [red] \\
\BS{foreach} \BS{s} in \AC{1,...,\BS{t}} \AC{(i-\BS{s}) circle (2pt) node[black,above right] \AC{\BS{s}}}
\\ \hline  
\begin{tikzpicture}
\draw [help lines] grid (4,2);
\draw [name path=XXX] (2,1) circle  (1cm);
\draw [name path=YYY] (0.5,0.5) rectangle +(3,1);
\fill [name intersections={of=XXX and YYY , name=i, total=\t}]
[red]
\foreach \s in {1,...,\t}{(i-\s) circle (2pt) node[black,above right] {\s}};
\end{tikzpicture}
\\ \hline 
\end{tabular} 



\newpage

\SbSbSSCT{Position calculée avec le module  \og  pgfmath \fg}{Calculated positions with  \og  pgfmath \fg }

\begin{center}
\RRR{13-2-1}
\end{center}

\TFRGB{Ce module est chargé automatiquement avec le module Tikz}{Package automatically loaded with Tikz} 

\begin{tabular}{|c|} \hline 
\begin{tikzpicture}
\draw[help lines] (0,0) grid (4,2);
\fill [red] (canvas cs:x=2cm+1.5cm,y=1.5cm-1cm) circle (3pt);
\fill [blue] (2cm,1.5cm) circle (3pt);
\draw[dashed] (2,1.5) -| (3.5,.5);
\end{tikzpicture}
\\ \hline 
\emph{\TFRGB{Explicite}{explicit}} 
 : \BS{fill} [red] (\RDD{canvas cs}:x=2cm+1.5cm,y=1.5cm-1cm) circle (3pt);
 \\  \hline 
\emph{\TFRGB{Implicite}{implicit}} :  \BS{fill} [red] {\color{red}(2cm+1.5cm,1.5cm-1cm)} circle (3pt);
\\ \hline 
\end{tabular} 

\bigskip
\begin{tabular}{|c|c|c|} \hline 
\begin{tikzpicture}[baseline=0pt]
\draw[help lines] (0,0) grid (4,4);
 \draw[dashed] (2,2) circle (2);
\fill[red](2+ 2*cos 30,2+2*sin 30) circle (3pt);
\fill[magenta](2+ 2*cos{(120)},2+2*sin{(120)}) circle (3pt);
\end{tikzpicture}
&
\parbox[c]{8cm}{
 \BS{draw}[dashed] (2,2) circle (2);\\
 \smallskip
 \BS{fill} [red]{\color{red}(2+ 2*cos 30 , 2+2*sin 30)} circle (3pt);\\
  \smallskip
 \BS{fill}[magenta] {\color{red}(2+2*cos\AC{(120)} , 2+2*sin\AC{(120)})} circle (3pt); 
 }
\\ \hline 
\end{tabular} 

\SbSbSSCT{Position calculée avec \og library calc \fg}{Calculated positions with \og  calc  library calc \fg}

\begin{center}
\RRR{13-5}
\end{center}
\label{lib-calc}

 \maboite{\BS{usetikzlibrary}\AC{calc}}
 
\begin{tabular}{|c|c|} \hline  
\begin{tikzpicture}[baseline=0pt]
\draw [help lines] (0,0) grid (3,2);
\node (a) at (1,1) {A};
\fill [red] ($(a) + 2/3*(1cm,0)$) circle (2pt);
\fill [red] ($(a) + 4/3*(1cm,0)$) circle (2pt);
\end{tikzpicture}
&
\parbox{8cm}{
\BS{node} (a) at (1,1) \AC{A}; \\
\BS{fill} [red] {\color{red} (\$(a) + 2/3*(1cm,0)\$)} circle (2pt); \\
\BS{fill} [red] {\color{red}(\$(a) + 4/3*(1cm,0)\$)} circle (2pt); \\
}
\\ 
\hline 
\end{tabular} 

\SbSbSSCT{Tangentes avec \og library calc \fg}{Tangents with  \og calc library  \fg}

\begin{center}
\RRR{13-2-4}
\end{center}

\begin{tabular}{|c|c|} \hline 
\multicolumn{2}{|l|}{\BS{node}[fill=green!20] (a) at (3,1.5) \AC{A}; } \\
\multicolumn{2}{|l|}{\BS{fill}[red] (\RDD{tangent cs}:\RDD{node}=c,\RDD{point}=\AC{(A)},\RDD{solution}=1);  }\\ 
\hline
\begin{tikzpicture}
\draw[help lines] (0,0) grid (4,2);
\node[fill=green!20] (A) at (3,1.5) {A};
\node [circle,draw] (c) at (1,1) [minimum size=1.5cm] {$c$};
\draw[red,dashed] (A) - -(tangent cs:node=c,point={(A)},solution=1) ;
\draw[red,dashed] (1,1) - -(tangent cs:node=c,point={(3,1.5)},solution=1) ;
\fill[red] (tangent cs:node=c,point={(A)},solution=1) circle (3pt);
\end{tikzpicture}
&
\begin{tikzpicture}
\draw[help lines] (0,0) grid (4,2);
\node[fill=green!20] (A) at (3,1.5) {A};
\node [circle,draw] (c) at (1,1) [minimum size=1.5cm] {$c$};
\draw[red,dashed] (A) - -(tangent cs:node=c,point={(A)},solution=2) ;
\draw[red,dashed] (1,1) - -(tangent cs:node=c,point={(A)},solution=2) ;
\fill[red] (tangent cs:node=c,point={(A)},solution=2) circle (3pt);
\end{tikzpicture}
\\ \hline
\RDD{solution}=1 & \RDD{solution}=2
\\ \hline
\end{tabular} 

\newpage

\SbSbSSCT{Point à pourcentage donné }{Percentage position }

\begin{center}
\RRR{13-5-3}
\end{center}


\begin{tabular}{|c|c|} \hline  
\multicolumn{2}{|c|}{\BS{fill}[red] ({\color{red}\$(0,1)!.25!(4,1)\$}) circle (4pt); } \\  \hline  

\begin{tikzpicture}
\draw [help lines] (0,0) grid (4,2);
\draw [line width= 3pt] (0,1) -- (4,1);
\fill[red] ($(0,1)!.25!(4,1)$) circle (4pt);
\end{tikzpicture}
&  
\begin{tikzpicture}
\draw [help lines] (0,0) grid (4,2);
\draw [line width= 3pt] (0,1) -- (4,1);
\fill[red] ($(0,1)!.75!(4,1)$) circle (4pt);
\end{tikzpicture}
\\ \hline (0,1)!{\color{red}0.25}!(4,1) & (0,1)!{\color{red}0.75}!(4,1) \\ 
\hline 
\end{tabular} 

\bigskip

\begin{tabular}{|c|} \hline  
\begin{tikzpicture}
\draw [help lines] (0,0) grid (4,3);
\draw [line width=2pt ](0,2) -- (4,2);
\draw[red] ($(0,2)!.75!(4,2)$) -- (0,0);
\fill[red] ($(0,2)!.75!(4,2)!.66!(0,0)$) circle (4pt);
\end{tikzpicture}
\\ \hline 
\BS{fill}[red] (\${\color{blue}(0,2)!0.75!(4,2)}!{\color{red}0.66!(0,0)}\$) circle (2pt);
\\ \hline 
\end{tabular} 


\SbSbSSCT{Point à distance donnée}{Position at a given distance }

\begin{center}
\RRR{13-5-4}
\end{center}

\begin{tabular}{|c|c|} \hline  
\multicolumn{2}{|c|}{\BS{fill}[red] ({\color{red}\$(0,1)!1.5cm!(4,1)\$}) circle (4pt); } \\  \hline  

\begin{tikzpicture}
\draw [help lines] (0,0) grid (4,2);
\draw [line width= 2pt] (0,1) -- (4,1);
\fill[red] ($(0,1)!1.5cm!(4,1)$) circle (4pt);
\end{tikzpicture}
&  
\begin{tikzpicture}
\draw [help lines] (0,0) grid (4,2);
\draw [line width= 2pt] (0,1) -- (4,1);
\fill[red] ($(0,1)!3cm!(4,1)$) circle (4pt);
\end{tikzpicture}
\\ \hline (0,1)!{\color{red}1.5cm}!(4,1) & (0,1)!{\color{red}3cm}!(4,1) \\ 
\hline 
\end{tabular} 

\bigskip

\begin{tabular}{|c|} \hline  
\begin{tikzpicture}
\draw [help lines] (0,0) grid (4,4);
\coordinate (a) at (1,0);
\coordinate (b) at (4,1);
\draw [line width= 3pt] (0,0) -- (4,1);
\draw [line width= 2pt,red](2,.5) -- ($ (2,.5)!2cm!90:(4,1) $);
\end{tikzpicture}
\\ \hline
\BS{draw} (2,.05) - - (\$ (2,0.5)!{\color{red}2cm!90:(4,1)} \$);
\\ \hline 
\end{tabular} 

\newpage

\SbSbSSCT{Coordonnées relatives}{Relative coordinates}


\Par{Cartésienne}{Cartesian coordinates}

\begin{center}
\RRR{13-4-1}
\end{center}

\begin{tabular}{|c|c|c|} \hline  
\TFRGB{relative à l'origine}{relative to the origin}  & \TFRGB{relative à une position}{relative to a position}  &  \TFRGB{relative à la dernière position}{relative to the last position}   
\\ \hline  
 
\begin{tikzpicture}
\draw[help lines] (0,-1) grid (3,1); 
 \draw[blue,very thick] (0,0) -- (1,0) - - (2,1) - - (2,-1);
 \fill[red] (0,0) circle (4pt);
\end{tikzpicture}
&
\begin{tikzpicture} %[scale=.8]
\draw[help lines] (0,-1) grid (4,1);
 \draw[blue,very thick] (0,0) - - (1,0) -- +(2,1) -- +(2,-1) ; %–- +(2,-1) ;
 \fill[red] (1,0) circle (4pt);
\end{tikzpicture}
&
\begin{tikzpicture} %[scale=.8]
\draw[help lines] (0,-1) grid (5,1);  
 \draw[blue,very thick] (0,0) -- (1,0)  - - ++(2,1) - - ++(2,-1);
 \fill[red] (1,0) circle (4pt);
 \fill[red] (3,1) circle (4pt);
\end{tikzpicture}
\\ \hline 
\tikz \fill node[fill=green!20,inner sep=0pt]{(0,0)}; - - (1,0) &
 (0,0) - - \tikz \fill node[fill=green!20,inner sep=0pt]{(1,0)};  & (0,0) - - \tikz \fill node[fill=green!20,inner sep=0pt]{(1,0)}; \\
 - - (2,1) - - (2,-1)  &
   - - +(2,1) - - +(2,-1) & - - ++\tikz \fill node[fill=green!20,inner sep=0pt]{(2,1)}; - - ++(2,-1)
\\ \hline 
\end{tabular} 

\bigskip

\begin{tabular}{|c|c|c|} \hline  
\begin{tikzpicture} [scale=.5]
\draw[help lines] (0,-1) grid (6,6);
 \draw[red,dotted,line width=2pt] (0,0) rectangle (2,2) ;
  \draw[green,dotted,line width=2pt] (0,0) rectangle (3,3) ;  
 \draw[blue,line width=2pt] (0,0) rectangle (1,1)  rectangle (2,2) rectangle (3,3);

\end{tikzpicture}

&  
\begin{tikzpicture} [scale=.5]
\draw[help lines] (0,-1) grid (6,6); 
  \draw[green,dotted,line width=2pt] (1,1) rectangle (4,4) ;   
 \draw[blue,line width=2pt] (0,0) rectangle (1,1)  rectangle +(2,2) rectangle +(3,3);
    \fill[red] (1,1) circle (4pt);
\end{tikzpicture}
&  
\begin{tikzpicture} [scale=.5]
\draw[help lines] (0,-1) grid (6,6);  
 \draw[blue,line width=2pt] (0,0) rectangle (1,1)  rectangle ++(2,2) rectangle ++(3,3);
    \fill[red] (1,1) circle (4pt);
     \fill[green] (3,3) circle (4pt); 
\end{tikzpicture}
\\ 
\hline 
\BS{draw} (0,0) rectangle (1,1)   &
\BS{draw} (0,0) rectangle (1,1)   & 
\BS{draw} (0,0) rectangle (1,1)  \\
rectangle (2,2) rectangle (3,3);  &
rectangle +(2,2) rectangle +(3,3);  &
rectangle ++(2,2) rectangle ++(3,3); \\
\hline 
\end{tabular}


\Par{Polaire }{Polar} {}

\bigskip


\noindent

\begin{tabular}{|c|c|c|c|} \hline
\TFRGB{relative à l'origine}{relative to the origin}  & \TFRGB{relative à une position}{relative to a position}  &  \TFRGB{relative à la dernière position}{relative to the last position}   
\\ \hline    
\begin{tikzpicture} %[scale=.8] 
\draw[help lines] (0,-1) grid (3,1);
 \fill[red] (0:0) circle (4pt);
 \draw[blue,very thick] (0:0)-- (0:1) -- (30:2) -- (-30:2);
\end{tikzpicture}
&
\begin{tikzpicture} %[scale=.8] 
\draw[help lines] (0,-1) grid (4,1);
 \fill[red] (1,0) circle (4pt);
 \draw[blue,very thick] (0:0) -- (0:1) -- +(30:2) -- +(-30:2);
\end{tikzpicture}
&
\begin{tikzpicture} %[scale=.8] 
\draw[help lines] (0,-1) grid (5,1);
 \fill[red] (1,0) circle (4pt);
 \fill[red] (2.732,1) circle (4pt);
 \draw[blue,very thick] (0:0)-- (0:1) -- ++(30:2) -- ++(-30:2);
\end{tikzpicture}
\\ \hline
\tikz \fill node[fill=green!20,inner sep=0pt] {(0:0)}; - - (0:1)&
 (0:0) - - \tikz \fill node[fill=green!20,inner sep=0pt] {(0:1)}; & (0:0)- - \tikz \fill node[fill=green!20,inner sep=0pt] {(0:1)}; \\
 - - (30:2) - - (-30:2)  &  - -  +(30:2) - - +(-30:2) & - -  ++\tikz \fill node[fill=green!20,inner sep=0pt] {(30:2)}; - - ++(-30:2)
\\ \hline 
\end{tabular} 

%\subsubsection{coordonnée relative en polaire}
\Par{coordonnée relative en polaire}{Relative polar coordinate}

\begin{center}
\RRR{13-4-2}
\end{center}
\bigskip

\begin{tabular}{|c|c|} \hline 
\multicolumn{2}{|c|}{ \BS{draw}[blue,very thick] (0,0) -- (2,1) -- ([turn]-45:1cm);}
 \\ \hline
\begin{tikzpicture} %[scale=.8] 
\draw[help lines] (0,0) grid (4,2);
 \draw[dotted] (0,0) -- (4,2);
 \draw[blue,very thick] (0,0) -- (2,1) -- ([turn]-45:1cm);
\end{tikzpicture}
&  
\begin{tikzpicture} %[scale=.8] 
\draw[help lines] (0,0) grid (4,2);
 \draw[dotted] (0,0) -- (4,2);
 \draw[blue,very thick] (0,0) -- (2,1) -- ([turn]45:1cm);
\end{tikzpicture}
\\ \hline ([\RDD{turn}]-45:1cm) & ([\RDD{turn}]45:1cm) \\ 
\hline 
\end{tabular}

\bigskip

\begin{tabular}{|c|c|} \hline  
\begin{tikzpicture}  
\draw[help lines] (-1,0) grid (4,3);
\draw [line width=2pt] (4,0) arc (0 :120 :2)  -- ([turn]90:2cm) ;

\end{tikzpicture}
&  
\begin{tikzpicture} %[scale=.8] 
\draw[help lines] (0,0) grid (4,3);
\draw [line width=2pt]  (0,0) to [bend left] (2,2) --  ([turn]0:2cm);
\fill [red](2,2) circle (4pt);
\end{tikzpicture}
\\ \hline  
\BS{draw} (4,0) arc (0 :120 :2)  - - ([\RDD{turn}]90:2cm) ;
& \BS{draw}  (0,0) to [bend left] (2,2) - -  ([\RDD{turn}]0:2cm); \\

\hline 
\end{tabular} 


%\bigskip 
%
%
%\tikz [delta angle=30, radius=1cm]
%\draw (0,0) arc [start angle=0] -- ([turn]0:1cm)
%arc [start angle=30] -- ([turn]0:1cm)
%arc [start angle=60] -- ([turn]30:1cm);



\bigskip

\begin{tabular}{|c|c|c|} \hline  
\multicolumn{3}{|c|}{ \BS{draw}(1,2)
.. controls ([turn]0:2cm) .. ([turn]-90:2cm); }
\\ \hline
\begin{tikzpicture} %[scale=.8] 
\draw[help lines] (0,0) grid (4,4);
 \draw [line width=2pt] (1,2)
.. controls ([turn]0:2cm) .. ([turn]-90:2cm);
\end{tikzpicture}
&  
\begin{tikzpicture} %[scale=.8] 
\draw[help lines] (0,0) grid (4,4);
 \draw [line width=2pt] (1,2)
.. controls ([turn]30:2cm) .. ([turn]-90:2cm);
\end{tikzpicture}
&  
\begin{tikzpicture} %[scale=.8] 
\draw[help lines] (-2,0) grid (2,4);
 \draw [line width=2pt] (1,2)
.. controls ([turn]0:2cm) .. ([turn]90:2cm);

\end{tikzpicture}
\\ \hline ([turn]0:2cm) .. ([turn]-90:2cm) & ([turn]30:2cm) .. ([turn]-90:2cm) & ([turn]0:2cm) .. ([turn]90:2cm) \\ 
\hline 
\end{tabular} 


\tikzset{every picture/.style=blue,very thick,inner sep=.3333em}

 
\newpage

\SSCT{Les n\oe uds }{Nodes}

\SbSSCT{Définition des  n\oe uds}{Creation of nodes}
\tikzset{blue}

\label{noeuds}
\noindent

\begin{tabular}{|c | c | c | c | c |} \hline
\multicolumn{5}{|c|}{  \BS{draw} (1,1) node[\RDD{fill}=red!20] \AC{};   }\\ 
\hline 
\tikz \draw (0,0) grid (2,2) (1,1) node[fill=red!20] {};
&
\tikz \draw (0,0) grid (2,2) (1,1) node[fill=red!20,draw] {}; 
&
\tikz \draw (0,0) grid (2,2) (1,1) node[circle,fill=red!20] {};
&
\tikz \draw (0,0) grid (2,2) (1,1) node[circle,fill=red!20,draw] {};
&
\tikz \draw (0,0) grid (2,2) (1,1) node[coordinate] {};
\\  \hline
\dft
&
node[\RDD{draw}] 
&
 node[\RDD{circle}]  
&
 node[\RDD{circle},\RDD{draw}]
 &
  node[\RDD{coordinate}]
 \\  \hline
\end{tabular}
\bigskip

\begin{tabular}{|c | c | c | c | } \hline
\multicolumn{4}{|c|}{ \BSS{node} \RDD{at} (1,1) [fill=red!20] \AC{};   }\\ 
\hline 
 \begin{tikzpicture}
\draw (0,0) grid (2,2) ; 
\node at (1,1) [fill=red!20] {};
 \end{tikzpicture}
&
 \begin{tikzpicture}
\draw (0,0) grid (2,2) ; 
\node at (1,1) [draw] {};
 \end{tikzpicture}
&
 \begin{tikzpicture}
\draw (0,0) grid (2,2) ; 
\node at (1,1) [fill=red!20,circle] {};
 \end{tikzpicture}
&
 \begin{tikzpicture}
\draw (0,0) grid (2,2) ; 
\node at (1,1) [circle,draw] {};
 \end{tikzpicture}

\\  \hline
[fill=red!20]
&
[\RDD{draw}] 
&
[\RDD{circle},fill=red!20]
 &
[\RDD{circle},draw] 
 \\  \hline
\end{tabular}
\bigskip

\TFRGB{Autres types de n\oe uds voir page}{Other type of nodes see page} \pageref{noeudboite}
\bigskip


\begin{tabular}{|c|c|} \hline 
\BS{draw} (0,0) node at (1,0) \AC{1} node at (2,0) \AC{2} & \BS{draw}(0,0) node foreach \BS{x} in \AC{1,2,...,5}\\ 
node at (3,0) \AC{3} node at (4,0) \AC{4} node at (5,0) \AC{5}; &  at (\BS{x},0) \AC{\BS{x}};\\ 
\hline 
\tikz \draw (0,0) node at (1,0) {1} node at (2,0) {2} node at (3,0) {3} node at (4,0) {4} node at (5,0) {5};
&
\tikz \draw (0,0) node foreach \x in {1,2,...,5} at (\x,0) {\x};
\\ \hline 
\end{tabular} 



\bigskip

\begin{tabular}{|c|} \hline 
\BS{draw}[\rouge{every node/.style=\AC{draw,red}}](0,0) node foreach \BS{x} in \AC{1,2,...,5} at (\BS{x},0) \AC{\BS{x}};
\\ \hline 
\rule[-3pt]{0pt}{.8cm}\tikz \draw[every node/.style={draw,red}] (0,0) node foreach \x in {1,2,...,5} at (\x,0) {\x};
\\ \hline 
\end{tabular} 

\bigskip

\begin{tabular}{|c|} \hline 
\BS{draw}[\rouge{every rectangle node/.style=\AC{draw,red}},\\
\rouge{every circle node/.style=\AC{draw,double}}]\\ (0,0) node at (1,0) \AC{1} node[circle] at (2,0) \AC{2} \\ node[circle] at (3,0) \AC{3} node at (4,0) \AC{4} node at (5,0) \AC{5};
\\ \hline 
\rule[-3pt]{0pt}{1cm} \tikz \draw[every rectangle node/.style={draw,red},
every circle node/.style={draw,double}] (0,0) node at (1,0) {1} node[circle] at (2,0) {2} node[circle] at (3,0) {3} node at (4,0) {4} node at (5,0) {5};
\\ \hline 
\end{tabular} 

\SbSSCT{Nom des  n\oe uds}{Node name}


\begin{tabular}{|c|c|c|}
\hline 
\multicolumn{3}{|c|}{} \\ 
\hline 
\begin{tikzpicture}
\node[name=A,fill=red] at (0,0) {};
\draw  (-1,-1) grid (1,1) ;
\draw (A) circle (.5) ;
\end{tikzpicture} 
&  
\begin{tikzpicture}
\node[name=A,alias=B,fill=red] at (0,0) {} ;
\draw  (-1,-1) grid (1,1) ;
\draw (B) circle (.5) ;
\end{tikzpicture}
& 
\begin{tikzpicture}
\node[fill=red] (C) at (0,0) {};
\draw  (-1,-1) grid  (1,1) ;
\draw (C) circle (.5);
\end{tikzpicture} \\ 
\hline 
\BS{node}[\RDD{name}=A] at (0,0) \AC{}  & \BS{node}[\RDD{name}=A,\RDD{alias}=B] at (0,0) \AC{}  & 
\BS{node}\rouge {(C)} at (0,0) \AC{} \\ 
\BS{draw} (A) circle (.5); & \BS{draw}  (B) circle (.5); &\BS{draw} (C) circle (.5);
\\ \hline 
\end{tabular} 
\newpage

\SbSSCT{Contenu des  n\oe uds}{Node contents}
\tikzset{blue}

\begin{center}
\RRR{17-2-1}
\end{center}

\begin{tabular}{|c|c|} \hline 
\BS{node} at (1,1) [fill=red!20]\rouge { \AC{XXX} };
&  
\BS{node} at (1,1) [fill=red!20,\RDD{node contents}=XXX] \AC{};
\\  \hline 
 \begin{tikzpicture}
\draw (0,0) grid (2,2) ; 
\node at (1,1) [fill=red!20] {XXX};
\end{tikzpicture}
&  
\begin{tikzpicture}
\draw (0,0) grid (2,2) ; 
\node at (1,1) [fill=red!20,node contents=XXX] {};
\end{tikzpicture} 
\\ \hline 
\end{tabular} 

\bigskip

\begin{tabular}{|c|c|} \hline 
\BS{node}[red] at (1,1) [fill=blue!20] \AC{XXX} ;
&  
\BS{node}[red] at (1,1) [fill=blue20,node contents=XXX] \AC{};
\\  \hline 
 \begin{tikzpicture}
\draw (0,0) grid (2,2) ; 
\node[red] at (1,1) [fill=blue!20] {XXX};
\end{tikzpicture}
&  
\begin{tikzpicture}
\draw (0,0) grid (2,2) ; 
\node[red] at (1,1) [fill=blue!20,node contents=XXX] {};
\end{tikzpicture} 
\\ \hline 
\end{tabular} 


\SbSSCT{Premier ou arrière plan}{Behind or in front}

\begin{tabular}{|c|c|} \hline 
\multicolumn{2}{|l|}{\BS{tikz} \BS{fill} [fill=blue!50, draw=blue, very thick]
(0,0) } \\ 
\multicolumn{2}{|l|}{node [\RDD{behind path}, fill=red!50] \AC{XXXXX} }  \\
\multicolumn{2}{|l|}{- - (1.5,0) - - (1.5,1) - - (0,1) ;}
\\ \hline 
\tikz \fill [fill=blue!50, draw=blue, very thick]
(0,0) node [behind path, fill=red!50] {XXXXX}
-- (1.5,0) 
-- (1.5,1) 
-- (0,1) ;
&  
\tikz \fill [fill=blue!50, draw=blue, very thick]
(0,0) node [in front of path, fill=red!50] {XXXXX}
-- (1.5,0) 
-- (1.5,1) 
-- (0,1) ;
\\ \hline 
\RDD{behind path}
&  
\RDD{in front of path}
\\ \hline 
\end{tabular}



\SbSSCT{Noms à préfixe ou suffixe}{Name prefix or name suffix}


\begin{tabular}{|c|c|}
\hline 
\begin{tikzpicture}[every node/.style={draw},baseline=0pt]
\draw[name prefix = top-] node (A) at (1,1) {A} node (B) at (2,1) {B} node (C) at (3,1) {C};
\draw[name prefix = bottom-] node (1) at (1,0) {1} node (2) at (2,0) {2} node(3) at  (3,0) {3};
\draw [red] (top-A) -- (bottom-3);
\end{tikzpicture} 
&
\parbox{12cm}{
\BS{draw}[\RDD{name prefix} = \blll{top-} ] node (A) at (1,1) \AC{A} node (B) at (2,1) \AC{B} node (C) at (3,1) \AC{C}; \\
\BS{draw}[\RDD{name prefix} = \blll{bottom-}] node (1) at (1,0) \AC{1} node (2) at (2,0) \AC{2} node(3) at  (3,0) \AC{3}; \\
\BS{draw} [red] (\blll{top-}A) -- (\blll{bottom-}3);}
\\ \hline
\begin{tikzpicture}[every node/.style={draw},baseline=0pt]
\draw[name suffix= -top] node (A) at (1,1) {A} node (B) at (2,1) {B} node (C) at (3,1) {C};
\draw[name suffix=  -bottom] node (1) at (1,0) {1} node (2) at (2,0) {2} node(3) at  (3,0) {3};
\draw [red] (A-top) -- (3-bottom);
\end{tikzpicture}
&
\parbox{12cm}{
\BS{draw}[\RDD{name suffix} = \blll{-top}] node (A) at (1,1) \AC{A} node (B) at (2,1) \AC{B} node (C) at (3,1) \AC{C}; \\
\BS{draw}[\RDD{name suffix} = \blll{-bottom}] node (1) at (1,0) \AC{1} node (2) at (2,0) \AC{2} node(3) at  (3,0) \AC{3}; \\
\BS{draw} [red] (A \blll{-top}) - - (3 \blll{-bottom});}
\\ \hline 

\end{tabular} 


\SbSSCT{Liaisons}{Links}
\label{liaisons}

\begin{tabular}{|c|c|c|} \hline 
\multicolumn{3}{|l|}{\BS{node}[draw] (A) at (0,0) \AC{A}; \hspace{.5cm} \BS{node}[draw] (B) at (1.5,1.5) \AC{B}; \hspace{.5cm} \BS{draw} (A) - - (B) } \\ \hline 
\begin{tikzpicture}[blue]
\node[draw] (A) at (0,0) {A};
\node[draw] (B) at (1.5,1.5) {B};
\draw (A) -- (B);
\end{tikzpicture}
&  
\begin{tikzpicture}[blue]
\node[draw] (A) at (0,0) {A};
\node[draw] (B) at (1.5,1.5) {B};
\draw (A) |- (B);
\end{tikzpicture}
&  
\begin{tikzpicture}[blue]
\node[draw] (A) at (0,0) {A};
\node[draw] (B) at (1.5,1.5) {B};
\draw (A) -| (B);
\end{tikzpicture}
\\ \hline  
(A){\color{red} - -} (B) & (A) {\color{red}|-} (B) &  (A) {\color{red}-|} (B)
\\ \hline 
\begin{tikzpicture}[blue]
\node[draw] (A) at (0,0) {A};
\node[draw] (B) at (1.5,1.5) {B};
\draw (A) to [bend right] (B);
\end{tikzpicture}
&  
\begin{tikzpicture}[blue]
\node[draw] (A) at (0,0) {A};
\node[draw] (B) at (1.5,1.5) {B};
\draw (A) to [bend left] (B);
\end{tikzpicture}
&  
\begin{tikzpicture}[blue]
\node[draw] (A) at (0,0) {A};
\node[draw] (B) at (1.5,1.5) {B};
\draw (A) to[bend left=0] (B);
\end{tikzpicture}
\\ \hline  
(A) to [\RDD{bend right}] (B) & (A) to [\RDD{bend left}] (B) &  (A) to[\RDD{bend left}=0] (B)
\\ \hline 
\begin{tikzpicture}[blue]
\node[draw] (A) at (0,0) {A};
\node[draw] (B) at (1.5,1.5) {B};
\draw (A) to[bend left=120]  (B);
\end{tikzpicture}
&  
\begin{tikzpicture}[blue]
\node[draw] (A) at (0,0) {A};
\node[draw] (B) at (1.5,1.5) {B};
\draw (A) to[bend left=45] (B);
\end{tikzpicture}
&  
\begin{tikzpicture}[blue]
\node[draw] (A) at (0,0) {A};
\node[draw] (B) at (1.5,1.5) {B};
\draw (A) to[bend left=90] (B);
\end{tikzpicture}
\\ \hline  
(A)  to[\RDD{bend left}=120]  (B) & (A) to[\RDD{bend left}=45] (B) &  (A) to[\RDD{bend left}=90] (B)
\\ \hline 
\begin{tikzpicture}[blue]
\node[draw] (A) at (0,0) {A};
\node[draw] (B) at (1.5,1.5) {B};
\draw (A)  to[out=90]  (B);
\end{tikzpicture}
&  
\begin{tikzpicture}[blue]
\node[draw] (A) at (0,0) {A};
\node[draw] (B) at (1.5,1.5) {B};
\draw (A) to[out=30] (B);
\end{tikzpicture}
&  
\begin{tikzpicture}[blue]
\node[draw] (A) at (0,0) {A};
\node[draw] (B) at (1.5,1.5) {B};
\draw (A)  to[in=-90]  (B);
\end{tikzpicture}
\\ \hline  
(A)  to[\RDD{out}=90] (B) & (A) to[\RDD{out}=30]  (B) &  (A)  to[\RDD{in}=-90]  (B)
\\ \hline  
\end{tabular} 

\bigskip
\begin{tabular}{|c|c|c|} \hline  
\multicolumn{2}{|c|}{ \BS{draw} (A) .. controls +(right:2cm) and +(down:2cm) .. (B);  }\\ 
\hline  
\begin{tikzpicture}[blue]
\node[draw] (A) at (0,0) {A};
\node[draw] (B) at (2,2) {B};
\draw  (A) .. controls +(right:2cm) and +(down:2cm) .. (B);
\end{tikzpicture}
&
\begin{tikzpicture}[blue]
\node[draw] (A) at (0,0) {A};
\node[draw] (B) at (2,2) {B};
\draw  (A) .. controls +(up:1cm) and +(left:1cm) .. (B);
\end{tikzpicture}
\\ \hline 
controls +(right:2cm) and +(down:2cm)  &
controls +(up:1cm) and +(left:1cm)
\\ \hline 
\begin{tikzpicture}[blue]
\node[draw] (A) at (0,0) {A};
\node[draw] (B) at (2,2) {B};
\draw  (A) .. controls +(right:1cm) and +(right:2cm) .. (B);
\end{tikzpicture}
&
\begin{tikzpicture}[blue]
\node[draw] (A) at (0,0) {A};
\node[draw] (B) at (2,2) {B};
\draw  (A) .. controls +(up:1cm) and +(right:2cm) .. (B);
\end{tikzpicture}
\\ \hline 
controls +(right:1cm) and +(right:2cm)  &
controls +(up:1cm) and +(right:2cm) 
\\ \hline 
\begin{tikzpicture}[blue]
\node[draw] (A) at (0,0) {A};
\node[draw] (B) at (2,2) {B};
\draw  (A) .. controls +(120:2cm) and +(200:1cm) .. (B);
\end{tikzpicture}
 &
 \begin{tikzpicture}[blue]
 \node[draw] (A) at (0,0) {A};
 \node[draw] (B) at (2,2) {B};T
 \draw  (A) .. controls +(120:2cm) and +(200:1cm) .. (A);
 \end{tikzpicture}
\\  \hline  
controls +(120:2cm) and +(200:1cm) & controls +(120:2cm) and +(200:1cm) 
\\ \hline 
\begin{tikzpicture}[blue]
\node[draw] (A) at (0,0) {A};
\node[draw] (B) at (2,2) {B};
\node[draw] (C) at (0,1) {C};
\node[draw] (D) at (3,0) {D};
\draw  (A) .. controls +(C) and +(D) .. (B);
\end{tikzpicture}
&
\begin{tikzpicture}[blue]
\node[draw] (A) at (0,0) {A};
\node[draw] (B) at (2,2) {B};
\node[draw] (C) at (0,1) {C};
\node[draw] (D) at (3,0) {D};
\draw (A) .. controls +(D)  .. (B);
\end{tikzpicture}
\\ \hline 
controls +(C) and +(D) &
controls +(D) 
\\ \hline 
\end{tabular} 
 \bigskip
 
\begin{tabular}{|c|c|c|} \hline 
\multicolumn{3}{|l|}{ \BS{node}[draw] (A) at (0,0) \AC{A}  }\\

\multicolumn{3}{|l|}{ \BS{node}[draw] (B) at (2,2) \AC{B} \RDD{edge}  [->] (A);  }\\
\multicolumn{3}{|c|}{\RRR{17-12-1}}  \\
\hline 
 \begin{tikzpicture}
 \node[draw] (A) at (0,0) {A};
 \node[draw] (B) at (2,2) {B} edge [->] (A);
 \end{tikzpicture}
 &
 \begin{tikzpicture}
 \node[draw] (A) at (0,0) {A};
 \node[draw] (B) at (2,2) {B} edge [red]  (A);
 \end{tikzpicture}
 &
 \begin{tikzpicture}
 \node[draw] (A) at (0,0) {A};
 \node[draw] (B) at (2,2) {B} edge [dashed] (A);
 \end{tikzpicture}
\\ \hline 
[->] & [red]  & [dashed]
\\ \hline 
\end{tabular}

\SbSSCT{\'Etiquettes sur les n\oe uds}{Node labels}

\begin{tabular}{|c|c|c|c|} \hline
\multicolumn{4}{|c|}{  \BS{fill}(0,0) circle (2pt) node[\RDD{above}] \AC{texte} ; \RRR{17-5-2}   }\\ 
\hline 
  
\begin{tikzpicture} \draw[help lines] (-1,-1) grid (1,1) ;\fill (0,0) circle (2pt) node[above] {texte};\end{tikzpicture}
& 
\begin{tikzpicture} \draw[help lines] (-1,-1) grid (1,1) ;\fill (0,0) circle (2pt) node[below] {texte};\end{tikzpicture}
 &  
\begin{tikzpicture} \draw[help lines] (-1,-1) grid (1,1);\fill (0,0) circle (2pt) node[left] {texte};\end{tikzpicture}
 &  
\begin{tikzpicture} \draw[help lines] (-1,-1) grid (1,1); \fill (0,0) circle (2pt) node[right] {texte};\end{tikzpicture}
 \\  \hline 
 [\RDD{above}] & [\RDD{below}] & [\RDD{left}] &  [\RDD{right}]
 \\ \hline 
 \begin{tikzpicture} \draw[help lines] (-1,-1) grid (1,1) ;\fill (0,0) circle (2pt) node[above left] {texte};\end{tikzpicture}
 & 
 \begin{tikzpicture} \draw[help lines] (-1,-1) grid (1,1) ;\fill (0,0) circle (2pt) node[below left] {texte};\end{tikzpicture}
  &  
 \begin{tikzpicture} \draw[help lines] (-1,-1) grid (1,1);\fill (0,0) circle (2pt) node[above right] {texte};\end{tikzpicture}
  &  
 \begin{tikzpicture} \draw[help lines] (-1,-1) grid (1,1); \fill (0,0) circle (2pt) node[below right] {texte};\end{tikzpicture}
  \\  \hline 
  [\RDD{above left}] & [\RDD{below left}] & [\RDD{above right}] &  [\RDD{below right}]
  \\ \hline 
 \begin{tikzpicture} \draw[help lines] (-1,-1) grid (1,1) ;\fill (0,0) circle (2pt) node[anchor=south] {texte};\end{tikzpicture}
 & 
 \begin{tikzpicture} \draw[help lines] (-1,-1) grid (1,1) ;\fill (0,0) circle (2pt) node[anchor=west] {texte};\end{tikzpicture}
  &  
 \begin{tikzpicture} \draw[help lines] (-1,-1) grid (1,1);\fill (0,0) circle (2pt) node[anchor=north] {texte};\end{tikzpicture}
  &  
 \begin{tikzpicture} \draw[help lines] (-1,-1) grid (1,1); \fill (0,0) circle (2pt) node[anchor=east] {texte};\end{tikzpicture}
  \\  \hline 
  [\RDD{anchor}=south] & [\RDD{anchor}=west] & [\RDD{anchor}=north] & [\RDD{anchor}=east]                                                                                                                                                               ]
  \\ \hline 
 \begin{tikzpicture} \draw[help lines] (-1,-1) grid (1,1) ;\fill (0,0) circle (2pt) node[anchor=south east] {texte};\end{tikzpicture}
 & 
\begin{tikzpicture} \draw[help lines] (-1,-1) grid (1,1) ;\fill (0,0) circle (2pt) node[anchor=south west] {texte};\end{tikzpicture}
&  
\begin{tikzpicture} \draw[help lines] (-1,-1) grid (1,1);\fill (0,0) circle (2pt) node[anchor=north west] {texte};\end{tikzpicture}
&  
\begin{tikzpicture} \draw[help lines] (-1,-1) grid (1,1); \fill (0,0) circle (2pt) node[anchor=east] {texte};\end{tikzpicture}
\\  \hline 
[\RDD{anchor}=south east] & [\RDD{anchor}=south west] & [\RDD{anchor}=north west] & [\RDD{anchor==north east                                                                                                                                                       }]
  \\ \hline 
\end{tabular} 


\bigskip
\begin{tabular}{|c|c|c|c|} \hline
\multicolumn{4}{|c|}{  \BS{fill}(0,0) circle (2pt) node[\RDD{above}=.3cm] \AC{texte} ; \RRR{17-5-2}  }\\ 
\hline 
  
\begin{tikzpicture} \draw[help lines] (-1,-1) grid (1,1) ;\fill (0,0) circle (2pt) node[above=.3cm] {texte};\end{tikzpicture}
& 
\begin{tikzpicture} \draw[help lines] (-1,-1) grid (1,1) ;\fill (0,0) circle (2pt) node[below=.3cm] {texte};\end{tikzpicture}
 &  
\begin{tikzpicture} \draw[help lines] (-1,-1) grid (1,1);\fill (0,0) circle (2pt) node[left=.3cm] {texte};\end{tikzpicture}
 &  
\begin{tikzpicture} \draw[help lines] (-1,-1) grid (1,1); \fill (0,0) circle (2pt) node[right=.3cm] {texte};\end{tikzpicture}
 \\  \hline 
 [\RDD{above}=.3cm] & [\RDD{below}=.3cm] & [\RDD{left}=.3cm] &  [\RDD{right}=.3cm]]
 \\ \hline 
\begin{tikzpicture} \draw[help lines] (-1,-1) grid (1,1) ;\fill (0,0) circle (2pt) node[above left=.3cm] {texte};\end{tikzpicture}
& 
\begin{tikzpicture} \draw[help lines] (-1,-1) grid (1,1) ;\fill (0,0) circle (2pt) node[below left=.3cm] {texte};\end{tikzpicture}
 &  
\begin{tikzpicture} \draw[help lines] (-1,-1) grid (1,1);\fill (0,0) circle (2pt) node[above right=.3cm] {texte};\end{tikzpicture}
 &  
\begin{tikzpicture} \draw[help lines] (-1,-1) grid (1,1); \fill (0,0) circle (2pt) node[below right=.3cm] {texte};\end{tikzpicture}
 \\  \hline 
 [\RDD{above left}=.3cm] & [\RDD{below left}=.3cm] & [\RDD{above right}=.3cm] &  [\RDD{below right}=.3cm]]
 \\ \hline 
 
 \end{tabular} 

 
 \newpage
\selectlanguage{french}
 
 \begin{tabular}{|c|c|c|c|c|} \hline
 \multicolumn{5}{|l|}{ \BSS{shorthandoff}\AC{:} \footnotemark[1]  } \\
 \multicolumn{5}{|l|}{  \BS{node} [draw,\RDD{label}=right:texte] \AC{}   }\\
 \multicolumn{5}{|l|}{ \BSS{shorthandon}\AC{:} } \\ 
 \hline 
     \shorthandoff{:} 
 \tikz \node [draw,label=right:texte] {};
 \shorthandon{:}
 &
  \shorthandoff{:}
 \tikz \node [draw,label=left:texte] {};
 \shorthandon{:}
 &
  \shorthandoff{:}
 \tikz \node [draw,label=above:texte] {};
 \shorthandon{:}
 &
  \shorthandoff{:}
 \tikz \node [draw,label=below:texte] {};
 \shorthandon{:}
 &
  \shorthandoff{:}
 \tikz \node [draw,label=45:texte] {};
    \shorthandon{:}
   \\ \hline
  label=right & label=left &  label=above & label=below & label=45
    \\ \hline 
 \end{tabular}
 \footnotetext[1]{\TFRGB{désactivation et ré-activation de \og : \fg  conflit entre les modules Tikz et Babel en français}{Only useful when the package babel is loaded with the frenchb option    }}
 
 \bigskip
  \begin{tabular}{|c|c|c|c|c|} \hline
  \BS{fill}(0,0) circle (2pt) node[below right=.3cm,draw,label=45:étiquette] \AC{texte} ;
      \\ \hline 
  
  \shorthandoff{:}
\begin{tikzpicture} \draw[help lines] (-1,-1) grid (2,1); \fill (0,0) circle (2pt) node[below right=.3cm,draw,label=45:étiquette] {texte};\end{tikzpicture}
 \shorthandon{:}
 
    \\ \hline 
 \end{tabular}
\bigskip

 \shorthandoff{:}

\SbSSCT{\'Etiquettes épinglées}{The Pin Option} 

\begin{center}
\RRR{17-10-3}
\end{center}
 
\begin{tabular}{|c|c|c|} \hline
\multicolumn{3}{|c|}{  \BSS{shorthandoff}\AC{:} \BS{node}[circle,draw,blue,\RDD{pin}=texte] \AC{} ;   \BSS{shorthandon}\AC{:}  \footnotemark[1] }\\ 
\hline
\begin{tikzpicture} 
\node [circle,draw,blue,pin=texte] {};
\end{tikzpicture}
&
\begin{tikzpicture} 
\node [circle,draw,blue,pin=60:texte] {};
\end{tikzpicture}
&
\begin{tikzpicture} 
\node [circle,draw,blue,pin=right:texte] {};
\end{tikzpicture}
 \\ \hline
[circle,pin=texte] &   [circle,pin=60:texte] & [circle,pin=right:texte]
 \\ \hline 
\end{tabular}  

\bigskip
\begin{tabular}{|c|c|c|} \hline
\multicolumn{3}{|c|}{  \BS{tikz}[\RDD{pin position}=60] \BS{node} [circle,pin=texte] \AC{} ;   }\\ 
\hline 
\tikz[pin position=60] \node [circle,draw,blue,pin=texte] {};
&
\tikz[pin distance=0 cm] \node [circle,draw,blue,pin=60:texte] {};
&
\tikz[pin distance=2 cm] \node [circle,draw,blue,pin=60:texte,pin distance=0cm] {};
  \\ \hline
  [\RDD{pin position}=60] & [\RDD{pin distance}=0 cm] & [\RDD{pin distance}=2 cm]
    \\ \hline
  \dft{ : above} & \multicolumn{2}{|c|}{ \dft{ : 3 ex}}
      \\ \hline
\end{tabular}  

\newpage

   \shorthandon{:} 
   
\selectlanguage{english}   

\SbSSCT{ N\oe uds  sur un chemin}{Nodes on a path}

\RRR{17-8}

\begin{tabular}{|c|c|c|} \hline
\multicolumn{3}{|c|}{  \BS{draw}(0,0) .. controls (1,2) and (2,-1) .. (4,0) node[\RDD{at end}] \AC{texte} ;   }\\ 
\hline 
\tikz \draw (0,0) .. controls (1,2) and (2,-1) .. (4,0) node[pos=0] {texte}; 
&
\tikz \draw (0,0) .. controls (1,2) and (2,-1) .. (4,0) node[pos=.33] {texte}; 
&
\tikz \draw (0,0) .. controls (1,2) and (2,-1) .. (4,0) node[at end] {texte}; 
  \\ \hline 
\RDD{pos}{\color{red}  =0} & \RDD{pos}{\color{red}  =.33} & \RDD{at end} (pos=1)
  \\ \hline 

\tikz \draw (0,0) .. controls (1,2) and (2,-1) .. (4,0) node[very near end] {texte}; 
&
\tikz \draw (0,0) .. controls (1,2) and (2,-1) .. (4,0) node[near end] {texte}; 
&
\tikz \draw (0,0) .. controls (1,2) and (2,-1) .. (4,0) node[midway] {texte}; 
  \\ \hline 
\RDD{very near end} (pos=0.875.) & \RDD{ near end} (pos=0.75) & \RDD{midway} (pos=0.5)
  \\ \hline 
  
\tikz \draw (0,0) .. controls (1,2) and (2,-1) .. (4,0) node[near start] {texte}; 
&
\tikz \draw (0,0) .. controls (1,2) and (2,-1) .. (4,0) node[very near start] {texte}; 
&
\tikz \draw (0,0) .. controls (1,2) and (2,-1) .. (4,0) node[at start] {texte};
\\ \hline 
\RDD{near start} (pos=0.25) & \RDD{very near start} (pos=0.125) & \RDD{at start} (pos=0)
  \\ \hline 
  
\end{tabular} 

\bigskip
\begin{tabular}{|c|c|c|} \hline
\multicolumn{3}{|c|}{  \BS{draw}(0,0) .. controls (1,2) and (2,1) .. (4,0) node[\RDD{sloped},midway] \AC{texte} ;   }\\ 
\hline 
\tikz \draw (0,0) .. controls (1,2) and (2,-1) .. (4,0) node[sloped,midway] {texte};
&
\tikz \draw (0,0) .. controls (1,2) and (2,-1) .. (4,0) node[above,midway] {texte};
&
\tikz \draw (0,0) .. controls (1,2) and (2,-1) .. (4,0) node[below,midway] {texte};
  \\ \hline
\RDD{sloped} & \RDD{above} &\RDD{below}
  \\ \hline
\end{tabular}
\bigskip

\begin{tabular}{|c|c|c|} \hline
\multicolumn{3}{|c|}{  \BS{draw}(0,0) .. controls (1,2) and (2,1) .. (5,0) node[\RDD{sloped},midway,allow upside down] \AC{texte} ;   }\\ 
\hline 
\tikz \draw (0,0) .. controls (1,2) and (2,-1) .. (4,0) node[sloped,midway,allow upside down] {texte};
&
\tikz \draw (0,0) .. controls (1,2) and (2,-1) .. (4,0) node[above,midway,allow upside down] {texte};
&
\tikz \draw (0,0) .. controls (1,2) and (2,-1) .. (4,0) node[below,midway,allow upside down] {texte};
  \\ \hline
\RDD{sloped} & \RDD{above} &\RDD{below}
  \\ \hline
\end{tabular}  


\begin{tabular}{|c|c|c|} \hline
\multicolumn{3}{|c|}{  \BS{draw}(A)  to [bend right]  node [\RDD{bend right}] \AC{texte} (B);   }\\ 
\hline 
\begin{tikzpicture} 
\node[draw] (A) at (0,0) {A};
\node[draw] (B) at (2,2) {B};
\draw (A) to [bend right] node [bend right] {texte} (B);
\end{tikzpicture}
&
\begin{tikzpicture} 
\node[draw] (A) at (0,0) {A};
\node[draw] (B) at (2,2) {B};
\draw (A) to [bend right] node [auto,bend right] {texte} (B);
\end{tikzpicture}
&
\begin{tikzpicture} 
\node[draw] (A) at (0,0) {A};
\node[draw] (B) at (2,2) {B};
\draw (A) to[bend right] node [auto,swap,bend right] {texte} (B);
\end{tikzpicture}
  \\ \hline
[bend right]  & [\RDD{auto},bend right] & [auto,\RDD{swap},bend right] 
  \\ \hline
\end{tabular}  

\SbSSCT{ N\oe uds  sur un \og edge\fg}{Nodes on an edge}

\begin{tabular}{|c|c|c|}\hline  
\multicolumn{3}{|c|}{  \BS{draw}(0,0) edge \rouge{["abc", ->]} (4,0);  }\\ 
\multicolumn{3}{|c|}{  \RRR{17-12-2} }\\ 
\hline 
\begin{tikzpicture}[blue] 
\useasboundingbox  (0,-.5) rectangle (4,.5); 
\draw (0,0) edge ["abc", ->] (4,0);
\end{tikzpicture}
&
\begin{tikzpicture}[blue] 
\useasboundingbox  (0,-.5) rectangle (4,.5); 
\draw (0,0) edge ["abc", near start] (4,0);
\end{tikzpicture}
&
\begin{tikzpicture}[blue] 
\useasboundingbox  (0,-.5) rectangle (4,.5); 
\draw (0,0) edge ["abc", style={auto=right}] (4,0);
\end{tikzpicture}
\\ \hline 
["abc", ->]
& 
["abc", near start] &  ["abc", style=\AC{auto=right}] 
\\ \hline  
\begin{tikzpicture}[blue] 
\useasboundingbox  (0,-.5) rectangle (4,.5); 
\draw (0,0) edge [font=\Large,"abc" ] (4,0);
\end{tikzpicture}
&
\begin{tikzpicture}[blue] 
\useasboundingbox  (0,-.5) rectangle (4,.5); 
\draw (0,0) edge ["abc" color=red ] (4,0);
\end{tikzpicture}
&
\begin{tikzpicture}[blue] 
\useasboundingbox  (0,-.5) rectangle (4,.5); 
 \draw (0,0) edge ["abc" '] (4,0);
\end{tikzpicture}
\\ \hline 
[font=\BS{Large},"abc" ] & ["abc" color=red ]
&["abc" ' ]
\\ \hline 

\begin{tikzpicture}[blue] 
\useasboundingbox  (0,-.5) rectangle (4,.75); 
\draw (0,0) edge ["abc" draw ] (4,0);
\end{tikzpicture}
&
\begin{tikzpicture}[blue] 
\useasboundingbox  (0,-.5) rectangle (4,.5); 
\draw (0,0) edge ["abc" inner sep=0pt ] (4,0);
\end{tikzpicture}
&
\begin{tikzpicture}[blue] 
\useasboundingbox  (0,-.5) rectangle (4,.5); 
\draw (0,0) edge ["abc" fill ,fill=yellow ] (4,0);
\end{tikzpicture}
\\ \hline
["abc" draw ]
&
["abc" inner sep=0pt ]
&
["abc" fill ,fill=yellow ]
\\ \hline
\end{tabular} 



\bigskip

\begin{tabular}{|c|} \hline  
\BS{draw}[every edge quotes/.style=\AC{fill=yellow}] (0,0) edge ["abc"] (4,0);
\\ \hline  
\begin{tikzpicture}[blue] 
\useasboundingbox  (0,-.5) rectangle (4,.5); 
 \draw[every edge quotes/.style={fill=yellow}] (0,0) edge ["abc"] (4,0);
\end{tikzpicture}
\\ \hline 
\end{tabular} 


\subsection{Positionnement relatif de n\oe uds}
\label{lib-pos}

\maboite{\BS{usetikzlibrary}\AC{positioning}}


\begin{center}
\RRR{17-5-3}
\end{center}

\begin{tabular}{|c|c|c|}  \hline 
\multicolumn{2}{|c|}{\BS{node} (a) at (1,0) [above=.4cm+.6cm,draw] \AC{XXX};} &  \\ \hline 
\begin{tikzpicture}
\draw[help lines] (0,0) grid (3,2);
\node (a) at (1,0) [above=.4cm+.6cm,draw] {XXX};
\draw[->,blue,line width=2pt,dotted] (1,0) -- (a.south) node [midway,right,draw=none,fill=red!10] {.4cm+.6cm} ;
\end{tikzpicture} 
&
\begin{tikzpicture}
\draw[help lines] (0,0) grid (3,2);
\node (a) at (1,0) [above=.5+sin(60),draw] {XXX};
\draw[->,blue,line width=2pt,dotted] (1,0) -- (a.south) node [midway,right,draw=none,fill=red!10] {.5+sin(60)} ;
\end{tikzpicture}  
&
\begin{tikzpicture}
\draw[help lines] (0,0) grid (2,2);
\node (a) at (1,0) [above=1,draw] {XXX};
\draw[->,blue,line width=2pt,dotted] (1,0) -- (a.south) node [midway,right,draw=none,fill=red!10] {1} ;
\end{tikzpicture}  
\\ \hline 
above = \rouge{0.4cm+0.6cm} & above = \rouge{.5+sin(60)}  & above = \rouge{1} \\ 
\hline 
\end{tabular} 

\bigskip

\begin{tabular}{|c|c|} \hline 
\multicolumn{2}{|c|}{\BS{node} (a) at (1,0) [\rouge{above right=3cm and 2cm},draw] \AC{XXX};} \\  \hline 
\begin{tikzpicture}
\draw[help lines] (0,0) grid (5,5);
\node (a) at (1,1) [above right=3cm and 2cm,draw] {XXX};
\draw[->,blue,line width=2pt,dotted] (1,1) |- (a.south west);
\end{tikzpicture}
&  
\begin{tikzpicture}
\draw[help lines] (0,0) grid (5,5);

\node (b) at (1,4) [below right=3cm and 2cm,draw] {XXX};
\draw[->,blue,line width=2pt,dotted] (1,4) |- (b.north west);
\end{tikzpicture}
\\ \hline 
\rouge{above right=3cm and 2cm} & \rouge{below right=3cm and 2cm}
\\ \hline 
\end{tabular}  

\bigskip
 
\begin{tabular}{|c|c|}  \hline 
\begin{tikzpicture}[every node/.style=draw,baseline=1.5cm]
\draw[help lines] (0,0) grid (5,4);
\node (a) at (1,1) {node a};
\node (b) [above=2cm of a.north east] {XXX};
\draw[->,blue,line width=2pt,dotted] (a.north) -- (b.south) node [midway,right,draw=none,fill=red!10] {2cm of a.north east} ;
\end{tikzpicture}
&  
\parbox{8cm}{
\BS{node} (a) at (1,1) \AC{node a}; \\
\BS{node} (b) [\rouge{above=2cm of a.north east}] \AC{XXX};}
\\ \hline 
\end{tabular} 

\bigskip

\begin{tabular}{|c|c|}  \hline 
\begin{tikzpicture}[every node/.style=draw]
\draw[help lines] (0,0) grid (2,3);
\node (a) at (1,0) {node a};
\node (b) [above=1cm of a] {node b};
\node (c) [above=1cm of b] {node c};
\draw[->,blue,line width=2pt,dotted] (a.north) -- (b.south) node [midway,right,draw=none,fill=red!10] {1cm} ;
\draw[->,blue,line width=2pt,dotted] (b.north) -- (c.south) node [midway,right,draw=none,fill=red!10] {1cm} ;
\end{tikzpicture}
&  
\begin{tikzpicture}[every node/.style=draw]
\draw[help lines] (0,0) grid (2,3);
\node (a) at (1,0) {node a };
\node (b) [on grid,above=1cm of a] {node b};
\node (c) [on grid,above=1cm of b] {node c};
\draw[->,blue,line width=2pt,dotted] (a.center) -- (b.center) node [midway,right,draw=none,fill=red!10] {1cm} ;
\draw[->,blue,line width=2pt,dotted] (b.center) -- (c.center) node [midway,right,draw=none,fill=red!10] {1cm} ;
\end{tikzpicture}
\\  \hline 
\BS{node} (a) at (1,0) \AC{node a};  &\BS{node} (a) at (1,0) \AC{node a};   \\ 
\BS{node} (b) [above=1cm of a] \AC{node b};  &\BS{node} (b) [\RDD{on grid},above=1cm of a] \AC{node b};   \\ 
\BS{node} (c) [above=1cm of b] \AC{node c};  &\BS{node} (c) [\RDD{on grid},above=1cm of b] \AC{node c};   \\ 
\hline 
\end{tabular} 

\begin{tabular}{|c|c|} \hline 
\begin{tikzpicture}[every node/.style=draw,node distance=1cm,baseline = 1.5cm]
\draw[help lines] (0,0) grid (2,3);
\node (a1) at (1,0) {node a};
\node (b) [above=of a] {node b};
\node (c) [above=of b] {node c};
\draw[->,blue,line width=2pt,dotted] (a.north) -- (b.south) node [midway,right,draw=none,fill=red!10] {1cm} ;
\draw[->,blue,line width=2pt,dotted] (b.north) -- (c.south) node [midway,right,draw=none,fill=red!10] {1cm} ;
\end{tikzpicture}
 & 
 \parbox{12cm}{ 
\BS{begin}\AC{tikzpicture}[every node/.style=draw,\RDD{node distance}=1mm] \\
\BS{node} (a1) at (1,0) \AC{node a}; \\
\BS{node} (b) [above=of a] \AC{node b}; \\
\BS{node} (c) [above=of b] \AC{node c}; \\
\BS{end}\AC{tikzpicture}
} 
 \\ 
\hline 
\end{tabular} 

\bigskip

\begin{tabular}{|l|l|} \hline 
\begin{tikzpicture}[node distance=2cm]
\draw[help lines] (0,-1) grid (6,1);
\huge
\node[draw] (X) at (0,0) {X};
\node[draw] (a) [right=of X] {a};
\node[draw] (y) [right=of a] {y};
\draw[->,blue,line width=2pt,dotted] (X.east) -- (a.west) node [midway,draw=none,fill=red!10] {\small{2cm}} ;
\draw[->,blue,line width=2pt,dotted] (a.east) -- (y.west) node [midway,draw=none,fill=red!10] {\small{2cm}} ;
\end{tikzpicture}
&  
\begin{tikzpicture}[node distance=2cm]
\draw[help lines] (0,-1) grid (6,1);
\huge
\node[draw] (X) at (0,0) {X};
\node[draw] (a) [base right=of X] {a};
\node[draw] (y) [base right=of a] {y};
\draw[->,blue,line width=2pt,dotted] (X.base east) -- (a.base west) node [midway,draw=none,fill=red!10] {\small{2cm}} ;
\draw[->,blue,line width=2pt,dotted] (a.base east) -- (y.base west) node [midway,draw=none,fill=red!10] {\small{2cm}} ;
\end{tikzpicture}
\\ \hline 
\BS{node}[draw] (X) at (0,0) \AC{X};
&  
\BS{node}[draw] (X) at (0,0) \AC{X};
\\
\BS{node}[draw] (a) [right=of X] \AC{a};
&
\BS{node}[draw] (a) [base right=of X] \AC{a};
\\
\BS{node}[draw] (y) [right=of a] \AC{y};
&
\BS{node}[draw] (y) [base right=of a] \AC{y};
\\ \hline 
\end{tabular} 



\SbSSCT{N\oe ud enveloppant}{Fitting nodes}


 \maboite{\BS{usetikzlibrary}\AC{fit}}

\label{lib-fit}
\begin{center}
\RRR{52}
\end{center}


\begin{tabular}{|c|l|}  \hline
\begin{tikzpicture}[baseline=0pt]
\draw[help lines] (0,0) grid (3,2.5);
\fill (.5,1) circle (3pt);
\fill (2,.25) circle (3pt);
\fill (1,2)  circle (3pt);
\fill (1.25,0.25)  circle (3pt);
\fill (1.75,1.5)  circle (3pt);
\node[draw=red,ultra thick,fit={(.5,1) (2,.25) (1,2) (1.25,0.25) (1.75,1.5) }]  {};
\end{tikzpicture}
&
\parbox[b]{10cm}{
\BS{fill } (.5,1) circle (3pt); \\
\BS{fill } (2,.25) circle (3pt);\\
\BS{fill }  (1,2)  circle (3pt); \\
\BS{fill } (1.25,0.25)  circle (3pt); \\
\BS{fill } (1.75,1.5)  circle (3pt);\\
\BS{node}[draw=red,ultra thick,\RDD{fit}=\AC{(.5,1) (2,.25) (1,2) (1.25,0.25) (1.75,1.5) }] \AC{} ;
}
\\ \hline
\end{tabular}

\bigskip

\begin{tabular}{|c|l|}  \hline
\begin{tikzpicture}
[dot/.style={inner sep=0pt,draw,circle,blue},baseline=0pt]
\draw[help lines] (0,0) grid (3,2.5);
\node[dot] (a) at (.5,1) {a};
\node[dot] (b) at (2,.25) {b};
\node[dot] (c) at (1,2) {c};
\node[dot] (d) at (1.25,0.25) {d};
\node[dot] (e) at (1.75,1.5) {e};
\node[draw=red,ultra thick,fit=(a) (b) (c) (d) (e)]  {};
\end{tikzpicture}
&
\parbox[b]{10cm}{
[dot/.style=\AC{inner sep=0pt,draw,circle,blue}]\\
\BS{node}[dot] (a) at (.5,1) \AC{a}; \\
\BS{node}[dot] (b) at (2,.25) \AC{b}; \\
\BS{node}[dot] (c) at (1,2) \AC{c}; \\
\BS{node}[dot] (d) at (1.25,0.25) \AC{d}; \\
\BS{node}[dot] (e) at (1.75,1.5) \AC{e}; \\
\BS{node}[draw=red,ultra thick,\RDD{fit}=(a) (b) (c) (d) (e)] \AC{} 
}
\\ \hline
\end{tabular}

\bigskip

\begin{tabular}{|c|c|c|} \hline 
\multicolumn{3}{|l|}{ \BS{node}[draw=red,ultra thick,fit=(a) (b) (c) (d) (e)] {\color{red}(xxx)} \AC{} }\\
\multicolumn{3}{|l|}{ \BS{node} at {\color{red} (xxx.east)} [fill=green!20] \AC{x};  }\\ 
\hline 
\begin{tikzpicture}
[dot/.style={inner sep=0pt,draw,circle,blue},baseline=0pt]
\draw[help lines] (0,0) grid (3,2.5);
\node[dot] (a) at (.5,1) {a};
\node[dot] (b) at (2,.25) {b};
\node[dot] (c) at (1,2) {c};
\node[dot] (d) at (1.25,0.25) {d};
\node[dot] (e) at (1.75,1.5) {e};
\node[draw=red,fit=(a) (b) (c) (d) (e)] (xxx) {};
\node at (xxx.east) [fill=green!20] {x};
\end{tikzpicture}
&  
\begin{tikzpicture}
[dot/.style={inner sep=0pt,draw,circle,blue},baseline=0pt]
\draw[help lines] (0,0) grid (3,2.5);
\node[dot] (a) at (.5,1) {a};
\node[dot] (b) at (2,.25) {b};
\node[dot] (c) at (1,2) {c};
\node[dot] (d) at (1.25,0.25) {d};
\node[dot] (e) at (1.75,1.5) {e};
\node[draw=red,fit=(a) (b) (c) (d) (e)] (xxx) {};
\node at (xxx.north east) [fill=green!20] {x};
\end{tikzpicture}
&
\begin{tikzpicture}
[dot/.style={inner sep=0pt,draw,circle,blue}]
\draw[help lines] (0,0) grid (3,2.5);
\node[dot] (a) at (.5,1) {a};
\node[dot] (b) at (2,.25) {b};
\node[dot] (c) at (1,2) {c};
\node[dot] (d) at (1.25,0.25) {d};
\node[dot] (e) at (1.75,1.5) {e};
\node[draw=red,fit=(a) (b) (c) (d) (e)] (xxx) {};
\node at (xxx.center) [fill=green!20] {x};
\end{tikzpicture}
\\ \hline xxx.east & xxx.north east & xxx.center \\ 
\hline 
\end{tabular} 

\bigskip

\begin{tabular}{|c|c|} \hline 
\multicolumn{2}{|l|}{ \BS{node} [draw=green,fit=(a) (b) (c) (d) (e)] {};  }\\ 
\multicolumn{2}{|l|}{ \BS{node} [\RDD{inner sep}=0pt,draw=red,fit=(a) (b) (c) (d) (e)] {};  }\\ 
\hline 
 \begin{tikzpicture}
[dot/.style={inner sep=0pt,draw,circle,blue},baseline=0pt]
\draw[help lines] (0,0) grid (3,2.5);
\node[dot] (a) at (.5,1) {a};
\node[dot] (b) at (2,.25) {b};
\node[dot] (c) at (1,2) {c};
\node[dot] (d) at (1.25,0.25) {d};
\node[dot] (e) at (1.75,1.5) {e};
\node[inner sep=0pt,draw=red,ultra thick,fit=(a) (b) (c) (d) (e)]  {};
\node[draw=green,fit=(a) (b) (c) (d) (e)] {};
\end{tikzpicture}
&  
\begin{tikzpicture}
[dot/.style={inner sep=0pt,draw,circle,blue},baseline=0pt]
\draw[help lines] (0,0) grid (3,2.5);
\node[dot] (a) at (.5,1) {a};
\node[dot] (b) at (2,.25) {b};
\node[dot] (c) at (1,2) {c};
\node[dot] (d) at (1.25,0.25) {d};
\node[dot] (e) at (1.75,1.5) {e};
\node[inner sep=.5cm,draw=red,ultra thick,fit=(a) (b) (c) (d) (e)]  {};
\node[draw=green,fit=(a) (b) (c) (d) (e)] {};
\end{tikzpicture}
\\ \hline  
inner sep=0pt & inner sep=.5cm \\ 
\hline 
\end{tabular} 

\bigskip

\begin{tabular}{|c|c|c|}\hline 
\multicolumn{3}{|c|}{ \BS{node}[circle,draw=red,inner sep=0pt,fit=(a) (b) (c) (d) (e)] \AC{};  }\\ 
\hline  
\begin{tikzpicture}
[dot/.style={inner sep=0pt,draw,circle,blue}]
\draw[help lines] (0,0) grid (3,2.5);
\node[dot] (a) at (.5,1) {a};
\node[dot] (b) at (2,.25) {b};
\node[dot] (c) at (1,2) {c};
\node[dot] (d) at (1.25,0.25) {d};
\node[dot] (e) at (1.75,1.5) {e};
\node[circle,draw=red,inner sep=0pt,fit=(a) (b) (c) (d) (e)]  {};
\end{tikzpicture}
&  
\begin{tikzpicture}
[dot/.style={inner sep=0pt,draw,circle,blue}]
\draw[help lines] (0,0) grid (3,2.5);
\node[dot] (a) at (.5,1) {a};
\node[dot] (b) at (2,.25) {b};
\node[dot] (c) at (1,2) {c};
\node[dot] (d) at (1.25,0.25) {d};
\node[dot] (e) at (1.75,1.5) {e};
\node[ellipse,draw=red,inner sep=0pt,fit=(a) (b) (c) (d) (e)] {};
\end{tikzpicture}
&
\begin{tikzpicture}
[dot/.style={inner sep=0pt,draw,circle,blue}]
\draw[help lines] (0,0) grid (3,2.5);
\node[dot] (a) at (.5,1) {a};
\node[dot] (b) at (2,.25) {b};
\node[dot] (c) at (1,2) {c};
\node[dot] (d) at (1.25,0.25) {d};
\node[dot] (e) at (1.75,1.5) {e};
\node[shape=starburst,draw=red,inner sep=0pt,fit=(a) (b) (c) (d) (e)] {};
\end{tikzpicture}
\\ \hline 
circle & ellipse & shape=starburst (\TFRGB{voir}{see} section \ref{ndbt} ) \\ 
\hline 
\end{tabular} 





\bigskip

\begin{tabular}{|c|c|} \hline 
\multicolumn{2}{|c|}{ \BS{node}[draw=red, \RDD{rotate fit}=45, fit=(a) (b) (c) (d) (e)]  \AC{};  }\\ 
\hline  
\begin{tikzpicture}
[inner sep=0pt,
dot/.style={draw,circle,blue}]
\draw[help lines] (0,0) grid (3,2.5);
\node[dot] (a) at (.5,1) {a};
\node[dot] (b) at (2,.25) {b};
\node[dot] (c) at (1,2) {c};
\node[dot] (d) at (1.25,0.25) {d};
\node[dot] (e) at (1.75,1.5) {e};
\node[draw=red, rotate fit=45, fit=(a) (b) (c) (d) (e)] {};
\end{tikzpicture}
&  
\begin{tikzpicture}
[inner sep=0pt,
dot/.style={draw,circle,blue}]
\draw[help lines] (0,0) grid (3,2.5);
\node[dot] (a) at (.5,1) {a};
\node[dot] (b) at (2,.25) {b};
\node[dot] (c) at (1,2) {c};
\node[dot] (d) at (1.25,0.25) {d};
\node[dot] (e) at (1.75,1.5) {e};
\node[ellipse,draw=red, rotate fit=45, fit=(a) (b) (c) (d) (e)] {};
\end{tikzpicture}
\\ \hline rotate fit=45 & ellipse, rotate fit=45 \\ 
\hline 
\end{tabular} 


\newpage


\SbSSCT{Cercle défini par deux points}{Circle defined by two points }


 \maboite{\BS{usetikzlibrary}\AC{through}}
\label{lib-through}


\begin{center}
\RRR{ 71 }
\end{center}

\begin{tabular}{|c|} \hline 
\BS{node} [draw] at (2,1) [\RDD{circle through}=\AC{(1,2)}] \AC{c};
\\ \hline 
\begin{tikzpicture}
\draw[help lines] (0,0) grid (3,2);
\filldraw [red] (1,2) circle (2pt);
\node [draw] at (2,1) [circle through={(1,2)}] {c};
\end{tikzpicture}
\\ \hline 
\end{tabular}

\newpage


\SbSSCT{Matrice de n\oe uds}{Matrices and Alignment}


\label{matrix}
\begin{center}
\RRR{20}
\end{center}

\begin{tabular}{|c|c|} \hline  
\begin{tikzpicture}[baseline=1cm]
\draw[help lines] (0,0) grid (4,2);
\node [matrix,fill=red!20,draw=blue,very thick] (my matrix) at (2,1)
{
\draw (0,0) circle (4mm); & \node[rotate=45] {Hello}; \\
\draw (0.2,0) circle (2mm); & \fill[red] (0,0) circle (3mm); \\
};
\end{tikzpicture}
& 
\parbox{10cm}{
\BS{node} [\RDD{matrix},fill=red!10,draw=blue,very thick] at (2,1) \\
\{ \\
\BS{draw} (0,0) circle (4mm); \& \BS{node} [rotate=45] {Hello}; \BS{}\BS{} \\
\BS{draw}  (0.2,0) circle (2mm); \& \BS{fill}[red] (0,0) circle (3mm); \BS{}\BS{} \\
\}; \\
}
\\ \hline 
\end{tabular} 

\bigskip

\begin{tabular}{|c|c|} \hline  
\begin{tikzpicture}[baseline=0pt]
\matrix [fill=red!20,draw=blue,very thick] 
{
\draw (0,0) circle (4mm); & \node[rotate=45] {Hello}; \\
\draw (0.2,0) circle (2mm); & \fill[red] (0,0) circle (3mm); \\
};
\end{tikzpicture}
&  
\parbox{10cm}{
\BSS{matrix} [fill=red!10,draw=blue,very thick] \\
\{ \\
\BS{draw} (0,0) circle (4mm); \& \BS{node} [rotate=45] {Hello}; \BS{}\BS{} \\
\BS{draw}  (0.2,0) circle (2mm); \& \BS{fill}[red] (0,0) circle (3mm); \BS{}\BS{} \\
\}; \\
}
\\ \hline 
\end{tabular} 


\SbSbSSCT{Alignement des cellules}{Cell Pictures}


\begin{center}
\RRR{20-3}
\end{center}

\begin{tabular}{|c|c|c|} \hline  
\begin{tikzpicture}
[every node/.style={draw=black,font=\huge}]
\matrix [draw=red]
{
\node {a}; \fill[blue] (0,0) circle (2pt); &
\node {X}; \fill[blue] (0,0) circle (2pt); &
\node {g}; \fill[blue] (0,0) circle (2pt); \\
};
\end{tikzpicture}
&  
\begin{tikzpicture}
[every node/.style={draw=black,anchor=base,font=\huge}]
\matrix [draw=red]
{
\node {a}; \fill[blue] (0,0) circle (2pt); &
\node {X}; \fill[blue] (0,0) circle (2pt); &
\node {g}; \fill[blue] (0,0) circle (2pt); \\
};
\end{tikzpicture}
&  
\begin{tikzpicture}[every node/.style={draw=black}]
\matrix [draw=red,anchor=north,font=\huge]
{
\node {a}; \fill[blue] (0,0) circle (2pt); &
\node {X}; \fill[blue] (0,0) circle (2pt); &
\node {g}; \fill[blue] (0,0) circle (2pt); \\
};
\end{tikzpicture}
\\ \hline  
 & anchor=base &  anchor=north \\ \hline 
\end{tabular} 

\bigskip
\begin{tabular}{|c|c|c|} \hline  
\begin{tikzpicture}
[every node/.style={draw=black,font=\huge}]
\matrix [draw=red]
{

\node[left]  {X}; \fill[blue] (0,0) circle (2pt);  \\
};
\end{tikzpicture}
&  
\begin{tikzpicture}
[every node/.style={draw=black,anchor=base,font=\huge}]
\matrix [draw=red]
{
\node {a}; \fill[blue] (0,0) circle (2pt); ²\\
\node[right] {X}; \fill[blue] (0,0) circle (2pt);  \\
\node {g}; \fill[blue] (0,0) circle (2pt); \\
};
\end{tikzpicture}
&  
\begin{tikzpicture}[every node/.style={draw=black}]
\matrix [draw=red,anchor=north,font=\huge]
{
\node {a}; \fill[blue] (0,0) circle (2pt); &
\node[right] {X}; \fill[blue] (0,0) circle (2pt); &
\node {g}; \fill[blue] (0,0) circle (2pt); \\
};
\end{tikzpicture}
\\ \hline  
 & anchor=base &  anchor=north \\ \hline 
\end{tabular} 

\bigskip

\begin{tabular}{|c|c|} \hline  
\begin{tikzpicture}[baseline=0pt]
\matrix [draw=red,nodes=draw]
{
\node[left] {A}; \fill[blue] (0,0) circle (2pt); \\
\node {B}; \fill[blue] (0,0) circle (2pt); \\
\node[right] {C}; \fill[blue] (0,0) circle (2pt); \\
};
\end{tikzpicture}
&  
\parbox{12cm}{
\BS{matrix} [draw=red,nodes=draw]
\AC{\\
\BS{node}\rouge{[left]} {A}; \BS{fill}[blue] (0,0) circle (2pt); \BS{} \BS{} \\
\BS{node} {B}; \BS{fill}[blue] (0,0) circle (2pt);\BS{} \BS{} \\
\BS{node}\rouge{[right]} {C}; \BS{fill}[blue] (0,0) circle (2pt); \BS{} \BS{}\\
}; \\
}

\\ \hline 
\end{tabular} 

\bigskip

\begin{tabular}{|c|c|} \hline  
\multicolumn{2}{|c|}{\BS{matrix} [draw,\RDD{column  sep}=1cm,nodes=draw]} 
\\ \hline 
\begin{tikzpicture}
\matrix [draw,column sep=1cm,nodes=draw]
{
\node(a) {123}; & \node (b) {1}; & \node {1}; \\
\node {12}; & \node {12}; & \node {1}; \\
\node(c) {1}; & \node (d) {123}; & \node {1}; \\
};
\draw [red,thick] (a.east) -- (a.east |- c)
(d.west) -- (d.west |- b);
\draw [<->,red,thick] (a.east) -- (d.west |- b)
node [above,midway] {1cm};
\end{tikzpicture}
&  
\begin{tikzpicture}
\matrix [draw,column sep={1cm,between origins},nodes=draw]
{
\node(a) {123}; & \node (b) {1}; & \node {1}; \\
\node {12}; & \node {12}; & \node {1}; \\
\node {1}; & \node {123}; & \node {1}; \\
};
\draw [<->,red,thick] (a.center) -- (b.center) node [above,midway] {1cm};
\end{tikzpicture}
\\ \hline \RDD{column sep}=1cm & column sep=\AC{1cm,\RDD{between origins} } 
\\ \hline 
\end{tabular} 

\bigskip

\begin{tabular}{|c|c|} \hline
\multicolumn{2}{|c|}{\BS{matrix} [draw,\RDD{row sep}=1cm,nodes=draw]} 
\\ \hline 
\begin{tikzpicture}
\matrix [draw,row sep=1cm,nodes=draw]
{
\node (a) {123}; & \node {1}; & \node {1}; \\
\node (b) {12}; & \node {12}; & \node {1}; \\
\node {1}; & \node {123}; & \node {1}; \\
};
\draw [<->,red,thick] (a.south) -- (b.north) node [right,midway] {1cm};
\end{tikzpicture}
&
\begin{tikzpicture}
\matrix [draw,row sep={1cm,between origins},nodes=draw]
{
\node (a) {123}; & \node {1}; & \node {1}; \\
\node (b) {12}; & \node {12}; & \node {1}; \\
\node {1}; & \node {123}; & \node {1}; \\
};
\draw [<->,red,thick] (a.center) -- (b.center) node [right,midway] {1cm};
\end{tikzpicture}
\\  \hline 
\RDD{row sep}=1cm  & row sep=\AC{1cm,\RDD{between origins} } 
\\ \hline 


\end{tabular} 




\bigskip

\begin{tabular}{|c|c|} \hline  
\multicolumn{2}{|c|}{\BS{matrix} [ \rouge{row sep=5mm},draw,nodes=draw]} \\
\multicolumn{2}{|c|}{ \{ \BS{node} \AC{1}; \& \BS{node} \AC{2}; \& \BS{node} \AC{3}; \BS{}\BS{}  } \\
\multicolumn{2}{|c|}{ \BS{node} \AC{4} ; \& \BS{node}  \AC{5}; \& \BS{node}  \AC{6};  \BS{}\BS{} \rouge{[1cm]} } \\
\multicolumn{2}{|c|}{ \BS{node} \AC{7}; \& \BS{node}\AC{8}; \& \BS{node}\AC{9}; \BS{}\BS{} \}  } 
\\ \hline  
\begin{tikzpicture}
\matrix [row sep=5mm,draw,nodes=draw]
{
\node {1}; & \node {2};& \node {3}; \\
\node(a) {4} ; & \node {5}; & \node {6};\\[1cm]
\node(b) {7}; &\node {8}; & \node {9}; \\
};
\draw [<->,red,thick] (a.center) -- (b.center) node [right,midway] {1,5cm};
\end{tikzpicture}
&  
\begin{tikzpicture}
\matrix [row sep=5mm,draw,nodes=draw]
{
\node {1}; & \node {2};& \node {3}; \\
\node(a) {4} ; & \node {5}; & \node {6};\\[10mm,between origins]
\node(b) {7}; &\node {8}; & \node {9}; \\
};
\draw [<->,red,thick] (a.center) -- (b.center) node [right,midway] {1,5cm};
\end{tikzpicture}
\\ \hline 
\rouge{[1cm]} & \rouge{[1cm,between origins]}
\\ \hline 
\end{tabular} 

\bigskip

\begin{tabular}{|c|c|} \hline  
\multicolumn{2}{|c|}{\BS{matrix} [ \rouge{column sep=5mm},draw,nodes=draw]} \\
\multicolumn{2}{|c|}{ \{ \BS{node} \AC{1}; \& \BS{node} \AC{2}; \& \BS{node} \AC{3}; \BS{}\BS{}  } \\
\multicolumn{2}{|c|}{ \BS{node} \AC{4} ; \& \BS{node}  \AC{5}; \& \rouge{[1cm]}\BS{node}  \AC{6};  \BS{}\BS{}  } \\
\multicolumn{2}{|c|}{ \BS{node} \AC{7}; \& \BS{node}\AC{8}; \& \BS{node}\AC{9}; \BS{}\BS{} \}  } 
\\ \hline  

\begin{tikzpicture}
\matrix [draw,nodes=draw,column sep=5mm]
{
\node {1}; & \node(a) {2}; &[1cm] \node(b) {3}; \\
\node {4}; & \node{5}; & \node {6}; \\
\node {7}; & \node{8}; & \node {9}; \\
};
\draw [<->,red,thick] (a.east) -- (b.west) node [above,midway] {15mm};
\end{tikzpicture}
&  
\begin{tikzpicture}
\matrix [draw,nodes=draw,column sep=5mm]
{
\node {1}; &[2mm] \node(a){2}; &[1cm,between origins] \node(b){3}; \\
\node {4}; & \node {5}; & \node {6}; \\
\node {7}; & \node {8}; & \node {9}; \\
};
\draw [<->,red,thick] (a.center) -- (b.center) node [above,midway] {15mm};
\end{tikzpicture}
\\ \hline  
\rouge{[1cm]}
&  
\rouge{[1cm,between origins]}
\\ \hline 
\end{tabular} 




\bigskip

\begin{tikzpicture}
\matrix [draw,nodes=draw,column sep={1cm,between origins}]
{
\node (a) {8}; & \node (b) {1}; &[between borders] \node (c) {6}; \\
\node {3}; & \node {5}; & \node {7}; \\
\node {4}; & \node {9}; & \node {2}; \\
};
\draw [<->,red,thick] (a.center) -- (b.center) node [above,midway] {10mm};
\draw [<->,red,thick] (b.east) -- (c.west) node [above,midway] {1cm};
\end{tikzpicture}



\SbSbSSCT{Format des cellules}{Cell Styles and Options}

\noindent 

\begin{tabular}{|c|} \hline  
\BS{matrix} [nodes=draw,nodes=\AC{\rouge{fill}=blue!10\rouge{,minimum size}=1cm}]
\\ \hline  
\begin{tikzpicture}
\matrix [nodes=draw,nodes={fill=blue!10,minimum size=1cm}]
{
\node {1}; & \node{2}; & \node {3}; \\
\node {4}; & \node{5}; & \node {6}; \\
\node {7}; & \node{8}; & \node {9}; \\
};
\end{tikzpicture}
\\ \hline 
\end{tabular} 


\bigskip 


\begin{tabular}{|c|c|c|} \hline 
\multicolumn{3}{|c|}{\BS{matrix}[\rouge{row 2/.style}=\AC{red}]}
 \\ \hline 
\begin{tikzpicture}
\matrix[row 2/.style={red}]
{
\node {8}; & \node{1}; & \node {6}; \\
\node {3}; & \node{5}; & \node {7}; \\
\node {4}; & \node{9}; & \node {2}; \\
};
\end{tikzpicture}
&  
\begin{tikzpicture}
\matrix[column 2/.style={red}]
{
\node {8}; & \node{1}; & \node {6}; \\
\node {3}; & \node{5}; & \node {7}; \\
\node {4}; & \node{9}; & \node {2}; \\
};
\end{tikzpicture}
&  
\begin{tikzpicture}
\matrix[row 2 column 2/.style={red}]
{
\node {8}; & \node{1}; & \node {6}; \\
\node {3}; & \node{5}; & \node {7}; \\
\node {4}; & \node{9}; & \node {2}; \\
};
\end{tikzpicture}
\\ \hline 
row 2/.style=\AC{red} & column 2/.style=\AC{red}  & row 2 column 2/.style=\AC{red}\\ 
\hline 
\end{tabular} 

\bigskip 

\begin{tabular}{|c|c|c|} \hline 
\multicolumn{3}{|c|}{\BS{matrix}[column 1/.style=\AC{anchor=west}]}
 \\ \hline 
\begin{tikzpicture}
\matrix[column 1/.style={anchor=west}]
{
\node {12345};  & \node {67890}; \\
\node {123}; & \node{67};  \\
\node {1}; & \node{6}; & \\
};
\end{tikzpicture}
&  
\begin{tikzpicture}
\matrix[column 1/.style={anchor=east}]
{
\node {12345};  & \node {67890}; \\
\node {123}; & \node{67};  \\
\node {1}; & \node{6}; & \\
};
\end{tikzpicture}
&  
\begin{tikzpicture}
\matrix[column 1/.style={anchor=base}]
{
\node {12345};  & \node {67890}; \\
\node {123}; & \node{67};  \\
\node {1}; & \node{6}; & \\
};
\end{tikzpicture}
\\  \hline  
[\rouge{column 1/.style}={anchor=west}]& [\rouge{column 1/.style}={anchor=east}] & [\rouge{column 1/.style}={anchor=base}]\\ 
\hline 
\end{tabular} 

\bigskip

\begin{tabular}{|c|c|c|c|} \hline
\multicolumn{4}{|c|}{\BS{matrix}[matrix of nodes,\RDD{every odd column}/.style={red}]}
 \\ \hline 
\begin{tikzpicture}
\matrix [matrix of nodes,every odd column/.style={red}]
{
a & b & c & d \\
e & f & g & h \\
i & j & k & l \\
};
\end{tikzpicture}
&  
\begin{tikzpicture}
\matrix [matrix of nodes,every even column/.style={red}]
{
a & b & c & d \\
e & f & g & h \\
i & j & k & l \\
};
\end{tikzpicture}
&  
\begin{tikzpicture}
\matrix [matrix of nodes,every odd row/.style={red}]
{
a & b & c & d \\
e & f & g & h \\
i & j & k & l \\
};
\end{tikzpicture}
&  
\begin{tikzpicture}
\matrix [matrix of nodes,every even row/.style={red}]
{
a & b & c & d \\
e & f & g & h \\
i & j & k & l \\
};
\end{tikzpicture}
\\ 
\hline 
\RDD{every odd column} & \RDD{every even column} & \RDD{every odd row}  & \RDD{every even row} \\ 
\hline 
\end{tabular} 


\bigskip


\begin{tabular}{|c|} \hline  
\BS{matrix} [draw,matrix of nodes,\rouge{execute at begin cell}=\AC{(}]
\\ \hline  
\begin{tikzpicture}
\matrix [draw,matrix of nodes,execute at begin cell={(}]
{
1 & 2 &   \\
4 &   & 6 \\
  &   & 9 \\
};
\end{tikzpicture}
\\ \hline 
\end{tabular} 

\bigskip

\begin{tabular}{|c|} \hline  
\BS{tikz} 
[matrix of nodes/.style=\AC{
execute at begin cell=\BS{node}\BS{bgroup} , \\
\rouge{execute at end cell}=\$m\wedge 2\$\BS{egroup}; 
}] \\
\BS{matrix} [draw,matrix of nodes
]
\\ \hline  
\tikz 
[matrix of nodes/.style={
execute at begin cell=\node\bgroup ,
execute at end cell=$m^2$\egroup;
}]
\matrix [draw,matrix of nodes
]
{1 & 2 &  \\
4 &   & 6 \\
  & 8 & 9 \\
};
\\ \hline 
\end{tabular}

\bigskip

\begin{tabular}{|c|} \hline 

 \BS{matrix} [raw,matrix of nodes, \rouge {execute at empty cell}=\BS{node}\AC{- -}; ]
\\ \hline 
 
\begin{tikzpicture}
\matrix [draw,matrix of nodes,execute at empty cell=\node{--};]
{
1 & 2 & \\
4 & & 6 \\
& & 9 \\
};
\end{tikzpicture}
\\ \hline  
\end{tabular} 


\newpage
\SbSbSSCT{Points d'ancrage}{Anchoring a Matrix}

\begin{center}
\RRR{20-4}
\end{center}

\begin{tabular}{|c|c|c|} \hline 
\multicolumn{3}{|c|}{
\BS{matrix} [draw=red,nodes=draw,\RDD{matrix anchor}=east](XXX) at (1,1) }
\\ \hline  
\begin{tikzpicture}
\draw[help lines] (0,0) grid (3,3);
\matrix [draw=red,nodes=draw,matrix anchor=west](XXX) at (1,1)
{
\node {123}; \\ 
\node {12}; \\
\node {1}; \\
};
\fill[red](XXX.west) circle (3pt);
\end{tikzpicture}
&  
\begin{tikzpicture}
\draw[help lines] (0,0) grid (3,3);
\matrix [draw=red,nodes=draw,matrix anchor=east](XXX) at (1,1)
{
\node {123}; \\ 
\node {12}; \\
\node {1}; \\
};
\fill[red] (XXX.east) circle (3pt);
\end{tikzpicture}
&  
\begin{tikzpicture}
\draw[help lines] (0,0) grid (3,3);
\matrix [draw=red,nodes=draw,matrix anchor=south](XXX) at (1,1)
{
\node {123}; \\ 
\node {12}; \\
\node {1}; \\
};
\fill[red](XXX.south) circle (3pt);
\end{tikzpicture}

\\  \hline 
matrix anchor=west & matrix anchor=east & matrix anchor=south 
\\ \hline 
\end{tabular} 

\bigskip 
\begin{tabular}{|c|c|c|c|} \hline 
\multicolumn{2}{|c|}{\BS{matrix} [draw=red,nodes=draw,\rouge{anchor=west}] }
\\ \hline  
\begin{tikzpicture}
\matrix [draw=red,nodes=draw,anchor=west] 
{
\node {123}; & \node {abc}; \\ 
\node {12}; & \node {ab}; \\
\node {1}; & \node {a}; \\
};
\end{tikzpicture}
&  
\begin{tikzpicture}
\matrix [draw=red,nodes=draw,anchor=east] 
{
\node {123};& \node {abc}; \\ 
\node {12};  &\node {ab};\\
\node {1};  & \node {a}; \\
};
\end{tikzpicture}

\\ \hline  
anchor=west & anchor=east  \\ 
\hline 
\end{tabular} 

\bigskip 


\begin{tabular}{|c|c|}\hline  
\begin{tikzpicture}[baseline=1cm]
\draw[help lines] (0,0) grid (4,3);
\matrix[draw=red,nodes=draw ,matrix anchor=inner node.south,anchor=base, row sep=5mm, column sep=5mm] at (2,1)
{
\node {a}; & \node {b}; & \node {c}; & \node {d}; \\
\node {a}; & \node {b}; & \node(inner node){c}; & \node {d}; \\
\node {a}; & \node {b}; & \node {c}; & \node {d}; \\
};
\fill[red] (inner node.south) circle (3pt);
\end{tikzpicture}
&  
\parbox{10.5cm}{
\BS{matrix}[draw=red,nodes=draw, \\ 
\RDD{ matrix anchor}=\blll{inner node}.south, anchor=base, \\
  row sep=5mm,column sep=5mm] at (2,1) \\
\{ \\
\BS{node} \AC{a}; \& \BS{node} \AC{b}; \& \BS{node} \AC{c}; \& \BS{node} \AC{d};  \BS{}\BS{} \\
\BS{node} \AC{a}; \& \BS{node} \AC{b}; \& \BS{node}(\blll{inner node})\AC{c}; \& \BS{node} \AC{d};  \BS{}\BS{} \\
\BS{node}\AC{a}; \& \BS{node} \AC{b}; \& \BS{node}\AC{c}; \& \BS{node} \AC{d}; \BS{}\BS{}  \\
\};
}
\\ \hline 
\end{tabular} 


\SbSbSSCT{Changement du séparateur}{Considerations Concerning Active Characters}

\begin{center}
\RRR{20-5}
\end{center}

\begin{tabular}{|c|c|} \hline  
\tikz[baseline=0pt]
\matrix [ampersand replacement=\|]
{
\draw (0,0) circle (4mm); \| \node[rotate=10] {Hello}; \\
\draw (0.2,0) circle (2mm); \| \fill[red] (0,0) circle (3mm); \\
};
& 
\parbox{12cm}{ 
\BS{tikz}
\BS{matrix} [\RDD{ampersand replacement}=\blll{\BS{|}} ] \\
\{ \\
\BS{draw} (0,0) circle (4mm); \blll{\BS{|} }  \BS{node}[rotate=10] \AC{Hello}; \BS{}\BS{} \\
\BS{draw} (0.2,0) circle (2mm);  \blll{\BS{|} }   \BS{fill}[red] (0,0) circle (3mm); \BS{}\BS{} \\
\}; \\
}
\\ \hline 
\end{tabular} 


\SbSSCT{Matrice de n\oe uds (compléments) }{Matrix Library}

 \maboite{\BS{usetikzlibrary}\AC{matrix}}
\label{lib-matrix}


\begin{center}
\RRR{57-1}
\end{center}

\begin{tabular}{|c|c|} \hline  
\begin{tikzpicture}[baseline=0pt]
\matrix (XXX) [matrix of nodes]
{
1 & 2 & 3 \\
4 & 5 & 6 \\
7 & 8 & 9 \\
};
\end{tikzpicture}
& 
\parbox{10cm}{ 
\BS{begin}\AC{tikzpicture} \\
\BSS{matrix}  [matrix of nodes]\\
\{ \\
1 \hspace{3mm} \& \hspace{3mm}  2 \hspace{3mm} \& \hspace{3mm} 3 \hspace{3mm} \BS{}\BS{}   \\
4 \hspace{3mm} \& \hspace{3mm}  5 \hspace{3mm} \& \hspace{3mm} 6 \hspace{3mm} \BS{}\BS{}  \\
7 \hspace{3mm} \& \hspace{3mm}  8 \hspace{3mm} \& \hspace{3mm} 9 \hspace{3mm} \BS{}\BS{} \\
\}; \\
\BS{end}\AC{tikzpicture}
}
\\ \hline  
\end{tabular} 

\bigskip

\begin{tabular}{|c|c|} \hline  
\begin{tikzpicture}[baseline=0pt]
\matrix (XXX) [matrix of nodes,column sep=.5cm,row sep=.5cm,every node/.style=draw]
{
1 & 2 & 3 \\
4 & 5 & 6 \\
7 & 8 & 9 \\
};
\draw[thick,red,->] (XXX-1-1) -- (XXX-2-3);
\end{tikzpicture}
& 
\parbox{10cm}{ 
\BS{begin}\AC{tikzpicture} \\
\BSS{matrix} \blll{(XXX)} [matrix of nodes,column sep=.5cm,row sep=.5cm,every node/.style=draw]\\
\{ \\
1 \hspace{3mm} \& \hspace{3mm} 2 \hspace{3mm} \& \hspace{3mm} 3 \hspace{3mm} \BS{}\BS{}   \\
4 \hspace{3mm} \& \hspace{3mm} 5 \hspace{3mm} \& \hspace{3mm} 6 \hspace{3mm} \BS{}\BS{}  \\
7 \hspace{3mm} \& \hspace{3mm} 8 \hspace{3mm} \& \hspace{3mm} 9 \hspace{3mm} \BS{}\BS{} \\
\}; \\
\BS{draw}[thick,red,->] \blll{(XXX-1-1)} - - \blll{(XXX-2-3)} ; \\
\BS{end}\AC{tikzpicture}
}
\\ \hline  
\end{tabular} 

\bigskip


\begin{tabular}{|c|c|} \hline  
\begin{tikzpicture}
\matrix [matrix of nodes,column sep=.5cm,row sep=.5cm,every node/.style=draw]
{
8 & 1 & 6 \\
3 & 5 & |[red]| 7 \\
4 & 9 & 2 \\
};
\end{tikzpicture}
&  
\begin{tikzpicture}
\matrix [matrix of nodes]
{
1 & \& &  2 & \& &  3 				& \BS{}\BS{} \\
4 & \& & 5 	& \& & \rouge{ $|[$red$]|$} 6 & \BS{}\BS{} \\
7 & \& & 8 	& \& & 9 				& \BS{}\BS{} \\
};
\end{tikzpicture}
\\ \hline 
\end{tabular}  


\bigskip

\begin{tabular}{|c|c|} \hline 
\begin{tikzpicture}[baseline=-1cm] 
\matrix [matrix of nodes,column sep=.5cm,row sep=.5cm,every node/.style=draw]
{
AAA 			& |[circle]| BBB \\
CCC & |(d) [isosceles triangle]| DDD \\
| [ellipse]| EEE &  FFF \\
};
\end{tikzpicture}
& 
\begin{tikzpicture}
\matrix [matrix of nodes]
{
AAA & \& & \rouge{ $|[$circle$]|$} BBB &  \BS{}\BS{} \\
CCC & \& &\rouge{ $|[$isosceles triangle$]|$} DDD 	&  \BS{}\BS{} \\
\rouge{ $|[$ellipse$]|$} EEE & \& & FFF & \BS{}\BS{} \\
};
\end{tikzpicture}
\\ \hline 
\end{tabular} 


\bigskip

\begin{tabular}{|c|c|} \hline 
\begin{tikzpicture}[baseline=-2cm] 
\matrix [matrix of nodes,column sep=.5cm,row sep=.5cm,every node/.style=draw]
{
|(a)| AAA 	& |(b)| BBB \\
|(c)| CCC 	& |(d)| DDD \\
|(e)| EEE 	& |(f)| FFF \\
};
\draw (a) -- (d);
\draw (d) -- (f);
\end{tikzpicture}
&  
\begin{tikzpicture}
\node at (0,1.5) [text width=10cm]
{\BS{matrix} [matrix of nodes,column sep=.5cm,row sep=.5cm,every node/.style=draw] \\
\{ 
};
\matrix [matrix of nodes]
{
\rouge{ $|$(a)$|$} AAA & \& & \rouge{ $|$(b)$|$} BBB &  \BS{}\BS{} \\
\rouge{ $|$(c)$|$} CCC & \& & \rouge{ $|$(d)$|$} DDD 	&  \BS{}\BS{} \\
\rouge{ $|$(e)$|$} EEE & \& & \rouge{ $|$(f)$|$} FFF & \BS{}\BS{} \\
};

\node at (0,-1.2) [text width=10cm]
{  \}; \\ 
\BS{draw} (a) - - (d); \\ \BS{draw} (d) - - (f);
};
\end{tikzpicture}
\\ \hline 
\end{tabular} 

\bigskip


\begin{tabular}{|c|c|} \hline  
\begin{tikzpicture}
\matrix [matrix of nodes]
{
1 &[1cm] 2 &[5mm] |[red]| 3 \\
4 & 5 &  6 \\
7 & 8 & 9 \\
};
\end{tikzpicture}
&
\begin{tikzpicture}
\matrix [matrix of nodes]
{
1 & \& & \rouge{\lbrack 1cm \rbrack} 2 & \& &\rouge{\lbrack 5mm \rbrack} |[red]| 3 & \BS{}\BS{} \\
4 & \& & 5 & \& & 6 & \BS{}\BS{} \\
7 & \& & 8 & \& & 9 & \BS{}\BS{} \\
};
\end{tikzpicture}

\\ \hline 
\end{tabular} 



\bigskip

\begin{tabular}{|c|c|} \hline  
\begin{tikzpicture}[baseline=0pt]
\matrix [matrix of math nodes]
{
A_1 & A_2 & A_3 \\
a_4 & a_5 &  a_6 \\
a^7 & a^8 & a^9 \\
};
\end{tikzpicture}
&  
\parbox{8cm}{ 
\BSS{matrix}  [\rouge{ matrix of math nodes}]\\
\{ \\
A\_1 \hspace{2mm}  \& \hspace{2mm}  A\_2 \hspace{2mm}  \& \hspace{2mm}  A\_3 \hspace{2mm}   \BS{}\BS{}   \\
a\_4 \hspace{2mm}  \& \hspace{2mm}  a\_5 \hspace{2mm}  \&  \hspace{2mm}  a\_6 \hspace{2mm}  \BS{}\BS{}  \\
a\land 7 \hspace{2mm}  \& \hspace{2mm}  a\land 8 \hspace{2mm}  \& \hspace{2mm}  a\land 9 \hspace{2mm}   \BS{}\BS{} \\
\}; 
}
\\ \hline  
\end{tabular} 

\bigskip

\begin{tabular}{|c|c|} \hline  
\begin{tikzpicture}[baseline=0pt]
\matrix [matrix of math nodes,nodes={circle,draw}]
{
a_1 & & a_3 \\
a_4 & & a_6 \\
a_7 & a_8 & \\
};
\end{tikzpicture}
&  
\parbox{10cm}{ 
\BSS{matrix}  [matrix of math nodes,\rouge{nodes={circle,draw}}]\\
\{ \\
A\_1 \hspace{2mm}  \& \hspace{12mm}  \& \hspace{2mm}  A\_3 \hspace{2mm}   \BS{}\BS{}   \\
a\_4 \hspace{2mm}  \& \hspace{12mm}  \& \hspace{2mm}   a\_6 \hspace{2mm}  \BS{}\BS{}  \\
a\_ 7 \hspace{2mm}  \& \hspace{2mm}  a\_ 8 \hspace{2mm}  \& \hspace{12mm}    \BS{}\BS{} \\
\}; 
}
\\ \hline 
\end{tabular} 

\bigskip

\begin{tabular}{|c|c|} \hline  
\begin{tikzpicture}[baseline=0pt]
\matrix [matrix of math nodes,nodes={circle,draw},nodes in empty cells]
{
a_1 & & a_3 \\
a_4 & & a_6 \\
a_7 & a_8 & \\
};
\end{tikzpicture}
&  
\parbox{10cm}{ 
\BSS{matrix}  [matrix of math nodes,nodes={circle,draw} ,\rouge{nodes in empty cells}]\\
\{ \\
A\_1 \hspace{2mm}  \& \hspace{12mm}  \& \hspace{2mm}  A\_3 \hspace{2mm}   \BS{}\BS{}   \\
a\_4 \hspace{2mm}  \& \hspace{12mm}  \& \hspace{2mm}   a\_6 \hspace{2mm}  \BS{}\BS{}  \\
a\_ 7 \hspace{2mm}  \& \hspace{2mm}  a\_ 8 \hspace{2mm}  \& \hspace{12mm}    \BS{}\BS{} \\
\}; 
}
\\ \hline 
\end{tabular} 

\SbSbSSCT{Texte dans les n\oe uds}{Characters in Matrices of Nodes}

\begin{center}
\RRR{57-2}
\end{center}


\begin{tabular}{|c|c|} \hline  
\begin{tikzpicture}[baseline=0pt]
\matrix [matrix of nodes,nodes={text width=2cm,draw}]
{
aaa  & bbb \\ 
ccc \\
eee & fff\\
};
\end{tikzpicture}
&  
\parbox{10cm}{ 
\BSS{matrix}  [matrix of nodes,\rouge{nodes=\AC{text width=2cm,draw}} ]\\
\{ \\
aaa \&  bbb \BS{}\BS{}  \\
ccc \BS{}\BS{}  \\
eee \& fff \BS{}\BS{}  \\
\}; 
}
\\ \hline 
\end{tabular} 

\bigskip

\begin{tabular}{|c|c|}  \hline  
\begin{tikzpicture}[baseline=0cm]
\matrix [matrix of nodes,nodes={text width=2cm,draw}]
{
1 & {aaa \\ bbb \\ ccc } \\
2 & ddd \\
};
\end{tikzpicture}
&  
\parbox{10cm}{ 
\BSS{matrix}  [matrix of nodes,nodes=\AC{text width=2cm,draw} ]\\
\{ \\
1 \& \& \rouge { \AC{aaa \BS{}\BS{} bbb \BS{}\BS{} ccc } } \BS{}\BS{}   \\
2 \& \& ddd \BS{}\BS{}  \\
\}; 
}
\\ \hline 
\end{tabular} 

\bigskip

\SbSbSSCT{Délimiteurs}{Delimiters}


\begin{center}
\RRR{57-3}
\end{center}

\bigskip

\begin{tabular}{|c|c|c|c|} \hline 
\multicolumn{4}{|c|}{\BS{matrix} [matrix of math nodes,\RDD{left delimiter}=( ]}
\\ \hline  
\begin{tikzpicture}
\matrix [matrix of math nodes,left delimiter=( ]
{
a_1 & a_2 & a_3 \\
a_4 & a_5 & a_6 \\
a_7 & a_8 & a_9 \\
};
\end{tikzpicture}
&  
\begin{tikzpicture}
\matrix [matrix of math nodes,right delimiter=\}]
{
a_1 & a_2 & a_3 \\
a_4 & a_5 & a_6 \\
a_7 & a_8 & a_9 \\
};
\end{tikzpicture}
&
\begin{tikzpicture}
\matrix [matrix of math nodes,above delimiter=\| ]
{
a_1 & a_2 & a_3 \\
a_4 & a_5 & a_6 \\
a_7 & a_8 & a_9 \\
};
\end{tikzpicture}
&
\begin{tikzpicture}
\matrix [matrix of math nodes,below delimiter=\rmoustache ]
{
a_1 & a_2 & a_3 \\
a_4 & a_5 & a_6 \\
a_7 & a_8 & a_9 \\
};
\end{tikzpicture}


\\  \hline 
\RDD{left delimiter}=(  & \RDD{right delimiter}=\BS{\}} & \RDD{above delimiter}=\BS{|} & \RDD{below delimiter}=\BS{rmoustache}
\\  \hline
\end{tabular} 

\bigskip
\begin{tabular}{|c|} \hline  
\BS{tikz}
\BS{node} [fill=red!20,text width=2cm,\rouge{left delimiter}=\BS{\{} ] \\
\AC{Ceci est une démonstration d'un texte  sur une largeur de 2cm.};
\\ \hline  
\tikz
\node [fill=red!20,text width=2cm,left delimiter=\{]
{Ceci est une démonstration d'un texte  sur une largeur de 2cm.};
\\ \hline 
\end{tabular} 




\newpage

\SbSSCT{Matrice de n\oe uds}{Chaine de n\oe uds}

\SbSbSSCT{Création d'une chaine de n\oe euds}{Starting and Continuing a Chain}

 \maboite{\BS{usetikzlibrary}\AC{chains}}
\label{lib-chains}


\begin{center}
\RRR{46-2}
\end{center}

\bigskip

\begin{tabular}{|l|} \hline  
\BS{begin}\AC{tikzpicture}[\RDD{start chain}] \\
\BS{node} [\RDD{on chain}] \AC{A};\\
\BS{node}  [\RDD{on chain}] \AC{B};\\
\BS{node}  [\RDD{on chain}] \AC{C};\\
\BS{end}\AC{tikzpicture} \\ \hline  
\begin{tikzpicture}[start chain]
\node [on chain] {A};
\node [on chain] {B};
\node [on chain] {C};
\end{tikzpicture}
\\ \hline 
\end{tabular} 

\bigskip

\begin{tabular}{|c|}  \hline  
\BS{begin}\AC{tikzpicture}[start chain, \RDD{node distance}= 0.5 cm] 
\\ \hline  
\begin{tikzpicture}[start chain, node distance= .5 cm]
\node [on chain] {A};
\node [on chain] {B};
\node [on chain] {C};
\end{tikzpicture}
\\ \hline 
\end{tabular} 

\bigskip

\begin{tabular}{|c|}  \hline 
\BS{begin}\AC{tikzpicture}[start chain=\rouge {going below} ]
\\   \hline 
\begin{tikzpicture}[start chain=going below]
\node [on chain] {A};
\node [on chain] {B};
\node [on chain] {C};
\end{tikzpicture}
\\   \hline 
\end{tabular} 

\bigskip

\begin{tabular}{|c|}  \hline 
\BS{begin}\AC{tikzpicture}[start chain=\rouge {going left} ] 
\\   \hline 
\rule[0cm]{0pt}{.7cm}  
\begin{tikzpicture}[start chain=going left]
\node [on chain] {A};
\node [on chain] {B};
\node [on chain] {C};
\end{tikzpicture} 
\\ \hline 
\end{tabular} 


\bigskip

\begin{tabular}{|c|}  \hline  
\BS{begin}\AC{tikzpicture}[start chain, \rouge{every node/.style=draw} ] 
\\ \hline 
\rule[0cm]{0pt}{.7cm}  
\begin{tikzpicture}[start chain, every node/.style=draw]
\node [on chain] {A};
\node [on chain] {B};
\node [on chain] {C};
\end{tikzpicture}
\\ \hline 
\end{tabular} 

\bigskip

\begin{tabular}{|c|c|}\hline
\begin{tikzpicture}[start chain=1 going right,
start chain=2 going left]
\node [draw,on chain=1] {A};
\node [draw,on chain=1] {B};
\node [draw,on chain=1] {C};
\node [draw,on chain=2] at (3,1) {0};
\node [draw,on chain=2] {1};
\node [draw,on chain=2] {2};
\node [draw,on chain=1] {D};
\end{tikzpicture} 
 &  
\parbox{10cm}{
\BS{begin}\AC{tikzpicture}[\rouge{start chain=1} going right , \\
\blll{start chain=2} going left] \\
\BS{node} [draw,\rouge{on chain=1}] \AC{A}; \\
\BS{node} [draw,\rouge{on chain=1}] \AC{B}; \\
\BS{node}[draw,\rouge{on chain=1}] \AC{C}; \\
\BS{node} [draw,\blll{on chain=2}] at (3,1) \AC{0}; \\
\BS{node} [draw,\blll{on chain=2}] \AC{1}; \\
\BS{node} [draw,\blll{on chain=2}] \AC{2}; \\
\BS{node}[draw,\rouge{on chain=1}] \AC{D}; \\
\BS{end}\AC{tikzpicture}} 
\\ \hline 
\end{tabular} 

\bigskip


\begin{tabular}{|c|c|} \hline  
\rule[-2cm]{0pt}{4cm} 
\begin{tikzpicture}[start chain=going right,baseline=-1.5cm]
\node [draw,on chain] {A};
\node [draw,on chain] {B};
\node [draw,continue chain=going below,on chain] {C};
\node [draw,on chain] {D};
\node [draw,continue chain=going right,on chain] {E};
\end{tikzpicture}
&  
\parbox{11cm}{
\BS{begin}\AC{tikzpicture}[start chain going right]
\BS{node} [draw,on chain] \AC{A}; \\
\BS{node} [draw,on chain] \AC{B}; \\
\BS{node} [draw,\RDD{continue chain}=going below,on chain] \AC{C}; \\
\BS{node}[draw,on chain] \AC{D}; \\
\BS{node} [draw,\RDD{continue chain}=going right,on chain] \AC{E}; \\
\BS{end}\AC{tikzpicture}} 
\\ \hline 
\end{tabular} 

\bigskip

\begin{tabular}{|c|c|}  \hline 
\begin{tikzpicture}[every node/.style=draw,baseline=-1.5cm]
{ [start chain=1]
\node [on chain] {A};
\node [on chain] {B};
\node [on chain] {C};
}
{ [start chain=2 going below]
\node [on chain=2] at (0.5,-.5) {0};
\node [on chain=2] {1};
\node [on chain=2] {2};
}
{ [continue chain=1]
\node [on chain] {D};
}
\end{tikzpicture}
&  
\parbox{10cm}{
\BS{begin}\AC{tikzpicture}[start chain going right] \\
\{ [\RDD{start chain}=1] \\
\BS{node} [draw,on chain] \AC{A}; \\
\BS{node} [draw,on chain] \AC{B}; \\
\BS{node} [draw,on chain] \AC{C}; \\
\} \\
\{ [\RDD{start chain}=2] \\
\BS{node}[draw,on chain=2] \AC{0}; \\
\BS{node}[draw,on chain=2] \AC{1}; \\
\BS{node}[draw,on chain=2] \AC{2}; \\
\} \\
\{ [\RDD{continue chain}=1] \\
\BS{node} [draw,on chain] \AC{D}; \\
\} \\
\BS{end}\AC{tikzpicture}} 
\\  \hline 
\end{tabular} 

\bigskip

\SbSbSSCT{N\oe uds sur la chaine}{Nodes on a Chain}

\begin{center}
\RRR{46-3} 
\end{center}

\bigskip

\begin{tabular}{|c|c|} \hline 
 \begin{tikzpicture}[start chain=XXX placed  {at=(\tikzchaincount*-30+90:1.5)},baseline=0pt]
 \foreach \i in {1,...,12}
 \node [on chain] {\i};
 \draw (0,0) -- (XXX-10);
 \draw (0,0) -- (XXX-2);
 \end{tikzpicture}
&
\parbox{11cm}{
\BS{begin}\AC{tikzpicture}[start chain=\blll{XXX} \RDD{placed} \\ \AC{at=(\BSS{tikzchaincount}*-30+90:1.5)}] \\
 \BS{foreach} \BS{i} in \AC{1,...,12} \\
\BS{node} [on chain] \AC{\BS{i}}; \\
\BS{draw }(0,0) -- \blll{(XXX-10)}; \\
\BS{draw }(0,0) -- \blll{(XXX-2)}; \\
\BS{end}\AC{tikzpicture}} 
\\ \hline 
\end{tabular} 

\bigskip


\begin{tabular}{|c|c|}  \hline 
\begin{tikzpicture}[start chain,baseline=-1cm]
\node [draw,on chain] {A};
\node [draw,on chain] {B};
\node [draw,on chain=going below] {C};
\node [draw,on chain] {D};
\node [draw,on chain] {E};
\end{tikzpicture}
&  
\parbox{11cm}{
\BS{begin}\AC{tikzpicture}[start chain] \\
\BS{node} [draw,on chain] \AC{A}; \\
\BS{node} [draw,on chain] \AC{B}; \\
\BS{node} [draw,on chain=\rouge{going below}] \AC{C}; \\
\BS{node} [draw,on chain] \AC{D}; \\
\BS{node} [draw,on chain] \AC{E}; \\
\BS{end}\AC{tikzpicture}} 
\\  \hline 
\end{tabular} 


\bigskip

\begin{tabular}{|c|c|} \hline 
\begin{tikzpicture}[start chain=going {at=(\tikzchainprevious),shift=(30:1)},baseline=1cm]
\node [draw,on chain] {A};
\node [draw,on chain] {B};
\node [draw,on chain] {C};
\node [draw,on chain] {D};
\end{tikzpicture}  
&  
\parbox{11cm}{
\BS{begin}\AC{tikzpicture}[start chain=going \\ \AC{at=(\BSS{tikzchainprevious},shift=(30:1)}] \\
\BS{node} [draw,on chain] \AC{A}; \\
\BS{node} [draw,on chain] \AC{B}; \\
\BS{node} [draw,on chain] \AC{C}; \\
\BS{node} [draw,on chain] \AC{D}; \\
\BS{end}\AC{tikzpicture}} 
\\ \hline 
\end{tabular} 

\bigskip

\begin{tabular}{|c|c|} \hline 
\begin{tikzpicture}[baseline=1cm]
\node[draw,red] (A) at (0,2) {A};
{ [start chain]
\node [draw,on chain] {B};
\node [draw,on chain] {C};
\chainin (A) [join];
\node [draw,on chain] {D};
\node [draw,on chain] {E};
}
\end{tikzpicture}
&  
\parbox{11cm}{
\BS{begin}\AC{tikzpicture} \\
\BS{node} [draw,red] (A) at (0,2)  \AC{A}; \\
\{ [start chain] \\
\BS{node} [draw,on chain] \AC{B}; \\
\BS{node} [draw,on chain] \AC{C}; \\
\BSS{chainin} (A) [join]; \\
\BS{node} [draw,on chain] \AC{D}; \\
\BS{node} [draw,on chain] \AC{E}; \\
\} \\
\BS{end}\AC{tikzpicture}} 
\\  \hline 
\end{tabular} 



\bigskip

\begin{tabular}{|c|c|} \hline 
\begin{tikzpicture}[baseline=-1cm]
\matrix [matrix of nodes,column sep=1cm,row sep=1cm,every node/.style=draw]
{
|(a) | A 	& |(b) |  B 	& |(c) | C \\
|(d) | D 	& |(e) | E 		& |(f) | F \\
};
{ [start chain,every on chain/.style={join=by ->}]
\chainin (a);
\chainin (b);
\chainin (d);
\chainin (c);
\chainin (f);
\chainin (e);
}
\end{tikzpicture}
&  
\parbox{11cm}{
\BS{begin}\AC{tikzpicture} \\
\BS{matrix} [matrix of nodes,column sep=5mm,row sep=5mm] ,every node/.style=draw \\
\{ \\
|(a) | A 	\& |(b) |  B 	\& |(c) | C \BS{}\BS{} \\
|(d) | D 	\& |(e) | E 	\& |(f) | F \BS{}\BS{} \\
\}; \\
\{ [start chain,every on chain/.style=\AC{join=by ->}] \\
\BSS{chainin} (a);
\BSS{chainin}(b);
\BSS{chainin}(d); \\
\BSS{chainin} (c);
\BSS{chainin}(f);
\BSS{chainin}(e);
\}
\BS{end}\AC{tikzpicture}
} 
\\ \hline 
\end{tabular} 

\bigskip

\SbSbSSCT{Jonction de n\oe uds}{Joining Nodes on a Chain}

\begin{center}
\RRR{46-4}
\end{center} 

\bigskip

\begin{tabular}{|c|c|} \hline 
\begin{tikzpicture}[start chain]
\node [draw,on chain] {A};
\node [draw,on chain,join] {B};
\node [draw,on chain] {C};
\node [draw,on chain,join] {D};
\end{tikzpicture}
&  
\parbox{11cm}{
\BS{begin}\AC{tikzpicture}[start chain] \\
\BS{node} [draw,on chain] \AC{A}; \\
\BS{node} [draw,on chain,\RDD{join}] \AC{B}; \\
\BS{node} [draw,on chain] \AC{C}; \\
\BS{node} [draw,on chain,\RDD{join}] \AC{D}; \\
\BS{end}\AC{tikzpicture}} 
\\ \hline 
\end{tabular} 

\bigskip

\begin{tabular}{|c|c|} \hline 
\begin{tikzpicture}[start chain, every on chain/.style=join, every join/.style=->]
\node [draw,on chain] {A};
\node [draw,on chain] {B};
\node [draw,on chain] {C};
\node [draw,on chain] {D};
\end{tikzpicture}
&  
\parbox{11cm}{
\BS{begin}\AC{tikzpicture}[start chain, \RDD{every on chain}/.style=join, \\ \RDD{every join}/.style=->] \\
\BS{node} [draw,on chain] \AC{A}; \\
\BS{node} [draw,on chain,\RDD{join}] \AC{B}; \\
\BS{node} [draw,on chain] \AC{C}; \\
\BS{node} [draw,on chain,\RDD{join}] \AC{D}; \\
\BS{end}\AC{tikzpicture}} 
\\ \hline 
\end{tabular} 

\bigskip

\begin{tabular}{|c|c|}  \hline 
\begin{tikzpicture}[start chain,baseline=-1cm]
\node [draw,on chain] {A};
\node [draw,on chain] {B};
\node [draw,on chain] {C};
\node [draw,on chain=going below,join=with chain-2 ] {D};
\end{tikzpicture} 
&  
\parbox{11cm}{
\BS{begin}\AC{tikzpicture}[start chain] \\
\BS{node} [draw,on chain] \AC{A}; \\
\BS{node} [draw,on chain] \AC{B}; \\
\BS{node} [draw,on chain] \AC{C}; \\
\BS{node} [draw,on chain=going below,\rouge{join=with chain-2} ] \AC{D}; \\
\BS{end}\AC{tikzpicture}} 
\\ \hline 
\begin{tikzpicture}[start chain,baseline=-1cm]
\node [draw,on chain] {A};
\node [draw,on chain] {B};
\node [draw,on chain] {C};
\node [draw,on chain=going below,join=with chain-1 by {blue,<-}] {D};
\end{tikzpicture}
&
\parbox{12cm}{
\BS{begin}\AC{tikzpicture}[start chain] \\
\BS{node} [draw,on chain] \AC{A}; \\
\BS{node} [draw,on chain] \AC{B}; \\
\BS{node} [draw,on chain] \AC{C}; \\
\BS{node} [draw,on chain=going below,join=with chain-1 \rouge{ by \AC{blue,<-}} ] \AC{D}; \\
\BS{end}\AC{tikzpicture}} 
\\ \hline 
\end{tabular} 



\bigskip

\SbSbSSCT{Branches}{Branches}

\begin{center}
\RRR{46-5}
\end{center} 


\bigskip

\begin{tabular}{|c|c|}  \hline 
\begin{tikzpicture} [baseline=-2cm]
{ [start chain=XXX]
\node [draw,on chain] {A};
\node [draw,on chain] {B};
{ [start branch=YYY going below]
\node [draw,on chain] {1};
\node [draw,on chain] {2};
\node [draw,on chain] {3};
}
\node [draw,on chain,join=with XXX/YYY-end,join=with XXX/YYY-2 ] {C};
}
\end{tikzpicture}
&  
\parbox{12cm}{
\BS{begin}\AC{tikzpicture}\\
\{ [start chain=\blll{XXX}] \\
\BS{node} [draw,on chain] \AC{A}; \\
\BS{node} [draw,on chain] \AC{B}; \\
\{ [\RDD{start branch}=\blll{YYY} going below] \\
\BS{node} [draw,on chain] \AC{1}; \\
\BS{node} [draw,on chain] \AC{2}; \\
\BS{node} [draw,on chain] \AC{3}; \\
\} \\
\BS{node} [ draw,on chain,join=with \blll{XXX/YYY}\rouge{-end}, \\ join=with \blll{XXX/YYY}\rouge{-2}]  \AC{C}; \\
\} \\
\BS{end}\AC{tikzpicture}   } 

\\ \hline 
\end{tabular} 

\bigskip

\begin{tabular}{|c|} \hline 
\BS{begin}\AC{tikzpicture}[ \RDD{node distance}=.2cm and 3cm]
\\ \hline 
\begin{tikzpicture}[ node distance=.2cm and 3cm]
{ [start chain=XXX]
\node [on chain] {A};
\node [on chain] {B};
{ [start branch=YYY going below]
\node [on chain] {1};
\node [on chain] {2};
\node [on chain] {3};
}
\node [on chain,join=with XXX/YYY-end] {C};
}
\end{tikzpicture}
\\ \hline 
\end{tabular} 

\bigskip

\begin{tabular}{|c|c|} \hline 
\begin{tikzpicture}[ node distance=2mm and 1cm,baseline=-2cm]
{ [start chain=XXX]
\node [draw,on chain] {A};
\node [draw,on chain] {B};
{ [start branch=YYY going below]
\node [draw,on chain] {1};
\node [draw,on chain] {2};
\node [draw,on chain] {3};
}
\node [draw,on chain,join=with XXX/YYY-end] {C};
{
[continue branch=YYY]
\node [draw,on chain] {4};
\node [draw,on chain] {5};
}
}
\end{tikzpicture}
&  
\parbox{12cm}{
\BS{begin}\AC{tikzpicture}[ node distance=2mm and 1cm]\\
\{ [start chain=\blll{XXX}] \\
\BS{node} [draw,on chain] \AC{A}; \\
\BS{node} [draw,on chain] \AC{B}; \\
\{ [start branch=\blll{YYY} going below] \\
\BS{node} [draw,on chain] \AC{1}; \\
\BS{node} [draw,on chain] \AC{2}; \\
\BS{node} [draw,on chain] \AC{3}; \} \\
\BS{node}  [draw,on chain,join=with \blll{XXX/YYY}-end]  \AC{C}; \\
\{ [\RDD{continue branch}=\blll{YYY}]\\
\BS{node} [on chain] \AC{4}; \\
\BS{node} [on chain] \AC{5}; \} \\
\} \\
\BS{end}\AC{tikzpicture}   } 
\\ \hline 
\end{tabular} 


\bigskip

\begin{tabular}{|c|c|} \hline 
\begin{tikzpicture}[node distance=2mm and 1cm, every node/.style=draw,baseline=-1cm]
{ [start chain]
\node [on chain] {1};
\node [on chain] {2};
{ [start branch=XXX going below] }
\node [on chain] {3};
{ [start branch=YYY going above] }
\node [on chain] {4};
{ [continue branch=XXX]
\node [on chain] {a};
\node [on chain] {b};
}{
[continue branch=YYY]
\node [on chain] {A};
\node [on chain] {B};
}
}
\end{tikzpicture}
&  
\parbox{12cm}{
\BS{begin}\AC{tikzpicture}[node distance=2mm and 1cm, every node/.style=draw]\\
\{ [start chain] \\
\BS{node} [on chain] \AC{1};  \\
\BS{node} [on chain] \AC{2}; \\
\{ [\RDD{start branch}=\blll{XXX} going below] \} \\
\BS{node} [on chain] \AC{3}; \\
\{ [\RDD{start branch}=\blll{YYY} going above] \} \\
\BS{node} [on chain] \AC{4}; \\
\{ [\RDD{continue branch}=\blll{XXX} ] \\
\BS{node} [on chain] \AC{a}; \\
\BS{node} [on chain] \AC{b};\} \\
\{ [\RDD{continue branch}=\blll{YYY} ] \\
\BS{node} [on chain] \AC{A}; \\
\BS{node} [on chain] \AC{B}; \}  }
\\ \hline 
\end{tabular} 




\newpage

\SSCT{Constructions particulières  }{Transformations}



\begin{center}
\RRR{25-3}
\end{center}


\begin{tabular}{|c|c|c|c|} \hline 
\multicolumn{4}{|c|}{  \BS{draw}[\RDD{rotate},blue] (0,0)  rectangle  (2,2) ;   }\\ 
\hline  
\begin{tikzpicture}
\draw[dashed,red] (0,0) rectangle  (2,2) ; 
\draw[rotate=40,blue] (0,0) rectangle  (2,2) ;
\end{tikzpicture}
&  
\begin{tikzpicture}
\draw[dashed,red] (0,0) rectangle  (2,2) ; 
\draw[x=1cm,y=.5cm,blue] (0,0) rectangle  (2,2); 
\end{tikzpicture}
&  
\begin{tikzpicture}
\draw[dashed,red] (0,0) rectangle  (2,2) ; 
\draw[xslant=.75,blue] (0,0) rectangle  (2,2);  
\end{tikzpicture}
&
\begin{tikzpicture}
\draw[dashed,red] (0,0) rectangle  (2,2) ; 
\draw[yslant=.75,blue] (0,0) rectangle  (2,2);  
\end{tikzpicture}
\\ \hline  
\RDD{rotate}=40 & \RDD{x}=1cm,\RDD{y}=0.5cm & \RDD{xslant}=0.75 & \RDD{yslant}=0.75\\ 
\hline 
  
\begin{tikzpicture}
\draw[dashed,red] (0,0) rectangle  (2,2) ; 
\draw[scale=1.5,blue] (0,0) rectangle  (2,2) ; 
\end{tikzpicture}
&  
\begin{tikzpicture}
\draw[dashed,red] (0,0) rectangle  (2,2) ; 
\draw[scale=-1,y=.5cm,blue] (0,0) rectangle  (2,2); 
\end{tikzpicture}
&  
\begin{tikzpicture}
\draw[dashed,red] (0,0) rectangle  (2,2) ; 
\draw[xshift=.5cm,blue] (0,0) rectangle  (2,2); 
\end{tikzpicture}
&
\begin{tikzpicture}
\draw[dashed,red] (0,0) rectangle  (2,2) ; 
\draw[yshift=.5cm,blue] (0,0) rectangle  (2,2);  
\end{tikzpicture}
\\ \hline  
\RDD{scale}=1.5 & \RDD{scale}=-1 & \RDD{xshift}=0.5cm & \RDD{yshift}=0.5cm
\\ \hline 
\end{tabular} 

\bigskip
 

\newpage


\SSCT{Placer son dessin}{Placing the picture}

\SbSSCT{Dans le texte}{In the text}

\SbSbSSCT{Sans option de décalage}{Without offset}

\begin{center}
\RRR{12-2}
\end{center}

\TFRGB{dessin directement  dans le texte}{picture in the text}  \tikz \draw[blue] (0,0) rectangle(2,2); \tikz \draw[blue] (0,0) circle (1);\TFRGB{ici est inclus le code suivant}{here is the following code} : \BS{tikz} \BS{draw} (0,0) rectangle(2,2);\BS{tikz} \BS{draw} (0,0) circle (1);
\bigskip

\SbSbSSCT{Avec décalage nul}{With zero offset}

\TFRGB{dessin directement  dans le texte}{picture in the text}  \tikz[baseline=0pt] \draw[blue] (0,0) rectangle(2,2); \tikz[baseline=0pt] \draw[blue] (0,0) circle (1);\TFRGB{ici est inclus le code suivant}{here is the following code} : \BS{tikz}[\RDD{baseline}=0pt] \BS{draw} (0,0) rectangle(2,2);\BS{tikz}[\RDD{baseline}=0pt] \BS{draw} (0,0) circle (1);
\bigskip


\bigskip

\SbSbSSCT{Avec décalage}{With an offset}

\TFRGB{dessin directement  dans le texte}{picture in the text}  \tikz[baseline=1cm] \draw[blue] (0,0) rectangle(2,2); \tikz[baseline=1cm] \draw[blue] (0,0) circle (1);\TFRGB{ici est inclus le code suivant}{here is the following code} : \BS{tikz}[\RDD{baseline}=1cm] \BS{draw} (0,0) rectangle(2,2);\BS{tikz}[\RDD{baseline}=1cm] \BS{draw} (0,0) circle (1);


\newpage

\SbSSCT{Dans un environnement tikzpicture}{In a tikzpicture environment}

\begin{center}
\RRR{12-1}
\end{center}

\begin{tabular}{|l|l|} \hline  
 
\TFRGB{texte avant}{text before} 
\begin{tikzpicture}[blue]
\draw (0,0) rectangle(2,2); 
 \draw (0,0) circle (1);
\end{tikzpicture}
\TFRGB{texte après}{text after} 
&  
\parbox[b]{8cm}{
\TFRGB{texte avant}{text before} \\
\ESS{tikzpicture}[blue] \\
\BS{draw} (0,0) rectangle(2,2);  \\
\BS{draw} (0,0) circle (1); \\
\BS{end\AC{tikzpicture}}\\
\TFRGB{texte après}{text after} \\
}
\\ \hline 
\end{tabular} 

\bigskip


\begin{tabular}{|l|l|} \hline 
\TFRGB{texte avant}{text before} 
\begin{tikzpicture}[blue,baseline=0pt]
\draw (0,0) rectangle(2,2); 
 \draw (0,0) circle (1);
\end{tikzpicture}
\TFRGB{texte après}{text after} 
&  
\parbox[c]{8cm}{
\TFRGB{texte avant}{text before} \\
\BS{begin}\AC{tikzpicture}[blue,\RDD{baseline}=0pt] \\
\BS{draw} (0,0) rectangle(2,2);  \\
\BS{draw} (0,0) circle (1); \\
\BS{end}\AC{tikzpicture} \\
\TFRGB{texte après}{text after} \\
}
\\ \hline 
\end{tabular} 

\bigskip
\noindent

\begin{tabular}{|l|l|} \hline 
\TFRGB{texte avant}{text before} 
\begin{tikzpicture}[blue,baseline=1cm]
\draw (0,0) rectangle(2,2); 
 \draw (0,0) circle (1);
\end{tikzpicture}
\TFRGB{texte après}{text after} 
&  
\parbox[t]{8cm}{
\TFRGB{texte avant}{text before} \\
\BS{begin}\AC{tikzpicture}[blue,\RDD{baseline}=1cm] \\
\BS{draw} (0,0) rectangle(2,2);  \\
\BS{draw} (0,0) circle (1); \\
\BS{end}\AC{tikzpicture} \\
\TFRGB{texte après}{text after} \\
}
\\ \hline 
\end{tabular} 

\SbSSCT{Dans un environnement fbox}{In a fbox environment}

\noindent

\begin{tabular}{|l|l|} \hline 
\TFRGB{texte avant}{text before}
\fbox{ 
\begin{tikzpicture}[blue,baseline=0pt]

\draw (0,0) rectangle(2,2); 
 \draw (0,0) circle (1);
\end{tikzpicture}}
\TFRGB{texte après}{text after} 
&  
\parbox[c]{8cm}{
\TFRGB{texte avant}{text before}\\
\BSS{fbox}\{ \\ 
\BS{begin}\AC{tikzpicture}[blue,\RDD{baseline}=0pt] \\
\BS{draw} (0,0) rectangle(2,2);  \\
\BS{draw} (0,0) circle (1); \\
\BS{end}\AC{tikzpicture} \\
 \}\\
\TFRGB{texte après}{text after} \\
}
\\ \hline 
\end{tabular}

\SbSSCT{Modification du cadrage}{Bounding box}

\begin{center}
\RRR{15-8}
\end{center}

\bigskip

\begin{tabular}{|c|c|}  \hline  
\multicolumn{2}{|l|}{\BS{draw} [\RDD{use as bounding box}] (1,0) rectangle (2,1);}\\ 
\multicolumn{2}{|l|}{\BS{draw}[blue] (-1,0) - - (3,1);} \\\hline 
texte avant\begin{tikzpicture}
\draw [use as bounding box] (1,0) rectangle (2,1);
\draw[blue] (-1,0) - - (3,1);
\draw [red](1,0) rectangle (2,1);
\end{tikzpicture}texte après
&
texte avant 
\begin{tikzpicture}
\draw [use as bounding box] (0,0) rectangle (0,0);
\draw[blue] (-1,0) - - (3,1);
\draw [red](0,0) rectangle (0,0);
\end{tikzpicture}
texte après  
\\ \hline 
(1,0) rectangle (2,1) & (0,0) rectangle (0,0)
\\ \hline 
\end{tabular} 

\bigskip

\begin{tabular}{|c|c|} \hline
\multicolumn{2}{|l|}{texte avant. \BS{begin\AC{tikzpicture}} [\RDD{trim left}=1cm] }\\ 
\multicolumn{2}{|l|}{\BS{draw}[blue] (-1,0) - - (3,1); \BS{draw}[red] (0,0) grid (2,1);} \\
\multicolumn{2}{|l|}{ \BS{end\AC{tikzpicture}}texte après }\\ 
\hline 
  
texte avant.%
\begin{tikzpicture}[trim left=1cm]
\draw[blue] (-1,0) - - (3,1);
\draw [red](0,0) grid (2,1);
\end{tikzpicture}%
texte après
&  
texte avant.%
\begin{tikzpicture}[trim right= 1cm]
\draw[blue] (-1,0) - - (3,1);
\draw [red](0,0) grid(2,1);
\end{tikzpicture}%
texte après
\\ \hline  
[\RDD{trim left}=1cm]
&  
[\RDD{trim right}= 1cm]
\\ \hline 
\end{tabular} 

\bigskip

\begin{tabular}{|l|l|} \hline  
\TFRGB{texte avant}{text before} 
\begin{tikzpicture}[blue]
\draw [red,use as bounding box] (-1.5,-1.5) rectangle (2.5,2.5);
\draw (0,0) rectangle(2,2); 
 \draw (0,0) circle (1);
\end{tikzpicture}
\TFRGB{texte après}{text after} 
&  
\parbox[b]{10cm}{
\TFRGB{texte avant}{text before} \\
\ESS{tikzpicture}[blue] \\
\BS{draw} [red,\RDD{use as bounding box}] (-1.5,-1.5) rectangle (2.5,2.5); \\
\BS{draw} (0,0) rectangle(2,2);  \\
\BS{draw} (0,0) circle (1); \\
\BS{end\AC{tikzpicture}}\\
\TFRGB{texte après}{text after} \\
}
\\ \hline 
\end{tabular}

\bigskip

\begin{tabular}{|l|l|} \hline 
\TFRGB{texte avant}{text before} 
\begin{tikzpicture}[blue,baseline=0pt]
\draw [red,use as bounding box] (-1.5,-1.5) rectangle (2.5,2.5);
\draw (0,0) rectangle(2,2); 
 \draw (0,0) circle (1);
\end{tikzpicture}
\TFRGB{texte après}{text after} 
&  
\parbox[c]{10cm}{
\TFRGB{texte avant}{text before} \\
\BS{begin}\AC{tikzpicture}[blue,\RDD{baseline}=0pt] \\
\BS{draw} [red,\RDD{use as bounding box}] (-1.5,-1.5) rectangle (2.5,2.5);\\
\BS{draw} (0,0) rectangle(2,2);  \\
\BS{draw} (0,0) circle (1); \\
\BS{end}\AC{tikzpicture} \\
\TFRGB{texte après}{text after} \\
}
\\ \hline 
\end{tabular}

\bigskip

\begin{tabular}{|l|l|} \hline 
\TFRGB{texte avant}{text before} 
\begin{tikzpicture}[blue,baseline=0pt]
\useasboundingbox  (-1.5,-1.5) rectangle (2.5,2.5);
\draw (0,0) rectangle(2,2); 
 \draw (0,0) circle (1);
\end{tikzpicture}
\TFRGB{texte après}{text after} 
&  
\parbox[c]{10cm}{
\TFRGB{texte avant}{text before} \\
\BS{begin}\AC{tikzpicture}[blue,baseline=0pt] \\
\BSS{useasboundingbox}  (-1.5,-1.5) rectangle (2.5,2.5);\\
\BS{draw} (0,0) rectangle(2,2);  \\
\BS{draw} (0,0) circle (1); \\
\BS{end}\AC{tikzpicture} \\
\TFRGB{texte après}{text after} \\
}
\\ \hline 
\end{tabular}

\bigskip

\begin{tabular}{|c|c|} \hline  
\begin{tikzpicture}[blue,baseline=0pt]
\fill (0,0) circle (5pt);
\fill (2,1) circle (5pt);
\draw[red] (current bounding box.south west) rectangle
(current bounding box.north east);
\end{tikzpicture}
&  
\parbox[c]{14cm}{
\BS{begin}\AC{tikzpicture}[blue] \\
\BS{fill} (0,0) circle (5pt); \\
\BS{fill}  (2,1) circle (5pt);\\
\BS{draw}[red] (\RDD{current bounding box.south west}) rectangle
(\RDD{current bounding box.north east});\\
\BS{end}\AC{tikzpicture}
}
\\ \hline 
\end{tabular}
\newpage
\SbSSCT{Coupure de l'image}{Clipping the picture}
	
\begin{center}
\RRR{15-9}
\end{center}

\begin{tabular}{|c|c|} \hline  
\tikzpicture[red]
\draw[help lines] (-2,-2) grid (2,2);
\draw[blue] (-1,-1) --(0,2) -- (1,-1) -- cycle;
\draw (0,0) circle (.5);
\draw (0,0) circle (1);
\draw (0,0) circle (1.5);
\endtikzpicture
&  
\tikzpicture[red]
\clip (-1,-1) --(0,2) -- (1,-1) -- cycle;
\draw[help lines] (-2,-2) grid (2,2);
\draw[blue] (-1,-1) --(0,2) -- (1,-1) -- cycle;
\draw (0,0) circle (.5);
\draw (0,0) circle (1);
\draw (0,0) circle (1.5);
\endtikzpicture
\\ \hline  
\TFRGB{sans coupure}{no clipping}
&  
\BSS{clip} (-1,-1) - -(0,2) - - (1,-1) - - cycle;
\\ \hline 
\end{tabular} 



\SbSSCT{Rognage partiel}{ Partial clipping}

\noindent

\begin{tabular}{|c|c|} \hline   
\tikzpicture[red,scale=.7,baseline=0pt]
\draw[help lines] (-2,-2) grid (2,2);
\draw[blue] (-1.1,-0.2) rectangle (2,1.5);
\draw (0,0) circle (1.5);
\clip (-1.1,-0.2) rectangle (2,1.5);
\draw (0,0) circle (.5);
\draw (0,0) circle (1);
\endtikzpicture
& 
\parbox{8cm}{ 
\BS{tikzpicture}[red,scale=.7] \\
\BS{draw}[help lines] (-2,-2) grid (2,2); \\
\BS{draw}[blue] (-1.1,-0.2) rectangle (2,1.5); \\
\BS{draw} (0,0) circle (1.5); \\
\BSS{clip} (-1.1,-0.2) rectangle (2,1.5); \\
\BS{draw} (0,0) circle (.5); \\
\BS{draw} (0,0) circle (1); \\
\BS{endtikzpicture}} 
\\ \hline 
\end{tabular}

\SbSbSSCT{Changement d'échelle}{Scaling}

\noindent

\begin{tabular}{|c|c|} \hline  
\tikzpicture[blue]
\draw[help lines] (-2,-2) grid (2,2);
\draw (0,0) circle (.5);
\draw (0,0) circle (1);
\draw (0,0) circle (1.5);
\endtikzpicture
&  
\tikzpicture[blue,scale=.5]
\draw[help lines] (-2,-2) grid (2,2);
\draw (0,0) circle (.5);
\draw (0,0) circle (1);
\draw (0,0) circle (1.5);
\endtikzpicture
\\ \hline  
\TFRGB{Taille normale}{Normal size}
&  
\BS{tikzpicture}[blue,\RDD{scale}=.5]
\\ \hline 
\end{tabular} 



\newpage

\section{Scope}


\SbSSCT{Environnement Scope}{Environment Scope}

\begin{center}
\RRR{12-3}
\end{center}

\begin{tabular}{|c|c|} \hline
\parbox[b]{8cm}{
\BS{begin}\AC{tikzpicture}[line width = 3mm] \\ \\
\BS{draw}  (0.5,6) - - (2.5,6);\\

\ESS{scope}[{\color{red}red}] \\
\BS{draw} (0.5,5) - - (2.5,5); \\
\BS{draw}  (0.5,4) - - (2.5,4);\\
\BS{end\AC{scope}} \\ \\
\BS{draw}  (0.5,3) - - (2.5,3);\\ \\
\ESS{scope}[{\color{red}green}] \\
\BS{draw}  (0.5,2) - - (2.5,2);\\
\BS{draw} [{\color{red}red}] (0.5,1) - - (2.5,1);\\
\BS{draw}  (0.5,0) - - (2.5,0);\\
\BS{end\AC{scope}} \\ \\
\BS{end}\AC{tikzpicture}
}
&  
\begin{tikzpicture}[line width = 3mm,baseline=-.5cm]
\draw[help lines] (0,0) grid (3,6);
\draw (0.5,6) -- (2.5,6);
\begin{scope}[red]
\draw (0.5,5) -- (2.5,5);
\draw (0.5,4) -- (2.5,4);
\end{scope}
\draw (0.5,3) -- (2.5,3);
\begin{scope}[green]
\draw (0.5,2) -- (2.5,2);
\draw[red]  (0.5,1) -- (2.5,1);
\draw (0.5,0) -- (2.5,0);
\end{scope}
\end{tikzpicture}
\\ \hline 
\end{tabular} 

\subsection{library scopes} 

\SbSbSSCT{Simplification d'un environnement scope}{Shorthand for Scope Environments}

\begin{center}
\RRR{12-3-2}
\end{center}

 \maboite{\BS{usetikzlibrary}\AC{scopes}}
\label{lib-scopes}

\begin{tabular}{|c|c|} \hline
\parbox[b]{8cm}{
\BS{begin}\AC{tikzpicture}[line width = 3mm] \\ \\
\BS{draw}  (0.5,6) - - (2.5,6);\\ \\
{\color{red} \{} [red] \\
\BS{draw} (0.5,5) - - (2.5,5); \\
\BS{draw}  (0.5,4) - - (2.5,4);\\
{\color{red} \} } \\ \\
\BS{draw}  (0.5,3) - - (2.5,3);\\ \\
{\color{red} \{ }[green] \\
\BS{draw}  (0.5,2) - - (2.5,2);\\
\BS{draw} [{\color{red}red}] (0.5,1) - - (2.5,1);\\
\BS{draw}  (0.5,0) - - (2.5,0);\\
{\color{red} \} }\\ \\
\BS{end}\AC{tikzpicture}
}
&  
\begin{tikzpicture}[line width = 3mm,baseline=-.5cm]
\draw[help lines] (0,0) grid (3,6);
\draw (0.5,6) -- (2.5,6);
{[red]
\draw (0.5,5) -- (2.5,5);
\draw (0.5,4) -- (2.5,4);
}
\draw (0.5,3) -- (2.5,3);
{[green]
\draw (0.5,2) -- (2.5,2);
\draw [red] (0.5,1) -- (2.5,1);
\draw (0.5,0) -- (2.5,0);
}
\end{tikzpicture}
\\ \hline 
\end{tabular} 

\SbSbSSCT{Portée d'un seul élément} {Single Command Scopes}

\begin{tabular}{|c|c|} \hline
\begin{tikzpicture}
\node [fill=white] at (1,1) {\DFR};
\scoped [on background layer]
\draw (0,0) grid (3,2);
\end{tikzpicture}
&
\begin{tikzpicture}
\node [fill=white] at (1,1) {\DFR};
\draw (0,0) grid (3,2);
\end{tikzpicture}
\\ \hline 
\BS{node} [fill=white] at (1,1) \AC{\BS{DFR}};  &\BS{node} [fill=white] at (1,1) \AC{\BS{DFR}}; \\
\BSS{scoped} [on background layer] & \\
\BS{draw} (0,0) grid (3,2); &  \BS{draw} (0,0) grid (3,2);
\\ \hline  
\end{tabular} 

\newpage

\SSCT{Position absolue sur une page}{Absolute position on a page}


\begin{tikzpicture}[remember picture,overlay,red]
\fill(current page.north) circle (5pt) node[below left=4mm] {\Huge north} ;
\fill(current page.north east) circle (5pt) node[below left=4mm] {\Huge north east} ;
\fill(current page.north west) circle (5pt) node[below right=4mm] {\Huge north west} ;
\fill(current page.east) circle (5pt) node[above left=4mm] {\Huge east} ;
\fill(current page.center) circle (5pt) node[above left=4mm] {\Huge center} ;
\fill(current page.west) circle (5pt) node[above right=4mm] {\Huge west} ;
\fill(current page.south) circle (5pt) node[above right=4mm] {\Huge south} ;
\fill(current page.south west) circle (5pt) node[above right=4mm] {\Huge south west} ;
\fill(current page.south east) circle (5pt) node[above left=4mm] {\Huge south east} ;
\end{tikzpicture}

\begin{tabular}{|l|} \hline
\BS{begin}\AC{tikzpicture}[remember picture,overlay]\\
\BS{fill}(\RDD{current page.north}) circle (5pt) node[below left=4mm] {\BS{Huge} north} ; \\
\BS{fill}(\RDD{current page.north east}) circle (5pt) node[below left=4mm] {\BS{Huge} north east} ;\\
\BS{fill}(\RDD{current page.north west}) circle (5pt) node[below right=4mm] {\BS{Huge} north west} ;\\
\BS{fill}(\RDD{current page.east}) circle (5pt) node[above left=4mm] {\BS{Huge} east} ;\\
\BS{fill}(\RDD{current page.center}) circle (5pt) node[above left=4mm] {\BS{Huge}center} ;\\
\BS{fill}(\RDD{current page.west}) circle (5pt) node[above right=4mm] {\BS{Huge} west} ;\\
\BS{fill}(\RDD{current page.south}) circle (5pt) node[above right=4mm] {\BS{Huge} south} ;\\
\BS{fill}(\RDD{current page.south west}) circle (5pt) node[above right=4mm] {\BS{Huge} south west} ;\\
\BS{fill}(\RDD{current page.south east}) circle (5pt) node[above left=4mm] {\BS{Huge} south east} ;\\
\BS{end}\AC{tikzpicture} \\
\hline 
\end{tabular} 

 \begin{tikzpicture}[remember picture,overlay,red]
 \node  [opacity=.15] at (current page.center) {\includegraphics[width=8cm]{tiger} };
 \end{tikzpicture} 
\bigskip

\begin{tabular}{|l|} \hline
\BS{begin}\AC{tikzpicture}[remember picture,overlay]\\
 \BS{node}  [opacity=.15] at (\RDD{current page.center}) \AC{\BS{includegraphics}[width=8cm]\AC{tiger} };\\
\BS{end}\AC{tikzpicture} \\
\hline 
\end{tabular}


 \begin{tikzpicture}[remember picture,overlay,red]
\draw[dotted,opacity=.4] (current page.south west) -- (current page.north east)  node[near start] {\Huge TIKZ} ;
 \end{tikzpicture} 

\bigskip
\begin{tabular}{|l|} \hline
\BS{begin}\AC{tikzpicture}[remember picture,overlay]\\
\BS{draw}[dotted,opacity=.4] (\RDD{current page.south west}) - - (\RDD{current page.north east}) \\
\hspace{1cm}  node[near start] \AC{\BS{Huge} TIKZ} ;\\
\BS{end}\AC{tikzpicture} \\
\hline 
\end{tabular}

\newpage
 
\newpage 

\SSCT{Arrière plan du dessin}{Background}

  \tikzset{background rectangle/.style={fill=cyan!20,draw=blue,line width=2pt}}

\SbSSCT{Encadrement}{Framing}
\label{lib-bkgd}
 
 \begin{tabular}{|c|l|} \hline  
 \begin{tikzpicture}[baseline=0pt,show background rectangle]
 \filldraw[fill=yellow] (0,0) ellipse (1 and .5);
 \end{tikzpicture}
&
\parbox{10cm}{
 \footnotemark \\
\BS{begin}\AC{tikzpicture}[\RDD{show background rectangle}]\\
\BS{filldraw}[fill=yellow] (0,0) ellipse (1 and .5 );\\
\BS{end}\AC{tikzpicture}\\
\\
\emph{\TFRGB{Autre syntaxe}{Other syntax}} : \\
\BS{begin}\AC{tikzpicture}[\RDD{framed}]\\
}
\\ \hline 
\end{tabular}

 \footnotetext[1]{ \BS{tikzset}\AC{background rectangle/.style=\AC{fill=cyan!20,draw=blue,line width=2pt}}}
  
\bigskip

\subsubsection{Options}

\begin{tabular}{|c|c|c|} \hline  
\multicolumn{3}{|c|}{  [show background rectangle,\RDD{inner frame xsep}=1cm]  }
\\ \hline
\begin{tikzpicture}[show background rectangle,inner frame xsep=1cm]
 \filldraw[fill=yellow](0,0) ellipse (1cm and .5cm);
\end{tikzpicture}
&  
\begin{tikzpicture}[show background rectangle,inner frame ysep=1cm]
 \filldraw[fill=yellow](0,0) ellipse (1cm and .5cm);
\end{tikzpicture}
&  
\begin{tikzpicture}[show background rectangle,inner frame sep=1cm]
 \filldraw[fill=yellow] (0,0) ellipse (1cm and .5cm);
\end{tikzpicture}
\\ \hline  
\RDD{inner frame xsep}=1cm & \RDD{inner frame ysep}=1cm & \RDD{inner frame sep}=1cm
\\ \hline 
\multicolumn{3}{|c|}{\dft : inner frame xsep=1ex
,  inner frame ysep=1ex
}  

\\ \hline 
\begin{tikzpicture}[show background rectangle,tight background]
 \filldraw[fill=yellow] (0,0) ellipse (1cm and .5cm);
\end{tikzpicture}
&
\begin{tikzpicture}[show background rectangle,loose background]
 \filldraw[fill=yellow](0,0) ellipse (1cm and .5cm);
\end{tikzpicture}
&
\begin{tikzpicture}[show background rectangle,rounded corners]
 \filldraw[fill=yellow] (0,0) ellipse (1cm and .5cm);
\end{tikzpicture}
\\ \hline 
\RDD{tight background} & \RDD{loose background} & \RDD{rounded corners} \\ \hline 
(inner frame sep = 0pt) & (inner frame sep = 2ex) & 
 \\ \hline
 \end{tabular}
   
  
\subsubsection{Style}
 
 \begin{tabular}{|c|c|c|c|c|} \hline  
 \multicolumn{5}{|c|}{  [\RDD{background rectangle/.style}=\AC{double,draw=blue},framed]  }
 \\ \hline 
 \begin{tikzpicture}[framed,background rectangle/.style={double,draw=blue}]
 \filldraw[fill=yellow] (0,0) ellipse (1cm and .5cm);
 \end{tikzpicture}
 &
 \begin{tikzpicture}[show background rectangle,background rectangle/.style={fill=green,draw=blue}]
 \filldraw[fill=yellow] (0,0) ellipse (1cm and .5cm);
 \end{tikzpicture}
  &
  \begin{tikzpicture}[show background rectangle,background rectangle/.style={top color=green,draw=blue}]
 \filldraw[fill=yellow](0,0) ellipse (1cm and .5cm);
  \end{tikzpicture}
 &
 \begin{tikzpicture}[show background rectangle,background rectangle/.style={line width=4pt,draw=blue}]
 \filldraw[fill=yellow](0,0) ellipse (1cm and .5cm);
 \end{tikzpicture}
  &
  \begin{tikzpicture}[show background rectangle,background rectangle/.style={rounded corners=.5cm,draw=blue}]
 \filldraw[fill=yellow](0,0) ellipse (1cm and .5cm);
  \end{tikzpicture}
 \\ \hline 
 \RDD{double} & \RDD{fill}=green & \RDD{top color}=green & \RDD{line width}=4pt & \RDD{rounded corners}=0.5cm
  \\ \hline 

\end{tabular} 

\SbSSCT{Encadrement partiel}{Partial framing}

 \tikzset{background top/.style={fill=green!20,draw=blue,line width=2pt}}
  \tikzset{background bottom/.style={fill=green!20,draw=blue,line width=2pt}}
   \tikzset{background left/.style={fill=green!20,draw=blue,line width=2pt}}
    \tikzset{background right/.style={fill=green!20,draw=blue,line width=2pt}}
 
 \begin{tabular}{|c|c|c|c|c|} \hline  
\begin{tikzpicture}[show background top]
 \filldraw[fill=yellow] (0,0) ellipse (10mm and 5mm);
\end{tikzpicture}
&
\begin{tikzpicture}[show background bottom]
 \filldraw[fill=yellow](0,0) ellipse (10mm and 5mm);
\end{tikzpicture}
&
\begin{tikzpicture}[show background left]
 \filldraw[fill=yellow] (0,0) ellipse (10mm and 5mm);
\end{tikzpicture}
&
\begin{tikzpicture}[show background right]
 \filldraw[fill=yellow] (0,0) ellipse (10mm and 5mm);
\end{tikzpicture}
\\ \hline
\RDD{show background top} & \RDD{show background bottom} & \RDD{show background left} & \RDD{show background right}
\\ \hline
\end{tabular}

\bigskip

 \begin{tabular}{|c|c|c|} \hline 
 \multicolumn{3}{|c|}{ [framed,show background top,\RDD{outer frame xsep}=1cm]  }
 \\ \hline  
  
\begin{tikzpicture}[framed,show background top,outer frame xsep=1cm]
 \filldraw[fill=yellow](0,0) ellipse (1cm and .5cm);
\end{tikzpicture}
&
\begin{tikzpicture}[framed,show background left,outer frame ysep=1cm]
 \filldraw[fill=yellow](0,0) ellipse (1cm and .5cm);
\end{tikzpicture}
&
\begin{tikzpicture}[framed,show background left,show background left,show background top,outer frame sep=1cm]
 \filldraw[fill=yellow](0,0) ellipse (1cm and .5cm);
\end{tikzpicture}
\\ \hline 
\RDD{outer frame xsep}=1cm & \RDD{outer frame ysep}=1cm
& \RDD{outer frame sep}=1cm
\\ \hline 
\end{tabular} 

\subsubsection{Style}

 \begin{tabular}{|c|c|c|c|c|} \hline  
 \multicolumn{4}{|l|}{  \BS{begin}\AC{tikzpicture}[show background left,  }\\
 \multicolumn{4}{|l|}{ \hspace{0.5cm}  [\RDD{background left/.style}=\AC{double,ultra thick,draw=blue}]  }
 \\ \hline
  
 \begin{tikzpicture}[show background left,background left/.style={double,ultra thick,draw=blue}]
 \filldraw[fill=yellow] (0,0) ellipse (1cm and .5cm);
 \end{tikzpicture}
 &
 \begin{tikzpicture}[show background left,background left/.style={<->,ultra thick,draw=blue}]
 \filldraw[fill=yellow](0,0) ellipse (1cm and .5cm);
 \end{tikzpicture}
 &
 \begin{tikzpicture}[show background left,background left/.style={line width=10pt,draw=blue}]
 \filldraw[fill=yellow](0,0) ellipse (1cm and .5cm);
 \end{tikzpicture}
  &
  \begin{tikzpicture}[show background left,background left/.style={dashed,ultra thick,draw=blue}]
 \filldraw[fill=yellow] (0,0) ellipse (1cm and .5cm);
  \end{tikzpicture}
 \\ \hline 
 \RDD{double} & \RDD{<->} &  \RDD{line width}=10pt & \RDD{dashed}
  \\ \hline 

\end{tabular}

\SbSbSSCT{Quadrillage}{Gridding}

 \begin{tabular}{|c|l|} \hline  
 \begin{tikzpicture}[baseline=0pt,show background grid]
 \filldraw [fill=yellow](0,0) ellipse (2 and 1);
 \end{tikzpicture}
&
\parbox{8cm}{
\BS{begin}\AC{tikzpicture}[\RDD{show background grid}]\\
\BS{filldraw}[fill=yellow] (0,0) ellipse (2 and 1);\\
\BS{end}\AC{tikzpicture}\\
\\
\emph{\TFRGB{Autre syntaxe}{Other syntax}}   : \\
\BS{begin}\AC{tikzpicture}[\RDD{gridded}]\\
}
\\ \hline 
\end{tabular}

\subsubsection{Style}

 
 \begin{tabular}{|c|c|c|} \hline  
 \multicolumn{3}{|l|}{  [\RDD{background  grid/.style}=\AC{ultra thick,draw=blue},show background grid]  }
 \\ \hline
\begin{tikzpicture}[background grid/.style={ultra thick,draw=blue},show background grid]
 \filldraw [fill=yellow](0,0) ellipse (2 and 1);
\end{tikzpicture}
&
\begin{tikzpicture}[background grid/.style={draw=red},show background grid]
 \filldraw [fill=yellow](0,0) ellipse (2 and 1);
\end{tikzpicture}
&
\begin{tikzpicture}[background grid/.style={step=.5cm,draw=blue},show background grid]
 \filldraw [fill=yellow](0,0) ellipse (2 and 1);
\end{tikzpicture}
\\ \hline
\RDD{ultra thick} ,draw=blue,\RDD{draw}=blue & draw=red & \RDD{step}=.5cm,draw=blue
\\ \hline
\end{tabular}

\SbSbSSCT{Encadrement et quadrillage}{Framing and gridding}
 
 \begin{tabular}{|c|l|} \hline  
 \begin{tikzpicture}[baseline=0pt,framed,gridded]
 \filldraw [fill=yellow](0,0) ellipse (2 and 1);
 \end{tikzpicture}
&
\parbox{8cm}{
\BS{begin}\AC{tikzpicture}[\RDD{framed , gridded }]\\
\BS{filldraw}[fill=yellow] (0,0) ellipse (2 and 1);\\
\BS{end}\AC{tikzpicture}\\
\\

}
\\ \hline 
\end{tabular}




\newpage 


\SSCT{Créer ses couleurs}{Defining your own colors}

 
\SbSSCT{Couleurs de base }{Basic colors}

\begin{tabular}{|c|c|c|c|c|} \hline
\fbox{\tikz \fill [black] (0,0)rectangle (2,1);}
&
\fbox{\tikz \fill [blue] (0,0)rectangle (2,1);}
&
\fbox{\tikz \fill [brown] (0,0)rectangle (2,1);}
&
\fbox{\tikz \fill [cyan] (0,0)rectangle (2,1);}
&
\fbox{\tikz \fill [darkgray] (0,0)rectangle (2,1);}
\\ \hline 
black & blue &  brown & cyan & darkgray
\\ \hline 
\fbox{\tikz \fill [gray] (0,0)rectangle (2,1);}
&  
\fbox{\tikz \fill [green] (0,0)rectangle (2,1);}
&
\fbox{\tikz \fill [lightgray] (0,0)rectangle (2,1);}
&
\fbox{\tikz \fill [lime] (0,0)rectangle (2,1);}
&
\fbox{\tikz \fill [magenta] (0,0)rectangle (2,1);}
\\ \hline 
gray & green &  lightgray & lime & magenta
\\ \hline 
\fbox{\tikz \fill [olive] (0,0)rectangle (2,1);}
&
\fbox{\tikz \fill [orange] (0,0)rectangle (2,1);}
&
\fbox{\tikz \fill [pink] (0,0)rectangle (2,1);}
&
\fbox{\tikz \fill [purple] (0,0)rectangle (2,1);}
&
\fbox{\tikz \fill [red] (0,0)rectangle (2,1);}
\\ \hline  
olive & orange & pink & purple & red
\\ \hline 
\fbox{\tikz \fill [teal] (0,0)rectangle (2,1);}
&
\fbox{\tikz \fill [violet] (0,0)rectangle (2,1);}
&
\fbox{\tikz \fill [white] (0,0)rectangle (2,1);}
&
\fbox{\tikz \fill [yellow] (0,0)rectangle (2,1);}
&
\\ \hline 
teal & violet & white & yellow &
\\ \hline 
\end{tabular} 

\bigskip

\noindent
\begin{tabular}{|c|c|c|c|c|} \hline  
\fbox{\begin{tikzpicture}[baseline=-.5]
 \fill [blue!10](0,0)rectangle (2,1);
 \end{tikzpicture}}
& 
\fbox{\begin{tikzpicture}[baseline=-.5]
 \fill [blue!30](0,0)rectangle (2,1);
 \end{tikzpicture}}
 & 
\fbox{\begin{tikzpicture}[baseline=-.5]
 \fill [blue!50](0,0)rectangle (2,1);
  \end{tikzpicture}}
   & 
  \fbox{\begin{tikzpicture}[baseline=-.5]
   \fill [blue!70](0,0)rectangle (2,1);
    \end{tikzpicture}}
      & 
     \fbox{\begin{tikzpicture}[baseline=-.5]
      \fill [blue!70](0,0)rectangle (2,1);
       \end{tikzpicture}}
 \\ 
\hline [blue!10]  & [blue!30] &[blue!50] & [blue!70] & [blue!90]\\ 
\hline 
\end{tabular}

\SbSSCT{Mélange de couleurs}{Colors mixing}


\begin{tabular}{|c|c|c|c|} \hline  
\fbox{\begin{tikzpicture}[baseline=-.5]
 \fill [blue!30!red](0,0)rectangle (2,1);
 \end{tikzpicture}}
& 
\fbox{\begin{tikzpicture}[baseline=-.5]
 \fill [red!80!blue!20](0,0)rectangle (2,1);
 \end{tikzpicture}}
 & 
\fbox{\begin{tikzpicture}[baseline=-.5]
 \fill [red!80!blue!50](0,0)rectangle (2,1);
  \end{tikzpicture}}
   & 
  \fbox{\begin{tikzpicture}[baseline=-.5]
   \fill [red!80!blue!50!black!40](0,0)rectangle (2,1);
    \end{tikzpicture}}
 \\ 
\hline [blue!30!red]  & [red!80!blue!20] &[red!80!blue!50] & [red!80!blue!50!black!40] \\ 
\hline 
\end{tabular}

\SbSSCT{Créer son nom de couleur}{Naming a color}

\begin{center}
\RRR{15-2}
\end{center}

\SbSbSSCT{A pourcentage de rouge vert et  bleue}{Percentage of red , green and blue}


\begin{tabular}{|c|c|} \hline  
\definecolor{macouleur}{rgb}{.75,0.5,0.25}

\fbox{\begin{tikzpicture}[baseline=-.5]
\fill [macouleur] (0,0)rectangle (2,1);
\end{tikzpicture}}
&  
\parbox[b]{8cm}{
\BSS{definecolor}\AC{{\color{red} macouleur}}\AC{rgb}\AC{.75,0.5,0.25}\\
(75\% de rouge 50\% de vert 25\% de bleu)\\
 \BS{fill} [{\color{red} macouleur}] (0,0) rectangle (2,1);
}
\\ \hline 
\end{tabular} 

\SbSbSSCT{A partir d'une couleur existante}{From existing color}

\begin{tabular}{|c|c|} \hline  
\colorlet{monrouge}{red!25}

\fbox{\begin{tikzpicture}[baseline=-.5]
 \fill [monrouge] (0,0)rectangle (2,1);
 \end{tikzpicture}} 
&  
\parbox[b]{8cm}{
\BSS{colorlet}\AC{{\color{red} monrouge}}\AC{red!25}
 \\
 \BS{fill} [{\color{red} monrouge}] (0,0) rectangle (2,1);
}
\\ \hline
\colorlet{monviolet}{red!25!blue}
\fbox{\begin{tikzpicture}[baseline=-.5]
 \fill [monviolet] (0,0)rectangle (2,1);
 \end{tikzpicture}} 
&  
\parbox[b]{8cm}{
\BSS{colorlet}\AC{{\color{red} monviolet}}\AC{red!25!blue}
 \\
 \BS{fill} [{\color{red} monviolet}] (0,0) rectangle (2,1);
}
\\ \hline 
\end{tabular}

\newpage

\SSCT{Opacité}{Opacity} 

\begin{center}
\RRR{23-2}
\end{center}

\begin{tabular}{|c|c|c|c|c|} \hline
\multicolumn{5}{|c|}{\BS{draw}[red] (0,0) -- (2,1);\hspace{.5cm} \BS{draw} [blue,\RDD{draw opacity}=0] (0,1) - - (2,0);}
 \\ \hline  
\begin{tikzpicture}[line width=.5cm]
\draw[red] (0,0) -- (2,1);
\draw [blue,draw opacity=0] (0,1) -- (2,0);
\end{tikzpicture}
&  
\begin{tikzpicture}[line width=.5cm]
\draw[red] (0,0) -- (2,1);
\draw [blue,draw opacity=0.25] (0,1) -- (2,0);
\end{tikzpicture}
&  
\begin{tikzpicture}[line width=.5cm]
\draw[red] (0,0) -- (2,1);
\draw [blue,draw opacity=0.5] (0,1) -- (2,0);
\end{tikzpicture}
&  
\begin{tikzpicture}[line width=.5cm]
\draw[red] (0,0) -- (2,1);
\draw [blue,draw opacity=0.75] (0,1) -- (2,0);
\end{tikzpicture}
&  
\begin{tikzpicture}[line width=.5cm]
\draw[red] (0,0) -- (2,1);
\draw [blue,draw opacity=1] (0,1) -- (2,0);
\end{tikzpicture}

\\ \hline  
draw opacity=0 & draw opacity=0.25 & draw opacity=0.5 & draw opacity=0.75 & draw opacity=1\\ 
\hline 
\end{tabular} 

\bigskip

\begin{tabular}{|c|c|c|c|} \hline 
\multicolumn{4}{|c|}{\BS{fill}[red] (0,0) rectangle (1,1);\hspace{.5cm} \BS{fill}[blue,\RDD{transparent}] (0.5,0) rectangle (1.5,1);}
 \\ \hline
\fbox{\begin{tikzpicture}
\fill[red] (0,0) rectangle (1,1);
\fill[blue,transparent] (0.5,0) rectangle (1.5,1);
\end{tikzpicture}}   
&
\fbox{\begin{tikzpicture}
\fill[red] (0,0) rectangle (1,1);
\fill[blue,ultra nearly transparent] (0.5,0) rectangle (1.5,1);
\end{tikzpicture}}   
&
\fbox{\begin{tikzpicture}
\fill[red] (0,0) rectangle (1,1);
\fill[blue,very nearly transparent] (0.5,0) rectangle (1.5,1);
\end{tikzpicture}}   

&
\fbox{\begin{tikzpicture}
\fill[red] (0,0) rectangle (1,1);
\fill[blue,nearly transparent] (0.5,0) rectangle (1.5,1);
\end{tikzpicture}}

\\ \hline  
\RDD{transparent} & \RDD{ultra nearly transparent} & \RDD{very nearly transparent} & \RDD{nearly transparent} \\ 
\hline 
\fbox{\begin{tikzpicture}
\fill[red] (0,0) rectangle (1,1);
\fill[blue,semitransparent] (0.5,0) rectangle (1.5,1);
\end{tikzpicture}}

&
\fbox{\begin{tikzpicture}
\fill[red] (0,0) rectangle (1,1);
\fill[blue,nearly opaque] (0.5,0) rectangle (1.5,1);
\end{tikzpicture}}

&
\fbox{\begin{tikzpicture}
\fill[red] (0,0) rectangle (1,1);
\fill[blue,very nearly opaque] (0.5,0) rectangle (1.5,1);
\end{tikzpicture}}

&
\fbox{\begin{tikzpicture}
\fill[red] (0,0) rectangle (1,1);
\fill[blue,ultra nearly opaque] (0.5,0) rectangle (1.5,1);
\end{tikzpicture}}

\\ \hline 
\RDD{semitransparent} & \RDD{nearly opaque} & \RDD{very nearly opaque} & \RDD{ultra nearly opaque}
\\ \hline
\fbox{\begin{tikzpicture}
\fill[red] (0,0) rectangle (1,1);
\fill[blue,opaque] (0.5,0) rectangle (1.5,1);
\end{tikzpicture}} 

&
\fbox{\begin{tikzpicture}
\fill[red] (0,0) rectangle (1,1);
\fill[blue,fill opacity=.25] (0.5,0) rectangle (1.5,1);
\end{tikzpicture}}
&
\fbox{\begin{tikzpicture}
\fill[red] (0,0) rectangle (1,1);
\fill[blue,fill opacity=.5] (0.5,0) rectangle (1.5,1);
\end{tikzpicture}}
&
\fbox{\begin{tikzpicture}
\fill[red] (0,0) rectangle (1,1);
\fill[blue,fill opacity=.75] (0.5,0) rectangle (1.5,1);
\end{tikzpicture}}

\\ \hline 
\RDD{opaque} & \RDD{fill opacity}=.25 & \RDD{fill opacity}=.5 & \RDD{fill opacity}=.75
\\ \hline 
\end{tabular} 

\bigskip


\begin{tabular}{|c|c|c|c|c|} \hline 
\multicolumn{5}{|c|}{\BS{node} at (1,1) [\RDD{text opacity}=1] \AC{ \BS{Huge} texte} ;}
 \\ \hline  
\fbox{\tikz{
\node at (1,1) [text opacity=1] {\Huge texte} ; }}
&  
\fbox{\tikz{
\node at (1,1) [text opacity=.75] {\Huge texte} ; }}
&  
\fbox{\tikz{
\node at (1,1) [text opacity=.5] {\Huge texte} ; }}
&  
\fbox{\tikz{
\node at (1,1) [draw,text opacity=.25] {\Huge texte} ; }}
&  
\fbox{\tikz{
\node at (1,1) [draw,text opacity=0] {\Huge texte} ;= }}
\\ \hline  
text opacity=1 & text opacity=0.75 & text opacity=0.5 & opacity=0.25 & text opacity=0 
\\ \hline 
\end{tabular} 

\newpage

\subsection{Blend Modes} 

\begin{center}
\RRR{23-3} 
\end{center}

\begin{tabular}{|c|c|c|} \hline  
\tikz [blend group=normal] {
\fill[red!50] ( 90:.6) circle (1);
\fill[green!50] (210:.6) circle (1);
\fill[blue!50] (330:.6) circle (1);
}
&  
\tikz [blend group=multiply ] {
\fill[red!50] ( 90:.6) circle (1);
\fill[green!50] (210:.6) circle (1);
\fill[blue!50] (330:.6) circle (1);
}
&  
\tikz [blend group=screen] {
\fill[red!50] ( 90:.6) circle (1);
\fill[green!50] (210:.6) circle (1);
\fill[blue!50] (330:.6) circle (1);
}
\\ \hline 
blend group=\BDD{normal} & blend group=\BDD{multiply} & blend group=\BDD{screen}  
\\ \hline
\tikz [blend group=overlay] {
\fill[red!50] ( 90:.6) circle (1);
\fill[green!50] (210:.6) circle (1);
\fill[blue!50] (330:.6) circle (1);
}
&
\tikz [blend group=darken] {
\fill[red!50] ( 90:.6) circle (1);
\fill[green!50] (210:.6) circle (1);
\fill[blue!50] (330:.6) circle (1);
}
&  
\tikz [blend group=lighten] {
\fill[red!50] ( 90:.6) circle (1);
\fill[green!50] (210:.6) circle (1);
\fill[blue!50] (330:.6) circle (1);
}
\\ \hline 
 blend group=\BDD{overlay} & blend group=\BDD{darken} & blend group=\BDD{lighten} 
\\ \hline    
\tikz [blend group=difference] {
\fill[red!50] ( 90:.6) circle (1);
\fill[green!50] (210:.6) circle (1);
\fill[blue!50] (330:.6) circle (1);}
&  
\tikz [blend group=exclusion] {
\fill[red!50] ( 90:.6) circle (1);
\fill[green!50] (210:.6) circle (1);
\fill[blue!50] (330:.6) circle (1);}
&
\tikz [blend group=hue] {
\fill[red!50] ( 90:.6) circle (1);
\fill[green!50] (210:.6) circle (1);
\fill[blue!50] (330:.6) circle (1);
}
\\ \hline 
 blend group=\BDD{difference} & blend group=\BDD{exclusion} & blend group=\BDD{hue} 
\\ \hline  
\tikz [blend group=saturation] {
\fill[red!50] ( 90:.6) circle (1);
\fill[green!50] (210:.6) circle (1);
\fill[blue!50] (330:.6) circle (1);
}
&  
\tikz [blend group=color] {
\fill[red!50] ( 90:.6) circle (1);
\fill[green!50] (210:.6) circle (1);
\fill[blue!50] (330:.6) circle (1);
}
&  
\tikz [blend group=luminosity] {
\fill[red!50] ( 90:.6) circle (1);
\fill[green!50] (210:.6) circle (1);
\fill[blue!50] (330:.6) circle (1);}
\\ \hline 
 blend group=\BDD{saturation} & blend group=\BDD{color} & blend group=\BDD{luminosity}
\\ \hline 
\end{tabular} 

\bigskip
\begin{tabular}{|c|c|c|c|} \hline
\multicolumn{4}{|c|}{\TFRGB{A revoir message d'erreur}{Error message } Unknow blend mode ! }
\\ \hline  
%\tikz [blend group=colordodge] {
%\fill[red!50] ( 90:.6) circle (1);
%\fill[green!50] (210:.6) circle (1);
%\fill[blue!50] (330:.6) circle (1);
%}
&  
%\tikz [blend group=colorburn] {
%\fill[red!50] ( 90:.6) circle (1);
%\fill[green!50] (210:.6) circle (1);
%\fill[blue!50] (330:.6) circle (1);
%} 

&  
%\tikz [blend group=hardlight] {
%\fill[red] ( 90:.6) circle (1);
%\fill[green] (210:.6) circle (1);
%\fill[blue] (330:.6) circle (1);
%}
&  
%\tikz [blend group=softlight] {
%\fill[red] ( 90:.6) circle (1);
%\fill[green] (210:.6) circle (1);
%\fill[blue] (330:.6) circle (1);
%}
\\ \hline  
blend group=colordodge & blend group=colorburn  & blend group=hardlight & blend group=softlight \\ 
\hline 
\end{tabular} 

\newpage

%==================
\subsection{Fading} 


 \maboite{\BS{usetikzlibrary}\AC{fadings}}

\label{lib-fadings}

\SbSbSSCT{Modèles prédéfinis }{Preset patterns}

\begin{center}
 \RRR{51}
\end{center}
 
\begin{tabular}{|c|c|c|c|} \hline  
\multicolumn{4}{|l|}{ \BS{fill} [blue,\RDD{path fading}=north] (-1,-1) rectangle (1,1);}
\\ \hline
\begin{tikzpicture}
 \draw (-1,-1) rectangle (1,1);
\fill [blue,path fading=north] (-1,-1) rectangle (1,1);
\end{tikzpicture}
&  
\begin{tikzpicture}
 \draw (-1,-1) rectangle (1,1);
\fill [blue,path fading=south] (-1,-1) rectangle (1,1);
\end{tikzpicture}
&  
\begin{tikzpicture}
\draw (-1.2,-1.2) rectangle (1.2,1.2);
\fill [path fading=east] (-1,-1) rectangle (1,1);
\end{tikzpicture} 
& 
\begin{tikzpicture}
\draw (-1.2,-1.2) rectangle (1.2,1.2);
\fill [path fading=west] (-1,-1) rectangle (1,1);
\end{tikzpicture}
\\ \hline  
path fading=north & path fading=south & path fading=east  & path fading=west   
\\ \hline 
\end{tabular}

\begin{tabular}{|c|c|} \hline  
\begin{tikzpicture}
 \draw (-1,-1) rectangle (1,1);
\fill [blue,path fading=circle with fuzzy edge 10 percent] (-1,-1) rectangle (1,1);
\end{tikzpicture}
&
\begin{tikzpicture}
 \draw (-1,-1) rectangle (1,1);
\fill [blue,path fading=circle with fuzzy edge 15 percent] (-1,-1) rectangle (1,1);
\end{tikzpicture}
\\ \hline 
path fading=circle with fuzzy edge 10 percent & path fading=circle with fuzzy edge 15 percent 
\\ \hline 
\begin{tikzpicture}
 \draw (-1,-1) rectangle (1,1);
\fill [blue,path fading=circle with fuzzy edge 20 percent] (-1,-1) rectangle (1,1);
\end{tikzpicture}
&
\begin{tikzpicture}
 \draw (-1,-1) rectangle (1,1);
\fill [blue,path fading=fuzzy ring 15 percent] (-1,-1) rectangle (1,1);
\end{tikzpicture}
\\ \hline 
path fading=circle with fuzzy edge 20 percent & path fading=fuzzy ring 15 percent
\\ \hline 
\end{tabular}

%\subsubsection{Création de décoloration avec tikzfadingfrompicture}
\SbSbSSCT{Création de décoloration avec tikzfadingfrompicture}{Own patterns of fading with  tikzfadingfrompicture}

\begin{center}
 \RRR{23-4-1}
\end{center}

\noindent 
\begin{tabular}{|c|c|} \hline 
\emph{\TFRGB{Création}{Creation}} & \emph{\TFRGB{Visualisation}{Visualization}}
\\ \hline
\parbox[c]{11cm}{ 
\ESS{tikzfadingfrompicture}[\RDD{name}=filtre] \\
\BS{shade}[left color=yellow,right color=blue!100] (0,0) rectangle (2,2); \\
\BS{fill}[blue!50] (1,1) circle (0.7);\\
\BS{end\AC{tikzfadingfrompicture}} }
&  
\begin{tikzpicture}[baseline=1cm]
\shade[left color=yellow,right color=blue!100] (0,0) rectangle (2,2);
\fill[blue!50] (1,1) circle (0.7);
\end{tikzpicture}
\\ \hline 
\parbox[c]{11cm}{ 
\ESS{tikzfadingfrompicture}[\RDD{name}=tikz] \\
\BS{node} [draw,text=transparent!20] \\
\AC{\BS{fontfamily}\AC{ptm}\BS{fontsize}\AC{25}\AC{25}\BS{bfseries}\BS{selectfont} TikZ};\\
\BS{end\AC{tikzfadingfrompicture}} }
& 
\begin{tikzpicture}[baseline=-.3cm]
\node [draw,text=transparent!20]
{\fontfamily{ptm}\fontsize{25}{25}\bfseries\selectfont TikZ};
\end{tikzpicture}
 \\ \hline 
\end{tabular} 

%\begin{tabular}{|c|c|c|} \hline  
\begin{tikzfadingfrompicture}[name=filtre]
\shade[left color=yellow,right color=blue!100] (0,0) rectangle (2,2);
\fill[blue!50] (1,1) circle (0.7);
\end{tikzfadingfrompicture}


\begin{tikzfadingfrompicture}[name=tikz]
\node [text=transparent!20]
{\fontfamily{ptm}\fontsize{25}{25}\bfseries\selectfont Ti\emph{k}Z};
\end{tikzfadingfrompicture}

\bigskip

\begin{tabular}{|c|c|} \hline 
\multicolumn{2}{|c|}{ \TFRGB{Utilisation dans un rectangle}{Use in a frame}}
 \\ \hline 
\multicolumn{2}{|c|}{\BS{fill}[\RDD{path fading}=filtre] (-2,-1) rectangle (2,1); }

\\ \hline 
\begin{tikzpicture}
\draw(-2,-1) rectangle (2,1);
\fill[path fading=filtre] (-2,-1) rectangle (2,1);
\end{tikzpicture}
&  
\begin{tikzpicture}
\draw(-2,-1) rectangle (2,1);
\fill[path fading=tikz] (-2,-1) rectangle (2,1);
\end{tikzpicture}
\\ \hline  
[\RDD{path fading}=filtre] &  [\RDD{path fading}=tikz]
\\ \hline 
\begin{tikzpicture}
\draw(-2,-1) rectangle (2,1);
\fill[path fading=filtre ,fit fading=false] (-2,-1) rectangle (2,1);
\end{tikzpicture}
&  
\begin{tikzpicture}
% Background
\draw(-2,-1) rectangle (2,1);
\fill[path fading=tikz,fit fading=false] (-2,-1) rectangle (2,1);
\end{tikzpicture}
\\ \hline 
[path fading=filtre ,\RDD{fit fading}=false] & [path fading=tikz,\RDD{fit fading}=false]
\\ 
\hline 
\begin{tikzpicture}
% Background
\draw(-2,-1) rectangle (2,1);
\fill[path fading=filtre ,left color=blue,right color=red] (-2,-1) rectangle (2,1);
\end{tikzpicture}
&  
\begin{tikzpicture}
\draw(-2,-1) rectangle (2,1);
\fill[path fading=tikz,left color=blue,right color=red] (-2,-1) rectangle (2,1);
\end{tikzpicture}
\\ \hline left color=blue,right color=red & [path left color=blue,right color=red  \\ 
\hline

\begin{tikzpicture}
 Background
\draw(-2,-1) rectangle (2,1);
\fill[path fading=filtre ,red] (-2,-1) rectangle (2,1);
\end{tikzpicture}
&  
\begin{tikzpicture}
 Background
\draw(-2,-1) rectangle (2,1);
\fill[path fading=tikz,red] (-2,-1) rectangle (2,1);
\end{tikzpicture}
\\ \hline 
[path fading=filtre ,red] &  [path fading=tikz,red] \\ 
\hline 
\end{tabular}
 
\bigskip


\begin{tabular}{|c|c|} \hline 
\multicolumn{2}{|c|}{\TFRGB{Utilisation dans un ellipse}{Use in an ellipse}}
 \\ \hline 
\multicolumn{2}{|c|}{\BS{fill}[\RDD{path fading}=filtre] (-2,-1) ellipse (2 and 1); }
\\ \hline
\begin{tikzpicture}
\draw  (-2,-1) ellipse (2 and 1);
\fill[path fading=filtre] (-2,-1) ellipse (2 and 1);
\end{tikzpicture}
&
\begin{tikzpicture}
\draw  (-2,-1) ellipse (2 and 1 );
\fill[path fading=tikz] (-2,-1) ellipse (2 and 1);
\end{tikzpicture}
\\ \hline
[\RDD{path fading}=filtre] &  [\RDD{path fading}=tikz]
\\ \hline
\end{tabular} 



%\subsection{Création de décoloration avec tikzfading}
\SbSSCT{Création de décoloration avec tikzfading}{Creating fading patterns with tikzfading}

\begin{tabular}{|c|c|} \hline
\parbox[c]{11cm}{ 
\BSS{tikzfading}[\RDD{name}={\color{purple} fade right},
left color=transparent!0,
right color=transparent!100]\\ \\
\BS{tikz} \BS{filldraw} [red,\RDD{path fading}={\color{purple} fade right}] (-1,-1) rectangle (1,1);}
&  
\tikzfading[name=fade right,
left color=transparent!0,
right color=transparent!100]

\tikz[baseline=0pt] \filldraw [red,path fading=fade right] (-1,-1) rectangle (1,1);
\\ \hline  
\parbox[c]{11cm}{ 
\BSS{tikzfading}[\RDD{name}={\color{purple}fade out},
inner color=transparent!0,
outer color=transparent!100]\\ \\
\BS{tikz} \BS{filldraw} [blue,\RDD{path fading}={\color{purple} fade out}] (-1,-1) rectangle (1,1);}
&  
\tikzfading[name=fade out,
inner color=transparent!0,
outer color=transparent!100]

\tikz[baseline=0pt] \filldraw [blue,path fading=fade out] (-1,-1) rectangle (1,1);

\\ \hline 
\parbox[c]{11cm}{ 
\BSS{tikzfading}[\RDD{name}={\color{purple}fade inside},
inner color=transparent!80,
outer color=transparent!10]\\ \\
\BS{tikz} \BS{filldraw} [blue,\RDD{path fading}={\color{purple} fade inside}] (-1,-1) rectangle (1,1);}
&  
\tikzfading[name=fade inside,
inner color=transparent!80,
outer color=transparent!10]

\tikz[baseline=0pt] \filldraw [blue,path fading= fade inside] (-1,-1) rectangle (1,1);
\\ \hline 
\parbox[c]{11cm}{ 
\BSS{tikzfading}[\RDD{name}={\color{purple}middle},
top color=transparent!80,
bottom color=transparent!80,
middle color=transparent!20]\\ \\
\BS{tikz} \BS{filldraw} [blue,\RDD{path fading}={\color{purple} middle}] (-1,-1) rectangle (1,1);}
&  
\tikzfading[name=middle,
top color=transparent!80,
bottom color=transparent!80,
middle color=transparent!20]

\tikz[baseline=0pt] \filldraw [blue,path fading= middle] (-1,-1) rectangle (1,1);

\\ \hline 
\end{tabular} 

\SbSbSSCT{Modification de la décoloration }{Modification of the fading pattern}

\begin{center}
\RRR{23-4-2}
\end{center}

  
\begin{tabular}{|c|c|c|} \hline
\multicolumn{3}{|l|}{ \BS{fill} [blue,path fading=north,fading transform=\AC{yshift=-.5cm}] (-1,-1) rectangle (1,1);}
\\ \hline
\begin{tikzpicture}
 \draw (-1,-1) rectangle (1,1);
\fill [blue,path fading=north,fading transform={yshift=-.5cm}](-1,-1) rectangle (1,1);
\end{tikzpicture}
& 
\begin{tikzpicture}
 \draw (-1,-1) rectangle (1,1);
\fill [blue,path fading=north,fading transform={rotate=30}](-1,-1) rectangle (1,1);
\end{tikzpicture}
&
\begin{tikzpicture}
 \draw (-1,-1) rectangle (1,1);
\fill [blue,path fading=north,fading angle=30](-1,-1) rectangle (1,1);
\end{tikzpicture}
\\ \hline 
\RDD{fading transform}=\AC{yshift=-.5cm} & \RDD{fading transform}=\AC{yshift=-.5cm} & \RDD{fading angle}=30
\\ \hline 
\end{tabular} 

\bigskip

\begin{center}
\RRR{23-4-3}
\end{center}

\begin{tabular}{|c|c|} \hline
\parbox[b]{10cm}{
 \BS{begin}\AC{tikzpicture} \\
 \BS{draw}  (-1,-1) rectangle (1,1);\\
 \BS{path} [\RDD{scope fading}=east] (-1,-1) rectangle (1,1);\\
 \BS{fill}[red] ( 90:1) circle (1);\\
 \BS{fill}[green] (210:1) circle (1);\\
 \BS{fill}[blue] (330:1) circle (1); \\
 \BS{end}\AC{tikzpicture} \\
  }
&
 \begin{tikzpicture}
 \draw  (-1,-1) rectangle (1,1);
 \path [scope fading=east] (-1,-1) rectangle (1,1);
 \fill[red] ( 90:1) circle (1);
 \fill[green] (210:1) circle (1);
 \fill[blue] (330:1) circle (1);
 \end{tikzpicture}
 \\ \hline 
\end{tabular}


\bigskip
\begin{tabular}{|c|c|} \hline
\parbox[c]{8cm}{
\BS{tikz} \BS{node} [black,\RDD{scope fading}=south,fading angle=45,text width=5cm] \\
\AC{
VisualTIKZ VisualTIKZ VisualTIKZ VisualTIKZ VisualTIKZ VisualTIKZ VisualTIKZ VisualTIKZ VisualTIKZ VisualTIKZ VisualTIKZ VisualTIKZ VisualTIKZ 
}; 
 }
&
\tikz \node [black,scope fading=south,fading angle=45,text width=5cm]
{
VisualTIKZ VisualTIKZ VisualTIKZ VisualTIKZ VisualTIKZ VisualTIKZ VisualTIKZ VisualTIKZ VisualTIKZ VisualTIKZ VisualTIKZ VisualTIKZ VisualTIKZ 
};
 \\ \hline 
\end{tabular}

\subsection{Transparency Groups} 

\begin{center}
\RRR{23-5}
\end{center}

\begin{tabular}{|c|c|} \hline 
\multicolumn{2}{|l|}{\BS{begin}\AC{tikzpicture}[opacity=.5]} \\
\multicolumn{2}{|l|}{ \BS{draw} [line width=1cm] (0,0) -- (2,2); }\\
\multicolumn{2}{|l|}{ \BS{draw} [line width=1cm] (0,2) -- (2,0); }\\
\multicolumn{2}{|l|}{\BS{end}\AC{tikzpicture}}
 \\ \hline 
\begin{tikzpicture}[opacity=.5]
\draw [line width=1cm] (0,0) -- (2,2);
\draw [line width=1cm] (0,2) -- (2,0);
\end{tikzpicture}
& 
\begin{tikzpicture}[opacity=.5,transparency group]
\draw [line width=1cm] (0,0) -- (2,2);
\draw [line width=1cm] (0,2) -- (2,0);
\end{tikzpicture} 
 \\ 
\hline [opacity=.5] & [opacity=.5,\RDD{transparency group}] \\ 
\hline 
\end{tabular} 

\bigskip

\begin{tabular}{|c|c|} \hline 
\multicolumn{2}{|c|}{\TFRGB{A revoir : ne fonctionne pas}{Not working !} }
\\ \hline  
\parbox[c]{9cm}{ 
\BS{begin}\AC{tikzpicture} \\
\BS{shade} [left color=red,right color=blue] (-2,-1) rectangle (2,1); \\
\BS{begin}\AC{scope}[transparency group=knockout] \\
\BS{fill]}[white] (-1.9,-.9) rectangle (1.9,.9); \\
\BS{node} [opacity=0] {TikZ}; \\
\BS{end}\AC{scope} \\
\BS{end}\AC{tikzpicture}
}
&  
\begin{tikzpicture}[baseline=0pt]
\shade [left color=red,right color=blue] (-2,-1) rectangle (2,1);
\fill [white] (-1.9,-.9) rectangle (1.9,.9);
\node [opacity=0,transparency group=knockout]{TikZ};
\end{tikzpicture}
\\ \hline 
\end{tabular} 

 

\newpage

\SSCT{Créer ses commandes}{Create command}



\maboite{ \TFRGB{Atention : la création de la commande doit être placée avant}{Warning: the creation of the command must be placed before} \BS{begin}\AC{document} ! }

\TFRGB{syntaxe :\BSS{newcommand}\AC{\BS{}nom}[nombre de variables]\AC{Description}}{syntax :\BSS{newcommand}\AC{\BS{}name}[
number of variables]\AC{Description}}
\bigskip


\textbf{\TFRGB{Exemple : commande avec une variable}{Example : command with one variable} :}

 \emph{\TFRGB{Création}{Creation}}


\begin{tabular}{ll}
\BS{newcommand} 	& \\ 
\AC{\BS{}maboite}[1]{\color{red} \{	}	 					& \% \TFRGB{commande nommée maboite et 1 seul d'argument}{command   named \og maboite \fg with one variable } \\ 
\BS{begin}\AC{center}								& \% \TFRGB{centrage sur la ligne}{centering the box  } \\ 
\BS{tikzpicture} \BS{node}[fill=yellow				& \% \TFRGB{un n\oe ud de texte de couleur jaune}{a yellow text box} \\ 
,text centered										& \% \TFRGB{centrage du texte dans la boite}{centering the text in the box} \\ 
,text width=.5\BS{linewidth}] 							& \% \TFRGB{largeur : la moitié de la ligne}{ to set the width of the box} \\ 
\#1\} ; \BS{end}\AC{center}				& \%  \TFRGB{\#1 correspond à l'argument}{\#1 will be replaced by the variable} \\ 
{\color{red} \}	} 		&  \\  
\end{tabular} 	

\bigskip
\emph{Utilisation} : \BS{maboite}\AC{contenu}

\maboite{contenu}


\bigskip
\label{DFR}

\textbf{\TFRGB{Exemple : commande sans variable}{Example : command without variable} :}
 
 \emph{\TFRGB{Création}{ creation}}
 
\BS{newcommand}\AC{\BS{DFR}}\AC{ \BS{tikzpicture}[scale=.25]
 \BS{draw} [fill=blue](0,0) rectangle (3,1.5);
\BS{draw} [fill=white](1,0) rectangle (2,1.5);
\BS{draw}[fill=red](2,0) rectangle (3,1.5);\BS{endtikzpicture} }
 

\emph{Utilisation}  :  \BS{DFR} \DFR


\newpage

\SSCT{Créer ses styles}{Creating styles}


\SbSSCT{Styles sans variable}{Styles without variable}

\begin{tabular}{|c|c|} \hline  
\begin{tikzpicture}[mon style/.style={draw=blue,fill=red!20,very thick},baseline=0pt]
\draw (0,0) circle (2cm);
\draw[mon style] (0,0) circle (1cm);
\end{tikzpicture}
&  
\parbox{10cm}{ 
\BS{begin}\AC{tikzpicture} [mon style/.\RDD{style}=\AC{draw=blue, fill=red!20, very thick}]\\
\BS{draw} (0,0) circle (2cm); \\
\BS{draw}[mon style] (0,0) circle (1cm); \\
\BS{end}\AC{tikzpicture} \\ } 
\\ \hline 
\end{tabular} 


\SbSSCT{Styles avec variable}{Styles with variable}

\begin{tabular}{|c|c|} \hline  
\begin{tikzpicture}[mon style/.style={draw=#1,thick,fill=#1!50,scale=.5}]
\filldraw [mon style=red] (0,0) rectangle (2,1);
\filldraw [mon style=blue] (3,0) rectangle (5,1);
\end{tikzpicture}
&  
\parbox{12cm}{ 
\BS{begin}\AC{tikzpicture} [mon style/.\RDD{style}=\AC{draw={\color{red}  \#1}, thick, fill={\color{red}  \#1}!50, scale=.5}]\\
\BS{filldraw} [mon style=red] (0,0) rectangle (2,1);\\
\BS{filldraw} [mon style=blue] (3,0) rectangle (5,1);\\
\BS{end}\AC{tikzpicture} \\ 
} 
\\ \hline   

\end{tabular} 




\begin{tabular}{|c|c|} \hline 
 \multicolumn{2}{|c|}{\TFRGB{valeur par défaut}{With a default value }}
 \\ \hline
\begin{tikzpicture}[mon style/.style={draw=#1,thick,fill=#1!50,scale=.5},
mon style/.default=black]
\filldraw [mon style] (0,0) rectangle (2,1);
\filldraw [mon style=blue] (3,0) rectangle (5,1);
\end{tikzpicture}
&  
\parbox{12cm}{ 
\BS{begin}\AC{tikzpicture} [mon style/.\RDD{style}=\AC{draw={\color{red}  \#1},fill={\color{red}  \#1}!20,very thick},\\
mon style/\RDD{default}=black] \\
\BS{filldraw} [mon style] (0,0) rectangle (2,1);\\
\BS{filldraw} [mon style=blue] (3,0) rectangle (5,1);\\
\BS{end}\AC{tikzpicture} }  
\\ \hline   

\end{tabular} 


\newpage

\SSCT{Mettre du texte  en valeur}{Text highlighting}

\label{ndbt}

\tikzset{blue}


\SbSSCT{Dans un n\oe ud de Tikz}{In a TikZ node}
\label{noeudboite}

\begin{tabular}{|c | c | c | c |} \hline
\multicolumn{4}{|c|}{ \BS{tikz} \BS{draw} (0,0) grid (2,2) (1,1) node[ fill=red!20 ] \AC{texte};   }\\ 
\hline 
\tikz \draw (0,0) grid (2,2) (1,1) node[fill=red!20] {texte};
&
\tikz \draw (0,0) grid (2,2) (1,1) node[fill=red!20,draw] {texte}; 
&
\tikz \draw (0,0) grid (2,2) (1,1) node[circle,fill=red!20] {texte};
&
\tikz \draw (0,0) grid (2,2) (1,1) node[circle,fill=red!20,draw] {texte};
\\  \hline
node[fill=red!20] 
&
node[fill=red!20,\RDD{draw}] 
&
 node[fill=red!20,\RDD{circle}]  
&
 node[fill=red!20,\RDD{circle},\RDD{draw}]
 \\  \hline
\end{tabular}
\bigskip


\subsubsection{Options}
\begin{tabular}{|c | c | c | c |c |c |c |c |} \hline
\multicolumn{8}{|c|}{ \BS{tikz} \BS{draw} node[draw,\RDD{double},blue] \AC{texte};   }\\ 
\hline 

\tikz \draw  node[draw,double,blue] {texte};
&
\tikz \draw  node[draw,rounded corners,blue] {texte};
&
\tikz \draw  node[draw,ultra thick,blue] {texte};
&
\tikz \draw  node[draw,dashed,blue] {texte};
&
\tikz \draw  node[draw,red] {texte};
&
\tikz \draw  node[draw,rotate=45,blue] {texte};
&
\tikz \draw  node[draw,shading=radial,blue] {texte};
&
\tikz \draw  node[draw,blue,text=red] {texte};
\\ \hline
\RDD{double} & \RDD{rounded corners} &  ultra thick & dashed & red & rotate=45 & shading=radial & text=red 
\\ \hline
\end{tabular}
\bigskip


\begin{tabular}{|c | c | c | c |c |} \hline
\multicolumn{4}{|c|}{ \BS{tikz} \BS{draw}  node[draw,\RDD{inner sep}=0pt] \AC{texte}; \RRR{17-2-3}  }\\ 
\hline 
\tikz \draw  node[draw,inner sep=0pt,blue] {texte};
&
\tikz \draw node[draw,inner sep=1cm,blue] {texte};
&
\tikz \draw  node[draw,inner xsep=1cm,blue] {texte};
&
\tikz \draw  node[draw,inner ysep=1cm,blue] {texte};
\\ \hline
 \RDD{inner sep}=0pt & \RDD{inner sep}=1cm & \RDD{inner xsep}=1cm & \RDD{inner ysep}=1cm
\\ \hline
\multicolumn{4}{|c|}{ \dft{} : 0.3333em }\\ 
\hline 

\end{tabular}

\bigskip

\begin{tabular}{|c | c | c | c |} \hline
\multicolumn{4}{|l|}{ \BS{node} [fill=red!20,\RDD{outer sep}=1cm] (A) at (1,1) \AC{texte}; \RRR{17-2-3} } \\ 
\multicolumn{4}{|l|}{ \BS{fill} (node cs:name=A,anchor=east) circle (3pt);  }\\ 
\multicolumn{4}{|l|}{ \BS{fill} (node cs:name=A,anchor=south) circle (3pt);  }\\ 
\hline 
\begin{tikzpicture}
\draw[help lines] (0,0) grid (3,2);
\node[fill=red!20,outer sep=1cm] (A) at (1,1) {texte};
\fill[red] (node cs:name=A,anchor=east) circle (3pt);
\fill[red] (node cs:name=A,anchor=south) circle (3pt);
\end{tikzpicture}
&
\begin{tikzpicture}
\draw[help lines] (0,0) grid (3,2);
\node[fill=red!20,outer sep=0pt] (A) at (1,1) {texte};
\fill[red] (node cs:name=A,anchor=east) circle (3pt);
\fill[red] (node cs:name=A,anchor=south) circle (3pt);
\end{tikzpicture}
&
\begin{tikzpicture}
\draw[help lines] (0,0) grid (3,2);
\node[fill=red!20,outer xsep=1cm] (A) at (1,1){texte};
\fill[red] (node cs:name=A,anchor=east) circle (3pt);
\fill[red] (node cs:name=A,anchor=south) circle (3pt);
\end{tikzpicture}
&
\begin{tikzpicture}
\draw[help lines] (0,0) grid (3,2);
\node[fill=red!20,outer ysep=1cm] (A) at (1,1) {texte};
\fill[red] (node cs:name=A,anchor=east) circle (3pt);
\fill[red] (node cs:name=A,anchor=south) circle (3pt);
\end{tikzpicture}
\\ \hline
 \RDD{outer sep}=1cm & \RDD{outer sep}=0pt & \RDD{outer xsep}=1cm & \RDD{outer ysep}=1cm
\\ \hline
\multicolumn{4}{|c|}{ \dft{} : 0.5\BS{pgflinewidth} }\\ 
\hline 
\end{tabular}

\SbSbSSCT{Taille minimale des noeuds}{Minimum size}

\begin{tabular}{|c|c|} \hline  
\multicolumn{2}{|c|}{  \BS{draw}((0,0) node[fill=blue!20,\RDD{minimum height}=1.5cm,draw]  \AC{texte} ;  \RRR{17-2-3}  }\\ 
\hline 
\tikz \draw (0,0) node[fill=red!20,minimum height=1.5cm,draw] {texte};
&  
\tikz \draw (0,0) node[fill=red!20,minimum width=3cm,draw] {texte};

\\ \hline  

\RDD{minimum height}=1.5cm
&  
\RDD{minimum width}=3cm
\\ \hline  
\tikz \draw (0,0) node[fill=red!20,minimum size=1.5cm,draw] {texte};
&  
\tikz \draw (0,0) node[fill=red!20,minimum size=1.5cm,draw,circle] {texte};

\\ \hline 
\RDD{minimum size}=1.5cm,draw
&  
\RDD{minimum size}=1.5cm,circle

\\ \hline 
\end{tabular} 

\newpage

\SbSSCT{Dans un n\oe ud à formes géométriques}{Geometric Shapes nodes}

\label{lib-geom}
\label{formes}


 \maboite{\BS{usetikzlibrary}\AC{shapes.geometric}}
 
 
\begin{center}
\RRR{67-3}
\end{center}

\SbSbSSCT{Formes disponibles}{Available shapes}

\label{nd1}

\begin{tabular}{|c|c|c|c|} \hline  
\multicolumn{4}{|l|}{ 2 syntaxes :   }\\ 
\multicolumn{4}{|l|}{ \BS{tikz} \BS{node}[fill=green!20,\RDD{shape}=diamond,draw,blue] \AC{texte};   }\\ 
\multicolumn{4}{|l|}{ \BS{tikz} \BS{node}[fill=green!20,\RDD{diamond},draw] \AC{texte};   }\\ 
\hline 
\tikz  \node[fill=green!20,diamond,draw] {texte}; 
&  
\tikz  \node[fill=green!20,ellipse,draw] {texte};
&  
\tikz  \node[fill=green!20,trapezium, regular polygon sides=6,draw] {texte};
&
\tikz  \node[fill=green!20,semicircle,draw] {texte}; 
\\ \hline 
diamond & ellipse  & trapezium & semicircle
\\ \hline 
\tikz  \node[fill=green!20,star,draw] {texte};
&  
\tikz  \node[fill=green!20,regular polygon,draw] {texte};
&  
\tikz  \node[fill=green!20,isosceles triangle,draw] {texte};
&
\tikz  \node[fill=green!20,kite,draw] {texte};
\\ \hline 
star & regular polygon  & isosceles triangle & kite 
\\ \hline 
\tikz  \node[fill=green!20,dart,draw] {texte};
&
\tikz  \node[fill=green!20,circular sector,draw] {texte};
&
\tikz  \node[fill=green!20,cylinder,draw] {texte};
&

\\ \hline 
dart & circular sector & cylinder &
\\ \hline 
\end{tabular} 

\subsubsection{Options}

\begin{tabular}{|c|c|c|} \hline
\multicolumn{3}{|c|}{  \BS{node} [trapezium,draw,\RDD{trapezium left angle}=90,draw,blue] \AC{texte};   }\\ 
\hline
\begin{tikzpicture}
\node[trapezium,draw,red,dashed] {texte};
\node[trapezium,draw,trapezium left angle=90,draw,blue] {texte};
\end{tikzpicture}
& 
\begin{tikzpicture}
\node[trapezium,draw,red,dashed] {texte};
\node[trapezium,draw,trapezium right angle=90,draw,blue] {texte};
\end{tikzpicture} 
& 
\begin{tikzpicture}
\node[trapezium,draw,red,dashed] {texte};
\node[trapezium,draw,trapezium angle=120,draw,blue] {texte};
\end{tikzpicture} 
\\ \hline
\RDD{trapezium left angle}=90  & \RDD{trapezium right angle}=90  & \RDD{trapezium  angle}=120 \\ 
\hline 
\begin{tikzpicture}
\node[trapezium,draw,red,dashed] {texte};
\node[trapezium,draw,minimum height=1.5cm,trapezium stretches=true,draw,blue] {texte};
\end{tikzpicture}
& 
\begin{tikzpicture}
\node[trapezium,draw,red,dashed] {texte};
\node[trapezium,draw,minimum height=1.5cm,trapezium stretches=false,draw,blue] {texte};
\end{tikzpicture} 
& 
\begin{tikzpicture}
\node[trapezium,draw,red,dashed] {texte};
\node[trapezium,draw,minimum width=3cm,trapezium stretches =false,draw,blue] {texte};
\end{tikzpicture} 

\\ \hline
minimum height=1.5cm & minimum height=1.5cm & minimum width=1.5cm \\
\RDD{trapezium stretches}=true & \RDD{trapezium stretches}=false & \RDD{trapezium stretches}  \\ 
\hline

\end{tabular} 


\bigskip
\begin{tabular}{|c|c|c|} \hline
\multicolumn{3}{|c|}{ \BS{tikz} \BS{node} [fill=green!20,star,\RDD{star points}=6,draw] \AC{texte};   }\\ 
\hline
\begin{tikzpicture}
\node[star,draw,red,dashed] {texte};
\node[star,star points=7,draw,blue] {texte};
\end{tikzpicture}
&  
\begin{tikzpicture}
\node[star,draw,red,dashed] {texte};
\node[star,star point height = 2cm,draw,blue] {texte};
\end{tikzpicture} 
&  
\begin{tikzpicture}
\node[star,draw,red,dashed] {texte};
\node[star,star point ratio = 3,draw,blue] {texte};
\end{tikzpicture} 
\\ \hline  
\RDD{star points}=7 & \RDD{star point height} = 2cm & \RDD{star point ratio} = 3 \\ \hline
\dft{5} & \dft.5cm &  \dft{1.5}\\ 
\hline 
\end{tabular} 
\bigskip

\begin{tabular}{|c|c|c|} \hline
\multicolumn{3}{|c|}{  \BS{node} [isosceles triangle,\RDD{isosceles triangle apex angle}=90,draw,blue] \AC{texte};   }\\ 
\multicolumn{3}{|c|}{  \BS{node} [regular polygon, \RDD{regular polygon sides}=6,draw,blue] \AC{texte};   }\\ 
\hline
\begin{tikzpicture}
\node[isosceles triangle,draw,red,dashed] {texte};
 \node[isosceles triangle,isosceles triangle apex angle=90,draw,blue] {texte};
\end{tikzpicture} 
& 
\begin{tikzpicture}
\node[isosceles triangle,draw,red,dashed] {texte};
 \node[isosceles triangle,isosceles triangle stretches=true,draw,blue] {texte};
\end{tikzpicture}
&  
\begin{tikzpicture}
\node[regular polygon,draw,red,dashed] {texte};
\node[regular polygon, regular polygon sides=6,draw,blue] {texte};
\end{tikzpicture} 
\\ \hline  
\RDD{isosceles triangle apex angle}=90 & \RDD{isosceles triangle stretches} & \RDD{regular polygon sides}=6 \\ 
\hline 
\end{tabular} 
\bigskip

\begin{tabular}{|c|c|c|} \hline 
\multicolumn{3}{|c|}{  \BS{node} [kite,\RDD{kite upper vertex angle}=90,draw,blue] \AC{texte};   }\\ 
\hline 
\begin{tikzpicture}
\node[red,kite,draw,dashed] {texte} ;
 \node[kite,kite upper vertex angle=90,draw,blue] {texte};
\end{tikzpicture} 
&  
\begin{tikzpicture}
\node[red,kite,draw,dashed] {texte} ;
 \node[kite,kite lower vertex angle=90,draw,blue] {texte};
\end{tikzpicture} 
&  
\begin{tikzpicture}
\node[red,kite,draw,dashed] {texte} ;
\node[kite,kite vertex angles=90,draw,blue] {texte};
\end{tikzpicture} 
\\ \hline  
\RDD{kite upper vertex angle}=90 & \RDD{kite lower vertex angle}=90 &\RDD{kite vertex angles}=90
\\ \hline 
initially 120 & initially 60 &  \\ 
\hline 
\end{tabular} 

\bigskip

\begin{tabular}{|c|c|c|} \hline
\multicolumn{3}{|c|}{  \BS{node} [dart,\RDD{dart tip angle}=90,draw,blue] \AC{texte};   }\\ 
\hline 
\begin{tikzpicture}
\node[dart,draw,red,dashed] {texte};
\node[dart,dart tip angle=90,draw,blue] {texte};
\end{tikzpicture} 
&  
\begin{tikzpicture}
\node[dart,draw,red,dashed] {texte};
\node[dart,dart tail angle=90,draw,blue] {texte};
\end{tikzpicture} 
&  
\begin{tikzpicture}
\node[,circular sector,draw,red,dashed] {texte};
\node[circular sector,circular sector angle=90,draw,blue] {texte};
\end{tikzpicture} 
\\ \hline  
\RDD{dart tip angle}=90 & \RDD{dart tail angle}=90  & \RDD{circular sector angle}=90
\\ \hline  
initially 45 & initially 135 & initially 60  \\ 
\hline 
\end{tabular} 

\bigskip

\begin{tabular}{|c|c|} \hline  
\multicolumn{2}{|c|}{  \BS{node} [cylinder,\RDD{aspect=2},draw,blue] \AC{texte};   }\\ 
\hline
\tikz  \node[cylinder,aspect=2,draw,blue] {texte};
& 
 \tikz  \node[cylinder,aspect=4,draw,blue] {texte};
\\ \hline 
\RDD{aspect}=2 & \RDD{aspect}=4 
\\ \hline
\tikz  \node[cylinder,cylinder uses custom fill, cylinder end fill=yellow,draw,blue] {texte};
&  
\tikz  \node[cylinder,cylinder uses custom fill, cylinder body fill=yellow,draw,blue] {texte};
\\ \hline
\RDD{cylinder uses custom fill}, & \RDD{cylinder uses custom fill}, \\ 
\RDD{cylinder end fill}=yellow & \RDD{cylinder body fill}=yellow  \\ 
\hline 
\end{tabular} 

\bigskip

\begin{tabular}{|c|c|c|c|} \hline 
\multicolumn{4}{|c|}{  \BS{draw}(0,0) node[\RDD{shape aspect}=1,diamond,draw]  \AC{texte} ;   }
\\ \hline
 
\tikz \draw (0,0) node[shape aspect=1,diamond,draw,blue] {texte};
&  
\tikz \draw (0,-2) node[shape aspect=2,diamond,draw,blue] {texte};
&
\tikz \draw (0,0) node[shape aspect=3,diamond,draw,blue] {texte};
&
\tikz \draw (0,0) node[shape aspect=4,diamond,draw,blue] {texte};
\\ \hline  
\RDD{shape aspect}=1
&  
\RDD{shape aspect}=2
&
\RDD{shape aspect}=3
&
\RDD{shape aspect}=4
\\ \hline 
\end{tabular} 

\bigskip

\begin{tabular}{|c|} \hline 
\BS{draw} node[\rouge {shape border rotate}=30,shape=dart, draw, \rouge {shape border uses incircle}] \AC{texte};
\\ \hline 
\tikz[] \draw node[shape border rotate=30,shape=dart, draw, shape border uses incircle] {texte};
\\ \hline 
\end{tabular} 

\newpage

\SbSSCT{Dans un n\oe ud en forme de symboles}{Symbol Shapes nodes}

\label{lib-symb}

\maboite{\BS{usetikzlibrary}\AC{shapes.symbols}}

\begin{center}
\RRR{67-4}
\end{center}

\SbSbSSCT{Formes disponibles}{Available shapes}

\label{nd2}

\begin{tabular}{|c|c|c|} \hline  
\tikz  \node[fill=green!20,forbidden sign,draw] {texte};
&  
\tikz  \node[fill=green!20,magnifying glass,draw] {texte};
&  
\tikz  \node[fill=green!20,cloud,draw] {texte};
\\ \hline 
forbidden sign & magnifying glass & cloud
\\ \hline  
\tikz  \node[fill=green!20,starburst,draw] {texte};
&  
\tikz  \node[fill=green!20,signal,draw] {texte};

&  
\tikz  \node[fill=green!20,tape,draw] {texte};
\\ \hline 
starburst & signal & tape
\\ \hline 
\end{tabular} 
\bigskip

\subsubsection{Options}

\begin{tabular}{|c|c|c|} \hline  
\multicolumn{3}{|c|}{  \BS{node}[magnifying glass,\RDD{magnifying glass handle angle}=45,draw,blue]  \AC{texte} ;   }
\\ \hline
\tikz  \node[magnifying glass,magnifying glass handle angle=45,draw,blue] {texte};
&  
\tikz  \node[,magnifying glass,magnifying glass handle aspect=3,draw,blue] {texte};
& 
\tikz  \node[magnifying glass,line width=1ex,draw,blue] {texte};

\\ \hline  
\RDD{magnifying glass handle angle}=45 & \RDD{magnifying glass handle aspect}=3  & line width=1ex  
\\ \hline 
\dft{ : -45} & \dft{ : 1.5}& 
\\ \hline 
\end{tabular} 

\bigskip

\begin{tabular}{|c|c|c|c|} \hline 
\multicolumn{4}{|c|}{  \BS{node} [cloud,\RDD{cloud puffs}=5,draw,blue] \AC{texte};   }\\ 
\hline 
\begin{tikzpicture}
\node[cloud,draw,red,dashed] {texte};
\node[cloud,cloud puffs=5,draw,blue] {texte};
\end{tikzpicture} 
&  
\begin{tikzpicture}
\node[cloud,draw,red,dashed] {texte};
\node[cloud,cloud puff arc=270,draw,blue] {texte};
\end{tikzpicture} 
&  
\begin{tikzpicture}
\node[cloud,draw,red,dashed] {texte};
\node[cloud,cloud ignores aspect=true,draw,blue] {texte};
\end{tikzpicture} 
&
\begin{tikzpicture}
\node[cloud,draw,red,dashed] {texte};
\node[cloud,cloud ignores aspect=false,draw,blue] {texte};
\end{tikzpicture} 
\\ \hline  
\RDD{cloud puffs}=5 & \RDD{cloud puff arc}=270 & \RDD{cloud ignores aspect}=false & \RDD{cloud ignores aspect}=true  \\ 
\hline 
\dft :  10 & \dft :  135 &\multicolumn{2}{|c|}{ \dft :  true } \\ \hline
\end{tabular} 

\bigskip

\begin{tabular}{|c|c|c|c|} \hline 
\multicolumn{4}{|c|}{  \BS{node} [starburst,\RDD{starburst points}=5,draw,blue] \AC{texte};   }\\ 
\hline  
\tikz  \node[starburst,starburst points=5,draw,blue] {texte};
&  
\tikz  \node[starburst,starburst point height=1cm,draw,blue] {texte};
&  
\tikz  \node[starburst,random starburst=50,draw,blue] {texte};
&
\tikz  \node[,starburst,random starburst=0,draw,blue] {texte};
\\ \hline  
\RDD{starburst points}=5 & \RDD{starburst point height}=1cm & \RDD{random starburst}=50 & \RDD{random starburst}=0  \\ 
\hline 
\end{tabular} 

\bigskip


\begin{tabular}{|c|c|c|} \hline 
\multicolumn{3}{|c|}{  \BS{node} [signal,\RDD{signal pointer angle}=45,draw,blue] \AC{texte};   }\\ 
\hline 
\tikz  \node[signal,signal pointer angle=45,draw,blue] {texte};
&
\tikz  \node[signal,signal pointer angle=10,draw,blue] {texte};
&
\tikz  \node[signal,signal pointer angle=300,draw,blue] {texte};
\\ \hline 
\RDD{signal pointer angle}=45
&
signal pointer angle=10
&
signal pointer angle=300
\\ \hline 
\multicolumn{3}{|c|}{  \dft{ : signal pointer angle= 90}  }
\\  \hline 

\end{tabular} 
\bigskip

\begin{tabular}{|c|c|c|c|c|} \hline 
\multicolumn{4}{|c|}{  \BS{node} [signal,\RDD{signal to}=above,draw,blue] \AC{texte};   }
\\ \hline 
\tikz  \node[signal,signal to=above,draw,blue] {texte};
&  
\tikz  \node[signal,signal to=below,draw,blue] {texte};
&
\tikz  \node[signal,signal to=right,draw,blue] {texte};
&
\tikz  \node[signal,signal to=above,draw,blue] {texte};
\\ \hline  
  \RDD{signal to}=above  & \RDD{signal to}=below & \RDD{signal to}=right  & \RDD{signal to}=above \\ 
\hline 
\end{tabular} 
\bigskip

\begin{tabular}{|c|c|c|c|c|} \hline 
\multicolumn{4}{|c|}{ \BS{tikz} [signal to=nowhere] \BS{node} [signal,\RDD{signal from=above}=45,draw,blue] \AC{texte};   }\\ 
\hline 
\tikz [signal to=nowhere] \node[signal,signal from=above,draw,blue] {texte};
&  
\tikz [signal to=nowhere] \node[signal,signal from=below,draw,blue] {texte};
&
\tikz [signal to=nowhere] \node[signal,signal from=right,draw,blue] {texte};
&
\tikz [signal to=nowhere] \node[signal,signal from=above,draw,blue] {texte};
\\ \hline  
  \RDD{signal from}=above  & \RDD{signal from}=below & \RDD{signal from}=right  & \RDD{signal from}=above \\ 
\hline 
\end{tabular} 

\bigskip
\begin{tabular}{|c|c|c|c|} \hline
\multicolumn{2}{|c|}{ \tikz  \node[draw,signal, signal from=east , signal to=west,blue] at (0,0) {texte};}
&
\multicolumn{2}{|c|}{ \tikz  \node[draw,signal,signal from=south, signal to=north,blue] at (0,0) {texte};}
\\ \hline 
\multicolumn{2}{|c|}{ \RDD{signal from}=east , \RDD{signal to}=west}
&
\multicolumn{2}{|c|}{\RDD{signal from}=south, \RDD{signal to}=north}

\\ \hline 
\end{tabular}
\bigskip

\begin{tabular}{|c | c | c | c |} \hline
\multicolumn{3}{|c|}{ \BS{tikz} \BS{node}  [tape, draw,\RDD{tape bend top}=out and in] \AC{texte};   }\\ 
\hline  
\tikz \node [tape, draw,tape bend top=out and in,blue] {texte};
&
\tikz \node [tape, draw, tape bend bottom=out and in,blue] {texte};
&
\tikz \node [tape, draw, tape bend bottom=in and in,blue] {texte};
 \\  \hline
 \RDD{tape bend top}=out and in & \RDD{tape bend bottom}=out and in &  \RDD{tape bend bottom}=in and in 
  \\  \hline
 \tikz \node [tape, draw, tape bend top=none,blue] {texte};
 &
 \tikz \node [tape, draw,tape bend top=out and in,tape bend bottom=out and in,blue] {texte};
 &
  \tikz \node [tape, draw,tape bend top=in and out,tape bend bottom=in and out,blue] {texte};
  \\  \hline
 \RDD{tape bend top}=none & \RDD{tape bend bottom}=out and in 	&  \RDD{tape bend bottom}=in and out  \\
 					& \RDD{tape bend top}=out and in 		& \RDD{tape bend top}=in and out  \\
 					& & (\dft{} ) 
  \\  \hline 
\end{tabular}
\bigskip

\begin{tabular}{|c | c | c | c |} \hline
\BS{tikz} \BS{node} [tape, draw, \RDD{tape bend height}=1cm,blue] \AC{texte}; 
  \\  \hline 
\tikz \node [tape, draw, tape bend height=1cm,blue] {texte};

  \\  \hline 
\dft{ : tape bend height = 5pt}
  \\  \hline 
\end{tabular}

\newpage

\SbSSCT{Dans un n\oe ud en forme de flèche}{Arrow Shapes nodes}

\label{lib-arr}

\maboite{\BS{usetikzlibrary}\AC{shapes.arrows}}

\begin{center}
\RRR{67-5}
\end{center}

\SbSbSSCT{Formes disponibles}{Available shapes}
\label{nd3}

\begin{tabular}{|c|c|c|} \hline  
\tikz \node[fill=green!20,single arrow,draw] {texte};
&  
\tikz  \node[fill=green!20,double arrow,draw] {texte};
&  
\tikz  \node[fill=green!20,arrow box,draw] {texte};
\\ \hline 
single arrow & double arrow & arrow box \\ 
\hline 
\end{tabular} 

\subsubsection{Options}

\begin{tabular}{|c|c|c|c|c|} \hline  
 \multicolumn{5}{|c|}{  \BS{node}[single arrow,draw,\RDD{single arrow tip angle}=45] \AC{texte};   }\\ 
  \multicolumn{5}{|c|}{  \BS{node}[single arrow,draw,\RDD{single arrow head extend}=.75cm] \AC{texte};   }\\
 \hline
\begin{tikzpicture}
 \node[single arrow,draw,red,dashed,text=black] {texte};
 \node[single arrow,draw,single arrow tip angle=45,blue] {texte};
\end{tikzpicture}
&
\begin{tikzpicture}
 \node[single arrow,draw,red,dashed,text=black] {texte};
\node[single arrow,draw,single arrow tip angle=120,blue] {texte};
\end{tikzpicture}
&
\begin{tikzpicture}
 \node[single arrow,draw,red,dashed,text=black] {texte};
 \node[single arrow,draw,single arrow head extend=.75cm,blue] {texte};
\end{tikzpicture}
&
\begin{tikzpicture}
 \node[single arrow,draw,red,dashed,text=black] {texte};
 \node[single arrow,draw,single arrow head extend=0cm,blue] {texte};
 \end{tikzpicture}
 &
 \begin{tikzpicture}
  \node[single arrow,draw,red,dashed,text=black] {texte};
  \node[single arrow,draw,single arrow head extend=-1mm,blue] {texte};
 \end{tikzpicture}

\\ \hline
angle=45 & angle=120 & extend=.75cm] & extend=0cm & extend=-1mm
\\ \hline 
\multicolumn{2}{|c|}{  \dft : single arrow tip angle= 90   }
&
\multicolumn{3}{|c|}{  \dft : single arrow head extend=0.5cm   }
\\ \hline 
\end{tabular} 
\bigskip


\begin{tabular}{|c|c|c|c|} \hline
 \multicolumn{4}{|c|}{  \BS{node}[minimum size=2cm,single arrow,draw,\RDD{single arrow head indent}=1cm,blue] \AC{texte};   }\\ 
 \hline   
\begin{tikzpicture}
 \node[minimum size=2cm,single arrow,draw,red,dashed,text=black] {texte};
\node[minimum size=2cm,single arrow,draw,single arrow head indent=1cm,blue] {texte};
\end{tikzpicture}
&
\begin{tikzpicture}
 \node[minimum size=2cm,single arrow,draw,red,dashed,text=black] {texte};
  \node[minimum size=2cm,single arrow,draw,single arrow head indent=10pt,blue] {texte};
  \end{tikzpicture}
&
\begin{tikzpicture}
 \node[minimum size=2cm,single arrow,draw,red,dashed,text=black] {texte};
  \node[minimum size=2cm,single arrow,draw,single arrow head indent=1ex,blue] {texte};
  \end{tikzpicture}
  &
  \begin{tikzpicture}
   \node[minimum size=2cm,single arrow,draw,red,dashed,text=black] {texte};
    \node[minimum size=2cm,single arrow,draw,single arrow head indent=-1ex,blue] {texte};
    \end{tikzpicture}
\\ \hline
indent=1cm & indent=10pt & indent=1ex & indent=-1ex
\\ \hline 
\end{tabular}
\bigskip

 



\begin{tabular}{|c|c|c|c|c|} \hline
 \multicolumn{5}{|c|}{  \BS{node}[minimum size=2cm,double arrow,draw,\RDD{double arrow tip angle}=45] \AC{texte};   }\\ 
  \multicolumn{5}{|c|}{  \BS{node}[minimum size=2cm,double arrow,draw,\RDD{double arrow head extend}=1ex] \AC{texte};   }\\
   \multicolumn{5}{|c|}{  \BS{node}[minimum size=2cm,double arrow,draw,\RDD{double arrow head indent}=1ex] \AC{texte};   }\\ 
 \hline  
\begin{tikzpicture}
\node[minimum size=2cm,double arrow,draw,red,dashed,text=black] {texte};
\node[minimum size=2cm,double arrow,draw,double arrow tip angle=45,blue] {texte};
\end{tikzpicture}
&
\begin{tikzpicture}
\node[minimum size=2cm,double arrow,draw,red,dashed,text=black] {texte};
\node[minimum size=2cm,double arrow,draw,double arrow tip angle=120,blue] {texte};
\end{tikzpicture}
&
\begin{tikzpicture}
 \node[minimum size=2cm,double arrow,draw,red,dashed,text=black] {texte};
 \node[minimum size=2cm,double arrow,draw,double arrow head extend=1ex,blue] {texte};
   \end{tikzpicture}
&
\begin{tikzpicture}
 \node[minimum size=2cm,double arrow,draw,red,dashed,text=black] {texte};
  \node[minimum size=2cm,double arrow,draw,double arrow head extend=0,blue] {texte};
    \end{tikzpicture}
&
\begin{tikzpicture}
 \node[minimum size=2cm,double arrow,draw,red,dashed,text=black] {texte};
  \node[,minimum size=2cm,double arrow,draw,double arrow head indent=1ex,blue] {texte};
    \end{tikzpicture}
\\ \hline 
angle=45 & angle=120 & extend=1ex & extend=0 & indent=1ex
\\ \hline
\end{tabular}

\bigskip

\begin{tabular}{|c|c|c|c|c|} \hline
\multicolumn{4}{|c|}{ \BS{node} [arrow box, draw, \RDD{arrow box arrows}=\AC{north:.25cm}] \AC{texte}; }\\ 
\hline 
\begin{tikzpicture}
\node[arrow box, draw,red,text=white,dashed] {texte};
\node[arrow box, draw, arrow box arrows={north:.25cm},blue] {texte};
\end{tikzpicture}
& 
\begin{tikzpicture}
\node[arrow box, draw,red,text=white,dashed] {texte};
\node[arrow box, draw, arrow box arrows={west:.25cm},blue] {texte};
\end{tikzpicture}
 &
 \begin{tikzpicture}
 \node[arrow box, draw,red,text=white,dashed] {texte};
 \node[arrow box, draw, arrow box arrows={south:.25cm},blue] {texte};
 \end{tikzpicture}
&
 \begin{tikzpicture}
 \node[arrow box, draw,red,text=white,dashed] {texte};
 \node[arrow box, draw, arrow box arrows={east:.25cm},blue] {texte};
 \end{tikzpicture}   
 \\ \hline
\AC{north:.25cm} & \AC{west:.25cm} & \AC{south:.25cm}& \AC{east:.25cm} 
\\ \hline
\multicolumn{4}{|c|}{  \dft{} : 0.5 cm}
 \\ \hline 
 \end{tabular}
 
 
 \bigskip
 
 \begin{tabular}{|c|c|} \hline
 \multicolumn{2}{|c|}{ \BS{node} [arrow box, draw, \RDD{arrow box tip angle}=45] \AC{texte}; }\\ 
 \hline 
  \begin{tikzpicture}
  \node[arrow box, draw,red,text=white,dashed] {texte};
  \node[arrow box, draw, arrow box tip angle=45,blue] {texte};
  \end{tikzpicture} 
  &
    \begin{tikzpicture}
   \node[arrow box, draw,red,text=white,dashed] {texte};
   \node[arrow box, draw, arrow box head extend=.25cm,blue] {texte};
   \end{tikzpicture}
\\ \hline  
\RDD{arrow box tip angle}=45 & \RDD{arrow box head extend}=.25cm
\\ \hline 
\dft : 90  & \dft : 0.125cm 
\\ \hline 
   \begin{tikzpicture}
   \node[arrow box, draw,red,text=white,dashed] {texte};
   \node[arrow box, draw, arrow box head indent=.25cm,blue] {texte};
   \end{tikzpicture} 
 &
    \begin{tikzpicture}
    \node[arrow box, draw,red,text=white,dashed] {texte};
    \node[arrow box, draw,arrow box shaft width=.25cm,blue] {texte};
    \end{tikzpicture} 
 \\ \hline 
\RDD{arrow box head indent}=.25cm  &  \RDD{arrow box shaft width}=.25cm
 \\ \hline  
 \dft{ : 0cm } &  \dft{ : 0.125cm }
 \\ \hline  
 \end{tabular}

\newpage

\SbSSCT{Dans un n\oe ud en forme de bulle}{Callout Shapes nodes}
\label{lib-call}

 \maboite{\BS{usetikzlibrary}\AC{shapes.callouts}}
 
\begin{center}
\RRR{67-7}
\end{center}

\SbSbSSCT{Formes disponibles}{Available shapes}

\begin{tabular}{|c|c|c|} \hline 
\tikz  \node[fill=green!20,ellipse callout,draw] {texte};
 &  
 \tikz  \node[fill=green!20,rectangle callout,draw] {texte};
  &  
  \tikz  \node[fill=green!20,cloud callout,draw] {texte};
 \\ \hline
 ellipse callout  &  rectangle callout  & cloud callout \\ 
\hline 
\end{tabular} 

\subsubsection{Options}


\begin{tabular}{|c | c | c | c |} \hline
\multicolumn{4}{|c|}{  \BS{node} [rectangle callout,draw,\RDD{callout absolute pointer}={(0,1)}] at (2,1) \AC{texte};   }\\ 
\hline 
\begin{tikzpicture} 
\draw [help lines] grid(3,3);
\node [rectangle callout,draw,blue, callout relative pointer={(0,1)}] at (2,1) {texte};
\end{tikzpicture}
&
\begin{tikzpicture} 
\draw [help lines] grid(3,3);
\node [ellipse callout,draw, callout relative pointer={(0,1)},blue] at (2,1) {texte};
\end{tikzpicture}
&
\begin{tikzpicture} 
\draw [help lines] grid(3,3);
\node [rectangle callout,draw,blue,callout absolute pointer={(0,1)}] at (2,1) {texte};
\end{tikzpicture}
&
\begin{tikzpicture} 
\draw [help lines] grid(3,3);
\node [ellipse callout,draw, callout absolute pointer={(0,1)},blue] at (2,1) {texte};
\end{tikzpicture}
 \\  \hline
\multicolumn{2}{|c|}{ \RDD{callout relative pointer}=\AC{(0,1)} } & 
\multicolumn{2}{|c|}{  \RDD{callout absolute pointer}=\AC{(0,1)} }
 \\  \hline 
 \begin{tikzpicture} 
 \draw [help lines] grid(3,3);
 \node [rectangle callout,draw, callout relative pointer={(0,1)},callout pointer shorten=.5cm,blue] at (2,1) {texte};
 \end{tikzpicture}
 &
  \begin{tikzpicture} 
  \draw [help lines] grid(3,3);
  \node [ellipse callout,draw, callout relative pointer={(0,1)},callout pointer shorten=.5cm,blue] at (2,1) {texte};
  \end{tikzpicture}
  &
 \begin{tikzpicture} 
 \draw [help lines] grid(3,3);
 \node [rectangle callout,draw, callout absolute pointer={(0,1)},callout pointer shorten=.5cm,blue] at (2,1) {texte};
 \end{tikzpicture}
  &
  \begin{tikzpicture} 
  \draw [help lines] grid(3,3);
  \node [ellipse callout,draw, callout absolute pointer={(0,1)},callout pointer shorten=.5cm,blue] at (2,1) {texte};
  \end{tikzpicture}
  \\  \hline
\multicolumn{4}{|c|}{ \RDD{callout pointer shorten}=.5cm} 
  \\  \hline 
\end{tabular}


\bigskip

\begin{tabular}{|c | c | c | c |} \hline
\multicolumn{3}{|c|}{  \BS{node} [ellipse callout,draw,\RDD{callout pointer arc}=1] at (0,1.5) \AC{texte};   }\\ 
\hline
\begin{tikzpicture}
\node[ellipse callout,draw, callout pointer arc=1,blue] at (0,1.5) {texte};
\end{tikzpicture}
&
\begin{tikzpicture}
\node[ellipse callout,draw, callout pointer arc=30,blue] at (0,1.5) {texte};
\end{tikzpicture}
 &
\begin{tikzpicture}
\node[ellipse callout,draw, callout pointer arc=90,blue] at (0,1.5) {texte};
\end{tikzpicture}
  \\  \hline 
   callout pointer arc=1 & callout pointer arc=30 & callout pointer arc=90
  \\  \hline  
  \multicolumn{3}{|c|}{  \dft{ : callout pointer arc=15}}
 \\  \hline  
 \end{tabular}

\bigskip

\begin{tabular}{|c | c | c | c |} \hline
\multicolumn{3}{|c|}{  \BS{node}[draw,cloud callout, aspect=2.5] \AC{texte};   }\\ 
\hline 
 \begin{tikzpicture}
  \node[draw,cloud callout, dashed,red,text=black] {texte};
 \node[draw,cloud callout, cloud puffs=5,blue] {texte};
 \end{tikzpicture}
&
 \begin{tikzpicture}
 \node[draw,cloud callout, dashed,red,text=black] {texte};
 \node[draw,cloud callout, aspect=2.5,blue] {texte};
 \end{tikzpicture}
&
  \begin{tikzpicture}
  \node[draw,cloud callout, dashed,red,text=black] {texte};
  \node[draw,cloud callout,cloud puff arc=120,blue] {texte};
  \end{tikzpicture}
   \\  \hline 
cloud puffs=5 & aspect=2.5 &  cloud puff arc=120
\\  \hline 
 \end{tabular}

\bigskip

\begin{tabular}{|c | c | c | c |c |} \hline
\multicolumn{3}{|c|}{  \BS{node} [draw,cloud callout,\RDD{callout pointer start size}=.1] \AC{texte};   }\\ 
\hline 
  \begin{tikzpicture}
  \node[draw,cloud callout, dashed,red,text=black] {texte};
  \node[draw,cloud callout,callout pointer start size=.1,blue] {texte};
  \end{tikzpicture}
&
  \begin{tikzpicture}
  \node[draw,cloud callout, dashed,red,text=black] {texte};
  \node[draw,cloud callout,callout pointer start size=.8cm,blue] {texte};
  \end{tikzpicture}
&
  \begin{tikzpicture}
  \node[draw,cloud callout, dashed,red,text=black] {texte};
 \node[draw,cloud callout,callout pointer start size=1cm and 0.1cm,blue] {texte};
  \end{tikzpicture}
\\  \hline 
\RDD{callout pointer start size}=.1 &start size=.8cm & start size=20pt and 1pt
\\  \hline 
\multicolumn{3}{|c|}{  \dft{} : callout pointer start size =.2 of callout  }
\\ 
\hline 
  \begin{tikzpicture}
  \node[draw,cloud callout, dashed,red,text=black] {texte};
  \node[draw,cloud callout,callout pointer end size=5,blue] {texte};
  \end{tikzpicture}
&
  \begin{tikzpicture}
  \node[draw,cloud callout, dashed,red,text=black] {texte};
  \node[draw,cloud callout,callout pointer end size=.8cm,blue] {texte};
  \end{tikzpicture}
&
    \begin{tikzpicture}
    \node[draw,cloud callout, dashed,red,text=black] {texte};
    \node[draw,cloud callout,callout pointer segments=3,blue] {texte};
    \end{tikzpicture}
\\  \hline 
\RDD{callout pointer end size}=.5 & \RDD{callout pointer end size}=.8cm & \RDD{callout pointer segments}=3
\\  \hline 
\multicolumn{2}{|c|}{  \dft{} : callout pointer start size = .1 of callout  }
& \dft{} : segments=2
\\  \hline  

 \end{tabular}

\newpage


\SbSSCT{Dans un n\oe ud en diverses formes  diverses}{Miscellaneous Shapes nodes}

\label{lib-misc}


 \maboite{\BS{usetikzlibrary}\AC{shapes.misc}}
 
\begin{center}
\RRR{67-8}
\end{center}

\SbSbSSCT{Formes disponibles}{Available shapes}

\begin{tabular}{|c|c|c|c|} \hline  
\tikz  \node[fill=green!20,cross out,draw] {texte};
&  
\tikz  \node[fill=green!20,strike out,draw] {texte};
&  
\tikz  \node[fill=green!20,rounded rectangle,draw] {texte};
&  
\tikz  \node[fill=green!20,chamfered rectangle,draw] {texte};
\\ \hline  
cross out & strike out & rounded rectangle & chamfered rectangle \\ 
\hline 
\end{tabular} 


\subsubsection{Options}

\paragraph{Options \TFRGB{pour}{for} \og rounded rectangle \fg} :


\begin{tabular}{|c|c|c|c|c|} \hline
\multicolumn{5}{|c|}{  \BS{node} [draw, rounded rectangle,\RDD{rounded rectangle arc length}=270] \AC{texte};   }\\ 

\hline 

\tikz \node[draw, rounded rectangle,rounded rectangle arc length=270,blue] {texte}; 
&
\tikz \node[draw, rounded rectangle,rounded rectangle arc length=180,blue]  {texte}; 
&
\tikz \node[draw, rounded rectangle,rounded rectangle arc length=120,blue] {texte}; 
&
\tikz \node[draw, rounded rectangle,rounded rectangle arc length=90,blue]  {texte}; 
&
\tikz \node[draw, rounded rectangle,rounded rectangle arc length=45,blue] {texte}; 
 \\ \hline 
270 & 180 & 120 & 90& 45 
\\ \hline 


\end{tabular} 

\bigskip


\begin{tabular}{|c|c|c|c|} \hline 
\multicolumn{4}{|c|}{  \BS{node} [draw, rounded rectangle,\RDD{rounded rectangle west arc}=concave] \AC{texte};   }\\ 
\multicolumn{4}{|c|}{  \BS{node} [draw, rounded rectangle,\RDD{rounded rectangle left arc}=concave] \AC{texte};   }\\ 
\hline 
\tikz \node[draw, rounded rectangle,rounded rectangle west arc=concave,blue] {texte}; 
&
\tikz \node[draw, rounded rectangle,rounded rectangle left arc=concave,blue] {texte}; 
&
\tikz \node[draw, rounded rectangle,rounded rectangle west arc=convex,blue] {texte}; 
&
\tikz \node[draw, rounded rectangle,rounded rectangle left arc=none,blue] {texte};
 \\\hline 
concave & convex & none 
 \\\hline 
\end{tabular} 

\bigskip

\begin{tabular}{|c|c|c|c|} \hline 
\multicolumn{3}{|c|}{  \BS{node} [draw, rounded rectangle,\RDD{rounded rectangle east arc}=concave] \AC{texte};   }\\ 
\multicolumn{3}{|c|}{  \BS{node} [draw, rounded rectangle,\RDD{rounded rectangle right arc}=concave] \AC{texte};   }\\ 

\hline 
\tikz \node[draw, rounded rectangle,rounded rectangle east arc=concave,blue] {texte}; 
&
\tikz \node[draw, rounded rectangle,rounded rectangle  east arc=convex,blue] {texte}; 
&
\tikz \node[draw, rounded rectangle,rounded rectangle right arc=none,blue] {texte};
 \\\hline 
concave & convex & none 
 \\\hline 
\end{tabular} 

\paragraph{Options  \TFRGB{pour}{for} \og chamfered rectangle \fg} :


\begin{tabular}{|c|c|c|c|} \hline 
\multicolumn{4}{|c|}{  \BS{node} [draw, chamfered rectangle,\RDD{chamfered rectangle angle}=30] \AC{texte};   }\\ 
\hline 
\tikz \node[draw, chamfered rectangle,chamfered rectangle angle=10,blue] {texte}; 
&
\tikz \node[draw, chamfered rectangle,chamfered rectangle angle=30,blue] {texte}; 
&
\tikz \node[draw,chamfered rectangle,chamfered rectangle angle=60,blue] {texte};
&
\tikz \node[draw,chamfered rectangle,chamfered rectangle angle=80,blue] {texte};
 \\ \hline 
10 & 30 & 60 & 80
\\ \hline 
\multicolumn{4}{|c|}{  \dft :  45 }
  \\\hline  

\end{tabular}

\bigskip

\begin{tabular}{|c|c|c|c|c|} \hline 
\multicolumn{5}{|c|}{  \BS{node} [draw, chamfered rectangle,\RDD{chamfered rectangle xsep}=10pt] \AC{texte};   }\\ 
\hline 
\tikz \node[draw, chamfered rectangle,chamfered rectangle xsep=0pt,blue] {texte}; 
&
\tikz \node[draw, chamfered rectangle,chamfered rectangle xsep=5pt,blue] {texte}; 
&
\tikz \node[draw, chamfered rectangle,chamfered rectangle xsep=10pt,blue] {texte}; 
&
\tikz \node[draw,chamfered rectangle,chamfered rectangle xsep=-10pt,blue] {texte};
&
\tikz \node[draw,chamfered rectangle,chamfered rectangle xsep=2cm,blue] {texte};
 \\\hline 
  xsep=0pt & xsep=5pt & xsep=10pt & xsep=-10pt  & xsep=2cm
  \\\hline  
\multicolumn{5}{|c|}{  \dft :  0.666ex }
  \\\hline   
\end{tabular}

\bigskip

\begin{tabular}{|c|c|c|c|c|} \hline 
\multicolumn{5}{|c|}{  \BS{node} [draw, chamfered rectangle,\RDD{chamfered rectangle ysep}=10pt] \AC{texte};   }\\ 
\hline 
\tikz \node[draw, chamfered rectangle,chamfered rectangle ysep=0pt,blue] {texte}; 
&
\tikz \node[draw, chamfered rectangle,chamfered rectangle ysep=5pt,blue] {texte}; 
&
\tikz \node[draw,chamfered rectangle,chamfered rectangle ysep=10pt,blue] {texte};
&
\tikz \node[draw,chamfered rectangle,chamfered rectangle ysep=-10pt,blue] {texte};
&
\tikz \node[draw,chamfered rectangle,chamfered rectangle ysep=1cm,blue] {texte};
 \\ \hline 
 ysep=0pt & ysep=5pt & ysep=10pt & ysep=-10pt & ysep=1cm
 \\\hline  
\end{tabular}

\bigskip

\begin{tabular}{|c|c|c|c|c|} \hline 
\multicolumn{5}{|c|}{  \BS{node} [draw, chamfered rectangle,\RDD{chamfered rectangle ysep}=10pt] \AC{texte};   }\\ 
\hline 
\tikz \node[draw, chamfered rectangle,chamfered rectangle sep=0pt,blue] {texte}; 
&
\tikz \node[draw, chamfered rectangle,chamfered rectangle sep=5pt,blue] {texte}; 
&
\tikz \node[draw, chamfered rectangle,chamfered rectangle sep=10pt,blue] {texte}; 

&
\tikz \node[draw, chamfered rectangle,chamfered rectangle sep=-10pt,blue] {texte}; 
&
\tikz \node[draw,chamfered rectangle,chamfered rectangle sep=1cm,blue] {texte};
 \\\hline 
 sep=0pt & sep=5pt & sep=10pt& sep=-10pt & sep=1cm
 \\\hline  
\end{tabular}

\bigskip

\begin{tabular}{|c|c|c|c|} \hline 
\multicolumn{3}{|c|}{  \BS{node} [draw, chamfered rectangle,\RDD{chamfered rectangle corners}=north west] \AC{texte};   }\\ 
\hline
\tikz \node[draw, chamfered rectangle,chamfered rectangle corners=north west,blue] {texte}; 
&
\tikz \node[draw, chamfered rectangle,chamfered rectangle corners={north east, south east},blue] {texte}; 
&
\tikz \node[draw,chamfered rectangle,chamfered rectangle corners={north east, south west},blue] {texte};
 \\ \hline 
 north west & \AC{north east, south east}  & \AC{north east, south west}
 \\ \hline 
\end{tabular}

\newpage

\SbSSCT{N\oe uds à plusieurs parties}{Shapes with Multiple Text Parts}

\label{lib-mult}


 \maboite{\BS{usetikzlibrary}\AC{shapes.multipart}}

\begin{center}
\RRR{67-6}
\end{center}



\begin{tabular}{|c|c|c|c|} \hline 
\multicolumn{4}{|c|}{  \BS{node} [\RDD{circle split},draw,fill=green!20]\AC{haut  \BSS{nodepart}\AC{lower} bas };   }\\ 
\hline 
 
\tikz  \node [circle split,draw,blue,fill=green!20] {haut  \nodepart{lower} bas }; 

&  
\tikz  \node [circle solidus,draw,blue,fill=green!20]{haut  \nodepart{lower} bas };
&  
\tikz  \node [ellipse split,draw,blue,fill=green!20]{texte haut  \nodepart{lower} texte bas };
& 
\tikz  \node [rectangle split,draw,blue,fill=green!20]{haut  \nodepart{lower} bas}; 

\\ \hline 
\RDD{circle split} & \RDD{circle solidus} & \RDD{ellipse split} & \RDD{rectangle split} \\ 
\hline 
\end{tabular} 

 \bigskip
 
 \begin{tabular}{|c|c|}  \hline  
 \begin{tikzpicture} [baseline=0pt]
 \node[rectangle split,rectangle split parts=5,draw,blue,fill=green!20] at(0,0)
 {texte 1
 \nodepart{second}
 texte 2
 \nodepart{four}
 texte 3};
 \end{tikzpicture}
&
\parbox[c]{10cm}{
 \BS{node}[rectangle split,\RDD{rectangle split parts}=5,\\
 draw] \\
 \AC{texte 1 \\
 \BSS{nodepart}\AC{second} texte 2 \\
 \BSS{nodepart}\AC{four} texte 3}; \\
 \\
\dft : rectangle split parts=4 }
 \\  \hline 
 \end{tabular} 
 
\bigskip

\begin{tabular}{|c|}\hline  
\BS{node} [rectangle split,rectangle split parts=3,\RDD{rectangle split horizontal},draw,blue] \\
\AC{texte1\BSS{nodepart}\AC{two}texte2\BSS{nodepart}\AC{three}texte3};
\\ \hline  
\tikz \node [rectangle split,rectangle split parts=3, rectangle split horizontal,draw,blue]
{texte 1\nodepart{two}texte 2\nodepart{three}texte 3}; 
\\ \hline 
\end{tabular} 
 
 \bigskip
 
% % % <<<<<<<<<<<<<<<<< A Voir rectangle split allocate boxes= >>>>>>>>>>>>>>>>>>>>>>>>>>>>>>>>

% \begin{tikzpicture} [baseline=0pt]%[every text node part/.style={text centered}]
% \node[rectangle split,draw,rectangle split parts=5,fill=green!20,rectangle split allocate boxes=3] at(0,0)
% {texte 1  \nodepart{second}  texte 2  \nodepart{four}  texte 3};
% \end{tikzpicture}
% 
 
\bigskip
 \begin{tabular}{|c|c|}  \hline  
\begin{tikzpicture}[baseline=0pt]
\node[rectangle split, rectangle split parts=3, draw,blue, text width=2.75cm]
{texte 1
\nodepart{two}
texte 2a \\
texte 2b \\
texte 2c
\nodepart{three}
texte 3a \\
texte 3b};
\end{tikzpicture}
&
\parbox{8cm}{
 \BS{node}[rectangle split,\RDD{rectangle split parts}=5, draw] \\
 \AC{texte 1 \\
 \BSS{nodepart}\AC{second} texte 2a  \BS{}\BS{}texte 2b  \BS{}\BS{}  texte 2c \\
 \BSS{nodepart}\AC{three} texte 3a \BS{}\BS{} texte 3b }; \\
}
 \\  \hline 
 \end{tabular} 
\bigskip


 \begin{tabular}{|c|c|}  \hline  
 \multicolumn{2}{|c|}{  \BS{node}[rectangle split, draw,blue,minimum size = 2cm,\RDD{rectangle split draw splits}= true] } \\
  \multicolumn{2}{|c|}{ 
  \AC{texte 1 \BS{nodepart}\AC{two} texte 2 \BS{nodepart}\AC{three} texte 3 \BS{nodepart}\AC{four} texte 4};   }\\ 
 \hline 
\tikz \node[rectangle split, draw,blue,minimum size = 2cm,rectangle split draw splits= true] {texte 1 \nodepart{two} texte 2 \nodepart{three} texte 3 \nodepart{four} texte 4};
&
\tikz \node[rectangle split, draw,blue,minimum size = 2cm,rectangle split draw splits= false] {texte 1 \nodepart{two} texte 2 \nodepart{three} texte 3 \nodepart{four} texte 4};
 \\ \hline
 \RDD{rectangle split draw splits}= true & \RDD{rectangle split draw splits}= false \\
 \dft &
 \\ \hline 
 \end{tabular}
 
\bigskip

 \begin{tabular}{|c|c|}  \hline  
\multicolumn{2}{|c|}{  
\BS{node} [rectangle split,rectangle split parts=3,draw,\RDD{rectangle split ignore empty parts}=false] }\\
 \multicolumn{2}{|c|}{ \AC{texte 1 \BS{nodepart}\AC{second} \BS{nodepart}\AC{third}texte 3};} 
\\ \hline  
\begin{tikzpicture} 
\node[rectangle split,rectangle split parts=3,draw,blue,rectangle split ignore empty parts=false] {texte 1 \nodepart{second} \nodepart{third}texte 3};
\end{tikzpicture}
&
\begin{tikzpicture}
\node[rectangle split,rectangle split parts=3,draw,blue,rectangle split ignore empty parts] 
{texte 1 \nodepart{second} \nodepart{third}texte 3};
\end{tikzpicture}
 \\  \hline 
\RDD{rectangle split ignore empty parts}=false & \RDD{rectangle split ignore empty parts}=true 
\\ \hline
 \end{tabular}
 
\bigskip

 \begin{tabular}{|c|c|}  \hline  
\multicolumn{2}{|c|}{  
\BS{node} [rectangle split,rectangle split parts=3,draw,\RDD{rectangle split empty part depth}=1cm] }\\
 \multicolumn{2}{|c|}{ \AC{texte 1 \BS{nodepart}\AC{second} \BS{nodepart}\AC{third}texte 3};} 
\\ \hline 
\begin{tikzpicture} 
\node[rectangle split,rectangle split parts=3,draw,blue,rectangle split empty part depth=1cm] {texte 1 \nodepart{second} \nodepart{third}texte 3};
\end{tikzpicture}
&
\begin{tikzpicture} 
\node[rectangle split,rectangle split parts=3,draw,blue,text depth=1cm] {texte 1 \nodepart{second} \nodepart{third}texte 3};
\end{tikzpicture}
\\ \hline 
\RDD{rectangle split empty part depth}=1cm & \RDD{text depth}=1cm
\\ \hline
\dft : 0ex & \dft : 0ex
\\ \hline 
\begin{tikzpicture}
\node[rectangle split,rectangle split parts=3,draw,blue,rectangle split empty part  height=1cm] 
{texte 1 \nodepart{second} \nodepart{third}texte 3};
\end{tikzpicture}
&
\begin{tikzpicture}
\node[rectangle split,rectangle split parts=3,draw,blue,text height=1cm] 
{texte 1 \nodepart{second} \nodepart{third}texte 3};
\end{tikzpicture}
\\  \hline 
\RDD{rectangle split empty part height}=1cm & \RDD{text height}=1cm
\\ \hline
\dft : 1ex & \dft : 1ex
\\ \hline 
 \end{tabular}
 
\bigskip



 \begin{tabular}{|c|c|}  \hline 
 \multicolumn{2}{|c|}{ 
 \BS{node} [rectangle split,rectangle split parts=3,draw,\RDD{rectangle split empty part width}=1cm]   \AC{};  } 
 \\ \hline 
\begin{tikzpicture} 
\node[rectangle split,rectangle split parts=3,draw,blue,rectangle split empty part width=2cm]{};
\end{tikzpicture}

&
\begin{tikzpicture} 
\node[rectangle split,rectangle split parts=3,draw,blue]{}; 
\end{tikzpicture}
\\  \hline 
 \RDD{rectangle split empty part width}=2cm  &  \dft : 1ex
\\ \hline
 \end{tabular} 
 
 \bigskip



% % % % <<<<<<<<<< A voir   /pgf/rectangle split use custom fill= (default true) <<<<<<<<<<<<<<<<<<<<<<<<<<<<
 


 \begin{tabular}{|c|c|}  \hline 
 \tikz[baseline=0pt] \node[rectangle split, draw,blue,minimum size = 2cm,rectangle split part align={center, left,right}] {texte 1 \nodepart{two} texte 2 \nodepart{three} texte 3 \nodepart{four} texte 4};
&
\parbox{8cm}{
\BS{node}[rectangle split, draw,blue,minimum size = 2cm,\\
\RDD{rectangle split part align}=\AC{center, left,right}]\\
 \AC{texte 1 \BS{nodepart}\AC{two} texte 2  \\
 \BS{nodepart}\AC{three} texte 3  \BS{nodepart}\AC{four} texte 4};
}
\\ \hline
 \tikz[baseline=0pt] \node[rectangle split, draw,blue,minimum size = 2cm, rectangle split horizontal,rectangle split part align={center,base, top,bottom}] {texte 1 \nodepart{two} texte 2 \nodepart{three} texte 3 \nodepart{four} texte 4};
 &
 \parbox{8cm}{
 \BS{node}[rectangle split, draw,blue,minimum size = 2cm,\\
 rectangle split horizontal,\\
 \RDD{rectangle split part align}=\AC{center,base, top,bottom}]\\
  \AC{texte 1 \BS{nodepart}\AC{two} texte 2  \\
  \BS{nodepart}\AC{three} texte 3  \BS{nodepart}\AC{four} texte 4};
 }
 \\ \hline
 \end{tabular}
 
\bigskip


 \begin{tabular}{|c|c|}  \hline  
\tikz[baseline=0pt] \node[rectangle split, draw,blue, minimum width=1cm,rectangle split part fill={red, green,cyan}]{};
&
\parbox{12cm}{
\BS{node}[rectangle split, draw,blue, minimum width=1cm,\\
 \RDD{rectangle split part fill}=\AC{red, green,cyan}]\AC{};}
\\ \hline
\end{tabular} 

\newpage

\SbSSCT{Mise en forme du texte}{Text attributes}

\subsubsection{Position}

\begin{center}
\RRR{17-4-3}
\end{center}

\begin{tabular}{|c|c|c|c|} \hline  
\multicolumn{4}{|l|}{ \BS{tikz} \BS{draw} (0,0) node[fill=blue!10,\RDD{text width}=2cm,\RDD{text justified}]   }\\ 

\multicolumn{4}{|l|}{ \AC{Ceci est une démonstration d'un texte  sur une largeur de 2cm};  }\\ 
\hline 
\tikz \draw (0,0) node[fill=blue!10,text width=2cm]
{Ceci est une démonstration d'un texte  sur une largeur de 2cm.};
&  
\tikz \draw (0,0) node[fill=blue!10,text width=2cm,text justified]
{Ceci est une démonstration d'un texte  sur une largeur de 2cm};
&  
\tikz \draw (0,0) node[fill=blue!10,text width=2cm,text centered]
{Ceci est une démonstration d'un texte  sur une largeur de 2cm .};
&  
\tikz \draw (0,0) node[fill=blue!10,text width=2cm,text ragged]
{Ceci est une démonstration d'un texte  sur une largeur de 2cm .};
\\  \hline  
\TFRGB{sans}{without} option & \RDD{text justified} & \RDD{text centered }& \RDD{text ragged}   
\\ \hline  
\tikz \draw (0,0) node[fill=blue!10,text width=2cm,text badly ragged]
{Ceci est une démonstration d'un texte  sur une largeur de 2cm.};
&  
\tikz \draw (0,0) node[fill=blue!10,text width=2cm,text badly centered]
{Ceci est une démonstration d'un texte  sur une largeur de 2cm .};
&
\tikz \draw (0,0) node[fill=blue!10,text width=2cm,align=center]
{Ceci est une démonstration d'un texte  sur une largeur de 2cm .};
&
\tikz \draw (0,0) node[fill=blue!10,text width=2cm,align=flush center]
{Ceci est une démonstration d'un texte  sur une largeur de 2cm .};
\\  \hline 
\RDD{text badly ragged} &  \RDD{text badly centered} &  \RDD{align}=center & \RDD{align}=flush center 
\\  \hline 
\tikz \draw (0,0) node[fill=blue!10,text width=2cm,align=justify]
{Ceci est une démonstration d'un texte  sur une largeur de 2cm .};
&
\tikz \draw (0,0) node[fill=blue!10,text width=2cm,align=flush right]
{Ceci est une démonstration d'un texte  sur une largeur de 2cm .};
&
\tikz \draw (0,0) node[fill=blue!10,text width=2cm,align=right]
{Ceci est une démonstration d'un texte  sur une largeur de 2cm .};
&
\tikz \draw (0,0) node[fill=blue!10,text width=2cm,align=flush left]
{Ceci est une démonstration d'un texte  sur une largeur de 2cm .};
\\ \hline 
\RDD{align}=justify & \RDD{align}=flush right &  \RDD{align}=right & \RDD{align}=flush left
\\ \hline 

\end{tabular} 
\bigskip

\begin{tabular}{|c|c|} \hline 
\tikz[baseline=0cm] \node [draw] {
\begin{tabular}{|c|c|} \hline
AAA & BBB \\ \hline
CCC & DDD \\ \hline
\end{tabular}
};
& 
\parbox{8cm}{
\BS{tikz} \BS{node} [draw] \AC{
\BS{begin}\AC{tabular}\AC{|c|c|} \BS{hline} \\
AAA \& BBB \BS{}\BS{} \BS{hline} \\
CCC \& DDD \BS{}\BS{} \BS{hline} \\
\BS{end}\AC{tabular}
};}
\\ \hline 
\end{tabular} 

\bigskip


\begin{tabular}{|c|c|c|}  \hline 
\multicolumn{3}{|c|}{\BS{tikz}[align=left] \BS{node}[draw] \AC{AAA \rouge{ \BS{}\BS{} } BBBBBBBB \rouge{ \BS{}\BS{} } CC};} \\ \hline
\tikz[align=left] \node[draw] {AAA\\BBBBBBBB\\CC};
&  
\tikz[align=center] \node[draw] {AAA\\BBBBBBBB\\CC};
&
\tikz[align=right] \node[draw] {AAA\\BBBBBBBB\\CC};
\\ \hline
[align=left]  & [align=center] &[align=right] 
\\ \hline
\end{tabular} 


\bigskip

\begin{tabular}{|c|c|} \hline 
\multicolumn{2}{|c|}{\BS{tikz}[align=left] \BS{node}[draw] \AC{AAA  \BS{}\BS{} \rouge{[1cm] } BBBBBBBB };} 
\\ \hline 
\rule[-1cm]{0pt}{1,5cm} \tikz[align=left] \node[draw] {AAA\\[1cm]BBBBBBBB\\}; 
& 
\tikz[align=left] \node[draw] {AAA\\[-1cm]BBBBBBBB\\}; 
\\ \hline 
\rouge{ [1cm] } & \rouge{[ -1cm] }
\\ \hline 
\end{tabular} 

\SbSbSSCT{Couleur et fontes }{Colors and Fonts}

\begin{tabular}{|c|c|c|c|c|c|} \hline  
\tikz \draw (0,0) node[text= red]{Texte.};
&
\tikz \draw (0,0) node[font=\itshape]{Texte.};
&
\tikz \draw (0,0) node[font=\slshape]{Texte.};
&
\tikz \draw (0,0) node[font=\scshape]{Texte.};
&
\tikz \draw (0,0) node[font=\upshape]{Texte.};
&
\tikz \draw (0,0) node[font=\bfseries]{Texte.};
\\ \hline 



[text= red] & [font=\BS{itshape}]  & [font=\BS{slshape}] & [font=\BS{scshape}] & [font=\BS{upshape}] & [font=\BS{bfseries}]
\\ \hline 
\end{tabular} 



\bigskip
 
\SbSbSSCT{Taille des fontes}{Font Sizes}

\begin{tabular}{|c|c|c|c|c|c|c|}\hline
\multicolumn{7}{|c|}{ \BS{tikz} \BS{draw} (0,0) node[\RDD{font}=\BS{tiny}]\AC{Texte.}   }
\\  \hline
\tikz \draw (0,0) node[font=\tiny]{Texte.};
&
\tikz \draw (0,0) node[font=\footnotesize]{Texte.};
&
\tikz \draw (0,0) node[font=\small]{Texte.};
&
\tikz \draw (0,0) node[font=\large]{Texte.};
&
\tikz \draw (0,0) node[font=\Large]{Texte.};
&
\tikz \draw (0,0) node[font=\huge]{Texte.};
&
\tikz \draw (0,0) node[font=\Huge]{Texte.};
\\ \hline \BS{tiny} & \BS{footnotesize}  & \BS{small} & \BS{large} & \BS{Large} & \BS{huge} & \BS{Huge} \\ 
\hline 
\end{tabular} 

\bigskip
\begin{center}
\RRR{17-4-4}
\end{center}

\begin{tabular}{|c|c|c|} \hline  
\tikz \draw (0,0) node[fill=blue!10,text height=1cm,draw]{Texte.};
&  
\tikz \draw (0,0) node[fill=blue!10,text depth=1cm,draw]{Texte.};
&  
\tikz \draw (0,0) node[fill=blue!10,text depth=0.5cm,,text height=.5cm,draw]{Texte.};
\\ \hline  
\RDD{text height}=1cm
&  
\RDD{text depth}=1cm
&
\RDD{text height}=0.5cm, \RDD{text depth}=0.5cm
\\ \hline 
\end{tabular} 

\newpage

\SbSSCT{Positions prédéfinies  sur un n\oe ud}{Positions on a node}
\label{nomnoeud}

\SbSbSSCT{pour l'ensemble des n\oe uds}{For all types of node}
\begin{center}
\RRR{17-5-1}
\end{center}

\begin{tabular}{|c|c|c|c|} \hline  
\begin{tikzpicture}
\node[rectangle,draw,minimum size=3cm] (A) at (1,1) {\Huge texte};
\fill[red] (node cs:name=A,anchor=north west) circle (3pt);
\end{tikzpicture}
&
\begin{tikzpicture}
\node[rectangle,draw,minimum size=3cm] (A) at (1,1) {\Huge texte};
\fill[red] (node cs:name=A,anchor=north) circle (3pt);
\end{tikzpicture}
&
\begin{tikzpicture}
\node[rectangle,draw,minimum size=3cm] (A) at (1,1) {\Huge texte};
\fill[red] (node cs:name=A,anchor=north east) circle (3pt);
\end{tikzpicture}
&
\begin{tikzpicture}
\node[rectangle,draw,minimum size=3cm] (A) at (1,1) {\Huge texte};
\fill[red] (node cs:name=A,anchor=text) circle (3pt);
\end{tikzpicture}
\\ \hline 
north west & north & north east & text
\\ \hline 

\begin{tikzpicture}
\node[rectangle,draw,minimum size=3cm] (A) at (1,1) {\Huge texte};
\fill[red] (node cs:name=A,anchor= west) circle (3pt);
\end{tikzpicture}
&
\begin{tikzpicture}
\node[rectangle,draw,minimum size=3cm] (A) at (1,1) {\Huge texte};
\fill[red] (node cs:name=A,anchor=mid  west) circle (3pt);
\end{tikzpicture}
&
\begin{tikzpicture}
\node[rectangle,draw,minimum size=3cm] (A) at (1,1) {\Huge texte};
\fill[red] (node cs:name=A,anchor= base west) circle (3pt);
\end{tikzpicture}
&
\begin{tikzpicture}
\node[rectangle,draw,minimum size=3cm] (A) at (1,1) {\Huge texte};
\fill[red] (node cs:name=A,anchor= base) circle (3pt);
\end{tikzpicture}
\\ \hline 
west & mid west & base west &  base
\\ \hline
 
\begin{tikzpicture}
\node[rectangle,draw,minimum size=3cm] (A) at (1,1) {\Huge texte};
\fill[red] (node cs:name=A,anchor=east) circle (3pt);
\end{tikzpicture}
&
\begin{tikzpicture}
\node[rectangle,draw,minimum size=3cm] (A) at (1,1) {\Huge texte};
\fill[red] (node cs:name=A,anchor=mid east) circle (3pt);
\end{tikzpicture}
&
\begin{tikzpicture}
\node[rectangle,draw,minimum size=3cm] (A) at (1,1) {\Huge texte};
\fill[red] (node cs:name=A,anchor=base east) circle (3pt);
\end{tikzpicture}
&
\begin{tikzpicture}
\node[rectangle,draw,minimum size=3cm] (A) at (1,1) {\Huge texte};
\fill[red] (node cs:name=A,anchor= mid) circle (3pt);
\end{tikzpicture}
\\ \hline 
east & mid esat & base east & mid
\\ \hline 

\begin{tikzpicture}
\node[rectangle,draw,minimum size=3cm] (A) at (1,1) {\Huge texte};
\fill[red] (node cs:name=A,anchor= south east) circle (3pt);
\end{tikzpicture}
&
\begin{tikzpicture}
\node[rectangle,draw,minimum size=3cm] (A) at (1,1) {\Huge texte};
\fill[red] (node cs:name=A,anchor= south) circle (3pt);
\end{tikzpicture}
&
\begin{tikzpicture}                                       
\node[rectangle,draw,minimum size=3cm] (A) at (1,1) {\Huge texte};
\fill[red] (node cs:name=A,anchor= south west) circle (3pt);
\end{tikzpicture}
&
\begin{tikzpicture}
\node[rectangle,draw,minimum size=3cm] (A) at (1,1) {\Huge texte};
\fill[red] (node cs:name=A,anchor=center ) circle (3pt);
\end{tikzpicture}
\\ \hline 
south east & south & south west & center
\\ \hline
 
\begin{tikzpicture}
\node[rectangle,draw,minimum size=3cm] (A) at (1,1) {\Huge texte};
\fill[red] (node cs:name=A,anchor=0) circle (3pt);
\end{tikzpicture}
&
\begin{tikzpicture}
\node[rectangle,draw,minimum size=3cm] (A) at (1,1) {\Huge texte};
\fill[red] (node cs:name=A,anchor=120) circle (3pt);
\end{tikzpicture}
&
\begin{tikzpicture}
\node[rectangle,draw,minimum size=3cm] (A) at (1,1) {\Huge texte};
\fill[red] (node cs:name=A,anchor=-60) circle (3pt);
\end{tikzpicture}
&


\\ \hline 
0 & 120 & -60 &  
\\ \hline 
\end{tabular}
 
\newpage 

\SbSbSSCT{spécifique à un n\oe ud}{Specific to a node}

\TFRGB{Consultez }{see} \RRR{67 }


\begin{tabular}{|c|c|} \hline 
shape=circle & shape=diamond
\\  \hline 
\begin{tikzpicture}[]
\node[circle,draw,minimum size=3.5cm] (A) at (1,1) {\Huge XXX};
\foreach \anchor/\placement in
{north west/above left, north/above, north east/above right,
west/above left, center/above, east/right,
mid west/left, mid/below right, mid east/right,
base west/below left, base/below, base east/below right,
south west/below left, south/below, south east/below right,
text/below, 20/right, 120/above}
\fill[blue,pin position=\placement] (node cs:name=A,anchor= \anchor) circle (2pt) node[blue,pin=\scriptsize{ \anchor} ] {} ;
\end{tikzpicture}
&
\begin{tikzpicture}[]
\node[diamond,draw,minimum size=3.5cm] (A) at (1,1) {\Huge XXX};
\foreach \anchor/\placement in
{north west/above left, north/above, north east/above right,
west/left, center/above, east/right,
mid/10,
base/below,
south west/below left, south/below, south east/below right,
text/left, 10/right, 120/above}
\fill[blue,pin position=\placement] (node cs:name=A,anchor= \anchor) circle (2pt) node[blue,pin=\scriptsize{ \anchor} ] {} ;
\end{tikzpicture}
\\ \hline 
\end{tabular} 

\bigskip

\begin{tabular}{|c|} \hline 
shape=ellipse
\\  \hline 
\begin{tikzpicture}[]
\node[ellipse,draw,minimum size=3.5cm] (A) at (1,1) {\Huge XXXXXXX};
\foreach \anchor/\placement in
{north west/above left, north/above, north east/above right, west/left, center/above, east/right,
mid west/left, mid/-75, mid east/right,
base west/200, base/-105, base east/-20,
south west/below left, south/below, south east/below right,
text/-75, 10/right, 130/above}
\fill[blue,pin position=\placement] (node cs:name=A,anchor= \anchor) circle (2pt) node[blue,pin=\scriptsize{ \anchor} ] {} ;
\end{tikzpicture}
\\ \hline 
\end{tabular}

\bigskip

\begin{tabular}{|c|} \hline 
shape=trapezium
\\  \hline 
\begin{tikzpicture}[]
\node[ trapezium,draw,minimum size=3cm] (A) at (1,1) {\Huge XXX};
\foreach \anchor/\placement in
{center/120, text/below, mid/-45, base/below, mid west/left, base west/-175, mid east/right, base east/-25,
west/175, east/above, north/-75, south/-60,
north west/above, north east/above,
south west/-150, south east/-30, 150/above}
\fill[blue,pin position=\placement] (node cs:name=A,anchor= \anchor) circle (2pt) node[blue,pin=\scriptsize{ \anchor} ] {} ;

\foreach \anchor/\placement in
{bottom left corner/below, top right corner/right,
top left corner/left, bottom right corner/below,
bottom side/-120, left side/left, right side/right, top side/above}
\fill[red,pin position=\placement] (node cs:name=A,anchor= \anchor) circle (2pt) node[blue,pin=\scriptsize{ \anchor} ] {} ;
\end{tikzpicture}
\\ \hline 
\end{tabular}

\bigskip

\begin{tabular}{|c|} \hline 
shape=semicircle,shape border rotate=0
\\  \hline 
\begin{tikzpicture}[]
\node[ semicircle,shape border rotate=0,draw,minimum size=3cm] (A) at (1,1) {\Huge XXX};
\foreach \anchor/\placement in
{center/above, base/-160, mid/-40, text/left, base west/-120, base east/-60, mid west/left, mid east/right, north/below, south/-75, east/60, west/120, north west/above left, north east/above right, south west/-140, south east/-60, 30/right}
\fill[blue,pin position=\placement] (node cs:name=A,anchor= \anchor) circle (2pt) node[blue,pin=\scriptsize{ \anchor} ] {} ;
\foreach \anchor/\placement in
{apex/above, arc start/-60, arc end/-120, chord center/-100}
\fill[red,pin position=\placement] (node cs:name=A,anchor= \anchor) circle (2pt) node[blue,pin=\scriptsize{ \anchor} ] {} ;
\end{tikzpicture}
\\ \hline 
\end{tabular}

\bigskip

\begin{tabular}{|c|} \hline 
shape=regular polygon
\\  \hline 
\begin{tikzpicture}[]
\node[ regular polygon,draw,minimum size=3cm] (A) at (1,1) {\Huge XXX};
\foreach \anchor/\placement in
{ center/97, text/97 , mid/-30, base/below, 75/above,
west/left, east/right, north/-87, south/-60,
north east/right, south east/right, north west/left, south west/left}
\fill[blue,pin position=\placement] (node cs:name=A,anchor= \anchor) circle (2pt) node[blue,pin=\scriptsize{ \anchor} ] {} ;

\foreach \anchor/\placement in
{corner 1/above, corner 2/left, corner 3/left, corner 4/right, corner 5/right,
side 1/above, side 2/left, side 3/-120, side 4/right, side 5/above}
\fill[red,pin position=\placement] (node cs:name=A,anchor= \anchor) circle (2pt) node[blue,pin=\scriptsize{ \anchor} ] {} ;
\end{tikzpicture}
\\ \hline 
\end{tabular}


\bigskip

\begin{tabular}{|c|} \hline 
shape=star
\\  \hline 
\begin{tikzpicture}[]
\node[  shape=star, star points=5, star point ratio=1.65,draw,minimum size=3cm] (A) at (1,1) {\Huge XXX};
\foreach \anchor/\placement in
{center/above, 
text/below, 
mid/-30, 
base/-80, 
75/above,
west/left, 
east/right, 
north/below, 
south/94,
north east/right, 
south east/right, 
north west/left, 
south west/left}
\fill[blue,pin position=\placement] (node cs:name=A,anchor= \anchor) circle (2pt) node[blue,pin=\scriptsize{ \anchor} ] {} ;

\foreach \anchor/\placement in
{inner point 1/above left, 
inner point 2/left, 
inner point 3/below, 
inner point 4/right,
inner point 5/above right, 
outer point 1/above, 
outer point 2/left, 
outer point 3/left,
outer point 4/right, 
outer point 5/right}
\fill[red,pin position=\placement] (node cs:name=A,anchor= \anchor) circle (2pt) node[blue,pin=\scriptsize{ \anchor} ] {} ;
\end{tikzpicture}
\\ \hline 
\end{tabular}



\bigskip

\begin{tabular}{|c|c|} \hline 
shape= isosceles triangle & shape= kite
\\  \hline 
\begin{tikzpicture}[]
\node[ shape=isosceles triangle,draw,minimum size=3cm] (A) at (1,1) {\Huge XXX};
\foreach \anchor/\placement in
{center/above,text/above,150/left,mid/-5, mid west/left, mid east/right,base/-120, base west/-150, 
base east/below right ,west/left, east/right,  north/above, north west/left, north east/above right,
south /-120 , south east/below right}
\fill[blue,pin position=\placement] (node cs:name=A,anchor= \anchor) circle (2pt) node[blue,pin=\scriptsize{ \anchor} ] {} ;
\foreach \anchor/\placement in
{apex/above, left corner/left, right corner/left,left side/above, right side/below, lower side/160}
\fill[red,pin position=\placement] (node cs:name=A,anchor= \anchor) circle (2pt) node[blue,pin=\scriptsize{ \anchor} ] {} ;
\end{tikzpicture}
&
\begin{tikzpicture}[]
\node[ shape=kite,draw,minimum size=3cm] (A) at (1,1) {\Huge XXX};
\foreach \anchor/\placement in
{center/above, text/85, mid/-85, base/-95,mid west/left, base west/-160, 
mid east/right, base east/-20,west/left, east/right, north/80, south/below left,north west/above left, north east/above right,south west/left, south east/right, 
110/above left}
\fill[blue,pin position=\placement] (node cs:name=A,anchor= \anchor) circle (2pt) node[blue,pin=\scriptsize{ \anchor} ] {} ;
\foreach \anchor/\placement in
{upper vertex/110, 
left vertex/left, 
lower vertex/below right,
right vertex/right, 
upper left side/left, 
upper right side/right,
lower left side/left, 
lower right side/below right}
\fill[red,pin position=\placement] (node cs:name=A,anchor= \anchor) circle (2pt) node[blue,pin=\scriptsize{ \anchor} ] {} ;
\end{tikzpicture}
\\ \hline 
\end{tabular}


\bigskip

\begin{tabular}{|c|c|} \hline 
shape= dart & shape= circular sector
\\  \hline 
\begin{tikzpicture}[]
\node[shape=dart, shape border rotate=90,,draw,minimum size=3cm] (A) at (1,1) {\Huge XXX};
\foreach \anchor/\placement in
{west/left  , east/above right , north/below,south/left,
north west/left, north east/right, south west/below, south east/below,110/above left}
\fill[blue,pin position=\placement] (node cs:name=A,anchor= \anchor) circle (2pt) node[blue,pin=\scriptsize{ \anchor} ] {} ;
\foreach \anchor/\placement in
{tip/above, tail center/right, right tail/below,
left tail/below, right tail/below, left side/above left, right side/above right}
\fill[red,pin position=\placement] (node cs:name=A,anchor= \anchor) circle (2pt) node[blue,pin=\scriptsize{ \anchor} ] {} ;
\end{tikzpicture}
&
\begin{tikzpicture}[]
\node[shape=circular sector,draw,minimum size=3cm] (A) at (1,1) {\Huge XXX};
\foreach \anchor/\placement in
{west/170  , east/right , north/above , south/below, north west/left, north east/above, south west/left, south east/below, 120/left}
\fill[blue,pin position=\placement] (node cs:name=A,anchor= \anchor) circle (2pt) node[blue,pin=\scriptsize{ \anchor} ] {} ;
\foreach \anchor/\placement in
{sector center/above, arc start/above, arc end/below, arc center/190}
\fill[red,pin position=\placement] (node cs:name=A,anchor= \anchor) circle (2pt) node[blue,pin=\scriptsize{ \anchor} ] {} ;
\end{tikzpicture}
\\ \hline 
\end{tabular}



\bigskip

\begin{tabular}{|c|c|} \hline 
shape=cylinder & shape=cloud
\\  \hline 
\begin{tikzpicture}[]
\node[shape=cylinder,draw,minimum size=3cm] (A) at (1,1) {\Huge XXX};
\foreach \anchor/\placement in
{west/170  , east/-10 , north/above , south/below, north west/left, north east/above, south west/left, south east/below, 120/left}
\fill[blue,pin position=\placement] (node cs:name=A,anchor= \anchor) circle (2pt) node[blue,pin=\scriptsize{ \anchor} ] {} ;
\foreach \anchor/\placement in
{before top/10 , top/10, after top/below right, before bottom/below left, bottom/190, after bottom/above left}
\fill[red,pin position=\placement] (node cs:name=A,anchor= \anchor) circle (2pt) node[blue,pin=\scriptsize{ \anchor} ] {} ;
\end{tikzpicture}
&
\begin{tikzpicture}[]
\node[shape=cloud,draw,minimum size=3cm] (A) at (1,1) {\Huge XXX};
\foreach \anchor/\placement in
{west/west  , east/east , north/below , south/below left, north west/left, north east/above right, south west/left, south east/right, 110/above}
\fill[blue,pin position=\placement] (node cs:name=A,anchor= \anchor) circle (2pt) node[blue,pin=\scriptsize{ \anchor} ] {} ;
\foreach \anchor/\placement in
{puff 1/above, puff 2/above left , puff 3/left, puff 4/left,
puff 5/below left, puff 6/below right, puff 7/below right, puff 8/right,
puff 9/right, puff 10/above}
\fill[red,pin position=\placement] (node cs:name=A,anchor= \anchor) circle (2pt) node[blue,pin=\scriptsize{ \anchor} ] {} ;
\end{tikzpicture}

\\ \hline 
\end{tabular}

\bigskip

\begin{tabular}{|c|} \hline 
shape=starburst
\\  \hline 
\begin{tikzpicture}[]
\node[shape=starburst, starburst points=9, starburst point height=2cm,draw,minimum size=3cm] (A) at (1,1) {\Huge XXX};
\foreach \anchor/\placement in
{west/west  , east/east , north/70 , south/above, north west/below , north east/below, south west/below left, south east/-85, 30/above right}
\fill[blue,pin position=\placement] (node cs:name=A,anchor= \anchor) circle (2pt) node[blue,pin=\scriptsize{ \anchor} ] {} ;
\foreach \anchor/\placement in
{outer point 1/105, outer point 2/above left , 
outer point 3/left, outer point 4/left, 
outer point 5/below, outer point 6/below, 
outer point 7/below, outer point 8/right, 
outer point 9/above,
inner point 1/93, inner point 2/160, 
inner point 3/190, inner point 4/below left, 
inner point 5/below, inner point 6/-85,
inner point 7/-30, inner point 8/above right, 
inner point 9/above}
\fill[red,pin position=\placement] (node cs:name=A,anchor= \anchor) circle (2pt) node[blue,pin=\scriptsize{ \anchor} ] {} ;
\end{tikzpicture}
\\ \hline 
\end{tabular}


\bigskip

\begin{tabular}{|c|} \hline 
shape=signal
\\  \hline 
\begin{tikzpicture}[]
\node[signal,signal from=west,draw,minimum size=3.5cm] (A) at (1,1) {\Huge XXX};
\foreach \anchor/\placement in
{north west/above left, north/above, north east/above right,
west/left, center/above, east/right,
mid west/left, mid/below right, mid east/right,
base west/-160, base/below, base east/below right,
south west/below left, south/below, south east/below right,
text/below, 20/right, 120/above}
\fill[blue,pin position=\placement] (node cs:name=A,anchor= \anchor) circle (2pt) node[blue,pin=\scriptsize{ \anchor} ] {} ;
\end{tikzpicture}
\\ \hline 
\end{tabular}


\bigskip

\begin{tabular}{|c|} \hline 
shape=tape
\\  \hline 
\begin{tikzpicture}[]
\node[tape, tape bend height=1cm,draw,minimum size=3.5cm] (A) at (1,1) {\Huge XXX};
\foreach \anchor/\placement in
{north west/above left, north/above, north east/above right,
west/left, center/above, east/right,
mid west/left, mid/below right, mid east/right,
base west/-160, base/110, base east/below right,
south west/below left, south/below, south east/below right,
text/110, 20/right, 120/above}
\fill[blue,pin position=\placement] (node cs:name=A,anchor= \anchor) circle (2pt) node[blue,pin=\scriptsize{ \anchor} ] {} ;
\end{tikzpicture}
\\ \hline 
\end{tabular}

\begin{tabular}{|c|} \hline 
shape=magnetic tape
\\  \hline 
\begin{tikzpicture}[]
\node[shape=magnetic tape,draw,minimum size=3cm] (A) at (1,1) {\Huge XXX};
\foreach \anchor/\placement in
{west/west  , east/east , north/above , south/below, north west/above left , north east/above right, south west/left, south east/below, 30/above right}
\fill[blue,pin position=\placement] (node cs:name=A,anchor= \anchor) circle (2pt) node[blue,pin=\scriptsize{ \anchor} ] {} ;
\foreach \anchor/\placement in
{tail east/right, tail south east/below right, tail north east/above right}
\fill[red,pin position=\placement] (node cs:name=A,anchor= \anchor) circle (2pt) node[blue,pin=\scriptsize{ \anchor} ] {} ;
\end{tikzpicture}
\\ \hline 
\end{tabular}



\bigskip

\begin{tabular}{|c|} \hline 
shape=single arrow
\\  \hline 
\begin{tikzpicture}[]
\node[shape=single arrow,draw,minimum size=3cm] (A) at (1,1) {\Huge XXXXXX};
\foreach \anchor/\placement in
{west/170  , east/below right , north/above , south/below, north west/above left, north east/above right, south west/below left, south east/below right, 30/east}
\fill[blue,pin position=\placement] (node cs:name=A,anchor= \anchor) circle (2pt) node[blue,pin=\scriptsize{ \anchor} ] {} ;
\foreach \anchor/\placement in
{tip/above right, before tip/above, after tip/below, before head/190 , after head/170, after tail/left, before tail/left, tail/190}
\fill[red,pin position=\placement] (node cs:name=A,anchor= \anchor) circle (2pt) node[blue,pin=\scriptsize{ \anchor} ] {} ;
\end{tikzpicture}
\\ \hline 
\end{tabular}


\bigskip

\begin{tabular}{|c|} \hline 
shape=double arrow
\\  \hline 
\begin{tikzpicture}[]
\node[shape=double arrow, double arrow head extend=1.5cm,,draw,minimum size=3cm] (A) at (1,1) {\Huge XXXXXXXXX};
\foreach \anchor/\placement in
{west/170  , east/-10 , north/above , south/below, north west/above left, north east/above right, south west/below left, south east/below right, 35/above right}
\fill[blue,pin position=\placement] (node cs:name=A,anchor= \anchor) circle (2pt) node[blue,pin=\scriptsize{ \anchor} ] {} ;
\foreach \anchor/\placement in
{before head 1/above right, before tip 1/above, 
tip 1/10, after tip 1/below, 
after head 1/below right, before head 2/below left, 
before tip 2/below left, tip 2/190, 
after tip 2/above left, after head 2/above left}
\fill[red,pin position=\placement] (node cs:name=A,anchor= \anchor) circle (2pt) node[blue,pin=\scriptsize{ \anchor} ] {} ;
\end{tikzpicture}
\\ \hline 
\end{tabular}


\bigskip

\begin{tabular}{|c|} \hline 
shape=arrow box
\\  \hline 
\begin{tikzpicture}[]
\node[shape=arrow box,draw,minimum size=3cm,arrow box arrows={north:2cm from border, south, east:2cm from border, west},arrow box shaft width=1cm,arrow box head extend=0.25cm] (A) at (1,1) {\Huge XXXXXXXXX};
\foreach \anchor/\placement in
{west/right  , east/left , north/below , south/above, north west/left, north east/right, south west/left, south east/right}
\fill[blue,pin position=\placement] (node cs:name=A,anchor= \anchor) circle (2pt) node[blue,pin=\scriptsize{ \anchor} ] {} ;
\foreach \anchor/\placement in
{north arrow tip/above,
south arrow tip/below, 
east arrow tip/right, 
west arrow tip/left,
before north arrow/above left, 
before north arrow head/110, 
before north arrow tip/left,
after north arrow tip/right, 
after north arrow head/70, 
after north arrow/above right,
before south arrow/below right, 
before south arrow head/-70, 
before south arrow tip/right,
after south arrow tip/left, 
after south arrow head/-110, 
after south arrow/below left,
before east arrow/above right, 
before east arrow head/right, 
before east arrow tip/right,
after east arrow tip/right, 
after east arrow head/right, 
after east arrow/below right,
before west arrow/below left, 
before west arrow head/left, 
before west arrow tip/left,
after west arrow tip/west, 
after west arrow head/left, 
after west arrow/above left}
\fill[red,pin position=\placement] (node cs:name=A,anchor= \anchor) circle (2pt) node[blue,pin=\scriptsize{ \anchor} ] {} ;
\end{tikzpicture}
\\ \hline 
\end{tabular}


\bigskip

\begin{tabular}{|c|} \hline 
shape=circle split
\\  \hline 
\begin{tikzpicture}[]
\node[shape=circle split,draw,minimum size=3.5cm](A) at (1,1) {XXX\nodepart{lower}YYY}  ;
\foreach \anchor/\placement in
{north west/above left, north/above, north east/above right,
west/left, center/above, east/right,
mid west/left, mid/below right, mid east/right,
base west/-160, base/110, base east/below right,
south west/below left, south/below, south east/below right,
text/110, 20/right, 120/above}
\fill[blue,pin position=\placement] (node cs:name=A,anchor= \anchor) circle (2pt) node[blue,pin=\scriptsize{ \anchor} ] {} ;
\foreach \anchor/\placement in
{text/left, lower/left}
\fill[red,pin position=\placement] (node cs:name=A,anchor= \anchor) circle (2pt) node[blue,pin=\scriptsize{ \anchor} ] {} ;
\end{tikzpicture}
\\ \hline 
\end{tabular}

\begin{tabular}{|c|} \hline 
shape=circle solidus
\\  \hline 
\begin{tikzpicture}[]
\node[shape=circle solidus,draw,minimum size=3.5cm](A) at (1,1) {XXX\nodepart{lower}YYY}  ;
\foreach \anchor/\placement in
{north west/above left, north/above, north east/above right,
west/left, center/above, east/right,
mid west/left, mid/below right, mid east/right,
base west/-160, base/110, base east/below right,
south west/below left, south/below, south east/below right,
text/110, 20/right, 120/above}
\fill[blue,pin position=\placement] (node cs:name=A,anchor= \anchor) circle (2pt) node[blue,pin=\scriptsize{ \anchor} ] {} ;
\foreach \anchor/\placement in
{text/left, lower/left}
\fill[red,pin position=\placement] (node cs:name=A,anchor= \anchor) circle (2pt) node[blue,pin=\scriptsize{ \anchor} ] {} ;
\end{tikzpicture}
\\ \hline 
\end{tabular}


\bigskip

\begin{tabular}{|c|} \hline 
shape=ellipse split
\\  \hline 
\begin{tikzpicture}[]
\node[shape=ellipse split,draw,minimum size=3.5cm](A) at (1,1) {XXX\nodepart{lower}YYY}  ;
\foreach \anchor/\placement in
{north west/above left, north/above, north east/above right,
west/left, center/above, east/right,
mid west/left, mid/below right, mid east/right,
base west/-160, base/110, base east/below right,
south west/below left, south/below, south east/below right,
text/110, 20/right, 120/above}
\fill[blue,pin position=\placement] (node cs:name=A,anchor= \anchor) circle (2pt) node[blue,pin=\scriptsize{ \anchor} ] {} ;
;
\end{tikzpicture}
\\ \hline 
\end{tabular}

\bigskip

\begin{tabular}{|c|} \hline 
shape=rectangle split
\\  \hline 
\begin{tikzpicture}[]
\node[name=s,shape=rectangle split, rectangle split parts=4,draw,inner ysep=0.75cm](A) at (1,1)
{\nodepart{text}XXXXXXXXXXXXXX\nodepart{two}YYY
\nodepart{three}ZZZ\nodepart{four}four};
\foreach \anchor/\placement in
{north/above, south/below, east/10, west/170,
north west/above, north east/above, south west/below, south east/below,
center/145, 20/right, mid/30, base/-145}
\fill[blue,pin position=\placement] (node cs:name=A,anchor= \anchor) circle (2pt) node[blue,pin=\scriptsize{ \anchor} ] {} ;
\foreach \anchor/\placement in
{text split/10, text split east/0, text split west/180,two split/30, two split east/right, two split west/left,
three split/30, three split east/east, three split west/west,text/-170, text east/east, text west/west,
two/left, two east/east, two west/west,
three/left, three east/east, three west/west,
four/west, four east/east, four west/west
}
\fill[red,pin position=\placement] (node cs:name=A,anchor= \anchor) circle (2pt) node[blue,pin=\scriptsize{ \anchor} ] {} ;
\end{tikzpicture}
\\ \hline 
\end{tabular}

\bigskip

\begin{tabular}{|c|} \hline 
shape=rectangle callout
\\  \hline 
\begin{tikzpicture}[]
\node[shape=rectangle callout, callout relative pointer={(1.5cm,-.5cm)},draw,
callout pointer width=2cm, inner xsep=1cm, inner ysep=.5cm] (A) at (1,1) {\Huge XXXXXXX};
\foreach \anchor/\placement in
{west/west  , east/east , north/above , south/below, north west/west , north east/right, south west/left, south east/right, 25/right}
\fill[blue,pin position=\placement] (node cs:name=A,anchor= \anchor) circle (2pt) node[blue,pin=\scriptsize{ \anchor} ] {} ;
\foreach \anchor/\placement in
{pointer/right}
\fill[red,pin position=\placement] (node cs:name=A,anchor= \anchor) circle (2pt) node[blue,pin=\scriptsize{ \anchor} ] {} ;
\end{tikzpicture}
\\ \hline 
\end{tabular}

\bigskip

\begin{tabular}{|c|} \hline 
shape=ellipse callout
\\  \hline 
\begin{tikzpicture}[]
\node[shape=ellipse callout,draw] (A) at (1,1) {\Huge XXXXXX};
\foreach \anchor/\placement in
{west/west  , east/right , north/above, south/below, north west/above left, north east/above right, south west/below left, south east/below right}
\fill[blue,pin position=\placement] (node cs:name=A,anchor= \anchor) circle (2pt) node[blue,pin=\scriptsize{ \anchor} ] {} ;
\foreach \anchor/\placement in
{pointer/below right}
\fill[red,pin position=\placement] (node cs:name=A,anchor= \anchor) circle (2pt) node[blue,pin=\scriptsize{ \anchor} ] {} ;
\end{tikzpicture}
\\ \hline 
\end{tabular}


\bigskip

\begin{tabular}{|c|} \hline 
shape=cloud callout
\\  \hline 
\begin{tikzpicture}[]
\node[shape=cloud callout,draw,aspect=1.5] (A) at (1,1) {\Huge XXXXXX};
\foreach \anchor/\placement in
{west/west  , east/right , north/below , south/above, north west/above left, north east/above right, south west/below left, south east/below right}
\fill[blue,pin position=\placement] (node cs:name=A,anchor= \anchor) circle (2pt) node[blue,pin=\scriptsize{ \anchor} ] {} ;
\foreach \anchor/\placement in
{puff 1/above, puff 2/above, puff 3/left, puff 4/left,
puff 5/below left, puff 6/below, puff 7/below right, puff 8/right,
puff 9/right, puff 10/above,pointer/below right}
\fill[red,pin position=\placement] (node cs:name=A,anchor= \anchor) circle (2pt) node[blue,pin=\scriptsize{ \anchor} ] {} ;
\end{tikzpicture}
\\ \hline 
\end{tabular}


\bigskip

\begin{tabular}{|c|} \hline 
shape=cross out
\\  \hline 
\begin{tikzpicture}[]
\node[shape=cross out,draw,minimum size=3cm] (A) at (1,1) {\Huge XXXXXXXXXX};
\foreach \anchor/\placement in
{west/west  , east/right , north/above , south/below, north west/above left, north east/above right, south west/below left, south east/below right}
\fill[blue,pin position=\placement] (node cs:name=A,anchor= \anchor) circle (2pt) node[blue,pin=\scriptsize{ \anchor} ] {} ;
\end{tikzpicture}
\\ \hline 
\end{tabular}

\bigskip

\begin{tabular}{|c|} \hline 
shape=rounded rectangle
\\  \hline 
\begin{tikzpicture}[]
\node[shape=rounded rectangle,draw,minimum size=3cm] (A) at (1,1) {\Huge XXXXXXXXXX};
\foreach \anchor/\placement in
{west/west  , east/right , north/above , south/below, north west/above left, north east/above right, south west/below left, south east/below right}
\fill[blue,pin position=\placement] (node cs:name=A,anchor= \anchor) circle (2pt) node[blue,pin=\scriptsize{ \anchor} ] {} ;

\end{tikzpicture}
\\ \hline 
\end{tabular}


\bigskip

\begin{tabular}{|c|} \hline 
shape=chamfered rectangle
\\  \hline 
\begin{tikzpicture}[]
\node[shape=chamfered rectangle,draw,minimum size=3cm, chamfered rectangle sep=.5cm,] (A) at (1,1) {\Huge XXXXXX};
\foreach \anchor/\placement in
{west/west  , east/right , north/above , south/below, north west/above left, north east/above right, south west/below left, south east/below right}
\fill[blue,pin position=\placement] (node cs:name=A,anchor= \anchor) circle (2pt) node[blue,pin=\scriptsize{ \anchor} ] {} ;
\foreach \anchor/\placement in
{before north east/above right, after north east/above right, before south east/below right,after south east/below right, before north west/above left, after north west/above left, before south west/below left,after south west/below left}
\fill[red,pin position=\placement] (node cs:name=A,anchor= \anchor) circle (2pt) node[blue,pin=\scriptsize{ \anchor} ] {} ;
\end{tikzpicture}
\\ \hline 
\end{tabular}

\normalsize


 


\newpage

\section{Decorations}



\subsection{Library \og decorations.pathmorphing \fg}

\label{lib-morph}
\begin{center}
\RRR{48-2}
\end{center}

\subsubsection{\og lineto \fg}

\begin{tabular}{|c|c|c|} \hline  
\begin{tikzpicture}
\draw [dotted,red](0,0) -- (2,2) ;
\draw [decorate,decoration=lineto]
(0,0) -- (2,2) ;
\end{tikzpicture}
&  
\begin{tikzpicture}
\draw [dotted,red] (1,1) circle (1);
\draw [decorate,decoration=lineto]
(1,1) circle (1); 
\end{tikzpicture}
&  
\begin{tikzpicture}
\draw [dotted,red]
(0,0)  arc (0:180:3 and 2) ;
\draw [decorate,decoration=lineto]
(0,0)  arc (0:180:3 and 2);
\end{tikzpicture}
\\ \hline  
(0,0) - - (2,2) & (1,1) circle (1) & (0,0)  arc (0:180:3 and 2)\\ \hline 
\end{tabular}

\subsubsection{ \og  straight zigzag \fg}

\begin{tabular}{|c|c|c|} \hline 
\multicolumn{3}{|c|}{\BSS{draw}[decorate,\RDD{decoration}=\RDDX{straight zigzag}{decoration}] (0,0) - - (2,2) ;}
\\ \hline 
\begin{tikzpicture}
\draw [dotted,red](0,0) -- (2,2) ;
\draw [decorate,decoration=straight zigzag](0,0) -- (2,2) ;
\end{tikzpicture}
&  
\begin{tikzpicture}
\draw [dotted,red] (1,1) circle (1);
\draw [decorate,decoration=straight zigzag](1,1) circle (1); 
\end{tikzpicture}
&  
\begin{tikzpicture}
\draw [dotted,red]
(0,0)  arc (0:180:3 and 2);
\draw [decorate,decoration=straight zigzag]
(0,0)  arc (0:180:3 and 2);
\end{tikzpicture}
\\ \hline  
(0,0) - - (2,2) & (1,1) circle (1) & (0,0)  arc (0:180:3 and 2); \\ 
\hline 
\end{tabular}

\bigskip

\begin{tabular}{|l|c|c|} \hline 
\multicolumn{2}{|c|}{\BSS{draw}[decorate,decoration=\AC{straight zigzag,\RDD{meta-segment length}=2cm}] (0,0) - - (10,0);}& \dft
 \\ \hline 
\RDD{meta-segment length}=2cm
&  
\begin{tikzpicture}[baseline=0pt]
\draw[red!20] (0,-0.5) grid (10,0.5);
\draw[dotted,red] (0,0) -- (10,0);
\draw[decorate,decoration={straight zigzag,meta-segment length=2cm}] (0,0) -- (10,0);
\end{tikzpicture}
& 1cm
\\ \hline  
\RDD{amplitude}=0.5cm
&  
\begin{tikzpicture}[baseline=0pt]
\draw[red!20] (0,-0.5) grid (10,0.5);
\draw[dotted,red] (0,0) -- (10,0);
\draw[decorate,decoration={straight zigzag,amplitude=0.5cm}] (0,0) -- (10,0);
\end{tikzpicture}
& 2.5pt
\\ \hline 
\RDD{segment length}=1cm
& 
\begin{tikzpicture}[baseline=0pt]
\draw[red!20] (0,-0.5) grid (10,0.5);
\draw[dotted,red] (0,0) -- (10,0);
\draw[decorate,decoration={straight zigzag,segment length=1cm}] (0,0) -- (10,0);
\end{tikzpicture}
& 10pt
\\ \hline 
\end{tabular}

\bigskip

\begin{tabular}{|c|c|c|} \hline
\multicolumn{3}{|c|}{ \BSS{draw}[decorate,decoration=}\\
\multicolumn{3}{|c|}{\AC{straight zigzag,\RDD{meta-segment length}=0.5cm}] (1,1) circle (1); }
 \\ \hline 
\begin{tikzpicture}[baseline=0pt]
\draw [dotted,red](1,1) circle (1); 
\draw [decorate,decoration={straight zigzag,meta-segment length=0.5cm}]
(1,1) circle (1); 
\end{tikzpicture}
&  
\begin{tikzpicture}[baseline=0pt]
\draw [dotted,red](1,1) circle (1); 
\draw [decorate,decoration={straight zigzag,amplitude=0.5cm}]
(1,1) circle (1); 
\end{tikzpicture}
&  
\begin{tikzpicture}[baseline=0pt]
\draw [dotted,red](1,1) circle (1); 
\draw [decorate,decoration={straight zigzag,segment length=5pt}]
(1,1) circle (1); 
\end{tikzpicture}
\\ \hline 
 \RDD{meta-segment length}=2cm & \RDD{amplitude}=0.5cm & \RDD{segment length}=5pt 
\\ \hline 
\end{tabular}

\subsubsection{\og random steps \fg }
\label{alea}

\begin{tabular}{|c|c|c|} \hline  
\multicolumn{3}{|c|}{\BSS{draw}[decorate,\RDD{decoration}=\RDDX{random steps}{decoration}] (0,0) - - (2,2) ;}
\\ \hline 
\begin{tikzpicture}
\draw [dotted,red](0,0) -- (2,2) ;
\draw [decorate,decoration=random steps]
(0,0) -- (2,2) ;
\end{tikzpicture}
&  
\begin{tikzpicture}
\draw [dotted,red] (1,1) circle (1);
\draw [decorate,decoration=random steps]
(1,1) circle (1); 
\end{tikzpicture}
&  
\begin{tikzpicture}
\draw [dotted,red]
(0,0)  arc (0:180:3 and 2);
\draw [decorate,decoration=random steps]
(0,0)  arc (0:180:3 and 2);
\end{tikzpicture}
\\ \hline  
(0,0) -- (2,2) & (1,1) circle (1) & (0,0)  arc (0:180:3 and 2)\\ 
\hline 
\end{tabular}

\bigskip

\begin{tabular}{|l|c|c|} \hline 
\multicolumn{2}{|c|}{\BSS{draw}[decorate,decoration=\AC{random steps,\RDD{segment length}=2cm}] (0,0) - - (10,0);} & \dft
\\ \hline 
\RDD{segment length}=2pt
&  
\begin{tikzpicture}[baseline=0pt]
\draw[red!20] (0,-.5) grid (10,.5);
\draw[dotted,red] (0,0) -- (10,0); \draw[decorate,decoration={random steps,segment length=2pt}] (0,0) -- (10,0);
\end{tikzpicture}
& 10pt \\
\RDD{segment length}=1cm
&  
\begin{tikzpicture}[baseline=0pt]
\draw[red!20] (0,-.5) grid (10,.5);
\draw[dotted,red] (0,0) -- (10,0); \draw[decorate,decoration={random steps,segment length=1cm}] (0,0) -- (10,0);
\end{tikzpicture}
&
\\ \hline  
\RDD{amplitude}=0.5cm
&  
\begin{tikzpicture}[baseline=0pt]
\draw[red!20] (0,-.5) grid (10,.5);
\draw[dotted,red] (0,0) -- (10,0); \draw[decorate,decoration={random steps,amplitude=0.5cm}] (0,0) -- (10,0);
\end{tikzpicture}
& 2.5pt
\\ \hline 
\parbox{4cm}{
\RDD{amplitude}=0.5cm\\
,\RDD{segment length}=1cm}
&  
\begin{tikzpicture}[baseline=0pt]
\draw[red!20] (0,-.5) grid (10,.5);
\draw[dotted,red] (0,0) -- (10,0); \draw[decorate,decoration={random steps,amplitude=0.5cm,segment length=1cm}] (0,0) -- (10,0);
\end{tikzpicture}
&
\\ \hline 
\end{tabular} 

\bigskip

\begin{tabular}{|c|c|c|} \hline  
\multicolumn{3}{|c|}{ \BSS{draw}[decorate,decoration= 
\AC{random steps,\RDD{segment length}=2cm}] (1,1) circle (1); }
 \\ \hline 
\begin{tikzpicture}
\draw [dotted,red](1,1) circle (1);
\draw [decorate,decoration={random steps,meta-segment length=2cm}]
(1,1) circle (1); 
\end{tikzpicture}
&  
\begin{tikzpicture}
\draw [decorate,decoration={random steps,amplitude=0.5cm}]
(1,1) circle (1); 
\end{tikzpicture}
&  
\begin{tikzpicture}
\draw [dotted,red](1,1) circle (1);
\draw [decorate,decoration={random steps,segment length=5pt}]
(1,1) circle (1); 
\end{tikzpicture}
\\ \hline 
 \RDD{meta-segment length}=2cm & \RDD{amplitude}=0.5cm & \RDD{segment length}=5pt 
\\ \hline 
\end{tabular} 

\subsubsection{\og saw \fg }

\begin{tabular}{|c|c|c|} \hline 
\multicolumn{3}{|c|}{\BSS{draw}[decorate,\RDD{decoration}=\RDDX{saw}{decoration}] (0,0) - - (2,2) ;}
 \\ \hline 
\begin{tikzpicture}
\draw [dotted,red](0,0) -- (2,2) ;
\draw [decorate,decoration=saw]
(0,0) -- (2,2) ;
\end{tikzpicture}
&  
\begin{tikzpicture}
\draw [dotted,red] (1,1) circle (1);
\draw [decorate,decoration=saw]
(1,1) circle (1); 
\end{tikzpicture}
&  
\begin{tikzpicture}
\draw [dotted,red]
(0,0)  arc (0:180:3 and 2);
\draw [decorate,decoration=saw]
(0,0)  arc (0:180:3 and 2);
\end{tikzpicture}
\\ \hline  
(0,0) - - (2,2) & (1,1) circle (1) & (0,0)  arc (0:180:3 and 2);\\ 
\hline 
\end{tabular}

\bigskip

\begin{tabular}{|l|c|c|} \hline 
\multicolumn{2}{|c|}{\BSS{draw}[decorate,decoration=\AC{saw,\RDD{meta-segment length}=0.5cm}] (0,0) - - (10,0);} & \dft
 \\ \hline 
\RDD{segment length}=0.5cm
&  
\begin{tikzpicture}[baseline=0pt]
\draw[red!20] (0,-0.5) grid (10,0.5);
\draw[dotted,red] (0,0) -- (10,0); \draw[decorate,decoration={saw,segment length=0.5cm}] (0,0) -- (10,0);
\end{tikzpicture}
& 10 pt \\ 
\RDD{segment length}=2cm
&  
\begin{tikzpicture}[baseline=0pt]
\draw[red!20] (0,-0.5) grid (10,0.5);
\draw[dotted,red] (0,0) -- (10,0); \draw[decorate,decoration={saw,segment length=2cm}] (0,0) -- (10,0);
\end{tikzpicture}
&  \\ \hline 
\RDD{amplitude}=0.5cm
&  
\begin{tikzpicture}[baseline=0pt]
\draw[red!20] (0,-0.5) grid (10,0.5);
\draw[dotted,red] (0,0) -- (10,0);\draw[decorate,decoration={saw,amplitude=0.5cm}] (0,0) -- (10,0);
\end{tikzpicture}
& 2.5 pt \\ \hline 
\end{tabular}

\bigskip

\begin{tabular}{|c|c|c|} \hline 
\multicolumn{3}{|c|}{ \BSS{draw}[decorate,decoration=\AC{saw,\RDD{segment length}=20pt}] (1,1) circle (1); }
 \\ \hline  
\begin{tikzpicture}
\draw [dotted,red](1,1) circle (1);
\draw [decorate,decoration={saw,segment length=20pt}]
(1,1) circle (1); 
\end{tikzpicture}
&  
\begin{tikzpicture}
\draw [dotted,red](1,1) circle (1);
\draw [decorate,decoration={saw,segment length=5pt}]
(1,1) circle (1); 
\end{tikzpicture}
&  
\begin{tikzpicture}
\draw [decorate,decoration={saw,amplitude=0.5cm}]
(1,1) circle (1); 
\end{tikzpicture}
\\ \hline 
\RDD{segment length}=20pt & \RDD{segment length}=5pt & \RDD{amplitude}=0.5cm 
\\ \hline 
\end{tabular}

\subsubsection{\og zigzag \fg }

\begin{tabular}{|c|c|c|} \hline  
\multicolumn{3}{|c|}{\BSS{draw}[decorate,\RDD{decoration}=\RDDX{zigzag}{decoration}] (0,0) - - (2,2) ;}
\\ \hline 
\begin{tikzpicture}
\draw [dotted,red](0,0) -- (2,2) ;
\draw [decorate,decoration=zigzag]
(0,0) -- (2,2) ;
\end{tikzpicture}
&  
\begin{tikzpicture}
\draw [dotted,red] (1,1) circle (1);
\draw [decorate,decoration=zigzag]
(1,1) circle (1); 
\end{tikzpicture}
&  
\begin{tikzpicture}
\draw [dotted,red]
(0,0)  arc (0:180:3 and 2);
\draw [decorate,decoration=zigzag]
(0,0)  arc (0:180:3 and 2);
\end{tikzpicture}
\\ \hline  
(0,0) - - (2,2) & (1,1) circle (1) & (0,0)  arc (0:180:3 and 2);\\ 
\hline 
\end{tabular}

\bigskip

\begin{tabular}{|l|c|c|} \hline 
\multicolumn{2}{|c|}{\BSS{draw}[decorate,decoration=\AC{zigzag,\RDD{meta-segment length}=2cm}] (0,0) - - (10,0);} & \dft
 \\ \hline 
\RDD{segment length}=0.5cm
&  
\begin{tikzpicture}[baseline=0pt]
\draw[red!20] (0,-0.5) grid (10,0.5);
\draw[dotted,red] (0,0) -- (10,0);
\draw[decorate,decoration={zigzag,segment length=0.5cm}] (0,0) -- (10,0);
\end{tikzpicture}
& 10pt
\\
\RDD{segment length}=2cm
&  
\begin{tikzpicture}[baseline=0pt]
\draw[red!20] (0,-0.5) grid (10,0.5);
\draw[dotted,red] (0,0) -- (10,0); \draw[decorate,decoration={zigzag,segment length=2cm}] (0,0) -- (10,0);
\end{tikzpicture}
& 
\\ \hline  
\RDD{amplitude}=0.5cm
&  
\begin{tikzpicture}[baseline=0pt]
\draw[red!20] (0,-0.5) grid (10,0.5);
\draw[dotted,red] (0,0) -- (10,0); \draw[decorate,decoration={zigzag,amplitude=0.5cm}] (0,0) -- (10,0);
\end{tikzpicture}
& 2.5 pt
\\ \hline 
\end{tabular}

\bigskip

\begin{tabular}{|c|c|c|} \hline 
 \multicolumn{3}{|c|}{ \BSS{draw}[decorate,decoration= 
 \AC{saw,\RDD{segment length}=20pt }] (1,1) circle (1);}
  \\ \hline  
\begin{tikzpicture}
\draw [dotted,red](1,1) circle (1);
\draw [decorate,decoration={zigzag,segment length=20pt}]
(1,1) circle (1); 
\end{tikzpicture}
&  
\begin{tikzpicture}
\draw [dotted,red](1,1) circle (1);
\draw [decorate,decoration={zigzag,segment length=5pt}]
(1,1) circle (1); 
\end{tikzpicture}
&  
\begin{tikzpicture}
\draw [dotted,red](1,1) circle (1);
\draw [decorate,decoration={zigzag,amplitude=0.5cm}]
(1,1) circle (1); 
\end{tikzpicture}
\\ \hline 
\RDD{segment length}=20pt & \RDD{segment length}=5pt & \RDD{amplitude}=0.5cm 
\\ \hline 
\end{tabular}


\subsubsection{\og bent \fg }

\begin{tabular}{|c|c|c|} \hline  
\begin{tikzpicture}
\draw [dotted,red](0,0) -- (2,2) ;
\draw [decorate,decoration=bent]
(0,0) -- (2,2) ;
\end{tikzpicture}
&  
\begin{tikzpicture}
\draw [dotted,red] (1,1) circle (1);
\draw [decorate,decoration=bent]
(1,1) circle (1); 
\end{tikzpicture}
&  
\begin{tikzpicture}
\draw [dotted,red]
(0,0)  arc (0:180:3 and 2);
\draw [decorate,decoration=bent]
(0,0)  arc (0:180:3 and 2);
\end{tikzpicture}
\\ \hline  
(0,0) - - (2,2) & (1,1) circle (1) & (0,0)  arc (0:180:3 and 2); \\ 
\hline 
\end{tabular}

\bigskip

\begin{tabular}{|l|c|c|} \hline 
\multicolumn{2}{|c|}{\BSS{draw}[decorate,decoration=\AC{bent,\RDD{amplitude}=0.5cm}] (0,0) -- (10,0);} & \dft
 \\ \hline 
\RDD{amplitude}=0.5cm
&  
\begin{tikzpicture}[baseline=0pt]
\draw[red!20] (0,-0.5) grid (10,0.5);
\draw[dotted,red] (0,0) -- (10,0); \draw[decorate,decoration={bent,amplitude=0.5cm}] (0,0) -- (10,0);
\end{tikzpicture}
& 2.5 pt
\\ \hline  
\parbox{4cm}{
\RDD{aspect}=0.1 (en bleue)\\
\RDD{aspect}=0.9 (en vert)\\
amplitude=0.5cm\\
}
&  
\begin{tikzpicture}[baseline=0pt]
\draw[red!20] (0,-0.5) grid (10,0.5);
\draw[dotted,red] (0,0) -- (10,0); \draw[decorate,blue,decoration={bent,aspect=0.1,amplitude=0.5cm}] (0,0) -- (10,0);
\draw[decorate,green,decoration={bent,aspect=0.9,amplitude=0.5cm}] (0,0) -- (10,0);
\end{tikzpicture}
& 0.5
\\ \hline 
\end{tabular}

\bigskip

\begin{tabular}{|c|c|c|} \hline  
\begin{tikzpicture}
\draw [dotted,red](1,1) circle (1);
\draw [decorate,decoration={bent,amplitude=1cm}]
(1,1) circle (1); 
\end{tikzpicture}
&  
\begin{tikzpicture}
\draw [dotted,red](1,1) circle (1);
\draw [decorate,decoration={bent,amplitude=0.5cm}]
(1,1) circle (1); 
\end{tikzpicture}
&  
\begin{tikzpicture}
\draw [dotted,red](1,1) circle (1);
\draw [decorate,decoration={bent,aspect=.25}]
(1,1) circle (1); 
\end{tikzpicture}
\\ \hline 
amplitude=1cm & amplitude=0.5cm & aspect=0.25
\\ \hline 
\end{tabular}


\subsubsection{\og bumps \fg  }

\begin{tabular}{|c|c|c|} \hline
\multicolumn{3}{|c|}{\BSS{draw}[decorate,\RDD{decoration}=\RDDX{bumps}{decoration}] (0,0) - - (2,2) ;}
 \\ \hline 
\begin{tikzpicture}
\draw [dotted,red](0,0) -- (2,2) ;
\draw [decorate,decoration=bumps]
(0,0) -- (2,2) ;
\end{tikzpicture}
&  
\begin{tikzpicture}
\draw [dotted,red] (1,1) circle (1);
\draw [decorate,decoration=bumps]
(1,1) circle (1); 
\end{tikzpicture}
&  
\begin{tikzpicture}
\draw [dotted,red]
(0,0)  arc (0:180:3 and 2);
\draw [decorate,decoration=bumps]
(0,0)  arc (0:180:3 and 2);
\end{tikzpicture}
\\ \hline  
(0,0) - - (2,2) & (1,1) circle (1) & (0,0)  arc (0:180:3 and 2) \\ 
\hline 
\end{tabular}

\bigskip

\begin{tabular}{|l|c|c|} \hline 
\multicolumn{2}{|c|}{\BSS{draw}[decorate,decoration=\AC{bumps,\RDD{amplitude}=0.5cm}] (0,0) - - (10,0);} & \dft
 \\ \hline 
\RDD{amplitude}=0.5cm
&  
\begin{tikzpicture}[baseline=0pt]
\draw[red!20] (0,-0.5) grid (10,0.5);
\draw[dotted,red] (0,0) -- (10,0); \draw[decorate,decoration={bumps,amplitude=0.5cm}] (0,0) -- (10,0);
\end{tikzpicture}
& 2.5 pt
\\ \hline  
\RDD{segment length}=1cm
&  
\begin{tikzpicture}[baseline=0pt]
\draw[red!20] (0,-0.5) grid (10,0.5);
\draw[dotted,red] (0,0) -- (10,0); \draw[decorate,decoration={bumps,segment length=1cm}] (0,0) -- (10,0);
\end{tikzpicture}
& 10 pt
\\ \hline 
\end{tabular}

\bigskip

\begin{tabular}{|c|c|c|} \hline 
\multicolumn{3}{|c|}{ \BSS{draw}[decorate,decoration= 
\AC{bumps,\RDD{amplitude}=10pt}] (1,1) circle (1);}
 \\ \hline 
\begin{tikzpicture}
\draw [dotted,red](1,1) circle (1);
\draw [decorate,decoration={bumps,amplitude=10pt}]
(1,1) circle (1); 
\end{tikzpicture}
&  
\begin{tikzpicture}
\draw [dotted,red](1,1) circle (1);
\draw [decorate,decoration={bumps,amplitude=0.5cm}]
(1,1) circle (1); 
\end{tikzpicture}
&  
\begin{tikzpicture}
\draw [dotted,red](1,1) circle (1);
\draw [decorate,decoration={bumps,segment length=20pt}] (1,1) circle (1); 
\end{tikzpicture}
\\ \hline 
\RDD{amplitude}=10pt & \RDD{amplitude}=0.5cm & \RDD{segment length}=20pt
\\ \hline 
\end{tabular}


\subsubsection{\og coil \fg }

\begin{tabular}{|c|c|c|} \hline 
\multicolumn{3}{|c|}{\BSS{draw}[decorate,\RDD{decoration}=\RDDX{coil}{decoration}] (0,0) - - (2,2) ;}
 \\ \hline  
\begin{tikzpicture}
\draw [dotted,red](0,0) -- (2,2) ;
\draw [decorate,decoration=coil]
(0,0) -- (2,2) ;
\end{tikzpicture}
&  
\begin{tikzpicture}
\draw [dotted,red] (1,1) circle (1);
\draw [decorate,decoration=coil]
(1,1) circle (1); 
\end{tikzpicture}
&  
\begin{tikzpicture}
\draw [dotted,red]
(0,0)  arc (0:180:3 and 2);
\draw [decorate,decoration=coil]
(0,0)  arc (0:180:3 and 2);
\end{tikzpicture}
\\ \hline  
(0,0) - - (2,2) & (1,1) circle (1) & (0,0)  arc (0:180:3 and 2) \\ 
\hline 
\end{tabular}

\bigskip

\begin{tabular}{|l|c|c|} \hline 
\multicolumn{2}{|c|}{\BSS{draw}[decorate,decoration=\AC{coil,\RDD{amplitude}=0.5cm}] (0,0) - - (10,0);} & \dft
 \\ \hline 
\RDD{amplitude}=0.5cm
&  
\begin{tikzpicture}[baseline=0pt]
\draw[red!20] (0,-0.5) grid (10,0.5);
\draw[dotted,red] (0,0) -- (10,0); \draw[decorate,decoration={coil,amplitude=0.5cm}] (0,0) -- (10,0);
\end{tikzpicture}
& 2.5 pt
\\ \hline  
\RDD{segment length}=1cm
&  
\begin{tikzpicture}[baseline=0pt]
\draw[red!20] (0,-0.5) grid (10,0.5);
\draw[dotted,red] (0,0) -- (10,0);
\draw[decorate,decoration={coil,segment length=1cm}] (0,0) -- (10,0);
\end{tikzpicture}
& 10 pt 
\\ \hline
\parbox{4cm}{

\RDD{aspect}=0.1\\
(amplitude=0.5cm)\\
}
&  
\begin{tikzpicture}[baseline=0pt]
\draw[red!20] (0,-0.5) grid (10,0.5);
\draw[dotted,red] (0,0) -- (10,0);
\draw[decorate,decoration={coil,aspect=0.1,amplitude=0.5cm}] (0,0) -- (10,0);
\end{tikzpicture}
& 
\\
\RDD{aspect}=0.3
&  
\begin{tikzpicture}[baseline=0pt]
\draw[red!20] (0,-0.5) grid (10,0.5);
\draw[dotted,red] (0,0) -- (10,0);
\draw[decorate,decoration={coil,aspect=0.3,amplitude=0.5cm}] (0,0) -- (10,0);
\end{tikzpicture}
&  0.5
\\
\RDD{aspect}=0.9
& 
\begin{tikzpicture}[baseline=0pt]
\draw[red!20] (0,-0.5) grid (10,0.5);
\draw[dotted,red] (0,0) -- (10,0);
\draw[decorate,decoration={coil,aspect=0.9,amplitude=0.5cm}] (0,0) -- (10,0);
\end{tikzpicture}
& 
\\ \hline
\end{tabular}

\bigskip

\begin{tabular}{|c|c|c|} \hline  
\multicolumn{3}{|c|}{ \BSS{draw}[decorate,decoration= \AC{coil,\RDD{amplitude}=0.5cm}] (1,1) circle (1);}
 \\ \hline 
\begin{tikzpicture}
\draw [dotted,red](1,1) circle (1);
\draw [decorate,decoration={coil,amplitude=0.5cm}]
(1,1) circle (1); 
\end{tikzpicture}
&  
\begin{tikzpicture}
\draw [dotted,red](1,1) circle (1);
\draw[decorate,decoration={coil,segment length=1cm,amplitude=0.5cm}]
(1,1) circle (1); 
\end{tikzpicture}
&  
\begin{tikzpicture}
\draw [dotted,red](1,1) circle (1);
\draw [decorate,decoration={coil,aspect=.25,amplitude=0.5cm}]
(1,1) circle (1); 
\end{tikzpicture}
\\ \hline 
\RDD{amplitude}=0.5 cm & \RDD{segment length}=1cm & \RDD{aspect}=0.25 \\
& amplitude=0.5cm & amplitude=0.5cm
\\ \hline 
\end{tabular}

\subsubsection{\og curveto \fg }

\begin{tabular}{|c|c|c|} \hline  
\begin{tikzpicture}
\draw [dotted,red](0,0) -- (2,2) ;
\draw [decorate,decoration=curveto]
(0,0) -- (2,2) ;
\end{tikzpicture}
&  
\begin{tikzpicture}
\draw [dotted,red] (1,1) circle (1);
\draw [decorate,decoration=curveto]
(1,1) circle (1); 
\end{tikzpicture}
&  
\begin{tikzpicture}
\draw [dotted,red];
\draw [decorate,decoration=curveto] (0,0)  arc (0:180:3 and 2);
\end{tikzpicture}
\\ \hline  
(0,0) - - (2,2) & (1,1) circle (1) & (0,0)  arc (0:180:3 and 2) 
\\ \hline 
\end{tabular}


\subsubsection{\og snake \fg }

\begin{tabular}{|c|c|c|} \hline 
\multicolumn{3}{|c|}{\BSS{draw}[decorate,\RDD{decoration}=\RDDX{snake}{decoration}] (0,0) - - (2,2) ;}
 \\ \hline   
\begin{tikzpicture}
\draw [dotted,red](0,0) -- (2,2) ;
\draw [decorate,decoration=snake]
(0,0) -- (2,2) ;
\end{tikzpicture}
&  
\begin{tikzpicture}
\draw [dotted,red] (1,1) circle (1);
\draw [decorate,decoration=snake]
(1,1) circle (1); 
\end{tikzpicture}
&  
\begin{tikzpicture}
\draw [dotted,red]
(0,0)  arc (0:180:3 and 2);
\draw [decorate,decoration=snake]
(0,0)  arc (0:180:3 and 2);
\end{tikzpicture}
\\ \hline  
(0,0) - - (2,2) & (1,1) circle (1) &(0,0)  arc (0:180:3 and 2) \\ 
\hline 
\end{tabular}

\bigskip

\begin{tabular}{|l|c|c|} \hline 
\multicolumn{2}{|c|}{\BSS{draw}[decorate,decoration=\AC{snake,\RDD{segment length}=2cm}] (0,0) - - (10,0);} & \dft
 \\ \hline 
\RDD{amplitude}=0.5cm
&  
\begin{tikzpicture}[baseline=0pt]
\draw[red!20] (0,-0.5) grid (10,0.5);
\draw[dotted,red] (0,0) -- (10,0); \draw[decorate,decoration={snake,amplitude=0.5cm}] (0,0) -- (10,0);
\end{tikzpicture}
& 2.5 pt
\\ \hline  
\RDD{segment length}=1cm
&  
\begin{tikzpicture}[baseline=0pt]
\draw[red!20] (0,-0.5) grid (10,0.5);
\draw[dotted,red] (0,0) -- (10,0);
\draw[decorate,decoration={snake,segment length=1cm}] (0,0) -- (10,0);
\end{tikzpicture}
& 10 pt
\\ \hline  
\end{tabular}

\bigskip

\begin{tabular}{|c|c|c|} \hline  
\multicolumn{3}{|c|}{ \BSS{draw}[decorate,decoration=
snake,
\RDD{amplitude}=5pt] (1,1) circle (1);}
 \\ \hline
\begin{tikzpicture}
\draw [dotted,red](1,1) circle (1);
\draw [decorate,decoration={snake,amplitude=5pt}]
(1,1) circle (1); 
\end{tikzpicture}
&  
\begin{tikzpicture}
\draw [dotted,red](1,1) circle (1);
\draw [decorate,decoration={snake,amplitude=0.5cm}]
(1,1) circle (1); 
\end{tikzpicture}
&  
\begin{tikzpicture}
\draw [dotted,red](1,1) circle (1);
\draw [decorate,decoration={snake,,segment length=1cm}]
(1,1) circle (1); 
\end{tikzpicture}
\\ \hline 
\RDD{amplitude}=5pt & \RDD{amplitude}=0.5cm & \RDD{segment length}=5pt
\\ \hline 
\end{tabular}

\newpage
\subsection{Library \og decorations.pathreplacing \fg}


 \maboite{\BS{usetikzlibrary}\AC{decorations.pathreplacing}}
\label{lib-replac}

\begin{center}
\RRR{48-3}
\end{center}

\subsubsection{\og border \fg }

\begin{tabular}{|c|c|c|} \hline
\multicolumn{3}{|c|}{\BSS{draw}[decorate,\RDD{decoration}=\RDDX{border}{decoration}] (0,0) - - (2,2) ;}
 \\ \hline 
\begin{tikzpicture}
\draw [dotted,red](0,0) -- (2,2) ;
\draw [decorate,decoration=border]
(0,0) -- (2,2) ;
\end{tikzpicture}
&  
\begin{tikzpicture}
\draw [dotted,red] (1,1) circle (1);
\draw [decorate,decoration=border]
(1,1) circle (1); 
\end{tikzpicture}
&  
\begin{tikzpicture}
\draw [dotted,red]
(0,0)  arc (0:180:3 and 2);
\draw [decorate,decoration=border]
(0,0)  arc (0:180:3 and 2);
\end{tikzpicture}
\\ \hline  
(0,0) - - (2,2) & (1,1) circle (1) & (0,0)  arc (0:180:3 and 2) \\ 
\hline 
\end{tabular}

\bigskip

\begin{tabular}{|l|c|c|} \hline 
\multicolumn{2}{|c|}{\BSS{draw}[decorate,decoration=\AC{border,\RDD{amplitude}=0.5cm}] (0,0) - - (10,0);} & \dft
 \\ \hline 
\RDD{amplitude}=0.5cm
&  
\begin{tikzpicture}[baseline=0pt]
\draw[red!20] (0,-0.5) grid (10,0.5);
\draw[dotted,red] (0,0) -- (10,0); \draw[decorate,decoration={border,amplitude=0.5cm}] (0,0) -- (10,0);
\end{tikzpicture}
& 2.5 pt
\\ \hline  
\parbox{4cm}{
\RDD{segment length}=1cm ,\\
amplitude=0.5cm}
&  
\begin{tikzpicture}[baseline=0pt]
\draw[red!20] (0,-0.5) grid (10,0.5);
\draw[dotted,red] (0,0) -- (10,0); \draw[decorate,decoration={border,segment length=1cm,amplitude=0.5cm}] (0,0) -- (10,0);
\end{tikzpicture}
& 10 pt
\\ \hline
\parbox{4cm}{
\RDD{angle}=90 ,\\
amplitude=0.5cm
}
&  
\begin{tikzpicture}[baseline=0pt]
\draw[red!20] (0,-1) grid (10,1);
\draw[dotted,red] (0,0) -- (10,0); \draw[decorate,decoration={border,angle=90,amplitude=0.5cm}] (0,0) -- (10,0);
\end{tikzpicture}
& 45
\\ \hline 
\end{tabular}

\bigskip

\begin{tabular}{|c|c|c|} \hline  
\multicolumn{3}{|c|}{ \BSS{draw}[decorate,decoration=
\AC{border,\RDD{amplitude}=0.5cm}] (1,1) circle (1);}
\\ \hline 
\begin{tikzpicture}
\draw [decorate,decoration={border,amplitude=0.5cm}]
 (1,1) circle (1); 
\end{tikzpicture}
 & 
\begin{tikzpicture}
\draw [dotted,red](1,1) circle (1);
\draw [decorate,decoration={border,segment length=1cm,amplitude=0.5cm}]
(1,1) circle (1); 
\end{tikzpicture}
&  
\begin{tikzpicture}
\draw [dotted,red](1,1) circle (1);
\draw [decorate,decoration={border,angle=90,amplitude=0.5cm}]
(1,1) circle (1); 
\end{tikzpicture}
\\ \hline 
 \RDD{amplitude}=0.5cm & \RDD{segment length}=1cm &\RDD{angle}=90 \\
 & ,amplitude=0.5cm & ,amplitude=0.5cm
\\ \hline 

\end{tabular}

\subsubsection{\og brace \fg }

\begin{tabular}{|c|c|} \hline 
\begin{tikzpicture}[baseline=0pt]
\draw [decorate,decoration=brace] (0,0) -- (3,0);
\end{tikzpicture}
 &  
 \BS{draw} [decorate,\RDD{decoration}=\RDDX{brace}{decoration}] (0,0) - - (3,1);
 \\ \hline 
\end{tabular} 

\bigskip

\begin{tabular}{|c|c|c|c|} \hline 
\multicolumn{3}{|c|}{ \BSS{draw}[decorate,decoration=
\AC{brace,\RDD{amplitude}=0.5cm}] (1,1) circle (1); ;}
\\ \hline 
\begin{tikzpicture}
\draw [dotted,red](0,0) -- (2,2) ;
\draw [decorate,decoration={brace,amplitude=0.5cm}](0,0) -- (2,2) ;
\end{tikzpicture}
&  
\begin{tikzpicture}
\draw [dotted,red](0,0) -- (2,2) ;
\draw [decorate,decoration={brace,aspect=.65,amplitude=0.5cm}]
(0,0) -- (2,2) ; 
\end{tikzpicture}
&  
\begin{tikzpicture}
\draw [dotted,red](0,0) -- (2,2) ;
\draw [decorate,decoration={brace,raise=0.25cm,amplitude=0.5cm}]
(0,0) -- (2,2) ;
\end{tikzpicture}
&
\begin{tikzpicture}
\draw [dotted,red](0,0) -- (2,2) ;
\draw [decorate,decoration={brace,mirror,amplitude=0.5cm}]
(0,0) -- (2,2) ;
\end{tikzpicture}
\\ \hline 
\RDD{amplitude}=0.5cm & \RDD{aspect}=0.65 & \RDD{raise}= 0.25cm & \RDD{mirror}
\\ 
					& ,amplitude = 0.5cm & ,amplitude = 0.5cm& ,amplitude = 0.5cm
\\ \hline  
\dft : 2.5 & \dft : 0.5 &  \dft : 0 & \\ 
\hline 
\end{tabular}

\subsubsection{\fg expanding waves \fg }

\begin{tabular}{|c|c|} \hline  
\begin{tikzpicture}[baseline=0pt]
\draw [dotted,red](0,0) -- (2,0) ;
\draw [dashed,red](0,0) -- (20:2) ;
\draw [dashed,red](0,0) -- (-20:2) ;
\draw [decorate,decoration={expanding waves}](0,0) -- (2,0) ;
\end{tikzpicture}
&  
\parbox{10cm}{
\BS{draw} [dashed,red](0,0) - - (20:2) ;\\
\BS{draw} [dashed,red](0,0) - - (-20:2) ;\\
\BS{draw} [decorate,decoration=\AC{\RDD{expanding waves}}](0,0) - - (2,0) ;
}
\\ \hline 
\end{tabular} 

\bigskip

\begin{tabular}{|c|c|} \hline
\multicolumn{2}{|c|}{ \BSS{draw}[decorate,decoration=
\AC{expanding waves,\RDD{segment length}=0.5cm}] (1,1) circle (1);}
\\ \hline 
\begin{tikzpicture}
\draw [dotted,red](0,0) -- (2,2) ;
\draw [decorate,decoration={expanding waves,segment length=0.5cm}](0,0) -- (2,2) ;
\end{tikzpicture}
&  
\begin{tikzpicture}
\draw [dotted,red](0,0) -- (2,2) ;
\draw [decorate,decoration={expanding waves,angle=45}]
(0,0) -- (2,2) ;
\end{tikzpicture}
\\ \hline 
\RDD{segment length}=0.5cm & \RDD{angle}=45
\\ \hline 
 
\dft : 10pt &  \dft : 20\\ 
\hline 
\end{tabular}

\subsubsection{\og moveto \fg }
 
 \TFRGB{voir}{see} page \pageref{moveto}

\subsubsection{\og  ticks \fg }

\begin{tabular}{|c|c|c|} \hline 
\multicolumn{3}{|c|}{\BSS{draw}[decorate,\RDD{decoration}=\RDDX{ticks}{decoration}] (0,0) - - (2,2) ;}
 \\ \hline  
\begin{tikzpicture}
\draw [dotted,red](0,0) -- (2,2) ;
\draw [decorate,decoration=ticks]
(0,0) -- (2,2) ;
\end{tikzpicture}
&  
\begin{tikzpicture}
\draw [dotted,red] (1,1) circle (1);
\draw [decorate,decoration=ticks]
(1,1) circle (1); 
\end{tikzpicture}
&  
\begin{tikzpicture}
\draw [dotted,red]
(0,0)  arc (0:180:3 and 2);
\draw [decorate,decoration=ticks]
(0,0)  arc (0:180:3 and 2);
\end{tikzpicture}
\\ \hline  
(0,0) - - (2,2) & (1,1) circle (1) & (0,0)  arc (0:180:3 and 2) \\ 
\hline 
\end{tabular}
 \bigskip

\begin{tabular}{|l|c|c|} \hline 
\multicolumn{2}{|c|}{\BSS{draw}[decorate,decoration=\AC{ticks,\RDD{amplitude}=0.5cm}] (0,0) - - (10,0);} & \dft
 \\ \hline 
\RDD{amplitude}=0.5cm
&  
\begin{tikzpicture}[baseline=0pt]
\draw[red!20] (0,-1) grid (10,1);
\draw[dotted,red] (0,0) -- (10,0); \draw[decorate,decoration={ticks,amplitude=0.5cm}] (0,0) -- (10,0);
\end{tikzpicture}
& 2.5 pt
\\ \hline  
\RDD{segment length}=1cm
&  
\begin{tikzpicture}[baseline=0pt]
\draw[red!20] (0,-0.5) grid (10,0.5);
\draw[dotted,red] (0,0) -- (10,0); \draw[decorate,decoration={ticks,segment length=1cm}] (0,0) -- (10,0);
\end{tikzpicture}
& 10 pt
\\ \hline
\end{tabular}

\bigskip

\begin{tabular}{|c|c|c|} \hline 
\multicolumn{3}{|c|}{ \BSS{draw}[decorate,decoration=
\AC{ticks,\RDD{segment length}=1cm}] (1,1) circle (1); }
 \\ \hline  
\begin{tikzpicture}
\draw [dotted,red] (1,1) circle (1); 
\draw [decorate,decoration={ticks,segment length=1cm}](1,1) circle (1); 
\end{tikzpicture}
&  
\begin{tikzpicture}
\draw [dotted,red](1,1) circle (32pt); 
\draw [decorate,decoration={ticks,segment length=pi*8}](1,1) circle (32pt); 
\end{tikzpicture}
&  
\begin{tikzpicture}
\draw [dotted,red](1,1) circle (1); 
\draw [decorate,decoration={ticks,amplitude=0.5cm}]
(1,1) circle (1); 
\end{tikzpicture}
\\ \hline 
\RDD{segment length}=1cm & segment length=\RDD{pi*8} & \RDD{amplitude}=0.5cm \\
(1,1) circle (1) & (1,1) circle (32pt) & (1,1) circle (1)
\\ \hline 
\end{tabular}

\subsubsection{\fg waves \fg }

\begin{tabular}{|c|c|c|} \hline 
\multicolumn{3}{|c|}{\BSS{draw}[decorate,\RDD{decoration}=\RDDX{waves}{decoration}] (0,0) - - (2,2) ;}
 \\ \hline  
\begin{tikzpicture}
\draw [dotted,red](0,0) -- (2,2) ;
\draw [decorate,decoration=waves]
(0,0) -- (2,2) ;
\end{tikzpicture}
&  
\begin{tikzpicture}
\draw [dotted,red] (1,1) circle (1);
\draw [decorate,decoration=waves]
(1,1) circle (1); 
\end{tikzpicture}
&  
\begin{tikzpicture}
\draw [dotted,red]
(0,0)  arc (0:180:3 and 2);
\draw [decorate,decoration=waves]
(0,0)  arc (0:180:3 and 2);
\end{tikzpicture}
\\ \hline  
(0,0) - - (2,2) & (1,1) circle (1) & (0,0)  arc (0:180:3 and 2)\\ 
\hline 
\end{tabular}

\bigskip

\begin{tabular}{|l|c|c|} \hline 
\multicolumn{2}{|c|}{\BSS{draw}[decorate,decoration=\AC{waves,\RDD{angle}=60,radius=1cm}] (0,0) - - (10,0);} & \dft
 \\ \hline 
\RDD{angle}=60
&  
\begin{tikzpicture}[baseline=0pt]
\draw[red!20] (0,-0.5) grid (10,0.5);
\draw[dotted,red] (0,0) -- (10,0); \draw[decorate,decoration={waves,angle=60,radius=1cm}] (0,0) -- (10,0);
\end{tikzpicture}
& 45
\\ \hline  
\RDD{segment length}=1cm
&  
\begin{tikzpicture}[baseline=0pt]
\draw[red!20] (0,-0.5) grid (10,0.5);
\draw[dotted,red] (0,0) -- (10,0); \draw[decorate,decoration={waves,segment length=1cm,radius=1cm}] (0,0) -- (10,0);
\end{tikzpicture}
& 10 pt
\\ \hline
\RDD{radius}=2cm
&  
\begin{tikzpicture}[baseline=0pt]
\draw[red!20] (0,-0.5) grid (10,0.5);
\draw[dotted,red] (0,0) -- (10,0); \draw[decorate,decoration={waves,radius=2cm}] (0,0) -- (10,0);
\end{tikzpicture}
& 10 pt
\\ \hline
\end{tabular}

\bigskip

\begin{tabular}{|c|c|c|} \hline 
\multicolumn{3}{|c|}{ \BSS{draw}[decorate,decoration=
\{waves,\RDD{segment length}=pi*8,} \\
\multicolumn{3}{|c|}{radius=1cm\}] (1,1) circle (32pt); }
 \\ \hline  
\begin{tikzpicture}
\draw [dotted,red](1,1) circle (32pt); 
\draw [decorate,decoration={waves,segment length=pi*8,radius=1cm}]
(1,1) circle (32pt); 
\end{tikzpicture}
&  
\begin{tikzpicture}
\draw [dotted,red](1,1) circle (32pt); 
\draw [decorate,decoration={waves,angle=60,segment length=pi*8,radius=1cm}]
(1,1) circle (32pt); 
\end{tikzpicture}
&  
\begin{tikzpicture}
\draw [dotted,red](1,1) circle (32pt); 
\draw [decorate,decoration={waves,segment length=pi*8,radius=2cm}]
(1,1) circle (32pt); 
\end{tikzpicture}
\\ \hline 
\RDD{segment length} = pi*8 & \RDD{angle}=60 & \RDD{radius}=2cm \\
& , segment length = pi*8 & , segment length = pi*8
\\ \hline 
\end{tabular}

\subsubsection{\og show path construction \fg }


\begin{tabular}{|l|} \hline 
\multicolumn{1}{|c|}{ \emph{\TFRGB{Chemin à décorer}{path to decorate}} }
\\ \hline  
\BS{draw} [blue,dashed] (0,0) - - (2,1)  arc (-20:135:1) - - cycle \\
(3,2)   .. controls (7,0) and (2,0) .. (5,2) - - (6,2) sin (7.57,0)  - - (8,3) ;
\\ \hline 
\begin{tikzpicture}
\draw [blue,dashed] (0,0) -- (2,1)  arc (-20:135:1) -- cycle (3,2)   .. controls (7,0) and (2,0) .. (5,2) -- (6,2) sin (7.57,0)  -- (8,3) ;
\end{tikzpicture} 
\\ \hline 
\end{tabular} 

\bigskip


\begin{tabular}{|l|} \hline 
\multicolumn{1}{|c|}{ \textbf{\TFRGB{composantes linéaires  }{ Linear components} : \og  lineto \fg  }  }
\\ \hline 
 
decoration=\{ \RDD{show path construction},\\
\RDD{lineto code}=\AC{ \BS{draw} [red,ultra thick,->] \\ (\BSS{tikzinputsegmentfirst}) - - (\BSS{tikzinputsegmentlast});
},\}
\\ \hline 
\begin{tikzpicture}
\draw [blue,dashed] (0,0) -- (2,1)  arc (-20:135:1) -- cycle (3,2)   .. controls (7,0) and (2,0) .. (5,2) -- (6,2) sin (7.57,0)  -- (8,3) ;
\path [decorate,decoration={show path construction,lineto code={
\draw [red,ultra thick,->] (\tikzinputsegmentfirst) -- (\tikzinputsegmentlast);
},} ]  (0,0) -- (2,1)  arc (-20:135:1) -- cycle (3,2)   .. controls (7,0) and (2,0) .. (5,2) -- (6,2) sin (7.57,0)  -- (8,3) ;;
\end{tikzpicture} 
\\ \hline 
\end{tabular} 

\bigskip


\begin{tabular}{|l|} \hline 
\multicolumn{1}{|c|}{ \textbf{\TFRGB{Fermetures de chemin }{ Path terminations} : \og  closepath \fg  }  }
\\ \hline  
decoration=\{ \RDD{show path construction},\\
\RDD{closepath code}=\AC{ \BS{draw} [red,ultra thick,->] \\ (\BSS{tikzinputsegmentfirst}) - - (\BSS{tikzinputsegmentlast});
},\}
\\ \hline 
\begin{tikzpicture}
\draw [blue,dashed] (0,0) -- (2,1)  arc (-20:135:1) -- cycle (3,2)   .. controls (7,0) and (2,0) .. (5,2) -- (6,2) sin (7.57,0)  -- (8,3) ;
\path [decorate,decoration={show path construction,closepath code={
\draw [red,ultra thick,->] (\tikzinputsegmentfirst) -- (\tikzinputsegmentlast);
},} ]  (0,0) -- (2,1)  arc (-20:135:1) -- cycle (3,2)   .. controls (7,0) and (2,0) .. (5,2) -- (6,2) sin (7.57,0)  -- (8,3) ;
\end{tikzpicture} 
\\ \hline 
\end{tabular} 

\bigskip


\begin{tabular}{|l|} \hline 
\multicolumn{1}{|c|}{ \textbf{\TFRGB{coupure de chemin }{ Broken paths} : \og  moveto \fg  }  }
\\ \hline  
decoration=\{ \RDD{show path construction},\\
\RDD{moveto code}=\AC{ \BS{draw} [red,ultra thick,->] \\ (\BSS{tikzinputsegmentfirst}) - - (\BSS{tikzinputsegmentlast});
},\}
\\ \hline 
\begin{tikzpicture}
\draw [blue,dashed] (0,0) -- (2,1)  arc (-20:135:1) -- cycle (3,2)   .. controls (7,0) and (2,0) .. (5,2) -- (6,2) sin (7.57,0)  -- (8,3) ;
\path [decorate,decoration={show path construction,moveto code={
\draw [red,ultra thick,->] (\tikzinputsegmentfirst) -- (\tikzinputsegmentlast);
},} ]  (0,0) -- (2,1)  arc (-20:135:1) -- cycle (3,2)   .. controls (7,0) and (2,0) .. (5,2) -- (6,2) sin (7.57,0)  -- (8,3) ;
\end{tikzpicture} 
\\ \hline 
\end{tabular} 

\newpage
 

\begin{tabular}{|l|} \hline 
\multicolumn{1}{|c|}{ \textbf{\TFRGB{composants courbes }{ Curved segments} : \og  curveto \fg  }  }
\\ \hline  
decoration=\{ \RDD{show path construction},\\
\RDD{curveto code}=\AC{ \BS{draw} [red,ultra thick,->] \\ (\BSS{tikzinputsegmentfirst}) - - (\BSS{tikzinputsegmentlast});
},\}
\\ \hline 
\begin{tikzpicture}
\draw [blue,dashed] (0,0) -- (2,1)  arc (-20:135:1) -- cycle (3,2)   .. controls (7,0) and (2,0) .. (5,2) -- (6,2) sin (7.57,0)  -- (8,3) ;
\path [decorate,decoration={show path construction,curveto code={
\draw [red,ultra thick,->] (\tikzinputsegmentfirst) -- (\tikzinputsegmentlast);
},} ]  (0,0) -- (2,1)  arc (-20:135:1) -- cycle (3,2)   .. controls (7,0) and (2,0) .. (5,2) -- (6,2) sin (7.57,0)  -- (8,3) ;
\end{tikzpicture} 
\\ \hline 

\hline  
decoration=\{ \RDD{show path construction},\\
\RDD{curveto code}=\AC{ \BS{draw} [red,ultra thick,->] \\ (\BSS{tikzinputsegmentfirst}) - - (\BSS{tikzinputsegmentsupporta});
},\}
\\ \hline 
\begin{tikzpicture}
\draw [blue,dashed] (0,0) -- (2,1)  arc (-20:135:1) -- cycle (3,2)   .. controls (7,0) and (2,0) .. (5,2) -- (6,2) sin (7.57,0)  -- (8,3) ;
\path [decorate,decoration={show path construction,curveto code={
\draw [red,ultra thick,->] (\tikzinputsegmentfirst) -- (\tikzinputsegmentsupporta);
},} ]  (0,0) -- (2,1)  arc (-20:135:1) -- cycle (3,2)   .. controls (7,0) and (2,0) .. (5,2) -- (6,2) sin (7.57,0)  -- (8,3) ;
\end{tikzpicture} 
\\ \hline 

\hline  
decoration=\{ \RDD{show path construction},\\
\RDD{curveto code}=\AC{ \BS{draw} [red,ultra thick,->] \\ (\BSS{tikzinputsegmentlast}) - - (\BSS{tikzinputsegmentsupportb});
},\}
\\ \hline 
\begin{tikzpicture}
\draw [blue,dashed] (0,0) -- (2,1)  arc (-20:135:1) -- cycle (3,2)   .. controls (7,0) and (2,0) .. (5,2) -- (6,2) sin (7.57,0)  -- (8,3) ;
\path [decorate,decoration={show path construction,curveto code={
\draw [red,ultra thick,->] (\tikzinputsegmentlast) -- (\tikzinputsegmentsupportb);
},} ]  (0,0) -- (2,1)  arc (-20:135:1) -- cycle (3,2)   .. controls (7,0) and (2,0) .. (5,2) -- (6,2) sin (7.57,0)  -- (8,3) ;
\end{tikzpicture} 
\\ \hline 
\hline  
decoration=\{ \RDD{show path construction},\\
\RDD{curveto code}=\AC{ \BS{draw} [red,ultra thick,->] \\ (\BSS{tikzinputsegmentfirst}) .. controls  (\BSS{tikzinputsegmentsupporta}) \\
and (\BSS{tikzinputsegmentsupportb}) .. (\BSS{tikzinputsegmentlast})
;
},\}
\\ \hline 
\begin{tikzpicture}
\draw [blue,dashed] (0,0) -- (2,1)  arc (-20:135:1) -- cycle (3,2)   .. controls (7,0) and (2,0) .. (5,2) -- (6,2) sin (7.57,0)  -- (8,3) ;
\path [decorate,decoration={show path construction,curveto code={
\draw [red,ultra thick,->] (\tikzinputsegmentfirst) .. controls (\tikzinputsegmentsupporta) and (\tikzinputsegmentsupportb) .. (\tikzinputsegmentlast);} } ]  
(0,0) -- (2,1)  arc (-20:135:1) -- cycle (3,2)  -- (6,2) sin (7.57,0)  -- (8,3) ;
\end{tikzpicture} 
\\ \hline 
.. controls (7,0) and (2,0) .. (5,2) \DW{} 
\\ \hline 
\end{tabular}


\newpage


\subsection{Library \og decorations.markings \fg }

 \maboite{\BS{usetikzlibrary}\AC{decorations.markings}}
\label{lib-mark}

\begin{center}
\RRR{48-4}
\end{center}

\SbSbSSCT{Sa marque à une position}{Personal mark at one position}

\begin{tabular}{|c|} \hline  
\BS{draw} [decorate,decoration=\{markings,\RDD{mark}=\RDDX{at position}{mark} 1cm \\ \RDD{ with} \{ 
\textbf{\BS{draw}[red] (-2pt,-2pt) - - (2pt,2pt);} 
\textbf{\BS{draw}[red](2pt,-2pt) - - (-2pt,2pt);}\\
\textbf{\BS{draw}[red] (-2pt,-2pt) rectangle (2pt,2pt); }
\}\}] (1,1) circle (1);
\\ \hline  
\begin{tikzpicture}
\draw[dotted] (1,1) circle (1);
\draw [decorate,decoration={markings,mark=at position 1cm with {
\draw[red] (-2pt,-2pt) -- (2pt,2pt);
\draw[red] (2pt,-2pt) -- (-2pt,2pt);
\draw[red] (-2pt,-2pt) rectangle (2pt,2pt);
}}]
(1,1) circle (1);
\end{tikzpicture}
\\ \hline 
\end{tabular} 

\SbSbSSCT{Ses marques : origine, fin et  pas }{Marks between positions with step size}


\begin{tabular}{|c|c|} \hline 
\multicolumn{2}{|c|}{\BSS{draw}[decorate,\{markings,mark=\RDD{between positions} 0 \RDD{and} 1  \RDD{step} 5mm with ... \}] (1,1) circle (1);;}
\\ \hline   
\begin{tikzpicture}
\draw[dotted] (1,1) circle (1);
\draw [decorate,decoration={markings,mark=between positions 0 and 1 step 5mm with {
\draw[red] (-2pt,-2pt) -- (2pt,2pt);
\draw[red] (2pt,-2pt) -- (-2pt,2pt);
\draw[red] (-2pt,-2pt) rectangle (2pt,2pt);
}}]
(1,1) circle (1);
\end{tikzpicture}
&  
\begin{tikzpicture}
\draw[dotted] (1,1) circle (1);
\draw [decorate,decoration={markings,mark=between positions 0 and 0.5 step 5mm with {
\draw[red] (-2pt,-2pt) -- (2pt,2pt);
\draw[red] (2pt,-2pt) -- (-2pt,2pt);
\draw[red] (-2pt,-2pt) rectangle (2pt,2pt);
}}]
(1,1) circle (1);
\end{tikzpicture}
\\ \hline  
mark=\RDD{between positions} 0 \RDD{and} 1  \RDD{step} 5mm &
  \RDD{between positions} 0 \RDD{and} 0.5  \RDD{step} 5mm
\\ \hline 
\begin{tikzpicture}
\draw[dotted] (1,1) circle (1);
\draw [decorate,decoration={markings,mark=between positions 0 and 1 step 1/10 with {
\draw[red] (-2pt,-2pt) -- (2pt,2pt);
\draw[red] (2pt,-2pt) -- (-2pt,2pt);
\draw[red] (-2pt,-2pt) rectangle (2pt,2pt);
}}]
(1,1) circle (1);
\end{tikzpicture}
&  
\begin{tikzpicture}
\draw[dotted] (1,1) circle (1);
\draw [decorate,decoration={markings,mark=between positions 0 and 1 step .1 with {
\draw[red] (-2pt,-2pt) -- (2pt,2pt);
\draw[red] (2pt,-2pt) -- (-2pt,2pt);
\draw[red] (-2pt,-2pt) rectangle (2pt,2pt);
}}]
(1,1) circle (1);
\end{tikzpicture}
\\ \hline  
mark= \RDD{between positions} 0 \RDD{and} 1 \RDD{step} 1/10 &
	\RDD{between positions} 0 \RDD{and} 1  \RDD{step}0.1
\\ \hline

\end{tabular} 

\bigskip

\SbSbSSCT{Marque avec un n\oe ud contenant du texte}{Marks with a text node}

\begin{tabular}{|c|c|c|} \hline  
\multicolumn{3}{|c|}{
decoration=\AC{markings,mark=at position 1cm with {\color{red}{\BS{node}[red]}\AC{texte}}}}
\\ \hline  
\begin{tikzpicture}
\draw[dotted] (1,1) circle (1);
\draw [decorate,decoration={markings,mark=at position 1cm with \node[red]{texte};
}]
(1,1) circle (1);
\end{tikzpicture}
&  
\begin{tikzpicture}
\draw[dotted] (1,1) circle (1);
\draw [decorate,decoration={markings,mark=at position 0.5 with \node[red]{texte};
}]
(1,1) circle (1);
\end{tikzpicture}
&  
\begin{tikzpicture}
\draw[dotted] (1,1) circle (1);
\draw [decorate,decoration={markings,mark=at position -1cm with \node[red]{texte};
}]
(1,1) circle (1);
\end{tikzpicture}
\\ \hline  
at position 1cm & at position 0.5 & at position -1cm 
\\ \hline 
\begin{tikzpicture}
\draw[dotted] (1,1) circle (1);
\draw [decorate,decoration={markings,mark=at position 1cm/2 with \node[red]{texte};
}]
(1,1) circle (1);
\end{tikzpicture}
&  
\begin{tikzpicture}
\draw[dotted] (1,1) circle (1);
\draw [decorate,decoration={markings,mark=at position 0.5/2 with \node[red]{texte};
}]
(1,1) circle (1);
\end{tikzpicture}
&  
\begin{tikzpicture}
\draw[dotted] (1,1) circle (1);
\draw [decorate,decoration={markings,mark=at position -.3 with \node[red]{texte};
}]
(1,1) circle (1);
\end{tikzpicture}
\\ \hline  
at position 1cm/2 & at position 0.5/2 & at position -0.5/2 
\\ \hline 

\end{tabular} 
 \bigskip

\SbSbSSCT{Marque avec un n\oe ud contenant une image}
{Mark with a picture node}

\begin{tabular}{|c|c|} \hline
\multicolumn{2}{|c|}{
\BS{draw} [decorate,decoration=\AC{markings,mark=at position 1cm with {\color{red}{\BS{node}\AC{\BS{DFR}}}};
}]
(1,1) circle (1);}
\\  \hline
\begin{tikzpicture}
\draw[dotted] (1,1) circle (1);
\draw [decorate,decoration={markings,mark=at position 1cm with \node{\DFR};
}]
(1,1) circle (1);
\end{tikzpicture}
&  
\begin{tikzpicture}
\draw[dotted] (1,1) circle (1);
\draw [decorate,decoration={markings,mark=at position 1cm with \node[transform shape]{\DFR};
}]
(1,1) circle (1);
\end{tikzpicture}

\\ \hline  
\BS{node}\AC{\BS{DFR}} &  \BS{node}[\RDD{transform shape}]\AC{\BS{DFR}}
\\ \hline 
\begin{tikzpicture}
\draw[dotted] (1,1) circle (1);
\draw [decorate,decoration={markings,mark=at position 1cm with \node{\includegraphics[width=0.5cm]{tiger}};
}]
(1,1) circle (1);
\end{tikzpicture}
&  
\begin{tikzpicture}
\draw[dotted] (1,1) circle (1);
\draw [decorate,decoration={markings,mark=at position 1cm with \node[transform shape]{\includegraphics[width=0.5cm]{tiger}};
}]
(1,1) circle (1);
\end{tikzpicture}

\\ \hline  
 \BS{node}\{ 
&  
\BS{node}[transform shape]\{ \\
\BS{includegraphics}[width=0.5cm]\AC{tiger} \} 
& 
\BS{includegraphics}[width=0.5cm]\AC{tiger} \}
\\ \hline 
\end{tabular}

\bigskip

\SbSbSSCT{Numérotation des marques et affectation d'un nom }{ Numbered marks}

\begin{tabular}{|c|c|}\hline  
\begin{tikzpicture}[baseline=0pt,decoration={markings,
mark=between positions 0 and 1 step 0.2 with {
\node [red,draw,circle,fill=white,
name=marque-\pgfkeysvalueof{/pgf/decoration/mark info/sequence number},
transform shape]
{\pgfkeysvalueof{/pgf/decoration/mark info/sequence number}};}}]
\draw [postaction={decorate}] (0,0)  arc (180:0:2 and 1.5);
\end{tikzpicture}
& 
\parbox[c]{11cm}{ 
decoration=\{markings,\\
mark=between positions 0 and 1 step 0.2 \\
with \{ \BS{node} [draw , circle ,fill=white, name= \\
{\color{blue} marque-}\BSS{pgfkeysvalueof}\AC{{\color{red}/pgf/decoration/mark info/sequence number}},\\
transform shape] \\
\AC{\BSS{pgfkeysvalueof}\AC{{\color{red}/pgf/decoration/mark info/sequence number}}};\}\}
}
\\ \hline 
\begin{tikzpicture}[baseline=0pt,decoration={markings,
mark=between positions 0 and 1 step 0.2 with {
\node [draw,circle,fill=white,
name=marque-\pgfkeysvalueof{/pgf/decoration/mark info/sequence number},
transform shape]
{\pgfkeysvalueof{/pgf/decoration/mark info/sequence number}};}}]
\draw [postaction={decorate}] (0,0)  arc (180:0:2 and 1.5);
\draw [red,ultra thick] (marque-3) -- (marque-5);
\end{tikzpicture}
& 
\parbox[c]{11cm}{ 
\BS{draw} [red,ultra thick] ({\color{blue}   marque-3}) - - ({\color{blue} marque-5});
}
\\ \hline 
\end{tabular} 

\SbSbSSCT{Distance des n\oe uds }{Marks info}

\begin{tabular}{|c|} \hline 
\begin{tikzpicture}[baseline=0pt,decoration={markings,
mark=between positions 0 and 1 step 40pt with {
\node [red,draw,ellipse,fill=white,font=\tiny]
{ \pgfkeysvalueof{/pgf/decoration/mark info/distance from start}
\pgfkeysvalueof{/pgf/decoration/mark info/mark info/distance from start}
};}}]
\draw [postaction={decorate}] (0,0)  arc (180:0:3 and 2);
\end{tikzpicture}
\\ \hline  
decoration=\{markings,\\
mark=between positions 0 and 1 step 40pt with \\
\{ \BS{node} [red,draw,ellipse,fill=white,font=\BS{tiny}] \\
\AC{\BSS{pgfkeysvalueof}\AC{{\color{red}/pgf/decoration/mark info/distance from start}}
};\} \}
\\ \hline 
\end{tabular}

\bigskip 

/pgf/decoration/reset marks (no value)

/pgf/decoration/mark connection node=node name (no default, initially empty)

\SbSbSSCT{N\oe ud sur une liaison}{Mark with a connection node}

\begin{tabular}{|c|c|} \hline  
\begin{tikzpicture}[baseline=0pt]

\draw [decorate,decoration={markings,
mark connection node=noeud,
mark=at position 0.4 with
{\node [draw,ellipse,blue,transform shape] (noeud) {texte};}}]  (0,0) -- (3,2) ;
\end{tikzpicture}
&  
\parbox[b]{11cm}{
\BS{draw} [decorate,decoration=\{markings,\\
\RDD{mark connection node}={\color{blue}  mon noeud},mark=at position 0.4 with \\
\AC{\BSS{node} [draw,ellipse,blue,transform shape] ({\color{blue}  mon noeud}) \AC{texte};}\}] \\
 (0,0) -- (3,2) ;}
\\ \hline 
\end{tabular}
 
\subsubsection{Arrow Tip Markings}

\begin{tabular}{|c|c|c|c|} \hline  
\multicolumn{4}{|c|}{ \BS{draw}[decorate,decoration=\{ markings,mark=at position 1cm with } \\
\multicolumn{4}{|c|}{\AC{\BSS{arrow}[blue,line width=2mm]{\color{red}\AC{>}}};\}] (1,1) circle (1); }
\\ \hline
\begin{tikzpicture}
\draw [white] (-0.5,-0.5) rectangle (2.5,2.5);
\draw[dotted] (1,1) circle (1);
\draw [decorate,decoration={markings,mark=at position 1cm with {\arrow[blue,line width=2mm]{>}};}] (1,1) circle (1);
\end{tikzpicture}
&  
\begin{tikzpicture}
\draw [white] (-0.5,-0.5) rectangle (2.5,2.5);
\draw[dotted] (1,1) circle (1);
\draw [decorate,decoration={markings,mark=at position 1cm with {\arrow[blue,line width=2mm]{stealth}};}] (1,1) circle (1); 
\end{tikzpicture}
&  
\begin{tikzpicture}
\draw [white] (-0.5,-0.5) rectangle (2.5,2.5);
\draw[dotted] (1,1) circle (1);
\draw [decorate,decoration={markings,mark=at position 1cm with {\arrow[blue,line width=2mm]{|}};
}] (1,1) circle (1);
\end{tikzpicture}
&  
\begin{tikzpicture}
\draw [white] (-0.5,-0.5) rectangle (2.5,2.5);
\draw[dotted] (1,1) circle (1);
\draw [decorate,decoration={markings,mark=at position 1cm with {\arrow[blue,line width=2mm]{diamond}};
}] (1,1) circle (1);
\end{tikzpicture}
\\ \hline  
{\color{red}\AC{>}} & {\color{red}\AC{stealth }}  &{\color{red}\AC{|}}  &{\color{red}\AC{diamond}} \\ 
\hline 
\multicolumn{4}{|c|}{ \TFRGB{Autres possibilités et paramètres voir page \pageref{fleches} et suivantes}{Other possibilities see page \pageref{fleches} } }
\\ \hline 
\end{tabular}

\bigskip

\begin{tabular}{|c|c|c|c|} \hline 
\multicolumn{4}{|c|}{ \BS{draw}[decorate,decoration=\{markings,mark=at position 1cm with } \\
\multicolumn{4}{|c|}{ \AC{\BSS{arrowreversed}[blue,line width=2mm]{\color{red}\AC{>}}};\}] (1,1) circle (1);}
\\ \hline 
\begin{tikzpicture}
\draw [white] (-.5,-.5) rectangle (2.5,2.5);
\draw[dotted] (1,1) circle (1);
\draw [decorate,decoration={markings,mark=at position 1cm with {\arrowreversed[blue,line width=2mm]{>}}; }] (1,1) circle (1);
\end{tikzpicture}
&  
\begin{tikzpicture}
\draw [white] (-.5,-.5) rectangle (2.5,2.5);
\draw[dotted] (1,1) circle (1);
\draw [decorate,decoration={markings,mark=at position 1cm with {\arrowreversed[blue,line width=2mm]{stealth}}; }] (1,1) circle (1);
\end{tikzpicture}
&  
\begin{tikzpicture}
\draw [white] (-.5,-.5) rectangle (2.5,2.5);
\draw[dotted] (1,1) circle (1);
\draw [decorate,decoration={markings,mark=at position 1cm with {\arrowreversed[blue,line width=2mm]{|}}; }] (1,1) circle (1);
\end{tikzpicture}
&  
\begin{tikzpicture}
\draw [white] (-.5,-.5) rectangle (2.5,2.5);
\draw[dotted] (1,1) circle (1);
\draw [decorate,decoration={markings,mark=at position 1cm with {\arrowreversed[blue,line width=2mm]{diamond}}; }] (1,1) circle (1);
\end{tikzpicture}
\\ \hline  
{\color{red}\AC{>}} & {\color{red}\AC{stealth }}   &{\color{red}\AC{|}}  &{\color{red}\AC{diamond}} 
\\ \hline  
\end{tabular} 



\newpage
\subsection{Library \og decorations.footprints \fg }


 \maboite{\BS{usetikzlibrary}\AC{decorations.footprints}}
\label{lib-footprints}

\begin{center}
\RRR{48-5-2}
\end{center}

 \bigskip
\begin{tabular}{|c|} \hline  
\BS{tikz} \BS{draw}[decorate,\RDD{decoration}=\RDDX{footprints}{decoration}] (0,0) -- (10,0);

\\ \hline  
\tikz \draw[decorate,decoration=footprints] (0,0) -- (10,0);

\\ \hline 
\end{tabular} 

 \bigskip

\begin{tabular}{|c|c|c|c|} \hline  
\multicolumn{4}{|c|}{\BSS{draw}[decorate,decoration=\AC{footprints,\RDD{foot of} = \RDDX{gnome}{foot of}}] (0,2.5) - - (3,2.5);}
 \\ \hline  
\tikz \draw[decorate,decoration={footprints,foot of = gnome}] (0,2.5) -- (3,2.5);
&  
\tikz \draw[decorate,decoration={footprints,foot of = human}](0,2.5) -- (3,2.5);
&  
\tikz \draw[decorate,decoration={footprints,foot of = bird}] (0,2.5) -- (3,2.5);
&  

\tikz \draw[decorate,decoration={footprints,foot of = felis silvestris}]  (0,2.5) -- (3,2.5);
\\ \hline  
foot of = \RDDX{gnome}{foot of} & foot of = \RDDX{human}{foot of} & foot of = \RDDX{bird}{foot of} & foot of = \RDDX{felis silvestris}{foot of} \\ 
 & (\dft) & & \\
\hline 
\end{tabular} 

 \bigskip

\begin{tabular}{|c|c|c|c|} \hline  
\multicolumn{4}{|c|}{\BSS{fill}[decorate,decoration=\AC{footprints,foot of = gnome}] (0,2.5) - - (3,2.5);}
 \\ \hline  
\tikz \fill[decorate,decoration={footprints,foot of = gnome}] (0,2.5) -- (3,2.5);
&  
\tikz \fill[decorate,decoration={footprints,foot of = human}](0,2.5) -- (3,2.5);
&  
\tikz \fill[decorate,decoration={footprints,foot of = bird}] (0,2.5) -- (3,2.5);
&  

\tikz \fill[decorate,decoration={footprints,foot of = felis silvestris}]  (0,2.5) -- (3,2.5);
\\ \hline  
foot of = gnome & foot of = human & foot of = bird & foot of = felis silvestris \\ 
\hline 
\end{tabular} 

 \bigskip
\begin{tabular}{|c|c|}\hline  
\multicolumn{2}{|c|}{\BS{fill}[decorate,decoration=\AC{footprints,\RDD{foot length}=20pt}] (0,2.5) - - (3,2.5);}
 \\ \hline 
\begin{tikzpicture}[baseline=0pt]
\draw[red!20] (0,-1) grid (6,1); 
\draw[decorate,decoration={footprints,foot length=1cm}] (0,0) -- (6,0);
\end{tikzpicture} 
&  
\begin{tikzpicture}[baseline=0pt]
\draw[red!20] (0,-1) grid (6,1); 
\draw[decorate,decoration={footprints,stride length=2cm}] (0,0) -- (6,0);
\end{tikzpicture} 

\\ \hline 
 \RDD{foot length}=1cm  &  \RDD{stride length}=2cm  \\ 
\hline 
\dft{} : 10pt & \dft{} : 30pt
 \\ \hline 
\begin{tikzpicture}[baseline=0pt]
\draw[red!20] (0,-1) grid (6,1);
 \draw[decorate,decoration={footprints,foot sep=1cm}] (0,0) -- (6,0);
 \end{tikzpicture} 
&  
\begin{tikzpicture}[baseline=0pt]
\draw[red!20] (0,-1) grid (6,1);
 \draw[decorate,decoration={footprints,foot angle =45}] (0,0) -- (6,0);
\end{tikzpicture} 
\\ \hline 
\RDD{foot sep}=1cm  &  \RDD{foot angle} = 45  \\ 
\hline 
\dft{} : 4pt & \dft{} : 10
 \\ \hline
\end{tabular} 

 \bigskip


\begin{tabular}{|c|c|c|c|}\hline  
\multicolumn{4}{|c|}{\BS{fill}[decorate,decoration=\AC{footprints,\RDD{foot length}=20pt}] (0,2.5) - - (3,2.5);}
 \\ \hline 
\begin{tikzpicture}[baseline=0pt]
\draw[red!20] (0,-0.5) grid (3,0.5); 
\draw[decorate,decoration={footprints,foot length=20pt}] (0,0) -- (3,0);
\end{tikzpicture}
& 
\begin{tikzpicture}[baseline=0pt]
\draw[red!20] (0,-0.5) grid (3,0.5); 
\draw[decorate,decoration={footprints,foot length=1cm}] (0,0) -- (3,0);
\end{tikzpicture} 
&
\begin{tikzpicture}[baseline=0pt]
\draw[red!20] (0,-0.5) grid (3,0.5); 
\draw[decorate,decoration={footprints,stride length=15pt}] (0,0) -- (3,0);
\end{tikzpicture}
&  
\begin{tikzpicture}[baseline=0pt]
\draw[red!20] (0,-0.5) grid (3,0.5); 
\draw[decorate,decoration={footprints,stride length=2cm}] (0,0) -- (3,0);
\end{tikzpicture} 

\\ \hline 
\RDD{foot length}=20pt & \RDD{foot length}=1cm  & \RDD{stride length}=15pt & \RDD{stride length}=2cm  \\ 
\hline 
\multicolumn{2}{|c|}{\dft{} : foot length=10pt} &
\multicolumn{2}{|c|}{\dft{} : stride length=30pt}
 \\ \hline 
\tikz \draw[decorate,decoration={footprints,foot sep=10pt}] (0,2.5) -- (3,2.5);
&  
\tikz \draw[decorate,decoration={footprints,foot sep=1cm}] (0,2.5) -- (3,2.5);
&
\tikz \draw[decorate,decoration={footprints,foot angle = -45}] (0,2.5) -- (3,2.5);
&  
\tikz \draw[decorate,decoration={footprints,foot angle =45}] (0,2.5) -- (3,2.5);

\\ \hline 
\RDD{foot sep}=10pt & \RDD{foot sep}=1cm  & \RDD{foot angle} = -45 & \RDD{foot angle} = 45  \\ 
\hline 
\multicolumn{2}{|c|}{\dft{} : foot sep=4pt} &
\multicolumn{2}{|c|}{\dft{} : foot angle=10}
 \\ \hline
\end{tabular} 


\newpage
\subsection{Library \og decorations.shapes \fg }
\subsubsection{Introduction}


 \maboite{\BS{usetikzlibrary}\AC{decorations.shapes}}
\label{lib-shapes}

\begin{center}
\RRR{48-5-3}
\end{center}
 \bigskip

\begin{center}
\begin{tabular}{|c|c|c|c|} \hline  
\multicolumn{3}{|c|}{\BSS{draw}[decorate,\RDD{decoration}=\RDDX{crosses}{decoration}] (0,0) - - (3,0);}
 \\ \hline  
\tikz \draw[decorate,decoration=crosses] (0,0) -- (3,0);
&  
\tikz \draw[decorate,decoration=triangles] (0,0) -- (3,0);
&  
\tikz \draw[decorate,decoration=shape backgrounds] (0,0) -- (3,0);
\\ \hline  
\RDD{crosses} & \RDD{triangles} & \RDD{shape backgrounds}  \\ 
\hline 
\end{tabular}
\end{center} 

 \bigskip

\begin{tabular}{|l|c|} \hline 
\multicolumn{2}{|c|}{\BSS{draw}[decorate,decoration=\AC{crosses,\RDD{segment length}=1cm}](0,0) -  - (10,0);} 
\\ \hline 

\RDD{segment length} = 1cm
&  
\tikz \draw[decorate,decoration={crosses,segment length=1cm}] (0,0) -- (10,0);
\\ \hline  
\RDD{shape width} = 1cm
&  
\tikz \draw[decorate,decoration={crosses,shape width=1cm}] (0,0) -- (10,0);
\\ \hline  
\RDD{shape height} = 1cm
&  
\tikz \draw[decorate,decoration={crosses,shape
 height=1cm}] (0,0) -- (10,0);
\\ \hline 
\RDD{shape size} = 1cm
&  
\tikz \draw[decorate,decoration={crosses,shape size=1cm}] (0,0) -- (10,0);
\\ \hline 
\multicolumn{2}{|c|}{\dft :  shape width = shape height =  2.5pt}
 \\ \hline 
\end{tabular} 



\subsubsection{\og shape backgrounds \fg }



\tikzset{paint/.style={ draw=#1!50!black, fill=#1!50 },
decorate with/.style=
{decorate,decoration={shape backgrounds,shape=#1,shape size=2mm}}}

\begin{tabular}{|c|c|c|c|} \hline  
 \multicolumn{4}{|c|}{\BS{draw}[\RDD{decorate with}=dart] (0,2.5) - - (3,2.5); }  
 \\ \hline 
\tikz \draw[decorate with=dart] (0,2.5) -- (3,2.5);
&  
\tikz \draw[decorate with=diamond] (0,2.5) -- (3,2.5);
&  
\tikz \draw[decorate with=rectangle] (0,2.5) -- (3,2.5);
&  
\tikz \draw[decorate with=circle] (0,2.5) -- (3,2.5);
\\ \hline 
dart & diamond & rectangle &  circle\\ 
\hline 
\tikz \draw[decorate with=star] (0,2.5) -- (3,2.5);
&  
\tikz \draw[decorate with=regular polygon] (0,2.5) -- (3,2.5);
&  
\tikz \draw[decorate with=signal] (0,2.5) -- (3,2.5);
&  
\tikz \draw[decorate with=kite] (0,2.5) -- (3,2.5);
\\ \hline 
star & regular polygon & signal & kite 
\\ \hline 
\multicolumn{4}{|c|}{\TFRGB{Autres possibilités et paramètres voir page \pageref{formes} et suivantes}{Other possibilities or parameters see from page \pageref{formes} }}

\\ \hline
\end{tabular} 

\bigskip 

\begin{tabular}{|l|c|}\hline 
\multicolumn{2}{|c|}{  \TFRGB{Formes disponibles}{Shapes available} }
\\ \hline 
\emph{\TFRGB{Syntaxe}{Syntax}} &\BSS{draw}[decorate,decoration=\{ \RDD{shape backgrounds},\RDD{shape}=dart,\\
 & shape size=.5cm,shape sep=1cm\}] (0,0) - - (10,0);
 \\ \hline 
\emph{\TFRGB{Autre syntaxe}{Other syntax}}
 &
\BS{draw}[\RDD{decorate with}=dart,decoration=\AC{shape size=.5cm,shape sep=1cm}] \\
 & (0,0) -- (10,0); 
 
 \\ \hline \hline   
\RDD{dart}
&  
\tikz \draw[decorate,decoration={shape backgrounds, shape=dart,shape size=.5cm,shape sep=1cm}] (0,2.5) -- (10,2.5);
\\ \hline  
\RDD{rectangle}
&  
\tikz \draw[decorate,decoration={shape backgrounds, shape=rectangle,shape size=.5cm,shape sep=1cm}] (0,2.5) -- (10,2.5);
\\ \hline 
\RDD{cloud}
&  
\tikz \draw[decorate,decoration={shape backgrounds, shape=cloud,shape size=.5cm,shape sep={1cm}}] (0,2.5) -- (10,2.5);
\\ \hline
\RDD{star}
&  
\tikz \draw[decorate,decoration={shape backgrounds, shape=star,shape size=.5cm,shape sep={1cm}}] (0,2.5) -- (10,2.5);
\\ \hline   
\RDD{starburst}
&  
\tikz \draw[decorate,decoration={shape backgrounds, shape=starburst,shape size=.5cm,shape sep={1cm}}] (0,2.5) -- (10,2.5);
\\ \hline  
\RDD{tape}
&  
\tikz \draw[decorate,decoration={shape backgrounds, shape=tape,shape size=.5cm,shape sep={1cm}}] (0,2.5) -- (10,2.5);
\\ \hline  
\RDD{kite}
&   
\tikz \draw[decorate,decoration={shape backgrounds, shape=kite,shape size=.5cm,shape sep={1cm}}] (0,2.5) -- (10,2.5);
\\ \hline 
\RDD{signal}
&   
\tikz \draw[decorate,decoration={shape backgrounds, shape=signal,shape size=.5cm,shape sep={1cm}}] (0,2.5) -- (10,2.5);
\\ \hline 
\multicolumn{2}{|c|}{\dft :   shape= circle }
 \\ \hline
 \multicolumn{2}{|c|}{ \TFRGB{Autres possibilités  voir page \pageref{formes} et suivantes} {Other possibilities see page  \pageref{formes} }}
  \\ \hline 
\end{tabular} 

\bigskip

\begin{tabular}{|c|c|c|c|} \hline
\multicolumn{2}{|c|}{  \TFRGB{Paramètres}{Parameters} }
\\ \hline  
 \multicolumn{4}{|l|}{\BS{draw}[decorate with=star,\RDD{star points}=3,decoration=\AC{shape size=.5cm,shape sep=1cm}]  }\\
 \multicolumn{4}{|l|}{ (0,2.5) - - (3,2.5); } 
 \\ \hline
\tikz \draw[decorate with=star,star points=3,decoration={shape size=.5cm,shape sep=1cm}] (0,2.5) -- (3,2.5);
& 
\tikz \draw[decorate with=star,star points=4,decoration={shape size=.5cm,shape sep=1cm}] (0,2.5) -- (3,2.5);
& 
\tikz \draw[decorate with=star,star points=5,,decoration={shape size=.5cm,shape sep=1cm}] (0,2.5) -- (3,2.5);
&  
\tikz \draw[decorate with=star,star points=8,decoration={shape size=.5cm,shape sep=1cm}] (0,2.5) -- (3,2.5);
\\ \hline  
star points=3 & star points=4 & star points=5 & star points=8\\ \hline 

 \hline 
 \multicolumn{4}{|c|}{\BS{draw}[decorate with=star,\RDD{paint}=green,decoration=\AC{shape size=.5cm,shape sep=1cm}] } \\
\\
 \multicolumn{4}{|l|}{ (0,2.5) - - (3,2.5); }  
 \\ \hline 
\tikz \draw[decorate with=star,paint=green,decoration={shape size=.5cm,shape sep=1cm}] (0,2.5) -- (3,2.5);
&  
\tikz \draw[decorate with=star,double,decoration={shape size=.5cm,shape sep=1cm}] (0,2.5) -- (3,2.5);
&  
\tikz \draw[decorate with=star,star points=8,ultra thick,decoration={shape size=.5cm,shape sep=1cm}] (0,2.5) -- (3,2.5);
&  
\tikz \draw[decorate with=star,star point ratio = 3,decoration={shape size=.5cm,shape sep=1cm}] (0,2.5) -- (3,2.5);
\\ \hline  
\RDD{paint}=green
&  
\RDD{double}
&  
\RDD{ultra thick}
&  
\RDD{star point ratio} = 3
\\ \hline 
\end{tabular} 

\bigskip


\begin{tabular}{|c|c|} \hline
\multicolumn{2}{|c|}{  \TFRGB{Espacement}{Spacing} }
\\ \hline  
 
\multicolumn{2}{|c|}{\BSS{draw}[decorate with=dart,decoration=\{shape size=.5cm,}\\
\multicolumn{2}{|c|}{\RDD{shape sep}=1cm\}] (0,2.5) -  - (10,2.5);}
 \\ \hline 
 
shape sep=\AC{1cm}
&  
\tikz \draw[decorate with=dart,decoration={shape size=.5cm,shape sep=1cm}] (0,2.5) -- (10,2.5);
\\ \hline  
shape sep=\AC{2cm}
&  
\tikz \draw[decorate with=dart,decoration={shape size=.5cm ,shape sep=2cm}] (0,2.5) -- (10,2.5);
\\ \hline 
\multicolumn{2}{|c|}{\dft :  shape sep=                     0.25cm}
 \\ \hline 
\end{tabular} 

\bigskip


\begin{tabular}{|l|c|} \hline 
\multicolumn{2}{|c|}{  \TFRGB{Type d'espacement}{Type of spacing} }
\\ \hline  

\multicolumn{2}{|c|}{\BSS{draw}[decorate with=dart,decoration=\{shape size=.5cm,}\\
\multicolumn{2}{|c|}{
shape sep=\AC{1cm,\RDD{between centers}}\}] (0,2.5) - - (10,2.5);}
 \\ \hline 
\RDD{between centers}
&  
\begin{tikzpicture}
\draw[dotted,red] (0,2.5) -- (10,2.5) ;
\draw[decorate with=dart,decoration={shape size=.5cm,shape sep={1cm,between centers}}] (0,2.5) -- (10,2.5);
\end{tikzpicture}
\\ \hline  
\RDD{between borders}
&  
\begin{tikzpicture}
\draw[dotted,red] (0,2.5) -- (10,2.5) ;
\draw[decorate with=dart,decoration={shape size=.5cm ,shape sep={1cm,between borders}}] (0,2.5) -- (10,2.5);
\end{tikzpicture}
\\ \hline 
\multicolumn{2}{|c|}{\dft :  between centers }
 \\ \hline 
\end{tabular}

\bigskip


\begin{tabular}{|l|c|} \hline
\multicolumn{2}{|c|}{  \TFRGB{Espacement automatique}{Automatic spacing } }
\\ \hline 
 
\multicolumn{2}{|c|}{\BSS{draw}[decorate with=dart,decoration=\{shape size=.5cm,}\\
\multicolumn{2}{|c|}{\RDD{shape evenly spread}=5\}] (0,0) -  - (10,0);}
 \\ \hline 
shape evenly spread=5
&  
\begin{tikzpicture}
\draw[dotted,red] (0,0) -- (10,0) ; \draw[decorate with=dart,decoration={shape size=.5cm,shape evenly spread=5}] (0,0) -- (10,0);
\end{tikzpicture}
\\ \hline  
shape evenly spread=10
&  
\begin{tikzpicture}
\draw[dotted,red] (0,2.5) -- (10,2.5) ;
\draw[decorate with=dart,decoration={shape size=.5cm,shape evenly spread=10}] (0,2.5) -- (10,2.5); 
\end{tikzpicture}

\\ \hline 
\end{tabular} 


\paragraph{Orientation}:


\begin{tabular}{|c|c|} \hline 
\multicolumn{2}{|c|}{ '' shape border rotate `` }
\\ \hline
shape border rotate=90 
& 
\tikz \draw[decorate with=dart,shape border rotate=90,decoration={shape sep=1cm,shape width=.5cm}] (0,0) -- (10,0); 
\\ \hline  
shape border rotate=45
& 
\tikz \draw[decorate with=dart,shape border rotate=45,decoration={shape sep=1cm,shape width=.5cm}] (0,0) -- (10,0); 
\\ \hline  
shape border rotate=180
& 
\tikz \draw[decorate with=dart,shape border rotate=180,decoration={shape sep=1cm,shape width=.5cm}] (0,0) -- (10,0); 
\\ \hline 
\end{tabular} 

 \bigskip


\begin{tabular}{|c|c|} \hline 
\multicolumn{2}{|c|}{ \og shape sloped \fg }
\\ \hline
 \multicolumn{2}{|c|}{\BSS{draw}[decorate with=dart,decoration=\{shape width=.5cm,shape sep=1cm, }\\
 \multicolumn{2}{|c|}{  \RDD{shape sloped}=true \}] (0,0) - - (3,3);}
  \\ \hline
\begin{tikzpicture}
\draw[dotted,red] (0,0) -- (3,3);
\draw[decorate with=dart,decoration={shape width=.5cm ,shape sep=1cm,shape sloped=true}] (0,0) -- (3,3);
\end{tikzpicture}
&  
\begin{tikzpicture}
\draw[dotted,red] (0,0) -- (3,3);
 \draw[decorate with=dart,decoration={shape width=.5cm ,shape sep=1cm,shape sloped=false}] (0,0) -- (3,3);
\end{tikzpicture}
\\ \hline  
shape sloped=true
&  
shape sloped=false
\\ \hline
\multicolumn{2}{|c|}{\dft :  shape sloped=true }
 \\ \hline  
\end{tabular} 
 \bigskip

\begin{tabular}{|c|c|} \hline 
 \multicolumn{2}{|c|}{\BSS{draw}[decorate with=dart,decoration=\{shape width=.5cm,shape sep=1cm, }\\
 \multicolumn{2}{|c|}{  \RDD{shape sloped}=true\}] (0,0)  arc (0:180:3 and 2);}
  \\ \hline
\begin{tikzpicture}
\draw[dotted,red]  (0,0)  arc (0:180:3 and 2);
\draw[decorate with=dart,decoration={shape width=.5cm ,shape sep=1cm,shape sloped=true}] (0,0)  arc (0:180:3 and 2);
\end{tikzpicture}
&  
\begin{tikzpicture}
\draw[dotted,red] (0,0)  arc (0:180:3 and 2);
 \draw[decorate with=dart,decoration={shape width=.5cm ,shape sep=1cm,shape sloped=false}] (0,0)  arc (0:180:3 and 2);
\end{tikzpicture}
\\ \hline  
shape sloped=true
&  
shape sloped=false
\\ \hline
\multicolumn{2}{|c|}{\dft :  shape sloped=true }
 \\ \hline  
\end{tabular} 

 \bigskip

\begin{tabular}{|c|c|} \hline  
 \multicolumn{2}{|c|}{\BSS{draw}[decorate with=dart,decoration=\{shape width=.5cm,shape sep=1cm, }\\
 \multicolumn{2}{|c|}{  \RDD{shape border rotate}=90,shape sloped=true \}] (0,0) - - (3,3);}
  \\ \hline

\begin{tikzpicture}
\draw[dotted,red] (0,0) -- (3,3);
\tikz \draw[decorate with=dart,shape border rotate=90,decoration={shape sep=1cm,shape sloped=true,shape width=.5cm}] (0,0) -- (3,3);
\end{tikzpicture}
&  
\begin{tikzpicture}
\draw[dotted,red] (0,0) -- (3,3);
\draw[decorate with=dart,shape border rotate=90,decoration={shape sep=1cm,shape sloped=false,shape width=.5cm}] (0,0) -- (3,3);
\end{tikzpicture}
\\ \hline  
shape sloped=true
&  
shape sloped=false
\\ \hline
\end{tabular} 

 \bigskip



\begin{tabular}{|c|c|}\hline
\multicolumn{2}{|c|}{ \og shift only \fg }
\\ \hline
 \multicolumn{2}{|c|}{  decoration={
 {\color{red}transform=\AC{shift only}},shape width=5mm,segment length=.5cm,shape sep=1cm}}
  \\ \hline
\begin{tikzpicture}
\draw (0,0)  arc (0:180:3 and 2);
\draw[decorate with=dart,decoration={
transform={shift only},shape width=5mm,segment length=.5cm,shape sep=1cm}]
(0,0)  arc (0:180:3 and 2);
\end{tikzpicture}
& 
\begin{tikzpicture} 
\draw (0,0)  arc (0:180:3 and 2); 
\draw[decorate with=dart,decoration={
shape width=5mm,segment length=.5cm,shape sep=1cm}]
(0,0)  arc (0:180:3 and 2);
\end{tikzpicture}
\\ \hline  
avec &  sans\\ 
\hline 
\end{tabular}


\bigskip

\begin{tabular}{|c|c|} \hline 
\multicolumn{2}{|c|}{ Dimensions }
\\ \hline 
\multicolumn{2}{|c|}{\BSS{draw}[decorate with=dart,decoration=\{shape size=.5cm,}\\
\multicolumn{2}{|c|}{\RDD{shape height}= 1cm \}] (0,0) -  - (10,0);}
\\ \hline 
\RDD{shape height}=1cm
&
\begin{tikzpicture} [baseline=0pt] 
\draw[decorate with=dart,decoration={shape sep=1cm,shape height=1cm}] (0,0) -- (10,0);
\end{tikzpicture}
\\ \hline
\RDD{shape width}=1cm
&
\begin{tikzpicture} [baseline=0pt] 
\draw[decorate with=dart,decoration={shape sep=1cm,shape width=1cm,shape scaled}] (0,0) -- (10,0);
\end{tikzpicture}
\\ \hline 
\RDD{shape size}=1cm
&
\begin{tikzpicture} [baseline=0pt] 
\draw[decorate with=dart,decoration={shape sep=1cm,shape size=1cm,shape scaled}] (0,0) -- (10,0);
\end{tikzpicture}
\\ \hline 
\end{tabular} 


 \bigskip


\begin{tabular}{|l|c|} \hline 
\multicolumn{2}{|c|}{\BSS{draw}[decorate with=dart,decoration=\{shape size=.5cm,}\\
\multicolumn{2}{|c|}{\RDD{shape start size}=1cm,\RDD{shape scaled} \}] (0,2.5) -  - (10,2.5);}
 \\ \hline 
 
\RDD{shape start size}=1cm
&  
\tikz \draw[decorate with=dart,decoration={shape sep=1cm,shape start size=1cm,shape scaled}] (0,0) -- (10,0);
\\ \hline  
\RDD{shape start height}=1cm
&  
\tikz \draw[decorate with=dart,decoration={shape sep=1cm,shape start height=1cm,shape scaled}] (0,0) -- (10,0);

\\ \hline  
\RDD{shape start width}=1cm
&  
\tikz \draw[decorate with=dart,decoration={shape sep=1cm,shape start width=1cm,shape scaled}] (0,0) -- (10,0);
\\ \hline  
\RDD{shape end size}=1cm
&  
\tikz \draw[decorate with=dart,decoration={shape sep=1cm,shape end size=1cm,shape scaled}] (0,0) -- (10,0);

\\ \hline  
\RDD{shape end height}=1cm
&  
\tikz \draw[decorate with=dart,decoration={shape sep=1cm,shape end height=1cm,shape scaled}] (0,0) -- (10,0);
\\ \hline  
\RDD{shape end width}=1cm
&  
\tikz \draw[decorate with=dart,decoration={shape sep=1cm,shape end width=1cm,shape scaled}] (0,0) -- (10,0);
\\ \hline 
\end{tabular} 


\newpage
\subsection{Library \og decorations.text \fg }


 \maboite{\BS{usetikzlibrary}\AC{decorations.text}}
\label{lib-text}

\begin{center}
\RRR{48-6}
\end{center}


\begin{tabular}{|c|} \hline  
 \BS{draw}[decorate,decoration=\AC{{\color{red}text along path,text=\AC{texte}}}] (1,1) circle (1); 
\\ \hline  
\begin{tikzpicture}
\draw[dotted] (1,1) circle (1);
\draw[decorate,decoration={text along path,text={texte}}] (1,1) circle (1); 
\end{tikzpicture}
\\ \hline 
\end{tabular} 


\begin{tabular}{|c|} \hline 
\multicolumn{1}{|c|}{ \TFRGB{Texte trop long}{Text too long} }
\\ \hline 
 
 \BS{draw}[decorate,decoration=\AC{text along path,\\ 
 text=\AC{Un Deux Trois Quatre Cinq Six sept Huit Neuf Dix}}] (1,1) circle (1);
\\ \hline  
\begin{tikzpicture}
\draw[dotted] (1,1) circle (1);
 \draw[decorate,decoration={text along path,text={Un Deux Trois Quatre Cinq Six sept Huit Neuf Dix}}] (1,1) circle (1); 
\end{tikzpicture}
\\ \hline 
\end{tabular} 


\begin{tabular}{|c|c|c|} \hline
\multicolumn{3}{|c|}{ \TFRGB{Format du texte}{Text format}} 
\\ \hline
\multicolumn{3}{|c|}{\BS{draw} [decorate,decoration=\AC{text along path,
text={avant {\color{red}|\BS{red} | texte | |} après }}]}  
\\ \hline
\begin{tikzpicture}
\draw[dotted] (1,1) circle (1);
\draw [decorate,decoration={text along path,
text={avant |\color{red} |texte|| après }}]
(1,1) circle (1);
\end{tikzpicture}
&  
\begin{tikzpicture}
\draw[dotted] (1,1) circle (1);
\draw [decorate,decoration={text along path,
text={|\color{red} |texte||  }}]
(1,1) circle (1);
\end{tikzpicture}
&  
\begin{tikzpicture}
\draw[dotted] (1,1) circle (1);
\draw [decorate,decoration={text along path,
text={|\color{red} |texte|| {} }}]
(1,1) circle (1);
\end{tikzpicture}
\\ \hline  
text=\AC{avant |\BS{red}|texte|| après }  
&  
text=\AC{ |\BS{red}|texte|| } 
&  
text=\AC{ |\BS{red}|texte|| \AC{} } 
\\ \hline 
\end{tabular} 

\bigskip
\begin{tabular}{|c|c|c|} \hline

\begin{tikzpicture}
\draw[dotted] (1,1) circle (1);
\draw [decorate,decoration={text along path,
text={ avant |\color{red}  | texte || après }}]
(1,1) circle (1);
\end{tikzpicture}
&  
\begin{tikzpicture}
\draw[dotted] (1,1) circle (1);
\draw [decorate,decoration={text along path,
text={ avant |\it| texte || après }}]
(1,1) circle (1);
\end{tikzpicture}
&  
\begin{tikzpicture}
\draw[dotted] (1,1) circle (1);
\draw [decorate,decoration={text along path,
text={ avant |\Huge| texte || après }}]
(1,1) circle (1);
\end{tikzpicture}
\\ \hline 
avant | {\color{red} \BS{red}}| texte || après  
& avant | {\color{red}\BS{it}}| texte || après 
& avant | {\color{red}\BS{Huge}}| texte || après  \\ 
\hline 
\end{tabular} 

\bigskip


\begin{tabular}{|c|} \hline  
\BS{draw} [decorate,decoration=\AC{text along path,\\
text=\AC{avant |{\color{red}\BS{Large}}|Visual |{\color{red}+\BS{bf}\BS{color}}\AC{red}|Tikz|| après  }}]
(1,1) circle (1);
\\ \hline  
\begin{tikzpicture}
\draw[dotted] (1,1) circle (1);
\draw [decorate,decoration={text along path,
text={avant |\Large|Visual |+\bf\color{red}|Tikz|| après  }}]
(1,1) circle (1);
\end{tikzpicture}
\\ \hline 
\end{tabular} 


\bigskip

\begin{tabular}{|c|} \hline  
\BS{draw} [decorate,decoration=\AC{text along path,{\color{red}text format delimiters=\AC{[}\AC{]}},\\
text=\AC{ {\color{red} [} \BS{red} {\color{red}  ]} texte {\color{red} [ ]}  }}]
(1,1) circle (1);
\\ \hline  
\begin{tikzpicture}
\draw[dotted] (1,1) circle (1);
\draw [decorate,decoration={text along path,text format delimiters={[}{]},
text={  [ \color{red}  ] texte []  }}]
(1,1) circle (1);
\end{tikzpicture}
\\ \hline 
\end{tabular} 

\bigskip

\begin{tabular}{|c|} \hline
\multicolumn{1}{|c|}{ \TFRGB{Sens du texte}{Text orientation} }
\\ \hline 
\BS{draw}[decorate,decoration=\AC{text along path,text=\AC{texte},\\
text color=blue, \RDD{reverse path} }] (1,1) circle (1);
\\ \hline  
\begin{tikzpicture}
\draw[dotted] (1,1) circle (1);
\draw[decorate,decoration={text along path,text={texte},text color=red}] (1,1) circle (1);
\draw[decorate,decoration={text along path,text={texte},text color=blue, reverse path}] (1,1) circle (1); 
\end{tikzpicture}
\\ \hline 
\end{tabular} 

\bigskip

\begin{tabular}{|c|c|c|} \hline 
\multicolumn{3}{|c|}{ \TFRGB{Position du texte}{Text position} }
\\ \hline  
\multicolumn{3}{|c|}{  \BS{draw}[decorate,decoration=\{ text along path,text=\AC{texte},}\\
\multicolumn{3}{|c|}{  {\color{red}text align=\AC{align=left}}\}] (1,1) circle (1);}  
\\ \hline 
\begin{tikzpicture}
\draw[dotted] (1,1) circle (1);
 \draw[decorate,decoration={text along path,text={texte},text align={align=left}}] (1,1) circle (1); 
\end{tikzpicture}
&  
\begin{tikzpicture}
\draw[dotted] (1,1) circle (1);
 \draw[decorate,decoration={text along path,text={texte},text align={align=center}}] (1,1) circle (1); 
\end{tikzpicture}
&  
\begin{tikzpicture}
\draw[dotted] (1,1) circle (1);
\draw[decorate,decoration={text along path,text={texte},text align={align=right}}] (1,1) circle (1); 
\end{tikzpicture}
\\ \hline  

align=\AC{\RDD{align}=\RDDX{left}{align}} & align=\AC{\RDD{align}=\RDDX{center}{align}} & align=\AC{\RDD{align}=\RDDX{right}{align}} \\ 
\hline 
\end{tabular} 

\bigskip

\begin{tabular}{|c|c|} \hline 
 \multicolumn{2}{|c|}{  \BS{draw}[ decorate,decoration=\{text along path,text=\AC{texte},} \\
 \multicolumn{2}{|c|}{  text align=\AC{align=left,\RDD{left indent}=1cm} \} ] (1,1) circle (1);}  
 \\ \hline 
\begin{tikzpicture}
\draw[dotted] (1,1) circle (1);
\tikz \draw[decorate,decoration={text along path,text={texte},text align={align=left,left indent=1cm}}] (1,1) circle (1); 
\end{tikzpicture}
&  
\begin{tikzpicture}
\draw[dotted] (1,1) circle (1);
\tikz \draw[decorate,decoration={text along path,text={texte},text align={align=right,right indent=1cm}}] (1,1) circle (1); 
\end{tikzpicture}
\\  \hline
 align=\AC{align=left,\RDD{left indent}=1cm}
&  
align=\AC{align=right,\RDD{right indent}=1cm}
 \\ \hline 
\end{tabular} 


\begin{tabular}{|c|} \hline 
\multicolumn{1}{|c|}{ \TFRGB{Justification du texte}{Fit to path} }
\\ \hline 
 
\BS{draw} [decoration=\AC{text along path, text=\AC{Un deux trois quatre },\\
text align=\AC{\RDD{fit to path}}}, decorate]
(1,1) circle (1);
\\ \hline  
\begin{tikzpicture}
\draw[dotted] (1,1) circle (1);
\draw [decoration={text along path, text={Un deux trois quatre },text align={fit to path}}, decorate]
(1,1) circle (1);
\end{tikzpicture}
\\ \hline 
\end{tabular} 



\begin{tabular}{|c|} \hline
\multicolumn{1}{|c|}{ \TFRGB{Justification des espaces}{Fit to path stretching spaces} }
\\ \hline  
\BS{draw} [decoration=\AC{text along path, text=\AC{Un deux trois quatre },\\
text align=\AC{\RDD{fit to path stretching spaces}}}, decorate]
(1,1) circle (1);
 \\ \hline  
\begin{tikzpicture}
\draw[dotted] (1,1) circle (1);
\draw [decoration={text along path, text={Un deux trois quatre },text align={fit to path stretching spaces}}, decorate]
(1,1) circle (1);
\end{tikzpicture}
 \\ \hline 
\end{tabular} 

\newpage

\subsection{Library \og decorations.fractals \fg }


 \maboite{\BS{usetikzlibrary}\AC{decorations.fractals}}
\label{lib-fractals}

\begin{center}
\RRR{48-7}
\end{center}
 \bigskip

\begin{tabular}{|c|c|c|c|} \hline 
\multicolumn{4}{|c|}{\BSS{draw}[decorate,decoration=\RDD{Koch curve type 1}] (0,0) - - (3,0);}
 \\ \hline 
\tikz \draw[decorate,decoration=Koch curve type 1] (0,0) -- (3,0);
 &  
\tikz \draw[decorate,decoration=Koch curve type 2] (0,0) -- (3,0);
 &  
\tikz \draw[decorate,decoration=Koch snowflake] (0,0) -- (3,0);
 &
\tikz \draw[decorate,decoration=Cantor set] (0,0) -- (3,0); 
 \\ \hline  
 
\RDD{Koch curve type 1}  & \RDD{Koch curve type 2}  & \RDD{Koch snowflake} & \RDD{Cantor set} \\ 
\hline 
\end{tabular} 

 \bigskip


\begin{tabular}{|c|c|c|c|} \hline 
\multicolumn{4}{|c|}{\BS{begin}\AC{tikzpicture}[decoration=Koch curve type 1] }
 \\ 
\multicolumn{4}{|c|}{\BS{draw} \color{green}  decorate \AC{ \color{red}  decorate \AC{ \color{black}  (0,0) -- (3,0) \color{red}  }\color{green} };}
 \\ 
\multicolumn{4}{|c|}{\BS{end}\AC{tikzpicture}}  \\  \hline 
\begin{tikzpicture}[decoration=Koch curve type 1]
\draw decorate{ decorate{ (0,0) -- (3,0) }};
\end{tikzpicture}
&  
\begin{tikzpicture}[decoration=Koch curve type 2]
\draw decorate{ decorate{ (0,0) -- (3,0) }};
\end{tikzpicture}
&  
\begin{tikzpicture}[decoration=Koch snowflake]
\draw decorate{ decorate{ ((0,0) -- (3,0) }};
\end{tikzpicture}
&  
\begin{tikzpicture}[decoration=Cantor set]
\draw decorate{ decorate{ (0,0) -- (3,0)}};
\end{tikzpicture}
\\ \hline  &  &  &  \\ 
\hline 
Koch curve type 1  & Koch curve type 2  & Koch snowflake & Cantor set \\ 
\hline 
\end{tabular} 

 \bigskip



\begin{tabular}{|c|c|c|c|} \hline  
\multicolumn{4}{|c|}{\BSS{draw} \color{green} decorate \AC{ \color{red}  decorate \AC{ \color{blue} decorate \AC{ \color{black} (0,0) - - (3,0) \color{blue} } \color{red}  } \color{green}};}
 \\  \hline 
 
\begin{tikzpicture}[decoration=Koch curve type 1]
\draw decorate{ decorate{ decorate{ (0,0) -- (3,0) }}};
\end{tikzpicture}
&  
\begin{tikzpicture}[decoration=Koch curve type 2]
\draw decorate{ decorate{ decorate{ (0,0) -- (3,0) }}};
\end{tikzpicture}
&  
\begin{tikzpicture}[decoration=Koch snowflake ]
\draw decorate{ decorate{ decorate{ (0,0) -- (3,0) }}};
\end{tikzpicture}
&  
\begin{tikzpicture}[decoration=Cantor set]
\draw decorate{ decorate{ decorate{ (0,0) -- (3,0) }}};
\end{tikzpicture}
\\ \hline 
Koch curve type 1  & Koch curve type 2  & Koch snowflake & Cantor set \\ 
\hline 
\end{tabular} 

 \bigskip

\begin{tabular}{|c|c|c|c|} \hline 
\begin{tikzpicture}[decoration=Koch snowflake,draw=blue,fill=blue!20,thick]
\filldraw  (0,0) -- ++(60:3) -- ++(-60:3) -- cycle ;
\end{tikzpicture}
& 
 
\begin{tikzpicture}[decoration=Koch snowflake,draw=blue,fill=blue!20,thick]
\filldraw decorate{ (0,0) -- ++(60:3) -- ++(-60:3) -- cycle };

\end{tikzpicture}
&  
\begin{tikzpicture}[decoration=Koch snowflake,draw=blue,fill=blue!20,thick]
\filldraw decorate{ decorate{ (0,0) -- ++(60:3) -- ++(-60:3) -- cycle }};
\end{tikzpicture}
&  
\begin{tikzpicture}[decoration=Koch snowflake,draw=blue,fill=blue!20,thick]
\filldraw decorate{ decorate{ decorate{ (0,0) -- ++(60:3) -- ++(-60:3) -- cycle }}};
\end{tikzpicture}
\\ \hline  
sans & 1 decorate
&  2 decorate &  3 decorate \\ \hline
\end{tabular}

\newpage


\subsection{Applications}


\SbSbSSCT{Décoration d'un n\oe ud}{Node decoration}

\begin{tabular}{|c|c|} \hline 
 \multicolumn{2}{|c|}{ \BS{node} [draw,decorate,decoration=\{bumps,
  minimum height=2cm, minimum width=3cm\}]
 \AC{texte}; }
  \\  \hline 
\begin{tikzpicture}
\node [fill=green,draw,decorate,decoration={bumps},
 minimum height=2cm, minimum width=3cm,]
{texte};
\end{tikzpicture}
&  
\begin{tikzpicture}
\node [fill=green,draw,decorate,decoration=footprints,
 minimum height=2cm, minimum width=3cm]
{texte};
\end{tikzpicture}
\\ \hline  
decoration=\RDD{bumps}&  decoration=\RDD{footprints} \\ \hline 

\begin{tikzpicture}
\node [fill=green,draw,thick, minimum height=2cm, minimum width=3cm,decorate, decoration={random steps, amplitude=1pt}] {texte};
\end{tikzpicture}
&
\begin{tikzpicture}[decoration={random steps,segment length=3pt , amplitude=2pt}]
\node at (0,0) [fill=green,decorate,starburst,
 minimum height=2cm, minimum width=3cm] {Texte};
\end{tikzpicture}
\\ \hline
 decoration=\{random steps , amplitude = 1pt \} 
 &
  starburst,decoration=\{random steps, \\
  &
 segment length=3pt , amplitude=2pt\} 
\\ \hline 
\begin{tikzpicture}
\node at (0,0) [fill=green,decorate,ellipse,decoration=zigzag,
 minimum height=2cm, minimum width=3cm] {Texte};
\end{tikzpicture}
&  
\begin{tikzpicture}
\node at (0,0) [inner sep=6mm,fill=green,decorate,ellipse,decoration=
{text along path,text={Un Deux Trois Quatre Cinq Six Sept Huit Neuf}}] {texte};
\end{tikzpicture}
\\ \hline  
 ellipse,decoration=zigzag &  decoration= \{text along path,text= \\

 & \AC{Un Deux Trois Quatre Cinq Six Sept Huit Neuf} \}
\\ \hline 
\end{tabular} 

\SbSbSSCT{Décoration de liaisons de noeuds}{Node link decoration}

\begin{tabular}{|c|c|c|} \hline 
\multicolumn{3}{|c|}{\BS{draw} [decorate,decoration=snake](A) -- (B);}
 \\  \hline 
\begin{tikzpicture}[blue]
\node[draw] (A) at (0,0) {A};
\node[draw] (B) at (2,2) {B};
\draw [decorate,decoration=snake](A) -- (B);
\end{tikzpicture}
&  
\begin{tikzpicture}[blue]
\node[draw] (A) at (0,0) {A};
\node[draw] (B) at (2,2) {B};
\draw [decorate,decoration=coil](A) |- (B);
\end{tikzpicture}
&  
\begin{tikzpicture}[blue]
\node[draw] (A) at (0,0) {A};
\node[draw] (B) at (2,2) {B};
\draw [decorate,decoration=footprints](A) -| (B);
\end{tikzpicture}
\\ \hline 
decoration=snake & decoration=coil & decoration=footprints \\
(A){\color{red} - -} (B) & (A) {\color{red}|-} (B) &  (A) {\color{red}-|} (B)
\\ \hline 
\begin{tikzpicture}[blue]
\node[draw] (A) at (0,0) {A};
\node[draw] (B) at (2,2) {B};
\draw [decorate,decoration=coil] (A) to [bend right] (B);
\end{tikzpicture}
&  
\begin{tikzpicture}[blue]
\node[draw] (A) at (0,0) {A};
\node[draw] (B) at (2,2) {B};
\draw[decorate,decoration=zigzag] (A) to[bend left=120]  (B);
\end{tikzpicture} 
&  
\begin{tikzpicture}[blue]
\node[draw] (A) at (0,0) {A};
\node[draw] (B) at (2,2) {B};
\draw [decorate,decoration=ticks](A) to[out=30] (B);
\end{tikzpicture} 
\\ \hline  
decoration=coil & decoration=zigzag & decoration=ticks \\
(A) to [bend right] (B) & (A) to[bend left=120]  (B) & (A) to[out=30] (B)
\\ \hline 

\end{tabular} 

 \bigskip 


\SbSbSSCT{Décoration d'un graphe}{Graph decoration}

\begin{tabular}{|c|c|}\hline 
\multicolumn{2}{|c|}{\BSS{draw}[
\RDD{decorate},decoration=footprints] plot coordinates {(0,0) (2,1) (4,-2)  (6,1) };}
 \\ \hline  
\begin{tikzpicture}[baseline=0pt]
\draw[->,blue] (-.1,0) -- (6,0);
\draw[->,blue] (0,-2.5) -- (0,2.5);
 \draw[red,dashed] plot coordinates {(0,0) (2,1) (4,-2)  (6,1) };
 \draw[decorate,decoration=footprints] plot coordinates {(0,0) (2,1) (4,-2)  (6,1) };
\end{tikzpicture}
&  
\begin{tikzpicture}[domain=0:6.28,x=0.7cm,baseline=0pt]
\draw[->,blue] (-.1,0) -- (7,0);
\draw[->,blue] (0,-2.5) -- (0,2.5);
\draw[red,dashed] plot  (\x,{sin(\x r)});
\draw[decorate,decoration=footprints] plot  (\x,{sin(\x r)});
\end{tikzpicture}
\\ \hline  
plot coordinates {(0,0) (2,1) (4,-2)  (6,1) } 
&
 plot  (\BS{x},\AC{sin(\BS{x} r)})
\\ \hline 
\end{tabular} 


\SbSbSSCT{Décorations variables}{Various decoration}

\begin{tabular}{|c|} \hline 
\BS{draw} [decorate, 
decoration=\AC{zigzag,\RDD{pre}=footprints,\RDD{pre length}=5cm}](0,0) -- (10,0);
\\ \hline  
\begin{tikzpicture}[baseline=0pt]
\draw[red!20] (0,-0.5) grid (10,0.5);
\draw[dotted,red] (0,0) -- (10,0);
\draw [decorate,decoration={zigzag,pre=footprints,pre length=5cm}] (0,0) -- (10,0);
\end{tikzpicture}
\\ \hline  
decoration=\AC{zigzag,\RDD{pre}=footprints,\RDD{pre length}=5cm}
\\ \hline  
\begin{tikzpicture}[baseline=0pt]
\draw[red!20] (0,-0.5) grid (10,0.5);
\draw[dotted,red] (0,0) -- (10,0);
\draw [decorate,decoration={zigzag,post=footprints,post length=5cm}] (0,0) -- (10,0);
\end{tikzpicture}
\\ \hline  
decoration=\AC{zigzag,\RDD{post}=footprints,\RDD{post length}=5cm}
\\ \hline  
\begin{tikzpicture}[baseline=0pt]
\draw[red!20] (0,-0.5) grid (10,0.5);
\draw[dotted,red] (0,0) -- (10,0);
\draw [decorate,decoration={zigzag,pre=footprints,pre length=3cm,post=expanding waves,post length=3cm}] (0,0) -- (10,0);
\end{tikzpicture}
\\ \hline  
decoration=\AC{zigzag,\RDD{pre}=footprints,\RDD{pre length}=3cm,
,\RDD{post}=expanding waves,\RDD{post length}=3cm}
\\ \hline 
\end{tabular} 


\SbSbSSCT{Décoration partielle}{Partial decoration}

\begin{tabular}{|c|l|} \hline  
\begin{tikzpicture}[baseline=0pt]
\draw [decorate,decoration={zigzag}]
 (0,0) -- (2,0) -- (2,1) -- (0,1)-- cycle;
 \end{tikzpicture}
&
\BS{draw} [decorate,decoration=zigzag]
 (0,0) -- (2,0) -- (2,1) -- (0,1)-- cycle;
\\ 
\hline 
  
\begin{tikzpicture}[baseline=0pt]
 \draw  [decoration=zigzag]
 (0,0) -- (2,0) decorate{ -- (2,1)} -- (0,1)-- cycle;
  \end{tikzpicture}
&
\BS{draw} [decoration=zigzag]
 (0,0) -- (2,0) \RDD{decorate}\AC{-- (2,1)} -- (0,1)-- cycle;
\\ 
\hline 
\begin{tikzpicture}[baseline=0pt]
 \draw  [decoration={zigzag}]
 (0,0) -- (2,0)  -- (2,1) decorate{-- (0,1)}-- cycle;
  \end{tikzpicture}
&
\BS{draw} [decorate,decoration=zigzag]
 (0,0) -- (2,0) -- (2,1) -- \RDD{decorate}\AC{(0,1)}-- cycle;
\\ \hline 
\begin{tikzpicture}[baseline=0pt]
 \draw  [decoration={zigzag}]
 (0,0) decorate{-- (2,0)}  -- (2,1) decorate{-- (0,1)}-- cycle;
  \end{tikzpicture}
&
\BS{draw} [decorate,decoration=zigzag]
 (0,0) \RDD{decorate}\AC{-- (2,0)} -- (2,1) -- \RDD{decorate}\AC{(0,1)}-- cycle;
\\ \hline 
 
\end{tabular}

 \newpage



\begin{tabular}{|c|} \hline 
\og lineto \fg
\textbf{}
\BS{draw} [decorate, 
decoration=\AC{zigzag,lineto,\RDD{pre length}=5cm}](0,0) -- (10,0);
\\ \hline  
\begin{tikzpicture}[baseline=0pt]
\draw[red!20] (0,-0.5) grid (10,0.5);
\draw[dotted,red] (0,0) -- (10,0);]
\draw [decorate,decoration={zigzag,pre=lineto,pre length}=5cm] (0,0) -- (10,0);
\end{tikzpicture}
\\ \hline  
decoration=\AC{ zigzag,\RDD{pre}=lineto,\RDD{pre length}=5cm }
\\ \hline  
\begin{tikzpicture}[baseline=0pt]
\draw[red!20] (0,-0.5) grid (10,0.5);
\draw[dotted,red] (0,0) -- (10,0);
\draw [decorate,decoration={zigzag,post=lineto,post length=5cm}] (0,0) -- (10,0);
\end{tikzpicture}
\\ \hline  
decoration=\AC{zigzag,\RDD{post}=lineto,\RDD{post length}=5cm}
\\ \hline  
\begin{tikzpicture}[baseline=0pt]
\draw[red!20] (0,-0.5) grid (10,0.5);
\draw[dotted,red] (0,0) -- (10,0);
\draw [decorate,decoration={zigzag,pre=lineto,pre length=3cm,post=lineto,post length=3cm}] (0,0) -- (10,0);
\end{tikzpicture}
\\ \hline  
decoration=\AC{zigzag,\RDD{pre}=lineto,\RDD{pre length}=3cm,
,\RDD{post}=curveto,\RDD{post length}=3cm}
\\ \hline 
\end{tabular} 


 \bigskip



\begin{tabular}{|c|} \hline 
\og curveto \fg
\\ \hline 
\BS{draw} [decorate, 
decoration=\AC{zigzag,\RDD{pre}=curveto,\RDD{pre length}=5cm}](0,0) -- (10,0);
\\ \hline  
\begin{tikzpicture}[baseline=0pt]
\draw[red!20] (0,-0.5) grid (10,0.5);
\draw[dotted,red] (0,0) -- (10,0);]
\draw [decorate,decoration={zigzag,pre=curveto,pre length=5cm}] (0,0) -- (10,0);
\end{tikzpicture}
\\ \hline  
decoration=\AC{zigzag,\RDD{pre}=curveto,\RDD{pre length}=5cm}
\\ \hline  
\begin{tikzpicture}[baseline=0pt]
\draw[red!20] (0,-0.5) grid (10,0.5);
\draw[dotted,red] (0,0) -- (10,0);
\draw [decorate,decoration={zigzag,post=curveto,post length=5cm}] (0,0) -- (10,0);
\end{tikzpicture}
\\ \hline  
decoration=\AC{zigzag,\RDD{post}=curveto,\RDD{post length}=5cm}
\\ \hline  
\begin{tikzpicture}[baseline=0pt]
\draw[red!20] (0,-0.5) grid (10,0.5);
\draw[dotted,red] (0,0) -- (10,0);
\draw [decorate,decoration={zigzag,pre=curveto,pre length=3cm,post=curveto,post length=3cm}] (0,0) -- (10,0);
\end{tikzpicture}
\\ \hline  
decoration=\AC{zigzag,\RDD{pre}=curveto,\RDD{pre length}=3cm,
,\RDD{post}=curveto,\RDD{post length}=3cm}
\\ \hline 
\end{tabular}

 \bigskip



\label{moveto}
\begin{tabular}{|c|} \hline
\og  moveto \fg 
\\ \hline 
\BS{draw} [decorate, 
decoration=\AC{zigzag,\RDD{pre}=moveto,\RDD{pre length}=5cm}](0,0) -- (10,0);
\\ \hline  
\begin{tikzpicture}[baseline=0pt]
\draw[red!20] (0,-0.5) grid (10,0.5);
\draw[dotted,red] (0,0) -- (10,0);]
\draw [decorate,decoration={zigzag,pre=moveto,pre length=5cm}] (0,0) -- (10,0);
\end{tikzpicture}
\\ \hline  
decoration=\AC{zigzag,\RDD{pre}=moveto,\RDD{pre length}=5cm}
\\ \hline  
\begin{tikzpicture}[baseline=0pt]
\draw[red!20] (0,-0.5) grid (10,0.5);
\draw[dotted,red] (0,0) -- (10,0);
\draw [decorate,decoration={zigzag,post=moveto,post length=5cm}] (0,0) -- (10,0);
\end{tikzpicture}
\\ \hline  
decoration=\AC{zigzag,\RDD{post}=moveto,\RDD{post length}=5cm}
\\ \hline  
\begin{tikzpicture}[baseline=0pt]
\draw[red!20] (0,-0.5) grid (10,0.5);
\draw[dotted,red] (0,0) -- (10,0);
\draw [decorate,decoration={zigzag,pre=moveto,pre length=3cm,post=moveto,post length=3cm}] (0,0) -- (10,0);
\end{tikzpicture}
\\ \hline  
decoration=\AC{zigzag,\RDD{pre}=moveto,\RDD{pre length}=3cm,
,\RDD{post}=moveto,\RDD{post length}=3cm}
\\ \hline 
\end{tabular}

\SbSbSSCT{Paramètres globaux ou particuliers}{Global and  partial parameters }

\begin{tabular}{|l|} \hline  
\begin{tikzpicture}[baseline=0pt,ultra thick,decoration={straight zigzag,amplitude=0.5cm,segment length=1cm}]
\draw[red!20,ultra thin] (0,-2) grid (10,3);
\draw[magenta] (0,2) --  (10,2);
\draw[blue,decorate] (0,1) -- (10,1);
\draw[red,decorate,decoration=saw] (0,0) -- (10,0);
\draw[cyan,decorate,decoration={meta-segment length=2cm}] (0,-1) -- (10,-1);
\end{tikzpicture}

\\ \hline 
\BS{begin}\AC{tikzpicture}[baseline=0pt,ultra thick,\\
{\color{red}decoration=\AC{straight zigzag,amplitude=0.5cm,segment length=1cm}}] \\
\BS{draw}[red!20,ultra thin] (0,-2) grid (10,3); \\
\BS{draw}[magenta] (0,2) --  (10,2); \\
\BS{draw}[blue,\RDD{decorate}] (0,1) -- (10,1); \\
\BS{draw}[red,\AC{\color{red}decorate,decoration=saw}] (0,0) -- (10,0); \\
\BS{draw}[cyan,{\color{red}decorate,decoration={meta-segment length=2cm}}] (0,-1) -- (10,-1); \\
\BS{end}\AC{tikzpicture}

 \\ \hline 
\end{tabular} 


\SbSbSSCT{Tracer le chemin et sa décoration avec  \og  Postaction \fg }{Path and its decoration \og  Postaction \fg }

\begin{tabular}{|c|c|}\hline  
\begin{tikzpicture}[baseline=0pt]
\draw [postaction={decorate,blue,draw,ultra thick,decoration={straight zigzag,amplitude=0.5cm}}][red,line width=10pt] (0,0)  arc (0:180:3 and 2);
\end{tikzpicture}
&  
\parbox[b]{8cm}{
\BS{draw} [\RDD{postaction}=\{decorate,blue,draw,ultra thick, \\
decoration=\AC{straight zigzag,amplitude=0.5cm}\}] \\
 
$[$red,line width = 10pt $]$ (0,0)  arc (0:180:3 and 2);
 
  }
\\ \hline 
\end{tabular} 



\newpage


\newpage

\SSCT{Insertion images dans un environnement TikZ}{Pictures in a TikZ picture}



%\subsubsection{Dans un noeud}
\SbSbSSCT{Dans un noeud}{In a node}

 \begin{tabular}{|c | c | } \hline
 \begin{tikzpicture}[baseline=0pt]
 \draw (0,0) grid (4,3);
 \node [fill=green!20,trapezium,draw] at (1,2) {\DFR};
 \node  [draw] at (3,1) {\includegraphics[width=1cm]{tiger} };
 \end{tikzpicture} 
 &
  \parbox[b]{10cm}{
  \BS{begin}\AC{tikzpicture}   \\
   \BS{draw} (0,0) grid (5,3); \\
  \BS{node} [fill=green!20,trapezium,draw] at (1,2) \AC{\BS{DFR} }; \pageref{DFR} \\
\BS{node}  [draw] at (3,1) \AC{\BS{includegraphics}[width=1cm]\AC{tiger} };\\
  \BS{end}\AC{tikzpicture}
 }
   \\  \hline
  \end{tabular}
  
 
 
%\subsubsection{En déclarant l'image dans pgf}
\SbSbSSCT{En déclarant l'image dans pgf}{With pgfdeclareimage}


 \begin{tabular}{|c | c | } \hline
 \pgfdeclareimage[width=3cm]{ttt}{tiger}
% \pgfuseimage{tiger}
 
 \begin{tikzpicture}[baseline=1cm]
 \draw (0,0) grid (5,5);
\draw (3,2) node {\pgfuseimage{ttt}} ;
 \end{tikzpicture} 
 &
 \parbox[b]{8cm}{
  \BSS{pgfdeclareimage}[width=3cm]\AC{ttt}\AC{tiger}\\
\\
\\  
  \BS{begin}\AC{tikzpicture}\\
  \BS{draw} (0,0) grid (5,5); \\
  \BS{draw} (3,2) node \AC{\BSS{pgfuseimage}\AC{ttt}} ; \\
  \BS{end}\AC{tikzpicture}
 }
  \\  \hline
 \end{tabular}
 



\SSCT{Trait à main levée }{Freehand drawing}

\TFRGB{voir page}{see page} \pageref{alea}


\begin{tabular}{|c|c|} \hline  

\begin{tikzpicture}[baseline=0pt]
\draw [decorate,blue,ultra thick,red,decoration={random steps,amplitude=1pt,segment length=3pt}]
(0,0)  arc (0:320:2.5 and 1.5);
\end{tikzpicture}
&  
\parbox{8cm}{ 
\BS{draw}[decorate,decoration=\AC {random steps, amplitude=1pt,segment length=3pt}]  (0,0)  arc (0:320:2.5 and 1.5);}
\\ \hline
\begin{tikzpicture}[baseline=0pt]
\draw[->,blue,ultra thick] (-.1,0) -- (5.5,0);
\draw[->,blue,ultra thick] (0,-0.5) -- (0,2.5);
 \draw[decorate,blue,ultra thick,red,decoration={random steps,amplitude=1pt,segment length=3pt}] plot coordinates {(0,0) (1,1) (2,0) (3,1) (4,1) (5,2)};
\end{tikzpicture}
&  
\parbox{8cm}{ 
\BS{draw}[decorate,decoration=\AC {random steps, amplitude=1pt,segment length=3pt}] plot coordinates {(0,0) (1,1) (2,0) (3,1) (4,1) (5,2)};}
\\ \hline  
\begin{tikzpicture}[domain=0:6.28,ultra thick,x=0.7cm,baseline=0pt]
%\draw[very thin,color=gray] (-0.1,-2.1) grid (4.1,2.1);
\draw[->,blue,ultra thick] (-.1,0) -- (7,0);
\draw[->,blue,ultra thick] (0,-1.5) -- (0,1.5);
\draw[decorate,red,decoration={random steps,amplitude=1pt,segment length=3pt}] plot  (\x,{sin(\x r)});
\end{tikzpicture}
&
\parbox{8cm}{ 
\BS{draw}[decorate, decoration=\AC{random steps, amplitude=1pt,segment length=3pt}] plot  (\BS{x},{sin(\BS{x} r)});} 
\\ \hline 
\end{tabular} 





\SSCT{Effets spéciaux}{Special effect}


\SbSSCT{Le peuple TikZ}{Tikzpeople}

\label{people}

 \maboite{\BS{usepackage}\AC{tikzpeople} \cite {tikzpeople} \footnote{ conflit \BS{usetikzlibrary}\AC{patterns} page \pageref{lib-patterns} : placer cette commande en premier} }

\bigskip
\begin{tabular}{|c|c|}\hline  
\BS{tikz} \BS{node}[\RDD{alice}] at (0,0) {};  &  \tikz \node[alice] at (0,0) {};\\ 
\hline 
\end{tabular} 
 
\SbSbSSCT{Personages disponibles}{available characters}

\noindent



\begin{tabular}{|c|c|c|c|c|c|c|}\hline 
\multicolumn{7}{|c|}{ \BS{tikz} \BS{node}[\RDD{alice},minimum size=1.5cm] at (0,0) {};  }
\\ \hline  
\tikz \node[alice,minimum size=1.5cm] at (0,0) {}; &  
\tikz \node[bob,minimum size=1.5cm] at (0,0) {}; &  
\tikz \node[bride,minimum size=1.5cm] at (0,0) {}; &  
\tikz \node[builder,minimum size=1.5cm] at (0,0) {}; &  
\tikz \node[businessman,minimum size=1.5cm] at (0,0) {}; &  
\tikz \node[charlie,minimum size=1.5cm] at (0,0) {}; &  
\tikz \node[chef,minimum size=1.5cm] at (0,0) {}; 
\\  \hline  
\RDD{alice} & \RDD{bob} & \RDD{bride} & \RDD{builder} & \RDD{businessman} & \RDD{charlie}  & \RDD{chef} 
\\  \hline  
\tikz \node[conductor,minimum size=1.5cm] at (0,0) {}; &  
\tikz \node[cowboy,minimum size=1.5cm] at (0,0) {}; &  
\tikz \node[criminal,minimum size=1.5cm] at (0,0) {}; &  
\tikz \node[dave,minimum size=1.5cm] at (0,0) {}; &  
\tikz \node[graduate,minimum size=1.5cm] at (0,0) {}; &  
\tikz \node[groom,minimum size=1.5cm] at (0,0) {}; & 
\tikz \node[guard,minimum size=1.5cm] at (0,0) {}; 
\\ \hline  
\RDD{conductor} & \RDD{cowboy} & \RDD{criminal} & \RDD{dave} & \RDD{graduate} & \RDD{groom} & \RDD{guard} 
\\ \hline  
\tikz \node[jester,minimum size=1.5cm] at (0,0) {}; &  
%\tikz \node[judge,minimum size=1.5cm] at (0,0) {}; 
&  
\tikz \node[mexican,minimum size=1.5cm] at (0,0) {}; &  
\tikz \node[nun,minimum size=1.5cm] at (0,0) {}; &  
\tikz \node[nurse,minimum size=1.5cm] at (0,0) {}; &  
\tikz \node[physician,minimum size=1.5cm] at (0,0) {}; &  
\tikz \node[pilot,minimum size=1.5cm] at (0,0) {};\\ \hline  
\RDD{jester} &  \RDD{judge} &  \RDD{mexican}  & \RDD{nun} &  \RDD{nurse} & \RDD{physician} 
&  \RDD{pilot}
\\ \hline  

\tikz \node[police,minimum size=1.5cm] at (0,0) {}; &  
\tikz \node[priest,minimum size=1.5cm] at (0,0) {}; &  
\tikz \node[sailor,minimum size=1.5cm] at (0,0) {}; &  
\tikz \node[santa,minimum size=1.5cm] at (0,0) {}; &  
\tikz \node[surgeon,minimum size=1.5cm] at (0,0) {};&  &  \\ 
\hline \RDD{police} & \RDD{priest}  & \RDD{sailor} & \RDD{santa} & \RDD{surgeon} &  &  \\ 
\hline 
\end{tabular} 

\subsubsection{Options}

\noindent

\begin{tabular}{|c|c|c|c|c|}\hline
\multicolumn{5}{|c|}{ \BS{tikz} \BS{node}[businessman,\RDD{evil},minimum size=1.5cm] at (0,0) {};  }
\\ \hline  
\tikz \node[businessman,evil,minimum size=1.5cm] at (0,0) {}; &  
\tikz \node[businessman,female,minimum size=1.5cm] at (0,0) {}; &  
\tikz \node[businessman,good,minimum size=1.5cm] at (0,0) {}; &  
\tikz \node[businessman,mirrored,minimum size=1.5cm] at (0,0) {}; &  
\tikz \node[businessman,monitor,minimum size=1.5cm] at (0,0) {};  
\\  \hline
\RDD{evil} & \RDD{female} & \RDD{good} & \RDD{mirrored} & \RDD{monitor}

\\  \hline 
\end{tabular}

\SbSbSSCT{Point d'ancrage spécifique}{Anchor specific}

\noindent

\begin{tabular}{|c|c|} \hline  
\begin{tikzpicture}[baseline=0pt,blue]
 \node[name=a,shape=bob,minimum
size=1.5cm] {};
 \node at (1.25,.5) [ellipse callout, draw,
callout absolute pointer={(a.mouth)},
font=\tiny] {Hey!};
\end{tikzpicture}
&  
\parbox{12cm}{
\BS{begin}\AC{tikzpicture}[blue] \\
\BS{node}[name=a,shape=bob,minimum size=1.5cm] \AC{};\\
\BS{node} at (1.25,.5) [ellipse callout, draw,
callout absolute pointer\AC{(a.\RDD{mouth})},
font=\BS{tiny}] {Hey!};\\
\BS{end}\AC{tikzpicture} \\
}
\\ \hline 
\end{tabular} 


\SbSbSSCT{Couleurs }{Colors}

\noindent


\begin{tabular}{|c|c|c|c|}\hline
\multicolumn{4}{|c|}{ \BS{tikz} \BS{node}[\blll{alice},\RDD{hair}=red,minimum size=1.5cm] at (0,0) {};  }
\\ \hline    
\tikz \node[alice,hair=red,minimum size=1.5cm] at (0,0) {}; &  
\tikz \node[alice,skin=red,minimum size=1.5cm] at (0,0) {}; &  
\tikz \node[alice,shirt=red,minimum size=1.5cm] at (0,0) {}; &  
\tikz \node[alice,undershirt=red,minimum size=1.5cm] at (0,0) {};  
\\  \hline
\RDD{hair}=red & \RDD{skin}=red & \RDD{shirt}=red & \RDD{details}=red 

\\  \hline 
\end{tabular}

\bigskip
\begin{tabular}{|c|c|c|c|}\hline 
\multicolumn{4}{|c|}{ \BS{tikz} \BS{node}[\blll{bob},\RDD{hair}=red,minimum size=1.5cm] at (0,0) {};  }
\\ \hline   
\tikz \node[bob,hair=red,minimum size=1.5cm] at (0,0) {}; &  
\tikz \node[bob,skin=red,minimum size=1.5cm] at (0,0) {}; &  
\tikz \node[bob,shirt=red,minimum size=1.5cm] at (0,0) {}; &  
\tikz \node[bob,details=red,minimum size=1.5cm] at (0,0) {};  
\\  \hline
\RDD{hair}=red & \RDD{skin}=red & \RDD{shirt}=red & \RDD{details}=red 
\\  \hline 
\end{tabular}

\bigskip
\begin{tabular}{|c|c|c|c|c|}\hline 
\multicolumn{5}{|c|}{ \BS{tikz} \BS{node}[\blll{bride},\RDD{hair}=red,minimum size=1.5cm] at (0,0) {};  }
\\ \hline    
\tikz \node[bride,hair=red,minimum size=1.5cm] at (0,0) {}; &  
\tikz \node[bride,skin=red,minimum size=1.5cm] at (0,0) {}; &  
\tikz \node[bride,shirt=red,minimum size=1.5cm] at (0,0) {}; &  
\tikz \node[bride,pearls=red,minimum size=1.5cm] at (0,0) {}; &
\tikz \node[bride,veil=red,minimum size=1.5cm] at (0,0) {};  
\\  \hline
\RDD{hair}=red & \RDD{skin}=red & \RDD{shirt}=red & \RDD{pearls}=red & \RDD{veil}=red 

\\  \hline 
\end{tabular}

\bigskip
\begin{tabular}{|c|c|c|c|c|}\hline
\multicolumn{5}{|c|}{ \BS{tikz} \BS{node}[\blll{builder},\RDD{hair}=red,minimum size=1.5cm] at (0,0) {};  }
\\ \hline    
\tikz \node[builder,hair=red,minimum size=1.5cm] at (0,0) {}; &  
\tikz \node[builder,skin=red,minimum size=1.5cm] at (0,0) {}; &  
\tikz \node[builder,shirt=red,minimum size=1.5cm] at (0,0) {}; &  
\tikz \node[builder,trousers=red,minimum size=1.5cm] at (0,0) {}; &
\tikz \node[builder,hat=red,minimum size=1.5cm] at (0,0) {};  
\\  \hline
\RDD{hair}=red & \RDD{skin}=red & \RDD{shirt}=red & \RDD{trousers}=red & \RDD{hat}=red   
\\  \hline 
\end{tabular}

\bigskip
\begin{tabular}{|c|c|c|c|c|c|}\hline
\multicolumn{6}{|c|}{ \BS{tikz} \BS{node}[\blll{businessman},\RDD{hair}=red,minimum size=1.5cm] at (0,0) {};  }
\\ \hline    
\tikz \node[businessman,hair=red,minimum size=1.5cm] at (0,0) {}; &  
\tikz \node[businessman,skin=red,minimum size=1.5cm] at (0,0) {}; &  
\tikz \node[businessman,shirt=red,minimum size=1.5cm] at (0,0) {}; &  
\tikz \node[businessman,tie=red,minimum size=1.5cm] at (0,0) {}; &
\tikz \node[businessman,undershirt=red,minimum size=1.5cm] at (0,0) {};  &
\tikz \node[businessman,monogram=red,minimum size=1.5cm] at (0,0) {};
\\  \hline
\RDD{hair}=red & \RDD{skin}=red & \RDD{shirt}=red & \RDD{tie}=red & \RDD{undershirt}=red & \RDD{monogram}=red 
\\  \hline 
\end{tabular}


\bigskip
\begin{tabular}{|c|c|c|c|}\hline
\multicolumn{4}{|c|}{ \BS{tikz} \BS{node}[\blll{charlie},\RDD{hair}=red,minimum size=1.5cm] at (0,0) {};  }
\\ \hline    
\tikz \node[charlie,hair=red,minimum size=1.5cm] at (0,0) {}; &  
\tikz \node[charlie,skin=red,minimum size=1.5cm] at (0,0) {}; &  
\tikz \node[charlie,shirt=red,minimum size=1.5cm] at (0,0) {}; &  
\tikz \node[charlie,buttons=red,minimum size=1.5cm] at (0,0) {}; 

\\  \hline
\RDD{hair}=red & \RDD{skin}=red & \RDD{shirt}=red & \RDD{buttons}=red 
\\  \hline 
\end{tabular}


\bigskip
\begin{tabular}{|c|c|c|c|c|}\hline
\multicolumn{5}{|c|}{ \BS{tikz} \BS{node}[\blll{chef},\RDD{hair}=red,minimum size=1.5cm] at (0,0) {};  }
\\ \hline 
\tikz \node[chef,hair=red,minimum size=1.5cm] at (0,0) {}; &  
\tikz \node[chef,skin=red,minimum size=1.5cm] at (0,0) {}; &  
\tikz \node[chef,shirt=red,minimum size=1.5cm] at (0,0) {}; &  
\tikz \node[chef,hat=red,minimum size=1.5cm] at (0,0) {}; &
\tikz \node[chef,details=red,minimum size=1.5cm] at (0,0) {};  
\\  \hline
\RDD{hair}=red & \RDD{skin}=red & \RDD{shirt}=red & \RDD{hat}=red & \RDD{details}=red 
\\  \hline 
\end{tabular}


\bigskip
\begin{tabular}{|c|c|c|c|c|}\hline
\multicolumn{5}{|c|}{ \BS{tikz} \BS{node}[\blll{conductor},\RDD{hair}=red,minimum size=1.5cm] at (0,0) {};  }
\\ \hline 
\tikz \node[conductor,hair=red,minimum size=1.5cm] at (0,0) {}; &  
\tikz \node[conductor,skin=red,minimum size=1.5cm] at (0,0) {}; &  
\tikz \node[conductor,shirt=red,minimum size=1.5cm] at (0,0) {}; &  \tikz \node[conductor,hat=red,minimum size=1.5cm] at (0,0) {}; &
\tikz \node[conductor,hatshield=red,minimum size=1.5cm] at (0,0) {};  
\\  \hline
\RDD{hair}=red & \RDD{skin}=red & \RDD{shirt}=red & \RDD{hat}=red & \RDD{hatshield}=red  
\\  \hline 
\tikz \node[conductor,undershirt=red,minimum size=1.5cm] at (0,0) {}; &  
\tikz \node[conductor,tie=red,minimum size=1.5cm] at (0,0) {}; &  
\tikz \node[conductor,hatbadge=red,minimum size=1.5cm] at (0,0) {}; &
\tikz \node[conductor,badge=red,minimum size=1.5cm] at (0,0) {}; &
\\  \hline 
\RDD{undershirt}=red &  \RDD{shirt}=red & \RDD{hatbadge}=red & \RDD{badge}=red &
\\  \hline 
\end{tabular}

\bigskip
\begin{tabular}{|c|c|c|c|}\hline
\multicolumn{4}{|c|}{ \BS{tikz} \BS{node}[\blll{cowboy},\RDD{hair}=red,minimum size=1.5cm] at (0,0) {};  }
\\ \hline 
\tikz \node[cowboy,hair=red,minimum size=1.5cm] at (0,0) {}; &  
\tikz \node[cowboy,skin=red,minimum size=1.5cm] at (0,0) {}; &  
\tikz \node[cowboy,shirt=green,minimum size=1.5cm] at (0,0) {}; &  
\tikz \node[cowboy,hat=red,minimum size=1.5cm] at (0,0) {};   
\\  \hline
\RDD{hair}=red & \RDD{skin}=red & \RDD{shirt}=green & \RDD{hat}=red 
\\  \hline 
\tikz \node[cowboy,patches=red,minimum size=1.5cm] at (0,0) {}; &  
\tikz \node[cowboy,tie=green ,minimum size=1.5cm] at (0,0) {}; &  
\tikz \node[cowboy,stitching=red,minimum size=1.5cm] at (0,0) {}; &
\tikz \node[cowboy,vest=red,minimum size=1.5cm] at (0,0) {}; 
\\  \hline 
\RDD{patches}=red &  \RDD{tie}=green & \RDD{stitching}=red & \RDD{vest}=red
\\  \hline 
\end{tabular}


\bigskip
\begin{tabular}{|c|c|c|c|}\hline 
\multicolumn{4}{|c|}{ \BS{tikz} \BS{node}[\blll{criminal},\RDD{hat}=red,minimum size=1.5cm] at (0,0) {};  }
\\ \hline 
\tikz \node[criminal,hat=red,minimum size=1.5cm] at (0,0) {}; &  
\tikz \node[criminal,skin=red,minimum size=1.5cm] at (0,0) {}; &  
\tikz \node[criminal,shirt=red,minimum size=1.5cm] at (0,0) {}; &  
\tikz \node[criminal,details=red,minimum size=1.5cm] at (0,0) {}; 
\\  \hline
\RDD{hat}=red & \RDD{skin}=red & \RDD{shirt}=red & \RDD{details}=red 
\\  \hline 
\end{tabular}

\bigskip
\begin{tabular}{|c|c|c|c|c|}\hline
\multicolumn{5}{|c|}{ \BS{tikz} \BS{node}[\blll{dave},\RDD{hair}=red,minimum size=1.5cm] at (0,0) {};  }
\\ \hline 
\tikz \node[dave,hair=red,minimum size=1.5cm] at (0,0) {}; &  
\tikz \node[dave,skin=red,minimum size=1.5cm] at (0,0) {}; &  
\tikz \node[dave,shirt=red,minimum size=1.5cm] at (0,0) {}; &  
\tikz \node[dave,undershirt=green,minimum size=1.5cm] at (0,0) {}; &
\tikz \node[dave,tie=green,minimum size=1.5cm] at (0,0) {};
\\  \hline
\RDD{hair}=red & \RDD{skin}=red & \RDD{shirt}=red & \RDD{undershirt}=green & \RDD{tie}=green
\\  \hline 
\end{tabular}

\bigskip
\begin{tabular}{|c|c|c|c|c|c|}\hline
\multicolumn{6}{|c|}{ \BS{tikz} \BS{node}[\blll{graduate},\RDD{hair}=red,minimum size=1.5cm] at (0,0) {};  }
\\ \hline 
\tikz \node[graduate,hair=red,minimum size=1.5cm] at (0,0) {}; &  
\tikz \node[graduate,skin=red,minimum size=1.5cm] at (0,0) {}; &  
\tikz \node[graduate,shirt=red,minimum size=1.5cm] at (0,0) {}; &  
\tikz \node[graduate,undershirt=red,minimum size=1.5cm] at (0,0) {}; &
\tikz \node[graduate,stripes=red,minimum size=1.5cm] at (0,0) {};
&
\tikz \node[graduate,hat=red,minimum size=1.5cm] at (0,0) {};
\\  \hline
\RDD{hair}=red & \RDD{skin}=red & \RDD{shirt}=red & \RDD{undershirt}=red & \RDD{stripes}=red & \RDD{hat}=red
\\  \hline 
\end{tabular}

\bigskip
\begin{tabular}{|c|c|c|c|c|c|}\hline
\multicolumn{6}{|c|}{ \BS{tikz} \BS{node}[\blll{groom},\RDD{hair}=red,minimum size=1.5cm] at (0,0) {};  }
\\ \hline
\tikz \node[groom,hair=red,minimum size=1.5cm] at (0,0) {}; &  
\tikz \node[groom,skin=red,minimum size=1.5cm] at (0,0) {}; &  
\tikz \node[groom,shirt=red,minimum size=1.5cm] at (0,0) {}; &  
\tikz \node[groom,undershirt=green,minimum size=1.5cm] at (0,0) {}; &
\tikz \node[groom,tie=green,minimum size=1.5cm] at (0,0) {}; &
\tikz \node[groom,hat=red,minimum size=1.5cm] at (0,0) {};
\\  \hline
\RDD{hair}=red & \RDD{skin}=red & \RDD{shirt}=red & \RDD{undershirt}=green & \RDD{tie}=green & \RDD{hat}=red
\\  \hline 
\end{tabular}


\bigskip
\begin{tabular}{|c|c|c|c|c|c|}\hline 
\multicolumn{6}{|c|}{ \BS{tikz} \BS{node}[\blll{guard},\RDD{hat}=red,minimum size=1.5cm] at (0,0) {};  }
\\ \hline
\tikz\node[guard,hat=red,minimum size=1.5cm] at (0,0) {}; &  
\tikz \node[guard,skin=red,minimum size=1.5cm] at (0,0) {}; &  
\tikz \node[guard,shirt=red,minimum size=1.5cm] at (0,0) {}; &  
\tikz \node[guard,collar=red,minimum size=1.5cm] at (0,0) {}; &
\tikz \node[guard,lining=red,minimum size=1.5cm] at (0,0) {}; &
\tikz \node[guard,details=red,minimum size=1.5cm] at (0,0) {};
\\  \hline
\RDD{hat}=red & \RDD{skin}=red & \RDD{shirt}=red & \RDD{collar}=red & \RDD{lining}=red & \RDD{details}=red
\\  \hline 
\end{tabular}


\bigskip
\begin{tabular}{|c|c|c|c|c|c|}\hline
\multicolumn{6}{|c|}{ \BS{tikz} \BS{node}[\blll{jester},\RDD{hat}=red,minimum size=1.5cm] at (0,0) {};  }
\\ \hline
\tikz \node[jester,hair=red,minimum size=1.5cm] at (0,0) {}; &  
\tikz \node[jester,skin=red,minimum size=1.5cm] at (0,0) {}; &  
\tikz \node[jester,shirt=yellow,minimum size=1.5cm] at (0,0) {}; &  
\tikz \node[jester,hat=red,minimum size=1.5cm] at (0,0) {}; &
%\tikz \node[jester,pattern=yellow,minimum size=1.5cm] at (0,0) {};
&
\tikz \node[jester,details=blue,minimum size=1.5cm] at (0,0) {};
\\  \hline
\RDD{hair}=red & \RDD{skin}=red & \RDD{shirt}=yellow & \RDD{hat}=red & \RDD{pattern}=yellow \footnote{voir confit} & \RDD{details}=blue
\\  \hline 
\end{tabular}

\bigskip
\begin{tabular}{|c|c|c|c|c|}\hline
\multicolumn{5}{|c|}{ \BS{tikz} \BS{node}[\blll{judge},\RDD{hair}=red,minimum size=1.5cm] at (0,0) {};  }
\\ \hline
%\tikz \node[judge,hair=red,minimum size=1.5cm] at (0,0) {}; 
&  
%\tikz \node[judge,skin=red,minimum size=1.5cm] at (0,0) {}; &  
%\tikz \node[judge,shirt=red,minimum size=1.5cm] at (0,0) {}; &  
%\tikz \node[judge,undershirt=red,minimum size=1.5cm] at (0,0) {}; &
%\tikz \node[judge,hairshadow=red,minimum size=1.5cm] at (0,0) {};

\\  \hline
\RDD{hair}=red & \RDD{skin}=red & \RDD{shirt}=red & \RDD{undershirt}=red & \RDD{hairshadow}=red 
\\  \hline 
\end{tabular}

\bigskip
\begin{tabular}{|c|c|c|c|c|c|c|}\hline
\multicolumn{7}{|c|}{ \BS{tikz} \BS{node}[\blll{mexican},\RDD{hair}=red,minimum size=1.5cm] at (0,0) {};  }
\\ \hline
\tikz \node[mexican,hair=red,minimum size=1.5cm] at (0,0) {}; &  
\tikz \node[mexican,skin=red,minimum size=1.5cm] at (0,0) {}; &  
\tikz \node[mexican,shirt=red,minimum size=1.5cm] at (0,0) {}; &  
\tikz \node[mexican,hat=green,minimum size=1.5cm] at (0,0) {}; &
\tikz \node[mexican,ringtop=red,minimum size=1.5cm] at (0,0) {};
&
\tikz \node[mexican,ringmid=red,minimum size=1.5cm] at (0,0) {};
&
\tikz \node[mexican,ringbot=yellow,minimum size=1.5cm] at (0,0) {};
\\  \hline
\RDD{hair}=red & \RDD{skin}=red & \RDD{shirt}=red & \RDD{hat}=green & \RDD{ringtop}=red &\RDD{ringmid}=red & \RDD{ringbot}=yellow
\\  \hline 
\end{tabular}


\bigskip

\begin{tabular}{|c|c|c|}\hline 
\multicolumn{3}{|c|}{ \BS{tikz} \BS{node}[\blll{nun},\RDD{plaid}=red,minimum size=1.5cm] at (0,0) {};  }
\\ \hline
\tikz \node[nun,plaid=red,minimum size=1.5cm] at (0,0) {}; &  
\tikz \node[nun,skin=red,minimum size=1.5cm] at (0,0) {}; &  
\tikz \node[nun,shirt=red,minimum size=1.5cm] at (0,0) {}; 
\\  \hline
\RDD{plaid}=red & \RDD{skin}=red & \RDD{shirt}=red 
\\  \hline 
\end{tabular}


\bigskip
\begin{tabular}{|c|c|c|c|c|c|c|}\hline 
\multicolumn{7}{|c|}{ \BS{tikz} \BS{node}[\blll{nurse},\RDD{hair}=red,minimum size=1.5cm] at (0,0) {};  }
\\ \hline
\tikz \node[nurse,hair=red,minimum size=1.5cm] at (0,0) {}; &  
\tikz \node[nurse,skin=red,minimum size=1.5cm] at (0,0) {}; &  
\tikz \node[nurse,shirt=red,minimum size=1.5cm] at (0,0) {}; &  
\tikz \node[nurse,badgeclip=green,minimum size=1.5cm] at (0,0) {}; &
\tikz \node[nurse,redcross=green,minimum size=1.5cm] at (0,0) {};
&
\tikz \node[nurse,badge=red,minimum size=1.5cm] at (0,0) {};
&
\tikz \node[nurse,badgename=red,minimum size=1.5cm] at (0,0) {};
\\  \hline
\RDD{hair}=red & \RDD{skin}=red & \RDD{shirt}=red & \RDD{badgeclip}=green & \RDD{redcross}=green & \RDD{badge}=red &  \RDD{badgename}=red
\\  \hline 
\end{tabular}


\bigskip
\begin{tabular}{|c|c|c|c|c|c|}\hline
\multicolumn{6}{|c|}{ \BS{tikz} \BS{node}[\blll{physician},\RDD{hair}=red,minimum size=1.5cm] at (0,0) {};  }
\\ \hline
\tikz \node[physician,hair=red,minimum size=1.5cm] at (0,0) {}; &  
\tikz \node[physician,skin=red,minimum size=1.5cm] at (0,0) {}; &  
\tikz \node[physician,shirt=red,minimum size=1.5cm] at (0,0) {}; &  
\tikz \node[physician,hat=red,minimum size=1.5cm] at (0,0) {}; &
\tikz \node[physician,stethoscope=red,minimum size=1.5cm] at (0,0) {};
&
\tikz \node[physician,tube=red,minimum size=1.5cm] at (0,0) {};
\\  \hline
\RDD{hair}=red & \RDD{skin}=red & \RDD{shirt}=red & \RDD{hat}=red & \RDD{stethoscope}=red &   \RDD{tube}=red
\\  \hline 
\end{tabular}


\bigskip
\begin{tabular}{|c|c|c|c|c|c|c|}\hline
\multicolumn{7}{|c|}{ \BS{tikz} \BS{node}[\blll{pilot},\RDD{hat}=red,minimum size=1.5cm] at (0,0) {};  }
\\ \hline
\tikz \node[pilot,hat=red,minimum size=1.5cm] at (0,0) {}; &  
\tikz \node[pilot,skin=red,minimum size=1.5cm] at (0,0) {}; &  
\tikz \node[pilot,shirt=red,minimum size=1.5cm] at (0,0) {}; &  
\tikz \node[pilot,undershirt=red,minimum size=1.5cm] at (0,0) {}; &
\tikz \node[pilot,visor=red,minimum size=1.5cm] at (0,0) {}; &
\tikz \node[pilot,straps=red,minimum size=1.5cm] at (0,0) {}; &
%\tikz \node[pilot,decoration=red,minimum size=1.5cm] at (0,0) {};
\\  \hline
\RDD{hat}=red & \RDD{skin}=red & \RDD{shirt}=red & \RDD{undershirt}=red & \RDD{visor}=red & \RDD{straps}=red & \RDD{decoration}=red
\\  \hline 
\end{tabular}


\bigskip
\begin{tabular}{|c|c|c|c|}\hline
\multicolumn{4}{|c|}{ \BS{tikz} \BS{node}[\blll{police},\RDD{hair}=red,minimum size=1.5cm] at (0,0) {};  }
\\ \hline 
\tikz \node[police,hair=red,minimum size=1.5cm] at (0,0) {}; &  
\tikz \node[police,skin=red,minimum size=1.5cm] at (0,0) {}; &  
\tikz \node[police,shirt=red,minimum size=1.5cm] at (0,0) {}; &  
\tikz \node[police,hat=red,minimum size=1.5cm] at (0,0) {};   
\\  \hline
\RDD{hair}=red & \RDD{skin}=red & \RDD{shirt}=red & \RDD{hat}=red
\\  \hline
\tikz \node[police,badge=red,minimum size=1.5cm] at (0,0) {}; &  
\tikz \node[police,hatbadge=red ,minimum size=1.5cm] at (0,0) {}; &  
\tikz \node[police,hatshield=red,minimum size=1.5cm] at (0,0) {}; &
\tikz \node[police,undershirt=red,minimum size=1.5cm] at (0,0) {}; 
\\  \hline 
\RDD{badge}=red &  \RDD{hatbadge}=red & \RDD{hatshield}=red & \RDD{undershirt}=red
\\  \hline 
\end{tabular}



\bigskip
\begin{tabular}{|c|c|c|c|c|c|}\hline
\multicolumn{6}{|c|}{ \BS{tikz} \BS{node}[\blll{priest},\RDD{hair}=red,minimum size=1.5cm] at (0,0) {};  }
\\ \hline
\tikz \node[priest,hair=red,minimum size=1.5cm] at (0,0) {}; &  
\tikz \node[priest,skin=red,minimum size=1.5cm] at (0,0) {}; &  
\tikz \node[priest,shirt=red,minimum size=1.5cm] at (0,0) {}; &  
\tikz \node[priest,hat=red,minimum size=1.5cm] at (0,0) {}; &
\tikz \node[priest,collar=red,minimum size=1.5cm] at (0,0) {}; &
\tikz \node[priest,cross=red,minimum size=1.5cm] at (0,0) {};
\\  \hline
\RDD{hair}=red & \RDD{skin}=red & \RDD{shirt}=red & \RDD{hat}=red & \RDD{collar}=red &   \RDD{cross}=red
\\  \hline 
\end{tabular}


\bigskip
\begin{tabular}{|c|c|c|c|c|c|c|}\hline 
\multicolumn{7}{|c|}{ \BS{tikz} \BS{node}[\blll{sailor},\RDD{hair}=red,minimum size=1.5cm] at (0,0) {};  }
\\ \hline
\tikz \node[sailor,hair=red,minimum size=1.5cm] at (0,0) {}; &  
\tikz \node[sailor,skin=red,minimum size=1.5cm] at (0,0) {}; &  
\tikz \node[sailor,shirt=red,minimum size=1.5cm] at (0,0) {}; &  
\tikz \node[sailor,hat=red,minimum size=1.5cm] at (0,0) {}; &
\tikz \node[sailor,undershirt=red,minimum size=1.5cm] at (0,0) {};
&
\tikz \node[sailor,stripes=red,minimum size=1.5cm] at (0,0) {};
&
\tikz \node[sailor,details=red,minimum size=1.5cm] at (0,0) {};
\\  \hline
\RDD{hair}=red & \RDD{skin}=red & \RDD{shirt}=red & \RDD{hat}=red & \RDD{undershirt}=red & \RDD{stripes}=red &  \RDD{details}=red
\\  \hline 
\end{tabular}


\bigskip
\begin{tabular}{|c|c|c|c|c|}\hline 
\multicolumn{5}{|c|}{ \BS{tikz} \BS{node}[\blll{santa},\RDD{hat}=green,minimum size=1.5cm] at (0,0) {};  }
\\ \hline
\tikz \node[santa,hat=green,minimum size=1.5cm] at (0,0) {}; &  
\tikz \node[santa,skin=green,minimum size=1.5cm] at (0,0) {}; &  
\tikz \node[santa,shirt=green,minimum size=1.5cm] at (0,0) {}; &  
\tikz \node[santa,beard=green,minimum size=1.5cm] at (0,0) {}; &
\tikz \node[santa,details=green,minimum size=1.5cm] at (0,0) {};

\\  \hline
\RDD{hat}=green & \RDD{skin}=green& \RDD{shirt}=green & \RDD{beard}=green & \RDD{details}=green 
\\  \hline 
\end{tabular}


\bigskip
\begin{tabular}{|c|c|c|c|c|}\hline 
\multicolumn{5}{|c|}{ \BS{tikz} \BS{node}[\blll{surgeon},\RDD{hat}=red,minimum size=1.5cm] at (0,0) {};  }
\\ \hline
\tikz \node[surgeon,hat=red,minimum size=1.5cm] at (0,0) {}; &  
\tikz \node[surgeon,skin=red,minimum size=1.5cm] at (0,0) {}; &  
\tikz \node[surgeon,shirt=red,minimum size=1.5cm] at (0,0) {}; &  
\tikz \node[surgeon,hair=red,minimum size=1.5cm] at (0,0) {}; &
\tikz \node[surgeon,mask=red,minimum size=1.5cm] at (0,0) {};

\\  \hline
\RDD{hat}=red & \RDD{skin}=red & \RDD{shirt}=red & \RDD{hair}=red & \RDD{mask}=red 
\\  \hline 
\end{tabular}



\newpage

\subsection{Ducks}

\label{ducks}

 \maboite{\BS{usepackage}\AC{tikzducks} \cite {tikzducks}}


\begin{center}
\begin{tabular}{|c|}\hline  
\BS{tikz} \BSS{duck} ;
\\ \hline  
\tikz \duck ;
\\ \hline 
\end{tabular} 
\end{center}


\subsubsection{Options}

\noindent

\begin{tabular}{|c|c|c|c|c|} \hline 
\multicolumn{4}{|c|}{\BS{tikz} \BS{duck}[\RDD{body}=red] ;} 
\\ \hline
\tikz \duck[body=red] ;
&  
\tikz \duck[head=red] ;
&
\tikz \duck[bill=red] ;
  &
  \tikz \duck[eye=red] ;
    \\ 
\hline  
[\RDD{body}=red] & [\RDD{head}=red] & [\RDD{bill}=red] & [\RDD{eye}=red] \\ 
\hline 
\end{tabular} 
\begin{tabular}{|c|} \hline
\BS{tikz}  \BS{duck}[\RDD{grumpy}] ;
\\ \hline   
\tikz  \duck[grumpy] ;
\\ \hline  

\end{tabular} 



\bigskip

\begin{tabular}{|c|c|c|c|c|c|} \hline  
\tikz \duck[longhair] ;
&  
\tikz \duck[shorthair] ;
&
\tikz \duck[crazyhair] ;
&
\tikz \duck[recedinghair] ;
&
\tikz \duck[mohican] ;
&
\tikz \duck[mullet] ;
\\ \hline  
[\RDD{longhair}] & [\RDD{shorthair}] & [\RDD{crazyhair}] & [\RDD{recedinghair}] &  [\RDD{mohican}] &  [\RDD{mullet}]\\ 
\hline
\tikz \duck[longhair=red] ;
&  
\tikz \duck[shorthair=red] ;
&
\tikz \duck[crazyhair=red] ;
  &
  \tikz \duck[recedinghair=red] ;
  &
  \tikz \duck[mohican=red] ;
  &
  \tikz \duck[mullet=red] ;
    \\ 
\hline  
[longhair=red] & [shorthair=red] & [crazyhair=red] & [recedinghair=red] &  [mohican=red] &  [mullet=red]  \\ 
\hline 
\end{tabular}

\bigskip





\begin{tabular}{|c|c|c|c|c|} \hline  
\tikz \duck[eyebrow] ;
&  
\tikz \duck[eyebrow=red] ;
&
\tikz \duck[beard] ;
  &
  \tikz \duck[beard=red] ;
    \\ 
\hline  
[\RDD{eyebrow}] & [eyebrow=red] & [\RDD{beard}] & [beard=red] \\ 
\hline
 
\end{tabular}

\bigskip

\begin{tabular}{|c|c|c|c|c|} \hline  
\tikz \duck[tshirt] ;
&  
\tikz \duck[tie] ;
&
\tikz \duck[jacket] ;
&
\tikz \duck[cape] ;
&
\tikz \duck[tshirt,tie ,jacket ,cape] ;
\\ \hline
[\RDD{tshirt}] & [\RDD{tie}] & [\RDD{jacket}] & [\RDD{cape}]& [tshirt,tie ,jacket ,cape]
\\ \hline
\dft{white} & \dft{blue} & \dft{blue} & \dft{red}&
\\ \hline
\tikz \duck[tshirt=red] ;
&  
\tikz \duck[tie=red] ;
&
\tikz \duck[jacket=red] ;
&
\tikz \duck[cape=blue] ;
&

\\ \hline
[tshirt=red] & [tie=red] & [jacket=red] & [cape=blue]& 
\\ \hline
\end{tabular}

\bigskip

\begin{tabular}{|c|c|c|c|c|} \hline 
\tikz \duck[water];
&
\tikz \duck[alien];
&
\tikz \duck[hat];
&
\tikz \duck[tophat];
&
\tikz \duck[cap];
\\ \hline
[\RDD{water}] & [\RDD{alien}] & [\RDD{hat}]& [\RDD{tophat}] & [\RDD{cap}]
\\ \hline
\tikz \duck[santa];
&
\tikz \duck[graduate];
&
\tikz \duck[graduate,tassel];
&
\tikz \duck[beret];
&
\tikz \duck[peakedcap];
\\ \hline
[\RDD{santa}] & [\RDD{graduate}] & [graduate,\RDD{tassel}] & [\RDD{beret}] & [\RDD{peakedcap}]
\\ \hline
\tikz \duck[crown];
&
\tikz \duck[queencrown];
&
\tikz \duck[kingcrown];
&
\tikz \duck[sheep];
&
\tikz \duck[horsetail];
\\ \hline
[\RDD{crown}] &[\RDD{queencrown}]&[\RDD{kingcrown}] & [\RDD{sheep}] &[\RDD{horsetail}]
\\ \hline

\tikz \duck[crozier];
&
\tikz \duck[unicorn];
&

\tikz \duck[bunny];
&
\tikz \duck[bunny=red,inear=blue];
&
\tikz \duck[witch];
\\ \hline
 [\RDD{crozier}] & [\RDD{unicorn}] &[\RDD{bunny}] & [bunny=red,\RDD{inear}=blue] & [\RDD{witch}]
\\ \hline
\tikz \duck[magicwand];
&
\tikz \duck[magichat];
&
\tikz \duck[magichat=teal,
magicstars=blue!30!cyan,
magicwand];
&
\tikz \duck[glasses];
&
\tikz \duck[sunglasses];
\\ \hline
[\RDD{magicwand}] & [\RDD{magichat}] & \parbox{3cm}{[magichat,\\ \RDD{magicstars}]} & [\RDD{glasses}] & [\RDD{sunglasses}]
\\ \hline
\end{tabular}  


\begin{tabular}{|c|c|c|c|c|} \hline

\tikz \duck[squareglasses];
&
\tikz \duck[signpost=42];
&
\tikz \duck[signpost=XXX,signcolour=green];
&
\tikz \duck[signpost=XXX,signback=green];
&
\tikz \duck[speech={XXX}];
\\ \hline
[\RDD{squareglasses}] & [\RDD{signpost}=42] & \parbox{3cm}{[signpost=XXX,\\ \RDD{signcolour}=green]} & \parbox{3cm}{[signpost=XXX, \\ \RDD{signback}=green]} & [\RDD{speech}=\AC{XXX}] 
\\ \hline 
\end{tabular}


\begin{tabular}{|c|c|c|c|c|} \hline 
\tikz \duck[speech={XXX},bubblecolour=green];
&
\tikz \duck[think={XXX}];
&
\tikz \duck[think=XXX,bubblecolour=green];
&
\tikz \duck[book={XXX}];
\\ \hline
\parbox{3cm}{[speech={XXX},\\ \RDD{bubblecolour}=green]} & [\RDD{think}=\AC{XXX}] & 
\parbox{3cm}{[think={XXX},\\ \RDD{bubblecolour}=green]}
&[\RDD{book}=\AC{XXX}] 
\\ \hline
%\end{tabular}  
%
%\begin{tabular}{|c|c|c|c|c|} \hline
\tikz \duck[book=XXX,bookcolour=green];
&
\tikz \duck[book=\scalebox{0.5}{XXX}];
&
\multicolumn{2}{|c|}{\tikz 
\duck[signpost=\scalebox{0.4}{
\parbox{2cm}{
\centering XXX \\ XXXXX}}]
;}
\\ \hline

\parbox{3cm}{[book={XXX},\\ \RDD{bookcolour}=green]} 
&
\parbox{3.5cm}{\BS{tikz} \BS{duck}[book=\\ \BS{scalebox}\AC{0.5}\AC{XXX}]; }
&
\multicolumn{2}{|c|}{ \parbox{7cm}{ \BS{tikz} 
\BS{duck}[signpost=\BS{scalebox}\AC{0.4}\AC{ \\
\BS{parbox}\AC{2cm}{  
\BS{centering} XXX \ XXXXX}}]}
;}
\\ \hline
\end{tabular}


\begin{tabular}{|c|c|c|c|c|} \hline 
\tikz \duck[cricket];
&
\tikz \duck[hockey];
&
\tikz \duck[football];
&
\tikz \duck[lightsaber];
&
\tikz \duck[torch];
\\ \hline
[\RDD{cricket}]& [\RDD{hockey}] & [\RDD{football}] & [\RDD{lightsaber}] & [\RDD{torch}]
\\ \hline
\tikz \duck[prison];
&
\tikz \duck[necklace];
&
\tikz \duck[icecream];
&
\tikz \duck[icecream,flavoura=green];
&
\tikz \duck[icecream,flavourb=green];
\\ \hline
[\RDD{prison}] & [\RDD{necklace}] & [\RDD{icecream}] &
\parbox{3cm}{[icecream,\\ \RDD{flavoura}=green]} & 
\parbox{3cm}{[icecream,\\ \RDD{flavourb}=green]}
\\ \hline
\tikz \duck[icecream,flavourc=green];
&
\tikz \duck[chef];
&
\tikz \duck[rollingpin];
&
\tikz \duck[cake];
&
\tikz \duck[pizza];
\\ \hline
\parbox{3cm}{[icecream,\\ \RDD{flavourc}=green]} &
 [\RDD{chef}] & [\RDD{rollingpin}] & [\RDD{cake}] & [\RDD{pizza}]
\\ \hline
\tikz \duck[baguette];
&
\tikz \duck[milkshake];
&
\tikz \duck[wine];
&
\tikz \duck[mask];
&
\tikz \duck[buttons];
\\ \hline
  [\RDD{baguette}] & [\RDD{milkshake}] & [\RDD{wine}] & [\RDD{mask}] & [\RDD{buttons}]
\\ \hline
\end{tabular}

\begin{tabular}{|c|c|c|c|c|} \hline 
\tikz \duck[basket];
&
\tikz \duck[easter];
&
\tikz \duck[easter,egga=red];
&
\tikz \duck[easter,eggb=red];
&
\tikz \duck[easter,eggc=red];
\\ \hline
[\RDD{basket}]& [\RDD{easter}] & [easter,\RDD{egga}=red] & [easter,\RDD{eggb}=red] & [easter,\RDD{eggc}=red]
\\ \hline

\end{tabular}

\bigskip

\begin{tabular}{|c|c|c|c|} \hline 
\multicolumn{4}{|c|}{\BS{tikz} \BS{duck} \BS{path}[preaction=\AC{fill,green},pattern=dots, pattern  color=red]  \BSS{duckpathbody} ;} 
\\ \hline
\tikz \duck
\path[preaction={fill,green},pattern=dots, pattern  color=red]  \duckpathbody;
&
\tikz   \duck
\path[preaction={fill,green},pattern=dots, pattern  color=red]  \duckpathgrumpybill;
&
\tikz   \duck
\path[preaction={fill,green},pattern=dots, pattern  color=red]  \duckpathbill;
&
\tikz   \duck
\path[preaction={fill,green},pattern=dots, pattern  color=red]  \duckpathtshirt;
\\ \hline 
\BSS{duckpathbody } & \BSS{duckpathgrumpybill} & \BSS{duckpathbill} & \BSS{duckpathtshirt} 
\\ \hline
\tikz   \duck
\path[preaction={fill,green},pattern=dots, pattern  color=red]  \duckpathjacket;
&
\tikz   \duck
\path[preaction={fill,green},pattern=dots, pattern  color=red]  \duckpathcape ;
&
\tikz   \duck
\path[preaction={fill,green},pattern=dots, pattern  color=red]  \duckpathshorthair ;
&
\tikz   \duck
\path[preaction={fill,green},pattern=dots, pattern  color=red]  \duckpathlonghair;
\\ \hline 
\BSS{duckpathjacket} & \BSS{duckpathcape}  & \BSS{duckpathshorthair} & \BSS{duckpathlonghair} 
\\ \hline 
\tikz   \duck
\path[preaction={fill,green},pattern=dots, pattern  color=red]  \duckpathcrazyhair;
&
\tikz   \duck
\path[preaction={fill,green},pattern=dots, pattern  color=red]  \duckpathrecedinghair;
&
\tikz   \duck
\path[preaction={fill,green},pattern=dots, pattern  color=red]  \duckpathcrown ;
&
\tikz   \duck
\path[preaction={fill,green},pattern=dots, pattern  color=red]   \duckpathmohican ;
\\ \hline 
\BSS{duckpathcrazyhair} & \BSS{duckpathrecedinghair} & \BSS{duckpathcrown} & \BSS{duckpathmohican}
\\ \hline 
\tikz   \duck
\path[preaction={fill,green},pattern=dots, pattern  color=red]   \duckpathmullet;
&
\tikz   \duck
\path[preaction={fill,green},pattern=dots, pattern  color=red]   \duckpathqueencrown ;
&
\tikz   \duck
\path[preaction={fill,green},pattern=dots, pattern  color=red]   \duckpathkingcrown ;
&
\tikz   \duck
\path[preaction={fill,green},pattern=dots, pattern  color=red]  \duckpathdarthvader ;
\\ \hline 
\BSS{duckpathmullet} & \BSS{duckpathqueencrown} & \BSS{duckpathkingcrown} & \BSS{duckpathdarthvader}
\\ \hline 
\tikz   \duck
\path[preaction={fill,green},pattern=dots, pattern  color=red]  \duckpathhorsetail ;
& & & 
\\ \hline 
\BSS{duckpathhorsetail}& & & 
\\ \hline 
\end{tabular}


\SbSbSSCT{Canards aléatoires}{Random ducks}

\noindent

\begin{tabular}{|c|}\hline  
\BS{tikz} \BSS{randuck} ; \BS{tikz} \BSS{randuck} ; \BS{tikz} \BSS{randuck} ; \BS{tikz} \BSS{randuck} ; \BS{tikz} \BSS{randuck} ; 
\\ \hline    
\tikz \randuck ; \tikz \randuck  ; \tikz  \randuck  ; \tikz  \randuck  ; \tikz  \randuck ;
\\ \hline 
\end{tabular} 

\bigskip

\begin{tabular}{|c|} \hline  
\BS{tikz} \BSS{shuffleducks} \BS{duck}[\BSS{randomhead}] ;
\\ \hline  
\tikz \shuffleducks \duck[\randomhead] ; \tikz \shuffleducks \duck[\randomhead] ; \tikz \shuffleducks \duck[\randomhead] ; \tikz \shuffleducks \duck[\randomhead] ;
\tikz \shuffleducks \duck[\randomhead] ;
\\ \hline 
\end{tabular} 

\bigskip

\begin{tabular}{|c|} \hline  
\BS{tikz} \BSS{shuffleducks} \BS{duck}[\BSS{randomaccessories}] ;
\\ \hline  
\tikz \shuffleducks \duck[\randomaccessories] ; \tikz \shuffleducks \duck[\randomaccessories] ; \tikz \shuffleducks \duck[\randomaccessories] ; \tikz \shuffleducks \duck[\randomaccessories] ; \tikz \shuffleducks \duck[\randomaccessories] ; 
\\ \hline 
\end{tabular} 


\SbSbSSCT{Coordonnées}{Coordinates}

\noindent


\begin{tabular}{|c|c|c|} \hline 
\multicolumn{3}{|c|}{\BS{tikz} \BS{duck} \BS{fill}[red] (wing) circle (3pt);}
\\ \hline  
\tikz \duck \fill[red] (wing) circle (3pt);
&
\tikz \duck \fill[red] (head) circle (3pt);
&
\tikz \duck \fill[red] (bill) circle (3pt);
\\ \hline wing &  head & bill \\ 
\hline 
\end{tabular} 

\bigskip

\begin{tabular}{|c|} \hline
\BS{tikz} \BS{duck}[\RDD{name}=XXX] \\ \BS{begin}\AC{scope} [xshift=4cm] \BS{duck}[\RDD{name}=YYY] 
\BS{end}\AC{scope} \\ \BS{draw}[red] (XXX-wing) - - (YYY-bill) ;
\\ \hline
\tikz \duck[name=XXX] \begin{scope} [xshift=4cm] 
\duck[name=YYY]
\end{scope}  \draw[red] (XXX-wing) -- (YYY-bill);
\\ \hline
\end{tabular} 


\SbSbSSCT{Rayures}{Stripes}

\noindent

\begin{tabular}{|c|c|} \hline  
\tikz \duck \stripes ;
&  
\tikz \duck[stripes] ;
\\ \hline  
\BS{tikz} \BS{duck} \BSS{stripes} ; 
&  
\BS{tikz} \BS{}duck[\RDD{stripes}] ;
\\ \hline 
\end{tabular} 

\bigskip

\begin{tabular}{|c|c|} \hline  
\tikz \duck[rollingpin] \stripes ;
&  
\tikz  \duck[rollingpin,stripes] ;
\\ \hline  
\BS{tikz} \BS{duck}[rollingpin] \BS{stripes} ;
&  
\BS{tikz}  \BS{duck}[rollingpin,stripes] ;
\\ \hline 
\end{tabular} 



\bigskip

\begin{tabular}{|c|c|c|c|}\hline 
\multicolumn{4}{|c|}{\BS{tikz} \BS[duck] \BS{stripes}[\RDD{color}=red];}
\\ \hline  
\tikz \duck \stripes[color=red];
& 
\tikz \duck \stripes[distance=.5]; 
&  
\tikz \duck \stripes[width=.05];
&
\tikz \duck \stripes[height=1];  
\\ \hline 
[\RDD{color}=red] & [\RDD{distance}=.5] & [\RDD{width}=.05] & [\RDD{height}=1] 
\\ \hline 
\dft{black} & \dft{0.3}  & \dft{0.15}  &  \dft{2.7} \\ 
\hline  

\tikz \duck \stripes[rotate=45]; & \tikz \duck \stripes[initialx=1]; & \tikz \duck \stripes[initialy=1]; &  
\\ \hline 
[\RDD{rotate=}45] & [\RDD{initialx}=1] & [\RDD{initialy}=1] &
\\ \hline 
\dft{-10} & \dft{0.1} & \dft{-0.3} & 
\\ \hline 
\end{tabular} 


\bigskip

\begin{tabular}{|c|c|c|} \hline
\multicolumn{3}{|c|}{\BS{tikz} \BS[duck] \BS{stripes}[\RDD{emblem}=XXX];}
\\ \hline 
\tikz \duck \stripes[emblem={XXX}];
&  
\tikz \duck \stripes[emblem={\includegraphics[width=6mm]{LogoIUT}}];
&  
\tikz \duck \stripes[emblem={\DFR}];
\\ \hline  
[emblem=XXX]
& \parbox{5cm}{ [emblem=\{\BS{includegraphics}  [width=6mm]\AC{LogoIUT} \} ] }  
&  [emblem=\AC{\BS{DFR}}  ] 
\\ \hline 
& &  \BS{DFR} : \TFRGB{voir}{see} page \pageref{DFR} 
\\ \hline   
\end{tabular} 

\bigskip

\begin{tabular}{|c|c|c|} \hline
\tikz \duck[stripes={ \stripes \stripes[rotate=45]}] ;
\\ \hline 
\BS{tikz}
\BS{duck}[stripes=\AC{
\BS{stripes}
\BS{stripes}[rotate=45] } ]
;

\\ \hline  
\end{tabular}




\newpage


\subsection{symbol}

\input{tkzsymbol}

\newpage
 %==================================================
 
\SSCT{Créer un graphe }{Creating Graphs}


\SbSSCT{Graphe avec TikZ}{Graph with TikZ}

\SbSbSSCT{Graphe à partir d'une liste de points}{From a list of points}
\label{plot}

\RRR{22-2}
\begin{tabular}{|c | } \hline
\BS{tikz} \BS{draw} plot \RDD{coordinates} \AC{(0,0) (1,1) (2,0) (3,1) (4,1) (5,2)}; \\ 
\hline
\tikz \draw plot coordinates {(0,0) (1,1) (2,0) (3,1) (4,1) (5,2)};
\\ \hline
\end{tabular}


\SbSbSSCT{Graphe à partir partir d'un fichier de données}{From a data file}

\begin{tabular}{|c | c | c | c |} \hline
\multicolumn{4}{|c|}{ \BS{tikz} \BS{draw}  plot[mark=x] \RDD{file} \AC{table.dat} ;   }\\ 
\hline
& 
\tikz \draw plot[mark=x,smooth] file {table.dat};
&
\tikz \draw plot[mark=x,smooth,tension=.2] file {table.dat};
&
\tikz \draw plot[mark=x,smooth,tension=1] file {table.dat};
\\ \hline
[mark=x] & [mark=x,\RDD{smooth}] & [mark=x,smooth,\RDD{tension}=.2] & [mark=x,smooth,\RDD{tension}=1]
\\ \hline
\multicolumn{4}{|c|}{ \dft : tension= 0:55}
\\ \hline
\end{tabular}

\bigskip


\begin{tabular}{|c  c |} \hline
\multicolumn{2}{|c|}{\TFRGB{Contenu du fichier}{content of the file} table.dat}
\\ \hline
0.0 & 0.3 \\
0.3 & 0.6 \\
0.6 & 0.9 \\
0.9 & 1.5  \\
1.2 & 0.6  \\
1.5 & 1.2  \\
1.8 & 1.5  \\
2.1 & 2.0 \\
2.4 & 3.0 \\
\hline
\end{tabular}

\bigskip

\SbSbSSCT{Les types de graphes}{Graph types}

\begin{tabular}{|c | c | c | c |} \hline
\multicolumn{4}{|c|}{ \BS{tikz} \BS{draw}  plot[mark=*,\RDD{const plot}] file \AC{table.dat} ;   }\\ 
\hline
\tikz \draw plot[mark=*,const plot] file {table.dat};
&

\tikz \draw plot[const plot mark left,mark=*] file {table.dat};
&
\tikz \draw plot[const plot mark right,mark=*] file {table.dat};
&
\tikz \draw plot[jump mark left, mark=*] file {table.dat};
\\ \hline
\RDD{const plot} & \RDD{const plot mark left} & \RDD{const plot mark right} & \RDD{jump mark left}
\\ \hline
\tikz \draw plot[jump mark right, mark=*] file {table.dat};
&
\tikz \draw plot[ycomb,thin,mark=*] file {table.dat};
&
\tikz \draw plot[xcomb,mark=*] file {table.dat};
&
\tikz \draw plot[only marks,mark=*] file {table.dat};
\\ \hline
\RDD{jump mark right} & \RDD{ycomb} & \RDD{xcomb} & \RDD{only marks}
\\ \hline
\end{tabular}

\bigskip
\begin{tabular}{|c | c | c |c |} \hline

\tikz  \draw plot[polar comb,mark=*]coordinates {(0:1) (60:0.5) (120:1.5) (180:3) (240:.5) (300:1) (0:1)};
\\ \hline
\BS{tikz}  \BS{draw} plot[\RDD{polar comb},mark=*]coordinates \\
\AC{(0:1) (60:0.5) (120:1.5) (180:3) (240:.5) (300:1) (0:1)};
\\ \hline
\end{tabular}

\bigskip

\begin{tabular}{|c | c | c |c |} \hline
\multicolumn{4}{|c|}{ \BS{tikz} \BS{draw}  plot[\RDD{ybar}] file \AC{table.dat} ;   }\\ 
\hline
\tikz \draw plot[ybar] file {table.dat};
&
\tikz \draw plot[ybar interval] file {table.dat};
&
\tikz \draw plot[ybar interval,x=2cm] file {table.dat};
&
\tikz \draw plot[ybar interval,y=.5cm] file {table.dat};
\\ \hline
[\RDD{ybar}] & [\RDD{ybar interval}] & [ybar interval,\RDD{x}=2cm] & [ybar interval,\RDD{y}=.5cm]
\\ \hline
\end{tabular}

\bigskip
 \begin{tabular}{|c|c|}  \hline 
\begin{tikzpicture}[baseline=0pt]
\draw[red,fill=cyan,ybar,bar width=.5cm]plot coordinates{(0,1) (1,1.2) (2,.6) (3,.7) (4,.9)};
\draw[blue,fill=green,ybar,bar width=.5cm,bar shift=.3cm]plot coordinates{(0,1.2) (1,1.3) (2,.5) (3,.2) (4,.5)};
\end{tikzpicture}
&
\parbox[c]{10cm}{
\BS{begin}\AC{tikzpicture} \\
\BS{draw}[red,fill=cyan,ybar,bar width=.5cm] \\
\rule{1cm}{.0pt} plot coordinates \AC{(0,1) (1,1.2) (2,.6) (3,.7) (4,.9)}; \\
\BS{draw}[blue,fill=green,ybar,bar width=.5cm,\RDD{bar shift}=.3cm] \\
\rule{1cm}{.0pt} plot coordinates \AC{(0,1.2) (1,1.3) (2,.5) (3,.2) (4,.5)}; \\
\BS{end}\AC{tikzpicture} }
 \\  \hline 
 \end{tabular} 

\bigskip

\begin{tabular}{|c | c | c | c |c |} \hline
\multicolumn{4}{|c|}{ \BS{tikz} \BS{draw}  plot[xbar interval] file \AC{table.dat} ;   }\\ 
\hline
\tikz \draw[blue] plot[xbar] coordinates{(2,0) (3,1) (1,2) (2,3)};
&
\tikz \draw[blue] plot[xbar interval]  coordinates {(2,0) (3,1) (1,2) (2,3)};
&
\tikz \draw[blue] plot[xbar interval,x=.5cm]  coordinates {(2,0) (3,1) (1,2) (2,3)};
&
\tikz \draw[blue] plot[xbar interval,y=.5cm]  coordinates {(2,0) (3,1) (1,2) (2,3)};
\\ \hline
[\RDD{xbar}] & [\RDD{xbar interval}] & [xbar interval,\RDD{x}=.5cm] & [xbar interval,\RDD{y}=.5cm] 
\\ \hline
\end{tabular}

\newpage

\SbSbSSCT{Graphe à partir d'une fonction}{Graph of a function}


\begin{tabular}{|c | c | c | } \hline
\multicolumn{3}{|c|}{  \BS{draw}  [color=red] plot (\BS{x},\BS{x});   }\\ 
\hline
\begin{tikzpicture}[domain=0:4,ultra thick]

\draw[->,blue,ultra thick] (-.1,0) -- (4.5,0);
\draw[->,blue,ultra thick] (0,-1.1) -- (0,04);
\draw[color=red] plot (\x,\x);
\end{tikzpicture} 
&
\begin{tikzpicture}[domain=0:6.28,ultra thick,x=0.7cm]
\draw[->,blue,ultra thick] (-.1,0) -- (7,0);
\draw[->,blue,ultra thick] (0,-2.5) -- (0,2.5);
\draw[color=red] plot  (\x,{sin(\x r)});
\end{tikzpicture} 
&
\begin{tikzpicture}[domain=0:360,x=0.3,ultra thick]
\draw[->,blue,ultra thick] (-.1,0) -- (370,0);
\draw[->,blue,ultra thick] (0,-2.5) -- (0,2.5);
\draw[color=red] plot (\x,{sin(\x)});
\end{tikzpicture} 
\\ \hline
(\BS{x},\BS{x}) &  (\BS{x},\AC{sin(\BS{x} r)}) & (\BS{x},\AC{sin(\BS{x})}) \\
& x en radian & x en degré
\\ \hline
\end{tabular}

Options 

\begin{tabular}{|c | c |} \hline
\multicolumn{2}{|l|}{ \BS{draw}[color=red,dashed] plot(\BS{x},\AC{sin(\BS{x} r)});}  \\
\multicolumn{2}{|l|}{ \BS{draw}[color=blue,\RDD{samples}=5,mark=*,ultra thick] plot(\BS{x},\AC{sin(\BS{x} r)});} 
\\ \hline
\begin{tikzpicture}[domain=0:6.28]
\draw[very thin,color=gray] (-0.1,-1.1) grid (6.28,1.1);
\draw[color=red,dashed] plot  (\x,{sin(\x r)});
\draw[color=blue,samples=5,mark=*,ultra thick] plot  (\x,{sin(\x r)});
\end{tikzpicture} 
&
\begin{tikzpicture}
\draw[very thin,color=gray] (-0.1,-1.1) grid (6.28,1.1);
\draw[color=red,dashed,domain=0:6.28] plot  (\x,{sin(\x r)});
\draw[color=blue,domain=0:4,ultra thick] plot  (\x,{sin(\x r)});
\end{tikzpicture} 
  \\ \hline
[color=blue,\RDD{samples}=5,mark=*] & [color=blue,\RDD{domain}=0:4]
\\ \hline
\begin{tikzpicture}
\draw[very thin,color=gray] (-0.1,-1.1) grid (6.28,1.1);
\draw[color=red,dashed,domain=0:6.28] plot  (\x,{sin(\x r)});
\draw[color=blue,domain=1:5,ultra thick] plot  (\x,{sin(\x r)});
\end{tikzpicture} 
&
\begin{tikzpicture}[domain=0:6.28]
\draw[very thin,color=gray] (-0.1,-1.1) grid (6.28,1.1);
\draw[color=red,dashed] plot  (\x,{sin(\x r)});
\draw[color=blue,samples at={1,2,4,5},mark=*,ultra thick] plot  (\x,{sin(\x r)});
\end{tikzpicture} 
\\ \hline
[color=blue,\RDD{domain}=1:5] & [color=blue,\RDD{samples at}=\AC{1,2,4,5},mark=*]
\\ \hline
\end{tabular}

\SbSbSSCT{Fonctions paramétriques}{Parametric function}


\begin{tabular}{|c | c |} \hline
\multicolumn{2}{|l|}{  \BS{draw}[domain=-3.141:3.141,smooth,variable=\BS{t}]plot (\AC{sin(\BS{t} r)},\AC{sin(2 *\BS{t} r)});} \\
\multicolumn{2}{|l|}{  \BS{draw}[domain=0:720,smooth,variable=\BS{t}]plot (\AC{sin(\BS{t})},\BS{t}/360,\AC{cos(\BS{t})});} 
\\ \hline

\tikz \draw[domain=-3.141:3.141,smooth,variable=\t,ultra thick]plot ({sin(\t r)},{sin(2*\t r)});
&
\tikz \draw[domain=0:720,smooth,variable=\t,ultra thick] plot ({sin(\t)},\t/360,{cos(\t)});
\\ \hline
(\AC{sin(\BS{t} r)},\AC{sin(2 *\BS{t} r)}) & (\AC{sin(\BS{t})},\BS{t}/360,\AC{cos(\BS{t})})
\\ \hline
\end{tabular} 

\SbSSCT{Marques}{Marks}

\SbSbSSCT{Marques avec TikZ}{Marks with TikZ}

\begin{tabular}{|c | c | c | c |} \hline
\tikz \draw plot[mark=+,mark size=5pt] coordinates {(0,0) (1,1) (2,0)};
&
\tikz \draw plot[mark=x,mark size=5pt] coordinates {(0,0) (1,1) (2,0) };
&
\tikz \draw plot[mark=*,mark size=5pt] coordinates {(0,0) (1,1) (2,0)};
&
\tikz \draw plot[mark=ball,mark size=5pt] coordinates {(0,0) (1,1) (2,0)};
\\ \hline
mark=+ & mark=x & mark=* & mark=ball
\\ \hline
\end{tabular}

\bigskip

\begin{tabular}{|c | c |} \hline
\begin{tikzpicture}[domain=0:6.28]
\draw[very thin,color=gray] (-0.1,-1.1) grid (6.28,1.1);
\draw[color=red,dashed,mark=+] plot  (\x,{sin(\x r)});
\draw[color=blue,mark repeat=3,mark=*] plot  (\x,{sin(\x r)});
\end{tikzpicture} 
&
\begin{tikzpicture}[domain=0:6.28]
\draw[very thin,color=gray] (-0.1,-1.1) grid (6.28,1.1);
\draw[color=red,dashed,mark=+] plot  (\x,{sin(\x r)});
\draw[color=blue,mark repeat=3,mark phase=5,mark=*] plot  (\x,{sin(\x r)});
\end{tikzpicture} 
\\ \hline
[color=blue,\RDD{mark repeat}=3,mark=*] & [color=blue,mark repeat=3,\RDD{mark phase}=5,mark=*]
\\ \hline
\begin{tikzpicture}[domain=0:6.28]
\draw[very thin,color=gray] (-0.1,-1.1) grid (6.28,1.1);
\draw[color=red,dashed,mark=+] plot  (\x,{sin(\x r)});
\draw[color=blue,mark indices={1,4,...,15,17,20},mark=*] plot  (\x,{sin(\x r)});
\end{tikzpicture} 
&
\begin{tikzpicture}[domain=0:6.28]
\draw[very thin,color=gray] (-0.1,-1.1) grid (6.28,1.1);
\draw[color=red,dashed,mark=+] plot  (\x,{sin(\x r)});
\draw[color=blue,mark size=5pt,mark=*] plot  (\x,{sin(\x r)});
\end{tikzpicture} 
\\ \hline
[color=blue,\RDD{mark indices}={1,4,...,15,17,20},mark=*] & [color=blue,\RDD{mark size}=5pt,mark=*]
\\ \hline
\begin{tikzpicture}[domain=0:6.28]
\draw[very thin,color=gray] (-0.1,-1.1) grid (6.28,1.1);

\draw[color=blue,mark size=5pt,mark options={color=magenta},mark=+] plot  (\x,{sin(\x r)});
\end{tikzpicture}
&
\begin{tikzpicture}[domain=0:6.28]
\draw[very thin,color=gray] (-0.1,-1.1) grid (6.28,1.1);

\draw[color=blue,mark size=5pt,mark options={rotate=10},mark=+] plot  (\x,{sin(\x r)});
\end{tikzpicture}
\\ \hline
\RDD{mark options}=\AC{color=magenta},mark=+ & \RDD{mark options}=\AC{rotate=10},mark=+
\\ \hline
\end{tabular}

\SbSbSSCT{Marques personnalisées avec text mark}{Marks with text mark}

\begin{tabular}{|c | c | c |} \hline
\multicolumn{3}{|l|}{ \BS{draw}[\RDD{mark}=\RDDX{text}{mark},\RDD{text mark}=A,mark size=5pt] coordinates \AC{(0,0) (1,1) (2,0)};} 
\\ \hline
\tikz \draw plot[mark=text,text mark=A,mark size=5pt] coordinates {(0,0) (1,1) (2,0)};
&
\tikz \draw plot[mark=text,text mark=Texte,mark size=5pt] coordinates {(0,0) (1,1) (2,0)};
&
\begin{tikzpicture}
\draw[white]  (-1,0)-- (-1,1.5);
 \draw plot[mark=text,text mark=\DFR,mark size=5pt] coordinates {(0,0) (1,1) (2,0)};
\end{tikzpicture} 
\\ \hline
\RDD{text mark}=A &  \RDD{text mark}=Texte & \RDD{text mark}=\BS{DFR} \pageref{DFR} 
\\ \hline 
\multicolumn{3}{|c|}{ 
\begin{tikzpicture}
\draw[white]  (-1,0)-- (-1,1.5);
\draw plot[mark=text,text mark={\includegraphics[width=.5cm]{tiger}} ,mark size=5pt] coordinates {(0,0) (1,1) (2,0)};  
\end{tikzpicture} }
\\ \hline  
\multicolumn{3}{|c|}{ \RDD{text mark}=\AC{\BS{includegraphics}[width=.5cm]\AC{tiger}} }
\\ \hline   
\end{tabular}


\newpage

\SbSbSSCT{Marques avec l'extension plotmarks }{Marks with plotmarks library}

\label{plotmarks}


 \maboite{\BS{usetikzlibrary}\AC{plotmarks}}
 
\begin{center}
\RRR{63}
\end{center}

\begin{tabular}{|c | c | c | c |} \hline
\tikz \draw plot[mark=-,mark size=5pt] coordinates {(0,0) (1,1) (2,0)};
& 
\tikz \draw plot[mark=|,mark size=5pt] coordinates {(0,0) (1,1) (2,0)};
 &
\tikz \draw plot[mark=o,mark size=5pt] coordinates {(0,0) (1,1) (2,0)};
 &
\tikz \draw plot[mark=asterisk,mark size=5pt] coordinates {(0,0) (1,1) (2,0)};
\\ \hline 
mark=\rouge{-} & mark=\rouge{|} & mark=\RDDX{o}{mark} & mark=\RDDX{asterisk}{mark}
\\ \hline
\tikz \draw plot[mark=star,mark size=5pt] coordinates {(0,0) (1,1) (2,0)};
&
\tikz \draw plot[mark=10-pointed star,mark size=5pt] coordinates {(0,0) (1,1) (2,0)};
&
\tikz \draw plot[mark=oplus,mark size=5pt] coordinates {(0,0) (1,1) (2,0)};
&
\tikz \draw plot[mark=oplus*,mark size=5pt] coordinates {(0,0) (1,1) (2,0)};
\\ \hline
mark==\RDDX{star}{mark}  & mark==\RDDX{10-pointed star}{mark}  & mark=\RDDX{oplus}{mark} & mark=\RDDX{oplus*}{mark} 
\\ \hline
 
\tikz \draw plot[mark=otimes,mark size=5pt] coordinates {(0,0) (1,1) (2,0)};
&
\tikz \draw plot[mark=otimes*,mark size=5pt] coordinates {(0,0) (1,1) (2,0)};
&
\tikz \draw plot[mark=square,mark size=5pt] coordinates {(0,0) (1,1) (2,0)};
&
\tikz \draw plot[mark=square*,mark size=5pt] coordinates {(0,0) (1,1) (2,0)};
\\ \hline
 mark=\RDDX{otimes}{mark} & mark=\RDDX{otimes*}{mark} & mark=\RDDX{square}{mark} & mark=\RDDX{square*}{mark}
  \\ \hline
  
\tikz \draw plot[mark=triangle,mark size=5pt] coordinates {(0,0) (1,1) (2,0)};
& 
\tikz \draw plot[mark=triangle*,mark size=5pt] coordinates {(0,0) (1,1) (2,0)};
& 
\tikz \draw plot[mark=diamond,mark size=5pt]  coordinates {(0,0) (1,1) (2,0)};
 &
\tikz \draw plot[mark=diamond*,mark size=5pt] coordinates {(0,0) (1,1) (2,0)};
\\ \hline 
mark=\RDDX{triangle}{mark} & mark=\RDDX{triangle*}{mark} & mark=\RDDX{diamond}{mark} & mark=\RDDX{diamond*}{mark}
\\ \hline 

\tikz \draw plot[mark=halfdiamond*,mark size=5pt] coordinates {(0,0) (1,1) (2,0)};
 &
\tikz \draw plot[mark=halfsquare*,mark size=5pt] coordinates {(0,0) (1,1) (2,0)};
 &
\tikz \draw plot[mark=halfsquare right*,mark size=5pt] coordinates {(0,0) (1,1) (2,0)};
 &
\tikz \draw plot[mark=halfsquare left*,mark size=5pt] coordinates {(0,0) (1,1) (2,0)};
\\ \hline 
mark=\RDDX{halfdiamond*}{mark} & mark=\RDDX{halfsquare*}{mark} & mark=\RDDX{halfsquare right*}{mark} & mark=\RDDX{halfsquare left*}{mark}
\\ \hline 

\tikz \draw plot[mark=pentagon,mark size=5pt] coordinates {(0,0) (1,1) (2,0)};
 &
\tikz \draw plot[mark=pentagon*,mark size=5pt] coordinates {(0,0) (1,1) (2,0)};
 &
\tikz \draw plot[mark=Mercedes star,mark size=5pt] coordinates {(0,0) (1,1) (2,0)};
 &
\tikz \draw plot[mark=Mercedes star flipped,mark size=5pt] coordinates {(0,0) (1,1) (2,0)};
 \\ \hline
 mark=\RDDX{pentagon}{mark} & mark=\RDDX{pentagon*}{mark} & mark=\RDDX{Mercedes star}{mark} & mark=\RDDX{Mercedes star flipped}{mark}
 \\ \hline 
 
\tikz \draw plot[mark=halfcircle,mark size=5pt] coordinates {(0,0) (1,1) (2,0)};
 &
\tikz \draw plot[mark=halfcircle*,mark size=5pt] coordinates {(0,0) (1,1) (2,0)};
& 
\tikz \draw plot[mark=heart,mark size=5pt] coordinates {(0,0) (1,1) (2,0)};
 &
\tikz \draw plot[mark=text,mark size=5pt] coordinates {(0,0) (1,1) (2,0)};
 \\ \hline
 mark=\RDDX{halfcircle}{mark} & mark=\RDDX{halfcircle*}{mark} & mark=\RDDX{heart}{mark} & mark=\RDDX{text}{mark}
  \\ \hline
\end{tabular}

\bigskip

\begin{tabular}{|c | c | c | c |} \hline
\multicolumn{4}{|l|}{ \BS{draw}[mark=halfcircle,\RDD{mark color}=red,mark size=5pt] coordinates \AC{(0,0) (1,1) (2,0)};} 
\\ \hline
\tikz \draw plot[mark=halfcircle,mark color=red,mark size=5pt] coordinates {(0,0) (1,1) (2,0)};
&
\tikz \draw plot[mark=halfcircle*,mark color=red,mark size=5pt] coordinates {(0,0) (1,1) (2,0)};
&
\tikz \draw plot[mark=halfdiamond*,mark color=red,mark size=5pt] coordinates {(0,0) (1,1) (2,0)};
&
\tikz \draw plot[mark=halfsquare*,mark color=red,mark size=5pt] coordinates {(0,0) (1,1) (2,0)};
  \\ \hline
  mark=halfcircle & mark=halfcircle* & mark=halfdiamond* & mark=halfsquare*
   \\ \hline 
\end{tabular}

\SbSSCT{Graphes avec Gnuplot}{Graph with Gnuplot}
 
 \begin{tabular}{|l| } \hline

\BS{draw}[color=red] plot[\RDD{id}=sin] function\AC{sin(x)} ;
   \\ \hline
\\
==> plot[id=sin] \TFRGB{crée le fichier}{create the file} \og sin.gnuplot \fg \\
==>  \TFRGB{Ouvrir le fichier}{Open the file} \og sin.gnuplot \fg \TFRGB{avec le programme gnuplot pour créer le fichier}{with the program gnuplot : creation of the file }   \og sin.table \fg\\
==> \TFRGB{Utiliser le fichier de données} {Use the datafile }
 \og sin.table  \fg   \\ \hline 
\end{tabular}

\newpage

\SSCT{Créer un graphe avec pgfplot}{Creation of a graph with pgfplots}



 \maboite{\BS{usepackage}\AC{pgfplots} \cite {pgfplots} }
\label{pgfplots}


\SbSSCT{Courbes 2 D}{2D Graph}

\subsubsection{Axes}

\begin{tabular}{|c|c|c|c|} \hline 

 \multicolumn{4}{|c|}{ \RRP{4-1}}  
 \\ \hline 
\begin{tikzpicture}[scale=.5,blue]
\begin{axis}
...
\end{axis}
\end{tikzpicture}
&
\begin{tikzpicture}[scale=.5,blue]
\begin{semilogxaxis}
...
\end{semilogxaxis}
\end{tikzpicture}
&
\begin{tikzpicture}[scale=.5,blue]
\begin{semilogyaxis}

\end{semilogyaxis}
\end{tikzpicture}
&
\begin{tikzpicture}[scale=.5,blue]
\begin{loglogaxis}

\end{loglogaxis}
\end{tikzpicture}
\\ \hline 
\ESS{axis} & \ESS{semilogxaxis} & \ESS{semilogyaxis} & \ESS{loglogaxis} \\
& & &\\

\BS{end}\AC{axis} & \BS{end}\AC{semilogxaxis} & \BS{end}\AC{semilogyaxis} & \BS{end}\AC{loglogaxis}
\\ \hline 
\end{tabular}


\SbSSCT{Tracé de la courbe}{Drawing of the graph}


\begin{tabular}{|c|c|c|} \hline 
 \multicolumn{3}{|c|}{ \RRP{4-2}}  
 \\ \hline 
\begin{tikzpicture}[scale=.5]
\begin{axis}
\addplot coordinates {(0,0) (1,1) (2,0) (3,1) (4,1) (5,2)};
\end{axis}
\end{tikzpicture}
&
\begin{tikzpicture}[scale=.5]
\begin{axis}
\addplot {x^2 - x +4};
\end{axis}
\end{tikzpicture}
& 

\\ \hline 
\BSS{addplot} coordinates  & \BSS{addplot}  \AC{x\^{}2 - x +4}; & \BSS{addplot}  gnuplot[id=sin]\AC{sin(x)};\\
\AC{(0,0) (1,1) (2,0) (3,1) (4,1) (5,2)}; & &
\\ \hline 
\end{tabular}

\bigskip
\begin{tabular}{|c|c|c|c|} \hline 
\begin{tikzpicture}[scale=.5]
\begin{semilogxaxis}
\addplot coordinates {(0,0) (1,1) (2,0) (3,1) (4,1) (5,2)};
\end{semilogxaxis}
\end{tikzpicture}
&
\begin{tikzpicture}[scale=.5]
\begin{semilogxaxis}
\addplot[domain=1:3] {x^2 - x +4};
\end{semilogxaxis}
\end{tikzpicture}
& 
\begin{tikzpicture}[scale=.5]
\begin{semilogyaxis}
\addplot {x^2 - x +4};
\end{semilogyaxis}
\end{tikzpicture}
\\ \hline 
axes : \RDD{semilogxaxis} & axes : \RDD{semilogxaxis} & axes : \RDD{semilogyaxis }
\\ \hline
\BSS{addplot} coordinates  & \BSS{addplot}  \AC{x\^{}2 - x +4}; & \BSS{addplot}  \AC{x\^{}2 - x +4};\\
\AC{(0,0) (1,1) (2,0) (3,1) (4,1) (5,2)}; & &
\\ \hline 
\end{tabular}
\bigskip

%\begin{tabular}{|c|c|c|c|} \hline 
%\begin{tikzpicture}[scale=.5]
%\begin{axis}
%\addplot[red,dashed]  file {table2.dat};
%\addplot[surf]  file {table2.dat};
%\end{axis}
%\end{tikzpicture}
%&
%\begin{tikzpicture}[scale=.5]
%\begin{axis}
%\addplot[red,dashed]  file {table2.dat};
%\addplot[mesh]  file {table2.dat};
%\end{axis}
%\end{tikzpicture}
%&
%\begin{tikzpicture}[scale=.5]
%\begin{axis}
%\addplot[red,dashed]  file {table2.dat};
%\addplot[patch]  file {table2.dat};
%\end{axis}
%\end{tikzpicture}
%\\ \hline 
%patch type=triangle & patch type=rectangle & patch type=line
%\\ \hline 
%\end{tabular}




\begin{tabular}{|c|c|c|c|} \hline 
\begin{tikzpicture}[scale=.5]
\begin{axis}[domain=-1:3]
\addplot {x^2 - x +4};
\end{axis}
\end{tikzpicture}
&
\begin{tikzpicture}[scale=.5]
\begin{axis}[samples=5]
\addplot {x^2 - x +4};
\end{axis}
\end{tikzpicture}
&
\begin{tikzpicture}[scale=.5]
\begin{axis}[domain=-1:3,samples=5]
\addplot {x^2 - x +4};
\end{axis}
\end{tikzpicture}
\\ \hline 
\BS{begin}\AC{axis}[\RDD{domain}=-1:3] &\BS{begin}\AC{axis}[\RDD{samples}=5] & \BS{begin}\AC{axis}[\RDD{domain}=-1:3,\RDD{samples}=5] 
\\ \hline 
\end{tabular}

\begin{tabular}{|c|c|c|c|} \hline 
\begin{tikzpicture}[scale=.4]
\begin{axis}[ymax=20,blue]
\addplot {x^2 - x +4};
\end{axis}
\end{tikzpicture}
&
\begin{tikzpicture}[scale=.4]
\begin{axis}[ymin=10,blue]
\addplot {x^2 - x +4};
\end{axis}
\end{tikzpicture}
&

\begin{tikzpicture}[scale=.5]
\begin{axis}[xmax=2,blue]
\addplot {x^2 - x +4};
\end{axis}
\end{tikzpicture}
&
\begin{tikzpicture}[scale=.5]
\begin{axis}[xmin=-2,blue]
\addplot {x^2 - x +4};
\end{axis}
\end{tikzpicture}
\\ \hline 
\RDD{ymax}=20 & \RDD{ymin}=10 & \RDD{xmax}=2 & \RDD{xmin}=-2
\\ \hline
\end{tabular}

\SbSbSSCT{Dimension unitaire en X et Y}{Xunit and Yunit}

\begin{tabular}{|c|c|c|c|} \hline 
\begin{tikzpicture}
\begin{axis}[x=.2cm,blue]
 \addplot {x^2 - x +4};
\end{axis}
\end{tikzpicture}
&
\begin{tikzpicture}
\begin{axis}[y=.2cm,blue]
 \addplot {x^2 - x +4};
\end{axis}
\end{tikzpicture}
&
\begin{tikzpicture}
\begin{axis}[x=.2cm,y=.2cm,blue]
 \addplot {x^2 - x +4};
\end{axis}
\end{tikzpicture}

\\ \hline 
\BS{begin}\AC{axis}[\RDD{x}=.2cm] & \BS{begin}\AC{axis}[\RDD{y}=.2cm]& \BS{begin}\AC{axis}[\RDD{x}=.2cm,\RDD{y}=.2cm]
\\ \hline 
\end{tabular}

\SbSbSSCT{Type de graphiques}{Graph type}

\begin{tabular}{|c|c|c|c|} \hline 
\begin{tikzpicture}[scale=.5]
\begin{axis}[const plot,blue]
\addplot file {table2.dat};
\end{axis}
\end{tikzpicture}
&
\begin{tikzpicture}[scale=.5]
\begin{axis}[const plot mark right,blue]
\addplot  file {table2.dat};
\end{axis}
\end{tikzpicture}
&
\begin{tikzpicture}[scale=.5]
\begin{axis}[const plot mark mid,blue]
\addplot  file {table2.dat};
\end{axis}
\end{tikzpicture}
\\ \hline 
\RDD{const plot} & \RDD{const plot mark right} & \RDD{const plot mark mid}
\\ \hline 
\end{tabular}


\begin{tabular}{|c|c|c|c|} \hline 
\begin{tikzpicture}[scale=.5]
\begin{axis}[jump mark left,blue]
\addplot file {table2.dat};
\end{axis}
\end{tikzpicture}
&
\begin{tikzpicture}[scale=.5]
\begin{axis}[jump mark right,blue]
\addplot  file {table2.dat};
\end{axis}
\end{tikzpicture}
&
\begin{tikzpicture}[scale=.5]
\begin{axis}[jump mark mid,blue]
\addplot  file {table2.dat};
\end{axis}
\end{tikzpicture}
\\ \hline 
\RDD{jump mark left} & \RDD{jump mark right} & \RDD{jump mark mid}
\\ \hline 
\end{tabular}

\begin{tabular}{|c|c|c|c|} \hline 
\begin{tikzpicture}[scale=.5]
\begin{axis}[xbar,blue]
\addplot coordinates {(1,0) (3,1) (2,2) (3,3) (2,4) (1,5)};
\end{axis}
\end{tikzpicture}
&
\begin{tikzpicture}[scale=.5]
\begin{axis}[ybar,blue]
\addplot  file {table2.dat};
\end{axis}
\end{tikzpicture}
&
\begin{tikzpicture}[scale=.5]
\begin{axis}[ybar interval,blue]
\addplot  file {table2.dat};
\end{axis}
\end{tikzpicture}
\\ \hline 
\RDD{xbar} & \RDD{ybar} & \RDD{ybar interval}
\\ \hline 
\end{tabular}


\begin{tabular}{|c|c|c|c|} \hline 
\begin{tikzpicture}[scale=.5]
\begin{axis}[xbar interval,blue]
\addplot  coordinates {(1,0) (3,1) (2,2) (3,3) (2,4) (1,5)};
\end{axis}
\end{tikzpicture}
&
\begin{tikzpicture}[scale=.5]
\begin{axis}[xcomb,blue]
\addplot  file {table2.dat};
\end{axis}
\end{tikzpicture}
&
\begin{tikzpicture}[scale=.5]
\begin{axis}[ycomb,blue]
\addplot  file {table2.dat};
\end{axis}
\end{tikzpicture}
\\ \hline 
\RDD{xbar interval} & \RDD{xcomb} & \RDD{ycomb}
\\ \hline 
\end{tabular}


\begin{tabular}{|c|c|c|c|} \hline 
\begin{tikzpicture}[scale=.5]
\begin{axis}[only marks,blue]
\addplot {x^2 - x +4};
\end{axis}
\end{tikzpicture}
&
\begin{tikzpicture}[scale=.5]
\begin{axis}
\addplot [scatter] {x^2 - x +4};
\end{axis}
\end{tikzpicture}
&

\\ \hline 
\RDD{only marks} & \RDD{scatter} & \RDD{mesh}
\\ \hline
\end{tabular}


\smallskip
\begin{tabular}{|c|c|c|c|} \hline 
\multicolumn{2}{|c|}{  \BS{addplot}  [\RDD{quiver}=\AC{u=1,v=2*x}],->,samples=5,blue,ultra thick]  \AC{x\^{}2 - x +4};   }
\\ \hline
\begin{tikzpicture}[scale=.5]

\begin{axis}
\addplot [red,dashed,no marks] {x^2 - x +4};
\addplot [quiver={u=1,v=2*x},->,samples=5,blue,ultra thick] {x^2 - x +4};
\end{axis}
\end{tikzpicture}
&
\begin{tikzpicture}[scale=.5,domain=0:360,ultra thick]
\begin{axis}
\addplot [red,dashed,no marks] (\x,{sin(\x)});
\addplot [quiver={u=180/3.14,v=cos(x)},->,samples=5,blue,ultra thick] (\x,{sin(\x)});
\end{axis}
\end{tikzpicture}
\\ \hline
\RDD{quiver}={u=1,v=2*x} & \RDD{quiver}=\AC{u=180/3.14,v=cos(x)}
\\ \hline
\multicolumn{2}{|c|}{ \dft :   u=0  et  v = 0}
\\ \hline
\end{tabular}




%\subsection{stack}
\smallskip
\begin{tabular}{|c|c|c|c|} \hline 
\begin{tikzpicture}[scale=.5]
\begin{axis}[stack plots=y,blue]
\addplot {x^2 - x +4};
\addplot {x^2 - x +4};
\end{axis}
\end{tikzpicture}
&

\begin{tikzpicture}[scale=.5]
\begin{axis}[stack plots=y,blue]
\addplot  file {table2.dat};
\addplot   file {table2.dat};
\end{axis}
\end{tikzpicture}
&

\begin{tikzpicture}[scale=.5]
\begin{axis}[ybar stacked,blue]
\addplot  file {table2.dat};
\addplot   file {table2.dat};
\end{axis}
\end{tikzpicture}
\\ \hline
[\RDD{stack plots}=y,blue] & [\RDD{stack plots=y},blue] & [\RDD{ybar stacked},blue]
\\ \hline
\end{tabular}

\bigskip
\begin{tabular}{|c|c|c|c|} \hline 
\begin{tikzpicture}[scale=.5]
\begin{axis}[stack plots=y,area style]
\addplot  file {table2.dat};
\addplot   file {table2.dat};
\end{axis}
\end{tikzpicture}
&
\begin{tikzpicture}[scale=.5]
\begin{axis}[const plot,stack plots=y,area style]
\addplot  file {table2.dat};
\addplot   file {table2.dat};
\end{axis}
\end{tikzpicture}
&
\begin{tikzpicture}[scale=.5]
\begin{axis}[stack plots=y,area style,smooth]
\addplot  file {table2.dat};
\addplot   file {table2.dat};
\end{axis}
\end{tikzpicture}
\\ \hline
[stack plots=y,area style] & [const plot,stack plots=y,area style] & [stack plots=y,area style,smooth]
\\ \hline

\end{tabular}


\bigskip

\begin{tabular}{|c|c|c|c|} \hline 
\multicolumn{3}{|c|}{  \BS{addplot}  [\RDD{error bars/y dir}=both,\RDD{error bars/y fixed} =2.5]  \AC{x\^{}2 - x +4};   }
\\ \hline

\begin{tikzpicture}[scale=.5]
\begin{axis}
\addplot [error bars/y dir =both,error bars/y fixed =2.5,blue]{x^2 - x +4};
\end{axis}
\end{tikzpicture}
&
\begin{tikzpicture}[scale=.5]
\begin{axis}
\addplot [error bars/y dir =plus,error bars/y fixed =2.5,blue]{x^2 - x +4};
\end{axis}
\end{tikzpicture}
&
\begin{tikzpicture}[scale=.5]
\begin{axis}
\addplot [error bars/y dir =minus,error bars/y fixed =2.5,blue]{x^2 - x +4};
\end{axis}
\end{tikzpicture}
\\ \hline
\RDD{error bars/y dir} =both & \RDD{error bars/y dir} =plus & \RDD{error bars/y dir} =minus
\\ \hline
\end{tabular}





\bigskip

\begin{tabular}{|c|c|c|c|} \hline 
\multicolumn{3}{|c|}{  \BS{addplot}  [\RDD{error bars/x dir}=both,\RDD{error bars/x fixed} =.5]  \AC{x\^{}2 - x +4};   }
\\ \hline
\begin{tikzpicture}[scale=.5]
\begin{axis}
\addplot [error bars/x dir =both,error bars/x fixed =.5,blue]{x^2 - x +4};
\end{axis}
\end{tikzpicture}
&
\begin{tikzpicture}[scale=.5]
\begin{axis}
\addplot [error bars/x dir =plus,error bars/x fixed =.5,blue]{x^2 - x +4};
\end{axis}
\end{tikzpicture}
&
\begin{tikzpicture}[scale=.5]
\begin{axis}
\addplot [error bars/x dir =minus,error bars/x fixed =.5,blue]{x^2 - x +4};
\end{axis}
\end{tikzpicture}
\\ \hline
\RDD{error bars/x dir} =both &\RDD{ error bars/x dir} =plus & \RDD{error bars/x dir} =minus
\\ \hline
\end{tabular}

\bigskip
\begin{tabular}{|c|c|c|c|} \hline 
\multicolumn{3}{|c|}{  \BS{addplot}  [\RDD{error bars/y dir}=both,\RDD{error bars/x fixed relative} =.2]  \AC{x\^{}2 - x +4};   }
\\ \hline

\begin{tikzpicture}[scale=.5]

\begin{axis}
\addplot [error bars/y dir =both,error bars/y fixed relative =.2,blue]{x^2 - x +4};
\end{axis}
\end{tikzpicture}
&
\begin{tikzpicture}[scale=.5]
\begin{axis}
\addplot [error bars/y dir =plus,error bars/y fixed relative =1,blue]{x^2 - x +4};
\end{axis}
\end{tikzpicture}
&
\begin{tikzpicture}[scale=.5]
\begin{axis}
\addplot [error bars/x dir =minus,error bars/x fixed relative =.2,blue]{x^2 - x +4};
\end{axis}
\end{tikzpicture}
\\ \hline
\RDD{error bars/y fixed relative} =.2 & \RDD{error bars/y fixed relative} =1 & \RDD{error bars/x fixed relative} =.2
\\ \hline
\end{tabular}


\SbSSCT{Habillage du graphe}{Graph information}

\SbSbSSCT{Titres}{Titles}

\begin{tabular}{|c|c|c|c|} \hline 
\begin{tikzpicture}[scale=.5]
\begin{axis}[xlabel=axe X,blue]
...
\end{axis}
\end{tikzpicture}
&
\begin{tikzpicture}[scale=.5]
\begin{axis}[ylabel=axe Y,blue]
...
\end{axis}
\end{tikzpicture}
&
\begin{tikzpicture}[scale=.5]
\begin{axis}[title=Titre du graphe,blue]
 
\end{axis}
\end{tikzpicture}


\\ \hline 
\BS{begin}\AC{axis}[\RDD{xlabel}=axe X] & \BS{begin}\AC{axis}[\RDD{ylabel}=axe Y] & \BS{begin}\AC{axis}[\RDD{title}=Titre du graphe]
\\ \hline 
\end{tabular}

\SbSbSSCT{Légende}{Legend}

\begin{tabular}{|c|c|c|c|} \hline 
\begin{tikzpicture}[blue ,baseline=0pt,scale=.5]
\begin{axis}
\addplot {x^2 - x +4};
\addplot {x^2 - x +2};
\addplot {x^2 - x };
\addplot {x^2 - x -2 };
\addplot {x^2 - x -4 };
\legend{$x^2 - x +4$,$x^2 - x +2$,$x^2 - x $,$x^2 - x -2 $,$x^2 - x -4 $}
\end{axis}
\end{tikzpicture}
&
\parbox[c]{10cm}{
\BS{begin}\AC{axis}\\
\BSS{addplot} \AC{x\^{}2 - x +4};\\
\BSS{addplot} \AC{x\^{}2 - x +2};\\
\BSS{addplot} \AC{x\^{}2 - x };\\
\BSS{addplot} \AC{x\^{}2 - x -2 };\\
\BSS{addplot} \AC{x\^{}2 - x -4 };\\

\BSS{legend}\AC{\$x\^{}2 - x +4\$,\$x\^{}2 - x +2\$,\$x\^{}2 - x \$,\$x\^{}2 - x -2 \$,\$x\^{}2 - x -4 \$}\\
\BS{end}\AC{axis}}
\\ \hline 
 
\begin{tikzpicture}[blue ,baseline=0pt,scale=.5]
\begin{axis}[legend entries={$x^2 - x +4$,$x^2 - x +2$,$x^2 - x $,$x^2 - x -2 $,$x^2 - x -4 $}]
\addplot {x^2 - x +4};
\addplot {x^2 - x +2};
\addplot {x^2 - x };
\addplot {x^2 - x -2 };
\addplot {x^2 - x -4 };

\end{axis}
\end{tikzpicture}
&
\parbox[c]{10cm}{
\BS{begin}\AC{axis}[\RDD{legend entries}= \AC{\$ x\^{}2 - x +4 \$,\$ x\^{}2 - x +2 \$,\$ x\^{}2 - x \$,\$ x\^{}2 - x -2 \$,\$ x\^{}2 - x -4 \$} ] \\

\BSS{addplot} \AC{x\^{}2 - x +4};\\
\BSS{addplot} \AC{x\^{}2 - x +2};\\
\BSS{addplot} \AC{x\^{}2 - x };\\
\BSS{addplot} \AC{x\^{}2 - x -2 };\\
\BSS{addplot} \AC{x\^{}2 - x -4 };\\
\BS{end}\AC{axis}}
\\ \hline 
\end{tabular}


Options

\begin{tabular}{|c|c|c|c|} \hline 
\begin{tikzpicture}[blue ,baseline=0pt,scale=.5]
\begin{axis}[legend entries={$x^2 - x +4$,$x^2 - x +2$,$x^2 - x $,$x^2 - x -2 $,$x^2 - x -4 $},legend style={font=\tiny}]
\addplot {x^2 - x +4};
\addplot {x^2 - x +2};
\addplot {x^2 - x };
\addplot {x^2 - x -2 };
\addplot {x^2 - x -4 };
\end{axis}
\end{tikzpicture}
&
\begin{tikzpicture}[blue ,baseline=0pt,scale=.5]
\begin{axis}[legend entries={$x^2 - x +4$,$x^2 - x +2$,$x^2 - x $,$x^2 - x -2 $,$x^2 - x -4 $},legend style={draw=none}]
\addplot {x^2 - x +4};
\addplot {x^2 - x +2};
\addplot {x^2 - x };
\addplot {x^2 - x -2 };
\addplot {x^2 - x -4 };
\end{axis}
\end{tikzpicture}
&
\begin{tikzpicture}[blue ,baseline=0pt,scale=.5]
\begin{axis}[legend entries={$x^2 - x +4$,$x^2 - x +2$},legend style={shape=ellipse}]
\addplot {x^2 - x +4};
\addplot {x^2 - x +2};
\end{axis}
\end{tikzpicture}
\\ \hline 
\RDD{legend style}=\AC{\RDD{font}=\BS{tiny}} & \RDD{legend style}=\AC{\RDD{draw}=none} & \RDD{legend style}=\AC{\RDD{shape}=ellipse} 
\\ \hline 
\end{tabular}




\bigskip
\begin{tabular}{|c|c|c|c|} \hline 
\begin{tikzpicture}[blue ,baseline=0pt,scale=.5]
\begin{axis}[legend style={at={(.5,.5)}}]
\addplot {x^2 - x +4};
\addplot {x^2 - x +2};
\addplot {x^2 - x };
\addplot {x^2 - x -2 };
\addplot {x^2 - x -4 };
\legend{$x^2 - x +4$,$x^2 - x +2$,$x^2 - x $,$x^2 - x -2 $,$x^2 - x -4 $}
\end{axis}
\end{tikzpicture}
&
\begin{tikzpicture}[blue ,baseline=0pt,scale=.5]
\begin{axis}[legend style={legend pos=outer north east}]
\addplot {x^2 - x +4};
\addplot {x^2 - x +2};
\addplot {x^2 - x };
\addplot {x^2 - x -2 };
\addplot {x^2 - x -4 };
\legend{$x^2 - x +4$,$x^2 - x +2$,$x^2 - x $,$x^2 - x -2 $,$x^2 - x -4 $}
\end{axis}
\end{tikzpicture}
\\ \hline 
legend style=\AC{\RDD{at}=\AC{(.5,.5)}} & legend style=\AC{\RDD{legend pos}=outer north east}
\\ \hline 
\end{tabular}

\bigskip
\begin{tabular}{|c|c|c|c|} \hline 
\begin{tikzpicture}[blue ,baseline=0pt,scale=.5]
\begin{axis}[legend style={legend columns=2}]
\addplot {x^2 - x +4};
\addplot {x^2 - x +2};
\addplot {x^2 - x };
\addplot {x^2 - x -2 };
\addplot {x^2 - x -4 };
\legend{A,B,C,D,E}
\end{axis}
\end{tikzpicture}
&
\begin{tikzpicture}[blue ,baseline=0pt,scale=.5]
\begin{axis}[legend style={legend columns=3}]
\addplot {x^2 - x +4};
\addplot {x^2 - x +2};
\addplot {x^2 - x };
\addplot {x^2 - x -2 };
\addplot {x^2 - x -4 };
\legend{A,B,C,D,E}
\end{axis}
\end{tikzpicture}
&
\begin{tikzpicture}[blue ,baseline=0pt,scale=.5]
\begin{axis}[legend style={legend columns=-1}]
\addplot {x^2 - x +4};
\addplot {x^2 - x +2};
\addplot {x^2 - x };
\addplot {x^2 - x -2 };
\addplot {x^2 - x -4 };
\legend{A,B,C,D,E}
\end{axis}
\end{tikzpicture}

\\ \hline 
legend style=\AC{\RDD{legend columns}=2} & legend style=\AC{\RDD{legend columns}=3}  & legend style=\AC{\RDD{legend columns}=-1}
\\ \hline 
\end{tabular}

\bigskip

\begin{tabular}{|c|c|c|c|} \hline 
\begin{tikzpicture}[blue ,baseline=0pt,scale=.5]
\begin{axis}[legend cell align=left]
\addplot {x^2 - x +4};
\addplot {x^2 - x +2};
\addplot {x^2 - x };
\addplot {x^2 - x -2 };
\addplot {x^2 - x -4 };
\legend{$x^2 - x +4$,f(x),$x^2 - x $,courbe,Y}
\end{axis}
\end{tikzpicture}
&
\begin{tikzpicture}[blue ,baseline=0pt,scale=.5]
\begin{axis}[legend cell align=center,]
\addplot {x^2 - x +4};
\addplot {x^2 - x +2};
\addplot {x^2 - x };
\addplot {x^2 - x -2 };
\addplot {x^2 - x -4 };
\legend{$x^2 - x +4$,f(x),$x^2 - x $,courbe,Y}
\end{axis}
\end{tikzpicture}
&
\begin{tikzpicture}[blue ,baseline=0pt,scale=.5]
\begin{axis}[legend cell align=right]
\addplot {x^2 - x +4};
\addplot {x^2 - x +2};
\addplot {x^2 - x };
\addplot {x^2 - x -2 };
\addplot {x^2 - x -4 };
\legend{$x^2 - x +4$,f(x),$x^2 - x $,courbe,Y}
\end{axis}
\end{tikzpicture}
\\ \hline 
[\RDD{legend cell align}=left] & [\RDD{legend cell align}=center] & [\RDD{legend cell align}=right]
\\ \hline 
\end{tabular}

\SbSbSSCT{Taille du graphe}{Size of the graph}

\begin{tabular}{|c|c|c|c|} \hline 
\begin{tikzpicture}
\begin{axis}[width=3cm,grid=major,blue]
 \addplot {x^2 - x +4};
\end{axis}
\end{tikzpicture}
&
\begin{tikzpicture}
\begin{axis}[height=5cm,grid=major,blue]
 \addplot {x^2 - x +4};
\end{axis}
\end{tikzpicture}
&
\begin{tikzpicture}
\begin{axis}[width=3cm,height=5cm,grid=major,blue]
 \addplot {x^2 - x +4};
\end{axis}
\end{tikzpicture}

\\ \hline
\RDD{width}=3cm & \RDD{height}=5cm & \RDD{width}=3cm,\RDD{height}=5cm
\\ \hline 


\end{tabular}

\SbSbSSCT{Quadrillage}{Grids}

\begin{tabular}{|c|c|c|c|} \hline 
\begin{tikzpicture}[scale=.5]
\begin{axis}[xmajorgrids=true,blue]
\addplot {x^2 - x +4};
\end{axis}
\end{tikzpicture}
&
\begin{tikzpicture}[scale=.5]
\begin{axis}[ymajorgrids=true,blue]
\addplot {x^2 - x +4};
\end{axis}
\end{tikzpicture}
&
\begin{tikzpicture}[scale=.5]
\begin{axis}[grid=major,blue]
\addplot {x^2 - x +4};
\end{axis}
\end{tikzpicture}
\\ \hline 
\BS{begin}\AC{axis}[\RDD{xmajorgrids}=true] &\BS{begin}\AC{axis}[\RDD{ymajorgrids}=true] & \BS{begin}\AC{axis}[\RDD{grid}=major] 
\\ \hline 
\end{tabular}

%\subsubsection{minorgrids}
%
%\begin{tabular}{|c|c|c|c|} \hline 
%\begin{tikzpicture}[scale=.4]
%\begin{loglogaxis}[grid=major,blue]
%\addplot {x^2 - x +4};
%\end{loglogaxis}
%\end{tikzpicture}
%&
%\begin{tikzpicture}[scale=.4]
%\begin{loglogaxis}[grid=both,blue]
%\addplot {x^2 - x +4};
%\end{loglogaxis}
%\end{tikzpicture}
%\\ \hline 
%\end{tabular}

\begin{tabular}{|l|l|c|c|} \hline 
\begin{tikzpicture}
\begin{axis}[nodes near coords,blue,samples=10]
\addplot {x^2 - x +4};
\end{axis}
\end{tikzpicture}
 &
 \begin{tikzpicture}
 \begin{axis}[nodes near coords,blue]
 \addplot file {table2.dat};
 \end{axis}
 \end{tikzpicture}
 \\ \hline 
 \BS{begin}AC{axis}[\RDD{nodes near coords},samples=10] &  \BS{begin}AC{axis}[\RDD{nodes near coords}]\\
 \BS{addplot} \AC{x\^{} 2- x +4}; &   \BS{addplot}  file {table2.dat}; 
 
 \\ \hline  
 \end{tabular}
 

\newpage



\SSCT{Courbes 3D}{3D graph}

\input{tkzgraph3D} % très lourd à compiler

%
%%%
%%%\essais{pstgraph2ess.tex}
%%\newpage
%%\section[Créer un graphe d'après une équation]{Créer un graphe d'après une équation  \cite{pst-user} \cite{pst-plot}}
%%
%
%%%
%%%\essais{pstgraph3ess.tex} 
%%\newpage
%% \section[Des outils pour les graphes]{Des outils pour les graphes \cite{pst-add} }
%% 
%
%%
%%\newpage
%% \section[Créer un graphe en camembert]{Créer un graphe en camembert \cite{pst-add} }
%% 
%%\input{chart} % camembert

\newpage

\SSCT{Les Tableaux de variation}{Table of a function variation }

\label{tabl}
%Insérer dans le préambule :

 \maboite{\BS{usepackage}\AC{tkz-tab}  \cite {tikstab}}


%\subsection{Déclaration du tableau}
\SbSSCT{Déclaration du tableau}{Creation of the table}

\begin{tabular}{|l|c|}\hline 
\begin{tikzpicture}
\tkzTabInit{1° ligne / 1 ,2° ligne /1 }{ a , b, c }
%\tkzTabLine{ 1, 2, 3 , 4,5 }
\end{tikzpicture}
\\ \hline 
\BS{begin}\AC{tikzpicture} \\
\BSS{tkzTabInit}\AC{1° ligne / 1 ,2° ligne /1 } \AC{ a , b, c } \\
% \BSS{tkzTabLine}\AC{ 1, 2, 3 , 4,5 } \\
\BS{end}\AC{tikzpicture}
 \\ \hline 
 \end{tabular} 

\subsubsection{Options}
 
%\paragraph{Hauteur des lignes}:

\begin{tabular}{|l|c|}\hline  
 \multicolumn{1}{|c|}{\textbf{\TFRGB{Hauteur des ligne}{Row width }} }
 \\ \hline

\begin{tikzpicture} \tkzTabInit{1° ligne /1  , 2° ligne /.5  , 3° ligne /1.5 }{a , b , c }\end{tikzpicture}
\\ \hline 
\BS{tikz}  \BSS{tkzTabInit}\AC{1° ligne {'\color{red}  /1}  , 2° ligne {\color{red}   /.5}  , 3° ligne {\color{red}  /1.5} }\AC{a , b , c };
\\ \hline 
\end{tabular} 
 
\bigskip
%\paragraph{Largeur de la première colonne } :

\begin{tabular}{|l|c|}\hline 
 \multicolumn{1}{|c|}{\textbf{\TFRGB{Largeur de la première colonne  }{First column width }} }
 \\ \hline
\begin{tikzpicture} 
\tkzTabInit[lgt=4]{ $x$ / 1}{ a , b , c  }
\end{tikzpicture}
\\ \hline 
\BS{tkzTabInit}[\RDD{lgt}=4]\AC{ $x$ / 1}\AC{ a , b , c  }; \\
\dft :  lgt==2 cm 
\\ \hline 
\end{tabular} 

\bigskip

%\paragraph{Espacement entre deux valeurs} :
%\Par{Espacement entre deux valeurs}{Space between two values} :


\begin{tabular}{|l|c|}\hline 
 \multicolumn{1}{|c|}{\textbf{\TFRGB{Espacement entre deux valeurs}{Space between two values}} }
 \\ \hline

\begin{tikzpicture} 
\tkzTabInit[espcl=2]{ $x$ / 1}{ a , b , c  }
\end{tikzpicture}
\\ \hline 
\BS{tkzTabInit}[\RDD{espcl}=1]\AC{ $x$ / 1}\AC{ a , b , c  }; \\
\dft :  espcl=2 cm
\\ \hline 
\end{tabular}


\bigskip
%\paragraph{Marge de début et de fin} :

\begin{tabular}{|l|c|}\hline 
 \multicolumn{1}{|c|}{\textbf{\TFRGB{Marge de début et de fin  }{Margin  }} }
 \\ \hline
\begin{tikzpicture} 
\tkzTabInit[deltacl=2]{ $x$ / 1}{ a , b , c  }
\end{tikzpicture} 
\\ \hline 
\BS{tkzTabInit}[\RDD{deltacl}=1]\AC{ $x$ / 1}\AC{ a , b , c  }; \\
\dft :  deltacl=0.5 cm
\\ \hline 
\end{tabular}



\newpage
%\paragraph{\'Epaisseur des lignes du tableau } : 

\begin{tabular}{|l|c|}\hline 
 \multicolumn{1}{|c|}{\textbf{\TFRGB{\'Epaisseur des lignes du tableau }{Line width }} }
 \\ \hline
\begin{tikzpicture} 
\tkzTabInit[lw=2pt]{ $x$ / 1}{ a , b , c  }
\end{tikzpicture} 
\\ \hline 
\BS{tkzTabInit}[\RDD{dlw}=2pt]\AC{ $x$ / 1}\AC{ a , b , c  }; \\
\dft :  lw=0,4 pt
\\ \hline 
\end{tabular}



\bigskip
%\paragraph{Absence de cadre} :

\begin{tabular}{|l|c|}\hline
 \multicolumn{1}{|c|}{\textbf{\TFRGB{Absence de cadre}{No cadre}} }
 \\ \hline 
\begin{tikzpicture} 
\tkzTabInit[nocadre]{ $x$ / 1}{ a , b , c  }
\end{tikzpicture} 
\\ \hline 
\BS{tkzTabInit}[nocadre]\AC{ $x$ / 1}\AC{ a , b , c  }; \\
\dft :  nocadre=false
\\ \hline 
\end{tabular}


\bigskip
%\paragraph{Mise en couleur}:\\
\begin{tabular}{|c|c|}\hline
 \multicolumn{2}{|c|}{\textbf{\TFRGB{Mise en couleur  }{ Coloring }} }
 \\ \hline 
\multicolumn{2}{|c|}{ \BS{tkzTabInit} [\RDD{color},\RDD{colorT} = yellow]\AC{1°ligne/1 , 2°ligne/1}\AC{ a , b  }   }\\ 
\hline
\begin{tikzpicture}
\tkzTabInit[color,colorT = yellow]{ 1°ligne/1 , 2°ligne/1}{ a , b   }
\end{tikzpicture}
 &
\begin{tikzpicture}
\tkzTabInit[color,colorC = cyan]{ 1°ligne/1 , 2°ligne/1}{ a , b }
\end{tikzpicture}
\\ \hline
[color,\RDD{colorT} = yellow] & [color,\RDD{colorC} = cyan]
\\ \hline
\begin{tikzpicture}
\tkzTabInit[color,colorL = green]{1°ligne/1 , 2°ligne/1}{ a , b  }
\end{tikzpicture}
&
\begin{tikzpicture}
\tkzTabInit[color,colorV = magenta]{1°ligne/1 , 2°ligne/1}{ a , b  }
\end{tikzpicture}
\\ \hline 
[color,\RDD{colorL} = green] & [color,\RDD{colorV} = magenta]
\\ \hline 
\multicolumn{2}{|c|}{ \dft : color = false \hspace{1cm}  colorT=colorC=colorL=colorV =white   }
\\ \hline 
\end{tabular} 




%\subsection{Création d'une ligne de signes}
\SbSSCT{Création d'une ligne de signes}{Creation of a sign row}

\begin{tabular}{|c|c|}\hline  
\begin{tikzpicture}
\tkzTabInit[espcl=1.5]
{$x$ / 1 ,$f(x)$ /1 }%
{ a , b, c  }
\tkzTabLine{ t, 2, t ,4 ,t }
\end{tikzpicture}
&  
\begin{tikzpicture}
\tkzTabInit[espcl=1.5]
{$x$ / 1 ,$f(x)$ /1 }%
{ a , b, c  }
\tkzTabLine{ z, 2, z ,4 ,z }
\end{tikzpicture}
\\ \hline  
\BSS{tkzTabLine}\AC{ {\color{red}  t}, 2,{\color{red}  t} ,4 ,{\color{red}  t} } & \BSS{tkzTabLine}\AC{ {\color{red}  z}, 2, {\color{red}  z} ,4 ,{\color{red}  z} } 
\\ \hline  
\begin{tikzpicture}
\tkzTabInit[espcl=1.5]
{$x$ / 1 ,$f(x)$ /1 }%
{ a , b, c  }
\tkzTabLine{ d, 2, d ,4 ,d }
\end{tikzpicture}
&  
\begin{tikzpicture}
\tkzTabInit[espcl=1.5]
{$x$ / 1 ,$f(x)$ /1 }%
{ a , b, c  }
\tkzTabLine{ 1,h, 3,4 ,5}
\end{tikzpicture}
\\ \hline
\BSS{tkzTabLine}\AC{ {\color{red}  d}, 2, {\color{red}  d} ,4 ,{\color{red}  d} } & \BSS{tkzTabLine}\AC{ 1, {\color{red}  h}, 3 ,4 ,5 } 
\\ \hline 
\end{tabular} 


\newpage
%\paragraph{Exemple}:

\begin{tabular}{|l|c|}\hline 
 \multicolumn{1}{|c|}{\textbf{\TFRGB{Exemple }{Example }} }
 \\ \hline
\begin{tikzpicture}
\tkzTabInit[espcl=1.5]{$x$ / 1 ,$f(x)$ /1 }%
{ $-\infty$ , -4, 4 , 10 , $+\infty$ }
\tkzTabLine{ t,+, d ,h ,d,-,z,+ }
\end{tikzpicture}
\\ \hline 
\BS{begin}\AC{tikzpicture} \\
\BS{tkzTabInit}[espcl=1.5]\AC{\$x\$ / 1 ,\$f(x)\$ /1 } %\\
\AC{ $-\infty$ , -4, 4 , 10 , $+\infty$ } \\
\BS{tkzTabLine}\AC{ t,+, d ,h ,d,-,z,+ } \\
\BS{end}\AC{tikzpicture}
\\ \hline 
\end{tabular}

%\subsection{Création d'une ligne de variations}
\SbSSCT{Création d'une ligne de variations}{Creation of a variation row}

\begin{tabular}{|c|c|}\hline  
\begin{tikzpicture}
\tkzTabInit[espcl=1.5]
{$x$ / 1 ,$f(x)$ /1 }%
{ a , b, c  }
%\tkzTabLine{ , t, , ,t }
\tkzTabVar{+/1 , -/2}
\end{tikzpicture}
&  
\begin{tikzpicture}
\tkzTabInit[espcl=1.5]
{$x$ / 1 ,$f(x)$ /1 }%
{ a , b, c  }
%\tkzTabLine{ 1, z, 3 ,4 ,z }
\tkzTabVar{-/1 , +/2}
\end{tikzpicture}
\\ \hline  
\BSS{tkzTabVar}\AC{ {\color{red}  +/}1 , {\color{red}  -/}2} & \BSS{tkzTabVar}\AC{ {\color{red}  -/}1 , {\color{red}  +/}2} 
%\tkzTabVar{+/1 , +/2}
\\ \hline  
\begin{tikzpicture}
\tkzTabInit[espcl=1.5]
{$x$ / 1 ,$f(x)$ /1 }%
{ a , b, c  }
%\tkzTabLine{ 1, d, 3 ,4 ,d }
\tkzTabVar{-/1 , -/2}
\end{tikzpicture}
&  
\begin{tikzpicture}
\tkzTabInit[espcl=1.5]
{$x$ / 1 ,$f(x)$ /1 }%
{ a , b, c  }
%\tkzTabLine{ 1,h, 3,4 ,h }
\tkzTabVar{+/1 , +/2}
\end{tikzpicture}
\\ \hline
\BSS{tkzTabVar}\AC{{\color{red}  -/}1 , {\color{red}  -/}2} & \BSS{tkzTabVar}\AC{ {\color{red}  +/}1 , {\color{red}  +/}2 } 
\\ \hline 
\end{tabular}

\bigskip

\begin{tabular}{|c|c|}\hline  
\begin{tikzpicture}
\tkzTabInit[espcl=1.5]
{$x$ / 1 ,$f(x)$ /1 }%
{ a , b, c  }
%\tkzTabLine{ , t, , ,t }
\tkzTabVar{+C/1 , -/2}
\end{tikzpicture}
&  
\begin{tikzpicture}
\tkzTabInit[espcl=1.5]
{$x$ / 1 ,$f(x)$ /1 }%
{ a , b, c  }
%\tkzTabLine{ 1, z, 3 ,4 ,z }
\tkzTabVar{-C/1 , +/2}
\end{tikzpicture}
\\ \hline  
\BSS{tkzTabVar}\AC{ {\color{red}  +C/}1 , -/2} & \BSS{tkzTabVar}\AC{ {\color{red}  -C/}1 , +/2} 
%\tkzTabVar{+/1 , +/2}
\\ \hline  
\begin{tikzpicture}
\tkzTabInit[espcl=1.5]
{$x$ / 1 ,$f(x)$ /1 }%
{ a , b, c  }
%\tkzTabLine{ 1, d, 3 ,4 ,d }
\tkzTabVar{+/1 , -C/2}
\end{tikzpicture}
&  
\begin{tikzpicture}
\tkzTabInit[espcl=1.5]
{$x$ / 1 ,$f(x)$ /1 }%
{ a , b, c  }
%\tkzTabLine{ 1,h, 3,4 ,h }
\tkzTabVar{-/1 , +C/2}
\end{tikzpicture}
\\ \hline
\BSS{tkzTabVar}\AC{-/1 , {\color{red}  -C/}2} & \BSS{tkzTabVar}\AC{ +/1 , {\color{red}  +C/}2 } 
\\ \hline 
\end{tabular}


\bigskip

\begin{tabular}{|c|c|}\hline  
\begin{tikzpicture}
\tkzTabInit[espcl=1.5]
{$x$ / 1 ,$f(x)$ /1 }%
{ a , b, c  }
%\tkzTabLine{ , t, , ,t }
\tkzTabVar{+H/1 , -/2}
\end{tikzpicture}
&  
\begin{tikzpicture}
\tkzTabInit[espcl=1.5]
{$x$ / 1 ,$f(x)$ /1 }%
{ a , b, c  }
%\tkzTabLine{ 1, z, 3 ,4 ,z }
\tkzTabVar{-H/1 , +/2}
\end{tikzpicture}
\\ \hline  
\BSS{tkzTabVar}\AC{ {\color{red}  +H/1} , -/2} & \BSS{tkzTabVar}\AC{ {\color{red}  -H/}1 , +/2} 
%\tkzTabVar{+/1 , +/2}
\\ \hline  
\begin{tikzpicture}
\tkzTabInit[espcl=1.5]
{$x$ / 1 ,$f(x)$ /1 }%
{ a , b, c  }
%\tkzTabLine{ 1, d, 3 ,4 ,d }
\tkzTabVar{+/1 , -H/2}
\end{tikzpicture}
&  
\begin{tikzpicture}
\tkzTabInit[espcl=1.5]
{$x$ / 1 ,$f(x)$ /1 }%
{ a , b, c  }
%\tkzTabLine{ 1,h, 3,4 ,h }
\tkzTabVar{-/1 , +H/2}
\end{tikzpicture}
\\ \hline
\BSS{tkzTabVar}\AC{-/1 , {\color{red}  -H/}2} & \BSS{tkzTabVar}\AC{ +/1 , {\color{red}  +H/}2 } 
\\ \hline 
\end{tabular}

\bigskip

\begin{tabular}{|c|c|}\hline  
\begin{tikzpicture}
\tkzTabInit[espcl=1.5]
{$x$ / 1 ,$f(x)$ /1 }%
{ a , b, c  }
%\tkzTabLine{ , t, , ,t }
\tkzTabVar{+D/1 , -/2}
\end{tikzpicture}
&  
\begin{tikzpicture}
\tkzTabInit[espcl=1.5]
{$x$ / 1 ,$f(x)$ /1 }%
{ a , b, c  }
%\tkzTabLine{ 1, z, 3 ,4 ,z }
\tkzTabVar{-D/1 , +/2}
\end{tikzpicture}
\\ \hline  
\BSS{tkzTabVar}\AC{ {\color{red}  +D/}1 , -/2} & \BSS{tkzTabVar}\AC{ {\color{red}  -D/}1 , +/2} 
%\tkzTabVar{+/1 , +/2}
\\ \hline  
\begin{tikzpicture}
\tkzTabInit[espcl=1.5]
{$x$ / 1 ,$f(x)$ /1 }%
{ a , b, c  }
%\tkzTabLine{ 1, d, 3 ,4 ,d }
\tkzTabVar{+/1 , -D/2}
\end{tikzpicture}
&  
\begin{tikzpicture}
\tkzTabInit[espcl=1.5]
{$x$ / 1 ,$f(x)$ /1 }%
{ a , b, c  }
%\tkzTabLine{ 1,h, 3,4 ,h }
\tkzTabVar{-/1 , +D/2}
\end{tikzpicture}
\\ \hline
\BSS{tkzTabVar}\AC{-/1 , {\color{red}  -D/}2} & \BSS{tkzTabVar}\AC{ +/1 , {\color{red}  +D/}2 } 
\\ \hline 
\end{tabular}

\bigskip

\begin{tabular}{|c|c|}\hline  
\begin{tikzpicture}
\tkzTabInit[espcl=1.5]
{$x$ / 1 ,$f(x)$ /1 }%
{ a , b, c  }
%\tkzTabLine{ , t, , ,t }
\tkzTabVar{D+/1 , -/2}
\end{tikzpicture}
&  
\begin{tikzpicture}
\tkzTabInit[espcl=1.5]
{$x$ / 1 ,$f(x)$ /1 }%
{ a , b, c  }
%\tkzTabLine{ 1, z, 3 ,4 ,z }
\tkzTabVar{D-/1 , +/2}
\end{tikzpicture}
\\ \hline  
\BSS{tkzTabVar}\AC{ {\color{red}  D+/}1 , -/2} & \BSS{tkzTabVar}\AC{{\color{red}  D-/}1 , +/2} 
%\tkzTabVar{+/1 , +/2}
\\ \hline  
\begin{tikzpicture}
\tkzTabInit[espcl=1.5]
{$x$ / 1 ,$f(x)$ /1 }%
{ a , b, c  }
%\tkzTabLine{ 1, d, 3 ,4 ,d }
\tkzTabVar{+/1 , D-/2}
\end{tikzpicture}
&  
\begin{tikzpicture}
\tkzTabInit[espcl=1.5]
{$x$ / 1 ,$f(x)$ /1 }%
{ a , b, c  }
%\tkzTabLine{ 1,h, 3,4 ,h }
\tkzTabVar{-/1 , D+/2}
\end{tikzpicture}
\\ \hline
\BSS{tkzTabVar}\AC{-/1 , {\color{red} D-/}2} & \BSS{tkzTabVar}\AC{ +/1 , {\color{red}  D+/}2 } 
\\ \hline 
\end{tabular}

\bigskip

\begin{tabular}{|c|c|}\hline  
\begin{tikzpicture}
\tkzTabInit[espcl=1.5]
{$x$ / 1 ,$f(x)$ /1 }%
{ a , b, c  }
%\tkzTabLine{ , t, , ,t }
\tkzTabVar{+DH/1 , -/2}
\end{tikzpicture}
&  
\begin{tikzpicture}
\tkzTabInit[espcl=1.5]
{$x$ / 1 ,$f(x)$ /1 }%
{ a , b, c  }
%\tkzTabLine{ 1, z, 3 ,4 ,z }
\tkzTabVar{-DH/1 , +/2}
\end{tikzpicture}
\\ \hline  
\BSS{tkzTabVar}\AC{ {\color{red}  +DH/}1 , -/2} & \BSS{tkzTabVar}\AC{ {\color{red}  -DH/}1 , +/2} 
%\tkzTabVar{+/1 , +/2}
\\ \hline  
\begin{tikzpicture}
\tkzTabInit[espcl=1.5]
{$x$ / 1 ,$f(x)$ /1 }%
{ a , b, c  }
%\tkzTabLine{ 1, d, 3 ,4 ,d }
\tkzTabVar{+/1 , -DH/2}
\end{tikzpicture}
&  
\begin{tikzpicture}
\tkzTabInit[espcl=1.5]
{$x$ / 1 ,$f(x)$ /1 }%
{ a , b, c  }
%\tkzTabLine{ 1,h, 3,4 ,h }
\tkzTabVar{-/1 , +DH/2}
\end{tikzpicture}
\\ \hline
\BSS{tkzTabVar}\AC{-/1 , {\color{red}  -DH/}2} & \BSS{tkzTabVar}\AC{ {\color{red}  +DH/}1 , +/2 } 
\\ \hline 
\end{tabular}

\bigskip

\begin{tabular}{|c|c|}\hline  
\begin{tikzpicture}
\tkzTabInit[espcl=1.5]{$x$ / 1 ,$f(x)$ /1 }{ a , b, c  }
%\tkzTabLine{ , t, , ,t }
\tkzTabVar{+CH/1 , -/2}
\end{tikzpicture}
&  
\begin{tikzpicture}
\tkzTabInit[espcl=1.5]{$x$ / 1 ,$f(x)$ /1 }{ a , b, c  }
%\tkzTabLine{ 1, z, 3 ,4 ,z }
\tkzTabVar{-CH/1 , +/2}
\end{tikzpicture}
\\ \hline  
\BSS{tkzTabVar}\AC{ {\color{red}  +CH/}1 , -/2} & \BSS{tkzTabVar}\AC{ {\color{red}  -CH/}1 , +/2} 
%\tkzTabVar{+/1 , +/2}
\\ \hline  
\begin{tikzpicture}
\tkzTabInit[espcl=1.5]{$x$ / 1 ,$f(x)$ /1 }{ a , b, c  }
\tkzTabVar{+/1 , -CH/2}
\end{tikzpicture}
&  
\begin{tikzpicture}
\tkzTabInit[espcl=1.5]{$x$ / 1 ,$f(x)$ /1 }{ a , b, c  }
\tkzTabVar{-/1 , +CH/2}
\end{tikzpicture}
\\ \hline
\BSS{tkzTabVar}\AC{-/1 , {\color{red}  -CH/}2} & \BSS{tkzTabVar}\AC{ +/1 , {\color{red}  +CH/}2 } 
\\ \hline 
\end{tabular}

\bigskip

\begin{tabular}{|c|c|}\hline  
\begin{tikzpicture}
\tkzTabInit[espcl=1.5]{$x$ / 1 ,$f(x)$ /1 }{ a , b, c  }
\tkzTabVar{-/1 , +D-/2 , +/3}
\end{tikzpicture}
&  
\begin{tikzpicture}
\tkzTabInit[espcl=1.5]{$x$ / 1 ,$f(x)$ /1 }{ a , b, c  }
\tkzTabVar{+/1 , -D+/2 , -/3}
\end{tikzpicture}
\\ \hline  
\BSS{tkzTabVar}\AC{ -/1 , {\color{red}  +D-/}2 , +/3} & \BSS{tkzTabVar}\AC{ +/1 , {\color{red}  -D+/}2 , -/3} 
\\ \hline  
\begin{tikzpicture}
\tkzTabInit[espcl=1.5]{$x$ / 1 ,$f(x)$ /1 }{ a , b, c  }
\tkzTabVar{+/1 , -D-/2 , +/3}
\end{tikzpicture}
&  
\begin{tikzpicture}
\tkzTabInit[espcl=1.5]{$x$ / 1 ,$f(x)$ /1 }{ a , b, c  }
\tkzTabVar{-/1 , +D+/2 , -/3}
\end{tikzpicture}
\\ \hline
\BSS{tkzTabVar}\AC{+/1 , {\color{red}  -D-/}2 , +/3} & \BSS{tkzTabVar}\AC{-/1 , {\color{red}  +D+/}2 , -/3 } 
\\ \hline 
\end{tabular}

\bigskip

\begin{tabular}{|c|c|}\hline  
\begin{tikzpicture}
\tkzTabInit[espcl=1.5]{$x$ / 1 ,$f(x)$ /1 }{ a , b, c  }
\tkzTabVar{-/1 , +CD-/2 , +/3}
\end{tikzpicture}
&  
\begin{tikzpicture}
\tkzTabInit[espcl=1.5]{$x$ / 1 ,$f(x)$ /1 }{ a , b, c  }
\tkzTabVar{+/1 , -CD+/2 , -/3}
\end{tikzpicture}
\\ \hline  
\BSS{tkzTabVar}\AC{ -/1 , {\color{red}  +CD-/}2 , +/3} & \BSS{tkzTabVar}\AC{ +/1 , {\color{red}  -CD+/}2 , -/3} 
\\ \hline  
\begin{tikzpicture}
\tkzTabInit[espcl=1.5]{$x$ / 1 ,$f(x)$ /1 }{ a , b, c  }
\tkzTabVar{+/1 , -CD-/2 , +/3}
\end{tikzpicture}
&  
\begin{tikzpicture}
\tkzTabInit[espcl=1.5]{$x$ / 1 ,$f(x)$ /1 }{ a , b, c  }
\tkzTabVar{-/1 , +CD+/2 , -/3}
\end{tikzpicture}
\\ \hline
\BSS{tkzTabVar}\AC{+/1 , {\color{red}  -CD-/}2 , +/3} & \BSS{tkzTabVar}\AC{-/1 , {\color{red}  +CD+/}2 , -/3 } 
\\ \hline 
\end{tabular}

\bigskip

\begin{tabular}{|c|c|}\hline  
\begin{tikzpicture}
\tkzTabInit[espcl=1.5]{$x$ / 1 ,$f(x)$ /1 }{ a , b, c  }
\tkzTabVar{-/1 , +DC-/2 , +/3}
\end{tikzpicture}
&  
\begin{tikzpicture}
\tkzTabInit[espcl=1.5]{$x$ / 1 ,$f(x)$ /1 }{ a , b, c  }
\tkzTabVar{+/1 , -DC+/2 , -/3}
\end{tikzpicture}
\\ \hline  
\BSS{tkzTabVar}\AC{ -/1 , {\color{red}  +DC-/}2 , +/3} & \BSS{tkzTabVar}\AC{ +/1 , {\color{red}  -DC+/}2 , -/3} 
\\ \hline  
\begin{tikzpicture}
\tkzTabInit[espcl=1.5]{$x$ / 1 ,$f(x)$ /1 }{ a , b, c  }
\tkzTabVar{+/1 , -DC-/2 , +/3}
\end{tikzpicture}
&  
\begin{tikzpicture}
\tkzTabInit[espcl=1.5]{$x$ / 1 ,$f(x)$ /1 }{ a , b, c  }
\tkzTabVar{-/1 , +DC+/2 , -/3}
\end{tikzpicture}
\\ \hline
\BSS{tkzTabVar}\AC{+/1 , {\color{red}  -DC-/}2 , +/3} & \BSS{tkzTabVar}\AC{-/1 , {\color{red}  +DC+/}2 , -/3 } 
\\ \hline 
\end{tabular}

\bigskip

\begin{tabular}{|c|c|}\hline  
\begin{tikzpicture}
\tkzTabInit[espcl=1.5]{$x$ / 1 ,$f(x)$ /1 }{ a , b, c  }
\tkzTabVar[color=red]{-/1 , +V-/2 , +/3}
\end{tikzpicture}
&  
\begin{tikzpicture}
\tkzTabInit[espcl=1.5]{$x$ / 1 ,$f(x)$ /1 }{ a , b, c  }
\tkzTabVar[color=red]{+/1 , -V+/2 , -/3}
\end{tikzpicture}
\\ \hline  
\BSS{tkzTabVar}\AC{ -/1 , {\color{red}  +V-/}2 , +/3} & \BSS{tkzTabVar}\AC{ +/1 , {\color{red}  -V+/}2 , -/3} 
\\ \hline  
\begin{tikzpicture}
\tkzTabInit[espcl=1.5]{$x$ / 1 ,$f(x)$ /1 }{ a , b, c  }
\tkzTabVar[color=red]{+/1 , -V-/2 , +/3}
\end{tikzpicture}
&  
\begin{tikzpicture}
\tkzTabInit[espcl=1.5]{$x$ / 1 ,$f(x)$ /1 }{ a , b, c  }
\tkzTabVar[color=red]{-/1 , +V+/2 , -/3}
\end{tikzpicture}
\\ \hline
\BSS{tkzTabVar}\AC{+/1 , {\color{red}  -V-/}2 , +/3} & \BSS{tkzTabVar}\AC{-/1 , {\color{red}  +V+/}2 , -/3 } 
\\ \hline 
\end{tabular}

\newpage

%\paragraph{Mise en évidence d'une valeur} :

\begin{tabular}{|c|c|}\hline
 \multicolumn{1}{|c|}{\textbf{\TFRGB{Mise en évidence d'une valeur  }{Emphasizing a value }} }
 \\ \hline  
\begin{tikzpicture}
\tkzTabInit[espcl=1.5]{$x$ / 1 ,$f(x)$ /1 }{ a , b, c  }
\tkzTabVar[color=red]{+/1 , -V-/\colorbox{yellow}{2} , +/3}
\end{tikzpicture}
\\ \hline 
\BS{tkzTabVar}\AC{+/1 , -V-/\BSS{colorbox}\AC{yellow}\AC{2} , +/3}
\\ \hline 
\end{tabular}

\bigskip

%\paragraph{Variation sur plusieurs colonnes}:

\begin{tabular}{|c|c|}\hline
 \multicolumn{1}{|c|}{\textbf{\TFRGB{Variation sur plusieurs colonnes }{Multicolumn variation }} }
 \\ \hline   
\begin{tikzpicture}
\tkzTabInit[espcl=1.5,color]{$x$ / 1 ,$f(x)$ /1 }{ a , b, c  }
\tkzTabVar[color=red]{-/1 , R/ , +/3}
\end{tikzpicture}
\\ \hline 
\BS{tkzTabVar}\AC{-/1 , {\color{red}  R/} , +/3}
\\ \hline 
\end{tabular}

\bigskip
%\paragraph{Valeurs intermédiaires}:

\begin{tabular}{|c|c|}\hline 
 \multicolumn{2}{|c|}{\textbf{\TFRGB{Valeurs intermédiaires }{Intermediate values }} }
 \\ \hline   
\begin{tikzpicture}
\tkzTabInit[espcl=1.5]{$x$ / 1 ,$f(x)$ /1 }{ a , b, c  }
%\tkzTabLine{d,+,}%
\tkzTabVar{ - / 1 , R/ , + / 3 }
\tkzTabVal{1}{3}{0.25}{A}{x}
\end{tikzpicture}
&
\begin{tikzpicture}
\tkzTabInit[espcl=1.5]{$x$ / 1 ,$f(x)$ /1 }{ a , b, c  }
%\tkzTabLine{d,+,}%
\tkzTabVar{ - / 1 , R/ , + / 3 }
\tkzTabVal{1}{3}{0.75}{\colorbox{yellow}{A}}{\colorbox{yellow}{x}}
\end{tikzpicture}
\\ \hline 
\BSS{tkzTabVal}\AC{1}\AC{3}\AC{0.25}\AC{A}\AC{x} & \BSS{tkzTabVal}\AC{1}\AC{3}\AC{0.75}\AC{A}\AC{x}
\\ \hline 
\end{tabular}
\bigskip

\begin{tabular}{|c|c|}\hline 
\begin{tikzpicture}
\tkzTabInit[espcl=1.5]{$x$ / 1, /1 ,$f(x)$ /1 }{ a , b, c  }
\tkzTabVar{   }
\tkzTabVar{ - / 1 , R/ , + / 3 }
\tkzTabVal[draw]{1}{3}{0.33}{A}{x}
\end{tikzpicture}
\\ \hline 
\BSS{tkzTabVal}[\RDD{draw}]\AC{1}\AC{3}\AC{0.25}\AC{A}\AC{x}
\\ \hline 
\end{tabular}

\bigskip 
%\paragraph{Ajout d’images} :

\begin{tabular}{|c|c|}\hline
  \multicolumn{2}{|c|}{\textbf{\TFRGB{Ajout d'images }{Picture insertion }} }
  \\ \hline 
\begin{tikzpicture}
\tkzTabInit[espcl=1.5]{$x$ / 1 ,$f(x)$ /1 }{ a , b, c,d  }
%\tkzTabLine{d,+,}%
\tkzTabVar{ - / 1 , R/  , R/, + / 3 }
\tkzTabIma{1}{4}{2}{x}
\end{tikzpicture}
&
\begin{tikzpicture}
\tkzTabInit[espcl=1.5]{$x$ / 1 ,$f(x)$ /1 }{ a , b, c,d  }
%\tkzTabLine{d,+,}%
\tkzTabVar{ - / 1 , R/  , R/, + / 3 }
\tkzTabIma{1}{4}{3}{x}
\end{tikzpicture}
\\ \hline 
\BSS{tkzTabIma}\AC{1}\AC{4}\AC{{\color{red}  2}}\AC{x} & \BSS{tkzTabIma}\AC{1}\AC{4}\AC{{\color{red}  3}}\AC{x}
\\ \hline 
\end{tabular}


\newpage

\SSCT{Les répétitions}{Repetitions}


\TFRGB{Utilisation du module  \og pgffor \fg  chargé automatiquement avec TikZ }{Package used :  \og pgffor \fg (automatically loaded with TikZ) }


\SbSSCT{Répétition à 1 variable}{One variable repetition}


\begin{tabular}{|c|} \hline  

\tikz \foreach \x in {1,...,10} \fill[blue](\x,0) circle (0.4cm);
\\  \hline  
\BS{tikz} \BSS{foreach} \BSR{x} in \AC{1,...,10} \BS{fill}[blue](\BSR{x},0) circle (0.4cm);
\\ \hline 
Variable \BSR{x} : position en X 
\\ \hline 
\end{tabular} 


\SbSSCT{Répétition à 2 variables}{Two variables repetition}

\begin{tabular}{|c|} \hline  
\TFRGB{Liste de variables numériques}{Numerical variables}
\\ \hline 
\tikz \foreach \pos/\y in {1/10,2/20,3/30,4/40,5/50,6/60,7/70,8/80,9/90,10/100} \fill[color=blue!\y](\pos,0) circle (0.5cm);
\\ \hline  
\BS{tikz} \BS{foreach} \BSR{pos}/\BSB{y} in \AC{1/10,2/20,3/30,4/40,5/50,6/60,7/70,8/80,9/90,10/100} \\ \BS{fill}[color=blue!\BSB{y}](\BSR{pos},0) circle (0.5cm);
\\ \hline 
Variable \BSR{pos} : position en X \hspace{1cm} Variable \BSB{y} : couleur
\\ \hline 
\end{tabular} 

\bigskip

\begin{tabular}{|c|} \hline
\TFRGB{Liste de variables mixtes}{Composite variables}
\\ \hline   
\tikz \foreach \x/\col in {1/red,3/green,5/magenta,7/blue}  \shade[ball color=\col](\x,0) circle (1);
\\ \hline  
\BS{tikz} \BS{foreach} \BSR{x}/\BSB{col} in {1/red,3/green,5/magenta,7/blue}  \BS{shade}[ball color=\BSB{col}](\BSR{x},0) circle (1);
\\  \hline 
Variable \BSR{x} : position en X  \hspace{1cm}  Variable \BSB{col} : couleur 
\\ \hline 
\end{tabular} 



\bigskip

\begin{tabular}{|c|} \hline
\TFRGB{Liste de variables avec un pas}{Variables with a step}
\\ \hline   
\begin{tikzpicture}
  \foreach \x in {1,2,...,4,7,8,...,10}
    \foreach \y in {1,...,3}
    {      \draw (\x,\y) +(-.5,-.5) rectangle ++(.5,.5);
      \draw (\x,\y) node{\x,\y};
    }
\end{tikzpicture}
\\ \hline  
\parbox{12cm}{ 
\BS{begin}\AC{tikzpicture}\\
\BS{foreach} \BSR{x} in\AC{1,2,...,4,7,8,...,10} \\
\BS{foreach} \BSB{y} in \AC{1,...,3} \\
\AC{  \BS{draw} (\BSR{x},\BSB{y}) +(-.5,-.5) rectangle ++(.5,.5);
\BS{draw} (\BSR{x},\BSB{y}) node{\BSR{x},\BSB{y}}; }\\
\BS{end}\AC{tikzpicture} \\
}
\\ \hline 
Variable \BSR{x} : position en X  \hspace{1cm}  Variable \BSR{y} : position en Y 
\\ \hline 

\end{tabular}

\bigskip
\begin{tabular}{|l|l|} \hline 
 \multicolumn{2}{|c|}{ \TFRGB{Exemples de liste}{List example }}
 \\ \hline 
\foreach \x in {1,...,6} {\x, }
&  
\BS{foreach} \BSR{x} in \AC{1,...,6} \AC{\BSR{x}, }
\\ \hline 
\foreach \x in {1,3,...,11} {\x, }
&  
\BS{foreach} \BSR{x} in \AC{1,3,...,11} \AC{\BSR{x}, }
\\ \hline 
\foreach \x in {Z,X,...,M} {\x, }
&  
\BS{foreach} \BSR{x} in \AC{Z,X,...,M} \AC{\BSR{x}, }
\\ \hline 
\foreach \x in {2^1,2^...,2^7} {$\x$, }
&  
\BS{foreach} \BSR{x} in \AC{2\^{}1,2\^{}...,2\^{}7} \AC{\BSR{x}, }
\\ \hline
\foreach \x in {0cm,0.5cm,...cm,3cm} {$\x$, }
&  
\BS{foreach} \BSR{x} in \AC{0cm,0.5cm,...cm,3cm} \AC{\BSR{x}, }
\\ \hline
\foreach \x in {A_1,..._1,H_1} {$\x$, } 
&  
\BS{foreach} \BSR{x} in \AC{A\_1,...\_1,H\_1} \AC{\BSR{x}, }
\\ \hline
\end{tabular} 




\bigskip
\begin{tabular}{|c|} \hline 
\TFRGB{Variables numériques avec opération}{Calculation on variables}
\\ \hline  
\begin{tikzpicture}
   \foreach \x in {0,20,...,360}{ \filldraw[red] (0,0) .. controls (\x+10:1) .. (\x:3) .. controls (\x-10:1) .. (0,0);}
    \foreach \x in {10,30,...,370}{ \filldraw[blue] (0,0) .. controls (\x+10:1) .. (\x:3) .. controls (\x-10:1) .. (0,0);}  
\end{tikzpicture}
\\ \hline  
\parbox{12cm}{ 
\BS{begin}\AC{tikzpicture}\\
   \BS{foreach} \BSR{x} in {0,20,...,360}\AC{ \BS{filldraw}[red] (0,0) .. controls (\BSR{x}+10:1) .. (\BSR{x}:1) .. controls (\BSR{x}-10:1) .. (0,0);} \\
    \BS{foreach}  \BSR{x} in {10,30,...,370}\AC{ \BS{filldraw}[blue] (0,0) .. controls (\BSR{x}+10:3) .. (\BSR{x}:3) .. controls (\BSR{x}-10:3) .. (0,0);}  \\
\BS{end}\AC{tikzpicture} \\

}
\\ \hline 
Variable \BSR{x} : angle 
\\ \hline 
\end{tabular} 


\SbSSCT{Répétition à 2 variables - boucles imbriquées}{Nested loops}

\begin{tabular}{|c|c|} \hline  
 \multicolumn{2}{|c|}{\TFRGB{Ordre des boucles imbriquées}{Order of the nested loops }}
\\ \hline 

\begin{tikzpicture}[blue]
\draw (0,0)
\foreach \x in {1,2,3}
{\foreach \y in {0,1,2}
{-- (\x,\y) node{X}}};
\end{tikzpicture}
&  
\begin{tikzpicture}[blue]
\draw (0,0)
\foreach \y in {0,1,2}
\foreach \x in {1,2,3}
{-- (\x,\y) node{X}};
\end{tikzpicture}
\\ \hline 
\parbox{5cm}{ 
\BS{begin}\AC{tikzpicture} \\
\BS{draw} (0,0) \\
\BS{foreach} \BSR{x} in \AC{1,2,3} \\
\BS{foreach} \BSB{y} in \AC{0,1,2} \\
\AC{-- (\BSR{x},\BSB{y}) node\AC{X}};\\
\BS{end}\AC{tikzpicture} \\ } 
&  
\parbox{5cm}{ 
\BS{begin}\AC{tikzpicture} \\
\BS{draw} (0,0) \\
\BS{foreach} \BSB{y} in \AC{0,1,2}\\
\BS{foreach} \BSR{x} in   \AC{1,2,3}\\
\AC{-- (\BSR{x},\BSB{y}) node\AC{X}};\\
\BS{end}\AC{tikzpicture} \\ } 
\\ \hline 
\end{tabular} 




 


\newpage
\SSCT{Dessin robotisé}{turtle graphics}



 \maboite{\BS{usetikzlibrary}\AC{turtle}}
\label{lib-turtle}


\begin{center}
\RRR{ 73 }
\end{center}

\begin{tabular}{|c|c|c|c|} \hline 
\multicolumn{4}{|c|}{  \BS{draw} [blue,line width=3pt,turtle={home,forward}];} \\  \hline 
\begin{tikzpicture}
\draw[help lines] (-1.5,-2) grid (1.5,2) ; 
\draw [blue,line width=3pt,turtle={home,forward}];
\end{tikzpicture}
&  
\begin{tikzpicture}
\draw[help lines] (-1.5,-2) grid (1.5,2) ;  
\draw [blue,line width=3pt,turtle={home,forward=1.5cm}];
\end{tikzpicture}
&  
\begin{tikzpicture}
\draw[help lines] (-1.5,-2) grid (1.5,2) ;  
\draw [blue,line width=3pt,turtle={home,fd}];
\end{tikzpicture}
&  
\begin{tikzpicture}
\draw[help lines] (-1.5,-2) grid (1.5,2) ; 
\draw [blue,line width=3pt,turtle={home,fd=1.5cm}];
\end{tikzpicture}
\\ \hline 
turtle=\AC{home,forward}  & turtle=\AC{home,forward=1.5cm} & turtle=\AC{home,fd} & 
turtle=\AC{home,fd=1.5cm} \\ 
\hline 
\end{tabular} 

\bigskip


\begin{tabular}{|c|c|c|c|}
\hline 
\multicolumn{4}{|c|}{  \BS{draw} [blue,line width=3pt,turtle={home,left,fd];}} \\  \hline  
\hline 
\begin{tikzpicture}
\draw (-1,-1) grid (1,1) ; 
\draw [blue,line width=3pt,turtle={home,left,fd}];
\end{tikzpicture} 
&  
\begin{tikzpicture}
\draw (-1,-1) grid (1,1) ; 
\draw [blue,line width=3pt,turtle={home,left=45,fd}];
\end{tikzpicture}
&  
\begin{tikzpicture}
\draw (-1,-1) grid (1,1) ; 
\draw [blue,line width=3pt,turtle={home,lt,fd}];
\end{tikzpicture}
&  
\begin{tikzpicture}
\draw (-1,-1) grid (1,1) ; 
\draw [blue,line width=3pt,turtle={home,lt=45,fd}];
\end{tikzpicture}
\\ \hline
turtle=\AC{home,left,fd}  & turtle=\AC{home,left=45,fd} & turtle=\AC{home,lt,fd} & 
turtle=\AC{home,lt=45,fd} \\ 
\hline 
\end{tabular} 

\bigskip

\begin{tabular}{|c|c|c|c|}
\hline 
\multicolumn{4}{|c|}{  \BS{draw} [blue,line width=3pt,turtle={home,right,fd];}} \\  \hline  
\hline 
\begin{tikzpicture}
\draw (-1,-1) grid (1,1) ; 
\draw [blue,line width=3pt,turtle={home,right,fd}];
\end{tikzpicture} 
&  
\begin{tikzpicture}
\draw (-1,-1) grid (1,1) ; 
\draw [blue,line width=3pt,turtle={home,right=45,fd}];
\end{tikzpicture}
&  
\begin{tikzpicture}
\draw (-1,-1) grid (1,1) ; 
\draw [blue,line width=3pt,turtle={home,rt,fd}];
\end{tikzpicture}
&  
\begin{tikzpicture}
\draw (-1,-1) grid (1,1) ; 
\draw [blue,line width=3pt,turtle={home,rt=45,fd}];
\end{tikzpicture}
\\ \hline
turtle=\AC{home,right,fd}  & turtle=\AC{home,right=45,fd} & turtle=\AC{home,rt,fd} & 
turtle=\AC{home,rt=45,fd} \\ 
\hline 
\end{tabular} 

\bigskip

\begin{tabular}{|c|c|} \hline 
\tikz[blue,line width=3pt]
\draw [->,turtle={home,rt,fd,fd,lt,fd,lt,fd}];
&  
\tikz[blue,line width=3pt]
\draw [->,turtle/distance=2cm,turtle={home,rt,fd,fd,lt,fd,lt,fd}];
\\ \hline 
[->,turtle={home,rt,fd,fd,lt,fd,lt,fd}] & [->,turtle/distance=2cm,turtle={home,rt,fd,fd,lt,fd,lt,fd}] 
\\ \hline 
\end{tabular} 

\bigskip


\begin{tabular}{|c|} \hline 
\begin{tikzpicture}[turtle/distance=2cm]
\draw[help lines] (-1.5,-1) grid (6,3) ; 
\draw [blue,line width=3pt,dotted,turtle={home,forward,right,forward},fd];
\draw [red,line width=3pt,turtle={how/.style={bend left},home,fd,rt,fd,fd}] ;
\end{tikzpicture}
\\  \hline 
[red,turtle=\AC{\rouge{how/.style}=\AC{bend left},home,fd,rt,fd,fd}]
\\ \hline 
\end{tabular} 

\bigskip

\begin{tabular}{|c|c|}  \hline 
\tikz
\filldraw [turtle/distance=2cm,thick,blue,fill=red!20]
[turtle=home]
\foreach \i in {1,...,5}
{
[turtle={forward,right=144}]
}; 
& 
 
\parbox[b]{10cm}{
\BS{filldraw}[turtle/distance=2cm,thick,blue,fill=red!20] \\
$[$ turtle=home $]$ \\
\BS{foreach} \BS{i} in \AC{1,...,5} \\
{
[ turtle=\AC{forward,right=144} ]
};
}
\\ \hline  
\end{tabular} 

\bigskip



\begin{tabular}{|c|c|}  \hline 
\tikz \draw [thick,blue]
[turtle=home]
\foreach \i in {1,...,25}
{
[turtle={forward=\i/5,right=120}]
};
& 
 
\parbox[b]{10cm}{
\BS{draw}[thick,blue] \\
$[$ turtle=home $]$ \\
\BS{foreach} \BS{i} in \AC{1,...,25} \\
{
[turtle=\AC{forward=\BS{i}/5,right=120} ]
}; \\
\vspace{1cm}
}
\\ \hline  
\end{tabular}





%%\subsection[Commande multido]{Commande multido \cite{pst-user} \cite{multido} }
%%
%
%%
%%\essais{pstrep2ess.tex}
%%
%%\subsection[Commande psforeach]{Commande psforeach \cite{pst-news10} }
%
%
%%
%%\newpage
%%% % % %======================================================================
%%\section[La géométrie]{La géométrie  \cite{pst-eucl} }
%%
%%Utilisation du module \textbf{pst-eucl} \label{pst-eucl}(consultez le fichier\textbf{ pst-eucl-doc.pdf} )
%%
%%
%%\psset{fillcolor=yellow,linecolor=blue,dotscale=2}
%%\subsection{\'Elements de base}
%%
%
%%
%%\subsection[Transformations géométriques]{Transformations géométriques \cite{pst-eucl} }
%%
%%
%
%%
%%
%%\subsection[Constructions particulières en géométrie ]{Constructions particulières en géométrie }
%%
%
%%
%%\subsection[Intersections]{Intersections  }
%%
%
%%
%%%--------------------------------------------------------------
%%\section[Les vecteurs]{Les vecteurs  }
%%
%
%%%==============================================================
\newpage
 
\SSCT{Les diagrammes arborescents }{Tree diagram}



\begin{center}
\RRR{21}
\end{center}

\subsection{Structure}

\begin{tabular}{|c|c|} \hline 
 \begin{tikzpicture}[baseline=0pt]
 \node {}
 child 
 child { child  child }
 child ;
 \end{tikzpicture} 
 &  
\begin{tikzpicture}[baseline=0pt]
\coordinate
 child 
 child { child  child }
 child ;
\end{tikzpicture} 
 \\ \hline
 \BS{node} \AC{}
{\color{red} child 
 child} \AC{ {\color{red}child  child }}
 {\color{red}child} ;
 &
  \BS{coordinate}
 {\color{red}child 
  child} \AC{ {\color{red}child  child }}
 {\color{red}child} ;
 \\ \hline 
\end{tabular} 

\bigskip

 
 \begin{tabular}{|c|c|}  \hline  
 \begin{tikzpicture}[baseline=0pt]
 \node {père}
 child {node {frère}}
 child {node {moi}
 child {node {fils}}
 child {node {fille}}}
 child {node{soeur}};
 \end{tikzpicture} 
 & 
 \parbox[t]{8cm}{ 
  \BS{begin}\AC{tikzpicture} \\
  \BS{node} \AC{père} \\
  {\color{red}child} \AC{node \AC{frère}} \\
  {\color{red}child} \AC{node \AC{moi}\\
   {\color{red}child} \AC{node \AC{fils}}\\
   {\color{red}child} \AC{node \AC{fille}}}\\
  {\color{red}child} \AC{node\AC{soeur}};\\
  \BS{end}\AC{tikzpicture}\\
  } 
 \\  \hline  
 
 \end{tabular}

\bigskip
\begin{tabular}{|c|} \hline  
  \begin{tikzpicture}[baseline=0pt]
\node {racine} child  foreach \name in {a,b,c,d} {node {\name}}; 
 \end{tikzpicture} 
\\  \hline  
\BS{node} \AC{racine} child  \RDD{foreach} \BS{name} in \AC{a,b,c,d} \AC{node \AC{\BS{name}}};
\\ \hline 
\end{tabular}  

 
  
\subsection{Orientation}

\begin{tabular}{|c|c|c|} \hline  
\begin{tikzpicture} 
  \node {père}[grow=-30] 
   child {node {frère}}
   child {node {moi}
   child {node {fils}}
   child {node {fille}}}
   child {node{soeur}};
   \end{tikzpicture}
&  
\begin{tikzpicture} 
  \node {père}[grow=30] 
   child {node {frère}}
   child {node {moi}
   child {node {fils}}
   child {node {fille}}}
   child {node{soeur}};
   \end{tikzpicture}
&  
\begin{tikzpicture} 
  \node {père}[grow'=30] 
   child {node {frère}}
   child {node {moi}
   child {node {fils}}
   child {node {fille}}}
   child {node{soeur}};
   \end{tikzpicture}
\\ \hline 
\BS{node} \AC{père}[\RDD{grow}=-30]
&  
\BS{node} \AC{père}[\RDD{grow}=30]
&  
\BS{node} \AC{père}[\RDD{grow'}=30]
\\ \hline 
\end{tabular}
\bigskip  


\begin{tabular}{|c|c|c|} \hline  
 \begin{tikzpicture}
 \node {père}[grow=up]
 child {node {frère}}
 child {node {moi}
 child {node {fils}}
 child {node {fille}}}
 child {node{soeur}};
 \end{tikzpicture}
&  
  \begin{tikzpicture}
  \node {père}[grow=left]
  child {node {frère}}
  child {node {moi}
  child {node {fils}}
  child {node {fille}}}
  child {node{soeur}};
  \end{tikzpicture}
&  
 \begin{tikzpicture}
 \node {père}[grow=right]
 child {node {frère}}
 child {node {moi}
 child {node {fils}}
 child {node {fille}}}
 child {node{soeur}};
 \end{tikzpicture} 
\\ \hline  
\BS{node} \AC{père}[\RDD{grow}=up]
&  
\BS{node} \AC{père}[\RDD{grow}=left]
&  
\BS{node} \AC{père}[\RDD{grow}=right]
\\ \hline 
 \begin{tikzpicture}
 \node {père}[grow=north]
 child {node {frère}}
 child {node {moi}
 child {node {fils}}
 child {node {fille}}}
 child {node{soeur}};
 \end{tikzpicture}
&  
  \begin{tikzpicture}
  \node {père}[grow=east]
  child {node {frère}}
  child {node {moi}
  child {node {fils}}
  child {node {fille}}}
  child {node{soeur}};
  \end{tikzpicture}
&  
 \begin{tikzpicture}
 \node {père}[grow=north east]
 child {node {frère}}
 child {node {moi}
 child {node {fils}}
 child {node {fille}}}
 child {node{soeur}};
 \end{tikzpicture} 
\\ \hline  
\BS{node} \AC{père}[\RDD{grow}=north]
&  
\BS{node} \AC{père}[\RDD{grow}=east]
&  
\BS{node} \AC{père}[\RDD{grow}=north east ]
\\ \hline
\end{tabular}  

\bigskip 

\begin{tabular}{|c|c|} \hline  
\begin{tikzpicture} [baseline=0pt]
  \node {père}
   child[grow=right,red] {node {frère}}
   child {node {moi}
   child {node {fils}}
   child {node {fille}}}
   child[grow=north west,red] {node{soeur}};
   \end{tikzpicture}
& 
\parbox[c]{8cm}{ 
  \BS{node} \AC{père}\\
   child[\RDD{grow}=right,red] \AC{node \AC{frère}}\\
   child \AC{node \AC{moi}\\
   child \AC{node \AC{fils}}\\
   child \AC{node \AC{fille}}}\\
   child[\RDD{grow}=north west,red] \AC{node\AC{soeur}};\\
   }
\\ \hline 
\end{tabular} 

\subsection{Distance}

\SbSSCT{Distance père fils}{Parent-child distance}

\begin{tabular}{|c|c|} \hline  
   \begin{tikzpicture}
   \node {père}[level distance=3cm,red]
   child {node {frère}}
   child {node {moi}
   child {node {fils}}
   child {node {fille}}}
   child {node{soeur}};
   \end{tikzpicture}
&  
    \begin{tikzpicture}
    \node {père}
    child[level distance=3cm,red] {node {frère}}
    child {node {moi}
    child {node {fils}}
    child[level distance=.5cm,red] {node {fille}}}
    child {node{soeur}};
    \end{tikzpicture} 
\\ \hline  
\BS{node} \AC{père}[level distance=3cm,red]
&  
child[level distance=3cm,red] \AC{node \AC{frère}}
\\
&  
child[level distance=.5cm,red] \AC{node \AC{fille}}
\\ \hline 
\multicolumn{2}{|c|}{\dft{} : level distance=15 mm} 
\\ \hline
\end{tabular} 

\bigskip

\begin{tabular}{|c|c|}  \hline  
\begin{tikzpicture}
\node {père}[level 1/.style={level distance=1cm}]
child{node {frère}}
child {node {moi}
child {node {fils}}
child {node {fille}}}
child {node{soeur}};
\end{tikzpicture}
&
\begin{tikzpicture}
\node {père}[level 2/.style={level distance=.5cm}]
child{node {frère}}
child {node {moi}
child {node {fils}}
child {node {fille}}}
child {node{soeur}};
\end{tikzpicture}
 \\  \hline  
\BS{node} \AC{père}[\RDD{level 1/.style}=\AC{level distance=1cm}]
 &  
\BS{node} \AC{père}[\RDD{level 2/.style}=\AC{level distance=.5cm}]
 \\  \hline 
 \end{tabular} 

\SbSSCT{Distance père fils}{Two children distance}
    
 \begin{tabular}{|c|c|}  \hline  
\begin{tikzpicture} 
  \node {père}[sibling distance=1cm,red]
   child {node {frère}}
   child {node {moi}
   child {node {fils}}
   child {node {fille}}}
   child {node{soeur}};
   \end{tikzpicture} 
 &  
 \begin{tikzpicture} 
   \node {père}[sibling distance=3cm,red]
    child {node {frère}}
    child {node {moi}
    child {node {fils}}
    child {node {fille}}}
    child {node{soeur}};
    \end{tikzpicture} 
 \\  \hline  
    \BS{node} \AC{père}[\RDD{sibling distance}=1cm,red]
 &  
   \BS{node} \AC{père}[\RDD{sibling distance}=3cm,red]
 \\  \hline 
 \multicolumn{2}{|c|}{\dft{} : sibling distance=15 mm} 
 \\ \hline
 \end{tabular} 

\bigskip
\begin{tabular}{|c|c|} \hline  
\TFRGB{Problème}{Problem} & solution 
\\ \hline 
 \begin{tikzpicture} 
   \node {père}[sibling distance=2cm]
    child {node {frère}}
    child {node {moi}
    child {node {fils}}
    child [red]{node {fille}}}
    child {node{soeur}
    child [red]{node {neveu}}
    child {node {nièce}}
    };
    \end{tikzpicture}
&  
 \begin{tikzpicture} 
   \node {père}[level 1/.style={sibling distance=2cm},
   level 2/.style={sibling distance=1cm}]
    child {node {frère}}
    child {node {moi}
    child {node {fils}}
    child  [red]{node {fille}}}
    child {node{soeur}
    child  [red]{node {neveu}}
    child {node {nièce}}
    };
    \end{tikzpicture}
\\ \hline  
[sibling distance=2cm]
&  
[level 1/.style={sibling distance=2cm},\\
&
   level 2/.style={sibling distance=1cm}]
\\ \hline 
\end{tabular} 

\SbSSCT{Personnalisation des noeuds}{Nodes customization}
  
  \begin{tabular}{|c|c|}   \hline

   \begin{tikzpicture}[baseline=0pt]
   \node[starburst,draw] {père}[grow=right]
   child {node[diamond,draw] {frère}}
   child {node[diamond,draw] {moi}
   child {node[ellipse,draw] {fils}}
   child {node[ellipse,draw] {fille}}}
   child {node[diamond,draw] {soeur}};
   \end{tikzpicture}
  &
   \parbox{8cm}{   
    \BS{node}[\RDD{starburst} \footnotemark[1] ,draw] \AC{père}[grow=right]\\ \\
    child \AC{node[\RDD{diamond},draw] {frère}}\\
    child \AC{node[\RDD{diamond},draw] {moi}\\
    child \AC{node[\RDD{ellipse},draw] {fils}}\\
    child \AC{node[\RDD{ellipse},draw] {fille}}}\\
    child \AC{node[\RDD{diamond},draw] {soeur}}; \\
    \\
 }
  \\   \hline 
    \begin{tikzpicture}[baseline=0pt,inner sep=4pt]
    \node[rectangle,double,draw,text width=1cm,text centered] {père et mère }[grow=right]
    child {node[red,ultra thick , draw,rotate=45] {frère}}
    child {node[blue,dashed, draw] {moi}
    child {node[ellipse,draw] {fils}}
    child {node[ellipse,fill] {fille}}}
    child {node[magenta,pattern=dots,draw] {soeur}};
    \end{tikzpicture} 
  &
    \parbox{10cm}{  
    \BS{node}[rectangle,double,draw,text width=1cm,text centered] \\ \AC{père}[grow=right,level distance=2cm]\\ \\
    child \AC{node{\color{red}[red,ultra thick,draw,rotate=45]} \AC{frère}}\\
    child \AC{node{\color{red}[blue,dashed, draw]} \AC{moi}\\
    child \AC{node{\color{red}[ellipse,draw]} \AC{fils}}\\
    child \AC{node {\color{red}[ellipse,fill]} \AC{fille}}}\\
    child \AC{node {\color{red}[magenta,pattern=dots,draw]} \AC{soeur}};\\
    }
  \\   \hline 
  \end{tabular} 
  
   \footnotetext[1]{ \TFRGB{autres types de n\oe uds voir}{Other types of nodes see} section \ref{ndbt} } 

\SbSbSSCT{Nom des noeuds}{Nodes name}

\begin{tabular}{|c|c|} \hline  
\begin{tikzpicture}[baseline=0pt]
\node (a) {a}
child
child {
child {child child}
child {child }
};
\node at (a-1) {a-1};
\node at (a-2) {a-2};
\node at (a-2-2) {a-2-2};
\node at (a-2-1) {a-2-1};
\node at (a-2-1-2) {a-2-1-2};
\draw[red,,ultra thick] (a-1) -- (a-2);
\end{tikzpicture}
& 

 \parbox[t]{8cm}{  
\BS{node} {\color{red}(a)} \AC{a}\\
child\\
child \AC{\\
child \AC{child child}\\
child \AC{child }\\
};\\
\BS{node} at {\color{red}(a-1)} \AC{a-1};\\
\BS{node} at {\color{red}(a-2)} \AC{a-2};\\
\BS{node} at {\color{red}(a-2-2)} \AC{a-2-2};\\
\BS{node} at {\color{red}(a-2-1)} \AC{a-2-1};\\
\BS{node} at {\color{red}(a-2-1-2)} \AC{a-2-1-2};\\
\\
\BS{draw}[red,ultra thick] {\color{red}(a-1)} -- {\color{red}(a-2)}; \\
}
\\ \hline 
\end{tabular} 

\bigskip

\begin{tabular}{|c|c|} \hline  
\begin{tikzpicture}[baseline=0pt]
\node (a) {a}
child
child {
child {coordinate (b) child child}
child
};
\node at (a-1) {a-1};
\node at (a-2) {a-2};
\node at (b) {b};
\node at (a-2-2) {a-2-2};
\node at (b-1) {b-1};
\node at (a-2-1-2) {a-2-1-2};
\draw[red,,ultra thick] (a-1) -- (b-1);

\end{tikzpicture}
&  
\parbox[t]{8cm}{  
\BS{node}  (a) \AC{a} \\
child  \\
child { \\
child {coordinate (b) child child} \\
child \\
}; \\
\BS{node}  at (a-1) \AC{a-1}; \\
\BS{node}  at (a-2) \AC{a-2}; \\
\BS{node}  at {\color{red}(b)} \AC{b}; \\
\BS{node}  at (a-2-2) \AC{a-2-2}; \\
\BS{node}  at {\color{red}(b-1)} \AC{b-1}; \\
\BS{node}  at (a-2-1-2) \AC{a-2-1-2}; \\
\\
\BS{draw}[red,ultra thick] {\color{red}(a-1)} -- {\color{red}(b-1)}; \\
}
\\ \hline 
\end{tabular} 

\bigskip

\begin{tabular}{|c|c|}\hline  
\begin{tikzpicture}
\node(a) {père}
child {node (b) {frère}}
child {node (c) {moi}
child {node (d) {fils}}
child {node (e) {fille}}}
child {node (f) {soeur}};
\draw[red,,ultra thick] (b) -- (d);
\end{tikzpicture}
& 
\parbox[b]{8cm}{

\BS{node} {\color{red}(a)} \AC{père}\\
child \AC{node {\color{red}(b)} \AC{frère}}\\
child \AC{node {\color{red}(c)} \AC{moi}\\
child \AC{node {\color{red}(d)} \AC{fils}}\\
child \AC{node {\color{red}(e)} \AC{fille}}}\\
child \AC{node {\color{red}(f)} \AC{soeur}};\\
\\
\BS{draw}[red,,ultra thick] {\color{red}(b)} -- {\color{red}(d)};\\
}
\\ \hline 
\end{tabular} 


\SbSbSSCT{Omission d'un noeud}{Missing a node}

\begin{tabular}{|c|} \hline  
 \begin{tikzpicture}
 \node {0} 
 child { node {1} }
 child { node {2} }
 child { node {3} }
 child[missing] { node {4} }
 child { node {5} }
 child { node {6} };
 \end{tikzpicture}
\\ \hline  
 child[\RDD{missing}] \AC{node \AC{4} }
\\ \hline 
\end{tabular} 

\SbSbSSCT{Modification du point d'accrochage}{Attachment point modification}

 \begin{tabular}{|l|l|} \hline  
  \begin{tikzpicture}
  \node {pére} [child anchor=east,red]
  child {node {frère}}
  child {node {moi}
  child  {node {fils}}
  child {node {fille} }
  };
  \end{tikzpicture}
 &  
  \begin{tikzpicture}
  \node {pére} 
  child {node {frère}}
  child {node {moi}
  child [child anchor=west,red] {node {fils}}
  child {node {fille} }
  };
  \end{tikzpicture}
 \\ \hline  
  \BS{node} \AC{pére} [{\color{red}child anchor=east},red]
 &  
  \BS{node} \AC{pére}
 \\ 
  child \AC{node \AC{frère}} &  child \AC{node \AC{frère}} \\
  child \{ node \AC{moi} &   child \{ node \AC{moi}\\
  child  \AC{node \AC{fils}} & child [\RDD{child anchor}=west,red]  \AC{node \AC{fils}} \\
   child  \AC{node \AC{fils}} \}; & child  \AC{node \AC{fils}} \}; 
   \\  \hline 
 \end{tabular} 
\bigskip

\begin{tabular}{|l|l|} \hline  
 \begin{tikzpicture}
 \node {pére} [parent anchor=east,red]
 child {node {frère}}
 child {node {moi}
 child  {node {fils}}
 child {node {fille} }
 };
 \end{tikzpicture}
&  
 \begin{tikzpicture}
 \node {pére} 
 child {node {frère}}
 child {node {moi}
 child [parent anchor=west,red] {node {fils}}
 child {node {fille} }
 };
 \end{tikzpicture}
\\ \hline  
 \BS{node} \AC{pére} [\RDD{parent anchor}=east,red]
&  
 \BS{node} \AC{pére}
\\ 
 child \AC{node \AC{frère}} &  child \AC{node \AC{frère}} \\
 child \{ node \AC{moi} &   child \{ node \AC{moi}\\
 child  \AC{node \AC{fils}} & child [\RDD{parent anchor}=west,red]  \AC{node \AC{fils}} \\
  child  \AC{node \AC{fils}} \}; & child  \AC{node \AC{fils}} \}; \\
\hline 
\end{tabular} 

\SbSbSSCT{Liaison}{Links}

\begin{tabular}{|c|c|c|} \hline  
 \begin{tikzpicture}
 \node {pére}  
child {node {frère}}
 child {node {moi}edge from parent[red,ultra thick]
 child  {node {fils}}
 child {node {fille} } }
 child {node{soeur}}; 
 \end{tikzpicture}
&  
 \begin{tikzpicture}
 \node {pére}
 child {node {frère}}
 child {node {moi} 
 child  {node {fils} edge from parent[red,ultra thick]}
 child {node {fille} } }
 child {node{soeur}};
 \end{tikzpicture}
&  
 \begin{tikzpicture}
   \node {père} 
    child {node {frère}}
    child {node {moi}
    child {node {fils}}
    child {node {fille} edge from parent[draw=none]}}
    child {node{soeur}};
 \end{tikzpicture}
\\ \hline  

child \{node \AC{moi} &  child  \{node \AC{fils} & child \{ node \AC{fille} \\ 
\RDD{edge from parent}[red,ultra thick] & \RDD{edge from parent}[red,ultra thick] \} & \RDD{edge from parent}[draw=none] \} \\
\hline 
\end{tabular} 

\bigskip

 
\begin{tabular}{|c|} \hline  
 \begin{tikzpicture}
 [edge from parent/.style={draw,red,ultra thick}]
  \node {père} 
   child {node {frère}}
   child {node {moi}
   child {node {fils}}
   child {node {fille}}}
   child {node{soeur}};
 \end{tikzpicture}
\\ \hline  
 [\RDD{edge from parent/.style}=\AC{draw,red,ultra thick}] \\
 \BS{node} \AC{père} 
\\ \hline 
\end{tabular} 
 
\SbSbSSCT{\'Etiquetes sur liaisons}{Labels on link}

\begin{tabular}{|c|c|c|c|} \hline 
\multicolumn{4}{|c|}{\BS{node} \AC{père}
child \AC{node \AC{fils}  \RDD{edge from parent}
 node[left,red] \AC{texte}};} 
 \\ \hline
 \begin{tikzpicture}
\node {père}
child {node {fils}  edge from parent
 node[left,red] {texte}};
 \end{tikzpicture}
&  
\begin{tikzpicture}
\node {père}
child {node {fils}  edge from parent
 node[right,red] {texte}};
 \end{tikzpicture}

&  
\begin{tikzpicture}
\node {père}
child {node {fils}  edge from parent
 node[near end,red] {texte}};
 \end{tikzpicture}
&  
\begin{tikzpicture}
\node {père}
child {node {fils}  edge from parent
 node[draw,red] {texte}};
 \end{tikzpicture}
\\ \hline  
node[\RDD{left},red] & node[\RDD{right},red] & node[\RDD{near end},red] & node[\RDD{draw},red] \\ 
\hline 
\end{tabular} 


\SbSbSSCT{Personalisation des liaisons}{Links customization}
 
 \begin{tabular}{|c|c|c|}  \hline  
 \multicolumn{3}{|c|} { [ edge from parent path=
   \{(\BSS{tikzparentnode.south}) {\color{red}.. controls +(0,-1) and +(0,1) ..  }} \\
 \multicolumn{3}{|c|}{ (\BSS{tikzchildnode.north})\} ]}
 \\ \hline
  \begin{tikzpicture}[edge from parent path=
  {(\tikzparentnode.south) .. controls +(0,-1) and +(0,1)
  .. (\tikzchildnode.north)}]
 \node {père}
 child {node {frère}}
 child {node {moi}
 child {node {fils}}
 child {node {fille}}}
 child {node{soeur}};
  \end{tikzpicture}
 &  
 \begin{tikzpicture}[edge from parent path=
 {(\tikzparentnode.south) -|(\tikzchildnode.north)}]
 \node {père}
 child {node {frère}}
 child {node {moi}
 child {node {fils}}
 child {node {fille}}}
 child {node{soeur}};
 \end{tikzpicture}
&
 \begin{tikzpicture}[edge from parent path=
  {(\tikzparentnode.south) to[in=90,out=-90] (\tikzchildnode.north)}]
 \node {père}
 child {node {frère}}
 child {node {moi}
 child {node {fils}}
 child {node {fille}}}
 child {node{soeur}};
  \end{tikzpicture}     
 \\ \hline 
{\color{red} .. controls +(0,-1) and +(0,1)
  ..} &
{\color{red}-|} &  {\color{red}to[in=90,out=-90]}
  \\ \hline 
   \multicolumn{3}{|c|}{ \TFRGB{voir liaison de noeuds}{see links available : } section \ref{liaisons} }
   \\ \hline
 \end{tabular} 
 

\newpage

\SbSSCT{Options supplémentaires avec « library trees »}{More options  with « library trees »}

\label{lib-trees}


 \maboite{\BS{usetikzlibrary}\AC{trees}}
 
\begin{center}
\RRR{72}
\end{center}

\SbSbSSCT{Positions d'un fils et de  deux fils}{One child and two childrenn position}

%\begin{tabular}{|c|c|c|} \hline  
%\begin{tikzpicture}
%[grow via three points={%
%one child at (0,0) and two children at (-.5,1) and (.5,1)}]
%\node at (0,0) {racine} child child child child;
%\draw[help lines] (-3,-1) grid (3,6); 
%\end{tikzpicture}
%&  
%\begin{tikzpicture}
%[grow via three points={%
%one child at (0,1) and two children at(-.5,1) and (.5,1)}]
%\node at (0,0) {racine} child child child child;
%\draw[help lines] (-3,-1) grid (3,6); 
%\end{tikzpicture}
%
%&  
%\begin{tikzpicture}
%[grow via three points={%
%one child at (0,-1) and two children at (-.5,1) and (.5,1)}]
%\node at (0,0) {racine} child child child child;
%\draw[help lines] (-3,-1) grid (3,6);
%\end{tikzpicture}
%\\ \hline  
%one child at (0,0)
%&  
%one child at (0,1)
%&  
%one child at (0,-1)
%\\ \hline 
%\end{tabular} 
%\bigskip
%
%\begin{tabular}{|c|c|c|} \hline  
%\begin{tikzpicture}[grow via three points={%
%one child at (0,0) and two children at (-.5,1) and (.5,1)}]
%\node at (0,-4.5) {racine} child child child child;
%\end{tikzpicture}
%&  
%\begin{tikzpicture}[grow via three points={%
%one child at (0,0) and two children at (0,1) and (1,1)}]
%\node at (0,-4.5) {racine} child child child child;
%\end{tikzpicture}
%&  
%\begin{tikzpicture}[grow via three points={%
%one child at (0,0)and two children at (-1,-.5) and (0,-.5)}]
%\node at (0,-4.5) {racine} child child child child;
%\end{tikzpicture}
%\\ \hline  
%at (-.5,1) and (.5,1)
%&  
%at (0,1) and (1,1)
%&  
%at (-1,-.5) and (0,-.5)
%\\ \hline 
%\end{tabular} 
%
%\bigskip

\begin{tabular}{|c|c|c|c|} \hline
\multicolumn{4}{|c|}{{\color{red}grow via three points}=\AC{
{\color{red} one child at} (0,1) {\color{red}and two children at} (-.5,1) {\color{red}and} (.5,1)} }  \\
\begin{tikzpicture}[red,ultra thick,grow via three points={%
one child at (0,1) and two children at (-.5,1) and (.5,1)}]
\node at (0,0) {un} child ;
\draw[help lines] (-1,-1) grid (1,2);
\end{tikzpicture}
&  
\begin{tikzpicture}[red,ultra thick,grow via three points={%
one child at (0,1) and two children at (-.5,1) and (.5,1)}]
\node at (0,0) {deux} child child;
\draw[help lines] (-1,-1) grid (1,2);
\end{tikzpicture}
&  
\begin{tikzpicture}[red,ultra thick,grow via three points={%
one child at (0,1) and two children at (-.5,1) and (.5,1)}]
\node at (0,0) {trois} child child child;
\draw[help lines] (-2,-1) grid (2,2);
\end{tikzpicture}
&  
\begin{tikzpicture}[red,ultra thick,grow via three points={%
one child at (0,1) and two children at (-.5,1) and (.5,1)}]
\node[] at (0,0) {quatre} child child child child;
\draw[help lines] (-2,-1) grid (2,2);
\end{tikzpicture}
\\ \hline  
&  &  &  \\ 
\hline 
\end{tabular} 

\bigskip

\begin{tabular}{|c|c|c|c|} \hline
\multicolumn{4}{|c|}{grow via three points=\AC{
one child at (0,1) and two children at {\color{red}(0,1)} and {\color{red}(1,1)}} }  \\ \hline
\begin{tikzpicture}[red,ultra thick,grow via three points={%
one child at (0,1) and two children at (0,1) and (1,1)}]
\node at (0,0) {un} child ;
\draw[help lines] (-1,-1) grid (1,2);
\end{tikzpicture}
&  
\begin{tikzpicture}[red,ultra thick,grow via three points={%
one child at (0,1) and two children at (0,1) and (1,1)}]
\node at (0,0) {deux} child child;
\draw[help lines] (-1,-1) grid (1,2);
\end{tikzpicture}
&  
\begin{tikzpicture}[red,ultra thick,grow via three points={%
one child at (0,1) and two children at (0,1) and (1,1)}]
\node at (0,0) {trois} child child child;
\draw[help lines] (-1,-1) grid (3,2);
\end{tikzpicture}
&  
\begin{tikzpicture}[red,ultra thick,grow via three points={%
one child at (0,1) and two children at (0,1) and (1,1)}]
\node[] at (0,0) {quatre} child child child child;
\draw[help lines] (-1,-1) grid (3,2);
\end{tikzpicture}
\\ \hline  
&  &  &  \\ 
\hline 
\end{tabular} 

\bigskip

\begin{tabular}{|c|c|c|c|} \hline
\multicolumn{4}{|c|}{grow via three points=\AC{
one child at (0,1) and two children at {\color{red}(-.5,1)} and {\color{red}(.5,1.5)}} }  \\  \hline  
\begin{tikzpicture}[red,ultra thick,grow via three points={%
one child at (0,1) and two children at (-.5,1) and (.5,1.5)}]
\node at (0,0) {un} child ;
\draw[help lines] (-1,-1) grid (1,3);
\end{tikzpicture}
&  
\begin{tikzpicture}[red,ultra thick,grow via three points={%
one child at (0,1) and two children at (-.5,1) and (.5,1.5)}]
\node at (0,0) {deux} child child;
\draw[help lines] (-1,-1) grid (1,3);
\end{tikzpicture}
&  
\begin{tikzpicture}[red,ultra thick,grow via three points={%
one child at (0,1) and two children at (-.5,1) and (.5,1.5)}]
\node at (0,0) {trois} child child child;
\draw[help lines] (-2,-1) grid (2,3);
\end{tikzpicture}
&  
\begin{tikzpicture}[red,ultra thick,grow via three points={%
one child at (0,1) and two children at (-.5,1) and (.5,1.5)}]
\node[] at (0,0) {quatre} child child child child;
\draw[help lines] (-2,-1) grid (2,3);
\end{tikzpicture}
\\ \hline  

\end{tabular} 
%
%\bigskip
%
%\begin{tabular}{|c|c|c|c|} \hline
%\multicolumn{4}{|c|}{grow via three points=\AC{
%one child at (0,1) and two children at (-.5,.5) and (.5,1)} }  \\
%\begin{tikzpicture}[red,ultra thick,grow via three 
%\end{tikzpicture}
%&  
%\begin{tikzpicture}[red,ultra thick,grow via three points={one child at (0,-.5) and two children at (-.5,.5) and (.5,1)} 
%\node at (0,0) {deux} child child;
%\draw[help lines] (-1,-1) grid (1,2);
%\end{tikzpicture}
%&  
%\begin{tikzpicture}[red,ultra thick,grow via three points={one child at (0,-.5) and two children at (-.5,.5) and (.5,1)}  
%\node at (0,0) {trois} child child child;
%\draw[help lines] (-1,-1) grid (3,2);
%\end{tikzpicture}
%%&  
%\begin{tikzpicture}[red,ultra thick,grow via three points={one child at (0,-.5) and two children at (-.5,.5) and (.5,1)}  
%\node[] at (0,0) {quatre} child child child child;
%\draw[help lines] (-1,-1) grid (3,2);
%\end{tikzpicture}
%\\ \hline  
%&  &  &  \\ 
%\hline 
%\end{tabular}
%
%\begin{tikzpicture}[grow via three points={%
%one child at (-1,-.5) and two children at (-1,-.5) and (0,-.75)}]
%\node at (0,0) {one} child;
%\node at (0,-1.5) {two} child child;
%\node at (0,-3) {three} child child child;
%\node at (0,-4.5) {four} child child child child;
%\end{tikzpicture}

%---------------------------------------

%\subsubsection{Liaison angulaire }
\SbSbSSCT{Liaison angulaire }{Angular linking}

\begin{tabular}{|c|c|c|} \hline  
\begin{tikzpicture}
[grow cyclic]
\node {racine} child child child child;
\end{tikzpicture}
&  
\begin{tikzpicture}
[grow cyclic,sibling angle=45]
\node {racine} child child child child;
\end{tikzpicture}
&  
\begin{tikzpicture}
[grow cyclic,sibling angle=90]
\node {racine} child child child child;
\end{tikzpicture}
\\ \hline  
[\RDD{grow cyclic}]
&  
[grow cyclic,\RDD{sibling angle}=45]
&  
[grow cyclic,\RDD{sibling angle}=90]
\\ \hline 
\end{tabular} 


\bigskip

%\begin{tabular}{|c|c|} \hline  
%\begin{tikzpicture}[baseline=0pt]
%[grow cyclic,level 1/.style={level distance=1cm,sibling angle=40},level 2/.style={level distance=5mm,sibling angle=20},level 3/.style={sibling angle=10}]
%\node {racine} child child  child {child child child }child child;
%\end{tikzpicture}
%& 
% \parbox[t]{8cm}{ 
%[grow cyclic,\\
%\RDD{level 1/.style}=\AC{sibling angle=40},\\
%\RDD{level 2/.style}=\AC{sibling angle=20}]\\
%\\
%\BS{node} \AC{racine} child \\
% child \\
% child \AC{child child child } \\
% 
% child \\
%  child;
%}
%\\ \hline 
%\end{tabular} 

%\bigskip
%
%\begin{tikzpicture}[baseline=0pt]
%[grow cyclic,
%level 1/.style={sibling angle=40},
%level 2/.style={sibling angle=10}]
%\node {racine} child child  {child child child child }child child;
%\end{tikzpicture}

%\bigskip
%\begin{tikzpicture}
%[grow cyclic,
%level 1/.style={level distance=8mm,sibling angle=60},
%level 2/.style={level distance=4mm,sibling angle=45},
%level 3/.style={level distance=2mm,sibling angle=30}]
%\coordinate [rotate=-90] % going down
%child foreach \x in {1,2,3}
%{child foreach \x in {1,2,3}
%{child foreach \x in {1,2,3}}};
%\end{tikzpicture}
%
%\begin{tikzpicture}
%[grow cyclic,
%level 1/.style={level distance=8mm},
%level 2/.style={level distance=4mm},
%level 3/.style={level distance=2mm}]
%\coordinate [rotate=-90] % going down
%child foreach \x in {1,2,3}
%{child foreach \x in {1,2,3}
%{child foreach \x in {1,2,3}}};
%\end{tikzpicture}
%
%
%\bigskip

\begin{tabular}{|c|c|} \hline  
\begin{tikzpicture}[baseline=0pt]
\node {root}
[clockwise from=30,sibling angle=30]
child {node {$30$}}
child {node {$0$}}
child {node {$-30$}}
child {node {$-60$}};
\end{tikzpicture}
&  
 \parbox[t]{8cm}{ 
\BS{node} \AC{racine}
[\RDD{clockwise from}=30,\RDD{sibling angle}=30] \\
\\
child \AC{node \AC{\$30\$} }\\
child \AC{node \AC{\$0\$} }\\
child \AC{node \AC{\$-30\$} }\\
child \AC{node \AC{\$-60\$ } };
}
\\ \hline 
\end{tabular} 

%-----------------------------------------------

%\subsubsection{Liaisons en fourchette}
\SbSbSSCT{Liaisons en fourchette}{Forking links}

\begin{tabular}{|c|c|} \hline  
\begin{tikzpicture}[baseline=0pt]
\node {père}
[edge from parent fork down]
child {node {frère}}
child {node {moi}
child[child anchor=north east] {node {fils}}
child {node {fille}}
};
\end{tikzpicture}
&  
\parbox[t]{9cm}{ 
\BS{node} \AC{père}
[\RDD{edge from parent fork down}] \\
\\
child \AC{node \AC{frère}} \\
child \AC{node \AC{moi} \\
child [child anchor=north east] \AC{node \AC{fils}} \\
child \AC{node \AC{fille}} \\
};
}
\\ \hline 
\end{tabular} 

\bigskip

\begin{tabular}{|c|c|} \hline  
\begin{tikzpicture}[baseline=0pt]
\node {père}
[edge from parent fork right]
child {node {frère}}
child {node {moi}
child {node {fils}}
child {node {fille}}
};
\end{tikzpicture}
&  
\parbox[t]{9cm}{ 
\BS{node} \AC{père} 
[\RDD{edge from parent fork right}] \\
\\
child \AC{node \AC{frère}} \\
child \AC{node \AC{moi} \\
child \AC{node \AC{fils}} \\
child \AC{node \AC{fille}} \\
};
}
\\ \hline 
\end{tabular} 

\bigskip

\begin{tabular}{|c|c|} \hline  
\begin{tikzpicture}[baseline=0pt]
\node {père}
[edge from parent fork right,grow=right]
child {node {frère}}
child {node {moi}
child {node {fils}}
child {node {fille}}
};
\end{tikzpicture}
&  
\parbox[t]{9cm}{ 
\BS{node} \AC{père} 
[edge from parent fork right,\RDD{grow=right}] \\
\\
child \AC{node \AC{frère}} \\
child \AC{node \AC{moi} \\
child \AC{node \AC{fils}} \\
child \AC{node \AC{fille}} \\
};
}
\\ \hline 
\end{tabular} 

  

%%%==================================================

\newpage

\SSCT{Les schemas électriques }{Electrical Engineering Circuits}

%\section{Circuit électrique}
\label{lib-ee}
%Insérer dans le préambule :

 \maboite{\BS{usepackage}\AC{circuits.ee.IEC}}
 
 
%\subsection{Symboles}
\SbSSCT{Symboles}{Symbols}

\begin{center}
\RRR{47-4}
\end{center}

\begin{tabular}{|c|c|c|c|} \hline 
\TFRGB{sur un noeud}{On a node} & \TFRGB{sur un chemin}{On a path}
\\   \hline
\begin{tikzpicture}[blue,line width=2pt]
\useasboundingbox (-.2,-.2) grid (2.2,1.2);
\draw [help lines] (0,0) grid (2,1);
\node[circuit ee IEC] at (1,.5) [resistor] {} ; 
\end{tikzpicture}
&
\begin{tikzpicture}[blue,line width=2pt]
\useasboundingbox (-.2,-.2) grid (2.2,1.2);
\draw [help lines] (0,0) grid (2,1);
\draw [circuit ee IEC] (0,.5) to [resistor] (2,.5) ; 
\end{tikzpicture} 
\\   \hline
\BS{node}  \rouge{[circuit ee IEC]}  at (1,0.5) \rouge{ to [resistor] } \AC{} ;
&
\BS{draw}   \rouge{[circuit ee IEC]}(0,0.5) \rouge{ to [resistor] } (2,.5) ;
\\   \hline   
\end{tabular} 

\bigskip

\begin{tabular}{|c|c|c|c|} \hline
\multicolumn{4}{|c|}{\textbf{\TFRGB{Composants de base}{Basic Elements}}  }\\ 
\hline  
\multicolumn{4}{|c|}{\BS{draw}  [circuit ee IEC] (0,.5) to [\RDD{resistor}] (2,.5) ;  }\\ 
\hline
\multicolumn{4}{|c|}{\RRR{47-4-3} }
\\ \hline   
\begin{tikzpicture}[blue]
\useasboundingbox (-.5,0) rectangle (2.5,1);
\draw [circuit ee IEC] (0,0.5) to [resistor] (2,0.5) ; 
\end{tikzpicture} 
&
\begin{tikzpicture}[blue]
\useasboundingbox (-.5,0) rectangle (2.5,1);
\draw [circuit ee IEC] (0,.5) to [inductor] (2,.5) ; 
\end{tikzpicture}
&
\begin{tikzpicture}[blue]
\useasboundingbox (-.5,0) rectangle (2.5,1);
\draw [circuit ee IEC] (0,.5) to [capacitor] (2,.5) ; 
\end{tikzpicture}
&
\begin{tikzpicture}[blue]
\useasboundingbox (-.5,0) rectangle (2.5,1);
\draw [circuit ee IEC] (0,.5) to [battery] (2,.5) ; 
\end{tikzpicture} 
\\   \hline 
[\RDD{resistor}] &[\RDD{inductor}] & [\RDD{capacitor}] &[\RDD{battery}]
\\   \hline 
\begin{tikzpicture}[blue]
\useasboundingbox (-.5,0) rectangle (2.5,1);
\draw [circuit ee IEC] (0,.5) to [bulb] (2,.5) ; 
\end{tikzpicture}
&
\begin{tikzpicture}[blue]
\useasboundingbox (-.5,0) rectangle (2.5,1);
\draw [circuit ee IEC] (0,.5) to [current source] (2,.5) ; 
\end{tikzpicture}
&
\begin{tikzpicture}[blue]
\useasboundingbox(-.5,0) rectangle (2.5,1);
\draw [circuit ee IEC] (0,.5) to [voltage source] (2,.5) ; 
\end{tikzpicture}; 
&
\begin{tikzpicture}[blue]
\useasboundingbox (-.5,0) rectangle (2.5,1);
\draw [circuit ee IEC] (0,.5) to [ground] (2,.5) ; 
\end{tikzpicture} 
\\   \hline
[\RDD{bulb}] &[\RDD{current source}] & [\RDD{voltage source}] &[\RDD{ground}]
\\   \hline 
\multicolumn{4}{|c|}{\RRR{47-4-4} }
\\ \hline  
\begin{tikzpicture}[blue]
\useasboundingbox (-.5,0) rectangle (2.5,1);
\draw [circuit ee IEC] (0,.5) to [diode] (2,.5) ; 
\end{tikzpicture} 
&
\begin{tikzpicture}[blue]
\useasboundingbox (-.5,0) rectangle (2.5,1);
\draw [circuit ee IEC] (0,.5) to [Zener diode]  (2,.5) ; 
\end{tikzpicture} 
&
\begin{tikzpicture}[blue]
\useasboundingbox (-.5,0) rectangle (2.5,1);
\draw [circuit ee IEC] (0,.5) to [Schottky diode]  (2,.5) ; 
\end{tikzpicture}
&
\begin{tikzpicture}[blue]
\useasboundingbox(-.5,0) rectangle (2.5,1);
\draw [circuit ee IEC] (0,.5) to [tunnel diode]  (2,.5) ; 
\end{tikzpicture} 
\\   \hline 
[\RDD{diode}] &[\RDD{Zener diode}] & [\RDD{Schottky diode}] &[\RDD{tunnel diode}]
\\   \hline
\begin{tikzpicture}[blue]
\useasboundingbox (-.5,0) rectangle (2.5,1); 
\draw [circuit ee IEC] (0,.5) to [backward diode]  (2,.5) ; 
\end{tikzpicture}  
&
\begin{tikzpicture}[blue]
\useasboundingbox (-.5,0) rectangle (2.5,1);
\draw [circuit ee IEC] (0,.5) to [breakdown diode]  (2,.5) ; 
\end{tikzpicture} 
&
& 
\\   \hline 
[\RDD{backward diode}] &[\RDD{breakdown diode}] &  &
\\   \hline 
\multicolumn{4}{|c|}{\RRR{47-4-5} }
\\ \hline 
\begin{tikzpicture}[blue]
\useasboundingbox (-.5,0) rectangle (2.5,1);
\draw [circuit ee IEC] (0,.5) to [contact] (2,.5) ; 
\end{tikzpicture} 
&
\begin{tikzpicture}[blue]
\useasboundingbox (-.5,0) rectangle (2.5,1); 
\draw [circuit ee IEC] (0,.5) to [make contact]  (2,.5) ; 
\end{tikzpicture}
&
\begin{tikzpicture}[blue]
\useasboundingbox (-.5,0) rectangle (2.5,1);
\draw [circuit ee IEC] (0,.5) to [break contact] (2,.5) ; 
\end{tikzpicture}
&

\\   \hline 
[\RDD{contact}] &[\RDD{make contact}] & [\RDD{break contact}] & 
\\   \hline 
\end{tabular}

\bigskip


%\subsection{Alternate appearance}

\begin{tabular}{|c|c|c|} \hline
\multicolumn{3}{|c|}{\textbf{\TFRGB{Autre apparence}{Alternate appearance }}  }
\\ \hline   
\multicolumn{3}{|l|}{\BS{draw} [circuit ee IEC,\rouge{set resistor graphic=var resistor IEC graphic} ]  }\\
\multicolumn{3}{|l|}{(0,0.5) to [resistor] (2,0.5) ;  }\\ 
\hline 
\begin{tikzpicture}[blue]
\useasboundingbox  (-.5,0) rectangle (2.5,1); 
\draw [circuit ee IEC,set resistor graphic=var resistor IEC graphic ] (0,.5) to [resistor] (2,.5) ; 
\end{tikzpicture}
&
\begin{tikzpicture}[blue]
\useasboundingbox  (-.5,0) rectangle (2.5,1); 
\draw [circuit ee IEC,set inductor graphic=var inductor IEC graphic ] (0,.5) to [inductor] (2,.5) ; 
\end{tikzpicture} 
&  
\begin{tikzpicture}[blue]
\useasboundingbox (-.5,0) rectangle (2.5,1);
\draw [circuit ee IEC,set diode graphic=var diode IEC graphic ] (0,.5) to [diode] (2,.5) ; 
\end{tikzpicture}
\\   \hline 
resistor & inductor &  diode 
\\   \hline
\begin{tikzpicture}[blue]
\useasboundingbox  (-.5,0) rectangle (2.5,1); 
\draw [circuit ee IEC,set Zener diode graphic=var Zener diode IEC graphic ] (0,.5) to [Zener diode] (2,.5) ; 
\end{tikzpicture}
&
\begin{tikzpicture}[blue]
\useasboundingbox  (-.5,0) rectangle (2.5,1);
\draw [circuit ee IEC,set Schottky diode graphic=var Schottky diode IEC graphic ] (0,.5) to [Schottky diode] (2,.5) ; 
\end{tikzpicture}
&
\begin{tikzpicture}[blue]
\useasboundingbox  (-.5,0) rectangle (2.5,1);
\draw [circuit ee IEC,set tunnel diode graphic=var tunnel diode IEC graphic ] (0,.5) to [tunnel diode] (2,.5) ; 
\end{tikzpicture}

\\   \hline 
Zener diode & Schottky diode & tunnel diode 
\\   \hline
 
\begin{tikzpicture}[blue]
\useasboundingbox  (-.5,0) rectangle (2.5,1); 
\draw [circuit ee IEC,set backward diode graphic=var backward diode IEC graphic ] (0,.5) to [backward diode] (2,.5) ; 
\end{tikzpicture}
&
\begin{tikzpicture}[blue]
\useasboundingbox  (-.5,0) rectangle (2.5,1);
\draw [circuit ee IEC,set breakdown diode graphic=var breakdown diode IEC graphic ] (0,.5) to [breakdown diode] (2,.5) ; 
\end{tikzpicture}
&
\begin{tikzpicture}[blue]
\useasboundingbox  (-.5,0) rectangle (2.5,1);
\draw [circuit ee IEC,set make contact graphic=var make contact IEC graphic ] (0,.5) to [make contact] (2,.5) ; 
\end{tikzpicture}
\\   \hline 
backward diode & breakdown diode & make contact 
\\   \hline 
\end{tabular}


\bigskip

%\subsection{Symbol Size}
%\begin{center}
%\RRR{47-2-1}
%\end{center}


\begin{tabular}{|c|c|c|c|c|} \hline
\multicolumn{5}{|c|}{ \textbf{\TFRGB{Taille des symboles}{Symbol Size}}}\\
\multicolumn{5}{|c|}{\RRR{47-2-1}}
\\ \hline 
\multicolumn{5}{|c|}{\BS{draw}  [circuit ee IEC]  (0,0.5) to [diode,\RDD{large circuit symbols}] (2,0.5) ;  }\\ 
\hline 
\begin{tikzpicture}[blue]
\useasboundingbox  (0,0) rectangle (2,1); 
\draw [circuit ee IEC] (0,.5) to [diode,huge circuit symbols] (2,.5) ; 
\end{tikzpicture}
&
\begin{tikzpicture}[blue]
\useasboundingbox  (0,0) rectangle (2,1); 
\draw [circuit ee IEC](0,.5) to [diode,large circuit symbols] (2,.5) ; 
\end{tikzpicture}
&
\begin{tikzpicture}[blue]
\useasboundingbox  (0,0) rectangle (2,1); 
\draw [circuit ee IEC](0,.5) to [diode,medium circuit symbols] (2,.5) ; 
\end{tikzpicture}
&
\begin{tikzpicture}[blue]
\useasboundingbox  (0,0) rectangle (2,1); 
\draw [circuit ee IEC] (0,.5) to [diode,small circuit symbols] (2,.5) ; 
\end{tikzpicture}
&
\begin{tikzpicture}[blue]
\useasboundingbox  (0,0) rectangle (2,1); 
\draw [circuit ee IEC] (0,.5) to [diode,tiny circuit symbols] (2,.5) ; 
\end{tikzpicture}
\\   \hline
\RDD{huge circuit symbols} & \RDD{large circuit symbols} & \RDD{medium circuit symbols} & \RDD{small circuit symbols} & \RDD{tiny circuit symbols}
\\   \hline
(10pt) & (8pt) & (7pt)& (6pt)& (5pt)
\\   \hline
\end{tabular}


\bigskip\begin{tabular}{|c|c|c|} \hline 
\multicolumn{3}{|c|}{\BS{draw}  [circuit ee IEC,\RDD{circuit symbol unit}=14pt] (0,0.5) to [diode] (2,0.5) ;  }\\ 
\hline  
\begin{tikzpicture}[blue]
\useasboundingbox  (0,0) rectangle (2,1); 
\draw [circuit ee IEC,circuit symbol unit=14pt] (0,.5) to [diode] (2,.5) ;
\end{tikzpicture}
&  
\begin{tikzpicture}[blue]
\useasboundingbox  (0,0) rectangle (2,1); 
\draw [circuit ee IEC,{circuit symbol size=width 3 height 1}] (0,.5) to [diode] (2,.5) ;
\end{tikzpicture}
&  
\begin{tikzpicture}[blue]
\useasboundingbox  (0,0) rectangle (2,1); 
\node[circuit ee IEC,circuit symbol size=width 1 height 5]at (1,.5) [diode] {} ; 
\end{tikzpicture}
\\ \hline  
\RDD{circuit symbol unit}=14pt & \RDD{circuit symbol size}=width 3 height 1  & \RDD{circuit symbol size}=width 1 height 5 \\
&
\multicolumn{2}{|c|}{ \DW{} }
\\ \hline 
\end{tabular} 

%\subsection{Declaring New Symbols}
\bigskip

\begin{tabular}{|c|c|c|c|} \hline 
\multicolumn{4}{|c|}{\textbf{\TFRGB{Création de nouveaux symboles}{Declaring New Symbols }}  }
\\ % \hline
\multicolumn{4}{|c|}{\RRR{47-2-2}  }
\\ \hline  
\begin{tikzpicture}
[baseline=0pt,circuit declare symbol=xxx,
set xxx graphic={draw,shape=rectangle,minimum size=5mm}]
\useasboundingbox (0,0) rectangle (3.2,1);
\node [xxx] at (0.5,0.5) {};
\draw[circuit ee IEC]  (1,0.5) to [xxx] (3,0.5) ;
\end{tikzpicture}
& 
\multicolumn{3}{|c|}{ 
\parbox[B]{10cm}{
\BS{begin}\AC{tikzpicture}
[\RDD{circuit declare symbol}={\color{blue} xxx},  \\
\rouge{set xxx graphic}=\AC{draw,\rouge{shape}=rectangle,minimum size=5mm}]  \\
\BS{node} [{\color{blue} xxx}] at (.5,.5) {}; \\
\BS{draw}[circuit ee IEC]  (1,.5) to [{\color{blue} xxx}] (3,.5) ; \\
\BS{end}\AC{tikzpicture}} } 
\\ \hline  
  
\begin{tikzpicture}
[circuit declare symbol=xxx,
set xxx graphic={draw,shape=circle,minimum size=5mm}] 
\useasboundingbox (0,0) rectangle (3.2,1);
\node [xxx] at (.5,.5) {};
\draw[circuit ee IEC]  (1,.5) to [xxx] (3,.5) ;
\end{tikzpicture}
&  
\begin{tikzpicture}
[circuit declare symbol=xxx,
set xxx graphic={draw,shape=dart,minimum size=5mm}] 
\useasboundingbox (0,0) rectangle (3.2,1);
\node [xxx] at (.5,.5) {};
\draw[circuit ee IEC]  (1,.5) to [xxx] (3,.5) ;
\end{tikzpicture}
&  
\begin{tikzpicture}
[circuit declare symbol=xxx,
set xxx graphic={draw,shape=star,minimum size=5mm}] 
\useasboundingbox (0,0) rectangle (3.2,1);
\node [xxx] at (.5,.5) {};
\draw[circuit ee IEC]  (1,.5) to [xxx] (3,.5) ;
\end{tikzpicture}
&  
\begin{tikzpicture}
[circuit declare symbol=xxx,
set xxx graphic={draw,shape=forbidden sign,minimum size=5mm}] 
\useasboundingbox (0,0) rectangle (3.2,1);
\node [xxx] at (.5,.5) {};
\draw[circuit ee IEC]  (1,.5) to [xxx] (3,.5) ;
\end{tikzpicture}
\\ \hline 
\RDD{shape}=circle & \RDD{shape}=dart & \RDD{shape}=star &  \RDD{shape}=forbidden sign \\ 
\hline 
\multicolumn{4}{|c|}{ \textbf{voir les \og different shape libraries \fg }{see the different shape libraries } }
\\ \hline 
\end{tabular} 

\bigskip
%\subsection{placement}


\begin{tabular}{|c|} \hline 
\textbf{\TFRGB{Placement des symboles sur un chemin}{Placement of symbol on a path }} 
\\ \hline  
\BS{draw} [circuit ee IEC] (0,0.5) to  [contact=\AC{\RDD{at start}},make contact=\AC{\RDD{very near start}},voltage source=\AC{\RDD{near start}},\\ resistor,
bulb=\AC{\RDD{near end}},
bulb=\AC{\RDD{very near end}},contact=\AC{\RDD{at end}}] (12,0.5) ;
\\ \hline  
\begin{tikzpicture}[blue]
\useasboundingbox (-.5,0) rectangle (12.5,1);
\draw [circuit ee IEC] (0,.5) to  [contact={at start},make contact={very near start},voltage source={near start},resistor,
bulb={near end},bulb={very near end},contact={at end}] (12,.5) ; 
\end{tikzpicture}
\\ \hline 
%\end{tabular}
%
%\bigskip
%\begin{tabular}{|c|} \hline  
\BS{draw} [circuit ee IEC] (0,0.5) to  [contact=\{ \rouge{pos=0}{} \},make contact=\{\rouge{pos=0.2}{}\},voltage source=\{\rouge{pos=0.3}{} \},    \\ 
resistor=\{ \rouge{pos=0.5}{} \},
bulb=\{\rouge{pos=0.75} {} \},contact=\{\RDD{pos} \rouge{=1}{} \}] (12,0.5) ;
\\ \hline  
\begin{tikzpicture}[blue]
\useasboundingbox (-.5,0) rectangle (12.5,1);
\draw [circuit ee IEC] (0,.5) to  [contact={pos=0},make contact={pos=0.2},voltage source={pos=0.3},resistor={pos=0.5},
bulb={pos=0.75},contact={pos=1}] (12,.5) ; 
\end{tikzpicture}
\\ \hline 
\end{tabular}

%===================
%\subsection{Orientation}

\bigskip

\begin{tabular}{|c|c|c|c|} \hline
\multicolumn{4}{|c|}{\textbf{\TFRGB{Orientation des symboles}{Symbol orientation }}  } \\
\multicolumn{4}{|c|}{ \RRR{47-2-3} }
\\ \hline   
\multicolumn{4}{|c|}{\BS{node}  [circuit ee IEC] at (1,.5) [diode,\RDD{point up}] \AC{} ;  }\\ 
\hline 
\begin{tikzpicture}[blue]
\useasboundingbox  (-.5,0) rectangle (2.5,1);
\node[circuit ee IEC] at (1,.5) [diode,point up] {} ; 
\end{tikzpicture}
&
\begin{tikzpicture}[blue]
\useasboundingbox  (-.5,0) rectangle (2.5,1);
\node[circuit ee IEC] at (1,.5) [diode,point down] {} ; 
\end{tikzpicture}
&
\begin{tikzpicture}[blue]
\useasboundingbox  (-.5,0) rectangle (2.5,1);
\node[circuit ee IEC] at (1,.5) [diode,point left] {} ; 
\end{tikzpicture}
&
\begin{tikzpicture}[blue]
\useasboundingbox  (-.5,0) rectangle (2.5,1);
\node[circuit ee IEC] at (1,.5) [diode,point right] {} ; 
\end{tikzpicture}
\\   \hline 
[diode,\RDD{point up}] & 
[diode,\RDD{point down}] & [diode,\RDD{point left}] & [diode,\RDD{point right}]
\\   \hline
\end{tabular}

\bigskip
\begin{tabular}{|c|c|} \hline 
\multicolumn{2}{|c|}{\textbf{\TFRGB{Orientation automatique}{Automatic orientation }}  } 

\\ \hline    
\begin{tikzpicture}[baseline=0pt,blue]
\useasboundingbox (-.5,-1.5) rectangle (2.5,1.5);
\draw [circuit ee IEC] (0,0) to  [voltage source] (1,1) to [resistor] (2,0) to 
[bulb] (1,-1) to [diode]  (0,0) ; 
\end{tikzpicture} 
 &  
\parbox[c]{10cm}{\BS{draw} [circuit ee IEC] (0,0) \\ to  [voltage source] (1,1) \\ to [resistor] (2,0) \\to 
[bulb] (1,-1) \\to [diode]  (0,0) ;} 
 \\ \hline 
\end{tabular} 

\bigskip

%===================
%\subsection{Current Directions}
\subsection{Annotations}
%\begin{center}
%\RRR{47-4-2}
%\end{center}

\noindent

\begin{tabular}{|c|c|} \hline 
\multicolumn{2}{|c|}{ \textbf{\TFRGB{Sens du courant}{Indicating Current Directions}}  } \\
\multicolumn{2}{|c|}{\RRR{47-4-2} }
\\ \hline 
\multicolumn{2}{|c|}{\BS{draw}  [circuit ee IEC] (0,0.5) to [\RDD{current direction}]  (2,0.5) ;  }
\\ \hline 
\begin{tikzpicture}[blue]
\useasboundingbox  (-1,0) rectangle (3,1); 
\draw [circuit ee IEC] (0,.5) to[current direction] (2,.5); 
\end{tikzpicture}
& 
\begin{tikzpicture}[blue]
\useasboundingbox  (-1,0) rectangle (3,1);
\draw [circuit ee IEC] (0,.5) to[current direction'] (2,.5); 
\end{tikzpicture} 
\\   \hline
[\RDD{current direction}]  & [\RDD{current direction' } ]

\\   \hline

\end{tabular}

\bigskip

%\subsection{units}
%\begin{center}
%\RRR{47-4-6}
%\end{center}

 
\begin{tabular}{|c|c|c|c|c|} \hline
\multicolumn{5}{|c|}{\textbf{ \TFRGB{Unités disponibles}{Units available}}  } \\
\multicolumn{5}{|c|}{\RRR{47-4-6}   }
\\ \hline   
\multicolumn{5}{|c|}{\BS{node}  [draw,circuit ee IEC] at(1,.5) [\RDD{ampere}=5] \AC{}  }\\ 
\hline 
\begin{tikzpicture}[blue]
\draw[help lines,white] (-.5,0) rectangle (2.5,1);
\node [draw,circuit ee IEC] at(1,0.5) [ampere=5] {} ;  
\end{tikzpicture}
&
\begin{tikzpicture}[blue]
\draw[help lines,white] (-.5,0) rectangle (2.5,1);
\node [draw,circuit ee IEC] at(1,0.5) [volt=5] {} ;  
\end{tikzpicture}
&
\begin{tikzpicture}[blue]
\draw[help lines,white] (-.5,0) rectangle (2.5,1);
\node [draw,circuit ee IEC] at(1,0.5) [ohm=5] {} ;  
\end{tikzpicture}

&
\begin{tikzpicture}[blue]
\draw[help lines,white] (-.5,0) rectangle (2.5,1);
\node [draw,circuit ee IEC] at(1,.5) [siemens=5] {} ;  
\end{tikzpicture}
&
\begin{tikzpicture}[blue]
\draw[help lines,white] (-.5,0) rectangle (2.5,1); 
\node [draw,circuit ee IEC] at(1,.5) [henry=5] {} ;  
\end{tikzpicture}
\\   \hline
[\RDD{ampere}=5] & [\RDD{volt}=5] & [\RDD{ohm}=5]  \DW{} & [\RDD{siemens}=5] & [\RDD{henry}=5]
\\   \hline
\begin{tikzpicture}[blue]
\draw[help lines,white] (-.5,0) rectangle (2.5,1);
\node [draw,circuit ee IEC] at(1,.5) [farad=5] {} ;  
\end{tikzpicture}
&
\begin{tikzpicture}[blue]
\draw[help lines,white] (-.5,0) rectangle (2.5,1); 
\node [draw,circuit ee IEC] at(1,.5) [coulomb=5] {} ;  
\end{tikzpicture}
&
\begin{tikzpicture}[blue]
\draw[help lines,white] (-.5,0) rectangle (2.5,1);
\node [draw,circuit ee IEC] at(1,.5) [voltampere=5] {} ;  
\end{tikzpicture}
&
\begin{tikzpicture}[blue]
\draw[help lines,white] (-.5,0) rectangle (2.5,1); 
\node [draw,circuit ee IEC] at(1,.5) [watt=5] {} ;  
\end{tikzpicture}
&
\begin{tikzpicture}[blue]
\draw[help lines,white] (-.5,0) rectangle (2.5,1);
\node [draw,circuit ee IEC] at(1,.5) [hertz=5] {} ;  
\end{tikzpicture}
\\   \hline
[\RDD{farad}=5] & [\RDD{coulomb}=5] &  [\RDD{voltampere}=5] & [\RDD{watt}=5] & [\RDD{hertz}=5] 
\\   \hline
\begin{tikzpicture}[blue]
\draw[help lines,white] (-.5,0) rectangle (2.5,1); 
\node [draw,circuit ee IEC] at(1,.5)[ampere=5k]{} ;  
\end{tikzpicture}
&
\begin{tikzpicture}[blue]
\draw[help lines,white] (-.5,0) rectangle (2.5,1);
\node [draw,circuit ee IEC] at(1,.5) [ampere=5m] {} ;  
\end{tikzpicture}
&
\begin{tikzpicture}[blue]
\draw[help lines,white] (-.5,0) rectangle (2.5,1);
\node [draw,circuit ee IEC] at(1,.5) [ampere=5\mu] {} ;  
\end{tikzpicture}
&
\begin{tikzpicture}[blue]
\draw[help lines,white] (-.5,0) rectangle (2.5,1); 
\node [draw,circuit ee IEC] at(1,.5) [watt=5k] {} ;  
\end{tikzpicture}
&
\begin{tikzpicture}[blue]
\draw[help lines,white] (-.5,0) rectangle (2.5,1);
\node [draw,circuit ee IEC] at(1,.5) [watt=5M] {} ;  
\end{tikzpicture}
\\   \hline
[ampere=\rouge{5k}] & [ampere=\rouge{5m}] &  [ampere=\rouge{5\BS{mu}}] & [watt=\rouge{5k}] & [watt=\rouge{5M}] 
\\   \hline
\end{tabular}

\bigskip

%\subsection{créer sa propre unité }

\begin{tabular}{|c|} \hline 
\textbf{ \TFRGB{créer sa propre unité}{Declare unit}} \\  
\RRR{47-2-4}
\\ \hline   
\BS{tikz}[circuit ee IEC,\RDD{circuit declare unit}=\AC{{\color{blue} xxx}}\AC{ Unit}] \\
\BS{draw} (0,0) to[resistor=\AC{{\color{blue} xxx}' sloped=3}] (3,2) to [resistor=\AC{{\color{blue} xxx}= 10\BS{mu}}] (5,2) to [resistor=\AC{{\color{blue} xxx}= 10M}] (7,0);
\\ \hline  
%\begin{tikzpicture}[circuit ee IEC,circuit declare unit={XXX}{ Unit}]
\tikz[circuit ee IEC,circuit declare unit={XXX}{ Unit}]
\draw (0,0) to[resistor={XXX' sloped=3}] (3,2) to [resistor={XXX= 10\mu}] (5,2) to [resistor={XXX= 10M}] (7,0);
%\end{tikzpicture}
\\ \hline 
\end{tabular} 

\bigskip
%\subsection{Annotations}

%\begin{center}
%\RRR{47-4-7}
%\end{center}

\begin{tabular}{|c|c|c|c|} \hline 
\multicolumn{4}{|c|}{ \textbf{\TFRGB{Annotations}{Annotations}}  } \\
\multicolumn{4}{|c|}{\RRR{47-4-7}}
\\ \hline 
\multicolumn{4}{|c|}{\BS{draw}  [circuit ee IEC] (0,0.5) to [resistor=\RDD{light emitting}] (2,0.5) ;  }\\ 
\hline 
\begin{tikzpicture}[blue]
\useasboundingbox (-.5,0) rectangle (2.5,1); 
\draw [circuit ee IEC] (0,.5) to [resistor=light emitting] (2,.5) ; 
\end{tikzpicture} 
&
\begin{tikzpicture}[blue]
\useasboundingbox  (-.5,0) rectangle (2.5,1); 
\draw [circuit ee IEC] (0,.5) to [resistor=light dependent]  (2,.5) ; 
\end{tikzpicture} 
&
\begin{tikzpicture}[blue]
\useasboundingbox (-.5,0) rectangle (2.5,1); 
\draw [circuit ee IEC] (0,.5) to [resistor=direction info]  (2,.5) ;  
\end{tikzpicture}  
&
\begin{tikzpicture}[blue]
\useasboundingbox  (-.5,0) rectangle (2.5,1);
\draw [circuit ee IEC] (0,.5) to [resistor=adjustable]  (2,.5) ;  
\end{tikzpicture} 
\\   \hline 
[resistor=\RDD{light emitting}] & [resistor=\RDD{light dependent}]  & [resistor=\RDD{direction info}]  & [resistor=\RDD{adjustable}]
\\   \hline 
\begin{tikzpicture}[blue]
\useasboundingbox  (-.5,0) rectangle (2.5,1); 
\draw [circuit ee IEC] (0,.5) to [diode=light emitting] (2,.5) ; 
\end{tikzpicture} 
&
\begin{tikzpicture}[blue]
\useasboundingbox  (-.5,0) rectangle (2.5,1);
\draw [circuit ee IEC] (0,.5) to [diode=light dependent]  (2,.5) ; 
\end{tikzpicture} 
&
\begin{tikzpicture}[blue]
\useasboundingbox  (-.5,0) rectangle (2.5,1);
\draw [circuit ee IEC] (0,.5) to [diode=direction info]  (2,.5) ;  
\end{tikzpicture}  
&
\begin{tikzpicture}[blue]
\useasboundingbox  (-.5,0) rectangle (2.5,1);
\draw [circuit ee IEC] (0,.5) to [diode=adjustable]  (2,.5) ;  
\end{tikzpicture} 
\\   \hline 
[diode=\RDD{light emitting}] & [diode=\RDD{light dependent}]  & [diode=\RDD{direction info}]  & [diode=\RDD{adjustable}]
\\   \hline 
\begin{tikzpicture}[blue]
\useasboundingbox  (-.5,0) rectangle (2.5,1);
\draw [circuit ee IEC] (0,.5) to [diode=light emitting'] (2,.5) ; 
\end{tikzpicture} 
&
\begin{tikzpicture}[blue]
\useasboundingbox  (-.5,0) rectangle (2.5,1); 
\draw [circuit ee IEC] (0,.5) to [diode=light dependent']  (2,.5) ; 
\end{tikzpicture} 
&
\begin{tikzpicture}[blue]
\useasboundingbox  (-.5,0) rectangle (2.5,1);
\draw [circuit ee IEC] (0,.5) to [diode=direction info']  (2,.5) ;  
\end{tikzpicture}  
&
\begin{tikzpicture}[blue]
\draw[help lines,white] (-.5,0) rectangle (2.5,1);
\draw [circuit ee IEC] (0,.5) to [diode=adjustable']  (2,.5) ;  
\end{tikzpicture} 
\\   \hline 
[diode=\RDD{light emitting'}] & [diode=\RDD{light dependent'}]  & [diode=\RDD{direction info'}]  & [diode=\RDD{adjustable'}]
\\   \hline 
\end{tabular}



\bigskip

%\subsection{Position}
%
%\begin{center}
%\RRR{47-2-4}
%\end{center}

\begin{tabular}{|c|c|} \hline 
\multicolumn{2}{|c|}{ \textbf{\TFRGB{Position des unités}{Units position}}  } \\
\multicolumn{2}{|c|}{  \RRR{47-2-4}}
\\ \hline 
\multicolumn{2}{|c|}{\BS{draw} [circuit ee IEC] (0,0) to [capacitor=\AC{farad=5\BS{mu}}]  (2,2)  ;  }\\ 
\hline 
\begin{tikzpicture}[blue]
\useasboundingbox   (0,-.5) rectangle (2,2.5); 
\draw [circuit ee IEC] (0,0) to [capacitor={farad=5\mu}]  (2,2) ; 
\end{tikzpicture}
&
\begin{tikzpicture}[blue]
\useasboundingbox   (0,-.5) rectangle (2,2.5);  
\draw [circuit ee IEC] (0,0) to [capacitor={farad'=5\mu}]  (2,2) ; 
\end{tikzpicture}
\\ \hline
[capacitor=\rouge{\AC{farad=5\BS{mu}}}] & [capacitor=\rouge{\AC{farad'=5\BS{mu}}}]
\\ \hline
\begin{tikzpicture}[blue]
\useasboundingbox   (0,-.5) rectangle (2,2.5); 
\draw [circuit ee IEC] (0,0) to [capacitor={farad sloped=5\mu}]  (2,2) ; 
\end{tikzpicture}
&
\begin{tikzpicture}[blue]
\useasboundingbox   (0,-.5) rectangle (2,2.5); 
\draw [circuit ee IEC] (0,0) to [capacitor={farad' sloped=5\mu}]  (2,2) ; 
\end{tikzpicture}
\\   \hline
[capacitor=\rouge{\AC{farad sloped=5\BS{mu}}}] & [capacitor=\rouge{\AC{farad' sloped=5\BS{mu}}}]
\\   \hline
\end{tabular}

\bigskip
%\subsection{Info Labels}

\begin{tabular}{|c|c|c|} \hline 
\multicolumn{3}{|c|}{ \textbf{ \TFRGB{Informations}{Info Labels}}  }\\ 
\multicolumn{3}{|c|}{ \RRR{47-2-4}  }\\ 
\hline 

\multicolumn{3}{|c|}{\BS{draw} [circuit ee IEC] (0,0.5) to [diode=\AC{light emitting=\AC{\RDD{info}=D1}}] (2,0.5) ;  }\\ 
\hline 

\begin{tikzpicture}[blue]
\useasboundingbox   (0,-1) rectangle (2,2); 
\draw [circuit ee IEC] (0,.5) to [diode={light emitting={info=D1}} ]  (2,.5) ; 
\end{tikzpicture}
&
\begin{tikzpicture}[blue]
\useasboundingbox   (0,-1) rectangle (2,2); 
\draw [circuit ee IEC] (0,.5) to [diode={light emitting={info'=D2}}]  (2,.5) ; 
\end{tikzpicture}
&
\begin{tikzpicture}[blue]
\useasboundingbox   (0,-1) rectangle (2,2);
\draw [circuit ee IEC] (0,.5) to [diode={light emitting,info'=D3}] (2,.5) ; 
\end{tikzpicture}
\\   \hline 
[diode=\AC{light emitting=\AC{\RDD{info}=D1}} ] & 
[diode=\AC{light emitting=\AC{\RDD{info'}=D2}} ] &
[diode=\AC{light emitting,\RDD{info'}=D3}]
\\   \hline
\end{tabular}

\bigskip

\begin{tabular}{|c|c|} \hline  
\TFRGB{sur un noeud}{On a node} & \TFRGB{sur un chemin}{On a path}

\\ \hline  
\begin{tikzpicture}[blue]
\node[circuit ee IEC] at (1,.5) [resistor,info=$3\Omega$,info'=R1] {} ; 
\end{tikzpicture}
&  
\begin{tikzpicture}[blue]
\useasboundingbox (-.2,-.2) grid (2.2,1.2);
\draw [circuit ee IEC] (0,.5) to [resistor={info=$3\Omega$,info'=R1}] (2,.5) ; 
\end{tikzpicture}
\\ \hline 
[resistor,\RDD{info}=\$3\BS{Omega}\$,\RDD{info'}=R1] &[resistor=\AC{\RDD{info}=\$3\BS{Omega}\$,\RDD{info'}=R1}] 
\\ \hline 
\end{tabular}

\bigskip

\begin{tabular}{|c|c|} \hline  
\begin{tikzpicture}[blue,circuit ee IEC]
\node [resistor,info=center:$3\Omega$] {};
\end{tikzpicture}
&  
\begin{tikzpicture}[blue,circuit ee IEC]
\node [resistor,point up,info=center:$3\Omega$] {};
%\node [resistor,point up,info=center:$R_1$] ;
\end{tikzpicture}
\\ \hline [resistor,point up,info=\RDD{center}:\$3\BS{Omega}\$] & 
[resistor,point up,info=\RDD{center}:\$3\BS{Omega}\$] \\ 
\hline 
\end{tabular} 


\bigskip



\begin{tabular}{|c|c||c|c|}  \hline 
\multicolumn{2}{|c||}{ \BS{node}  [voltage source,\RDD{direction info}=\AC{volt=10}] \AC{}  } & \multicolumn{2}{|c|}{ \BS{node}  [voltage source,\RDD{direction info'}=\AC{volt=10}] \AC{}}
\\ \hline 
\begin{tikzpicture}[blue,circuit ee IEC]
\useasboundingbox (-2,-1.2) grid (2,1.2);
\node  [voltage source,direction info={volt=10}] {};
\end{tikzpicture}
&  
\begin{tikzpicture}[blue,circuit ee IEC]
\useasboundingbox (-2,-1.2) grid (2,1.2);
\node  [voltage source,direction info={->,volt'=10}] {};
\end{tikzpicture}
&  
\begin{tikzpicture}[blue,circuit ee IEC]
\useasboundingbox (-2,-1.2) grid (2,1.2);
\node [voltage source,direction info'={volt=10}] {};
\end{tikzpicture}
&  
\begin{tikzpicture}[blue,circuit ee IEC]
\useasboundingbox (-2,-1.2) grid (2,1.2);
\node [voltage source,direction info'={volt'=10}] {};
\end{tikzpicture}
\\ \hline 
\AC{volt=10} &   \AC{volt'=10} & \AC{volt=10}& \AC{volt'=10}  \\
\TFRGB{ou}{or} \AC{->,volt=10} &  \TFRGB{ou}{or}  \AC{->,volt'=10} &\TFRGB{ou}{or} \AC{->,volt=10} & \TFRGB{ou}{or}  \AC{->,volt'=10} 
\\ \hline 
\begin{tikzpicture}[blue,circuit ee IEC]
\useasboundingbox (-2,-1.2) grid (2,1.2);
\node [voltage source,direction info={<-,volt=10}] {};
\end{tikzpicture} 
&  
\begin{tikzpicture}[blue,circuit ee IEC]
\useasboundingbox (-2,-1.2) grid (2,1.2);
\node   [voltage source,direction info'={<-,volt=10}] {};
\end{tikzpicture}
&  
\begin{tikzpicture}[blue,circuit ee IEC]
\useasboundingbox (-2,-1.2) grid (2,1.2);
\node   [voltage source,direction info'={<-,volt=10}] {};
\end{tikzpicture}
&  
\begin{tikzpicture}[blue,circuit ee IEC]
\useasboundingbox (-2,-1.2) grid (2,1.2);
\node   [voltage source,direction info'={<-,volt'=10}] {};
\end{tikzpicture}
\\ \hline 
\AC{<-,volt=10} & \AC{<-,volt=10} & \AC{<-,volt=10} & \AC{<-,volt'=10}\\ 
\hline 
\end{tabular} 


%===================
\bigskip

\begin{tabular}{|c|c|}\hline 
\multicolumn{2}{|c|}{  \textbf{\TFRGB{Créer sa propre annotation}{Declare annotation}} } \\
\multicolumn{2}{|c|}{ \RRR{47-2-5} }
\\ \hline  
\tikzset{circuit declare annotation={XXX}{9pt}
{ (-0.5cm,0.5cm) edge[to path={- -(0pt,2pt) - - (8pt,8pt)}] ()} }

\tikz[blue,circuit ee IEC]
\draw (0,0) to [resistor={XXX}] (3,0);
&  
\parbox[b]{12cm}{
\BS{tikzset}\AC{circuit \RDD{declare annotation}=\AC{\blll{XXX}}\AC{9pt} \\
\hspace{1cm}  \AC{ (-0.5cm,0.5cm) edge[to path=\AC{- -(0pt,2pt) - - (8pt,8pt)}] ()} }  \\
\BS{tikz}[blue,circuit ee IEC]
\BS{draw} (0,0) to [resistor={\blll{XXX}}] (3,0);
}
\\ \hline  
\tikzset{circuit declare annotation={xxx}{9pt}
{ (-.5cm,.5cm) edge[to path={--(0pt,2pt) -- (8pt,8pt)}] ()} }

\tikz[circuit ee IEC]
\draw (0,0) to [resistor={xxx={info=abc}}] (3,0);
&  
\parbox[b]{12cm}{
\BS{tikzset}\AC{circuit declare annotation=\AC{xxx}\{ \rouge{9pt} \} \}\\
\hspace{1cm}  \AC{ (-0.5cm,0.5cm) edge[to path=\AC{- -(0pt,2pt) - - (8pt,8pt)}] ()} }  \\
\BS{tikz}[blue,circuit ee IEC]
\BS{draw} (0,0) to [resistor=\AC{xxx\blll{=\AC{info=abc}}}] (3,0);
}
\\ \hline 
\tikzset{circuit declare annotation={xxx}{1cm}
{ (-.5,.5) edge[to path={--(0pt,2pt) -- (8pt,8pt)}] ()} }


\tikz[circuit ee IEC]
\draw (0,0) to [resistor={xxx={info=abc}}] (3,0);
&  
\parbox[b]{12cm}{
\BS{tikzset}\{circuit declare annotation=\AC{xxx}\{\rouge{1cm} \} \} \\
\hspace{1cm}  \AC{ (-0.5,0.5) edge[to path=\AC{- -(0pt,2pt) - - (8pt,8pt)}] ()} \}  \\
\BS{tikz}[blue,circuit ee IEC]
\BS{draw} (0,0) to [resistor=\AC{xxx\blll{=\AC{info=abc}}}] (3,0);
}
\\ \hline 
\end{tabular}


\bigskip
%47.2.6 Theming Symbols

\begin{tabular}{|c|}\hline 
\textbf{ \TFRGB{Style des symboles}{Theming Symbols
}}\\
 \RRR{47-2-6} 
 
\\  \hline 
\BS{draw}[\RDD{circuit symbol lines/.style}=\AC{draw,red,very thick}] (0,0) \\to [capacitor=\AC{near start},resistor,
make contact=\AC{near end}] (5,0);
\\ \hline  
\begin{tikzpicture}[blue,circuit ee IEC]
\useasboundingbox  (-1,-1) rectangle (6,1);
\draw[circuit symbol lines/.style={draw,red,very thick}] (0,0) to [capacitor={near start},resistor,
make contact={near end}] (5,0);
\end{tikzpicture}
 \\ \hline 

\hline  
\BS{draw}[\RDD{circuit symbol wires/.style}=\AC{draw,red,very thick}] (0,0) \\to [capacitor=\AC{near start},resistor,
make contact=\AC{near end}] (5,0);
\\ \hline  
\begin{tikzpicture}[blue,circuit ee IEC]
\useasboundingbox  (-1,-1) rectangle (6,1);
\draw[circuit symbol wires/.style={draw,red,very thick}] (0,0) to [capacitor={near start},resistor,
make contact={near end}] (5,0);
\end{tikzpicture}
 \\ \hline 

\hline  
\BS{draw}[\RDD{circuit symbol open/.style}=\AC{thick,draw,red,fill=yellow}] (0,0) \\to [capacitor=\AC{near start},resistor,
make contact=\AC{near end}] (5,0);
\\ \hline  
\begin{tikzpicture}[blue,circuit ee IEC]
\useasboundingbox  (-1,-1) rectangle (6,1);
\draw[circuit symbol open/.style={thick,draw,red,fill=yellow}] (0,0) to [capacitor={near start},resistor,
make contact={near end}] (5,0);
\end{tikzpicture}
 \\ \hline 
\end{tabular}

\bigskip

\begin{tabular}{|c|c|} \hline 
\multicolumn{2}{|l|}{\BS{tikz}[blue,circuit ee IEC,\RDD{every info/.style}=red] }\\
\multicolumn{2}{|l|}{\BS{draw} (0,0) to[resistor=\AC{info=\AC{\$3\BS{Omega}\$},info'=\AC{\$R\_1\$}}] (3,0) }\\
\multicolumn{2}{|l|}{to[resistor=\AC{info=\{\$4\BS{Omega}\$\},info'=\AC{\$R\_2\$}}] (3,2); }\\ 
\hline  
 
\tikz[blue,circuit ee IEC,every info/.style=red]
\draw (0,0) to[resistor={info={$3\Omega$},info'={$R_1$}}] (3,0)
to[resistor={info={$4\Omega$},info'={$R_2$}}] (3,2);

&  
\tikz[blue,circuit ee IEC,every info/.style={font=\tiny}]
\draw (0,0) to[resistor={info={$3\Omega$},info'={$R_1$}}] (3,0)
to[resistor={info={$4\Omega$},info'={$R_2$}}] (3,2);
\\ \hline  
\RDD{every info/.style}=red
&  
\RDD{every info/.style}=\AC{font=\BS{tiny}}
\\ \hline 
\end{tabular} 





\bigskip
%\subsection{Example}

\SbSSCT{Exemple}{Example}

\begin{tabular}{|c|c|} \hline
\multicolumn{2}{|c|}{ \textbf{\TFRGB{3 méthodes pour le même schéma}{3 methods for the same circuit}}  }
\\ \hline 
\begin{tikzpicture}[blue,circuit ee IEC,baseline=0pt]
\useasboundingbox (-1,-1) rectangle (3,3); 
\draw (0,0) to [voltage source={direction info={->,volt=10}}]  (0,2)   to [resistor={info=center:$3 k\Omega$}] (2,2)   to [diode=light emitting]  ( 2,0)  to [make contact]  (0,0);
\end{tikzpicture}
&
\parbox[b]{10cm}{
\BS{begin}\AC{tikzpicture}[blue,circuit ee IEC] \\ 
\BS{draw} (0,0) \\
to [voltage source=\AC{direction info=\AC{->,volt=10}}]  (0,2) \\ 
to [resistor=\AC{info=center:\$3 k\BS{Omega}\$}] (2,2) \\ 
to [diode=light emitting]  ( 2,0) \\ 
to [make contact]  (0,0); \\
\BS{end}\AC{tikzpicture}

}
\\   \hline
\begin{tikzpicture}[blue,circuit ee IEC,baseline=0pt]
\useasboundingbox  (-1,-1) rectangle (3,3); 
\draw (0,0) to [voltage source={direction info={->,volt=10}}]  ++(up:2)   
to [resistor={info=center:$ 3 k\Omega$}] ++(right:2)  
 to [diode=light emitting]  ++(down:2)  
 to [make contact]  ++(left:2) ;
%\node at (0,1) [volt=10,point up] {};
%\node at (1,2) [ohm=10k] {}; 
\end{tikzpicture}
&
\parbox[b]{10cm}{
\BS{begin}\AC{tikzpicture}[blue,circuit ee IEC] \\ 
\BS{draw} (0,0) to [voltage source=\AC{direction info=\AC{->,volt=10}}] ++(up:2) \\ 
to [resistor=\AC{info=center:\$ 3 k\BS{Omega}\$}] ++(right:2) \\ 
to [diode=light emitting]  ++(down:2) \\ 
to [make contact]  ++(left:2) ; \\
%\BS{node} at (0,1) [volt=10,point up] \AC{}; \\
%\BS{node} at (1,2) [ohm=10k] \AC{}; \\
\BS{end}\AC{tikzpicture}

}
\\   \hline
\begin{tikzpicture}[blue,circuit ee IEC]
\useasboundingbox  (-1,-1) rectangle (3,3); 
\node (A) at (0,1) [voltage source,point up,volt=10]{} ;
\node  (B) at (1,2) [resistor,ohm=10k] {};
\node(C) at (2,1)  [diode=light emitting,point down] {} ;
\node (D) at  ( 1,0)   [make contact]  {}; 
\draw (A) |- (B) -| (C) |- (D) -|  (A); 
\end{tikzpicture}
&
\parbox[b]{10cm}{
\BS{begin}\AC{tikzpicture}[blue,circuit ee IEC] \\
\BS{node} (A) at (0,1) [voltage source,point up,volt=10]\AC{}; \\
\BS{node} (B) at (1,2) [resistor,ohm=10k] \AC{}; \\
\BS{node} (C) at (2,1)  [diode=light emitting,point down] \AC{} ; \\
\BS{node} (D) at  ( 1,0)   [make contact]  \AC{}; \\
\BS{draw} (A) |- (B) -| (C) |- (D) -|  (A); \\
\BS{end}\AC{tikzpicture}
}
\\   \hline
\end{tabular}
 












\SSCT{Les circuits logiques }{Logical circuits}

\label{lib-gate}
International Electrotechnical Commission :
\maboite{\BS{usepackage}\AC{circuits.logic.IEC}}

American logic gates  :
\maboite{\BS{usepackage}\AC{circuits.logic.US}}

logic symbols used in A. Croft, R. Davidson, and M.
Hargreaves (1992), Engineering Mathematics, Addison-Wesley, 82–95 :

\maboite{\BS{usepackage}\AC{circuits.logic.CDH}}

%\usetikzlibrary{circuits.logic.IEC}
%\usetikzlibrary[circuits.logic.US]
%\usetikzlibrary{circuits.logic.CDH}

\begin{tabular}{|c|c|c|} \hline
\multicolumn{3}{|c|}{\textbf{\TFRGB{Composants de base}{Basic Elements}}  }\\ 
\hline  
\multicolumn{3}{|c|}{\BS{node}  [\blll{circuit logic IEC}]  at (1,.5) [\RDD{and gate} ] \AC{A} ;   }\\ 
%\hline
\multicolumn{3}{|c|}{\RRR{47-3-2} }
\\ \hline 
\begin{tikzpicture}[blue]
\useasboundingbox (-.2,-.2) grid (2.2,1.2);
\node[circuit logic IEC]  at (1,.5) [and gate] {} ; 
\end{tikzpicture}  
&
\begin{tikzpicture}[blue]
\useasboundingbox (-.2,-.2) grid (2.2,1.2);
\node[circuit logic US]  at (1,.5) [and gate] {} ; 
\end{tikzpicture} 
&
\begin{tikzpicture}[blue]
\useasboundingbox (-.2,-.2) grid (2.2,1.2);
\node[circuit logic CDH]  at (1,.5) [and gate] {} ; 
\end{tikzpicture} 
\\ \hline 
[\blll{circuit logic IEC}] & [\blll{circuit logic US}]  & [\blll{circuit logic CDH}] \\

\RDD{and gate}& \RDD{and gate} & \RDD{and gate} 
\\ \hline 

\begin{tikzpicture}[blue]
\useasboundingbox (-.2,-.2) grid (2.2,1.2);
\node[circuit logic IEC]  at (1,.5) [nand gate] {} ; 
\end{tikzpicture} 
&
\begin{tikzpicture}[blue]
\useasboundingbox (-.2,-.2) grid (2.2,1.2);
\node[circuit logic US]  at (1,.5) [nand gate] {} ; 
\end{tikzpicture} 
&
\begin{tikzpicture}[blue]
\useasboundingbox (-.2,-.2) grid (2.2,1.2);
\node[circuit logic CDH]  at (1,.5) [nand gate] {} ;  
\end{tikzpicture} 
\\   \hline 
[\blll{circuit logic IEC}] & [\blll{circuit logic US}]  & [\blll{circuit logic CDH}] \\
\RDD{nand gate}& \RDD{nand gate} & \RDD{nand gate} 

\\ \hline 
\begin{tikzpicture}[blue]
\useasboundingbox (-.2,-.2) grid (2.2,1.2);
\node[circuit logic IEC]  at (1,.5) [or gate] {} ;
\end{tikzpicture} 
&
\begin{tikzpicture}[blue]
\useasboundingbox (-.2,-.2) grid (2.2,1.2);
\node[circuit logic US]  at (1,.5) [or gate] {} ;
\end{tikzpicture} 
&
\begin{tikzpicture}[blue]
\useasboundingbox (-.2,-.2) grid (2.2,1.2);
\node[circuit logic CDH]  at (1,.5) [or gate] {} ; 
\end{tikzpicture} 
\\   \hline
[\blll{circuit logic IEC}] & [\blll{circuit logic US}]  & [\blll{circuit logic CDH}] \\
\RDD{or gate}& \RDD{or gate} & \RDD{or gate} 

\\ \hline  
\begin{tikzpicture}[blue]
\useasboundingbox (-.2,-.2) grid (2.2,1.2);
\node[circuit logic IEC]  at (1,.5) [nor gate] {} ; 
\end{tikzpicture} 
&
\begin{tikzpicture}[blue]
\useasboundingbox (-.2,-.2) grid (2.2,1.2);
\node[circuit logic US]  at (1,.5) [nor gate] {} ; 
\end{tikzpicture} 
&
\begin{tikzpicture}[blue]
\useasboundingbox (-.2,-.2) grid (2.2,1.2);
\node[circuit logic CDH]  at (1,.5) [nor gate] {} ; 
\end{tikzpicture} 
\\   \hline
[\blll{circuit logic IEC}] & [\blll{circuit logic US}]  & [\blll{circuit logic CDH}] \\
\RDD{nor gate}& \RDD{nor gate} & \RDD{nor gate} 

\\ \hline 
\begin{tikzpicture}[blue]
\useasboundingbox (-.2,-.2) grid (2.2,1.2);
\node[circuit logic IEC]  at (1,.5) [xor gate] {} ; 
\end{tikzpicture} 
&
\begin{tikzpicture}[blue]
\useasboundingbox (-.2,-.2) grid (2.2,1.2);
\node[circuit logic US]  at (1,.5) [xor gate] {} ; 
\end{tikzpicture} 
&
\begin{tikzpicture}[blue]
\useasboundingbox (-.2,-.2) grid (2.2,1.2);
\node[circuit logic CDH]  at (1,.5) [xor gate] {} ;
\end{tikzpicture} 
\\   \hline
[\blll{circuit logic IEC}] & [\blll{circuit logic US}]  & [\blll{circuit logic CDH}] \\
\RDD{xor gate}& \RDD{xor gate} & \RDD{xor gate} 

\\ \hline 
\begin{tikzpicture}[blue]
\useasboundingbox (-.2,-.2) grid (2.2,1.2);
\node[circuit logic IEC]  at (1,.5) [xnor gate] {} ; 
\end{tikzpicture} 
&
\begin{tikzpicture}[blue]
\useasboundingbox (-.2,-.2) grid (2.2,1.2);
\node[circuit logic US]  at (1,.5) [xnor gate] {} ;
\end{tikzpicture} 
&
\begin{tikzpicture}[blue]
\useasboundingbox (-.2,-.2) grid (2.2,1.2);
\node[circuit logic CDH]  at (1,.5) [xnor gate] {} ; 
\end{tikzpicture} 
\\   \hline
[\blll{circuit logic IEC}] & [\blll{circuit logic US}]  & [\blll{circuit logic CDH}] \\
\RDD{xnor gate}& \RDD{xnor gate} & \RDD{xnor gate} 

\\ \hline 
\begin{tikzpicture}[blue]
\useasboundingbox (-.2,-.2) grid (2.2,1.2);
\node[circuit logic IEC]  at (1,.5) [not gate] {} ; 
\end{tikzpicture} 
&
\begin{tikzpicture}[blue]
\useasboundingbox (-.2,-.2) grid (2.2,1.2);
\node[circuit logic US]  at (1,.5) [not gate] {} ;
\end{tikzpicture} 
&
\begin{tikzpicture}[blue]
\useasboundingbox (-.2,-.2) grid (2.2,1.2);
\node[circuit logic CDH]  at (1,.5) [not gate] {} ;
\end{tikzpicture} 
\\   \hline
[\blll{circuit logic IEC}] & [\blll{circuit logic US}]  & [\blll{circuit logic CDH}] \\
\RDD{not gate}& \RDD{not gate} & \RDD{not gate} 

\\ \hline 
\begin{tikzpicture}[blue]
\useasboundingbox (-.2,-.2) grid (2.2,1.2);
\node[circuit logic IEC]  at (1,.5) [buffer gate] {} ;
\end{tikzpicture} 
&
\begin{tikzpicture}[blue]
\useasboundingbox (-.2,-.2) grid (2.2,1.2);;
\node[circuit logic US]  at (1,.5) [buffer gate] {} ; 
\end{tikzpicture} 
&
\begin{tikzpicture}[blue]
\useasboundingbox (-.2,-.2) grid (2.2,1.2);
\node[circuit logic CDH]  at (1,.5) [buffer gate] {} ;
\end{tikzpicture} 
\\   \hline
[\blll{circuit logic IEC}] & [\blll{circuit logic US}]  & [\blll{circuit logic CDH}] \\
\RDD{buffer gate}& \RDD{buffer gate} & \RDD{buffer gate} 

\\ \hline 
\end{tabular}
\bigskip

\begin{tabular}{|c|c|c|} \hline
\multicolumn{3}{|c|}{\textbf{\TFRGB{Avec etiquette}{Labelled}}  }\\ 
\hline  
\multicolumn{3}{|c|}{\BS{node}  [\blll{circuit logic IEC}]  at (1,.5) [and gate] \rouge{\AC{A}} ;   }\\ 
%\hline
\multicolumn{3}{|c|}{\RRR{47-3-1} }
\\ \hline 
\begin{tikzpicture}[blue]
\useasboundingbox (-.2,-.2) grid (2.2,1.2);
\node[circuit logic IEC]  at (1,.5) [and gate] {$A$} ; 
\end{tikzpicture}  
&
\begin{tikzpicture}[blue]
\useasboundingbox (-.2,-.2) grid (2.2,1.2);
\node[circuit logic US]  at (1,.5) [and gate] {$A$} ; 
\end{tikzpicture} 
&
\begin{tikzpicture}[blue]
\useasboundingbox (-.2,-.2) grid (2.2,1.2);
\node[circuit logic CDH]  at (1,.5) [and gate] {$A$} ; 
\end{tikzpicture} 
\\ \hline 
[\blll{circuit logic IEC}] & [\blll{circuit logic US}]  & [\blll{circuit logic CDH}] 
\\ \hline 
\end{tabular}
\bigskip

\begin{tabular}{|c|c|c|} \hline
\multicolumn{3}{|c|}{\textbf{\TFRGB{Orientation}{Orientation}  }}
\\  \hline 
\multicolumn{3}{|c|}{\RRR{47-3-1} }
\\ \hline 
\multicolumn{3}{|c|}{\BS{node}  [\blll{circuit logic IEC}]  at (1,.5) [and gate,\RDD{point down}] \AC{A} ;   }
\\ \hline 
\begin{tikzpicture}[blue]
\useasboundingbox (-.2,-.2) grid (2.2,1.2);
\node[circuit logic IEC]  at (1,.5) [and gate,point down] {$A$} ; 
\end{tikzpicture}  
&
\begin{tikzpicture}[blue]
\useasboundingbox (-.2,-.2) grid (2.2,1.2);
\node[circuit logic US]  at (1,.5) [and gate,point down] {$A$} ; 
\end{tikzpicture} 
&
\begin{tikzpicture}[blue]
\useasboundingbox (-.2,-.2) grid (2.2,1.2);
\node[circuit logic CDH]  at (1,.5) [and gate,point down] {$A$} ; 
\end{tikzpicture} 
\\ \hline 
[\blll{circuit logic IEC}] & [\blll{circuit logic US}]  & [\blll{circuit logic CDH}] 
\\ \hline  \hline 

\multicolumn{3}{|c|}{\BS{node}  [\blll{circuit logic IEC}]  at (1,.5) [and gate,\RDD{point up}] \AC{A} ;   }
\\ \hline 
\begin{tikzpicture}[blue]
\useasboundingbox (-.2,-.2) grid (2.2,1.2);
\node[circuit logic IEC]  at (1,.5) [and gate,point up] {$A$} ; 
\end{tikzpicture}  
&
\begin{tikzpicture}[blue]
\useasboundingbox (-.2,-.2) grid (2.2,1.2);
\node[circuit logic US]  at (1,.5) [and gate,point up] {$A$} ; 
\end{tikzpicture} 
&
\begin{tikzpicture}[blue]
\useasboundingbox (-.2,-.2) grid (2.2,1.2);
\node[circuit logic CDH]  at (1,.5) [and gate,point up] {$A$} ; 
\end{tikzpicture} 
\\ \hline 
[\blll{circuit logic IEC}] & [\blll{circuit logic US}]  & [\blll{circuit logic CDH}] 
\\ \hline 
\multicolumn{3}{|c|}{\BS{node}  [\blll{circuit logic IEC}]  at (1,.5) [and gate,\RDD{point left}] \AC{A} ;   }
\\ \hline 
\begin{tikzpicture}[blue]
\useasboundingbox (-.2,-.2) grid (2.2,1.2);
\node[circuit logic IEC]  at (1,.5) [and gate,point left] {$A$} ; 
\end{tikzpicture}  
&
\begin{tikzpicture}[blue]
\useasboundingbox (-.2,-.2) grid (2.2,1.2);
\node[circuit logic US]  at (1,.5) [and gate,point left] {$A$} ; 
\end{tikzpicture} 
&
\begin{tikzpicture}[blue]
\useasboundingbox (-.2,-.2) grid (2.2,1.2);
\node[circuit logic CDH]  at (1,.5) [and gate,point left] {$A$} ; 
\end{tikzpicture} 
\\ \hline 
[\blll{circuit logic IEC}] & [\blll{circuit logic US}]  & [\blll{circuit logic CDH}] 
\\ \hline 
\end{tabular}
\bigskip

\begin{tabular}{|c|c|}\hline 
\multicolumn{2}{|c|}{\textbf{\TFRGB{Entrées sortie}{inputs exit}}}  \\ \hline 
\multicolumn{2}{|c|}{ \RRR{47-3-3} }
\\ \hline
\begin{tikzpicture}[blue]
\useasboundingbox (-.2,-.2) grid (2.2,1.2);
\node[and gate IEC, draw, logic gate inputs={inverted, normal, inverted}] at (1,.5)  (A) {};
\draw [red] (A.input 1) -| (0,0.5);
\draw[green] (A.input 2) -| (0,0.5);
\draw[cyan] (A.input 3) -| (0,0.5);
\draw (A.output) -| (2,0.5);
\end{tikzpicture}
&  
\parbox[c]{10cm}{
\BS{node} [and gate IEC, draw, \\
\RDD{logic gate inputs}=\{\RDDX{inverted}{gate},\RDDX{normal}{gate} , \RDDX{inverted}{gate}\}] at (1,.5)  (A) \AC{}; \\
\BS{draw} [red] (A.\DDD{input} 1) -| (0,0.5); \\
\BS{draw}[green] (A.\DDD{input} 2) -| (0,0.5); \\
\BS{draw}[cyan] (A.\DDD{input} 3) -| (0,0.5); \\ 
\BS{draw} (A.\DDD{output}) -| (2,0.5);}
\\ \hline  
\begin{tikzpicture}[blue]
\useasboundingbox (-.2,-.2) grid (2.2,1.2);
\node[and gate IEC, draw, logic gate inputs={ini}] at (1,.5)  (A) {};
\draw[red] (A.input 1) -| (0,0.5);
\draw[green] (A.input 2) -| (0,0.5);
\draw[cyan] (A.input 3) -| (0,0.5);
\draw (A.output) -| (2,0.5);
\end{tikzpicture}
&  
\parbox[c]{10cm}{
\BS{node} [and gate IEC, draw, \\
\RDD{logic gate inputs}=\AC{\rouge{ini}}] at (1,.5)  (A) \AC{}; \\
\BS{draw} [red] (A.\DDD{input} 1) -| (0,0.5); \\
\BS{draw}[green] (A.\DDD{input} 2) -| (0,0.5); \\
\BS{draw}[cyan] (A.\DDD{input} 3) -| (0,0.5); \\ 
\BS{draw} (A.\DDD{output}) -| (2,0.5);}
\\  \hline 
\end{tabular}

\bigskip
 
\begin{tabular}{|c|c|}\hline 
\multicolumn{2}{|c|}{\textbf{\TFRGB{Paramètres des entrées}{input parameter}}}  \\ \hline 
\multicolumn{2}{|c|}{\BS{node} [and gate IEC, draw, 
logic gate inputs={ini},\RDD{logic gate inverted radius}=4pt ]} \\
\multicolumn{2}{|c|}{ at (1,.5)  (A) \AC{}; }\\ 
\multicolumn{2}{|c|}{ \RRR{47-3-3} }

\\ \hline 
\begin{tikzpicture}[blue]
\useasboundingbox (-.2,-1.2) grid (2.2,2.2);
\node[and gate IEC, draw, logic gate inputs={ini},logic gate inverted radius=4pt] at (1,.5)  (A) {};
\draw (A.input 1) -| (0,0.5);
\draw (A.input 2) -| (0,0.5);
\draw (A.input 3) -| (0,0.5);
\draw (A.output) -| (2,0.5);
\end{tikzpicture}
&  
\begin{tikzpicture}[blue]
\useasboundingbox (-.2,-1.2) grid (2.2,2.2);
\node[and gate IEC, draw, logic gate inputs={ini},logic gate input sep=0.5cm] at (1,.5)  (A) {};
\draw (A.input 1) -| (0,0.5);
\draw (A.input 2) -| (0,0.5);
\draw (A.input 3) -| (0,0.5);
\draw (A.output) -| (2,0.5);
\end{tikzpicture}
\\ \hline 
\RDD{logic gate inverted radius}=4pt &  \RDD{logic gate input sep}=0.5cm\\ 
\hline 
\end{tabular} 

\bigskip

\begin{tabular}{|c|c|c|}\hline 
\multicolumn{3}{|c|}{\textbf{\TFRGB{Paramètres des symboles}{symbol parameter}}}  \\ \hline 
\multicolumn{3}{|c|}{\BS{node} [circuit logic IEC,\RDD{and gate IEC symbol}=AND ] at (1,.5) [and gate] \AC{} } \\
\multicolumn{3}{|c|}{ \RRR{47-3-5} }
\\ \hline 
\begin{tikzpicture}[blue]
\useasboundingbox (-.2,-.2) grid (2.2,1.2);
\node[circuit logic IEC,and gate IEC symbol=AND ]  at (1,.5) [and gate] {} ; 
\end{tikzpicture}
&  
\begin{tikzpicture}[blue]
\useasboundingbox (-.2,-.2) grid (2.2,1.2);
\node[circuit logic IEC,logic gate IEC symbol color=red ]  at (1,.5) [and gate] {} ; 
\end{tikzpicture}
&  
\begin{tikzpicture}[blue]
\useasboundingbox (-.2,-.2) grid (2.2,1.2);
\node[circuit logic IEC,logic gate IEC symbol align={bottom, right}]  at (1,.5) [and gate] {} ; 
\end{tikzpicture}
\\ \hline 
\RDD{and gate IEC symbol}& \RDD{logic gate IEC symbol color}& \RDD{logic gate IEC symbol align}\\ 
=AND  & =red  & =\AC{bottom, right} 
\\
\hline 
\end{tabular} 

\bigskip

\begin{tabular}{|c|c|c|}\hline 
\multicolumn{3}{|c|}{\textbf{\TFRGB{Paramètres des composants}{Composant parameter}}}  \\ \hline 
\multicolumn{3}{|c|}{\BS{node} [circuit logic IEC,\rouge{very thick} ] at (1,.5) [and gate] \AC{} } \\
\multicolumn{3}{|c|}{ \RRR{47-3-5} }
\\ \hline  
\begin{tikzpicture}[blue]
\useasboundingbox (-.2,-.2) grid (2.2,1.2);
\node[circuit logic IEC,very thick]  at (1,.5) [and gate] {} ; 
\end{tikzpicture}
&  
\begin{tikzpicture}[blue]
\useasboundingbox (-.2,-.2) grid (2.2,1.2);
\node[circuit logic IEC,
fill=blue!10]  at (1,.5) [and gate] {} ; 
\end{tikzpicture}
&  
\begin{tikzpicture}[blue]
\useasboundingbox (-.2,-.2) grid (2.2,1.2);
\node[circuit logic IEC,logic gate IEC symbol color=black,
fill=blue!10]  at (1,.5) [and gate] {} ; 
\end{tikzpicture}
\\ \hline 
\rouge{very thick} & \rouge{fill}=blue!10 & \rouge{fill}=blue!10, \\ 
& & logic gate IEC symbol color=black 
\\ \hline 
\end{tabular} 






\newpage

\SSCT{Optique }{Optics}

\label{optics}


 \maboite{\BS{usepackage}\AC{optics}  \cite {optics}}
 
\begin{tabular}{|c|c|}\hline  
\begin{tikzpicture}[blue,line width=2pt,baseline=0pt]
\useasboundingbox (-1.5,-1.2) rectangle (1.5,1.2);
\draw[help lines] (-1,-1) grid (1,1); 
\node[use optics,lens] (L) at (0,0) {}; 
\end{tikzpicture}
&  
\parbox{10cm}{
\BS{begin}\AC{tikzpicture}[blue,line width=2pt] \\
\BS{draw}[help lines] (-1,-1) grid (1,1); \\
\BS{node}[\RDD{use optics},lens] (L) at (0,0) {}; \\
\BS{end}\AC{tikzpicture}
}
\\ \hline 
\end{tabular}
 
%\subsection{Eléments optiques}
\SbSSCT{Éléments optiques}{Optic components } 

\SbSbSSCT{Éléments optiques disponibles}{Components available}

\noindent

\begin{tabular}{|c|c|c|c|c|} \hline
\multicolumn{5}{|c|}{Éléments optiques}
\\  \hline 
\multicolumn{5}{|c|}{ \BS{tikz}[use optics,blue] \BS{node}[\RDD{lens}] (L) at (0,0) \AC{};  }
\\  \hline   
\multicolumn{2}{|c|}{ 
\tikz[use optics,blue] \node[lens] (L) at (0,0) {}; 
}
&  
\tikz[use optics,blue] \node[slit] (S) at (0,0) {};
&  
\tikz[use optics,blue]  \node[double slit] (S) at (0,0) {};
&  
\tikz[use optics,blue] \node[mirror] (S) at (0,0) {};
\\  \hline 
\multicolumn{2}{|c|}{ \RDD{lens} } & \RDD{slit} & \RDD{double slit}  & \RDD{mirror} 
\\ \hline 

\tikz[use optics,blue] \node[convex mirror] at (0cm,0) {};
&
\tikz[use optics,blue] \node[concave mirror] at (4cm,0) {};
&
\tikz[use optics,blue] \node[polarizer] (S) at (0,0) {};
&
\tikz[use optics,blue] \node[beam splitter] at (0,0) {};
&
\tikz[use optics,blue]  \node[double amici prism] (PVD) at (0,0) {};
\\ \hline
\RDD{convex mirror} & \RDD{concave mirror} & \RDD{polarizer} & \RDD{beam splitter} & \RDD{double amici prism}
\\ \hline 
\multicolumn{2}{|c|}{ \tikz[use optics,scale=.5,blue] 
\node[thin optics element] (S) at (0,0) {};
}
&
\tikz[use optics,blue]  \node[thick optics element] at (0,0) {};
&
\tikz[use optics,blue] \node[heat filter] (S) at (0,0) {};
&
\tikz[use optics,blue] \node[screen] (S) at (0,0) {};
\\ \hline
\multicolumn{2}{|c|}{ \RDD{thin optics element} } & \RDD{thick optics element} & \RDD{heat filter} &  \RDD{screen}
\\ \hline
\multicolumn{2}{|c|}{ \tikz[use optics,scale=.5,blue] 
\node[diffraction grating] (S) at (0,0) {};
}
&
\tikz[use optics,blue] \node[grid] (S) at (0,0) {};
&
\tikz[use optics,blue] \node[semi-transparent mirror] (S) at (0,0) {};
&
\tikz[use optics,blue] \node[diaphragm] (S) at (0,0) {};
\\ \hline
\multicolumn{2}{|c|}{ \RDD{diffraction grating} } & \RDD{grid} & \RDD{semi-transparent mirror} & \RDD{diaphragm}
\\ \hline 
\end{tabular} 

\SbSbSSCT{Paramètres}{Parameters}

\noindent

\begin{tabular}{|c|c|c|c|} \hline 
%\multicolumn{4}{|c|}{ \textbf{Lens parameters} }
%\\ \hline 
\multicolumn{4}{|c|}{ \BS{node}[lens,\RDD{object height}=1cm] (L) at (0,0) \AC{}; }
\\ \hline 
\begin{tikzpicture}[use optics,blue,line width=2pt,baseline=0pt]
\useasboundingbox (-1.2,-1.2) rectangle (1.2,1.2);
\draw[help lines] (-1,-1) grid (1,1); 
\node[lens,object height=1cm] (L) at (0,0) {}; 
\end{tikzpicture}
&
\begin{tikzpicture}[use optics,blue,line width=2pt,baseline=0pt]
\useasboundingbox (-2.2,-1.2) rectangle (2.2,1.2);
\draw[help lines] (-2,-1) grid (2,1); 
\node[lens,draw focal points] (L) at (0,0) {}; 
\end{tikzpicture}
&
\begin{tikzpicture}[use optics,blue,line width=2pt,baseline=0pt]
\useasboundingbox (-2.2,-1.2) rectangle (2.2,1.2);
\draw[help lines] (-2,-1) grid (2,1); 
\node[lens,draw focal points, focal length=1.5cm] (L) at (0,0) {}; 
\end{tikzpicture}
&
\begin{tikzpicture}[use optics,blue,line width=2pt,baseline=0pt]
\useasboundingbox (-1.2,-1.2) rectangle (1.2,1.2);
\draw[help lines] (-1,-1) grid (1,1); 
\node[lens,lens height=.5] (L) at (0,0) {};
\draw (L.lens north) to[thin,short dim arrow={label=$50\%$,label near middle}] (L.lens south);
% \draw[line width =5pt,red] (L.lens north) -- (L.lens south) ;
\end{tikzpicture}
\\ \hline 
\RDD{object height}=1cm & \RDD{draw focal points} & \RDD{focal length}=1.5cm & \RDD{focal height}=0.5
\\
\dft{ 2cm} & \dft{ empty} &  \dft{ 1cm} & \dft{ 0.8} (80\%)
\\ \hline 
\end{tabular}

\bigskip

\begin{tabular}{|c|c|}\hline  
\multicolumn{2}{|c|}{ \textbf{Lens type} }
\\ \hline 
\multicolumn{2}{|c|}{ \BS{node}[lens,\RDD{lens type}=converging] (L) at (0,0) \AC{}; }
\\ \hline 
\begin{tikzpicture}[use optics,scale=.5,blue,line width=2pt]
\useasboundingbox (-.5,-2.5) rectangle (.5,2.5);
\node[lens,lens type=converging] (L) at (0,0) {};
\end{tikzpicture}
&  
\begin{tikzpicture}[use optics,scale=.5,blue,line width=2pt]
\useasboundingbox (-.5,-2.5) rectangle (.5,2.5);
\node[lens,lens type=diverging] (L) at (0,0) {};
\end{tikzpicture}
\\ 
\hline  \RDD{lens type}=converging & \RDD{lens type}=diverging \\ 
\hline 
\end{tabular} 
\hspace{1cm}
\begin{tabular}{|c|c|} \hline 
\multicolumn{2}{|c|}{ \textbf{slit parameters}}
\\ \hline 
\multicolumn{2}{|c|}{ \BS{node}[slit,\RDD{slit height}=0.5] (L) at (0,0) \AC{}; }
\\ \hline  
\begin{tikzpicture}[use optics,blue,line width=2pt,baseline=0pt]
\useasboundingbox (-1.2,-1.2) rectangle (1.2,1.2);
\draw[help lines,ystep=5mm] (-1,-1) grid (1,1); 
\node[slit, slit height=0.5 ](L) at (0,0) {};
\end{tikzpicture}
&  
\begin{tikzpicture}[use optics,blue,line width=2pt,baseline=0pt]
\useasboundingbox (-1.2,-1.2) rectangle (1.2,1.2);
\draw[help lines,ystep=2.5mm] (-1,-1) grid (1,1); 
\node[slit, slit height=0.5cm](L) at (0,0) {};
\end{tikzpicture}
\\ \hline 
 \RDD{slit height}=0.5 &   \RDD{slit height}=0.5cm 
\\ \hline 
\multicolumn{2}{|c|}{ \dft{ 0.075} (7.5\%  )}
\\ \hline 
\end{tabular} 

\bigskip
\begin{tabular}{|c|c|c|c|} \hline
\multicolumn{4}{|c|}{ \textbf{Double slit parameters}}
\\ \hline 
\multicolumn{4}{|c|}{ \BS{node}[double slit,\RDD{slit height}=0.15] (L) at (0,0) \AC{}; }
\\ \hline   
\begin{tikzpicture}[use optics,blue,line width=2pt,baseline=0pt]
\useasboundingbox (-1.2,-1.2) rectangle (1.2,1.2);
\draw[help lines,ystep=1.25mm] (-1,-1) grid (1,1); 
%\node[double slit,red,line width=1pt ](L) at (0,0) {};
\node[double slit, slit height=0.15 ](L) at (0,0) {};
\end{tikzpicture}
&  
\begin{tikzpicture}[use optics,blue,line width=2pt,baseline=0pt]
\useasboundingbox (-1.2,-1.2) rectangle (1.2,1.2);
\draw[help lines,ystep=1.25mm] (-1,-1) grid (1,1); 
%\node[double slit,red,line width=1pt ](L) at (0,0) {};
\node[double slit, slit height=0.25cm](L) at (0,0) {};
\end{tikzpicture}
&
\begin{tikzpicture}[use optics,blue,line width=2pt,baseline=0pt]
\useasboundingbox (-1.2,-1.2) rectangle (1.2,1.2);
\draw[help lines,ystep=5mm] (-1,-1) grid (1,1); 
\node[double slit, slit separation=0.5 ](L) at (0,0) {};
\end{tikzpicture}
&  
\begin{tikzpicture}[use optics,blue,line width=2pt,baseline=0pt]
\useasboundingbox (-1.2,-1.2) rectangle (1.2,1.2);
\draw[help lines,ystep=5mm] (-1,-1) grid (1,1); 
\node[double slit, slit separation=1cm](L) at (0,0) {};
\end{tikzpicture}
\\ \hline 
\RDD{slit height}=0.15 & \RDD{slit height}=0.25cm  &  \RDD{slit separation}=0.5 &  double slit, \RDD{slit separation}=1cm 
\\ \hline 
\multicolumn{2}{|c|}{ \dft{ 0.075} (7.5\%  x 2cm = 1.5 mm)} & \multicolumn{2}{|c|}{ \dft{ 0.2} (20\% x 2cm = 4mm)}
\\ \hline 
\end{tabular}

\bigskip
\begin{tabular}{|c|c|} \hline
\multicolumn{2}{|c|}{ \textbf{mirror parameters}}
\\ \hline 
\multicolumn{2}{|c|}{ \BS{node}[mirror,\RDD{mirror decoration separation}=0.25] (L) at (0,0) \AC{}; }
\\ \hline   
\begin{tikzpicture}[use optics,blue,line width=2pt,baseline=0pt]
\useasboundingbox (-1.2,-1.2) rectangle (1.2,1.2);
\draw[help lines,ystep=5mm] (-1,-1) grid (1,1); 
\node[mirror,mirror decoration separation=0.25 ](L) at (0,0) {};
\end{tikzpicture}
&  
\begin{tikzpicture}[use optics,blue,line width=2pt,baseline=0pt]
\useasboundingbox (-1.2,-1.2) rectangle (1.2,1.2);
\draw[help lines,ystep=5mm] (-1,-1) grid (1,1); 
\node[mirror,mirror decoration separation=0.5cm](L) at (0,0) {};
\end{tikzpicture}
\\ \hline  
\RDD{mirror decoration separation}=0.25 & \RDD{mirror decoration separation}=0.5cm 
\\ \hline 
\multicolumn{2}{|c|}{ \dft{ 0.15cm} } 
\\ \hline  
\begin{tikzpicture}[use optics,blue,line width=2pt,baseline=0pt]
\useasboundingbox (-1.2,-1.5) rectangle (1.2,1.2);
\draw[help lines] (-1,-1) grid (1,1); 
\node[mirror,mirror decoration amplitude=0.25 ](L) at (0,0) {};
\end{tikzpicture}
&  
\begin{tikzpicture}[use optics,blue,line width=2pt,baseline=0pt]
\useasboundingbox (-1.2,-1.2) rectangle (1.2,1.2);
\draw[help lines] (-1,-1) grid (1,1); 
\node[mirror,mirror decoration amplitude=.5cm](L) at (0,0) {};
\end{tikzpicture}
\\ \hline \hline
 \RDD{mirror decoration amplitude}=0.25 & \RDD{mirror decoration amplitude}=1cm 
\\ \hline 
 \multicolumn{2}{|c|}{ \dft{ 0.125cm} }
\\ \hline 
\end{tabular}

\bigskip


\bigskip\begin{tabular}{|c|c|} \hline 
 \multicolumn{2}{|c|}{ \textbf{spherical mirror type} }
\\ \hline 
 \multicolumn{2}{|c|}{\BS{node}[\RDD{convex mirror}](L) at (0,0) \AC{}; }
\\ \hline  
\begin{tikzpicture}[use optics,blue,line width=2pt,baseline=0pt]
\useasboundingbox (-1.2,-1.2) rectangle (1.2,1.2);
\draw[help lines] (-1,-1) grid (1,1); 
\node[convex mirror](L) at (0,0) {};
\end{tikzpicture}
&  
\begin{tikzpicture}[use optics,blue,line width=2pt,baseline=0pt]
\useasboundingbox (-1.2,-1.2) rectangle (1.2,1.2);
\draw[help lines] (-1,-1) grid (1,1); 
\node[concave mirror](L) at (0,0) {};
\end{tikzpicture}
\\ \hline \RDD{convex mirror} & \RDD{concave mirror} 
\\ \hline 
spherical mirror, \RDD{spherical mirror type}=\BDD{convex} &
spherical mirror, \RDD{spherical mirror type}=\BDD{concave} 
\\ \hline 
\end{tabular} 

\bigskip

\begin{tabular}{|c|c|} \hline  
 \multicolumn{2}{|c|}{ \textbf{spherical mirror orientation } }
\\ \hline 
 \multicolumn{2}{|c|}{ \BS{node}[convex mirror, \RDD{spherical mirror orientation}=\BDD{ltr}](L) at (0,0) \AC{};  }
\\ \hline  
\begin{tikzpicture}[use optics,blue,line width=2pt,baseline=0pt]
\useasboundingbox (-1.2,-1.2) rectangle (1.2,1.2);
\draw[help lines] (-1,-1) grid (1,1); 
\node[convex mirror, spherical mirror orientation=ltr](L) at (0,0) {};
\end{tikzpicture}
&  
\begin{tikzpicture}[use optics,blue,line width=2pt,baseline=0pt]
\useasboundingbox (-1.2,-1.2) rectangle (1.2,1.2);
\draw[help lines] (-1,-1) grid (1,1); 
\node[convex mirror, spherical mirror orientation=rtl](L) at (0,0) {};
\end{tikzpicture}
\\ \hline
convex mirror, & convex mirror, \\
\RDD{spherical mirror orientation}=\BDD{ltr} & \RDD{spherical mirror orientation}=\BDD{rtl} 
\\ \hline

\begin{tikzpicture}[use optics,blue,line width=2pt,baseline=0pt]
\useasboundingbox (-1.2,-1.2) rectangle (1.2,1.2);
\draw[help lines] (-1,-1) grid (1,1); 
\node[concave mirror, spherical mirror orientation=ltr](L) at (0,0) {};
\end{tikzpicture} 
&
\begin{tikzpicture}[use optics,blue,line width=2pt,baseline=0pt]
\useasboundingbox (-1.2,-1.2) rectangle (1.2,1.2);
\draw[help lines] (-1,-1) grid (1,1); 
\node[concave mirror, spherical mirror orientation=rtl](L) at (0,0) {};
\end{tikzpicture}
\\ \hline 
concave mirror & concave mirror, \\
 \RDD{spherical mirror orientation}=\BDD{ltr} &  \RDD{spherical mirror orientation}=\BDD{rtl}
\\ \hline 
\end{tabular} 
\bigskip

\begin{tabular}{|c|c|c|} \hline 
 \multicolumn{3}{|c|}{ \BS{node}[spherical mirror, \RDD{spherical mirror angle}=240](L) at (0,0) \AC{};  }
\\ \hline  

\begin{tikzpicture}[use optics,blue,line width=2pt,baseline=0pt]
\useasboundingbox (-1.2,-1.5) rectangle (1.2,1.5);
\draw[help lines] (-1,-1) grid (1,1); 
\node[spherical mirror, spherical mirror angle=240](L) at (0,0) {};
\end{tikzpicture}
&  
\begin{tikzpicture}[use optics,blue,line width=2pt,baseline=0pt]
\useasboundingbox (-1.2,-1.5) rectangle (1.2,1.5);
\draw[help lines,ystep=5mm] (-1,-1) grid (1,1); 
\node[spherical mirror, mirror decoration separation=0.25](L) at (0,0) {};
\end{tikzpicture}
&  
\begin{tikzpicture}[use optics,blue,line width=2pt,baseline=0pt]
\useasboundingbox (-1.2,-1.5) rectangle (1.2,1.5);
\draw[help lines,ystep=5mm] (-1,-1) grid (1,1); 
\node[spherical mirror,mirror decoration amplitude=.5cm](L) at (0,0) {};
\end{tikzpicture} 

\\ \hline 
\RDD{spherical mirror angle}=240 & \RDD{mirror decoration separation}=0.25 &  \RDD{mirror decoration amplitude}=0.5cm  
\\ \hline 
\dft{ 150} & \dft{ 0.15cm}  & \dft{ 0.125cm} 
\\ \hline 
\end{tabular} 

\bigskip


\begin{tabular}{|c|} \hline  
\BS{node}[spherical mirror,
spherical mirror angle=from\_radius(2cm)](L) at (0,0) \AC{}; 
\\ \hline  
\begin{tikzpicture}[use optics,blue,line width=2pt,baseline=0pt]
\useasboundingbox (-1.2,-1.2) rectangle (1.2,1.2);
\draw[help lines,ystep=5mm]  (-2,-1) grid (1,1); 
\node[spherical mirror,
spherical mirror angle=from_radius(2cm)] (M) {M};
\draw[red,fill] (M.mirror center)   circle (2pt) ;
\end{tikzpicture}
\\ \hline 
\end{tabular} 



\bigskip

\begin{tabular}{|c|c|c|} \hline
 \multicolumn{3}{|c|}{ \BS{node}[polarizer, \RDD{object height}=1.5cm](L) at (0,0) \AC{};  }
\\ \hline  
\begin{tikzpicture}[use optics,blue,line width=2pt,baseline=0pt]
\useasboundingbox (-1.2,-1.2) rectangle (1.2,1.2);
\draw[help lines,ystep=5mm]  (-1,-1) grid (1,1); 
\node[polarizer, object height=1.5cm](L) at (0,0) {};
\end{tikzpicture}  
&  
\begin{tikzpicture}[use optics,blue,line width=2pt,baseline=0pt]
\useasboundingbox (-1.2,-1.2) rectangle (1.2,1.2);
\draw[help lines,ystep=5mm]  (-1,-1) grid (1,1); 
\node[polarizer, object aspect ratio=.5](L) at (0,0) {};
\end{tikzpicture}
&  
\begin{tikzpicture}[use optics,blue,line width=2pt,baseline=0pt]
\useasboundingbox (-2.2,-1.2) rectangle (2.2,1.2);
\draw[help lines,ystep=5mm]  (-2,-1) grid (2,1); 
\node[polarizer, object aspect ratio=2](L) at (0,0) {};
\end{tikzpicture}
\\ \hline  
\RDD{object height}=1.5cm
&  
\RDD{object aspect ratio}=0.5
&  
\RDD{object aspect ratio}=2
\\ \hline 
\dft{ 2cm} & \dft{ 0.2} &
\\ \hline
\end{tabular} 

\bigskip

\begin{tabular}{|c|c|c|} \hline
 \multicolumn{3}{|c|}{ \BS{node}[beam splitter,\RDD{object height}=1.5cm](L) at (0,0) \AC{};  }
\\ \hline  
\begin{tikzpicture}[use optics,blue,line width=2pt,baseline=0pt]
\useasboundingbox (-1.2,-1.2) rectangle (1.2,1.2);
\draw[help lines,ystep=5mm]  (-1,-1) grid (1,1); 
\node[beam splitter,object height=1.5cm](L) at (0,0) {};
\end{tikzpicture}  
&  
\begin{tikzpicture}[use optics,blue,line width=2pt,baseline=0pt]
\useasboundingbox (-1.2,-1.2) rectangle (1.2,1.2);
\draw[help lines,ystep=5mm]  (-1,-1) grid (1,1); 
\node[beam splitter, object aspect ratio=.5](L) at (0,0) {};
\end{tikzpicture}
&  
\begin{tikzpicture}[use optics,blue,line width=2pt,baseline=0pt]
\useasboundingbox (-1.2,-1.2) rectangle (1.2,1.2);
\draw[help lines,ystep=5mm]  (-1,-1) grid (1,1); 
\node[beam splitter, object aspect ratio=2](L) at (0,0) {};
\end{tikzpicture}
\\ \hline  
\RDD{object height}=1.5cm
&  
\RDD{object aspect ratio}=.5
&  
\RDD{object aspect ratio}=2
\\ \hline 
\end{tabular} 

\bigskip

\begin{tabular}{|c|c|}\hline
 \multicolumn{2}{|c|}{ \BS{node}[double amici prism,\RDD{prism height}=1cm](L) at (0,0) \AC{};  }
\\ \hline  
\begin{tikzpicture}[use optics,blue,line width=2pt,baseline=0pt]
\useasboundingbox (-2.2,-1.2) rectangle (2.2,1.2);
\draw[help lines,ystep=5mm] (-2,-1) grid (2,1); 
\node[double amici prism,prism height=1cm](L) at (0,0) {};
\end{tikzpicture}
&  
\begin{tikzpicture}[use optics,blue,line width=2pt,baseline=0pt]
\useasboundingbox (-2.2,-1.2) rectangle (2.2,1.2);
\draw[help lines,ystep=5mm] (-2,-1) grid (2,1); 
\node[double amici prism, prism apex angle=90](L) at (0,0) {};
\end{tikzpicture}
\\ \hline 
\RDD{prism height}=1cm & \RDD{prism apex angle}=90   
\\ \hline 
\dft{ 1.5cm } & \dft{ 60 } 
\\ \hline 
\end{tabular} 


\bigskip

\begin{tabular}{|c|c|c|} \hline
 \multicolumn{3}{|c|}{ \BS{node}[thick optics element,\RDD{object height}=1.5cm](L) at (0,0) \AC{};  }
\\ \hline  

\begin{tikzpicture}[use optics,blue,line width=2pt,baseline=0pt]
\useasboundingbox (-1.2,-1.2) rectangle (1.2,1.2);
\draw[help lines,ystep=5mm]  (-1,-1) grid (1,1); 
\node[thick optics element,object height=1.5cm]  (L) at (0,0) {};
\end{tikzpicture}  
&  
\begin{tikzpicture}[use optics,blue,line width=2pt,baseline=0pt]
\useasboundingbox (-1.2,-1.2) rectangle (1.2,1.2);
\draw[help lines,ystep=5mm]  (-1,-1) grid (1,1); 
\node[thick optics element, object aspect ratio=.5](L) at (0,0) {};
\end{tikzpicture}
&  
\begin{tikzpicture}[use optics,blue,line width=2pt,baseline=0pt]
\useasboundingbox (-2.2,-1.2) rectangle (2.2,1.2);
\draw[help lines,ystep=5mm](-2,-1) grid (2,1); 
\node[thick optics element, object aspect ratio=1.5](L) at (0,0) {};
\end{tikzpicture}
\\ \hline  
\RDD{object height}=1.5cm
&  
\RDD{object aspect ratio}=0.5
&  
\RDD{object aspect ratio}=1.5
\\ \hline 
\end{tabular} 




\SbSbSSCT{Ancres}{Anchors}

\noindent

\begin{tabular}{|c|c|c|c|c|} \hline 
\multicolumn{5}{|c|}{\BS{node}[lens] (\blll{L}) at (0,0) \AC{} ;} \\
\multicolumn{5}{|c|}{\BS{node}[red,fill] (\blll{L}.\RDD{lens north})   circle (2pt) ;}
\\  \hline  
\begin{tikzpicture}[use optics,blue]
\useasboundingbox (-.5,-1.5) rectangle (.5,1.5);
\node[lens] (L) at (0,0) {} ;
\draw[red,fill] (L.lens north)   circle (2pt) ;
\end{tikzpicture}
&  
\begin{tikzpicture}[use optics,blue]
\useasboundingbox (-.5,-1.5) rectangle (.5,1.5);
\node[lens] (L) at (0,0) {} ;
\draw[red,fill] (L.lens south)   circle (2pt) ;
\end{tikzpicture}
&  
\begin{tikzpicture}[use optics,blue]
\useasboundingbox (-.5,-1.5) rectangle (.5,1.5);
\node[lens] (L) at (0,0) {} ;
\draw[red,fill] (L.east focus)  circle (2pt) ;
\end{tikzpicture}
&  
\begin{tikzpicture}[use optics,blue]
\useasboundingbox (-.5,-1.5) rectangle (.5,1.5);
\node[lens] (L) at (0,0) {} ;
\draw[red,fill] (L.west focus)  circle (2pt) ;
\end{tikzpicture}
&  
\begin{tikzpicture}[use optics,blue]
\useasboundingbox (-.5,-1.5) rectangle (.5,1.5);
\node[lens] (L) at (0,0) {} ;
\draw[red,fill] (L.center)  circle (2pt) ;
\end{tikzpicture}
\\ \hline 
(\blll{L}.\RDD{lens north}) & (\blll{L}.\RDD{lens south})  & (\blll{L}.\RDD{east focus}) & (\blll{L}.\RDD{west focus}) & (\blll{L}.\RDD{center}) \\ 
\hline 
\end{tabular} 

\bigskip
\begin{tabular}{|c|c|c|} \hline 
\multicolumn{3}{|c|}{\BS{node}[slit, slit height=0.5] (\blll{L}) at (0,0) \AC{} ;} \\
\multicolumn{3}{|c|}{\BS{node}[red,fill] (\blll{L}.\RDD{slit north})   circle (2pt) ;}
\\  \hline  
\begin{tikzpicture}[use optics,blue]
\useasboundingbox (-.5,-1.5) rectangle (.5,1.5);
\node[slit, slit height=0.5] (L) at (0,0) {} ;
\draw[red,fill] (L.slit north)   circle (2pt) ;
\end{tikzpicture}
&  
\begin{tikzpicture}[use optics,blue]
\useasboundingbox (-.5,-1.5) rectangle (.5,1.5);
\node[slit, slit height=0.5] (L) at (0,0) {} ;
\draw[red,fill] (L.slit south)   circle (2pt) ;
\end{tikzpicture}
&  
\begin{tikzpicture}[use optics,blue]
\useasboundingbox (-.5,-1.5) rectangle (.5,1.5);
\node[slit, slit height=0.5] (L) at (0,0) {} ;
\draw[red,fill] (L.slit center)  circle (2pt) ;
\end{tikzpicture}
\\ \hline 
(\blll{L}.\RDD{slit north}) & (\blll{L}.\RDD{slit south})  & (\blll{L}.\RDD{slit center})  \\ 
\hline 
\end{tabular}


\bigskip
\begin{tabular}{|c|c|c||c|c|c|} \hline 
\multicolumn{6}{|c|}{\BS{node}[double slit,slit height=0.2,slit separation=0.5] (\blll{L}) at (0,0) \AC{} ;} \\
\multicolumn{6}{|c|}{\BS{node}[red,fill] (\blll{L}.\RDD{slit 1 north})   circle (2pt) ;}
\\  \hline  
\begin{tikzpicture}[use optics,blue]
\useasboundingbox (-.5,-1.5) rectangle (.5,1.5);
\node[double slit,slit height=0.2,slit separation=0.5] (L) at (0,0) {} ;
\draw[red,fill] (L.slit 1 north)   circle (2pt) ;
\end{tikzpicture}
&  
\begin{tikzpicture}[use optics,blue]
\useasboundingbox (-.5,-1.5) rectangle (.5,1.5);
\node[double slit,slit height=0.2,slit separation=0.5] (L) at (0,0) {} ;
\draw[red,fill] (L.slit 1 south)   circle (2pt) ;
\end{tikzpicture}
&  
\begin{tikzpicture}[use optics,blue]
\useasboundingbox (-.5,-1.5) rectangle (.5,1.5);
\node[double slit,slit height=0.2,slit separation=0.5]  (L) at (0,0) {} ;
\draw[red,fill] (L.slit 1 center)  circle (2pt) ;
\end{tikzpicture}
&
\begin{tikzpicture}[use optics,blue]
\useasboundingbox (-.5,-1.5) rectangle (.5,1.5);
\node[double slit,slit height=0.2,slit separation=0.5] (L) at (0,0) {} ;
\draw[red,fill] (L.slit 2 north)   circle (2pt) ;
\end{tikzpicture}
&  
\begin{tikzpicture}[use optics,blue]
\useasboundingbox (-.5,-1.5) rectangle (.5,1.5);
\node[double slit,slit height=0.2,slit separation=0.5] (L) at (0,0) {} ;
\draw[red,fill] (L.slit 2 south)   circle (2pt) ;
\end{tikzpicture}
&  
\begin{tikzpicture}[use optics,blue]
\useasboundingbox (-.5,-1.5) rectangle (.5,1.5);
\node[double slit,slit height=0.2,slit separation=0.5]  (L) at (0,0) {} ;
\draw[red,fill] (L.slit 2 center)  circle (2pt) ;
\end{tikzpicture}
\\ \hline 
(\blll{L}.\RDD{slit 1 north}) & (\blll{L}.\RDD{slit 1 south})  & (\blll{L}.\RDD{slit 1 center})& (\blll{L}.\RDD{slit 2 north}) & (\blll{L}.\RDD{slit 2 south})  & (\blll{L}.\RDD{slit 2 center})  \\ 
\hline 
\end{tabular}

\bigskip

\begin{tabular}{|c|c|c|c|c|c|c|} \hline 
\multicolumn{7}{|c|}{\BS{node}[spherical mirror] (\blll{L}) at (0,0) \AC{} ;} \\
\multicolumn{7}{|c|}{\BS{node}[red,fill] (\blll{L}.\RDD{mirror center})   circle (2pt) ;}
\\  \hline  
\begin{tikzpicture}[use optics,blue]
\useasboundingbox (-.5,-1.5) rectangle (.5,1.5);
\node[spherical mirror] (L) at (0,0) {} ;
\draw[red,fill] (L.mirror center)   circle (2pt) ;
\end{tikzpicture}
&  
\begin{tikzpicture}[use optics,blue]
\useasboundingbox (-.5,-1.5) rectangle (.5,1.5);
\node[spherical mirror] (L) at (0,0) {} ;
\draw[red,fill] (L.focus)   circle (2pt) ;
\end{tikzpicture}
&  
\begin{tikzpicture}[use optics,blue]
\useasboundingbox (-.5,-1.5) rectangle (.5,1.5);
\node[spherical mirror] (L) at (0,0) {} ;
\draw[red,fill] (L.arc start)   circle (2pt) ;
\end{tikzpicture}
&  
\begin{tikzpicture}[use optics,blue]
\useasboundingbox (-.5,-1.5) rectangle (.5,1.5);
\node[spherical mirror] (L) at (0,0) {} ;
\draw[red,fill] (L.arc center)   circle (2pt) ;
\end{tikzpicture}
&
\begin{tikzpicture}[use optics,blue]
\useasboundingbox (-.5,-1.5) rectangle (.5,1.5);
\node[spherical mirror] (L) at (0,0) {} ;
\draw[red,fill] (L.arc end)   circle (2pt) ;
\end{tikzpicture}
&
\begin{tikzpicture}[use optics,blue]
\useasboundingbox (-.5,-1.5) rectangle (.5,1.5);
\node[spherical mirror] (L) at (0,0) {} ;
\draw[red,fill] (L.45)   circle (2pt) ;
\end{tikzpicture}
&
\begin{tikzpicture}[use optics,blue]
\useasboundingbox (-.5,-1.5) rectangle (.5,1.5);
\node[spherical mirror] (L) at (0,0) {} ;
\draw[red,fill] (L.-45)   circle (2pt) ;
\end{tikzpicture}
\\ \hline  
\blll{L}.\RDD{mirror center} & \blll{L}.\RDD{focus} & \blll{L}.\RDD{arc start} &  \blll{L}.\RDD{arc center}
& \blll{L}.\RDD{arc end} & \blll{L}.\rouge{45} & \blll{L}.\rouge{-45}
\\ \hline 
%\begin{tikzpicture}[use optics,blue]
%\useasboundingbox (-.5,-1.5) rectangle (.5,1.5);
%\node[spherical mirror] (L) at (0,0) {} ;
%\draw[red,fill] (L.arc end)   circle (2pt) ;
%\end{tikzpicture}
%&
%\begin{tikzpicture}[use optics,blue]
%\useasboundingbox (-.5,-1.5) rectangle (.5,1.5);
%\node[spherical mirror] (L) at (0,0) {} ;
%\draw[red,fill] (L.45)   circle (2pt) ;
%\end{tikzpicture}
%&
%\begin{tikzpicture}[use optics,blue]
%\useasboundingbox (-.5,-1.5) rectangle (.5,1.5);
%\node[spherical mirror] (L) at (0,0) {} ;
%\draw[red,fill] (L.-45)   circle (2pt) ;
%\end{tikzpicture}
%&
%
%\\ \hline 
%\blll{L}.\RDD{arc end} & \blll{L}.\rouge{45} & \blll{L}.\rouge{-45} &
%\\ \hline 
\end{tabular} 



\SbSSCT{Lampes et capteurs}{Lights and sensors}



\SbSbSSCT{Disponibles}{Available}

\noindent

\begin{tabular}{|c|c|c|c|}\hline 
%\multicolumn{4}{|c|}{ Lampes et capteurs }
%\\  \hline  
\multicolumn{4}{|c|}{ \BS{tikz}[use optics,scale=.5,blue] \BS{node}[\RDD{generic optics io}] (L) at (0,0) \AC{};  }
\\  \hline  
\tikz[use optics,scale=.5,blue]  \node[generic optics io] (S) at (0,0) {};
&  
\tikz[use optics,scale=.5,blue] \node[sensor line] (S) at (0,0) {};
&  
\tikz[use optics,scale=.5,blue] \node[generic sensor] (S) at (0,0) {};
&  
\tikz[use optics,scale=.5,blue] \node[generic lamp] (S) at (0,0) {};
\\ 
\hline 
\RDD{generic optics io} & \RDD{sensor line} & \RDD{generic sensor}  & \RDD{generic lamp} 
\\ \hline 
\tikz[use optics,scale=.5,blue] \node[halogen lamp] (S) at (0,0) {};
%\node[halogen lamp] (S) at (5cm,0) {QI};
&
\tikz[use optics,scale=.5,blue] \node[spectral lamp] (S) at (0,0) {};
%\node[spectral lamp] (S) at (5cm,0) {{Hg} \\ BP};
&
\tikz[use optics,scale=.5,blue] \node[laser] (S) at (0,0) {};
%\node[laser] (S) at (5cm,0) {{HeNe}}
;
&
\tikz[use optics,scale=.5,blue] \node[laser'] (S) at (0,0) {};
\\ \hline
\RDD{halogen lamp} & \RDD{spectral lamp } & \RDD{laser} & \RDD{laser'}
\\ \hline 
\end{tabular}
 
\SbSbSSCT{Paramètres}{Parameters}

\noindent

\begin{tabular}{|c|c|c|}\hline 
\multicolumn{3}{|c|}{\BS{node}[\blll{generic optics io}, \RDD{io body height}=1.5cm](L) at (0,0) \AC{};}
\\ \hline 
\multicolumn{3}{|c|}{ \TFRGB{Paramètres applicables pour}{Same parameters for } \blll{generic sensor} , \blll{generic lamp} , \blll{halogen lamp} , \blll{spectral lamp},\blll{laser} }
\\ \hline
\begin{tikzpicture}[use optics,blue,line width=2pt,baseline=0pt]
\useasboundingbox (-2.2,-1.2) rectangle (2.2,1.2);
\draw[help lines] (-2,-1) grid (2,1); 
\node[generic optics io, io body height=1.5cm](L) at (0,0) {};
\node[generic optics io,dotted,thick](A) at (0,0) {};
\end{tikzpicture}
&  
\begin{tikzpicture}[use optics,blue,line width=2pt,baseline=0pt]
\useasboundingbox (-2.2,-1.2) rectangle (2.2,1.2);
\draw[help lines] (-2,-1) grid (2,1); 
\node[generic optics io, io body aspect ratio=4](L) at (0,0) {};
\node[generic optics io,dotted,thick](A) at (0,0) {};
\end{tikzpicture}
&  
\begin{tikzpicture}[use optics,blue,line width=2pt,baseline=0pt]
\useasboundingbox (-2.2,-1.2) rectangle (2.2,1.2);
\draw[help lines] (-2,-1) grid (2,1); 
\node[generic optics io,io body width=4](L) at (0,0) {};
\node[generic optics io,dotted,thick](A) at (0,0) {};
\end{tikzpicture}
\\ \hline  
\RDD{io body height}=1.5cm & \RDD{io body aspect ratio}=4 &  \RDD{io body width}=4 \\ 
\hline 
\dft{ 0.75cm} & \dft{ 2} & 
\\ \hline   \hline 
\begin{tikzpicture}[use optics,blue,line width=2pt,baseline=0pt]
\useasboundingbox (-2.2,-1.2) rectangle (2.2,1.2);
\draw[help lines] (-2,-1) grid (2,1); 
\node[generic optics io,io body width=3cm](L) at (0,0) {};
\node[generic optics io,dotted,thick](A) at (0,0) {};
\end{tikzpicture}
&
\begin{tikzpicture}[use optics,blue,line width=2pt,baseline=0pt]
\useasboundingbox (-2.2,-1.2) rectangle (2.2,1.2);
\draw[help lines] (-2,-1) grid (2,1); 
\node[generic optics io,io aperture width=1](L) at (0,0) {};
\node[generic optics io,dotted,thick](A) at (0,0) {};
\end{tikzpicture}
&
\begin{tikzpicture}[use optics,blue,line width=2pt,baseline=0pt]
\useasboundingbox (-2.2,-1.2) rectangle (2.2,1.2);
\draw[help lines] (-2,-1) grid (2,1); 
\node[generic optics io,io aperture width=1cm](L) at (0,0) {};
\node[generic optics io,dotted,thick](A) at (0,0) {};
\end{tikzpicture}
\\ \hline  
\RDD{io body width}=3cm  & \RDD{io aperture width}=1  & \RDD{io aperture width}=1cm
\\ \hline 
 & \multicolumn{2}{|c|}{\dft{ 0.33} }
\\ \hline  \hline 
\begin{tikzpicture}[use optics,blue,line width=2pt,baseline=0pt]
\useasboundingbox (-2.2,-1.2) rectangle (2.2,1.2);
\draw[help lines] (-2,-1) grid (2,1); 
\node[generic optics io,io aperture height=2](L) at (0,0) {};
\node[generic optics io,dotted,thick](A) at (0,0) {};
\end{tikzpicture}
&
\begin{tikzpicture}[use optics,blue,line width=2pt,baseline=0pt]
\useasboundingbox (-2.2,-1.2) rectangle (2.2,1.2);
\draw[help lines] (-2,-1) grid (2,1); 
\node[generic optics io,io aperture height=1cm](L) at (0,0) {};
\node[generic optics io,dotted,thick](A) at (0,0) {};
\end{tikzpicture}
&
\begin{tikzpicture}[use optics,blue,line width=2pt,baseline=0pt]
\useasboundingbox (-2.2,-1.2) rectangle (2.2,1.2);
\draw[help lines] (-2,-1) grid (2,1); 
\node[generic optics io,io aperture shift=0.25](L) at (0,0) {};
\node[generic optics io,dotted,thick](A) at (0,0) {};
\end{tikzpicture}
\\ \hline 
\RDD{io aperture height}=2 & \RDD{io aperture height}=1cm & \RDD{io aperture shift}=0.25
\\ \hline 
\multicolumn{2}{|c|}{\dft{ 0.66} }  & \dft{ 0}
\\ \hline 
%\end{tabular} 
%
%\bigskip
%
%\begin{tabular}{|c|c|}\hline  
\begin{tikzpicture}[use optics,blue,line width=2pt,baseline=0pt]
\useasboundingbox (-2.2,-1.2) rectangle (2.2,1.2);
\draw[help lines] (-2,-1) grid (2,1); 
\node[generic optics io,io orientation=ltr](L) at (0,0) {};
\end{tikzpicture}
&  
\begin{tikzpicture}[use optics,blue,line width=2pt,baseline=0pt]
\useasboundingbox (-2.2,-1.2) rectangle (2.2,1.2);
\draw[help lines] (-2,-1) grid (2,1); 
\node[generic optics io,io orientation=rtl](L) at (0,0) {};
\end{tikzpicture}
&
\\ \hline \RDD{io orientation}=ltr &  \RDD{io orientation}=rtl &
\\ \hline 
\multicolumn{2}{|c|}{\dft{ ltr} } & 
\\ \hline 
\end{tabular} 

\bigskip

\begin{tabular}{|c|c|c|} \hline 
\multicolumn{3}{|c|}{\BS{node}[sensor line, \RDD{sensor line height}=1.5cm](L) at (0,0) \AC{};}
\\ \hline  
\begin{tikzpicture}[use optics,blue,line width=2pt,baseline=0pt]
\useasboundingbox (-2.2,-1.2) rectangle (2.2,1.2);
\draw[help lines] (-2,-1) grid (2,1); 
\node[sensor line, sensor line height=1.5cm](L) at (0,0) {};
\end{tikzpicture}
&  
\begin{tikzpicture}[use optics,blue,line width=2pt,baseline=0pt]
\useasboundingbox (-2.2,-1.2) rectangle (2.2,1.2);
\draw[help lines] (-2,-1) grid (2,1); 
\node[sensor line, sensor line aspect ratio=0.5](L) at (0,0) {};
\end{tikzpicture}
&  
\begin{tikzpicture}[use optics,blue,line width=2pt,baseline=0pt]
\useasboundingbox (-2.2,-1.2) rectangle (2.2,1.2);
\draw[help lines] (-2,-1) grid (2,1); 
\node[sensor line, sensor line pixel number=10](L) at (0,0) {};
\end{tikzpicture}
\\ \hline 
\RDD{sensor line height}=1.5cm & \RDD{sensor line aspect ratio}=0.5 &  \RDD{sensor line pixel number}=10 
\\ \hline
\dft{ 2cm} & \dft{ 0.2} & \dft{ 5} 
\\ \hline
 
\begin{tikzpicture}[use optics,blue,line width=2pt,baseline=0pt]
\useasboundingbox (-2.2,-1.2) rectangle (2.2,1.2);
\draw[help lines] (-2,-1) grid (2,1); 
\node[sensor line,sensor line pixel width=0.8](L) at (0,0) {};
\end{tikzpicture}
&
\begin{tikzpicture}[use optics,blue,line width=2pt,baseline=0pt]
\useasboundingbox (-2.2,-1.2) rectangle (2.2,1.2);
\draw[help lines] (-2,-1) grid (2,1); 
\node[sensor line,sensor line pixel width=0.2cm](L) at (0,0) {};
\end{tikzpicture}
&
\begin{tikzpicture}[use optics,blue,line width=2pt,baseline=0pt]
\useasboundingbox (-2.2,-1.2) rectangle (2.2,1.2);
\draw[help lines] (-2,-1) grid (2,1); 
\node[sensor line,sensor line inner ysep=0.2](L) at (0,0) {};
\end{tikzpicture}

\\ \hline 
\RDD{sensor line pixel width}=0.8 & \RDD{sensor line pixel width}=0.2cm  & \RDD{sensor line inner ysep}=0.2 
\\ \hline 
\multicolumn{2}{|c|}{\dft{0.4} }  & \dft{ 0.05 } 
\\ \hline 
\end{tabular} 


\SbSbSSCT{Points d'ancrages}{Anchors}

\noindent

\begin{tabular}{|c|c|c|c|c|} \hline  
\begin{tikzpicture}[use optics,blue]
\useasboundingbox (-.5,-1.5) rectangle (.5,1.5);
\node[generic optics io,name=s] {};
\draw[red,fill] (s.body north)   circle (2pt) ;
\end{tikzpicture}
&  
\begin{tikzpicture}[use optics,blue]
\useasboundingbox (-.5,-1.5) rectangle (.5,1.5);
\node[generic optics io,name=s] {};
\draw[red,fill] (s.body south)   circle (2pt) ;
\end{tikzpicture}
&  
\begin{tikzpicture}[use optics,blue]
\useasboundingbox (-.5,-1.5) rectangle (.5,1.5);
\node[generic optics io,name=s] {};
\draw[red,fill] (s.body east)   circle (2pt) ;
\end{tikzpicture}
&  
\begin{tikzpicture}[use optics,blue]
\useasboundingbox (-.5,-1.5) rectangle (.5,1.5);
\node[generic optics io,name=s] {};
\draw[red,fill] (s.body west)   circle (2pt) ;
\end{tikzpicture}
&  
\begin{tikzpicture}[use optics,blue]
\useasboundingbox (-.5,-1.5) rectangle (.5,1.5);
\node[generic optics io,name=s] {};
\draw[red,fill] (s.body center)   circle (2pt) ;
\end{tikzpicture}
\\ \hline 
s.body north & s.body south & s.body east & s.body west & s.body center
\\ \hline  
\begin{tikzpicture}[use optics,blue]
\useasboundingbox (-.5,-1.5) rectangle (.5,1.5);
\node[generic optics io,name=s] {};
\draw[red,fill] (s.body north east)   circle (2pt) ;
\end{tikzpicture}
&  
\begin{tikzpicture}[use optics,blue]
\useasboundingbox (-.5,-1.5) rectangle (.5,1.5);
\node[generic optics io,name=s] {};
\draw[red,fill] (s.body north west)   circle (2pt) ;
\end{tikzpicture}
&  
\begin{tikzpicture}[use optics,blue]
\useasboundingbox (-.5,-1.5) rectangle (.5,1.5);
\node[generic optics io,name=s] {};
\draw[red,fill] (s.body south east)   circle (2pt) ;
\end{tikzpicture}
&  
\begin{tikzpicture}[use optics,blue]
\useasboundingbox (-.5,-1.5) rectangle (.5,1.5);
\node[generic optics io,name=s] {};
\draw[red,fill] (s.body south west)   circle (2pt) ;
\end{tikzpicture}
&  

\\ \hline 
s.body north east & s.body north west & s.body south east & s.body south west &
\\ \hline 
%\end{tabular} 
%
%\begin{tabular}{|c|c|c|c|c|} \hline  
\begin{tikzpicture}[use optics,blue]
\useasboundingbox (-.5,-1.5) rectangle (.5,1.5);
\node[generic optics io,name=s] {};
\draw[red,fill] (s.aperture north)   circle (2pt) ;
\end{tikzpicture}
&  
\begin{tikzpicture}[use optics,blue]
\useasboundingbox (-.5,-1.5) rectangle (.5,1.5);
\node[generic optics io,name=s] {};
\draw[red,fill] (s.aperture south)   circle (2pt) ;
\end{tikzpicture}
&  
\begin{tikzpicture}[use optics,blue]
\useasboundingbox (-.5,-1.5) rectangle (.5,1.5);
\node[generic optics io,name=s] {};
\draw[red,fill] (s.aperture east)   circle (2pt) ;
\end{tikzpicture}
&  
\begin{tikzpicture}[use optics,blue]
\useasboundingbox (-.5,-1.5) rectangle (.5,1.5);
\node[generic optics io,name=s] {};
\draw[red,fill] (s.aperture west)   circle (2pt) ;
\end{tikzpicture}
&  
\begin{tikzpicture}[use optics,blue]
\useasboundingbox (-.5,-1.5) rectangle (.5,1.5);
\node[generic optics io,name=s] {};
\draw[red,fill] (s.aperture center)   circle (2pt) ;
\end{tikzpicture}
\\ \hline 
s.aperture north & s.aperture south & s.aperture east & s.aperture west & s.aperture center
\\ \hline  
\begin{tikzpicture}[use optics,blue]
\useasboundingbox (-.5,-1.5) rectangle (.5,1.5);
\node[generic optics io,name=s] {};
\draw[red,fill] (s.aperture north east)   circle (2pt) ;
\end{tikzpicture}
&  
\begin{tikzpicture}[use optics,blue]
\useasboundingbox (-.5,-1.5) rectangle (.5,1.5);
\node[generic optics io,name=s] {};
\draw[red,fill] (s.aperture north west)   circle (2pt) ;
\end{tikzpicture}
&  
\begin{tikzpicture}[use optics,blue]
\useasboundingbox (-.5,-1.5) rectangle (.5,1.5);
\node[generic optics io,name=s] {};
\draw[red,fill] (s.aperture south east)   circle (2pt) ;
\end{tikzpicture}
&  
\begin{tikzpicture}[use optics,blue]
\useasboundingbox (-.5,-1.5) rectangle (.5,1.5);
\node[generic optics io,name=s] {};
\draw[red,fill] (s.aperture south west)   circle (2pt) ;
\end{tikzpicture}
&  

\\ \hline 
s.aperture north east & s.aperture north west & s.aperture south east & s.aperture south west &
\\ \hline 
\end{tabular}

\bigskip

\begin{tabular}{|c|c|c|c|c|} \hline  
\begin{tikzpicture}[use optics,blue]
\useasboundingbox (-.5,-1.5) rectangle (.5,1.5);
\node[sensor line,name=s,sensor line aspect ratio= .5] {};
\draw[red,fill] (s.pixel 1 center)   circle (2pt) ;
\end{tikzpicture}
&  
\begin{tikzpicture}[use optics,blue]
\useasboundingbox (-.5,-1.5) rectangle (.5,1.5);
\node[sensor line,name=s,sensor line aspect ratio= .5] {};
\draw[red,fill] (s.pixel 2 center)   circle (2pt) ;
\end{tikzpicture}
&  
\begin{tikzpicture}[use optics,blue]
\useasboundingbox (-.5,-1.5) rectangle (.5,1.5);
\node[sensor line,name=s,sensor line aspect ratio= .5] {};
\draw[red,fill] (s.pixel 3 center)   circle (2pt) ;
\end{tikzpicture}
&  
\begin{tikzpicture}[use optics,blue]
\useasboundingbox (-.5,-1.5) rectangle (.5,1.5);
\node[sensor line,name=s,sensor line aspect ratio= .5] {};
\draw[red,fill] (s.pixel 4 center)   circle (2pt) ;
\end{tikzpicture}
&  
\begin{tikzpicture}[use optics,blue]
\useasboundingbox (-.5,-1.5) rectangle (.5,1.5);
\node[sensor line,name=s,sensor line aspect ratio= .5] {};
\draw[red,fill] (s.pixel 5 center)   circle (2pt) ;
\end{tikzpicture}
\\ \hline 
s.pixel 1 center & s.pixel 2 center & s.pixel 3 center & s.pixel 4 center & s.pixel 5 center
\\ \hline  
\begin{tikzpicture}[use optics,blue]
\useasboundingbox (-.5,-1.5) rectangle (.5,1.5);
\node[sensor line,name=s,sensor line aspect ratio= .5] {};
\draw[red,fill] (s.pixel 3 east)   circle (2pt) ;
\end{tikzpicture}
&  
\begin{tikzpicture}[use optics,blue]
\useasboundingbox (-.5,-1.5) rectangle (.5,1.5);
\node[sensor line,name=s,sensor line aspect ratio= .5] {};
\draw[red,fill] (s.pixel 3 west)   circle (2pt) ;
\end{tikzpicture}
&  
\begin{tikzpicture}[use optics,blue]
\useasboundingbox (-.5,-1.5) rectangle (.5,1.5);
\node[sensor line,name=s,sensor line aspect ratio= .5] {};
\draw[red,fill] (s.pixel 3 south)  circle (2pt) ;
\end{tikzpicture}
&  
\begin{tikzpicture}[use optics,blue]
\useasboundingbox (-.5,-1.5) rectangle (.5,1.5);
\node[sensor line,name=s,sensor line aspect ratio= .5] {};
\draw[red,fill] (s.pixel 3 north)   circle (2pt) ;
\end{tikzpicture}
&  

\\ \hline 
s.pixel 3 east & s.pixel 3 west & s.pixel 3 south &  s.pixel 3 north &
\\ \hline  
\begin{tikzpicture}[use optics,blue]
\useasboundingbox (-.5,-1.5) rectangle (.5,1.5);
\node[sensor line,name=s,sensor line aspect ratio= .5] {};
\draw[red,fill] (s.pixel 3 north east)   circle (2pt) ;
\end{tikzpicture}
&  
\begin{tikzpicture}[use optics,blue]
\useasboundingbox (-.5,-1.5) rectangle (.5,1.5);
\node[sensor line,name=s,sensor line aspect ratio= .5] {};
\draw[red,fill] (s.pixel 3 north west)   circle (2pt) ;
\end{tikzpicture}
&  
\begin{tikzpicture}[use optics,blue]
\useasboundingbox (-.5,-1.5) rectangle (.5,1.5);
\node[sensor line,name=s,sensor line aspect ratio= .5] {};
\draw[red,fill] (s.pixel 3 south east)  circle (2pt) ;
\end{tikzpicture}
&  
\begin{tikzpicture}[use optics,blue]
\useasboundingbox (-.5,-1.5) rectangle (.5,1.5);
\node[sensor line,name=s,sensor line aspect ratio= .5] {};
\draw[red,fill] (s.pixel 3 south west)   circle (2pt) ;
\end{tikzpicture}
&  

\\ \hline 
s.pixel 3 north east & s.pixel 3 north west & s.pixel 3 south east &  s.pixel 3 south west &
\\ \hline 
\end{tabular}

\SbSSCT{Outils}{Tools}


\SbSbSSCT{Marquer des rayons}{Marks on the ray}

\noindent

\begin{tabular}{|c|c|c|c|c|c|}\hline 
\multicolumn{6}{|c|}{\BS{draw} [\rouge{->-}] (0,0) -- (1.5,1;}
\\  \hline   
\begin{tikzpicture}[use optics,blue,line width=1pt,baseline=0pt]
\draw[->-] (0,0) -- (1.5cm,1cm);
\end{tikzpicture}
&  
\begin{tikzpicture}[use optics,blue,line width=1pt,baseline=0pt]
\draw[-<-] (0,0) -- (1.5cm,1cm);
\end{tikzpicture}
&  
\begin{tikzpicture}[use optics,blue,line width=1pt,baseline=0pt]
\draw[->>-] (0,0) -- (1.5cm,1cm);
\end{tikzpicture}
&  
\begin{tikzpicture}[use optics,blue,line width=1pt,baseline=0pt]
\draw[->n-={n=4}] (0,0) -- (1.5cm,1cm);
\end{tikzpicture}
&  
\begin{tikzpicture}[use optics,blue,line width=1pt,baseline=0pt]
\draw[->n-={n=5,at=0.25}] (0,0) -- (1.5cm,1cm);
\end{tikzpicture}
&  
\begin{tikzpicture}[use optics,blue,line width=1pt,baseline=0pt]
\draw[->>-={at=0.25}, ->-={at=0.75}] (0,0) -- (1.5cm,1.1cm);
\end{tikzpicture}
\\ \hline 
[\rouge{->-}] & [\rouge{-<-}] & [\rouge{-> >-}]   & [->\rouge{n}-=\AC{\RDD{n}=4}] & [->n=\AC{n=5,\RDD{at}=0.25}] & [-> >-={\RDD{at}=0.25}, ->-=\AC{\RDD{at}=0.75}]
\\ 
\hline 
\end{tabular} 
 

\bigskip

\noindent
\begin{tabular}{|c|c|c|c|} \hline
\multicolumn{4}{|c|}{\BS{draw} [\RDD{put arrow}] (0,0) to[bend left=120] (2,0);}
\\  \hline    
\begin{tikzpicture}[use optics,blue]
\draw[put arrow] (0,0) to[bend left=120] (1.5,1);
\end{tikzpicture}
&  
\begin{tikzpicture}[use optics,blue]
\draw[put arrow={arrow'}] (0,0) to[bend left=120](1.5cm,1cm);
\end{tikzpicture}
& 
\begin{tikzpicture}[use optics,blue]
\draw[put arrow={at=0.2}] (0,0) to[bend left=120] (1.5cm,1cm);
\end{tikzpicture}
& 
\begin{tikzpicture}[use optics,blue]
\draw[put arrow={style=red}] (0,0) to[bend left=120] (1.5cm,1cm);
\end{tikzpicture}
\\ \hline 
[\RDD{put arrow}] & [put arrow=\AC{\RDD{arrow'}}] & [put arrow=\AC{\RDD{at}=0.2}]& [put arrow=\AC{\RDD{style}=red}] \\ 
\hline 
\end{tabular} 

\bigskip

\begin{tabular}{|c|c|c|c|} \hline  
\begin{tikzpicture}[use optics,blue]
\draw[red,put arrow={arrow=latex}](0,0) to[bend left=120] (1.5cm,1cm);
\end{tikzpicture}
&  
\begin{tikzpicture}[use optics,blue]
\draw[put arrow={arrow'=Kite}](0,0) to[bend left=120] (1.5cm,1cm);
\end{tikzpicture}
& 
\begin{tikzpicture}[use optics,blue]
\draw[put arrow={pos=.25}](0,0) to[bend left=120] (1.5cm,1cm);
\end{tikzpicture}
\\ \hline 
[red,put arrow=\AC{\RDD{arrow}=latex}] & [put arrow=\AC{\RDD{arrow'}=Kite}] &   [put arrow=\AC{\RDD{pos}=.25}]
\\ \hline 
& & \dft{ pos=0.5}
\\ \hline 
\end{tabular}


\bigskip

\begin{tabular}{|c|} \hline  
\BS{draw}[red, put arrow/\RDD{every arrow}/.style=\AC{blue},
put arrow=\AC{at=0.2},\\ put arrow=\AC{at=0.5}, put arrow=\AC{at=0.8}]
(0,0) -- (5,0);
\\ \hline  
\begin{tikzpicture}[use optics]
\useasboundingbox (-0.5,-1.2) rectangle (5.5,1.2);
\draw[help lines] (0,-1) grid (5,1);
\draw[red, put arrow/every arrow/.style={blue},
put arrow={at=0.2}, put arrow={at=0.5}, put arrow={at=0.8}]
(0,0) -- (5,0);
\end{tikzpicture}
\\ \hline 
\end{tabular} 



\bigskip

\begin{tabular}{|c|c|} \hline  
\begin{tikzpicture}[use optics,baseline=0pt,blue]
\draw[put coordinate=A at 0.1,put coordinate=B at 0.9] (0,0) -- (1.5cm,1cm) -- (3cm, 0) -- (4.5cm,1cm);
\draw[red] (A) -- (B);
\fill(A) circle (2pt) node[above] {A} ;
\fill(B) circle (2pt) node[above] {B} ;
\end{tikzpicture}
&  
\parbox{10cm}{
\BS{begin}\AC{tikzpicture}[use optics,blue] \\
\BS{draw}[\RDD{put coordinate}=\blll{A} at 0.1,\RDD{put coordinate}=\blll{B} at 0.9] \\ (0,0) - - (1.5,1) - - (3, 0) - - (4.5,1); \\
\BS{draw}[red] (\blll{A}) - - (\blll{B});\\
\BS{fill}(A) circle (2pt) node[above] \AC{\blll{A}} ;\\
\BS{fill}(B) circle (2pt) node[above] \AC{\blll{B}} ;\\
\BS{end}\AC{tikzpicture}
}
\\  \hline  
& Point A à 10\% , point B à 90\%
\\ \hline 
\end{tabular} 

\bigskip


\begin{tabular}{|c|c|}\hline  
\begin{tikzpicture}[use optics,baseline=0pt]
\node[halogen lamp] (quartz iode) at (0,0) {Q.I.};
\node[heat filter,right=0.5cm of quartz iode.aperture east] (AC) {};
\node[slit,right=0.75cm of AC] (fente) {};
\node[lens,right=2cm of fente] (L) {};
\node[screen,right=3cm of fente] (screen) {};
\end{tikzpicture}
&  
\parbox{10.5cm}{
\BS{begin}\AC{tikzpicture}[use optics] \\
\BS{node}[halogen lamp] (\blll{quartz iode}) at (0,0) \AC{Q.I.};\\
\BS{node}[heat filter,\RDD{right}=0.5cm \rouge{of} \blll{quartz iode}.aperture east] (\blll{AC}) \AC{};\\
\BS{node}[slit,\RDD{right}=0.75cm \rouge{of} \blll{AC}] (\blll{fente}) \AC{};\\
\BS{node}[lens,\RDD{right}=2cm  \rouge{of}  \blll{fente}] (L) \AC{};\\
\BS{node}[screen,\RDD{right}=3cm  \rouge{of}  \blll{fente}] (screen) \AC{};\\
\BS{end}\AC{tikzpicture}
}
\\ \hline 
\end{tabular} 




\SbSbSSCT{Cotation}{Dimensions indicating}

\noindent

\begin{tabular}{|c|c|c|}\hline 
\multicolumn{3}{|c|}{\BS{draw} (0,0) to[\RDD{short dim arrow}=\AC{\RDD{label}=2cm}] (2,0);}
\\  \hline  
\begin{tikzpicture}[use optics,blue]
\useasboundingbox (-1.5,-1.2) rectangle (3.5,1.2);
\draw[help lines] (-1,-1) grid (3,1);
\draw[line width=2pt,dashed] (0,0) -- (2,0);
\draw (0,0) to[dim arrow={label=2cm}] (2,0);
\end{tikzpicture}
&  
\begin{tikzpicture}[use optics,blue]
\useasboundingbox (-1.5,-1.2) rectangle (3.5,1.2);
\draw[help lines] (-1,-1) grid (3,1);
\draw[line width=2pt,dashed] (0,0) -- (2,0);
\draw (0,0) to[dim arrow={label'=2cm}] (2,0);
\end{tikzpicture}
&  
\begin{tikzpicture}[use optics,blue]
\useasboundingbox (-1.5,-1.2) rectangle (3.5,1.2);
\draw[help lines] (-1,-1) grid (3,1);
\draw[line width=2pt,dashed] (0,0) -- (2,0);
\draw (0,0) to[dim arrow={label=2cm,label style/.append style=red}] (2,0);
\end{tikzpicture}  

\\ \hline
[\RDD{dim arrow}=\AC{\RDD{label}=2cm}] & to[dim arrow=\AC{\RDD{label'}=2cm}] & [dim arrow=\{label=2cm, \\
 & & \RDD{label style}/.append style=red\}]
\\ \hline
\begin{tikzpicture}[use optics,blue]
\useasboundingbox (-1.5,-1.2) rectangle (3.5,2.2);
\draw[help lines] (-1,-1) grid (3,2);
\draw[line width=2pt,dashed] (0,0) -- (2,0);
\draw (0,0) to[dim arrow={label=2cm,raise=1cm}] (2,0);
\end{tikzpicture}
&
\begin{tikzpicture}[use optics,blue]
\useasboundingbox (-1.5,-1.2) rectangle (3.5,2.2);
\draw[help lines] (-1,-1) grid (3,2);
\draw[line width=2pt,dashed] (0,0) -- (2,0);
\draw (0,0) to[dim arrow={label=2cm,no raise},red] (2,0);
\end{tikzpicture} 
&
\begin{tikzpicture}[use optics,blue]
\useasboundingbox (-1.5,-1.2) rectangle (3.5,2.2);
\draw[help lines] (-1,-1) grid (3,2);
\draw[line width=2pt,dashed] (0,0) -- (2,0);
\draw (0,0) to[dim arrow'={label=2cm}] (2,0);
\end{tikzpicture}
\\ \hline
[dim arrow=\AC{label=2cm,\RDD{raise}=1cm}] & [dim arrow=\AC{label=2cm,\RDD{no raise}},red] & [\RDD{dim arrow'}=\AC{label=2cm}]
\\ \hline
\dft{ raise = 0.5cm} & & 
\\ \hline
\end{tabular}

\bigskip

\begin{tabular}{|c|c|}\hline 
\multicolumn{2}{|c|}{\BS{draw} (0,0) to[\RDD{short dim arrow}=\AC{label=2cm}] (2,0);}
\\  \hline  
\begin{tikzpicture}[use optics,blue]
\useasboundingbox (-1.5,-1.2) rectangle (3.5,1.2);
\draw[help lines] (-1,-1) grid (3,1);
\draw[line width=2pt,dashed] (0,0) -- (2,0);
\draw (0,0) to[short dim arrow={label=2cm}] (2,0);
\end{tikzpicture}
&
\begin{tikzpicture}[use optics,blue]
\useasboundingbox (-1.5,-1.2) rectangle (3.5,1.2);
\draw[help lines] (-1,-1) grid (3,1);
\draw[line width=2pt,dashed] (0,0) -- (2,0);
\draw (0,0) to[short dim arrow={label=2cm,arrow length=1cm}] (2,0);
\end{tikzpicture}
\\ \hline
[\RDD{short dim arrow}=\AC{label=2cm}] & 
[short dim arrow=\AC{label=2cm,\RDD{arrow length}=1cm}]
\\ \hline
& \dft{ arrow length= 5mm}
\\ \hline
\begin{tikzpicture}[use optics,blue]
\useasboundingbox (-1.5,-1.2) rectangle (3.5,1.2);
\draw[help lines] (-1,-1) grid (3,1);
\draw[line width=2pt,dashed] (0,0) -- (2,0);
\draw (0,0) to[short dim arrow={label=2cm,label near end}] (2,0);
\end{tikzpicture}
&
\begin{tikzpicture}[use optics,blue]
\useasboundingbox (-1.5,-1.2) rectangle (3.5,1.2);
\draw[help lines] (-1,-1) grid (3,1);
\draw[line width=2pt,dashed] (0,0) -- (2,0);
\draw (0,0) to[short dim arrow={label=2cm,label near middle}] (2,0);
\end{tikzpicture}

\\ \hline
[short dim arrow=\AC{label=2cm,\RDD{label near end}}] & [short dim arrow=\AC{label=2cm,\RDD{label near middle}}]
\\ \hline
\multicolumn{2}{|c|}{\dft{ label near start}}
\\ \hline
\end{tabular} 


\newpage

\SSCT{Les animations }{Animate a TikZ picture}

\label{anim}

%Insérer dans le préambule :

 \maboite{\BS{usepackage}\AC{animate} \cite {animate} }

\SbSSCT{Animation à partir de fichiers d'image }{Animation from picture files}

%\begin{verbatim}\animategraphics[<options>]{<frame rate>}{<file basename>}{<first>}{<last>} \end{verbatim}


\begin{tabular}{|c|c|} \hline 
\TFRGB{première image}{first frame} & \TFRGB{seconde et dernière image}{second and last frame}
\\ \hline
\includegraphics[width=3cm]{XXX1}
&  
\includegraphics[width=3cm]{XXX2}
\\ 
\hline \BS{includegraphics}\AC{XXX1} &  \BS{includegraphics}\AC{XXX2}\\ 
\hline 
\end{tabular} 


\begin{minipage}{7cm}
\animategraphics[controls,loop,autoplay]{4}{XXX}{1}{2}  
\end{minipage}\hfill
\begin{minipage}{7cm}
\begin{tabular}{|l@{:}l|}
\hline \BSS{animategraphics} &  \\ 
\hline [ controls, & \TFRGB{boutons de contrôle}{Inserts control buttons} \\ 
\hline loop &  \TFRGB{en boucle}{animation restarts automatically}\\ 
\hline autoplay ] & \TFRGB{auto démarrage}{Start animation automatically } \\ 
\hline \AC{4} &  \TFRGB{4 fois par seconde}{4 frame per second}\\ 
\hline \AC{XXX} & \TFRGB{base du nom fichier}{file base name} \\ 
\hline \AC{1} & \TFRGB{numero de début}{number of the first frame} \\ 
\hline \AC{2} & \TFRGB{numero de fin}{number of the last frame} \\ 
\hline 
\end{tabular} 

\end{minipage}

%--------------------------------------------------------------
\subsection{Animateinline}




\begin{minipage}{5cm}
\begin{center}
\begin{animateinline}[ controls,loop,autoplay]{5}%
\begin{tikzpicture} 
\fill[blue] (45:2) -- (135:.5)-- (225:2)--(315:.5) -- cycle;
\fill[blue] (45:.5) -- (135:2)-- (225:.5)--(315:2) -- cycle;
 \end{tikzpicture}
\newframe%
\begin{tikzpicture} 
\fill[blue] (0:2) -- (90:.5)-- (180:2)--(270:.5) -- cycle;
\fill[blue] (0:.5) -- (90:2)-- (180:.5)--(270:2) -- cycle;
 \end{tikzpicture}
\end{animateinline}% 
\end{center}
\end{minipage}\hfill
\begin{minipage}{12cm}
\ESS{animateinline}[controls,loop,autoplay]\AC{5} \\

\emph{\% \TFRGB{première image}{first frame }}\\
\BS{begin\AC{tikzpicture} }
\BS{fill}[blue] (45:2) - - (135:.5)- - (225:2)- -(315:.5) - - cycle;
\BS{fill}[blue] (45:.5) - - (135:2)- - (225:.5)- -(315:2) - - cycle;
 \BS{end\AC{tikzpicture}}

\emph{\% \TFRGB{deuxième}{second frame }}\\
\BSS{newframe} \\
\BS{begin\AC{tikzpicture} } \\
\BS{fill}[blue] (0:2) - - (90:.5)- - (180:2)- -(270:.5) - - cycle; \\
\BS{fill}[blue] (0:.5) - - (90:2)- - (180:.5)- -(270:2) - - cycle; \\
 \BS{end\AC{tikzpicture}} \\
 \\
\BS{end\AC{animateinline}} \\
\end{minipage}




\subsection{Multiframe}

\begin{minipage}{5cm}
\begin{center}
\begin{animateinline}[poster=first, controls, palindrome,autoplay]{12}%
\multiframe{29}{iAngle=80+10,Rdim=2.0+-0.2}{
\begin{tikzpicture} 
\fill[blue] (\iAngle+45:\Rdim) -- (\iAngle+135:.5)-- (\iAngle+225:\Rdim)--(\iAngle+315:.5) -- cycle;
\fill[blue] (\iAngle+45:.5) -- (\iAngle+135:\Rdim)-- (\iAngle+225:.5)--(\iAngle+315:\Rdim) -- cycle;
 \end{tikzpicture}
%\end{pspicture} 
}%
\end{animateinline}%
\end{center}
\end{minipage}\hfill
\begin{minipage}{12cm}
\BS{begin}\AC{animateinline}[poster=first,controls, palindrome]\AC{12} \\
\BSS{multiframe}\AC{29}\AC{{\color{red} iAngle}=80+10, {\color{red} Rdim}=2.0+-0.2}\{ \\
\BS{begin\AC{tikzpicture} } \\
\BS{fill}[blue] (\BS{iAngle}+45:\BS{Rdim}) - - (\BS{iAngle}+135:.5)- - (\BS{iAngle}+225:\BS{Rdim})- -(\BS{iAngle}+315:.5) - - cycle; \\
\BS{fill}[blue] (\BS{iAngle}+45:.5) - - (\BS{iAngle}+135:\BS{Rdim})- - (\BS{iAngle}+225:.5)- -(\BS{iAngle}+315:\BS{Rdim}) - - cycle; \\
 \BS{end\AC{tikzpicture}}
\} \\
\BS{end}\AC{animateinline}%
\end{minipage}
\bigskip

\TFRGB{L'initiale de la variable définit son type}{The first  letter of the variable name determines his type }

\begin{tabular}{|c|l|}
\hline  entier &  initiale : i ou I \\ 
\hline  réelles &  initiale : n, N, r ou R \\ 
\hline  longueurs & initiale : d ou D \\ 
\hline 
\end{tabular} 



\bigskip

%---------------------------------------------------------------------
\begin{minipage}{5cm}

\begin{animateinline}[autoplay,loop]{12}%
\multiframe{24}{iAngle=0+15,icol=0+5}{\begin{tikzpicture}[rotate=90] %
  \draw[line width=0pt] (-2,-2) rectangle(6,2); %
  \draw  (0,0) node[fill=white,circle,rotate=\iAngle] {\includegraphics[width=2cm]{LogoIUT}}  (0,0) circle (1);
   \draw (0,0) circle (1);
   \coordinate (abc) at (${sqrt(9-sin(\iAngle)*sin(\iAngle))+cos(\iAngle)}*(1,0)$) ;
   \coordinate (xyz) at (\iAngle:1);
   \draw[ultra thick] (0,0) --(xyz); 
   \draw[ultra thick] (xyz) -- (abc) ;
   \fill[color=blue!\icol] (abc)++(0.5,-1) rectangle (5,1) ;
   \draw[ultra thick] (abc) ++(0,-1) rectangle ++(.5,2) ;
   \draw[ultra thick]  (1.5,1) -- (5,1) -- (5,-1) -- (1.5,-1);
   \fill[red] (xyz) circle (4pt);
   \fill[red] (abc) circle (4pt); 
 \end{tikzpicture}}
\end{animateinline}
\end{minipage}\hfill
\begin{minipage}{12cm}
\BS{begin\AC{animateinline}}[autoplay,loop]\AC{12}\\
\BS{multiframe}\AC{24}\AC{iAngle=0+15,icol=0+5}\{\BS{begin\AC{tikzpicture}} \\
 \BS{draw}[line width=0pt] (-2,-3) rectangle(6,3); \\
  \BS{draw} (0,0) node[fill=white,circle,rotate=\BS{iAngle}] \\
   \AC{\BS{includegraphics}[width=2cm]\AC{LogoIUT}}  (0,0) circle (1);\\
   \BS{draw} (0,0) circle (1); \\
   \BS{coordinate} (abc) at (\$\AC{sqrt(9-sin(\BS{iAngle})*sin(\BS{iAngle}))+cos(\BS{iAngle})}*(1,0)\$) ; \\
   \BS{coordinate} (xyz) at (\BS{iAngle}:1); \\
   \BS{draw}[ultra thick] (0,0) - -(xyz); \\
   \BS{draw}[ultra thick] (xyz) - - (abc) ; \\
   \BS{fill}[color=blue{}!\BS{icol}] (abc)++(0.5,-1) rectangle (5,1) ; \\
   \BS{draw}[ultra thick] (abc) ++(0,-1) rectangle ++(.5,2) ; \\
   \BS{draw}[ultra thick]  (1.5,1) - - (5,1) - - (5,-1) - - (1.5,-1); \\
   \BS{fill}[red] (xyz) circle (4pt); \\
   \BS{fill}[red] (abc) circle (4pt); \\
 \BS{end\AC{tikzpicture}}\} \\
\BS{end\AC{animateinline}}

\end{minipage}


%\newpage
%
%\SSCT{Les tableaux de variations}{Variation of a function}
%
%\label{tabl}
%Insérer dans le préambule :

 \maboite{\BS{usepackage}\AC{tkz-tab}  \cite {tikstab}}


%\subsection{Déclaration du tableau}
\SbSSCT{Déclaration du tableau}{Creation of the table}

\begin{tabular}{|l|c|}\hline 
\begin{tikzpicture}
\tkzTabInit{1° ligne / 1 ,2° ligne /1 }{ a , b, c }
%\tkzTabLine{ 1, 2, 3 , 4,5 }
\end{tikzpicture}
\\ \hline 
\BS{begin}\AC{tikzpicture} \\
\BSS{tkzTabInit}\AC{1° ligne / 1 ,2° ligne /1 } \AC{ a , b, c } \\
% \BSS{tkzTabLine}\AC{ 1, 2, 3 , 4,5 } \\
\BS{end}\AC{tikzpicture}
 \\ \hline 
 \end{tabular} 

\subsubsection{Options}
 
%\paragraph{Hauteur des lignes}:

\begin{tabular}{|l|c|}\hline  
 \multicolumn{1}{|c|}{\textbf{\TFRGB{Hauteur des ligne}{Row width }} }
 \\ \hline

\begin{tikzpicture} \tkzTabInit{1° ligne /1  , 2° ligne /.5  , 3° ligne /1.5 }{a , b , c }\end{tikzpicture}
\\ \hline 
\BS{tikz}  \BSS{tkzTabInit}\AC{1° ligne {'\color{red}  /1}  , 2° ligne {\color{red}   /.5}  , 3° ligne {\color{red}  /1.5} }\AC{a , b , c };
\\ \hline 
\end{tabular} 
 
\bigskip
%\paragraph{Largeur de la première colonne } :

\begin{tabular}{|l|c|}\hline 
 \multicolumn{1}{|c|}{\textbf{\TFRGB{Largeur de la première colonne  }{First column width }} }
 \\ \hline
\begin{tikzpicture} 
\tkzTabInit[lgt=4]{ $x$ / 1}{ a , b , c  }
\end{tikzpicture}
\\ \hline 
\BS{tkzTabInit}[\RDD{lgt}=4]\AC{ $x$ / 1}\AC{ a , b , c  }; \\
\dft :  lgt==2 cm 
\\ \hline 
\end{tabular} 

\bigskip

%\paragraph{Espacement entre deux valeurs} :
%\Par{Espacement entre deux valeurs}{Space between two values} :


\begin{tabular}{|l|c|}\hline 
 \multicolumn{1}{|c|}{\textbf{\TFRGB{Espacement entre deux valeurs}{Space between two values}} }
 \\ \hline

\begin{tikzpicture} 
\tkzTabInit[espcl=2]{ $x$ / 1}{ a , b , c  }
\end{tikzpicture}
\\ \hline 
\BS{tkzTabInit}[\RDD{espcl}=1]\AC{ $x$ / 1}\AC{ a , b , c  }; \\
\dft :  espcl=2 cm
\\ \hline 
\end{tabular}


\bigskip
%\paragraph{Marge de début et de fin} :

\begin{tabular}{|l|c|}\hline 
 \multicolumn{1}{|c|}{\textbf{\TFRGB{Marge de début et de fin  }{Margin  }} }
 \\ \hline
\begin{tikzpicture} 
\tkzTabInit[deltacl=2]{ $x$ / 1}{ a , b , c  }
\end{tikzpicture} 
\\ \hline 
\BS{tkzTabInit}[\RDD{deltacl}=1]\AC{ $x$ / 1}\AC{ a , b , c  }; \\
\dft :  deltacl=0.5 cm
\\ \hline 
\end{tabular}



\newpage
%\paragraph{\'Epaisseur des lignes du tableau } : 

\begin{tabular}{|l|c|}\hline 
 \multicolumn{1}{|c|}{\textbf{\TFRGB{\'Epaisseur des lignes du tableau }{Line width }} }
 \\ \hline
\begin{tikzpicture} 
\tkzTabInit[lw=2pt]{ $x$ / 1}{ a , b , c  }
\end{tikzpicture} 
\\ \hline 
\BS{tkzTabInit}[\RDD{dlw}=2pt]\AC{ $x$ / 1}\AC{ a , b , c  }; \\
\dft :  lw=0,4 pt
\\ \hline 
\end{tabular}



\bigskip
%\paragraph{Absence de cadre} :

\begin{tabular}{|l|c|}\hline
 \multicolumn{1}{|c|}{\textbf{\TFRGB{Absence de cadre}{No cadre}} }
 \\ \hline 
\begin{tikzpicture} 
\tkzTabInit[nocadre]{ $x$ / 1}{ a , b , c  }
\end{tikzpicture} 
\\ \hline 
\BS{tkzTabInit}[nocadre]\AC{ $x$ / 1}\AC{ a , b , c  }; \\
\dft :  nocadre=false
\\ \hline 
\end{tabular}


\bigskip
%\paragraph{Mise en couleur}:\\
\begin{tabular}{|c|c|}\hline
 \multicolumn{2}{|c|}{\textbf{\TFRGB{Mise en couleur  }{ Coloring }} }
 \\ \hline 
\multicolumn{2}{|c|}{ \BS{tkzTabInit} [\RDD{color},\RDD{colorT} = yellow]\AC{1°ligne/1 , 2°ligne/1}\AC{ a , b  }   }\\ 
\hline
\begin{tikzpicture}
\tkzTabInit[color,colorT = yellow]{ 1°ligne/1 , 2°ligne/1}{ a , b   }
\end{tikzpicture}
 &
\begin{tikzpicture}
\tkzTabInit[color,colorC = cyan]{ 1°ligne/1 , 2°ligne/1}{ a , b }
\end{tikzpicture}
\\ \hline
[color,\RDD{colorT} = yellow] & [color,\RDD{colorC} = cyan]
\\ \hline
\begin{tikzpicture}
\tkzTabInit[color,colorL = green]{1°ligne/1 , 2°ligne/1}{ a , b  }
\end{tikzpicture}
&
\begin{tikzpicture}
\tkzTabInit[color,colorV = magenta]{1°ligne/1 , 2°ligne/1}{ a , b  }
\end{tikzpicture}
\\ \hline 
[color,\RDD{colorL} = green] & [color,\RDD{colorV} = magenta]
\\ \hline 
\multicolumn{2}{|c|}{ \dft : color = false \hspace{1cm}  colorT=colorC=colorL=colorV =white   }
\\ \hline 
\end{tabular} 




%\subsection{Création d'une ligne de signes}
\SbSSCT{Création d'une ligne de signes}{Creation of a sign row}

\begin{tabular}{|c|c|}\hline  
\begin{tikzpicture}
\tkzTabInit[espcl=1.5]
{$x$ / 1 ,$f(x)$ /1 }%
{ a , b, c  }
\tkzTabLine{ t, 2, t ,4 ,t }
\end{tikzpicture}
&  
\begin{tikzpicture}
\tkzTabInit[espcl=1.5]
{$x$ / 1 ,$f(x)$ /1 }%
{ a , b, c  }
\tkzTabLine{ z, 2, z ,4 ,z }
\end{tikzpicture}
\\ \hline  
\BSS{tkzTabLine}\AC{ {\color{red}  t}, 2,{\color{red}  t} ,4 ,{\color{red}  t} } & \BSS{tkzTabLine}\AC{ {\color{red}  z}, 2, {\color{red}  z} ,4 ,{\color{red}  z} } 
\\ \hline  
\begin{tikzpicture}
\tkzTabInit[espcl=1.5]
{$x$ / 1 ,$f(x)$ /1 }%
{ a , b, c  }
\tkzTabLine{ d, 2, d ,4 ,d }
\end{tikzpicture}
&  
\begin{tikzpicture}
\tkzTabInit[espcl=1.5]
{$x$ / 1 ,$f(x)$ /1 }%
{ a , b, c  }
\tkzTabLine{ 1,h, 3,4 ,5}
\end{tikzpicture}
\\ \hline
\BSS{tkzTabLine}\AC{ {\color{red}  d}, 2, {\color{red}  d} ,4 ,{\color{red}  d} } & \BSS{tkzTabLine}\AC{ 1, {\color{red}  h}, 3 ,4 ,5 } 
\\ \hline 
\end{tabular} 


\newpage
%\paragraph{Exemple}:

\begin{tabular}{|l|c|}\hline 
 \multicolumn{1}{|c|}{\textbf{\TFRGB{Exemple }{Example }} }
 \\ \hline
\begin{tikzpicture}
\tkzTabInit[espcl=1.5]{$x$ / 1 ,$f(x)$ /1 }%
{ $-\infty$ , -4, 4 , 10 , $+\infty$ }
\tkzTabLine{ t,+, d ,h ,d,-,z,+ }
\end{tikzpicture}
\\ \hline 
\BS{begin}\AC{tikzpicture} \\
\BS{tkzTabInit}[espcl=1.5]\AC{\$x\$ / 1 ,\$f(x)\$ /1 } %\\
\AC{ $-\infty$ , -4, 4 , 10 , $+\infty$ } \\
\BS{tkzTabLine}\AC{ t,+, d ,h ,d,-,z,+ } \\
\BS{end}\AC{tikzpicture}
\\ \hline 
\end{tabular}

%\subsection{Création d'une ligne de variations}
\SbSSCT{Création d'une ligne de variations}{Creation of a variation row}

\begin{tabular}{|c|c|}\hline  
\begin{tikzpicture}
\tkzTabInit[espcl=1.5]
{$x$ / 1 ,$f(x)$ /1 }%
{ a , b, c  }
%\tkzTabLine{ , t, , ,t }
\tkzTabVar{+/1 , -/2}
\end{tikzpicture}
&  
\begin{tikzpicture}
\tkzTabInit[espcl=1.5]
{$x$ / 1 ,$f(x)$ /1 }%
{ a , b, c  }
%\tkzTabLine{ 1, z, 3 ,4 ,z }
\tkzTabVar{-/1 , +/2}
\end{tikzpicture}
\\ \hline  
\BSS{tkzTabVar}\AC{ {\color{red}  +/}1 , {\color{red}  -/}2} & \BSS{tkzTabVar}\AC{ {\color{red}  -/}1 , {\color{red}  +/}2} 
%\tkzTabVar{+/1 , +/2}
\\ \hline  
\begin{tikzpicture}
\tkzTabInit[espcl=1.5]
{$x$ / 1 ,$f(x)$ /1 }%
{ a , b, c  }
%\tkzTabLine{ 1, d, 3 ,4 ,d }
\tkzTabVar{-/1 , -/2}
\end{tikzpicture}
&  
\begin{tikzpicture}
\tkzTabInit[espcl=1.5]
{$x$ / 1 ,$f(x)$ /1 }%
{ a , b, c  }
%\tkzTabLine{ 1,h, 3,4 ,h }
\tkzTabVar{+/1 , +/2}
\end{tikzpicture}
\\ \hline
\BSS{tkzTabVar}\AC{{\color{red}  -/}1 , {\color{red}  -/}2} & \BSS{tkzTabVar}\AC{ {\color{red}  +/}1 , {\color{red}  +/}2 } 
\\ \hline 
\end{tabular}

\bigskip

\begin{tabular}{|c|c|}\hline  
\begin{tikzpicture}
\tkzTabInit[espcl=1.5]
{$x$ / 1 ,$f(x)$ /1 }%
{ a , b, c  }
%\tkzTabLine{ , t, , ,t }
\tkzTabVar{+C/1 , -/2}
\end{tikzpicture}
&  
\begin{tikzpicture}
\tkzTabInit[espcl=1.5]
{$x$ / 1 ,$f(x)$ /1 }%
{ a , b, c  }
%\tkzTabLine{ 1, z, 3 ,4 ,z }
\tkzTabVar{-C/1 , +/2}
\end{tikzpicture}
\\ \hline  
\BSS{tkzTabVar}\AC{ {\color{red}  +C/}1 , -/2} & \BSS{tkzTabVar}\AC{ {\color{red}  -C/}1 , +/2} 
%\tkzTabVar{+/1 , +/2}
\\ \hline  
\begin{tikzpicture}
\tkzTabInit[espcl=1.5]
{$x$ / 1 ,$f(x)$ /1 }%
{ a , b, c  }
%\tkzTabLine{ 1, d, 3 ,4 ,d }
\tkzTabVar{+/1 , -C/2}
\end{tikzpicture}
&  
\begin{tikzpicture}
\tkzTabInit[espcl=1.5]
{$x$ / 1 ,$f(x)$ /1 }%
{ a , b, c  }
%\tkzTabLine{ 1,h, 3,4 ,h }
\tkzTabVar{-/1 , +C/2}
\end{tikzpicture}
\\ \hline
\BSS{tkzTabVar}\AC{-/1 , {\color{red}  -C/}2} & \BSS{tkzTabVar}\AC{ +/1 , {\color{red}  +C/}2 } 
\\ \hline 
\end{tabular}


\bigskip

\begin{tabular}{|c|c|}\hline  
\begin{tikzpicture}
\tkzTabInit[espcl=1.5]
{$x$ / 1 ,$f(x)$ /1 }%
{ a , b, c  }
%\tkzTabLine{ , t, , ,t }
\tkzTabVar{+H/1 , -/2}
\end{tikzpicture}
&  
\begin{tikzpicture}
\tkzTabInit[espcl=1.5]
{$x$ / 1 ,$f(x)$ /1 }%
{ a , b, c  }
%\tkzTabLine{ 1, z, 3 ,4 ,z }
\tkzTabVar{-H/1 , +/2}
\end{tikzpicture}
\\ \hline  
\BSS{tkzTabVar}\AC{ {\color{red}  +H/1} , -/2} & \BSS{tkzTabVar}\AC{ {\color{red}  -H/}1 , +/2} 
%\tkzTabVar{+/1 , +/2}
\\ \hline  
\begin{tikzpicture}
\tkzTabInit[espcl=1.5]
{$x$ / 1 ,$f(x)$ /1 }%
{ a , b, c  }
%\tkzTabLine{ 1, d, 3 ,4 ,d }
\tkzTabVar{+/1 , -H/2}
\end{tikzpicture}
&  
\begin{tikzpicture}
\tkzTabInit[espcl=1.5]
{$x$ / 1 ,$f(x)$ /1 }%
{ a , b, c  }
%\tkzTabLine{ 1,h, 3,4 ,h }
\tkzTabVar{-/1 , +H/2}
\end{tikzpicture}
\\ \hline
\BSS{tkzTabVar}\AC{-/1 , {\color{red}  -H/}2} & \BSS{tkzTabVar}\AC{ +/1 , {\color{red}  +H/}2 } 
\\ \hline 
\end{tabular}

\bigskip

\begin{tabular}{|c|c|}\hline  
\begin{tikzpicture}
\tkzTabInit[espcl=1.5]
{$x$ / 1 ,$f(x)$ /1 }%
{ a , b, c  }
%\tkzTabLine{ , t, , ,t }
\tkzTabVar{+D/1 , -/2}
\end{tikzpicture}
&  
\begin{tikzpicture}
\tkzTabInit[espcl=1.5]
{$x$ / 1 ,$f(x)$ /1 }%
{ a , b, c  }
%\tkzTabLine{ 1, z, 3 ,4 ,z }
\tkzTabVar{-D/1 , +/2}
\end{tikzpicture}
\\ \hline  
\BSS{tkzTabVar}\AC{ {\color{red}  +D/}1 , -/2} & \BSS{tkzTabVar}\AC{ {\color{red}  -D/}1 , +/2} 
%\tkzTabVar{+/1 , +/2}
\\ \hline  
\begin{tikzpicture}
\tkzTabInit[espcl=1.5]
{$x$ / 1 ,$f(x)$ /1 }%
{ a , b, c  }
%\tkzTabLine{ 1, d, 3 ,4 ,d }
\tkzTabVar{+/1 , -D/2}
\end{tikzpicture}
&  
\begin{tikzpicture}
\tkzTabInit[espcl=1.5]
{$x$ / 1 ,$f(x)$ /1 }%
{ a , b, c  }
%\tkzTabLine{ 1,h, 3,4 ,h }
\tkzTabVar{-/1 , +D/2}
\end{tikzpicture}
\\ \hline
\BSS{tkzTabVar}\AC{-/1 , {\color{red}  -D/}2} & \BSS{tkzTabVar}\AC{ +/1 , {\color{red}  +D/}2 } 
\\ \hline 
\end{tabular}

\bigskip

\begin{tabular}{|c|c|}\hline  
\begin{tikzpicture}
\tkzTabInit[espcl=1.5]
{$x$ / 1 ,$f(x)$ /1 }%
{ a , b, c  }
%\tkzTabLine{ , t, , ,t }
\tkzTabVar{D+/1 , -/2}
\end{tikzpicture}
&  
\begin{tikzpicture}
\tkzTabInit[espcl=1.5]
{$x$ / 1 ,$f(x)$ /1 }%
{ a , b, c  }
%\tkzTabLine{ 1, z, 3 ,4 ,z }
\tkzTabVar{D-/1 , +/2}
\end{tikzpicture}
\\ \hline  
\BSS{tkzTabVar}\AC{ {\color{red}  D+/}1 , -/2} & \BSS{tkzTabVar}\AC{{\color{red}  D-/}1 , +/2} 
%\tkzTabVar{+/1 , +/2}
\\ \hline  
\begin{tikzpicture}
\tkzTabInit[espcl=1.5]
{$x$ / 1 ,$f(x)$ /1 }%
{ a , b, c  }
%\tkzTabLine{ 1, d, 3 ,4 ,d }
\tkzTabVar{+/1 , D-/2}
\end{tikzpicture}
&  
\begin{tikzpicture}
\tkzTabInit[espcl=1.5]
{$x$ / 1 ,$f(x)$ /1 }%
{ a , b, c  }
%\tkzTabLine{ 1,h, 3,4 ,h }
\tkzTabVar{-/1 , D+/2}
\end{tikzpicture}
\\ \hline
\BSS{tkzTabVar}\AC{-/1 , {\color{red} D-/}2} & \BSS{tkzTabVar}\AC{ +/1 , {\color{red}  D+/}2 } 
\\ \hline 
\end{tabular}

\bigskip

\begin{tabular}{|c|c|}\hline  
\begin{tikzpicture}
\tkzTabInit[espcl=1.5]
{$x$ / 1 ,$f(x)$ /1 }%
{ a , b, c  }
%\tkzTabLine{ , t, , ,t }
\tkzTabVar{+DH/1 , -/2}
\end{tikzpicture}
&  
\begin{tikzpicture}
\tkzTabInit[espcl=1.5]
{$x$ / 1 ,$f(x)$ /1 }%
{ a , b, c  }
%\tkzTabLine{ 1, z, 3 ,4 ,z }
\tkzTabVar{-DH/1 , +/2}
\end{tikzpicture}
\\ \hline  
\BSS{tkzTabVar}\AC{ {\color{red}  +DH/}1 , -/2} & \BSS{tkzTabVar}\AC{ {\color{red}  -DH/}1 , +/2} 
%\tkzTabVar{+/1 , +/2}
\\ \hline  
\begin{tikzpicture}
\tkzTabInit[espcl=1.5]
{$x$ / 1 ,$f(x)$ /1 }%
{ a , b, c  }
%\tkzTabLine{ 1, d, 3 ,4 ,d }
\tkzTabVar{+/1 , -DH/2}
\end{tikzpicture}
&  
\begin{tikzpicture}
\tkzTabInit[espcl=1.5]
{$x$ / 1 ,$f(x)$ /1 }%
{ a , b, c  }
%\tkzTabLine{ 1,h, 3,4 ,h }
\tkzTabVar{-/1 , +DH/2}
\end{tikzpicture}
\\ \hline
\BSS{tkzTabVar}\AC{-/1 , {\color{red}  -DH/}2} & \BSS{tkzTabVar}\AC{ {\color{red}  +DH/}1 , +/2 } 
\\ \hline 
\end{tabular}

\bigskip

\begin{tabular}{|c|c|}\hline  
\begin{tikzpicture}
\tkzTabInit[espcl=1.5]{$x$ / 1 ,$f(x)$ /1 }{ a , b, c  }
%\tkzTabLine{ , t, , ,t }
\tkzTabVar{+CH/1 , -/2}
\end{tikzpicture}
&  
\begin{tikzpicture}
\tkzTabInit[espcl=1.5]{$x$ / 1 ,$f(x)$ /1 }{ a , b, c  }
%\tkzTabLine{ 1, z, 3 ,4 ,z }
\tkzTabVar{-CH/1 , +/2}
\end{tikzpicture}
\\ \hline  
\BSS{tkzTabVar}\AC{ {\color{red}  +CH/}1 , -/2} & \BSS{tkzTabVar}\AC{ {\color{red}  -CH/}1 , +/2} 
%\tkzTabVar{+/1 , +/2}
\\ \hline  
\begin{tikzpicture}
\tkzTabInit[espcl=1.5]{$x$ / 1 ,$f(x)$ /1 }{ a , b, c  }
\tkzTabVar{+/1 , -CH/2}
\end{tikzpicture}
&  
\begin{tikzpicture}
\tkzTabInit[espcl=1.5]{$x$ / 1 ,$f(x)$ /1 }{ a , b, c  }
\tkzTabVar{-/1 , +CH/2}
\end{tikzpicture}
\\ \hline
\BSS{tkzTabVar}\AC{-/1 , {\color{red}  -CH/}2} & \BSS{tkzTabVar}\AC{ +/1 , {\color{red}  +CH/}2 } 
\\ \hline 
\end{tabular}

\bigskip

\begin{tabular}{|c|c|}\hline  
\begin{tikzpicture}
\tkzTabInit[espcl=1.5]{$x$ / 1 ,$f(x)$ /1 }{ a , b, c  }
\tkzTabVar{-/1 , +D-/2 , +/3}
\end{tikzpicture}
&  
\begin{tikzpicture}
\tkzTabInit[espcl=1.5]{$x$ / 1 ,$f(x)$ /1 }{ a , b, c  }
\tkzTabVar{+/1 , -D+/2 , -/3}
\end{tikzpicture}
\\ \hline  
\BSS{tkzTabVar}\AC{ -/1 , {\color{red}  +D-/}2 , +/3} & \BSS{tkzTabVar}\AC{ +/1 , {\color{red}  -D+/}2 , -/3} 
\\ \hline  
\begin{tikzpicture}
\tkzTabInit[espcl=1.5]{$x$ / 1 ,$f(x)$ /1 }{ a , b, c  }
\tkzTabVar{+/1 , -D-/2 , +/3}
\end{tikzpicture}
&  
\begin{tikzpicture}
\tkzTabInit[espcl=1.5]{$x$ / 1 ,$f(x)$ /1 }{ a , b, c  }
\tkzTabVar{-/1 , +D+/2 , -/3}
\end{tikzpicture}
\\ \hline
\BSS{tkzTabVar}\AC{+/1 , {\color{red}  -D-/}2 , +/3} & \BSS{tkzTabVar}\AC{-/1 , {\color{red}  +D+/}2 , -/3 } 
\\ \hline 
\end{tabular}

\bigskip

\begin{tabular}{|c|c|}\hline  
\begin{tikzpicture}
\tkzTabInit[espcl=1.5]{$x$ / 1 ,$f(x)$ /1 }{ a , b, c  }
\tkzTabVar{-/1 , +CD-/2 , +/3}
\end{tikzpicture}
&  
\begin{tikzpicture}
\tkzTabInit[espcl=1.5]{$x$ / 1 ,$f(x)$ /1 }{ a , b, c  }
\tkzTabVar{+/1 , -CD+/2 , -/3}
\end{tikzpicture}
\\ \hline  
\BSS{tkzTabVar}\AC{ -/1 , {\color{red}  +CD-/}2 , +/3} & \BSS{tkzTabVar}\AC{ +/1 , {\color{red}  -CD+/}2 , -/3} 
\\ \hline  
\begin{tikzpicture}
\tkzTabInit[espcl=1.5]{$x$ / 1 ,$f(x)$ /1 }{ a , b, c  }
\tkzTabVar{+/1 , -CD-/2 , +/3}
\end{tikzpicture}
&  
\begin{tikzpicture}
\tkzTabInit[espcl=1.5]{$x$ / 1 ,$f(x)$ /1 }{ a , b, c  }
\tkzTabVar{-/1 , +CD+/2 , -/3}
\end{tikzpicture}
\\ \hline
\BSS{tkzTabVar}\AC{+/1 , {\color{red}  -CD-/}2 , +/3} & \BSS{tkzTabVar}\AC{-/1 , {\color{red}  +CD+/}2 , -/3 } 
\\ \hline 
\end{tabular}

\bigskip

\begin{tabular}{|c|c|}\hline  
\begin{tikzpicture}
\tkzTabInit[espcl=1.5]{$x$ / 1 ,$f(x)$ /1 }{ a , b, c  }
\tkzTabVar{-/1 , +DC-/2 , +/3}
\end{tikzpicture}
&  
\begin{tikzpicture}
\tkzTabInit[espcl=1.5]{$x$ / 1 ,$f(x)$ /1 }{ a , b, c  }
\tkzTabVar{+/1 , -DC+/2 , -/3}
\end{tikzpicture}
\\ \hline  
\BSS{tkzTabVar}\AC{ -/1 , {\color{red}  +DC-/}2 , +/3} & \BSS{tkzTabVar}\AC{ +/1 , {\color{red}  -DC+/}2 , -/3} 
\\ \hline  
\begin{tikzpicture}
\tkzTabInit[espcl=1.5]{$x$ / 1 ,$f(x)$ /1 }{ a , b, c  }
\tkzTabVar{+/1 , -DC-/2 , +/3}
\end{tikzpicture}
&  
\begin{tikzpicture}
\tkzTabInit[espcl=1.5]{$x$ / 1 ,$f(x)$ /1 }{ a , b, c  }
\tkzTabVar{-/1 , +DC+/2 , -/3}
\end{tikzpicture}
\\ \hline
\BSS{tkzTabVar}\AC{+/1 , {\color{red}  -DC-/}2 , +/3} & \BSS{tkzTabVar}\AC{-/1 , {\color{red}  +DC+/}2 , -/3 } 
\\ \hline 
\end{tabular}

\bigskip

\begin{tabular}{|c|c|}\hline  
\begin{tikzpicture}
\tkzTabInit[espcl=1.5]{$x$ / 1 ,$f(x)$ /1 }{ a , b, c  }
\tkzTabVar[color=red]{-/1 , +V-/2 , +/3}
\end{tikzpicture}
&  
\begin{tikzpicture}
\tkzTabInit[espcl=1.5]{$x$ / 1 ,$f(x)$ /1 }{ a , b, c  }
\tkzTabVar[color=red]{+/1 , -V+/2 , -/3}
\end{tikzpicture}
\\ \hline  
\BSS{tkzTabVar}\AC{ -/1 , {\color{red}  +V-/}2 , +/3} & \BSS{tkzTabVar}\AC{ +/1 , {\color{red}  -V+/}2 , -/3} 
\\ \hline  
\begin{tikzpicture}
\tkzTabInit[espcl=1.5]{$x$ / 1 ,$f(x)$ /1 }{ a , b, c  }
\tkzTabVar[color=red]{+/1 , -V-/2 , +/3}
\end{tikzpicture}
&  
\begin{tikzpicture}
\tkzTabInit[espcl=1.5]{$x$ / 1 ,$f(x)$ /1 }{ a , b, c  }
\tkzTabVar[color=red]{-/1 , +V+/2 , -/3}
\end{tikzpicture}
\\ \hline
\BSS{tkzTabVar}\AC{+/1 , {\color{red}  -V-/}2 , +/3} & \BSS{tkzTabVar}\AC{-/1 , {\color{red}  +V+/}2 , -/3 } 
\\ \hline 
\end{tabular}

\newpage

%\paragraph{Mise en évidence d'une valeur} :

\begin{tabular}{|c|c|}\hline
 \multicolumn{1}{|c|}{\textbf{\TFRGB{Mise en évidence d'une valeur  }{Emphasizing a value }} }
 \\ \hline  
\begin{tikzpicture}
\tkzTabInit[espcl=1.5]{$x$ / 1 ,$f(x)$ /1 }{ a , b, c  }
\tkzTabVar[color=red]{+/1 , -V-/\colorbox{yellow}{2} , +/3}
\end{tikzpicture}
\\ \hline 
\BS{tkzTabVar}\AC{+/1 , -V-/\BSS{colorbox}\AC{yellow}\AC{2} , +/3}
\\ \hline 
\end{tabular}

\bigskip

%\paragraph{Variation sur plusieurs colonnes}:

\begin{tabular}{|c|c|}\hline
 \multicolumn{1}{|c|}{\textbf{\TFRGB{Variation sur plusieurs colonnes }{Multicolumn variation }} }
 \\ \hline   
\begin{tikzpicture}
\tkzTabInit[espcl=1.5,color]{$x$ / 1 ,$f(x)$ /1 }{ a , b, c  }
\tkzTabVar[color=red]{-/1 , R/ , +/3}
\end{tikzpicture}
\\ \hline 
\BS{tkzTabVar}\AC{-/1 , {\color{red}  R/} , +/3}
\\ \hline 
\end{tabular}

\bigskip
%\paragraph{Valeurs intermédiaires}:

\begin{tabular}{|c|c|}\hline 
 \multicolumn{2}{|c|}{\textbf{\TFRGB{Valeurs intermédiaires }{Intermediate values }} }
 \\ \hline   
\begin{tikzpicture}
\tkzTabInit[espcl=1.5]{$x$ / 1 ,$f(x)$ /1 }{ a , b, c  }
%\tkzTabLine{d,+,}%
\tkzTabVar{ - / 1 , R/ , + / 3 }
\tkzTabVal{1}{3}{0.25}{A}{x}
\end{tikzpicture}
&
\begin{tikzpicture}
\tkzTabInit[espcl=1.5]{$x$ / 1 ,$f(x)$ /1 }{ a , b, c  }
%\tkzTabLine{d,+,}%
\tkzTabVar{ - / 1 , R/ , + / 3 }
\tkzTabVal{1}{3}{0.75}{\colorbox{yellow}{A}}{\colorbox{yellow}{x}}
\end{tikzpicture}
\\ \hline 
\BSS{tkzTabVal}\AC{1}\AC{3}\AC{0.25}\AC{A}\AC{x} & \BSS{tkzTabVal}\AC{1}\AC{3}\AC{0.75}\AC{A}\AC{x}
\\ \hline 
\end{tabular}
\bigskip

\begin{tabular}{|c|c|}\hline 
\begin{tikzpicture}
\tkzTabInit[espcl=1.5]{$x$ / 1, /1 ,$f(x)$ /1 }{ a , b, c  }
\tkzTabVar{   }
\tkzTabVar{ - / 1 , R/ , + / 3 }
\tkzTabVal[draw]{1}{3}{0.33}{A}{x}
\end{tikzpicture}
\\ \hline 
\BSS{tkzTabVal}[\RDD{draw}]\AC{1}\AC{3}\AC{0.25}\AC{A}\AC{x}
\\ \hline 
\end{tabular}

\bigskip 
%\paragraph{Ajout d’images} :

\begin{tabular}{|c|c|}\hline
  \multicolumn{2}{|c|}{\textbf{\TFRGB{Ajout d'images }{Picture insertion }} }
  \\ \hline 
\begin{tikzpicture}
\tkzTabInit[espcl=1.5]{$x$ / 1 ,$f(x)$ /1 }{ a , b, c,d  }
%\tkzTabLine{d,+,}%
\tkzTabVar{ - / 1 , R/  , R/, + / 3 }
\tkzTabIma{1}{4}{2}{x}
\end{tikzpicture}
&
\begin{tikzpicture}
\tkzTabInit[espcl=1.5]{$x$ / 1 ,$f(x)$ /1 }{ a , b, c,d  }
%\tkzTabLine{d,+,}%
\tkzTabVar{ - / 1 , R/  , R/, + / 3 }
\tkzTabIma{1}{4}{3}{x}
\end{tikzpicture}
\\ \hline 
\BSS{tkzTabIma}\AC{1}\AC{4}\AC{{\color{red}  2}}\AC{x} & \BSS{tkzTabIma}\AC{1}\AC{4}\AC{{\color{red}  3}}\AC{x}
\\ \hline 
\end{tabular}

%%
%%\newpage
%%
%%\section[Créer un dessin en 3D]{Créer un dessin en 3D  }
%

%%\subsection{Les objets en 3D}
%%

%%\newpage
%%\subsection[Créer un graphe en 3D]{Créer un graphe en 3D } 
%%

\newpage

%%===============================================


\SSCT{Les modules étudiés dans ce document}{Packages studied in this document}


\begin{tabular}{|c|c|l c|}\hline 
\multicolumn{4}{|c|}{ \textbf{\TFRGB{module de base TikZ}{Basic TikZ package} : } }
\\ \hline

\TFRGB{nom}{name} & \TFRGB{A insérer dans le préambule}{Load package}& documentation \footnotemark[1] 	& \\  \hline 
tikz & \BS{usepackage}\AC{tikz}  	& pgfmanual.pdf			& \DGB \\

\hline 
\end{tabular} 

\bigskip

\begin{tabular}{|c|c|l c|}\hline 
\multicolumn{4}{|c|}{ \textbf{\TFRGB{Autres modules}{Other packages}} }
\\ \hline
\TFRGB{nom}{name} & \TFRGB{voir page}{see page} & documentation  \footnotemark[2] 	& \\  \hline 
animate 	& \pageref{anim} 	& animate.pdf 			& \DGB \\
tikz-optics 	& \pageref{optics} 	& tikz-optics.pdf 			& \DFR \\
pgfplots 	& \pageref{pgfplots} & pgfplots.pdf 		& \DGB \\
tikzpeople  & \pageref{people} 	& tikzpeople.pdf 		& \DGB \\
tikzducks  & \pageref{ducks} 	& tikzducks-doc.pdf 		& \DGB \\
tikzsymbols  & \pageref{symbol} 	& tikzsymbols.pdf 		& \DGB \\
tkz-tab  	& \pageref{tabl} 	& tkz-tab-screen.pdf 	& \DFR \\
\hline 
\end{tabular} 
\bigskip



\begin{tabular}{|l|c|l|}\hline 
\multicolumn{3}{|c|}{ \textbf{\TFRGB{Compléments optionnels}{Optional library} (documentation : pgfmanual.pdf)} }
\\ \hline
\TFRGB{nom}{name} 				& \TFRGB{voir page}{see page}						& \TFRGB{A insérer dans le préambule}{Load package}\\ \hline 
angles & \pageref{lib-angles} &  \BS{usetikzlibrary}\AC{angles}\\
arrows.meta	& \pageref{lib-arrows.meta}	&  \BS{usetikzlibrary}\AC{arrows.meta}\\
bending				& \pageref{lib-bending}			&  \BS{usetikzlibrary}\AC{bending}
\\
backgrounds			& \pageref{lib-bkgd} 			&  \BS{usetikzlibrary}\AC{backgrounds}
\\
calc				& \pageref{lib-calc}			&  \BS{usetikzlibrary}\AC{calc}
\\
chains			& \pageref{lib-chains} 			& \BS{usetikzlibrary}\AC{chains} 
\\
circuits.ee.IEC				& \pageref{lib-ee}			&  \BS{usetikzlibrary}\AC{circuits.ee.IEC}
\\
circuits.logic.IEC	& \pageref{lib-gate}			&  \BS{usetikzlibrary}\AC{circuits.logic.IEC}
\\ 
circuits.logic.US	& \pageref{lib-gate}			&  \BS{usetikzlibrary}\AC{circuits.logic.US}
\\ 
circuits.logic.CDH	& \pageref{lib-gate}			&  \BS{usetikzlibrary}\AC{circuits.logic.CDH}
\\ 
fit & \pageref{lib-fit} 	& \BS{usetikzlibrary}\AC{fit} 
\\
decorations.footprints & \pageref{lib-footprints} 	& \BS{usetikzlibrary}\AC{decorations.footprints} 
\\
decorations.fractals & \pageref{lib-fractals} 		& \BS{usetikzlibrary}\AC{decorations.fractals} 
\\
decorations.markings & \pageref{lib-mark} 			& \BS{usetikzlibrary}\AC{decorations.markings} 
\\
decorations.pathmorphing  & \pageref{lib-morph}		& \BS{usetikzlibrary}\AC{decorations.pathmorphing}
\\
decorations.pathreplacing & \pageref{lib-replac}	& \BS{usetikzlibrary}\AC{decorations.pathreplacing} 
\\
decorations.shapes & \pageref{lib-shapes} 			& \BS{usetikzlibrary}\AC{decorations.shapes} 
\\
decorations.text & \pageref{lib-text} 				& \BS{usetikzlibrary}\AC{decorations.text} 
\\
fadings 			& \pageref{lib-fadings}			&  \BS{usetikzlibrary}\AC{fadings }
\\
intersections		& \pageref{lib-intersections}	&  \BS{usetikzlibrary}\AC{intersections}
\\
matrix			& \pageref{lib-matrix} 			& \BS{usetikzlibrary}\AC{matrix} 
\\
patterns			& \pageref{lib-patterns}		&  \BS{usetikzlibrary}\AC{patterns}
\\
plotmarks			& \pageref{plotmarks} 			&  \BS{usetikzlibrary}\AC{plotmarks}
\\
positioning			& \pageref{lib-pos} 			&  \BS{usetikzlibrary}\AC{positioning}
\\ 
scopes				& \pageref{lib-scopes}			&  \BS{usetikzlibrary}\AC{scopes}
\\
shadings			& \pageref{lib-shadings}		&  \BS{usetikzlibrary}\AC{shadings}
\\
shapes.arrows		& \pageref{lib-arr}				&\BS{usetikzlibrary}\AC{shapes.arrows} 
\\shapes.callouts		& \pageref{lib-call}			& \BS{usetikzlibrary}\AC{shapes.callouts} 
\\
shapes.geometric	& \pageref{lib-geom} 			& \BS{usetikzlibrary}\AC{shapes.geometric}
\\

shapes.misc			& \pageref{lib-misc} 			& \BS{usetikzlibrary}\AC{shapes.misc} 
\\
shapes.multipart	& \pageref{lib-mult} 			& \BS{usetikzlibrary}\AC{shapes.multipart} 
\\
shapes.symbols		& \pageref{lib-symb}			& \BS{usetikzlibrary}\AC{shapes.symbols} 
\\
through				& \pageref{lib-through}			&  \BS{usetikzlibrary}\AC{through}
\\ 
trees				& \pageref{lib-trees}
\BS{usetikzlibrary}\AC{trees}
\\ 
through				& \pageref{lib-turtle}			&  \BS{usetikzlibrary}\AC{turtle}
\\ 
\hline
 \end{tabular} 

\TFRGB{ 
\footnotetext[1]{voir dans le répertoire :  \BS{texlive}\BS{2016}\BS{tesmf-dist}\BS{doc}\BS{generic}\BS{pgf}}
\footnotetext[2]{chercher  dans le répertoire  :  \BS{texlive}\BS{2016}\BS{tesmf-dist}\BS{doc}\BS{latex}} }
{ 
\footnotetext[1]{look in repertory :  \BS{texlive}\BS{2016}\BS{tesmf-dist}\BS{doc}\BS{generic}\BS{pgf}}
\footnotetext[2]{search in repertory :  \BS{texlive}\BS{2016}\BS{tesmf-dist}\BS{doc}\BS{latex}} }

\bigskip



\begin{tabular}{|l|c|}\hline
\multicolumn{2}{|c|}{ \TFRGB{dans une prochaine mise à jour}{In a a future update } }
\\ \hline
automata			& \RRR{41} \\
babel				& \RRR{42} \\
calendar			& \RRR{45} \\
%chains				& \RRR{46} \\ 
%circuits.ee		& \RRR{47-4} \\ 
 
circular graph drawing library 				& \RRR{32} \\
curvilinear library 						& \RRR{103-4-7} \\
datavisualization library					& \RRR{75} \\
datavisualization.formats.functions library & \RRR{76-4} \\
datavisualization.polar library 			& \RRR{80}  \\
 er 										& \RRR{49}  \\
examples graph drawing library 				& \RRR{35-8} \\ 
external 									& \RRR{50}  \\  
%fit 										& \RRR{52} \\ 
fixedpointarithmetic 						& \RRR{53} \\ 
folding 									& \RRR{59} \\
force graph drawing library 				& \RRR{31}  \\
fpu											& \RRR{54}  \\
graph.standard library 						& \RRR{19-10}\\
graphdrawing library 						& \RRR{27} \\
graphs library 								& \RRR{19} \\ 
layered graph drawing library 				& \RRR{30}  \\
lindenmayersystems							& \RRR{55} \\  
mindmap										& \RRR{58} \\ 
petri										& \RRR{61}  \\ 
phylogenetics graph drawing library 		& \RRR{33} \\
plothandlers								& \RRR{62}  \\  
profiler									& \RRR{64}   \\ 
quotes library 								& \RRR{17-10-4} \\
routing graph drawing library 				& \RRR{34} \\
shadows										& \RRR{66}   \\ 
 
spy											&  \RRR{68} \\ 
svg.path									&  \RRR{69} \\ 
%through										&  \RRR{71} \\ 
topaths										&  \RRR{70} \\ 
trees graph drawing library					& 
\\ \hline
\end{tabular}  


%\newpage
%%
%% \tableofcontents
%\renewcommand{\bibname}{Sources}
%
\label{sources}
%\input{bib}

\newpage

\begin{thebibliography}{99}
\bibitem{pgfmanual} pgfmanual.pdf  	\hspace{1cm}	version 3.0.1a \hspace{1cm} 	1161 pages 	\hspace{1cm}	\DGB
\bibitem{pgfplots} pgfplots.pdf 	\hspace{1cm}	version 1.80 \hspace{1cm} 	439 pages 	\hspace{1cm}	\DGB
\bibitem{tikstab} tkz-tab-screen.pdf 	\hspace{1cm}	version 1.1c \hspace{1cm} 	83 pages 	\hspace{1cm}	\DFR
\bibitem{tikzpeople} tikzpeople.pdf 	\hspace{1cm}	 \hspace{1cm} 	19 pages 	\hspace{1cm}	\DGB

\bibitem{tikzducks} tikzducks-doc.pdf 	\hspace{1cm}	version 0.6  \hspace{1cm} 	28 pages 	\hspace{1cm}	\DGB

\bibitem{tikzsymbols} tikzsymbols.pdf 	\hspace{1cm}	version sept 2017  \hspace{1cm} 	15 pages 	\hspace{1cm}	\DGB

\bibitem{animate} animate.pdf 	\hspace{1cm}	 \hspace{1cm} 	26 pages 	\hspace{1cm}	\DGB

\bibitem{optics} tikz-optics.pdf	\hspace{1cm}	version 0.2.2  \hspace{1cm} 	39 pages 	\hspace{1cm}	\DFR
\end{thebibliography}




\newpage 

%%%%XXXXXXXXXXXXXXXXXXXXXXXXXXXXXXXXXXXXXXX
%
%\section{Index}
%
%\begin{enumerate}
%\item environnements
%\item Commandes
%\item paramètres et options
%\item Valeurs TikZ
%\item Extrémités
%\end{enumerate}
%
%\printindex 
%
%%\newpage
%%
%%\SbSSCT{Mises à jour précédente}{Former updates}
%%
%%
\subsection{version 0.64}

\TFRGB{
\textbf{Quoi de neuf version 0.64! } :

\begin{itemize}
\item ajout du module tikzpeople  \pageref{people}
\item ajout du module circuits.logic  \pageref{lib-gate}
\item ajout du module tikz-optics  \pageref{optics}
\item restructuration de l'index
\end{itemize}
}{
\textbf{What's new  version 0.64} :

\begin{itemize}
\item tikzpeople package added \pageref{people}
\item circuits.logic package added \pageref{lib-gate}
\item tikz-optics package added  \pageref{optics}
\item 3 minors bugs signaled by Jim Diamond corrected
\item reorganization of the index
\end{itemize}
}

\subsection{version 0.65}

\TFRGB{
\textbf{Quoi de neuf version 0.65! } :

\begin{itemize}
\item Résolution partielle du conflit entre \BS{usetikzlibrary}\AC{patterns} (\pageref{lib-patterns}) et \BS{usepackage}\AC{tikzpeople} (\pageref{people})
\item Ajout matrices de n\oe uds \pageref{matrix}
\item Ajout \og library matrix \fg \pageref{lib-matrix}
\item Ajout du module tikzducks  \pageref{ducks}
\end{itemize}

}
{
\textbf{What's new version 0.65!} :
\begin{itemize}
\item Partial resolution of the conflict between \BS{usetikzlibrary}\AC{patterns} (\pageref{lib-patterns}) and \BS{usepackage}\AC{tikzpeople} (\pageref{people})

\item Matrices of nodes \pageref{matrix}
\item Library matrix added \pageref{lib-matrix}
\item Tikzducks package added \pageref{ducks}
\end{itemize}
}




\end{document}