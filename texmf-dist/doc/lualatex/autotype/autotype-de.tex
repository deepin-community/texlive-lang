% autotype package: German manual
% Version 0.5
% Date: 5. Januar 2024
\PassOptionsToPackage{hyphens}{url}
\documentclass{scrartcl}
\usepackage[german,ngerman]{babel}
\usepackage{autotype}
\usepackage{fontspec}
\usepackage{biblatex}
\usepackage{csquotes}
\usepackage{hvextern}
\usepackage{paracol}
\usepackage[right]{lineno}
\usepackage[colorlinks=true,
  allcolors=black,
  bookmarksopen=true,
  bookmarksopenlevel=0,
  bookmarksnumbered=true,
  pdfencoding=auto,
  pdftitle={Automatische sprachspezifische Typographie mit LuaLaTeX},
  pdfsubject={Anleitung zum Paket AUTOTYPE},
  pdfkeywords={lualatex autotype Typographie Typografie},
  pdfauthor={K. Wehr}]{hyperref}

\addbibresource{blackletter.bib}
\addbibresource{dtk-bibliography.bib}

\setkeys{hv}{compiler=lualatex,force=true,crop=true,code=true,aboveskip=0.7ex,belowpreambleskip=1ex,lstOptions={basicstyle=\ttfamily,breaklines=true},BGbody=blue!20,BObody=blue!20,mpvalign=c}

\setmonofont{Deja Vu Sans Mono}[Scale=MatchLowercase]
\autotypelangoptions{ngerman}{ligbreak,hyphenation=weighted}

\newcommand\AutorAVorname{Stephan}
\newcommand\AutorANachname{Hennig}
\newcommand\AutorBVorname{Keno}
\newcommand\AutorBNachname{Wehr}
\newcommand\EMail{trennmuster}
\newcommand\EMailDomain{dante}
\newcommand*\file[1]{\texttt{#1}}
\newcommand*\package[1]{\textsc{#1}}
\newcommand\autotype{\textcolor{blue!50!black}{\package{auto\-type}}}
\newcommand\TrennmusterWiki{\url{https://wiki.dante.de/doku.php?id=trennmuster:trennmuster}}
\newcommand\TrennmusterMailingList{\texttt{\EMail@\EMailDomain.de}}
\newcommand*\command[1]{\mbox{\texttt{\textbackslash\textcolor{red!40!black}{#1}}}}
\newcommand*\meta[1]{\bgroup\rmfamily$\langle$\emph{#1}$\rangle$\egroup}
\newcommand*\marg[1]{\texttt{\{}\meta{#1}\texttt{\}}}
\newcommand*\option[1]{\texttt{\textcolor{green!40!black}{#1}}}
\newcommand*\BspFrak[1]{\bgroup\fontspec{yfrak.otf}#1\egroup}

\newenvironment{commandlist}{%
  \begin{list}{}{%
    \setlength\leftmargin{1em}%
    \setlength\itemindent{-1em}%
    \setlength\parsep{0pt}%
    \setlength\listparindent{\parindent}%
    \setlength\itemsep{\topsep}}}{%
  \end{list}}

\newenvironment{optionlist}{%
  \begin{list}{}{%
    \setlength\leftmargin{1em}%
    \setlength\itemindent{-1em}%
    \setlength\parsep{0pt}%
    \setlength\listparindent{\parindent}%
    \setlength\itemsep{3pt}}}{%
  \end{list}}

\ExplSyntaxOn

\NewDocumentCommand \commanddescription {mo}
  {
    \item
    \command {#1}
    \IfValueT {#2} {#2}
    \\
  }

\NewDocumentCommand \optiondescription {mmo}
  {
    \item
    \option {#1}
    \texttt { \, = \, \clist_use:nn {#2} {|} }
    \IfValueT {#3} { \hfill Voreinstellung:~\texttt{#3} }
    \\
  }

\ExplSyntaxOff

\begin{document}

\begin{center}
\Huge
\autotype

\smallskip
\large
Automatische sprachspezifische Typographie mit Lua\LaTeX

\medskip
Version 0.5

\medskip
\normalsize
\today

\vspace{2\bigskipamount}
\large
\textit{Paketautoren}

\smallskip
\AutorAVorname\ \AutorANachname\quad
\AutorBVorname\ \AutorBNachname\\
\normalsize
\texttt{\TrennmusterMailingList}

\bigskip
\textit{Fehlermeldungen}

\smallskip
\normalsize
\url{https://codeberg.org/wehr/autotype/issues}
\end{center}

\bigskip
\begin{abstract}
\noindent Mit diesem Lua\LaTeX-Paket können sprachspezifische typographische Anforderungen automatisiert erfüllt werden. Zur Zeit wird nur die deutsche Sprache (alte und neue Rechtschreibung) unterstützt. Möglich sind die Wahl eines Silbentrennverfahrens mit unterschiedlich gewichteten Trennstellen, die nach Wichtung differenzierte Markierung von Trennstellen, der Aufbruch falscher Ligaturen und der differenzierte Einsatz von Lang-s und Rund-s gemäß den Regeln für den Fraktursatz.
\end{abstract}

\bigskip
\noindent An English version of this manual is not yet available.

\tableofcontents

\section{Installation}
Seit Version 0.4 ist \autotype\ auf dem CTAN verfügbar.\footnote{\url{https://ctan.org/tex-archive/macros/luatex/latex/autotype}} Die Installation müsste über die Paketverwaltung der \TeX-Distribution möglich sein. Falls die automatische Installation nicht möglich ist, legen Sie in Ihrem persönlichen \TeX-Baum das Verzeichnis \file{\mbox{TEXMFHOME}\slash\mbox{tex}\slash\mbox{lualatex}\slash\mbox{autotype}} an und kopieren Sie die folgenden Dateien vom CTAN in dieses Verzeichnis:
\begin{itemize}
\item \file{autotype.lua}
\item \file{autotype.sty}
\item \file{autotype-*.pat.txt} (8 Dateien)
\item \file{autotype-cls\_pdnm\_oop.lua}
\item \file{autotype-cls\_pdnm\_pattern.lua}
\item \file{autotype-cls\_pdnm\_spot.lua}
\item \file{autotype-cls\_pdnm\_trie\_simple.lua}
\item \file{autotype-pdnm\_nl\_manipulation.lua}
\end{itemize}

\section{Befehlsübersicht}
\begin{commandlist}
\commanddescription{autotypelangoptions}[\marg{Sprache}\marg{Optionen}]
Legt sprachspezifische Optionen fest.
Die \meta{Sprache} ist dabei die vom Sprachpaket verwendete Sprachbezeichnung. Für das Deutsche ist dies mit \package{polyglossia} immer \texttt{german}, mit \package{babel} \texttt{ngerman}, \texttt{naustrian} oder anders, je nach gewählter Varietät. Die Beispiele in dieser Anleitung setzen die Verwendung von \package{babel} voraus. Die folgenden Optionen stehen zur Verfügung:
\begin{optionlist}
\optiondescription{hyphenation}{default,primary,weighted}[default]
Silbentrennverfahren: \texttt{default} ist des Standardverfahren von \TeX, \texttt{primary} bedeutet eine Trennung nur an den Wortfugen mit der geringsten Bindungsstärke (Primärtrennstellen), \texttt{weighted} aktiviert einen Trennalgorithmus, der günstigere Trennstellen bevorzugt. Für das Nähere siehe Abschnitt \ref{hyphenation}.
\optiondescription{mark-hyph}{on,off}[off]
Farbige Markierung der Trennstellen wie im Abschnitt \ref{hyphenation} beschrieben. Der Code für diese Option basiert auf dem Paket \package{showhyphens} von Patrick Gundlach.
\optiondescription{ligbreak}{on,off}[off]
Ligaturaufbruch, siehe Abschnitt \ref{ligbreak}.
\optiondescription{long-s}{on,off}[off]
Lang- und Rund-s-Schreibung, siehe Abschnitt \ref{long-s}.
\end{optionlist}
\commanddescription{autotypefontoptions}[\marg{Schriftart}\marg{Optionen}]
Legt schriftartenspezifische Optionen fest. Die \meta{Schriftart} ist der \TeX-Name der Schrift. Der Befehl wird nur für Schriftarten mit irregulären Eigenschaften benötigt.
\begin{optionlist}
\optiondescription{long-s-codepoint}{\meta{Zahl}}[383]
Legt die Codeposition des Lang-s in der Schrift fest, siehe Abschnitt \ref{long-s}.
\optiondescription{round-s-codepoint}{\meta{Zahl}}[115]
Legt die Codeposition des Rund-s in der Schrift fest, siehe Abschnitt \ref{long-s}.
\optiondescription{final-round-s-codepoint}{\meta{Zahl}}
Legt die Codeposition eines Rund-s am Wortende in der Schrift fest. Wenn kein Wert gesetzt ist, wird der Wert von \option{round-s-codepoint} übernommen. Siehe Abschnitt \ref{suetterlin} für ein Beispiel.
\end{optionlist}
\commanddescription{noligbreak}[\marg{Text}]
Dient zur Unterdrückung des automatischen Ligaturaufbruchs für einzelne Wörter oder Buchstabengruppen, falls die Option \option{ligbreak} aktiviert ist.
\commanddescription{autotypelongs}
Dient zur Einfügung eines langen s, wenn die automatische Ersetzung scheitert.
\commanddescription{autotyperounds}
Dient zur Einfügung eines runden s, wenn die automatische Ersetzung scheitert.
\end{commandlist}

\section{Silbentrennung\label{hyphenation}}
Das gewöhnliche \TeX-Verfahren zur Silbentrennung behandelt alle Trennstellen innerhalb eines Wortes gleich. Im Deutschen gibt es zahlreiche zusammengesetzte Wörter; bei diesen ist die Trennung an der Wortfuge lesefreundlicher als an anderen möglichen Trennstellen. Beispielsweise lässt sich mit der Trennung »\mbox{Mineral}-\mbox{wasser}« das Wort besser erfassen als mit »Mi-\mbox{neralwasser}«, »\mbox{Mine}-\mbox{ralwasser}« oder »\mbox{Mineralwas}-ser«. Das von \autotype\ angebotene gewichtete Silbentrennverfahren bevorzugt Trennstellen an Wortfugen. Nennenswerte Auswirkungen hat dieses Verfahren nur in Texten mit längeren Absätzen, da ansonsten nicht viel Spielraum bei der Wahl der Trennstellen bleibt.

\autotype\ unterscheidet zwischen Primär-, Sekundär- und Tertiärtrennstellen. Primärtrennstellen liegen in zusammengesetzten Wörtern sowie Wörtern mit Vor- oder Nachsilben an den Fugen mit der geringsten Bindungsstärke. Sekundärtrennstellen liegen an Fugen der nächstgrößeren Bindungsstärke und Tertiärtrennstellen sind alle übrigen. Die Unterscheidung zwischen den Trennstellenarten erfolgt auf der Grundlage von drei verschiedenen Trennmustersätzen, die vom deutschsprachigen Trennmusterprojekt\footnote{\TrennmusterWiki} zur Verfügung gestellt werden.

In den folgenden Beispielwörtern sind Primärtrennstellen durch einen langen grünen Strich, Sekundärtrennstellen durch einen kurzen blauen Strich unten und Tertiärtrennstellen durch einen kurzen orangen Strich oben markiert:
\bgroup
\autotypelangoptions{ngerman}{mark-hyph}%
Mineralwasser,
Einarbeitung,
Müdigkeit,
Bilderbuchsommer,
Botaniklehrbuch,
Dienstaufsichtsbeschwerde,
Bäckereien,
Bedeutung,
Philosophiestudium.
\egroup

\subsection{Auswahl des Trennverfahrens}
Mit der Sprachoption \option{hyphenation=weighted} wird ein gewichtetes Trennverfahren aktiviert.
\begin{externalDocument}[showoutput=false]{hyphenation}
\documentclass{article}
\usepackage[ngerman]{babel}
\usepackage{autotype}
\begin{document}
\autotypelangoptions{ngerman}{hyphenation=weighted}
\end{document}
\end{externalDocument}

\medskip
Dabei stehen alle Trennstellen des gewöhnlichen Trennverfahrens zur Verfügung; sie erhalten jedoch unterschiedlich viele Strafpunkte (\emph{penalties}) und werden daher mit unterschiedlichen Wahrscheinlichkeiten zur Trennung ausgewählt. Während beim Standardtrennverfahren alle Trennstellen mit 50 Strafpunkten versehen werden, wichtet \autotype\ Primärtrennstellen mit \directlua{tex.print(autotype.get_penalty(1))}, Sekundärtrennstellen mit \directlua{tex.print(autotype.get_penalty(2))} und Tertiärtrennstellen mit \directlua{tex.print(autotype.get_penalty(3))} Strafpunkten.

In Wörtern mit Bindestrichen (z.\,B. »Konrad-Adenauer-Allee«) gelten die Bindestriche als Primärtrennstellen. Anders als beim Standardtrennverfahren werden auch innerhalb der Teilwörter Trennstellen eingefügt, die unterschiedslos als Tertiärtrennstellen behandelt werden.

Der Anhang \ref{Kant} stellt die Ergebnisse des gewichteten Trennalgorithmus denen des Standardtrennalgorithmus anhand eines philosophischen Textes von Immanuel Kant gegenüber.

Mit \option{hyphenation=primary} werden die möglichen Trennstellen auf die Primärtrennstellen beschränkt. Das erhöht die Lesbarkeit, ist aber nur in Texten mit hinreichend langen Zeilen empfehlenswert.
\begin{externalDocument}[showoutput=false]{hyphenation}
\documentclass{article}
\usepackage[ngerman]{babel}
\usepackage{autotype}
\begin{document}
\autotypelangoptions{ngerman}{hyphenation=primary}
\end{document}
\end{externalDocument}

\medskip
Außerdem ist es möglich, wieder zum Standardtrennverfahren zurückzukehren.
\begin{externalDocument}[showoutput=false]{hyphenation}
\documentclass{article}
\usepackage[ngerman]{babel}
\usepackage{autotype}
\begin{document}
\autotypelangoptions{ngerman}{hyphenation=default}
\end{document}
\end{externalDocument}

\medskip
Wenn für die Option \option{hyphenation} innerhalb des Dokuments ein neuer Wert gewählt wird, gilt dieser immer rückwirkend ab dem Anfang des Absatzes.

\subsection{Anzeige der Trennstellen}
Zum Anzeigen von Trennstellen existieren die Pakete \package{showhyphens}\footnote{\url{https://ctan.org/pkg/showhyphens}} und \package{showhyphenation}\footnote{\url{https://ctan.org/pkg/showhyphenation}}. Mit beiden ist es aber nicht möglich, die Trennstellenanzeige innerhalb des Dokuments ein- und auszuschalten und die Trennstellen nach Strafpunkten (\emph{penalties}) differenziert anzuzeigen.

Daher bietet \autotype\ die Option \option{mark-hyph}, mit der Primärtrennstellen grün (langer Strich), Sekundärtrennstellen blau (kurzer Strich unten), Tertiärtrennstellen orange und sonstige Trennstellen rot markiert werden (jeweils kurzer Strich oben). \option{mark-hyph} kann auch für andere Sprachen als Deutsch verwendet werden.%
\begin{externalDocument}{hyphenation}
\documentclass[11pt]{article}
\usepackage[ngerman]{babel}
\usepackage{autotype}
\pagestyle{empty}
\setlength\parindent{0pt}
\begin{document}
\autotypelangoptions{ngerman}{hyphenation=weighted,mark-hyph}
Sozialhilferegelsatz Konrad-Adenauer-Allee\par
\autotypelangoptions{ngerman}{hyphenation=default}
Sozialhilferegelsatz Konrad-Adenauer-Allee\\
\autotypelangoptions{ngerman}{mark-hyph=off}
Sozialhilferegelsatz Konrad-Adenauer-Allee
\end{document}
\end{externalDocument}

\subsection{Bekannte Probleme}
\begin{itemize}
\item Die Übersetzung des Dokuments verlangsamt sich bei Verwendung des gewichteten Trennalgorithmus (Option \option{hyphenation=weighted}) erheblich.
\item Die Werte von \command{lefthyphenmin} und \command{righthyphenmin} liegen mit \option{hyphenation=\mbox{weighted}} und \option{hyphenation=primary} konstant bei \texttt{2}. Sie sollten besser aus dem Sprachpaket übernommen werden, um anpassbar zu sein.
\item Mit \option{hyphenation=weighted} und \option{hyphenation=primary} werden auch Textteile in Schreibmaschinenschrift getrennt, was in \LaTeX\ eigentlich nicht vorgesehen ist, da für solche Schriften der Wert von \command{hyphenchar} auf \texttt{-1} gesetzt wird.
\begin{externalDocument}{hyphenation}
\documentclass[11pt]{article}
\usepackage[ngerman]{babel}
\usepackage{autotype}
\pagestyle{empty}
\begin{document}
\autotypelangoptions{ngerman}{mark-hyph=on}
Botaniklehrbuch \texttt{Botaniklehrbuch}\par
\autotypelangoptions{ngerman}{hyphenation=weighted}
Botaniklehrbuch \texttt{Botaniklehrbuch}
\end{document}
\end{externalDocument}
\item Nach einem explizit eingegebenen Gedankenstrich/Streckenstrich (\texttt{U+2013}) wird bei allen Trennverfahren keine Trennstelle eingefügt. Falls hier ein Umbruch gewünscht wird, muss der \package{babel}-Shorthand \verb:"": oder die alternative Eingabe \verb:--: verwendet werden.
\begin{externalDocument}{hyphenation}
\documentclass[11pt]{article}
\usepackage[ngerman]{babel}
\pagestyle{empty}
\begin{document}
\begin{minipage}{6cm}
Heute ist auf der Strecke Hamburg–Berlin mit Stau zu rechnen.\\
Heute ist auf der Strecke Hamburg–""Berlin mit Stau zu rechnen.\\
Heute ist auf der Strecke Hamburg--Berlin mit Stau zu rechnen.
\end{minipage}
\end{document}
\end{externalDocument}
\end{itemize}

\section{Ligaturaufbruch\label{ligbreak}}
An Wortfugen sind Ligaturen sinnentstellend und daher typographisch falsch. Statt »\noligbreak{Auffahrt}« (mit ff-Ligatur) muss »Auffahrt« gesetzt werden, statt »\noligbreak{Auflage}« (mit fl-Ligatur) »Auflage«. \autotype\ kann falsche Ligaturen auf der Grundlage von Pseudotrennmustern verhindern, die nach dem gleichen Prinzip wie die Silbentrennmuster von \TeX\ funktionieren. An den durch die Muster bezeichneten Stellen werden Ligaturen durch Einfügen eines Kernings von 0\,pt verhindert.

Die Pseudotrennmuster für den Ligaturaufbruch werden vom deutschsprachigen Trennmusterprojekt bereitgestellt.\footnote{\TrennmusterWiki} Sie bezeichnen Stellen nach Präfixen, an Wortfugen sowie vor Suffixen, sofern vor letzteren gleichzeitig eine Silbentrennstelle vorliegt.

\subsection{Anwendung}
Der automatische Ligaturaufbruch wird durch die Option \option{ligbreak} aktiviert.
\begin{externalDocument}{ligbreak}
\documentclass[11pt]{article}
\usepackage[ngerman]{babel}
\usepackage{autotype}
\pagestyle{empty}
\setlength\parindent{0pt}
\begin{document}
Auffahrt Auflage Tarifinformation Kunststoffindustrie\\
\autotypelangoptions{ngerman}{ligbreak}
Auffahrt Auflage Tarifinformation Kunststoffindustrie
\end{document}
\end{externalDocument}

Zwar kommen Ligaturen mit \emph{f} als erstem Buchstaben am häufigsten vor, doch sind die verwendeten Muster nicht auf solche beschränkt, sondern verhindern auch exotischere Ligaturen.
\begin{externalDocument}{ligbreak}
\documentclass[11pt]{article}
\usepackage[ngerman]{babel}
\usepackage{fontspec}
\usepackage{autotype}
\pagestyle{empty}
\setlength\parindent{0pt}
\begin{document}
\fontspec{Libertinus Serif}[Ligatures=Rare,Ligatures=Historic]
Abgastest Tagestourist Besetztzeichen Leitzinssatz\\
\autotypelangoptions{ngerman}{ligbreak}
Abgastest Tagestourist Besetztzeichen Leitzinssatz
\end{document}
\end{externalDocument}

Falls eine Ligatur nicht wie erwartet aufgebrochen wird, steht der herkömmliche \package{babel}-Shorthand \verb:"|: für diesen Zweck zur Verfügung. Falls eine Ligatur fälschlich aufgebrochen wird, kann mit dem Befehl \command{noligbreak}, der als Argument ein Wort oder eine Buchstabengruppe erwartet, der Aufbruch verhindert werden; alternativ ist es möglich, an der Wortfuge einen der \package{babel}-Shorthands \verb:"|: oder \verb:"-: zu setzen.
\begin{externalDocument}{ligbreak}
\documentclass[11pt]{article}
\usepackage[ngerman]{babel}
\usepackage{autotype}
\pagestyle{empty}
\begin{document}
\autotypelangoptions{ngerman}{ligbreak}
Graufliege \noligbreak{Graufliege} Grau"|fliege Grau"-fliege
\end{document}
\end{externalDocument}
Betroffene Wörter sollten in beiden Fällen an das deutschsprachige Trennmusterprojekt\footnote{\TrennmusterMailingList} gemeldet werden, damit die Muster verbessert werden können.

Um den automatischen Ligaturaufbruch gänzlich wieder abzuschalten, gibt es die Option \option{ligbreak=off}.
\begin{externalDocument}{ligbreak}
\documentclass[11pt]{article}
\usepackage[ngerman]{babel}
\usepackage{autotype}
\pagestyle{empty}
\setlength\parindent{0pt}
\begin{document}
\autotypelangoptions{ngerman}{ligbreak}
Schaffell Zupfinstrument Topflappen\\
\autotypelangoptions{ngerman}{ligbreak=off}
Schaffell Zupfinstrument Topflappen
\end{document}
\end{externalDocument}

\subsection{Bekannte Probleme und offene Fragen}
\begin{itemize}
\item Auch nach dem Aufbruch einer Ligatur stehen die betreffenden Buchstaben in vielen Schriftarten noch sehr eng zusammen und sind auf den ersten Blick kaum von einer Ligatur zu unterscheiden. Dies gilt namentlich für die Buchstabenkombination »f\kern0pt l«.
Es wäre ein Leichtes für \autotype, an solchen Stellen einen kleinen Zusatzabstand einzufügen. Fraglich ist, wie eine Nutzerschnittstelle mit maximaler Flexibilität (individuell wählbares Kerning für jede Schriftart und jede Buchstabenkombination) aussehen sollte.
\item In den folgenden Fällen bleiben die Ligaturen »fl« und »ft« (sofern vorhanden) erhalten, obwohl sie nach manchen Quellen aufgebrochen werden sollten:\footnote{\enquote{Ligaturen werden zwischen Stamm und Beugungs-Endungen wie -te, -ten [\dots] nicht gesetzt.} \textsc{Forssman/de Jong}: Detailtypografie, Mainz \textsuperscript{8}2021, S. 194. Als Beispiele werden \enquote{hofften} und \enquote{kaufte} genannt.\par\enquote{Keine Ligatur steht zwischen Wortstamm und Endung (Ausnahme: fi).} Duden. Die deutsche Rechtschreibung, Mannheim \textsuperscript{24}2006, S. 112. Als Beispiele werden \enquote{ich schauf"|le, ich kaufte, höf"|lich} genannt.}
\begin{itemize}
\item Vergangenheitsformen und Partizipien von Verben, deren Stamm auf \emph{f} endet: \emph{ich kaufte, du rafftest, gestreifte}
\item Ordnungszahlen: \emph{der fünfte, der elfte, der zwölfte}
\item Grundwort mit ausgefallenem \emph{e} in der Endsilbe vor einem Suffix, das mit einem Vokal beginnt: \emph{Verzweiflung, Zweifler, schweflig, knifflig}
\item 1. Person Singular von Verben auf \emph{eln}: \emph{ich schaufle, ich zweifle, ich löffle}
\end{itemize}
\item Wenn eine Dreierligatur (zu denken ist in erster Linie an \emph{ffi}, \emph{ſſi} und \emph{ffl}) möglich ist, die Schrift aber nur Zweierligaturen kennt, werden die ersten beiden Buchstaben ligiert. Wenn allerdings nach dem ersten Buchstaben der Dreiergruppe eine Silbentrennstelle vorliegt, sollten stattdessen der zweite und dritte Buchstabe ligiert werden. Beispiele hierfür wären \emph{Offizier, Meſſias, Soufflé}.

Da die meisten Digitalschriften über die Dreierligatur verfügen, wenn auch die entsprechenden Zweierligaturen vorhanden sind, ergibt sich das Problem nur selten. Das einzige bekannte Beispiel ist die recht spezielle Schrift \emph{Missaali}, eine spätmittelalterliche gotische Druckschrift (vgl. auch Abschnitt \ref{missaali}).

\begin{externalDocument}[mpwidth=0.5\textwidth]{long-s}
\documentclass[DIV=8,fontsize=16pt]{scrartcl}
\usepackage[german]{babel}
\usepackage{fontspec}
\pagestyle{empty}
\setlength\parindent{0pt}
\begin{document}
\fontspec{Missaali}
Offizier Meſſias Soufflé\\
Of"|fizier Meſ"|ſias Souf"|flé
\end{document}
\end{externalDocument}
\end{itemize}

\section{Lang-s und Rund-s\label{long-s}}
Gebrochene Schriften (auch Frakturschriften genannt) haben im Allgemeinen zwei Formen des kleinen s, das Rund-s (\BspFrak{\autotyperounds}) und das Lang-s (\BspFrak{\autotypelongs}). Im Deutschen gibt es differenzierte Regeln für der Einsatz der beiden s-Formen. Die Grundregel besagt, dass das Rund-s am Ende eines Wortes oder Teilwortes (z.\,B. \BspFrak{Hau\autotyperounds}, \BspFrak{Hau\autotyperounds tür}) steht, das Lang-s an anderen Stellen (z.\,B. \BspFrak{Häu\autotypelongs er}).\footnote{Für eine genauere Darstellung siehe etwa \url{https://www.typografie.info/3/faq.htm/wie-setzt-man-das-lange-s-im-klassischen-fraktursatz-r12/}.}

\autotype\ kann einen Text, der unterschiedslos mit Rund-s eingegeben wurde, mit Lang- und Rund-s an den korrekten Stellen ausgeben. Es greift hierfür auf Pseudotrennmuster zurück, die nach dem gleichen Prinzip wie die Silbentrennmuster von \TeX\ funktionieren. Diese werden vom deutschsprachigen Trennmusterprojekt\footnote{\TrennmusterWiki} bereitgestellt und bezeichnen Stellen, vor denen ein Rund-s gesetzt werden muss.

In der frühen Neuzeit war das Lang-s auch in Antiqua-Schriften üblich, wenn auch nicht nach den heutigen Regeln. In vielen Antiqua-Digitalschriften ist es ebenfalls verfügbar, sodass auch mit diesen prinzipiell die nach Lang- und Rund-s differenzierte Schreibung verwendet werden kann. Allerdings wäre dies angesichts der historischen Praxis fragwürdig; in erster Linie ist diese \autotype-Fähigkeit für gebrochene Schriften gedacht.

\subsection{Anwendung}
Die automatische Lang- und Rund-s-Schreibung wird durch die Option \option{long-s} aktiviert. Diese sollte immer mit der Option \option{ligbreak} kombiniert werden, da die korrekte Verwendung der Ligaturen im Fraktursatz noch wichtiger ist als im Antiquasatz. Für die folgenden Beispiele verwenden wir die Schrift \texttt{yfrak.otf} (vgl. Abschnitt \ref{yfonts}).
\begin{externalDocument}{long-s}
\documentclass[DIV=8,fontsize=15pt]{scrartcl}
\usepackage[german]{babel}
\usepackage{fontspec}
\usepackage{autotype}
\setmainfont{yfrak.otf}
\pagestyle{empty}
\setlength\parindent{0pt}
\begin{document}
aussichtslos Abschiedsbesuch Atemschutzmaske Massenarbeitslosigkeit\\
\autotypelangoptions{german}{ligbreak,long-s}
aussichtslos Abschiedsbesuch Atemschutzmaske Massenarbeitslosigkeit
\end{document}
\end{externalDocument}

Falls die automatische Ersetzung nicht zum richtigen Ergebnis führt, können die Befehle \command{autotypelongs} und \command{autotyperounds} verwendet werden, um manuell ein Lang- bzw. Rund-s einzufügen. Das ist etwa im folgenden Beispiel nützlich, wenn es sich um eine \emph{Wachs-Tube} und keine \emph{Wach-Stube} handelt. Alternativ helfen hier auch die \package{babel}-Shorthands \verb:"|: und \verb:"-: an der Wortfuge.\\
\begin{externalDocument}[mpwidth=0.5\textwidth]{long-s}
\documentclass[DIV=8,fontsize=15pt]{scrartcl}
\usepackage[german]{babel}
\usepackage{fontspec}
\usepackage{autotype}
\setmainfont{yfrak.otf}
\autotypelangoptions{german}{long-s}
\pagestyle{empty}
\setlength\parindent{0pt}
\begin{document}
Wachstube\\
Wach\autotyperounds tube\\
Wachs"|tube\\
Wachs"-tube
\end{document}
\end{externalDocument}
\\
Betroffene Wörter sollten außer bei Homographen wie den oben genannten an das deutschsprachige Trennmusterprojekt\footnote{\TrennmusterMailingList} gemeldet werden, damit die Muster verbessert werden können.

Die automatische Ersetzung kann mit der Option \option{long-s=off} wieder abgeschaltet werden. Die Befehle \command{autotypelongs} und \command{autotyperounds} bleiben auch dann verfügbar.

Falls keine weiteren Einstellungen vorgenommen wurden, geht \autotype\ davon aus, dass sich Lang- und Rund-s auf den durch Unicode definierten Codepositionen befinden, nämlich 383 (\texttt{U+017F}) für das Lang-s und 115 (\texttt{U+0073}) für das Rund-s. Für Opentype-Schriften ist diese Annahme zutreffend, für andere Schriftformate typischerweise nicht. Für solche Schriften müssen die Codepositionen daher mit Hilfe der Schriftoptionen \option{long-s-codepoint} und \option{round-s-codepoint} angegeben werden. Beispielsweise lautet für die Sütterlinschrift \texttt{schwell} (vgl. Abschnitt \ref{suetterlin}) der nötige Befehl:
\begin{externalDocument}[showoutput=false]{long-s}
\documentclass{article}
\usepackage{autotype}
\begin{document}
\autotypefontoptions{schwell}{long-s-codepoint=115,round-s-codepoint=28}
\end{document}
\end{externalDocument}

\medskip
Alternativ ist die hexadezimale Schreibweise möglich.
\begin{externalDocument}[showoutput=false]{long-s}
\documentclass{article}
\usepackage{autotype}
\begin{document}
\autotypefontoptions{schwell}{long-s-codepoint=0x73,round-s-codepoint=0x1C}
\end{document}
\end{externalDocument}
Für spezielle Anwendungen wird zusätzlich noch die Option \option{final-round-s-codepoint} benötigt, siehe Abschnitt \ref{suetterlin}.

Für den Satz mit gebrochenen Schriften ist generell die deutsche Rechtschreibregelung von 1901 (»alte Rechtschreibung«) vorzuziehen, da die Schreibung der s-Laute hier besser ins Schriftbild passt. Bei Verwendung der Konventionen von 1996 (»neue Rechtschreibung«) wird ein Doppel-s am Wortende als \BspFrak{\autotypelongs\autotyperounds} wiedergegeben.
\begin{externalDocument}{long-s}
\documentclass[DIV=8,fontsize=15pt]{scrartcl}
\usepackage[ngerman]{babel}
\usepackage{fontspec}
\usepackage{autotype}
\setmainfont{yfrak.otf}
\pagestyle{empty}
\begin{document}
\autotypelangoptions{ngerman}{ligbreak,long-s}
Abschiedskuss Kongressbeschluss Messstelle
\end{document}
\end{externalDocument}

\subsection{Beispiele}
Wir stellen die konkrete Anwendung der Option \option{long-s} mit verschiedenen frei verfügbaren gebrochenen Schriften vor.

\subsubsection{ygoth, yswab, yfrak\label{yfonts}}
Diese Schriften von Yannis Haralambous -- ursprünglich mit Metafont entworfen \cite{haralambous}, dann nach Type1 konvertiert \cite{dtk03.2:bronger:einfaches,dtk03.3:hoffmann-axthelm:fraktur} -- liegen seit 2022 auch im Opentype-Format\footnote{\url{https://ctan.org/pkg/yfonts-otf}} vor.

Bei \texttt{ygoth} handelt es sich um eine gotische Schrift nach dem Vorbild der Schrift Gutenbergs und anderer Druckschriften des 15. Jahrhunderts.
\begin{externalDocument}{long-s}
\documentclass[DIV=8,fontsize=15pt]{scrartcl}
\usepackage[german]{babel}
\usepackage{fontspec}
\usepackage{autotype}
\pagestyle{empty}
\setlength\parindent{0pt}
\begin{document}
\autotypelangoptions{german}{ligbreak,long-s}
\fontspec{ygoth.otf}
Oskar entschied sich angesichts des aussichtslosen Wassermangels zur Umsiedlung von Moskau nach Minsk.
\end{document}
\end{externalDocument}

\texttt{yswab} ist eine Schwabacher Schrift nach Vorlagen des 18. Jahrhunderts.
\begin{externalDocument}{long-s}
\documentclass[DIV=8,fontsize=15pt]{scrartcl}
\usepackage[german]{babel}
\usepackage{fontspec}
\usepackage{autotype}
\pagestyle{empty}
\setlength\parindent{0pt}
\begin{document}
\autotypelangoptions{german}{ligbreak,long-s}
\fontspec{yswab.otf}
Oskar entschied sich angesichts des aussichtslosen Wassermangels zur Umsiedlung von Moskau nach Minsk.
\end{document}
\end{externalDocument}

Die Schrift \texttt{yfrak} ist eine digitalisierte Variante der Breitkopf-Fraktur aus der Mitte des 18. Jahrhunderts, einer bis ins 20. Jahrhundert vielfach genutzten Frakturschrift.
\begin{externalDocument}{long-s}
\documentclass[DIV=8,fontsize=15pt]{scrartcl}
\usepackage[german]{babel}
\usepackage{fontspec}
\usepackage{autotype}
\pagestyle{empty}
\setlength\parindent{0pt}
\begin{document}
\autotypelangoptions{german}{ligbreak,long-s}
\fontspec{yfrak.otf}
Oskar entschied sich angesichts des aussichtslosen Wassermangels zur Umsiedlung von Moskau nach Minsk.
\end{document}
\end{externalDocument}

Das Paket \package{yfonts-otf.sty} stellt zur Schriftauswahl die Befehle \command{gothfamily}, \command{swabfamily} und \command{frakfamily} zur Verfügung. Zur Verwendung mit \autotype\ sind diese aber weniger geeignet, da mit ihnen auch das Opentype-Merkmal \texttt{ss11} aktiviert wird, hinter dem sich ein eigener Algorithmus zum Lang- und Rund-s-Satz verbirgt, dessen Fehler \autotype\ nicht in allen Fällen korrigieren kann. Wenn jene Befehle gemeinsam mit \autotype\ verwendet werden sollen, muss anschließend jeweils das Merkmal \texttt{ss11} deaktiviert werden.%
\begin{externalDocument}{long-s}
\documentclass[DIV=8,fontsize=15pt]{scrartcl}
%StartVisiblePreamble
\usepackage[german]{babel}
\usepackage{yfonts-otf}
\usepackage{autotype}
%StopVisiblePreamble
\pagestyle{empty}
\setlength\parindent{0pt}
\begin{document}
\autotypelangoptions{german}{ligbreak,long-s}
\frakfamily % falsche s-Schreibung mit ss11
Aischylos Couscous Disco Eschatologie Ischias Ismaning Sanskrit \\
\addfontfeature{RawFeature=-ss11} % richtige s-Schreibung mit autotype
Aischylos Couscous Disco Eschatologie Ischias Ismaning Sanskrit
\end{document}
\end{externalDocument}

\subsubsection{Missaali\label{missaali}}
\emph{Missaali} von Tommi Syrjänen ist eine freie gebrochene Schrift im Opentype-Format. Vorlage ist die gotische Schrift eines im späten Mittelalter für Finnland gedruckten Messbuchs. Eine umfangreiche Dokumentation liegt vor.\footnote{\url{https://ctan.org/pkg/missaali}}

Nach historischem Vorbild setzt \emph{Missaali} ein Lang-s im Wortinnern und ein Rund"~s am Wortende, wofür das Opentype-Merkmal \texttt{calt} (Contextual Alternates) genutzt wird. Sollen Lang- und Rund-s stattdessen nach den komplexeren deutschen Regeln gesetzt werden, muss dieses Merkmal mit Hilfe der entsprechenden \package{fontspec}-Option ausgeschaltet werden. Dadurch geht allerdings auch die automatische Einfügung des Rund-r an bestimmten Stellen\footnote{Vgl. \package{Missaali}-Anleitung, Abschnitt 3, S. 21 unten.} verloren.
\begin{externalDocument}{long-s}
\documentclass[DIV=8,fontsize=16pt]{scrartcl}
\usepackage[german]{babel}
\usepackage{fontspec}
\usepackage{autotype}
\pagestyle{empty}
\setlength\parindent{0pt}
\begin{document}
\fontspec{Missaali}[Contextuals=AlternateOff]
\autotypelangoptions{german}{ligbreak,long-s}
Oskar entschied sich angesichts des aussichtslosen Wassermangels zur Umsiedlung von Moskau nach Minsk.
\end{document}
\end{externalDocument}

\subsubsection{Ligafaktur-Schriften\label{ligafaktur}}
Auf der Seite \url{https://www.ligafaktur.de} wird eine Reihe von Frakturschriften in drei unterschiedlichen Varianten bezüglich der vorhandenen Ligaturen angeboten: LUC, LOB und LOV. Für die Verwendung mit \autotype\ sind die LOB-Schriften am besten geeignet.
\begin{externalDocument}{long-s}
\documentclass[DIV=8,fontsize=15pt]{scrartcl}
\usepackage[german]{babel}
\usepackage{fontspec}
\usepackage{autotype}
\pagestyle{empty}
\setlength\parindent{0pt}
\begin{document}
\autotypelangoptions{german}{ligbreak,long-s}
\fontspec{LOB.TheuerdankFraktur}
Oskar entschied sich angesichts des aussichtslosen Wassermangels zur Umsiedlung von Moskau nach Minsk.
\end{document}
\end{externalDocument}

\subsubsection{Sütterlinschrift\label{suetterlin}}
Auf dem CTAN findet man verschiedene Varianten der Sütterlinschrift, die in der ersten Hälfte des 20. Jahrhunderts als Schulausgangsschrift verwendet wurde. Sie wurden von Walter Entenmann\footnote{\url{https://ctan.org/pkg/schulschriften}} und Berthold Ludewig\footnote{\url{https://ctan.org/pkg/sueterlin}} bereitgestellt. Es handelt sich in beiden Fällen um T1-codierte Metafont-Schriften, die ausnahmsweise nicht mit der Option \option{ligbreak} benutzt werden sollten, da diese die Buchstabenverbindungen an den Wortfugen zerstört.

Das Paket \package{schulschriften} von W. Entenmann \cite{dtk12.4:entenmann:schulschriften} stellt eine ganze Sippe von zwölf Sütterlinschriften zur Verfügung, die sich durch die Strichstärke, die Neigung und die Art der Feder unterscheiden. Wir führen im folgenden Beispiel zwei dieser Schriften vor und verweisen für die übrigen auf die Paketanleitung. Die Schriften haben zwei verschiedene Rund-s-Formen: eine Binnenform mit Anschluss an den Folgebuchstaben und eine Schlussform. Die Schlussform muss \autotype\ mit der Schriftoption \option{final-round-s-codepoint} bekannt gemacht werden. Die Codepositionen der s-Formen sind für alle Schriften gleich, müssen aber für jede verwendete Variante separat angegeben werden, da alle Varianten unterschiedliche \TeX-Namen tragen. Die Verdopplung des Doppelpunkts bei Wörtern, die auf \emph{s} enden (vgl. Paketanleitung), entfällt bei Verwendung der \autotype-Option \option{long-s}.
\begin{externalDocument}{long-s}
\documentclass[DIV=8,fontsize=17pt]{scrartcl}
\setlength\parindent{0pt}
%StartVisiblePreamble
\usepackage[german]{babel}
\usepackage{autotype}
\autotypefontoptions{wesu14}{long-s-codepoint=115,round-s-codepoint=24,final-round-s-codepoint=25}
\autotypefontoptions{wesubsl14}{long-s-codepoint=115,round-s-codepoint=24,final-round-s-codepoint=25}
%StopVisiblePreamble
\pagestyle{empty}
\begin{document}
\autotypelangoptions{german}{long-s}
\usefont{T1}{wesu}{m}{n}
Oskar entschied sich angesichts des aussichtslosen Wassermangels zur Umsiedlung von Moskau nach Minsk.\\
\usefont{T1}{wesu}{b}{sl}
Oskar entschied sich angesichts des aussichtslosen Wassermangels zur Umsiedlung von Moskau nach Minsk.
\end{document}
\end{externalDocument}

Die Sütterlinschrift von B. Ludewig gibt es in zwei Varianten: einer aufrechten mit dem Namen \texttt{suet14} und einer geneigten mit dem Namen \texttt{schwell} (siehe auch \cite[39\psqq]{dtk96.1:neugebauer:krakelig}). Die Verwendung wird durch das Hilfspaket \package{suetterl}\footnote{\url{https://ctan.org/pkg/fundus-sueterlin}} erleichtert, das den Schriftauswahlbefehl \command{suetterlin} bereitstellt und \texttt{schwell} als Kursive von \texttt{suet14} definiert.
\begin{externalDocument}{long-s}
\documentclass[DIV=8,fontsize=20pt]{scrartcl}
\setlength\parindent{0pt}
%StartVisiblePreamble
\usepackage[german]{babel}
\usepackage{autotype}
\usepackage{suetterl}
\autotypefontoptions{suet14}{long-s-codepoint=115,round-s-codepoint=28}
\autotypefontoptions{schwell}{long-s-codepoint=115,round-s-codepoint=28}
%StopVisiblePreamble
\pagestyle{empty}
\begin{document}
\autotypelangoptions{german}{long-s}
\suetterlin
Oskar entschied sich angesichts des aussichtslosen Wassermangels zur Umsiedlung von Moskau nach Minsk.\\
\itshape
Oskar entschied sich angesichts des aussichtslosen Wassermangels zur Umsiedlung von Moskau nach Minsk.
\end{document}
\end{externalDocument}

\subsection{Bekannte Probleme und offene Fragen}
\begin{itemize}
\parskip=\dimexpr\baselineskip-3mm-\parsep
\item Die zur Zeit vorliegenden Pseudotrennmuster für die s-Schreibung berücksichtigen zwar die alte und neue deutsche Rechtschreibung, nicht jedoch die Schweizer Orthographie. Bei Wörtern wie »ausser« oder »Stoss« führt dies zu falschen s-Schreibungen (\BspFrak{\autotypelangoptions{ngerman}{long-s}ausser, Stoss}). Fraglich ist, ob die Schreibung mit Doppel-s im Fraktursatz bei solchen Wörtern überhaupt jemals praktiziert wurde oder ob auch in der Schweiz hier das ß genutzt wurde.
\item Bei Verwendung von Gesangstexten in \package{musixtex}/\package{musixlyr} muss der Text mit expliziten Trennstellen eingegeben werden. Dies kann zu falschen s-Schreibungen führen, insbesondere am Silbenende.
\begin{externalDocument}{long-s}
\documentclass[12pt]{article}
%StartVisiblePreamble
\usepackage[german]{babel}
\usepackage{fontspec}
\usepackage{autotype}
\usepackage{musixtex}
\input{musixlyr}
\setmainfont{yfrak.otf}
\autotypelangoptions{german}{long-s}
%StopVisiblePreamble
\pagestyle{empty}
\begin{document}
\begin{music}
\nobarnumbers
\generalsignature{1}
\generalmeter{\meterfrac{2}{4}}
\setlyrics{Text}{Ging ein Weib-lein Nüs-se schüt-teln}
\assignlyrics{1}{Text}
\startextract
\Notes\ca{g}\ca{g}\ca{f}\ca{e}\en\bar
\Notes\ca{d}\ca{d}\ca{g}\ca{g}\en
\endextract
\end{music}
\end{document}
\end{externalDocument}
\item Bei Opentype-Schriften, die das Merkmal \texttt{calt} (Contextual Alternates) oder ein anderes Opentype-Merkmal nutzen, um Rund- und Lang-s nach eigenen Regeln zu setzen, funktioniert die s-Ersetzung durch \autotype\ nicht. Das Merkmal muss daher ausgeschaltet werden; siehe den Abschnitt \ref{missaali} für ein Beispiel.
\item Falls in einer Abkürzung ein Lang-s am Wortende stehen soll, kann dieses nicht automatisch gesetzt werden, da die Trennmuster immer nur Stellen zwischen zwei Buchstaben bezeichnen und sich die Markierung im Falle der s-Schreibung vereinbarungsgemäß auf den vorangehenden Buchstaben bezieht. Am Wortende setzt der Algorithmus unabhängig von den Mustern stets ein Rund-s. Ein Beispiel wäre die Abkürzung »StAss« für »Studienassessor«. Fraglich ist, ob solche Abkürzungen überhaupt in Fraktur zu setzen sind.
\item Falls eine Schriftart eine Ligatur aus Rund-s und einem weiteren Buchstaben definiert, deren Glyphe ein Lang-s enthält, führt dies zu falscher s-Schreibung. Zwar wird von \autotype\ zunächst ein Rund-s eingefügt, doch verschwindet dieses anschließend wieder durch die Ligatur. Auch die Option \option{ligbreak} verhindert das in einigen Fällen nicht.
\begin{externalDocument}{long-s}
\documentclass[DIV=8,fontsize=16pt]{scrartcl}
\usepackage[german]{babel}
\usepackage{fontspec}
\usepackage{autotype}
\pagestyle{empty}
\setlength\parindent{0pt}
\begin{document}
\fontspec{Missaali}[Contextuals=AlternateOff]
\autotypelangoptions{german}{long-s}
Eislauf Auslosung Haustier Breslau Moslem\\
\autotypelangoptions{german}{ligbreak,long-s}
Eislauf Auslosung Haustier Breslau Moslem
\end{document}
\end{externalDocument}
\item Bei einer Nicht-Opentype-Schriftfamilie mit mehreren Schriften (z.\,B. die Sütterlinschrift aus dem Paket \package{schulschriften}, siehe Abschnitt \ref{suetterlin}) müssen die Codepositionen der s-Glyphen für alle Einzelschriften separat festgelegt werden, da die Schriften unterschiedliche \TeX-Namen haben. Besser wäre eine pauschale Festlegung für die ganze Schriftfamilie. Deren Name ist im \LaTeX-Makro\ \verb:\f@family: abgelegt. Wie er in Lua\TeX\ abgefragt werden kann, muss noch untersucht werden.
\item Bei den im Abschnitt \ref{suetterlin} beschriebenen Sütterlinschriften wird ein Lang-s vor einem Trennstrich durch ein Rund-s ersetzt. Die geschieht auch ohne die \autotype-Option \option{long-s} und liegt an einer entsprechenden Definition in den Ligaturtabellen der Schriften, die wohl eigentlich für Wörter mit Bindestrich gedacht war.%
\begin{externalDocument}{long-s}
\documentclass[DIV=8,fontsize=17pt]{scrartcl}
\usepackage[german]{babel}
\usepackage{autotype}
\pagestyle{empty}
\begin{document}
\usefont{T1}{wesu}{m}{n}
\autotypefontoptions{wesu14}{long-s-codepoint=115,round-s-codepoint=24,final-round-s-codepoint=25}
\autotypelangoptions{german}{long-s}
\parbox{6cm}{Wasser Wasser Wasser Wasser}
\end{document}
\end{externalDocument}
\end{itemize}

\printbibliography[heading=subbibnumbered,title={Literatur zum Satz gebrochener Schriften mit \TeX}]

\newpage
\appendix
\KOMAoptions{paper=landscape,DIV=14}
\recalctypearea
\setlength\columnsep{2em}
\newcommand\Abschnitt[1]{\begin{paracol}{2}
\small
\selectlanguage{german}
\autotypelangoptions{german}{hyphenation=default}
#1
\switchcolumn
\autotypelangoptions{german}{hyphenation=weighted}
\linenumbers
#1
\end{paracol}}
\newcommand\KantI{\textbf{I. Von dem Unterschiede der reinen und empirischen Erkenntnis}

Daß alle unsere Erkenntnis mit der Erfahrung anfange, daran ist gar
kein Zweifel; denn wodurch sollte das Erkenntnisvermögen sonst zur
Ausübung erweckt werden, geschähe es nicht durch Gegenstände, die
unsere Sinne rühren und teils von selbst Vorstellungen bewirken, teils
unsere Verstandestätigkeit in Bewegung bringen, diese zu vergleichen,
sie zu verknüpfen oder zu trennen, und so den rohen Stoff sinnlicher
Eindrücke zu einer Erkenntnis der Gegenstände zu verarbeiten, die
Erfahrung heißt? Der Zeit nach geht also keine Erkenntnis in uns vor
der Erfahrung vorher, und mit dieser fängt alle an.

Wenn aber gleich alle unsere Erkenntnis mit der Erfahrung anhebt, so
entspringt sie darum doch nicht eben alle aus der Erfahrung. Denn
es könnte wohl sein, daß selbst unsere Erfahrungserkenntnis ein
Zusammengesetztes aus dem sei, was wir durch Eindrücke empfangen, und
dem, was unser eigenes Erkenntnisvermögen (durch sinnliche Eindrücke
bloß veranlaßt) aus sich selbst hergibt, welchen Zusatz wir von jenem
Grundstoffe nicht eher unterscheiden, als bis lange Übung uns darauf
aufmerksam und zur Absonderung desselben geschickt gemacht hat.

Es ist also wenigstens eine der näheren Untersuchung noch benötigte
und nicht auf den ersten Anschein sogleich abzufertigende Frage: ob es
ein dergleichen von der Erfahrung und selbst von allen Eindrücken der
Sinne unabhängiges Erkenntnis gebe. Man nennt solche Erkenntnisse a
priori, und unterscheidet sie von den empirischen, die ihre Quellen a
posteriori nämlich in der Erfahrung, haben.

Jener Ausdruck ist indessen noch nicht bestimmt genug, um den ganzen
Sinn, der vorgelegten Frage angemessen, zu bezeichnen. Denn man pflegt
wohl von mancher aus Erfahrungsquellen abgeleiteten Erkenntnis zu
sagen, daß wir ihrer a priori fähig oder teilhaftig sind, weil wir sie
nicht unmittelbar aus der Erfahrung, sondern aus einer allgemeinen
Regel, die wir gleichwohl selbst doch aus der Erfahrung entlehnt
haben, ableiten. So sagt man von jemand, der das Fundament seines
Hauses untergrub: er konnte es a priori wissen, daß es einfallen
würde, d.\,i. er durfte nicht auf die Erfahrung, daß es wirklich
einfiele, warten. Allein gänzlich a priori konnte er dieses doch auch
nicht wissen. Denn daß die Körper schwer sind, und daher, wenn ihnen
die Stütze entzogen wird, fallen, mußte ihm doch zuvor durch Erfahrung
bekannt werden.

Wir werden also im Verfolg unter Erkenntnissen a priori nicht solche
verstehen, die von dieser oder jener, sondern die schlechterdings
von aller Erfahrung unabhängig stattfinden. Ihnen sind empirische
Erkenntnisse, oder solche, die nur a posteriori, d.\,i. durch Erfahrung,
möglich sind, entgegengesetzt. Von den Erkenntnissen a priori heißen
aber die jenigen rein, denen gar nichts Empirisches beigemischt ist.
So ist z.\,B. der Satz: eine jede Veränderung hat ihre Ursache, ein Satz
a priori, allein nicht rein, weil Veränderung ein Begriff ist, der nur
aus der Erfahrung gezogen werden kann.}
\newcommand\KantII{\textbf{II. Wir sind im Besitze gewisser Erkenntnisse a priori, und selbst der
gemeine Verstand ist niemals ohne solche}

Es kommt hier auf ein Merkmal an, woran wir sicher ein reines
Erkenntnis vom empirischen unterscheiden können. Erfahrung lehrt uns
zwar, daß etwas so oder so beschaffen sei, aber nicht, daß es nicht
anders sein könne. Findet sich also erstlich ein Satz, der zugleich
mit seiner Notwendigkeit gedacht wird, so ist er ein Urteil a priori,
ist er überdem auch von keinem abgeleitet, als der selbst wiederum als
ein notwendiger Satz gültig ist, so ist er schlechterdings a priori.
Zweitens: Erfahrung gibt niemals ihren Urteilen wahre oder strenge,
sondern nur angenommene und komparative Allgemeinheit (durch
Induktion), so daß es eigentlich heißen muß: soviel wir bisher
wahrgenommen haben, findet sich von dieser oder jener Regel keine
Ausnahme. Wird also ein Urteil in strengen Allgemeinheit gedacht, d.\,i.
so, daß gar keine Ausnahme als möglich verstattet wird, so ist es
nicht von der Erfahrung abgeleitet, sondern schlechterdings a priori
gültig. Die empirische Allgemeinheit ist also nur eine willkürliche
Steigerung der Gültigkeit, von der, welche in den meisten Fällen,
zu der, die in allen gilt, wie z.\,B. in dem Satze: alle Körper sind
schwer; wo dagegen strenge Allgemeinheit zu einem Urteile wesentlich
gehört, da zeigt diese auf einen besonderen Erkenntnisquell desselben,
nämlich ein Vermögen des Erkenntnisses a priori. Notwendigkeit und
strenge Allgemeinheit sind also sichere Kennzeichen einer Erkenntnis
a priori, und gehören auch unzertrennlich zueinander. Weil es aber
im Gebrauche derselben bisweilen leichter ist, die empirische
Beschränktheit derselben, als die Zufälligkeit in den Urteilen, oder
es auch manchmal einleuchtender ist, die unbeschränkte Allgemeinheit,
die wir einem Urteile beilegen, als die Notwendigkeit desselben zu
zeigen, so ist es ratsam, sich gedachter beider Kriterien, deren jedes
für sich unfehlbar ist, abgesondert zu bedienen.

Daß es nun dergleichen notwendige und im strengsten Sinne allgemeine,
mithin reine Urteile a priori, im menschlichen Erkenntnis wirklich
gebe, ist leicht zu zeigen. Will man ein Beispiel aus Wissenschaften,
so darf man nur auf alle Sätze der Mathematik hinaussehen, will man
ein solches aus dem gemeinsten Verstandesgebrauche, so kann der Satz,
daß alle Veränderung eine Ursache haben müsse, dazu dienen; ja in dem
letzteren enthält selbst der Begriff einer Ursache so offenbar den
Begriff einer Notwendigkeit der Verknüpfung mit einer Wirkung und
einer strengen Allgemeinheit der Regel, daß er gänzlich verlorengehen
würde, wenn man ihn, wie Hume tat, von einer öftern Beigesellung
dessen, was geschieht, mit dem, was vorhergeht, und einer daraus
entspringenden Gewohnheit, (mithin bloß subjektiven Notwendigkeit,)
Vorstellungen zu verknüpfen, ableiten wollte. Auch könnte man, ohne
dergleichen Beispiele zum Beweise der Wirklichkeit reiner Grundsätze
a priori in unserem Erkenntnisse zu bedürfen, dieser ihre
Unentbehrlichkeit zur Möglichkeit der Erfahrung selbst, mithin
a priori dartun. Denn wo wollte selbst Erfahrung ihre Gewißheit
hernehmen, wenn alle Regeln, nach denen sie fortgeht, immer wieder
empirisch, mithin zufällig wären; daher man diese schwerlich für
erste Grundsätze gelten lassen kann. Allein hier können wir uns damit
begnügen, den reinen Gebrauch unseres Erkenntnisvermögens als Tatsache
samt den Kennzeichen desselben dargelegt zu haben. Aber nicht bloß in
Urteilen, sondern selbst in Begriffen zeigt sich ein Ursprung einiger
derselben a priori. Lasset von eurem Erfahrungsbegriffe eines Körpers
alles, was daran empirisch ist, nach und nach weg: die Farbe, die
Härte oder Weiche, die Schwere, selbst die Undurchdringlichkeit, so
bleibt doch der Raum übrig, den er (welcher nun ganz verschwunden
ist) einnahm, und den könnt ihr nicht weglassen. Ebenso, wenn ihr
von eurem empirischen Begriffe eines jeden, körperlichen oder nicht
körperlichen, Objekts alle Eigenschaften weglaßt, die euch die
Erfahrung lehrt; so könnt ihr ihm doch nicht diejenige nehmen, dadurch
ihr es als Substanz oder einer Substanz anhängend denkt, (obgleich
dieser Begriff mehr Bestimmung enthält, als der eines Objekts
überhaupt). Ihr müßt also, überführt durch die Notwendigkeit, womit
sich dieser Begriff euch aufdringt, gestehen, daß er in eurem
Erkenntnisvermögen a priori seinen Sitz habe.}
\newcommand\KantIII{\textbf{III. Die Philosophie bedarf einer Wissenschaft, welche die
Möglichkeit, die Prinzipien und den Umfang aller Erkenntnisse a
priori bestimme}

Was noch weit mehr sagen will als alles vorige, ist dieses, daß
gewisse Erkenntnisse sogar das Feld aller möglichen Erfahrungen
verlassen, und durch Begriffe, denen überall kein entsprechender
Gegenstand in der Erfahrung gegeben werden kann, den Umfang unserer
Urteile über alle Grenzen derselben zu erweitern den Anschein haben.

Und gerade in diesen letzteren Erkenntnissen, welche über die
Sinnenwelt hinausgehen, wo Erfahrung gar keinen Leitfaden, noch
Berichtigung geben kann, liegen die Nachforschungen unserer Vernunft,
die wir, der Wichtigkeit nach, für weit vorzüglicher, und ihre
Endabsicht für viel erhabener halten, als alles, was der Verstand im
Felde der Erscheinungen lernen kann, wobei wir, sogar auf die Gefahr
zu irren, eher alles wagen, als daß wir so angelegene Untersuchungen
aus irgendeinem Grunde der Bedenklichkeit, oder aus Geringschätzung
und Gleichgültigkeit aufgeben sollten. Diese unvermeidlichen Aufgaben
der reinen Vernunft selbst sind Gott, Freiheit und Unsterblichkeit.
Die Wissenschaft aber, deren Endabsicht mit allen ihren Zurüstungen
eigentlich nur auf die Auflösung derselben gerichtet ist, heißt
Metaphysik, deren Verfahren im Anfange dogmatisch ist, d.\,i. ohne
vorhergehende Prüfung des Vermögens oder Unvermögens der Vernunft zu
einer so großen Unternehmung zuversichtlich die Ausführung übernimmt.

Nun scheint es zwar natürlich, daß, sobald man den Boden der Erfahrung
verlassen hat, man doch nicht mit Erkenntnissen, die man besitzt, ohne
zu wissen woher, und auf den Kredit der Grundsätze, deren Ursprung man
nicht kennt, sofort ein Gebäude errichten werde, ohne der Grundlegung
desselben durch sorgfältige Untersuchungen vorher versichert zu sein,
daß man also vielmehr die Frage vorlängst werde aufgeworfen haben, wie
denn der Verstand zu allen diesen Erkenntnissen a priori kommen könne,
und welchen Umfang, Gültigkeit und Wert sie haben mögen. In der Tat
ist auch nichts natürlicher, wenn man unter dem Worte natürlich das
versteht, was billiger- und vernünftigerweise geschehen sollte;
versteht man aber darunter das, was gewöhnlichermaßen geschieht,
so ist hinwiederum nichts natürlicher und begreiflicher, als daß
diese Untersuchung lange unterbleiben mußte. Denn ein Teil dieser
Erkenntnisse, als die mathematischen, ist im alten Besitze der
Zuverlässigkeit, und gibt dadurch eine günstige Erwartung auch für
andere, ob diese gleich von ganz verschiedener Natur sein mögen.
Überdem, wenn man über den Kreis der Erfahrung hinaus ist, so ist man
sicher, durch Erfahrung nicht widerlegt zu werden. Der Reiz, seine
Erkenntnisse zu erweitern, ist so groß, daß man nur durch einen klaren
Widerspruch, auf den man stößt, in seinem Fortschritte aufgehalten
werden kann. Dieser aber kann vermieden werden, wenn man seine
Erdichtungen nur behutsam macht, ohne daß sie deswegen weniger
Erdichtungen bleiben. Die Mathematik gibt uns ein glänzendes Beispiel,
wie weit wir es, unabhängig von der Erfahrung, in der Erkenntnis a
priori bringen können. Nun beschäftigt sie sich zwar mit Gegenständen
und Erkenntnissen bloß so weit, als sich solche in der Anschauung
darstellen lassen. Aber dieser Umstand wird leicht übersehen, weil
gedachte Anschauung selbst a priori gegeben werden kann, mithin von
einem bloßen reinen Begriff kaum unterschieden wird. Durch einen
solchen Beweis von der Macht der Vernunft eingenommen, sieht der
Trieb zur Erweiterung keine Grenzen. Die leichte Taube, indem sie im
freien Fluge die Luft teilt, deren Widerstand sie fühlt, könnte die
Vorstellung fassen, daß es ihr im luftleeren Raum noch viel l besser
gelingen werde. Ebenso verließ Plato die Sinnenwelt, weil sie dem
Verstande so enge Schranken setzt, und wagte sich jenseit derselben,
auf den Flügeln der Ideen, in den leeren Raum des reinen Verstandes.
Er bemerkte nicht, daß er durch seine Bemühungen keinen Weg gewönne,
denn er hatte keinen Widerhalt, gleichsam zur Unterlage, worauf er
sich steifen, und woran er seine Kräfte anwenden konnte, um den
Verstand von der Stelle zu bringen. Es ist aber ein gewöhnliches
Schicksal der menschlichen Vernunft in der Spekulation, ihr Gebäude
so früh, wie möglich, fertigzumachen, und hintennach allererst zu
untersuchen, ob auch der Grund dazu gut gelegt sei. Alsdann aber
werden allerlei Beschönigungen herbeigesucht, um uns wegen dessen
Tüchtigkeit zu trösten, oder auch eine solche späte und gefährliche
Prüfung lieber gar abzuweisen. Was uns aber während dem Bauen
von aller Besorgnis und Verdacht frei hält, und mit scheinbarer
Gründlichkeit schmeichelt, ist dieses. Ein großer Teil, und
vielleicht der größte, von dem Geschäfte unserer Vernunft, besteht in
Zergliederungen der Begriffe, die wir schon von Gegenständen haben.
Dieses liefert uns eine Menge von Erkenntnissen, die, ob sie gleich
nichts weiter als Aufklärungen oder Erläuterungen desjenigen sind, was
in unsern Begriffen (wiewohl noch auf verworrene Art) schon gedacht
worden, doch wenigstens der Form nach neuen Einsichten gleich
geschätzt werden, wiewohl sie der Materie, oder dem Inhalte nach die
Begriffe, die wir haben, nicht erweitern, sondern nur auseinander
setzen. Da dieses Verfahren nun eine wirkliche Erkenntnis a priori
gibt, die einen sichern und nützlichen Fortgang hat, so erschleicht
die Vernunft, ohne es selbst zu merken, unter dieser Vorspiegelung
Behauptungen von ganz anderer Art, wo die Vernunft zu gegebenen
Begriffen ganz fremde und zwar a priori hinzutut, ohne daß man weiß,
wie sie dazu gelangen und ohne sich eine solche Frage auch nur in
die Gedanken kommen zu lassen. Ich will daher gleich anfangs von dem
Unterschiede dieser zweifachen Erkenntnisart handeln.}
\newcommand\KantIV{\textbf{IV. Von dem Unterschiede analytischer und synthetischer Urteile}

In allen Urteilen, worinnen das Verhältnis eines Subjekts zum Prädikat
gedacht wird, (wenn ich nur die bejahenden erwäge, denn auf die
verneinenden ist nachher die Anwendung leicht,) ist dieses Verhältnis
auf zweierlei Art möglich. Entweder das Prädikat B gehört zum Subjekt
A als etwas, was in diesem Begriffe A (versteckterweise) enthalten
ist; oder B liegt ganz außer dem Begriff A, ob es zwar mit demselben
in Verknüpfung steht. Im ersten Fall nenne ich das Urteil analytisch,
in dem andern synthetisch. Analytische Urteile (die bejahenden) sind
also diejenigen, in welchen die Verknüpfung des Prädikats mit dem
Subjekt durch Identität, diejenigen aber, in denen diese Verknüpfung
ohne Identität gedacht wird, sollen synthetische Urteile
heißen. Die ersteren könnte man auch Erläuterungs-, die andern
Erweiterungs-Urteile heißen, weil jene durch das Prädikat nichts zum
Begriff des Subjekts hinzutun, sondern diesen nur durch Zergliederung
in seine Teilbegriffe zerfällen, die in selbigen schon (obgleich
verworren) gedacht waren: dahingegen die letzteren zu dem Begriffe
des Subjekts ein Prädikat hinzutun, welches in jenem gar nicht
gedacht war, und durch keine Zergliederung desselben hätte können
herausgezogen werden. Z.\,B. wenn ich sage: alle Körper sind
ausgedehnt, so ist dies ein analytisch Urteil. Denn ich darf nicht
über den Begriff, den ich mit dem Körper verbinde, hinausgehen, um
die Ausdehnung, als mit demselben verknüpft, zu finden, sondern
jenen Begriff nur zergliedern, d.\,i. des Mannigfaltigen, welches ich
jederzeit in ihm denke, mir nur bewußt werden, um dieses Prädikat
darin anzutreffen; es ist also ein analytisches Urteil. Dagegen, wenn
ich sage: alle Körper sind schwer, so ist das Prädikat etwas ganz
anderes, als das, was ich in dem bloßen Begriff eines Körpers
überhaupt denke. Die Hinzufügung eines solchen Prädikats gibt also ein
synthetisch Urteil.

Erfahrungsurteile, als solche, sind insgesamt synthetisch. Denn es
wäre ungereimt, ein analytisches Urteil auf Erfahrung zu gründen, weil
ich aus meinem Begriffe gar nicht hinausgehen darf, um das Urteil
abzufassen, und also kein Zeugnis der Erfahrung dazu nötig habe. Daß
ein Körper ausgedehnt sei, ist ein Satz, der a priori feststeht, und
kein Erfahrungsurteil. Denn, ehe ich zur Erfahrung gehe, habe ich alle
Bedingungen zu meinem Urteile schon in dem Begriffe, aus welchem ich
das Prädikat nach dem Satze des Widerspruchs nur herausziehen, und
dadurch zugleich der Notwendigkeit des Urteils bewußt werden kann,
welche mir Erfahrung nicht einmal lehren würde. Dagegen, ob ich schon
in dem Begriff eines Körpers überhaupt das Prädikat der Schwere gar
nicht einschließe, so bezeichnet jener doch einen Gegenstand der
Erfahrung durch einen Teil derselben, zu welchem ich also noch
andere Teile eben derselben Erfahrung, als zu dem ersteren gehörten,
hinzufügen kann. Ich kann den Begriff des Körpers vorher analytisch
durch die Merkmale der Ausdehnung, der Undurchdringlichkeit, der
Gestalt usw., die alle in diesem Begriffe gedacht werden, erkennen.
Nun erweitere ich aber meine Erkenntnis, und, indem ich auf die
Erfahrung zurücksehe, von welcher ich diesen Begriff des Körpers
abgezogen hatte, so finde ich mit obigen Merkmalen auch die Schwere
jederzeit verknüpft, und füge also diese als Prädikat zu jenem
Begriffe synthetisch hinzu. Es ist also die Erfahrung, worauf sich die
Möglichkeit der Synthesis des Prädikats der Schwere mit dem Begriffe
des Körpers gründet, weil beide Begriffe, ob zwar einer nicht in dem
anderen enthalten ist, dennoch als Teile eines Ganzen, nämlich der
Erfahrung, die selbst eine synthetische Verbindung der Anschauungen
ist, zueinander, wiewohl nur zufälligerweise, gehören.

Aber bei synthetischen Urteilen a priori fehlt dieses Hilfsmittel ganz
und gar. Wenn ich über den Begriff A hinausgehen soll, um einen andern
B als damit verbunden zu erkennen, was ist das, worauf ich mich
stütze, und wodurch die Synthesis möglich wird? da ich hier den
Vorteil nicht habe, mich im Felde der Erfahrung danach umzusehen.
Man nehme den Satz: Alles, was geschieht, hat seine Ursache. In dem
Begriff von etwas, das geschieht, denke ich zwar ein Dasein, vor
welchem eine Zeit vorhergeht usw. und daraus lassen sich analytische
Urteile ziehen. Aber der Begriff einer Ursache liegt ganz außer jenem
Begriffe, und zeigt etwas von dem, was geschieht, Verschiedenes an,
ist also in dieser letzteren Vorstellung gar nicht mit enthalten. Wie
komme ich denn dazu, von dem, was überhaupt geschieht, etwas davon
ganz Verschiedenes zu sagen, und den Begriff der Ursache, obzwar in
jenem nicht enthalten, dennoch, als dazu und sogar notwendig gehörig,
zu erkennen. Was ist hier das Unbekannte = X, worauf sich der Verstand
stützt, wenn er außer dem Begriff von A ein demselben fremdes Prädikat
B aufzufinden glaubt, welches er gleichwohl damit verknüpft zu sein
erachtet? Erfahrung kann es nicht sein, weil der angeführte Grundsatz
nicht allein mit größerer Allgemeinheit, sondern auch mit dem Ausdruck
der Notwendigkeit, mithin gänzlich a priori und aus bloßen Begriffen,
diese zweite Vorstellungen zu der ersteren hinzugefügt. Nun beruht
auf solchen synthetischen d.\,i. Erweiterungs-Grundsätzen die ganze
Endabsicht unserer spekulativen Erkenntnis a priori; denn die
analytischen sind zwar höchst wichtig und nötig, aber nur um zu
derjenigen Deutlichkeit der Begriffe zu gelangen, die zu einer
sicheren und ausgebreiteten Synthesis, als zu einem wirklich neuen
Erwerb, erforderlich ist.}
\newcommand\KantV{\textbf{V. In allen theoretischen Wissenschaften der Vernunft sind
synthetische Urteile a priori als Prinzipien enthalten}

1. Mathematische Urteile sind insgesamt synthetisch. Dieser Satz
scheint den Bemerkungen der Zergliederer der menschlichen Vernunft
bisher entgangen, ja allen ihren Vermutungen gerade entgegengesetzt
zu sein, ob er gleich unwidersprechlich gewiß und in der Folge sehr
wichtig ist. Denn weil man fand, daß die Schlüsse der Mathematiker
alle nach dem Satze des Widerspruchs fortgehen, (welches die Natur
einer jeden apodiktischen Gewißheit erfordert,) so überredet man sich,
daß auch die Grundsätze aus dem Satze des Widerspruchs erkannt würden;
worin sie sich irrten; denn ein synthetischer Satz kann allerdings
nach dem Satze des Widerspruchs eingesehen werden, aber nur so,
daß ein anderer synthetischen Satz vorausgesetzt wird, aus dem er
gefolgert werden kann, niemals aber an sich selbst.

Zuvörderst muß bemerkt werden: daß eigentliche mathematische Sätze
jederzeit Urteile a priori und nicht empirisch sind, weil sie
Notwendigkeit bei sich führen, welche aus Erfahrung nicht abgenommen
werden kann. Will man aber dieses nicht einräumen, wohlan, so schränke
ich meinen Satz auf die reine Mathematik ein, deren Begriff es
schon mit sich bringt, daß sie nicht empirische, sondern bloß reine
Erkenntnis a priori enthalte.

Man sollte anfänglich zwar denken: daß der Satz 7 + 5 = 12 ein bloß
analytischer Satz sei, der aus dem Begriffe einer Summe von Sieben und
Fünf nach dem Satze des Widerspruches erfolge. Allein, wenn man es
näher betrachtet, so findet man, daß der Begriff der Summe von 7 und
5 nichts weiter enthalte, als die Vereinigung beider Zahlen in eine
einzige, wodurch ganz und gar nicht gedacht wird, welches diese
einzige Zahl sei, die beide zusammenfaßt. Der Begriff von Zwölf ist
keineswegs dadurch schon gedacht, daß ich mir bloß jene Vereinigung
von Sieben und Fünf denke, und, ich mag meinen Begriff von einer
solchen möglichen Summe noch solange zergliedern, so werde ich
doch darin die Zwölf nicht antreffen. Man muß über diese Begriffe
hinausgehen, indem man die Anschauung zu Hilfe nimmt, die einem von
beiden korrespondiert, etwa seine fünf Finger, oder (wie Segner in
seiner Arithmetik) fünf Punkte, und so nach und nach die Einheiten der
in der Anschauung gegebenen Fünf zu dem Begriffe der Sieben hinzutut.
Denn ich nehme zuerst die Zahl 7, und, indem ich für den Begriff der 5
die Finger meiner Hand als Anschauung zu Hilfe nehme, so tue ich die
Einheiten, die ich vorher zusammennahm, um die Zahl 5 auszumachen, nun
an jenem meinem Bilde nach und nach zur Zahl 7, und sehe so die Zahl
12 entspringen. Daß 7 zu 5 hinzugetan werden sollten, habe ich zwar in
dem Begriffe einer Summe = 7 + 5 gedacht, aber nicht, daß diese Summe
der Zahl 12 gleich sei. Der arithmetische Satz ist also jederzeit
synthetisch; welches man desto deutlicher inne wird, wenn man etwas
größere Zahlen nimmt, da es dann klar einleuchtet, daß, wir möchten
unsere Begriffe drehen und wenden, wie wir wollen, wir, ohne die
Anschauung zu Hilfe zu nehmen, vermittels der bloßen Zergliederung
unserer Begriffe die Summe niemals finden könnten.

Ebensowenig ist irgendein Grundsatz der reinen Geometrie analytisch.
Daß die gerade Linie zwischen zwei Punkten die kürzeste sei, ist ein
synthetischen Satz. Denn mein Begriff vom Geraden enthält nichts von
Größe, sondern nur eine Qualität. Der Begriff des Kürzesten kommt also
gänzlich hinzu, und kann durch keine Zergliederung aus dem Begriffe
der geraden Linie gezogen werden. Anschauung muß also hier zu Hilfe
genommen werden, vermittels deren allein die Synthesis möglich ist.

Einige wenige Grundsätze, welche die Geometer voraussetzen, sind zwar
wirklich analytisch und beruhen auf dem Satze des Widerspruchs, sie
dienen aber auch nur, wie identische Sätze, zur Kette der Methode und
nicht als Prinzipien, z.\,B. a = a, das Ganze ist sich selber gleich,
oder (a + b) > a, d.\,i. das Ganze ist größer als sein Teil. Und doch
auch diese selbst, ob sie gleich nach bloßen Begriffen gelten, werden
in der Mathematik nur darum zugelassen, weil sie in der Anschauung
können dargestellt werden. Was uns hier gemeiniglich glauben macht,
als läge das Prädikat solcher apodiktischen Urteile schon in
unserm Begriffe, und das Urteil sei also analytisch, ist bloß die
Zweideutigkeit des Ausdrucks. Wir sollen nämlich zu einem gegebenen
Begriffe ein gewisses Prädikat hinzudenken, und diese Notwendigkeit
haftet schon an den Begriffen. Aber die Frage ist nicht, was wir zu
dem gegebenen Begriffe hinzudenken sollen, sondern was wir wirklich in
ihm, obzwar nur dunkel, denken, und da zeigt sich, daß das Prädikat
jenen Begriffen zwar notwendig, aber nicht als im Begriffe selbst
gedacht, sondern vermittels einer Anschauung, die zu dem Begriffe
hinzukommen muß, anhänge.

2. Naturwissenschaft (Physica) enthält synthetische Urteile a priori
als Prinzipien in sich. Ich will nur ein paar Sätze zum Beispiel
anführen, als den Satz: daß in allen Veränderungen der körperlichen
Welt die Quantität der Materie unverändert bleibe, oder daß, in aller
Mitteilung der Bewegung, Wirkung und Gegenwirkung jederzeit einander
gleich sein müssen. An beiden ist nicht allein die Notwendigkeit,
mithin ihr Ursprung a priori, sondern auch, daß sie synthetische Sätze
sind, klar. Denn in dem Begriffe der Materie denke ich mir nicht
die Beharrlichkeit, sondern bloß ihre Gegenwart im Raume durch die
Erfüllung desselben. Also gehe ich wirklich über den Begriff von der-
Materie hinaus, um etwas a priori zu ihm hinzuzudenken, was ich in ihm
nicht dachte. Der Satz ist also nicht analytisch, sondern synthetisch
und dennoch a priori gedacht, und so in den übrigen Sätzen des reinen
Teils der Naturwissenschaft.

3. In der Metaphysik, wenn man sie auch nur für eine bisher bloß
versuchte, dennoch aber durch die Natur der menschlichen Vernunft
unentbehrliche Wissenschaft ansieht, sollen synthetische Erkenntnisse
a priori enthalten sein, und es ist ihr gar nicht darum zu tun,
Begriffe, die wir uns a priori von Dingen machen, bloß zu zergliedern
und dadurch analytisch zu erläutern, sondern wir wollen unsere
Erkenntnis a priori erweitern, wozu wir uns solcher Grundsätze
bedienen müssen, die über den gegebenen Begriff etwas hinzutun, was in
ihm nicht enthalten war, und durch synthetische Urteile a priori wohl
gar so weit hinausgehen, daß uns die Erfahrung selbst nicht so weit
folgen kann, z.\,B. in dem Satze: die Welt muß einen ersten Anfang
haben, u.\,a.\,m. und so besteht Metaphysik wenigstens ihrem Zwecke nach
aus lauter synthetischen Sätzen a priori.}
\newcommand\KantVI{\textbf{VI. Allgemeine Aufgabe der reinen Vernunft}

Man gewinnt dadurch schon sehr viel, wenn man eine Menge von
Untersuchungen unter die Formel einer einzigen Aufgabe bringen kann.
Denn dadurch erleichtert man sich nicht allein selbst sein eigenes
Geschält, indem man es sich genau bestimmt, sondern auch jedem
anderen, der es prüfen will, das Urteil, ob wir unserem Vorhaben ein
Genüge getan haben oder nicht. Die eigentliche Aufgabe der reinen
Vernunft ist nun in der Frage enthalten: Wie sind synthetische Urteile
a priori möglich?

Daß die Metaphysik bisher in einem so schwankenden Zustande der
Ungewißheit und Widersprüche geblieben ist, ist lediglich der Ursache
zuzuschreiben, daß man sich diese Aufgabe und vielleicht sogar den
Unterschied der analytischen und synthetischen Urteile nicht früher
in Gedanken kommen ließ. Auf der Auflösung dieser Aufgabe, oder einem
genugtuenden Beweise, daß die Möglichkeit, die sie erklärt zu wissen
verlangt, in der Tat gar nicht stattfinde, beruht nun das Stehen und
Fallen der Metaphysik. David Hume, der dieser Aufgabe unter allen
Philosophen noch am nächsten trat, sie aber sich bei weitem nicht
bestimmt genug und in ihrer Allgemeinheit dachte, sondern bloß bei dem
synthetischen Satze der Verknüpfung der Wirkung mit ihren Ursachen
(Principium causalitatis) stehen blieb, glaubte herauszubringen, daß
ein solcher Satz a priori gänzlich unmöglich sei, und nach seinen
Schlüssen würde alles, was wir Metaphysik nennen, auf einen bloßen
Wahn von vermeinter Vernunfteinsicht dessen hinauslaufen, was in der
Tat bloß aus der Erfahrung erborgt und durch Gewohnheit den Schein
der Notwendigkeit überkommen hat; auf welche, alle reine Philosophie
zerstörende, Behauptung er niemals gefallen wäre, wenn er unsere
Aufgabe in ihrer Allgemeinheit vor Augen gehabt hätte, da er dann
eingesehen haben würde, daß, nach seinem Argumente, es auch keine
reine Mathematik geben könnte, weil diese gewiß synthetische Sätze a
priori enthält, vor welcher Behauptung ihn alsdann sein guter Verstand
wohl würde bewahrt haben.

In der Auflösung obiger Aufgabe ist zugleich die Möglichkeit
des reinen Vernunftgebrauches in Gründung und Ausführung aller
Wissenschaften, die eine theoretische Erkenntnis a priori von
Gegenständen enthalten, mit begriffen, d.\,i. die Beantwortung der
Fragen:

Wie ist reine Mathematik möglich?
Wie ist reine Naturwissenschaft möglich?

Von diesen Wissenschaften, da sie wirklich gegeben sind, läßt sich nun
wohl geziemend fragen: wie sie möglich sind; denn daß sie möglich sein
müssen, wird durch ihre Wirklichkeit bewiesen*. Was aber Metaphysik
betrifft, so muß ihr bisheriger schlechter Fortgang, und weil man von
keiner einzigen bisher vorgetragenen, was ihren wesentlichen Zweck
angeht, sagen kann, sie sei wirklich vorhanden, einen jeden mit Grund
an ihrer Möglichkeit zweifeln lassen.

* Von der reinen Naturwissenschaft könnte mancher dieses letztere noch
  bezweifeln. Allein man darf nur die verschiedenen Sätze, die im
  Anfange der eigentlichen (empirischen) Physik vorkommen, nachsehen,
  als den von der Beharrlichkeit derselben Quantität Materie, von
  der Trägheit, der Gleichheit der Wirkung und Gegenwirkung usw.,
  so wird man bald überzeugt werden, daß sie eine physicam puram
  (oder rationalem) ausmachen, die es wohl verdient, als eigene
  Wissenschaft, in ihrem engen oder weiten, aber doch ganzen Umfange,
  abgesondert aufgestellt zu werden.

Nun ist aber diese Art von Erkenntnis in gewissem Sinne doch
auch als gegeben anzusehen, und Metaphysik ist, wenngleich nicht
als Wissenschaft, doch als Naturanlage (metaphysica naturalis)
wirklich. Denn die menschliche Vernunft geht unaufhaltsam, ohne
daß bloße Eitelkeit des Vielwissens sie dazu bewegt, durch eigenes
Bedürfnis getrieben bis zu solchen Fragen fort, die durch keinen
Erfahrungsgebrauch der Vernunft und daher entlehnte Prinzipien
beantwortet werden können, und so ist wirklich in allen Menschen,
sobald Vernunft sich in ihnen bis zur Spekulation erweitert,
irgendeine Metaphysik zu aller Zeit gewesen, und wird auch immer darin
bleiben. Und nun ist auch von dieser die Frage:

Wie ist Metaphysik als Naturanlage möglich?

d.\,i. wie entspringen die Fragen, welche reine Vernunft sich
aufwirft, und die sie, so gut als sie kann, zu beantworten durch
ihr eigenes Bedürfnis getrieben wird, aus der Natur der allgemeinen
Menschenvernunft?

Da sich aber bei allen bisherigen Versuchen, diese natürlichen Fragen,
z.\,B. ob die Welt einen Anfang habe, oder von Ewigkeit her sei, usw.
zu beantworten, jederzeit unvermeidliche Widersprüche gefunden haben,
so kann man es nicht bei der bloßen Naturanlage zur Metaphysik, d.\,i.
dem reinen Vernunftvermögen selbst, woraus zwar immer irgendeine
Metaphysik (es sei welche es wolle) erwächst, bewenden lassen, sondern
es muß möglich sein, mit ihr es zur Gewißheit zu bringen, entweder
im Wissen oder Nicht-Wissen der Gegenstände, d.\,i. entweder der
Entscheidung über die Gegenstände ihrer Fragen, oder über das Vermögen
und Unvermögen der Vernunft in Ansehung ihrer etwas zu urteilen, also
entweder unsere reine Vernunft mit Zuverlässigkeit zu erweitern, oder
ihr bestimmte und sichere Schranken zu setzen. Diese letzte Frage, die
aus der obigen allgemeinen Aufgabe fließt, würde mit Recht diese sein:
Wie ist Metaphysik als Wissenschaft möglich?

Die Kritik der Vernunft führt also zuletzt notwendig zur Wissenschaft;
der dogmatische Gebrauch derselben ohne Kritik dagegen auf grundlose
Behauptungen, denen man ebenso scheinbare entgegensetzen kann, mithin
zum Skeptizismus.

Auch kann diese Wissenschaft nicht von großer abschreckender
Weitläufigkeit sein, weil sie es nicht mit Objekten der Vernunft,
deren Mannigfaltigkeit unendlich ist, sondern es bloß mit sich selbst,
mit Aufgaben, die ganz aus ihrem Schoße entspringen, und ihr nicht
durch die Natur der Dinge, die von ihr unterschieden sind, sondern
durch ihre eigene vorgelegt sind, zu tun hat; da es denn, wenn sie
zuvor ihr eigen Vermögen in Ansehung der Gegenstände, die ihr in der
Erfahrung vorkommen mögen, vollständig hat kennenlernen, leicht werden
muß, den Umfang und die Grenzen ihres über alle Erfahrungsgrenzen
versuchten Gebrauchs vollständig und sicher zu bestimmen.

Man kann also und muß alle bisher gemachten Versuche, eine Metaphysik
dogmatisch zustande zu bringen, als ungeschehen ansehen; denn was in
der einen oder der anderen Analytisches, nämlich bloße Zergliederung
der Begriffe ist, die unserer Vernunft a priori beiwohnen, ist
noch gar nicht der Zweck, sondern nur eine Veranstaltung zu der
eigentlichen Metaphysik, nämlich seine Erkenntnis a priori synthetisch
zu erweitern, und ist zu diesem untauglich, weil sie bloß zeigt, was
in diesen Begriffen enthalten ist, nicht aber, wie wir a priori zu
solchen Begriffen gelangen, um danach auch ihren gültigen Gebrauch
in Ansehung der Gegenstände aller Erkenntnis überhaupt bestimmen zu
können. Es gehört auch nur wenig Selbstverleugnung dazu, alle diese
Ansprüche aufzugeben, da die nicht abzuleugnenden und im dogmatischen
Verfahren auch unvermeidlichen Widersprüche der Vernunft mit sich
selbst jede bisherige Metaphysik schon längst um ihr Ansehen gebracht
haben. Mehr Standhaftigkeit wird dazu nötig sein, sich durch die
Schwierigkeit innerlich und den Widerstand äußerlich nicht abhalten zu
lassen, eine der menschlichen Vernunft unentbehrliche Wissenschaft,
von der man wohl jeden hervorgeschossenen Stamm abhauen, die Wurzel
aber nicht ausrotten kann, durch eine andere, der bisherigen ganz
entgegengesetzte, Behandlung endlich einmal zu einem gedeihlichen und
fruchtbaren Wuchse zu befördern.}
\newcommand\KantVII{\textbf{VII. Idee und Einteilung einer besonderen Wissenschaft, unter dem
Namen der Kritik der reinen Vernunft}

Aus diesem allein ergibt sich nun die Idee einer besonderen
Wissenschaft, die Kritik der reinen Vernunft heißen kann. Denn ist
Vernunft das Vermögen, welches die Prinzipien der Erkenntnis a priori
an die Hand gibt. Daher ist reine Vernunft diejenige, welche die
Prinzipien, etwas schlechthin a priori zu erkennen, enthält. Ein
Organon der reinen Vernunft würde ein Inbegriff derjenigen Prinzipien
sein, nach denen alle reinen Erkenntnisse a priori können erworben und
wirklich zustande gebracht werden. Die ausführliche Anwendung eines
solchen Organon würde ein System der reinen Vernunft verschaffen. Da
dieses aber sehr viel verlangt ist, und es noch dahin steht, ob auch
hier überhaupt eine Erweiterung unserer Erkenntnis, und in welchen
Fällen sie möglich sei; so können wir eine Wissenschaft der bloßen
Beurteilung der reinen Vernunft, ihrer Quellen und Grenzen, als die
Propädeutik zum System der reinen Vernunft ansehen. Eine solche würde
nicht eine Doktrin, sondern nur Kritik der reinen Vernunft heißen
müssen, und ihr Nutzen würde in Ansehung der Spekulation wirklich nur
negativ sein, nicht zur Erweiterung, sondern nur zur Läuterung unserer
Vernunft dienen, und sie von Irrtümern frei halten, welches schon sehr
viel gewonnen ist. Ich nenne alle Erkenntnis transzendental, die sich
nicht sowohl mit Gegenständen, sondern mit unserer Erkenntnisart
von Gegenständen, insofern diese a priori möglich sein soll,
überhaupt beschäftigt. Ein System solcher Begriffe würde
Transzendental-Philosophie heißen. Diese ist aber wiederum für den
Anfang noch zu viel. Denn, weil eine solche Wissenschaft sowohl die
analytische Erkenntnis, als die synthetische a priori vollständig
enthalten müßte, so ist sie, soweit es unsere Absicht betrifft, von zu
weitem Umfange, indem wir die Analysis nur so weit treiben dürfen, als
sie unentbehrlich notwendig ist, um die Prinzipien der Synthesis a
priori, als warum es uns nur zu tun ist, in ihrem ganzen Umfange
einzusehen. Diese Untersuchung, die wir eigentlich nicht Doktrin,
sondern nur transzendentale Kritik nennen können, weil sie nicht die
Erweiterung der Erkenntnisse selbst, sondern nur die Berichtigung
derselben zur Absicht hat, und den Probierstein des Werts oder Unwerts
aller Erkenntnisse a priori abgeben soll, ist das, womit wir uns jetzt
beschäftigen. Eine solche Kritik ist demnach eine Vorbereitung, wo
möglich, zu einem Organon, und wenn dieses nicht gelingen sollte,
wenigstens zu einem Kanon derselben, nach welchem allenfalls dereinst
das vollständige System der Philosophie der reinen Vernunft, es mag
nun in Erweiterung oder bloßer Begrenzung ihrer Erkenntnis bestehen,
sowohl analytisch als synthetisch dargestellt werden könnte. Denn daß
dieses möglich sei, ja daß ein solches System von nicht gar großem
Umfange sein könne, um zu hoffen, es ganz zu vollenden, läßt sich
schon zum voraus daraus ermessen, daß hier nicht die Natur der Dinge,
welche unerschöpflich ist, sondern der Verstand, der über die Natur
der Dinge urteilt, und auch dieser wiederum nur in Ansehung seiner
Erkenntnis a priori, den Gegenstand ausmacht, dessen Vorrat, weil wir
ihn doch nicht auswärtig suchen dürfen, uns nicht verborgen bleiben
kann, und allem Vermuten nach klein genug ist, um vollständig
aufgenommen, nach seinem Werte oder Unwerte beurteilt und unter
richtige Schätzung gebracht zu werden. Noch weniger darf man hier eine
Kritik der Bücher und Systeme der reinen Vernunft erwarten, sondern
die des reinen Vernunftvermögens selbst. Nur allein, wenn diese zum
Grunde liegt, hat man einen sicheren Probierstein, den philosophischen
Gehalt alter und neuer Werke in diesem Fache zu schätzen;
widrigenfalls beurteilt der unbefugte Geschichtsschreiber und Richter
grundlose Behauptungen anderer, durch seine eigenen, die ebenso
grundlos sind.

Die Transzendental-Philosophie ist die Idee einer Wissenschaft, wozu
die Kritik der reinen Vernunft den ganzen Plan architektonisch, d.\,i.
aus Prinzipien, entwerfen soll, mit völliger Gewährleistung der
Vollständigkeit und Sicherheit aller Stücke, die dieses Gebäude
ausmachen. Sie ist das System aller Prinzipien der reinen Vernunft.
Daß diese Kritik nicht schon selbst Transzendental-Philosophie heißt,
beruht lediglich darauf, daß sie, um ein vollständiges System zu sein,
auch eine ausführliche Analysis der ganzen menschlichen Erkenntnis a
priori enthalten müßte. Nun muß zwar unsere Kritik allerdings auch
eine vollständige Herzählung aller Stammbegriffe, welche die gedachte
reine Erkenntnis ausmachen, vor Augen legen. Allein der ausführlichen
Analysis dieser Begriffe selbst, wie auch der vollständigen Rezension
der daraus abgeleiteten, enthält sie sich billig, teils weil diese
Zergliederung nicht zweckmäßig wäre, indem sie die Bedenklichkeit
nicht hat, welche bei der Synthesis angetroffen wird, um deren willen
eigentlich die ganze Kritik da ist, teils, weil es der Einheit des
Planes zuwider wäre, sich mit der Verantwortung der Vollständigkeit
einer solchen Analysis und Ableitung zu befassen, deren man
in Ansehung seiner Absicht doch überhoben sein konnte. Diese
Vollständigkeit der Zergliederung sowohl, als der Ableitung aus den
künftig zu liefernden Begriffen a priori, ist indessen leicht zu
ergänzen, wenn sie nur allererst als ausführliche Prinzipien der
Synthesis da sind, und in Ansehung dieser wesentlichen Absicht nichts
ermangelt.

Zur Kritik der reinen Vernunft gehört demnach alles, was die
Transzendental-Philosophie ausmacht, und sie ist die vollständige Idee
der Transzendental-Philosophie, aber diese Wissenschaft noch nicht
selbst; weil sie in der Analysis nur so weit geht, als es zur
vollständigen Beurteilung der synthetischen Erkenntnis a priori
erforderlich ist.

Das vornehmste Augenmerk bei der Einteilung einer solchen Wissenschaft
ist: daß gar keine Begriffe hineinkommen müssen, die irgend etwas
Empirisches in sich enthalten; oder daß die Erkenntnis a priori völlig
rein sei. Daher, obzwar die obersten Grundsätze der Moralität und die
Grundbegriffe derselben, Erkenntnisse a priori sind, so gehören sie
doch nicht in die Transzendental-Philosophie, weil sie die Begriffe
der Lust und Unlust, der Begierden und Neigungen usw., die insgesamt
empirischen Ursprungs sind, zwar selbst nicht zum Grunde ihrer
Vorschriften legen, aber doch im Begriffe der Pflicht, als Hindernis,
das überwunden, oder als Anreiz, der nicht zum Bewegungsgrunde
gemacht werden soll, notwendig in die Abfassung des Systems
der reinen Sittlichkeit mit hineinziehen müssen. Daher ist die
Transzendental-Philosophie eine Weltweisheit der reinen bloß
spekulativen Vernunft. Denn alles Praktische, sofern es Triebfedern
enthält, bezieht sich auf Gefühle, welche zu empirischen
Erkenntnisquellen gehören.

Wenn man nun die Einteilung dieser Wissenschaft aus dem allgemeinen
Gesichtspunkte eines Systems überhaupt anstellen will, so muß die,
welche wir jetzt vortragen, erstlich eine Elementar-Lehre, zweitens
eine Methoden-Lehre der reinen Vernunft enthalten. Jeder dieser
Hauptteile würde seine Unterabteilung haben, deren Gründe sich
gleichwohl hier noch nicht vortragen lassen. Nur so viel scheint zur
Einleitung, oder Vorerinnerung, nötig zu sein, daß es zwei Stämme
der menschlichen Erkenntnis gebe, die vielleicht aus einer
gemeinschaftlichen, aber uns unbekannten Wurzel entspringen, nämlich
Sinnlichkeit und Verstand, durch deren ersteren uns Gegenstände
gegeben, durch den zweiten aber gedacht werden. Sofern nun die
Sinnlichkeit Vorstellungen a priori enthalten sollte, welche die
Bedingung ausmachen, unter der uns Gegenstände gegeben werden, so
würde sie zur Transzendental-Philosophie gehören. Die transzendentale
Sinnenlehre würde zum ersten Teile der Elementarwissenschaft gehören
müssen, weil die Bedingungen, worunter allein die Gegenstände der
menschlichen Erkenntnis gegeben werden, denjenigen vorgehen, unter
welchen selbige gedacht werden.}
\section{Vergleich von Trennalgorithmen anhand von Immanuel Kants Einleitung zur Kritik der reinen Vernunft\label{Kant}}

Textquelle: \url{https://www.gutenberg.org/cache/epub/6343/pg6343.html}

\bigskip
\begin{paracol}{2}
\small
\noindent\emph{\textbf{Linke Spalte:} Standardtrennalgorithmus, alle Trennstellen mit 50 Strafpunkten versehen}
\switchcolumn
\noindent\emph{\textbf{Rechte Spalte:} Gewichteter Trennalgorithmus, Primärtrennstellen mit \directlua{tex.print(autotype.get_penalty(1))}, Sekundärtrennstellen mit \directlua{tex.print(autotype.get_penalty(2))}, Tertiärtrennstellen mit \directlua{tex.print(autotype.get_penalty(3))} Strafpunkten gewichtet}
\end{paracol}
\Abschnitt{\KantI}
\Abschnitt{\KantII}
\Abschnitt{\KantIII}
\Abschnitt{\KantIV}
\Abschnitt{\KantV}
\Abschnitt{\KantVI}
\Abschnitt{\KantVII}

\end{document}
