\documentclass{letgut}

% On compose la présente documentation en Kp-Fonts, sauf en ce qui concerne la
% fonte à chasse fixe qui n'est pas ce qu'il y a de plus réussi chez elle.
\setmainfont{KpRoman}
\setsansfont{KpSans}
% \setmonofont{TeX Gyre Cursor}
\setmonofont[Scale = MatchLowercase]{RobotoMono}

% On ne veut pas d'éditorial dans cette « Lettre » et on veut que sa couleur de
% fond ne soit pas la couleur ivoire par défaut, mais du blanc.
\letgutsetup{
  , editorial=false
  , pagecolor= {1,1,1}
  , detailedtoc = subsection
  % ,final
}

% Bibliographie
\addbibresource{letgut.bib}

% On définit quelques commandes (utiles seulement pour la présente
% documentation), par exemple celle permettant d'accéder à la couleur de fond du
% papier lorsque l'option `paper' n'est pas employée.
\ExplSyntaxOn
\NewDocumentCommand { \letgutdefaultpagecolor } {  } {
  \c__letgut_default_pagecolor_screen_clist
}
\definecolor{letgutdefaultpagecolor}{rgb}{\c__letgut_default_pagecolor_screen_clist}

\NewDocumentCommand { \letgutalertboxdefaultcolor } {  } {
  \c__letgut_default_alert_box_color_tl
}

\NewDocumentCommand { \letgutallcolorslinks } {  } {
  \c__letgut_default_allcolors_links_color_tl
}

\NewDocumentCommand { \docker } { s } {
  \IfBooleanTF{#1}%
  {\software{docker}}%
  {\software[https://docs.docker.com/]{Docker}}%
}

\pdfstringdefDisableCommands{%
  \def\docker{Docker}
}
\ExplSyntaxOff

% Acronymes
\letgutacro[tag=thisdocument]{LD}{\LaTeX{} dépendant}
\letgutacro[tag=thisdocument]{CD}{Compact Disc}[]
\letgutacro[tag=thisdocument]{TLMGR}{\TeX~Live Manager}[gestionnaire \TeX~Live]
\letgutacro[tag=thisdocument,short=no-op,short-format=\upshape] % Désactive les petites capitales
{NOOP}
{No Operation}
[instruction nulle]
\letgutacro{CPGÉ}{Classes Préparatoires aux Grandes Écoles}

% Éléments du titre
\title{%
  Documentation de la
  \texorpdfstring{%
    \letgutcls%
  }{%
    classe letgut%
  }%
}
\author{Association GUTenberg}
\date{%
  Version 0.9.12 en date du \today%
  \texorpdfstring{%
    \\
    \url{https://framagit.org/gutenberg/letgut}%
  }{%
  }%
}

% \tcbset{index command name=\jobname-commands}
% \makeindex[name=\jobname-commands,title=Index des commandes,extout=odx,extin=ond,columnseprule]
% \makeindex[title=Index des concepts,extout=pdx,extin=pnd,columnseprule]
\begin{document}

\syntaxhl{%
  LaTeX,%
  acro,%
  babel-french,%
  biblatex,%
  biolinum,%
  % classes,%
  csquotes,%
  % graphicx,%
  hologo,%
  hyperref,%
  letgut,%
  listings,%
  tcolorbox,%
  varioref,%
}

\title{Introduction}
\label{sec:introduction}
L'association \gutenberg{} publie la \lettregut{}, son bulletin irrégulomestriel,
depuis février 1993 \autocite{AssociationGUTenbergLettreGUTenberg}.

Pour ce faire, une classe \hologo{(La)TeX} dédiée, maison, a peu à peu vu le
jour\footnote{Notamment grâce au concours de \person{André, Jacques},
  \person{Flipo, Daniel} et \person{Chupin, Maxime}.} mais, au gré des
nouveaux besoins et des personnes qui ont assuré la publication de la \lettre{},
son développement a été quelque peu erratique ; il n'aurait notamment pas été
possible de publier son code en l'état. En outre, sa documentation était
inexistante.

Le \ca{} de l'association élu en novembre 2020 a souhaité fournir une classe
mieux structurée, davantage pérenne et documentée, à même d'être publiée sur le
\ctan{}. C'est désormais chose faite avec la présente \letgutcls\footnote{À
  cette occasion, la classe a été légèrement renommée de \class[]{let-gut} en
  \letgut{}.}.

\title{Usage de la classe \letgut}
\label{sec:usage-de-la}

\section{Compilation}
\label{sec:compilation}

\begin{dbwarning}{\hologo{LuaLaTeX} (récent) et UTF-8 nécessaires}{lualatex-recent}
  Les documents recourant à la \letgutcls{} doivent :
  \begin{itemize}
  \item être compilés avec le moteur \hologo{LuaLaTeX}\footnote{Dans une version
      au minimum \texttt{1.13.2}, le format \software{lualatex} devant être dans
      une version au minimum \texttt{2021.6.6}.} ;
  \item avoir (de ce fait) comme codage d'entrée l'UTF-8\footnote{Y compris les
      fichiers sources auxiliaires tels que les \file{.bib}.}.
  \end{itemize}
\end{dbwarning}

La section \enquote{\nameref{recours-docker}}, \vpageref{recours-docker},
détaille un moyen de de disposer d'une telle version de \hologo{LuaLaTeX} sans
risque de perturber une installation de \hologo{(La)TeX} déjà existante.

\section{Options}
\label{sec:options}

\subsection{Options de la \class*{article}}
\label{sec:options-de-la}

La \letgutcls{} est basée sur la \class*{article}, chargée avec les options
\docAuxKey{twoside} et \docAuxKey{11pt} qui ne peuvent donc être
surchargées. Les autres options peuvent être employées mais sont déconseillées
par souci d'homogénéité de la mise en page des numéros successifs de la
\lettregut{}.

\subsection{Options de la \letgutcls}
\label{sec:options-de-la}

\subsubsection{Spécification}
\label{sec:specification}

Les options de la \letgutcls{} peuvent être spécifiées de deux façons :
\begin{enumerate}
\item en argument de la commande dédiée \refCom{letgutsetup} ;
\item à la compilation, au moyen de la commande (à lancer dans un terminal) :
\begingroup
\lstset{basicstyle=\ttfamily\small}
\terminal{lualatex '\PassOptionsToClass{£\meta{options}£}{letgut} \input{£\meta{fichier}£}'}{}
\endgroup
\end{enumerate}

\begin{dbwarning}{Options de \letgut{} : pas en argument optionnel
    de \docAuxCommand{documentclass}}{}
  On évitera de passer les options de la \letgutcls{} en argument optionnel de
  la commande \docAuxCommand{documentclass} et ce, de sorte à éviter les
  conflits d'options avec les différents packages chargés.
\end{dbwarning}

\begin{docCommand}{letgutsetup}{\marg{options}}
  \index{configuration}%
  Cette commande permet de spécifier les \meta{options} de la \letgutcls{}.
\end{docCommand}

\subsubsection{Liste des options}
\label{sec:liste-des-options}

\begin{docKeys}
  {
    {
      doc name = for-readers,
      doc description = {\valinitdef[\docValue*{true}][\docValue*{true}]},
    },
    {
      doc name = for-authors,
      doc description = {\valinitdef[\docValue*{false}][\docValue*{true}]},
    },
  }%
  %
  Ces clés booléennes, contraires l'une de l'autre, activent les versions
  respectivement \enquote{pour lecteurs} (par défaut) et \enquote{pour auteurs}
  de la \lettre{}. Celles-ci mettent en page la \lettre{} respectivement :
  \begin{itemize}
  \item telle que les lecteurs finaux la liront ;
  \item telle que les auteurs d'articles de la \lettre{} peuvent le souhaiter au
    cours de leur rédaction, notamment sans :
    \begin{itemize}
    \item titre ;
    \item (r)appel à cotisation (cf. clé \refKey{membership-reminder}) ;
    \item éditorial\footnote{Ou avertissement s'il n'est pas trouvé.} (cf. clé
      \refKey{editorial}) ;
    \item informations sur \gut{}\footnote{Ou avertissement si elles ne sont pas
        trouvées.}  (cf. clé \refKey{informations}) ;
    \item fichiers attachés au \pdf{} produit (cf. clé
      \refKey{reverse-files-attachement}).
    \end{itemize}
  \end{itemize}
\end{docKeys}

\begin{docKeys}
  {
    {
      doc name = draft,
      doc description = {\valinitdef[\docValue*{true}][\docValue*{true}]},
    },
    {
      doc name = final,
      doc description = {\valinitdef[\docValue*{false}][\docValue*{true}]},
    },
  }%
  %
  Ces clés booléennes, contraires l'une de l'autre, activent les versions
  respectivement \enquote{brouillon} (par défaut) et \enquote{final} de la
  \lettre{}.

  En version \enquote{brouillon}, et seulement dans cette version% \footnote{Avec
    % la version actuelle, une conséquence supplémentaire mais temporaire est
    % que }
  :
  \begin{enumerate}
  \item le \package*{lua-typo} est chargé\footnote{En fait, cela est
      momentanément désactivé car le \package*{luacolor}, chargé en sous-main
      par \package{lua-typo}, ne fonctionne actuellement pas bien avec une
      fonctionnalité du noyau \hologo{LaTeX} utilisée par la classe (plus de
      détails
      \href{https://github.com/ho-tex/luacolor/issues/4}{ici}.)}. Celui-ci met
    en lumière, par un changement de couleur, les lignes typographiquement
    imparfaites d’un fichier \pdf*{} produit par \hologo{LuaLaTeX} ;
  \item le mot \enquote{Brouillon} figure en filigrane sur chaque page ;
  \item les boîtes trop pleines (\foreignloc{Overfull \docAuxCommand*{hbox}})
    sont mises en évidence comme avec les classes standards.
  \end{enumerate}
\end{docKeys}

\begin{docKeys}
  {
    {
      doc name = screen,
      doc description = {\valinitdef[\docValue*{true}][\docValue*{true}]},
    },
    {
      doc name = paper,
      doc description = {\valinitdef[\docValue*{false}][\docValue*{true}]},
    },
  }%
  %
  Ces clés booléennes, contraires l'une de l'autre, activent les formats de
  sortie respectivement \enquote{écran} (par défaut) et \enquote{papier} de la
  \lettre{}.

  En version \enquote{écran}, et seulement dans cette version, la couleur de
  fond du papier est par défaut non pas le blanc mais celle spécifiée (et
  modifiable) par \refKey{pagecolor}.
\end{docKeys}

\begin{docKey}{number}{=\meta{numéro}}{\valinitdef}
  Cette clé permet de spécifier le \meta{numéro} de la \lettre{}.
\end{docKey}

{%
  % \tcbset{before lower=\vspace*{.5\baselineskip}\par}
  \begin{docKey}{date}{=\meta{année}\texttt{-}\meta{mois} ou \meta{texte}}{\valinitdef[année et mois en cours]}
    Cette clé permet de spécifier la date de la \lettre{}. Celle-ci est affichée sous
    la forme :
    \begin{itemize}
    \item \enquote{\meta{Mois} \meta{année}} dans les cas où l'option :
      \begin{itemize}
      \item n'est pas employée (les mois et année en cours sont alors
        utilisés) ;
      \item est employée sous la forme \refKey*{date}™=™\meta{année}™-™\meta{mois}
        où \meta{année} et \meta{mois} doivent être des nombres entiers
        positifs. Les garde-fous suivants sont mis en place :
        \begin{itemize}
        \item si \meta{année} n'est pas celle en cours ou la suivante, elle est
          remplacée par l'année en cours ;
        \item si \meta{mois} n'est pas entre $1$ et $12$, il est remplacé par le
          mois en cours ;
        \end{itemize}
      \end{itemize}
    \item \enquote{\meta{texte}} si l'option est employée sous la forme
      \refKey*{date}™=™\meta{texte}.
    \end{itemize}
  \end{docKey}
}

\begin{docKey}{pagecolor}{=\marg{couleur}}{\valinitdef[\docValue*{letgut_pagecolor}]}
  Cette clé permet, si (et seulement) l'option \refKey{paper} \emph{n'}est
  \emph{pas} employée, de spécifier (selon le modèle \enquote{rgb}) une
  \meta{couleur} de fond du papier autre que celle appliquée par
  défaut\footnote{C'est-à-dire \colorbox{letgutdefaultpagecolor}{celle-ci}% , dont
    % la valeur \enquote{rgb} est \docValue*{\letgutdefaultpagecolor}
    .}.
\end{docKey}

\begin{docKey}{allcolorslinks}{=\meta{couleur}}{\valinitdef[\docValue*{letgut_allcolors_links}]}
  Cette clé permet de spécifier (selon le modèle \enquote{\foreignloc{named}})
  une \meta{couleur} pour (tous) les liens hypertextes autre que celle
  \docColor{letgut_allcolors_links} par défaut\footnote{C'est-à-dire
    \textcolor{letgut_allcolors_links}{celle-ci}% , égale
    % à \docValue*{\letgutallcolorslinks}
    .}.
\end{docKey}

\begin{docKey}{membership-reminder}{}{\valinitdef[\docValue*{true}][\docValue*{true}]}
  Cette clé booléenne affiche automatiquement un (r)appel à cotisation en bas de
  1\iere{} page de la \lettre{}.
\end{docKey}

\begin{docKey}{editorial}{}{\valinitdef[\docValue*{true}][\docValue*{true}]}
  Cette clé booléenne importe automatiquement en tout début de la \lettre{}
  (néanmoins après le titre et le sommaire) le \file*{editorial.tex} contenant
  l'éditorial. Si aucun \file*{editorial.tex} n'est trouvé, un avertissement est
  émis lors de la compilation et une boîte d'alerte est affichée en 1\iere{}
  page.
\end{docKey}

\begin{docKey}{informations}{}{\valinitdef[\docValue*{true}][\docValue*{true}]}
  Cette clé booléenne importe automatiquement en dernière page de la \lettre{}
  le \file*{informations-gut.tex} contenant toutes les informations sur
  \gut{}. Si aucun \file*{informations-gut.tex} n'est trouvé, un avertissement
  est émis lors de la compilation et une boîte d'alerte est affichée en dernière
  page.
\end{docKey}

{\tcbset{before lower=\vspace*{\baselineskip}\par}
  %
  \begin{docKey}[][doc updated={2023-01-19}]{detailedtoc}{=\docValue{section}\textbar\docValue{subsection}\textbar\docValue{subsubsection}\textbar\docValue{paragraph}\textbar\docValue{subparagraph}}{\valinitdef[\docValue*{title}][\docValue*{all}]}
    Par défaut, une table des matières est automatiquement insérée en début de
    document, avec comme niveau de profondeur celui des titres des articles
    (saisis via la \refCom{title}), et seulement eux.  La clé \refKey{detailedtoc}
    permet de modifier le \enquote{niveau de profondeur} de cette table des
    matières, respectivement jusqu'aux sections, sous-sections,
    sous-sous-sections, paragraphes, sous-paragraphes. Sont également acceptées
    les valeurs spéciales :
    \begin{itemize}
    \item \docValue{all} (alias de \docValue{subparagraph}) ;
    \item \docValue{none} qui inhibe l'affichage de la table des matières.
    \end{itemize}

    \begin{dbremark}{Tables des matières locales}{}
      Chaque article peut contenir une table des matières locale, affichée au
      moyen de la commande \docAuxCommand{localtableofcontents} (fournie par le
      \package*{etoc} chargé en sous-main par la \letgutcls{}). Le niveau de
      profondeur est par défaut celui des sections mais cela peut être modifié en
      la faisant précéder de la commande \docAuxCommand{etocsetnexttocdepth} (dont
      l'argument est l'une des valeurs possibles de la clé \refKey{detailedtoc}).
    \end{dbremark}
  \end{docKey}
}

\begin{docKey}[][doc new={2023-01-14}]{reverse-files-attachement}{}{\valinitdef[pas de valeur]}
  Si, et seulement si, la \lettre{} est à la fois en version pour les lecteurs
  (cf. \refKey{for-readers}) et en sortie écran (cf. \refKey{screen}), chacun
  des fichiers nécessaires (et suffisants) à la compilation d'un de ses articles
  est :
  \begin{itemize}
  \item attaché au \pdf{} produit ;
  \item accessible en cliquant sur l'hyperlien correspondant en forme de
    trombone : \noattachfile[icon=Paperclip] ;
  \end{itemize}
  L'option \refKey{reverse-files-attachement} inverse ce comportement par
  défaut.
\end{docKey}

\begin{docKeys}[doc new={2024-10-07}]
  {
    {
      doc name = watermark letter,
      doc description = {\valinitdef[g]},
    },
    {
      doc name = watermark scale,
      doc description = {\valinitdef[1]},
    },
  }%
  La première page de la \lettre{} comporte, en filigrane et en gris clair,
  (entre autre) un (très grand) \enquote{g}, composé avec la fonte
  \enquote{normale}.
  \begin{itemize}
  \item La clé \refKey{watermark letter} permet de spécifier une lettre autre
    qu'un \enquote{g} et/ou composé avec une fonte autre que \enquote{normale}.
  \item La clé \refKey{watermark scale} permet de spécifier échelle autre que~1
    pour cette lettre (par exemple si celle-ci s'avère par défaut trop grande).
  \end{itemize}
\end{docKeys}

\subsection{Options autres}
\label{sec:options-autres}

D'autres options peuvent être passées à la \letgutcls{}. Il est ainsi possible
de faire usage de langues du \package*{babel}, autres que le français et
l'anglais déjà chargées par \letgut{}, en les stipulant en option de
\docAuxCommand{documentclass} et en les employant selon la syntaxe de ce
package.

\section{Titre et titres courants}
\label{sec:titre}
Si la commande \docAuxCommand{title}\marg{titre} est
\begin{description}
\item[\emph{non} utilisée :] le titre du document est construit à partir du
  \meta{numéro} et de la \meta{date} spécifiés (cf. clés \refKey{number} et
  \refKey{date}). Il figure alors automatiquement en 1\iere{} page sous la forme
  \enquote{Numéro \meta{numéro} -- \meta{date}}. Le titre courant est alors
  \enquote{La \lettregut{}, \meta{date}} ;
\item[\phantom{\emph{non}} utilisée\footnotemark{} :]\footnotetext{Ainsi que les
    habituelles commandes \docAuxCommand{author} et \docAuxCommand{date}.}  et
  ce, \emph{avant \lstinline+\\begin\{document\}+}, le \meta{titre} du document
  figure alors automatiquement en 1\iere{} page sous sa forme habituelle et est
  suivie d'un changement de page. Le titre courant est alors
  \enquote{\meta{titre}, \meta{date}}.
\end{description}

\begin{dbwarning}{\docAuxCommand{title} et \docAuxCommand{author}
    $\neq$ avant et après \lstinline+\\begin\{document\}+}{}
  Les commandes \docAuxCommand{title} et \docAuxCommand{author} ne se comportent
  pas de la même façon avant et après ™\begin{document}™ (cf. sections
    \nameref{sec:structuration} \vpageref{sec:structuration} et
    \nameref{sec:sign-des-articl} \vpageref{sec:sign-des-articl}).
\end{dbwarning}

\begin{dbwarning}{Commande \docAuxCommand{maketitle} à \emph{ne pas} employer}{}
  La commande \docAuxCommand{maketitle} est à \emph{ne pas} employer car elle
  l'est en sous-main par la classe.
\end{dbwarning}

\section{Importation d'articles}
\label{sec:import-dart}

\begin{dbwarning}{Importation de fichiers d'articles}{}
  Si le contenu d'un article est stocké dans un \meta{fichier enfant}\file{.tex},
  on l'importera dans un fichier parent recourant à la \letgutcls{} :
  \begin{itemize}
  \item \emph{non pas} au moyen de la commande ordinaire \docAuxCommand{input} ;
  \item \emph{mais}  au moyen de la commande \refCom{inputarticle}.
  \end{itemize}
\end{dbwarning}

\begin{docCommands}
  {
    {
      doc name = inputarticle,
      doc parameter = \marg{fichier enfant},
    },
    {
      doc name = inputarticle*,
      doc parameter = \marg{fichier enfant},
    },
  }
  Cette commande permet d'importer le contenu d'un article stocké dans un
  \meta{fichier enfant}\file{.tex}.

  En version étoilée, les fichiers nécessaires (et suffisants) à la compilation
  de l'article ne sont pas attachés au \pdf{} produit (cf. clé
  \refKey{reverse-files-attachement}).

  En plus de l'importation proprement dite, cette commande procède à un certain
  nombre de réinitialisations.
\end{docCommands}

\section{Structuration}
\label{sec:structuration}

\begin{docCommands}[
  doc parameter = \oarg{intitulé alternatif}\marg{intitulé}
  ]
  {
    { doc name = title },
    { doc name = subtitle },
    { doc name = section },
    { doc name = subsection },
    { doc name = subsubsection },
    { doc name = paragraph },
    { doc name = subparagraph },
  }
  %
  Ces commandes permettent de structurer le contenu de la \lettre{} :
  \begin{itemize}
  \item \refCom{title} est celle de plus haut niveau, introduisant
    l'\meta{intitulé} de chaque article (automatiquement composé en grandes
    capitales et précédé de l'ornement \aldineleft) ;
  \item \refCom{subtitle}, de niveau suivant et facultative, introduisant un
    \meta{intitulé} d'éventuel sous-titre d'article (automatiquement composé en
    grandes capitales). Ceci peut être utile par exemple pour distinguer des
    parties indépendantes d'un même article ;
  \item celles de niveaux suivants sont les habituelles commandes de
    structuration fournies par la \class*{article}.
  \end{itemize}
\end{docCommands}

\begin{dbremark}{Structure non numérotée}{}
  Les titres, sous-titres, sections, sous-sections, etc. de la \lettre{} ne sont
  pas numérotés. Aussi pourra-t-on, pour faire référence à l'une de ces
  rubriques, recourir aux commandes :
  \begin{itemize}
  \item \docAuxCommand{nameref} pour en citer l'\meta{intitulé} ;
  \item \docAuxCommand{vpageref} pour en citer la page ;
  \item \docAuxCommand{enquote} pour, le cas échéant, faire figurer
    l'\meta{intitulé} entre guillemets ;
  % \item \docAuxCommand{namecref} (fournie par le \package*{cleveref}, chargé par
  %   \letgut{}) pour en citer la nature (le cas échéant).
  \end{itemize}
  ces trois commandes étant directement utilisables puisque fournies par les
  packages respectivement \package{hyperref}, \package{varioref} et
  \package{csquotes}, chargés en sous-main par la \letgutcls{}.
\end{dbremark}

\begin{ltx-code-result}[title addon=références croisées aux rubriques,listing options app={deletekeywords={[3]{nameref,section}}}]
On lira avec intérêt la section \enquote{\nameref{sec:acronymes}}
\vpageref{sec:acronymes}.
\end{ltx-code-result}

\section[Personnes et auteurs]{Noms de personnes et d'auteurs d'articles}
\label{sec:sign-des-articl}

% Les prénom, nom et titre d'une personne peuvent être affichés au moyen de la
% commande \refCom{person} suivante.

\begin{docCommands}[
  doc name = person,
  doc name = author,
  ]
  {
    {
      doc name = person,
      doc parameter = \marg{données}
    },
    {
      doc name = person,
      doc parameter = \brackets{\meta{données$_1$} and \meta{données$_2$}[ and ...]}
    },
    {
      doc name = person*,
      doc parameter = \brackets{\meta{données$_1$} and \meta{données$_2$}[ and ...]}
    },
    {
      doc name = author,
      doc parameter = \marg{données}
    },
    {
      doc name = author,
      doc parameter = \brackets{\meta{données$_1$} and \meta{données$_2$}[ and ...]}
    },
    {
      doc name = author*,
      doc parameter = \brackets{\meta{données$_1$} and \meta{données$_2$}[ and ...]}
    },
  }
  \index{auteur}
  \index{personne}
  %
  Ces commandes affichent\footnote{Au fer à droite pour \refCom{author}.} les
  \meta{données} (noms et éventuels prénoms et titres) d'une ou plusieurs
  personnes ou d'un ou plusieurs auteurs d'articles, ces \meta{données}
  étant spécifiées :
  \begin{description}
  \item[pour un individu unique] selon l'un des formats suivants :
    \begin{itemize}
    \item \meta{nom}
    \item \meta{nom}™, ™\meta{prénom}
    \item \meta{nom}™, ™\meta{prénom}™, ™\meta{titre}
    \end{itemize}
  \item[pour des individus multiples :]\leavevmode
    \begin{itemize}
    \item selon le même schéma que pour un individu unique ;
    \item les \meta{données} de chacun des individus étant séparées par le mot clé
      ™and™.
    \end{itemize}
  \end{description}
  Les version étoilées de ces commandes trient alphabétiquement les listes de
  personnes ou d'auteurs.
\end{docCommands}

Indépendamment de la casse utilisée en entrée, pour chaque \meta{nom} et
\meta{prénom}, chacune des initiales et des premières lettres après un espace ou
un tiret est affiché en grande capitale.

\begin{ltx-code-result}[title addon=personnes]
On peut dire merci à
\person{Knuth, Donald E., dieu and Lamport, Leslie} !
\end{ltx-code-result}

\begin{ltx-code-result}[title addon=auteur,listing options app={deletekeywords={[3]{author}}}]
\begin{displayquote} % Fourni par `csquotes' chargé par `letgut'
  Wait, wait, I never said that.
  \author{knuth, dONALD e.}
\end{displayquote}
\end{ltx-code-result}

\section{Aide à la saisie et homogénéisation de la mise en forme}
\label{sec:aide-la-saisie}

Les articles de la \lettre{} sont émaillés de concepts (packages ou classes
\hologo{(La)TeX}, logiciels, etc.) et de termes et expressions
(\enquote{\gut{}}, \enquote{\lettre{}}, etc.) employés de façon
récurrente. Aussi des commandes spécifiques sont-elles prévues de façon à en
faciliter la saisie et à en homogénéiser la mise en forme.

\subsection{Packages et classes, logiciels, fichiers}
\label{sec:classes-packages-et}

\begin{docCommands}
  {
    {
      doc name = package,
      doc parameter = \oarg{URL}\marg{nom},
    },
    {
      doc name = package*,
      doc parameter = \oarg{URL}\marg{nom}\oarg{préfixe},
    },
    {
      doc name = class,
      doc parameter = \oarg{URL}\marg{nom},
    },
    {
      doc name = class*,
      doc parameter = \oarg{URL}\marg{nom}\oarg{préfixe},
    },
  }
  %
  Ces commandes affichent le \meta{nom} d'un package ou d'une classe
  \hologo{(La)TeX}. Le \meta{nom} affiché est un lien hypertexte si et seulement
  si l'argument optionnel est :
  \begin{description}
  \item[\emph{non} employé] la cible étant alors
    \url{https://ctan.org/pkg/}\meta{nom} ;
  \item[\phantom{\emph{non}} employé] mais non vide, la cible étant alors
    \meta{URL}.
  \end{description}
  Pour que le \meta{nom} affiché ne soit pas un lien hypertexte, il suffit
  d'employer un argument optionnel vide.

  Les versions étoilées font précéder le \meta{nom} d'un \meta{préfixe} qui, par
  défaut, est respectivement \enquote{package} et \enquote{classe}.
\end{docCommands}

\begin{ltx-code-result}[title addon=packages et classes,listing options app={deletekeywords={[4]{tables}}}]
La \class*{letgut} s'appuie entre autres sur le \package*{etoc}
(qui permet de personnaliser les tables des matières).

Une des classes s'appuyant sur le \package*[]{etoc} est
\class[https://framagit.org/gutenberg/letgut/]{letgut}.
\end{ltx-code-result}

\begin{docCommands}
  {
    {
      doc name = software,
      doc parameter = \oarg{URL}\marg{nom},
    },
    {
      doc name = software*,
      doc parameter = \oarg{URL}\marg{nom}\oarg{préfixe},
    },
  }
  %
  Ces commandes affichent le \meta{nom} d'un logiciel qui est optionnellement un
  lien hypertexte vers \meta{URL}.  La version étoilée fait précéder le
  \meta{nom} d'un \meta{préfixe} qui, par défaut, est \enquote{logiciel}.
\end{docCommands}

\begin{docCommands}
  {
    {
      doc name = file,
      doc parameter = \marg{nom},
    },
    {
      doc name = file*,
      doc parameter = \marg{nom}\oarg{préfixe},
    },
  }
  %
  Ces commandes affichent le \meta{nom} d'un fichier.  La version étoilée fait
  précéder le \meta{nom} d'un \meta{préfixe} qui, par défaut, est
  \enquote{fichier}.
\end{docCommands}

\begin{ltx-code-result}[title addon=logiciels et fichiers,listing options app={deletekeywords={[3]{file,plus,l,tex}}}]
Le \file*{test.tex} a été ouvert dans le
\software*[https://www.gnu.org/software/emacs/]{Emacs}, plus
précisément dans \software*{Emacs}[l'éditeur de texte].
\end{ltx-code-result}

\begin{dbwarning}{Commandes pas toutes bienvenues en \docAuxCommand{title} et
    \docAuxCommand{subtitle}}{}
  Lorsqu'elles sont employées en argument des commandes \refCom{title} et
  \refCom{subtitle}, les versions étoilées de ces commandes ont des effets
  indésirables (préfixes pas en grandes capitales et signets non conformes).
\end{dbwarning}

\subsection{Locutions étrangères, points de code Unicode}

\begin{docCommand}{foreignloc}{\marg{locution}}
  Cette commande est conçue pour afficher une \meta{locution} étrangère.
\end{docCommand}

\begin{docCommand}{latinloc}{\marg{locution}}
  Cette commande est conçue pour afficher une \meta{locution} latine.
\end{docCommand}

\begin{ltx-code-result}[title addon=locutions étrangères,listing options app={deletekeywords={[3]{options}}}]
Attention aux \foreignloc{load-time options} !
Mais... \latinloc{errare humanum est}.
\end{ltx-code-result}

\begin{docCommand}{Ucode}{\oarg{nom}\marg{point de code}}
  Cette commande est conçue pour afficher le \meta{point de code} et
  éventuellement le \meta{nom} d'un caractère Unicode sous la forme
  \enquote{U+\meta{point de code} (\meta{\textsc{nom}})}.
\end{docCommand}

\begin{ltx-code-result}[title addon=Point de code d'un caractère Unicode,listing options app={deletekeywords={[3]{options}}}]
Unicode a prévu le caractère \Ucode[symbole numéro]{2116}.
\end{ltx-code-result}

\subsection{Termes et expressions (figés)}
\label{sec:termes}

\begin{docCommands}
  {
    { doc name = gutenberg },
    { doc name = gut },
    { doc name = assogut, doc new={2023-01-14} },
    { doc name = Assogut, doc new={2023-01-14} },
    { doc name = lettres, doc new={2023-01-14} },
    { doc name = lettresgut, doc new={2023-01-14} },
    { doc name = cahier },
    { doc name = cahiers },
    { doc name = cahiergut },
    { doc name = cahiersgut },
    { doc name = letgut },
    { doc name = letgutcls },
    { doc name = knuth },
    { doc name = lamport },
    { doc name = tl },
    { doc name = tugboat },
    { doc name = linux },
    { doc name = macos },
    { doc name = windows },
  }
  %
  Ce que ces commandes affichent est répertorié dans le \vref{tab:raccourcis}.
\end{docCommands}

\begin{table}[htb]
  \centering
  \begin{tabular}{ll}
    \toprule
    \refCom{gutenberg}  & \gutenberg  \\
    \refCom{gut}        & \gut        \\
    \refCom{assogut}    & \assogut    \\
    \refCom{Assogut}    & \Assogut    \\
    \refCom{lettres}    & \lettres    \\
    \refCom{lettresgut} & \lettresgut \\
    \refCom{cahier}     & \cahier     \\
    \refCom{cahiergut}  & \cahiergut  \\
    \refCom{cahiers}    & \cahiers    \\
    \refCom{cahiersgut} & \cahiersgut \\ \midrule
    \refCom{letgut}     & \letgut     \\
    \refCom{letgutcls}  & \letgutcls  \\ \midrule
    \refCom{knuth}      & \knuth      \\
    \refCom{lamport}    & \lamport    \\
    \refCom{tl}         & \tl         \\
    \refCom{tugboat}    & \tugboat    \\ \midrule
    \refCom{linux}      & \linux      \\
    \refCom{macos}      & \macos      \\
    \refCom{windows}    & \windows    \\ \bottomrule
  \end{tabular}
  \caption{Effet des commandes de raccourcis}
  \label{tab:raccourcis}
\end{table}

Par homogénéité avec les commandes \refCom{class} et \refCom{class*}, on aurait
pu souhaiter que les terme et expression \enquote{\letgut} et
\enquote{\letgutcls} soient produits par \refCom{letgut} et
\docAuxCommand*{letgut*}. Mais cette dernière commande, étoilée, a dû être
remplacée par une commande non étoilée (\refCom{letgutcls}), sans quoi un
problème technique aurait empêché l'utilisation de \refCom{letgut} en argument
de \docAuxCommand{section}%
\footnote{Plus de détails
  \href{https://tex.stackexchange.com/q/493017/18401}{ici}.}.

\begin{dbwarning}{Commande \refCom{letgutcls} pas bienvenue en \docAuxCommand{title} et
    \docAuxCommand{subtitle}}{}
  Lorsqu'elle est employée en argument des commandes \refCom{title} et
  \refCom{subtitle}, la commande \refCom{letgutcls} a un effet indésirable
  (préfixe pas en grandes capitales).
\end{dbwarning}

\begin{docCommands}[
    doc new=2022-10-03,
    doc name=lettrenumber,
    doc parameter = \oarg{entier relatif signé}
  ]
  {
    { },
    { doc name=lettrenumber* },
  }
  \index{configuration}%
  Cette commande affiche le numéro de la \lettre{} :
  \begin{description}
  \item[en cours] si l'argument optionnel n'est pas employé ;
  \item[décalé de celui en cours] de l'\meta{entier relatif
      signé}\footnote{\label{entier-signe}C.-à-d. un \enquote{plus} ou un
      \enquote{moins} (\lstinline|+| ou \lstinline|-|) suivi d'un nombre
      entier.} spécifié sinon.
  \end{description}
  En version étoilée, la chaîne \enquote{\no{}} précède ce numéro.
\end{docCommands}

\ExplSyntaxOn
\int_gset:Nn \g__letgut_number_int {46}
\ExplSyntaxOff

\begin{ltx-code-result}[title addon=emploi de la commande \refCom{lettrenumber},listing options app={deletekeywords={[6]{cours,lettre}}}]
Si le numéro de la \lettre{} en cours est 46,
celui de la \lettre :
\begin{enumerate}
\item en cours est \lettrenumber ;
\item en cours est le \lettrenumber* ;
\item suivante est \lettrenumber[+1] ;
\item précédente est le \lettrenumber*[-1].
\end{enumerate}
\end{ltx-code-result}

\begin{docCommands}[]
  {
    { doc name = lettre },
    { doc name = lettre,  doc new=2022-10-03, doc parameter = \oarg{argument optionnel} },
    { doc name = lettre*, doc new=2022-10-03, doc parameter = \oarg{argument optionnel} },
    { doc name = lettregut },
    { doc name = lettregut,  doc new=2022-10-03, doc parameter = \oarg{argument optionnel} },
    { doc name = lettregut*, doc new=2022-10-03, doc parameter = \oarg{argument optionnel} },
  }
  %
  Ces commandes affichent les chaînes de caractères respectivement
  \enquote{\lettre} et \enquote{\lettregut}, le cas échéant suivies du numéro
  de la \lettre{} :
  \begin{description}
  \item[en cours] si l'\meta{argument optionnel} est un point (™.™) ;
  \item[décalé de celui en cours] de ce qui est spécifié si l'\meta{argument optionnel} est un
    entier relatif signé\footref{entier-signe} ;
  \item[spécifié] si l'\meta{argument optionnel} est autre.
  \end{description}
  En version étoilée, la chaîne \enquote{\no{}} précède le numéro (seulement si
  l'\meta{argument optionnel} est employé).
\end{docCommands}

\begin{ltx-code-result}[title addon=emplois des commandes \refCom{lettre} et \refCom{lettregut},listing options app={deletekeywords={[6]{cours,lettre}}}]
Si le numéro de la \lettre{} en cours est 46, on a :
\begin{enumerate}
\item \lettre
\item \lettre[.]
\item \lettre[+10]
\item \lettre[-10]
\item \lettre[43]
\item \lettre[coucou]
\item \lettre*[.]
\item \lettre*[+10]
\item \lettre*[-10]
\item \lettre*[43]
\item \lettre*
\end{enumerate}
On fait usage de ces commandes dans la \lettregut*[.].
\end{ltx-code-result}

\subsection{Touches de clavier}
\label{sec:touches-de-clavier}

Afin de disposer d'un moyen simple, riche et élégant pour composer des touches
de clavier, la \letgutcls{} s'appuie sur le \package*{biolinum} et notamment sa
commande ™\LKey™. Cette dernière a été légèrement étendue de façon à faciliter
la saisie pour toutes les touches des diacritiques utilisés en français.

\begin{ltx-code-result}[title addon=touches de clavier]
% De base (échantillon) :
\LKey{A} \LKey{Z} \LKey{0} \LKey{9}

\LKeyF{1} \LKeyF{12}

\LKeyCtrl \LKeyAlt \LKeyAltGr \LKeyShift \LKeyEnter \LKeyTab

\LKeyCtrlX{A} \LKeyShiftX{A} \LKeyAltX{A} \LKeyAltGrX{A}

\LKeyAt \LKeyScreenUp \LKeyScreenDown \LKeyCommand \LKeyOptionKey

\LMouseN \LMouseL \LMouseM \LMouseR

\LKey{exclam} \LKey{numbersign} \LKey{percent} \LKey{backslash}

% Ajoutés par la classe `letgut`
\LKey{à} \LKey{À} \LKey{â} \LKey{Â} \LKey{é} \LKey{É}
\LKey{è} \LKey{È} \LKey{ê} \LKey{Ê} \LKey{ë} \LKey{Ë}
\LKey{î} \LKey{Î} \LKey{ï} \LKey{Ï}
\LKey{ô} \LKey{Ô}
\LKey{ù} \LKey{Ù} \LKey{û} \LKey{Û} \LKey{ü} \LKey{Ü}
\LKey{ÿ} \LKey{Ÿ} \LKey{ç} \LKey{Ç}
\end{ltx-code-result}

\begin{dbwarning}{Touche de clavier du symbole € manquant}{}
  La touche de clavier du symbole € n'est pas fournie par le
  \package*{biolinum}.
\end{dbwarning}

\section{Codes informatiques}
\label{sec:code-informatique}

Cette section est consacrée aux outils spécifiques à la \letgutcls{} permettant
de faire figurer du code informatique dans la \lettre{}.

\subsection{Codes \hologo{(La)TeX}}
\label{sec:exemples-de-codes}

\subsubsection{Exemples de codes \hologo{(La)TeX}, possiblement avec résultats}
\label{sec:listings-}

Afin de présenter aisément et de façon homogène les exemples de codes
\hologo{(La)TeX}, possiblement avec leurs résultats, la \letgutcls{} fournit les
environnements \enquote{verbatim} suivants.

\begin{docEnvironments}[
  doc parameter = \oarg{options},
  doclang/environment content = code,
  ]{
    {
      doc name = ltx-code,
      % doc description = code (seulement),
    },
    {
      doc name = ltx-code-result,
      % doc description = % code
      % $+$ résultat (interne),
    },
    {
      doc name = ltx-code-external-result,
      doc parameter = \oarg{options}\marg{fichier},
      % doc description = % code
      % $+$ résultat (externe),
    },
  }
  Ces environnements affichent le \meta{code} \hologo{(La)TeX} qui y est inséré et
  pour :
  \begin{description}
  \item[\refEnv{ltx-code}] seulement ce \meta{code} ;
  \item[\refEnv{ltx-code-result}] également le résultat, compilé en même temps
    que la \lettre{} ;
  \item[\refEnv{ltx-code-external-result}] également le résultat, compilé
    indépendamment de la \lettre{} et dont le \meta{fichier} image est spécifié.
  \end{description}
\end{docEnvironments}

\begin{dbremark}{Mise en page des exemples de codes}{}
  \begin{enumerate}
  \item Les exemples de codes (avec ou sans résultats) sont par défaut
    automatiquement coupés en frontière de page.
  \item Les exemples de codes avec résultats (environnements
    \refEnv{ltx-code-result} et \refEnv{ltx-code-external-result}), présentent
    ces codes et résultats :
    \begin{itemize}
    \item l'un sous l'autre par défaut ;
    \item l'un à gauche de l'autre si l'option ™sidebyside™ est
      employée.
    \end{itemize}
  \end{enumerate}
\end{dbremark}

\begin{dbwarning}{Exemples de codes avec résultats : possiblement
    flottants}{exemples-flottants}
  Si l'option ™sidebyside™ est passée à l'un ou l'autre des
  environnements \refEnv{ltx-code-result} et \refEnv{ltx-code-external-result},
  l'exemple :
  \begin{itemize}
  \item présente ses code et résultat en regard ce qui rend impossible sa
    coupure en frontière de page ;
  \item est alors automatiquement flottant.
  \end{itemize}
  Dans le cas où cet exemple (\no\meta{n}) s'avère se trouver sur une page
  (\meta{q}) autre que celle (\meta{p}) de son point d'insertion, deux
  références croisées sont automatiquement insérées :
  \begin{description}
  \item[une \enquote{avant} :] au point d'insertion de l'exemple pour indiquer
    qu'il est à consulter plus loin ; son texte, \emph{par défaut}
    \enquote{Cf. exemple \meta{n} page \meta{q}.}, peut être surchargé au moyen
    de l'option \refKey{reference text} ;
  \item[une \enquote{arrière} :] à la fin du titre de l'exemple ; son texte est
    \enquote{ (cf. page \meta{p})}.
  \end{description}
\end{dbwarning}

Ces trois environnements admettent des \meta{options} :
\begin{itemize}
\item (toutes) celles acceptées par l'environnement
  \docAuxEnvironment{tcblisting} et la commande \docAuxCommand{newtcblisting}
  fournis par la bibliothèque ™listings™ du
  \package*{tcolorbox}\footnote{Et, aussi, les commandes
    \docAuxCommand{DeclareTCBListing} et assimilées fournies par la bibliothèque
    \lstinline+xparse+ de ce package.}. Elles permettent notamment de surcharger
  les réglages par défaut, par exemple :
  \begin{itemize}
  \item de faire figurer l'éventuel résultat, non pas sous le code comme c'est
    le cas par défaut, mais en regard (à droite) au moyen de l'option
    ™sidebyside™ ;
  \item de supprimer les numéros de ligne au moyen de l'option
\begin{ltx-code}
listing options={numbers=none}
\end{ltx-code}
  \end{itemize}
\item trois spécifiques à ces environnements :
  \begin{docKey}{title addon}{=\meta{supplément}}{\valinitdef}
    Cette option permet d'adjoindre au titre de ces exemples, qui sont par
    défaut et automatiquement \enquote{Exemple \meta{n}}, un \meta{supplément}.
  \end{docKey}
  \begin{docKey}{result width}{=\meta{longueur}}{\valinitdef[\docAuxCommand*{linewidth}]}
    Cette option, utile seulement pour l'environnement
    \refEnv{ltx-code-external-result}, permet de spécifier une largeur autre que
    celle initiale pour le fichier image du résultat, compilé indépendamment de
    la \lettre{}.
  \end{docKey}
  \begin{docKey}{reference text}{=\meta{texte}}{\valinitdef[Cf. exemple \meta{n} page \meta{q}.]}
    Cette option n'a d'effet que :
    \begin{itemize}
    \item avec l'un ou l'autre des environnements \refEnv{ltx-code-result} et
      \refEnv{ltx-code-external-result} ;
    \item lorsque l'option ™sidebyside™ leur est passée ;
    \item lorsque l'exemple s'avère se trouver sur une page autre que
      celle de son point d'insertion.
    \end{itemize}
    Elle permet alors de surcharger le texte \enquote{Cf. exemple \meta{n} page
      \meta{q}.} automatiquement inséré au point d'insertion de l'exemple
    (cf.~\vref{wa-exemples-flottants})\footnote{Il est par exemple possible de
      s'affranchir de ce texte en recourant à \lstinline+reference text=\{\}+.}.
    \begin{dbwarning}{\refKey{reference text} avant \lstinline+sidebyside+}{}
      Pour qu'elle soit prise en compte, l'option \refKey{reference text} doit
      être passée \emph{avant} l'option ™sidebyside™.
    \end{dbwarning}
  \end{docKey}
\end{itemize}

\subsubsection{Coloration syntaxique}
\label{sec:coloration}

Par défaut, en début de document et de chaque fichier importé au moyen de
\refCom{inputarticle}, le langage supposé dans ces exemples de codes est
\hologo{TeX}, chargé (seulement) avec ses \enquote{dialectes} :
\begin{itemize}
\item ™primitive™, ™common™, ™plain™, ™LaTeX™, ™AlLaTeX™ fournis par le
  \package*{listings} ;
\item ™classes™ fourni par \letgutcls{} (répertoriant les classes disponibles
  sur le \ctan{}).
\end{itemize}
Une conséquence notable est la suivante :
\begin{dbwarning}{Coloration syntaxique réduite par défaut}{}
  La coloration syntaxique dans les exemples de codes n'est par défaut active
  que pour le langage \hologo{TeX} et ses dialectes ™primitive™, ™common™,
  ™plain™, ™LaTeX™, ™AlLaTeX™ et ™classes™.
\end{dbwarning}

Il est néanmoins possible de spécifier d'autres langages et dialectes au moyen
de la commande \refCom{syntaxhl} suivante, à insérer avant le début de l'exemple
de code concerné.

\begin{docCommands}[
doc name = syntaxhl,
doc parameter = \marg{liste de dialectes},
]
{
  { },
  { doc parameter = \oarg{langage}\marg{liste de dialectes} },
}
Cette comande permet de spécifier :
\begin{itemize}
\item un \meta{langage} (par défaut \hologo{TeX}) ;
\item une \meta{liste de dialectes}, séparés par des virgules ;
\end{itemize}
auxquels on souhaite que s'applique la coloration syntaxique.
\end{docCommands}

\begin{dbwarning}{Dialectes colorés syntaxiquement seulement si définis}{}
  Ceci suppose que ces langages et dialectes sont définis (et saisis selon la
  syntaxe du \package*{listings}) dans le \file*{letgut-lstlang.sty} situé :
  \begin{itemize}
  \item soit dans le dossier de la \lettre{} en cours ;
  \item soit dans le dossier parent de celui de la \lettre{} en cours ;
  \item soit  dans un dossier de la \tds{}
  \end{itemize}
\end{dbwarning}

Pour le langage \hologo{TeX}, ces dialectes sont essentiellement les classes et
les packages \hologo{(La)TeX} et un exemple de déclaration de tel dialecte est
fourni section~\enquote{\nameref{sec:exemple-de-decl}},
\vpageref{sec:exemple-de-decl}.

\subsection{Entrées et sorties dans un terminal}
\label{sec:entrees-sorties}

Afin de présenter aisément et de façon homogène des exemples de commandes
entrées et éventuellement de leurs sorties correspondantes, la \letgutcls{}
fournit la commande à arguments \enquote{verbatim} suivante.

\begin{docCommand}{terminal}{\oarg{prompt}\oarg{options}\marg{stdin}\marg{stdout}}
  %
  Cette commande affiche les codes en entrée (\meta{stdin}) et en sortie
  (\meta{stdout}), chacun des deux étant possiblement vide.

  Le \meta{prompt}, ou \enquote{invite de commande}, est par défaut le symbole
  \texttt{\$} affiché en rouge.

  Il est possible de surcharger les réglages par défaut de cette commande au
  moyen d'\meta{options} qui sont (toutes) celles acceptées par l'environnement
  \docAuxEnvironment{tcblisting} et la commande \docAuxCommand{newtcblisting}
  fournis par la bibliothèque ™listings™ du \package*{tcolorbox}\footnote{Et,
    aussi, les commandes \docAuxCommand{DeclareTCBListing} et assimilées
    fournies par la bibliothèque \lstinline+xparse+ de ce package.}.
\end{docCommand}

Ainsi le code suivant :
\begingroup
% \lstset{basicstyle=\ttfamily\scriptsize}
\begin{ltx-code}
\terminal{lualatex}{
This is LuaHBTeX, Version 1.18.0 (TeX Live 2024)
 restricted system commands enabled.
**
}
\end{ltx-code}

donne-t-il :

\terminal{lualatex}{
This is LuaHBTeX, Version 1.18.0 (TeX Live 2024)
 restricted system commands enabled.
**
}
\endgroup

\subsection{Caractères d'échappement et de raccourci pour les
  extraits de code}
\label{sec:caract-dech-et}

\begin{dbwarning}{Caractère d'échappement des listings}{}
  \lstset{escapechar=}%
  La \letgutcls{} définit ™£™ comme caractère d'échappement dans \LaTeX{} au
  sein d'un listing.
\end{dbwarning}

Au besoin, on pourra désactiver ce caractère actif au moyen de
\begin{ltx-code}
\lstset{escapechar=}
\end{ltx-code}

\begin{dbwarning}{Équivalent court de \docAuxCommand{lstinline}}{}
  Les extraits de code peuvent être saisis au moyen de la commande
  \docAuxCommand{lstinline} du \package*{listings} mais, pour
  simplifier la tâche, la \letgutcls{} définit comme équivalent court
  de \docAuxCommand{lstinline} le caractère unicode %
  \lstDeleteShortInline™%
  \texttt{™} %
  \lstMakeShortInline™%
  (\Ucode{2122})\footnote{Peu susceptible d'être utilisé dans du texte
    ordinaire.}.
\end{dbwarning}

Autrement dit, la \letgutcls{} contient l'instruction\footnote{À peu
  de choses près.} :
  \lstDeleteShortInline™
\begin{ltx-code}
\lstMakeShortInline£\texttt{™}£
\end{ltx-code}

  Au besoin, on pourra désactiver ce caractère actif au moyen de :
\begin{ltx-code}
\lstDeleteShortInline£\texttt{™}£
\end{ltx-code}

\begin{dbremark}{Obtention du caractère \texttt{™}}{}
Le caractère %
\texttt{™} %
s'obtient :
\begin{itemize}
\item sous \linux{} : \LKeyShiftAltGrX{8}\footnote{Touche \LKey{8} du clavier
    principal.} ;
\item sous \macos{} : \LKeyShift+\LKeyOptionKey+\LKey{T} ;
\item sous \windows{} : \LKeyAltX{0}+\LKey{1}+\LKey{5}+\LKey{3}.
\end{itemize}
\end{dbremark}
  \lstMakeShortInline™%

\section{Nouveautés apparues sur le \ctan}
\label{sec:rubr-cons-aux}

Afin de pouvoir plus aisément lister les nouveautés (packages et classes
\hologo{(La)TeX}, etc.) apparues sur le \ctan{}, la \letgutcls{} fournit le
nouvel environnement de liste \refEnv{ctannews}, similaire à l'environnement
\docAuxEnvironment{description}.
%
\begin{docEnvironment}[doclang/environment content=liste des nouveautés]{ctannews}{}
  Cet environnement permet de dresser la \meta{liste des nouveautés} apparues
  sur le \ctan{}.
\end{docEnvironment}
%
Chaque \meta{nouveauté} est introduite au moyen de la commande \refCom{item}
suivante.
\begin{docCommands}[
  doc parameter = \oarg{nom},
  ]
  {
    { doc name = item },
    { doc name = item* },
  }
  %
  Cette commande affiche le \meta{nom} de la \meta{nouveauté} comme ce serait le
  cas pour le \enquote{label} d'une liste de description, ce qui permet ensuite
  de décrire la \meta{nouveauté} en question. Le \meta{nom} est en outre un lien
  hypertexte vers sa page sur le \ctan{}
  (\url{https://ctan.org/pkg/}\meta{nom}).

  La version étoilée \refCom{item*} est dédiée aux nouveautés œuvres de
  contributeurs francophones et le logo de la francophonie, alors
  automatiquement situé en regard dans la marge, les signale comme telles.
\end{docCommands}

\begin{ltx-code-external-result}[title addon=nouveautés,listing options app={deletekeywords={[6]{hologo,matapli}}}]{exemple-nouveautes}
\begin{ctannews}
\item[nl-interval] vise à simplifier le processus de
  représentation graphique des intervalles de l'axe réel.
\item*[matapli] classe \hologo{LaTeX} destinée à la composition
  de la revue Matapli (conçue par \person{Chupin, Maxime},
  secrétaire adjoint de \gutenberg{}).
\end{ctannews}
\end{ltx-code-external-result}

\begin{docCommand}{francophony}{}
  Cette commande affiche le logo de la francophonie, ainsi : \francophony.
\end{docCommand}

\section{Fiches de lecture}
\label{sec:fiches-de-lecture}

Les fiches de lecture d'un livre sont créées au moyen de l'environnement
\refEnv{bookreview} suivant.

\begin{docEnvironment}[doclang/environment content=fiche de lecture]{bookreview}{\marg{caractéristiques}}
  Cet environnement permet de mettre en page une \meta{fiche de lecture}
  caractérisée par les \meta{caractéristiques} suivantes qui sont, selon les
  cas :
  \begin{description}
  \item[obligatoires :]
    \begin{docKey}{title}{=\meta{titre}}{\valinitdef}
      Cette clé permet de spécifier le \meta{titre} introductif de la fiche.
    \end{docKey}
    \begin{docKey}{reviewer}{=\meta{rapporteur}}{\valinitdef}
      Cette clé permet de spécifier le \meta{rapporteur} de la fiche, à spécifier
      selon la syntaxe de la commande \refCom{author}.
    \end{docKey}
    \begin{docKey}{bibkey}{=\meta{clé}}{\valinitdef}
      Cette clé permet de spécifier la \meta{clé} identifiant l'entrée d'un
      \file*{.bib} contenant les données bibliographiques du document rapporté.
      \begin{dbwarning}{Fichier de bibliographie}{}
        Ces données bibliographiques doivent se trouver dans un fichier
        \meta{bibliographie}\file{.bib}, structurées selon le format du
        \package*{biblatex} et chargées en préambule (par exemple dans le
        fichier local de configuration, cf. \vpageref{sec:fichier-local-de}) au
        moyen de la commande :
\begin{ltx-code}
\addbibresource{£\meta{bibliographie}£.bib}
\end{ltx-code}
      \end{dbwarning}
    \end{docKey}
  \item[fortement conseillée :]
    \begin{docKey}{frontcover}{=\meta{fichier}}{\valinitdef}
      Cette clé permet de spécifier le \meta{fichier} image de la couverture du
      document rapporté.
    \end{docKey}
  \item[facultative :]
    \begin{docKey}{price}{=\meta{prix}}{\valinitdef}
      Cette clé permet le cas échéant de spécifier le \meta{prix} du document rapporté.
    \end{docKey}
  \end{description}
\end{docEnvironment}

\section{Acronymes}
\label{sec:acronymes}

Nombreux sont les articles de la \lettre{} susceptibles de contenir des
acronymes peut-être pas connus de tous. Aussi est-il opportun que, lors de leur
première occurrence, ceux-ci soient explicités.

Pour automatiser cela, la \letgutcls{} s'appuie sur le \package*{acro} ;
toutefois, pour à la fois simplifier la création desdits acronymes et étendre
(légèrement) les fonctionnalités offertes par \package{acro}, elle fournit la
commande dédiée \refCom{letgutacro}.

\begin{docCommand}{letgutacro}{\oarg{options}\marg{COURT}\marg{long}\oarg{traduction
      française}}
  \index{acronyme}%
  Cette commande permet de créer un nouvel acronyme en spécifiant :
  \begin{itemize}
  \item sa forme courte \meta{COURT}, \emph{obligatoirement en grandes
      capitales} ;
  \item sa forme longue \meta{long}.
  \end{itemize}
  En outre, le 1\ier{} et 2\ieme{} arguments \emph{optionnels} permettent de,
  respectivement :
  \begin{itemize}
  \item passer à la commande \docAuxCommand{DeclareAcronym} (de création
    d'acronymes du \package*{acro} agissant en sous-main) des \meta{options} qui
    lui sont propres, permettant ainsi de surcharger les options par défaut
    passées à cette commande par \refCom{letgutacro} ;
  \item signaler que l'acronyme provient de l'anglais et d'en spécifier la
    \meta{traduction française} (éventuellement vide si celle-ci n'est pas
    pertinente).
  \end{itemize}

  L'acronyme ainsi créé a pour identifiant \meta{court}, c'est-à-dire
  la version \emph{en bas de casse} de \meta{COURT}, et peut donc
  être employé au moyen des commandes fournies par le \package*{acro}, par
  exemple :
  \begin{itemize}
  \item ™\ac{™\meta{court}™}™\footnote{Acronyme automatiquement affiché sous sa
      forme complète à sa 1\iere{} occurrence, sous sa forme courte à ses
      occurrences suivantes.} ;
  \item ™\acs{™\meta{court}™}™\footnote{Acronyme affiché sous sa forme courte
      seulement.}.
  \end{itemize}
  Toutefois, pour simplifier l'usage de ces acronymes, la \letgutcls{} crée
  alors automatiquement une commande \docAuxCommand{\meta{court}}%
  \footnote{%
    Sauf si elle existe déjà, auquel cas la création d'une telle commande est
    silencieusement escamotée. Ainsi par exemple, l'acronyme
    %
    \lstinline+\\letgutacro[...]\{TIKZ\}\{...\}[...]+
    %
    fourni par \letgut{} (cf. \vpageref{liste-acronymes}) ne surcharge-t-il
    pas la commande \docAuxCommand{tikz} fournie notamment par le
    \package*{tikz}.%
  }
  %
  qui agit comme :
  \begin{itemize}
  \item ™\ac{™\meta{court}™}™ en version non étoilée ;
  \item ™\acs{™\meta{court}™}™ en version étoilée.
  \end{itemize}
\end{docCommand}
Ainsi, l'acronyme utilisé via \docAuxCommand{\meta{court}} figure, pour ses
occurrences :
\begin{description}
\item[première :] sous la forme \meta{\textsc{court}}\footnote{C'est-à-dire
    \meta{court} en petites capitales.} suivi d'une note de bas de
  page contenant \enquote{\meta{long}.} ;
\item[suivantes :] sous la forme \meta{\textsc{court}}.
\end{description}

En outre :
\begin{itemize}
\item cette commande peut être utilisée sans restriction en argument des
  commandes \refCom{title}, \refCom{subtitle}, \refCom{section},
  \refCom{subsection}, etc. et l'acronyme figure sous sa forme \meta{COURT} dans
  les \foreignloc{bookmarks} (signets) ;
\item un copié de \meta{\textsc{court}} colle \meta{COURT}.
\end{itemize}

\begin{dbwarning}{\refCom{letgutacro} : uniquement en préambule}{}
  La définition d'acronymes au moyen de \refCom{letgutacro} ne peut se faire
  qu'en préambule.
\end{dbwarning}

Ainsi, avec les définitions suivantes en préambule :

\begin{ltx-code}[title addon=définition d'acronymes,listing options app={deletekeywords={[3]{TeX,LaTeX,emph,n,no}}}]
% Acronyme français
\letgutacro{LD}{\LaTeX{} dépendant}

% Acronyme anglais avec traduction française
\letgutacro{TLMGR}{\TeX~Live Manager}[gestionnaire \TeX~Live]

% Acronyme anglais sans traduction française
\letgutacro{CD}{Compact Disc}[]

% Acronyme anglais avec surcharge :
% - `short=no-op` : l'acronyme est « no-op » (en bas de casse)
%   et le nom de la commande sous-jacente ne peut être \no-op
%   (tiret interdit)
% - `short-format=\upshape` : les petites capitales sont
%   désactivées
\letgutacro[short=no-op,short-format=\upshape]
{NOOP}
{No Operation}
[instruction nulle]

% Acronyme en allemand :
\letgutacro[
  short=\emph{Ti\emph{k}Z},
  short-format=\em,
  foreign-babel=german,
  foreign-locale=allemand]
{TIKZ}
{Ti\emph{k}Z ist \emph{kein} Zeichenprogramm}
[Ti\emph{k}Z \emph{n'}est \emph{pas} un programme de dessin]
\end{ltx-code}

a-t-on :

\begin{ltx-code-result}[title addon=utilisation d'acronymes,listing options app={deletekeywords={cd},deletekeywords={[3]{cd,tikz,LaTeX,on}},deletekeywords={[6]{cd}},deletekeywords={itemize},morekeywords={[2]{itemize}}}]
On dispose désormais pour \enquote{\LaTeX{} dépendant}
d'un acronyme qu'on peut utiliser par exemple
\begin{itemize}
\item ainsi : \ac{ld} ou \acs{ld} ;
\item ou bien ainsi : \ld{} ou \ld*{}.
\end{itemize}

On peut également employer les acronymes :
\begin{itemize}
\item \cd{}, \cd{} ;
\item \tlmgr{}, \tlmgr{} ;
\item \noop{}, \noop{} ;
\item \ac{tikz}, \ac{tikz}. % Noter le non emploi de `\tikz'
\end{itemize}
\end{ltx-code-result}

Les noms des commandes sous-jacentes ne doivent contenir que des lettres, mais
celles-ci peuvent être accentuées. Ainsi, avec la définition suivante en
préambule :

\begin{ltx-code}[title addon=définition d'acronyme avec lettres
  accentuées,drop lifted shadow]
\letgutacro{CPGÉ}{Classes Préparatoires aux Grandes Écoles}
\end{ltx-code}

a-t-on :

\begin{ltx-code-result}[title addon=utilisation d'acronyme avec lettres accentuées,listing options app={deletekeywords={[3]{l}}}]
On peut également employer l'acronyme \cpgé{}, \cpgé{}.
\end{ltx-code-result}

La \letgutcls{} fournit un \file*{letgut-acronyms.tex} dans lequel sont définis
plusieurs acronymes anglais et français, directement utilisables. Ceux-ci sont
répertoriés \vpageref{liste-acronymes}.

\section{Séparateurs}
\label{sec:filets}

Il est parfois utile d'accentuer la séparation entre les articles de la
\lettre. Ceci peut se faire au moyen de la commande \refCom{separator} qui
insère un filet horizontal.

\begin{docCommand}{separator}{}
  \index{séparateur}%
  Cette commande permet d'accentuer la séparation entre deux articles.
\end{docCommand}

\section{Annonces}
\label{sec:annonces}

Afin de présenter aisément et de façon homogène les annonces à paraître dans la
\lettre{}, la \letgutcls{} fournit l'environnement \refEnv{announcement}.

\begin{docEnvironment}[doclang/environment content=annonce,doc new and updated={2023-01-14}{2023-05-21}]{announcement}{\oarg{options}\marg{titre}}
  % \tcbdocmarginnote{\color{red}\faExclamationTriangle{} Compatibilité ascendante cassée le
  %   2023-05-21}
  Cet environnement est dédié aux \meta{annonce}s. La spécification d'un
  \meta{titre} (pouvant toutefois être vide) est obligatoire. S'il n'est pas
  vide, il figure alors par défaut dans la table des matières et dans les
  signets au même niveau que celui des titres d'articles.

  Cet environnement admet des \meta{options} :
  \begin{itemize}
  \item (toutes) celles acceptées par les environnements \refAux{tcolorbox} et
    assimilés du \package*{tcolorbox}, destinées à, le cas échéant, modifier la
    mise en forme par défaut de l'annonce ;
  \item deux qui lui sont propres :
    \begin{docKey}[][doc new and updated={2023-05-21}{2023-11-08}]{toc title}{=\meta{titre alternatif}}{\valinitdef}
      Cette clé permet de remplacer dans la table des matières et dans les
      signets le \meta{titre} par un \meta{titre alternatif}. Si cette option
      est utilisée avec un \meta{titre alternatif} vide, l'annonce ne figure ni
      dans la table des matières, ni dans les signets.
    \end{docKey}
    \begin{docKey}[][doc new={2023-05-21}]{color}{=\meta{couleur}}{\valinitdef[\docValue*{black}]}
      Cette clé permet une \meta{couleur} d'ornement et de titre autre que celle
      appliquée par défaut.
    \end{docKey}
  \end{itemize}
\end{docEnvironment}

\begin{ltx-code-result}[title addon=annonces]
\begin{announcement}[toc title={Exemple d'annonce},color=red!35!black]{Convocation \acs{ag} ordinaire}
  Les adhérents de l'\assogut{} sont invités à participer à
  l'\textbf{assemblée générale \emph{ordinaire}} de l'association
  le \textbf{dimanche 11 décembre 2022}.
\end{announcement}
\end{ltx-code-result}

\section{Rébus}
\label{sec:rebus}

\tcbstartrecording\relax
Afin de présenter aisément et de façon homogène les rébus à paraître dans la
\lettre{}, la \letgutcls{} fournit l'environnement \refEnv{rebus}.

\begin{docEnvironment}[doclang/environment content=rébus,doc new={2023-01-14}]{rebus}{\oarg{options}}
  Cet environnement affiche un \meta{rébus} et s'emploie différemment selon que
  la solution est prévue de figurer dans la \lettre{} :
  \begin{description}
  \item[en cours :] le rébus et sa solution doivent alors être séparées par
    la commande \refCom{solution} ;
  \item[suivante :] l'option \refKey{no solution} doit alors être employée.
  \end{description}
  La boîte contenant le rébus mentionne que sa solution se trouve,
  respectivement :
  \begin{itemize}
  \item \enquote{page \meta{n}} ou \enquote{ci-dessous} ;
  \item \enquote{dans la prochaine \lettre{}}.
  \end{itemize}

  La commande séparant le rébus et sa solution prévue de figurer dans la
  \lettre{} en cours est :
  \begin{docCommand}[doc new={2023-01-14}]{solution}{}
    Cette commande débute la solution d'un rébus à l'intérieur d'un environment
    \refEnv{rebus}.
  \end{docCommand}

  En outre, cet environnement admet des \meta{options} :
  \begin{itemize}
  \item (toutes) celles acceptées par les environnements \refAux{tcolorbox} et
    assimilés du \package*{tcolorbox}, destinées à, le cas échéant, modifier la
    mise en forme par défaut de la boîte contenant le rébus ;
  \item deux qui lui sont propres :
    \begin{docKey}[][doc new={2023-01-14}]{title addon}{=\meta{supplément}}{\valinitdef}
      Cette option permet d'adjoindre au titre d'un rébus, qui est par défaut et
      automatiquement \enquote{Rébus}, un \meta{supplément} alors en italique et
      entre parenthèses.
    \end{docKey}
    \begin{docKey}[][doc new={2023-01-14}]{no solution}{}{\valinitdef[\docValue*{false}][\docValue*{true}]}
      Cette option est à spécifier pour un rébus dont la solution ne doit figurer
      que dans la prochaine \lettre{}.
    \end{docKey}
  \end{itemize}
\end{docEnvironment}

Ainsi le code :

\begin{ltx-code}[title addon=rébus]
\begin{rebus}[no solution]
  Rébus \emph{sans} solution dans le présent numéro.
\end{rebus}
\begin{rebus}
  Rébus \emph{avec} solution dans le présent numéro.
  \solution
  Solution du rébus.
\end{rebus}
\end{ltx-code}

donne-t-il les résultats suivants\footnote{Éventuellement pas immédiatement à la
  suite car les boîtes contenant les rébus sont flottantes.}.

\begin{rebus}[no solution,nofloat]
  Rébus \emph{sans} solution dans le présent numéro.
\end{rebus}
\begin{rebus}[nofloat]
  Rébus \emph{avec} solution dans le présent numéro.
  \solution
  Solution du rébus.
\end{rebus}

La \letgutcls{} fournit bien sûr le moyen d'afficher les solutions des rébus de
la \lettre{}
soit en cours, soit précédente.

\begin{docCommand}[doc new={2023-01-14}]{rebussolution}{\oarg{solution}\oarg{numéro}\oarg{options}}
  Cette commande affiche la ou les solutions du ou des rébus qui figurent dans
  la \lettre{} :
  \begin{description}
  \item[en cours :] (commande \refCom{solution}) si ses 1\ier{} et
    2\ieme{} arguments optionnels \meta{solution} et \meta{numéro} sont
    inutilisés ;
  \item[précédente :] si son 1\ier{} argument optionnel \meta{solution} est
    utilisé. Selon que 2\ieme{} argument optionnel \meta{numéro} est spécifié ou
    pas, il est indiqué que la \meta{solution} est celle d'un rébus figurant
    dans la \lettre{} :
    \begin{itemize}
    \item \meta{numéro} ;
    \item précédente.
    \end{itemize}
  \end{description}
  Cette commande admet comme \meta{options} (toutes) celles acceptées par les
  environnements \refAux{tcolorbox} et assimilés du \package*{tcolorbox},
  destinées à, le cas échéant, modifier la mise en forme par défaut de la boîte
  contenant la solution du rébus.
\end{docCommand}

Ainsi le code :

\begin{ltx-code}[title addon=solution de rébus]
\rebussolution
\rebussolution[Solution du rébus]
\rebussolution[Solution du rébus][49]
\end{ltx-code}

donne-t-il donne-t-il les résultats suivants\footnote{Éventuellement pas
  immédiatement à la suite car les boîtes contenant les solutions des rébus sont
  flottantes.}.

\rebussolution
\rebussolution[Solution du rébus]
\rebussolution[Solution du rébus][49]

\section{Boîtes d'alertes}
\label{sec:boites-dalertes}

\begin{docCommand}{alertbox}{ \oarg{couleur}\marg{texte} }
  \index{alerte}%
  Cette commande insère une boîte d'alerte :
  \begin{itemize}
  \item optionnellement de \meta{couleur} de fond (à spécifier selon le modèle
    \enquote{\foreignloc{named}}) autre que celle par défaut\footnote{C'est-à-dire
      \colorbox{letgut_default_alert_box_color}{celle-ci}% , nommée
      % \docColor{letgut_default_alert_box_color} et égale
      % à \docValue*{\letgutalertboxdefaultcolor}
      .} ;
  \item contenant le \meta{texte} (qui peut contenir plusieurs paragraphes).
  \end{itemize}
\end{docCommand}

\begin{ltx-code-result}[title addon=boîtes d'alertes]
\alertbox{%
  Adhérez, adhérez, il en restera toujours quelque chose !%
}
\alertbox[yellow]{%
  Adhérez, adhérez !

  Il en restera toujours quelque chose...%
}
\end{ltx-code-result}

\section{Fichier local de configuration}
\label{sec:fichier-local-de}

Chaque numéro de la \lettre{} nécessite certaines configurations locales :
configuration dédiée au numéro en question, packages particuliers utilisés dans
les articles, configurations propres du \package*{listings}, etc. Afin de ne pas
encombrer le \file*{.tex} principal de la \lettre{}, un fichier de configuration
locale nommé
%
\ExplSyntaxOn
\file{
  \tl_use:N \c__letgut_local_config_file_tl.tex
}~
\ExplSyntaxOff
%
est, si présent dans le répertoire courant, automatiquement inclus à la fin
du préambule.

\title{Aspects de la 1\iere{} page}
\label{sec:mise-en-page}

La première page de la \lettre{} comporte :
\begin{itemize}
\item une bannière sous forme d'un très grand \enquote{L} en noir sur lequel
  figurent de la couleur de fond de la page, en gras, dans sa partie :
  \begin{itemize}
  \item verticale, \enquote{La} puis, chacune sur une ligne, les lettres du mot
    \enquote{Lettre} en grandes capitales ;
  \item horizontale, \enquote{\gut}.
  \end{itemize}
\item un très grand \enquote{g}, en filigrane et en gris clair.
\end{itemize}
La fonte de ces deux éléments est la principale utilisée (spécifiée au moyen de
\docAuxCommand{setmainfont}).

Pour ce faire, la classe charge le package maison \package{letgut-banner} qui
n'est pas décrit ici.

\title{Les dinosaures, leur écosystème et \letgut}
\label{recours-docker}

Pour à la fois :
\begin{itemize}
\item disposer d'une version suffisamment récente de \hologo{LuaLaTeX}
  pour la présente \letgutcls{} (cf. \vref{wa-lualatex-recent}) ;
\item éviter de perturber une installation existante de \hologo{(La)TeX} ;
\end{itemize}
on pourra recourir à \docker{}\footnote{Cerise sur le gâteau : un temps de
  compilation éventuellement réduit de façon significative. Ainsi, celui de la
  présente documentation est-il sur la machine de \person{Bitouzé, Denis} d'un
  peu plus de \SI{13}{\s} avec \docker{} et de plus de \SI{30}{\s} par le biais
  habituel.} dont cette section est un mode d'emploi :
\begin{itemize}
\item express ;
\item axé sur Linux, mais qui devant s'appliquer au moins en partie aux autres
  systèmes d'exploitation ;
\item axé sur la \tl{}.
\end{itemize}

\begin{dbwarning}{Commande \software{sudo} peut-être nécessaire}{}
  Les commandes \docker*{} ci-après ne sont pas précédées de \software{sudo}
  mais, selon les systèmes d'exploitation, elles peuvent devoir l'être.
\end{dbwarning}

On commence par installer \docker*{} puis à lancer le service
\docker*\footnote{Pour Ubuntu, cf. par exemple
  \href{https://doc.ubuntu-fr.org/docker}{ce guide}.}.

Ensuite, par exemple depuis un dossier où se trouve un \file*{mon-fichier.tex} (disons
à compiler avec \hologo{LuaLaTeX}), on lance la longue commande suivante
(\emph{qui doit être sur une seule ligne}\footnote{%
  Pour la copier d'un seul bloc, il devrait suffire de \emph{copier} (et non de
  \emph{cliquer sur}) l'icône suivante
% \marginpar{%
  \BeginAccSupp{method=plain,ActualText={%
      docker run -i --rm --name latex -v "$PWD":/usr/src/app -w /usr/src/app registry.gitlab.com/islandoftex/images/texlive:latest-with-cache lualatex mon-fichier
    }%
  }%
  \faCopy[regular]%$$
  \EndAccSupp{}%
% }%
}) :

\terminal{docker run -i --rm --name latex -v "$PWD":/usr/src/app -w /usr/src/app registry.gitlab.com/islandoftex/images/texlive:latest-with-cache lualatex mon-fichier}{}%$

La toute première fois, cela provoque le téléchargement de plusieurs fichiers,
dont celui assez lourd de l'image d'une version allégée de la \tl{}~2021
(délestée des sources et des documentations) puis lance la compilation
demandée.

Pour simplifier les compilations ultérieures, on aura intérêt à créer dans son
\file*{.bashrc} (ou \file{.zshrc}, etc.) un ou plusieurs alias de la forme :

\begin{tcblisting}{listing only,breakable,listing options={style=tcblatex,language=bash}}
alias docker-texlive='docker run -i --rm --name latex -v "$PWD":/usr/src/app -w /usr/src/app registry.gitlab.com/islandoftex/images/texlive:latest-with-cache'
alias docker-pdflatex='docker-texlive pdflatex'
alias docker-xelatex='docker-texlive xelatex'
alias docker-lualatex='docker-texlive lualatex'
alias docker-biber='docker-texlive biber'
alias docker-makeglossaries='docker-texlive makeglossaries'
alias docker-latexmk-pdf='docker-texlive latexmk -pdf'
alias docker-latexmk-xe='docker-texlive latexmk -pdfxe'
alias docker-latexmk-lua='docker-texlive latexmk -pdflua'
\end{tcblisting}

pour pouvoir compiler au moyen de seulement\footnote{En lançant préalablement
  \lstinline[language=bash]+source ~/.bashrc+ (ou assimilé) afin de pouvoir en
  bénéficier dans un terminal déjà ouvert.} :

\terminal{docker-latexmk-lua mon-fichier}{}

\title{Packages chargés par \letgut{}}
\label{sec:packages-charges-par}

La \letgutcls{} charge en sous-main un certain nombre de packages utiles, voire
nécessaires, à son codage. Elle en charge également certains pas indispensables,
mais considérés comme \enquote{incontournables} pour que les auteurs de la
\lettre{} puissent (aisément) composer un \enquote{joli} document. Nous en
dressons ci-après la liste en les regroupant selon ces deux catégories.

\section{Packages utiles aux auteurs de la \lettre{}}
\label{sec:utiles-aux-auteurs}

\begin{ctannews}
\item[fontspec] fontes \otf{}.
\item[microtype] raffinements subliminaux vers la perfection typographique.
  \begin{description}
  \item[Options :] \docAuxKey*{stretch=30}, \docAuxKey*{shrink=25},
    \docAuxKey*{letterspace=150}.
  \end{description}
\item[graphicx] prise en charge améliorée des graphiques.
\item[array] extension des environnements \docAuxEnvironment*{array} et
  \docAuxEnvironment*{tabular}.
\item[fancyvrb] notamment pour permettre l'usage de commandes \enquote{verbatim}
  dans les notes de bas de page.
\item[booktabs] tableaux de qualité.
\item[csquotes] facilités de citations, en ligne et hors-texte, sensibles au
  contexte.
\item[amsmath] nombreux outils utiles pour la composition mathématique.
\item[mathtools] étend les fonctionnalités et corrige certaines déficiences
  d'\package{amsmath} et de \hologo{LaTeX}.
  \begin{description}
  \item[Option :] \docAuxKey*{fleqn}.
  \end{description}
\item[siunitx] aide à la saisie et à l'affichage cohérent des nombres, unités et
  quantités.
  \begin{description}
  \item[Options :] \docAuxKey*{locale=FR}, \docAuxKey*{mode=text}.
  \end{description}
\item[hologo] collection de logos habituels (\hologo{LaTeX}, \hologo{LaTeX2e},
  etc.) avec support pour les signets.
\item[xcolor] accès facile, indépendant du pilote, à plusieurs types de teintes,
  de nuances, de tons et de mélanges de couleurs arbitraires.
  \begin{description}
  \item[Option :] \docAuxKey*{table}.
  \end{description}
\item[ninecolors] sélection de couleurs avec contraste \wcag{} approprié.
\item[tabularray] mise en page de tableaux et de matrices offrant une séparation
  complète des contenus et styles. Ce package très récent (sorti le 14 mai 2021)
  est utilisé dans le code de la classe pour la création de boîtes d'alertes
  \enquote{légères}, c.-à-d. ne nécessitant notamment pas le chargement
  (indirect) du package \ac{tikz} qui augmente significativement le temps de
  compilation ; mais il pourrait (devrait) être utile également aux auteurs de
  la \lettre{}.
\item[babel] support multilingue.
  \begin{description}
  \item[Options :] \docAuxKey*{english}, \docAuxKey*{french} ;
  \item[Configuration] ™\renewcommand*\frenchtablename{Tableau}™.
  \end{description}
\item[varioref] références de pages intelligentes.
  \begin{description}
  \item[Options :] \docAuxKey*{nospace}, \docAuxKey*{french}.
  \end{description}
\item[eurosym] symbole et montants en \euro{}.
  \begin{description}
  \item[Option :] \docAuxKey*{right}.
  \end{description}
\item[listings] composition des listings informatiques.
  \begin{description}
  \item[Options :] \docAuxKey*{basicstyle=\ttfamily}, \docAuxKey*{frame=single},
    \docAuxKey*{belowskip=0pt}\footnote{Cette dernière option du fait
      d'\href{https://github.com/FrankMittelbach/fmitex-parskip/issues/3}{un
        problème actuel} impliquant les packages \package{parskip} et
      \package{listings}.}.
  \end{description}
\item[floatrow] nombreuses possibilités de personnalisation de la disposition
  des flottants.
  \begin{description}
  \item[Options :]\leavevmode{}
    \begin{itemize}
    \item \docAuxKey*{objectset=justified} ;
    \item \docAuxKey*{style=__letgut_ruled}\footnote{Style propre à la classe.}
      et \docAuxKey*{margins=hangleft} pour les figures ;
    \item \docAuxKey*{capposition=top} pour les tableaux.
    \end{itemize}
  \end{description}
\item[biblatex] bibliographies sophistiquées.
  \begin{description}
  \item[Option :] \docAuxKey*{sorting=none}.
  \end{description}
\item[acro] création simple d'acronymes\footnote{Pour la gestion des acronymes,
    il était initialement prévu de recourir au \package*{glossaries-extra} mais
    celui-ci augmente significativement le temps de compilation.}.
  \begin{description}
  \item[Options :]\leavevmode{}
    \begin{itemize}
    \item \docAuxKey*{first-style=footnote} ;
    \item \docAuxKey*{format/short=}™\scshape™ ;
    \item \docAuxKey*{format/foreign=}™\em™ ;
    \item \docAuxKey*{foreign/display} ;
    \item \docAuxKey*{locale/display} ;
    \item \docAuxKey*{locale/format=}™\upshape™.
    \end{itemize}
  \end{description}
\item[hyperref] prise en charge étendue de l'hypertexte.
  \begin{description}
  \item[Options :]\leavevmode{}
    \begin{itemize}
    \item \docAuxKey*{draft} si l'option de classe \refKey{paper} est utilisée ;
    \item \docAuxKey*{colorlinks}, \docAuxKey*{allcolors={letgut_allcolors_links}}
      sinon.
    \end{itemize}
  \end{description}
\item[hypcap] ajustement des ancres des légendes.
  \begin{description}
  \item[Option :] \docAuxKey*{all}.
  \end{description}
\item[cleveref] détermination automatique du format des références en fonction
  du type de référence.
  \begin{description}
  \item[Option :] \docAuxKey*{french}.
  \end{description}
\item[lua-typo] mise en lumière, par un changement de couleur, des lignes
  typographiquement imparfaites avec \hologo{LuaLaTeX}.
  \begin{description}
  \item[Option :] \docAuxKey*{All}.
  \end{description}
\end{ctannews}

\section{Packages utiles au codage de la classe \letgut}
\label{sec:utiles-au-codage}

\begin{ctannews}
\item[l3keys2e] traitement \hologo{LaTeX2e} des options de classe en utilisant
  les clés \hologo{LaTeX3}.
\item[parskip] mise en page de paragraphes séparés par un blanc vertical au lieu
  (ou en plus) d'un retrait.
\item[fancyhdr] contrôle étendu des en-têtes et des pieds de page.
\item[geometry] interface flexible et complète pour les dimensions des
  documents.
  \begin{description}
  \item[Options :] \docAuxKey*{a4paper}, \docAuxKey*{asymmetric}.
  \end{description}
\item[etoc] tables des matières entièrement personnalisables.
\item[enumitem] contrôle de la mise en page de \docAuxEnvironment*{itemize},
  \docAuxEnvironment*{enumerate}, \docAuxEnvironment*{description} et permet de
  cloner les environnements standards et créer de nouveaux
  environnements. Utilisé pour la création de l'environnement \refEnv{ctannews}.
\item[titlesec] personnalisation aisée des titres de sections, etc.
\item[placeins] contrôle du placement des flottants permettant de s'assurer que
  ceux d'une section (donc d'un article dans le cas de la \lettre{})
  apparaissent avant la commande \docAuxCommand*{section} suivante.
  \begin{description}
  \item[Options :] \docAuxKey*{section}, \docAuxKey*{above}.
  \end{description}
\item[accsupp] notamment remplacement de texte lors des copiés-collés, utilisé
  en particulier pour que les acronymes, composés en petites capitales par
  \letgut{}, une fois copiés, soient collés en grandes capitales.
\item[letgut-banner] bannière de la 1\iere{} page de la \lettre{}.
\end{ctannews}

\section{Exemple de déclaration de dialecte du langage \hologo{TeX}}
\label{sec:exemple-de-decl}

Nous fournissons ci-dessous un exemple de déclaration de dialecte (ici le
package \hologo{(La)TeX} \package{graphicx}) du langage \hologo{TeX}
(cf. section~\enquote{\nameref{sec:coloration}}, \vpageref{sec:coloration}).

\begin{ltx-code}[listing options app={deletekeywords={[3]{
  ,draft
  ,final
  ,setpagesize
  ,dvips
  ,dvipdfm
  ,dvipdfmx
  ,xetex
  ,pdftex
  ,dvipsone
  ,dviwindo
  ,textures
  ,vtex
  ,alt
  ,width
  ,height
  ,totalheight
  ,scale
  ,clip
  ,draft
  ,type
  ,command
  ,page
}}},
listing options app={deletekeywords={[4]{
  ,draft
  ,final
  ,dvips
  ,dvipdfm
  ,dvipdfmx
  ,xetex
  ,pdftex
  ,luatex
  ,vtex
  ,scale
  ,true
  ,false
}}}]
\lst@definelanguage[graphicx]{TeX}[LaTeX]{TeX}{%
  % Control sequences names
  moretexcs={%
    DeclareGraphicsExtensions,DeclareGraphicsRule,graphicspath,%
    includegraphics*,includegraphics,reflectbox,resizebox*,%
    resizebox,rotatebox,scalebox,%
  },%
  % Keywords of class 1 : keywords that contain other characters
  % (since of the same class as the ones specified as
  % 'otherkeywords')
  morekeywords={%
  },%
  % Keywords of class 2 : environments names
  morekeywords=[2]{%
  },%
  % Keywords of class 3 : mandatory arguments (not environments)
  % & optional arguments which are keys (in key=value)
  morekeywords=[3]{%
    draft,final,hiresbb,demo,setpagesize,nosetpagesize,dvips,xdvi,%
    dvipdf,dvipdfm,dvipdfmx,xetex,pdftex,luatex,dvisvgm,dvipsone,%
    dviwindo,emtex,dviwin,oztex,textures,pctexps,pctexwin,pctexhp,%
    pctex32,truetex,tcidvi,vtex,debugshow,hiderotate,hidescale,%
    alt,%
    %
    bb,bbllx,bblly,bburx,bbury,natwidth,natheight,hiresbb,pagebox,%
    viewport,trim,angle,origin,width,height,totalheight,%
    keepaspectratio,scale,clip,type,ext,read,command,quiet,page,%
    interpolate,decodearray,origin,x,y,units,%
  },%
  % Keywords of class 4 : values of keys (in key=value)
  morekeywords=[4]{%
    mediabox,cropbox,bleedbox,trimbox,artbox,true,false,%
  },%
  % Keywords of class 5 : arguments specifications (after ":"
  % in expl3 syntax)
  morekeywords=[5]{%
  },%
  % Keywords of class 6 : current package name (and possibly
  % derived packages)
  morekeywords=[6]{%
    graphicx,%
  },%
  % otherkeywords={},%
  alsoletter={23},%
  % alsodigit={},%
  sensitive,%
}[keywords,tex,comments]%
\end{ltx-code}

\printacronyms[
, heading=title
% , template=longtable
, preamble={\label{liste-acronymes}}
, display=all
, exclude=thisdocument
, name={Liste des acronymes prédéfinis par \letgut}
]

\printbibliography[heading=title]

% \listoffigures
% \listoftables
\tcblistof[\title]{dbwarninglist}{Table des avertissements}%
% \tcblistof[\title]{dbremarklist}{Table des remarques}%
%
\phantomsection
\indexprologue{%
  Afin de différencier leurs natures, les entrées de cet index sont affichées
  en couleurs (variées) lorsqu'elles correspondent à des :
  \begin{itemize}
  \item commandes ;
  \item environnements ;
  \item clés ;
  \item valeurs de clé.
  \end{itemize}
}%
% \renewcommand{\indexname}{Index des commandes}
% \printindex[\jobname-commands]
% %
% \phantomsection
% \indexprologue{%
%   Dans cet index, un numéro de page :
%   \begin{description}
%   \item[en gras] indique une page contenant une information importante sur
%     l'entrée correspondante, par exemple sa définition ;
%   \item[en italique] indique une page contenant un exemple qui illustre
%     l'entrée correspondante.
%   \end{description}
% }%
% \renewcommand{\indexname}{Index des concepts}
\printindex
\end{document}

%%% Local Variables:
%%% mode: latex
%%% TeX-engine: luatex
%%% TeX-master: t
%%% End:
