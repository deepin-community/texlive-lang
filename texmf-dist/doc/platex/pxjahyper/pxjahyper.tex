% 文字コードは UTF-8
% uplatex で組版する
\documentclass[uplatex,dvipdfmx,a4paper]{jsarticle}
\renewcommand{\headfont}{\gtfamily\romanseries{sbc}\sffamily}
\usepackage[T1]{fontenc}
\usepackage{lmodern,textcomp}
\usepackage{color}
\definecolor{myblue}{rgb}{0,0,0.75}
\definecolor{mygreen}{rgb}{0,0.45,0}
\usepackage[colorlinks,hyperfootnotes=false]{hyperref}
\usepackage{pxjahyper}
\hypersetup{linkcolor=myblue,urlcolor=mygreen,
  pdftitle={pxjahyperパッケージ},
  pdfauthor={八登崇之}}
\usepackage{bxtexlogo}
\bxtexlogoimport{*,TL-e}
\usepackage{shortvrb}
\MakeShortVerb{\|}
\usepackage[verb]{bxghost}
\newcommand{\PkgVersion}{1.3}
\newcommand{\PkgDate}{2023/03/01}
\newcommand{\Pkg}[1]{\textsf{#1}}
\newcommand{\Meta}[1]{$\langle$\mbox{}#1\mbox{}$\rangle$}
\newcommand{\Note}{\par\noindent ※}
\newcommand{\Means}{:\quad}
\providecommand{\Strong}[1]{{\headfont#1}}
%-----------------------------------------------------------
\begin{document}
\title{\Pkg{pxjahyper} パッケージ}
\author{八登崇之\ (Takayuki YATO; aka.~``ZR'')}
\date{v\PkgVersion\quad[\PkgDate]}
\maketitle
\tableofcontents

%===========================================================
\section{概要}

(u){\pLaTeX} + hyperref + dvipdfmxの組み合わせで
日本語を含む\Strong{PDF文字列}%
(“しおり”などの文書情報の文字列)
をもつPDF文書を作成する場合に必要となる機能を提供する。
\begin{itemize}
\item dvipdfmx用の「tounicode special」について、
  内部漢字コードに応じて適切なものを出力する。
\item PDF文字列の中で{\LaTeX}カーネル(|\"a|や|\textsection|等)
  や\Pkg{japanese-otf}パッケージ(|\UTF|や|\ajMaru|等)の
  文字出力用命令が“可能な限り”正しく機能するようにする。
  %ただし、エンジンが{\pTeX}の場合は、out2uni/convbkmk%
  %(および試験的機能の“force-unicode”)を利用
  %する場合を除き、JIS~X~0208にない文字は出力できない
  %(hyperrefの警告が出る)。
\item {\TeX}の版面拡大機能が使われている
  (|\mag|値が1000でない)場合
  \footnote{典型的には\Pkg{jsclasses}の
    文書クラスで基底フォントサイズを10\,pt以外に設定している場合。}%
  には\Pkg{hyperref}が出力するpapersize specialの
  紙面サイズの値が不正になるが、この場合に|\mag|値を
  考慮して求めた正しいサイズによるpapersize specialを
  改めて出力する。
\item \Pkg{hyperref}が行う「テキスト装飾命令の無力化」の
  対象に、{(u)\pLaTeX}特有のいくつかの命令を追加する。
\end{itemize}

%===========================================================
\section{パッケージの読込}

|\usepackage|で読み込む。
\begin{quote}\small\begin{verbatim}
\usepackage[オプション,...]{pxjahyper}
\end{verbatim}\end{quote}

\paragraph{使用可能なオプション}\mbox{}
\begin{itemize}
\item \Strong{設定キー記述}\Means
  |\pxjahypersetup|命令の引数に書く設定記述をパッケージオプションに
  書くことができる。
  例えば
\begin{quote}\small\begin{verbatim}
\usepackage[fallback=delete]{pxjahyper}
\end{verbatim}\end{quote}
  と指定すると、表現不能文字を代替するゲタ文字が出力されなくなる。
\item \Strong{ドライバオプション}\Means
  |dvipdfmx|、|dvips|、および|nodvidriver|%
  (ドライバ依存動作の無効化を明示的に指示、
  現状ではほとんどの動作が無効になる)
  が指定できる。
  \Note 1.0版において|nodvidriver|の非推奨の別名の|none|は
    \Strong{廃止}された。
  %(0.5版から非推奨だった。)
\item |resetdvidriver|(既定)\Means
  ドライバオプションを\Pkg{hyperref}のドライバ指定から自動的に判定する。
  \Note 1.1版において名前を|auto|から|resetdvidriver|に変更した。
    旧名の|auto|は\Strong{非推奨}とする。
\item |tounicode|(既定)\Means
  以下の2つの設定を行う。
  \begin{enumerate}
  \item ドライバが|dvipdfmx|である場合は適切な
    「tounicode special」を発行する。
    \Note この設定は|notounicode|オプションにより打ち消される。
  \item 「tounicode special」を前提にした出力を行う。
    \Note この設定は|out2uni|・|convbkmk|オプションと排他である。
  \end{enumerate}
\item |notounicode|\Means
  dvipdfmx用の「tounicode special」を発行しない。
\item |out2uni|\Means
  out2uniフィルタを使うことを前提にした出力を行う。
\item |convbkmk|\Means
  convbkmkフィルタを使うことを前提にした出力を行う。
  \Note ドライバが|dvips|の場合は既定で|convbkmk|が有効になる。
  これにより元の既定値|tounicode|は実質的に
  \footnote{|tounicode|の説明中の項目1の機能はdvipsでは効果がなく、
    また項目2の機能は上書きされるため。}%
  無効化される。
%\item |fallback-geta|(既定)\Means
%  PDF文字列中で表現不能な文字をゲタ記号(〓)に置き換える。
%\item |fallback-delete|\Means
%  PDF文字列中で表現不能な文字を削除する。
\item |papersize|(既定)\Means
  papersize specialの補正を有効にする。
\item |nopapersize|\Means
  |papersize|の否定。
\end{itemize}

\paragraph{上級者向けオプション}\mbox{}
\begin{itemize}
\item |otfmacros|(既定)\Means
  \Pkg{japanese-otf}付属の\Pkg{ajmacros}パッケージが提供する
  文字入力命令(|\ajMaru|、|\ajLig|等)をPDF文字列中で
  “可能な限り”使えるようにする。
  \Note 詳細は\ref{ssec:otf-package}節を参照。
  \Note |otfmacros|を有効にする場合は|otfcid|も有効にする必要がある。
\item |nootfmacros|\Means
  |otfmacros|の否定。
  \Note 0.6版より既定を|otfmacros|に変更した。
\item |otfutf|(既定)\Means
  \Pkg{japanese-otf}パッケージの|\UTF|命令をPDF文字列中で
  使えるようにする。
  \Note 詳細は\ref{ssec:otf-package}節を参照。
\item |nootfutf|\Means
  |otfutf|の否定。
\item |otfcid|(既定)\Means
  \Pkg{japanese-otf}パッケージの|\CID|命令をPDF文字列中で
  “可能な限り”使えるようにする。
  \Note エンジンの{\eTeX}拡張および
  \Pkg{etoolbox}と\Pkg{bxjatoucs}パッケージのインストールが必要。
  \Note 詳細は\ref{ssec:otf-package}節を参照。
\item |nootfcid|\Means
  |otfcid|の否定。
\item |disablecmds|(既定)\Means
  「PDF文字列中のテキスト装飾命令の無効化」を有効にする。
  \Note 詳細は\ref{ssec:disablecmds}節を参照。
\item |nodisablecmds|\Means
  |disablecmds|の否定。
\item |bigcode|(既定)\Means
  {\upTeX}でのToUnicode CMapとしてUTF8-UTF16を用いる。
  %(当該のファイルが存在する必要がある。)
\item |nobigcode|\Means
  |bigcode|の否定。
  {\upTeX}でのToUnicode CMapとしてUTF8-UCS2を用いる。
  \Note 0.3a版より既定を|bigcode|に変更した。
\item |force-unicode|\Means
  このオプションは何もしない。
  \Note {\pLaTeX}での“unicodeモード”が正式にサポートされたため
  このオプションは不要になった。
\end{itemize}

%===========================================================
\section{機能}

「概要」で述べた機能は(オプション設定に応じて)
自動的に実施される。

%-------------------
\subsection{設定}
\label{ssec:setup}

パッケージの動作の設定を|\pxjahypersetup|命令で行える。

\begin{quote}\small
|\pxjahypersetup{|\Meta{キー}|=|\Meta{値}|,...}|
\end{quote}

有効な設定キーは以下の通り。

\begin{itemize}
\item |fallback=|\Meta{値}\Means
  PDF文字列中で表現不能な文字についての代替出力の方法を指定する。
  \begin{itemize}
  \item |geta|(既定)\Means
    表現不能な文字をゲタ記号(〓)に置き換える。
  \item |delete|\Means
    表現不能な文字を削除する。
  \end{itemize}
  \Note 以下、パッケージ動作の解説中で「ゲタ記号を出力」と
    ある場合は実際にはこの設定に従う。
\end{itemize}

%-------------------
\subsection{Unicode符号値による入力}
\label{ssec:ux-command}

PDF文字列入力中で、命令|\Ux|が以下の意味に変更される。
PDF文字列以外では|\Ux|は以前の定義(または未定義)に戻る。
\footnote{|\Ux|という命令名は\Pkg{bxbase}パッケージの
Unicode符号値入力用の命令が使っているものである。
従って、\Pkg{bxbase}パッケージを読み込んでいれば、
「PDF文字列と版面出力の両方に使われる」ようなテキストにおいて、
|\Ux|でUnicode符号値入力が可能になる。
ただし、Unicode符号値入力用の命令としては
「\Pkg{japanese-otf}パッケージの|\UTF|命令」
の方が有名であり、\Pkg{pxjahyper}は|\UTF|もサポートするので、
こちらを使う方が無難かもしれない。}

\begin{itemize}
\item |\Ux{|\Meta{Unicode符号値16進}|}|\Means
  その符号値の文字を出力する。
  具体的な動作は以下の通り:
  \begin{itemize}
  \item |out2uni|または|convbkmk|が有効の
  場合は、エスケープ表記(|\0xUUUU|)を出力する。
  \item エンジンが{\upLaTeX}の場合、あるいは“unicodeモード”
  (\ref{sec:Unicode-mode}節)が有効の場合は、
  当該のUnicode文字自体を書いたのと同等になる。
  \item 上記以外で、{\TeX} Live 2022以降の{\pLaTeX}の場合
  \footnote{正確にいうと、|\Uchar|と|\ucs|プリミティブをもつ
    {\TLe-(u)\pTeX}エンジンである場合。}%
  は、当該のUnicode文字に対応するJIS符号系の文字を書いたのと同等になる。
  JIS符号系にない文字の場合は出力できないので警告を出しだ上で
  ゲタ記号を出力する。
  \end{itemize}
  \item どの条件にも当てはまらない場合は、|\Ux|は無効になる
  (定義されない)。
\end{itemize}

%-------------------
\subsection{\Pkg{japanese-otf}パッケージの文字入力命令への対応}
\label{ssec:otf-package}

\paragraph{\textbackslash UTF命令}
\Pkg{japanese-otf}パッケージの|\UTF|命令は、PDF文字列中では
out2uni用の出力を行う。
本パッケージで|otfutf|オプションを有効にした場合は、
PDF文字列中の動作が以下のように変更される。

\begin{itemize}
\item |\Ux|命令(\ref{ssec:ux-command}節参照)が有効の場合は、
  |\Ux|と同じ動作
  \footnote{もし|\Ux|の出力がゲタ記号になる場合は、
    |\UTF|もゲタ記号になる。}%
  になる。
\item それ以外の場合は、
  常に(警告を出した上で)ゲタ記号を出力する。
\end{itemize}

\Note |\UTF|命令の多言語版、
すなわち|\UTFC|・|\UTFK|・|\UTFM|・|\UTFT|命令も|\UTF|%
と同じ扱いになる。

\paragraph{\textbackslash CID命令}
\Pkg{japanese-otf}パッケージの仕様では|\CID|命令は、PDF文字列中では
サポートされない(未定義動作となる)。
本パッケージで|otfcid|オプションを有効にした場合は、
PDF文字列中で|\CID|が“可能な限り”使えるようにする。
具体的な仕様は以下の通り。

\begin{itemize}
\item 当該のAJ1のグリフに“対応”する\textbf{単独の}Unicode文字が
  あればそれを出力し、なければ
  (警告を出した上で)ゲタ記号を出力する。
  \footnote{旧版では削除していたが、他の同様の場合と合わせるため
    1.0版よりゲタ記号を出力する仕様を変更した。}
  \Note 例えば、|\CID{8226}|(ローマ数字12)はUnicode文字の
  U+217Bに“対応”するので|\Ux{217B}|と同等になるが、
  |\CID{8297}|(ローマ数字15)については“対応”する
  単独のUnicode文字がないので、ゲタ記号に置き換えられる。
\item ただし|\Ux|命令(\ref{ssec:ux-command}節参照)が
  無効になる場合は、そもそもUnicode文字も出力できないため、
  常に(警告を出した上で)ゲタ記号を出力する。
  結局情報は欠落するが、それでも未定義動作(エラーになりえる)
  よりは好ましいであろう。
\end{itemize}

\Note |otfcid|の利用には、エンジンの{\eTeX}拡張および
\Pkg{etoolbox}と\Pkg{bxjatoucs}パッケージが必要。
\Note |\CID|命令の多言語版はサポートされない。

\paragraph{ajmacrosパッケージの命令}
本パッケージで|otfmacros|オプションを有効にした場合は、
\Pkg{japanese-otf}付属の\Pkg{ajmacros}パッケージが提供する
文字入力命令(|\ajMaru|、|\ajLig|等)をPDF文字列中で
“可能な限り”使えるようにする。
具体的な仕様は以下の通り。

\begin{itemize}
%\item 現状では、|\Ux|命令が有効になる場合のみが
%  サポートされる。
%  \Note それ以外の場合は|otfmacros|オプションは無効になる。
\item Unicode文字で表現可能であればそれを出力し、
  なければ代替表現を出力する。
\item Unicode文字を出力する場合の仕様は|\CID|と同じ。
  (|\Ux|が無効の場合はゲタ記号になる。)
  代替表現の場合は「普通の文字の出力に置き換えられる」
  可能性がある。
  \Note 例えば、|\ajLig{ドル}|(“ドル”の組文字)はUnicode文字の
  U+3326に“対応”するので|\Ux{3326}|と同等になるが、
  |\ajLig{ウルシ}|(“ウルシ”の組文字)はUnicodeに“対応”する
  文字がないため単に“\textgt{ウルシ}”と書いたのと同等になる。
\end{itemize}

\Note |otfmacros|を有効にする場合は|otfcid|も有効にする必要がある。
(従って|otfcid|と同じ前提条件が課される。)
|otfcid|が無効な場合は|otfmacros|も無効になる。
\Note \Pkg{ajmacros}パッケージの多くの命令は脆弱(fragile)である。
そのため、節見出し(|\section|等の引数)で|\ajMaru|等の命令を
使いたい場合は、命令の前に|\protect|を付ける必要がある。
\footnote{ちなみに、引数がPDF文字列として解釈される場合には、
  |\protect|は全く結果に影響しない。}

%-------------------
\subsection{PDF文字列用の文字命令のユーザ定義}

以下の命令が提供される。(プリアンブルでのみ使用可能。)

\begin{itemize}
\item |\pxDeclarePdfTextCommand{\制御綴}{|\Meta{JIS符号値}|}{|\Meta
{Unicode符号値}|}|\Means
  PDF文字列中の|\制御綴|の動作として、
  指定した符号値の文字を出力することを指定する。
\item |\pxDeclarePdfTextComposite{\制御綴}{|\Meta{引数}|}{|\Meta
{JIS符号値}|}{|\Meta{Unicode符号値}|}|\Means
  PDF文字列中の「|\制御綴|(アクセント命令)+ \Meta{引数}」の
  動作として、指定した符号値の文字を出力することを指定する。
\end{itemize}

これらの命令において、符号値は16進数で指定する。
Unicode文字の出力が可能な状況
(エンジンが{\upLaTeX}の場合は常に該当する)
では「JIS符号値」は使われないので省略して
(空にして)もよい
\footnote{一応、「JIS符号値」が使われることが確実な状況では
「Unicode符号値」も省略可能であるが、
そのような状況であるかの判断は困難であるため、
「Unicode符号値」の省略は推奨されない。}%
(或いはそもそもJIS~X~0208にない文字の場合は省略する)。

例えば、以下のように定義しておくと、
PDF文字列中で|\textschwa|(schwa記号)や|\d{t}|(\d{t})が
使えるようになる。
\begin{quote}\small\begin{verbatim}
\pxDeclarePdfTextCommand{\textschwa}{}{0259}
\pxDeclarePdfTextComposite{\d}{t}{}{1E6D}
\end{verbatim}\end{quote}

%-------------------
\subsection{PDF文字列用中のテキスト装飾命令の無効化}
\label{ssec:disablecmds}

PDF文字列は単なるUnicode文字列として扱われるものなので、
|\textit|や|\large|等のテキスト装飾用の命令は意味をなさず、
またそれらの命令の実装はPDF文字列の解釈中は正常に処理できない。
PDF文字列と版面出力の両方に使われるテキスト(節見出し等)
についてテキスト装飾命令が支障なく使えるように、
\Pkg{hyperref}では基本的なテキスト装飾命令
(多くは{\LaTeX}カーネルが提供するもの)
について、「PDF文字列として扱う場合は自動的に無力化
\footnote{例えば、“|\textit{text}|”や“|{\large text}|”は
  単に“|text|”と書いたものと見なされる。}
する」機構を実装している。
これにより、例えば節見出しのテキストに“|\textit{text}|”が
含まれていたとすると、
版面に出力する場合には“\textit{text}”のように装飾が施され、
一方で、PDF文字列としては“|text|”と解釈されることになる。

0.5版以降の\Pkg{pxjahyper}では、この無効化の対象に
「和文用のテキスト装飾命令(およびそれに準じるもの)」
を追加するようになった。
以下の命令が対象になる。

\begin{itemize}
\item \Pkg{hyperref}での無効化の対象である「フォント選択命令」の
  和文版に相当するもの。
  例えば、
  |\textmc| |\gtfamily| |\kanjifamily| |\romanshape|
  |\usekanji| |\useroman| |\userelfont|
  等が該当する。
\item {\pLaTeX}カーネル命令\Means
  |\<|
\item {\pTeX}プリミティブ\Means
  |\|(|dis|)|inhibitglue| |\|(|no|)|autospacing| |\|(|no|)|autoxspacing|
\item \Pkg{plext}の命令\Means
  |\bou| |\kasen| |\rensuji|
\item \Pkg{japanese-otf}の命令\Means
  |\textmg| |\mgfamily| |\ltseries| |\ebseries| |\propshape|
\item \Pkg{jsclasses}のクラスの命令\Means
  |\maybeblue| |\HUGE|
\item \Pkg{jlreq}クラスの命令\Means
  |\jafontsize| |\tatechuyoko| |\jidori| % |\jaspace| |\akigumi|
\end{itemize}

%===========================================================
\section{hyperrefの“unicodeモード”での動作}
\label{sec:Unicode-mode}

\Pkg{hyperref}パッケージの|unicode|オプションが有効である場合
(これを“unicodeモード”と呼ぶことにする
\footnote{“unicodeモード”を有効にする方法は|unicode|(|=true|)\,%
  の指定以外にも存在する。
  また、\Pkg{hyperref}の7.00g版[2021-02-04]より、
  {\pLaTeX}以外のエンジン(\Strong{{\upLaTeX}も含む})について
  “unicodeモード”は既定で有効になっていることに注意。}%
)で動作している場合は、PDF文字列のUnicodeへの変換は
(DVIドライバ等でなく)\Pkg{hyperref}自身により行われる。
\Pkg{hyperref}が“unicodeモード”である場合には、\Pkg{pxjahyper}は
それを自動的に検知してそれに適応した動作に切り替える。
\begin{itemize}
\item |\Ux|命令は\Pkg{hyperref}の|\unichar|命令を利用して出力する。
  このため、{\pLaTeX}でもUnicode文字の出力が可能になる。
\item PDF文字列中の和文文字やLICR命令の処理は
  \Pkg{hyperref}の側に任せられる。
  \Note ただし現状では「tounicode special」の発行は
  (特に害はないため)無効化されない。
\end{itemize}
ただし、現状での“unicodeモード”対応動作には以下の制限がある。
\begin{itemize}
\item \Pkg{hyperref}の(|pdftitle|等の)パッケージオプション中での
  和文文字の処理は失敗する。
  このため、文書情報は|\hypersetup|命令で指定する必要がある。
%\item {\pLaTeX}における“unicodeモード”対応動作は試験的機能の
%  扱いであるため、その旨の警告が表示される。
%  この警告は|force-unicode|オプションを指定すれば抑止できる。
\end{itemize}

\end{document}
