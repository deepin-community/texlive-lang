%
% texlive-cz/texlive-cz.tex - TeXlive 2022 documentation 
% CzechoSlovak translation (some parts are in Czech and some
% are in Slovak :-)
%
% file is in the UTF8 encoding  $Id: texlive-cz.tex 62854 2022-03-21 14:37:53Z sojka $
%
% Pokud chcete opravit chyby v~tomto textu, a nemáte oprávnění pro opravy
% přímo v~texlive svn repozitáři, posílejte návrhy změn na adresu
% sojka at fi.muni.cz
%
% TeX Live documentation.  Originally written by Sebastian Rahtz and
% Michel Goossens, now maintained by Karl Berry and others.
% Public domain.
%
% History:
% 2022/03/20: v1.71 last minute fixes and additions (PS)
% 2022/03/18: v1.70 last minute fixes and additions (PS)
% 2022/03/09: v1.69 translation of 2022 modifications (JB)
% 2022/03/05: v1.68 last minute fixes and additions (PS)
% 2021/03/07: v1.67 translation of 2021 modifications (JB)
% 2020/03/02: v1.66 last minute fixes and additions (PS)
% 2020/03/07: v1.65 translation of 2020 modifications (JB)
% 2019/04/05: v1.64 last minute fixes and additions (PS)
% 2019/03/17: v1.63 translation of 2019 modifications (JB)
% 2018/04/11: v1.62 last minute fixes and additions (PS)
% 2018/04/10: v1.61 translation of 2018 modifications (JB, PS)
% 2017/05/14: v1.59.,60 last minute translation additions (PS)
% 2017/05/09: v1.58 translation of 2017 modifications (JB, PS)
% 2016/05/15: v1.57 last minute changes from EN (PS)
% 2016/04/28: v1.56 translation of 2016 modifications (JB, PS)
% 2015/12/29: v1.55 proofreading corrections by Z. Michálek
% 2015/05/11: v1.54 feedback from J. Kuben (PS)
% 2015/05/06: v1.53 translation of 2015 modifications (JB, PS)
% 2015/05/05: v1.52 2015 version translation started (PS)
% 2015/02/13: v1.51 small corrections reported by users (PS) 
% 2014/05/09: v1.50 further edits and cleanup (PS)
% 2014/05/08: v1.49 small edits and cleanup (PS)
% 2014/05/08: v1.48 translation of 2014 modifications update (JB)
% 2014/05/04: v1.47 translation of 2014 modifications (JB)
% 2013/05/29: v1.46 small edits and cleanup (PS) 
% 2013/05/28: v1.45 translation of 2013 modifications (JB)
% 2012/06/09: v1.44 further typos (PS)
% 2012/06/05: v1.43 typos, checking 2012 modifications (PS) 
% 2012/05/21: v1.42 translation of 2012 modifications (JB) 
% 2011/06/28: v1.41 switch to utf8, typos (PS)
% 2011/06/24: v1.40 small fixes, deactivation of - (babel) (PS) 
% 2011/06/16: v1.39 translation of new things for 2011 (JB) 
% 2010/06/26: v1.38 translation of new things for 2010 (JB) 
% 2009/10/20: v1.37 typos (by Piska) fixed, together with other edits (PS) 
% 2009/09/23: v1.36 first 2009 version committed to TL 2009 (PS) 
% 2009/09/02: v1.35 translation of new things for 2009 (JB)
% 2008/08/12: v1.34 first corrections of 2008 translation
% 2008/08/02: v1.33 translation update 2008
% 2007/01/24: v1.32 XeTeX URL and HH (config.manpath) change
% 2007/01/22: v1.31 last minute corrections by Staszek (2x) and ZW
% 2007/01/15: v1.30 corrections and typos
% 2007/01/14: v1.29 texlive2007 version (JB&PS)
% 2005/11/01: v1.28 last minute Staszek additions
% 2005/10/31: v1.27 texlive2005 version (JB&PS)
% 2004/11/18: v1.26 final release cleanup, latest changes by Staszek
% 2004/11/17: v1.24,25 typos corrected
% 2004/11/15: v1.21,22,23 corrections by KB, JB and others entered
% 2004/11/14: v1.20 first texlive2004 version (JB&PS) 
% 2004/01/30: v1.19 postbulletin version release for CSTUG web
% 2004/01/22: v1.18 minor changes before final-2003 source release
% 2004/01/19: v1.16, v1.17 last-minute corrections before sending
%             CSTUG Bulletin to print
% 2004/01/18: modifications for CS bulletin by Z. Wagner
% 2004/01/07: version 1.15
% 2003/12/08-10: Slovak and other Czech typos corrected
% 2003/12/06: Czech language typos/errors entered
% 2003/09/23: prepositions checked, v1.03
% 2003/09/23: some typos, jch's email, v1.02
% 2003/09/22: further error corrections, v1.01
% 2003/09/18: final checking and error corrections, v1.0 release
% 2003/09/17: another changes in English version incorporated
% 2003/09/08: changes in English version from 09/04 to 09/06 incorporated
% 2003/09/06: fixes from Staszek
% 2003/09/05: touched by Karl
% 2003/09/04: fixes from Staszek and Fabrice
% 2003/09/05: translation started from English version as of Sep 1st by
%             Petra & Petr S.
% 2003/09/01: fixes from Volker.
% 2003/08/12: fixes from Christer Gustafsson <gustaf@powertech.no>.
% 2003/07/07: substantial revisions by Karl.
% ... deleted
%
% TODO
% * adding metadata for hyperref
% * adding markup for hyphenation switching between CZ and SK
%
% version, metadata as macros
%
\newcommand\thisyear{2022}
\newcommand\lastyear{2021}
\newcommand\cstlversion{1.71}
\newcommand\cstldate{20.\ b\v rezna 2022} %\thisyear}
%
% macros for different version of document:
% - in Zpravodaj CSTUG (with different layout)
% - for generation of html version
% With the advent of using htlatex in 2020 (so we can get better HTML
% output), the \Status setting from tex-live.sty should be used. If ever
% reprinted again in Zpravodaj, have to redo ...
\ifx \HCode\UnDef
 \def\classname{artikel1} % running (pdf)TeX
 \def\classoptions{11pt}
\else
  \def\classname{article} % running TeX4ht
  \def\classoptions{11pt}
\fi
%
\newif\ifforbulletin
\forbulletinfalse % relict from publishing tldoc as Zpravodaj issue
%%\forbulletintrue
%\ifforbulletin
% \providecommand\Status{3} % version for CSTUG bulletin Zpravodaj
% \else
% \providecommand\Status{1} % 2 html alebo 1 pdf
%\fi
%\ifnum\Status=2            % e.g. html generation via tex4ht
%  \def\classname{article}  % need different doc class for html
%  \def\classoptions{11pt}
%\else
%  \ifnum\Status=3          % Zpravodaj
%         \def\classname{article}
%         \def\classoptions{twoside,a5paper}
%%  \pdfpaperwidth=147.5mm
%%  \pdfpaperheight=210mm
%  \else \def\classname{artikel1}
%        \def\classoptions{11pt}
%  \fi
%\fi

%\listfiles
\documentclass[\classoptions,slovak,english,czech]{\classname}
%\usepackage{xkeyval}
\usepackage{tex-live}
\usepackage[T1]{fontenc}
\usepackage[utf8x]{inputenc}
\usepackage[slovak,czech]{babel}%
\usepackage{csquote} % do define \uv
%\usepackage[all]{svn-multi}
%
% document macros
%
\newcommand{\singleuv}[1]{,#1`}
\newcommand{\ctt}{\url{news:comp.text.tex}}
\newcommand{\kB}{KiB}
\newcommand{\MB}{MiB}
\newcommand{\GB}{GiB}
\newcommand{\CSTUG}{\CS TUG}
\providecommand*{\GS}{Ghost\-script\xspace}
\newcommand\TKCS{\textsf{\TeX-kolekce}} 
\def\p.{na straně~}
\makeatletter
\newcommand\tubreflect[1]{%
 \@ifundefined{reflectbox}{%
   \message{A graphics package must be loaded for \string\XeTeX}%
 }{%
   \ifdim \fontdimen1\font>0pt
     \raise 1.75ex \hbox{\kern.1em\rotatebox{180}{#1}}\kern-.1em
   \else
     \reflectbox{#1}%
   \fi
 }%
}
\makeatother
\newcommand\tubhideheight[1]{\setbox0=\hbox{#1}\ht0=0pt \dp0=0pt \box0 }
%
% layout
%
\clubpenalty 10000
\widowpenalty 10000
%
% hyperref setup
%
\newcommand\docauthor{Karl Berry, editor}
\newcommand\docmaintitle{\TL\ \thisyear}
\newcommand\dockeywords{\TL\ \thisyear; documentation; installation; TeX}
\newcommand\docsubtitle{Příručka \TL, CS verze \cstlversion}
\hypersetup{%
  pdfauthor={\docauthor},
  pdftitle={\docmaintitle: \docsubtitle},
  pdfsubject={\dockeywords},
  pdfkeywords={\dockeywords; \docauthor},
% unicode=true,
% plainpages=false,
% pdfpagelabels=true,
%%pdfborder={0 0 0},
%%backref=page,
% pdfpagemode=UseNone,
% pdfstartview={XYZ null null 1},
% pdfpagelayout=OneColumn,
  pdfdisplaydoctitle=true,
  bookmarksopen=true,
  bookmarksopenlevel=5,
  bookmarksnumbered=true
}

\usepackage{microtype}

\begin{document}
\ifnum\Status=2
   % this ugliness is because the normal definition of ' for tex4ht
   % tries to go into math mode for Czech. Don't know why.
   \def'{\char`\'}
   % And these are because the default html output for (La)TeX is a mess.
   % In English they somehow become the desired simple "TeX" and "LaTeX",
   % but not in Czech. Don't know why. Redefining here seems to suffice.
   \def\TeX{TeX}
   \def\LaTeX{LaTeX}
\fi

\selectlanguage{czech}
\hyphenation{Meta-Postu}
\errorcontextlines=20
\shorthandoff{-}
\title{{\huge\textsf{\docmaintitle}}\\[3mm]
       \docsubtitle}
\date{\cstldate}

\author{\docauthor \\[3mm]
        \url{https://tug.org/texlive/}\\[3mm]
%B%\ifnum \Status=2
%B%        \includegraphics[bb=0 0 824 741]{../pictures/front} \\[5mm]
%B%\else
%B%\ifnum \Status=3
%B%       \includegraphics[width=.5\linewidth]{../pictures/bw-front2} \\[5mm]
%B%\else
%B%       \includegraphics[width=.5\linewidth]{../pictures/front} \\[5mm]
%B%\fi \fi
%B%       \small \textit{Kontaktní osoby pro tuto dokumentaci:}\\[3mm]
%B%       \small \begin{tabular}{lcr}
%B%          Czech/Slovak & Petr Sojka & \email{sojka (at) fi.muni.cz} \\
%B%                       & Ján Buša & \email{jan.busa (at) tuke.sk} \\                 
%B%          English  & Karl Berry & \email{karl (at) freefriends.org} \\
%B%          French   & Fabrice Popineau & \email{fabrice.popineau (at) supelec.fr} \\
%B%          German   & Volker R.\,W.\ Schaa & \email{volker (at) dante.de} \\
%B%          Polish   & Staszek Wawrykiewicz & \email{staw (at) gust.org.pl} \\
%B%          Russian  & Boris Veytsman & \email{borisv (at) lk.net}
%B%         \end{tabular}
%       \tlimage{front}{bb=0 0 824 741}{.7\linewidth}
         }

% comes out too close to the toc, and we know it's page one anyway.
\thispagestyle{empty}
\maketitle

\ifnum\Status=3
  \thispagestyle{csbul}
  \ifforbulletin \setcounter{page}{113} % valid for 3-4/2003
  \else \raggedbottom
  \fi
\fi

Překlad 
  2004--2022 Ján Buša, 
  2001 Janka Chlebíková,
  2003--2022 Petr Sojka a 
  2003 Petra Sojková 
je šířen pod GNU FDL licencí.
  
Permission is granted to copy, distribute and/or modify this document
under the terms of the GNU Free Documentation License, Version 1.2
or any later version published by the Free Software Foundation;
with no Invariant Sections, no Front-Cover Texts, and no
Back-Cover Texts.

\bigskip  % \newpage?


%\begin{multicols}{2}
\tableofcontents
%\listoftables
%\end{multicols}

\section{Úvodem}
\label{sec:intro}

\subsection{\TeX\ Live a kolekce \TeX{}u}

Tento dokument popisuje základní vlastnosti distribuce \TeXLive{} \thisyear,
což je instalace \TeX{}u a příbuzných programů pro \acro{GNU}/Linux a další
unixové systémy, \MacOSX{} a (32bitové) systémy Windows.

\TL{} můžete získat stažením z~internetu nebo na \TKCS{} \DVD.
Některé skupiny uživatelů \TeX{}u distribuují \DVD\ svým členům. 
Obsah \DVD\ je stručně popsán v~oddíle~\ref{sec:tlcoll-dists}.
\TL{} a \TKCS{} spolu jsou výsledkem společného úsilí 
skupin uživatelů \TeX u. Tento dokument popisuje převážně samotný \TL{}.

\TL{} obsahuje \file{.exe} soubory pro \TeX, \LaTeXe, \ConTeXt, 
\MF, \MP, \BibTeX{} a mnoho dalších programů včetně obsáhlého
seznamu maker, fontů a dokumentace spolu s~podporou
sazby v~mnoha různých světových jazycích. 

%Aktuální verze použitého software je v~archivu
%\acro{CTAN} na \url{http://www.ctan.org/}.

Krátký seznam hlavních změn v~této verzi \TL{}
najdete na konci tohoto dokumentu, v~oddílu~\ref{sec:history}. 
%na straně~\pageref{sec:history}.

\htmlanchor{platforms}
\subsection{Podpora operačních systémů}
\label{sec:os-support}

\TL{} obsahuje binárky pro mnohé unixové platformy
včetně \GNU/Linux, \MacOSX a Cygwin. 
Obsažené zdrojové texty mohou být zkompilovány
pro platformy, pro které neposkytujeme binárky.

Co se týče Windows: podporovány jsou Windows~7 a pozdější verze.
Windows Vista a 2000 ještě bude \emph{pravděpodobně} z~větší části fungovat,
ale \TL{} se dokonce nenainstaluje pod Windows XP a dřívějšími. 
Pro Windows nejsou k~dispozici zvláštní 64bitové \file{.exe} soubory, ale
32bitové binárky by měly běžet na 64bitových systémech. Ale podívejte se na
\url{https://tug.org/texlive/windows.html} pro možnosti přidání
64bitových binárních souborů. 
			
Alternativní řešení pro Windows a \MacOSX{} najdete 
v~oddíle~\ref{sec:tlcoll-dists}.

\subsection{Základní instalace \protect\TL{}}
\label{sec:basic}

\TL{} můžete nainstalovat buď z~\DVD{} nebo z~internetu 
(\url{https://tug.org/texlive/acquire.html}). Samotný síťový
instalační program je malý a vše požadované stáhne z~internetu. 

Instalační program na \DVD{} vám umožní instalaci na lokálním 
disku, ale \TL{} nemůžete spustit přímo z~\TK{} \DVD{} 
(nebo z~\TK{} nebo \TL{} \code{.iso} obrazů), avšak \emph{můžete} připravit
spustitelnou instalaci, například na klíči \USB{} (viz
oddíl~\ref{sec:portable-tl}). Instalace je popsána v~následujících 
sekcích (na stránce~\pageref{sec:install}), zde jen souhrn:
\begin{itemize*}
\item Instalační dávka pro Unix se jmenuje \filename{install-tl}; na
Windows byste místo toho měli vyvolat \filename{install-tl-windows}.
Instalační program bude pracovat v~grafickém režimu při volbě 
\code{-gui} (výchozí pro Windows a starší \MacOSX) nebo v~textovém režimu
při zadání volby \code{-gui=text} (výchozí pro všechny ostatní). 

\item Součástí instalace je program
\singleuv{\TL\ Manager}, nazvaný \prog{tlmgr}.
Podobně jako instalační program může být použit 
v~režimu \GUI{} nebo v~textovém režimu. 
Můžete ho použít k~nainstalování nebo odinstalování 
balíků a na různé konfigurační činnosti.
\end{itemize*}

\htmlanchor{security}
\subsection{Úvahy o bezpečnosti}
\label{sec:security}

Pokud je nám známo, hlavní programy \TeX{u} jsou (a vždy byly)
extrémně robustní. Nicméně programy dodané v~rámci \TeX\ Live nemusí dosahovat
stejné úrovně, navzdory nejlepšímu úsilí všech. Jako vždy při použití programů
pro nedůvěryhodný vstup musíte být opatrní; pro zvýšení bezpečnosti použijte
nový podadresář nebo \singleuv{chroot}.

Tato potřeba opatrnosti je zvláště naléhavá pro Windows, protože 
Windows obvykle najdou programy v~aktuálním adresáři před jinými,
bez ohledu na cestu vyhledávaní. To otevírá široké varianty možných
útoků. Zavřeli jsme mnoho děr, ale nepochybně některé zůstaly,
obzvláště pro programy pocházející odjinud. Proto doporučujeme 
zkontrolovat podezřelé programy v~aktuálním adresáři, zvláště ty spustitelné 
(binárky nebo skripty). Běžně by neměly být přítomny a rozhodně
nemohou být normálně vytvořeny při zpracování dokumentů.

\TeX\ (a jeho doprovodné programy) jsou schopny při běhu
zapisovat do souborů. Tato schopnost může rovněž být zneužita 
mnohými způsoby. Zpracování neznámých dokumentů v~novém podadresáři 
je nejbezpečnější tip.

Dalším aspektem zabezpečení je zajištění toho, že stažený materiál nebyl
od~vytvoření změněn. Program \prog{tlmgr}
(sekce~\ref{sec:tlmgr}) automaticky provede kryptografické
ověření stahování, pokud je program \prog{gpg} (GNU Privacy Guard)
k~dispozici. Není distribuován jako součást \TL, ale vizte
\url{https://texlive.info/tlgpg/} pro informace o~\prog{gpg}, pokud
je potřebný.

\subsection{Nápověda}
\label{sec:help}

\TeX{}ovská komunita je aktivní, vstřícná a většina
seriózních otázek je obvykle zodpovězena. Podpora je
neformální, je prováděna příležitostnými uživateli a dobrovolníky,
a proto je důležité, abyste odpověď na svůj dotaz hledali nejdříve sami
než ho vznesete na fóru. Pokud toužíte po garantované komerční podpoře,
můžete na \TL{} zapomenout a zakoupit si komerční distribuci od
prodejců na \url{https://tug.org/interest.html#vendors}.

Níže je uveden seznam informačních zdrojů, přibližně v~pořadí, ve kterém
ho doporučujeme k~použití:
\begin{description}
\item [první kroky] Pokud jste \TeX ovský nováček, na stránce
\url{https://tug.org/begin.html} najdete krátký úvod do systému.

\item [CTAN] Pokud hledáte konkrétní balík,
  font, program ap., \CTAN{} je místo, kde začít.
  Je to obsáhlá sbírka všech \TeX{}ových položek. Katalogové záznamy vám také říkají, zda je
  balíček dostupný pro \TL{} nebo MiK\TeX. Viz
  \url{https://ctan.org}.

\item [\TeX{} FAQ] \TeX{} FAQ je studnice
znalostí obsahující všechny druhy otázek, od těch základních až po 
ty nejobskurnější. Dokument najdete na stránce  \url{https://texfaq.org}.

\item [\TeX{}ové odkazy na Webu] Na \url{https://tug.org/interest.html}
  najdete mnoho relevantních odkazů na různé příručky,
  knihy, manuály a články o~všech aspektech systému \TeX{}.

\item [archivy diskusních skupin]
  Základní fóra pro hledání řešení problémů \TeX{}u zahrnují 
  stránku \LaTeX{}ovské komunity \url{https://latex.org}, 
  stránku kolektivně editovaných dotazů a odpovědí \url{https://tex.stackexchange.com},
  newsovou skupinu \ctt{} a e-mailovou diskusní
  skupinu \email{texhax@tug.org}. V~archivech těchto zdrojů
  najdete tisíce předchozích dotazů a odpovědí z~minulých let čekajících na vaše
  hledání. Nahlédněte na dva poslední zmíněné zdroje
  \url{http://groups.google.com/group/comp.text.tex/topics}, a na
  \url{https://tug.org/mail-archives/texhax}. Dotazem do obecného
  vyhledávacího stroje %jako \url{http://google.com/} 
  také nic nepokazíte a pro specifika češtiny a slovenštiny najdete další zdroje
  odkazované na \url{http://www.cstug.cz/}.
  
\item [kladení dotazů] Pokud nemůžete najít odpověď na svou otázku,
  můžete ji položit na \url{http://latex-community.org/} a
  \url{https://tex.stackexchange.com/} prostřednictvím jejich webových rozhraní, 
  na \ctt{} přes Google nebo newsového klienta nebo e-mailem na \email{texhax@tug.org}.
  Ale dříve, než tak učiníte, \emph{prosím,} přečtěte si toto heslo FAQ,
  abyste maximalizovali vyhlídky na získání užitečné odpovědi:
  \url{https://texfaq.org/FAQ-askquestion}.

\item [podpora \TL{}] Pokud chcete poslat
  chybové hlášení, připomínku nebo poznámku k~distribuci
  \TL{}, její instalaci nebo dokumentaci,
  diskusní skupina k~tomu určená je \email{tex-live@tug.org}.
  Pokud však je vaše otázka specifická pro program na
  \TL{} umístěný, napište prosím přímo autorovi nebo do diskusní
  skupiny určené pro tento program. Spuštění programu 
  s~volbou \code{-{}-help} nezřídka
  poskytuje adresu pro zasílání chybových hlášení.
\end{description}

\noindent
Druhou stranou mince je odpovídání na dotazy těch, kteří kladou otázky.
 %\ctt{} i~list \code{texhax} 
Všechny výše uvedené zdroje jsou otevřeny pro kohokoliv.
Přihlaste se, čtěte a začněte odpovídat tam, kde můžete. 

% In the next headline 
% don't use \TL so the \uppercase in the headline works. Also so
% tex4ht ends up with the right TeX.  Likewise the \protect's.
\section{Přehled \protect\TeX Live}
\label{sec:overview.tl}

Tento oddíl popisuje obsah \TL{} a \TKCS{}, jejíž je částí.

%Dvě hlavní instalační dávky pro Unix a \MacOSX{} jsou
%\texttt{install-tl.sh} a \texttt{install-pkg.sh}. Jsou
%popsány v~oddíle~\ref{sec:unix.install} na
%straně~\pageref{sec:unix.install}.
%V~dalším oddíle popíšeme strukturu a obsah \TL{}.


\subsection{Kolekce \protect\TeX{}u: \protect\TL, pro\protect\TeX{}t, Mac\protect\TeX}
\label{sec:tlcoll-dists}

\DVD{} \TKCS{} zahrnuje následující:

\begin{description}
\item [\TL] Úplný systém \TeX{}, k~nainstalování na disk.
Domovská stránka: \url{https://tug.org/texlive/}.

\item [Mac\TeX] pro \MacOSX, přidává nativní \MacOSX\ 
(aktuálně pojmenován macOS by Apple, avšak v~tomto dokumentu budeme používat starý název) 
instalační program a jiné aplikace Mac k~\TL{}. Domovská stránka:
\url{https://tug.org/mactex/}.

\item [pro\TeX{}t] Rozšíření systému \MIKTEX\ pro Windows.
\ProTeXt\ přidává k~\MIKTEX u nové doplňkové nástroje 
a zjednodušuje instalaci. Je plně nezávislý na \TL{} 
a má své vlastní příkazy. 
Domovská stránka: \url{https://tug.org/protext/}.

\item [CTAN] Výpis obrazovky archivu \CTAN{} (\url{https://ctan.org}).

\end{description}

\CTAN{}, \pkgname{protext} a \dirname{texmf-extra}
nemají stejné podmínky pro kopírování jako \TL{}, 
proto buďte pozorní při šíření nebo modifikaci.


\subsection{Popis kořenových adresářů \protect\TL{}}
\label{sec:tld}

%V~kořenovém adresáři distribuce \TL{} najdete následující podadresáře. 
Zde uvádíme stručný seznam a popis kořenových adresářů instalace \TL{}. 
%Na %\pkgname{live} 
%\TK{} \DVD\ je celá \TL{} hierarchie v~podaresáři 
%\dirname{texlive}, a ne v~kořenovém adresáři disku.

\begin{ttdescription}
\item[bin] Binárky systému \TeX{}, s~podadresáři dle platforem.
%
%\item[readme.html] Webovské stránky se stručnými úvody a užitečnými
%odkazy, v~různých jazycích.
\item[readme-*.dir] Stručný přehled a užitečné odkazy 
na \TL{}, v~různých jazycích, ve formátu \HTML{} a textovém.
%
\item[source] Zdrojové kódy všech programů, včetně základní distribuce
   \TeX{}u založené na \Webc{}.
%
\item[texmf-dist] Hlavní strom, viz \dirname{TEXMFDIST} níže.
%
\item[tlpkg] Skripty, programy a údaje pro správu instalace a  
přídavnou podporu pro Windows.
%Obsahuje rovněž neveřejné kopie Perlu a Ghostscriptu pro Windows, které jsou mimo
%\TL{} neviditelné, a nový prohlížeč PostScriptu pro Windows PSView.
\end{ttdescription}

\smallskip

Co se týče dokumentace, užitečné mohou 
být obsáhlé odkazy v~kořenovém souboru \OnCD{doc.html}.
Dokumentace téměř všeho (balíků, formátů, manuálů, man-stránek, info-souborů) 
je v~\dirname{texmf-dist/doc}.
K~vyhledání dokumentace na libovolném místě můžete 
použít programy \cmdname{texdoc} nebo \cmdname{texdoctk}. 

Samotná tato příručka \TL\ je k~dispozici v~několika 
jazycích v~adresáři \dirname{texmf-dist/doc/texlive}:
\begin{itemize*}
\item{česko-slovenská:} \OnCD{texmf-dist/doc/texlive/texlive-cz}
\item{německá:} \OnCD{texmf-dist/doc/texlive/texlive-de}
\item{anglická:} \OnCD{texmf-dist/doc/texlive/texlive-en}
\item{francouzská:} \OnCD{texmf-dist/doc/texlive/texlive-fr}
\item{italská:} \OnCD{texmf-dist/doc/texlive/texlive-it}
\item{japonská:} \OnCD{texmf-dist/doc/texlive/texlive-ja}
\item{polská:} \OnCD{texmf-dist/doc/texlive/texlive-pl}
\item{ruská:} \OnCD{texmf-dist/doc/texlive/texlive-ru}
\item{srbská:} \OnCD{texmf-dist/doc/texlive/texlive-sr}
\item{španělská:} \OnCD{texmf-dist/doc/texlive/texlive-es}
\item{zjednodušená čínština:} \OnCD{texmf-dist/doc/texlive/texlive-zh-cn}
\end{itemize*}

\subsection{Přehled předdefinovaných stromů texmf}
\label{sec:texmftrees}

Tento oddíl uvádí seznam předdefinovaných proměnných určujících 
texmf stromy používané systémem, a jejich zamýšlený účel
ve standardním uspořádání systému \TL. 
Povel \texttt{tlmgr~conf} ukáže hodnoty těchto proměnných.
Můžete tak jednoduše zjistit zda a jak tyto hodnoty odpovídají
nastavení jednotlivých adresářů ve vaší instalaci.

Všechny stromy, včetně osobních, musí dodržovat strukturu adresářů \TeX\
Directory Structure (\TDS, \url{https://tug.org/tds}), s~jejími nesčetnými 
podadresáři, jinak soubory nebudou k~nalezení. Podrobněji je to 
popsáno v~oddíle~\ref{sec:local_personal_macros} 
(\p.\pageref{sec:local_personal_macros}).
Pořadí zde je opačné vůči pořadí, ve kterém se stromy 
prohledávají, tj. pozdější stromy v~seznamu přepíšou předcházející. 

\begin{ttdescription}
\item [TEXMFDIST] Strom obsahující téměř všechny soubory původní 
distribuce~-- konfigurační soubory, pomocné skripty, balíky maker,
fonty atd. Hlavní výjimky tvoří binárky závislé na platformách, 
které jsou uloženy v~sourozeneckém adresáři \code{bin/}\,. 
%
\item [TEXMFSYSVAR] Strom (uživateli v~instalaci sdílený) 
používaný programy 
\verb+texconfig-sys+, \verb+updmap-sys+, \verb+fmtutil-sys+ 
a také \verb+tlmgr+, na (cache) uložení runtime
údajů, jako jsou soubory formátů a generované \code{.map} soubory.
%
\item [TEXMFSYSCONFIG] Strom (uživateli v~instalaci sdílený) 
používaný nástroji 
\verb+texconfig-sys+, \verb+updmap-sys+ a \verb+fmtutil-sys+ 
na uložení modifikovaných konfiguračních údajů.
%
\item [TEXMFLOCAL] Strom, který mohou použít administrátoři na instalaci 
doplňkových nebo upravených maker, fontů atd. pro celý systém.
%
\item [TEXMFHOME] Strom, který mohou použít uživatelé 
na svoje osobní doplňková nebo upravená makra, fonty atd.
Tato proměnná pro
každého uživatele ukazuje na jeho vlastní osobní adresář. 

%
\item [TEXMFVAR] Strom (soukromý) používaný programy \verb+texconfig+, 
\verb+updmap-user+ a \verb+fmtutil-user+ na (cache) uložení runtime
údajů, jako jsou soubory formátů a generované \code{.map} soubory.
%
\item [TEXMFCONFIG] Strom (soukromý) používaný nástroji  
\verb+texconfig+, \verb+updmap-sys+ a \verb+fmtutil-sys+ na uložení
modifikovaných konfiguračních údajů.
%
\item [TEXMFCACHE] Strom(y) používaný \ConTeXt{}extem 
MkIV a Lua\LaTeX{}em
na uložení (cache) runtime údajů; implicitně do \code{TEXMFSYSVAR}, 
nebo (pokud tento neumožňuje zápis) \code{TEXMFVAR}.
\end{ttdescription}

\noindent
Standardní struktura je:
\begin{description}
  \item[system-wide root] může obsáhnout vícenásobné vydání \TL{}
   (\texttt{/usr/local/texlive} ve výchozím nastavení pro Unix):
  \begin{ttdescription}
    \item[\lastyear] Předchozí vydání.
    \item[\thisyear] Aktuální vydání.
    \begin{ttdescription}
      \item [bin] ~
      \begin{ttdescription}
        \item [i386-linux] binárky systému \GNU/Linux (32-bitové)
        \item [...]
        \item [x86\_64-darwin] binárky systému \MacOSX
        \item [x86\_64-linux] binárky systému \GNU/Linux (64-bitové)        
        \item [win32] binárky systému Windows
      \end{ttdescription}
			\item [texmf-dist\ \ ]      \envname{TEXMFDIST} a \envname{TEXMFMAIN}
      \item [texmf-var \ \ ]      \envname{TEXMFSYSVAR}, \envname{TEXMFCACHE}
      \item [texmf-config]        \envname{TEXMFSYSCONFIG}
    \end{ttdescription}
    \item [texmf-local] \envname{TEXMFLOCAL}, zamýšlený k~zachování od vydání k~vydání. 
  \end{ttdescription}
  \item[domovský adresář uživatele] (\texttt{\$HOME} nebo
      \texttt{\%USERPROFILE\%})
    \begin{ttdescription}
      \item[.texlive\lastyear] Soukromě generované a konfigurační údaje předchozího vydání.
      \item[.texlive\thisyear] Soukromě generované a konfigurační údaje aktuálního vydání.
      \begin{ttdescription}
        \item [texmf-var\ \ \ ] \envname{TEXMFVAR}, \envname{TEXMFCACHE}
        \item [texmf-config] \envname{TEXMFCONFIG}
      \end{ttdescription}
    \item[texmf] \envname{TEXMFHOME} Osobní makra atd.
  \end{ttdescription}
\end{description}


\subsection{Rozšíření \protect\TeX{}u}
\label{sec:tex-extensions}

Samotný Knuthův původní \TeX{} je zmrazený, kromě ojedinělých oprav chyb.
Je v~\TL\ přítomen jako program \prog{tex} a tak to zůstane
v~dohledné budoucnosti. \TL{} obsahuje též několik rozšířených verzí 
\TeX{}u (známé také jako \TeX{}ovské stroje):

\begin{description}

\item [\eTeX] přidává množinu nových
příkazů (nazývaných \TeX{}ové primitivy).
\label{text:etex}
Nové příkazy se týkají například makroexpanze, načítání znaků,
tříd značek (marks), rozšířených ladicích možností
a rozšíření \TeXXeT{} pro obousměrnou sazbu. Implicitně
je \eTeX{} 100\% kompatibilní se standardním \TeX{}em.
Viz~\OnCD{texmf-dist/doc/etex/base/etex_man.pdf}. 
%B% \eTeX{} je nyní sázecím programem pro \LaTeX{} i pdf\LaTeX{}.

\item [pdf\TeX] vybudován na rozšířeních \eTeX{}u přidává podporu 
zápisu ve formátu PDF stejně jako v~\dvi{} a 
četná rozšíření netýkající se výstupu. 
Tento program je používán pro mnoho běžných formátů, 
například, \prog{etex}, \prog{latex}, \prog{pdflatex}.
Jeho stránka je \url{https://www.pdftex.org/}. Viz 
návod \OnCD{texmf-dist/doc/pdftex/manual/pdftex-a.pdf}
a \OnCD{texmf-dist/doc/pdftex/samplepdftex/samplepdf.tex}
pro vzorové použití některých jeho vlastností.  

\item [Lua\TeX] přidává podporu pro vstup Unicode a OpenType\slash TrueType
a systémová písma. Zahrnuje také interpret Lua
(\url{https://lua.org/}), umožňujícího řešení mnoha ožehavých \TeX ovských
problémů. Volaný povelem \filename{texlua} funguje jako samostatný
interpret Lua. Jeho web je \url{http://www.luatex.org/} a
referenční příručka je \OnCD{texmf-dist/doc/luatex/base/luatex.pdf}. 

\item [(e)(u)p\TeX] má nativní podporu pro požadavky japonské sazby; 
p\TeX\ je základní motor, zatímco e- varianty přidávají funkce \eTeX\ a u- přidává podporu Unicode. 

\item [\XeTeX] přidává podporu vstupního kódování
Unicode a OpenType\slash TrueType a systémových fontů, implementovaných
zejména použitím knihoven třetích stran, srv. \url{https://tug.org/xetex}.

\item [\OMEGA\ (Omega)] je založena na Unicode.  
Umožňuje sázet v~téměř všech světových jazycích zároveň.
Dociluje toho tzv. překladovými procesy (\OMEGA{} Translation
Processes, OTP) pro realizaci složitých transformací na jakémkoliv vstupu. 
Omega už není součástí \TL{} jako samostatný program;
poskytnutý je jenom Aleph:

\item [Aleph] kombinuje rozšíření \OMEGA\ a \eTeX.  
Viz \OnCD{texmf-dist/doc/aleph/base}.
\end{description}


\subsection{Další za zmínku stojící programy na \protect\TeXLive}

Na \TL{} najdete několik často používaných programů:

\begin{cmddescription}

\item [bibtex, biber] podpora práce se seznamem literatury.

\item [makeindex, upmendex, xindex, xindy] vytváření rejstříku.
Pro češtinu a slovenštinu však potřebujete verzi programu s~názvem
\texttt{csindex}.
Program zatím není součástí distribuce, je potřeba instalovat 
zvlášť z~\url{https://www.ctan.org/pkg/csindex}.

\item [dvips] pro konverzi \dvi{} do \PS{}.

\item [dvipdfmx] konvertor \dvi{} do PDF, alternativní
přístup vedle pdf\TeX{}u zmíněného výše.

\item [xdvi] prohlížeč \dvi{} pro systém X~Window.

\item [dviconcat, dviselect] pro kopii a vkládání stránek
do/z~\dvi{} souborů.

\item [psselect, psnup, \ldots] programy pro práci s~\PS{}em.

\item [pdfjam, pdfjoin, \ldots] pomůcky pro PDF.

\item [context, mtxrun] Con\TeX{}t a PDF procesor.

\item [htlatex, \ldots] \cmdname{tex4ht}: konvertor \AllTeX{}
do HTML (a~XML, DocX  a dalších formátů).

\end{cmddescription}

\htmlanchor{installation}
\section{Instalace}
\label{sec:install}

\subsection{Spuštění instalačního programu}
\label{sec:inst-start}

Pro začátek si obstarejte \TK{} \DVD{} nebo si stáhněte
síťový instalační program \TL{}.
Na \url{https://tug.org/texlive/acquire.html}
najdete více informací a další způsoby získání softwaru.

\begin{description}
\item [Síťový instalátor, .zip nebo .tar.gz:] stáhněte z~\CTAN{}u, 
z~adresáře \dirname{systems/texlive/tlnet}; url
\url{http://mirror.ctan.org/systems/texlive/tlnet}
by Vás měl přesměrovat na blízký, aktuální mirror.
Můžete získat \filename{install-tl.zip}, který může
být použit pod Unixem a Windows, nebo jenom pro Unix
podstatně menší \filename{install-unx.tar.gz}.
Po rozbalení se \filename{install-tl} a
\filename{install-tl-windows.bat} objeví v~podadresáři \dirname{install-tl}.

\item[Síťový .exe instalátor pod Windows:] stáhněte z~\CTAN{}u,
jak uvedeno výše. Dvojklikem spustíte prvotní 
instalátor a rozbalovač, jak vidíte na 
obrázku~\ref{fig:nsis}. Objeví se dvě volby:
„Install“ a „Unpack only“.

\item [\DVD{} \TeX{} kolekce:] vejděte do podadresáře \DVD{} \dirname{texlive}. 
  Pod Windows by se instalátor mohl spustit automaticky po vložení \DVD,
  pokud to není z~bezpečnostních důvodů zakázané, jinak musíte instalaci
  spustit ručně.
  \DVD\ můžete získat, když se stanete členem skupiny uživatelů \TeX u
  (vřele doporučujeme \CSTUG, \url{https://tug.org/usergroups.html}),
  nebo si ho zvlášť zakoupíte (\url{https://tug.org/store}).
Můžete si vypálit svoje vlastní \DVD\ z~\ISO\
obrazu staženého z~\CTAN{}u, \url{https://tug.org/texlive/acquire.html}.
Ve většině systémů můžete \ISO\ namontovat přímo. 
Jestliže máte po instalaci z~\DVD\ nebo \ISO{}
zájem o~pokračující aktualizace z~Internetu, 
nahlédněte, prosím, do oddílu~\ref{sec:dvd-install-net-updates}.
\end{description}

\newcommand\figdesc{První fáze instalátoru \code{.exe} pod Windows}
\begin{figure}[tb]
%\centering 
\tlpng{nsis_installer}{.6\linewidth}{\figdesc}
\caption{\figdesc. Stisknutím tlačítka Instalovat získáte
	okno zobrazené na obrázku~\ref{fig:basic-w32}.}
\label{fig:nsis}
\end{figure}

Nezávisle na zdroji se spouští tentýž instalátor.
Nejvíc znatelný rozdíl mezi oběma možnostmi je ten, že po skončení
instalace z~Internetu získáte balíky, které jsou v~současné 
době k~dispozici. To je v~protikladu k~\DVD\ a 
\ISO\ obrazům, které se mezi významnějšími vydáními neaktualizují.

Pokud potřebujete stahovat přes proxy server, použijte soubor
\filename{~/.wgetrc} nebo proměnné prostředí s~nastavením proxy pro Wget
(\url{https://www.gnu.org/software/wget/manual/html_node/Proxies.html}),
nebo ekvivalent za cokoli, stáhněte program, který používáte.
Samozřejmě se to netýká instalace z~\DVD\ nebo z~\ISO\ obrazu.

\noindent
Následující oddíly vysvětlují spuštění instalátoru podrobněji.

\subsubsection{Unix}

Dále \texttt{>} označuje výzvu (prompt shellu); vstup 
uživatele je \Ucom{\texttt{zvýrazněn}}.
Program \filename{install-tl} je skript v~jazyce Perl. Nejjednodušší
způsob jeho spuštění v~unixovém systému je následující:
\begin{alltt}
> \Ucom{perl /path/to/installer/install-tl}
\end{alltt}
(Nebo můžete vyvolat \Ucom{/path/to/installer/install-tl}, když je spustitelný, 
nebo nejdříve použijte \texttt{cd} do adresáře atd.; 
nechceme opakovat všechny tyto variace.)
Možná zvětšíte okno terminálu tak, aby ukazovalo celou 
obrazovku textového instalátoru (obr.~\ref{fig:textmain}).

K~instalaci v~režimu \GUI\ (obr.~\ref{fig:advanced-lnx}) 
budete potřebovat nainstalovaný Tcl/Tk. Pak můžete spustit:
\begin{alltt}
> \Ucom{perl install-tl -gui}
\end{alltt}

Staré volby \code{-wizard} a \code{-perltk}/\code{-expert} nyní dělají tytéž věci jako \code{-gui}.
Úplný seznam různých voleb získáte povelem:
\begin{alltt}
> \Ucom{perl install-tl -help}
\end{alltt}

\textbf{O oprávněních Unixu:} Vaše nastavení \code{umask} v~čase instalace 
bude respektováno instalačním programem \TL{}. Proto když chcete, aby byla Vaše 
instalace použitelná i jinými uživateli než Vámi, ujistěte se, že jsou Vaše
nastavení dostatečně tolerantní, například, \code{umask
002}. Další informace o~nastavení \code{umask} hledejte v~dokumentaci k~Vašemu systému.

\textbf{Zvláštní vysvětlivky pro Cygwin:} Na rozdíl od jiných 
unixových systémů Cygwin implicitně
neobsahuje všechny nezbytné programy, které instalátor \TL{} potřebuje. 
Viz sekci~\ref{sec:cygwin}.

\subsubsection{\MacOSX}
\label{sec:macosx}

Jak již bylo zmíněno v~sekci~\ref{sec:tlcoll-dists}, pro \MacOSX\ je 
připravena samostatná distribuce, nazvaná 
Mac\TeX\ (\url{https://tug.org/mactex}).
Doporučujeme použít původní instalační program 
Mac\TeX u namísto instalátoru \TL\
pod \MacOSX, protože původní (nativní) instalátor provede  
několik nastavení specifických pro Mac, zejména umožňuje 
snadné přepínání mezi různými vydáními \TL\ na počítačích Mac, pomocí %takzvané 
datové struktury \TeX{}Dist. 

Mac\TeX\ je silně založen na \TL a hlavní \TeX ovská stromová 
struktura a binárky jsou identické. Přidává několik dalších adresářů 
s~dokumentací a aplikacemi specifickými pro Mac.

\subsubsection{Windows}\label{sec:wininst}

Jestliže používáte nerozbalený stažený .zip soubor nebo pokud se instalační 
program \DVD\ nespustí automaticky, klikněte 
dvakrát na soubor \filename{install-tl-windows.bat}. 

Můžete také spustit instalační program z~příkazového řádku.
Dále \texttt{>} označuje prompt shellu; 
vstup uživatele je \Ucom{\texttt{polotučný}}. 
Pokud jste v~adresáři instalačního programu, jenom spusťte:
\begin{alltt}
> \Ucom{install-tl-windows}
\end{alltt}

Můžete také zadat absolutní cestu, jako například:
\begin{alltt}
> \Ucom{D:\bs{}texlive\bs{}install-tl-windows}
\end{alltt}
pro \TKCS\ \DVD, za předpokladu, že \dirname{D:} je optický disk.
Obr.~\ref{fig:basic-w32} zobrazuje základní obrazovku průvodcovského 
instalátoru, který je pro Windows implicitní.

Pro instalaci v~textovém režimu použijte:
\begin{alltt}
> \Ucom{install-tl-windows -no-gui}
\end{alltt}

Pro úplný seznam různých voleb zadejte:
\begin{alltt}
> \Ucom{install-tl-windows -help}
\end{alltt}

\textbf{Poznámka.} Pokud stejný adresář obsahuje také \texttt{install-tl-windows.exe}, přidejte příponu \texttt{.bat}. 
Obvykle tomu tak nebude (pokud jste nezrcadlili adresář \dirname{tlnet} lokálně).

\begin{figure}[tb]
\begin{boxedverbatim}
Installing TeX Live 2022 from: ...
Platform: x86_64-linux => 'GNU/Linux on x86_64'
Distribution: inst (compressed)
Directory for temporary files: /tmp
...
Detected platform: GNU/Linux on Intel x86_64

<B> binary platforms: 1 out of 16

<S> set installation scheme: scheme-full

<C> customizing installation collections
    40 collections out of 41, disk space required: 7239 MB

<D> directories:
    TEXDIR (the main TeX directory):
    /usr/local/texlive/2022
   ...

 <O> options:
   [ ] use letter size instead of A4 by default
   ...
 
 <V> set up for portable installation
 
Actions:

 <I> start installation to hard disk
 <P> save installation profile to 'texlive.profile' and exit
 <H> help
 <Q> quit
\end{boxedverbatim}
\vskip-.5\baselineskip
\caption{Hlavní obrazovka textového instalačního programu (\GNU/Linux)}%
\label{fig:textmain}
\end{figure}

% nebudeme překládat
\begin{figure}[tb]
	\tlpng{basic-w32}{.6\linewidth}{Základní obrazovka instalátoru (Windows)}
	\caption{Základní obrazovka instalátoru (Windows); tlačítko \singleuv{Advanced} 
	způsobí něco jako na obr.~\ref{fig:advanced-lnx}}
	\label{fig:basic-w32}
\end{figure}

\begin{figure}[tb]
\tlpng{advanced-lnx}{\linewidth}{Obrazovka pokročilého \GUI{} instalátoru (\GNU/Linux)}
\caption{Obrazovka pokročilého \GUI{} instalátoru (\GNU/Linux)}
\label{fig:advanced-lnx}
\end{figure}

\htmlanchor{cygwin}
\subsubsection{Cygwin}
\label{sec:cygwin}

Před začátkem instalace použijte program Cygwinu 
\filename{setup.exe} k~instalaci programů 
\filename{perl} a \filename{wget}, pokud jste tak ještě neudělali.
Doporučené jsou následující doplňkové balíky:
\begin{itemize*}
\item \filename{fontconfig} [potřebný pro \XeTeX a Lua\TeX]
\item \filename{ghostscript} [potřebný pro různé pomůcky]
\item \filename{libXaw7} [potřebný pro \code{xdvi}]
\item \filename{ncurses} [umožní příkaz 
      \code{clear} používaný instalátorem]
\end{itemize*}

\subsubsection{Textový instalační program}

Obrázek~\ref{fig:textmain} ukazuje základní obrazovku textového režimu pod Unixem.
Pro Unix je textový instalační program nastaven implicitně.

Je to instalátor jenom s~příkazovým řádkem; vůbec nemá kurzorovou podporu.
Nemůžete se například pohybovat v~zatrhávacích rámečcích nebo vstupních polích. 
Jenom něco napíšete (s~rozlišováním velikosti písma) na příkazovém řádku a 
stlačíte klávesu Enter, poté se celá obrazovka přepíše s~přizpůsobeným obsahem.

Rozhraní textového instalátoru je tak primitivní z~prostého důvodu: je
navržené tak, aby se dalo spustit na tolika platformách, jak je to jen
možné, dokonce i~s~minimálním Perlem.

\subsubsection{Grafický instalační program}
\label{sec:graphical-inst}

Implicitní grafický instalátor začíná jednoduše, pouze s~několika
volbami; viz obr.~\ref{fig:basic-w32}.
Může být vyvolán pomocí
\begin{alltt}
	> \Ucom{install-tl -gui}
\end{alltt}
Tlačidlo \singleuv{Advanced} dává přístup ke většině voleb textového
instalátoru; viz obr.~\ref{fig:advanced-lnx}.

\subsubsection{Starší instalátory}

Režimy \texttt{perltk}/\texttt{expert} a \texttt{wizard} jsou ještě pořád 
k~dispozici pro systémy s~nainstalovaným Perl/Tk. Můžou být specifikované 
pomocí argumentů \texttt{-gui=perltk} resp. \texttt{-gui=wizard}.


\subsubsection{Jednoduchý průvodce instalací}

Pod Windows je implicitně nastaveno spuštění nejjednoduššího 
instalačního způsobu, který můžeme doporučit, 
nazvaného \uv{průvodce} instalací. %(obr.~\ref{fig:wizard-w32}).
Nainstaluje všechno a nezadává skoro žádné otázky. Pokud si 
chcete veškeré nastavení upravit, musíte spustit některý 
z~dalších instalátorů.

Pro jiné platformy může být tento režim vyvolán explicitně povelem
\begin{alltt}
> \Ucom{install-tl -gui=wizard}
\end{alltt}

\subsection{Spuštění instalačního programu}
\label{sec:runinstall}

Instalátor je zamýšlený jako co nejvíce samovysvětlující. 
Nicméně nyní následuje několik poznámek o~jednotlivých volbách 
a dílčích nabídkách:

\subsubsection{Nabídka binárních systémů (pouze Unix)}
\label{sec:binary}

\begin{figure}[tb]
\begin{boxedverbatim}
Available platforms:
===============================================================================
a [ ] Cygwin on Intel x86 (i386-cygwin)
b [ ] Cygwin on x86_64 (x86_64-cygwin)
c [ ] MacOSX current (10.14-) on ARM/x86_64 (universal-darwin)
d [ ] MacOSX legacy (10.6-) on x86_64 (x86_64-darwinlegacy)
e [ ] FreeBSD on x86_64 (amd64-freebsd)
f [ ] FreeBSD on Intel x86 (i386-freebsd)
g [ ] GNU/Linux on ARM64 (aarch64-linux)
h [ ] GNU/Linux on ARMhf (armhf-linux)
i [ ] GNU/Linux on Intel x86 (i386-linux)
j [X] GNU/Linux on x86_64 (x86_64-linux)
k [ ] GNU/Linux on x86_64 with musl (x86_64-linuxmusl)
l [ ] NetBSD on x86_64 (amd64-netbsd)
m [ ] NetBSD on Intel x86 (i386-netbsd)
o [ ] Solaris on Intel x86 (i386-solaris)
p [ ] Solaris on x86_64 (x86_64-solaris)
s [ ] Windows (win32)
\end{boxedverbatim}
\vskip-.5\baselineskip
\caption{Nabídka binárek}\label{fig:bin_text}
\end{figure}

Obrázek~\ref{fig:bin_text} ukazuje nabídku binárek textového
režimu. Standardně budou nainstalovány jenom binárky 
vaší aktuální platformy. Z~této nabídky si rovněž můžete
vybrat instalaci binárek pro jiné platformy. Toto 
může být užitečné, pokud sdílíte \TeX ovský strom
v~síti heterogenních strojů, nebo na systému
s~dvojitým zaváděcím procesem.


\subsubsection{Volba obsahu instalace}
\label{sec:components}

\begin{figure}[tbh]
\begin{boxedverbatim}
Select a scheme:
====================================================================
 a [X] full scheme (everything)
 b [ ] medium scheme (small + more packages and languages)
 c [ ] small scheme (basic + xetex, metapost, a few languages)
 d [ ] basic scheme (plain and latex)
 e [ ] minimal scheme (plain only)
 f [ ] ConTeXt scheme
 g [ ] GUST TeX Live scheme
 h [ ] infrastructure-only scheme (no TeX at all)
 i [ ] teTeX scheme (more than medium, but nowhere near full)
 j [ ] custom selection of collections
\end{boxedverbatim}
\vskip-.5\baselineskip
\caption{Nabídka schémat}\label{fig:schemetext}
\end{figure}

Obrázek~\ref{fig:schemetext} ukazuje nabídku schémat \TL; tady
vybíráte \uv{schéma}, což je souhrn kolekcí balíků.
Předvolené schéma \optname{full} nainstaluje vše, 
co je k~dispozici. To doporučujeme, avšak
můžete také zvolit schéma \optname{basic} 
pro pouze plain a \LaTeX, \optname{small} pro několik málo dalších programů
(ekvivalentní s~takzvanou instalací Basic\TeX\ Mac\TeX{}u),
\optname{minimal} pro účely testování, a schéma \optname{medium} nebo
\optname{teTeX} pro získání něčeho mezi tím. K~dispozici jsou také různá
specializovaná schémata a schémata specifická pro některé země.

\begin{figure}[tb]
\def\figdesc{Nabídka kolekcí}
\centering \tlpng{stdcoll}{.7\linewidth}{\figdesc}
\caption{\figdesc}\label{fig:collections.gui}
\end{figure}

Svůj výběr schématu můžete upřesnit pomocí nabídky \singleuv{collections} 
(obrázek~\ref{fig:collections.gui}, ukázáno pro změnu v~režimu \GUI).

Kolekce jsou o~jednu úroveň podrobnější než schémata -- v~podstatě
je schéma tvořeno několika kolekcemi, kolekci tvoří jeden nebo více
balíků, a balík (nejnižší úroveň seskupování v~\TL) obsahuje vlastní
soubory \TeX ovských maker, soubory fontů atd. 

Pokud chcete získat větší kontrolu, než jakou poskytuje nabídka kolekcí, 
po instalaci můžete použít program \TeX\ Live Manager (\prog{tlmgr}) 
(viz sekci~\ref{sec:tlmgr}); jeho použitím můžete řídit 
instalaci na úrovni balíků.

\subsubsection{Adresáře}
\label{sec:directories}

Standardní uspořádání je popsáno v~sekci~\ref{sec:texmftrees},
\p.\pageref{sec:texmftrees}. 

Standardní umístění instalačního adresáře je
\dirname{/usr/local/texlive/2022} pro Unix
a \verb|C:\texlive\2022| pod Windows.
Toto uspořádání umožňuje mít mnoho paralelních instalací \TL, jednu pro každé vydání
(typicky podle roku, jako tady),
a můžete mezi nimi přepínat pouhou změnou vyhledávací cesty.

Tento instalační adresář může být přepsán nastavením proměnné \dirname{TEXDIR} v~instalátoru.
Obrazovka \GUI\ pro toto a další nastavení je ukázána na obrázku~\ref{fig:advanced-lnx}.
Hlavní důvod pro změnu této předvolby je nedostatek diskového prostoru v~této části
(úplný \TL\ potřebuje několik gigabytů) nebo nedostatek práv
na zápis pro standardní umístění.
Nemusíte být zrovna rootem nebo administrátorem, když 
instalujete \TL, ale potřebujete oprávnění na zápis do cílového adresáře.

V systému Windows obvykle nemusíte být správcem
na vytvoření \verb|C:\texlive\2022| (nebo obecněji,
\verb|%SystemDrive%\texlive\2022|). 

Instalační adresáře mohou být také změněny nastavením různých proměnných 
prostředí před spuštěním instalátoru (pravděpodobně
\envname{TEXLIVE\_INSTALL\_PREFIX} nebo
\envname{TEXLIVE\_INSTALL\_TEXDIR}); viz dokumentaci z~|install-tl --help| 
(dostupná online na
\url{https://tug.org/texlive/doc/install-tl.html}) k~získání úplného seznamu 
nebo dalších detailů.

Rozumnou alternativou je adresář uvnitř vašeho domovského
adresáře, zvlášť když chcete být výhradním uživatelem. 
K~označení domovského adresáře použijte vlnku, `|~|',
například `|~/texlive/2022|'.

Doporučujeme do názvu začlenit rok, což umožní paralelní
společné zachování různých vydání \TL{} vedle sebe. Můžete 
také chtít udržovat název nezávislý na verzi, například 
\dirname{/usr/local/texlive-cur} pomocí symbolického odkazu, 
který může být později přepsán po přezkoušení nového vydání.

Změna \dirname{TEXDIR} v~instalačním programu vyvolá také změny 
\dirname{TEXMFLOCAL}, \dirname{TEXMFSYSVAR} a
\dirname{TEXMFSYSCONFIG}.

\dirname{TEXMFHOME} je doporučené umístění osobních souborů maker
 nebo balíků.  Předvolená hodnota je \verb|~/texmf| (|~/Library/texmf| na Macs). 
Na rozdíl od \dirname{TEXDIR} je nyní \verb|~| uchována 
v~nově vytvořených konfiguračních souborech, 
protože to užitečně odkazuje na domovský adresář kteréhokoliv uživatele \TeX u. 
Expanduje se na \dirname{$HOME} pod Unixem a \verb|%USERPROFILE%| pod Windows.
Zvláštní poznámka: \envname{TEXMFHOME}, jako všechny
stromy, musí být uspořádaný v~souladu s~\TDS, jinak
nemusí být soubory k~nalezení.

\dirname{TEXMFVAR} je umístění pro uložení většiny  
průběžně generovaných dočasných dat specifických pro každého uživatele. 
\dirname{TEXMFCACHE} je název proměnné, která se používá pro tento účel 
Lua\LaTeX{}em a \ConTeXt{}em 
MkIV (viz oddíl~\ref{sec:context_mkiv},
na straně~\pageref{sec:context_mkiv}); její implicitní hodnota je 
\dirname{TEXMFSYSVAR}, nebo (pokud tato neumožňuje zápis) \dirname{TEXMFVAR}.


\subsubsection{Volby}
\label{sec:options}

\begin{figure}[tbh]
\begin{boxedverbatim}
Options customization:
===============================================================================
 <P> use letter size instead of A4 by default: [ ]
 <E> execution of restricted list of programs: [X]
 <F> create all format files:                  [X]
 <D> install font/macro doc tree:              [X]
 <S> install font/macro source tree:           [X]
 <L> create symlinks in standard directories:  [ ]
            binaries to:
            manpages to:
                info to:
 <Y> after install, set CTAN as source for package updates: [X]								
\end{boxedverbatim}
\vskip-.5\baselineskip
\caption{Nabídka voleb (Unix)}\label{fig:optionstext}
\end{figure}

Obrázek~\ref{fig:optionstext} ukazuje nabídku voleb textového režimu.
Další informace o~každé volbě:

\begin{description}
\item[use letter size instead of A4 by default:] Výběr standardní velikosti papíru.  
Jednotlivé dokumenty mohou a měly by deklarovat zvláštní 
rozměr papíru, pokud je to žádoucí.

\item[execution of restricted list of programs:] 
Od \TL\ 2010 je implicitně povoleno vykonávání
několika externích programů. Velmi neúplný seznam 
povolených programů je uveden v~souboru \filename{texmf.cnf}.
Pro další podrobnosti viz novinky 2010 (oddíl~\ref{sec:2010news}).

\item[create all format files:] Doporučujeme ponechat tuto možnost
zaškrtnutou, abyste předešli zbytečným problémům při dynamickém vytváření formátů. 
Další podrobnosti najdete v~dokumentaci \prog{fmtutil}. 

\item[install font/macro \ldots\ tree:] Stahování/instalace 
dokumentace a zdrojových souborů ve většine balíků.
Nedoporučuje~se vypustit.

\item[create symlinks in standard directories:] 
Tato volba (pouze Unix) obchází potřebu změny proměnných prostředí. Bez této 
volby je obvykle potřebné přidat adresáře \TL{} do proměnných
\envname{PATH}, \envname{MANPATH} a \envname{INFOPATH}. 
Budete muset přidělit práva na zápis cílovým adresářům. Tato volba je 
určena pro zpřístupnění systému \TeX\ pomocí adresářů, 
které již uživatelé znají, jako například \dirname{/usr/local/bin}, 
které neobsahují žádné \TeX ovské soubory. Důrazně doporučujeme \emph{ne}přepsat 
stávající soubory vašeho \TeX ovského systému, 
který přišel s~touto volbou, tj. specifikací systémových adresářů.
Nejbezpečnější a doporučený přístup je ponechat volbu neoznačenou.

\item[after install, set CTAN as source for package updates:]
Pro instalaci z~\DVD je
tato volba implicitně umožněna, protože uživatel obvykle chce uskutečnit
následné aktualizace balíků z~archivu \CTAN, kde jsou průběžně 
aktualizovány po celý rok. Jediný důvod pro jejich potlačení 
přicháí v~úvahu, pokud instalujete jenom část z~\DVD\ a plánujete
rozšírit instalaci později. V~každém případě úložiště balíku pro
instalátor a pro poinstalační aktualizace mohou být nastaveny nezávisle
podle potřeby; viz oddíl~\ref{sec:location}
 a oddíl~\ref{sec:dvd-install-net-updates}.
\end{description}


Volby pro Windows, zobrazené v~grafickém instalátoru \GUI{} pro 
znalce:
\begin{description}
\item[adjust searchpath] 
Tohle zabezpečí, že všechny programy uvidí binární adresář 
\TL{} v~seznamu cest spustitelných programů.

\item[add menu shortcuts] V~případě nastavení vznikne ve Start menu
další položka \TL{} podmenu. Kromě `TeX Live menu' a `No shortcuts' 
zde bude třetí volba `Launcher entry'. Tato volba je popsána
v~oddíle~\ref{sec:sharedinstall}.

\item[File associations] Volby jsou `Only new' (vytvoření 
pouze nových souborových asociací, bez přepsání stávajících), `All' a `None'.

\item[install \TeX{}works front end]
\end{description}
Když jsou všechna nastavení podle vašich preferencí, stačí napsat
\singleuv{I} v~textovém okně, nebo stisknout tlačítko `Install' 
v~\GUI{} a spustit instalační proces. Po dokončení přeskočte na
sekci~\ref{sec:postinstall}, kde se dozvíte, co se případně
má udělat nakonec. 

\subsection{Volby příkazového řádku pro install-tl}
\label{sec:cmdline}

K~zobrazení voleb příkazového řádku napište
\begin{alltt}
> \Ucom{install-tl -help}
\end{alltt}
K~uvedení názvu volby mohou být použity |-| nebo také |--|.
Následují nejběžnější volby:

\begin{ttdescription}
\item[-gui] Použijte \GUI{} instalátor pokud je to možné. Toto si vyžaduje
Tcl/Tk ve verzi 8.5 nebo vyšší. Toto bylo distribuováno s~\MacOSX, až do
Monterey; poté budete muset nainstalovat Tcl/Tk sami, pokud nezvolíte
použití instalačního programu Mac\TeX\. Tcl/Tk je 
distribuován s~\TL{} pod Windows. Starší volby \texttt{-gui=perltk} 
a \texttt{-gui=wizard} jsou stále ještě k~dispozici, ale vyvolají stejné \GUI{} rozhraní; 
pokud Tcl/Tk a Perl/Tk nejsou k~dispozici, pokračuje instalace 
v~textovém režimu.   

\item[-no-gui] Vynutí si použití instalátoru v~textovém režimu.

%\item[-lang {\sl LL}] Specify the installer interface
%  language as a standard (usually two-letter) code.  The installer tries
%  to automatically determine the right language but if it fails, or if
%  the right language is not available, then it uses English as a
%  fallback.  Run \verb|install-tl --help| to get the list of available
%  languages.

\item[-lang {\sl LL}] Specifikuje jazyk instalačního rozhraní jako 
  jeho standardní, obvykle dvoupísmený, kód. Instalátor se pokusí 
  automaticky určit správný jazyk, ale když selže nebo když správný jazyk není
  k~dispozici, použije angličtinu jako nouzové řešení. 
  Pro získání seznamu všech podporovaných jazyků spusťte
  \verb|install-tl --help|.

\item[-portable] Instalace pro přenosné použití, například na klíč \USB{}.
  Dá se zvolit rovněž v~textovém instalátoru pomocí příkazu \code{V}
  a z~instalátoru \GUI{}. Viz oddíl~\ref{sec:portable-tl}.

\item[-profile {\sl soubor}] Načtěte instalační profilový 
  soubor a proveďte instalaci bez interakce s~uživatelem.
  Instalační program vždy uloží soubor
  \filename{texlive.profile} do podadresáře \dirname{tlpkg} vaší instalace.  
  Tento soubor může být zadán jako argument například pro znovuvytvoření
  identické instalace na jiném systému. Nebo můžete  
  použít uživatelský profil, který nejjednodušeji 
  vytvoříte změnou hodnot vygenerovaného
  souboru, nebo odstartováním s~prázdným souborem, 
  který převezme všechny předvolby.

\item [-repository {\sl soubor-nebo-adresář}] Určuje 
  repozitář balíků, z~kterého se má instalovat; viz následující oddíl.
  \htmlanchor{opt-in-place}

\item[-in-place] (Dokumentováno pouze pro úplnost: nepoužívejte, pokud
  si nejste jisti tím, co děláte!) 
  Pokud již máte \prog{rsync}, \prog{svn} nebo jinou kopii \TL{} (viz
  \url{https://tug.org/texlive/acquire-mirror.html}) tehdy tato volba
  použije ta data, která již máte stažena, a vykoná pouze nezbytné činnosti
  po instalaci. Upozorňujeme, že soubor \filename{tlpkg/texlive.tlpdb}
  může být přepsán; jeho uložení zůstává na vaší odpovědnosti. Také odstranění 
  balíku se musí vykonat ručně. Tato volba nemůže být zapnuta pomocí 
  rozhraní instalátoru.
\end{ttdescription}


\subsubsection{Volba \optname{-repository}}
\label{sec:location}

Implicitní síťový repozitář balíků je zrcadlo CTAN zvolené automaticky
použitím \url{http://mirror.ctan.org}.

Pokud ho chcete přepsat, může být hodnotou umístění 
adresa url s~\texttt{ftp:}, \texttt{http:}, \texttt{https:}, 
\texttt{file:/} na začátku nebo jednoduchá cesta k~adresáři.  
(Při zadání umístění \texttt{http:}, \texttt{https:} nebo \texttt{ftp:} 
jsou koncové znaky \singleuv{\texttt{/}} a/nebo koncová složka
 \singleuv{\texttt{/tlpkg}} ignorovány.)

Kupříkladu můžete zvolit určité zrcadlo \CTAN\ něčím 
jako: \url{http://ctan.example.org/tex-archive/systems/texlive/tlnet/},
s~nahrazením |ctan.example.org/tex-archive| skutečným 
hostitelským jménem (hostname) a jeho konkrétní kořenovou 
cestou k~\CTAN\ (jako třeba |ftp.cstug.cz/pub/CTAN|).   
Seznam zrcadel \CTAN\ je udržován na \url{https://ctan.org/mirrors}.

Pokud je zadaný argument lokální (buď cesta nebo \texttt{file:/} url),
jsou použity komprimované soubory v~podadresáři \dirname{archive}
cesty repozitáře (i kdyby byly rovněž k~dispozici nekomprimované soubory).

\htmlanchor{postinstall}
\subsection{Poinstalační činnosti}
\label{sec:postinstall}

Po instalaci mohou být požadovány některé další instalace.

\subsubsection{Proměnné prostředí pro Unix}
\label{sec:env}

Pokud se rozhodnete vytvořit symbolické odkazy v~standardních adresářích
(popsaných v~oddíle~\ref{sec:options}), pak není nutná editace proměnných
prostředí. Jinak v~systémech Unix musí být adresář binárek pro vaši platformu
přidán k~prohledávaným cestám. (Ve Windows se o~to postará instalátor.)

Každá podporovaná platforma má svůj vlastní podadresář 
pod \dirname{TEXDIR/bin}. 
Seznam podadresářů a odpovídajících platforem je na 
obrázku~\ref{fig:bin_text}.

Nepovinně můžete rovněž přidat dokumentační manuálové
stránky (man pages) a adresáře Info k~jejich 
příslušejícím vyhledávacím cestám, pokud chcete, 
aby je našly systémové nástroje.
Dokumentační stránky mohou být automaticky
nalezeny po přidání do proměnné \envname{PATH}.

For Bourne-compatible shells such as \prog{bash}, and using Intel x86
GNU/Linux and the \TL\ default directory setup as an example, the file to edit
might be \filename{$HOME/.profile} (or another file sourced by
\filename{.profile}), and the lines to add would look like this:


Pro shelly kompatibilní s~Bourneshell, jako je 
\prog{bash}, použijíc jako příklad Intel x86
GNU/Linux se standardním nastavením adresářů \TL, může 
být vhodné editovat soubor \filename{$HOME/.profile} a řádky, které je potřeba přidat, 
budou vypadat následovně:

\begin{sverbatim}
PATH=/usr/local/texlive/2022/bin/x86_64-linux:$PATH; export PATH
MANPATH=/usr/local/texlive/2022/texmf-dist/doc/man:$MANPATH; export MANPATH
INFOPATH=/usr/local/texlive/2022/texmf-dist/doc/info:$INFOPATH; export INFOPATH
\end{sverbatim}

Pro \prog{csh} nebo \prog{tcsh} je editovaný soubor 
typicky \filename{$HOME/.cshrc} a
řádky k~přidání mohou vypadat jako:
\begin{sverbatim}
setenv PATH /usr/local/texlive/2022/bin/x86_64-linux:$PATH
setenv MANPATH /usr/local/texlive/2022/texmf-dist/doc/man:$MANPATH
setenv INFOPATH /usr/local/texlive/2022/texmf-dist/doc/info:$INFOPATH
\end{sverbatim}

Pokud nejste na platformě \code{x86\_64-linux}, použijte příslušný
název platformy; podobně, pokud jste nenainstalovali ve výchozím
adresáři, změňte název adresáře. Instalační program \TL\ oznamuje
plné řádky k~použití na konci instalace.

Pokud již někde ve svých spouštěcích souborech máte nastavení \envname{PATH}, 
slučte v adresářích \TL\, jak uznáte za vhodné. 
% If you already have \envname{PATH} settings somewhere in your startup
% files, merge in the \TL\ directories as you see fit.


\subsubsection{Proměnné prostředí: globální konfigurace}
\label{sec:envglobal}
Volba, zda učinit tyto změny globálně, anebo pro uživatele právě
přidaného do systému, je na vás. Mezi různými systémy existuje 
příliš mnoho variant, kde a jak se tato nastavení provádějí.
Naše dvě rady jsou: 1)~můžete vyhledat soubor
\filename{/etc/manpath.config} a pokud existuje, přidejte řádky jako

\begin{sverbatim}
MANPATH_MAP /usr/local/texlive/2022/bin/x86_64-linux \
            /usr/local/texlive/2022/texmf-dist/doc/man
\end{sverbatim}

A za 2)~vyhledejte soubor \filename{/etc/environment},
který může definovat vyhledávací cestu a další standardní 
proměnné prostředí.

V~každém (Unixovém) adresáři binárek vytváříme také symbolický 
odkaz na adresář \dirname{texmf-dist/doc/man} s~názvem \code{man}.
Některé programy \code{man}, jako například standardní
\MacOSX\ \code{man}, ho automaticky najdou, což odstraňuje
potřebu jakéhokoliv nastavování dokumentačních stránek.

\subsubsection{Internetové aktualizace po instalaci z~\protect\DVD}
\label{sec:dvd-install-net-updates}

Pokud jste instalovali \TL\ z~\DVD\ a později si přejete
získat aktualizace z~internetu, budete potřebovat spuštění
tohoto povelu -- \emph{poté}, co jste aktualizovali vaši
vyhledávací cestu (jako to bylo popsané v~předcházejícím oddíle):

\begin{alltt}
> \Ucom{tlmgr option repository http://mirror.ctan.org/systems/texlive/tlnet}
\end{alltt}

Toto řekne programu \cmdname{tlmgr}, aby pro následující
aktualizace použil nejbližší zrcadlo \CTAN{}u.
To je implicitně nastaveno při instalaci
z~\DVD pomocí volby popsané v~sekci~\ref{sec:options}. 

Pokud se vyskytnou problémy s~automatickým výběrem zrcadla, 
můžete deklarovat konkrétní zrcadlo \CTAN{}u ze seznamu na stránce
\url{https://ctan.org/mirrors}. Použijte přesnou cestu k~podadresáři 
\dirname{tlnet} tohoto zrcadla, jak jsme uvedli výše.

\htmlanchor{xetexfontconfig}  % keep historical anchor working
\htmlanchor{sysfontconfig}
\subsubsection{Systémová konfigurace fontů pro \protect\XeTeX\protect\ a Lua\protect\TeX}
\label{sec:font-conf-sys}

\XeTeX\ a Lua\TeX\ mohou používat jakýkoli font instalovaný 
v~systému, nejenom ty, které se nacházejí
v~\TeX{}ovských stromech. Provádí to prostřednictvím 
podobných, ale ne identických metod.

Pod Windows jsou fonty dodané s~\TL\ automaticky
dostupné pro \XeTeX.
Na \MacOSX, na automatické hledání fontů dle jména
jsou potřeba další kroky; prosím vizte web Mac\TeX u
(\url{https://tug.org/mactex}). 
Pro ostatní unixové systémy potřebujete dokonfigurovat 
systém dle následujícího postupu tak, aby byl \XeTeX\ 
schopen najít fonty dodané s~\TL\ 
prostřednictvím vyhledávání jmen fontů systémem.

Pro usnadnění, když se instaluje balík \pkgname{xetex} 
(buď ve výchozí instalaci, nebo později), 
se vytváří potřebný konfigurační soubor 
\filename{TEXMFSYSVAR/fonts/conf/texlive-fontconfig.conf}.

Pro nastavení fontů \TL{} pro použití v~rámci 
celého systému (za předpokladu, že máte odpovídající oprávnění), 
postupujte následovně:
\begin{enumerate*}
\item Zkopírujte soubor \filename{texlive-fontconfig.conf} do adresáře
\dirname{/etc/fonts/conf.d/09-texlive.conf}.
\item Spusťte \Ucom{fc-cache -fsv}.
\end{enumerate*}
Pokud nemáte postačující práva k~provedení výše popsaných kroků,
nebo chcete-li učinit fonty \TL{} dosažitelné pro jediného uživatele, 
můžete učinit následující:
\begin{enumerate*}
\item Zkopírujte soubor \filename{texlive-fontconfig.conf} do
  \filename{~/.fonts.conf}, kde \filename{~} označuje váš 
  domovský adresář.
\item Spusťte \Ucom{fc-cache -fv}.
\end{enumerate*}

Pokud chcete uvidět jména všech systémových fontů, můžete 
spustit příkaz \code{fc-list}.  
Příkaz \code{fc-list : family style file spacing} 
(všechny argumenty jsou písmenkové řetězce) 
ukáže některou obecně zajímavou informaci.


\subsubsection{\protect\ConTeXt{} Mark IV}
\label{sec:context_mkiv}

\singleuv{Starý} \ConTeXt{} (Mark II) a \singleuv{nový} \ConTeXt{}
(Mark IV) by měly po instalaci \TL{} fungovat bez dalších zásahů
a neměly by vyžadovat zvláštní pozornost, pokud budete k~aktualizacím 
používat \verb+tlmgr+.

Protože však \ConTeXt{} MkIV nepoužívá knihovnu kpathsea, 
určité nastavení bude požadováno vždy, když budete instalovat nové 
soubory ručně (bez použití \verb+tlmgr+). Po ukončení takové instalace 
musí každý uživatel MkIV spustit:
\begin{sverbatim}
context --generate
\end{sverbatim}
pro obnovení diskové cache údajů \ConTeXt{}u.
Výsledné soubory jsou uloženy do proměnné \code{TEXMFCACHE},
jejíž přednastavená hodnota v~\TL\ je \verb+TEXMFSYSVAR;TEXMFVAR+. 

\ConTeXt\ MkIV bude číst ze všech cest uvedených 
v~\verb+TEXMFCACHE+ a zapisovat do první zapisovatelné cesty. 
Při čtení v~případě duplicitních údajů v~paměti cache získá
přednost poslední nalezený prvek.

Pro další informace viz
\url{https://wiki.contextgarden.net/Running_Mark_IV}.


\subsubsection{Začleňování lokálních a osobních maker}
\label{sec:local_personal_macros}

Toto je již implicitně zmíněno v~sekci~\ref{sec:texmftrees}:
adresář \dirname{TEXMFLOCAL} (standardně 
\dirname{/usr/local/texlive/texmf-local} nebo
\verb|%SystemDrive%\texlive\texmf-local| pod Windows)
je určen pro rozsáhlé systémové % system-wide 
lokální fonty a makra; adresář
\dirname{TEXMFHOME} (standardně \dirname{$HOME/texmf} nebo
\verb|%USERPROFILE%\texmf|) je určen pro osobní fonty a makra.
Pro oba stromy musí být soubory umístěny v~patřičných 
podadresářích \TDS\ (\TeX\ Directory Structure);
viz \url{https://tug.org/tds} nebo nahlédni do souboru
\filename{texmf-dist/web2c/texmf.cnf}. Například, \LaTeX{}ovský 
styl, třída nebo makrobalík by měl být umístěn
v~\dirname{TEXMFLOCAL/tex/latex} nebo
\dirname{TEXMFHOME/tex/latex}, nebo v~jejich podadresářích.

\dirname{TEXMFLOCAL} vyžaduje aktuální databázi jmen souborů, jinak
nebudou soubory nalezeny.
Můžete ji obnovit povelem \cmdname{mktexlsr} nebo použít tlačítko
\singleuv{Update file database} na záložce \singleuv{Actions} programu
\TeX\ Live Manager v~režimu \GUI.

Standardně je každá z~těchto proměnných definována jako 
samostatný adresář, jak je vidět. To však není nezbytně nutné. 
Pokud například potřebujete přepínat mezi dvěma verzemi velkých
balíků, můžete udržovat více stromů pro vaše vlastní potřeby.  
Toho dosáhnete nastavením \dirname{TEXMFHOME} na seznam adresářů uvnitř 
složených závorek oddělených čárkami:

\begin{verbatim}
  TEXMFHOME = {/my/dir1,/mydir2,/a/third/dir}
\end{verbatim}

Další popis expanze závorek je v~oddíle~\ref{sec:brace-expansion}.

\subsubsection{Začleňování fontů třetích stran}

Toto je naneštěstí nepříjemné téma pro \TeX\ a pdf\TeX. Zapomeňte na něj, 
pokud se nechcete probírat v~mnoha podrobnostech instalace \TeX u. Mnohé fonty již jsou zahrnuty v~\TL, proto
se podívejte, jestli chcete; \url{https://tug.org/FontCatalogue} je pohodlný
způsob, jak zobrazit písma dostupná na webu.

Pokud to potřebujete udělat, vyvinuli jsme maximální úsilí 
k~popsání postupu, viz \url{https://tug.org/fonts/fontinstall.html}.

Zvažte rovněž použití \XeTeX{}u nebo Lua\TeX{}u (viz
sekce~\ref{sec:tex-extensions}), které vám umožní používat provozní
systémová písma bez jakékoli instalace v~\TeX{}u. (Ale pozor, používání
systémových fontů obvykle způsobí, že zdrojáky vašich dokumentů budou nepoužitelné pro kohokoli v~jiném prostředí.) 
% your document sources

\subsection{Testování instalace}
\label{sec:test.install}

Po nainstalování \TL{} přirozeně chcete
systém otestovat, abyste mohli začít vytvářet nádherné dokumenty nebo fonty.

Jednou z~věcí, kterou byste mohli ihned hledat, je nástroj na editaci
souborů.  \TL{} nainstaluje \TeX{}works (\url{https://tug.org/texworks})
pro Windows (pouze) a Mac\TeX\ nainstaluje TeXShop
(\url{https://pages.uoregon.edu/koch/texshop}).  V~jiných Unixových systémech
je volba editoru ponechána na vás. Jsou k~dispozici mnohé možnosti, 
některé z~nich jsou uvedeny v~následujícím oddíle; viz
též \url{https://tug.org/interest.html#editors}. Bude fungovat libovolný
obyčejný editor; nic \TeX{}ovsky specifické se nevyžaduje.

Zbytek tohto oddílu udává některé základní postupy testování funkcionality
nového systému. Zde uvádíme příkazy Unixu; pod
\MacOSX{} nebo Windows pravděpodobně budete spouštět testy pomocí
grafického rozhraní, avšak principy jsou stejné.

\begin{enumerate}
\item Nejprve ověřte, zda se spustí program \cmdname{tex}:

\begin{alltt}
> \Ucom{tex -{}-version}
TeX 3.14159265 (TeX Live ...)
Copyright ... D.E. Knuth.
...
\end{alltt}
Pokud obdržíte hlášku s~\singleuv{command not found} místo výše 
uvedeného nebo se starší verzí, patrně nemáte nastavený správný podadresář
\dirname{bin} v~proměnné prostředí \envname{PATH}.  Vraťte se
k~informacím o~jejich nastavování na straně~\pageref{sec:env}.

\item Přeložte ukázkový soubor \LaTeX{}u a vytvořte PDF:
\begin{alltt}
> \Ucom{pdflatex sample2e.tex}
This is pdfTeX 3.14...
...
Output written on sample2e.pdf (3 pages, 142120 bytes).
Transcript written on sample2e.log.
\end{alltt}
Pokud selže nalezení souboru \filename{sample2e.tex} nebo jiných
souborů, můžete mít aktivní stará nastavení proměnných prostředí nebo
konfiguračních souborů; pro začátek doporučujeme zrušit 
nastavení všech proměnných prostředí souvisejících s~\TeX{}em.
Pro hlubší analýzu a dohledání problému můžete kdykoliv požádat
\TeX{} o~detaily toho, co a kde hledá: viz \uv{Ladicí činnosti}
na straně~\pageref{sec:debugging}.

\item Prohlédněte si PDF soubor, například:
\begin{alltt}
	> \Ucom{xpdf sample2e.pdf}
\end{alltt}
Mělo by se zobrazit nové okno s~pěkným dokumentem vysvětlujícím některé ze
základů \LaTeX{}u. (Mimochodem stojí za přečtení, pokud jste \TeX{}ovský nováček.)

Samozřejmě existuje mnoho dalších prohlížečů PDF; na unixových systémech se běžně používají
\cmdname {evince} a \cmdname {okular}. Pro Windows doporučujeme vyzkoušet Sumatra PDF
(\url{https://www.sumatrapdfreader.org/free-pdf-reader.html}). Žádné prohlížeče PDF nejsou součástí \TL{}, takže si musíte samostatně nainstalovat, co chcete
používat.

\item Samozřejmě stále můžete generovat původní \TeX{}ovský formát \dvi{}:
\begin{alltt}
> \Ucom {latex sample2e.tex}
\end{alltt}

\item A prohlédnout si \dvi{} na obrazovce:
\begin{alltt}
> \Ucom{xdvi sample2e.dvi}    # Unix
> \Ucom{dviout sample2e.dvi}  # Windows
\end{alltt}
Musíte mít spuštěny X~Window, aby \cmdname{xdvi} pracovalo. Pokud tomu tak není, nebo máte
špatně nastavenou proměnnou prostředí
\envname{DISPLAY}, dostanete chybovou hlášku \singleuv{Can't open display}.

\item Pro vytvoření \PS{}ového souboru z~\dvi{} použijte:
\begin{alltt}
> \Ucom{dvips sample2e.dvi -o sample2e.ps}
\end{alltt}

\item Nebo vytvořte PDF ze souboru \dvi{}, alternativní cestou k~použití
pdf\TeX{}u (nebo Xe\TeX{}u nebo Lua\TeX{}u), co může být někdy užitečné:
\begin{alltt}
	> \Ucom{dvipdfmx sample2e.dvi -o sample2e.pdf}
\end{alltt}

\item Další standardní testovací soubory, které mohou být 
užitečné kromě \filename{sample2e.tex}:

\begin{ttdescription}
\item [small2e.tex] Ukázkový dokument, ještě kratší než
\filename{sample2e}.
\item [testpage.tex] Test, jestli vaše tiskárna neposunuje tiskové zrcadlo.
\item [nfssfont.tex] Pro tisk tabulek fontů a testů fontů.
\item [testfont.tex] Pro totéž, ale pro plain \TeX{}.
\item [story.tex] Základní (plain) \TeX{}ový testovací soubor.
Musíte napsat \singleuv{\texttt{\ bye}} na výzvu \code{*} po \singleuv{\texttt{tex
story.tex}}.
\end{ttdescription}

\item Pokud máte nainstalovaný balík \filename{xetex} package, 
můžete prověřit jeho přístup k~systémovým fontům následovně:
\begin{alltt}
> \Ucom{xetex opentype-info.tex}
This is XeTeX, Version 3.14\dots
...
Output written on opentype-info.pdf (1 page).
Transcript written on opentype-info.log.
\end{alltt}

Jestliže obdržíte chybové hlášení \uv{Invalid fontname `Latin Modern
Roman/ICU'\dots}, pak potřebujete nakonfigurovat váš systém tak, aby 
fonty dodané s~\TL\ byly k~nalezení.  Viz
oddíl~\ref{sec:font-conf-sys}.
\end{enumerate}


\htmlanchor{uninstall}
\subsection{Odinstalování \TL}
\label{sec:uninstall}

Chcete-li odinstalovat \TL\ (po úspěšné instalaci) použijte: 

\begin{alltt}
	> \Ucom{tlmgr uninstall --all}
\end{alltt}

Budete požádáni o potvrzení, jinak se nic dělat nebude.
(Bez \code{-{}-all} se k~odstranění použije činnost \code{uninstall}
jednotlivé balíčky.) 

Tím se neodstraní adresáře specifické pro uživatele, konkrétně (viz také
oddíl~\ref{sec:texmftrees}): 

\begin{ttdescription}
\item [TEXMFCONFIG] To je určeno pro změny uživatelské konfigurace.
Pokud je chcete zachovat, před odstraněním sa ujistěte, že víte, jak je znovu vytvořit. 
	
\item [TEXMFVAR] To je určeno k~ukládání automaticky generovaných runtime dat, 
jako jsou lokální soubory formátů. Pokud jste je nepoužili
pro jiné účely, mělo by jejich odstranění být bezpečné. 
	
\item[TEXMFHOME] Obvykle obsahuje pouze soubory, které jste si sami nainstalovali,
které nejsou dostupné v~distribucích. Pravděpodobně toto nebudete chtít odstranit, pokud neplánujete úplně
přestat používat \TeX, nebo pokud nechcete začít znovu od nuly.
\end{ttdescription}

\noindent Cesty k adresářům pro tyto proměnné můžete najít spuštěním \code{kpsewhich -var-value=\ttvar{var}}.

Tato odinstalace \prog{tlmgr} také nezruší poinstalační
činnosti, jako jsou změny \envname{PATH} v~inicializačních souborech vašeho shellu 
a systémový přístup k fontům v~\TL\ (viz oddíl~\ref{sec:postinstall}). 
Takové akce musíte ručně zvrátit, pokud je to žádoucí. 

V~systému Windows lze odinstalaci provést pomocí \GUI; viz
oddíl~\ref{sec:winfeatures}. 


\subsection{Odkazy na doplňkový software s~možností stažení z~internetu}

Pokud jste \TeX{}ový začátečník nebo potřebujete pomoc s~psaním
\TeX{}ových, respektive \LaTeX{}ových dokumentů, navštivte
\url{https://tug.org/begin.html}, kde najdete úvodní informace
k~instalaci.

Odkazy na některé další pomůcky, o~jejichž instalaci můžete uvažovat:
\begin{description}
\item[Ghostscript] \url{https://ghostscript.com/}, bezplatný interpret PostScriptu a
PDF. 
\item[Perl] \url{https://perl.org/} 
      s~doplňujícími balíky z~CPAN, \url{https://cpan.org/}.
\item[ImageMagick] \url{https://imagemagick.org}, k~zpracování 
      a konverzi grafiky
\item[NetPBM] \url{http://netpbm.sourceforge.net}, rovněž pro grafiku.

\item[\TeX ovsky orientované editory] Existuje široký výběr a je to 
záležitost vkusu uživatele. Tady je výběr v~abecedním řazení
(několik málo je pouze pro Windows).
  \begin{itemize*}
\item \cmdname{GNU Emacs} je k~dispozici pro všechny hlavní platformy, viz
\url{https://www.gnu.org/software/emacs}.
\item \cmdname{AUC\TeX} běží pod Emacsem; je k dispozici přes správce balíčků Emacs \cmdname{ELPA}. 
Zdroje jsou také k~dispozici na CTAN. Domovská stránka AUC\TeX{}u je \url{https://www.gnu.org/software/auctex}. 
        
  \item \cmdname{SciTE} je k~dostání z~\url{https://www.scintilla.org/SciTE.html}.
  \item \cmdname{Texmaker} je volný (free) software, k~dispozici 
        z~\url{https://www.xmlmath.net/texmaker}.
  \item \cmdname{TeXstudio} začínalo jako odbočka programu \cmdname{Texmaker}
        s~dodatečnými rysy; dostupné z~\url{https://texstudio.org/}
        a v~distribuci pro\TeX{}t.
  \item \cmdname{TeXnicCenter} je volný software, 
        k~dispozici z~\url{https://www.texniccenter.org}.
  \item \cmdname{TeXworks} je volný software, k~dispozici 
        z~\url{https://tug.org/texworks} 
        a je nainstalovaný jako součást \TL{} pro Windows (pouze).
  \item \cmdname{Vim} je volný software, k~dispozici 
        z~\url{https://www.vim.org}.
  \item \cmdname{WinEdt} je shareware dostupný třeba
        na \url{https://tug.org/winedt} 
        nebo na \url{https://www.winedt.com}.
  \item \cmdname{WinShell} je k~dispozici z~\url{https://www.winshell.de}.
  \end{itemize*}
\end{description}
Pro mnohem delší seznam balíků a programů viz
\url{https://tug.org/interest.html}.

\section{Specializované instalace}

Předcházející oddíly popisovaly základní instalační proces.
Teď se zaměříme na některé speciální případy.

\htmlanchor{tlsharedinstall}
\subsection{Instalace sdílené uživateli}
\label{sec:sharedinstall}

\TL{} byl navržený tak, aby se dal sdílet mezi různými systémy na síti. 
Se standardní strukturou adresářů se nekonfigurují žádné 
pevné plné cesty: umístění souborů potřebných pro programy \TL{}
je zřízeno relativně k~programům.  Můžete ho 
najít v~nejdůležitějším konfiguračním souboru
\filename{$TEXMFDIST/web2c/texmf.cnf}, který obsahuje řádky jako jsou
\begin{sverbatim}
TEXMFROOT = $SELFAUTOPARENT
...
TEXMFDIST = $TEXMFROOT/texmf-dist
...
TEXMFLOCAL = $SELFAUTOGRANDPARENT/texmf-local
\end{sverbatim}
To znamená, že k~získání funkčního nastavení stačí 
přidat ke své vyhledávací cestě adresář binárek \TL{} 
pro jejich platformu.

Stejným způsobem můžete nainstalovat \TL{} lokálně a 
pak přesunout celou hierarchii později na místo v~síti.

Pro Windows \TL{} zahrnuje spouštěč \filename{tlaunch}. 
Jeho hlavní okno obsahuje položky menu a tlačítka pro 
pro různá programy a dokumentaci související s~\TeX{}em,
které se dají přizpůsobovat prostřednictvím \code{ini} souboru.
Při prvním použití spouští běžné poinstalační procesy specifické pro Windows, 
\emph{tj.} upravuje vyhledávací cesty pro \TL{} a
vytváří některé asociace souborů ale jenom pro aktuálního uživatele. 
Proto pracovní stanice s~přístupem k~\TL{} na lokální síti 
potřebují pouze link pro spouštěč. Viz příručku 
\code{tlaunch} (\code{texdoc tlaunch} nebo
\url{https://ctan.org/pkg/tlaunch}).

\htmlanchor{tlportable}
\subsection{Mobilní \USB{} instalace}
\label{sec:portable-tl}

Volba instalačního programu \code{-portable} (nebo 
příkaz \code{V} v~textové verzi instalátoru nebo 
odpovídající volba \GUI{}) vytváří úplně samostatnou 
instalaci \TL{} pod společným kořenem a předcházející integraci systému. 
Takovou instalaci můžete vytvořit přímo na klíči \USB{}, 
nebo ji zkopírovat na klíč \USB{} později.

Technicky přenosná instalace se stává samostatnou nastavením
výchozích hodnot \envname{TEXMFHOME}, \envname{TEXMFVAR} a
\envname{TEXMFCONFIG} tak, aby byly stejné jako \envname{TEXMFLOCAL},
\envname{TEXMFSYSVAR} a \envname{TEXMFSYSCONFIG}; tím pádem,
konfigurace a mezipaměti pro uživatele nebudou vytvořeny. 

Ke spuštění \TeX{}u při použití této přenosné instalace musíte přidat 
příslušný adresář binárek k~vyhledávané cestě 
během vaší práce na terminálu jako obvykle. 

Pod Windows můžete dvakrát kliknout
na \filename{tl-tray-menu} v~kořenovém adresáři instalace a 
vytvořit dočasné `tray menu' poskytující volby mezi 
několika běžnými úkoly, jak je to ukázáno na této obrazovce:

\medskip
\tlpng{tray-menu}{4cm}{Zásobníková nabídka Windows} %{Windows tray menu}
\smallskip

\noindent Vstup \singleuv{Více\ldots} vysvětluje, jak si můžete 
přizpůsobit tuto nabídku.


%\htmlanchor{tlisoinstall}
%\subsection{\ISO\ (nebo \DVD) instalace}
%\label{sec:isoinstall}
%
%Pokud nepotřebujete příliš často aktualizovat nebo jiným 
%způsobem měnit vaši instalaci
%a\slash nebo máte více systémů, na kterých provozujete \TL{}, 
%může pro vás být užitečné vytvořit
%\ISO\ obraz vaší instalace \TL{}, protože:
%
%\begin{itemize}
%\item Kopírování \ISO\ obrazů mezi různými počítači je mnohem rychlejší
%než zkopírování obyčejné instalace.
%\item Pokud máte duální zavádění různých operačních systémů a chcete 
%sdílet instalaci \TL, instalace \ISO{} není svázaná specifiky 
%a ohraničeními % is not tied to the idiosyncrasies 
%vzájemně rozdílných podporovaných souborových systémů
%(FAT32, NTFS, HFS+).
%\item Virtuální stroj dokáže jednoduše namontovat takové \ISO.
%\end{itemize}
%
%
%Samozřejmě si můžete také vypálit \ISO\ obraz na \DVD, pokud
%to uznáte za užitečné.
%
%Systémy \GNU/Linux/Unix s~pracovní plochou, včetně \MacOSX, 
%jsou schopné namontovat \ISO. Windows 8 je první (!) verzí Windows,
%která to dokáže. Nezávisle na tom se nic 
%nemění v~porovnání s~běžnou instalací na pevném disku,
%viz oddíl \ref{sec:env}.
%
%Při přípravě takové \ISO\ instalace je nejlepší vynechat 
%podadresář roku vydání a mít 
%\filename{texmf-local} na stejné úrovni jako ostatní stromy
%(\filename{texmf-dist}, \filename{texmf-var} atd.). Toto můžete udělat pomocí
%obyčejných voleb adresářů v~instalátoru.
%
%Pro fyzický (spíše než virtuální) systém Windows můžete 
%vypálit \ISO\ obraz na DVD. Ale předtím může stát za 
%to prozkoumat možné volby \ISO-montování, jakou je 
%kupříkladu WinCDEmu na \url{http://wincdemu.sysprogs.org/}.
%
%Pro integraci systému Windows, můžete zahrnout skripty \filename{w32client}a
%popsané v~oddíle~\ref{sec:sharedinstall} a na
%\url{http://tug.org/texlive/w32client.html}, které pracují téměř stejně 
%dobře pro \ISO\ jako pro síťovou instalaci.
%
%Pod \MacOSX bude TeXShop schopen použít DVD
%instalaci, pokud symbolický odkaz \filename{/usr/texbin} ukazuje na 
%příslušný adresář binárek, například
%\begin{verbatim}
%sudo ln -s /Volumes/MyTeXLive/bin/universal-darwin /usr/texbin
%\end{verbatim}
%
%Historická poznámka: \TL{} 2010 byla první edice \TL{}, která už nebyla
%distribuována \singleuv{live}.  Nicméně spouštění z~\DVD\ nebo \ISO\ vždy 
%vyžadovalo jistou obratnost; 
%především nebyla možnost nastavení alespoň jedné dodatečné 
%proměnné prostředí. Pokud vytváříte vaše \ISO\ ze stávající 
%instalace, není to potřebné.

\htmlanchor{tlmgr}
\section{\cmdname{tlmgr}: správa vaší instalace}
\label{sec:tlmgr}

\begin{figure}[tb]
	\def\figdesc{\prog{tlshell} \GUI, ukazující menu \singleuv{Actions} (\MacOSX)}
	\tlpng{tlshell-macos}{\linewidth}{\figdesc}
	\caption{\figdesc}
	\label{fig:tlshell}
\end{figure}

\begin{figure}[tb]
	\def\figdesc{\prog{tlcockpit} \GUI{} pro \prog{tlmgr}}
	\tlpng{tlcockpit-packages}{.8\linewidth}{\figdesc}
	\caption{\figdesc}
	\label{fig:tlcockpit}
\end{figure}

\begin{figure}[tb]
	\def\figdesc{Starší režim \prog{tlmgr} \GUI: hlavní okno, po kliknutí na tlačítko \singleuv{Load}}
	\tlpng{tlmgr-gui}{\linewidth}{\figdesc}
	\caption{\figdesc}
	\label{fig:tlmgr-gui}
\end{figure}

\TL{} obsahuje program nazvaný \prog{tlmgr} pro správu \TL{}
po výchozí instalaci. Jeho možnosti zahrnují:
\begin{itemize*}
%\item výpis seznamů schémat, kolekcí a balíků;
\item instalaci, aktualizaci, zálohování, obnovení a odinstalování 
jednotlivých balíků, volitelně i se započítáním závislostí mezi
balíky; % optionally taking dependencies into account;
\item vyhledávání a přehled balíků a jejich popisy;
\item výpis seznamu, přidání a odstranění platforem;
\item změna instalačních voleb jako například velikosti papíru a umístění 
zdrojů (viz sekci~\ref{sec:location}).
\end{itemize*}

Funkcionalita programu \prog{tlmgr} úplně zahrnuje program \prog{texconfig}.  
Pořád distribuujeme a udržujeme \prog{texconfig}, pokud někdo využívá jeho
rozhraní, avšak nyní doporučujeme používat \prog{tlmgr}.

\subsection{Rozhraní \GUI{} pro \cmdname{tlmgr}}

\TL{} obsahuje několik \GUI{} pro \prog{tlmgr}. 
Dva pozoruhodné: (1) Obr.~\ref{fig:tlshell} ukazuje \cmdname{tlshell},
který je napsán v~Tcl/Tk a pracuje mimo \singleuv{box}u pod Windows
a \MacOSX. (2) Obr.~\ref{fig:tlcockpit} ukazuje \prog{tlcockpit},
který vyžaduje Java ve verzi~8 nebo vyšší a JavaFX. 
Oba jsou dodány jako samostatné balíčky.

\prog{tlmgr} má také nativní režimu \GUI{} 
(viz obr.~\ref{fig:tlmgr-gui}), který se spouští pomocí
\begin{alltt}
> \Ucom{tlmgr -gui}
\end{alltt}
Avšak toto rozšíření \GUI\ vyžaduje Perl/Tk, kterého modul již není
zahrnut v~distribuci Perl v~\TL{} pro Windows.



\subsection{Vzorové realizace \cmdname{tlmgr} z~příkazového řádku}

Po výchozí instalaci můžete svůj systém aktualizovat
na nejnovější dostupnou verzi pomocí:
\begin{alltt}
> \Ucom{tlmgr update -all}
\end{alltt}
Pokud vás to znepokojuje, zkuste nejdříve
\begin{alltt}
> \Ucom{tlmgr update -all -dry-run}
\end{alltt}
nebo (méně upovídané):
\begin{alltt}
> \Ucom{tlmgr update -list}
\end{alltt}

Tento složitější příklad přidá z~místního adresáře kolekci
pro nástroj (engine) \XeTeX:

\begin{alltt}
> \Ucom{tlmgr -repository /local/mirror/tlnet install collection-xetex}
\end{alltt}
Vytvoří následující výstup (zkrácené):
\begin{fverbatim}
install: collection-xetex
install: arabxetex
...
install: xetex
install: xetexconfig
install: xetex.i386-linux
running post install action for xetex
install: xetex-def
...
running mktexlsr
mktexlsr: Updating /usr/local/texlive/2022/texmf-dist/ls-R...
...
running fmtutil-sys --missing
...
Transcript written on xelatex.log.
fmtutil: /usr/local/texlive/2022/texmf-var/web2c/xetex/xelatex.fmt installed.
\end{fverbatim}

Jak můžete vidět, \prog{tlmgr} nainstaluje závislosti
a postará se o~všechny potřebné poinstalační činnosti, 
včetně aktualizace databáze názvů souborů 
a (znovu)vygenerování formátů.  Výše jsme vytvořili
nové formáty pro \XeTeX.

K~popisu balíku (nebo kolekce či schématu) zadejte:
\begin{alltt}
> \Ucom{tlmgr show collection-latexextra}
\end{alltt}
což vytvoří výstup jako tento:
\begin{fverbatim}
package:    collection-latexextra
category:   Collection
shortdesc:  LaTeX supplementary packages
longdesc:   A very large collection of add-on packages for LaTeX.
installed:  Yes
revision:   46963
sizes:      657941k
\end{fverbatim}

Nakonec to nejdůležitější -- úplnou dokumentaci najdete na
\url{https://tug.org/texlive/tlmgr.html} nebo zadáním:
\begin{alltt}
> \Ucom{tlmgr -help}
\end{alltt}

\section{Poznámky k~Windows}
\label{sec:windows}

\subsection{Vlastnosti typické pro Windows}  %Windows-specific features
\label{sec:winfeatures}
Pod Windows dělá instalační program některé dodatečné věci:
\begin{description}
\item[Nabídky a zkratky.] Je nainstalována nová položka \singleuv{\TL{}} nabídky
  Start. Obsahuje vstupy pro některé programy \GUI{} jako \prog{tlshell}
  (\GUI\  pro \prog{tlmgr}) a \prog{dviout} a trochu dokumentace. 
  
%
\item[Přidružení souborů.] Pokud je to povoleno, \prog{TeXworks}, \prog{Dviout}
  a \prog{PS\_view} se buď stávají předvolenými programy pro jejich příslušné
  typy souborů, nebo pro tyto typu souborů získávají položku v~nabídce 
  \singleuv{Otevřít pomocí} dostupnou kliknutím pravým tlačítkem.
  Avšak přidružení souborů \singleuv{Uživatelská volba} s~vyšší prioritou, která mohou být pouze
  specifikované interaktivně, může překážet. 
%
\item[Podpora PostScriptu.] Pro soubory PostScript typ souboru PSviewer
nyní převede PostScript na dočasné PDF, které pak
zobrazí výchozí prohlížeč PDF. Různé bitmapové formáty získají
záznam \cmdname{bitmap2eps} v jejich nabídce \singleuv{Otevřít pomocí} kliknutím pravým tlačítkem
pro převod do EPS, skutečnou práci vykonají \cmdname{sam2p} nebo \cmdname{bmeps}. 

\item[Automatické nastavení proměnné path.] Nevyžadují se žádné
   kroky ruční konfigurace.
%
 \item[Odinstalátor.] Instalační program vytvoří položku 
  pro \TL{}, pod nabídkou \singleuv{Add/Remove Programs} (administrátorská instalace) nebo pod nabídkou \TL\ (instalace pro jednoho uživatele).

%
\item[Ochrana proti zápisu.] Pro administrátorskou instalaci jsou adresáře 
  \TL\ chráněny proti zápisu, přinejmenším pokud je \TL\ instalován na pevném disku
  s~formátováním NTFS.
\end{description}


Pro jiný přístup se také podívejte na \filename{tlaunch}, 
popsaný v~oddíle~\ref{sec:sharedinstall}.


\subsection{Dodatečný obsažený software pod Windows}

Pro úplnost, instalace \TL{} potřebuje podpůrné programy,
které na stroji s~Windows obvykle nenajdete.
\TL{} poskytuje chybějící součásti. Tyto programy jsou nainstalovány jako 
část \TL{} pouze pod Windows.

\begin{description}
\item[Perl, Tcl/Tk a Ghostscript.] Kvůli důležitosti Perlu a Ghostscriptu,
a protože \GUI{} instalačního programu a tlshell jsou napsány v Tcl/Tk, 
zahrnuje \TL{} \singleuv{skryté} kopie těchto programů.  
Programy \TL{}, které je potřebují, vědí, kde je najdou, ale
neprozrazují jejich přítomnost nastavením proměnných prostředí nebo 
registrů. Nejsou to úplné instalace a neměly by překážet žádným 
systémovým instalacím Perlu, Tcl/Tk nebo Ghostscriptu.

\item[dviout.] Nainstalován je také \prog{dviout}, prohlížeč
  DVI souborů.  Nejdříve, když prohlížíte soubory pomocí 
  \cmdname{dviout}, vytvoří fonty, 
  protože fonty pro obrazovku nebyly nainstalovány. Po chvilce 
  budete mít vytvořenu většinu fontů pro použití a okno 
  vytváření fontů uvidíte už jen zřídka.  Více informací
  je možné nalézt ve (velmi doporučené) odpovídající nápovědě.
\item[\TeX{}works.]  \TeX{}works je \TeX ovsky orientovaný
  editor se zabudovaným prohlížečem PDF. 
  Je již pro \TL\ nakonfigurován.

\item[Nástroje příkazového řádku.] Řada běžných unixových
  programů řízených z~příkazového řádku je nainstalovaných
  pod Windows spolu s~běžnými binárkami \TL. Ty zahrnují
  programy \cmdname{gzip}, \cmdname{zip},%\cmdname{chktex}, \cmdname{jpeg2ps}, 
  \cmdname{unzip} %, \cmdname{wget} 
  programy ze skupiny \cmdname{poppler} (\cmdname{pdfinfo},
  \cmdname{pdffonts}, \ldots).
     
\item[\prog{fc-listi}, \prog{fc-cache}, \ldots] Nástroj z~knihovny 
  fontconfig pomáhá \XeTeX{}u efektivněji zacházet s~fonty pod Windows.  
  Můžete použít \prog{fc-list} k~určení jmen fontů 
  k~předání příkazu \XeTeX{}u \cs{font} s~rozšířenou funkcionalitou. 
  Pokud je to potřebné, nejdříve spusťte \prog{\mbox{fc-cache}}
  k~aktualizaci informací o~fontech.
\end{description}

\subsection{User Profile je Home}
\label{sec:winhome}
Windowsovský protějšek domovského (home) adresáře Unixu je adresář \verb|%USERPROFILE%|.
Pod Windows Vista a pozdějšími je to \verb|C:\Users\<username>|.
V~souboru \filename{texmf.cnf} a obecně pro \KPS{}, se bude \verb|~| expandovat
přiměřeně v~obou systémech Windows a Unix.

\subsection{Registry Windows}
\label{sec:registry}
Windows ukládá téměř všechny konfigurační údaje do svých registrů. Registr 
obsahuje soubor hierarchicky uspořádaných klíčů s~několika kořenovými klíči. 
Nejdůležitější pro instalační programy jsou stručně řečeno 
\path{HKEY_CURRENT_USER} a \path{HKEY_LOCAL_MACHINE}, \path{HKCU} a
\path{HKLM}. Část \path{HKCU} registru je v~domovském adresáři 
uživatele (viz sekci~\ref{sec:winhome}). \path{HKLM} je 
obvykle v~podadresáři adresáře Windows.

V~některých případech je možné získat systémové informace z~proměnných
prostředí, ale pro další informace, kupříkladu umístění zkratek, je nutné
nahlédnout do registrů. Trvalé nastavení proměnných prostředí si rovněž
vyžaduje přístup k~registrům.

\subsection{Oprávnění Windows}
\label{sec:winpermissions}

V~novějších verzích Windows se rozlišuje mezi běžnými uživateli a
administrátory, když pouze posledně zmínění mají volný přístup k~téměř celému
operačnímu systému. Ve skutečnosti můžete tyto třídy uživatelů raději
označit jako neprivilegovaní uživatelé a normální uživatelé: být
administrátorem je pravidlo, ne výjimka. Snažili jsme se učinit
\TL{} instalovatelným bez administrátorských práv. 

Pokud je instalátor spuštěn s~administrátorským oprávněním, je možnost instalovat 
i pro všechny uživatele počítače. Pokud je tato volba zvolena,
odkazy (shortcuts) se vytvářejí pro 
všechny uživatele a systémová vyhledávací cesta %prostředí 
se upravuje. Jinak jsou odkazy a
položky nabídky vytvářeny pouze pro aktuálního uživatele a upravuje
se jen jeho vyhledávací cesta. %prostředí. 
%Použijte volbu \optname{non-admin} pro \prog{install-tl}, pokud
%upřednostňujete nastavení podle uživatele dokonce jako administrátor.

Bez ohledu na status administrátora je standardní kořenový adresář \TL{}
navržený instalačním programem vždy pod \verb|%SystemDrive%|.
Instalátor vždy testuje, zda je kořenový adresář
zapisovatelný pro aktuálního uživatele.

Problém se může vyskytnout, pokud uživatel není administrátor a \TeX{} již
existuje ve vyhledávací cestě. Protože skutečná vyhledávací cesta sestává ze
systémové vyhledávací cesty následované uživatelskou vyhledávací cestou, 
nový \TL{} by nikdy neměl získat přednost.  
Jako záložní opatření vytváří instalátor odkaz na příkazový řádek
(command-prompt), ve kterém je adresář binárek nového \TL{} 
předřazený lokální vyhledávací cestě.  Nový \TL{} bude pořád použitelný, když
bude běžet v~relaci příkazového řádku spuštěné z~takového odkazu. Odkaz na 
\TeX{}works, pokud je nainstalován, rovněž předřadí \TL{} 
k~vyhledávací cestě, proto by měl být imunní vůči tomuto problému cest.

Musíte si být vědomi, že dokonce i když jste 
přihlášen jako správce,
musíte explicitně požádat o~správcovská práva. Ve skutečnosti nemá význam
přihlašovat se jako správce. Místo toho kliknutí pravým 
tlačítkem na program nebo odkaz,
který chcete použít, vám obvykle nabídne volbu
\singleuv{Spustit jako správce/Run as administrator}.


\subsection{Zvětšení maxima paměti pod Windows a Cygwin}
\label{sec:cygwin-maxmem}

Uživatelé Windows a Cygwin (pro zvláštnosti
instalace Cygwin viz oddíl~\ref{sec:cygwin}) 
mohou zjistit, že při běhu některých programů dodaných s~\TL, 
trpí nedostatkem paměti.  Například \prog{asy} zhavaruje pro 
nedostatek paměti,
pokud se pokusíte alokovat pole 25,000,000 reálných čísel a
Lua\TeX\ může mít málo paměti, pokud zkusíte zpracovat dokument
s~velkým množstvím rozsáhlých fontů.

Pro Cygwin můžete zvětšit množství dostupné paměti podle
návodu v~příručce The Cygwin User's Guide
(\url{https://cygwin.com/cygwin-ug-net/setup-maxmem.html}).

Pro Windows musíte vytvořit soubor, například \code{moremem.reg}, obsahující
tyto čtyři řádky:

\begin{sverbatim}
Windows Registry Editor Version 5.00

[HKEY_LOCAL_MACHINE\Software\Cygwin]
"heap_chunk_in_mb"=dword:ffffff00
\end{sverbatim}

\noindent a pak spustit povel \code{regedit /s moremem.reg} jako
administrátor.  (Pokud si přejete změnit paměť pouze pro
stávajícího uživatele místo všech, použijte \code{HKEY\_CURRENT\_USER}.)


\begin{otherlanguage}{slovak}
\frenchspacing
% don't use \Webc so the \uppercase in the headline works.
\section{Používateľská príručka ku systému Web2C}
\Webc{} obsahuje množinu \TeX-príbuzných programov, t.\,j.\ samotný \TeX{},
\MF{}, \MP, \BibTeX{} atď. Je to srdce systému \TL{}. 
Domovská stránka \Webc{}, s~aktuálnou príručkou a ďalšími vecami, je na
\url{https://tug.org/web2c}. 

Trochu histórie: originálna implementácia pochádza
od Thomasa Rokického, ktorý v~roku 1987 vyvinul prvý
\TeX{}-to-C systém založený na zmenových súboroch systému pre Unix, ktoré
boli v~prvom rade pôvodnou prácou Howarda Trickeya a Pavla Curtisa.
Tim Morgan sa stal spravovateľom systému a počas jeho obdobia sa meno zmenilo
na Web-to-C\@. V~roku 1990 Karl Berry prebral túto prácu,
asistoval pri tuctoch dodatočných príspevkov a v~roku 1997 podal
taktovku Olafovi Weberovi, ktorý ju v~roku 2006 vrátil Karlovi.

\Webc{} systém beží pod Unixom, 32bitovými Windows,
\acro{MacOSX} a inými operačnými systémami. Používa originálne
\TeX{} zdrojové súbory od Donalda Knutha a ostatné základné programy napísané
v~systéme kultivovaného programovania \web{}, 
ktoré sú preložené do zdrojového kódu jazyka~C. 
Základné programy \TeX{}u sú spracované týmto spôsobom:
\begin{description}
\renewcommand{\makelabel}[1]{\descriptionlabel{\mdseries\cmdname{#1}}}
\item[bibtex]    Spravovanie bibliografií.
%\item[dmp]       Konverzia \cmdname{troff} do MPX (\MP{} obrázky).
\item[dvicopy]   Vytváranie modifikovanej kópie \dvi{} súboru.
\item[dvitomp]   Konverzia \dvi{} do MPX (\MP{} obrázky).
\item[dvitype]   Konverzia \dvi{} do ľudsky čitateľného textu.
\item[gftodvi]   Generovanie fontov pre náhľad.
\item[gftopk]    Konverzia gf formátu fontov do pakovaných fontov.
\item[gftype]    Konverzia gf formátu fontov do ľudsky čitateľného textu.
%\item[makempx]   \MP{} značkové sádzanie.
\item[mf]        \MF{} -- vytváranie rodín fontov.
\item[mft]       Preddefinované \MF ové zdrojové súbory.
\item[mpost]     \MP{} -- tvorba technických diagramov.
%\item[mpto]      \MP{} -- značkový výber.
%\item[newer]     Porovnanie modifikačných časov.
\item[patgen]    Vytváranie vzorov rozdeľovania slov.
\item[pktogf]    Konverzia pakovaných formátov fontov do
gf formátov.
\item[pktype]    Konverzia pakovaných písiem do ľudsky čitateľného textu.
\item[pltotf]    Konverzia \singleuv{Property list} do TFM.
\item[pooltype]  Zobrazovanie \singleuv{\web{} pool} súborov.
\item[tangle]    Konverzia \web{} súborov do Pascalu.
\item[tex]       \TeX{} -- sadzba.
\item[tftopl]    Konverzia TFM do \singleuv{property list}.
\item[vftovp]    Konverzia virtuálneho fontu do virtuálneho \singleuv{property list}.
\item[vptovf]    Konverzia virtuálneho \singleuv{property list} do virtuálneho fontu.
\item[weave]     Konverzia \web{} súborov do \TeX u.
\end{description}
\noindent Presné funkcie a syntax týchto programov sú popísané
v~dokumentáciách jednotlivých balíkov alebo v~dokumentácii
\Webc{}. Napriek tomu, poznanie niekoľkých princípov, ktoré
platia pre celý balík programov, vám pomôže vyťažiť čo najviac
z~vašej \Webc{} inštalácie.

Všetky programy dodržiavajú štandardné \GNU{} voľby:

\begin{description}
\item[\texttt{--help}] Vypisuje prehľad základného používania.
\item[\texttt{--version}] Vypisuje informáciu o~verzii, potom skončí.
\end{description}

A väčšina tiež dodržiava:
\begin{ttdescription}
\item[\texttt{--verbose}] Vypisuje detailnú správu spracovania.
\end{ttdescription}

Na vyhľadávanie súborov používajú \Webc{} programy prehľadávaciu
knižnicu \KPS{} (\url{https://tug.org/kpathsea}).
Táto knižnica používa kombináciu premenných
prostredia a niekoľkých konfiguračných súborov na optimalizáciu
prehľadávania adresárového stromu \TeX u{}. \Webc{}  zvládne
prácu s~viacerými adresárovými stromami súčasne, čo je
užitočné, ak niekto chce udržiavať štandardnú distribúciu \TeX u a jeho
lokálne a osobné rozšírenia v~rozličných stromoch. Na urýchlenie
vyhľadávania súborov obsahuje koreň každého stromu súbor
\file{ls-R} so záznamom obsahujúcim meno a relatívnu cestu ku
všetkým súborom umiestneným pod týmto koreňom.


\subsection{Vyhľadávanie ciest knižnicou Kpathsea}
\label{sec:kpathsea}
Najprv popíšeme všeobecný mechanizmus vyhľadávania ciest
knižnicou \KPS{}.

\emph{Vyhľadávacou cestou} nazveme zoznam \emph{elementov cesty},
ktorými sú v~prvom rade mená adresárov
oddelené dvojbodkou alebo bodkočiarkou.
Vyhľadávacia cesta môže pochádzať z~viacerých zdrojov.
Pri vyhľadávaní súboru \singleuv{\texttt{my-file}} podľa cesty \singleuv{\texttt{.:/dir}},
\KPS{} skontroluje každý element cesty: najprv \file{./my-file},
potom \file{/dir/my-file}, vracajúc prvý zodpovedajúci nájdený
prvok (alebo prípadne všetky zodpovedajúce prvky).

Aby bolo dosiahnuté prispôsobenie sa konvenciám čo možno najviac
operačných systémov, na neunixových systémoch \KPS{} môže používať
oddeľovače názvov súborov rôzne od dvojbodky (\singleuv{\texttt{:}}) a
lomítka (\singleuv{\texttt{/}}).

Pri kontrolovaní určitého elementu cesty \var{p} \KPS{} najprv
overí, či sa na naň nevzťahuje vopred vybudovaná databáza (pozri
\singleuv{Data\-báza názvov súborov} na
strane~\pageref{sec:filename.database}), t.\,j., či sa databáza nachádza
v~adresári, ktorý je prefixom \var{p}. Ak to tak je, špecifikácia
cesty sa porovnáva s~obsahom databázy.

Hoci najjednoduchší a najbežnejší element cesty je meno adresáru,
\KPS{} podporuje aj iné zdroje vo vyhľadávacích cestách: dedičné
(layered) štandardné hodnoty, mená premenných prostredia, hodnoty
súboru \file{config}, domáce adresáre používateľov a rekurzívne
prehľadávanie podadresárov. Preto ak hovoríme, že \KPS{}
\emph{rozbalí} element cesty, znamená to, že pretransformuje všetky
špecifikácie do základného mena alebo mien adresárov. Toto je
popísané v~nasledujúcich odsekoch.

Všimnite si, že keď je meno hľadaného súboru vyjadrené absolútne
alebo explicitne relatívne, t.j.\ začína \singleuv{\texttt{/}} alebo \singleuv{\texttt{./}}
alebo \singleuv{\texttt{../}}, \KPS{} jednoducho skontroluje, či taký súbor
existuje.
\ifSingleColumn
\else
\begin{figure*}
\verbatiminput{examples/ex5.tex}
\setlength{\abovecaptionskip}{0pt}
  \caption{Ilustračný príklad konfiguračného súboru}
  \label{fig:configsample}
\end{figure*}
\fi

\subsubsection{Zdroje cesty}
\label{sec:pathsources}
Vyhľadávacia cesta môže byť vytvorená z~rôznych zdrojov.
\KPS{} ich používa v~tomto poradí:
\begin{enumerate}
\item
  Používateľom nastavená premenná prostredia, napríklad \envname{TEXINPUTS}\@.
  Premenné prostredia s~pridanou bodkou a menom programu majú prednosť pred
premennými rovnakého mena, ale bez prípony.
  Napríklad, keď \singleuv{\texttt{latex}} je meno práve bežiaceho programu, potom
  premenná \envname{TEXINPUTS.latex} prepíše \envname{TEXINPUTS}.
\item
  Programovo-špecifický konfiguračný súbor, napríklad riadok
  \singleuv{\texttt{S /a:/b}} v~súbore \file{config.ps} \cmdname{dvips}.
\item Konfiguračný súbor \KPS{} -- \file{texmf.cnf}, obsahujúci riadok
  ako \singleuv{\texttt{TEXINPUTS=/c:/d}} (pozri ďalej).
\item Predvolené hodnoty počas kompilácie.
\end{enumerate}
\noindent Všetky tieto hodnoty vyhľadávacej cesty môžete prezerať
použitím ladiacích možností (pozri \singleuv{Ladenie} na
strane~\pageref{sec:debugging}).

\subsubsection{Konfiguračné súbory}

\KPS{} číta počas behu z~\emph{konfiguračných súborov}
s~menom \file{texmf.cnf} vyhľadávaciu cestu a ďalšie definície.
Vyhľadávacia cesta \envname{TEXMFCNF} sa používa na hľadanie týchto súborov, 
ale neodporúčame nastavovať túto (ani žiadnu inú) premennú prostredia 
na prepísanie systémových adresárov.

Namiesto toho normálna inštalácia vyústi do súboru
\file{.../\thisyear/texmf.cnf}. Ak musíte vykonať zmeny implicitných 
nastavení (obyčajne to nie je nutné), toto je miesto, kam sa majú vložiť.  
Hlavný konfiguračný súbor je \file{.../\thisyear/texmf-dist/web2c/texmf.cnf}.  
Nesmiete editovať tento neskorší súbor, pretože vaše zmeny budú stratené
pri obnove šírenej verzie.

Ak chcete iba pridať osobný adresár do konkrétnej cesty vyhľadávania, 
je rozumné nastavenie premennej prostredia:
\begin{verbatim}
TEXINPUTS =.:/my/macro/dir:
\end{verbatim}
Ak chcete zachovať udržiavateľnosť a prenosnosť nastavenia v priebehu rokov, 
použite koncové \samp{:} (\samp{;} v systéme Windows) na vloženie systémových ciest,
namiesto toho, aby sa ich všetky snažili napísať explicitne (viď
oddiel~\ref{sec:defaultexpansion}). Ďalšou možnosťou je použitie
stromu \envname{TEXMFHOME} (viď oddiel~\ref{sec:directories}).

\emph{Všetky} súbory
\file{texmf.cnf} vo vyhľadávacej ceste budú prečítané a definície
v~starších súboroch prepíšu definície v~novších súboroch.  
Napríklad, pri vyhľadávacej ceste \verb|.:$TEXMF|, 
hodnoty z~\file{./texmf.cnf} prepíšu hodnoty z~\verb|$TEXMF/texmf.cnf|.

\begin{itemize*}
\item
  Komentáre začínajú znakom \singleuv{\texttt{\%}} buď na začiatku riadku alebo s~medzerou pred ním a pokračujú do konca
  riadku.
\item
  Prázdne riadky sú ignorované.
\item
  Znak \bs{} na konci riadku slúži ako pokračovací znak, t.\,j.\
  nasledujúci riadok je k~nemu pripojený. Prázdne znaky na začiatku
  pripájaných riadkov nie sú ignorované.
\item
  Všetky ostatné riadky majú tvar:\\
  \hspace*{2em}\texttt{\var{variable} \textrm{[}.\var{progname}\textrm{]}
  	\textrm{[}=\textrm{]} \var{value}}\\[1pt]
  kde \singleuv{\texttt{=}} a prázdne znaky naokolo sú nepovinné.
%    (But if \var{value} begins with \samp{.}, it is simplest to use the
%  \samp{=} to avoid the period being interpreted as the program name
%  qualifier.)
  (Ale ak \var{value} začína znakom \samp{.}, je najjednoduchšie použiť
  \samp{=}, aby sa predišlo interpretácii bodky ako kvalifikátora názvu programu).
\item
  Názov premennej \singleuv{\texttt{\var{variable}}} môže obsahovať
  akékoľvek znaky okrem prázdnych znakov, \singleuv{\texttt{=}}, alebo \singleuv{\texttt{.}},
  ale najbezpečnejšie je obmedziť sa na znaky \singleuv{\texttt{A-Za-z\_}}.
\item
  Ak je \singleuv{\texttt{.\var{progname}}} neprázdne, definícia sa
  použije iba vtedy, keď práve bežiaci program má meno
  \texttt{\var{progname}} alebo \texttt{\var{progname}.exe}.
  Toto umožňuje napríklad mať pre rôzne nadstavby \TeX u rôzne
  vyhľadávacie cesty.
\item Hodnoty \singleuv{\texttt{\var{value}}} uvažované ako reťazce môžu obsahovať akýkoľvek znak.
V praxi však väčšina hodnôt \file{texmf.cnf} súvisí s~rozvinutím cesty
a keďže v~expanzii  používajú rôzne špeciálne znaky
(viď oddiel~\ref{sec:cnf-special-chars}), ako napríklad zátvorky alebo čiarky,
nemôžu byť použité v názvoch adresárov.

Znak \singleuv{\texttt{;}} vo \singleuv{\texttt{\var{value}}} je preložený do \singleuv{\texttt{:}}, 
ak sme pod operačným systémom Unix. Toto je užitočné, keď chceme mať jediný
súbor \file{texmf.cnf} pre obidva systémy Unix a Windows.
Táto transformácia nastane s akoukoľvek hodnotou, nielen s~vyhľadávacou cestou,
ale našťastie v praxi nie je \singleuv{\texttt{;}} potrebný
v~iných hodnotách.

Funkcia \code{\$\var{var}.\var{prog}} nie je na pravej strane dostupná;
namiesto nej musíte použiť dodatočnú premennú.

\item
  Všetky definície sú prečítané skôr, ako sa expandujú. Preto môžu
  existovať referencie na premenné skôr, ako sú tieto definované.
\end{itemize*}
Ukážkový úsek konfiguračného súboru, ilustrujúci väčšinu
týchto bodov\ifSingleColumn
\ignorespaces:
\verbatiminput{examples/ex5.tex}
\else
\space je na obrázku~\ref{fig:configsample}.
\fi

\subsubsection{Expanzia cesty}
\label{sec:pathexpansion}

\KPS{} rozpoznáva určité zvláštne znaky a konštrukcie vo
vyhľadávacích cestách podobné tým, čo existujú v~prostrediach
unixovských interprétov príkazového riadku (shells). Ako všeobecný
príklad uvedieme cestu \verb+~$USER/{foo,bar}//baz+, %$
ktorá sa expanduje do všetkých podadresárov pod adresármi
\file{foo} a \file{bar}
v~domovskom adresári používateľa \texttt{\$USER}, ktorý  obsahuje
adresár alebo súbor \file{baz}. Tieto konštrukcie sú popísané
v~ďalších odsekoch.
\subsubsection{Predvolená expanzia}
\label{sec:defaultexpansion}

Ak vyhľadávacia cesta s~najväčšou prioritou (pozri \singleuv{Zdroje cesty}
na strane~\pageref{sec:pathsources}) obsahuje \emph{dvojbodku
navyše} (t.\,j.\ začiatočnú, koncovú, alebo zdvojenú), \KPS{} vloží na
toto miesto vyhľadávaciu cestu s~druhou najvyššou prioritou, ktorá
je definovaná. Ak táto vložená cesta obsahuje dvojbodku navyše, to
isté sa stane s~ďalšou najvýznamnejšou cestou. Keby sme mali
napríklad dané takéto nastavenie premennej prostredia
\begin{alltt}
>> \Ucom{setenv TEXINPUTS /home/karl:}
\end{alltt}
a hodnotu \code{TEXINPUTS} v~súbore \file{texmf.cnf}
\begin{alltt}
  .:\$TEXMF//tex
\end{alltt}
potom konečná hodnota použitá na vyhľadávanie by bola:
\begin{alltt}
  /home/karl:.:\$TEXMF//tex
\end{alltt}
Keďže by bolo zbytočné vkladať predvolenú hodnotu na viac ako jedno
miesto, \KPS{} mení iba nadbytočnú \singleuv{\texttt{:}}\ a všetko ostatné
ponecháva na mieste. Najprv kontroluje začiatočnú \singleuv{\texttt{:}}, potom
koncovú \singleuv{\texttt{:}} a potom zdvojenú \singleuv{\texttt{:}}.

\subsubsection{Expanzia zátvoriek}
\label{sec:brace-expansion}

Užitočná črta je expanzia zátvoriek, ktorá funguje tak, že
napríklad \verb+v{a,b}w+ sa expanduje na \verb+vaw:vbw+. Vnáranie je
povolené. Toto sa používa na implementáciu viacnásobných
\TeX{}ovských hierarchií, priradením hodnoty \code{\$TEXMF}
s~použitím zátvoriek. V~dodanom súbore \file{texmf.cnf} nájdete 
definíciu podobnú tejto (zjednodušenú pre tento príklad):
\begin{verbatim}
  TEXMF = {$TEXMFVAR,$TEXMFHOME,!!$TEXMFLOCAL,!!$TEXMFDIST}
\end{verbatim}
Použijeme to potom na definovanie, napríklad, \TeX{}ovskej cesty pre vstupy:
\begin{verbatim}
  TEXINPUTS = .;$TEXMF/tex//
\end{verbatim}
%$
bude to znamenať, že po hľadaní v~aktuálnom adresári sa najprv prehľadajú
stromy \code{\$TEXMFVAR/tex}, \code{\$TEXMFHOME/tex}, \code{\$TEXMFLOCAL/tex}
a \code{\$TEXMFDIST/tex} (posledné dva
s~použitím databázových súborov \file{ls-R}). 

\subsubsection{Expanzia podadresárov}
\label{sec:subdirectory.expansion}

Dva alebo viac za sebou nasledujúcich znakov \singleuv{\texttt{/}} 
v~elemente cesty nasledujúcom za adresárom \var{d} je nahradených všetkými
podadresármi \var{d\/}: najprv podadresármi priamo pod \var{d}, potom
podadresármi pod nimi atď. Poradie, v~akom sú prehľadávané
podadresáre na každej úrovni, \emph{nie je špecifikované}.

Ak po \singleuv{\texttt{//}} špecifikujete akékoľvek komponenty mena súboru,
pridajú sa iba pod\-adre\-sáre so zodpovedajúcimi komponentami.
Napríklad \singleuv{\texttt{/a//b}} sa expanduje do
adresárov \file{/a/1/b}, \file{/a/2/b},
\file{/a/1/1/b}, atď, ale nie do \file{/a/b/c} alebo \file{/a/1}.

Viacnásobné konštrukcie \singleuv{\texttt{//}} v~ceste sú možné, ale
použitie \singleuv{\texttt{//}} na začiatku cesty je ignorované.

\subsubsection{Zhrnutie špeciálnych znakov v~súboroch \file{texmf.cnf}}
\label{sec:cnf-special-chars}

Nasledujúci zoznam zahŕňa špeciálne znaky a konštrukcie v~konfiguračných
súboroch \KPS{}.

% need a wider space for the item labels here. 3em instead of 2em
\newcommand{\CODE}[1]{\makebox[3em][l]{\code{#1}}}
\begin{description}
\item[\CODE{:}]Oddeľovač v~špecifikácii cesty; na začiatku alebo
  na konci cesty, alebo zdvojený uprostred, nahrádza predvolenú expanziu cesty.\par
\item[\CODE{;}]Oddeľovač v~neunixových systémoch (správa sa ako
   \singleuv{\texttt{:}}).
\item[\CODE{\$}]Expanzia premennej.
\item[\CODE{\string~}] Reprezentuje domovský adresár používateľa.
\item[\CODE{\char`\{...\char`\}}] Expanzia zátvoriek, napr.\
  z~\verb+a{1,2}b+ sa stane \verb+a1b:a2b+.
\item[\CODE{,}] Oddeľuje členy v~expanzii zátvoriek.
\item[\CODE{//}] Expanzia podadresárov. (Môže sa
  vyskytnúť kdekoľvek v~ceste, okrem jej začiatku.)
\item[\CODE{\%{\rm\ a }\#}] Začiatok komentáru.
\item[\CODE{\bs}] Na konci riadku, znak pokračovania na umožnenie viacriadkových vstupov.
\item[\CODE{!!}] Povel na hľadanie súboru \emph{iba} v~databáze,
  neprehľadáva disk.
\end{description}

Kedy presne bude znak považovaný za špeciálny alebo bude predstavovať 
samého seba závisí od kontextu, v ktorom sa používa. Pravidlá sú obsiahnuté vo
viacerých úrovniach interpretácie konfigurácie (analýza,
expanzia, vyhľadávanie, \ldots), a preto, nanešťastie, to nie je možné stručne ustanoviť. 
Neexistuje žiadny všeobecný únikový mechanizmus; konkrétne,
\samp{\bs} nie je \uv{únikový znak} v~ súboroch \file{texmf.cnf}.

Pokiaľ ide o výber názvov adresárov na inštaláciu, je najbezpečnejšie
vyhnúť sa im všetkým.

\subsection{Databázy názvov súborov}
\label{sec:filename.database}

\KPS{} minimalizuje prístupy na disk pri vyhľadávaní. Predsa však
pri štandardnej alebo ľubovoľnej inštalácii 
s~dostatočným množstvom adresárov, vyhľadávanie súboru
v~každom možnom adresári môže zabrať prehnane veľa času.
\KPS{} preto môže používať externe vytvorený \singleuv{databázový} súbor
nazývaný \file{ls-R}, ktorý mapuje súbory v~adresároch a pomáha tak
vyhnúť sa vyčerpávajúcemu prehľadávaniu disku.

A second database file \file{aliases} allows you to give additional
names to the files listed in \file{ls-R}.

Druhý databázový súbor (\file{aliases}) vám umožňujú dať dodatočné mená súborom nachádzajúcim sa
v~zozname \file{ls-R}. 

\subsubsection{Súborová databáza}
\label{sec:ls.R}

Ako bolo vysvetlené hore, meno hlavnej databázy súborov musí
byť \file{ls-R}. Môžete umiestniť jednu do koreňa každej hierarchie
\TeX u{} vo svojej inštalácii, ktorú chcete, aby bola prehľadávaná
(predvolená je \code{\$TEXMF}). 
%; väčšinou sa jedná iba o~jednu hierarchiu. 
\KPS{} hľadá \file{ls-R} súbory podľa cesty
v~\code{TEXMFDBS}.

Odporúčaný spôsob, ako vytvoriť a udržiavať \samp{ls-R}, je spustiť
skript \code{mktexlsr} zahrnutý v~distribúcii. Je vyvolávaný
rôznymi \singleuv{\texttt{mktex...}} skriptami. Tento skript v~princípe iba
spúšťa príkaz
\begin{alltt}
cd \var{/your/texmf/root} && \path|\|ls -1LAR ./ >ls-R
\end{alltt}
predpokladajúc, že \code{ls} vášho systému vytvára správny výstup
(výstup \GNU \code{ls} je v~poriadku). Aby ste sa ubezpečili,
že databáza bude vždy aktuálna, najjednoduchšie je pravidelne ju
prebudovávať cez \code{cron}, 
takže po zmenách v~inštalovaných
súboroch -- napríklad pri inštalácii alebo aktualizácii balíka
\LaTeX u{} -- bude súbor \file{ls-R} automaticky aktualizovaný.

Ak súbor nie je v~databáze nájdený, podľa predvoleného nastavenia
\KPS{} začne vyhľadávať na disku. Ak však určitý element cesty
začína \singleuv{\texttt{!!}}, bude prehľadávaná \emph{iba} databáza, nikdy nie
disk.

\subsubsection{kpsewhich: samostatné prehľadávanie cesty}
\label{sec:invoking.kpsewhich}

Program \texttt{kpsewhich} vykonáva prehľadávanie cesty nezávislé
od každej aplikácie. Môže byť užitočný ako vyhľadávací \code{find}
program na nájdenie súborov v~hierarchiách \TeX u{} (veľmi sa
využíva v~distribuovaných \singleuv{\texttt{mktex}}\dots\ skriptoch).

\begin{alltt}
>> \Ucom{kpsewhich \var{option}\dots{} \var{filename}\dots{}}
\end{alltt}
Voľby špecifikované v~\singleuv{\texttt{\var{option}}} môžu začínať buď
\singleuv{\texttt{-}} alebo \singleuv{\texttt{-{}-}} a 
každá skratka, ktorá nie je viacznačná, je akceptovaná.

\KPS{} považuje každý element vstupného riadku, ktorý nie je
argumentom nejakej voľby, za meno súboru, ktorý hľadá, a
vracia prvý súbor, ktorý nájde. Neexistuje voľba umožňujúca
vrátiť všetky súbory s~určitým menom (na to môžete použiť
nástroj Unixu \singleuv{\texttt{find}}).

Najbežnejšie voľby sú popísané nižšie.
\begin{description}
\item[\CODE{--dpi=num}]\hfill\break
  Nastav rozlíšenie na \singleuv{\texttt{\var{num}}}; toto má vplyv iba na
  \singleuv{\texttt{gf}} a \singleuv{\texttt{pk}} vyhľadávanie. \singleuv{\texttt{-D}} je synonymom, kvôli
  kompatibilite s~\cmdname{dvips}.  Predvolená hodnota je~600.
\item[\CODE{--format=name}]\hfill\break
  Nastav formát na vyhľadávanie na \singleuv{\texttt{\var{name}}}. Podľa
  predvoleného nastavenia je formát uhádnutý z~mena súboru. Pre formáty,
ktoré nemajú asociovanú jednoznačnú príponu, ako napríklad podporné
  súbory \MP u a konfiguračné súbory \cmdname{dvips}, musíte
  špecifikovať meno, ako známe pre \KPS{}, také ako
   \texttt{tex} alebo \texttt{enc files}. Zoznam získate spustením 
   \texttt{kpsewhich -{}-help-formats}.

\item[\texttt{--mode=\var{string}}]\mbox{}\\
  Nastav meno módu na \singleuv{\texttt{\var{string}}}; toto má vplyv iba na
  \singleuv{\texttt{gf}} a \singleuv{\texttt{pk}} vyhľadávanie. Žiadna predvolená hodnota:
  každý mód bude nájdený.
\item[\texttt{--must-exist}]\mbox{}\\
  Urob všetko preto, aby si našiel súbory. Ak je to potrebné,
vrátane hľadania na disku. Normálne je v~záujme efektívnosti
  prehľadávaná iba databáza \file{ls-R}.
\item[\texttt{--path=\var{string}}]\mbox{}\\
  Vyhľadávaj podľa cesty \singleuv{\texttt{\var{string}}} (oddeľovaná dvojbodkou
  ako zvyčajne) namiesto hádania vyhľadávacej cesty z~mena súboru.
  Podporované sú \singleuv{\texttt{//}} a všetky bežné expanzie. Voľby \singleuv{\texttt{--path}}
  a \singleuv{\texttt{--format}} sa vzájomne vylučujú.
\item[\texttt{--progname=\var{name}}]\mbox{}\\
  Nastav meno programu na \singleuv{\texttt{\var{name}}}. Toto nastavenie ovplyvňuje
  použitie vyhľadávacej cesty cez nastavenie \singleuv{\texttt{.\var{progname}}}
  v~konfiguračných súboroch. Predvolená hodnota je \singleuv{\texttt{kpsewhich}}.
\item[\texttt{--show-path=\var{name}}]\mbox{}\\
  Zobrazí cestu použitú na vyhľadávanie súboru s~typom
\singleuv{\texttt{\var{name}}}.
  Môže byť použitá buď súborová prípona (\singleuv{\texttt{.pk}}, \singleuv{\texttt{.vf}} a pod.)
  alebo meno, podobne ako vo voľbe \singleuv{\texttt{--format}}.
\item[\texttt{--debug=\var{num}}]\mbox{}\\
  Nastaví masku výberu ladiacích možností na \singleuv{\texttt{\var{num}}}.
\end{description}

\subsubsection{Príklady použitia}
\label{sec:examplesofuse}

Pozrime sa na \KPS{} v~akcii. Za podčiarknutým príkazom nasleduje
výsledok vyhľadávania v~nasledujúcich riadkoch.
\begin{alltt}
> \Ucom{kpsewhich  article.cls}
/usr/local/texmf-dist/tex/latex/base/article.cls
\end{alltt}
Hľadáme súbor \file{article.cls}. Keďže prípona \singleuv{\texttt{.cls}} je
jednoznačná, nemusíme špecifikovať, že hľadáme súbor typu \singleuv{tex}
(zdrojový súbor \TeX{}). Nájdeme ho v~podadresári
\file{tex/latex/base} pod koreňovým adresárom
\singleuv{\texttt{TEXMF-dist}}. Podobne všetky nasledujúce súbory budú nájdené bez
problémov vďaka ich jednoznačnej prípone.
\begin{alltt}
> \Ucom{kpsewhich array.sty}
   /usr/local/texmf-dist/tex/latex/tools/array.sty
> \Ucom{kpsewhich latin1.def}
   /usr/local/texmf-dist/tex/latex/base/latin1.def
> \Ucom{kpsewhich size10.clo}
   /usr/local/texmf-dist/tex/latex/base/size10.clo
> \Ucom{kpsewhich small2e.tex}
   /usr/local/texmf-dist/tex/latex/base/small2e.tex
> \Ucom{kpsewhich tugboat.bib}
   /usr/local/texmf-dist/bibtex/bib/beebe/tugboat.bib
\end{alltt}

Mimochodom, posledným súborom je bibliografická databáza \BibTeX u pre
články \textsl{TUGBoatu}.

\begin{alltt}
> \Ucom{kpsewhich cmr10.pk}
\end{alltt}
Bitmapové súbory fontov typu \file{.pk} sa používajú zobrazovacími
programami ako \cmdname{dvips} a \cmdname{xdvi}. V~tomto prípade
je vrátený prázdny výsledok, keďže neexistujú žiadne vopred
generované Computer Modern \singleuv{\texttt{.pk}} súbory v~našom systéme
(vzhľadom na to, že v~\TL{} implicitne používame verzie Type1 ).
\begin{alltt}
> \Ucom{kpsewhich wsuipa10.pk}
\ifSingleColumn   /usr/local/texmf-var/fonts/pk/ljfour/public/wsuipa/wsuipa10.600pk
\else /usr/local/texmf-var/fonts/pk/ljfour/public/
...                         wsuipa/wsuipa10.600pk
\fi\end{alltt}
Pre tieto fonty (fonetickú abecedu z~University of Washington) sme museli vygenerovať\
\singleuv{\texttt{.pk}} súbory. Keďže predvolený mód \MF u v~našej inštalácii
je \texttt{ljfour} so základným rozlíšením 600~dpi (dots per inch),
je vrátená táto inštancia.
\begin{alltt}
> \Ucom{kpsewhich -dpi=300 wsuipa10.pk}
\end{alltt}
V~tomto prípade po špecifikovaní, že nás zaujíma rozlíšenie 300~dpi
(\texttt{-dpi=300}), vidíme, že taký font nie je v~systéme
k~dispozícii. Programy ako \cmdname{dvips} alebo \cmdname{xdvi}
by v~tomto prípade vytvorili požadované \texttt{.pk} súbory, použijúc skript \cmdname{mktexpk}.

Teraz obráťme našu pozornosť na hlavičkové a konfiguračné súbory
\cmdname{dvips}. Najprv sa pozrieme na jeden z~bežne používaných
súborov, všeobecný  prológový \file{tex.pro} na pod\-poru \TeX u{},
potom pohľadáme konfiguračný súbor (\file{config.ps}) a
PostScriptovú mapu fontov \file{psfonts.map} --  
mapové a kódové súbory majú svoje vlastné cesty na 
vyhľadávanie a nové umiestnenie v~stromoch \dirname{texmf}.  
Keďže prípona
\singleuv{\texttt{.ps}} je ne\-jed\-no\-značná, musíme pre súbor \texttt{config.ps}
špecifikovať explicitne, o~ktorý typ sa zaujímame (\optname{dvips
config}).
\begin{alltt}
> \Ucom{kpsewhich tex.pro}
  /usr/local/texmf/dvips/base/tex.pro
> \Ucom{kpsewhich --format=`dvips config' config.ps}
   /usr/local/texmf/dvips/config/config.ps
> \Ucom{kpsewhich psfonts.map}
   /usr/local/texmf/fonts/map/dvips/updmap/psfonts.map
\end{alltt}
Teraz sa pozrieme na podporné súbory URW Times PostScript.
V~Berryho schéme meno pre
tieto pomenovania fontov je \uv{\texttt{utm}}. Prvý
súbor, ktorý hľadáme, je konfiguračný súbor, ktorý obsahuje meno
mapového súboru:
\begin{alltt}
> \Ucom{kpsewhich --format="dvips config" config.utm}
   /usr/local/texmf-dist/dvips/psnfss/config.utm
\end{alltt}
Obsah tohoto súboru je
\begin{alltt}
  p +utm.map
\end{alltt}
čo odkazuje na súbor \file{utm.map}, ktorý ideme ďalej hľadať.
\begin{alltt}
> \Ucom{kpsewhich utm.map}
   /usr/local/texmf-dist/fonts/map/dvips/times/utm.map
\end{alltt}
Tento mapový súbor definuje mená súborov fontov typu Type1
PostScript v~kolekcii URW. Jeho obsah vyzerá takto (zobrazili sme
iba jeho časť):
\begin{alltt}
  utmb8r  NimbusRomNo9L-Medi    ... <utmb8a.pfb
  utmbi8r NimbusRomNo9L-MediItal... <utmbi8a.pfb
  utmr8r  NimbusRomNo9L-Regu    ... <utmr8a.pfb
  utmri8r NimbusRomNo9L-ReguItal... <utmri8a.pfb
  utmbo8r NimbusRomNo9L-Medi    ... <utmb8a.pfb
  utmro8r NimbusRomNo9L-Regu    ... <utmr8a.pfb
\end{alltt}
Zoberme napríklad inštanciu Times Regular \file{utmr8a.pfb} a
nájdime jej pozíciu v~adresárovom strome \file{texmf} použitím
vyhľadávania fontových súborov Type1:
\begin{alltt}
> \Ucom{kpsewhich utmr8a.pfb}
\ifSingleColumn   /usr/local/texmf-dist/fonts/type1/urw/times/utmr8a.pfb
\else   /usr/local/texmf-dist/fonts/type1/
... urw/utm/utmr8a.pfb
\fi\end{alltt}

Z~týchto príkladov by malo byť zrejmé, ako ľahko môžete nájsť
umiestnenie daného súboru. Toto je zvlášť dôležité, keď máte
podozrenie, že ste narazili na zlú verziu súboru,
pretože \cmdname{kpsewhich} emuluje vyhľadávanie
úplne rovnakým spôsobom ako skutočný program (\TeX{}, dvips a pod).

\subsubsection{Ladiace činnosti}
\label{sec:debugging}
Niekedy je potrebné vyšetriť, ako program rozpoznáva referencie na
súbory. Aby toto bolo možné vhodne uskutočniť, \KPS{} ponúka rôzne
stupne ladenia:
\begin{itemize}
\item[\texttt{\ 1}] Volania \texttt{stat} (testy súborov). Pri behu
   s~aktuálnou \file{ls-R} databázou by nemal dať takmer žiaden výstup.
\item[\texttt{\ 2}] Referencie do hašovacích tabuliek (ako \file{ls-R}
  databázy, mapové súbory, konfiguračné súbory).
\item[\texttt{\ 4}] Operácie otvárania a zatvárania súboru.
\item[\texttt{\ 8}] Všeobecná informácia o~ceste pre typy súborov
  hľadaných \KPS. Toto je užitočné pri zisťovaní, kde bola definovaná
  určitá cesta pre daný súbor.
\item[\texttt{16}] Adresárový zoznam pre každý element cesty (vzťahuje
  sa iba na vyhľadávanie na disku).
\item[\texttt{32}] Vyhľadávanie súborov.
\item[\texttt{64}] Premenlivé hodnoty.
\end{itemize}
Hodnota \texttt{-1} nastaví všetky horeuvedené voľby, v~praxi
pravdepodobne vždy použijete tieto úrovne, ak budete potrebovať
akékoľvek ladenie.

Podobne s~programom \cmdname{dvips} nastavením kombinácie
ladiacich prepínačov môžete detailne sledovať, odkiaľ sa berú
používané súbory. Aktuálny popis parametrov je možné nájsť
v~\path{../texmf/doc/html/dvips/dvips_2.html}. Alternatívne,
keď súbor nie je nájdený,
ladiaca cesta ukazuje, v~ktorých adresároch program daný súbor
hľadal, čo môže naznačovať, v~čom sa asi vyskytol problém.
Všeobecne povedané, keďže väčšina programov volá knižnicu \KPS{}
vnútorne, ladiace voľby je možné nastaviť pomocou premennej
prostredia \envname{KPATHSEA\_DEBUG} na potrebnú kombináciu, ako je
to popísané v~horeuvedenom zozname.
(Poznámka pre používateľov Windows: nie je jednoduché
presmerovať všetky hlášky v~tomto systéme do súboru.
Na diagnostické účely môžete dočasne priradiť\hfill\\
\texttt{SET KPATHSEA\_DEBUG\_OUTPUT=err.log}.)
Uvažujme ako príklad malý zdrojový súbor
\LaTeX u{}, \file{hello-world.tex}, ktorý obsahuje nasledujúci vstup.
\begin{verbatim}
  \documentclass{article}
  \begin{document}
  Hello World!
  \end{document}
\end{verbatim}
Tento malý súbor používa iba font \file{cmr10}, takže pozrime sa,
ako \cmdname{dvips} pripravuje PostScriptový súbor (chceme použiť
Type1 verziu písiem Computer Modern, preto je nastavená
voľba \texttt{-Pcms}\footnote{Od verzie \TL{}~7 nie
je nutné túto voľbu nastavovať, pretože Type1 fonty sú
načítané implicitne.}.
\begin{alltt}
> \Ucom{dvips -d4100 hello-world -Pcms -o}
\end{alltt}
V~tomto prípade sme skombinovali \cmdname{dvips} ladiacu
triedu~4 (cesty k~fontom) s~expanziou elementu cesty \KPS\
(pozri Referenčnú príručku \cmdname{dvips}).
Výstup, trochu preusporiadaný, je zobrazený na obrázku~\ref{fig:dvipsdbga}.

\begin{figure*}[tp]
\centering
\input{examples/ex6a.tex}
\caption{Vyhľadávanie konfiguračných súborov}\label{fig:dvipsdbga}
\end{figure*}

\cmdname{dvips} začne lokáciou svojich pracovných súborov. Najprv je
nájdený \file{texmf.cnf}, ktorý obsahuje definície vyhľadávacích
ciest ostatných súborov, potom databáza súborov \file{ls-R} (na
optimalizáciu vyhľadávania súborov) a skratky mien súborov
(\file{aliases}), čo robí možným deklarovať viacero mien (napr.\
krátke meno typu \singleuv{8.3} ako v~DOSe a dlhšiu prirodzenejšiu
verziu) pre ten istý súbor. Potom \cmdname{dvips} pokračuje
v~hľadaní všeobecného konfiguračného súboru \file{config.ps} skôr,
ako začne hľadať súbor nastavení \file{.dvipsrc} (ktorý, v~tomto
prípade, \emph{nie je nájdený}). Nakoniec, \cmdname{dvips} nájde
konfiguračný súbor pre font Computer Modern PostScript,
\file{config.cms} (toto bolo iniciované voľbou\texttt{-Pcms}
v~príkaze \cmdname{dvips}). Tento súbor obsahuje zoznam \uv{mapových}
súborov, ktoré definujú vzťah medzi menami fontov v~\TeX u,
PostScripte a systéme súborov.
\begin{alltt}
> \Ucom{more /usr/local/texmf/dvips/cms/config.cms}
   p +ams.map
   p +cms.map
   p +cmbkm.map
   p +amsbkm.map
\end{alltt}
\cmdname{dvips} preto pokračuje v~hľadaní všetkých týchto súborov plus
všeobecného mapového súboru \file{psfonts.map}, ktorý sa načíta
vždy (obsahuje deklarácie bežne používaných PostScriptových fontov;
pozri poslednú časť sekcie~\ref{sec:examplesofuse}, kde sa nachádza
viac detailov o~narábaní s~mapovými súbormi PostScriptu).

V~tomto bode sa \cmdname{dvips} identifikuje používateľovi\,\ldots
\begin{alltt}\ifSingleColumn
This is dvips(k) 5.92b Copyright 2002 Radical Eye Software (www.radicaleye.com)
\else\small{}This is dvips(k) 5.92b Copyright 2002 Radical Eye ...
\fi\end{alltt}% decided to accept slight overrun in tub case
\ifSingleColumn
\noindent\ldots potom pokračuje v~hľadaní prológového súboru
\file{texc.pro}:

\begin{alltt}\small
kdebug:start search(file=texc.pro, must\_exist=0, find\_all=0,
  path=.:~/tex/dvips//:!!/usr/local/texmf/dvips//:
  ~/tex/fonts/type1//:!!/usr/local/texmf/fonts/type1//).
  kdebug:search(texc.pro) => /usr/local/texmf/dvips/base/texc.pro
\end{alltt}
\else
potom pokračuje v~hľadaní prológového súboru
\file{texc.pro} (pozri obrázok~\ref{fig:dvipsdbgb}).
\fi

Po nájdení tohoto súboru \cmdname{dvips} vypíše na výstup dátum a
čas a informuje nás, že vygeneruje súbor \file{hello-world.ps}, že
potrebuje súbor s~fontom \file{cmr10}, ktorý bude deklarovaný ako
\uv{rezidentný}:
\begin{alltt}\small
TeX output 1998.02.26:1204' -> hello-world.ps
Defining font () cmr10 at 10.0pt
Font cmr10 <CMR10> is resident.
\end{alltt}
Teraz sa rozbehne hľadanie súboru \file{cmr10.tfm}, ktorý je
nájdený, potom je referencovaných ešte niekoľko prológových súborov
(nezobrazené) a nakoniec je nájdená inštancia fontu
Type1, \file{cmr10.pfb}, ktorá je pridaná do výstupného súboru
(pozri posledný riadok).
\begin{alltt}\small
kdebug:start search(file=cmr10.tfm, must\_exist=1, find\_all=0,
  path=.:~/tex/fonts/tfm//:!!/usr/local/texmf/fonts/tfm//:
       /var/tex/fonts/tfm//).
kdebug:search(cmr10.tfm) => /usr/local/texmf/fonts/tfm/public/cm/cmr10.tfm
kdebug:start search(file=texps.pro, must\_exist=0, find\_all=0,
   ...
<texps.pro>
kdebug:start search(file=cmr10.pfb, must\_exist=0, find\_all=0,
  path=.:~/tex/dvips//:!!/usr/local/texmf/dvips//:
       ~/tex/fonts/type1//:!!/usr/local/texmf/fonts/type1//).
kdebug:search(cmr10.pfb) => /usr/local/texmf/fonts/type1/public/cm/cmr10.pfb
<cmr10.pfb>[1]
\end{alltt}

\subsection{Možnosti nastavenia za behu programu}

Ďalšou z~pekných čŕt distribúcie \Webc{} je možnosť kontroly
množstva pamäťových parametrov (najmä veľkosti polí) za behu
prostredníctvom súboru \file{texmf.cnf},
ktorý číta knižnica \KPS{}. Nastavenia všetkých parametrov môžete
nájsť v~časti~3 tohto súboru. Najdôležitejšie riadiace
premenné (čísla riadkov sa vzťahujú na súbor \file{texmf.cnf}):

\begin{description}
\item[\texttt{main\_memory}]
  Celkový počet dostupných slov v~pamäti pre \TeX{}, \MF{} a \MP.
  Musíte vytvoriť nový formátový súbor pre každé odlišné nastavenie.
  Napríklad môžete vygenerovať \uv{obrovskú} verziu \TeX u{}
  a zavolať súbor s~formátom \texttt{hugetex.fmt}. S~po\-užitím
  štandardnej špecifikácie mena programu používaného knižnicou
  \KPS{}, konkrétna hodnota premennej \texttt{main\_memory} sa načíta zo
  súboru \file{texmf.cnf}. 
\item[\texttt{extra\_mem\_bot}]
  Dodatočný priestor pre \uv{veľké} dátové štruktúry \TeX u{}:
  \uv{boxy}, \uv{glue}, \uv{break\-point(y)} a podobne.
  Je to užitočné hlavne ak používate \PiCTeX{}.
\item[\texttt{font\_mem\_size}]
  Počet dostupných slov pre informáciu o~fontoch v~\TeX u. Toto je
  viac-menej celková veľkosť všetkých prečítaných TFM súborov.
\item[\texttt{hash\_extra}]
  Dodatočný priestor pre hašovaciu tabuľku mien riadiacej sekvencie, jeho implicitná hodnota  
  je \texttt{600000}.  
\end{description}
\noindent Tento prvok nemôže nahradiť naozajstné dynamické polia
a alokácie pamäte, ale keďže tieto sa veľmi ťažko implementujú
v~súčasnej verzii \TeX u, tieto parametre počas behu programu
poskytujú praktický kompromis, ktorý dovoľuje aspoň nejakú
flexi\-bilitu.

\htmlanchor{texmfdotdir}
\subsection{\texttt{\$TEXMFDOTDIR}}
\label{sec:texmfdotdir}

Na rôznych miestach uvedených vyššie sme uviedli rôzne cesty vyhľadávania
začínajúce znakom \code{.} (na vyhľadávanie najprv v~aktuálnom adresári), ako v
\begin{alltt}\small
TEXINPUTS=.;$TEXMF/tex//
\end{alltt}
	
Toto je zjednodušenie. Súbor \code{texmf.cnf}, ktorý je súčasťou distribúcie
\TL{} používa \filename{$TEXMFDOTDIR} namiesto jednoduchého \samp{.}, ako v:
\begin{alltt}\small
TEXINPUTS=$TEXMFDOTDIR;$TEXMF/tex//
\end{alltt}
(V~dodanom súbore je aj druhý prvok cesty o niečo komplikovanejší
ako \filename{$TEXMF/tex//}. Ale to je drobnosť; na tomto mieste chceme
pojednať o črte \filename{$TEXMFDOTDIR}.)
	
Dôvod na použitie premennej \filename{$TEXMFDOTDIR} v~definíciach ciest
namiesto jednoduchého \samp{.} je čisto taký, že môže byť prepísaný. 
Napríklad zložitý dokument môže pozostávať z~mnohých zdrojových súborov 
uložených vo viacerých podadresároch. Aby ste to zvládli, môžete
nastaviť \filename{TEXMFDOTDIR} na \filename{.//} (napríklad, v~prostredí
keď zostavujete dokument) a všetky budú prehľadávané. 
(Upozornenie: nepoužívajte \filename{.//} ako implicitné nastavenie; 
je to zvyčajne veľmi nežiadúce a
potenciálne nezabezpečené, 
aby ste hľadali hocijaký dokument vo všetkých podadresároch.)
	
	As another example, you may wish not to search the current directory at
	all, e.g., if you have arranged for all the files to be found via
	explicit paths. You can set \filename{$TEXMFDOTDIR} to, say,
	\filename{/nonesuch} or any other nonexistent directory for this.
	
Ďalším príkladom je, že nebudete chcieť vôbec prehľadávať aktuálny adresár, 
napríklad, ak ste zariadili, aby sa všetky súbory našli prostredníctvom
explicitných ciest. Môžete nastaviť \filename{$TEXMFDOTDIR}, napríklad, na 
\filename{/nonesuch} alebo na akýkoľvek neexistujúci adresár.	
		
Implicitná hodnota \filename{$TEXMFDOTDIR} je jednoducho \samp{.}, 
ako je to nastavené v~našom súbore \filename{texmf.cnf}.


\end{otherlanguage}


\htmlanchor{ack}
\section{Poděkování}

\TL{} je výsledkem společného úsilí téměř všech skupin uživatelů \TeX u.
Toto vydání \TL{} redigoval Karl Berry. 
Seznam ostatních hlavních přispěvatelů, minulých i současných, následuje. 
Děkujeme:
%B% Kaja Christiansen prováděla neúnavně opakované
%B% rekompilace na různých unixových platformách. Vladimir Volovich odvedl
%B% vynikající práci na vyčištění zdrojových textů a provedl mnohá
%B% vylepšení, zatímco Gerben Wierda dělal veškerou práci pro \MacOSX.

\begin{itemize*}
\item Anglickému, německému, holandskému a polskému sdružení 
uživatelů \TeX{}u (TUG, DANTE e.V., NTG resp. GUST), kteří společně zajistili 
nezbytnou technickou a administrativní infrastrukturu.
Připojte se, prosím, k~\TeX ovskému sdružení ve vaší 
blízkosti (\acro{CSTUG}, \url{https://www.cstug.cz}),
v~jehož gesci vznikl i tento překlad!
(Viz \url{https://tug.org/usergroups.html}.)

\item Týmu CTANu (\url{https://ctan.org}), který distribuuje obrazy \TL{} a
poskytuje společnou infrastrukturu pro aktualizaci balíků, 
na kterých je \TL{} závislý.

\item Nelsonu Beebemu, který zpřístupnil mnohé platformy vývojařům \TL\
a za jeho vlastní obsáhlé testování a bezpříkladné bibliogtrafické úsilí.

\item Johnu Bowmanovi za vykonání mnoha změn v~jeho pokročilém 
grafickém programu Asymptote, aby fungoval v~\TL.

\item Peteru Breitenlohnerovi a \eTeX\ týmu, kteří poskytují 
stabilní základ budoucnosti \TeX u, a výslovně Peterovi za 
skvělou pomoc s~používáním osobních nástrojů \GNU\ a udržování zdrojů 
v aktuálním stavu. Peter zemřel v~říjnu 2015 a pokračující dílo věnujeme jeho památce.

\item Jin-Hwan Choovi a celému týmu DVIPDFM$x$ za jejich vynikající 
ovladač a za schopnost reagovat na konfigurační problémy. 

\item Thomasi Esserovi za překrásný balík \teTeX{}, bez něhož by \TL{}
nikdy neexistoval.

\item Michelu Goossensovi, který je spoluautorem původní dokumentace.

\item Eitanu Gurarimu, jehož \TeX{}4ht je použito pro
\HTML{} verzi této dokumentace a který každý rok obratem
neúnavně pracoval na jeho rozšířeních. Eitan nás předčasně
opustil v~červnu 2009 a tuto dokumentaci věnujeme jeho památce.

\item Hansi Hagenovi za mnohá testování a přípravu jeho balíku \ConTeXt\ 
(\url{https://pragma-ade.com}) pracujícího uvnitř systému \TL a za neustály rozvoj \TeX{}u.

\item \Thanh ovi, Martinu Schröderovi a pdf\TeX\ týmu (\url{http://pdftex.org}) 
za pokračující rozšiřování možností \TeX u.

\item Hartmutu Henkelovi za významný příspěvek k~vývoji
pdf\TeX u, Lua\TeX u atp.

\item Shunshaku Hirata, za originálnejší a pokračování práce na DVIPDFM$x$. 
%for much original and continuing on DVIPDFM$x$.

\item Tacu Hoekwaterovi za významné úsilí při obnovení vývoje \MP u i
samotného (Lua)\TeX{}u (\url{http://luatex.org}), za začlenění 
\ConTeXt{}u do systému \TL, 
za přidání vícevláknové funkčnosti programu Kpathsea a mnoho dalšího.

\item Khaledu Hosnymu, za podstatnou práci na \XeTeX{}u, DVIPDFM$x$ a
za úsilí s~arabskými i jinými fonty.

\item Paw{\l}u Jackowskému za windowsový instalátor \cmdname{tlpm} 
a Tomaszi {\L}uczakovi za \cmdname{tlpmgui}, používaný 
v~předchozích vydáních.

\item Akiru Kakutovi, za poskytnutí windowsovských binárek
z~jeho distribucí W32TEX a W64TEX pro japonský \TeX{} 
(\url{http://w32tex.org}) a za množství dalších příspěvků k~vývoji.

\item Jonathanu Kewovi a \acro{SIL} za vyvinutí pozoruhodného systému \XeTeX{} a 
za čas a trápení při jeho integraci do \TL{}, 
stejně tak za výchozí verzi instalačního programu Mac\TeX{} a kromě toho za námi 
doporučený pomocný program \TeX{}works. % front-end

\item Hironorimu Kitagawa za údržbu (e)p\TeX{}u a související podporu.

\item Dicku Kochovi za údržbu Mac\TeX{}u (\url{http://tug.org/mactex})
ve velmi blízkém tandemu s~\TL{} a za jeho skvělý přístup. 
% for his great good cheer in doing so.

\item Reinhardu Kotuchovi za důležitý příspěvek k~infrastruktuře a 
instalačnímu programu \TL{} 2008, rovněž za úsilí při výzkumu Windows, 
za skript \texttt{getnonfreefonts} a mnoho dalšího.

\item Siep Kroonenbergové rovněž za důležitý příspěvek k~infrastruktuře 
a instalačnímu programu \TL{} 2008, zvláště pod Windows, a za množství práce 
při aktualizaci této příručky, popisující tyto vlastnosti.

\item Clerku Ma za opravu a rozšíření stroje.  %for engine bug fixes and extensions.

\item Mojce Miklavec za množství pomoci s~\ConTeXt{}em, vybudování 
mnoha souborů binárek a mnohem víc.


\item Heikovi Oberdiekovi za balík \pkgname{epstopdf} a mnohé další, 
za kompresi velikých datových souborů \pkgname{pst-geo} tak, že 
jsme je mohli zařadit do instalace, a především za jeho mimořádnou práci 
na balíku \pkgname{hyperref}.

\item Phelypemu Oleiniku za %group-delimited 
skupinově oddělený \cs{input} pro různé stroje v~roce 2020 a mnohé další.

\item Petru Olšákovi, který velmi pečlivě kontroloval 
svou českou a slovenskou podporu na \TeXLive.

\item Toshiu Oshimovi za jeho prohlížeč \cmdname{dviout} do Windows.

\item Manuelu Pégourié-Gonnardovi za pomoc při aktualizaci balíků,
vylepšení dokumentace a rozvoj dokumentu \cmdname{texdoc}.

\item Fabrice Popineau, za původní podporu Windows na \TL{}
a za francouzskou dokumentaci.

\item Norbertu Preiningovi, hlavnímu architektovi současné infrastruktury a 
instalačního programu \TL{}, za koordinaci Debian 
verze \TL{} (společně s~Frankem K\"usterem) a za vykonání obrovského množství práce
v~průběhu naší cesty.

\item Sebastianu Rahtzovi za původní vytvoření systému \TL{}
a za jeho údržbu po mnoho let. Sebastian zemřel v~březnu 2016 
a pokračující dílo věnujeme jeho památce.

\item Luigimu Scarsovi za pokračující vývoj MetaPostu, Lua\TeX{}u a
mnoho dalšího.

\item Andreasu Schererovi za \texttt{cwebbin}, implementaci CWEB použitou \TL{}, a pokračující údržbu původního CWEBu.

\item Takujimu Tanakovi za údržbu (e)p\TeX{}u a související podporu.

\item Tomaszi Trzeciakovi za všestrannou pomoc s~Windows.

\item Vladimiru Volovichovi za významnou pomoc s~přenositelností
a jinými problémy údržby, obzvláště za to, že udělal 
realizovatelným zahrnutí \cmdname{xindy} do~\TL.

\item Staszku Wawrykiewiczovi, hlavnímu testérovi všeho na \TL{} a
koordinátorovi mnoha důležitých polských příspěvků:
fontů, windowsové instalace a dalších. Staszek zemřel v únoru 2018 a my
věnujeme pokračující práci jeho paměti.

\item Olafu Weberovi za jeho pečlivou údržbu \Webc v~minulých letech.

\item Gerbenu Wierdovi za vytvoření a údržbu původní podpory \MacOSX.

\item Grahamu Williamsovi, tvůrci \TeX\ Catalogue.

\item Josephovi Wrightovi za množství práce umožňující dostupnost tytéž primitivní funkcionality 
pro různé stroje.
%, for much work on making the same primitive functionality available across engines.

\item Hironobu Yamashitovi za množství práce na p\TeX{}u a souvisící podporu.

\end{itemize*}

Tvůrci binárek:
Ettore Aldrovandi (\pkgname{i386-solaris}, \pkgname{x86\_64-solaris}),
Marc Baudoin (\pkgname{amd64-netbsd}, \pkgname{i386-netbsd}),
Ken Brown (\pkgname{i386-cygwin}, \pkgname{x86\_64-cygwin}),
Johannes Hielschier (\pkgname{aarch64-linux}),
Simon Dales (\pkgname{armhf-linux}),
Akira Kakuto (\pkgname{win32}),
Dick Koch (\pkgname{x86\_64-darwin}),
Mojca Miklavec (\pkgname{amd64-freebsd},
\pkgname{i386-freebsd},
\pkgname{x86\_64-darwinlegacy},
\pkgname{i386-solaris}, \pkgname{x86\_64-solaris},
\pkgname{sparc-solaris}),
Norbert Preining (\pkgname{i386-linux},
\pkgname{x86\_64-linux},
\pkgname{x86\_64-linuxmusl}).
Pro informaci o~procesu budování \TL{}, viz
\url{https://tug.org/texlive/build.html}.

Překladatelé této příručky:
Carlos Enriquez Figueras (španělština),
Jjgod Jiang, Jinsong Zhao, Yue Wang, \& Helin Gai (čínština),
Nikola Lečić (srbština),
Marco Pallante \& Carla Maggi (italština),
Denis Bitouz\'e \& Patrick Bideault (francouzština),
Petr Sojka \& Ján Buša (čeština\slash slovenština),%
\footnote{Ke korektuře českého a slovenského překladu přispěli 
v~letech 2001--2015 kromě výše uvedených autorů Jaromír Kuben, 
Milan Matlák, Zbyněk Michálek, Tomáš Obšívač,
Karel Píška, Tomáš Polešovský, Libor Škarvada, Zdeněk Wagner a další.
Michal Mádr editoval \code{cs.po} a přeložil soubor \code{README.EN}.}
Boris Veytsman (ruština), 
Zofia Walczak (polština),
Uwe Ziegenhagen (němčina).  
Webovská stránka dokumentace \TL{} je \url{https://tug.org/texlive/doc.html}.

Samozřejmě nejdůležitější poděkování patří Donaldu Knuthovi, 
především za vymyšlení \TeX u a také za to, že ho věnoval světu.


\section{Historie vydání}
\label{sec:history}

\subsection{Minulost}

Diskuse začala koncem roku 1993, kdy holandská skupina uživatelů \TeX{}u 
NTG začala práci na \CD{} 4All\TeX{} pro uživatele MS-DOSu, doufajíc,
že doba nazrála pro vydání jednoho \CD{} pro všechny
systémy. Byl to na svou dobu příliš ambiciózní cíl, ale
nenastartoval jen velmi úspěšné 4All\TeX{} \CD{}, ale také
pracovní skupinu TUGu o~\emph{\TeX{} Directory Structure}
(\url{https://tug.org/tds}), která specifikovala,
jak vytvořit konzistentní a spravovatelnou
kolekci \TeX{}ových souborů. Kompletní draft \TDS{} byl publikován 
v~prosincovém čísle časopisu \textsl{TUGboat} v~roce 1995 a
hned ze začátku bylo jasné, že jedním z~žádaných produktů bude
vzorová struktura \CD{}. Distribuce, kterou nyní máte, je přímým
výstupem práce této pracovní skupiny. Evidentní úspěch \CD{} 4All\TeX{}  
ukázal, že i unixoví uživatelé by toužili po podobně jednoduchém 
systému, a to je také jedno z~hlavních aktiv~\TeXLive.

Nejprve jsme vytvořili unixové \TDS{} \CD{} na podzim
1995 a rychle identifikovali \teTeX{} Thomase Essera
jako ideální systém, jelikož již měl multi\-platformní
podporu a byl koncipován s~perspektivou
přenositelnosti. Thomas souhlasil s~pomocí a seriózní práce započala
začátkem roku 1996. První vydání se uskutečnilo
v~květnu 1996. Začátkem 1997 Karl Berry dokončil nové hlavní
vydání \Webc, které obsahovalo téměř všechny vlastnosti,
které Thomas Esser přidal do \teTeX{}u, a tak jsme se rozhodli
druhé vydání \CD{} postavit na standardním
\Webc, s~přidáním skriptu \texttt{texconfig} z~\teTeX{}u.
Třetí vydání \CD{} bylo založeno na další revizi
\Webc, 7.2, provedené Olafem Weberem; a jelikož zároveň
byla hotova nová verze \teTeX{}u, \TL{} obsahoval 
téměř všechna její vylepšení. Podobně čtvrté vydání používalo novou 
verzi \teTeX{}u a nové vydání \Webc{} (7.3).
\TL{} nyní obsahuje i kompletní systém pro Windows díky Fabrice Popineau.

Pro páté vydání (březen 2000) bylo mnoho částí \CD{} revidováno
a zkontrolováno a byly aktualizovány stovky
balíků. Detaily o~balících byly uloženy v~souborech \acro{XML}.
Ale hlavní změnou pro \TeX\ Live~5 bylo vynětí
softwaru, na kterém byla jakákoliv omezení na šíření
(non-free software). Vše uložené na \TL{} je nyní
slučitelné s~tzv.\ \singleuv{Debian Free Software Guide\-lines}
(\url{https://debian.org/intro/free}); udělali jsme vše možné,
abychom zkontrolovali licenční podmínky všech balíků,
ale budeme vděčni za upozornění na jakékoli chyby.

Šesté vydání (červenec 2001) mělo aktualizovaného materiálu
ještě více. Hlavní změnou byl nový instalační přístup:
uživatel může volit instalační kolekce.
Byly kompletně reorganizovány jazykové kolekce,
takže jejich výběrem se instalují
nejen makra, fonty, ale je také připraven odpovídající
soubor \texttt{language.dat}.

Sedmé vydání v~roce 2002 mělo podstatné rozšíření v~přidání
podpory \MacOSX{}, kromě množství aktualizací balíků a programů.
Důležitým cílem byla opětná integrace s~\teTeX{}em
a korekce odchylek z~pátého a šestého vydání.

\subsubsection{2003}

V~roce 2003 se neustálou smrští oprav a rozšíření stalo to, že
velikost \TL{} již neumožnila jeho směstnání
na jedno \CD{}, a tak došlo k~rozdělení na tři různé distribuce
(viz oddíl~\ref{sec:tlcoll-dists} na
straně~\pageref{sec:tlcoll-dists}). Navíc:

\begin{itemize*}
\item Na žádost \LaTeX{} týmu jsme změnili standardní příkazy
      \cmdname{latex} a \cmdname{pdflatex} tak, že nyní používají
      \eTeX{} (viz strana~\pageref{text:etex}).
\item Byly přidány a jsou nyní doporučovány k~používání
      nové fonty Latin Modern.
\item Byla zrušena podpora OS Alpha OSF
      (podpora HPUX byla zrušena již dříve), jelikož se 
      nenašel nikdo, kdo by na těchto platformách zkompiloval nové binárky.
\item Instalační program Windows byl změněn podstatným způsobem; poprvé
      bylo integrováno uživatelské prostředí založené na XEmacsu.
\item Důležité pomocné programy pro Windows
      (Perl, \GS{}, Image\-Magick, Ispell) jsou nyní instalovány
      do instalačního adresáře \TL{}.
\item Mapovací soubory jmen fontů pro \cmdname{dvips}, \cmdname{dvipdfm}
      a \cmdname{pdftex} jsou nyní generovány novým programem
      \cmdname{updmap} a instalovány do \dirname{texmf/fonts/map}.
%+\item \TeX{}, \MF{}, and \MP{} now output 8-bit input
%+      characters as themselves in output (e.g., \verb|\write|) files,
%+      log files, and the terminal, i.e., \emph{not} translated using the
%+      \verb|^^| notation.  In \TL{}~7, this translation was
%+      dependent on the system locale settings; now, locale settings do
%+      not influence the \TeX{} programs' behavior.  If for some reason
%+      you need the \verb|^^| output, rename the file
%+      \verb|texmf/web2c/cp8bit.tcx|.  (Future releases will have cleaner
%+      ways to control this.)
\item \TeX{}, \MF{} a \MP{} nyní dávají na výstup  
      většinu vstupních 8-bitových
      znaků (32 a výš) bez konverze (například do souborů
      zapisovaných pomocí \verb|\write|,
      souborů log, na terminál), a tedy \emph{nejsou překládány} do
      sedmibitové \verb|^^| notace.  V~\TL{}~7 bylo toto mapování
      závislé na nastavení systémových locale; nyní již nastavení locale
      \emph{neovlivní} chování \TeX{}u.  Pokud z~nějakých důvodů
      potřebujete výstup s~\verb|^^|, přejmenujte soubor
      \verb|texmf/web2c/cp8bit.tcx|. Příští verze bude mít čistší řešení.
\item Tato dokumentace byla podstatným způsobem přepracována.
\item Konečně, jelikož čísla verzí \TL{} již příliš narostla,
      verze je nyní identifikována rokem vydání: \TL{} 2004.
\end{itemize*}

\subsubsection{2004}
Rok 2004 přinesl mnoho změn:
\begin{itemize}

\item Pokud máte lokálně instalovány fonty, které 
využívají své vlastní podpůrné soubory
\filename{.map} nebo (mnohem méně pravděpodobně) \filename{.enc}, 
možná budete nuceni tyto soubory přesunout.

Soubory \filename{.map}  jsou nyní vyhledávány jen v~podaresářích
\dirname{fonts/map} (v~každém stromě \filename{texmf}), podle cesty
\envname{TEXFONTMAPS}.  Podobně soubory \filename{.enc} jsou hledány
jen v~podadresářích \dirname{fonts/enc}, podle cesty \envname{ENCFONTS}.
\cmdname{updmap} se pokusí vypsat varování o~problematických souborech.

O~metodách zpracování té které informace prosíme viz
\url{https://tug.org/texlive/mapenc.html}.

\item \TKCS\ byla rozšířena přidáním instalovatelného \CD, 
založeného na \MIKTEX u, pro ty, kteří dávají přednost 
této implementaci před Web2C. 
%Také loňské demo \CD{} bylo nahrazeno instalovatelným \CD{} pro Windows. 
Viz oddíl~\ref{sec:overview.tl} (stránka~\pageref{sec:overview.tl}).

\item Uvnitř \TL byl velký strom \dirname{texmf} předešlých vydání 
nahrazen třemi: \dirname{texmf}, \dirname{texmf-dist} a \dirname{texmf-doc}.  
Viz oddíl~\ref{sec:tld} (stránka~\pageref{sec:tld})
a soubory \filename{README} pro každý z~nich.

\item Všechny vstupní soubory týkajíci se \TeX u jsou teď soustředěny 
v~podadresáři \dirname{tex} stromů  \dirname{texmf*} a nemají oddělené 
sourozenecké adresáře \dirname{tex}, \dirname{etex},
\dirname{pdftex}, \dirname{pdfetex} atd.  Viz
\CDref{texmf.doc/doc/english/tds/tds.html\#Extensions}
{\texttt{texmf-doc/doc/english/tds/tds.html\#Extensions}}.

\item Pomocné dávky (neměly by být volány uživateli) jsou teď 
umístěny v~novém podadresáři \dirname{scripts} stromů \dirname{texmf*}
a mohou být vyhledávány prostřednictvím 
\verb|kpsewhich -format=texmfscripts|.  Pokud tedy máte programy 
volající tyto dávky, budou muset být nastaveny. Viz
\CDref{texmf.doc/doc/english/tds/tds.html\#Scripts} 
{\texttt{texmf-doc/doc/english/tds/tds.html\#Scripts}}. 

\item Téměř všechny formáty umožňují většinu znaků tisknout 
bez konverze pomocí překladového souboru 
\filename{cp227.tcx} místo jejich konverze pomocí 
\verb|^^| notace.  Konkrétně znaky na pozicích 32--256, 
plus \uv{tab}, \uv{vertikální tab} a \uv{form feed} jsou považovány 
za tisknutelné a nejsou konvertovány. Výjimky tvoří plain \TeX\
(jen 32--127 jsou tisknutelné), \ConTeXt\ (0--255 tisknutelné) a formáty
systému \OMEGA. Toto implicitní chování je téměř stejné jako 
v~\TL\,2003, ale je implementováno čistěji, s~více možnostmi přizpůsobení.
Viz \CDref{texmf-dist/doc/web2c/web2c.html\#TCX-files}
{\texttt{texmf-dist/doc/web2c/web2c.html\#TCX-files}}. 
(Mimochodem, se vstupem Unicode může \TeX\ na výstupu vypsat posloupnosti
částečných znaků při výpisu chybových kontextů, protože je bytově orientován.)

\item \textsf{pdfetex} je teď implicitní stroj pro všechny formáty 
kromě samotného (plain) \textsf{tex}u.  (Ten samozřejmě generuje DVI,
když je spuštěn jako \textsf{latex} atd.)  To znamená, kromě jiných
věcí, že  mikrotypografické rozšíření \textsf{pdftex}u je dostupné
v~\LaTeX u, \ConTeXt u atd., stejně jako rozšíření 
\eTeX u (\OnCD{texmf-dist/doc/etex/base/}). 

To také znamená, že užití balíku \pkgname{ifpdf} (pracuje s~plainem 
i \LaTeX em) nebo ekvivalentního kódu je \emph{důležitější 
než kdykoliv předtím}, protože jednoduché testování, zda 
je \cs{pdfoutput} nebo nějaký jiný primitiv definován, 
není spolehlivý způsob určení, zda je generován PDF 
výstup. Tento zpětný krok jsme udělali co nejvíc kompatibilní 
letos, ale v~příštím roce \cs{pdfoutput} může 
být definován, dokonce i když se zapisuje do DVI.

\item pdf\TeX\ (\url{http://pdftex.org/}) má množství nových vlastností:
 \begin{itemize*}
  \item \cs{pdfmapfile} a \cs{pdfmapline} poskytují podporu mapování 
    fontů uvnitř dokumentu.  
  \item Mikrotypografické rozšíření fontu může být použito jednodušeji.\newline
    \url{http://www.ntg.nl/pipermail/ntg-pdftex/2004-May/000504.html}.
  \item Všechny parametry, které byly předtím nastavovány ve 
    zvláštním konfiguračním souboru \filename{pdftex.cfg}, musejí 
    teď být nastaveny pomocí primitivů, obyčejně 
    v~\filename{pdftexconfig.tex}; \filename{pdftex.cfg} není 
    dále podporován.  Všechny existující \filename{.fmt} soubory 
    musí být znovu vytvořeny při změně souboru \filename{pdftexconfig.tex}.
  \item Další informace viz manuál pdf\TeX u: \OnCD{texmf/doc/pdftex/manual}.
 \end{itemize*}

\item Primitiv \cs{input} v~\cmdname{tex}u (a \cmdname{mf} a
\cmdname{mpost}) teď akceptuje dvojité uvozovky se jmény obsahujícími 
mezery a s~jinými zvláštními znaky. Typické příklady:
\begin{verbatim}
\input "filename with spaces"   % plain
\input{"filename with spaces"}  % latex
\end{verbatim}
Pro další informace viz manuál Web2C: \OnCD{texmf-dist/doc/web2c}.

\item Podpora enc\TeX u je nyní zahrnuta uvnitř Web2C, 
v~důsledku toho ve všech \TeX ových programech, použitím 
volby \optname{-enc} -- \emph{jen v~případě, že formáty 
jsou vytvořeny}. enc\TeX\ podporuje obecné překódování 
vstupu a výstupu, čímž umožňuje plnou podporu kódování 
Unicode (v~UTF-8). Viz \OnCD{texmf-dist/doc/generic/enctex/} a 
\url{http://olsak.net/enctex.html}.

\item Aleph, nový stroj kombinující \eTeX\ a \OMEGA, je nyní k~dispozici.
Informace najdete na \OnCD{texmf-dist/doc/aleph/base}
a \url{https://texfaq.org/FAQ-enginedev}. Formát pro 
Aleph, založený na \LaTeX u, se jmenuje \textsf{lamed}.  

\item Nejnovější vydání \LaTeX u má novou verzi LPPL -- teď
oficiálně schválená Debian licence. Rozmanité další aktualizace najdete
v~souborech \filename{ltnews} v~\OnCD{texmf-dist/doc/latex/base}. 
 
\item Je dodán \cmdname{dvipng}, nový program pro konvertování DVI na 
obrazové soubory PNG. Viz \url{https://ctan.org/pkg/dvipng}. 

\item Zredukovali jsme balík \pkgname{cbgreek} na \uv{středně} velkou 
sadu fontů, se souhlasem a na radu autora (Claudio Beccari).
Odstraněné fonty jsou neviditelné, obrysové, a průhledné,
relativně zřídka se používají a my jsme potřebovali místo.  
Úplná sada je dostupná z~CTAN (\url{https://ctan.org/pkg/cbgreek-complete}).
 
\item \cmdname{oxdvi} byl odebrán; použijte jednoduše \cmdname{xdvi}.

\item Příkazy \cmdname{ini} a \cmdname{vir} (linky) pro
\cmdname{tex}, \cmdname{mf} a \cmdname{mpost} se již nevytvářejí,
například \cmdname{initex}.  Funkčnost instrukce \cmdname{ini} byla po
celá léta přístupná prostřednictvím volby \optname{-ini} na~příkazovém řádku. 

\item Podpora platformy \textsf{i386-openbsd} byla zrušena. 
Jelikož balíček \pkgname{tetex} v~BSD Ports systému 
je dostupný a GNU/Linux a FreeBSD binárky byly 
dostupné, zdálo se nám, že čas dobrovolníků může být využit
lépe někde jinde.
\item Na \textsf{sparc-solaris} (přinejmenším) jste možná 
museli nastavovat proměnné prostředí
\envname{LD\_LIBRARY\_PATH}, aby běžely programy \pkgname{t1utils}.  
Je to tím, že jsou kompilovány v~C++,
a neexistuje standardní umístění \singleuv{runtime} knihoven.  
(To není novinka roku 2004, ale nebylo to dřív zdokumentováno.)  
Podobně na \textsf{mips-irix} jsou \singleuv{runtime} knihovny pro MIPS
7.4 nezbytné.
\end{itemize}

\subsubsection{2005}

Rok 2005 přinesl jako obvykle množství modernizací balíčků a programů.
Infrastruktura zůstala relativně stabilní z~roku 2004, ale nutně také
nastaly určité změny:

\begin{itemize}

\item Byly zavedeny nové skripty \cmdname{texconfig-sys}, \cmdname{updmap-sys} a
      \cmdname{fmtutil-sys}, které mění konfiguraci v~systémových stromech.  
      Skripty \cmdname{texconfig},
      \cmdname{updmap} a \cmdname{fmtutil} teď mění uživatelské (user-specific) soubory
      v~\dirname{$HOME/.texlive2005}.  %$

\item Na specifikaci stromů obsahujících konfigurační soubory (uživatelské, resp. systémové)
      byly zavedeny odpovídající nové proměnné \envname{TEXMFCONFIG}, resp.
      \envname{TEXMFSYSCONFIG}.  Budete tedy možná potřebovat přesunout osobní verze
      souborů \filename{fmtutil.cnf} a \filename{updmap.cfg} na tato místa; jiná volba je 
      předefinování \envname{TEXMFCONFIG} nebo \envname{TEXMFSYSCONFIG} v~souboru 
      \filename{texmf.cnf}. V~každém případě skutečná pozice těchto souborů a
      hodnoty \envname{TEXMFCONFIG} a \envname{TEXMFSYSCONFIG}
      se musejí shodovat. Viz oddíl~\ref{sec:texmftrees}, 
      strana~\pageref{sec:texmftrees}.

\item Loni jsme ponechali \verb|\pdfoutput| a jiné primitivy nedefinovány
      pro výstup \dvi, přestože byl používán program \cmdname{pdfetex}.  
      Letos, jak jsme slíbili, jsme odstranili toto zpětně kompatibilní
      opatření. Tedy když váš dokument používá 
      \verb|\ifx\pdfoutput\undefined|
      na testování výstupu do formátu \acro{PDF}, je nutné ho změnit. Můžete
      k~tomu použít balík \pkgname{ifpdf.sty} (který funguje 
      v~plain \TeX u i \LaTeX u), nebo použít jeho logiku.
      
\item Loni jsme změnili většinu formátů tak, aby vypisovaly (8bitové) znaky 
      tak, jak jsou (viz předcházející sekci).  Nový TCX soubor
      \filename{empty.tcx} teď poskytuje jednodušší cestu k~dosažení 
      původního zápisu \verb|^^|, když si to budete přát, jako v~případě:
\begin{verbatim}
latex --translate-file=empty.tcx yourfile.tex
\end{verbatim}

\item Nový program \cmdname{dvipdfmx} je zařazen pro 
převedení z~\acro{DVI} do \acro{PDF}; 
ten je platnou aktualizací programu
\cmdname{dvipdfm} (který je též ještě k~dispozici, i když ho nedoporučujeme).
      
\item Nové programy \cmdname{pdfopen} a \cmdname{pdfclose} byly přidány, aby
poskytly možnost znovu otevřít PDF 
soubory v~programe Adobe Acrobat Reader bez jeho restartu.  
(Jiné prohlížeče PDF, především \cmdname{xpdf},
\cmdname{gv} a \cmdname{gsview}, nikdy netrpěly tímto problémem.)      
            
\item Kvůli důslednosti proměnné \envname{HOMETEXMF} a
      \envname{VARTEXMF} byly přejmenovány na \envname{TEXMFHOME}, resp. na
      \envname{TEXMFSYSVAR}.  Je tu také \envname{TEXMFVAR}, která je
      implicitně uživatelsky závislá (user-specific). Viz první bod výše.
\end{itemize}

\subsubsection{2006--2007}

V~letech 2006--2007 byl rozsáhlým přírůstkem na \TL{} program \XeTeX{},
přístupný jako programy \texttt{xetex} a \texttt{xelatex}; viz
\url{https://scripts.sil.org/xetex}.

\MP\ byl také podstatně aktualizován, s~mnoha plány do budoucnosti 
(\url{https://tug.org/metapost/articles}), podobně pdf\TeX{}
(\url{https://tug.org/applications/pdftex}).

Formát \TeX\ \filename{.fmt} (vysokorychlostní formát) a podobně soubory
\MP\ a \MF\ jsou teď uloženy v~podadresářích \dirname{texmf/web2c},
namísto přímého uložení v~něm (ačkoliv je adresář stále prohledáván, 
v~zájmu stávajících formátů \filename{.fmt}). Podadresáře nesou jména 
používaných programů, například \filename{tex}
nebo \filename{pdftex} nebo \filename{xetex}. Tato změna by měla při 
běžném používání zůstat nepostřehnuta.

Program (plain) \texttt{tex} již nečte první řádky \texttt{\%\&}, aby
určil, jaký formát má spustit; je to čistý knuthovský \TeX.
(\LaTeX\ a všechny ostatní programy stále čtou řádky \texttt{\%\&}).
Pochopitelně se během roku (jako obvykle) vyskytly stovky jiných
aktualizací balíků a programů.  Jako obyčejně, zkontrolujte, prosím,
aktualizace na CTANu (\url{https://ctan.org}).

Strom zdrojových textů je nyní uložen v~Subversion, se 
standardním webovským rozhraním pro jeho prohlížení, kam směruje 
odkaz z~naší domovské stránky. Třebaže není v~konečné verzi
viditelný, očekáváme, že to poskytne základ pro stabilní rozvoj
v~letech následujících.

Nakonec, v~květnu 2006 Thomas Esser oznámil, že už nebude aktualizovat
te\TeX{} (\url{https://tug.org/tetex}).  Výsledkem je nárůst zájmu o~\TL{}, 
především mezi distributory \GNU/Linuxu.  
(Na \TL{} se nachází nové instalační schéma \texttt{tetex}u, 
poskytující přibližný ekvivalent.)  Doufáme, že to možná povede
ke zlepšení prostředí \TeX{}u pro všechny.

\subsubsection{2008}

V~roce 2008 byla celá infrastruktura \TL{} přebudována a znovu
implementována. Úplná informace o~instalaci je nyní
uložena v~textovém souboru \filename{tlpkg/texlive.tlpdb}.

Mezi dalšími věcmi je konečně možná aktualizace instalace \TL{}
z~internetu po předchozí instalaci. Tuto vlastnost poskytoval 
MiK\TeX\ již řadu let.  Předpokládáme pravidelnou aktualizaci nových
balíků po jejich vydání na \CTAN{}u.

Obsažený je významnější nový stroj (engine) Lua\TeX\ (\url{http://luatex.org});
kromě lepší přizpůsobivosti v~sázení je možno tento vynikající skriptovací jazyk
použít jak uvnitř, tak i~mimo \TeX{}ovské dokumenty.

Podpora Windows a unixových platforem je nyní jednotnější.
Zejména většina skriptů Perlu a Lua je teď k~dispozici pod Windows, 
s~použitím Perlu distribuovaného na~\TL.

Nový skript \cmdname{tlmgr} (sekce~\ref{sec:tlmgr}) je všeobecné rozhraní
pro správu \TL{} po předchozí instalaci. Ovládá aktualizaci balíků
a následující znovuvytvoření formátů, mapovacích souborů a jazykových souborů,
volitelně zahrnující lokální doplňky.

S~příchodem programu \cmdname{tlmgr} jsou nyní činnosti programu
\cmdname{texconfig} na editaci formátů a konfiguračních souborů dělení
slov blokované.

Program \cmdname{xindy} pro tvorbu rejstříků
(\url{http://xindy.sourceforge.net/}) je nyní zahrnut na většině
platforem.  

Nástroj \cmdname{kpsewhich} může nyní ohlásit všechny
výskyty pro daný soubor (option \optname{-{}-all}) a omezené výskyty
pro daný podadresář (option \optname{-{}-subdir}).

Program \cmdname{dvipdfmx} zahrnuje nyní funkčnost extrakce informace
o \singleuv{bounding boxu}, pomocí povelu \cmdname{extractbb}; toto
byl jeden z~posledních rysů, které poskytoval \cmdname{dvipdfm}, avšak ne
\cmdname{dvipdfmx}.

Fontové přezdívky \filename{Times-Roman}, \filename{Helvetica} atd.
byly odstraněny.  Různé balíky očekávaly jejich různé chování (především,
že budou mít různé kódování) a nenalezlo se vhodné řešení tohoto problému.

Formát \pkgname{platex} byl odstraněn pro konflikt jmen s~japonštinou
\pkgname{platex}; podporu polštiny nyní zajišťuje balík \pkgname{polski}.

Soubory \web{}ovských řetězců (pool) jsou nyní zkompilované do binárek pro
usnadnění aktualizací.

A nakonec, v~tomto vydání jsou zahrnuty změny provedené Donaldem Knuthem
v~jeho úpravách \TeX u roku 2008 (\singleuv{\TeX\ tuneup of 2008}).
Viz \url{https://tug.org/TUGboat/Articles/tb29-2/tb92knut.pdf}.

\subsubsection{2009}

Od roku 2009 je standardní výstupní formát Lua\AllTeX\ PDF, pro využití
výhody Lua\TeX{}ovské podpory Open\-Type a jiné. Nové binárky nazvané
\code{dviluatex} a \code{dvilualatex} spouští Lua\TeX\ při výstupu DVI.
Domovská stránka Lua\TeX{}u je \url{http://luatex.org}.

Původní systém (engine) Omega a formát Lambda format byly odstraněny
po diskusích s~autory systému Omega. Zůstaly aktualizované
programy Aleph a Lamed, podobně jako pomocné programy systému Omega.

Obsaženo je nové vydání fontů AMS \TypeI včetně Computer
Modern: do zdrojáků Metafontu bylo zapracováno několik málo změn tvarů,
které za léta udělal Donald Knuth a byl aktualizován hinting fontů. 
Tvary fontů Euler byly důkladně překresleny Hermannem Zapfem (viz
\url{https://tug.org/TUGboat/Articles/tb29-2/tb92hagen-euler.pdf}). Ve všech 
případech zůstaly metriky \emph{nezměněny}. Domovská stránka AMS fontů je
\url{https://ams.org/tex/amsfonts.html}.

Pomocný program -- nový \GUI{} editor -- \TeX{}works 
je zahrnut pro Windows, ale také v~Mac\TeX u.  
Pro jiné platformy a další informace viz domovskou 
stránku \TeX{}works \url{https://tug.org/texworks}. 
Je to multi-platformní prostředí
inspirované editorem TeXShop v~\MacOSX, zaměřené 
na jednoduché použití.

Grafický program Asymptote je zahrnutý pro více platforem. Realizuje
textově založený jazyk pro popis grafiky, blízký k~MetaPostu,
avšak s~pokročilou podporou 3D a jinými vlastnostmi. 
Jeho domovská stránka je \url{https://asymptote.sourceforge.io}.

Samostatný program \code{dvipdfm} byl nahrazen programem 
\code{dvipdfmx}, který pod tímto jménem pracuje ve zvláštním 
režimu kompatibility.  % under that name
\code{dvipdfmx} zahrnuje podporu CJK a má nahromaděny 
mnohé další úpravy za léta od posledního vydání
\code{dvipdfm}. 

Binárky pro platformy \pkgname{cygwin} a \pkgname{i386-netbsd} jsou nyní
v~\TL{} zahrnuty, zatímco nám bylo oznámeno, že uživatelé 
OpenBSD získají \TeX\ pomocí jejich
systémů balíků a navíc se objevily potíže při vytváření binárek,
které by měly šanci fungovat na více než jedné verzi.

Z dalších menších změn: nyní používáme \pkgname{xz} kompresi,
stabilní náhradu za \pkgname{lzma}
(\url{https://tukaani.org/xz/}); a literál~|$| je povolen v~názvech
souborů pokud není uveden na začátku jména známé proměnné;
knihovna Kpathsea je teď více\-vláknová (použitelné v~Meta\-Postu); 
budování celého \TL{} je nyní založeno na systému Automake.

Závěrečná poznámka o~minulosti: všechna vydání \TL{} spolu s~podpůrným 
materiálem jako např. \CD\ labels jsou dostupná na
\url{ftp://tug.org/historic/systems/texlive}.

\subsubsection{2010}
\label{sec:2010news}

V~roce 2010 je předvolenou verzí pro výstup PDF verze 1.5, umožňující větší kompresi.  
To se týká všech nástrojú \TeX{}u používaných na vytváření PDF a \code{dvipdfmx}.  
Načtením \LaTeX{}ovského balíčku \pkgname{pdf14} se provede zpětná změna na PDF~1.4, 
nebo nastavte |\pdfminorversion=4|.

pdf\AllTeX\ nyní \emph{automaticky} konvertuje požadovaný soubor ve formátu Encapsulated
PostScript (EPS) na PDF prostřednictvím balíku \pkgname{epstopdf}, když a pokud je načten 
konfigurační soubor \LaTeX{}u \code{graphics.cfg} a pokud je výstup do PDF.  
Implicitní nastavení jsou zamýšlena pro eliminaci možností přepsání 
ručně vytvořených PDF souborů, ale můžete také docela zakázat načtení \code{epstopdf} zadáním
|\newcommand{\DoNotLoadEpstopdf}{}| (nebo |\def...|) před deklarací 
\cs{documentclass}. Balík \pkgname{epstopdf} rovněž nebude zaveden
pokud bude použit balík \pkgname{pst-pdf}. 
Pro další podrobnosti viz dokumentaci balíku \pkgname{epstopdf}
(\url{https://ctan.org/pkg/epstopdf-pkg}).

Další podobnou změnou je, že vykonání několika málo externích příkazů z~\TeX{}u, 
prostřednictvím vlastnosti \cs{write18}, je nyní implicitně povoleno. Tyto
příkazy jsou \code{repstopdf}, \code{makeindex}, \code{kpsewhich},
\code{bibtex} a \code{bibtex8}; seznam je uveden v~\code{texmf.cnf}.  
Prostředí, která musí zakázat všechny takové externí povely,
mohou zrušit tuto volbu v~instalátoru (viz
oddíl~\ref{sec:options}), nebo po instalaci přepsat hodnotu
spuštěním |tlmgr conf texmf shell_escape 0|.

Ještě další podobnou změnou je to, že \BibTeX\ a Makeindex nyní implicitně odmítají zapsat
své výstupní soubory do libovolného adresáře (jako samotný \TeX).  
Je to proto, že nyní mohou být povolené pro použití omezeným \cs{write18}.  
Aby se to změnilo, může být nastavena proměnná prostředí \envname{TEXMFOUTPUT}
nebo změneno nastavení |openout_any|.

\XeTeX\ nyní podporuje posun (kerning) okrajů podél stejných linií jako pdf\TeX.
(Expanze fontů není aktuálně podporována.)

Program \prog{tlmgr} nyní standardně ukládá jednu zálohu každého 
aktualizovaného balíku (\code{tlmgr option autobackup 1}), 
tudíž přerušené aktualizace balíků mohou být snadno vráceny příkazem \code{tlmgr restore}.  
Pokud děláte poinstalační aktualizace a nemáte dostek místa na disku pro zálohy, 
spusťte \code{tlmgr option autobackup 0}.

Byly zařazeny nové programy: nástroj (engine) p\TeX\ a příbuzné pomůcky pro sazbu Japonštiny; 
program \BibTeX{}U pro \BibTeX umožňující použití Unicode; utility \prog{chktex} 
(původně z~\url{https://www.nongnu.org/chktex/})
na kontrolu dokumentů \AllTeX{}u; překladač \prog{dvisvgm} z~DVI do SVG
(\url{https://dvisvgm.de}).

Jsou dodány binárky těchto nových platforem: \code{amd64-freebsd},
\code{amd64-kfreebsd}, \code{i386-freebsd}, \code{i386-kfreebsd},
\code{x86\_64-darwin}, \code{x86\_64-solaris}.

Změna \TL{} 2009, které jsme si nevšimli: četné binárky týkající se \TeX4ht
(\url{https://tug.org/tex4ht}) byly odstraněny z~adresářů binárek.  
Obecně použitelný (generic) program \code{mk4ht} může být použit na spuštění 
libovolné z~rozličných kombinací \code{tex4ht}.

Nakonec, vydání \TL{} na \TK\ \DVD\ již nemůže být (kupodivu) spouštěno živě.
Samostatné \DVD\ již nemá dostatek místa.  
Výhodou je, že instalace z~fyzického \DVD\ je mnohem rychlejší.

\subsubsection{2011}

Binárky \MacOSX\ (\code{universal-darwin} a
\code{x86\_64-darwin}) nyní pracují jenom pro Leopard nebo pozdější; Panther a
Tiger již nejsou podporovány.

Program \code{biber} pro zpracování bibliografie je zahrnut pro běžné platformy.  
Jeho rozvoj je úzce spojený s~balíkem
\code{biblatex}, který úplně přebudovává %which completely reimplements
bibliografické prostředky poskytované \LaTeX{}em.

Program MetaPost (\code{mpost}) již nevytváří nebo nepoužívá soubory
\code{.mem}.  Potřebné soubory, jako je \code{plain.mp}, se jednoduše načítají 
při každém spuštění.  To souvisí s~podporou MetaPostu jako knihovny, 
což je další důležitá změna, třebas neviditelná pro uživatele.

Implementace \code{updmap} v~programu Perl, předtím používaná pouze
pod Windows, byla vylepšena a nyní je používána na všech
platformách. Výsledkem toho je, že uživatel by neměl 
vidět žádné změny, kromě toho, že program běží mnohem rychleji.

Programy \cmdname{initex} a \cmdname{inimf} byly obnoveny 
(ale žádné jiné \cmdname{ini*} varianty).

\subsubsection{2012}

\code{tlmgr} podporuje aktualizace z~vícenásobných 
síťových repozitářů. Více obsahuje oddíl o~vícenásobných 
repozitářích ve výstupu příkazu \code{tlmgr help}.

Parametr \cs{XeTeXdashbreakstate} je implicitně nastaven na~1, pro
\code{xetex} i \code{xelatex}.  To umožňuje zalomení řádek po
pomlčkách a spojovnících, % a em-dashes and en-dashes, 
což vždy bylo chováním  %which has always been the behavior of plain
plain \TeX{}u, \LaTeX{}u, Lua\TeX{}u atd.  Stávající dokumenty 
\XeTeX{}u, které si musí udržet perfektní kompatibilitu
zalomení řádek, musí explicitně nastavit hodnotu
\cs{XeTeXdashbreakstate} na~0.

Výstupní soubory generované programy \code{pdftex} a \code{dvips} 
teď mohou mimo jiné překročit velikost 2\,\GB.

Do výstupu programu \code{dvips} je implicitně zahrnuto 35 standardních 
Post\-Scriptových fontů, protože existuje příliš mnoho jejich různých verzí.

V omezeném režimu vykonávání \cs{write18}, který je implicitně nastaven,
je teď \code{mpost} povoleným programem.

Soubor \code{texmf.cnf} je také k~nalezení v~adresáři \filename{../texmf-local},
například \filename{/usr/local/texlive/texmf-local/web2c/texmf.cnf}, 
pokud existuje.

Skript \code{updmap} čte soubor \code{updmap.cfg} podle stromu místo
globálního konfiguračního souboru. Tato změna by neměla být viditelná,
pokud needitujete vaše soubory updmap.cfg přímo.  
Více obsahuje výstup příkazu \code{updmap --help}.

Platformy: byly přidány \pkgname{armel-linux} a \pkgname{mipsel-linux};
\pkgname{sparc-linux} a \pkgname{i386-netbsd} již nejsou  
v~základní distribuci.

\subsubsection{2013}

Rozvržení distribuce: kořenový adresář \code{texmf/} přešel do 
\code{texmf-dist/} kvůli zjednodušení. Obě
proměnné \code{TEXMFMAIN} a \code{TEXMFDIST} Kpathsea nyní odkazují na \code{texmf-dist}.

Mnohé malé jazykové kolekce byly sloučeny pro zjednodušení instalace.

\MP: byla přidána původní podpora pro výstup PNG a pro pohyblivou čárku (IEEE double).

Lua\TeX: aktualizován na Lua 5.2 a zahrnuje novou knihovnu
(\code{pdfscanner}) pro zpracování obsahu externí stránky PDF, 
kromě množství dalšího (viz jeho stránky).

\XeTeX\ (pro doplnění viz také jeho stránky):
\begin{itemize*}
\item Na navrhování fontů je nyní použita knihovna The HarfBuzz místo
ICU.  (ICU je stále používán pro podporu vstupních kódování, obousměrnost
a zvláštní zalamování řádků v~Unicode.)
\item Na návrh Graphite se nyní používají Graphite2 a HarfBuzz namísto SilGraphite.
\item Na počítačích Mac se používá Core Text namísto (kritizovaného) ATSUI.
\item Preferují se TrueType/OpenType fonty před Type1, pokud mají stejné názvy.
\item Opraveny jsou občasné neshody při hledání fontů mezi \XeTeX{}em a
\code{xdvipdfmx}.
\item Podpora OpenType math cut-ins.
\end{itemize*}

\cmdname{xdvi}: nyní používá pro vyobrazení FreeType namísto \code{t1lib}.

\pkgname{microtype.sty}: trochu podpory pro \XeTeX\ (vyčuhování) a
Lua\TeX\ (vyčuhování, rozpínavost fontů, mezipísmenný proklad -- tracking), 
kromě dalších zlepšení.

\cmdname{tlmgr}: nová činnost \code{pinning} pro usnadnění
konfigurace násobných repozitářů; 
více obsahuje tato sekce v~\verb|tlmgr --help|, online na
\url{https://tug.org/texlive/doc/tlmgr.html#MULTIPLE-REPOSITORIES}.

Platformy: \pkgname{armhf-linux}, \pkgname{mips-irix},
\pkgname{i386-netbsd} a \pkgname{amd64-netbsd} přidány nebo oživeny;
\pkgname{powerpc-aix} odstraněna. 


\subsubsection{2014}

Rok 2014 zažil další doladění \TeX{}u od Donalda Knutha;
to ovlivnilo všechny stroje,
ale pravděpodobně jedinou viditelnou změnou je 
navrácení řetězce \code{preloaded format} 
ve výstupním řádku. Podle Knutha toto nyní odráží formát, 
který \emph{bude} standardně načten, a ne %undumped ???
formát, který je ve skutečnosti natažen 
již v~binárce; toto může být potlačeno různými způsoby.

pdf\TeX: nový parametr na potlačení varování 
\cs{pdfsuppresswarningpagegroup}; nové primitivy 
pro fiktivní mezislovní mezery na pomoc s~přeformátováváním textu 
v~PDF: \cs{pdfinterwordspaceon},
\cs{pdfinterwordspaceoff}, \cs{pdffakespace}.

Lua\TeX: význačné změny a úpravy byly provedeny pro čtení fontů
a dělení slov. Největší přírůstek je nová varianta stroje,
\code{luajittex} a jeho sourozenci \code{texluajit} a \code{texluajitc}. Používá
just-in-time Lua kompilátor (podrobný článek v~\textsl{TUGboatu} je na 
\url{https://tug.org/TUGboat/tb34-1/tb106scarso.pdf}). \code{luajittex} 
je stále ve vývoji, není k~dispozici pro všechny platformy a je
podstatně méně stabilní než \code{luatex}. Ani my, ani jeho
vývojáři nedoporučujeme jeho použití s~výjimkou zvláštního použití
pro účel experimentu s~jit na kódech Lua. 

\XeTeX: Stejné grafické formáty jsou nyní podporovány 
na všech platformách (včetně Mac); 
tím se vyhýbá problému kompatibility dekompozice v~Unicode.
Preferuje fonty OpenType před Graphite kvůli
kompatibilitě s~předchozími verzemi \XeTeX{}u.

\MP: Podporován je nový číselný systém \code{decimal}, společně s~interním  
parametrem \code{numberprecision}; nová definice makra \code{drawdot} 
v~\filename{plain.mp}, podle Donalda Knutha, mimo jiné odstraňuje chyby ve výstupech 
SVG a PNG.

Con\TeX{}tová pomůcka \cmdname{pstopdf} bude odstraněna jako samostatný 
příkaz v~určité době po vydání kvůli konfliktům s~pomůckami OS
téhož názvu. Stále může být (i teď) vyvolána jako 
\code{mtxrun --script pstopdf}.

\cmdname{psutils} byl podstatně revidován novým vývojářem. 
Ve výsledku jsou nyní mnohé zřídka užívané pomůcky (\code{fix*}, \code{getafm},
\code{psmerge}, \code{showchar}) pouze v~adresáři \dirname{scripts/}
a nejsou vykonavatelné na uživatelské úrovni.
Toto může být navráceno, pokud se to ukáže jako problematické.  
Byl přidán nový skript \code{psjoin}.

Přerozdělení \TeX\ Live pro Mac\TeX\ (sekce~\ref{sec:macosx}) již
nezahrnuje výběrové balíky pouze pro Mac pro fonty Latin Modern a
\TeX\ Gyre, jelikož pro jednotlivé uživatele je dostatečně jednoduché 
začlenit je do systému. Program \cmdname{convert} z~ImageMagick 
byl rovněž odstraněn, protože \TeX4ht (konkrétně
\code{tex4ht.env}) nyní používá přímo Ghostscript.

Kolekce \pkgname{langcjk} pro čínskou, japonskou a korejskou
podporu byla rozdělena na jednotlivé jazykové kolekce z~důvodu 
rozumnějších velikostí.

Platformy: \pkgname{x86\_64-cygwin} byla přidána; \pkgname{mips-irix} odstraněna. 
Microsoft již nepodporuje Windows XP, tudíž naše programy mohou kdykoliv 
začít selhávat.

\subsubsection{2015}

\LaTeXe\ nyní implicitně zahrnuje změny, které byly předtím
zahrnuty pouze explicitním načtením balíčku \pkgname{fixltx2e}, 
který je nyní \uv{no-op} (prázdný). Nový balík \pkgname{latexrelease}
a další mechanismy umožňují kontrolu toho, co je již doděláno
v~základním \LaTeX u. Podrobnosti obsahují zahrnuté dokumenty 
\LaTeX\ News \#22 a \uv{\LaTeX\ changes}.  Mimochodem, balíky 
\pkgname{babel} a \pkgname{psnfss}, které jsou součástí jádra 
\LaTeX{}u, se spravují odděleně a nejsou dotčeny těmito změnami
(a měly by být stále ještě funkční).

Vnitřně nyní \LaTeXe\ zahrnuje konfiguraci Unicode-related stroje
(jehož znaky jsou písmena, názvy primitivů atd.), který byl
původně součástí \TeX\ Live.  Tato změna je zamýšlena
jako neviditelná pro uživatele; 
několik vnitřních řídících příkazů nízké úrovně
bylo přejmenováno nebo odstraněno, ale vnější chování 
by mělo zůstat stejné.

pdf\TeX: Podpora JPEG Exif stejně JFIF; dokonce 
nevydává výstrahu při záporném \cs{pdfinclusionerrorlevel}; 
synchronizace s~\prog{xpdf}~3.04.

Lua\TeX: Nová knihovna \pkgname{newtokenlib} na skenování tokenů; 
odstraněna chyba v~generátoru normálních náhodných čísel a na
jiných místech.

\XeTeX: Opraveno zacházení s~obrázky; binárka \prog{xdvipdfmx} 
nyní poprvé vypadá jako příbuzná \prog{xetex}u; změněn
vnitřní operační kód \code{XDV}.

MetaPost: Nový číselný systém \code{binary}; nové programy 
\prog{upmpost} a \prog{updvitomp} umožňující japonštinu, 
analogické k~\prog{up*tex}.

Mac\TeX:\ Aktualizace zařazeného balíku Ghostscript  
pro podporu CJK. Panel preferencí \TeX ové distribuce
pracuje nyní v~Yosemite (\MacOSX~10.10).  
Resource-fork font suitcases (obecně bez přípony) \XeTeX\ už 
nepodporuje; podpora data-fork suitcases
(\code{.dfont}) zůstává.

Infrastruktura: Skript \prog{fmtutil} byl předělán pro čtení
\filename{fmtutil.cnf} na stromovém základě, analogicky k~\prog{updmap}.
Skripty Web2C \prog{mktex*} (včetně \prog{mktexlsr}, \prog{mktextfm},
\prog{mktexpk}) upřednostňují nyní programy v~jejich vlastních adresářích,
místo vždy používané existující proměnné \envname{PATH}.

Platformy: \pkgname{*-kfreebsd} jsou odstraněny, protože \TeX\ Live je nyní 
snadno dostupný prostřednictvím mechanismu systémových platforem.
Podpora pro několik dalších platforem je dostupná ve formě uživatelských binárek
(\url{https://tug.org/texlive/custom-bin.html}). Navíc jsou některé 
platformy nyní vynechány na \DVD\ (jednoduše pro ušetření místa), 
avšak mohou být normálně nainstalovány z~Internetu.

\subsubsection{2016}

Lua\TeX: %Sweeping 
Rozsáhlé změny primitivů, jak přejmenování, tak i odstranění
společně s~reorganizací struktury některých uzlů. Změny jsou shrnuty 
v~článku Hanse Hagena, \uv{Lua\TeX\ 0.90 backend changes
for PDF and more} (\url{http://tug.org/TUGboat/tb37-1/tb115hagen-pdf.pdf}); 
pro všechny podrobnosti viz příručku Lua\TeX{}u,
\OnCD{texmf-dist/doc/luatex/base/luatex.pdf}.

\MF: Nové vysoce experimentální příbuzné programy MFlua a MFluajit,
integrující Lua s~\MF, pro účely pokusného testování.

\MP: Opravy chyb a interní příprava pro MetaPost 2.0.

\code{SOURCE\_DATE\_EPOCH} má podporu všech překladačů s~výjimkou
Lua\TeX u (ta přijde v další verzi)
a originálního \code{tex}u (cíleně vynechána): 
pokud je proměnná prostředí \code{SOURCE\_DATE\_EPOCH} nastavena,
její hodnota je použita jako časová známka PDF výstupu.
Pokud je nastavena i proměnná  
\code{SOURCE\_DATE\_EPOCH\_TEX\_PRIMITIVES}, hodnota
\code{SOURCE\_DATE\_EPOCH} je použita k~inicializaci
\TeX ových primitiv \cs{year}, \cs{month}, \cs{day}
a \cs{time}. Manuál pdf\TeX u má příklady a detaily.

pdf\TeX: Nové primitivy \cs{pdfinfoomitdate}, \cs{pdftrailerid},
\cs{pdfsuppressptexinfo}, na nastavení hodnot objevujících se v~PDF
výstupu, které se normálně mění při každém spuštění
(časové známky). Ovlivní pouze výstup PDF, ne DVI.

Xe\TeX: Nové primitivy \cs{XeTeXhyphenatablelength},
\cs{XeTeXgenerateactualtext},\newline
\cs{XeTeXinterwordspaceshaping}, \cs{mdfivesum};
limit počtu tříd znaků byl zvětšen na 4096; a byl zvýšen DVI id byte.

Ostatní nástroje:
\begin{itemize*}
\item \code{gregorio} je nový program, část balíku \code{gregoriotex}
  pro sazbu not Gregoriánských chorálů; implicitně je zařazen
  do \code{shell\_escape\_commands}.

\item \code{upmendex} je program na vytváření indexů, 
většinou slučitelný s~programem \code{makeindex}, 
s~podporou řazení pro Unicode sorting, kromě jiných změn.

\item \code{afm2tfm} nyní provádí výškové nastavení jenom na základě akcentů,
%accent-based height adjustments upward; 
nová volba \code{-a} vynechává všechny úpravy.

\item \code{ps2pk} umí zacházet s~rozšířenými fonty PK/GF.
\end{itemize*}

Mac\TeX:\ The \TeX\ Distribution Preference Panel byl zrušen; jeho 
funkcionalita je nyní v~\TeX\ Live Utility; přibalené aplikace GUI
jsou aktualizovány; nový skript \code{cjk-gs-integrate} 
pro spuštění uživateli, kteří si přejí začlenit různé
fonty CJK do Ghostscriptu.

Infrastruktura: %System-level \code{tlmgr} configuration file supported;
Podporován je systémový konfigurační soubor \code{tlmgr}; 
ověření kontrolního součtu balíku; pokud je dostupný program \code{gpg}, 
dojde k~ověření podpisu síťové aktualizace. Ověření
se týká jak instalačního programu tak \code{tlmgr}. 
Pokud \code{gpg} k~dispozici není, aktualizace probíhají jako obvykle.

Platformy \code{alpha-linux} a \code{mipsel-linux} byly odstraněny.


\subsubsection{2017}

Lua\TeX: Více zpětných volání (callbacks), více typografického řízení,
více přístupů k~interním datovým strukturám; 
pro některé platformy přidáná knihovna \code{ffi}
pro dynamické načtení kódu.

pdf\TeX: Proměnná prostředí |SOURCE_DATE_EPOCH_TEX_PRIMITIVES| z~minulého
roku přejmenována na |FORCE_SOURCE_DATE| bez změn ve 
funkcionalitě; pokud seznam tokenů \cs{pdfpageattr} obsahuje řetězec 
\code{/MediaBox}, potlačí se výstup implicitního \code{/MediaBox}.

Xe\TeX: Unicode/OpenType matematika je nyní založena na podpoře tabulky HarfBuzz MATH; odstraněno pár chyb.

Dvips: Poslední nastavení \code{\bs special papersize} \uv{vyhrává}, pro
konzistenci s~\code{dvipdfmx} a očekáváním balíků;
konfigurační nastavení \code{L0} (přepínač \code{-L0}) obnovuje předchozí
chování, tedy to, kdy \uv{první nastavení vyhrává}.

\sloppypar ep\TeX, eup\TeX: Nová primitiva \cs{pdfuniformdeviate},
\cs{pdfnormaldeviate}, \cs{pdfrandomseed}, \cs{pdfsetrandomseed},
\cs{pdfelapsedtime}, \cs{pdfresettimer} z~pdf\TeX{}u.

Mac\TeX:\ Jako v~tomto roce, pouze vydání \MacOSX, pro která Apple ještě 
vydává bezpečnostní záplaty, budou podporovány v~Mac\TeX{}u pod
platformou s~názvem |x86_64-darwin|; v~současnosti sa tím myslí Yosemite,
El~Capitan a Sierra (10.10 a novší). Binárky pro starší verze \MacOSX\
nejsou zahrnuty v~Mac\TeX{}u, ale stále ještě jsou dostupné v~\TeX\
Live (|x86_64-darwinlegacy|, \code{i386-darwin}, \code{powerpc-darwin}).

Infrastruktura: Strom \envname{TEXMFLOCAL} je nyní prohledáván před
\envname{TEXMFSYSCONFIG} a \envname{TEXMFSYSVAR} (implicitně); je naděje,
že to povede k~lepšímu naplnění očekávání 
použití lokálních souborů před systémovými.
Program \code{tlmgr} má také nový režim \code{shell}
pro interaktivní a dávkové použití a novou funkci
\code{conf auxtrees} pro jednodušší přidání a odebrání
doplňkových stromů.

\code{updmap} a \code{fmtutil}: Tyto skripty nyní vydávají varování 
pokud jsou vyvolány bez specifikace buď v~tzv. systémovém režimu
(\code{updmap-sys}, \code{fmtutil-sys} nebo volba \code{-sys}) nebo v~uživatelském
režimu (\code{updmap-user}, \code{fmtutil-user} nebo volba \code{-user}).
Změna byla dělána v~naději, že toto povede k~snížení setrvalého problému náhodného spouštění 
uživatelského režimu, a tím ke~ztrátě následných aktualizací systému. 
Pro podrobnosti viz \url{https://tug.org/texlive/scripts-sys-user.html}.

\code{install-tl}: Osobní cesty jako \envname{TEXMFHOME} jsou nyní přiřazeny hodnotám Mac\TeX{}u
(|~/Library/...|\ implicitně na Macích). Nová volba 
\code{-init-from-profile} pro zahájení instalace s~hodnotami zadaného 
profilu; nový příkaz \code{P} pro explicitní uložení profilu; nové
názvy proměnných profilu (ale předchozí jsou ještě stále akceptovány).

\sloppypar Sync\TeX: dočasný soubor se nyní jmenuje \code{foo.synctex(busy)}, namísto
\code{foo.synctex.gz(busy)}. Frontendy a dávky, které mažou
dočasné soubory by se měly upravit (kvůli zrušené koncovce \code{.gz}).

Ostatní nástroje: \code{texosquery-jre8} je nový multi-platformní program
pro získávání lokálních nastavení (\texttt{locale}) a jiných
systémových informací ze zdrojového dokumentu \TeX{}u;
je to implicitně zahrnuto v~příkazech |shell_escape_commands|
pro omezené vykonávání shellu. (Starší verze \code{JRE}
jsou podporovány programem \code{texosquery}, 
ale nemohou být k~dispozici v~omezeném režimu vykonávání, protože už nejsou 
podporovány Oracle, kvůli bezpečnostním problémům.)

Platformy: Viz položku Mac\TeX\ výše; žádné další změny.

% 
\subsubsection{2018}

Kpathsea: implicitně je hledání v~nesystémových adresářích 
nezávislé na velikosti písmenek v~názvech souborů;
nastav \code{texmf.cnf} nebo proměnnou prostředí 
\code{texmf\_casefold\_search} na \code{0} pro potlačení nezávislosti.
Plné detaily jsou v~manuálu Kpathsea (\url{https://tug.org/kpathsea}).

ep\TeX, eup\TeX: Nový primitiv \cs{epTeXversion}.

Lua\TeX: Příprava na migraci na Lua 5.3 v~roce 2019: binárka
\code{luatex53} je dostupná pro většinu platform, ale musí být přejmenována na
\code{luatex} aby byla účinná. Nebo použijte soubory \ConTeXt\ Garden
(\url{https://wiki.contextgarden.net}); více informací tamtéž.

MetaPost: Oprava chybných směrů cest, TFM a PNG výstupy.

pdf\TeX: Umožňuje kódování vektorů pro bitmapové fonty; 
aktuální adresář není zakódován do PDF ID; odstranění chyb pro 
\cs{pdfprimitive} a souvisejíci věci.

Xe\TeX: Podpora \code{/Rotate} při vkládání PDF obrázků; nenulový chybový kód
programu pokud selže výstupní ovladač; opraveny různé UTF-8 a další primitivy.

Mac\TeX:\ Viz změny podpory verze níže. 
Navíc, soubory instalovány do \code{/Applications/TeX/} programem Mac\TeX\ 
byly reorganizovány pro větší srozumitelnost; nyní toto místo obsahuje čtyři GUI programy 
(BibDesk, LaTeXiT, \TeX\ Live Utility a TeXShop) na nejvyšší úrovni a adresáře
s~dalšími nástroji a dokumentací.

\code{tlmgr}: nové front-end \code{tlshell} (Tcl/Tk) a
\code{tlcockpit} (Java); výstup JSON; \code{uninstall} je nyní synonymem 
pro \code{remove}; nová akce/volba \code{print-platform-info}.

Platformy:
\begin{itemize*}
\item Odstraněny: \code{armel-linux}, \code{powerpc-linux}.

\item \code{x86\_64-darwin} podporuje 10.10--10.13
(Yosemite, El~Capitan, Sierra a High~Sierra).

\item \code{x86\_64-darwinlegacy} podporuje 10.6--10.10 (i když
pro 10.10 je preferován \code{x86\_64-darwin}).  Veškerá podpora pro 10.5
(Leopard) je pryč, tj.\ platformy \code{powerpc-darwin} a
\code{i386-darwin platforms} byly odstraněny.

\item Windows: XP už není podporován.
\end{itemize*}

% 
\subsubsection{2019}

Kpathsea: Důslednější expanze závorek a rozdělení cesty; nová
proměnná \code{TEXMFDOTDIR} namísto \singleuv{hard-coded} \code{.}\ v~cestách 
umožňuje snadné vyhledávání dalších adresářů nebo
podadresářů (viz komentáře v~\code{texmf.cnf}).

ep\TeX, eup\TeX: Nové primitivy \cs{readpapersizespecial} a \cs{expanded}.

Lua\TeX: Lua 5.3 nyní používán, s průvodními aritmetickými změnami a změnami rozhraní.
Domácí knihovna pplib slouží ke čtení souborů PDF, co odstraňuje závislost na poppleru
(a potřebu C++); odpovídajícím způsobem se změnilo rozhraní Lua. 

MetaPost: název příkazu \code{r-mpost} rozpoznán jako
alias pro vyvolání pomocí volby \code{-{}-restricted} a přidán do
seznamu omezených příkazů dostupních ve výchozím nastavení.
Minimální přesnost je nyní 2 pro desetinný a binární režim.
Binární režim již není v~MPlib k~dispozici, ale stále je k~dispozici 
v~samostatném MetaPostu.

pdf\TeX: Nový primitiv \cs{expanded}; pokud je parametr
\cs{pdfomitcharset} nového primitivu nastaven na 1, řetězec \code{/CharSet} 
je vypuštěn z~výstupu PDF, protože nemůže být garantovaně správný, jak to
požadují PDF/A-2 a PDF/A-3.

Xe\TeX: Nové primitivy \cs{expanded},
\cs{creationdate},
\cs{elapsedtime},
\cs{filedump}, 
\cs{filemoddate}, 
\cs{filesize}, 
\cs{resettimer}, 
\cs{normaldeviate}, 
\cs{uniformdeviate}, 
\cs{randomseed}; extend \cs{Ucharcat} to produce active
characters.

\code{tlmgr}: Podporuje \code{curl} jako program na stahování;
pro lokální zálohování používá \code{lz4} a gzip před \code{xz}, pokud jsou k~dispozici;
dává přednost systémovým binárkám před binárkami, které poskytuje \TL\ pro programy 
pro kompresi a stahování, pokud není nastavena proměnná
prostředí \code{TEXLIVE\_PREFER\_OWN}.

\code{install-tl}: Nová volba \code{-gui} (bez argumentu) je výchozí pod Windows a Macs, 
a vyvolává nové GUI Tcl/TK (viz oddíly~\ref{sec:basic} a~\ref{sec:graphical-inst}).

Nástroje:
\begin{itemize*}
	\item \code{cwebbin} (\url{https://ctan.org/pkg/cwebbin}) je nyní 
	implementace CWEB v~\TeX\ Live, s~podporou pro více jazykových dialektů,
	včetně programu \code{ctwill} na výrobu mini-indexů. 
	
	\item \code{chkdvifont}: poskytuje informace o fontech z~\dvi{} souborů, také
	z~tfm/ofm, vf, gf, pk.
	
	\item \code{dvispc}: udělá soubor DVI nezávislým na stránce % page-independent 
	vzhledem na \singleuv{specials}.
\end{itemize*}

Mac\TeX:\ \code{x86\_64-darwin} nyní podporuje 10.12 a vyšší (Sierra,
High Sierra, Mojave); \code{x86\_64-darwinlegacy} ještě pořád podporuje 10.6
a novější. Kontrola pravopisu Excalibur již není zahrnuta, protože vyžaduje 
32-bitovou podporu.

Platformy: odstraněna \code{sparc-solaris}.

\subsection{2020}

Všeobecně: 
\begin{itemize}
	\item Primitiv \cs{input} ve všech strojích \TeX{}u, včetně
	\texttt{tex}, nyní rovněž přijímá argument group-delimited název souboru, jako 
	systémově závislé rozšíření.  
%	accepts a group-delimited filename argument, as a system-dependent extension. 
	Použití se snandardním space/token-delimited názvem souboru je úplně nezměněno.
	The group-delimited argument byl předtím implementován v~Lua\TeX{}u; nyní je 
	k~dispozici pro všechny stroje. ASCII znaky dvojitých uvozovek (\texttt{"})
	jsou odstraněny z~názvu souboru, ale jinak to zůstane nezměněno po
	tokenizaci. Toto v~současnosti neovlivní příkaz \LaTeX{}u \cs{input},
	protože se jedná o redefinici standardního primitivu \cs{input}.
	
	\item Nová volba \optname{-{}-cnf-line} pro \texttt{kpsewhich}, \texttt{tex},
	\texttt{mf} a všechny další stroje, pro podporu libovolné nastavení 
	konfigurace na příkazovém řádku.
	
	\item Přidání různých primitivů do různých strojů v~tomto roce a
	v~předchozích letech je zamýšleno aby vyústilo do společné sady funkcionality
	dostupné ve všech strojích (\textsl{\LaTeX\ News \#31},
	\url{https://latex-project.org/news}).
	
\end{itemize}

ep\TeX, eup\TeX: Nové primitivy \cs{Uchar}, \cs{Ucharcat},
\cs{current(x)spacingmode}, \cs{ifincsname}; opraveny \cs{fontchar??} a
\cs{iffontchar}. Jenom pro eup\TeX: \cs{currentcjktoken}.

Lua\TeX: Integrace s~knihovnou HarfBuzz, dostupná jako nové stroje 
\texttt{luahbtex} (použito pro \texttt{lualatex}) a \texttt{luajithbtex}.
Nové primitivy: \cs{eTeXgluestretch}, \cs{eTeXglueshrink},
\cs{eTeXglueorder}.

pdf\TeX: Nový primitiv \cs{pdfmajorversion}; to pouze mění
číslo verze ve výstupu PDF; nemá žádný vliv na obsah PDF.
\cs{pdfximage} a podobné nyní hledají obrazové soubory 
stejným způsobem jako \cs{openin}.

p\TeX: Nové primitivy \cs{ifjfont}, \cs{iftfont}. Rovněž v~ep\TeX{}u,
up\TeX{}u, eup\TeX{}u.

Xe\TeX: Opravy pro \cs{Umathchardef}, \cs{XeTeXinterchartoks}, \cs{pdfsavepos}.

Dvips: Výstupní kódování bitmapových fontů pro lepší schopnosti copy/paste
(\url{https://tug.org/TUGboat/tb40-2/tb125rokicki-type3search.pdf}).

Mac\TeX:\ Mac\TeX\ a \texttt{x86\_64-darwin} nyní požadují 10.13 nebo
vyšší (High~Sierra, Mojave a Catalina);
\texttt{x86\_64-darwinlegacy} podporuje 10.6 a novější.
Mac\TeX\ je na systému autorizován pro běh a programy příkazové řádky mají zajištěné spouštění, 
jak nyní vyžaduje Apple pro instalační balíčky. 
BibDesk a \TeX\ Live Utility v~Mac\TeX{}u protože nejsou pro systém autorizovány, 
ale soubor \filename{README} obsahuje seznam url, kde je možné je získat.

\code{tlmgr} a infrastruktura: \begin{itemize*}
	\item Automaticky (jednou) zopakuje balíčky, které se nepodaří stáhnout.
	\item Nová volba \texttt{tlmgr check texmfdbs}, na kontrolu souhlasu (konzistence)
	souborů \texttt{ls-R} a specifikací \texttt{!!}\ pro každý strom.
	\item Používejí verzované názvy souborů pro balíčkové kontejnery, jako\newline 
	v~\texttt{tlnet/archive/\textsl{pkgname}.rNNN.tar.xz}; 
	mělo by to být neviditelné pro uživatele, ale je to výrazná změna v~distribuci.
	\item \texttt{catalogue-date} informace již nejsou šířeny z~\TeX~Catalogue, 
	protože to často nesouviselo s aktualizacemi balíčků.
\end{itemize*}

\subsubsection{2021}

Všeobecně:
\begin{itemize}
\item Začleněny jsou změny Donalda Knutha pro jeho vyladění \TeX{}u\ a Metafontu v~roce 2021
(\url{https://tug.org/TUGboat/tb42-1/tb130knuth-tuneup21.pdf}. Jsou
k~dispozici také na CTAN jako balíčky \code {knuth-dist} a \code {knuth-local}.
Opravy se podle očekávání týkají obskurních případů a nemají vliv
na jakékoli chování v~praxi.
	
\item S~výjimkou původního \TeX{}u: pokud je \cs{tracinglostchars} nastaveno na 3 nebo
více, chybějící znaky budou mít za následek chybu, nejen zprávu v~log-souboru,
a chybějící kód znaku se zobrazí v~hexadecimálním formátu.
	
\item S~výjimkou původního \TeX{}u: nový celočíselný parametr
\cs{tracingstacklevels}, pokud je kladný, a \cs{tracingmacros}
je také kladný, způsobí předponu označující hloubku expanze makra
na výstup na každém příslušném řádku protokolu (např. |~..| v~hloubce~2).
Také protokolování maker je zkráceno na hloubku $\ge$ hodnotu parametru. 
\end{itemize}

Aleph: \LaTeX{}ovský formát na bázi Aleph, pojmenovaný \code{lamed}, byl
odstraněn. Samotný binární soubor \code{aleph} je stále zahrnut a podporován. 

Lua\TeX:
\begin{itemize*}
\item Lua 5.3.6.
\item Zpětné volání pro úroveň vnoření použitou v~\cs{tracingmacros}, jako
zobecněná varianta nového \cs{tracingstacklevels}.
\item Označí matematické glyfy jako chráněné, aby se zabránilo zpracování jako textu.
\item Odstraněna kompenzace šířky/italické korekce pro tradiční cestu matematického kódu. 	
\end{itemize*}

MetaPost:
\begin{itemize*}
\item Podpora proměnné prostředí |SOURCE_DATE_EPOCH| pro reprodukovatelný výstup.
\item Vyhne se chybnému konečnému \texttt {\%} v~mpto.
\item Popsaná volba \texttt {-T}, další opravy příručky.
\item Hodnota \texttt{epsilon} se změnila v~binárním a desítkovém režimu, takže\newline
	|mp_solve_rising_cubic| nyní funguje podle očekávání. 
\end{itemize*}

pdf\TeX{}:
\begin{itemize*}
\item Nové primitivy \cs{pdfrunninglinkoff} a \cs{pdfrunninglinkon};
   např.\ pro deaktivaci generování odkazů v~záhlavích a zápatích.
\item Varování místo přerušení, když \uv{\cs{pdfendlink} skončí v~jiné úrovni vnoření než \cs{pdfstartlink}}.
\item Záložní (dump) přiřazení \cs{pdfglyphtounicode} v~souboru \texttt{fmt}.
\item Zdroj: podpora poppler byla odstraněna, protože bylo příliš těžké ji aktualizovat.
V~nativním TL, pdf\TeX\ vždy používal \texttt{libs/xpdf},
což je výřez a upravený kód z~\texttt{xpdf}. 
\end{itemize*}

Xe\TeX{}: Opravy matematického vyrovnání párů (kerningu).

Dvipdfmx:
\begin{itemize*}
\item Pokud soubor s~obrázkem nebyl nalezen, ukončení se špatným stavem.
\item Rozšířená speciální syntaxe pro podporu barev.
\item Specials pro manipulaci |ExtGState|.
\item Specials \code{pdfcolorstack} a \code{pdffontattr} pro kompatibilitu.
\item Experimentální podpora rozšířeného |fnt_def| \code{dviluatex}u.
\item Podpora nové vlastnosti virtuálního písma pro záložní definici japonského písma. 
\end{itemize*}

Dvips:
\begin{itemize*}
\item Výchozí název dokumentu PostScript je nyní základním názvem vstupního
souboru a lze jej přepsat novou volbou \texttt{-title}.
\item Pokud soubor \texttt{.eps} nebo jiný obrazový soubor nebyl nalezen, ukončení se špatným
stavem.
\item Podpora nové vlastnosti virtuálního písma pro záložní definici japonského písma. 
\end{itemize*}

Mac\TeX{}: Mac\TeX{} a jeho nová binární složka \texttt{universal-darwin} nyní
vyžadují macOS 10.14 nebo vyšší (Mojave, Catalina a Big~Sur);
binární složka |x86_64-darwin| již není k~dispozici. Binární složka |x86_64-darwinlegacy|, dostupná pouze s~Unixovým
\texttt{install-tl}, podporuje verzi 10.6 a novější. 

Tento rok je důležitý pro Macintosh, protože společnost Apple v~listopadu představila
stroje ARM a bude prodávat a podporovat stroje ARM i Intel po mnoho let. 
Všechny programy v~\texttt{universal-darwin} mají spustitelný kód pro ARM i Intel. Oba binární soubory jsou kompilovány ze
stejného zdrojového kódu. 

Doplňkové programy Ghostscript, LaTeXiT, \TeX{} Live Utility a TeXShop jsou všechny univerzální a jsou podepsány s~potvrzeným modulem runtime, takže všechny
jsou letos součástí Mac\TeX{}u.

\code{tlmgr} a infrastruktura: \begin{itemize*}
\item ponechávají pouze jednu zálohu hlavního úložiště \texttt{texlive.tlpdb}.
\item ještě větší přenositelnost napříč systémy a verzemi Perlu.
\item \texttt{tlmgr info} hlásí nové pole \texttt{lcat-*} a \texttt {rcat-*} pro místní vs. vzdálená data katalogu.	
\item úplné protokolování dílčích příkazů přesunuto do nového log-souboru\newline
\texttt{texmf-var/web2c/tlmgr-commands.log}. 
\end{itemize*}

\htmlanchor{news}
\subsection{Současnost -- 2022}
\label{sec:tlcurrent}

Všeobecné: \begin{itemize}
	\item Nový stroj \code{hitex}, který vytváří svůj vlastní HINT formát,
	navržen speciálně pro responzivní čtení technických dokumentů na mobilních zařízeních.
	Odděleně od \TL jsou k~dispozici prohlížeče HINT pro GNU/Linux, Windows a Android. 
	
\item \code{tangle}, \code{weave}: podporují volitelný třetí argument
pro specifikaci výstupní soubor. 

\item Nyní zahrnut Knuthův program \code{twill} pro vytváření miniindexů pro původní
programy \texttt{WEB}u. 
\end{itemize}

Rozšíření příkazů pro různé verze sázecích programů (kromě původních \TeX{}, Aleph a hi\TeX{}):
\begin{itemize}
	\item Nový primitiv \cs{showstream} pro přesměrování výstupu \cs{show} do souboru. 
	\item Nové primitivy \cs{partokenname} a \cs{partokencontext} umožňují
	přepsat název tokenu \cs{par} vygenerovaného na prázdných řádcích,
	konec vboxů atd. 
\end{itemize}

ep\TeX{}, eup\TeX{}: \begin{itemize*}\raggedright
	\item Nové primitivy: \cs{lastnodefont}, \cs{suppresslongerror},
	\cs{suppressoutererror}, \cs{suppressmathparerror}.
	\item Nyní dostupné rozšíření pdf\TeX{}u \cs{vadjust pre}.
\end{itemize*}

Lua\TeX{}:
\begin{itemize*}
	\item Podporuje strukturované destinace z~PDF 2.0.
	\item PNG /Smask pro PDF 2.0.
	\item Variabilní rozhraní písma pro luahbtex.
	\item Různé výchozí styly odmocnin v~mathdefaultsmode. 
	\item Možnost volit specifické dělení (\cs{discretionary}
          ve vybraném bloku.  % Optionally block selected discretionary creation.
	\item Vylepšení implementace písem TrueType.
	\item Efektivnější alokace \cs{fontdimen}.
	\item Ignoruje odstavce s~pouze lokálním uzlem \texttt{par}
          následovaným synchronizačními uzly směru. 
\end{itemize*}

MetaPost: Oprava chyby pro nekonečnou expanzi maker. 

pdf\TeX{}:
\begin{itemize*}%\raggedright
	\item Podporuje strukturované destinace z~PDF 2.0.
	\item Pro písma s~mezipísmenným prokladem se použije explicitní \cs{fontdimen}6,
          pokud je specifikován.  %For letterspaced fonts, use explicit \cs{fontdimen}6 if specified.
	\item Vždy začne varování na začátku řádku.
	\item Pro znaky s~automatickým kerningem (\cs{pdfappendkern} a
	  \cs{pdfprependkern}) se přesto provede vysunutí (protrusion);
          podobně se použije automatika prokladu použije u implicitního
          a explicitního rozdělovníku. 	
%	for characters with autokern (\cs{pdfappendkern} and
%	\cs{pdfprependkern}), still do protrusion; likewise, autokern
%	both implicit and explicit hyphens.
\end{itemize*}

p\TeX{}\ et al.: \begin{itemize*}
	\item Významná důležitá aktualizace p\TeX{}u na 4.0.0 pro lepší podporu aktuálního \LaTeX{}u.
	\item Nové primitivy \cs{ptexlineendmode} a \cs{toucs}.
	\item \cs{ucs} (dříve k~dispozici v~uptex, euptex) nyní dostupný také v~p\TeX{}u a ep\TeX{}u.
	\item Rozlišují se 8bitové znaky a japonské znaky,
	jak je diskutováno v~článku TUGboat od Hironori Kitagawy 
%	 Distinguish 8-bit characters and Japanese characters
%	as discussed in a TUGboat article by Hironori Kitagawa
	(\url{https://tug.org/TUGboat/tb41-3/tb129kitagawa-char.pdf}).
\end{itemize*}

% wrapper scripts
Xe\TeX{}: Nové skripty na spouštění \texttt{xetex-unsafe} a
\texttt{xelatex-unsafe} pro jednodušší překlad dokumentů požadujících 
oba operátory průhlednosti \XeTeX{} a PSTricks, což je ze své podstaty
nebezpečné (dokud a pokud nedojde k~reimplementaci v~Ghostscriptu).
Pro bezpečnost použijte Lua\AllTeX{}. 

Dvipdfmx: \begin{itemize*}
	\item Podpora pro PSTricks bez nutnosti \texttt{-dNOSAFER}, kromě transparentnosti. 
	\item Volba \texttt{-r} pro nastavení rozlíšení bitmapových fontů opět funguje.
\end{itemize*}

Dvips: Ve výchozím nastavení se nepokoušejte o automatické nastavení média pro otočené
velikosti papíru; nová volba \texttt{--landscaperotate} ho znovu povolí. 

\code{upmendex}: Experimentální podpora pro arabské a hebrejské skripty;
vylepšená klasifikace znaků a jazyková podpora. 

Kpathsea: První cesta vrácená z~\texttt{kpsewhich -all} je nyní
stejná jako při běžném (ne \texttt{-all}) vyhledávání. 

\code{tlmgr} a infrastruktura: \begin{itemize*}
	\item standardně používájí https pro \code{mirror.ctan.org}.
	\item používájí \code{TEXMFROOT} místo \code{SELFAUTOPARENT} pro snazší
	přemisťování.
	\item \code{install-tl}: pokud stahování nebo instalace pro daný balíček selže, 
	bude automaticky pokračovat a později to zopakuje (jednou).
\end{itemize*} 

Mac\TeX{}: Mac\TeX{} a jeho binární složka \texttt{universal-darwin}
vyžadují macOS 10.14 nebo vyšší (Mojave, Catalina, Big~Sur, Monterey). Binární složka
|x86_64-darwinlegacy|, dostupná pouze s~Unixem \texttt{install-tl},
podporuje 10.6 (Snow~Leopard) a novější. 

Platformy: Žádné změny v~podpoře platforem pro tento rok (2022). Nicméně,
pro vydání v~příštím roce (2023) plánujeme přechod binárních souborů Windows z~32-bit na 64-bit. 
Bohužel nedokážeme realizovat podporu pro oba současně. 

\subsection{Budoucnost}

Máme v~úmyslu pokračovat v~pravidelných vydáních \TL{} a rádi bychom poskytovali
více dokumentace, více programů, a také stále aktualizovaný 
a lépe zkontrolovaný strom maker a fontů, a vše ostatní pro \TeX.
Tato práce je prováděna dobrovolníky v~jejich
omezeném volném čase a tedy vždy je co dodělávat. 
Viz, prosím, \url{https://tug.org/texlive/contribute.html}.

Prosíme, pošlete opravy, návrhy, náměty a nabídky pomoci na:
\begin{quote}
\email{tex-live@tug.org}\\
\url{https://tug.org/texlive}
\end{quote}
\smallskip\nobreak
\noindent \textsl{Šťastné \TeX ování!}
\end{document}

