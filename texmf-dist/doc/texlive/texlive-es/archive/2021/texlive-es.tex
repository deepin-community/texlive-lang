
\documentclass{article}

\usepackage[spanish]{babel}
\let\tldocenglish=1 
\usepackage{tex-live}
%\usepackage[T1]{fontenc}
%\usepackage[utf8]{inputenc}
\usepackage{hyperref}
\begin{document}

\title{%
{\huge \textit{La Guía de \TeX\ Live---2021}}}

\author{Karl Berry, editor \\[3mm]
        \url{https://tug.org/texlive/}}

\date{Marzo del 2021}

\maketitle

\begin{multicols}{2}
\tableofcontents
%\listoftables
\end{multicols}

\section{Introducción}
\label{sec:intro}

\subsection{\TeX\ Live y la Colección de \TeX\ }


Este documento describe las principales características de la
distribución\Dash \TeX{} y los programas relacionados con el mismo, en
los sistemas \GNU/Linux, Unix, \MacOSX, y Windows.

Puedes adquirir \TL{} por descarga, o mediante el \TK{} \DVD, el
cual es distribuido por los grupos de usuarios de \TeX{} a todos sus
miembros, o de cualquier otra manera. La sección
\ref{sec:tl-coll-dists} describe el contenido del \DVD. Ambos \TL{} y
el \TK{} son posible gracias al esfuerzo por los grupos de usuarios de
\TeX{}. Este documento principalmente describe \TL{}.

\TL{} incluye ejecutables para \TeX{}, \LaTeXe{}, \ConTeXt, \MF, \MP,
\BibTeX{} y muchos más programas; una extensa colección de macros,
fuentes, y documentación; y también ayuda para la composición de
documentos de imprenta, mediante varios scripts que se usan alrededor
del mundo.

Para un sumario de los cambios más significativos en esta edición de
\TL{}, vea el final de este documento, section~\ref{sec:history}
(\p.\pageref{sec:history}).

\htmlanchor{plataformas}
\subsection{Respaldo con el sistema operativo}
\label{sec:os-support}

\TL{} contiene los binarios para muchas de las plataformas, incluyendo
a \GNU/Linux, \MacOSX, y Cygwin. El código original puede ser
compilado en todas las plataformas para las cuales no se han proveído
los binarios.

En cuanto a Windows: Windows 7 y versiones más recientes
son también respaldadas. Windows Vista puede aún funcionar en la mayoría de los casos, pero en cuanto a Windows \acro{XP} y versiones anteriores a este, \TL{} ya no funcionará más. No hay
ejecutables especiales de 64-bit para Windows, pero los ejecutables de
32-bit deben funcionar sin problema alguno, en los sistemas de
64-bit. Pero no obstante vea el enlace en \url{https://tug.org/texlive/windows.html} para las opciones en cómo añádir los binarios de 64-bit.

Vea la sección~\ref{sec:tl-coll-dists} para soluciones alternativas,
tanto para Windows como para \MacOSX.

\subsection{Instalación básica de \protect\TL{}}
\label{sec:basic}

Puedes instalar \TL{} mediante el \DVD{} o por descarga en el Internet
(\url{https://tug.org/texlive/acquire.html}). El instalador del net, es
pequeño, y descarga todo sin mayores problemas. 

El instalador del \DVD{} te permite instalar a un disco local. No se
puede ejecutar directamente \TL{} desde la \TK{} \DVD{} (o la
imagen \code{.iso}), pero puedes preparar una instalación, e.g,
mediante el uso de un disco portátil de \USB{} (vea la
sección~\ref{sec:portable-tl}). La instalación se detalla más tarde en
las secciones (\p.\pageref{sec:install}), pero aquí hay un breve
sumario:

\begin{itemize*}

\item El script de la instalación es nombrado \filename{intall-tl}. El mismo
	puede operar en el modo gráfico o ``modo gui'' una vez dada la opción
		\code{-gui} (estándar en Windows y Macs), o un modo de texto
		mediante \code{-gui=text} (estándar en todos los otros sistemas
		operativos). 

\item Otro de los programas que se instala, es el `\TL\ Manager', nombrado
	\prog{tlmgr}. Que al igual que el instalador, puede usarse con el mode
		\GUI{} o mediante el modo de texto. Puedes usarlo para instalar
		y desinstalar paquetes, y para realizar varias configuraciones.

\end{itemize*}

\htmlanchor{security}
\subsection{Consideraciones de seguridad}
\label{sec:security}

Con lo mejor de nuestro conocimiento, la base de los programas
que constituyen  \TeX\ como tal, es (y siempre han sido)
extremadamente robusta. Sin embargo, los programas que han sido
contribuidos a \TeX\ Live quizás no estén al mismo nivel, a
pesar de sobresalientes esfuerzos. Y de más está recalcar, que
se debe tener precaución y cuidado, cuando se ejecutan
aquellos programas originales no confiables. 

Esta necesidad de cautela es especialmente urgente en Windows, debido
entre tantos factores a que Windows encuentra los programas en el
directorio presente, antes cualquier otro, sin tener en cuenta el
search path, o la ruta de acceso que se haya establecido. Y esto trae
consigo un sin número de ataques. Y aunque hemos logrado cerrar muchos
de estos huecos, indudablemente algunos de estos, aún permanecen ahí,
especialmente con programas de third-party, o tercera-persona. Por lo
tanto, recomendamos que se revisen los ficheros sospechosos en el actual
directorio, principalmente aquellos que son ejecutables (como los
scripts y binarios). Normalmente, no deben estar presentes, ni tampoco
se recrean mediante el simple procesamiento o compilación de un
documento. 
  
Finalmente, \TeX\ (y todos sus programas acompañantes) pueden escribir
archivos cuando se están procesando los documentos, una característica
que puede ser abusada de varias maneras. Aún así, procesar documentos
desconocidos, en un nuevo subdirectorio, es lo más recomendable. 

\subsection{Consiguiendo ayuda}
\label{sec:help}

La comunidad de \TeX{} es activa y amistosa, y las preguntas más
complicadas, son respondidas. Sin embargo, el respaldo ofrecido es
informal, debido a que el mismo es ofrecido por voluntarios y lectores
casuales. De esta manera es importante que se familiarice con el
formato antes de hacer una pregunta. (Si prefieres un respaldo
garantizado y comercial, puedes abstenerte de \TL{} y comprar un
sistema del vendedor; \url{https://tug.org/interest.html#vendors} tiene
una lista.)

Aquí está una lista de recursos, aproximadamente en el orden, por el
cual las usamos:

\begin{description}
	\item[Comenzando] Si eres un principiante con \TeX, la página en el
	web \url{https://tug.org/begin.html} te ofrece en Inglés una
	introducción con el sistema.

\item [\TeX{} FAQ] El \TeX{} FAQ es un compendio de preguntas y
respuestas, también en Inglés, desde las más básicas, hasta las más
arcanas. Y está incluido en \TL{} en
\OnCD{texmf-dist/doc/generic/FAQ-en/}, y está disponible en el
Internet a través de la página en \url{https://www.tex.ac.uk.faq}. Por
favor, vaya allí primero.  \item [\TeX{} Catalogue] Si estás
	interesado en un específico paquete para un proyecto, fuente,
	programa, etc., el Catálogo de \TeX{} sería el primer lugar
	donde encontrarías esta información. Es una colección inmensa
	de todo lo relacionado con \TeX{}. Vea por ejemplo
	\url{https://ctan.org/pkg/catalogue/}.

\item [\TeX{} Recursos en el web] La página en el web
	\url{https://tug.org/interest.html} tiene muchos enlaces
	relacionados con el sistema \TeX{}, y en particular sobre
	muchos libros, manuales, guías y artículos relacionados con
	este sistema.

\item [Archivos de ayuda] Los principales foros de ayuda incluyen la comunidad
	de \LaTeX{} en \url{https://latex-community.org} el sitio de p\&r
		\url{https://tex.stackexchange.com}, el newsgroup Usenet en
		\url{news:comp.text.tex}, la lista de correo
		\email{texhax@tug.org}, Otros archivos tienen preguntas y
		respuestas.  como
		\url{https://groups.google.com/groupcomp.text.tex/topics}, al
		igual que \url{https://tug.org/mail-archives/texhax}. Estos
		últimos dos enlaces ofrecen un placentero mecanismo de búsqueda
		con lo que quieras averiguar.  También una búsqueda rápida en
		el web, \url{https://google.com} no está de más.

\item [Preguntar] Si no puedes encontrar una respuesta, siempre puedes
	escribir a \dirname{comp.text.tex} a través de Google o el
	lector de noticias, newsreader, o contactar mediante email a
	\email{texhax@tug.org}. Pero antes de enviar algo, \emph{por
	favor} de leer este FAQ, para maximizar los chances en
	encontrar una práctica respuesta:
	\url{https://www.tex.ac.uk/cgi-bin/texfaq2html?label=askquestion}.
	También puede acceder la Comunidad \LaTeX{} en
	\url{https://www.latex-community.org/} y el foro en
	\url{https://www.latex-community.org/forum/}, y \TeX\
	StackExchange en \url{https://tex.stackexchange.com/}.

\item [\TL{} support] si deseas reportar errores, o tiene una sugerencia
	o comentario sobre la distribución de \TL{}, la instalación, o
	la documentación del sistema, la lista de correo es
	\email{tex-live@tug.org}. Sin embargo, si la pregunta es
	acerca de un programa en particular incluido en \TL{}, por
	favor de escribirle al autor o mantenedor del programa en la
	lista de correo. Muchos de los programas incluyen una opción
	\code{-{}-help} que muestra la dirección donde se reporta el
	error. 

\end{description}

Por otra parte, siempre puede ayudar a otros que tengan preguntas.
Tanto \dirname{comp.text.tex} como \code{texhax} están disponibles
para todos. Usted puede formar parte del grupo, comenzar a leer muchas
de las preguntas, y ayudar en lo que sea necesario. 


\section{Sumario de \protect\TeX\protect\ Live}
\label{sec:overview-tl}

Esta sección describe los contenidos de \TL{} y de \TK{} que es parte
del mismo.

\subsection{La Colección de \protect\TeX\protect:~\TL{}, pro\TeX{}, Mac\TeX}
\label{sec:tl-coll-dists}

El \TK{} \DVD{} se compone de lo siguiente:

\begin{description}

\item [\TL] Un sistema completo de \TeX{} para instalar en el disco.
	Página en el web: \url{https://tug.org/texlive/}.

\item [Mac\TeX] para \MacOSX, que añade un instalador nativo de
	\MacOSX\ y otras aplicaciones relevantes de \TL{}. Página en
	el web en \url{https://tug.org/mactex/}.

\item [pro\TeX{}t] Una mejoría de la distribución de \MIKTEX\ para
	Windows, \ProTeXt\ añade unas extra herramientas para \MIKTEX,
	y también simplifica la instalación. Es completamente
	independiente de \TL{}, y tiene sus propias instrucciones para
	la instalación. La página en la web es en
	\url{https://tug.org/protext/}.

\item [\CTAN{}] Un panorama del repositorio de \CTAN{} en
	(\url{https://www.ctan.org/}).

\end{description}

\CTAN{} y \pkgname{protext} no siguen las mismas condiciones de copia
que \TL{}, así que tenga cuidado con redistribuirlo o modificarlo. 

\subsection{Directorios en el alto nivel de \TL{}} \label{sec:tld}

Aquí esta una breve lista y descripción de los directorios que están
en el alto nivel de la instalación de \TL{}. 

\begin{ttdescription}
\item[bin] Los programas del sistema \TeX{}, organizados según la
	plataforma.
%
\item[readme-*.dir] Esbozo rápido y enlaces importantes para \TL{}, en
	varios idiomas, en ambos \HTML{} y texto. 
%
\item[source] El código de los programas incluidos, al igual que las
	distribuciones de \TeX{} basadas en \Webc{}.
%
\item[texmf-dist] El árbol principal; ver también \dirname[TEXMFDIST]
	en las siguientes líneas.
%
\item[tlpkg] Scripts, programas y datos para administrar la
	instalación, y respaldo especial para el sistema operativo
	Windows.
%
\end{ttdescription}

En adición a los directorios ya mencionados, los scripts de la
instalación, junto a los ficheros \filename{README} y \filename{LEEME}
(en varios idiomas) se encuentran en el nivel más alto de la
distribución.

Para la documentación, los importantes enlaces en el fichero
\OnCD{doc.html} en el alto nivel, pueden servir de ayuda. Y la
documentación para todo lo demás (paquetes, formatos, fuentes,
manuales de los programas, documentación a través de man pages, y
ficheros de Info) que se encuentran en \dirname{texmf-dist/doc}. Usted
también puede usar el programa \cmdname{texdoc}, para encontrar
cualquier documentación.

Esta documentación \TL\ se encuentra en
\dirname{texmf-dist/doc/texlive}, disponible en varios idiomas:

\begin{itemize*}
\item{Alemán:} \OnCD{texmf-dist/doc/texlive/texlive-de}
\item{Checoslovaco/Eslovaco:} \OnCD{texmf-dist/doc/texlive/texlive-cz}
\item{Chino simplificado:} \OnCD{texmf-dist/doc/texlive/texlive-zh-cn}
\item{Francés:} \OnCD{texmf-dist/doc/texlive/texlive-fr}
\item{Inglés:} \OnCD{texmf-dist/doc/texlive/texlive-en}
\item{Italiano:} \OnCD{texmf-dist/doc/texlive/texlive-it}
\item{Polaco:} \OnCD{texmf-dist/doc/texlive/texlive-pl}
\item{Ruso:} \OnCD{texmf-dist/doc/texlive/texlive-ru}
\item{Serbio:} \OnCD{texmf-dist/doc/texlive/texlive-sr}
\item{Español:} \OnCD{texmf-dist/doc/texlive/texlive-es}
\end{itemize*}

\subsection{Resumen de los árboles predefinidos de texmf}
\label{sec:texmftrees}

Esta sección enumera las variables predefinidas que especifican los
árboles de texmf en el sistema, al igual que su propósito, y también
el diseño de \TL{}. El comando \texttt{tlmgr-conf} muestra los valores
de estas variables. Así usted puede aprender el proceso por el cual,
estas variables aplican o se representan en los directorios de su
instalación.

Todos los árboles, incluyendo los personales, deben guiarse por \TeX\
Directory Structure, o la Estructura del Directorio de \TeX\ (\TDS,
\url{https://tug.org/tds}), con la innumerable lista de
subdirectorios, o de lo contrario, los ficheros no se encontrarán. Vea
la sección \ref{sec:local-personal-macros}
(\p.\pageref{sec:local-personal-macros}) donde se detalla más acerca
de esto.

\begin{ttdescription}
\item [TEXMFDIST] El árbol que contiene casi todos los
	archivos de la distribución original--ficheros de configuración, scripts, paquetes, fuentes, etc. (La excepción aquí, son los ejecutables por-plataforma, los cuales están guardados en un directorio pariente \code{bin/}.)
\item [TEXMFSYSVAR] El árbol (nivel global) usado por \verb+texconfig-sys+, \verb+updmap-sys+, y \verb+fmtutil-sys+, y también usado por \verb+tlmgr+, para almacenar todo los datos del cache, que hayan sido causados por ficheros formateados, o de ficheros esquematizados, con mapas reasignados y regenerados.
\item [TEXMFSYSCONFIG] El árbol a nivel de sitio global, que es usado por las utilidades \verb+texmfconfig-sys+, \verb+updmap-sys+, y \verb+fmtutil-sys+ para guardar datos de configuración que hayan sido modificados.
\item [TEXMFLOCAL] El árbol el cual es usado por administradores para instalaciones globales, o la adicional instalación de macros actualizados, fuentes, etc.
\item [TEXMFHOME] El árbol el cual es usado por usuarios para sus propias instalaciones de adicionales fuentes, macros actualizados, etc. La expansión de esta variable, se ajusta dinámicamente para cada usuario, en sus propios directorios.
\item [TEXMFCONFIG] El árbol (personal) usado por las utilidades \verb+texconfig+, \verb+updmap-sys-, y también \verb+fmtutil-sys+, para guardar datos de configuración que hayan sido modificados.  
\item [TEXMFVAR] El árbol (personal) usado por \verb+texconfig+, \verb+updmap-user y \verb+fmtutil+user+ para guardar todo los datos que hayan sido generados tanto por el cache de ficheros regenerados y mapas reasignados, como también de ficheros formateados.
\item [TEXMFCACHE] El árbol o árboles usado por \ConTeXt\ Mk\acro{IV} y Lua\LaTeX\ para almacenar los datos del cache; estándar es \code{TEXMFSYSVAR}, o (si esto no se permite escribirlo), \code{TEXMFVAR}.
\end{ttdescription}

\noindent
El diseño estándar, o predeterminado, es el siguiente:
\begin{description}
	\item[raíz del sistema] puede abarcar múltiples versiones de \TL{}. 
	(\texttt{/usr/local/texlive} estándar en Unix):
  \begin{ttdescription}
    \item[2020] Una versión anterior.
    \item[2021] La versión actual
    \begin{ttdescription}
      \item [bin] ~
      \begin{ttdescription}
        \item [i386-linux] \GNU/Linux binarios
        \item [...]
        \item [universal-darwin] \MacOSX\ binarios
        \item [win32] Windows binarios
      \end{ttdescription}
      \item [texmf-dist\ \ ]      \envname{TEXMFDIST} and \envname{TEXMFMAIN}
      \item [texmf-var \ \ ]      \envname{TEXMFSYSVAR}, \envname{TEXMFCACHE}
      \item [texmf-config]        \envname{TEXMFSYSCONFIG}
    \end{ttdescription}
    \item [texmf-local] \envname{TEXMFLOCAL}, su uso es reservado de una versión a la otra.
  \end{ttdescription}
  \item[directorio de casa del usuario] (\texttt{\$HOME} o el
      \texttt{\%USERPROFILE\%})
    \begin{ttdescription}
      \item[.texlive2019] Datos privados generados de configuración para la versión anterior.
      \item[.texlive2020] Datos privados generados de configuración para la versión anterior.
      \item[.texlive2021] Datos privados generados de configuración para la versión actual.
      \begin{ttdescription}
        \item [texmf-var\ \ \ ] \envname{TEXMFVAR}, \envname{TEXMFCACHE}
        \item [texmf-config]    \envname{TEXMFCONFIG}
      \end{ttdescription}
    \item[texmf] \envname{TEXMFHOME} macros personales, etc.
  \end{ttdescription}
\end{description}

\subsection{Extensiones para \protect\TeX}
\label{sec:tex-extensions}

El \TeX{} original de Knuth está congelado, con ciertas reparaciones
de errores, de vez en cuando. El \TeX{} original está presente en \TL{} como el
programa \prog{tex}, y permanecerá así en el futuro. \TL{} también contiene
varias versiones extendidas de \TeX\ (conocidas también como los motores de
\TeX\ ):

\begin{description}

\item [\eTeX] añade un set de primitivos nuevos
\label{text:etex} (relacionado con la expansión de macros, escaneo de
caracteres, clases de marcas, otras características de depuración que
eliminan fallos, y otras más) y las extensiones de \TeXXeT{} para la
compilación bi-direccional de documentos de imprenta. En el modo
estándar, \eTeX{} tiene 100\% de compatibilidad con el \TeX\
básico.
Por favor, vea \OnCD{texmf-dist/doc/etex/base/etex_man.pdf}.

\item [pdf\TeX] se edifica en las extensiones de \eTeX{}, añadiendo
	respaldo para los resultados en \acro{PDF}, al igual que en
	\dvi{}. También ofrece otras extensiones. Este es el programa
	que se invoca para la mayoría de los formatos, e.g.,
	\prog{etex}, \prog{latex}, \prog{pdflatex}. La página del web
	para accedir al mismo se encuentra en
	\url{https://www.pdftex.org/}. En el CD, el manual se encuentra
	en \OnCD{texmf-distn/doc/pdftex/manual/pdftex-a.pdf}, al igual
	que en
	\OnCD{texmf-dist/doc/pdftex/samplepdf/samplepdf.tex}.
	Este último contiene muchos de los ejemplos para su
	aplicación.

\item [Lua\TeX] añade respaldo para ingreso de Unicode, al igual que para fuentes OpenType\slash TrueType, como también fuentes del sistema. Además de incorporar el intérprete Lua
	(\url{https://www.lua.org/}) el cual permite soluciones elegantes para
	muchos de los problemas de \TeX{}. Cuando se le invoca como
	\filename{texlua}, funciona como un intérprete independiente
	de Lua, y como tal, es usado dentro de \TL{}. La página de su
	sitio web se encuentra en \url{https://www.luatex.org/}, y el
	manual de referencia en
	\OnCD{texmf-dist/doc/luatex/base/luatexref-t.pdf}.

\item [(e)(u)p\TeX] tiene respaldo nativo para los requisitos de tipografía en japonés; p\TeX\ es el motor básico, mientras que las variantes e- añaden funconalidad en \eTeX\ y u- añade respaldo en Unicode.
 
\item [\XeTeX] añade respaldo para el ingreso de Unicode y OpenType,
	al igual que para las fuentes del sistema operativo. El mismo
	es implementado por librerías the tercer-party. Vea
	\url{https://tug.org/xetex}.

\item [\OMEGA\ (Omega)] es basado en Unicode (caracteres de 16-bit), y
	de esta forma respalda el trabajo con casi todos los scripts,
	simultáneamente. También respalda los llamados Procesos de
	Traducciones de Omega, o `\OMEGA{} Translation
	Processes'(\acro{OTP}s), que desempeñan transformaciones
	complejas en procesos de ingresos arbitrarios. Omega no está
	más incluido en \TL{} como un programa separado; solamente
	Aleph es proveído. 

\item [Aleph] combina las extensiones de ambos \OMEGA\ e \eTeX. Vea
	\OnCD{texmf-dist/doc/aleph/base} para más información.

\end{description}

\subsection{Otros notables programas en \protect\TL}

Aquí aparecen otros programas comúnmente usados en \TL{}:

\begin{cmddescription}

\item [bibtex, bibtex8] respaldo de bibliografía.

\item [makeindex, xindy] respaldo de índices.

\item [dvips] conversor \dvi{} a \PS{}.

\item [xdvi] \dvi{} presentador preliminar para el Sistema de X Window.

\item [dvilj] \dvi{} drive para la familia de HP LaserJet. 

\item [dviconcat, dviselect] cortar, copiar y pegar páginas de ficheros de \dvi{}.

\item [dvipdfmx] conversor \dvi{} a \acro{PDF}, una alternativa a pdf\TeX\ (que ya se mencionó).

\item [psselect, psnup, \ldots] Utilidades de \PS{}.

\item [pdfjam, pdfjoin, \ldots] Utilidades de \acro{PDF}..

\item [context, mtxrun] procesador de Con\TeX{}t y \acro{PDF}.

\item [htlatex, \ldots] Conversor \cmdname{tex4ht}: \AllTeX{} a \acro{HTML} (y
\acro{XML}).

\end{cmddescription}


\htmlanchor{instalación}
\section{Instalación}
\label{sec:install}

\subsection{Comenzando con el instalador}
\label{sec:inst-start}

Para comenzar, consiga el \TK{} \DVD{} o descargue el instalador de
net \TL{}, y localice el script instalador: \filename{install-tl} para
los sistemas Unix, o \filename{install-tl.bat} para Windows. Por
favor, vea
\url{https://tug.org/texlive/acquire.html} para más información, y
otros métodos para conseguir el programa.

\begin{description}
	\item [Instalador de net] Descargue de \CTAN{}, bajo
		\dirname{systems/texlive/tlnet}; página de web en
		\url{https://mirror.ctan.org/systems/texlive/tlnet}
		automáticamente lo dirigirá hacia el repositorio
		actualizado, más cercano. Usted puede  descargar
		\filename{install-tl.zip}, el cual puede ser usado
		bajo cualquier plataforma basada en Unix, o Windows, o
		como otra alternativa, puede hacerlo con
		\filename{install-unx.tar.gz} para Unix, que es mucho
		más pequeño. Después que lo hayas descomprimido, de
		su estado original, \filename{install-tl} y
		\filename{install-tl.bat} se encontrarán en el
		subdirectorio \dirname{install-tl}.

\item [Colección de \TeX{} en \DVD:] vaya al subdirectorio del \DVD en
	\dirname{texlive}. En Windows, el instalador normalmente
	arranca automáticamente cuando insertes el \DVD. Puedes
	conseguir el \DVD\, registrándote como miembro de cualquier
	Grupo de Usuarios de \TeX\ (altamente recomendable, en
	\url{https://tug.org/usergroups.html}). También lo puedes
	conseguir, comprándolo separado de \url{https://tug.org/store},
	o puedes grabar tu propia copia de la imagen \ISO\ que se
	provee. Después de instalarlo de tu \DVD\, o de la imagen
	\ISO\, si deseas continuar con las actualizaciones del
	Internet, por favor vea \ref{sec:dvd-install-net-updates}.

\end{description}

\begin{figure}[tb]
\tlpng{nsis_installer}{.6\linewidth}
\caption{Primera fase del instalador de Windows \code{.exe} }\label{fig:nsis}
\end{figure}

El instalador como tal realiza lo mismo, sin importar el método que
se utilice. La diferencia más visible entre ellos, es que con el
instalador del Net, usted termina con todos los paquetes que están
disponibles, mientras que con el \DVD\ o con la imagen de
\ISO, estos no proveen las actualizaciones más recientes de
ellos.

Si necesitas realizar las descargas a través de proxies, utilice el fichero \filename{~/.wgetrc} o las variables del entorno del sistema con las preferencias parar Wget (\url{https://www.gnu.org/software/wget/manual/html_node/Proxies.html}), o el equivalente de cualquier otro programa de descarga. Esto no aplica si se instala de un \DVD o the una imagen de \ISO.

\noindent
Las siguientes secciones explican el comienzo del instalador con más
detalles.

\subsubsection{Unix}

Debajo de este párrafo, el carácter \texttt{>} denota el
punto de la terminal o del intérprete de línea de comando u órdenes shell.
El ingreso del usuario se encuentra en
\Ucom{\texttt{negritas}}.  El script que lleva de nombre
\filename{install-tl} es un script de Perl. La manera más
simple de inicializarlo en un sistema compatible con Unix, es
de la siguiente manera: 

\begin{alltt} 
	> \Ucom{perl /ruta/hacia/install-tl} 
\end{alltt} 

(O puedes invocar \Ucom{/ruta/hacia/installer/install-tl} si
permaneció como ejecutable, o \texttt{cd} al directorio
primero, etc.; no repetiremos todas las variaciones.)
Posiblemente tuvieses que agrandar la ventana de la
terminal, para que muestre el texto completo del instalador,
en la pantalla.  (Figura~\ref{fig:text-main}).

Para instalarlo en modo experto con Interfaz Gráfica de
Usuario \GUI\, necesitarás tener Tcl/Tk instalado. Dado
esto, puedes ejecutar:
\begin{alltt}
	> \Ucom{perl install-tl -gui}
\end{alltt}

En cuanto al antiguo instalador mago \code{wizard} y las opciones
\code{perltk}/\code{expert} están aún disponibles. Pero actualmente hace la misma función que \code{-gui}.

Para un listado completo de las varias opciones:
\begin{alltt}
> \Ucom{perl install-tl -help}
\end{alltt}

\textbf{Precaución con los permisos en Unix:} Tu \code{umask} en el
momento de instalación, será respetado por el instalador de \TL{}. De
tal manera, si quieres que tu instalación sea accesible por otros
usuarios, asegúrate que los settings o los ajustes, así lo
indiquen, por ejemplo, \code{umask002}. Para más
información acerca de 
\code{umask}, consulte la documentación de tu sistema. 

\textbf{Consideraciones especiales con Cygwin:} A contrario de los
sistemas compatibles con Unix, Cygwin, por norma o estándar, no
incluye los programas pre-requeridos y necesitados por el instalador
\TL{}. Vea la sección~\ref{sec:cygwin}.

\subsubsection{MacOSX}
\label{sec:macosx}

Como se mencionó en la sección~\ref{sec:tl-coll-dists}, para el
sistema operativo de \MacOSX existe una distribución de \TL{} separada, 
y que fue preparada exclusivamente para este sistema, nombrada
Mac\TeX\ (\url{https://tug.org/mactex}).  Recomendamos que utilices
este instalador nativo Mac\TeX\ en vez de \TL\, porque el instalador
nativo hace algunos ajustes que son específicos con el sistema
operativo \MacOSX, y permite la alternación entre varias
distribuciones de \TeX\ para este sistema. (Mac\TeX, Fink, MacPorts,
\ldots) mediante la estructura de datos de \TeX{}Dist.

Mac\TeX\ está basado en \TL, y los árboles y binarios de \TeX\ son los mismos que
este. Lo que hace, es añadir unos ficheros extras que tienen documentación y
aplicaciones específicas con el sistema operativo de Mac.

\subsubsection{Windows}\label{sec:wininst}

Si estás usando el archivo descomprimido zip de descarga, o el
instalador del \DVD\ falló en arrancar automáticamente, dele doble
clic a \filename{install-tl-windows.bat}. Y si necesitas más opciones
aún, e.g., selección de colecciones de paquetes específicos, en vez de
abrir el anterior, ejecuta el \filename{install-tl-advanced.bat}. 

También puedes arrancar el instalador, mediante el comando de la
terminal. Más abajo, el carácter \texttt{>} denota el indicador del comando;
mientras que el ingreso del comando por el usuario se denota con
\Ucom{\texttt{bold}}. Si estás en el directorio del instalador, sólo
tienes que ejecutar:
\begin{alltt}
	> \Ucom{install-tl-windows}
\end{alltt}

O también lo puedes invocar con una localización absoluta, tal como:
\begin{alltt}
	\Ucom{D:\bs{}texlive\bs{}install-tl-windows}
\end{alltt}
para el \TK\ \DVD, suponiendo que el directorio \dirname{D:} es donde
esté localizado el \textsc{DVD} o \textsc{CD}.

Figura~\ref{fig:basic-w32} muestra la pantalla básica
inicial del instalador \GUI\ que es el estándar en
Windows.\\

Para instalarlo mediante el modo de texto:
\begin{alltt}
	> \Ucom{install-tl-windows -no-gui}
\end{alltt}

Para un listado completo de las varias opciones disponibles:
\begin{alltt}
	> \Ucom{install-tl-windows -help}
\end{alltt}

\begin{figure}[tb]
\begin{boxedverbatim}

Installing TeX Live 2021 from: ...
Platform: x86_64-linux => 'GNU/Linux on x86_64'
Distribution: inst (compressed)
Directory for temporary files: /tmp
...
 Detected platform: GNU/Linux on Intel x86_64
 
 <B> binary platforms: 1 out of 16

 <S> set installation scheme: scheme-full

 <C> customizing installation collections
     40 collections out of 41, disk space required: 7172 MB

 <D> directories:
   TEXDIR (the main TeX directory):
     /usr/local/texlive/2021
   ...

 <O> options:
   [ ] use letter size instead of A4 by default
   ...
 
 <V> set up for portable installation

Actions:
 <I> start installation to hard disk
 <P> save installation profile to 'texlive.profile' and exit
 <H> help
 <Q> quit
\end{boxedverbatim}
\vskip-.5\baselineskip
\caption{Pantalla del instalador principal de texto (\GNU/Linux)}\label{fig:text-main}
\end{figure}

\begin{figure}[tb]
\tlpng{basic-w32}{.6\linewidth}
	\caption{Pantalla del instalador básico (Windows);
	el botón Avanzado resultará a algo similar como lo
	que se muestra en la
	figura~\ref{fig:advanced-lnx}}\label{fig:basic-w32}
\end{figure}

\begin{figure}[tb]
\tlpng{basic-w32}{\linewidth}
\caption{Pantalla del instalador mago
	(Windows)}\label{fig:basic-w32}
\end{figure}

\begin{figure}[tb]
\tlpng{advanced-lnx}{\linewidth}
\caption{Pantalla del Instalador Avanzado \GUI{}
	(\GNU/Linux)}\label{fig:advanced-lnx}
\end{figure}

\htmlanchor{cygwin}
\subsubsection{Cygwin}
\label{sec:cygwin}

Antes de comenzar la instalación, ejecute \filename{setup.exe} del
programa de Cygwin, para instalar los paquetes \filename{perl} y
\filename{wget} si no lo ha hecho. Se recomiendan los siguientes
paquetes:
\begin{itemize*}
	\item \filename{fontconf}[necesitado por \XeTeX\ y Lua\TeX]
	\item \filename{ghostscript} [necesitado por varias utilidades]
	\item \filename{libXaw7} [necesitado por \code{xdvi}]
	\item \filename{ncurses} [provee el comando \code{clear} usado por el instalador]
\end{itemize*}

\subsubsection{El instalador de texto}

Figura~\ref{fig:text-main} muestra una pantalla con el modo principal
de texto en Unix. El instalador de texto es el estándar en Unix. 

Este es solamente un instalador a través de la línea de comando; ni el
indicador, ni la selección mediante el ratón es posible. Por ejemplo,
tampoco es posible tabular con el teclado, ni sombrear las cajas o las
tablas que normalmente se haría con el \GUI. Solamente se permite
teclear (con minúsculas) en la línea de comando, y presionar la tecla
de Enter una vez terminado. La pantalla es reescrita y ajustada con el
contenido apropiado. 

La interfaz del instalador de texto tiene esta
característica primitiva por una razón: está diseñado para
ser ejecutado en varias plataformas, incluso bajo un
sistema, con un mínimo código Perl. 

\subsubsection{El instalador gráfico}
\label{sec:graphical-inst}

El instalador gráfico estándar comienza sencillamente con
unas cuantas opciones: vea la figura~\ref{fig:basic-w32}.
Puede ser iniciado con:
\begin{alltt}
	> \Ucom{install-tl -gui}
\end{alltt}

El botón Avanzado ofrece acceso a la mayoría de las opciones
del instalador de texto;  vea la
figura~\ref{fig:advanced-lnx}.

\subsubsection{Los instaladores de legado}

El modo experto y el modo mago de
Las opciones de \texttt{perltk}\texttt{expert} y \GUI{} \texttt{wizard} ahora invocan al instalador regular gráfico.

\subsection{Ejecutando el instalador}
\label{sec:runinstall}

El instalador es explícito por diseño, pero a continuación hay unas
notas acerca de sus opciones y sub-menús. 

\subsubsection{Menú de sistemas binarios (Unix solamente)}
\label{sec:binary}

\begin{figure}[tb]
\begin{boxedverbatim}
Plataformas disponibles
================================================================================   
   a [ ] Cygwin en Intel x86 (i386-cygwin)
   b [ ] Cygwin en x86_64 (x86_64-cygwin)
   c [ ] MacOSX actual (10.14-) en ARM/x86_64 (universal-darwin)
   d [ ] MacOSX legacy (10.6-) on x86_64 (x86_64-darwinlegacy)
   e [ ] FreeBSD en x86_64 (amd64-freebsd)
   f [ ] FreeBSD en Intel x86 (i386-freebsd)
   g [ ] GNU/Linux en ARM64 (aarch64-linux)
   h [ ] GNU/Linux en ARMv6/RPi (armhf-linux)
   i [ ] GNU/Linux en Intel x86 (i386-linux)
   j [X] GNU/Linux en x86_64 (x86_64-linux)
   k [ ] GNU/Linux en x86_64 with musl (x86_64-linuxmusl)
   l [ ] NetBSD en x86_64 (amd64-netbsd)
   m [ ] NetBSD en Intel x86 (i386-netbsd)
   o [ ] Solaris en Intel x86 (i386-solaris)
   p [ ] Solaris en x86_64 (x86_64-solaris)
   s [ ] Windows (win32)
\end{boxedverbatim}
\vskip-.5\baselineskip
\caption{Menú de los binarios}\label{fig:bin-text}
\end{figure}

Figura~\ref{fig:bin-text} muestra el menú binario en el modo de texto.
Por estándar, solamente los binarios de su plataforma actual serán
instalados. Desde este menú, usted puede seleccionar la instalación de
los binarios para otras plataformas. Esto puede ser provechoso, si
estás compartiendo un árbol de \TeX\ a través de un network de
máquinas heterogéneas, o también en aquellos sistemas que tienen dos o
más sistemas operativos. 

\subsubsection{Seleccionar lo que va a ser instalado}
\label{sec:components}

\begin{figure}[tbh]
\begin{boxedverbatim}
Select scheme:
===============================================================================
 a [X] full scheme (everything)
 b [ ] medium scheme (small + more packages and languages)
 c [ ] small scheme (basic + xetex, metapost, a few languages)
 d [ ] basic scheme (plain and latex)
 e [ ] minimal scheme (plain only)
 f [ ] ConTeXt scheme
 g [ ] GUST TeX Live scheme
 h [ ] teTeX scheme (more than medium, but nowhere near full)
 i [ ] XML scheme
 j [ ] custom selection of collections
\end{boxedverbatim}
\caption{Menú del esquema}\label{fig:scheme-text}
\end{figure}

Figura~\ref{fig:scheme-text} muestra el esquema del menú de \TL. Desde
aquí, puedes seleccionar un ``esquema'', el cual es un conjunto de las
colecciones de los paquetes. El esquema estándar \optname{full},
instala todo lo que está disponible. Esto es recomendable, pero
también puedes escoger e instalar el esquema básico \optname{basic}
para aquellos sistemas pequeños, el esquema mínimo \optname{minimal}
para los sistemas de evaluación y pruebas, y mediano \optname{medium} o
el \optname{teTeX}, que ofrece un set de paquetes con ambos propósitos
de evaluación al igual que básico. También hay varios esquemas
especializados y específicos a regiones o países.

\begin{figure}[tb]
\centering \tlpng{stdcoll}{.7\linewidth}
\caption{Menú con las Colecciones disponibles}\label{fig:collections-gui}
\end{figure}

Se puede refinar la selección del esquema con el menú `colecctions'
(figura~\ref{fig:collections-gui}, mostrado en el modo \GUI\, para
cambiar.)

Las colecciones son un nivel más detallado que los esquemas\Dash en
esencia, un esquema consiste de varias colecciones, lo que una colección
consiste en uno o más paquetes, y un paquete (el grupo del nivel más
abajo en la estructura de \TL) contiene los archivos de macros de
\TeX{}, archivos de fuentes, y otros.

Si deseas más control sobre la colección que el menú provee, puedes
usar el programa \TeX\ Live Manager (\prog{tlmgr}) después de la
instalación (vea la sección~\ref{sec:tlmgr}); usando este último,
usted puede controlar la instalación e información de los paquetes, al
nivel donde estos se encuentren, en la estructura del sistema.

\subsubsection{Directorios}
\label{sec:directories}

El diseño estándar está descrito en la sección~\ref{sec:texmftrees},
\p.\pageref{sec:texmftrees}. La localización estándar de
\dirname{TEXDIR} es \dirname{/usr/local/texlive/2021} en los sistemas
Unix y |%SystemDrive%\texlive\2021| en Windows.  Acomodándolo de esta manera,
usted tendrá instalaciones paralelas de \TL. Una para cada versión, y puede
cambiar entre ambas, simplemente cambiando la ruta de búsqueda. 

Esto puede ser sobrescrito mediante la especificación de
configuraciones
de \dirname{TEXDIR} en el instalador. La razón principal en cambiarlo,
es por ejemplo si usted no tiene los permisos necesarios para
escribirlos en la localización estándar. No se tiene que ser el
administrador del sistema, o la `raíz', para instalar \TL, pero hay que recalcar que sí
tiene que tener acceso para escribirlo al directorio de destino. 

Los directorios de instalación pueden ser también modificados, ajustando las preferencias a través de las variables del entorno del sistema antes de ejecutar el instalador (lo más propenso en, 
\envname{TEXLIVE\_INSTALL\_PREFIX} o
\envname{TEXLIVE\_INSTALL\_TEXDIR}); vea la documentación de 
|install-tl --help| (disponible en línea en
\url{https://tug.org/texlive/doc/install-tl.html}) para la lista completa que ofrece más detalles.

Una alternativa razonable es un directorio local, o el
directorio de la casa (home directory), especialmente si
usted es el único usuario. Use `|~|' para indicar esto, como
por ejemplo `|~/texlive/2021|'.

Recomendamos incluir el año del calendario, en el nombre del susodicho
directorio. De esa manera, puede tener varias versiones de \TL{}
juntas en el sistema. También puede tener un nombre con versión
independiente, tal como \dirname{/usr/local/texlive-cur}, mediante un
enlace simbólico, el cual puede redirigir después de haber evaluado la
nueva versión de la instalación.

El cambiar \dirname{TEXDIR} en el instalador, también cambiará
\dirname{TEXMFLOCAL}, \dirname{TEXMFSYSVAR} y
\dirname{TEXMFSYSCONFIG}.

\dirname{TEXMFHOME} es la localización recomendable para los archivos
personales de macros, o de paquetes. El valor estándar es |~/texmf| (|~/Library/texmf| en las Macs). A
diferencia con \dirname{TEXDIR}, aquí una |~| es preservada en los
recién escritos archivos de configuración, debido a que se refiere al
directorio local del usuario que opera \TeX. Se expande a
\dirname{$HOME} en Unix y \verb|%USERPROFILE%| en Windows.  

Nota especial redundante:
\envname{TEXMFHOME}, como todos los árboles, tiene que ser organizado
de acuerdo al \TDS\ (Estructura del Directorio de \TeX{}), o de lo
contrario, los archivos no se encontrarán. 

\dirname{TEXMFVAR} es la localización para almacenar la mayor cantidad
de cache, de los datos de ejecución, específicos a cada usuario.
\dirname{TEXMFCACHE} es el nombre de la variable que es usada para ese
propósito por Lua\LaTeX\ y \ConTeXt\ Mk\acro{IV} (vea la
sección~ \ref{sec:context-mkiv},) \p.\pageref{sec:context-mkiv}); su
valor pre-configurado es \dirname{TEXMFSYSVAR}, o (si no es permisible en
los permisos de escritos), \dirname{TEXMFVAR}.

\subsubsection{Opciones}
\label{sec:options}

\begin{figure}[tbh]
\begin{boxedverbatim}
Options setup:
===============================================================================
 <P> use letter size instead of A4 by default: [ ]
 <E> execution of restricted list of programs: [X]
 <F> create all format files:                  [X]
 <D> install font/macro doc tree:              [X]
 <S> install font/macro source tree:           [X]
 <L> create symlinks in standard directories:  [ ]
            binaries to:
            manpages to:
                info to:
 <Y> after installation, get package updates from CTAN: [X]
\end{boxedverbatim}
\vskip-.5\baselineskip
\caption{Menú con opciones disponibles (Unix)}\label{fig:options-text}
\end{figure}

Figura~\ref{fig:options-text} muestra el menú de opciones bajo el modo de texto. 
Más información de cada uno:

\begin{description}
\item[use letter size instead of A4 by default:] La selección del
	papel estándar. Por supuesto, para documentos individuales, el
	tamaño del papel se puede, y debe ser especificado.  

\item[execution of restricted list of programs:] A partir de \TL\
	2010, la ejecución de varios programas externos es permisible.
	La (breve) lista de los programas que son permitidos, se puede
	encontrar en el archivo \filename{texmf.cnf}. Si necesita más
	detalles, vea las noticias del 2010 \ref{sec:2010news}.

\item[create all format files:] Aunque muchos archivos formateados,
	lleven tiempo no solo en generar, sino también en almacenar, es
	aún recomendable que esta opción se mantenga chequeada,
	de lo contrario, estos archivos formateados se
	generarían en el árbol privado del usuario, así como sea
	necesario. En esa localización, los mismos no serán
	actualizados automáticamente, digamos por ejemplo, los binarios,
	o los patrones de las separaciones por guión, son
	actualizados, y por esa razón, usted puede terminar con
	archivos formateados incompatibles. 

\item[install font/macro \ldots\ tree:] Omita descargar/instalar la
	documentación y archivos originales, incluídos en la mayoría de
	los paquetes. No es recomendable. 

\item[create symlinks in standard directories:] Esta opción (solamente
	en Unix) evita cambiar las variables del sistema. Sin esta
	opción, los directorios de \TL{} usualmente tienen que ser
	añadidos a \envname{PATH}, \envname{MANPATH}, e
	\envname{INFOPATH}. Necesitaría los permisos para escribir en
	los directorios de destino. Es extremadamente aconsejable
	\emph{no} sobrescribir un sistema de \TeX\ en tu sistema, con
	esta opción. Es principalmente utilizado para acceder el
	sistema de \TeX\ a través de aquellos directorios, que son
	conocidos para los usuarios, tales como
	\dirname{/usr/local/bin}, que no contienen ningún archivo de
	\TeX.

\item[después de instalación, especifique \CTAN\ como la fuente para
	todas las actualizacinoes:] Cuando está instalando de \DVD, esta
	opción está autorizada por estándar, debido a que
	usualmente, uno actualizaría los subsiguientes paquetes,
	mediante las actualizaciones de \CTAN que ocurren
	durante el año. La única razón para no autorizarlo, es
	si usted instala solamente un subset del \DVD\ y planea
	en resumir la instalación después. De cualquier manera,
	tanto el repositorio de paquetes del instalador, al
	igual que las actualizaciones después de instalación,
	pueden ser fijadas, establecidas, tanto como sea
	necesario; ver la sección~\ref{sec:location}, y la
sección \ref{sec:dvd-install-net-updates}~.para más detalles.  
\end{description}

Opciones específicas en Windows, como se muestra en la interfaz avanzada \GUI{}

\begin{description}
\item[adjust searchpath] Esto asegura que todos los programas verán el directorio binario de \TL{} en la ruta de búsqueda. 

\item[add menu shortcuts] Si está configurado, habrá un submenú \TL{} en el Start menu. Hay también una tercera opción `Launcher entry` al lado de `TeX Live menu` y `No shortcuts`. Esta opción está descrita en la sección 
\ref{sec:sharedinstall}.

\item[File associations] Las opciones aquí son `Only new', solamente nuevas asociaciones, (que crea las asociaciones de los archhivos, pero no sobrescriben las ya existentes), `All' (todas las asociaciones) y `None' (ninguna asociación).

\item[install \TeX{}works front end]
\end{description}

Cuando todas las preferencias personalizadas y configuraciones, hayan
sido especificadas, usted puede ingresar la `I', para comenzar el
proceso de instalación.  Cuando esto finalice, vaya a la
sección~\ref{sec:postinstall} para leer qué más se necesitaría hacer. 

\subsection{Opciones con la Línea de Comando install-tl}
\label{sec:cmdline}

Ingrese 
\begin{alltt}
	> \Ucom{install-tl -help}
\end{alltt}
para una lista de opciones de la línea de comandos. Así sea |-| o |--|
pueden ser usados para los nombres de las opciones. Los
siguientes opciones, son los más comunes:

\begin{ttdescription}
\item[-gui] Si es posible, use el instalador del \GUI{}. Esto requiere
		el módulo de Tcl/Tk versión 8.5 o más reciente.  Este es el
		caso en \MacOSX\ y con \TL{} que se distribuye en Windows.  Las
		opciones de legado \texttt{-gui=perltk} y \texttt{-gui=wizard}
		están aún disponibles pero actualmente invocan la misma interfaz \GUI{} y si por alguna razón Tcl/Tk no está disponible, la instalación entonces continúa, en el modo de texto. 

\item[-no-gui] Obligue el instalador de texto, incluso bajo Windows. 

\item[-lang {\sl LL}] Especifique el interfaz del idioma del
	instalador, mediante un código estándar (usualmente
	dos-letras). El instalador determina automáticamente el idioma
	correcto, pero si falla, o si el idioma seleccionado no está
	disponible, se utilizará el inglés como reemplazo. Ingrese
	\verb|install-tl --help| para ver la lista de idiomas
	disponibles.

\item[-portable] Instale para uso portátil, e.g., una tarjeta flash
	drive \USB{}. También seleccionable, a través del instalador
	con el comando \code{V}, y desde el instalador gráfico de
	\GUI{}. Vea la sección~\ref{sec:portable-tl}.

\item[-profile {\sl file}] Cargue el perfil de instalación \var{file},
	y haga la instalación sin interacción del usuario. El
	instalador siempre escribe un archivo
	\filename{texlive-profile} al subdirectorio \dirname{tlpkg} de
	su instalación. Ese archivo, puede ser dado como un argumento,
	para replicar la misma y exacta instalación en un sistema
	diferente. Alternativamente, usted puede usar un perfil
	personalizado, que es fácilmente creado, comenzando con uno
	que haya sido generado, y cambiar los valores, o también de un
	archivo vacío, el cual tomaría todos los valores estándares.

\item[-repository {\sl url-or-directory}] Especifique el repositorio
	de paquetes de donde instalarlo; vea lo siguiente.


\htmlanchor{opt-in-place}
\item[-in-place] (Documentado solamente para concluir: No use esto, a
	menos que sepa lo que está haciendo.) Si ya tiene un rsync,
	svn, o otra copia de \TL{} (vea
	\url{https://tug.org/texlive/acquire-mirror.html}) entonces
	esta opción implementará lo que tiene, tal y como es, y hará
	solamente la necesaria post-instalación. Se le advierte que el
	archivo \filename{tlpkg/texlive.tlpdb} puede ser sobrescrito;
	guardarlo es su responsabilidad. No se olvide, que la
	eliminación de paquetes, tiene que hacerse manualmente. Esta
	opción no puede ser intercambiada mediante el interfaz del
	instalador. 
\end{ttdescription}

\subsubsection{La opción \optname{-repository}}
\label{sec:location}

El repositorio estándar de la red, es un espejo de \CTAN{}, escogido
automáticamente por \url{http://mirror.ctan.org}.

Si deseas sobrescribir eso, el valor de la localización, puede ser un
url comenzando con \texttt{ftp:}, \texttt{http:}, \texttt{https:}, o \texttt{file:/}, o
la ruta del directorio. (Cuando se especifica un \texttt{http:}\ o
\texttt{ftp:}\ o \texttt{https:}\ o cualquier retazo de caracteres como `\texttt{/}' y/o
retazos de componentes como `\texttt{/tlpkg}', estos son ignorados.)

Por ejemplo, usted puede escoger un espejo en particular de \CTAN\ con
algo como:
\url{http://ctan.example.org/tex-archive/systems/texlive/tlnet/},
sustituyendo un servidor del network, con el nivel alto de la ruta de
\CTAN\ por |ctan.example.org/tex-archive|. La lista de espejos de
\CTAN\ es mantenida en \url{https://ctan.org/mirrors}.

Si el argumento es local (así sea una ruta, o un \texttt{file:/} url),
los archivos comprimidos en el subdirectorio \dirname{archive} de la ruta
del directorio, son entonces los que son así utilizados (incluso cuando los archivos
descomprimidos estén disponibles también). 

\subsection{Acciones de post-instalación}
\label{sec:postinstall}

Alguna post-instalación puede ser requerida.

\subsubsection{Las variables del sistema en Unix}
\label{sec:env}

Si elegiste crear enlaces simbólicos o symlinks en los directorios
estándar (descrito en la sección~\ref{sec:options}), no hay necesidad
por lo tanto de editar las variables. De otra manera, en los sistemas
de Unix, el directorio de los binarios de tu plataforma tienen que ser
añadido a la ruta de acceso, o el search path. (En Windows, el
instalador se ocupa de esto.)

Cada plataforma respaldada tiene su propio subdirectorio bajo
\dirname{TEXDIR/bin}. Vea la figura~\ref{fig:bin-text} para la lista
de subdirectorios y su correspondiente plataforma. 

Opcionalmente, si quieres que las herramientas del sistema lo
encuentren, también puedes añadir la documentación man y los
directorios Info, a su respectiva ruta de acceso. Las páginas man
pudiesen ser encontradas manualmente después de la adición a
\envname{PATH}. 

Para las terminales compatibles de Bourne, como la \prog{bash}, y
usando un Intel x86 GNU/Linux y un directorio estándar como ejemplo,
el archivo que podría ser editado es \filename{$HOME/.profile} (o
cualquier otro archivo), especificado por \filename{.profile} y las
líneas que consiguientemente se añadirían, serían:

\begin{sverbatim}
PATH=/usr/local/texlive/2021/bin/i386-linux:$PATH; export PATH
MANPATH=/usr/local/texlive/2021/texmf-dist/doc/man:$MANPATH; export MANPATH
INFOPATH=/usr/local/texlive/2021/texmf-dist/doc/info:$INFOPATH; export INFOPATH
\end{sverbatim}

Para csh o tcsh, el archivo para editar es típicamente
\filename{$HOME/.cshrc}, y las líneas que se tuviesen que añadir, serían:

\begin{sverbatim}
setenv PATH /usr/local/texlive/2021/bin/i386-linux:$PATH
setenv MANPATH /usr/local/texlive/2021/texmf-dist/doc/man:$MANPATH ; export MANPATH
setenv INFOPATH /usr/local/texlive/2021/texmf-dist/doc/info:$INFOPATH ; export INFOPATH
\end{sverbatim}

Si ya tienes los ajustes necesarios en los archivos ``dot'',
naturalmente los directorios de \TL\ deben ser integrados
apropiadamente. 

\subsubsection{Variables del sistema: configuración global}
\label{sec:envglobal}

Si quieres hacer cambios globales, o para un usuario recién añadido,
será bajo tu propia responsabilidad; simplemente hay mucha variación
entre los sistemas, y cómo y dónde serían las configuraciones. 

Nuestras dos recomendaciones son: 

1)~quizás chequea un file
\filename{/etc/manpath.config} y, si está presente, añade líneas como:

\begin{sverbatim}
MANPATH_MAP /usr/local/texlive/2021/bin/i386-linux \
            /usr/local/texlive/2021/texmf-dist/doc/man
\end{sverbatim}

Y 2)~verifica un archivo \filename{/etc/environment}, el cual puede
definir la ruta de acceso y otras variables del sistema.

En cada directorio binario (Unix), también creamos un enlace simbólico
nombrado \code{man}, al directorio \dirname{texmf-dist/doc/man}.
Algunos programas \code{man}, tales como el estándar \MacOSX\
\code{man}, lo encontrarán automáticamente, evitando la necesidad de
una configuración adicional para la página del programa
\code{man}.

\subsubsection{Actualizaciones del Internet después de instalación}
\label{sec:dvd-install-net-updates}

Si instalastes \TL\ de un \DVD\ y después quieres obtener
actualizaciones a través del Internet, necesitas ejecutar este
comando---\emph{después} que hayas actualizado la ruta de
búsqueda (como se describe en la sección anterior):

\begin{alltt}
> \Ucom{tlmgr option repository http://mirror.ctan.org/systems/texlive/tlnet}
\end{alltt}

Esto le dice a \cmdname{tlmgr}, que utilice el espejo de \CTAN\ más
cercano, para futuras actualizaciones. 

Este es el estándar cuando se instala de un \DVD con la opción
descrita en la sección~\ref{sec:options}

Si hay problemas con la selección automática de espejos, puedes
especificar un espejo particular de \CTAN, de la lista en
\url{https://ctan.org/mirrors}. Usa la ruta exacta al subdirectorio
\dirname{tlnet} en ese espejo, como se mostró anteriormente.

\htmlanchor{xetexfontconfig}  % keep historical anchor working
\htmlanchor{sysfontconfig}
\subsubsection{Configuración de la fuente para \XeTeX\ y Lua\TeX}
\label{sec:font-conf-sys}

\XeTeX\ y Lua\TeX\ pueden usar cualquier fuente instalada en el
sistema, no solo aquellas que están instaladas en los árboles \TeX.
Esto lo hacen relacionadas entre sí, pero no mediante métodos
idénticos. 

En Windows, las fuentes incluidas con \TL\ están disponibles
automáticamente en \XeTeX\ a través del nombre de la fuente. En el
sistema operativo \MacOSX, el respaldo para la búsqueda de las
fuentes, requiere pasos adicionales; por favor consulte las páginas en el web de Mac\TeX\ (\url{https://tug.org/mactex}) para más detalles acerca de esto. Para otros sistemas
operativos basados en Unix, el procedimiento para encontrar estas
fuentes es de la siguiente manera. 

Cuando el paquete \pkgname{xetex} es instalado (así sea
durante la instalación inicial o después), el archivo
necesario para la configuración es creado en:
\filename{TEXMFSYSVAR/fonts/conf/texlive-fontconfig.conf}. 

Para configurar y fijar las fuentes de \TL{} para el uso
global del sistema (asumiendo que usted tenga privilegios),
es de la siguiente manera: 
\begin{enumerate*} 
	\item Copia el archivo \filename{texlive-fontconfig.conf} a
\dirname{/etc/fonts/conf.d/09-texlive.conf}.  \item Ejecuta
\Ucom{fc-cache -fsv}.  
\end{enumerate*}

Si no tienes suficientes privilegios con las indicaciones anteriores, o
si quieres que las fuentes estén disponibles solamente para un
usuario, puedes hacer lo siguiente:
\begin{enumerate*}
	\item Copia el archivo \filename{texlive-fontconfig.conf} a \filename{~/.fonts.conf}, donde \filename{~} es el directorio de la casa. 
	\item Ejecuta \Ucom{fc-cache -fv}. 
\end{enumerate*}

Puedes ingresar \code{fc-list} para ver los nombres de las fuentes del
sistema. La incantación \code{fc-list: family style file spacing}
(todos los argumentos son cadenas literales) muestra alguna
información.

\subsubsection{\ConTeXt\ Mark \acro{IV}}
\label{sec:context-mkiv}

Ambos el `antiguo' \ConTeXt{} (Mark \acro{II}) y el `nuevo' \ConTeXt{}
(Mark \acro{IV}) deben de operar sin problemas después de la
instalación de \TL{}, y no requieren atención especial, mientras que
uno utilice \verb+tlmgr+ para las actualizaciones. 

Sin embargo, debido a que \ConTeXt{} Mk\acro{IV} no hace uso
de la librería kpathsea, alguna que otra configuración será
requerida, como por ejemplo cuando se instalan nuevos
archivos manualmente (sin el uso de
\verb+tlmgr+). Después de tal instalación, cada usuario de Mk\acro{IV}
debe ejecutar:
\begin{sverbatim}
context --generate
\end{sverbatim}
para actualizar la memoria del cache del disco.  Los
archivos resultantes son por consiguiente guardados bajo
\code{TEXMFCACHE} cuyo valor estándar en \TL\ es
\verb+TEXMSYSVAR;TEXMFVAR+.

\ConTeXt\ Mk\acro{IV} leerá todas las rutas mencionadas en
\verb+TEXMFCACHE+, y escribirá a la primera ruta que sea grabable.
Mientras esto se lee, la última ruta que corresponda, tomará precedencia,
en el caso de memoria duplicada del cache de datos. 

Para más información, vea 
\url{https://wiki.contextgarden.net/Running_Mark_IV}.

\subsubsection{Integrando macros locales y personales}
\label{sec:local-personal-macros}

Esto ya está implícitamente mencionado en la
sección~\ref{sec:texmftrees}: 
\dirname{TEXMFLOCAL} (por estándar,
\dirname{/usr/local/texlive/texmf-local} o en
\verb|%SystemDrive%\texlive\texmf-local| en Windows)
está dirigido hacia las fuentes locales y los macros a nivel global
del sistema; y \dirname{TEXMFHOME} (estándar, \dirname{$HOME/texmf} o
\verb|%USERPROFILE%\texmf|), es para las fuentes personales y macros.

Estos directorios están diseñados a permanecer de una versión a otra,
mientras que su contenido sea visto automáticamente por cada nueva
versión de \TL{}. De esta manera, lo mejor es no cambiar la definición
de \dirname{TEXMFLOCAL} que difiera mucho del principal directorio de
\TL{}, o de lo contrario necesitaría cambiar manualmente las versiones
futuras. 


Para ambos árboles, los archivos deben ponerse en los subdirectorios
de la apropiada Estructura del Directorio de \TeX\ (\TDS); vea
\url{https://tug.org/tds} o consulte
\filename{texmf-dist/web2c/texmf.cnf}. Por ejemplo, un archivo de una
clase de \LaTeX{}
o 
el paquete debe ser puesto en \dirname{TEXMFLOCAL/tex/latex} o
\dirname{TEXMFHOME/tex/latex}, o en un subdirectorio de este.

\dirname{TEXMFLOCAL} requiere una base de datos actualizada, o los
archivos no se encontrarán. Puedes actualizarla con el comando
\cmdname{mktexlsr} o usa el botón `Reinit file database' en el
tabulador de la configuración del \GUI\ \TeX\ Live Manager. 

Por norma o estándar, como se muestra, cada una de estas variables está
definida en un sólo directorio. Esto no es
un requisito que es forzado y tomado a la ligera. Si necesitas alternar entre
diferentes versiones o paquetes grandes, puedes por ejemplo, 
mantener múltiples árboles para tu propio
objetivo. Esto se logra, configurando \dirname{TEXMFHOME} a la lista de directorios, 
encerrados entre llaves, y separados por comas.

\begin{verbatim}
	TEXMFHOME = {/my/dir1,/mydir2,/a/third/dir}
\end{verbatim}

Sección~\ref{sec:brace-expansion} describe aún más la expansión de llaves. 

\subsubsection{Integrando fuentes de tercera persona}

Desafortunadamente, esto es un tópico un poco complicado, tanto para \TeX\ como para pdf\TeX{}. Es mejor no
tomarlo en cuenta, a menos que desees profundizar en todos los
detalles acerca de la instalación de \TeX{}. Muchas fuentes ya están
incluidas en \TL, así que échale un vistazo si deseas, para asegurarte que la fuente no se encuentre ahí; la página en el web de \url{https://tug.org/FontCatalogue} muestran casi todas las fuentes de texto que estan incluidas en las principales distribuciones de \TeX\, categorizadas de varias maneras.
 
Si necesitas instalar tus propias fuentes, vea
\url{https://tug.org/fonts/fontinstall.html} que con nuestro mejor esfuerzo tratamos en describir el procedimiento.
  
También puedes considerar en utilizar \XeTeX\ o Lua\TeX\ (vea detalladamente la sección~\ref{sec:tex-extensions}), el cual te permite el uso de fuentes del sistema sin instalación alguna de \TeX\. Pero que sirva de advertencia/precaución, que al usar las fuentes del sistema, tus documentos serán instantáneamente inutilizables para todos aquellos que esten en diferentes entornos de sistemas. 
   
\subsubsection{Evaluando la instalación}
\label{sec:test-install}

Después de instalar \TL{}, naturalmente quieres evaluarlo, y así
comenzar a crear bellos documentos y\slash o fuentes. 

Una cosa que inmediatamente puedes estar buscando es un programa que te
permita editar los archivos. \TL{} instala \TeX{}Works
(\url{https://tug.org/texworks}) en Windows, y Mac\TeX\ instala TeXShop
(\url{https://pages.uoregon.edu/koch/texshop}). En otros sistemas de
Unix, es tu decisión escoger un editor. Por supuesto que hay muchas
opciones disponibles, algunas de las cuales aparecen en la próxima
sección; vea también \url{https://tug.org/interest.html#editors}. Pero
en sentido general, cualquier editor de texto funcionará.

El resto de esta sección te ofrece algunos procedimientos básicos para
evaluar que el nuevo sistema esté operando correctamente. En este
manual, ofrecemos comandos basados en Unix; bajo \MacOSX{} y Windows, estás
más propenso en ejecutar las pruebas a través de una interfaz gráfica,
pero el proceso es en sí el mismo. 

\begin{enumerate}
	\item Asegúrate de ejecutar el programa \cmdname{tex} antes
		que todo:
	\begin{alltt}
		> \Ucom{tex -{}-version}
		TeX 3.14159265 (TeX Live ...)
		Copyright ... D.E. Knuth.
		...
	\end{alltt}
Si esto resulta en `command not found', en vez de aparecer la versión y la
información de derechos de copia, o si te aparece una versión
anterior, lo más probable es que no tienes el correcto subdirectorio
\dirname{bin} en la ruta \envname{PATH}. Vea la información acerca de
la configuración del programa en \p.\pageref{sec:env}. 

\item Procesar un archivo básico de \LaTeX{}, generando PDF:
\begin{alltt}
	> \Ucom{latex sample2e.tex}
	This is pdfTeX 3.14 ...
	...
	Output written on sample2e.pdf (3 pages, 142120 bytes).
	Transcript written on sample2e.log
\end{alltt}

Si lo anterior, falla en encontrar \filename{sample2e.tex} u otro
archivo, lo más probable en este caso es una interferencia con
las variables o con los archivos de configuración de una
versión anterior; recomendamos primero antes que todo, remover
cualquier variable que esté relacionada con \TeX.  (Para un
análisis más profundo, puedes preguntarle a \TeX{} que reporte
exactamente lo que está buscando, y encontrando; vea
``Acciones de depuración'' en la página~\ref{sec:debugging}.)

\item Vista previa del resultado:
\begin{alltt}
> \Ucom{xpdf sample2e.pdf}    
\end{alltt}
Debes ver una ventana nueva con un documento que explica alguna
información básica acerca de \LaTeX{}. (Vale la pena leerlo, por
cierto, si eres un principiante con \TeX.)

Por supuesto que hay muchos previsualizadore; en los sistemas Unix; \cmdname{evince} y \cmdname{okular} son usados frecuentemente. Para Windws recomendamos Sumatra PDF (\url{https://www.sumatrapdfreader.org/free-pdf-reader.html}). No hay previsualizadores de PDF incluidos en \TL(), por ese motiv debes instalar por separado, lo que quisieras usar.

\item Of course you can still generate \TeX's original \dvi{} format:
Por supuesto que aún puedes generar el formato \dvi() original de \TeX\
\begin{alltt}
> \Ucom{latex sample2e.tex}
\end{alltt}

\item And preview the \dvi{} online:
\begin{alltt}
> \Ucom{xdvi sample2e.dvi}    # Unix
> \Ucom{dviout sample2e.dvi}  # Windows
\end{alltt}

Tienes que estar bajo el
sistema de X para que el \cmdname{xdvi} funcione; si no lo estás, o 
si la variable del sistema \envname{DISPLAY} está configurada
incorrectamente, obtendrá el error \samp{Can't open display}.

\item Crea un archivo \PS{} para imprimir o mostrar en la pantalla:
\begin{alltt}
> \Ucom{dvips sample2e.dvi -o sample2e.ps}
\end{alltt}

\item O para crear un PDF del archivo \dvi{}, una ruta alternativa antes el uso de pdf\TeX\ (o Xe\TeX\ o Lua\TeX), el cual es útil en algunas ocasiones:

\begin{alltt}
> \Ucom{dvipdfmx sample2e.dvi -o sample2e.pdf}
\end{alltt}
  
\item Archivos estándar de evaluación que puede considerar útiles
	además de \filename{sample2e.tex}:
\begin{ttdescription}
	\item [small2e.tex] Un documento más simple que
		\filename{sample2e.tex}, para reducir el tamaño del
		archivo si tienes problemas. 
\item [testpage.tex] Evalúa si tu imprenta tiene muchos valores fuera
	del margen. 
\item [nfssfont.tex] Para imprimir tablas con fuentes y pruebas. 
\item [testfont.tex] También para cuadros de fuentes, pero usando esta
	vez plain \TeX{}.
\item [story.tex] El más canónico archivo de prueba para plain \TeX{}, o \TeX{} básico. 
Tienes que ingresar \samp{\bs bye} en la línea \code{*} del intérprete de la terminal, después de \samp{texstory.tex}
\end{ttdescription}

\item Si instalaste el paquete \filename{xetex}, puedes evaluar el
	acceso a las fuentes del sistema de la siguiente manera:
\begin{alltt}
> \Ucom{xetex opentype-info.tex}
This is XeTeX, Version 3.14\ldots
...
Output written on opentype-info.pdf (1 page).
Transcript written on opentype-info.log.
\end{alltt}

Si te aparece un mensaje de error que dice ``Invalid fontname `Latin
Modern Roman/ICU'\ldots'', entonces necesitas configurar tu sistema,
de tal manera que las fuentes que están incluidas en \TL\ puedan ser
encontradas. Vea 
Sección~\ref{sec:font-conf-sys}.

\end{enumerate}

\subsection{Enlaces para programas adicionales de descarga}

Si eres principiante con \TeX{}, o necesitas ayuda preparando
documentos en \TeX{} o \LaTeX{}, por favor visita
\url{https://tug.org/begin.html} para algunos recursos introductorios.

Enlaces para algunas herramientas que puedes considerar en instalar:
\begin{description}
\item[Ghostscript] \url{https://www.cs.wisc.edu/~ghost/}
\item[Perl] \url{https://www.perl.org/} con
      paquetes suplementarios de \acro{CPAN}, \url{https://www.cpan.org/}
\item[ImageMagick] \url{https://www.imagemagick.com}, para
	procesamiento y conversión de gráficas. 
\item[NetPBM] \url{https://netpbm.sourceforge.net/}, también para
	gráficas.

\item[Editores de texto orientados para \TeX] Hay una amplia selección, y es cuestión
	de gusto. Aquí hay una selección en orden alfabético (unos
	cuantos que aparecen aquí son para Windows solamente).
  \begin{itemize*}
  \item \cmdname{GNU Emacs} está disponible nativamente para Windows,
	  vea
        \url{https://www.gnu.org/software/emacs/emacs.html}.
  \item \cmdname{Emacs con Auc\TeX} para Windows está disponible de \CTAN.
        La página de AuC\TeX\ está en \url{https://www.gnu.org/software/auctex}.
  \item \cmdname{SciTE} está disponible de
        \url{https://www.scintilla.org/SciTE.html}.
  \item \cmdname{Texmaker} es software libre, disponible de 
        \url{https://www.xm1math.net/texmaker}.
  \item \cmdname{TeXstudio} se bifurcó de \cmdname{Texmaker} con
        características adicionales; disponible en \url{https://texstudio.org/} y en la distribución de pro\TeX{}t.  
  \item \cmdname{TeXnicCenter} es software libre, disponible en
        \url{https://www.texniccenter.org}.
  \item \cmdname{TeXworks} es software libre, disponible de 
        \url{https://tug.org/texworks} e instalado para Windows y
        \MacOSX\ como parte de \TL.
  \item \cmdname{Vim} es software libre, disponible de
        \url{https://www.vim.org}.
  \item \cmdname{WinEdt} es programa shareware a través de 
        \url{https://tug.org/winedt} or \url{https://www.winedt.com}.
  \item \cmdname{WinShell} está disponible en \url{https://www.winshell.de}.
  \end{itemize*}
\end{description}
Para una lista más extensa de paquetes y programas, vea
\url{https://tug.org/interest.html}

\section{Instalaciones especializadas}

Las secciones previas describen el proceso básico de instalación. Aquí
veremos algunos casos que son especializados. 

\htmlanchor{tlsharedinstall}
\subsection{Instalación con varios usuarios (o múltiples ordenadores)}
\label{sec:sharedinstall}

\TL{} ha sido diseñado para ser compartido entre varios
usuarios en un sistema, y/o entre diferentes sistemas en una
red. Con una estándar organización del directorio, ninguna
ruta forzada es configurada: la localización de los archivos
que los programas de \TL{} necesitan, pueden ser encontrados
relativos con estos programas. Puedes ver esto en el archivo
principal de configuración:
\filename{$TEXMFDIST/web2c/texmf.cnf}, que contiene líneas tales como:
\begin{sverbatim}
TEXMFROOT = $SELFAUTOPARENT
...
TEXMFDIST = $TEXMFROOT/texmf-dist
...
TEXMFLOCAL = $SELFAUTOPARENT/../texmf-local
\end{sverbatim}
Esto quiere decir, que añadiendo un directorio para los ejecutables de
\TL{} en la ruta de acceso de estos, es más que suficiente para
tener una configuración operable. 

De la misma manera, puedes instalar \TL{} localmente, y
luego mover, traspasar toda la jerarquía, a una localización
en la red. 

Para Windows, \TL{} incluye un programa lanzador \filename{tlaunch} para la instalación.

La ventana principal de este, contiene diferentes funciones en el menú y
botones para varios programas relacionados con \TeX\ y también con
documentación. Todo esto se puede personificar a través de un fichero \code{ini}.

Durante el inicio del program, este lanzador modifica la ruta de acceso para
\TL\ y crea algunas asociaciones de los ficheros/archivos. Es decir, replica lo
que usualmente transcurre después de la post-instalación pero solamente para el
usuario actual del sistema. De tal manera, aquellas computadoras u ordenadores
conectadas en la misma red local, pueden acceder \TL{} a través de enlaces
rápidos de este lanzador \filename{tlaunch}. 

Vea el manual completo de \code{tlaunch} (\code{texdoc tlaunch}, o visitando \url{https://ctan.org/pkg/tlaunch}).

\htmlanchor{tlportable}
\subsection{Instalaciones portables (\USB{})}
\label{sec:portable-tl}

La opción \code{-portable} del instalador (o el comando
\code{V} en el instalador de texto, o la correspondiente
opción en el \GUI{}) crea una instalación completamente
independiente bajo una raíz común, y renuncia a cualquier
integración con el sistema. Puedes crear una instalación
directamente en un lector de \USB{}, o copiarlo después a la
memoria del \USB{}.

Técnicamente, la instalación portátil es hecha independiente, al configurar los valores estándar de \envname{TEXMFHOME}, \envname{TEXMFVAR}, y
\envname{TEXMFCONFIG} que sean los mismos de \envname{TEXMFLOCAL},
\envname{TEXMFSYSVAR}, y \envname{TEXMFSYSCONFIG} respectivamente; de esta manera, cualquier configuración predefinida del usuario, al igual que el caché, no serán creados. 
 
Para ejecutar \TeX\ con esta instalación, necesitas añadir el
apropiado directorio del binario, a la ruta de acceso, durante la sesión
de la terminal, como es usual.

En Windows, puedes darle dos veces a 
\filename{tl-tray-menu} en la raíz de la instalación para escoger
entre varias opciones, como se muestra en esta captura de pantalla:

\medskip
\tlpng{tray-menu}{4cm}
\smallskip

\noindent La función de `More\ldots' explica como usted puede
personalizar este menú. 

%\htmlanchor{tlisoinstall}
%\subsection{Instalaciones con \ISO\ (o \DVD) }
%\label{sec:isoinstall}
%
%Si no necesitas actualizar o modificar tu sistema periódicamente,
%y\slash o tienes varios sistemas para instalar \TL{}, puedes
%encontrar conveniente en crear una imagen \ISO\ de tu instalación de
%\TL{}, porque:
%
%\begin{itemize}
%	\item Copiando un \ISO\ entre diferentes computadoras es mucho más
%		rápido que copiando una instalación estándar. 
%	\item Si estás inicializando tu sistema, con dos sistemas
%		operativos, y quieres que ambos compartan una
%		instalación \TL{}, la instalación del \ISO\ no está
%		expuesta a las idiosincrasias y limitaciones de
%		otros sistemas de archivos (\acro{FAT32}, \acro{NTFS},
%		\acro{HFS+}).
%	\item Sistemas virtuales pueden simplemente montar un \ISO.
%\end{itemize}
%
%Por supuesto, también puedes grabar la imagen \ISO\ a un \DVD, si te
%es útil. 
%
%Sistemas operativos \GNU/Linux/Unix, incluyendo \MacOSX, pueden montar
%un \ISO. Windows 8 es la primera (!) versión Windows que puede hacer
%esto. Excepto esto, nada cambia en respecto a una instalación normal a
%un disco duro, vea la sección \ref{sec:env}.
%
%Cuando prepares tal instalación \ISO, lo mejor es omitir el
%subdirectorio para el año del lanzamiento, y que tengas
%\filename{texmf-local}, en el mismo nivel que los otros árboles,
%(\filename{texmf-var}, etc.). Puedes hacer esto con las opciones
%normales del directorio en el instalador.
%
%Para un sistema físico de Windows (en vez de virtual), puedes grabar
%la imagen \ISO\ a \acro{DVD}. Sin embargo, es mejor que investigues
%las opciones libres de montajes\Dash\ISO, tales como WinCDEmu en 
%\url{http://wincdemu.sysprogs.org/}.
%
%Para integración con el sistema Windows, puedes incluir los programas
%\filename{w32client} descrito en la sección~\ref{sec:sharedinstall} y
%en \url{http://tug.org/texlive/w32client.html}, que funciona de igual
%manera como un \ISO\ para una instalación en red. 
%
%En \MacOSX, TeXShop podrá hacer uso de la instalación con \acro{DVD}, 
%si un enlace simbólico \filename{/usr/texbin} apunta hacia el
%apropiado binario del directorio, e.g., 
%\begin{verbatim}
%sudo ln -s /Volumes/MyTeXLive/bin/universal-darwin /usr/texbin
%\end{verbatim}
%
%Nota histórica: \TL{} 2010 fue la primera edición de \TL{}, la cual no
%fue más distribuida `live', o en `vivo'. Sin embargo, siempre requirió algunas
%acrobacias para ejecutar el \DVD\ o el \ISO; en particular, no había
%manera de configurar por lo menos una variable extra en el sistema. Si
%creas tu \ISO\ de una existente instalación, cabe señalar, que no hay necesidad de esto.

\htmlanchor{tlmgr}
\section{\cmdname{tlmgr}: Administrando tu instalación}
\label{sec:tlmgr}

\begin{figure}[tb]
\tlpng{tlshell-macos}{\linewidth}
\caption{\prog{tlshell} \GUI, mostrando las acciones del
	Menú (\MacOSX)}
\label{fig:tlshell}
\end{figure}

\begin{figure}[tb]
\tlpng{tlcockpit-packages}{.8\linewidth}
	\caption{\prog{tlcockpit} Interfaz Gráfica del Usuario (\GUI{}) para \prog{tlmgr}}
\label{fig:tlcockpit}
\end{figure}

\begin{figure}[tb]
\tlpng{tlmgr-gui}{\linewidth}
\caption{\prog{tlmgr} en modo \GUI: página principal, después de
`cargarse'.}
\label{fig:tlmgr-gui}
\end{figure}

\TL{} incluye un programa nombrado \prog{tlmgr} para administrar \TL{}
después de la instalación inicial. Sus capacidades incluye:

\begin{itemize*}
\item instalando, actualizando, archivando, restaurando, y
	desinstalando paquetes individuales, opcionalmente tomando
	dependencias en consideración;
\item buscando y listando paquetes y sus descripciones;
\item listando, añadiendo, y removiendo plataformas;
\item cambiando las opciones de instalación, tales como el tamaño del
	papel y el origen de la localización (vea la
	sección~\ref{sec:location}).
\end{itemize*}

La funcionalidad de \prog{tlmgr} abarca el programa \prog{texconfig} en su
totalidad. Por lo tanto distribuiremos y mantendremos \prog{texconfig} para
aquellos que están acostumbrados a su interfaz, pero hoy en día recomendamos
usar \prog{tlmgr}. 

\subsection{\cmdname{tlmgr} Modo GUI}

\TL{} contiene varias Interfaces Gráficas de Usuario \GUI para el
programa \prog{tlmgr}. Dos notables de estas interfaces son:
Figura~\ref{fig:tlshell} muestra \cmdname{tlshell}, el cual es escrito
en Tcl/Tk y trabaja inmediatamente tanto en Windows como en \MacOSX.
Figura~\ref{fig:tlcockpit} muestra \prog{tlcockpit}, el cual requiere
Java con una versión por lo menos 8, o más alta, y también JavaFX. Ambos
son programas separados.

\prog{tlmgr} tiene también un modo nativo \GUI{}
(como se muestra en figura~\ref{fig:tlmgr-gui}), que es iniciado con:
\begin{alltt}
> \Ucom{tlmgr -gui}
\end{alltt}

Esta extensión de la Interfaz Gráfica \GUI\ requiere de
Perl/Tk, un módulo el cual ya no está incluido en la
distribución \TL\ basada en Perl para Windows.

\subsection{Ejemplos de invocaciones en la línea de comando de
\cmdname{tlmgr}}. 

Después de la instalación inicial, puedes actualizar el sistema a las
últimas versiones disponibles con: 
\begin{alltt}
> \Ucom{tlmgr update -all}
\end{alltt}
Si esto te pone nervioso, primero trata
\begin{alltt}
> \Ucom{tlmgr update -all -dry-run}
\end{alltt}
o (menos verboso):
\begin{alltt}
> \Ucom{tlmgr update -list}
\end{alltt}

Este ejemplo más complejo, añade una colección, para el motor \XeTeX,
desde un directorio local:

\begin{alltt}
> \Ucom{tlmgr -repository /local/mirror/tlnet install collection-xetex}
\end{alltt}
Genera el siguiente resultado (abreviado):
\begin{fverbatim}
install: collection-xetex
install: arabxetex
...
install: xetex
install: xetexconfig
install: xetex.i386-linux
running post install action for xetex
install: xetex-def
...
running mktexlsr
mktexlsr: Updating /usr/local/texlive/2021/texmf-dist/ls-R...
...
running fmtutil-sys --missing
...
Transcript written on xelatex.log.
fmtutil: /usr/local/texlive/2021/texmf-var/web2c/xetex/xelatex.fmt installed.
\end{fverbatim}

Como puedes ver, \prog{tlmgr} instala dependencias, y se ocupa de
cualquier acción necesaria post-instalación, incluyendo la
actualización de la base de datos de los nombres de los archivos, y
(re)generando formatos. Arriba, generamos nuevos formatos para \XeTeX.

Para describir un paquete (o colección o esquema):
\begin{alltt}
> \Ucom{tlmgr show collection-latexextra}
\end{alltt}
el cual produce resultados de la siguiente manera:
\begin{fverbatim}
package:    collection-latexextra
category:   Collection
shortdesc:  LaTeX supplementary packages
longdesc:   A very large collection of add-on packages for LaTeX.
installed:  Yes
revision:   46963
sizes:      657941k
\end{fverbatim}

Por último, y muy importante, la documentación completa se puede ver en
\url{https://tug.org/texlive/tlmgr.html}, o:
\begin{alltt}
> \Ucom{tlmgr -help}
\end{alltt}

\section{Notas sobre Windows}
\label{sec:windows}

\subsection{Características específicas de Windows}
\label{sec:winfeatures}

Bajo Windows, el instalador hace varias cosas extra:
\begin{description}
\item[Menús y acceso rápido.] Un nuevo submenú del menú de Start es
	instalado, el cual contiene funciones para algunas programas
	del \GUI{} (\prog{tlmgr}, \prog{texdoctk}, el PS\_View
	(\prog{psv}) pre-visualizador de PostScript) y alguna
	documentación.
\item [Associación de archivos.] Si está activado, le permitirán a
	programas como \prog{TeXworks}, \prog{Dviout}, y
	en convertirse en el programa estándar para sus
	respectivos tipos de archivos, o tener una función de doble
	clic en el menú de `Open with', para esos tipos de archivos.
\item[Conversor de bitmap a eps.] Varios formatos de bitmap tienen una
	función \cmdname{bitmap2eps} en sus `Open with' del menú accedido
	mediante el doble clic. Bitmap2eps es un programa simple el
	cual permite que \cmdname{sam2p} o \cmdname{bmeps} hagan el
	trabajo. 
\item[Ajuste automático de acceso a la ruta.] No requiere
	procedimientos manuales de configuración.
\item[Desinstalador.] El instalador crea una función debajo de
	`Añadir/Remover Programas' (`Add/Remove
		Programs') para \TL. El tabulador `uninstall' del \GUI\ de
	\TeX\ Live Manager, se encarga de esto. Para la
	instalación con un usuario solamente, el instalador
	también crea un ingreso de desinstalación, que se
	encuentra bajo el menú de Start o Comienzo. 
\item[Write-protect] Para una instalación con privilegios
	administrativos, los directorios de \TL\ son
	write-protected, o protegidos-contra-escritura,
	siempre y cuando \TL\ haya sido instalado en un disco
	que no sea removible y formateado bajo el sistema de
	partición de NTFS. 
\end{description}
	
\subsection{Software adicional incluido en Windows}
Para finalizar, una instalación de \TL{} necesita paquetes de ayuda, los cuales no
son comúnmente encontrados en ordenador de Windows. Por lo tanto \TL{} provee las
piezas perdidas:

\begin{description} 
\item[Perl y Ghostscript] Por la importancia de Perl
			y Ghostscript, \TL{} incluye copias
			`escondidas', de estos programas. Hay programas
			de \TL{} que necesitan saber donde encontrarlos,
			pero no engañan su presencia, a través de
			variables del sistema, o las preferencias del
			registro. En sí, no son instalaciones a mayor
			escala, y no deben interferir con instalaciones
			de Perl o Ghostscript. 

\item[dviout.] También está instalado \prog{dviout}, un visualizador de
	\acro{DVI}. Al comienzo, cuando pre-visualices archivos con
	\cmdname{dviout}, creará fuentes, porque las fuentes del
	sistema no fueron instaladas. Después de esto, habrás
	creado la mayoría de las fuentes que uses, y raramente
	verás la ventana de creación de fuentes nuevamente. Para
	más información acerca de esto (recomendable) puede
	encontrarse en la ayuda en línea. 

\item[\TeX{}works.] \TeX{}works es un editor de texto, diseñado para
	\TeX{}, y con un visualizador integrado de \acro{PDF}.

\item[Herramientas de línea de Comando.] Un número de puertos en
	Windows, de programas basados en el intérprete de la línea de
	comando de Unix, son instalados también con los binarios
	de \TL{}. Estos incluye \cmdname{gzip}, \cmdname{unzip},
	y las utilidades de del conjunto de \cmdname{xpdf}
	(\cmdname{pdfinfo}, \cmdname{pdffonts}, \ldots).  El
	visualizador \cmdname{xpdf} no está disponible para
	Windows.  Una alternativa es el visualizador Sumatra
	PDF, disponible de \url{https://sumatrapdfreader.org/} 	

\item[fc-list, fc-cache, \ldots] Las herramientas de la librería de
	fontconfig, le permite a \XeTeX{} a manipular fuentes del
	sistema en Windows. Usted puede usar \prog{fc-list} para
	determinar los nombres de las fuentes que pasan con el
	extendido comando \cs{font}. Y si es necesario, ejecute		\prog{fc-cache} primero para actualizar la información
	de la fuente.
\end{description}

\subsection{El Perfil del Usuario es Casa}
\label{sec:winhome}

Las partes homólogas en Windows, del directorio de la casa en Unix, es el
directorio \verb|%USERPROFILE|. Bajo Windows \acro{XP}, esto es usualmente
\verb|C:\Documents and Settings\<username>|, y bajo Windows Vista y
versiones más recientes, es \verb|C:\Users\<username>|. En el archivo
\filename{texmf.cnf}, y \KPS{} en general, \verb|~| expandirá
apropiadamente en ambos Windows y Unix. 

\subsection{El registro de Windows}
\label{sec:registry}

Windows almacena casi todos los datos de configuración en su registro. El
registro contiene un set de claves organizadas en jerarquía, con claves de
la raíz del sistema. Las más importantes para los programas de
instalación, son \path{HKEY_CURRENT_USER} y \path{HKEY_LOCAL_MACHINE}, que
abreviando serían la \path{HKCU} y \path{HKLM}. La parte \path{HKCU} del
registro, está en el directorio de la casa del usuario (vea la
sección~\ref{sec:winhome}). \path{HKLM} está normalmente en un
subdirectorio del directorio de Windows.

En algunos casos, la información puede ser obtenida a través de variables
del sistema, pero otras informaciones, como la localización de accesos
directos, es necesario consultar el registro. También la fijación
permanente de variables del sistema, requieren acceso al registro.

\subsection{Permisos en Windows}
\label{sec:winpermissions}

En versiones más recientes de Windows, una distinción es hecha, entre
usuarios regulares y administradores, y son estos últimos los que tienen
acceso sin permisos, al sistema operativo. Por lo tanto, hemos hecho un
esfuerzo, en hacer la instalación de \TL{} permisible, sin privilegios
administrativos. 

Si el instalador se comienza, con permisos administrativos, hay una
opción que permite seleccionar, que esto sea asequible para todos los
usuarios. Si esta opción es seleccionada, enlaces de acceso directo,
serán creados para todos los usuarios, y el entorno del sistema será
modificado. De otra manera, los enlaces de acceso directo, al igual
que las funciones del menú, serán creadas para el usuario actual, y el
entorno del sistema para el usuario, es por lo tanto modificado. 

Cualquiera que sea el estatus del administrador, la raíz estándar de
\TL{} que es propuesta por el instalador, estará siempre bajo
\verb|%SystemDrive%|. El instalador siempre evalúa si la raíz tiene
permisos de escritura para el usuario actual.

Puede resultar un problema, si un usuario no es un administrador y \TeX{}
ya existe en la ruta de búsqueda. Debido a que la ruta efectiva, consiste
de la ruta del sistema, seguida por la ruta del usuario, el nuevo \TL{}
nunca tendrá precedencia. De manera retroactiva o de retroceso, el instalador crea un
enlace de acceso directo a la línea de comando, en el cual, el nuevo
binario de \TL{}, es antepuesto a la ruta de búsqueda local. El nuevo
\TL{} siempre se podrá usar, dentro de tal intérprete de la línea de
comando. El enlace de acceso directo a \TeX{}works, si está instalado,
siempre antepondrá \TL{} a la ruta de búsqueda, por lo tanto, debe ser
también inmune a este problema de la ruta. 

Debe saber, que aunque haya accedido al sistema, como administrador, no
obstante necesita, en obtener privilegios administrativos. En realidad, no es
necesario en acceder al sistema como administrador. En vez de ello, con un clic
a mano derecha, del programa, o en el enlace del programa que se desea
ejecutar, usualmente te ofrece una opción `Ejecutar como administrador' o `Run as administrator'

\subsection{Incrementando la memoria máxima en Windows y Cygwin}
\label{sec:cygwin-maxmem}

Usuarios de Windows y Cygwin (vea la sección~\ref{sec:cygwin} para los
métodos específicos de instalación en Cygwin) pueden inesperadamente
verse en la situación que han agotado la memoria del sistema, cuando
estén ejecutando algunos de los programas que están en \TL{}.  Por
ejemplo, \prog{asy} puede agotar la memoria, si usted asigna un array
de 25,000 reales, y Lua\TeX\ puede agotar la memoria, si se trata de
procesar un documento con muchas fuentes grandes. 

Para Cygwin, puede incrementar la cantidad de memoria disponible, a través
de las instrucciones de la Guía del Usuario de Cygwin.  
(\url{https://www.cygwin.com/cygwin-ug-net/setup-maxmem.html}).

En Windows, tiene que crear un archivo, digamos \code{moremem.reg}, con
estas líneas:

\begin{sverbatim}
Windows Registry Editor Version 5.00

[HKEY_LOCAL_MACHINE\Software\Cygwin]
"heap_chunk_in_mb"=dword:ffffff00   
\end{sverbatim}

\noindent y después ejecute el comando \code{regedit /s moremem.reg} como
administrador. (Si desea cambiar la memoria solamente para el usuario
actual, en vez de sistema global, utilice \code{HKEY\_CURRENT\_USER}.)

\section{Una guía de usuario para Web2C}

\Webc{} es una colección integrada de programas relacionados con \TeX{}:
\TeX{} como tal, \MP{}, \BibTeX{}, etc. Es el corazón de \TL{}. La
página en el Internet, para \Webc{}, junto con el actual manual y más es
\url{https://tug.org/web2c}.

Un poco de historia: La implementación original fue por Tomas Rokicki,
quien en 1987, desarrolló el primer sistema \TeX{}-to-C basado en archivos
que se cambiaron bajo Unix, los cuales fueron primariamente el trabajo
original de Howard Trickey y Pavel Curtis. Tim Morgan se convirtió en el
mantenedor del sistema, y durante este período, el nombre como tal cambió
a Web-to-C\@. En 1990, Karl Berry se hizo responsable del trabajo, ayudado
por docenas de contribuidores adicionales, y en 1997 le dio el batón de
relevo a Olaf Weber, quien se lo devolvió a Karl, en el 2006.

El sistema \Webc{} opera en Unix, versiones de 32-bits en Windows,
\MacOSX{}, y otros sistemas operativos. Utiliza el código original de
Knuth, para \TeX{} y otros programas básicos escritos en el sistema
literal \web{} y los traduce, a código C. Los principales programas de
\TeX{} que operan así, son los siguientes:

\begin{cmddescription}
\item[bibtex, biber]   Respaldo bibliográfico.  
\item[dvicopy]   Expande las referencias virtuales de fuentes, a archivos
	\dvi{}.
\item[dvitomp]   \dvi{} a MPX (MetaPost pictures).
\item[dvitype]   \dvi{} a texto leíble por humanos.
\item[gftodvi]   Conversor de fuentes genéricas.
\item[gftopk]    Conversor de fuentes \code{.pk}.
\item[gftype]    Archivo de \code{.gf} a texto leíble por humanos.
\item[mf]        Creando familias de fuentes.
\item[mft]       Imprimiendo código de \MF{}.
\item[mpost]     Creación de diagramas técnicos. 
\item[patgen]    Creación de patrones de separaciones silábicas.
\item[pktogf]    Conversor de fuente de formato \code{.pk} a fuentes genéricas
\item[pktype]    Conversor de fuente de formato \code{.pk} a texto leíble por
humanos. 
\item[pltotf]    Lista de texto regular a TFM. 
\item[pooltype]  Archivos pool \web{} de visualización. 
\item[tangle]    Programación \web{} a Pascal.
\item[tex]       Tipografía.
\item[tftopl]    Conversor de TFM a lista de propiedad de texto regular.
\item[vftovp]    Conversor de fuente virtual a lista de propiedad virtual. 
\item[vptovf]    Conversor de lista de propiedad virtual a fuente virtual.
\item[weave]     \web{} 
\end{cmddescription}

\noindent Las funciones precisas y sintaxis de estos
programas están descritas en la documentación de los
paquetes individuales y en el propio \Webc{}. Sin embargo,
el conocer los principios básicos que gobiernan toda la
familia de programas, le ayudará con la instalación de
\Webc{}. 

Todos los programas honoran estas opciones estándar de \GNU{}:
\begin{ttdescription}
	\item[-{}-help] imprimir sumario de uso básico. 
	\item[-{}-version] imprimir información de la versión, y
		salida.
\end{ttdescription}

Y  tienen que honorar también:
\begin{ttdescription}
\item[-{}-verbose] imprimir reporte detallado del progreso
\end{ttdescription}

Para localizar los archivos, los programa de \Webc{} usan la
librería de ruta de búsqueda \KPS{}
(\url{https://tug.org/kpathsea}). Esta librería utiliza una
combinación de variables de entorno del sistema, y unos
archivos de configuración para optimizar la búsqueda de la inmensa
colección de archivos de \TeX{}. \Webc{} puede escanear varios
árboles de directorios simultáneamente, lo cual tiene uso en
la mantención de la distribución estándar de \TeX{} y
extensiones personales y locales en los árboles. Para agilizar
las búsquedas de archivos, la raíz de cada árbol tiene un
archivo \file{ls-R}, que contiene un ingreso, que muestra el
nombre y la ruta, para todos estos archivos bajo la raíz. 

\subsection{Ruta de acceso Kpathsea}
\label{sec:kpathsea}

Vamos a describir el mecanismo de la ruta de búsqueda genérica
de la librería de \KPS{}.

Le llamamos una ruta de búsqueda a un guión- o semi guión\hyph de una
lista separada de \emph{elementos de ruta}, que son
básicamente nombres de directorios. Una ruta de búsqueda puede
provenir (de una combinación) de muchos orígenes. Para
encontrar un archivo \samp{mi-archivo} alrededor de una ruta
\samp{.:/dir}, \KPS{} revisa cada uno de los elementos de la
ruta en orden: primero \file{./mi-archivo}, luego
\file{/dir/mi-archivo}, que indica el primer archivo (o
archivos) que correspondan con el mismo. 

Para poder adaptar óptimamente las convenciones de todos los
sistemas operativos, en sistemas no-Unix \KPS{} puede usar
separadores de los nombres de archivos, que difieren de dos puntos
(\samp{:}) y barra oblicua (\samp{/}). 

Para revisar un elemento particular de ruta \var{p}, \KPS{}
primero revisa si una base de datos prefabricada (vea
``Base de datos de nombres de archivos'' Filename data\-base) en la
página~\ref{sec:filename-database} aplica a \var{p}, i.e., si
la base de datos se encuentra en un directorio que es un
prefijo de \var{p}. Si esto es así, la especificación de la
ruta tiene que corresponder, con los contenidos de la base de
datos. 

Si la base de datos no existe, o no aplica a este elemento de
ruta, o no corresponde con ningún archivo, el sistema de
archivos es escaneado (si esto no ha sido prohibido por alguna
especificación que comience con \samp{!!}\ y si el archivo que
está siendo buscado, ya existe). \KPS{} construye la
lista de directorios que corresponde con este elemento de la
ruta, y luego revisa en cada uno de estos elementos, por el
archivo que se está buscando. 

El ``archivo tiene que existir'' es una condición que viene a
relucir con los archivos \samp{.vf} y los archivos de ingreso
que son leídos por el comando \cs{openin} de \TeX{}. Estos
archivos quizás no existan (e.g., \file{cmr10.vf}), y por lo
lo tanto, sería un error, en escanear el disco para
encontrarlos. Por tal motivo, si usted comete el error en no
actualizar \file{ls-R} cuando instala un archivo nuevo
\samp{.vf}, este archivo nunca será encontrado. 

Aunque el elemento más simple y más común es el nombre de un
directorio, \KPS{} respalda características adicionales, en las
rutas de búsquedas: valores superpuestos estándar, nombres de
variables del sistema, valores de configuración del archivo,
directorios de usuarios, y búsqueda de subdirectorios
recursivos. Por lo tanto, decimos que \KPS{} \emph{expande} un
elemento de la ruta, que quiere decir que transforma todas las
especificaciones, en nombre o nombres básicos de directorios.
Esto es descrito en las siguientes secciones en el mismo
orden que esto se lleva a cabo. 

Note que si el nombre de archivo que se está buscando, es un
relativo absoluto, o implícitamente, i.e., comienza con
\samp{/} o \samp{./} o \samp{../}, \KPS{} simplemente revisa
si este archivo existe. 

\ifSingleColumn
\else
\begin{figure*}
\verbatiminput{examples/ex5.tex}
\setlength{\abovecaptionskip}{0pt}
\caption{Un ejemplo ilustrativo de archivo de configuración}
  \label{fig:config-sample}
\end{figure*}
\fi

\subsubsection{Orígenes de la ruta}
\label{sec:path-sources}

Una ruta de acceso, puede provenir de muchos orígenes.
\KPS{} las utiliza, en el siguiente orden:

\begin{enumerate}
	\item
		Una variable especificada por el usuario, por
		ejemplo, \envname{TEXINPUTS}\@. Variables de
		entorno del sistema con un punto y un nombre
		de programa impuesto, sobreponen; e.g., si
		\samp{latex} es el nombre del programa que
		está siendo ejecutado, entonces
		\envname{TEXINPUTS.latex} sobrepondrá
		\envname{TEXINPUTS}.
	\item
		Un archivo de configuración de un
		programa-específico, por ejem\-plo, una línea
		\samp{S /a:/b} en \file{config.ps} de
		\cmdname{dvips}. 
	\item 
		Un archivo de configuración \KPS{}
		\file{texmf.cnf}, conteniendo una línea
		\samp{TEXINPUTS=/c:/d} (vea más abajo).
	\item
		El tiempo de compilación estándar. 
\end{enumerate}
\noindent Puede ver cada uno de estos valores para alguna ruta
de búsqueda, usando opciones de depuración (vea ``Acciones de
depuración'' en la página~\pageref{sec:debugging}).

\subsubsection{Archivos Config}

\KPS{} lee \emph{archivos de configuración ejecutables} nombrado
\file{texmf.cnf} para la ruta de búsqueda y otras definiciones. La ruta
de búsqueda que se utiliza para buscar estos archivos, es nombrada
\envname{TEXMFCNF}, pero no recomendamos en especificar esta (o
cualquier otra) variable que sobrescriba los directorios del sistema. 

En vez de esto, una instalación normal resulta en un archivo
\file{.../2021/texmf.cnf}. Si tienes que hacer modificaciones o
cambios a las configuraciones estándares (algo normalmente
innecesario), este es el lugar para ponerlos. El principal archivo de
configuración está en el archivo
\file{.../2021/texmf-dist/web2c/texmf.cnf}.  Usted no debe editar este
último archivo, debido a que los cambios serán perdidos, tan pronto
como la versión de la distribución sea actualizada. 

Como una nota adjunta, si lo que se desea es añadir un directorio personal a una ruta de búsqueda, especificando entonces una variable del entorno del sistema es un método razonable:
\begin{verbatim}
  TEXINPUTS=.:/mi/macro/dir:
\end{verbatim}
Para que esta especificación pueda ser actualizada, mantenida y portable a través de los años, utilice un sobrante \samp{:} (y \samp{;} en Windows) para ingresar las rutas de búsqueda del sistema, en vez de escribirlo explícitamente (vea la sección~\ref{sec:default-expansion}). Otra opción sería utilizar el árbol de \envname{TEXMFHOME} (vea la sección~\ref{sec:directories}).



\emph{Todos} los archivos \file{texmf.cnf} en la ruta de
búsqueda serán leídos, y las definiciones en archivos más
recientes, sobrepondrán aquellas definiciones de los primeros archivos. Por
ejemplo, con una ruta de búsqueda de \verb|.:$TEXMF|, los
valores de \file{./texmf.cnf} sobrepondrán aquellos de
\verb|$TEXMF/texmf.cnf|.

\begin{itemize*}
	\item 
		Comentarios comienzan con ~\code{\%}, tanto al principio de la línea o precedida por espacio en blanco, y continúan así hasta el final de la línea.
		 
		\item
			Líneas en blanco, son ignoradas.
		\item
			Un \bs{} en el final de la línea,
			actúa como un carácter de
			continuación, i.e., la próxima línea
			es adjunta. Espacio en blanco al
			comienzo de líneas continuativas no es
			ignorado. 
		\item
			Cada línea restante, tiene la forma:\\
\hspace*{2em}\texttt{\var{variable} \textrm{[}.\var{progname}\textrm{]}
  \textrm{[}=\textrm{]} \var{value}}\\[1pt]
			donde el \samp{=} y espacio en blanco
			alrededor son opcionales. 
			(Pero si el \var{valor} comienza con \samp{.}, es más simple utilizar \samp{=}, para evitar que el signo de punto sea interpretado como el calificativo del nombre del programa.
		\item
			El nombre de la \ttvar{variable} puede
			contener cualquier carácter otro que
			no sea un espacio en blanco, \samp{=},
			o \samp{.}, pero lo más seguro es
			quedarse con \samp{A-Za-z\_}.
		\item
			Si \samp{.\var{progname}} está
			presente, la definición solamente
			aplica si el programa que está
			ejecutándose es nombrado
			\texttt{\var{progname}} o
			\texttt{\var{progname}.exe}. Esto 
			permite, por ejemplo, diferentes
			sabores de \TeX{} que tengan
			diferentes rutas de búsquedas.  
		\item  Considerado como cadenas literales, el
			\var{value} puede consistir de cualquier
			carácter. Sin embargo, en la práctica la mayoría
			de los valores de \file{texmf.cnf} están
			relacionados con la expansión de la ruta, y
			debido a que varios caracteres especiales son
			usados durante esta expansión, (vea la sección
			~\ref{sec:cnf-special-chars}), tales como llaves
			y comas, estos no se pueden usar en los nombres
			de los directorios.  Un \samp{;} en \var{:}, es
			traducido a \samp{:} si el programa está
			ejecutándose bajo Unix, y de esta manera tener
			tan solo un \file{texmf.cnf} que pueda respaldar
			ambos sistemas Unix y Windows. Esta traducción
			sucede con cualquier valor, no tan solo con
			rutas de búsquedas, pero afortunadamente en la
			práctica \samp{;} no se necesita en otros
			valores.  La función \code{\$\var{var},
			\var{prog}} no está disponible  a mano derecha;
			en vez de ello, se debe utilizar una variable
			adicional.  
		\item
			Todas las definiciones son leídas
			antes de que cualquier cosa se
			expanda, de esta forma las variables
			pueden ser referidas antes que sean
			definidas. 
	\end{itemize*}

	Un fragmento de un archivo de configuración que
	ilustra la mayoría de estos puntos es
	\ifSingleColumn
	mostrado abajo:

	\verbatiminput{examples/ex5.tex}
	\else
	muestra en Figura~\ref{fig:config-sample}.
	\fi

	\subsubsection{Expansión de la ruta}
	\label{sec:path-expansion}

	\KPS{} reconoce ciertos caracteres especiales y construcciones en rutas
	de búsqueda, similares a aquellos que están disponibles
	en las shells (intérprete de línea de órdenes o comandos) de Unix.
	Como ejemplo general, la ruta
	\verb+~$USER/{foo,bar}//baz+, expande a todos los
	subdirectorios bajo directorios \file{foo} y \file{bar}
	en el directorio de la casa de \texttt{\$USER} que
	contiene un directorio o archivo \file{baz}. Estas
	expansiones son explicadas en la sección más adelante.
	
	\subsubsection{Expansión estándar}
	\label{sec:default-expansion}

        Si la ruta de búsqueda de más alta prioridad (vea ``Orígenes
        de la ruta'' Path Sources) en la página~\ref{sec:path-sources}
        contiene un \emph{dos puntos} adicional (i.e., líderes,
        sobrantes, o dobles), \KPS{} inserta en ese punto, la próxima
        ruta de búsqueda de más alta prioridad que es definida. Si esa
        ruta insertada, tiene un extra punto, lo mismo sucede después con la
        próxima ruta que tiene más alta prioridad.  Por ejemplo, dada la
        variable del sistema:

\begin{alltt}
	> \Ucom{setenv TEXINPUTS /home/karl:}
\end{alltt}
y un valor \code{TEXINPUTS} del archivo \file{texmf.cnf} de

\begin{alltt}
	.:\$TEXMF/tex
\end{alltt}
entonces el valor final usado para la búsqueda será:

\begin{alltt}
	/home/karl:.:\$TEXMF//tex
\end{alltt}

Debido a que es innecesario insertar el valor estándar en más
de un lugar, \KPS{} cambia solamente un extra \samp{:}\ y deja
cualquier otro en su lugar. Primero revisa un líder \samp{:},
y luego un sobrante \samp{:}, y más tarde un doble \samp{:}.

\subsubsection{Expansión de llaves}
\label{sec:brace-expansion}

Una función que es útil, es la expansión de 
llaves, lo cual quiere decir que, por ejemplo,
\verb+v{a,b}w+ se expande a \verb+vaw:vbw+. Nidos son
permitidos. Esto tiene su uso, para implementar múltiples
jerarquías de \TeX{}, asignando una lista de llaves a
\code{\$TEXMF}.
En el archivo distribuido \file{texmf.cnf}, una definición
como esta (simplificada para este ejemplo) es hecha:
\begin{verbatim}
	TEXMF = {$TEXMFVAR,$TEXMFHOME,
	!!$TEXMFLOCAL,!!$TEXMFDIST}
\end{verbatim}
Después podemos usar esto para definir, por ejemplo, la ruta de
ingreso de \TeX{}. 
\begin{verbatim}
	TEXINPUTS = .;$TEXMF/tex//
\end{verbatim}

lo cual quiere decir, que después de buscar en el directorio
actual, los árboles \code{\$TEXMFVAR/tex},
\code{\$TEXMFHOME/tex}, \code{\$TEXMFLOCAL/tex}, y
\code{\$TEXMFDIST/tex}, \emph{solamente} serán buscados (los
últimos dos, a través de los archivos de la base de datos
\file{ls-R}). 

\subsubsection{Expansión del subdirectorio}
\label{sec:subdirectory-expnsion}

Dos o más barras oblicuas consecutivas, en un elemento de ruta después
de un directorio \var{d}, es reemplazado por todos los subdirectorios
de \var{d\/}: primero aquellos subdirectorios bajo \var{d}, después los
sub-subdirectorios bajo ellos, y así sucesivamente. En cada nivel, el
orden por el cual los directorios son buscados, \emph{no es
especificado}.

Si se especifica cualquier componente de un nombre de un archivo
después de \samp{//}, solamente los subdirectorios con los componentes
que correspondan, son incluidos. Por ejemplo, \samp{/a//b} expande a
los directorios \file{/a/1/b}, \file{/a2b}, \file{a/1/1/b}, y así
continúa, pero no a directorios como \file{/a/b/c}, o \file{/a/1}.

Construcciones múltiples \samp{//} en una ruta, son posibles, pero
\samp{//} en el comienzo de una ruta, es ignorado. 

\subsubsection{Un sumario de la lista de caracteres especiales en el archivo \file{texmf.cnf}}
\label{sec:cnf-special-chars}

La siguiente lista es un sumario de los caracteres especiales de los
archivos de configuración de \KPS{}.

% necesita un espacio más ancho para estos labels.
\newcommand{\CODE}[1]{\makebox[3em][l]{\code{#1}}}
\begin{ttdescription}
        \item[\CODE{:}] Separador en la especificación de la ruta, en
            el principio o el final de la ruta, o doble en el medio,
		substituye la expansión de la ruta estándar. \par
	\item[\CODE{;}] Separador en sistemas non-Unix (actúa como
		\code{:}).
	\item[\CODE{\$}] Expansión de variable.
        \item[\CODE{\string~}] Representa el directorio de casa del
            usuario.
	\item[\CODE{\char`\{...\char`\}}] Expansión de llaves.
	\item[\CODE{\,}] Separa todos los caracteres en expansión de llaves.
        \item[\CODE{//}] Expansión de subdirectorio (puede ocurrir en
            cualquier lugar de la ruta, excepto en el principio).
	\item[\CODE{\%}] Comienzo de comentario.
	\item[\CODE{\bs}] Al final de la línea, carácter de continuación
		(permite ingresos de múltiples líneas).
	\item[\CODE{!!}] Búsqueda \emph{solamente} en la base de datos
            para localizar el archivo, \emph{no} busca en el disco.
\end{ttdescription}

Exactamente saber cuando un carácter será considerado especial o actúe pos si solo, depende en el contexto en el cual este carácter es usado. Las reglas son inherentes en los niveles múltiples de configuración (análisis sintáctico, expansión, búsqueda, \ldots)\ y desafortunadamente por ese motivo, no se puede explicar concisamente. No hay ningún mecanismo general de escape, en particular \samp{\bs} no es un ``carácter de escape'' en los archivos \file{texmf.cnf}.

Cuando se refiere a escoger nombres de directorios para una instalación, lo más confiable y seguro, es tratar de evitarlos todos.

\subsection{Base de datos de nombres de archivos}
\label{sec:filename-database}

\KPS{} implementa todos los medios para minimizar el acceso al disco para las
búsquedas. No obstante a ello, en el estándar \TL, buscar un posible
directorio para un archivo dado, entre tantas instalaciones con suficientes
directorios, puede tomar un tiempo excesivo. Por este motivo, \KPS{} puede utilizar una base de datos con texto regular que sea edificada externamente con el archivo nombrado \file{ls-R} que asigna estos archivos a los directorios, y evitando así que se agote una búsqueda en el disco. 

Un archivo \file{aliases} de una segunda base de datos, permite dar
nombres adicionales a los archivos que están listados en \file{ls-R}.
Esto puede ser útil, para confirmar las convenciones de nombre de
archivos \acro{DOS} 8.3 en aquellos archivos originales.

\subsubsection{La base de datos de nombres de archivos}
\label{sec:ls-R}

Como se explicó anteriormente, el nombre del principal nombre de
archivo de la base de datos tiene que ser \file{ls-R}. Puede poner
uno en la raíz de cada jerarquía de su instalación de \TeX{} 
 para que así sea buscada (\code{\$TEXMF} por estándar. \KPS{} busca
archivos \file{ls-R} alrededor de la ruta \code{TEXMFDBS}).

La manera recomendable para crear y mantener \samp{ls-R} es mediante
la ejecución de \code{mktexlsr}, que está incluído con la
distribución. Es invocado por varios scripts \samp{mktex}. Como regla general, este script solo ejecuta el comando
\begin{alltt}
	cd \var{/tu/raíz/texmf} && \path|/|ls -1LAR ./ >ls-R
\end{alltt}
asumiendo que tu sistema de \code{ls} produce el formato correcto en el
resultado final (\GNU \code{ls} está bien). Para asegurarnos que la base de
datos siempre esté actualizada, es más fácil en reedificar esta base
de datos, regularmente, a través de \code{cron}, de esa manera es
automáticamente actualizada cuando los archivos de instalación
se hayan modificado, tal como ocurre después de instalar o actualizar un paquete de
\LaTeX{}.

Si un archivo no puede ser encontrado en la base de datos, \KPS{} por
estándar o por norma, continúa la búsqueda en el disco. Sin embargo,
si un elemento de la búsqueda comienza con \samp{!!}, solamente la
búsqueda en la base de datos será para ese elemento como tal, y nunca
en el disco. 

\subsubsection{kpsewhich: ruta de búsqueda independiente}
\label{sec:invoking-kpsewhich}

El programa \texttt{kpsewhich} ejercita la ruta de búsqueda,
independiente de cualquier otra aplicación. Esto puede ser útil, como si
fuese un programa \code{find} para localizar archivos en las
jerarquías de \TeX{} ~(esto se utiliza bastante con los scripts o programas
distribuidos por \samp{mktex}).

\begin{alltt}
> \Ucom{kpsewhich \var{option}\dots{} \var{filename}\dots{}}
\end{alltt}
Las opciones especificadas en \ttvar{option} comienzan con \samp{-} o
\samp{-{}-}, y muchas abreviaciones que no son ambiguas, son
aceptables.

\KPS{} busca cada argumento sin-opción, en la línea de comando, tal como
un nombre de archivo, y enumera los resultados con el primer archivo
que haya sido encontrado. No hay opción alguna para que produzca la lista de
todos los archivos que se hayan encontrados, a través de un nombre en
particular (usted puede ejecutar la utilidad de Unix \samp{find} para
obtener eso).

Las opciones más frecuentes son descritas a continuación:

\begin{ttdescription}
	\item[\texttt{-{}-dpi=\var{num}}]\mbox{}
	Ajusta la resolución a \ttvar{num}; esto solamente afecta
	\samp{gf} y búsquedas de \samp{pk}. \samp{-D} es un sinónimo,
	para compatibilidad con \cmdname{dvips}. Estándar es 600.  
\item[\texttt{-{}-format=\var{name}}]\mbox{}\\
	Ajusta el formato para buscar, a \ttvar{name}. Por estándar,
	el formato es averiguado por el nombre del archivo. Para
	aquellos formatos que no tienen un sufijo que no sea ambiguo,
	tales como los archivos de respaldo \MP{} y los archivos de
	configuración de \cmdname{dvips}, usted tiene que especificar
	el nombre, tal y como \KPS{} lo reconoce. Ejemplo de esto es
	\texttt{tex} o \texttt{enc archivos}. Ejecute en la línea del intérprete de
	comandos u órdenes \texttt{kpsewhich -{}-help-formats} para una lista con estas opciones. 
\item[\texttt{-{}-mode=\var{string}}]\mbox{}\\
        Ajusta el modo de nombre para \ttvar{string} (cadena de
        caracteres); esto solamente afecta las búsquedas de \samp{gf}
        y \samp{pk}. Cuando no es estándar: cualquier modo será
        encontrado. 
\item[\texttt{-{}-must-exist}]\mbox{}\\
	Haz todo lo posible para encontrar los archivos, incluyendo
	comenzar la búsqueda en el disco. Por estándar, solamente la
	base de datos de \file{ls-R} es revisada, en el interés de
	eficiencia. 
\item[\texttt{-{}-path=\var{string}}]\mbox{}\\
	Búsqueda a través de la ruta \ttvar{string} - cadena de caracteres - (separado
	por \emph{dos puntos}, como es usual), en vez de adivinar la ruta
	de búsqueda desde el nombre del archivo. \samp{//} y
	todas las expansiones que usualmente se utilizan, son respaldadas. Las
	opciones \samp{-{}-path} y \samp{-{}-format} son
	exclusivas.
\item[\texttt{-{}-progname=\var{name}}]\mbox{}\\
	Especifique el nombre del programa a \texttt{\var{name}}.
	Esto puede afectar las rutas de búsquedas a través de
	\texttt{.\var{progname}}. El estándar es \cmdname{kpsewhich}.
\item[\texttt{-{}-show-path=\var{name}}]\mbox{}\\
	muestra la ruta que es utilizada para las búsquedas de archivo de tipo
	\texttt{\var{name}}. 
        Tanto una extensión de nombre de archivo (\code{.pk},
        \code{.vf}, etc.) o un nombre pueden ser usados, al igual que
        la opción \samp{-{}-format}.
\item[\texttt{-{}-debug=\var{num}}]\mbox{}\\
	selecciona las opciones de depuración a \texttt{\var{num}}.
\end{ttdescription}

\subsubsection{Ejemplos de uso}
\label{sec:examples-of-use}

Vamos ahora a ver a \KPS{} en acción. Aquí hay una búsqueda transparente:

\begin{alltt}
	> \Ucom{kspewhich article.cls}
	/usr/local/texmf-dist/tex/latex/base/article.cls
\end{alltt}

Estamos buscando el archivo \file{article.cls}. Debido a que el sufijo
\samp{.cls} no es ambiguo, no necesitamos especificar que necesitamos encontrar
un archivo de tipo \optname{tex} (los directorios de \TeX{} con los archivos
originales). Lo encontramos en el subdirectorio \file{tex/latex/base}, justo
debajo de \samp{texmf-dist} en el directorio de \TL{}. Similarmente, todos los
ficheros que continúan posteriormente, son encontrados sin problemas gracias al
sufijo no ambiguo.

\begin{alltt}
> \Ucom{kpsewhich array.sty}
   /usr/local/texmf-dist/tex/latex/tools/array.sty
> \Ucom{kpsewhich latin1.def}
   /usr/local/texmf-dist/tex/latex/base/latin1.def
> \Ucom{kpsewhich size10.clo}                        
   /usr/local/texmf-dist/tex/latex/base/size10.clo
> \Ucom{kpsewhich small2e.tex}
   /usr/local/texmf-dist/tex/latex/base/small2e.tex
> \Ucom{kpsewhich tugboat.bib}
   /usr/local/texmf-dist/bibtex/bib/beebe/tugboat.bib
\end{alltt}

Hablando de esto, el último es una base de datos de bibliografía \BibTeX{} para
artículos en \textsl{TUGboat}.

\begin{alltt}
	> \Ucom{kpsewhich cmr10.pk}
\end{alltt}
Archivos bitmap (mapas representados con bits) del glifo de la fuente \file{.pk} son usados por los
programas de visualización, como \cmdname{dvips} y \cmdname{xdvi}.
Nada es generado en este caso, debido a que no hay archivos de
Computer Modern pre-generados \samp{.pk} en \TL{}\Dash las variantes
Type-1 son usadas por estándar.
\begin{alltt}
	> \Ucom{kspewhich wsuipa10.pk}
	\ifSingleColumn
	/usr/local/texmf-var/fonts/pk/ljfour/public/wsuipa/wsuipa10.600pk
	\else /usr/local/texmf-var/fonts/pk/ljfour/public/wsuipa/wsuipa10.600pk
\fi\end{alltt}
Para estas fuentes (utilizando un alfabeto fonético de la Universidad de
Washington) tuvimos que generar archivos \samp{.pk}, y debido a que el
modo estándar \MF{} de nuestra instalación es \texttt{ljfour} con una
base de resolución de 600\dpi{} (puntos por pulgadas), este valor es
ingresado.
\begin{alltt}
	> \Ucom{kpsewhich -dpi=300 wsuipa10.pk}
\end{alltt}
En este caso, cuando se especifica que estamos interesados en una
resolución de 300\dpi (\texttt{-dpi=300}) vemos que tal fuente, no
está disponible en este sistema.
Un programa como \cmdname{dvips} o \cmdname{xdvi} edificaría sin
problemas, los archivos requeridos \texttt{.pk}, usando el script
\cmdname{mktexpk}.

El próximo punto a tocar, son los archivos de configuración y cabecera
de \cmdname{dvips}. Primero miramos a uno de los archivos comúnmente
usados, el prólogo general \file{tex.pro} para respaldo de ayuda con
\TeX{}, antes de hablar sobre la configuración genérica del archivo
(\file{config.ps}) y el fichero del mapa \file{psfonts.map} de la fuente
\PS{}\Dash que desde el 2004, estos archivos de mapas y codificaciones
han tenido sus rutas de búsquedas y nuevas localizaciones, en los
árboles de \dirname{texmf}.  Debido a que el sufijo \samp{.ps} es
ambiguo, tenemos que especificar explícitamente cuáles de estos tipos
de fuentes, tenemos que considerar de la configuración (\optname{dvips
config}) para el archivo \texttt{config.ps}.
\begin{alltt}
> \Ucom{kpsewhich tex.pro}
   /usr/local/texmf/dvips/base/tex.pro
> \Ucom{kpsewhich --format="dvips config" config.ps}
   /usr/local/texmf/dvips/config/config.ps
> \Ucom{kpsewhich psfonts.map}
   /usr/local/texmf/fonts/map/dvips/updmap/psfonts.map
\end{alltt}

Ahora le damos un vistazo más cerca, a los archivos de respaldo \PS{}
para \acro{URW} Times. El prefijo para estos en el esquema estándar de
fuentes es \samp{utm}. El primer archivo que tenemos es el archivo de
configuración, el cual contiene el nombre de los archivos con mapas de
las fuentes. 
\begin{alltt}
> \Ucom{kpsewhich --format="dvips config" config.utm}
   /usr/local/texmf-dist/dvips/psnfss/config.utm
\end{alltt}
Los contenidos de ese archivo son 
\begin{alltt}
	p +utm.map
\end{alltt}
el cual apunta al archivo \file{utm.map}, que queremos localizar a continuación.
\begin{alltt}
> \Ucom{kpsewhich utm.map}
   /usr/local/texmf-dist/fonts/map/dvips/times/utm.map
\end{alltt}
Este archivo con el mapa de la fuente, define los nombres de archivos,
de Type-1 de la colección de URW. Sus contenidos son parecidos a estos
(solamente mostramos parte de las líneas):
\begin{alltt}
utmb8r  NimbusRomNo9L-Medi    ... <utmb8a.pfb
utmbi8r NimbusRomNo9L-MediItal... <utmbi8a.pfb
utmr8r  NimbusRomNo9L-Regu    ... <utmr8a.pfb
utmri8r NimbusRomNo9L-ReguItal... <utmri8a.pfb
utmbo8r NimbusRomNo9L-Medi    ... <utmb8a.pfb
utmro8r NimbusRomNo9L-Regu    ... <utmr8a.pfb
\end{alltt}
Tomemos por ejemplo, la instancia de la fuente Times Roman
\file{utmr8a.pfb} y encontremos su posición en el árbol del directorio
\file{texmf} con una búsqueda para archivos de fuente de Type-1:
\begin{alltt}
	> \Ucom{kpsewhich utmr8A.pfb}
\ifSingleColumn   /usr/local/texmf-dist/fonts/type1/urw/times/utmr8a.pfb
\else   /usr/local/texmf-dist/fonts/type1/                              
... urw/utm/utmr8a.pfb                                                 
\fi\end{alltt}

Debe ser evidente, mediante estos ejemplos, como se puede localizar
fácilmente las localizaciones de algún archivo. Esto es especialmente
importante, si sospechas que la versión equivocada de algún archivo,
es identificada primero, debido a que \cmdname{kpsewhich} mostrará el
primer archivo con que se tropieza. 

\subsubsection{Acciones de depuración o ``debugging''}
\label{sec:debugging}

Algunas veces es necesario, investigar como un programa resuelve las
referencias de un archivo. Para hacer esto más práctico, \KPS{} ofrece
varios niveles de la salida de depuración, resultados de la
depuración, o debugging:

 \begin{ttdescription} 
                    \item[\texttt{\ 1}] \texttt{Llamadas inmediatas}
                        (búsquedas en el disco) Cuando se está
                        operando con una base de datos \file{ls-R}
                        actualizada, esto casi siempre, no devuelve un
                        resultado en la salida.
                    \item[\texttt{\ 2}] Referencias a las cuadros hash
                        (tales como \file{ls-R} base de datos,
                        archivos de los mapas, y archivos de
                        configuración).  
                    \item[\texttt{\ 4}] Operaciones para abrir y cerrar
                        archivos.  
                    \item[\texttt{\ 8}] Información general de la
                        ruta, para los tipos de archivos que son
                        buscados por \KPS{}. Esto se ha mostrado útil,
                        para encontrar el origen donde una ruta en
                        particular de algún archivo, haya sido
                        definida.  
                     \item[\texttt{16}] Lista de directorio, para cada
                         elemento en la ruta (solamente búsquedas
                         relevantes en el disco).  
                     \item[\texttt{32}] Búsqueda de archivos.  
                     \item[\texttt{64}] Valores de las variables.  
 \end{ttdescription}
 Un valor de \texttt{-1} fijará todas las opciones
 anteriores; en práctica, esto es usualmente lo más
 conveniente. 

 De igual manera, con el programa de \cmdname{dvips}, se
 puede especificar una combinación de interruptores de
 depuración (o debugging switches), uno puede averiguar con
 detalles, la localización donde los archivos están siendo
 identificados. Alternativamente, cuando un archivo no es
 encontrado, el rastro de esta depuración, (trace debug)
 muestra los directorios, donde el programa busca un archivo
 determinado. De esta manera uno puede tener una
 idea que le indique, donde está el problema.

 Por lo general, debido a que muchos programas contactan la
 librería interna de \KPS{}, uno puede seleccionar una
 opción de depuración (debug), usando la variable del
 sistema \envname{KPATHSEA\_DEBUG}, y configurándola así a (una
 combinación) de valores, como fue descrito anteriormente. 

 (Nota para los usuarios de Windows: no es tan fácil,
 redirigir todos los mensajes a un archivo dentro de este
 sistema. Por razones de diagnósticos, uno puede
 temporalmente ajustar \texttt{SET
 KPATHSEA\_DEBUG\_OUTPUT=err.log}).

Consideremos por un momento, el siguiente ejemplo, un pequeño
archivo original de \LaTeX{}, el archivo
\file{hello-world.tex}, el cual contiene el siguiente
ingreso.
\begin{verbatim}
  \documentclass{article}
  \begin{document}
  Hello World!
  \end{document}
\end{verbatim}
Este pequeño archivo, solamente utiliza la fuente
\file{cmr10}. Veamos entonces como \cmdname{dvips} prepara
el archivo de \PS{} (queremos hacer uso de la versión Type-1 de las
fuentes de Computer Modern, por tal motivo la opción
\texttt{-Pcms}).

\begin{alltt}
> \Ucom{dvips -d4100 hello-world -Pcms -o} 
\end{alltt}
En este caso hemos combinado, la clase 4 de depuración
\cmdname{dvips} con el elemento de expansión de \KPS{} (vea
el manual de referencia de \cmdname{dvips}).
El resultado (reordenado un poco) aparece en la
figura~\ref{fig:dvipsdbga}.
\begin{figure*}[tp]
	\centering
	\input{examples/ex6a.tex}
	\caption{Encontrando archivos de
	configuración}\label{fig:dvipsdbga}

	\bigskip

	\input{examples/ex6b.tex}
	\caption{Encontrando el archivo
	prolog}\label{fig:dvipsdbgb}

	\bigskip

	\input{examples/ex6c.tex}
	\caption{Encontrando el archivo de
	la fuente}\label{fig:dvipsdbgc}
\end{figure*}

\cmdname{dvips} comienza localizando los archivos de
trabajo. Primero, \file{texmf.cnf} es encontrado, el cual provee
las definiciones de las rutas de búsqueda para los otros
archivos, y luego la base de datos de archivos \file{ls-R}
(para optimizar la búsqueda de archivo) y el archivo
\file{aliases}, el cual hace posible la declaración de
varios nombres (e.g., una breve versión similar a \acro{DOS}
8.3 y una versión más natural larga) para el mismo archivo.

Después de esto, el \cmdname{dvips} continúa para encontrar la
configuración genérica de \file{config.ps} antes de seguir buscando el
archivo de ajustes de personalización \file{.dvipsrc} (el cual, en este
caso \emph{no es encontrado}). Finalmente, \cmdname{dvips} localiza el
archivo de configuración \file{config.cms} para las fuentes \PS{} de
Computer Modern (esto fue iniciado con la opción de \texttt{-Pcms} en
\cmdname{dvips}). Este archivo contiene la lista de los archivos de
mapas, el cual define la relación entre los nombres de las fuentes de
\TeX{}, \PS{}, y el sistema de archivos. 
\begin{alltt}
	> \Ucom{más /usr/local/texmf/dvips/cms/config.cms}
	p +ams.map
        p +cms.map
        p +cmbkm.map
        p +amsbkm.map
\end{alltt}
\cmdname{dvips} de esa manera continúa para encontrar todos
estos archivos, en adición al archivo del mapa
\file{psfonts.map}, el cual siempre es cargado (contiene
declaraciones para fuentes de \PS{} que son comúnmente
usadas; vea la última parte de Sección
\ref{sec:examples-of-use} para más detalles acerca de la
manipulación de archivos de mapas \PS{}).

Ya a este punto, \cmdname{dvips} se auto-identifica al
usuario:
\begin{alltt}
	This is dvips(k) 5.92b Copyright 2002 Radical Eye
	Software (www.radicaleye.com)
\end{alltt}
\ifSingleColumn
Luego comienza a buscar el archivo prolog \file{texc.pro}:
\begin{alltt}\small
kdebug:start search(file=texc.pro, must\_exist=0, find\_all=0,
  path=.:~/tex/dvips//:!!/usr/local/texmf/dvips//:             	
       ~/tex/fonts/type1//:!!/usr/local/texmf/fonts/type1//).
kdebug:search(texc.pro) => /usr/local/texmf/dvips/base/texc.pro
\end{alltt}
\else
Luego comienza a buscar el prolog file \file{texc.pro}(vea
Figura~\ref{fig:dvipsdbgb}).
\fi

Después de encontrar este archivo, \cmdname{dvips} nos da
los resultados con la fecha y la hora, y nos informa que
generará el archivo \file{hello-world.ps}. Consiguientemente, 
necesita el archivo de fuente \file{cmr10}, y que este
último es declarado como ``residente''(bitmaps o mapas de bits no son
necesitadas):
\begin{alltt}\small
TeX output 1998.02.26:1204' -> hello-world.ps
Defining font () cmr10 at 10.0pt
Font cmr10 <CMR10> is resident.
\end{alltt}
Ahora la búsqueda es por el archivo \file{cmr10.tfm}, el
cual es encontrado. Después de esto, unos cuantos más
archivos prolog (no se muestran) son ingresados como
referencias, y finalmente la instancia de Typo-1
\file{cmr10.pfb} de la fuente, es localizada e incluida en
el archivo de resultados (vea la última línea).
\begin{alltt}\small
kdebug:start search(file=cmr10.tfm, must\_exist=1, find\_all=0,
  path=.:~/tex/fonts/tfm//:!!/usr/local/texmf/fonts/tfm//:                  
/var/tex/fonts/tfm//).
kdebug:search(cmr10.tfm) => /usr/local/texmf/fonts/tfm/public/cm/cmr10.tfm
kdebug:start search(file=texps.pro, must\_exist=0, find\_all=0),
   ...
<texps.pro>
kdebug:start search(file=cmr10.pfb, must\_exist=0, find\_all=0,
  path=.:~/tex/dvips//:!!/usr/local/texmf/dvips//:
       ~/tex/fonts/type1//:!!/usr/local/texmf/fonts/type1//).
kdebug:search(cmr10.pfb) => /usr/local/texmf/fonts/type1/public/cm/cmr10.pfb
<cmr10.pfb>[1]
\end{alltt}

\begin{alltt}\small
kdebug:start search(file=cmr10.tfm, must\_exist=1, find\_all=0,
  path=.:~/tex/fonts/tfm//:!!/usr/local/texmf/fonts/tfm//:
       /var/tex/fonts/tfm//).
kdebug:search(cmr10.tfm) => /usr/local/texmf/fonts/tfm/public/cm/cmr10.tfm
kdebug:start search(file=texps.pro, must\_exist=0, find\_all=0,
   ...
<texps.pro>
kdebug:start search(file=cmr10.pfb, must\_exist=0, find\_all=0,
  path=.:~/tex/dvips//:!!/usr/local/texmf/dvips//:
       ~/tex/fonts/type1//:!!/usr/local/texmf/fonts/type1//).
kdebug:search(cmr10.pfb) => /usr/local/texmf/fonts/type1/public/cm/cmr10.pfb
<cmr10.pfb>[1]
\end{alltt}

\subsubsection{Opciones de ejecución}

Una función útil de \Webc{} es la posibilidad de controlar
un número de parámetros de memoria (en particular, tamaños
de matrices) a través de la ejecución del archivo
\file{texmf.cnf} que es leído por \KPS{}. Las
configuraciones
de la memoria pueden ser encontradas en Parte-3 de ese
archivo en la distribución de \TL{}. Las más importantes
son:

\begin{ttdescription}
	\item[\texttt{main\_memory}]
		Total de palabras de memoria disponible,
		para \TeX{}, \MF{} y \MP. Usted puede hacer
		un nuevo archivo de formato, para todas las
		diferentes configuraciones. Por ejemplo, usted
		pudiese generar una ``inmensa'' versión de
		\TeX{}, y nombrar el archivo formateado
		\texttt{inmensotex.fmt}. Utilizando la
		manera estándar de especificar el nombre del
		programa que es usado por \KPS{}, el valor
		particular de la variable \texttt{main\_memory}
		se leerá de \file{texmf.cnf}.
	\item[\texttt{extra\_mem\_bot}]
		Espacio extra para estructuras ``grandes''
		de \TeX: cajas, pegamentos, puntos de
		interrupción, etc. Especialmente útil si se
		utiliza \PiCTeX{}.
	\item[\texttt{extra\_mem\_bot}]
		Número de palabras para la información de
		fuente disponible para \TeX. Esto es más o
		menos el tamaño total de todos los archivos
		\acro{TFM} que fueron leídos. 
	\item[\texttt{hash\_extra}]
		Espacio adicional para el cuadro hash, de la
		secuencia de control de nombres. El valor estándar es \texttt{600000}. 
\end{ttdescription}

\noindent Esta función no es substituto con las matrices
realmente dinámicas y con la asignación de memoria, pero debido a que es
extremadamente difícil implementar en el archivo original \TeX{}, estos
parámetros de ejecución proveen una práctica comprometedora, que permite
cierta flexibilidad. 

\htmlanchor{texmfdotdir}
\subsection{\texttt{\$TEXMFDOTDIR}}
\label{sec:texmfdotdir}

En varios lugares anteriormente, dimos varias rutas de búsqueda que comenzaban con \code{.} (para encontrar el actual directorio primero), como en
\begin{alltt}\small
TEXINPUTS=.;$TEXMF/tex//
\end{alltt}

Esto es una simplificación. El archivo \code{texmf.cnf} que distribuimos en \TL{} utiliza \filename{$TEXMFDOTDIR} en vez de:
\begin{alltt}\small
TEXINPUTS=$TEXMFDOTDIR;$TEXMF/tex//
\end{alltt}
(En el fichero que se distribuye, el segundo elemento de ruta es un poco más complicado que \filename{$TEXMF-/tex//}. Pero eso es leve; aquí queremos hablar de la característica de \filename{$TEXMFDOTDIR}.)

La razón para utilizar la variable \filename{$TEXMFDOTDIRR} en las definiciones de la ruta en vez de simplemente \samp{.}, es puramente porque puede ser sobrescrita. Por ejemplo, un documento complejo puede tener muchos más ficheros de origen que estén organizados en muchos directorios. Para ocuparnos de esto, usted puede configurar \filename{TEXMFDOTDIR} con \filename{.//} (por ejemplo, en el entorno del sistema cuando se está preparando el documento) y de esta manera todos serán encontrados durante esa búsqueda. (Advertencia: No utilices \filename{.//} predeterminado o estándar; esto no es recomendable, y potencialmente inseguro el escanear todos los subdirectorios po

Como otro ejemplo, usted quizás no quiera buscar nada en el actual directorio, e.g., si haz organizado todos los archivos que se encuentren mediante rutas de búsqueda explícitas. Puede configurar \filename{$TEXMFDOTDIR} a digamos \filename{/nonesuch} o cualquier directorio inexistente para esto.

El valor estándar de \filename{$TEXMFDOTDIR} es solamente \samp{.}, como está configurado en \filename{texmf.cnf}


\htmlanchor{ack}
\section{Agradecimientos}

\TL{} es un esfuerzo unido, por virtualmente todos los grupos
de usuarios de \TeX{}. Esta edición de \TL{} fue supervisada
por Karl Berry. Los otros principales contribuidores, pasados y
presentes, aparecen en la siguiente lista:

\begin{itemize*}

\item Los grupos de usuarios en inglés, alemán, holandés, y polaco de \TeX{}
	(\acro{TUG}, \acro{DANTE} e.V., \acro{NTG}, y \acro{GUST},
	respectivamente), que proveyeron la infraestructura técnica y
	administrativa.  Por favor, !`forme parte de un grupo \TeX\ cerca de
	usted! (Vea \url{https://tug.org/usergroups.html}.) 

\item El equipo de \acro{CTAN}, (\url{https://ctan.org}) que
	distribuye las imágenes de \TL{} y proveen la
		infraestructura común para actualizaciones
		de paquetes, en los cuales \TL{} depende.  

\item Nelson Beebe, por hacer muchas plataformas disponibles a constructores de
	\TL\, y su propia comprehensiva revisión.  

\item John Bowman, por hacer muchos cambios a su avanzado programa gráfico
Asymptote, que trabaje con \TL.  \item Peter Breitenlohner y el equipo \eTeX\
	por la fundación estable del futuro de \TeX, y Peter específicamente
	por la ayuda estelar por la utilización de \GNU\ autotools a través de \TL.  

\item	Jin-Hwan Cho y todo el equipo de DVIPDFM$x$, por su excelentes drivers
	y respuestas con problemas de configuración. 

\item Thomas Esser, quien sin su contribución, el maravilloso paquete \teTeX{}
	para \TL{} nunca hubiese existido. 

\item Michel Goosens, por la co-autoría de la documentación original.

\item Eitan Gurari, cuyo \TeX4ht es usado para crear la version de
	esta documentación en \HTML{}, y quien trabajó esforzadamente todos
	los años para mejorar este código.  Eitan falleció en junio del 2009,
	y dedicamos esta documentación a su memoria. 

\item Hans Hagen, por mucha revisión y por hacer su formato de \ConTeXt\
	(\url{https://pragma-ade.com}) que funcione dentro del marco de
	funcionalidad de \TL.

\item \Thanh, Martin Schr\"oder, y el equipo de pdf\TeX\
	(\url{http://pdftex.org}) por la continuación de adiciones a las
	habilidades de \TeX. 

\item Hartmut Henkel, por significantes contribuciones al desarrollo de
	pdf\TeX, Lua\TeX, y mucho más.

\item Shunshaku Hirata, por su continuo trabajo y originalidad en 
DVIPDFM$x$.

\item Taco Hoekwater, por renovados esfuerzos de desarrollo en MetaPost y el
	mismo (Lua)\TeX\ (\url{http://luatex.org}), incorporando \ConTeXt\ en
		\TL, y dándole a Kpathsea una característica de
		multi-procesos, y mucho más.

\item Pawe{\l} Jackowski, por el instalador de Windows \cmdname{tlpm}, y
	Tomasz {\L}uczak, por \cmdname{tlpmgui}, que ha sido usado en
		versiones anteriores. 

\item Akira Kakuto, por proveer los binarios de Windows de las
    distribuciones \acro{W32TEX} y \acro{W64TEX} para \TeX\ en japonés
        (\url{http://w32tex.org}), y muchas otras contribuciones de
        desarrollo.

\item Hironori Kitagawa, por el mantenimiento de p\TeX\ y respaldo relacionado al mismo.

\item Jonathan Kew, por desarollar el impresionante motor de \XeTeX{} y por
	tomar su tiempo, en resolver el problema de integrarlo en \TL{}, al
	igual que la versión inicial del instalador de Mac\TeX{}, y también
	por nuestro recomendable programa \TeX{}works.  

\item Dick Koch, por la mantención de Mac\TeX\ (\url{https://tug.org/mactex})
	en estrecha colaboración con \TL{}, y por su gran entusiasmo en
	hacerlo.

\item Reinhard Kotucha, por sus grandes contribuciones a la infraestructura y
	el instalador de \TL{} 2008, al igual que por su esfuerzo con la
	investigación de Windows, el script de \texttt{getnonfreefonts}, y
	mucho más. 

\item Siep Kroonenberg, también por sus enormes contribuciones a la
	infraestructura e instalador de \TL{} 2008, especialmente en Windows,
	y por su trabajo en actualizar este manual, donde describe muchas de
	estas funciones. 

\item Clerk Ma, por reparación de errores en el motor y extensiones.

\item Mojca Miklavec, por muchísima ayuda con \ConTeXt, edificando muchos
	conjuntos de binarios y mucho más.

\item Heiko Oberdiek, por su paquete \pkgname{epstopdf} y muchos otros,
	también por comprimir los archivos de datos del extenso
	\pkgname{pst-geo} y de esa manera poder incluirlo, pero por encima de
	todo, por su notable trabajo con \pkgname{hyperref}. 

\item Phelype Oleinik, por el \cs{input} de delimitados de grupos en todos los motores en 2020, y mucho más 

\item Petr Ol\v{s}ak, quien coordinó y revisó cuidadosamente todo el material
	eslovaco y checoslovaco.

\item Toshio Oshima, por su pre-visualizador \cmdname{dviout} para Windows. 

\item Manuel P\'egouri\'e-Gonnard, por ayudar con actualizaciones de paquetes,
	mejorías en la documentación, y por su participación con el desarrollo
	de \cmdname{texdoc}.

\item Fabrice Popineau, por su respaldo original de Windows en \TL{} y trabajo
	con la documentación francesa. 

\item Norbert Preining, el principal arquitecto de la actual infraestructura e
	instalador de \TL{}, y también por la coordinación de la versión de
	\TL{} para Debian (junto con Frank K\"uster), y trabajando tanto
	durante todo el proceso. 

\item Sebastian Rahtz, por originalmente crear \TL{} y mantenerlo por varios
	años. 

\item Phil Taylor, por ajustar las descargas BitTorrent. 

\item Luigi Scarso, por el continuo desarrollo de MetaPost, LuaTeX, y mucho
	más. 

\item Andreas Scherer, por \texttt{cwebbin}, la implementación de CWEB que es usada en \TL{}, y el continuo mantenimiennto del CWEB original.

\item Takuji Tanaka, por el mantenimiento de (e)(u)p\TeX\ y respaldo relacionado al mismo.
 
\item Tomasz Trzeciak, por su amplia ayuda con Windows.

\item Vladimir Volovich, por su substancial ayuda con la migración y otros
	problemas de mantenimiento, y especialmente por hacer más asequible
	el programa \cmdname{xindy}.

\item Staszek Wawrykiewicz, el principal evaluador para todo \TL{}, y
	coordinador de las mayores contribuciones polacas: fuentes, instalación
	en Windows, y mucho más. Staszek falleció en Febrero del 2018, y
	dedicamos el continuo trabajo a su memoria.  

\item Olaf Weber, por su paciente mantenimiento de \Webc.

\item Gerben Wierda, por crear y mantener el original apoyo técnico de
	\MacOSX.

\item Graham Williams, el creador del Catálogo de \TeX.

\item Joseph Wright, por su extenso trabajo en que la funncionlidad primitiva esté disponible a través de todos los motores.

\item Hironobu Yamashita, por muchísimo trabajo en en p\TeX\ y respaldo relacionado con el mismo.
 
\end{itemize*}

Edificadores de los binarios:
Marc Baudoin (\pkgname{amd64-netbsd}, \pkgname{i386-netbsd}),
Ken Brown (\pkgname{i386-cygwin}, \pkgname{x86\_64-cygwin}),
Simon Dales (\pkgname{armhf-linux}),
Johannes Hielscher (\pkgname{aarch64-linux}),
Akira Kakuto (\pkgname{win32}),
Dick Koch (\pkgname{x86\_64-darwin}),
Mojca Miklavec (\pkgname{amd64-freebsd},
                \pkgname{i386-freebsd},
                \pkgname{x86\_64-darwinlegacy},
                \pkgname{i386-solaris}, \pkgname{x86\_64-solaris},
                \pkgname{sparc-solaris}),
Norbert Preining (\pkgname{i386-linux},
                  \pkgname{x86\_64-linux},
                  \pkgname{x86\_64-linuxmusl}).
Para información en el proceso de edificación de \TL{} vea \url{https://tug.org/texlive/build.html}.

 Traductores de este manual:
 Takuto Asakura (japonés),
 Boris Veytsman (ruso),
 Jjgod Jiang, Jinsong Zhao, Yue Wang, y Helin Gai (chino),
 Uwe Ziegenhagen (alemán),
 Manuel P\'egouri\'e-Gonnard, Denis Bitouz\'e, y Patrick Bideault (francés),
 Marco Pallante y Carla Maggi (italiano),
 Nikola Le\v{c}i\'c (serbio),
 Petr Sojka \& Jan Busa (eslovaco checoslovaco),
 Staszek Wawrykiewicz y Zofia Walczak (polaco),
 Carlos Enriquez Figueras (español).  
 La página en el internet de la documentación de \TL{} 
 es \url{https://tug.org/texlive/doc.html}.
 
Por supuesto que el más importante reconocimiento tiene que ir a Donald Knuth, primero antes que todo por inventar \TeX, y después por dárselo al mundo. 

\section{Historia del lanzamiento}
\label{sec:history}

\subsection{Pasado}

Conversaciones comenzaron a finales de 1993, cuando el
Grupo de Usuarios Holandés de \TeX{} comenzó a trabajar en
su \CD{} 4All\TeX{} para usuarios de \acro{MS-DOS}, y en
ese momento sólo se esperó en producir un sencillo, y
racional \CD{} para todos los sistemas. El gol de este
proyecto era bastante ambicioso, durante el tiempo que
esto se quería llevar a cabo. No obstante a ello esto
generó no tan solo el primer \CD{} 4Aall\TeX{}, pero
también el grupo de trabajo y Consejo Técnico \acro{TUG}
en la \emph{Estructura del Directorio de TeX}
(\url{https://tug.org/tds}), el cual especificó cómo crear
colecciones consistentes y manejables de los archivos de
apoyo y ayuda de \TeX{}. Una muestra completa del proyecto de \TDS{}
fue publicado en el ejemplar de diciembre de 1995 de
\textsl{TUGboat}, y fue evidente desde temprano, que un
producto deseado, estaría basado en un modelo de
estructura mediante un \CD{}. La distribución que usted
tiene en sus manos, es un resultado directo, de las
deliberaciones de este grupo de trabajo. También estaba
esclarecido que el éxito del \CD{} 4All\TeX{} mostró que
los usuarios de Unix se beneficiarían de un sistema fácil,
y este es el otro principal motivo de \TL. 

En el otoño de 1995, fue cuando nos dedicamos en realizar un
\CD{} \TDS{} basado en Unix, y rápidamente identificamos el
\teTeX{} de Thomas Esser, como la configuración ideal, debido a
que ya tenía apoyo para múltiples plataformas y fue construido
con portabilidad a través de varios sistemas. Thomas acordó en
ayudar, y el trabajo comenzó seriamente en el comienzo de 1996.
La primera edición fue lanzada en mayo de 1996. A principios de
1997, Karl Berry completó una edición nueva de Web2c, el cual
incluyó casi todas las funciones que Thomas Esser había añadido
en \teTeX, y decidimos basar la 2a edición del \CD{} en el
estándar \Webc, con la adición del script \texttt{texconfig} de
\teTeX. La 3a edición del \CD{} fue basada en una mayor
revisión de \Webc, 7.2, por Olaf Weber; y durante el mismo
tiempo, una nueva revisión de \teTeX estaba siendo llevado a
cabo, y \TL{} incluyó casi todas sus funciones. La 4a edición
seguía el mismo patrón, usando una nueva versión de \teTeX, y
un nuevo lanzamiento de \Webc{} (7.3). El sistema ahora incluye
configuraciones completas para Windows. 

Para la 5a edición (marzo del 2000), muchas partes del \CD{}
fueron revisadas y chequeadas, actualizando cientos de
paquetes. Los detalles de los paquetes fueron almacenados en
archivos XML. Pero el cambio más grande para \TeX\ Live 5, fue
que todos los programas no-libres, fueron removidos. Todo en
\TL{} está ahora con el propósito de que sea compatible con las
Directrices de Programas Libres de Debian, como está reflejado
en Debian Free Software Guidelines
(\url{https://www.debian.org/intro/free.es.html}); hemos hecho nuestro
mejor esfuerzo en revisar las condiciones de la licencia para
todos los paquetes, pero agradeceríamos muchísimo, escuchar de
cualquier error cometido. 

La 6a edición (julio del 2001) tuvo mucho más material
actualizado. El mayor cambio fue el de un concepto nuevo de
instalación: el usuario pudiese seleccionar un set de
colecciones más exacto. Las colecciones relacionadas con
lenguajes, fueron completamente reorganizadas, de esa manera,
seleccionando cualesquiera de estas, instala no solo los
macros, sino también las fuentes, etc., pero también prepara un
apropiado archivo \texttt{language.dat}

La 7a edición de 2002 tuvo la notable adición de respaldo a
\MacOSX{}, y las vastas posibilidades de actualizaciones a
todo tipo de programas y paquetes. Un gol importante fue la
integración de los archivos originales con \teTeX, para
corregir la separación entre versiones 5 y 6. 

\subsubsection{2003} En el 2003, con la continua
inundación de actualizaciones y adiciones, nos dimos
cuenta 
que \TL{} había crecido de tal manera, que un solo \CD{} no
podía contenerlo, y por ese motivo lo separamos en tres
diferentes distribuciones (vea la
sección~\ref{sec:tl-coll-dists},
\p.\pageref{sec:tl-coll-dists}).  En adición a esto:

\begin{itemize*}
  \item A petición del equipo de \LaTeX{}, modificamos los comandos estándares 
\cmdname{latex} y \cmdname{pdflatex} (vea
\p. \pageref{text:etex}). 
\item Las nuevas fuentes de Latin Modern fueron incluidas (y ahora son
  recomendadas).
\item Respaldo para Alpha \acro{OSF} fue removido (ayuda para
  \acro{HPUX} fue removida previamente), debido a que nadie tenía (o
  voluntariamente ofreció) equipos que estuviesen disponibles para
  compilar nuevos binarios. 
\item Las configuraciones de Windows, fueron substancialmente
  cambiados; por primera vez, todo un sistema basado en XEmacs, fue
  introducido. 
\item Programas suplementarios muy importantes para Windows, (Perl,
  Ghost\-script, Image\-Magick, Ispell) ahora fueron instalados en el
  directorio de instalación de \TL.
\item Archivos con los mapas de las fuentes que son usados por \cmdname{dvips},
  \cmdname{dvipdfm}, y \cmdname{pdftex} ahora son generados por el
  nuevo programa \cmdname{updmap} e instalados en
  \dirname{texmf/fonts/map}.
\item \TeX{}, \MF{}, y \MP{} ahora, como estándar, ofrece los
  resultados de ingresos de caracteres (32 y por encima) como ellos
  mismos, en la producción de resultados (e.g., archivos
  \verb|\write|), archivos de registro de datos, y la terminal de
  intérprete de comandos, i.e., \emph{no} traducido mediante el uso de
  la notación \verb|^^|. en \TL{}-7, esta traducción dependía en las
  configuraciones locales del sistema; mientras que ahora, las
  configuraciones locales no influyen el funcionamiento de los
  programas \TeX{}. Si por alguna razón se necesita el resultado que
  produce \verb|^^|, renombra el archivo
  \verb|texmf/web2c/cp8bit.tcx|. (Las versiones futuras tendrán maneras
  más limpias en controlar esto). 
\item Esta documentación fue substancialmente revisada.
\item Finalmente, debido a que los números de las versiones, han
  incrementado de manera inconsistente, el número de la versión es
  identificada por el año: \TL{} 2003.
\end{itemize*}

\subsubsection{2004}

En el 2004 se vieron muchos cambios:

\begin{itemize*}

\item Si se habían instalado las fuentes a los directorios locales,
  los cuales usan sus propios archivos de soporte, como
  \filename{.map} o \filename{.enc} (mucho menos propenso este último)
  usted tendrá que mover estos archivos de respaldo. 

  Los archivos \filename{.map} se buscan ahora en los
subdirectorios \dirname{fonts/map}, solamente (en cada árbol
de \filename{texmf}), junto con la ruta del sistema
\envname{TEXFONTMAPS}. De similar manera, los archivos
\filename{.enc}, son ahora buscados en los directorios
\dirname{fonts/enc} solamente, a través de la ruta del
sistema \envname{ENCFONTS}. El comando \cmdname{updmap},
atentará de advertirle sobre cualquier problema. 

Para métodos en manipular esto y otra información, por favor
de ver \url{https://tug.org/texlive/mapenc.html}.

\item El \TK\ ha sido expandido con la adición de un \CD{}
	de instalación basado en \MIKTEX{}, para aquellos que
	prefieren la implementación a Web2C. Vea la
	sección~\ref{sec:overview-tl} (\p.
	\pageref{sec:overview-tl}). 

\item Dentro de \TL, el único e inmenso árbol
	\dirname{texmf}, de versiones anteriores, fue
	reemplazado por tres: \dirname{texmf},
	\dirname{texmf-dist}, y \dirname{texmf-doc}. Vea la
	sección~\ref{sec:tld} (\p. \pageref{sec:tld}), y los
	archivos \filename{README} para cada uno. 

\item Todos los archivos de ingreso relacionados con \TeX{}
	son ahora coleccionados en el subdirectorio
	\dirname{tex} de los árboles de \dirname{texmf*}, en
	vez de tener directorios parientes separados
	\dirname{tex}, \dirname{etex}, \dirname{pdftex},
	\dirname{pdfetex}, etc. Vea
	\CDref{texmf-dist/doc/generic/tds/tds.html\#Extensions}
	{\texttt{texmf-dist/doc/generic/tds/tds.html\#Extensions}}.

\item Los scripts de ayuda (que no están supuestos a ser
	invocados por los usuarios) están ahora localizados
	en un nuevo subdirectorio \dirname{scripts} de los
	árboles \dirname{texmf*}, y pueden ser buscados
	mediante la vía de 
	\verb|kpsewhich	-format=texmfscripts|. Por lo tanto si tiene
	programas que operan bajo estos scripts, necesitarán
	ser ajustados. Vea
	\CDref{texmf-dist/doc/generic/tds/tds.html\#Scripts}
	{\texttt{texmf-dist/doc/generic/tds/tds.html\#Scripts}}.

\item Casi todos los formatos, dejan los caracteres imprimibles como ellos
	mismos mediante la vía de ``archivo de traducción''
	\filename{cp227.tcx}, en vez de ser traducidos a la notación \verb|^^|.
	Específicamente, caracteres en las posiciones 32--256, tabulador de
	adición, tabulador vertical, y marcadores de saltos de páginas son
	considerados imprimibles y no son traducidos. Las excepciones son los
	formatos de \TeX\ (solamente 32--126 imprimible), \ConTeXt\ (0--255
	imprimible), y aquellos relacionados con \OMEGA{}. Este funcionamiento
	estándar es casi el mismo en \TL\,2003, pero es implementado más
	limpiamente, con más posibilidades para la personalización que el
	usuario necesite. Vea 
	\CDref{texmf-dist/doc/web2c/web2c.html\#TCX-files}
	{\texttt{texmf-dist/doc/web2c/web2c.html\#TCX-files}}.
	(Hablando de esto, con el ingreso de Unicode, \TeX\ puede producir
	resultados con secuencias parciales de caracteres cuando muestre
	contextos de errores, por la razón que es octeto-orientado u orientado
	con byte).

\item	\textsf{pdfetex} es el motor estándar para todos los formatos,
    excepto el \textsf{tex} básico como tal. (Por supuesto que genera
    \acro{DVI} cuando se ejecuta como \textsf{latex}, etc.) Esto
    quiere decir, entre varias cosas, que las funciones
    micro-tipográficas de \textsf{pdftex} están disponibles en
    \LaTeX{}, \ConTeXt, etc., al igual que las funciones de \eTeX{}
    (\OnCD{texmf-dist/doc/etex/base/}).

\item 	También quiere decir que \emph{más importante que nunca antes}
	la funcionalidad del paquete \pkgname{ifpdf} (trabaje con
	ambos, con \TeX\ básico y \LaTeX) o código equivalente, porque
	con simplemente evaluar si \cs{pdfoutput} o cualquier otro primitivo
	es definido, esto no es una manera confiable en determinar si el
	resultado de \acro{PDF} está siendo generado. Hicimos esto
	retroactivamente compatible, de la mejor manera que pudimos
	hacerlo este año, pero el próximo año, \cs{pdfoutput} pudiese
	ser definido incluso cuando \acro{DVI} haya sido escrito. 

\item pdf\TeX\ (\url{http://pdftex.org}) tiene muchas nuevas
    funciones:
	
	\begin{itemize*}
		\item \cs{pdfmapfile} y \cs{pdfmapline} proveen respaldo con
			los archivos de los mapas de las fuentes, dentro del
			mismo documento.

                \item Expansión micro-tipográfica puede ser usado más
                    fácilmente.\\
                    \url{http://www.ntg.nl/pipermail/ntg-pdftex/2004-May/000504.html}

                \item Todos los parámetros previamente configurados, a
                    través del archivo de configuración
                    \filename{pdftex.cfg} tienen que ser ahora a
                    través de primitivos, típicamente en
                    \filename{pdftexconfig.tex}; \filename{pdftex.cfg}
                    no es respaldado más.  Cualquier subsiguiente
                    archivo \filename{.fmt} tiene que desecharse
                    nuevamente, tan pronto cuando
                    \filename{pdftexconfig.tex} haya sido modificado. 

                \item Vea el manual de pdf\TeX\ para más información:
                    \OnCD{texmf-dist/doc/pdftex/manual/pdftex-a.pdf}.  
\end{itemize*}

\item El primitivo \cs{input} en \cmdname{tex} (y \cmdname{mf} y
	\cmdname{mpost}) ahora acepta doble apóstrofe, y puede
	contener espacios, y caracteres especiales. Ejemplos
	típicos:
	\begin{verbatim}
	\input "archivo con espacios" %plain
	\input{"archivo con espacios"} %latex
	\end{verbatim}
	Vea el manual de Web2c para más información sobre esto:
	\OnCD{texmf-dist/doc/web2c}.

\item Respaldo técnico para enc\TeX\ está ahora incluído dentro de
	Web2C y consiguientemente para todos los programas de \TeX{},
	mediante la opción \optname{~enc} \Dash \emph{solamente
	cuando los formatos sean edificados}. enc\TeX\ respalda la
	re-codificación de ingreso y salida, permitiendo el apoyo
	completo de Unicode (en \acro{UTF}-8). Vea
	\OnCD{texmf-dist/doc/generic/enctex/} y
	\url{http://olsak.net/enctex.html}.  

\item Aleph, un nuevo motor que combina \eTeX\ y \OMEGA, está
	disponible. Un poco de información está disponible en 
	\OnCD{texmf-dist/doc/aleph/base} y en 
	\url{https://texfaq.org/FAQ-enginedev} 
	El formato que está basado en \LaTeX\ para Aleph es nombrado
	\textsf{lamed}. 

\item El último lanzamiento de \LaTeX\ tiene una nueva versión de la
	licencia \acro{LPPL}\Dash ahora oficialmente una
	licencia-aprobada por Debian. Para otras actualizaciones,
	vea los otros archivos \filename{ltnews} en el 
	\OnCD{texmf-dist/doc/latex/base}.

\item \cmdname{dvipng}, un nuevo programa conversor de archivos de
	imágenes de \acro{DVI} a \acro{PNG} está incluido. Vea 
	\url{https://www.ctan.org/pkg/dvipng}.

\item  Reducimos el paquete \pkgname{cbgreek} a un tamaño ``mediano'' de
	set de fuentes, con el consejo del autor (Claudio Beccari). Las
	fuentes extraídas, son las invisibles, esbozadas, y
	transparentes, las cuales son relativamente y raramente
	utilizadas, y necesitábamos el espacio de almacenamiento. El set
	completo, por supuesto que está disponible
	(\url{https://ctan.org/pkg/cbgreek-complete}).

\item Los comandos (enlaces) \cmdname{ini} y \cmdname{vir} para
	\cmdname{tex}, \cmdname{mf}, y \cmdname{mpost} ya no son
	creados, tales como \cmdname{initex}. La funcionalidad del
	comando \cmdname{ini} ha estado disponible a través de las
	opciones de la línea de comandos \optname{-ini} desde hace
	varios años. 

\item El respaldo y ayuda técnica con la plataforma
	\textsf{i386-openbsd} fue removido. Debido a que el
	paquete \pkgname{tetex} está disponible en el sistema
	de Puertos de \acro{BSD}, al igual que los binarios
	tanto para \acro{GNU/}Linux y \acro{BSD} que también
	están disponibles, parecía que emplear el tiempo para
	trabajar voluntariamente en esto, era preferible que fuese
	con otros asuntos que necesitasen de ello. 

\item Acerca de \textsf{sparc-solaris} (por lo menos), se tuviera
	que configurar la variable del sistema
	\envname{LD\_LIBRARY\_PATH} para que ejecutara los programas
	\pkgname{tlutils}. Esto es debido a que las mismas son
	compiladas con C++, y no existe una localización estándar
	para las librerías de ejecución. (Esto no fue algo nuevo en
	el 2004, pero no estaba previamente documentado.) De igual
	manera, en \textsf{mips-irix}, las ejecuciones de
	\acro{MIPS}pro 7.4, son requeridas. 

\end{itemize*}

\subsubsection{2005}

El 2005 presenció un inmenso número de actualizaciones de
paquetes y programas. La estructura se mantuvo,
relativamente estable desde el 2004, pero inevitablemente
hubieron algunos cambios también:

\begin{itemize}

	\item  Los nuevos scripts \cmdname{texconfig-sys},
		\cmdname{updamp-sys}, y
		\cmdname{fmtutil-sys} fueron introducidos,
		lo cual modificó la configuración en los
		árboles del sistema. Los scripts
		\cmdname{texconfg}, \cmdname{updmap}, y
		\cmdname{fmtutil}, ahora modifican
		archivos específicos de usuarios, bajo el
		directorio
		\dirname{$HOME/.texlive2005}

	\item  Nuevas variables correspondientes a estos scripts como
		\envname{TEXMFCONFIG} y
		\envname{TEXMFSYSCONFIG} para especificar
		los árboles donde los archivos de
		configuración (del usuario o del sistema)
		fueron encontrados. Por lo tanto, versiones
		personales de \filename{fmtutil.cnf} y
		\filename{updmap.cfg} pueden moverse hacia
		estos lugares; otra opción es en redefinir
		\envname{TEXMFCONFIG} o
		\envname{TEXMFSYSCONFIG} en los archivos de
		\filename{texmf.cnf}. Cualquiera que sea el
		caso, la localización real de estos
		archivos y los valores de
		\envname{TEXMFCONFIG} y
		\envname{TEXMFSYSCONFIG} deben de
		corresponder apropiadamente. Vea la
		sección~\ref{sec:texmftrees}, \p.
		\pageref{sec:texmftrees}.

	\item El año pasado, primitivos indefinidos, como
		\verb|\pdfoutput| para la producción de
		\dvi\ a pesar que el programa
		\cmdname{pdfetex} estaba siendo usado. Este
		año, como prometimos, revertimos estas
		medidas de compatibilidad. De tal manera,
		que si su documento utiliza
		\verb|ifx\pdfoutput\undefined| para evaluar
		si el PDF está sido producido, esto
		necesitará cambiarse. Puede usar el
		\pkgname{ifpdf.sty} (el cual funciona en
		ambos, \TeX\ básico y \LaTeX) para hacer
		esto, o robarse su lógica. 

	\item El año pasado, cambiamos la mayoría de los
		formatos de resultados de caracteres
		(8-bit) como los mismos (vea la sección
		anterior). El nuevo archivo TCK
		\filename{empty.tcx} ahora provee una
		manera más fácil para obtener la original
		notación \verb|^^| si así se desea, como
		en:
		\begin{verbatim}
		latex --translate-file=vacío.tcx tuarchivo.tex
		\end{verbatim}

	\item El nuevo programa \cmdname{dvipdfmx} está
		incluido para traducción de DVI a PDF; esto
		es una actualización activamente mantenida
		de \cmdname{dvipdfm} (el cual está también
		disponible por ahora, pero no más
		recomendado).

	\item Los nuevos programas \cmdname{pdfopen}
		\cmdname{pdfclose} están incluidos para
		permitir la recarga de archivos pdf, en el
		Adobe Acrobat Reader, sin recomenzar el
		programa. (Otros lectores de pdf,
		notablemente \cmdname{xpdf}, \cmdname{gv},
		y \cmdname{gsview}, nunca han sufrido de
		este problema.)

	\item Para consistencia, las variables
		\envname{HOMETEXMF} y \envname{VARTEXMF}
		han sido renombradas a \envname{TEXMFHOME}
		y \envname{TEXMFSYSVAR}, respectivamente.
		También está \envname{TEXMFVAR}, el cual es
		por estándar, específico con el usuario.
		Vea el primer punto encima de esto para más
		información.
\end{itemize}

\subsubsection{2006--2007}

Durante el 2006--2007, la mayor adición a \TL{} fue el
programa \XeTeX{}, disponible como los programas
\texttt{xetex} y \texttt{xelatex}; vea 
\url{https://scripts.sil.org/xetex}.

Metapost también recibió una notable actualización, con más
planes para el futuro.
(\url{https://tug.org/metapost/articles}), de igual manera,
pdf\TeX{} (\url{https://tug.org/applications/pdftex}).

El archivo de \TeX\ \filename{.fmt} (formato de alta-velocidad), y
otras archivos similares para MetaPost y \MF\ están ahora
almacenados en los subdirectorios de \dirname{texmf/web2c},
en vez del directorio mismo (aunque el directorio esté
siendo buscado, por el hecho que algún \filename{.fmt}'s) exista en el
sistema. Los subdirectorios son nombrados por el
`motor' en uso, tal como \filename{tex} o \filename{pdftex}
o \filename{xetex}. Este cambio debe ser invisible en uso
normal. 

El programa de \texttt{tex} (básico) no lee ya más, las primeras
líneas \texttt{\%\&} para determinar el formato que ejecutar; esto es
divisado por el \TeX\ de Knuth. (\LaTeX\ y todo lo demás, aún leen las
líneas \texttt{\%\&}). 

Por supuesto que este año también presenció (como
usualmente) los cientos de actualizaciones a paquetes y
programas. Por favor de revisar
\acro{CTAN} (\url{https://mirror.ctan.org}) para
actualizaciones. 

Internamente, el árbol original es ahora almacenado en
Subversión, con una interfaz de web para ver el árbol, como
el enlace de nuestra página muestra. Aunque no está visible
en la distribución final, esperamos que esto seguirá
proveyendo una fundación estable en el desarrollo y
construcción en los años venideros. 

Finalmente, en mayo del 2006, Thomas Esser anunció que él
no continuaría con las actualizaciones de te\TeX{}
(\url{https://tug.org/tetex}). Y producto de esto, ha
existido un alto interés en \TL{}, especialmente entre
los distribuidores de \GNU/Linux. (Existe un nuevo esquema
de instalación para \texttt{tetex} en \TL{}, el cual provee
un aproximado equivalente.) Esperamos que esto
eventualmente se traduzca a mejorías para todos, en el
sistema \TeX.

\subsubsection{2008}

En el 2008, la infraestructura completa de \TL{} fue
rediseñada y re-implementada. Toda la información acerca de
la instalación está ahora guardada en un archivo de texto
regular \filename{tlpkg/texlive.tlpdb}. 

Entre otras cosas, esto finalmente posibilita actualizar
una instalación de \TL{} a través del Internet después de
una instalación inicial, una función que ha sido proveída
por MiK\TeX\ por años. Esperamos en actualizar regularmente
los paquetes, tan pronto como estos sean introducidos a
\CTAN. 

El nuevo y mayor motor de compilación Lua\TeX\
(\url{https://luatex.org}) está incluido; que aparte de
proveer un nuevo nivel de flexibilidad en la composición
tipográfica, es a su vez, un excelente lenguaje de
inscripción para uso adentro y afuera de documentos \TeX.

Ayuda técnica entre Windows y plataformas basadas en Unix,
es ahora mucho más uniforme. En particular, la mayoría de
los scripts de Perl y Lua están ahora disponibles en
Windows, usando el Perl que internamente se distribuye con
\TL. 

El nuevo script \cmdname{tlmgr} (section~\ref{sec:tlmgr})
es la interfaz general para administrar \TL{} después de
una instalación inicial. Manipula actualizaciones de
paquetes, y consiguientes regeneración de formatos,
regeneración de archivos de mapas, y archivos de lenguajes, opcionalmente
incluyendo adiciones locales. 

Con el advenimiento de \cmdname{tlmgr}, las acciones de
\cmdname{texconfig} para editar los archivos de
configuración de formateo, y de separación silábica, están
ahora desactivados. 

El programa de índice \cmdname{xindy}
(\url{https://xindy.sourceforge.net/}) está ahora incluido
en la mayoría de las plataformas. 

La herramienta \cmdname{kpsewhich}, puede reportar ahora
todas aquellos ingresos que concuerden de un archivo
(option \optname{--all}) y limita aquellos que coincidan en
un subdirectorio dado (option \optname{--subdir}). 

El programa \cmdname{dvipdfmx} ahora incluye nuevas
funciones para extraer la información del borde de las
cajas, a través de la línea de comando
\cmdname{extractbib}; esto fue una de las últimas funciones
proveídas por \cmdname{dvipdfm}, pero no en
\cmdname{dvipdfmx}. 

Los aliases de las fuentes \filename{Times-Roman}, y
\filename{Helvetica}, y muchos más, han sido
removidos. Diferentes paquetes esperan que funcionen
diferentemente (en particular, que tengan diferentes
codificaciones), y no hubo manera en resolver esto. 

El formato \pkgname{platex} ha sido removido, para resolver
un conflicto de nombre con un completamente diferente
\pkgname{platex}; el \pkgname{polski}, es ahora el de ayuda
principal en polaco. 

Internamente, los archivos de cadenas de datos, son ahora
compilados en los binarios, para facilitar las
actualizaciones. 

Finalmente, los cambios hechos por Donald Knuth en su
`afinación de \TeX\ del 2008', está incluida en esta
versión. Vea
\url{http://tug.org/TUGboat/Articles/tb29-2/tb92knut.pdf}.

\subsubsection{2009}

En el 2009, el formato de resultado estándar para
Lua\AllTeX\ es ahora PDF, para así tomar ventaja del apoyo
de Lua\TeX\ con OpenType, et al. Nuevos ejecutables
nombrados \code{dviluatex} y \code{dvilualatex} ejecutan
Lua\TeX\ con resultados DVI. 
La página de Lua\TeX\ es \url{http://luatex.org}.

El motor original Omega, y formato Lambda han sido
extraídos, después de discusiones con los autores de Omega.
El actualizado Aleph y Lamed permanecen, al igual que las
utilidades de Omega. 

Un nueva versión de las fuentes \TypeI\ de AMS, está incluida,
incluyendo Computer Modern: con unos cuantos cambios en la
configuración de estos, que a través de los años, han sido
llevado a cabo por Knuth, con los archivos originales de
Metafont, han sido integrados y actualizados. Las fuentes
Euler han sido completamente reconfiguradas por Hermann Zapf
(vea
\url{http://tug.org/TUGboat/Articles/tb29-2/tb92hagen-euler.pdf}).
En todos los casos, los métricos no han cambiado. La página en
el Internet de las fuentes de AMS se encuentra en
\url{https://www.ams.org/tex/amsfonts.html}.

El nuevo editor con GUI \TeX{}works está incluido para
Windows, y también en Mac\TeX. Para otras plataformas, y para
más información, vea la página en el Web de \TeX{}works,
\url{https://tug.org/texworks}. Este editor es
multi-plataforma, inspirado por TeXShop para \MacOSX{} con el
objetivo en ser fácil de usar. 

El programa de gráficas Asymptote está incluido para varias
plataformas. Esto implementa un lenguaje de descripción de
lenguajes, basado en texto, con cierto parecido con Metapost,
pero con apoyo para un avanzado formato de tres dimensiones, y
otras funciones. Su página en el Web está en
\url{https://asymptote.sourceforge.net}.

El programa por separado, \code{dvipdfm}, ha sido reemplazado por
\code{dvipdfmx}, el cual opera en un modo de compatibilidad
especial, bajo ese nombre. \code{dvipdfmx} incluye apoyo para
\acro{CJK} y tiene muchos otros arreglos que se acumularon,
a través de los años, desde la última versión de
\code{dvipdfm}. La página principal de DVIPDFMx está en
\url{http://project.ktug.or.kr/dvipdfmx}.

Ejecutables para \pkgname{cygwin} y plataformas
\pkgname{i386-netbsd}, están ahora incluidos, aunque se nos
fue aconsejado que muchos usuarios de Open\acro{BSD} obtienen
\TeX\ a través de los paquetes de sus sistemas, y encima de esto,
habían dificultades en hacer que los binarios funcionaran en
más de una versión. 

Una miscelánea de pequeños cambios: ahora usamos comprensión
\pkgname{xz}, el estable programa que reemplaza a
\pkgname{lzma} (\url{https://tukaani.org/xz/}); el carácter
literal |$| 
es permitido en los nombres de archivos, cuando no introduce
un nombre de una variable conocida; la librería Kpathsea es
ahora multi-hilo, multi-procesos, (hizo uso de esto, en Metapost); y la edificación
entera de \TL{} está basada ahora en Automake. 

Nota final del pasado: todas las versiones de \TL{},
acompañado de material auxiliar como las etiquetas de los \CD,
están ahora disponibles en
\url{ftp://tug.org/historic/systems/texlive}.

\subsubsection{2010}
\label{sec:2010news} % manténgase al tanto con 2010

En el 2010, la versión estándar para los resultados en PDF es
ahora 1.5, permitiendo más compresión. Esto aplica también a
todos los motores que fueron usados para producir PDF y
\code{dvipdfmx}. Cargando los paquetes \pkgname{pdf14} de
\LaTeX\ revierte a PDF-1.4, o set |\pdfminorversion=4|. 

Pdf\AllTeX\ ahora \emph{automáticamente} convierte un archivo
encapsulado de PostScript (EPS) a un PDF, mediante la vía de
del paquete \pkgname{epstopdf}, siempre y cuando el archivo de
configuración \code{graphics.cfg} de \LaTeX\ esté cargado, y
si el formato PDF es seleccionado como estándar en la salida. Las
opciones estándar, tienen como objetivo, eliminar cualquier
chance de que los archivos que hayan sido creados a mano,
puedan ser sobrescritos, pero puedes también prevenir que por
ejemplo un \code{epstopdf} se cargue por completo, mediante el
ingreso de |\newcommand{\DoNotLoadEpstopdf}{}| (o |\def...|)
antes de la declaración \cs{documentclass}. Este archivo
tampoco es cargado, si \pkgname{pst-pdf} es utilizado. Para
más detalles, vea la documentación del paquete
\pkgname{epstopdf} (\url{https://ctan.org/pkg/epstopdf-pkg}).

Un cambio relacionado con esto, es la ejecución de unos
cuantos comandos externos de \TeX, mediante la vía de la
función \cs{write18}, que está ahora activado por estándar.
Estos comandos son \code{repstopdf}, \code{makeindex},
\code{kpsewhich}, \code{bibtex}, y \code{bibtex8}; la lista
está definida en \code{texmf.cnf}. En aquellos sistemas, que
tienen que prohibir todos estos comandos externos, esta opción
se puede desactivar, en el instalador (vea la
sección~\ref{sec:options}), o sobrescribe el valor después de
la instalación mediante la ejecución de 
|tlmgr conf texmf shell escape 0|.

Y aún otro cambio fue en \BibTeX\ y Makeindex, que ahora rehúsan
producir los archivos en un directorio arbitrario (como el
mismo \TeX), como valor estándar. Esto es así, porque ahora
su uso puede ser activado, mediante el restricto \cs{write18}.
Para cambiar esto, la variable del entorno del sistema
\envname{TEXMFOUTPUT} puede ser fijada, o la configuración
|openout_any| modificada. 

\XeTeX\ ahora respalda el ajuste de margen de un bloque de
texto, de la misma manera que pdf\TeX. (La expansión de la
fuente, no es actualmente respaldado.)

Por configuración estándar, \prog{tlmgr} ahora guarda una
copia automáticamente de cada paquete actualizado (\code{tlmgr
option autobackup 1}), de esta manera, las actualizaciones que
hayan tenido rupturas, puedan ser fácilmente revertidas con
\code{tlmgr restore}. Si usted realiza actualizaciones,
posteriores a la instalación, y no cuenta con el espacio de
almacenamiento en el disco para estas copias de seguridad,
ejecute \code{tlmgr option autobackup 0}. 

Nuevos programas incluyen: el motor p\TeX\ y utilidades
relacionadas con el mismo para composición tipográfica en
japonés; el programa \BibTeX{}U para activación de Unicode en
\BibTeX; la utilidad \prog{chxtex}
(\url{https://baruch.ev-en.org/proj/chktex}) para la revisión
de los documentos \AllTeX{}; el programa \pkgname{dvisvgm}
(\url{https://dvisvgm.sourceforge.net}) que es un traductor
DVI-a-SVG. 

Ejecutables para estas nuevas plataformas están ahora
incluidas: \code{amd64-freebsd}, \code{amd64-kfreebsd},
\code{i386-freebsd}, \code{i386-kfreebsd},
\code{x86\_64-darwin}, \code{x86\_64-solaris}.

Un cambio en \TL{} 2009 que fallamos en mencionar: numerosos
ejecutables relacionados con \TeX4ht\
(\url{https://tug.org/tex4ht}), fueron removidos de los
directorios de los binarios. El programa genérico \code{mk4ht}
puede ser utilizado para ejecutar cualquiera de las
combinaciones de \code{tex4ht}.

Finalmente, la versión de \TL{} en el \TK\ \DVD\ no puede ser
más ejecutada en vivo (extrañamente curioso). Un \DVD\
sencillo, ya no cuenta con más espacio. Un beneficioso efecto
secundario de esto, es que la instalación directa de un \DVD\
como tal, es mucho más rápida.

\subsubsection{2011}

Los binarios para \MacOSX\ (\code{universal-darwin} y
\code{x86\_64-darwin}) ahora solamente funcionan en Leopard y
versiones recientes; Panther y Tiger, no tienen compatibilidad. 

\code{Biber}, el programa para procesamiento de bibliografías
está incluido en plataformas comunes. Su construcción está
estrechamente acompañada por el paquete \code{biblatex}, el cual
re-implementa completamente las facilidades bibliográficas
proveídas por LaTeX.

El programa (\code{mpost}) ya no crea ni utiliza más, los
archivos \code{.mem}. Los archivos necesarios, tales como
\code{plain.mp}, son simplemente leídos durante la ejecución. 
Esto es relacionado, con el apoyo a Metapost como una
librería, que es un significante avance, y que no es un cambio
visible al usuario. 

La implementación \code{updmap} en Perl, previamente utilizada
solamente en Windows, ha sido reconfigurada y ahora está
siendo utilizada en todas las plataformas. No debe haber
ningún cambio visual para el usuario, como resultado de esto,
excepto que opera mucho más rápido. 

Los programas \cmdname{initex} e \cmdname{initex} fueron
restaurados, (pero sin otras variantes \cmdname{ini*}). 

\subsubsection{2012}

\code{tlmgr} respalda actualizaciones de múltiples repositorios
en la red. La sección sobre múltiples repositorios en los
resultados que provee \code{tlmgr help} tiene más de estos.

El parámetro \cs{XeTeXdashbreakstate} está fijado a 1 como
estándar, para ambos \code{xetex} y \code{xelatex}. Esto permite
saltos de líneas, después de guiones cortos y largos,
lo cual siempre ha sido el funcionamiento de \TeX\
básico,
\LaTeX, Lua\TeX, etc. Aquellos documentos existentes de \XeTeX\
los cuales deben retener la compatibilidad perfecta de saltos 
de líneas, necesitarán configurar \cs{XeTeXdashbreakstate} a-0
explícitamente. 

Los archivos de salida, que son generados por \code{pdftex} y
\code{dvips}, pueden exceder ahora, hasta 2gb. 

Las 35 fuentes estándar de PostScript, están incluidas en los
resultados de \code{dvips}, por norma o estándar, debido a que numerosas
versiones de estas, están ahora presentes. 

En el modo restricto de ejecución \cs{write18}, configurado como
estándar, el programa \code{mpost} es ahora permitido. 

Un archivo \code{texmf.cnf} es también encontrado en
\filename{../texmf-local}, e.g.,
\filename{/usr/local/texlive/texmf-local/web2c/texmf.cnf}, si el
mismo existe. 

El script \code{updamp} lee \code{updmap.cfg} por cada árbol en
vez de una configuración global. Este cambio, debe ser
invisible, a menos que updmap.cfg's hayan sido editados
directamente. Puedes ver más sobre estos, invocando 
\verb|updmap --help|. 

Las plataformas: \pkgname{armel-linux} y \pkgname{mipsel-linux}
fueron añadidas; \pkgname{sparc-linux} y \pkgname{i386-netbsd}
no se encuentran más en la distribución. 

\subsubsection{2013}
La organización de la distribución: el directorio en alto-nivel
\code{texmf/} ha sido unido en \code{texmf-dist/}, para
simplicidad. Ambas variables de Kpathsea \code{TEXMFMAIN}, y
\code{TEXMFDIST}, ahora apuntan hacia \code{texmf-dist}. 

Muchas colecciones pequeñas de lenguajes han sido unidas, para
simplificar la instalación. 

\MP: respaldo nativo para la producción de PNG y
puntos-flotantes (IEEE doble) han sido añadidos. 

Lua\TeX: se actualizó con Lua 5.2, y ahora incluye una nueva
librería (\code{pdfscanner}) para procesar contenido PDF
externo, entre muchas cosas más (vea el sitio en el Internet).

\XeTeX\ (también vea sus páginas para más):
\begin{itemize}
	\item La librería HarfBuzz es usada para la
		organización de las fuentes, en vez de ICU. (ICU
		es aún usada para el respaldo de codificación de
		archivos de ingreso, bi-dirección, y el opcional
		salto de línea de Unicode.)
	\item Para diseño de Grafito, Graphite2 y HarfBuzz son
		ahora utilizados. 
	\item En Macs, Core Text es usado en vez del (deprecado)
		ATSUI.
	\item La preferencia de fuentes TrueType/OpenType en vez
		de Type1, cuando los nombres son los mismos.
	\item El arreglo de búsquedas que no coincidan, entre
		\XeTeX\ y \code{xdvipdfmx}. 
	\item Apoyo de fuentes para matemática de OpenType.
\end{itemize}

\cmdname{xdvi}: ahora utiliza las fuentes FreeType en
vez de \code{t1lib} para la visualización.

El \pkgname{microtype.sty}: ahora respalda \XeTeX\
(protuberancia) y Lua\TeX\ (protuberancia, expansión de
fuente, rastreo), entre otras mejorías. 

\cmdname{tlmgr}: nueva acción \code{pinning} para
facilitar la configuración de múltiples repositorios;
esa sección en \verb|tmlgr --help| tiene más, y en el
Internet en
\url{https://tug.org/texlive/doc/tlmgr.html#MULTIPLE-REPOSITORIES}.

Plataformas: \pkgname{armhf-linux}, \pkgname{mips-irix},
\pkgname{i386-netbsd}, y \pkgname{amd64-netbsd}
añadido o revivido; \pkgname{powerpc-aix} removido.

\subsubsection{2014}

El 2014, presenció otra afinación de \TeX\ por parte de
Knuth; esto afectó todos los motores, pero el único
cambio visible, fue la restauración de la cadena
\code{preloaded format} en la línea de banner. De
acuerdo a Knuth, esto ahora refleja el formato que sería
cargado por norma, en vez de un formato que no haya
sido desechado y que esté pre-cargado en el binario;
puede ser rescrito de varias maneras.

pdf\TeX: nuevo parámetro de supresión-de-advertencia
\cs{pdfsuppresswarningpagegroup}; nuevos primitivos para
espacios falsos entre-palabras, que ayuden con 
el reflujo del texto: \cs{pdfinterwordspaceon},
\cs{pdfinterwordspaceoff}, \cs{pdffakespace}.

Lua\TeX: Modificaciones notables y arreglos fueron
realizados con la carga de fuentes, y la separación
silábica. La adición más grande es una nueva variante de
motor de compilación, \code{luajittex}
(\url{http://foundry.supelec.fr/projects/luajittex}) y
sus hermanos \code{texluajit} y \code{texluajitc}. Esto
utiliza, justo-a-tiempo, un compilador Lua (artículo
detallado de TUGboat en
\url{https://tug.org/TUGboat/tb34-1/tb106scarso.pdf}).
\code{luajittex} está aún en construcción, no está
disponible en todas las plataformas, y es
considerablemente menos estable que \code{luatex}. Ni
nosotros, ni sus constructores recomiendan utilizarlo,
excepto para propósitos específicos de experimentación con
jit en el código Lua. 

\XeTeX: Las mismas imágenes de formato, son ahora
respaldadas en todas las plataformas (incluyendo Mac);
evitar el retorno de descomposición de compatibilidad
con Unicode (pero no todas las variantes); la
preferencia de OpenType antes de fuentes de Graphite,
para compatibilidad con versiones previas de \XeTeX.

\MP: Un nuevo sistema-numérico \code{decimal} es
respaldado, acompañado de una opción interna
\code{numberprecision}; una nueva definición de
\code{drawout} en el archivo \filename{plain.mp}, por
Knuth; reparaciones de errores, depuraciones, en archivos
producidos \acro{SVG}, y \acro{PNG}, entre otros.

La utilidad \cmdname{pstopdf} en Con\TeX{}t, será
removida como comando solitario, después de esta
versión, debido a conflictos con las utilidades de OS
con el mismo nombre. Puede ser (aún) invocada como 
\code{mtxrun --script pstopdf}.

\cmdname{psutils} han sido substancialmente revisadas
por un nuevo administrador. Como resultado de ello,
diferentes utilidades, raramente utilizadas, (\code{fix*},
\code{getafm}, \code{psmerge}, \code{showchar}) están
ahora solamente en el directorio \dirname{scripts/}, en
vez de ser ejecutables a nivel-de-usuario (esto puede
ser revertido, si resulta ser problemático). Un nuevo
script, \code{psjoin}, ha sido añadido.

La redistribución de Mac\TeX\ de \TeX\ Live
(section~\ref{sec:macosx}) no incluye más los paquetes
opcionales basados en Mac solamente, para las fuentes de
Latin Modern, ni \TeX\ Gyre, por la razón que es
bastante fácil para que los usuarios individualmente,
los seleccionen para el sistema. El programa
\cmdname{convert} de ImageMagick también ha sido
removido, debido a que \TeX4ht (específicamente
\code{tex4ht.env}) ahora utiliza Ghostscript
directamente.  

La colección \pkgname{langcjk} para el soporte de
idiomas chino, japonés, y coreano, ha sido dividida en
colecciones individuales, con tamaños más moderados. 

Plataformas: \pkgname{x86\_64-cygwin} fue añadida,
\pkgname{mips-irix} fue removido; Microsoft no ofrece ayuda
más para Windows XP, y por esa razón, nuestros
programas pueden comenzar a fallar allí en cualquier
momento, sin previo aviso. 

Plataformas: \pkgname{*-kfreebsd} removido, debido a que \TeX\ Live está
ahora disponible fácilmente a través de los mecanismos de la plataforma
del sistema.  Ayuda para plataformas adicionales están disponibles como
binarios personalizados (\url{https://tug.org/texlive/custom-bin.html}).
En adición a esto, algunas de las plataformas se han omitido del \DVD\
(simplemente para ahorrar espacio), pero pueden ser instaladas
normalmente a través del Internet.

\subsubsection{2015}

\LaTeXe\ ahora incorpora, por norma estándar, cambios previamente
incluidos solamente cuando se cargaba el paquete \pkgname{fixltx2e}, el
cual es ahora un no-op. Un nuevo paquete \pkgname{latexrelease} y otros
mecanismos permiten controlar lo que se hace. Los documentos incluidos
\LaTeX\ News \#22 y ``cambios en \LaTeX'' tienen más detalles acerca de
esto. Incidentalmente, los paquetes \pkgname{babel} y \pkgname{psnfss},
aunque son partes principales de \LaTeX, están siendo mantenidos
separados, y no están afectados por estos cambios (y deben trabajar
aún).

Internamente, \LaTeXe\ ahora incluye configuración del motor relacionada
con Unicode (caracteres que son letras, nombres de primitivos, etc.) el
cual era anteriormente parte de \TeX\ Live. Este cambio será invisible
para los usuarios; unas cuantas secuencias internas de control de
bajo-nivel, han sido renombradas o removidas, pero la funcionalidad debe
ser la misma. 

Pdf\TeX: apoyo técnico con \acro{JPEG} Exif, al igual que \acro{JFIF},
no emiten ni incluso una advertencia si \cs{pdfinclusionerrorlevel} es
negativo; sync con \prog{xpdf}~3.04.

Lua\TeX: Nueva librería \pkgname{newtokenlib} para escanear fichas o tokens;
depuración en el generador de números al azar \code{normal}, y en otros
lugares. 

\XeTeX: Arreglos de manipulación de imágenes; el binario de
\prog{xdvipdfm} es buscado primero como hermano para \prog{xetex}; los
opcodes o códigos operativos internos de \code{XDV} fueron modificados.

Metapost: Nuevo sistema numérico \code{binario}; nuevos programas
\pkgname{upmpost} y \pkgname{updvitomp} con japonés-activado, análogo a
\prog{up-tex}.

Mac\TeX: Actualizaciones al paquete incluido GhostScript para respaldo
de \acro{CJK}. El Panel de Configuración de la Distribución \TeX\ ahora
trabaja en Yosemite (\MacOSX-10.10). Portafolios de recursos de
bifurcación para las fuentes (generalmente sin una extensión) no son más
respaldados por \XeTeX; portafolios de bifurcación de datos
(\code{.dfont}) permanecen con respaldo. 

Infraestructuras: El script \prog{fmtutil} ha sido re-implementado para
leer \filename{fmtutil.cnf} en una base por-árbol, análogo a
\pkgname{updmap}. Programas scripts de Web2C \pkgname{mktex*}
(incluyendo \prog{mktexlsr}, \prog{mktextfm}, \prog{mktexpk}) ahora
prefieren programas en sus propios directorios, en vez de utilizar la
existente ruta (\envname{PATH}). 

Plataformas: \pkgname{*-kfreebsd} fue removido, debido a que \TeX\
Live está ahora disponible fácilmente a través de los mecanismos de la
plataforma del sistema.  Ayuda para plataformas adicionales están
disponibles como binarios personalizados
(\url{https://tug.org/texlive/custom-bin.html}).  En adición a esto,
algunas de las plataformas se han omitido del \DVD\ (simplemente para
ahorrar espacio), pero pueden ser instaladas normalmente a través del
Internet.  

% 
\subsubsection{2016}

Lua\TeX: Cambios radicales de los primitivos, tanto en renombres como
en eliminaciones de estos, al igual que la reorganización de la
estructura de nodos. Los cambios están resumidos en un artículo por
Hans Hagen, ``Lua\TeX\ 0.90 backend changes for PDF and more''
(\url{https://tug.org/TUGboat/tb37-1/tb115hagen-pdf.pdf}); para todos
los detalles acerca de esto, vea el manual de Lua\TeX,
\OnCD{texmf-dist/doc/luatex/base/luatex.pdf}.

Metafont: Nuevos programas parientes altamente experimentales MFlua y
MFluajit, que integran Lua con \MF, con propósitos de evaluación.

Metapost: reparaciones de errores, depuraciones, y preparaciones
internas para MetaPost 2.0.

\code{SOURCE\_DATE\_EPOCH} respaldo en todos los motores
excepto para Lua\TeX\ (el cual estará disponible para el
próximo lanzamiento) y el código original \code{tex}
(que fue intencionalmente omitido): si la variable del entorno
del sistema está configurada y fijada, este valor es utilizado
para las fechas, en los resultados del PDF. Si el código
\code{SOURCE\_DATE\_EPOCH\_TEX\_PRIMITIVES} está configurado y
establecido, entonces el valor del código
\code{SOURCE\_DATE\_EPOCH} es utilizado para la inicialización
de los primitivos de \TeX\ \cs{year}, \cs{month}, \cs{day},
y \cs{time}. El manual de pdf\TeX\ tiene ejemplos y detalles. 

pdf\TeX: nuevos primitivos \cs{pdfinfoomitdate},
\cs{pdftrailerid}, y \cs{pdfsuppressptexinfo} que controlan
los valores que aparecen en la salida de datos, que
normalmente se modifican con cada ejecución. Estas funciones
son para los resultados de salida del PDF, no del DVI.

Xe\TeX: Nuevos primitivos \cs{XeTeXhyphenatablelength},
\cs{XeTeXgenerateactualtext},\\ \cs{XeTeXinterwordspaceshaping},
\cs{mdfivesum}; el límite de caracteres de clases se incrementó a
4096; y el id de la unidad de octeto o byte incrementó. 

Otras utilidades:
\begin{itemize}
\item \code{gregorio} es un nuevo programa, parte del paquete
\code{gregoriotex} que es para la composición tipográfica de notaciones
musicales de cantos Gregorianos. Está incluido
estándar en \code{shell\_escape\_commands}. 

\item \code{upmendex} es un programa de creación de índices,
mayormente compatible con \code{makeindex}, con respaldo para
la clasificación de Unicode; entre otros cambios. 

\item \code{afm2tfm} hace ahora ajustes de altura hacia arriba
de los métricos de las fuentes que están basadas en acentos; una
nueva opción \code{-a} omite todos estos ajustes. 

\item \code{ps2pk} puede manipular las fuentes extendidas PK/GF.
\end{itemize}

Mac\TeX: El Panel de Configuración de la Distribución \TeX\ ya no
existe; su funcionalidad ahora se encuentra en TeX Live Utility;
conjunto de aplicaciones GUI (Interfaz Gráfica de Usuario)
actualizadas; un nuevo script \code{cjk-gs-integrate} puede ser
ejecutado por aquellos usuarios que desean incorporar varias fuentes
CJK en el visualizador Ghostscript. 

Infraestructura: Respaldo y apoyo para el fichero de configuración a
nivel-del-sistema de \code{tlmgr}; verificación de las sumas de
control de los paquetes; si el GPG está disponible, verifique la
firma, la credencial de las actualizaciones en la red. Estos
chequeos ocurren mediante ambos tanto en el instalador como en el
programa \TL\ Manager \code{tlmgr} (Si el GPG no está
disponible, las actualizaciones proceden como es usual.)

Plataformas: \code{alpha-linux} y \code{mipsel-linux} fueron
removidas. 

\subsubsection{2017}

Lua\TeX: Más funciones de devolución de llamadas o funciones de
callback, más control en la composición tipográfica, más acceso a los
comandos internos; librería \code{ffi} para cargar código dinámico fue
añadida en algunas plataformas. 

pdf\TeX: Variables de entorno |SOURCE_DATE_EPOCH_TEX_PRIMITIVES| del
año pasado fueron renombrabas a |FORCE_SOURCE_DATE|, sin cambios en la
funcionalidad; la lista de fichas o tokens de \cs{pdfpageattr} que
contiene la cadena del código \code{/MediaBox}, omite la salida del
código predeterminado de \code{/MediaBox}. 

Xe\TeX: Matemática de Unicode/Opentype está ahora basada en el
respaldo de la tabla de matemáticas de HarfBuzz; y algunas
correcciones de errores. 

Dvips: implementación que el último papersize especial o el último
especial del tamaño del papel gane, cuando múltiples dimensiones son
especificadas, para consistencia con el programa \code{dvipdfmx} y
expectativas del paquete; la opción \code{-LO} (\code{LO} config
setting) restaura la funcionalidad anterior del primer especial
ganador. 

ep\TeX, eup\TeX: Nuevos primitivos \cs{pdfuniformdeviate},
\cs{pdfnormaldeviate}, \cs{pdfrandomseed}, \cs{pdfsetrandomseed},
\cs{pdfelapsedtime}, \cs{pdfresettimer}, de pdf\TeX.

Mac\TeX: Empezando este año, solamente aquellas versiones de \MacOSX\
para las cuales Apple aún lanza parches de seguridad, serán
respaldadas en Mac\TeX, bajo la plataforma de nombre |x86_64-darwin|;
actualmente esto quiere decir Yosemite, El-Capitan, y Sierra (10.10 y
más recientes). Los binarios para versiones anteriores de \MacOSX\ no
están incluidas en Mac\TeX, pero están aún disponibles en \TeX\
Live (|x86_64-darwinlegacy|, \code{i386-darwin}, \code{powerpc-darwin}). 

Infraestructura: El árbol \envname{TEXMFLOCAL} es ahora buscado antes
de \envname{TEXMFSYSCONFIG} y \envname{TEXMFSYSVAR} (por norma
estándar); se espera y se tiene la esperanza que esto coincida mejor
con las expectativas de los archivos locales que sobrescriben los
archivos del sistema. También, \code{tlmgr} tiene un modo nuevo
\code{shell} para uso interactivo y de programación, y una nueva
acción \code{conf auxtrees} que fácilmente añade y remueve árboles
extra. 

\code{updmap} y \code{fmtutil}: Estos scripts ahora avisan e informan
con una advertencia cuando son invocados sin explícitamente haber
especificado tanto el así llamado modo (\code{updmap-sys},
\code{fmtutil-sys}, o la opción \code{-sys}), o en el modo de usuario
(\code{updmap-user}, \code{fmtutil-user}, o la opción \code{-user}).
Se espera que esto reducirá el problema perenne en invocar el modo de
usuario por accidente y que por consiguiente se pierdan las
actualizaciones futuras del sistema. Vea
\url{https://tug.org/texlive/scripts-sys-user.html} para más detalles. 

\code{install-tl}: acceso de rutas personales son ahora configuradas bajo los valores de Mac\TeX\ (|~/Library/...|)\, predeterminados en Macs. Nueva opción \code{-init-from-profile} para comenzar una instalación con los valores de un perfil dado; un nuevo comando \code{P} para que explícitamente guarde un perfil; nuevos nombres de variable de perfil (pero los anteriores son aún aceptados). 

Sync\TeX: el nombre temporal de un archivo aparece ahora como
\code{foo.synctex(busy)}, en vez de \code{foo.synctex.gz(busy)}
(no~\code{.gz}). Compiladores y sistemas de edificación que quieren
remover los archivos temporales pueden necesitar ajustes.  

Otras utilidades: \code{texosquery-jre8} es un nuevo programa
multi-plataforma que extrae información del sistema operativo como la
configuración de la localización, entre otros aspectos, de un
documento \TeX\ y está incluido estándar en el intérprete de la línea
de órdenes o comandos |shell_escape_commands| para la ejecución restricta del
shell. (Viejas versiones de JRE están respaldadas por texosquery, pero
no pueden ser habilitadas en modo restricto, debido a que no son
respaldadas por Oracle, incluso para problemas de seguridad.)

Plataformas: Vea la anotación acerca de Mac\TeX\ encima; no hay otros
cambios. 


\subsubsection{2018}

Kpathsea: Se ignora las mayúsculas con todos los nombres de archivos que
coincidan, es ahora el estándar en los directorios que no son parte
del sistema; establezca un \code{texmf.cnf} o la variable del entorno
del sistema \code{texmf\_casefold\_search} con el valor \code{0} para
desactivarlo. 
Detalles completos se encuentran en el manual de Kpathsea
(\url{https://tug.org/kpathsea}). 

ep\TeX, eup\TeX: Nuevo primitivo \cs{epTeXversion}.

Lua\TeX: Preparación para el lanzamiento de Lua 5.3 en el 2019: el
binario \code{luatex53} está disponible en la mayoría de las
plataformas, pero debe ser renombrado \code{luatex} para que sea
efectivo. O utilice los ficheros de \ConTeXt\ Garden
(\url{https://wiki.contextgarden.net}); más información acerca de esto
se puede encontrar allí.

MetaPost: Arreglos con direcciones equivocadas en la ruta de acceso.
Salidas de TFM y de PNG.

pdf\TeX: Permite vectores de codificación para las fuentes de bitmap;
el actual directorio no está configurado con un hash en el PDF ID;
arreglos de depuración para \cs{pdfprimitive} y todo lo relacionado
con el mismo.

Xe\TeX: Respaldo para \code{Rotate} en la inclusión de imagen en PDF;
escape con valores sin ceros si los resultados de la salida del driver
fallan; varios arreglos con UTF-8 y otros primitivos.

Mac\TeX: Vea más abajo, los cambios en el respaldo de la versión.
Adicionalmente, los ficheros instalados en \code{Applications/TeX/} a
través de Mac\TeX\ han sido reorganizados para tener mejor
clarificación; ahora esta localización contiene cuatro programas de
interfaz gráfica GUI (BibDesk, LaTeXiT, TeX Live Utility, y TeXShop)
en el nivel más alto, y directorios con utilidades adicionales y
documentación.

\code{tlmgr}: nuevos instaladores \code{tlshell} (Tcl/Tk) y
\code{tlcockpit} (Java); resultados de salida JSON; \code{uninstall}
tiene ahora un sinónimo para \code{remove}; nueva acción/opción
\code{print-platform-info}.

Plataformas:
\begin{itemize*}
        \item Nuevo: \code{x86\_64-linuxmusl} y \code{aarch64-linux}
		Removido: \code{armel-linux}, y \code{powerpc-linux}

        \item \code{x86\_64-darwin} respalda 10.10--10.13 (Yosemite,
            El-Capitan, Sierra, y High-Sierra).

        \item \code{x86\_64-darwinlegacy} respalda 10.6--10.10 (aunque
            \code{x86\_64-darwin} es preferido por 10.10). Todo el
            respaldo para 10.5 (leopard) ha cesado, esto quiere decir,
            que ambos \code{powerpc-darwin} y \code{i386-darwin
            platforms} han sido removidos.

	\item Windows: XP ya no es respaldado.
\end{itemize*}

\subsubsection{2019}

Kpathsea: Una más consistente expansión de llaves y
separación de ruta; nueva variable \code{TEXMFDOTDIR} en vez
del código forzado \code{.}\ en las rutas de acceso permite
realizar búsquedas más fácil o en los subdirectorios (vea
los comentarios en el archivo \code{texmf.cnf}.)

ep\TeX eup\TeX;  Nuevos primitivos \cs{readpapersizespecial}
y \cs{expanded}.

Lua\TeX: Lua 5.3 es ahora utilizada con una aritmética
concomitante con cambios en la interfaz. La creciente librería
pplib es ahora utilizada para leer los archivos pdf, y de
esa manera se elimina la dependencia en poppler (y la
necesidad de C++); la interfaz de Lua se modificó
respectivamente.

MetaPost: El nombre del comando \code{r-mpost} es reconocido
como un alias para la invocación de la opción
\code{--restricted}\, y es añadido a la lista de comandos
restrictos que son el estándar y que están disponibles.
Precisión mínima consiste de dos números para el modo
binario y el modo decimal.
El modo binario ya no está disponible en MPLib pero aún se
encuentra en MetaPost por sí solo.

pdf\TeX: Nuevos primitivos \cs{expanded}; si el parámetro 
\cs{pdfomitcharset} es especificado con 1, la cadena
\code{CharSet} es omitido en los resultados de la salida de
PDF, debido a que no se puede garantizar que sea correcto o
no, y tal como es requerido por PDF/A-2 y PDF/A-3.


Xe\TeX: Nuevos primitivos \cs{expanded},
\cs{creationdate},
\cs{elapsedtime},
\cs{filedump}, 
\cs{filemoddate}, 
\cs{filesize}, 
\cs{resettimer}, 
\cs{normaldeviate}, 
\cs{uniformdeviate}, 
\cs{randomseed}; extiende \cs{Ucharcat}  a producir
caracteres activos. 

\code{tlmgr}: Respaldo de \code{curl} como un programa para
descargas; utilice \code{lz4} y gzip antes de utilizar
\code{xz} para realizar copias locales de reserva, siempre
que esté
disponible; y a menos que la variable de entorno del sistema \code{TEXLIVE\_PREFER\_OWN} esté configurada, \code{tlmgr} 
prefiere utilizar los binarios-del-sistema antes que los
binarios que son proveídos por \TL\ para programas de
compresión y descargas.

\code{install-tl}: Nueva opción \code{-gui} (sin argumento)
es el estándar en Windows y Macs, e invoca una nueva
Interfaz Gráfica de Usuario Tcl/TK GUI.
(vea ~\ref{sec:basic} and~\ref{sec:graphical-inst}.) 

Utilidades:
\begin{itemize*}
	\item \code{cwebbin}
		(\url{https://ctan.org/pkg/cwebbin}) es
		ahora la implementación de CWEB en \TeX\
		Live, con respaldo para más dialectos de
		lenguajes, que incluye el programa
		\code{ctwill} para compilar mini-índices.

	\item \code{chkdvifont}: reporta información de la
		fuente a través de los archivos \dvi{}, al
		igual que de tfm/ofm, vf, gf, pk.

	\item \code{dvispo}: compila un archivo DVI de
		página-independiente con respecto a los
		especiales.
\end{itemize*}

Mac\TeX: \code{x86\_64-darwin} ahora respalda versiones
10.12 y más reciente (Sierra, High Sierra, Mojave);
\code{x86\_64-darwinlegacy} aún respalda versiones 10.6 y
otras más nuevas. El corrector ortográfico Excalibur ya no
está incluido, debido a que requiere respaldo de 32-bit.

Plataformas: \code{sparc-solaris} fue removido.

%
\subsubsection{2020}
\label{sec:tlcurrent}

General: 
\begin{itemize}
	\item El primitivo \cs{input} en todos los motores, incluyendo \texttt{tex}, ahora acepta un argumento de archivo de grupo-delimitado como extensión dependiente del sistema. El uso con un archivo de espacio/token-delimitado es completamente inalterable. El argumento de grupo-delimitado fue implementado anteriormente en Lua\TeX; ahora disponible en todos los motores. ASCII caracteres de comillas (\texttt{"}) han sido removidos de los nombres de ficheros, pero está no obstante inalterado después de la tokenización de las fichas. Esto no afecta actualmente la orden de la línea de comando \cs{input} de \LaTeX, debido a que este es un macro redefinido del primitivo estándar \cs{input}.

	\item Una nueva opción \texttt{--cnf-line} para \texttt{kpsewhich}, \texttt{tex}, \texttt{mf}, y todos los otros motores, que respaldan las preferencias de configuración en la línea de intérprete de comandos. 

	\item La adición de varios primitivos con varios motores en este año y los años anteriores, ha sido llevado a cabo con la intención que resulte en un conjunto común de funcionalidad a través de todos los motores. (\textsl{Noticias de \LaTeX\ \#31}, 
\url{https://latex-project.org/news}).
\end{itemize}

ep\TeX, eup\TeX: Nuevos primitivos \cs{Uchar}, \cs{Ucharcat},
\cs{current(x)spacingmode}, \cs{ifincsname}; revisión de \cs{fontchar??} y
\cs{iffontchar}. Para eup\TeX\ solamente: \cs{currentcjktoken}.

Lua\TeX: Integración con la librería de HarfBuzz, disponible como nuevos motores
\texttt{luahbtex} (utilizado por \texttt{lualatex}) y \texttt{luajithbtex}.
Nuevos primitivos: \cs{eTeXgluestretch}, \cs{eTeXglueshrink},
\cs{eTeXglueorder}.

p\TeX: Nuevos primitivos \cs{ifjfont}, \cs{iftfont}. También en ep\TeX,
up\TeX, eup\TeX.

Xe\TeX: Reparaciones para \cs{Umathchardef}, \cs{XeTeXinterchartoks}, \cs{pdfsavepos}.

Dvips: Codificaciones de salida para las fuentes de bitmap, que ofrecen mejores capacidades de copiar/pegar. 
(\url{https://tug.org/TUGboat/tb40-2/tb125rokicki-type3search.pdf}).

Mac\TeX: Mac\TeX\ y \texttt{x86\_64-darwin} ahora requieren 10.13 o 
las más recientes versiones (High~Sierra, Mojave, y Catalina);
\texttt{x86\_64-darwinlegacy} respalda 10.6 y las más recientes. Mac\TeX\ está 
notariado y programas de línea de intérprete de comandos tienen procesos de ejecución más fortalecidos, como es requerido por Apple para los paquetes de instalación. BibDesk y \TeX\ Live Utility
no están en Mac\TeX\ porque los mismos no están notariados,  pero esto se puede obtener mediante el archivo \filename{README} que provee una lista de urls.

\code{tlmgr} y la infraestructura: \begin{itemize*}
\item Automáticamente reintenta (solo una vez) paquetes que fallan durante las descargas. 
\item Nueva opción \texttt{tlmgr check texmfdbs}, para chequear consistencia con los archivos \texttt{ls-R} y con las especificaciones \texttt{!!} para cada uno de los árboles.
\item Uso de control de revisión con los nombres de archivos para los envases de los paquetes, como en \texttt{tlnet/archive/\textsl{pkgname}.rNNN.tar.xz}; debe ser invisible para los usuarios, pero es un notable cambio en la distribución.
\item \texttt{catalogue-date} información no está más propagada desde el Catálogo de \TeX, debido a que esto no estaba relacionado con las actualizaciones de los paquetes. 
\end{itemize*}

% 
\htmlanchor{news}
\subsection{Presente: 2021}
\label{sec:tlcurrent}

General: \begin{itemize}
\item Los cambios por parte de Donald Knuth para la afinación de \TeX\ de 2021 y Metafont están incorporados.
(\url{https://tug.org/TUGboat/tb42-1/tb130knuth-tuneup21.pdf}. Estos cambios están también disponibles en CTAN en los paquetes \code{knuth-dist} y \code{knuth-local}. Como es de esperar, estos reparaciones son con casos aislados, y no afectan funcionamiento alguno en la práctica.

\item Excepto en el original \TeX: si \cs{tracinglostchars} está configurado a 3 o más, aquellos caracteres que estén ausentes, resultaría en un error, y no solo un mensaje en el archivo del log, y el caracter ausente se mostrará en código hex.

\item Excepto en el original \TeX: un nuevo parámetro de integral \cs{tracingstacklevels}, siempre que sea positivo, y \cs{tracingmacros}
sea también positivo, esto por lo tanto causa un prefijo que indica que la expansión del macro es salida, en cada línea del log (e.g., |~..| de profundidad 2). También, el historial del log del macro se acorta en una profundidad $\ge$ del valor del parámetro de integral.

\end{itemize}

Aleph: El formato de \LaTeX\ basado en Aleph, el cual se nombra \code{lamed}, ha sido removido. El binario \code{aleph} como tal, está aún incluido y respaldado.
 
Lua\TeX: \begin{itemize*}
\item Lua 5.3.6.
\item Devolución de llamadas en niveles de adinamiento son usadas en \cs{tracingmacros}, como variantes generalizadas del nuevo \cs{tracingstacklevels}.\item La calificación de glifos de matemática, que sean protegidos para evitar que estos sean procesados como texto. 
\item Se ha removido la compensación de ancho/ic para la ruta tradicional de código en matemática.
\end{itemize*}

MetaPost: \begin{itemize*}
\item |SOURCE_DATE_EPOCH| Respaldo para variables de entorno con salida reproducible. 
\item Evita un erróneo final \texttt{\%} en mpto.
\item Documenta la opción \texttt{-T}, otras reparaciones en el manual.
\item Valor de \texttt{epsilon} es modificado en los modos binarios y decimales, y así |mp_solve_rising_cubic| funciona como se espera.
\end{itemize*}

pdf\TeX{}: \begin{itemize*}
\item Nuevos primitivos \cs{pdfrunninglinkoff} y
\cs{pdfrunninglinkon}; e.g., para desactivar la producción de enlaces en 
las cabeceras y los pies de páginas.
\item Advertencia en vez de anular cuando ``\cs{pdfendlink} terminaba en diferentes niveles de adinamiento que \cs{pdfstartlink}''.
\item Desecha las asignaciones \cs{pdfglyphtounicode} en los archivos \texttt{fmt}.
\item Source: respaldo para Poppler fue removido, debido a que era difícil en mantenerlo en sync
con upstream. En el TL nativo, pdf\TeX\ siempre ha usado \texttt{libs/xpdf}, el cual es un código acortado y adaptado de \texttt{xpdf}.
\end{itemize*}

Xe\TeX{}: Reparaciones de ajustes de los interletrajes para matemática. 

Dvipdfmx: \begin{itemize*}
\item Si un archivo de imagen no se encuentra, se termina con un mal estatus.
\item Syntax especial para respaldo con color.
\item Especiales para manipular |ExtGState|.
\item Compatibilidad de especiales \code{pdfcolorstack} y \code{pdffontattr}.
\item Respaldo experimental para el |fnt_def| de \code{dviluatex} 
\item Respaldo para las nuevas funciones de las fuentes virtuales que retrocedan a definiciones de fuentes japonesas.
\end{itemize*}

Dvips: \begin{itemize*}
\item Títulos estándar de Postscript son ahora el nombre base del fichero ingresado, y puede ser sobrescrito con la nueva opción \texttt{-title}.
\item Si una imagen \texttt{.eps} u otro archivo de imágenes no se encuentra, termina entonces con un mal estatus.
\item Respaldo de nueva función de fuente virtual que retroceda a la definición de fuentes japonesas.
\end{itemize*}

Mac\TeX{}: Mac\TeX{} y su nuevo archivo de binario \texttt{universal-darwin} ahora requiere macOS 10.14 o más reciente (Mojave, Catalina, y Big~Sur); el archivo |x86_64-darwin| de binario, no está más presente. El directorio binario |x86_64-darwinlegacy|, disponible solamente con la instalación \texttt{install-tl} de Unix, respalda 10.6 y versiones más recientes.

Este es un importante año para Macintosh porque Apple presentó máquinas con arquitectura ARM en Noviembre y venderán y respaldarán ambos ARM e INTEL por muchos años. Todos los programas en \texttt{universal-darwin} tienen código ejecutable tanto para ARM como para INTEL. Ambos binarios son compilados del mismo código fuente.

Los programas adicionales GhostScript, LaTeXiT, \TeX{} Live Utility, y TeXShop son todos universales y están firmados con tiempos de ejecución fortalecidos, y así todos están incluidos en Mac\TeX{} este año.

\code{tlmgr} y la infraestructura: \begin{itemize*}
\item guarde una copia de reserva del principal repositorio \texttt{texlive.tlpdb}.
\item mucha más portabilidad entre diferentes sistemas y versiones Perl.
\item \texttt{tlmgr info} reporta nuevos campos de datos \texttt{lcat-*} y \texttt{rcat-*} para el Catálogo local vs. remoto.
\item completo historial de log de sub-órdenes, sub-comandos, trasladados a un nuevo archivo log \texttt{texmf-var/web2c/tlmgr-commands.log}.
\end{itemize*}
   
\subsection{Futuro}

\emph{!`\TL{} no es perfecto!} (Y nunca lo será.) Nuestro
propósito es continuar con el lanzamiento de nuevas
versiones, y desearíamos proveer más ayuda material, más
utilidades, más programas de instalación, y (por
supuesto) un árbol de macros y fuentes con más
mejorías y con mejor revisión. Este trabajo está siendo
llevado a cabo por voluntarios, en sus tiempos libres, y
siempre habrá más que hacer. Por favor vea
\url{https://tug.org/texlive/contribute.html}. 

Por favor, envíe correcciones, referencias y ofertas de
ayuda a:
\begin{quote}
\email{tex-live@tug.org} \\
\url{https://tug.org/texlive}
\end{quote}

\medskip
\noindent \textsl{!`Feliz \TeX ing!}

\end{document}
