% French translation of the TeX Live documentation.
% See ../texlive-en for original version.
% This translation is released under the terms of the WTFPL v2.
%
% Last update based on ../texlive-en/texlive-en.tex r54019
%
% Change history (started May 18th 2002)
%
% 2023/03/03: synced with 2023 English version, by Denis Bitouzé
% 2022/04/18: corrections by Patrick Bideault
% 2022/03/20: synced with 2022 English version, by Denis Bitouzé
% 2021/02/28: synced with 2021 English version, by Denis Bitouzé
% 2020/02/03: synced with 2020 English version, by Denis Bitouzé
% 2019/03/23: synced with 2019 English version, by Denis Bitouzé
% 2018/08/14: synced with 2018 English version, by Denis Bitouzé & Patrick Bideault
% 2017/05/13: synced with 2017 English version, by Denis Bitouzé & Patrick Bideault
% 2016/05/08: synced with 2016 English version, by Denis Bitouzé
% 2015/05/17: synced with 2014/2015 English version, by Denis Bitouzé
% 2009/09/21: synced with 2009 English version, by Manuel Pégourié-Gonnard
% 2008/07/26: synced with 2008 English version, by Daniel Flipo
% 2007/01/18: synced with 2007 English version, by Daniel Flipo
% 2005/11/01: synced with 2005 English version, by Daniel Flipo
% 2005/05/05: re-added section on building binaries (Daniel Flipo)
% 2004/10/28: synced with 2004 English version, by Daniel Flipo
% 2003/08/28: synced with 2003 English version, by Éric Picheral & Jacques André
% 2002/05/25: proof-reading and corrections, by Daniel Flipo
% 2002/05/18: synced with English version, by Fabrice

\documentclass[german, english, french]{article}

\usepackage[utf8]{inputenc}
\usepackage[T1]{fontenc}
\let\tldocfrench=1 % pour live4ht.cfg (ou en fait inutilisé?) (mpg)

\usepackage{csquotes}
\usepackage{mathtools}
\usepackage{amssymb}
\usepackage{babel}
\usepackage{xspace}
\frenchbsetup{% AutoSpacePunctuation=false,
  og=«, fg=»}

\usepackage{tex-live}
\ifpdf
\usepackage{tocloft}
\addtolength{\cftsubsubsecnumwidth}{5pt}
\usepackage[breaklinks]{hyperref}
\usepackage[all]{hypcap}
\fi

% Francisation de tex-live.sty (df)
\renewcommand{\samp}[1]{\enquote{\texttt{#1}}}
% (df) La commande \TeXLive n'est pas utilisée en v.f.
\renewcommand{\TL}{\TeX{} Live\xspace}% Robuste (pas besoin de \protect)...
\renewcommand{\TK}{\TeX{} Collection\xspace}% idem.
% tex-live.sty fait précéder les no de page d'une fine...
% en français, une insécable normale est préférable
\def\p.{page~}

% correction du support pour tex4ht relatif à frenchb.ldf 2.3 ou supérieur
\ifnum\Status=2 % generating html
  \makeatletter
  \AtBeginDocument{\let\FDP@space\FDP@colonspace}
  \makeatother
\fi

% plus de macros (mpg)
\newcommand\eng[1]{\foreignlanguage{english}{\emph{#1}}}
\newcommand\ger[1]{\foreignlanguage{german}{\emph{#1}}}
\newcommand*\semail[1]{\href{mailto:#1}{#1}}

\setlength{\parindent}{0mm}
\addtolength{\parskip}{.25\baselineskip}

\begin{document}

\title{%
  \huge \emph{Guide pratique de \TL 2023}%
}

\author{
  Karl Berry, éditeur \\[3mm]
  \url{https://tug.org/texlive/} \\[6mm]
  \textit{Version française} \\[3mm]
  Denis Bitouzé \& Patrick Bideault\thanks{%
    Précédentes versions françaises par Manuel Pégourié-Gonnard, Daniel Flipo,
    Éric Picheral, Jacques André, Fabrice Popineau et peut-être d'autres avant
    eux --- qu'ils en soient tous remerciés. Nos remerciements aussi aux
    aimables relecteurs ; nous sommes bien sûr responsables des erreurs
    restantes. N'hésitez pas à les signaler par mail
    à \semail{denis.bitouze@univ-littoral.fr} !%
  }%
}

% relecteurs 2009 :
%   Denis Bitouzé,
%   Élie Roux,
%   Hervé Vigier,
%   Jean-Yves Beaudais,
%   Laurent Méhats,
%   tdc
% relecteurs 2010 :
%   Denis Bitouzé,
%   Hervé Vigier,
%   Arnaud Schmittbuhl

\date{Mars 2023}

\maketitle

\begin{multicols}{2}
  \tableofcontents
\end{multicols}

\section{Introduction}
\label{sec:intro}

\subsection{\protect\TL{} et \protect\TeX\ Collection}

Ce document décrit les principales caractéristiques de la distribution \TL{},
une distribution complète de \TeX{} pour Linux et autres Unix, \macOS{} et
systèmes Windows.

Cette distribution peut être obtenue par téléchargement, sur le \DVD{} \TK{}
distribué par les groupes d'utilisateurs de \TeX{} à leurs membres, ou par
d'autres moyens.  La section \ref{sec:tl-coll-dists} décrit brièvement le
contenu du \DVD.  \TL{} et \TK{} sont le fruit des efforts des groupes
d'utilisateurs de \TeX. La description qui suit porte essentiellement sur \TL{}.

La distribution \TL{} comprend les binaires précompilés de \TeX, \LaTeXe,
\ConTeXt, \MF, \MP, \BibTeX{} et de nombreux autres programmes ainsi qu'une
bibliothèque étendue de macros, de fontes et de documentations. Elle permet
aussi la composition de textes dans la plupart des langues utilisées dans le
monde.

On trouvera à la fin du document, section~\ref{sec:history}
(page~\pageref{sec:history}), un bref historique des principales modifications
apportées au fil du temps à la distribution \TL{}.

Les utilisateurs des éditions précédentes de \TL{} sont invités à lire,
\emph{avant toute mise à jour}, la section~\ref{sec:tlcurrent}
page~\pageref{sec:tlcurrent} qui présente les principaux changements intervenus
cette année.

\htmlanchor{platforms}
\subsection{Support des différents systèmes d'exploitation}
\label{sec:os-support}

\TL{} contient les exécutables pour les principales architectures Unix, dont
\GNU/Linux, \macOS et Cygwin. Les sources inclus dans la distribution devraient
être compilables sur des plateformes pour lesquelles nous ne fournissons pas
d'exécutables.

Concernant Windows, les versions~7 et ultérieures sont prises en charge.
Windows Vista \emph{peut} probablement fonctionner en bonne partie, mais \TL{} ne
saurait dorénavant être installée sur XP ou versions antérieures.  La \TL{}
inclut des exécutables 64~bit pour Windows.

Consulter la section~\ref{sec:tl-coll-dists} pour des solutions alternatives
sous Windows et \macOS.

\subsection{Installation élémentaire de \protect\TL{}}
\label{sec:basic}

Vous pouvez installer \TL{} de deux façons différentes : soit depuis le \DVD,
soit depuis Internet (\url{https://tug.org/texlive/acquire.html}).
L'installateur réseau est petit et télécharge tout ce qu'il faut depuis
Internet.

L'installateur du \DVD vous permet d'installer sur votre disque dur mais il
n'est plus possible de lancer \TL{} directement depuis le \DVD{} \TK{} (ou une
image \code{.iso}). Vous \emph{pouvez} créer une installation portable, par
exemple sur une clé \USB{} (voir section~\ref{sec:portable-tl}). La procédure
d'installation est décrite en détail dans les sections suivantes
(\p.\pageref{sec:install}), mais voici de quoi commencer rapidement.

\begin{itemize*}

\item Sous Unix, le script d'installation est \filename{install-tl}.  Sous
  Windows, il faut plutôt invoquer \filename{install-tl-windows}.
  L'installateur fonctionnera en mode graphique avec l'option \code{-gui}
  (par défaut pour Windows) ou en mode texte avec l'option
  \code{-gui=text} (par défaut pour tout le reste).

\item Ce script installe en particulier le programme « \TL\ Manager » appelé
  \prog{tlmgr}. Comme l'installateur, il peut être utilisé en mode graphique ou
  en mode texte. Il permet d'ajouter ou de supprimer des composants et de
  procéder à différents réglages de configuration.

\end{itemize*}

\htmlanchor{security}
\subsection{Remarques sur la sécurité}
\label{sec:security}

À notre connaissance, les programmes \TeX{} de base en eux-mêmes sont (et ont
toujours été) extrêmement robustes. Cependant, d'autres programmes livrés avec
\TL{} ne le sont peut-être pas autant, bien que chacun fasse de son mieux. De
façon générale, il faut faire preuve de prudence avant de lancer des programmes
sur des données en lesquelles vous n'avez pas toute confiance. Pour un maximum
de sécurité, utilisez un nouveau répertoire créé à cet effet ou chroot.

Il convient d'être particulièrement vigilant sous Windows, qui en général
cherche les programmes dans le répertoire courant avant tout, quel que soit le
chemin de recherche défini. Nous avons comblé de nombreuses failles, mais il en
reste certainement d'autres, notamment avec les programmes maintenus
extérieurement. C'est pourquoi nous recommandons de vérifier s'il n'y a pas de
fichiers suspects dans le répertoire courant, en particulier les exécutables
(binaires ou scripts). En général, il ne devrait pas y en avoir, et le simple
fait de compiler un document devrait encore moins en créer.

Enfin, \TeX{} (et les programmes associés) ont la capacité d'écrire des fichiers
en compilant des documents, capacité qui peut être exploitée à des fins malignes
de nombreuses façons. Ici aussi, la meilleure protection est de compiler les
documents inconnus dans un répertoire nouvellement créé.

Un autre aspect de la sécurité consiste à s'assurer que le matériel téléchargé
n'a pas a été modifié par rapport à ce qui a été créé. Le programme \prog{tlmgr}
(section~\ref{sec:tlmgr}) effectuera automatiquement des vérifications des
téléchargements si le \prog{gpg} (GNU Privacy Guard) est disponible. Il n'est
pas distribué par la \TL, mais voir \url{https://texlive.info/tlgpg/} pour des
informations sur \prog{gpg} si nécessaire.

\subsection{Obtenir de l'aide}
\label{sec:help}
\subsubsection{Aide en français}
Si vous connaissez mal le système \TeX{}, ou que vous n'êtes pas certain de
vouloir installer \TL{}, vous pouvez avoir un aperçu du fonctionnement de l'un
de ses logiciels principaux en lisant la \emph{Une courte (\string?)
  introduction à \LaTeXe{}}, dont la traduction française est disponible
à l'adresse \url{https://mirrors.ctan.org/info/lshort/french/lshort-fr.pdf}.

À la suite de cette lecture, vous n'aurez de cesse d'utiliser \LaTeXe{}, et donc
d'installer \TL{} : il vous suffira tout simplement de reprendre ici la lecture
du présent document pour effectuer sereinement cette installation.  Ce document
a été rédigé pour vous guider lors de celle-ci.

Une fois installée, votre distribution \TL{} sera prête à l'emploi.  C'est alors
que la commande\footnote{Si vous utilisez un système Mac, il vous faut utiliser
  l'application « Terminal » pour avoir accès à l'émulateur de commandes.  Sur
  systèmes Windows, il vous faut taper \code{Win-R} puis \code{cmd}.  Enfin, si
  vous utilisez un système de type Unix, il est vraisemblable que vous sachiez
  déjà comment utiliser cette commande.} \code{texdoc} vous sera d'une aide
précieuse, puisqu'elle vous permettra d'afficher la documentation des différents
composants de \TL{}.

Ainsi, pour lire la \emph{Courte Introduction} citée plus haut, vous n'aurez
qu'à taper \code{texdoc lshort-fr} (vous remarquerez que la chaîne de caractères
\code{lshort-fr} est également présente dans l'url citée plus haut : c'est son
nom au sein de \TL{}).  Et pour afficher de nouveau le présent document, vous
taperez \code{texdoc texlive-fr}. Faites l'essai !

En toute logique, la commande \code{texdoc texdoc} vous indiquera comment vous
servir au mieux de cette commande, par exemple pour qu'elle utilise votre
lecteur PDF favori, ou pour qu'elle affiche automatiquement les documentations
en français (si elles ont été traduites).

Car voilà : l'intégralité de la documentation des innombrables composants de
\TL{} n'est pas encore disponible en français, tant s'en faut.

Fort heureusement, des ressources francophones de qualité existent.  La
documentation de \LaTeXe{} est traduite : elle est accessible via la commande
\code{texdoc latex2e-fr}. Une communauté d'utilisateurs a recensé d'autres
ressources à l'adresse suivante :
\url{https://texnique.fr/osqa/questions/2559/des-sources-dinformation-pour-debutants}.

En effet, la communauté \TeX{} est active et conviviale ; la plupart des
questions finissent par obtenir une réponse. En revanche, le support est
informel, assuré par des volontaires et des usagers occasionnels, aussi est-il
particulièrement important que vous fassiez votre propre travail de recherche
avant de poser une question (si vous préférez un support commercial, vous pouvez
renoncer à \TL{} et acheter un système payant ; voir une liste, en anglais,
à \url{https://tug.org/interest.html#vendors}).

Avant de poser une question à d'autres utilisateurs, nous vous invitons
à vérifier qu'elle n'a pas déjà été posée : un moteur de recherche, par exemple
\url{https://duckduckgo.com/}, peut à lui seul fournir des réponses pertinentes,
et les archives des différents espaces de discussion contiennent des milliers de
questions et réponses déjà traitées ; elles permettent une recherche fructueuse.

Enfin, avant de poser une question, n'hésitez pas à prendre conseil sur la façon
de rédiger la question pour avoir des chances d'obtenir une réponse pertinente.
Soyez clair, précis et concis. Et illustrez votre propos avec un \emph{exemple
  complet minimal} (les trois termes sont importants), comme indiqué ici :
\url{https://texnique.fr/osqa/faq/#custom-id-ecm}.

Ces précautions étant prises, il est temps de vous présenter les trois espaces
de discussion en ligne des utilisateurs francophones.

\begin{description}
\item [FAQ \TeX{} francophone] -- La \TeX{} FAQ francophone est un recueil de
  réponses à toutes sortes de questions, des plus élémentaires aux plus
  obscures. Elle est disponible sur Internet à l'adresse
  \url{https://faq.gutenberg.eu.org/}.

\item[\texttt{texnique.fr}] est un site de questions \& réponses.  D'utilisation
  très simple, il permet aux utilisateurs ayant besoin d'aide de poser des
  questions, à d'autres de leur apporter des réponses, et aux visiteurs du site
  de noter celles-ci selon leur pertinence, ce qui permet de les repérer
  efficacement.  Les questions sont triées par mots-clés, et le site offre la
  compilation en ligne des exemples de code qui y sont postés.  Il est
  accessible à l'adresse \url{https://texnique.fr/}.

\item[la liste GUT] est une liste de discussion par courriel comptant un grand
  nombre d'abonnés.  Elle permet de demander ou d'apporter de l'aide, mais aussi
  de débattre ou d'exprimer des opinions, ce qui n'est pas l'objet du site cité
  plus haut.  Pour vous inscrire à cette liste, envoyez un message
  à \email{sympa@ens.fr} avec \code{subscribe gut} comme corps ou sujet du
  message.

\item[\texttt{fctt}] est l'appellation usuelle du groupe de nouvelles Usenet
  très fréquenté \code{fr.comp.text.tex}. Pour l'utiliser de façon ergonomique,
  il est conseillé de recourir à un lecteur de \emph{news}, par exemple celui
  intégré au courrielleur
  \href{https://www.mozilla.org/fr/thunderbird/}{Thunderbird}.  Ses archives
  sont consultables à l'adresse
  \url{https://groups.google.com/g/fr.comp.text.tex}.
\end{description}
Il est à noter que le site \dirname{texnique.fr} et la liste GUT sont soutenus
par l'association GUTenberg, qui est le groupe francophone des utilisateurs de
\TeX{}, \LaTeX{} et logiciels compagnons. Le site de l'association est
accessible à l'adresse \url{https://www.gutenberg-asso.fr/}.  Les traducteurs de
la présente documentation sont également membres de l'association, qui elle-même
contribue au développement de \TL{}.

\subsubsection{Aide en anglais}
Donald Knuth, l'auteur de \TeX, est anglophone : il est donc logique qu'une
large documentation existe en anglais. Voici une liste de ressources
anglophones, classées selon l'ordre dans lequel nous recommandons de les
utiliser :

\begin{description}
\item[Getting Started] -- pour ceux qui débutent en \TeX, la page web
  \url{https://tug.org/begin.html} contient une courte description du système.

\item [CTAN] Si vous recherchez un package, une police, un programme
  particulier, etc., \CTAN{} est l'endroit où commencer à chercher. Il s'agit
  d'une énorme collection de tous les éléments liés à \TeX{}. Les entrées du
  catalogue vous indiquent également la disponibilité de la ressource
  correspondante pour \TL{} ou MiK\TeX. Voir \url{https://ctan.org}.

\item [FAQ \TeX{}] -- La \TeX{} FAQ est un recueil de réponses à toutes sortes
  de questions, des plus élémentaires aux plus obscures. Elle est disponible sur
  Internet à l'adresse \url{https://texfaq.org}.

\item[Catalogue \TeX{}] -- si vous recherchez une extension, une fonte, un
  programme, etc., le mieux est de consulter le catalogue \TeX{} %(appelé CTAN)
  ici : \url{https://ctan.org/pkg/catalogue}.

\item[Ressources \TeX{} sur le Web] -- la page web
  \url{https://tug.org/interest.html} propose beaucoup de liens relatifs
  à \TeX{}, en particulier concernant des livres, manuels et articles portant
  sur tous les aspects du système.

\item[Forums d'aide] -- les principaux forums dédiés au support de \TeX{} sont
  le site de la communauté \LaTeX{} \url{https://latex.org/}, le site de
  questions \& réponses \url{https://tex.stackexchange.com/}, le groupe de
  nouvelles Usenet \url{news:comp.text.tex}, la liste de diffusion
  \email{texhax@tug.org} (dont les archives sont à l'adresse
  \url{https://tug.org/mail-archives/texhax}).  Une recherche en ligne ne fait
  jamais de mal.

\item[Poster une question] -- si vous ne trouvez pas réponse à votre question,
  vous pouvez, vous pouvez la poser sur \url{https://latex.org/forum/} et
  \url{https://tex.stackexchange.com/} via leurs interfaces Web, sur
  \dirname{comp.text.tex} via Google ou votre lecteur de nouvelles, ou sur
  \email{texhax@tug.org} par mail. Mais, avant de poser votre question, nous
  vous conseillons de lire la FAQ \url{https://texfaq.org/FAQ-askquestion} afin
  de maximiser vos chances d'obtenir une réponse utile.
\end{description}

Enfin, l'organisation internationale des utilisateurs de \TeX, le \eng{\TeX{}
  Users Group}, ou TUG, est anglophone. Basée aux États-Unis, elle fédère toutes
les associations nationales et son site est à l'adresse \url{https://tug.org/}.

Parmi ses très nombreuses activités, elle organise une conférence annuelle,
publie une revue, le \emph{TUGboat}, qui paraît trois fois par an, ainsi qu'une
collection de logiciels appelée\dots{} \TL{} !

\subsubsection{Aide en allemand}
La communauté germanophone est très active. Elle organise de solides séminaires
ainsi que des rencontres informelles, publie des ouvrages de grande
qualité\dots{} et a produit de très nombreuses lignes de code utilisées par
\TL{} ! Le groupe germanophone des utilisateurs de \TeX{} s'appelle
\textsc{dante} et son site est à l'adresse \url{https://www.dante.de/}. Il publie
régulièrement une revue, \emph{Die TeXnische Komödie}, et soutient le forum
\url{https://golatex.de/}.

Enfin, le site germanophone de questions \& réponses \url{https://texwelt.de/}
est le grand frère du francophone \dirname{texnique.fr} !

\subsubsection{Aide dans d'autres langues}
\TeX est répandu dans le monde entier, et des groupes d'utilisateurs existent
dans de nombreux pays, de la Bulgarie à l'Afrique du Sud ! On en trouvera la
liste à l'adresse \url{https://tug.org/usergroups.html}.

\subsection{Apporter de l'aide}
Il vous est possible d'apporter une aide précieuse aux autres utilisateurs :
\begin{description}
\item[Support de \TL{}] -- si vous voulez faire un rapport d'anomalie, émettre
  des suggestions ou des commentaires sur la distribution \TL{}, son
  installation ou sa documentation, utilisez la liste de diffusion
  \email{tex-live@tug.org}. Mais attention, si la question concerne
  l'utilisation d'un programme particulier inclus dans \TL{}, il vaut mieux que
  vous vous adressiez directement à la personne ou à la liste de diffusion qui
  maintient le programme. Il suffit souvent d'ajouter l'option \code{-{}-help}
  lors de l'exécution du programme pour trouver à qui adresser le rapport.

\item[Aide en ligne] -- des personnes soumettent des questions, et leur répondre
  est très utile, tant pour eux que pour vous : formuler clairement une réponse
  est une source d'apprentissage non négligeable !  Les sites \LaTeX{}, tels
  \dirname{texnique.fr}, les groupes de nouvelles comme
  \dirname{fr.comp.text.tex} ainsi que la liste \code{gut} sont ouverts à tous :
  n'hésitez pas à vous y joindre, à commencer la lecture et à fournir de l'aide
  là où cela vous est possible.

\item[Traduction] -- le présent document a été rédigé puis traduit bénévolement,
  pour le bien de la communauté. Il en va de même pour l'essentiel des logiciels
  et de leur documentation. Comme la rédaction de réponses, la traduction est un
  exercice très profitable en termes d'apprentissage.  Si la documentation qui
  vous est nécessaire n'est pas encore disponible en français, n'hésitez pas
  à la traduire vous-même et à en faire profiter les autres utilisateurs !  Les
  coordonnées des auteurs des logiciels figurent généralement dans la
  documentation originale de ceux-ci ; dès lors, il est très simple de prendre
  contact avec eux pour publier votre traduction.
\end{description}

% No \protect needed as \TL is defined in French with \newcommand (robust).
\section{Structure de \protect\TK}
\label{sec:overview-tl}

Nous décrivons ici le contenu de \TK{} qui est un sur-ensemble de \TL.

\subsection {\protect\TK{} : \protect\TL, Mac\protect\TeX, MiK\protect\TeX,
  CTAN}
\label{sec:tl-coll-dists}

Le \DVD{} \TK{} contient les éléments suivants :

\begin{description}

\item[\TL{}] -- Un système multiplateforme complet \TeX{} à installer sur
  disque. Page web : \url{https://tug.org/texlive/}.

\item[Mac\TeX] -- pour \macOS. Cela ajoute un installateur natif pour \macOS\ et
  d'autres applications Mac à \TL{}.  Page web : \url{https://tug.org/mactex/}.

\item [\MIKTEX] Une autre distribution complète et multiplateforme \TeX\ pour
  Windows, GNU/Linux et \macOS\ (mais seuls les binaires Windows sont inclus
  dans le DVD). Elle dispose d'un gestionnaire de paquets intégré qui, si
  nécessaire, installe les composants manquants à partir d'Internet. Page
  web : \url{https://miktex.org/}.

\item[CTAN] -- une image du site d'archives \CTAN{} (\url{https://ctan.org/}).
  \CTAN{} ne suit pas les mêmes conditions de copie que \TL{}, pensez à lire les
  licences si vous envisagez de modifier ou de redistribuer certains de ces
  fichiers.

\end{description}

\subsection{Répertoires situés à la racine de \protect\TL{}}
\label{sec:tld}

Voici une petite liste des répertoires situés à la racine d'une installation \TL
avec une courte description.

\begin{ttdescription}
\item[bin] -- programmes de la famille \TeX{}, rangés dans des sous-répertoires
  selon les plateformes.
%
\item[readme-*.dir] -- une brève introduction et quelques liens utiles pour \TL,
  dans divers langages, sous forme de pages \HTML\ ou de texte brut.
%
\item[source] -- le code source de tous les programmes, dont les principaux
  programmes \TeX{} basés sur \Webc{}.
%
\item[texmf-dist] -- le répertoire principal ; voir \envname{TEXMFDIST} dans la
  section suivante.
%
\item[tlpkg] -- scripts et programmes d'installation, et éléments spécifiques
  pour Windows.
\end{ttdescription}

On trouve également, au même niveau que les répertoires mentionnés ci-dessus,
les scripts d'installation et le fichier \filename{README} principal en anglais
(disponible en différentes langues dans les répertoires \filename{readme-*.dir}
susmentionnés).

En plus des répertoires ci-dessus, les scripts d'installation et les fichiers
\filename{README} (dans différentes langues) sont situés dans le dossier racine
de la distribution.

Concernant la documentation, le fichier \OnCD{doc.html}, qui contient une liste
exhaustive de liens vers les fichiers de documentation, pourra s'avérer utile.
La documentation pour presque tout (paquets, formats, fontes, manuels de
logiciels, pages de \cmdname{man}, fichiers \cmdname{info}) se trouve dans
\dirname{texmf-dist/doc}.  Vous pouvez utiliser le programme \cmdname{texdoc}
pour trouver une documentation, où qu'elle soit.

La documentation de \TL\ proprement dite (que vous consultez actuellement) est
disponible dans le répertoire \dirname{texmf-dist/doc/texlive} en plusieurs
langues :

\begin{itemize*}
\item{anglais :} \OnCD{texmf-dist/doc/texlive/texlive-en},
\item{allemand :} \OnCD{texmf-dist/doc/texlive/texlive-de},
\item{espagnol :} \OnCD{texmf-dist/doc/texlive/texlive-es},
\item{chinois (simplifié) :} \OnCD{texmf-dist/doc/texlive/texlive-zh-cn},
\item{français :} \OnCD{texmf-dist/doc/texlive/texlive-fr},
\item{italien :} \OnCD{texmf-dist/doc/texlive/texlive-it}
\item{polonais :} \OnCD{texmf-dist/doc/texlive/texlive-pl},
\item{russe :} \OnCD{texmf-dist/doc/texlive/texlive-ru},
\item{tchèque \& slovaque :} \OnCD{texmf-dist/doc/texlive/texlive-cz}.
\end{itemize*}

\subsection{Description des arborescences « texmf » de \protect\TL}
\label{sec:texmftrees}

Nous donnons ici la liste des variables prédéfinies qui contiennent les noms des
différentes arborescences de type « texmf », l'usage qui en est fait, et leurs
valeurs par défaut dans \TL. La commande \cmdname{tlmgr~conf} montre les valeurs
de ces variables, ce qui permet de savoir où se trouvent ces différentes
arborescences dans une installation donnée.

Notez bien que toutes ces arborescences, y compris les arborescences
personnelles, doivent suivre la structure standard des répertoires de \TeX{}
(\TDS{} : \url{https://tug.org/tds}) avec sa pléthore de sous-répertoires, sans
quoi les fichiers ne seront pas pris en compte. La section
\ref{sec:local-personal-macros} page \pageref{sec:local-personal-macros} donne
plus de détails à ce sujet. L'ordre ici est l'inverse de celui dans lequel les
arborescences sont cherchées, c'est-à-dire que les dernières arborescences dans
la liste outrepassent les précédentes.

\begin{ttdescription}
\item[TEXMFDIST] -- contient à peu près tous les fichiers de la distribution
  originale : fichiers de configuration, scripts, paquets, polices, etc. (La
  principale exception concerne les exécutables de chaque plateforme placés
  dans le répertoire apparenté \code{bin/}).  Rien ne devrait être ajouté,
  modifié ni supprimé manuellement dans cette arborescence.
\item[TEXMFSYSVAR] -- c'est là que les scripts \cmdname{texconfig-sys},
  \cmdname{updmap-sys} et \cmdname{fmtutil-sys} rangent les (versions globales
  des) données générées, comme les formats et les fichiers \verb+.map+.
\item[TEXMFSYSCONFIG] -- c'est là que les scripts \cmdname{texconfig-sys},
  \cmdname{updmap-sys} et \cmdname{fmtutil-sys} recherchent leurs fichiers de
  configuration (globaux) en priorité.
\item[TEXMFLOCAL] -- c'est l'arborescence que les administrateurs peuvent
  utiliser pour installer pour tous les utilisateurs les extensions, fontes,
  etc. additionnelles ou mises à jour.
\item[TEXMFHOME] -- c'est l'arborescence que les utilisateurs peuvent utiliser
  pour leurs installations individuelles de macros, polices,
  etc. supplémentaires ou de mises à jour.  Pour chaque utilisateur, cette
  variable pointe vers son propre répertoire individuel.
\item[TEXMFVAR] -- c'est là que les scripts \cmdname{texconfig},
  \cmdname{updmap-user} et \cmdname{fmtutil-user} rangent les (versions
  personnelles de) données générées, comme les formats et les fichiers
  \verb+.map+.
\item[TEXMFCONFIG] -- c'est là que les scripts \cmdname{texconfig}
  \cmdname{updmap-sys} et \cmdname{fmtutil-sys} recherchent leurs fichiers de
  configuration (personnelles) en priorité.
\item[TEXMFCACHE] -- arborescence(s) utilisée(s) par \ConTeXt\ MkIV et LuaLaTeX\
  pour stocker des (versions de cache de) fichiers de travail. Par défaut,
  utilise \code{TEXMFSYSVAR} ou, s'il n'est pas possible d'y écrire,
  \code{TEXMFVAR}.
\end{ttdescription}

La structure de l'arborescence \TL{} est par défaut la suivante :
\begin{description}
\item[racine multi-utilisateur] (par exemple \verb|/usr/local/texlive| sous
  Unix, \verb|D:\texlive| sous Windows,\dots) qui contient éventuellement
  plusieurs distributions :
  \begin{ttdescription}
  \item[2022] -- une distribution précédente,
  \item[2023] -- la distribution actuelle,
    \begin{ttdescription}
    \item[bin] ~
      \begin{ttdescription}
      \item[i386-linux] -- exécutables Linux (32~bit),
      \item[...]
      \item [universal-darwin] -- exécutables \macOS,
      \item [x86\_64-linux] -- exécutables \GNU/Linux (64~bit),
      \item[windows] -- exécutables Windows (64~bit),
      \end{ttdescription}
    \item[texmf-dist] -- correspond à \envname{TEXMFDIST} et
      \envname{TEXMFMAIN},
    \item[texmf-var] -- correspond à \envname{TEXMFSYSVAR},
    \item[texmf-config] -- correspond à \envname{TEXMFSYSCONFIG},
    \end{ttdescription}
  \item[texmf-local] -- correspond à \dirname{TEXMFLOCAL}, normalement commun
    à plusieurs distributions \TL{} (2007, 2008, etc.),
  \end{ttdescription}
\item[répertoire personnel de l'utilisateur] \dirname{$HOME} %$
  ou |%USERPROFILE%| :
  \begin{ttdescription}
  \item[.texlive2022] -- fichiers propres à l'utilisateur générés
    automatiquement pour une distribution \TL{} précédente,
  \item[.texlive2023] -- fichiers propres à l'utilisateur générés
    automatiquement pour la distribution \TL{} actuelle,
    \begin{ttdescription}
    \item[texmf-var] -- correspond à \envname{TEXMFVAR}, \envname{TEXMFCACHE},
    \item[texmf-config] -- correspond à \envname{TEXMFCONFIG},
    \end{ttdescription}
  \item[texmf] -- correspond à \dirname{TEXMFHOME}, répertoire regroupant tous
    les répertoires de macros personnelles : \dirname{texmf/tex/latex} par
    exemple pour les fichiers de macros \LaTeX, etc.
  \end{ttdescription}
\end{description}

\subsection{Extensions de \protect\TeX}
\label{sec:tex-extensions}

Le programme \TeX\ original de Knuth est figé, sauf corrections de bugs,
rares. C'est toujours le programme \prog{tex} présent dans \TL, et ceci ne
changera pas autant dans un futur prévisible. \TL contient plusieurs variantes
étendues de \TeX\ (aussi appelées « moteurs \TeX{} ») :

\begin{description}

\item [\eTeX{}] \label{text:etex} ajoute un jeu de nouvelles primitives et les
  extensions \TeXXeT{} pour l'écriture de droite à gauche ; \eTeX{} est, en mode
  normal, 100\% compatible avec \TeX{} standard.  Pour plus de détails, consulter
  le fichier \OnCD{texmf-dist/doc/etex/base/etex_man.pdf} du \DVD{}.

\item[pdf\TeX{}] inclut les extensions \eTeX{} et produit, au choix, des
  fichiers au format Acrobat PDF ou au format \dvi{} ; il inclut aussi de
  nombreuses extensions indépendantes du format de sortie. C'est le programme
  invoqué pour de nombreux formats courants tels que \prog{etex}, \prog{latex}
  et \prog{pdflatex}. Son site web est
  \url{https://tug.org/applications/pdftex/}.  Le manuel d'utilisation se trouve
  sur le \DVD{} dans \OnCD{texmf-dist/doc/pdftex/manual/pdftex-a.pdf}. Le
  fichier \OnCD{texmf-dist/doc/pdftex/samplepdftex/samplepdf.tex} donne des
  exemples d'utilisation.

\item[Lua\TeX] ajoute la prise en charge d'un codage d'entrée Unicode et des
  polices OpenType/TrueType et système. Il intègre également un interprète Lua
  (\url{https://lua.org/}), permettant des solutions à de nombreux épineux
  problèmes \TeX{}. Lorsqu'il est appelé sous le nom de \filename{texlua}, il
  fonctionne comme un interpréteur Lua autonome. Son site web est
  \url{http://luatex.org/}, et le manuel de référence est
  \OnCD{texmf-dist/doc/luatex/base/luatex.pdf}.

\item[(e)(u)p\TeX] ont un support natif pour la composition en japonais ; p\TeX\
  est le moteur de base, tandis que les variantes e- ajoutent \eTeX\ et u-
  ajoutent le support Unicode.

\item[\XeTeX] ajoute la prise en charge d'un codage d'entrée Unicode et des
  polices OpenType/TrueType et système, implémentées à l'aide de bibliothèques
  standard tierces. Voir \url{https://tug.org/xetex}.

\item[\OMEGA{} (Omega)] fonctionne en utilisant Unicode et des caractères 16
  bit, ce qui lui permet de travailler directement et simultanément avec
  presque toutes les écritures du monde.  L'extension prend en charge aussi le
  chargement dynamique d'OTP (\OMEGA{} \emph{Translation Processes}) qui permet
  à l'utilisateur d'appliquer sur des flux d'entrée quelconques des
  transformations complexes qu'il aura lui-même définies. \TL n'inclut plus
  Omega en tant que programme séparé ; seul Aleph est fourni.

\item[Aleph] combine les fonctionnalités de \OMEGA\ et de \eTeX ; voir
  \OnCD{texmf-dist/doc/aleph/base}.

\end{description}

\subsection{Quelques autres programmes intéressants dans \protect\TL}

Voici quelques autres programmes couramment utilisés et présents dans \TL{} :

\begin{cmddescription}

\item[bibtex, biber] -- traitement des bibliographies.

\item[makeindex, upmendex, xindex, xindy] -- traitement des index.

\item[dvips] -- conversion \dvi{} vers \PS{}.

\item[dvipdfmx] -- conversion \dvi{} vers PDF, une alternative à pdf\TeX{}
  (mentionné ci-dessus) pour produire des fichiers PDF.

\item[xdvi] -- visualisation \dvi{} pour le système X Window.

\item[dviconcat, dviselect] -- couper/coller de pages à partir de fichiers
  \dvi{}.

\item[psselect, psnup, \ldots] -- utilitaires \PS{}.

\item[pdfjam, pdfjoin, \ldots] -- utilitaires PDF.

\item[context, mtxrun] -- processeurs \ConTeXt{} et PDF.

\item [htlatex, \ldots] \cmdname{tex4ht} : \AllTeX{} vers HTML (et XML et DocX
  et autres).

\end{cmddescription}

\htmlanchor{installation}
\section{Installation}
\label{sec:install}

\subsection{Démarrage de l'installation}
\label{sec:inst_start}

La première chose à faire est de télécharger le programme d'installation par le
réseau ou de récupérer le \DVD{} \TK.  La page
\url{https://tug.org/texlive/acquire.html} donne plus d'informations sur les
différentes façons d'obtenir le logiciel.

\begin{description}
\item[Installateur réseau, .zip ou .tar.gz] -- à télécharger sur CTAN dans
  \dirname{systems/texlive/tlnet} ; l'url
  \url{https://mirror.ctan.org/systems/texlive/tlnet/} devrait vous rediriger vers
  le serveur d'archive le plus proche de chez vous.  Deux versions sont
  disponibles, \filename{install-tl.zip} utilisable sous Unix et Windows, et
  \filename{install-unx.tar.gz} pour Unix seulement (y compris \macOS). Après
  décompression de l'un de ces deux fichiers, les scripts \filename{install-tl}
  et \filename{install-tl-windows.bat} se trouveront dans le sous-répertoire
  \dirname{install-tl}.

\item[Installateur réseau, Windows .exe :] à télécharger sur \CTAN{} comme
  ci-dessus, et à exécuter en double-cliquant. Cela lance un installateur
  préliminaire et décompacteur (voir figure~\ref{fig:nsis}) proposant deux
  options : « Installation » et « Décompactage seulement ».

\item[\DVD \TK{}] -- aller dans son répertoire \dirname{texlive}.  Sous Windows,
  l'installation démarre en principe automatiquement quand vous insérerez le
  \DVD, sinon, il faudra exécuter le script d'installation \filename{install-tl}
  ou \filename{install-tl.bat}.  Le \DVD \TK{} est distribué gratuitement aux
  membres des groupes d'utilisateurs de \TeX{} (GUTenberg par exemple en France,
  voir la liste des groupes sur \url{https://tug.org/usergroups.html}). Il est
  également possible de l'acheter séparément (\url{https://tug.org/store}) ou de
  graver votre propre \DVD\ pour \TL\ à partir du image \ISO\ téléchargée sur
  \CTAN. Sur la plupart des systèmes, vous pouvez aussi monter directement cette
  image \ISO.  Après avoir installé depuis le \DVD{}, si vous voulez accéder aux
  mises à jour en continu depuis Internet, consultez la
  section~\ref{sec:dvd-install-net-updates}.
\end{description}

\begin{figure}[tb]
  \def\figdesc{Première étape de l'installateur \code{.exe} de Windows}
  \tlpng{nsis_installer}{.6\linewidth}{\figdesc}
  \caption{\figdesc. En appuyant sur le bouton Installer, vous obtiendrez la
    fenêtre affichée dans la figure~\ref{fig:basic-w32}.}\label{fig:nsis}
\end{figure}

Le même programme d'installation est utilisé, quelle que soit la source
d'installation. La différence la plus notable entre ces deux modes est qu'avec
l'installation par le réseau, vous obtenez les versions courantes des différents
paquets, contrairement au \DVD{} (ou image \ISO) qui n'est pas mis à jour entre
deux versions majeures.

Si vous devez télécharger par le biais des serveurs mandataires
(\emph{proxies}), utilisez un fichier \filename{~/.wgetrc} ou des variables
d'environnement avec une configuration de serveur mandataire pour Wget
(\url{https://www.gnu.org/software/wget/manual/html_node/Proxies.html}), ou
équivalent pour tout autre programme de téléchargement que vous utilisez.  \TL{}
utilise toujours \GNU\ Wget pour le téléchargement. Bien sûr, ceci ne vous
concerne pas si vous installez depuis le \DVD\ ou l'image \ISO.

Les sections suivantes expliquent plus en détail le fonctionnement de
l'installateur.

\subsubsection{Installation sous Unix}

Dans ce qui suit, l'invite du \textit{shell} est notée \texttt{>} ; les
commandes de l'utilisateur sont en \Ucom{\texttt{gras}}. Le programme
\filename{install-tl} est un script Perl ; la façon la plus simple de le
démarrer sur un système Unix est la suivante.
\begin{alltt}
> \Ucom{perl /chemin/de/l/installateur/install-tl}
\end{alltt}
Si le fichier \filename{install-tl} est encore exécutable, vous pouvez aussi
n'invoquer que :
\begin{alltt}
> \Ucom{/chemin/de/l/installateur/install-tl}
\end{alltt}
ou changer de répertoire auparavant, via la commande \Ucom{cd} ; nous ne
répéterons plus toutes ces variantes. Il est possible que vous deviez agrandir
la fenêtre de votre terminal pour voir le texte complet de l'installateur
(figure~\ref{fig:text-main}).

Pour travailler via une interface graphique (\GUI{}),
cf. figure~\ref{fig:advanced-lnx}, il est nécessaire que Tcl/Tk soit
installé. Vous pouvez alors lancer :
\begin{alltt}
> \Ucom{cd /chemin/de/l/installateur}
> \Ucom{./install-tl -gui}
\end{alltt}

Les anciennes options \code{-wizard} and \code{-perltk}/\code{-expert} ont
désormais le même effet que \code{-gui}. La liste complète des options est
donnée par
\begin{alltt}
> \Ucom{perl install-tl -help}
\end{alltt}

\textbf{Concernant les permissions Unix} : le matériel installé doit normalement
être accessible à tous les utilisateurs de la machine ; si celui qui installe
n'est pas \textit{root}, il devra s'assurer que son \code{umask} est adapté, par
exemple \code{umask 022} ou \code{umask 002}.  Consulter la documentation
système pour plus de précisions.

\textbf{Remarques particulières pour Cygwin} : contrairement aux autres systèmes
compatibles Unix, Cygwin ne comprend pas par défaut tous les programmes requis
pour faire fonctionner l'installateur \TL. Voir la section~\ref{sec:cygwin}.

\subsubsection{Installation sous \macOS}
\label{sec:macosx}

Une distribution spécifique, Mac\TeX\ (\url{https://tug.org/mactex}), a été mise
au point pour \macOS.  Nous recommandons d'utiliser son installateur natif,
plutôt que d'installer la distribution \TL{} en suivant la procédure indiquée
pour Unix. En effet, Mac\TeX{} comprend des ajustements spécifiques pour le
système \macOS{} et facilite la cohabitation entre plusieurs distributions
\TeX{} (Mac\TeX, Fink, MacPorts, etc.).

Mac\TeX\ est fortement basée sur \TL et les arborescences principales et les
exécutables sont en tous points identiques. Quelques répertoires contenant de la
documentation et des applications spécifiques au Mac sont ajoutés.

\subsubsection{Installation sous Windows}\label{sec:wininst}

Si vous utilisez le fichier zip téléchargé et décompacté ou si l'insertion du
\DVD{} ne lance pas l'installation automatiquement, double-cliquez sur
\filename{install-tl-windows.bat}.

Il est également possible de travailler en ligne de commande ; dans ce qui suit
l'invite du \textit{shell} est notée \texttt{>}, les commandes de l'utilisateur
sont notées \Ucom{\texttt{en gras}}. Voici les commandes à lancer (à partir du
répertoire du script d'installation) :
\begin{alltt}
> \Ucom{install-tl-windows}
\end{alltt}
Si l'on opère à l'extérieur du répertoire dudit script d'installation, il faut
écrire :
\begin{alltt}
> \Ucom{D:\bs{}texlive\bs{}install-tl-windows}
\end{alltt}
où \texttt{D:} désigne le lecteur de \DVD où se trouve la collection \TeX.

La figure~\ref{fig:basic-w32} montre l'écran initial de base de l'installateur
en mode \GUI{}, qui est le mode par défaut pour Windows.

Pour travailler en mode texte :
\begin{alltt}
> \Ucom{install-tl-windows -no-gui}
\end{alltt}
La liste complète des options est donnée par
\begin{alltt}
> \Ucom{install-tl-windows -help}
\end{alltt}

\textbf{N.B.} Ajoutez une extension \texttt{.bat} si le même répertoire contient
également \texttt{install-tl-windows.exe}. Ce ne sera normalement pas le cas (à
moins que vous n'ayez mis en miroir le répertoire \dirname{tlnet} localement).

\begin{figure}[tb]
  \centering
\begin{boxedverbatim}
Installing TeX Live 2023 from: ...
Platform: x86_64-linux => 'GNU/Linux on x86_64'
Distribution: inst (compressed)
Directory for temporary files: /tmp
...
 Detected platform: GNU/Linux on Intel x86_64

 <B> binary platforms: 1 out of 16

 <S> set installation scheme (scheme-full)

 <C> customizing installation collections
     40 collections out of 41, disk space required: 7620 MB (free: 138718 MB)

 <D> directories:
   TEXDIR (the main TeX directory):
     /usr/local/texlive/2023
   ...

 <O> options:
   [ ] use letter size instead of A4 by default
   ...

 <V> set up for portable installation

Actions:
 <I> start installation to hard disk
 <P> save installation profile to 'texlive.profile' and exit
 <H> help
 <Q> quit
\end{boxedverbatim}
  \vskip-.5\baselineskip
  \caption{Fenêtre principale de l'installateur en mode texte (\GNU/Linux)}%
  \label{fig:text-main}
\end{figure}

\begin{figure}[tb]
  \tlpng{basic-w32}{.6\linewidth}{Fenêtre de l'installateur de base (Windows)}
  \caption{Fenêtre de l'installateur de base (Windows); le bouton « Avancé »
    quelque chose ressemblant à la
    figure~\ref{fig:advanced-lnx}}\label{fig:basic-w32}
\end{figure}

\begin{figure}[tb]
  \tlpng{advanced-lnx}{\linewidth}{Fenêtre de installateur en mode avancé (\GNU/Linux)}
  \caption{Fenêtre de installateur en mode \GUI{} avancé
    (\GNU/Linux)}\label{fig:advanced-lnx}
\end{figure}

\htmlanchor{cygwin}
\subsubsection{Installation sous Cygwin}
\label{sec:cygwin}

Avant de commencer l'installation, utilisez le programme \filename{setup.exe} de
Cygwin pour installer les paquets \filename{perl} et \filename{wget} (si ce
n'est déjà fait).

Il est également recommandé d'installer les paquets suivants :
\begin{itemize*}
\item \filename{fontconfig}, utilisé par \XeTeX\ et Lua\TeX{} ;
\item \filename{ghostscript}, nécessaire pour divers utilitaires ;
\item \filename{libXaw7}, utilisé par \code{xdvi} ;
\item \filename{ncurses}, qui fournit la commande \code{clear} utilisée par
  l'installateur.
\end{itemize*}

\subsubsection{Installation en mode texte}

La figure~\ref{fig:text-main} présente l'écran principal de configuration en
mode texte sous Unix/Linux, où c'est le mode par défaut.

Il s'agit d'un installateur en mode ligne de commande, il n'y a pas de notion de
curseur ; par exemple, vous ne pouvez pas naviguer entre les cases à cocher et
les champs d'entrée avec la touche tabulation. Tapez une des lettres proposées
(en respectant la casse) suivie de « entrée » (retour chariot) et l'écran se
mettra automatiquement à jour.

L'interface en mode texte est volontairement rudimentaire afin qu'elle fonctionne
sur le plus grand nombre possible de systèmes --- même avec des versions très
minimales de Perl.

\subsubsection{Installation en mode graphique}
\label{sec:graphical-inst}

L'installateur en mode graphique par défaut est au départ très simple, avec
juste quelques options ; voir figure~\ref{fig:basic-w32}. Il peut être lancé
avec la commande :
\begin{alltt}
> \Ucom{install-tl -gui}
\end{alltt}

Le bouton « Avancé » donne accès à la plupart des options de l'installateur en
mode texte ; voir figure~\ref{fig:advanced-lnx}.

\subsubsection{Les anciens installateurs}

Les modes \texttt{perltk}/\texttt{expert} et \texttt{wizard} sont toujours
disponibles sur les systèmes sur lesquels Perl/Tk est installé. Ils peuvent être
spécifiés avec les arguments respectivement \texttt{-gui=perltk} and
\texttt{-gui=wizard}.

\subsection{Choix des options d'installation}
\label{sec:runinstall}

Les options proposées sont censées être assez explicites, voici cependant
quelques précisions.

\subsubsection{Choix des binaires (Unix seulement)}
\label{sec:binary}

\begin{figure}[tbh]
  \centering
\begin{boxedverbatim}
Available platforms:
===============================================================================
   a [ ] Cygwin on x86_64 (x86_64-cygwin)
   b [ ] MacOSX current (10.14-) on ARM/x86_64 (universal-darwin)
   c [ ] MacOSX legacy (10.6-) on x86_64 (x86_64-darwinlegacy)
   d [ ] FreeBSD on x86_64 (amd64-freebsd)
   e [ ] FreeBSD on Intel x86 (i386-freebsd)
   f [ ] GNU/Linux on ARM64 (aarch64-linux)
   g [ ] GNU/Linux on RPi(32-bit) and ARMv7 (armhf-linux)
   h [ ] GNU/Linux on Intel x86 (i386-linux)
   i [X] GNU/Linux on x86_64 (x86_64-linux)
   j [ ] GNU/Linux on x86_64 with musl (x86_64-linuxmusl)
   k [ ] NetBSD on x86_64 (amd64-netbsd)
   l [ ] NetBSD on Intel x86 (i386-netbsd)
   m [ ] Solaris on Intel x86 (i386-solaris)
   o [ ] Solaris on x86_64 (x86_64-solaris)
   p [ ] Windows (64-bit) (windows)
\end{boxedverbatim}
  \vskip-.5\baselineskip
  \caption{Menu pour le choix des binaires}\label{fig:bin-text}
\end{figure}

La figure~\ref{fig:bin-text} présente les choix de binaires possibles en mode
texte.  Seuls les binaires correspondant à l'architecture détectée seront
installés par défaut.  Rien n'empêche d'en sélectionner d'autres si nécessaire :
cela peut être intéressant dans le cas de machines en réseau ou sur des systèmes
en « double boot ».

\subsubsection{Sélection de ce qui va être installé}
\label{sec:components}

\begin{figure}[tbh]
  \centering
\begin{boxedverbatim}
Select scheme:
===============================================================================
 a [X] full scheme (everything)
 b [ ] medium scheme (small + more packages and languages)
 c [ ] small scheme (basic + xetex, metapost, a few languages)
 d [ ] basic scheme (plain and latex)
 e [ ] minimal scheme (plain only)
 f [ ] infrastructure-only scheme (no TeX at all)
 g [ ] book publishing scheme (core LaTeX and add-ons)
 h [ ] ConTeXt scheme
 i [ ] GUST TeX Live scheme
 j [ ] teTeX scheme (more than medium, but nowhere near full)
 k [ ] custom selection of collections
\end{boxedverbatim}
  \vskip-.5\baselineskip
  \caption{Menu « Scheme »}\label{fig:scheme-text}
\end{figure}

Dans le menu « Scheme » (voir figure~\ref{fig:scheme-text}) on choisit un schéma
général de configuration qui détermine un ensemble de collections à installer.
Le schéma par défaut (\optname{full}) consiste à tout installer, ce qui est
recommandé. Vous pouvez également choisir \optname{basic} pour juste plain et
\LaTeX{}, \optname{small} pour quelques programmes supplémentaires (équivalents
à l'installation de Mac\TeX dite « Basic\TeX »), \optname{minimal} pour des
tests uniquement, \optname{medium} ou \optname{teTeX} pour quelque chose
d'intermédiaire. Il existe aussi des schémas spécialisés ou spécifiques à un
pays.

\begin{figure}[tb]
  \def\figdesc{Menu « Collections »}
  \centering \tlpng{stdcoll}{.7\linewidth}{\figdesc}
  \caption{Menu « Collections » (Linux)}\label{fig:collections-gui}
\end{figure}

Une fois un schéma choisi, vous pouvez affiner votre sélection avec le menu
« collections » (voir figure~\ref{fig:collections-gui}, montrée ici en mode
graphique pour changer).

Il est possible de raffiner encore les choix : pour ce faire, il faudra recourir
ultérieurement au gestionnaire de paquets \TL{} (\prog{tlmgr}) --- voir la
section~\ref{sec:tlmgr}.

\subsubsection{Répertoires d'installation}
\label{sec:directories}

La disposition par défaut des différents répertoires est donnée à la
section~\ref{sec:texmftrees}, page~\pageref{sec:texmftrees}.  Le répertoire
d'installation par défaut est \dirname{/usr/local/texlive/2023} sous Unix et
|%SystemDrive%\texlive\2023|
sous Windows. Cela permet d'avoir en parallèle plusieurs installations \TL{},
par exemple une par version (typiquement par année, comme ici) et vous pouvez
basculer de l'une à l'autre simplement en modifiant votre chemin de recherche
(votre « \emph{PATH} »).

Il peut être nécessaire de changer la valeur de \dirname{TEXDIR} lorsque celui
qui procède à l'installation n'a pas les droits d'écriture sur le répertoire
\dirname{TEXDIR} : l'installation n'est pas réservée au superutilisateur
\eng{root} ou « Administrateur », il suffit d'avoir les droits en écriture sur
le répertoire \dirname{TEXDIR}.

Ce répertoire d'installation peut être modifié en configurant le
\dirname{TEXDIR} dans l'installateur. L'écran de l'interface pour ceci -- et
pour d'autres choix -- est montré à la figure~\ref{fig:advanced-lnx}.  Les
principales raisons pour lesquelles il peut être nécessaire de le modifier sont
soit le manque de place sur cette partition (une installation \TL{} complète
nécessite plusieurs Go), soit l'absence de droit d'écriture sur le répertoire
par défaut : comme indiqué précédemment, l'installation n'est pas réservée au
super-utilisateur \eng{root} ou « Administrateur » ; il suffit d'avoir les
droits en écriture sur le répertoire \dirname{TEXDIR}.

Sous Windows, vous n'avez normalement pas besoin d'être un administrateur pour
créer |C:\texlive\2023| (ou, plus généralement, |%SystemDrive%\texlive\2023|).

Les répertoires d'installation peuvent aussi être modifiés en configurant
différentes variables d'environnement avant de lancer l'installateur
(vraisemblablement \envname{TEXLIVE\_INSTALL\_PREFIX} ou
\envname{TEXLIVE\_INSTALL\_TEXDIR}) ; consultez la documentation au moyen de
|install-tl --help| (disponible en ligne
à \url{https://tug.org/texlive/doc/install-tl.html}) pour la liste complète et
plus de détails.

Une alternative raisonnable est d'installer \TL{} dans votre répertoire
personnel, surtout si vous prévoyez d'en être le seul utilisateur. Vous pouvez
utiliser |~| à cet effet, par exemple |~/texlive/2023|.

Dans tous les cas, il est recommandé d'inclure l'année dans le chemin afin de
pouvoir conserver plusieurs versions de \TL{} en parallèle.  Rien n'empêche
d'ajouter aussi un lien symbolique (par exemple
\dirname{/usr/local/texlive-current}) pointant sur la version actuellement
utilisée. Ceci facilite les basculements d'une version à une autre.

Le répertoire \dirname{TEXMFHOME} est destiné à regrouper les répertoires de
macros personnelles. Par défaut, son emplacement est |~/texmf|
(|~/Library/texmf| isur Mac) ; ici, le |~| est préservé dans les fichiers de
configuration créés, de façon à être remplacé dynamiquement pour chaque
utilisateur de \TeX, par la valeur des variables d'environnement
\dirname{$HOME} %$
sous Unix et |%USERPROFILE%| sous Windows.
Attention, comme tous les autres répertoires, \dirname{TEXMFHOME} doit respecter
la structure \TDS{}, ce sans quoi les fichiers ne seront pas trouvés.

Enfin, \dirname{TEXMFVAR} est l'emplacement où sont stockés les fichiers de
cache spécifiques à chaque utilisateur. Le nom \dirname{TEXMFCACHE} est utilisé
par Lua\LaTeX\ et \ConTeXt\ MkIV pour le même but (voir la
section~\ref{sec:context-mkiv} page~\pageref{sec:context-mkiv}) ; par défaut il
coïncide avec \dirname{TEXMFSYSVAR} ou, s'il n'est pas possible d'y écrire, avec
\dirname{TEXMFVAR}.


\subsubsection{Options}
\label{sec:options}

\begin{figure}[tbh]
  \centering
\begin{boxedverbatim}
Options customization:
===============================================================================
 <P> use letter size instead of A4 by default: [ ]
 <E> execution of restricted list of programs: [X]
 <F> create all format files:                  [X]
 <D> install font/macro doc tree:              [X]
 <S> install font/macro source tree:           [X]
 <L> create symlinks in standard directories:  [ ]
            binaries to:
            manpages to:
                info to:
 <Y> after install, set CTAN as source for package updates: [X]
\end{boxedverbatim}
  \vskip-.5\baselineskip
  \caption{menu « Options » (Unix)}\label{fig:options-text}
\end{figure}

La figure~\ref{fig:options-text} présente le menu « Options » en mode texte, sur
lequel voici quelques précisions.

\begin{description}
\item[use letter size instead of A4 by default] -- la taille du papier par
  défaut pour des outils comme \prog{dvips}, \prog{pdftex}, \prog{xdvi}.  Cette
  option n'a pas d'influence sur les tailles par défaut utilisées par des jeux
  de macros, tels que les classes standard de \LaTeX\ ou ses extensions : ces
  derniers resteront prioritaires. Dans tous les cas, il est possible (et
  recommandé) de préciser la taille de papier souhaitée au sein de chaque
  document.

\item[execution of restricted list of programs] -- à compter de \TL 2010,
  l'exécution de certains programmes externes depuis \TeX{} est autorisée par
  défaut. La (très courte) liste de programmes autorisés se trouve dans le
  fichier \filename{texmf.cnf}. Voir la section~\ref{sec:2010news} pour plus de
  détails.

\item [create all format files] Nous recommandons de laisser cette option
  cochée, afin d'éviter des problèmes inutiles lors de la création dynamique de
  formats.  Consultez la documentation \prog{fmtutil} pour plus de détails.

  % \item[create all format files] -- création de tous les formats
  %   à l'installation.  Bien que cette opération prenne un peu de temps, il est
  %   conseillé de ne pas la supprimer, sinon les formats seront créés au coup
  %   par coup dans les répertoires personnels des utilisateurs (sous
  %   \envname{TEXMFVAR}). Les formats personnels ainsi créés ne bénéficieront
  %   pas des mises à jour éventuelles (par exemple packages de bases ou motifs
  %   de césure) effectuées ultérieurement sur l'installation \TL{}, ce qui peut
  %   engendrer des incompatibilités.

\item[install font/macro \ldots\ tree] -- ces options permettent d'installer les
  fichiers de documentation et les fichiers source présents dans la plupart des
  paquets. Décocher cette option n'est pas recommandé.

\item[create symlinks in standard directories] -- cette option (pour Unix
  seulement) permet d'éviter d'avoir à modifier les variables d'environnement
  \envname{PATH}, \envname{MANPATH} et \envname{INFOPATH} après
  l'installation. Elle nécessite les droits d'écriture dans les répertoires
  cibles. Cette option est destinée à l'accès au système \TeX\ depuis des
  répertoires standard tels que \dirname{/usr/local/bin} qui ne contiennent pas
  déjà de fichiers \TeX. Veillez à ne pas écraser de fichiers existants sur
  votre système avec cette option, par exemple en spécifiant des répertoires
  système. L'approche la plus sûre et recommandée est de laisser cette option
  décochée.
  \begin{description}
  \item[Note du traducteur.] Le traducteur (Denis Bitouzé) est d'un avis
    opposé (mais controversé) et recommande au contraire cette option : elle
    simplifie grandement l'installation puisqu'elle ne nécessite pas les
    configurations post-installation indiquées section~\ref{sec:env}. Cette
    méthode ne nécessite des précautions que dans les cas suivants.
    \begin{description}
    \item[Ajout d'un exécutable :] après que la \TL{} a été installée, elle peut
      être mise à jour, par exemple au moyen de :
\begin{alltt}
> tlmgr update --self --all
\end{alltt}
      Les éléments mis à jour sont la plupart du temps des packages et
      classes. Eux ne posent pas de problème mais il arrive aussi qu'un nouvel
      exécutable soit ajouté et, par défaut, celui-ci n'est pas directement
      accessible au système. Pour qu'il le soit, il est nécessaire de mettre
      à jour les liens symboliques et, pour ce faire, il suffit de lancer :
\begin{alltt}
> tlmgr path add
\end{alltt}
    \item[Installations multiples :] supposons par exemple que vous conserviez
      la \TL{} 2022 à côté de la \TL{} 2023 pour le cas où un document qui
      compilait avec la 1\iere{} ne compile plus avec la 2\ieme{}. Pour basculer
      de la 2023 à 2022, il suffit de lancer\footnote{En supposant que
        ces \TL{} ont été installées dans le dossier
        \dirname{/usr/local/texlive/}, sous-dossiers respectivement
        \dirname{2022} et \dirname{2023}.} :
\begin{alltt}
> /usr/local/texlive/2022/bin/x86_64-linux/tlmgr path add
\end{alltt}
      puis, pour revenir à la \TL{} 2023 :
\begin{alltt}
> /usr/local/texlive/2023/bin/x86_64-linux/tlmgr path add
\end{alltt}
    \item[Liens symboliques morts :] à la suite de l'installation de multiples
      \TL{}, il se peut que certains liens symboliques morts se trouvent dans le
      dossier des binaires (typiquement \dirname{/usr/local/bin}). Pour les
      supprimer, il suffit de lancer :
\begin{alltt}
> find /usr/local/bin -xtype l -delete
\end{alltt}
    \end{description}
  \end{description}

\item[after install, set CTAN as source for package updates] -- lors d'une
  installation depuis le \DVD, cette option est activée par défaut, car on
  souhaite généralement bénéficier des mises à jour de paquets depuis la partie
  du \CTAN\ les hébergeant, qui est mise à jour toute l'année. La seule raison
  probable de la désactiver est si vous installez seulement un sous-ensemble de
  \TL\ depuis le \DVD\ et prévoyez de la compléter ultérieurement, toujours
  depuis le \DVD. Dans tous les cas, le dépôt de paquets utilisé par
  l'installateur et celui utilisé après installation peuvent être choisis de
  façon indépendante ; voir les sections~\ref{sec:location}
  et~\ref{sec:dvd-install-net-updates}.
\end{description}

Options spécifiques à Windows, telles qu'affichées dans l'interface avancée
Perl/Tk :
\begin{description}
\item[adjust PATH setting in registry] -- ceci assure que tous les programmes
  vont voir le répertoire contenant les binaires \TL{} dans leur chemin de
  recherche.

\item[Add menu shortcuts] -- si coché, le menu \enquote{Démarrage} contiendra un
  sous-menu \TL{}. Il y a une 3\ieme{} option \enquote{Entrée de lanceur} en
  plus de \enquote{Menu \TL{}} et \enquote{Pas de raccourcis}. Cette option est
  décrite section~\ref{sec:sharedinstall}.

\item[Change file associations] -- ces options sont \enquote{Seulement les
    nouveaux} (créant des associations de fichier, mais n'en écrasant aucune),
  \enquote{Tous} et \enquote{Aucun}.

\item[Install \TeX{}works front end]
\end{description}

Lorsque vous êtes satisfait des réglages effectués, il vous reste à taper
\code{I} dans l'interface textuelle ou cliquer sur le bouton \enquote{Installer}
dans l'interface graphique Perl/Tk pour lancer le processus d'installation.
Lorsque celui-ci sera terminé, allez à la section~\ref{sec:postinstall} pour
voir s'il y a d'autres choses à faire.

\subsection{Options de install-tl en ligne de commande}
\label{sec:cmdline}

Tapez
\begin{alltt}
> \Ucom{install-tl -help}
\end{alltt}
pour obtenir la liste de toutes les options disponibles. Vous pouvez utiliser
|-| ou |--| pour introduire le noms des options.  Voici les plus courantes :

\begin{ttdescription}
\item[-gui] : exécution en mode \GUI{} si possible. Ceci nécessite Tcl/Tk
  version 8.5 ou plus. Il était distribué avec \macOS ; pour Big Sur et
  suivants, vous devrez installer Tcl/Tk vous-même, si vous ne si vous ne
  choisissez pas d'utiliser l'installateur MacTeX. Tcl/Tk est distribué avec
  \TL{} sous Windows. Les anciennes options \texttt{-gui=perltk} et
  \texttt{-gui=wizard} sont toujours disponibles et requièrent le module Perl/Tk
  (\url{https://tug.org/texlive/distro.html#perltk}) avec support de XFT ; au
  cas où ni Tcl/Tk, ni Perl/Tk ne sont disponible, l'installation se poursuit en
  mode texte.

\item[-no-gui] : exécution en mode texte.

\item[-lang {\sl LL}] : langue utilisée par l'interface de l'installateur,
  spécifiée par son code standard (généralement sur deux lettres).  Le programme
  s'efforce de déterminer automatiquement la langue à utiliser et se rabat sur
  l'anglais en cas d'échec. Vous pouvez obtenir la liste des langues disponibles
  avec \code{install-tl --help}.

\item[-portable] : créer une installation utilisable de façon portable sur une
  clé USB ou un \DVD ; peut aussi être activé depuis l'interface textuelle de
  l'installateur avec la commande \code{V}, ainsi que depuis l'interface
  graphique.  Voir la section~\ref{sec:portable-tl} pour les détails.

\item[-profile {\sl fichier}] : charger le profil d'installation depuis le
  \var{fichier} et installer sans interaction avec l'utilisateur.  À chaque
  exécution, le script d'installation écrit un compte-rendu dans le fichier
  \filename{texlive.profile} du sous-répertoire \dirname{tlpkg} de votre
  installation \TL. Ce fichier peut être donné en argument pour refaire
  exactement la même installation sur une machine différente, par
  exemple. Sinon, vous pouvez utiliser un profil personnalisé ; la façon la plus
  facile de le créer est de partir d'un profil généré et de modifier les
  valeurs, ou d'utiliser un fichier vide pour utiliser toutes les valeurs par
  défaut.

\item[-repository {\sl url-ou-répertoire}] : choix d'une source où récupérer le
  matériel à installer, voir ci-dessous.

  \htmlanchor{opt-in-place}
\item[-in-place] : si vous possédez déjà une copie de \TL{} acquise par rsync,
  svn ou un autre moyen (voir
  \url{https://tug.org/texlive/acquire-mirror.html}), cette option vous permet
  de l'utiliser directement, tel quel, et de ne procéder qu'aux actions
  post-installation. Attention, ceci peut écraser le fichier
  \filename{tlpkg/texlive.tlpdb}, c'est à vous de le sauvegarder auparavant si
  vous le désirez. Aussi, une éventuelle suppression de paquet est à faire
  manuellement. N'utilisez pas cette option à moins de savoir ce que vous
  faites. Cette option ne peut pas être activée depuis l'interface de
  l'installateur.
\end{ttdescription}

\subsubsection{L'option \optname{-repository}}
\label{sec:location}

L'emplacement par défaut du dépôt de paquets en ligne est un miroir du \CTAN{}
choisi automatiquement par le service de redirection
\url{https://mirror.ctan.org/}.

Si vous voulez en utiliser un autre, vous pouvez utiliser l'option
\optname{-repository} avec pour valeur une url commençant par \texttt{ftp:},
\texttt{http:}, \texttt{https:} ou \texttt{file:/}, ou un chemin vers un
répertoire local.  Dans le cas d'une url en \texttt{https:}, \texttt{http:} ou
\texttt{ftp:}, un éventuel caractère « \texttt{/} » à la fin, de même qu'une
éventuelle composante \texttt{tlpkg/} finale, sont ignorés.

Par exemple, vous pouvez choisir un miroir du \CTAN{} en particulier avec une
valeur comme \url{https://ctan.example.org/tex-archive/systems/texlive/tlnet/},
en substituant le nom d'un vrai miroir et le chemin vers l'archive \TeX{}
spécifique à la place de |ctan.example.org|. La liste des miroirs du CTAN est
disponible sur \url{https://ctan.org/mirrors}.

Si l'argument donné est local (un chemin ou une url en \texttt{file:/}) et que
les paquets sont présents à la fois sous forme de fichiers non compressés et
d'archives compressées, ces dernières seront utilisées.

\htmlanchor{postinstall}
\subsection{Étapes post-installation}
\label{sec:postinstall}

Selon les cas, quelques opérations supplémentaires peuvent être nécessaires.

\subsubsection{Variables d'environnement sous Unix}
\label{sec:env}

Si vous avez choisi de créer des liens symboliques dans les répertoires standard
(voir la section~\ref{sec:options}), alors il est inutile de modifier vos
variables d'environnement. Sinon, sur les systèmes Unix, le répertoire contenant
les binaires pour votre plateforme doit être ajouté au \envname{PATH} (pas sous
Windows où l'installateur s'en occupe).

À chaque architecture correspond un sous-répertoire de \dirname{TEXDIR/bin},
voir la liste à la figure~\ref{fig:bin-text} \pageref{fig:bin-text}.

Si vous voulez que votre système trouve les fichiers de documentation concernant
\TL{} aux formats man et Info, il faut également ajuster les variables
\envname{MANPATH} et \envname{INFOPATH}. Sur certains systèmes, ceci ne sera pas
nécessaire et il suffira de régler le \envname{PATH} pour que les pages de man
et d'Info soient trouvées.

Pour les interpréteurs de commandes (\textit{shells}) dits « Bourne-compatible »
tels que \prog{bash} sous GNU/Linux (et la configuration du répertoire par
défaut \TL\ à titre d'exemple), le fichier à éditer peut être
\filename{$HOME/.bash_profile} (ou \filename{$HOME/.profile}) et les lignes
à ajouter sont de la forme suivante :

\begin{sverbatim}
PATH=/usr/local/texlive/2023/bin/x86_64-linux:$PATH; export PATH
MANPATH=/usr/local/texlive/2023/texmf-dist/doc/man:$MANPATH; export MANPATH
INFOPATH=/usr/local/texlive/2023/texmf-dist/doc/info:$INFOPATH; export INFOPATH
\end{sverbatim}
% stopzone $

Pour les \textit{shells} \prog{csh} ou \prog{tcsh}, le fichier à éditer est en
principe \filename{$HOME/.cshrc} %$
et les lignes à ajouter sont de la forme suivante :
\begin{sverbatim}
setenv PATH /usr/local/texlive/2023/bin/x86_64-linux:$PATH
setenv MANPATH /usr/local/texlive/2023/texmf-dist/doc/man:$MANPATH
setenv INFOPATH /usr/local/texlive/2023/texmf-dist/doc/info:$INFOPATH
\end{sverbatim}
% $

Si vous n'êtes pas sur une plateforme \code{x86\_64-linux}, utilisez le nom de
plateforme approprié ; de même, si vous n'avez pas effectué l'installation dans
le répertoire par défaut, modifiez le nom du répertoire. Le programme
d'installation \TL\ indique les lignes complètes à utiliser à la fin de
l'installation.

Si vous avez déjà des paramètres \envname{PATH} quelque part dans vos fichiers
de démarrage, fusionnez les répertoires \TL\ comme vous le souhaitez.

\subsubsection{Variables d'environnement : configuration multi-utilisateur}
\label{sec:envglobal}

Il est possible d'ajuster les variables \envname{PATH}, \envname{MANPATH} et
\envname{INFOPATH} globalement pour tous les utilisateurs présents et futurs
sans avoir à éditer les fichiers personnels de chacun, mais les façons de le
faire sont trop différentes d'un système à l'autre pour être présentées ici.

Voici quelques pistes : pour \envname{MANPATH} chercher un fichier
\filename{/etc/manpath.config}, s'il est présent lui ajouter des lignes comme
\begin{sverbatim}
MANPATH_MAP /usr/local/texlive/2023/bin/i386-linux \
            /usr/local/texlive/2023/texmf-dist/doc/man
\end{sverbatim}
Pour les deux autres, chercher un fichier\filename{/etc/environment}, il est
éventuellement possible d'y définir les valeurs des variables d'environnement
telles que \envname{PATH} et \envname{INFOPATH}.

Nous créons aussi à l'installation un lien symbolique nommé \code{man} dans les
répertoires des binaires Unix. Certains programmes \code{man}, comme celui livré
en standard sur \macOS{}, l'utilisent pour trouver automatiquement les pages de
\code{man}, rendant inutile tout réglage de \envname{MANPATH}.

\subsubsection{Mises à jour par Internet après une installation par le
  \protect\DVD}
\label{sec:dvd-install-net-updates}

Si vous avez installé \TL{} depuis le \DVD{} et souhaitez ensuite accéder aux
mises à jour par Internet, il vous faudra exécuter la commande suivante (après
avoir réglé votre \envname{PATH} si nécessaire, comme expliqué à la section
précédente) :

\begin{alltt}
> \Ucom{tlmgr option repository https://mirror.ctan.org/systems/texlive/tlnet}
\end{alltt}

Ceci dit à \cmdname{tlmgr} d'utiliser pour les futures mises à jour un miroir du
\CTAN\ proche. Ceci est fait par défaut lors d'une installation depuis le \DVD,
via l'option décrite dans la section~\ref{sec:options}.

Si vous rencontrez des problèmes avec la sélection automatique du miroir, vous
pouvez en sélectionner un en particulier depuis la liste disponible en
\url{https://ctan.org/mirrors}. Utilisez le chemin complet vers le répertoire
\dirname{tlnet} pour ce miroir, comme ci-dessus.


\htmlanchor{xetexfontconfig} % keep historical anchor working
\htmlanchor{sysfontconfig}
\subsection{Configuration des fontes pour \protect\XeTeX\ et Lua\protect\TeX}
\label{sec:font-conf-sys}

\XeTeX\ et Lua\TeX\ peuvent utiliser n'importe quelle police installée sur le
système, et pas seulement celles des arborescences \TeX. Ces polices système
(qui ne font pas partie de \TL) sont généralement accessibles en donnant le nom
de la police, par exemple `\code{Liberation Serif}', bien que le nom de fichier
système puisse être également utilisé.

Une question connexe consiste à rendre les polices de la distribution \TL\
disponibles en tant que polices système, ce qui les rendra à leur tour
accessibles par leur nom.

\begin{description}
\item[Pour Lua\TeX :] pour l'accès par nom de police, il n'y a rien de spécial
  à faire. Toutes les polices de la distribution \TL\ devraient être également
  accessibles par nom de police ou par nom de fichier pour Lua\TeX, via le
  paquet \pkgname{luaotfload} qui prend en charge à la fois \LaTeX\ et plain
  \TeX{}. L'index des noms de police \pkgname{luaotfload} peut avoir besoin
  d'être reconstruit pour les nouvelles polices ; ceci est déclenché
  automatiquement lors de la tentative de chargement d'une police qui n'est pas
  encore connue.
\item[Pour \XeTeX :] sous :
  \begin{description}
  \item[Windows :] les polices livrées avec \TL\ sont automatiquement mises
    à disposition (en exécutant le programme \cmdname{fc-cache} fourni pour
    Windows par la \TL) ;
  \item[\macOS{} :] vous devrez consulter d'autres documents ;
  \item[systèmes Unix autres que \macOS{} :] la procédure est la suivante :
    lorsque le paquetage \pkgname{xetex} est installé (lors de l'installation
    initiale ou ultérieurement), le fichier de configuration nécessaire est créé
    dans \filename{TEXMFSYSVAR/fonts/conf/texlive-fontconfig.conf}. Pour rendre
    les polices \TL\ disponibles en tant que polices système :
    \begin{enumerate*}
    \item Copiez le fichier \filename{texlive-fontconfig.conf} dans
      (généralement) \dirname{/etc/fonts/conf.d/09-texlive.conf}.
    \item Exécutez \Ucom{fc-cache -fsv}.
    \end{enumerate*}
    Si vous n'avez pas les privilèges suffisants pour utiliser la méthode
    ci-dessus, ou si vous préférez rendre les polices disponibles seulement pour
    l'utilisateur en cours, vous pouvez procéder comme suit :
    \begin{enumerate*}
    \item copiez le fichier \filename{texlive-fontconfig.conf} dans
      (généralement) \filename{~/.fonts.conf.d/09-texlive.conf}, où \filename{~}
      représente votre répertoire personnel.
    \item exécutez \Ucom{fc-cache -fv}.
    \end{enumerate*}
  \end{description}
\end{description}

Vous pouvez exécuter \code{fc-list} pour afficher les noms des polices système
disponibles. L'incantation \code{fc-list : family style file spacing} (tous ces
arguments sont des chaînes littérales) affiche des informations généralement
intéressantes.

\subsubsection{\protect\ConTeXt{} LMTX et Mark IV}
\label{sec:context-mkiv}

Les \enquote{vieilles} versions de \ConTeXt{} (Mark II et Mark IV) ainsi que la
\enquote{nouvelle} version (LMTX) devraient fonctionner directement après
l'installation de \TL et ne réclamer aucune action particulière tant que vous
n'utilisez que \code{tlmgr} pour les mises à jour.

Cependant, comme \ConTeXt{} n'utilise pas la bibliothèque kpathsea, il
faudra mettre à jour manuellement le cache de fichiers de \ConTeXt{} si jamais
vous installez des nouveaux fichiers manuellement (c'est-à-dire sans utiliser
\code{tlmgr}). Pour cela, et après chaque installation manuelle, exécutez :
\begin{description}
\item[pour LMTX :]\leavevmode{}
\begin{sverbatim}
context --generate
\end{sverbatim}
\item[pour Mark IV :]\leavevmode{}
\begin{sverbatim}
context --luatex --generate
\end{sverbatim}
\end{description}
afin de rafraîchir les données \ConTeXt{} du cache du disque.

Les fichiers résultants de l'exécution de cette commande sont stockés dans
\code{TEXMFCACHE}, dont la valeur par défaut sous \TL{} est
\envname{TEXMFSYSVAR;TEXMFVAR}.

\ConTeXt\ lira les fichiers de toutes les arborescences mentionnées dans
\envname{TEXMFCACHE} et les écrira dans le premier de ces répertoires
accessibles en écriture. Lors de la lecture, les dernières informations lues
sont prises en compte de façon prioritaire par rapport à celles lues
précédemment, dans le cas où certaines données de cache seraient dupliquées.

Pour plus d'informations, voir :
\begin{itemize}
\item \url{https://wiki.contextgarden.net/LMTX} ;
\item \url{https://wiki.contextgarden.net/Running_Mark_IV}.
\end{itemize}


\subsubsection{Ajout de fichiers locaux ou personnels}
\label{sec:local-personal-macros}

Comme cela a déjà été indiqué à la section~\ref{sec:texmftrees},
\dirname{TEXMFLOCAL} (par défaut \dirname{/usr/local/texlive/texmf-local} ou
\verb|%SystemDrive%\texlive\texmf-local| sous Windows) est la racine de
l'arborescence prévue pour regrouper les fichiers de macros, les fontes et le
matériel utilisable par l'ensemble des utilisateurs du système.  D'autre part
\dirname{TEXMFHOME} (par défaut \dirname{$HOME/texmf} %$
(ou |%USERPROFILE%\texmf|) regroupe le matériel personnel de chaque
utilisateur.

Dans les deux cas, les fichiers ajoutés doivent être placés, non pas en vrac
à la racine de \dirname{TEXMFLOCAL} ou \dirname{TEXMFHOME} mais dans des
sous-répertoires bien choisis (voir \url{https://tug.org/tds} ou lire le fichier
\filename{texmf.cnf}). Par exemple, une classe ou une extension \LaTeX{} ne sera
trouvée que si elle est dans \dirname{TEXMFLOCAL/tex/latex} ou
\dirname{TEXMFHOME/tex/latex} ou dans un sous-répertoire de ceux-ci.

Enfin, si les ajouts ont été faits sous \dirname{TEXMFLOCAL}, il y a lieu de
régénérer les bases de données \filename{ls-R} (commande \cmdname{mktexlsr} ou,
en mode graphique, bouton « Mettre à jour l'index de fichiers » dans l'onglet
« Actions ») de l'interface graphique du \TL{} « Manager ».

L'emplacement des répertoires \dirname{TEXMFLOCAL} et \dirname{TEXMFHOME} est
fixe (il ne change pas d'une version à l'autre de \TL{}). Leur contenu est pris
en compte par toutes les versions de \TL{}, aussi est-il préférable de ne pas
changer la valeur des variables \dirname{TEXMFLOCAL} et \dirname{TEXMFHOME}
à l'installation.

Par défaut, chacune de ces variables est définie comme étant un seul répertoire,
comme illustré ci-dessus. Cependant, ceci n'est pas obligatoire.  Si vous voulez
par exemple changer rapidement de version pour des paquets importants, vous
pouvez maintenir plusieurs arborescences pour votre usage personnel en
spécifiant dans \dirname{TEXMFHOME} une liste de répertoires entre accolades,
séparés par des virgules :

\begin{verbatim}
  TEXMFHOME = {/my/dir1,/mydir2,/a/third/dir}
\end{verbatim}

La section~\ref{sec:brace-expansion} décrit plus en détail la façon dont de
telles listes entre accolades sont traitées.


\subsubsection{Ajout de fontes externes à \protect\TL{}}

Ce sujet est malheureusement assez délicat en ce qui concerne \TeX\ et pdf\TeX{}
donc ne vous y attardez pas à moins que vous souhaitiez vous plonger dans les
arcanes d'une installation \TeX{}.

De nombreuses polices étant déjà intégrées à \TL, nous vous conseillons de
vérifier en premier lieu si ce que vous cherchez ne s'y trouve pas : les pages
de \url{https://tug.org/FontCatalogue} listent, classées de différentes façons,
la quasi-totalité des polices de texte incluses dans les principales
distributions \TeX.

Si vous voulez toutefois installer vos propres polices, vous pouvez consulter le
document :

\url{https://tug.org/fonts/fontinstall.html}

qui essaie de décrire au mieux la procédure pour ce faire.

Une alternative possible est d'utiliser \XeTeX{} ou Lua\TeX{} (voir
section~\ref{sec:tex-extensions}) qui permettent l'accès aux fontes disponibles
sur le système sans aucune installation supplémentaire. (Mais prenez garde au
fait que l'utilisation l'utilisation de polices système rend vos sources
instantanément inutilisables par toute personne travaillant dans un
environnement différent.)

\subsection{Test de l'installation}
\label{sec:test-install}

Après avoir installé \TL{} aussi bien que possible, il faut la tester avant de
créer des documents ou des fontes.  Les tests doivent être faits par un
utilisateur non privilégié (autre que \eng{root}).

Une chose que vous devez immédiatement chercher est une interface graphique (un
éditeur) avec laquelle éditer des fichiers. \TL{} installe \TeX{}works
(\url{https://tug.org/texworks}) sur Windows (seulement) et Mac\TeX\ installe
TeXShop (\url{https://pages.uoregon.edu/koch/texshop}). Sur les autres systèmes
Unix, le choix d'un éditeur vous revient. Il y en a de nombreux et certains
d'entre eux sont listés à la section suivante (cf. aussi
\url{https://tug.org/interest.html#editors}). Tout éditeur de texte brut
convient : il n'est pas requis qu'il soit spécifique à \TeX{}.

Le reste de cette section donne quelques procédures de base pour vérifier que le
nouveau système est opérationnel. On les décrit pour Unix ; pour \macOS{} ou
Windows, il vaut mieux tester au travers d'une interface graphique, mais les
principes sont les mêmes.

\begin{enumerate}

\item S'assurer en premier lieu que le programme \cmdname{tex} fonctionne :

\begin{alltt}
> \Ucom{tex -{}-version}
TeX 3.14159265 (TeX Live ...)
Copyright ... D.E. Knuth.
...
\end{alltt}

Si la réponse est \texttt{command not found} ou si le numéro de version est
différent, il est fort probable que vous n'ayez pas le bon répertoire de
binaires dans votre \envname{PATH}.  Voir les informations sur l'environnement
page~\pageref{sec:env}.

\item Traiter un fichier \LaTeX{} simple :

\begin{alltt}
> \Ucom{pdflatex sample2e.tex}
This is pdfTeX 3.14...
...
Output written on sample2e.pdf (3 pages, 142120 bytes).
Transcript written on sample2e.log.
\end{alltt}
Si le fichier \filename{sample2e.tex} ou d'autres ne sont pas trouvés, il est
possible qu'il y ait des interférences avec vos anciennes variables
d'environnement ou fichiers de configuration. Pour analyser en détail votre
problème, vous pouvez demander à \TeX{} de dire exactement ce qu'il cherche et
trouve ; voir la section~\ref{sec:debugging} page~\pageref{sec:debugging}.

\item Prévisualiser le fichier PDF, par exemple au moyen de :
\begin{alltt}
> \Ucom{xpdf sample2e.pdf}
\end{alltt}
Vous devriez voir une nouvelle fenêtre avec un joli document expliquant
certaines des bases de \LaTeX{}.  (Au fait, ça vaut la peine de lire si
vous êtes nouveau à \TeX.)

Bien sûr, il existe de nombreux autres lecteurs de PDF ; sur les systèmes Unix,
\cmdname{evince} et \cmdname{okular} sont couramment utilisés. Pour Windows,
nous recommandons d'essayer Sumatra PDF
(\url{https://www.sumatrapdfreader.org/free-pdf-reader.html}). Aucun afficheur
PDF n'est inclus dans \TL{} et vous devez donc installer séparément celui que
vous voulez utiliser.

\item Bien sûr, vous pouvez toujours générer le format originel \dvi{} de
  \TeX{} :
\begin{alltt}
> \Ucom{latex sample2e.tex}
\end{alltt}

\item Prévisualiser les résultats :
\begin{alltt}
> \Ucom{xdvi sample2e.dvi}    # Unix
> \Ucom{dviout sample2e.dvi}  # Windows
\end{alltt}
Nota : vous devez travailler sous X pour que \cmdname{xdvi} fonctionne.  Dans le
cas contraire, votre variable d'environnement \envname{DISPLAY} ne sera pas
correcte et vous obtiendrez l'erreur \samp{Can't open display}.

\item Créer un fichier \PS{} à partir du \dvi{} :
\begin{alltt}
> \Ucom{dvips sample2e.dvi -o sample2e.ps}
\end{alltt}

\item Ou pour créer un fichier PDF à partir du \dvi{}, une méthode alternative
  à l'utilisation pdf\TeX\ (ou Xe\TeX\ ou Lua\TeX) qui peut parfois être utile :
\begin{alltt}
> \Ucom{dvipdfmx sample2e.dvi -o sample2e.pdf}
\end{alltt}

\item Autres fichiers de tests utiles en plus de \filename{sample2e.tex} :

  \begin{ttdescription}

  \item[small2e.tex] à compiler avant \filename{sample2e} si celui-ci pose des
    problèmes.

  \item[testpage.tex] teste que l'imprimante n'introduit pas de décalages.

  \item[nfssfont.tex] imprime des tables des fontes et des tests.
  \item[testfont.tex] aussi pour les tables de fontes, mais en (plain)\TeX.
  \item[story.tex] le fichier de test de (plain)\TeX{} le plus canonique de
    tous. Il faut taper \samp{\bs bye} à l'invite \code{*} après \samp{tex
      story.tex}.
  \end{ttdescription}

\item Si vous avez installé le paquet \pkgname{xetex}, vous pouvez vérifier s'il
  a bien accès aux polices du système ainsi :
  \begin{alltt}
> \Ucom{xetex opentype-info.tex}
This is XeTeX, Version 3.14\dots
...
Output written on opentype-info.pdf (1 page).
Transcript written on opentype-info.log.
\end{alltt}

En cas de message d'erreur contenant \samp{Invalid fontname `Latin Modern
  Roman/ICU'\dots}, alors vous devez revoir la configuration de votre système
pour que les polices installées par \TL\ soient reconnues, voir
section~\ref{sec:font-conf-sys}.

\end{enumerate}

\htmlanchor{uninstall}
\subsection{Désinstaller la \TL}
\label{sec:uninstall}

Pour désinstaller \TL\ (après une installation réussie ; pour Windows, voir
ci-dessous) :

\begin{alltt}
> \Ucom{tlmgr uninstall --all}
\end{alltt}

Une confirmation vous sera demandée, sinon rien ne sera fait.  (Sans
\code{-{}-all}, l'action \code{uninstall} est utilisée pour supprimer les
packages individuellement).

Ceci ne supprime pas les répertoires spécifiques à l'utilisateur, à savoir (voir
aussi section~\ref{sec:texmftrees}) :

\begin{ttdescription}
\item [TEXMFCONFIG] Ceci est destiné aux changements de configuration de
  l'utilisateur.  Si vous souhaitez les conserver, assurez-vous de savoir
  comment les recréer avant de les supprimer.

\item [TEXMFVAR] Ceci est destiné à stocker les données d'exécution générées
  automatiquement, telles que les fichiers de format local. À moins que vous ne
  l'ayez utilisé à d'autres fins, vous pouvez le supprimer en toute sécurité.

\item [TEXMFHOME] Contient uniquement les fichiers que vous avez vous-même
  installés, généralement ceux qui ne sont pas disponibles dans les
  distributions. À moins que vous n'arrêtiez complètement d'utiliser \TeX, ou
  que vous souhaitiez repartir de zéro, vous n'avez probablement pas envie de
  supprimer cet élément.

\end{ttdescription}

\noindent Vous pouvez trouver les chemins de répertoire de ces variables en
exécutant \code{kpsewhich -var-value=\ttvar{var}}.

Cette désinstallation de \prog{tlmgr} n'annule pas non plus les actions
post-installation, telles que les modifications de \envname{PATH} dans les
fichiers d'initialisation de votre shell et l'accès du système aux polices de
\TL\ (voir la section~\ref{sec:postinstall}). Vous devez annuler manuellement
ces actions, si vous le souhaitez.

Sous Windows, la désinstallation peut être effectuée via l'interface utilisateur
graphique (\GUI) ; voir la section~\ref{sec:winfeatures}.

\subsection{Liens vers d'autres logiciels téléchargeables}

Si vous êtes débutant ou si vous avez besoin d'aide pour réaliser des documents
\TeX{} ou \LaTeX{}, n'hésitez pas à consulter \url{https://tug.org/begin.html}.

% liens GUTenberg

Voici quelques liens vers d'autres outils qui peuvent être utiles à installer.
\begin{description}
\item[Ghostscript] \url{https://ghostscript.com/}, un interpréteur PostScript et
  PDF gratuit.
\item[Perl] \url{https://perl.org/} avec des paquets supplémentaires du CPAN,
  \url{https://cpan.org/}
\item[ImageMagick] \url{https://imagemagick.com} pour les conversions entre
  formats graphiques, notamment.
\item[NetPBM] \url{http://netpbm.sourceforge.net/} également pour les
  graphiques.

\item[Éditeurs orientés \TeX] Le choix est large, et est en bonne partie une
  question de goût personnel. Voici une sélection classée par ordre alphabétique
  (dont certains ne sont disponibles que sous Windows).
  \begin{itemize*}
  \item \cmdname{GNU Emacs} est disponible pour toutes les principales
    plateformes ; voir \url{https://www.gnu.org/software/emacs}.
  \item \cmdname{AUC\TeX} fonctionne sous Emacs ; il est disponible par le biais
    du gestionnaire de paquets \cmdname{ELPA} d'Emacs. Les sources sont
    également disponibles sur le CTAN. La page d'accueil de AUC\TeX\ est
    \url{https://www.gnu.org/software/auctex}.
  \item \cmdname{SciTE} (Windows seulement) est disponible sur
    \url{https://www.scintilla.org/SciTE.html}.
  \item \cmdname{Texmaker} est disponible sur
    \url{https://www.xm1math.net/texmaker/index_fr.html }.
  \item \cmdname{TeXstudio} était au départ un \eng{fork} de \cmdname{Texmaker}
    avec des fonctionnalités supplémentaires : \url{https://texstudio.org/}.
  \item \cmdname{TeXnicCenter} est disponible sur
    \url{https://www.texniccenter.org}.
  \item \cmdname{TeXworks} est disponible sur \url{https://tug.org/texworks} et
    inclus dans \TL sur Windows (seulement).
  \item \cmdname{Vim} est disponible sur \url{https://www.vim.org}.
  \item \cmdname{WinEdt} (Windows seulement, non libre) est disponible sur
    \url{https://tug.org/winedt} or \url{https://www.winedt.com}.
  \item \cmdname{WinShell} (Windows seulement) est disponible sur
    \url{https://www.winshell.de}.
  \end{itemize*}
\end{description}
Voir \url{https://tug.org/interest.html} pour une liste plus complète de
programmes.


\section{Installations spécialisées}

Les sections précédentes décrivaient le processus pour installation
\enquote{normale}. Celle-ci concerne des cas plus spécialisés.

\htmlanchor{tlsharedinstall}
\subsection{Installation partagée entre plusieurs utilisateurs ou machines}
\label{sec:sharedinstall}

\TL a été conçue pour pouvoir être partagée par différents systèmes sur un
réseau.  Avec la disposition standard des répertoires, aucun chemin n'est codé
en dur : les emplacements des fichiers dont \TL a besoin sont trouvés
automatiquement à partir des emplacements des programmes. Vous pouvez le
constater dans le fichier de configuration principal
\filename{$TEXMFDIST/web2c/texmf.cnf}, %$
qui contient des lignes comme
\begin{sverbatim}
TEXMFROOT = $SELFAUTOPARENT
...
TEXMFDIST = $TEXMFROOT/texmf-dist
...
TEXMFLOCAL = $SELFAUTOGRANDPARENT/texmf-local
\end{sverbatim}
% $
Cela signifie qu'il suffit aux utilisateurs d'ajouter à leur \envname{PATH} le
chemin des exécutables pour leurs plateformes, pour obtenir une configuration
qui marche.

Pour la même raison, vous pouvez aussi installer \TL localement et ensuite
déplacer l'arborescence complète vers un emplacement réseau.

Pour Windows, \TL{} inclut un programme de lancement nommé
\filename{tlaunch}. Sa fenêtre principale contient des entrées de menus et des
boutons pour différents programmes et documentations liés à \TeX{}
personnalisables via un fichier \code{ini}. Lors de son premier usage, il
reproduit la post-installation usuelle sous Windows, c'est-à-dire qu'il modifie
le chemin de recherche pour la \TL\ et crée quelques associations de fichiers,
mais seulement pour l'utilisateur en cours. De ce fait, les stations de travail
ayant accès à la \TL{} sur un réseau local ne nécessitent qu'un raccourci de
menu pour le lanceur. Cf. le manuel de \code{tlaunch} (\code{texdoc tlaunch} ou
\url{https://ctan.org/pkg/tlaunch}).

\htmlanchor{tlportable}
\section{Installations « portables » de \protect\TL}
\label{sec:portable-tl}

L'option \code{-portable} de l'installateur (ou la commande \code{V} dans
l'installateur en mode texte ou l'option correspondante en mode graphique) crée
une installation de \TL{} entièrement contenue dans un seul répertoire et
n'effectue aucune intégration au système. Vous pouvez créer une telle
installation directement sur une clé \USB{} ou la copier sur une clé
ultérieurement.

Techniquement, l'installation portable est rendue autonome en rendant les
valeurs par défaut de \dirname{TEXMFHOME}, \envname{TEXMFVAR}, et
\envname{TEXMFCONFIG} respectivement identiques à \dirname{TEXMFLOCAL},
\envname{TEXMFSYSVAR}, et \envname{TEXMFSYSCONFIG} ; la configuration par
utilisateur et les caches ne seront donc pas créés.

Pour utiliser \TeX\ depuis cette installation portable, il suffit d'ajouter le
bon répertoire de binaires à votre \code{path} pour la session de terminal en
cours, comme d'habitude.

Sous Windows, vous pouvez double-cliquer sur le fichier \filename{tl-tray-menu}
à la racine de l'installation et créer un \enquote{menu receveur} temporaire
offrant quelques actions communes, comme dans cette capture d'écran :

\medskip
\tlpng{tray-menu}{4cm}{Windows tray menu}
\smallskip

\noindent L'entrée « More\dots » explique comment personnaliser ce menu.

% \htmlanchor{tlisoinstall}
% \subsection{Installation sur \protect\DVD ou image \protect\ISO}
% \label{sec:isoinstall}
%
% Si vous n'avez pas besoin de mettre à jour ni de modifier votre installation
% souvent, et/ou que vous désirez utiliser \TL{} sur plusieurs systèmes, il peut
% être commode de créer une image \ISO\ de votre installation \TL, pour les
% raisons suivantes :
%
% \begin{itemize}
% \item il est nettement plus rapide de copier d'une machine à l'autre une image
%   \ISO\ qu'une installation classique ;
% \item si vous êtes en multi-boot sur plusieurs systèmes et voulez leur faire
%   partager une même installation de \TL, une image \ISO\ vous isole des
%   idiosyncrasies et limitations des divers systèmes de fichiers (FAT32, NTFS,
%   HFS+).
% \item les machines virtuelles peuvent monter une image \ISO\ très simplement.
% \end{itemize}
%
% Bien sûr vous pouvez aussi graver votre image \ISO\ sur une \DVD, si cela vous
% est utile.
%
% Les systèmes \GNU/Linux et autres Unix de bureau, y compris \macOS, sont
% capables de monter une image \ISO. Windows 8 est la première (!) version de
% Windows qui peut le faire. À part ce montage, rien de change par rapport à une
% installation normale sur le disque dur, voir la section~\ref{sec:env}.
%
% Dans la préparation d'une telle installation, il est préférable d'omettre le
% sous-répertoire avec le numéro de l'année et d'avoir \filename{texmf-local} au
% même niveau que les autres arborescences (\filename{texmf-dist},
% \filename{texmf-var}, etc.), ce qui est possible avec les options habituelles
% de l'installateur.
%
% Pour une machine Windows physique (et non virtuelle) vous pouvez graver
% l'\ISO\ sur un \DVD. Cependant, il est peut-être plus rentable pour vous de
% vous renseigner sur les différentes options libres ou gratuites de montage
% d'images \ISO, par exemple WinCDEmu : \url{http://wincdemu.sysprogs.org/}.
%
% Pour l'intégration au système sous Windows, vous pouvez inclure dans l'image
% le script \filename{w32client} décrit en section~\ref{sec:sharedinstall} et
% à l'adresse \url{https://tug.org/texlive/w32client.html} ; il fonctionne aussi
% bien pour une image \ISO\ que pour une installation partagée par le réseau.
%
% Sous \macOS, TeXShop peut utiliser une telle installation si un lien
% symbolique \filename{/usr/texbin} existe et pointe vers le bon répertoire, par
% exemple
% \begin{verbatim}
% sudo ln -s /Volumes/MyTeXLive/bin/universal-darwin /usr/texbin
% \end{verbatim}
%
% Remarque historique : \TL\ 2010 a été la première édition à ne pas être
% distribuée en version « live ». Cependant, toutes les éditions précédentes
% exigeaient quelques acrobaties pour être utilisées depuis le \DVD ; en
% particulier il n'y avait pas moyen de s'en sortir sans devoir régler au moins
% une variable d'environnement supplémentaire. Si vous créez votre \ISO\ ou
% \DVD\ depuis une installation existante ceci n'est plus nécessaire.

\htmlanchor{tlmgr}
\section{Maintenance de l'installation avec \cmdname{tlmgr}}
\label{sec:tlmgr}

\begin{figure}[tb]
  \def\figdesc{\GUI{} \prog{tlshell}, montrant le menu Actions (\GNU/Linux)}
  \tlpng{tlshell-linux}{\linewidth}{\figdesc}
  \caption{\figdesc}
  \label{fig:tlshell}
\end{figure}

\begin{figure}[tb]
  \def\figdesc{\GUI{} \prog{tlcockpit} pour \prog{tlmgr}}
  \tlpng{tlcockpit-packages}{.8\linewidth}{\figdesc}
  \caption{\figdesc}
  \label{fig:tlcockpit}
\end{figure}

\begin{figure}[tb]
  \def\figdesc{Ancien mode \GUI\ pour \prog{tlmgr}: fenêtre principale, après
    « Chargement »}
  \tlpng{tlmgr-gui}{\linewidth}{\figdesc}
  \caption{\figdesc}
  \label{fig:tlmgr-gui}
\end{figure}

\TL fournit un programme appelé \prog{tlmgr} pour assurer la maintenance de la
distribution après son installation initiale. Il permet en particulier
\begin{itemize}
\item d'installer, de mettre à jour ou de désinstaller des paquets
  individuellement, éventuellement en respectant les dépendances ;
\item de rechercher des paquets, d'obtenir leur liste et leurs descriptions,
  etc. ;
\item de voir la liste des plateformes binaires et d'en installer ou d'en
  supprimer ;
\item de modifier la configuration, par exemple la taille du papier par défaut,
  le dépôt de paquets par défaut (voir la section~\ref{sec:location}).
\end{itemize}

Les fonctionnalités de \prog{tlmgr} surpassent celles de \prog{texconfig}. Nous
continuons à distribuer et maintenir ce dernier pour le confort de ceux habitués
à son interface, mais nous recommandons d'utiliser \prog{tlmgr} désormais.

\subsection{\GUI{} actuelles pour le \cmdname{tlmgr}}

\TL{} fournit plusieurs \GUI{} pour \prog{tlmgr}. Deux notables d'entre elles
sont :
\begin{enumerate}
\item \cmdname{tlshell} (figure~\ref{fig:tlshell}) qui est écrit en Tcl/Tk et
  fonctionne d'emblée sous Windows ;
\item \prog{tlcockpit} (figure~\ref{fig:tlcockpit}) qui nécessite Java version~8
  ou supérieure et JavaFX.
\end{enumerate}
Les deux sont des paquets séparés.

\prog{tlmgr} dispose lui-même d'une \GUI{} native (figure~\ref{fig:tlmgr-gui})
qui peut être lancée au moyen de :
\begin{alltt}
> \Ucom{tlmgr -gui}
\end{alltt}
Cependant, cette extension \GUI\ requiert Perl/Tk, dont le module n'est plus
fourni par la distribution Perl de \TL{} pour Windows.

\subsection{Exemples d'utilisation de \cmdname{tlmgr} en ligne de commande}

Après l'installation initiale, vous pouvez mettre à jour votre système en
utilisant :

\begin{alltt}
> \Ucom{tlmgr update -all}
\end{alltt}
Si cela vous inquiète, vous pouvez commencer par :
\begin{alltt}
> \Ucom{tlmgr update -all -dry-run}
\end{alltt}
ou (moins bavard) :
\begin{alltt}
> \Ucom{tlmgr update -list}
\end{alltt}

L'exemple suivant, plus complexe, ajoute une collection, pour le moteur \XeTeX,
depuis un dépôt local :

\begin{alltt}
> \Ucom{tlmgr -repository /local/mirror/tlnet install collection-xetex}
\end{alltt}
La sortie, abrégée, ressemble à ceci :
\begin{fverbatim}
install: collection-xetex
install: arabxetex
...
install: xetex
install: xetexconfig
install: xetex.i386-linux
running post install action for xetex
install: xetex-def
...
running mktexlsr
mktexlsr: Updating /usr/local/texlive/2023/texmf-dist/ls-R...
...
running fmtutil-sys --missing
...
Transcript written on xelatex.log.
fmtutil: /usr/local/texlive/2023/texmf-var/web2c/xetex/xelatex.fmt installed.
\end{fverbatim}
Comme vous pouvez le constater, \prog{tlmgr} prend en compte les dépendances et
effectue automatiquement toutes les opérations nécessaires, comme la mise à jour
des bases de données de fichiers et la génération de formats (ici un nouveau
format a été créé pour \XeTeX).

La commande suivante permet d'obtenir la description d'un paquet.
\begin{alltt}
> \Ucom{tlmgr show collection-latexextra}
\end{alltt}
Elle retourne par exemple quelque chose comme :
\begin{fverbatim}
package:    collection-latexextra
category:   Collection
shortdesc:  LaTeX supplementary packages
longdesc:   A very large collection of add-on packages for LaTeX.
installed:  Yes
revision:   46963
sizes:	    657941k
\end{fverbatim}

Enfin, le plus important, pour la documentation complète, tapez tout
simplement :
\begin{alltt}
> \Ucom{tlmgr help}
\end{alltt}
ou consultez l'adresse suivante :

\url{https://tug.org/texlive/tlmgr.html}

\section{Notes concernant Windows}
\label{sec:windows}

\subsection{Fonctionnalités supplémentaires à l'installation}
\label{sec:winfeatures}

Sous Windows le programme d'installation effectue quelques tâches
supplémentaires :
\begin{description}
\item[Menus et raccourcis] -- un sous-menu « \TL{} » est ajouté au menu
  « Démarrer ». Il contient des entrées pour quelques programmes graphiques,
  tels que \prog{tlshell} (une \GUI{} pour \prog{tlmgr}) et \prog{dviout}, et
  quelques entrées pour la documentation.

\item[Associations de fichiers] -- si cette option n'est pas désactivée,
  \prog{TeXworks} et \prog{Dviout} deviennent le programme par défaut pour
  ouvrir leurs types de fichiers respectifs ou, s'il y a déjà un tel programme,
  sont ajoutés à la liste « Ouvrir avec... » du menu contextuel. Cependant, les
  associations de fichiers plus prioritaires « choisies par l'utilisateur », qui
  ne peuvent être spécifiées que de manière interactive, peuvent prendre le
  dessus.

\item[Convertisseur bitmap vers eps] -- une entrée \cmdname{bitmap2eps} est
  ajoutée à l'entrée « Ouvrir avec... » du menu contextuel pour de nombreux
  formats d'images bitmap. \prog{Bitmap2eps} est un script simple qui utilise
  \cmdname{sam2p} ou \cmdname{bmeps} pour faire le vrai travail.

\item[Support PostScript] -- pour les fichiers PostScript, un type de fichier
  PSviewer convertit désormais PostScript en un PDF temporaire, qui est ensuite
  affiché par le visualiseur PDF par défaut. Pour la conversion en EPS, une
  entrée \cmdname{bitmap2eps} est ajoutée à l'entrée « Ouvrir avec... » du menu
  contextuel pour de nombreux formats d'images bitmap, laissant
  \cmdname{sam2p} ou \cmdname{bmeps} faire le vrai travail.

\item[Ajustement automatique du \code{path}] -- aucune intervention manuelle
  n'est nécessaire.

\item[Désinstallation] -- une entrée est ajoutée pour \TL{}, soit dans le menu
  « Ajout et suppression de programmes » (pour une installation en tant
  qu'administrateur), soit dans le menu \TL{} (pour une installation
  mono-utilisateur)

\item[Protection en écriture] -- Pour une installation en tant
  qu'administrateur, les répertoires de la \TL\ sont protégés en écriture, au
  moins si la \TL\ est installée sur un disque normal formaté en NTFS et non
  amovible.
\end{description}

Pour une autre approche, cf. \filename{tlaunch}, décrit à la
section~\ref{sec:sharedinstall}.

\subsection{Programmes supplémentaires}

Pour être complète, une installation \TL a besoin de quelques utilitaires qui ne
sont pas présents en général sur les machines Windows. \TL fournit donc les
outils suivants (installés sur Windows seulement).
\begin{description}
\item [Perl, Tcl/Tk et Ghostscript]-- en  raison de l'importance de Perl et de
  Ghostscript, et parce que l'installateur et les \GUI{} de tlshell sont écrits
  en Tcl/Tk, \TL{} inclut des copies `cachées' de ces programmes. Les programmes
  \TL{} qui en ont besoin savent où les trouver, mais ils ne trahissent pas leur
  présence par des variables d'environnement ou des paramètres de registre. Il
  ne s'agit pas d'installations complètes (sauf pour Ghostscript), et elles ne
  devraient pas interférer avec les installations système de Perl, Tcl/Tk ou
  Ghostscript. Voir la sous-section \ref{sec:externalwndws} pour savoir comment
  indiquer à \TL{} que vous souhaitez utiliser vos propres installations
  externes pour les scripts contribués dans \TL.

\item[dviout] -- lecteur de DVI. La première fois que vous visualisez un fichier
  avec \prog{dviout}, il va créer des polices, car les versions des fontes pour
  écran ne sont pas installées. Au bout d'un moment, la plupart des polices que
  vous utilisez auront été créées, et vous ne verrez plus que rarement la
  fenêtre de création des fontes. Vous pouvez trouver plus d'informations dans
  le menu d'aide du logiciel (dont la lecture est recommandée).

\item[\TeX{}works] -- \TeX{}works est un éditeur conçu pour les fichiers \TeX,
  avec un lecteur de PDF intégré.

\item[Outils en ligne de commande] -- des versions pour Windows de programmes
  Unix sont installées, en particulier \cmdname{gzip}, \cmdname{zip},
  \cmdname{unzip}, \cmdname{jpeg2ps}, \cmdname{chktex}, \cmdname{wget} et
  quelques utilitaires de la suite \cmdname{poppler} (comme \cmdname{pdfinfo} ou
  \cmdname{pdffonts}) ; aucun afficheur PDF autonome pour Windows n'est
  inclus. Une option est l'afficheur Sumatra PDF disponible
  à \url{https://www.sumatrapdfreader.org/}.
\item[\prog{fc-list}, \prog{fc-cache}, etc.] -- outils de la bibliothèque
  \pkgname{fontconfig} permettant à \XeTeX{} d'accéder aux polices système sous
  Windows. Vous pouvez utiliser \prog{fc-list} pour connaître la liste des noms
  de police utilisables avec la commande \cs{font} de \XeTeX.
\end{description}

\subsection{Utilisation d'installations externes de Perl, Tcl/Tk et Ghostscript}
\label{sec:externalwndws}

Normalement, \TL{} utilise ses installations intégrées Perl, Tcl/Tk et
Ghostscript également pour les scripts contribués dans \TL. Si vous souhaitez
utiliser vos propres versions externes, vous pouvez le configurer dans le
fichier \file{texmf.cnf} \emph{à la racine de l'installation}.

Pour Perl, vous devez ajouter une ligne
\begin{verbatim}
TEXLIVE_WINDOWS_TRY_EXTERNAL_PERL = 1
\end{verbatim}
\TL{} (plus précisément, \file{bin/windows/runscript.tlu}) recherchera alors
\file{perl.exe} sur le chemin de recherche, sauf pour les scripts qui
appartiennent à l'infrastructure \TL{}. Il s'agit du paramètre le plus
susceptible d'être utile ; bien que l'infrastructure \TL{} Perl inclut de
nombreux modules supplémentaires, il ne peut pas prendre en charge tous les
scripts tiers.

De même, pour Tcl/Tk, vous avez besoin d'une ligne
\begin{verbatim}
TEXLIVE_WINDOWS_TRY_EXTERNAL_TCL = 1
\end{verbatim}
\TL{} recherchera alors les fichiers \file{tclkit.exe}, \file{wish.exe},
\file{wish85.exe}, \file{wish86.exe} et \file{wish87.exe} sur le chemin de
recherche.

Ghostscript est traité différemment, dans la mesure où vous devez spécifier le
nom de fichier ou le chemin complet de votre Ghostscript en ligne de commande :
\begin{alltt}
TEXLIVE_WINDOWS_EXTERNAL_GS = \var{chemin de la ligne de commande ghostscript}
\end{alltt}
Une autre différence est que le Ghostscript fourni avec \TL\ est complet, seuls
la documentation et les pilotes d'imprimante étant omis.  Il est donc peu
probable que vous ayez besoin de le remplacer.

Voir également la section \ref{sec:configfiles} à propos du fichier \file{texmf.cnf}.

\subsection{Répertoire personnel}
\label{sec:winhome}

L'équivalent du répertoire personnel noté \envname{\$HOME} sous Unix s'appelle
|%USERPROFILE%| sous Windows. Cette variable vaut en général
|C:\Utilisateurs\<username>| sous Vista et versions suivantes.  La notation
\verb|~|, utilisée dans \filename{texmf.cnf} et dans \KPS{} en général pour
désigner un répertoire personnel, est correctement interprétée sous Windows
comme sous Unix.


\subsection{Base de registre Windows}
\label{sec:registry}
% NdT% choix : "registre", pas "registres"

Windows stocke pratiquement tous les paramètres de configuration dans sa base de
registre. Celle-ci contient un ensemble de clés organisées par niveau.  Les clés
les plus importantes pour l'installation de programmes sont
\path{HKEY_CURRENT_USER} et \path{HKEY_LOCAL_MACHINE}, \path{HKCU} et
\path{HKLM} en abrégé. La partie \path{HKCU} de la base de registre se trouve
dans le répertoire personnel de l'utilisateur (voir
section~\ref{sec:winhome}). La partie \path{HKLM} est normalement dans un
sous-répertoire du répertoire Windows.

Certaines informations système peuvent s'obtenir à partir des variables
d'environnement mais, pour d'autres, la localisation des raccourcis par exemple,
la consultation de la base de registre est indispensable.  La modification des
variables d'environnement nécessite l'accès à la base de registre.


\subsection{Droits d'accès sous Windows}
\label{sec:winpermissions}

Dans les versions récentes de Windows, la distinction est faite entre
« utilisateurs » et « administrateurs », ces derniers ayant accès en écriture
à la totalité du système. Nous nous sommes efforcés de rendre l'installation de
\TL{} possible aux utilisateurs non privilégiés.

Si l'installateur est lancé avec des droits « administrateur », il dispose d'une
option procédant à l'installation pour tous les utilisateurs : si elle est
choisie, les raccourcis et les entrées de menu sont créés pour tous les
utilisateurs, et le chemin de recherche est modifié au niveau du système. Sinon,
les raccourcis et entrées de menu sont créés pour l'utilisateur courant, et seul
le chemin de recherche dudit utilisateur est modifié.

Dans tous les cas, le répertoire proposé comme racine de l'installation est
\verb|%SystemDrive%|. %
Le programme \prog{install-tl} vérifie si le répertoire choisi comme racine est
accessible en écriture pour celui qui procède à l'installation.

Lorsqu'une installation \TeX{} est présente sur la machine, l'installation de
\TL{} par un utilisateur non privilégié est problématique : cet utilisateur
n'aura jamais accès aux exécutables \TL{} car la recherche s'effectue d'abord
dans les répertoires système, puis dans les répertoires de l'utilisateur.  Une
parade sommaire a été prévue : l'installateur crée un raccourci vers une fenêtre
de ligne de commande où les exécutables \TL sont prioritaires : \TL{} est ainsi
utilisable en ligne de commande à partir de ce raccourci. De même, le raccourci
pour \TeX{}works, s'il est installé, fait ce qu'il faut pour que les outils \TL
soient utilisés.

Vous devez savoir que, même si vous êtes connecté en tant qu'administrateur,
vous devez explicitement demander les privilèges d'administrateur.  En fait, il
ne sert pas à grand-chose de se connecter en tant qu'administrateur ; le mieux
est de faire un clic droit sur le programme ou raccourci à utiliser, et de
choisir l'entrée « exécuter en tant qu'administrateur ».


\subsection{Augmentation de la limite de mémoire sous Windows et Cygwin}
\label{sec:cygwin-maxmem}

Les utilisateurs de Windows et Cygwin (voir la section~\ref{sec:cygwin} pour les
spécificités de l'installation sous Cygwin) peuvent parfois arriver à court de
mémoire en utilisant certains des programmes fournis avec \TL. Par exemple,
\prog{asy} peut manquer de mémoire si vous essayez d'allouer un tableau de
25\,000\,000 réels, et Lua\TeX\ aussi si vous essayez de traiter un document
avec beaucoup de fontes lourdes.

Pour Cygwin, vous pouvez augmenter la quantité de mémoire disponible en suivant
les instructions données dans le guide de l'utilisateur Cygwin
(\url{https://www.cygwin.com/cygwin-ug-net/setup-maxmem.html}).

Pour Windows, il vous faudra créer un fichier, nommé par exemple
\code{ajoutmemoire.ref}, contenant les quatre lignes suivantes.

\begin{sverbatim}
Windows Registry Editor Version 5.00

[HKEY_LOCAL_MACHINE\Software\Cygwin]
"heap_chunk_in_mb"=dword:ffffff00
\end{sverbatim}

\noindent puis exécuter la commande \code{regedit /s ajoutmemoire.reg} en tant
qu'administrateur. Si vous souhaitez modifier la limite mémoire uniquement pour
l'utilisateur courant plutôt qu'au niveau du système, utilisez
\code{HKEY\_CURRENT\_USER}.


\section{Guide d'utilisation du système Web2C}

\Webc{} est une collection intégrée de programmes relatifs à \TeX,
c.-à-d. \TeX{} lui-même, \MF{}, \MP, \BibTeX{}, etc.  C'est le cœur de \TL{}. Le
site de référence est \url{https://tug.org/web2c}.

Un peu d'histoire : la première implémentation a été réalisée par Tomas Rokicki
qui, en 1987, a développé un premier système \TeX{}-to-C en adaptant les
\textit{change files} pour Unix (travail de Howard Trickey et Pavel Curtis
principalement).  Tim Morgan assura la maintenance du système, dont le nom fut
remplacé durant cette période par Web-to-C.  En 1990, Karl Berry reprit le
travail, assisté par des dizaines de contributeurs et en 1997 il passa le relais
à Olaf Weber qui le rendit à Karl en 2006.

% NdT% ci-dessous, "sources originales", on ne pense pas au /fichiers/ source.
% (enfin, je crois... (mpg))
Le système \Webc{} fonctionne sur Unix, sur les systèmes Windows, sur
\macOS{} et sur de nombreux autres systèmes d'exploitation.  Il utilise les
sources originales de Knuth pour \TeX{} et les autres programmes de base écrits
en \web{} (système de programmation documentée) qui sont tous traduits en
langage C. Les composants du noyau de \TeX{} issus de ce processus sont :

\begin{cmddescription}
\item[bibtex] -- gestion des bibliographies.
\item[dvicopy] -- copie de fichier \dvi{} en supprimant les fontes virtuelles.
\item[dvitomp] -- conversion de fichier \dvi{} en MPX (dessins \MP{}).
\item[dvitype] -- conversion le fichier \dvi{} en un texte lisible.
\item[gftodvi] -- visualisation de fontes génériques GF.
\item[gftopk] -- conversion des fontes génériques GF en fontes bitmap PK.
\item[gftype] -- conversion de fichier GF en un texte lisible.
\item[mf] -- création de fontes.
\item[mft] -- mise en page de code source \MF{}.
\item[mpost] -- création de diagrammes techniques.
\item[patgen] -- création de motifs de césure.
\item[pktogf] -- conversion des fontes bitmap PK en fontes génériques GF.
\item[pktype] -- conversion des fontes PK en un texte lisible.
\item[pltotf] -- conversion des fichiers PL (lisibles) en TFM.
\item[pooltype] -- affichage des fichiers \web{} pool.
\item[tangle] -- \web{} vers Pascal.
\item[tex] -- composition de textes.
\item[tftopl] -- conversion des fichiers TFM en PL (lisibles).
\item[vftovp] -- conversion des fontes virtuelles VF en VPL (lisibles).
\item[vptovf] -- conversion des fontes VPL en fontes virtuelles VF.
\item[weave] -- \web{} vers \TeX.
\end{cmddescription}

La syntaxe et les fonctions précises de ces programmes sont décrites dans la
documentation des composants individuels et dans le manuel \Webc{} lui-même.
Toutefois, connaître un certain nombre de principes régissant l'ensemble de la
famille de programmes peut aider à exploiter de façon optimale votre
installation \Webc{}.

Presque tous ces programmes suivent les options standard de \GNU{} :
\begin{ttdescription}
\item[\texttt{-{}-help}] imprime le sommaire de l'utilisation ;
\item[\texttt{-{}-version}] imprime seulement le numéro de version.
\end{ttdescription}

Et la plupart honorent :
\begin{ttdescription}
\item[\texttt{-{}-verbose}] imprime le rapport détaillé du processus.
\end{ttdescription}

Pour localiser les fichiers, les programmes \Webc{} utilisent la bibliothèque de
recherche \KPS{} (\url{https://tug.org/kpathsea}). Cette bibliothèque utilise
une combinaison de variables d'environnement et un certain nombre de fichiers de
paramètres pour optimiser la recherche dans l'énorme arborescence \TeX{}.
\Webc{} peut exécuter une recherche dans plusieurs arborescences simultanément,
ce qui est utile si l'on souhaite maintenir la distribution standard de \TeX{}
et les extensions locales dans deux arborescences distinctes. Afin d'accélérer
la recherche de fichiers, la racine de chaque arborescence possède un fichier
\file{ls-R} contenant une entrée donnant le nom et le chemin de chaque fichier
situé sous la racine.

\subsection{Kpathsea et la recherche de fichiers}
\label{sec:kpathsea}

Décrivons en premier lieu le mécanisme de recherche de la bibliothèque \KPS{}.

Nous appelons \emph{chemin de recherche} une liste d'éléments, séparés par
« deux-points» ou « point-virgule», et appelés \emph{éléments de chemin}, qui
sont des noms de répertoires.  Un chemin de recherche peut provenir de plusieurs
sources.  Pour rechercher un fichier \samp{my-file} le long d'un chemin
\samp{.:/dir}, \KPS{} vérifie chaque élément du chemin : d'abord
\filename{./my-file}, puis \filename{/dir/my-file}, etc.  Puis \KPS{} renvoie la
première occurrence (voire toutes).

Afin d'optimiser l'adaptation à tous les systèmes d'exploitation, \KPS{} peut
utiliser dans les noms de fichiers des séparateurs différents de deux-points
(\samp{:}) et barre oblique (\samp{/}) pour les systèmes non-Unix.

Pour vérifier un élément de chemin particulier \var{p}, \KPS{} vérifie d'abord
si une base de données existante (voir page~\pageref{sec:filename-database})
contient \var{p}, c.-à-d. si la base de données se trouve dans un répertoire qui
est un préfixe de \var{p}. Si oui, la spécification du chemin est comparée avec
le contenu de la base de données.

Bien que l'élément de chemin le plus simple et le plus fréquent soit un nom de
répertoire, \KPS{} prend en charge d'autres types d'éléments dans les chemins de
recherche : des valeurs par défaut différentes pour chaque programme, des noms
de variables d'environnement, des valeurs de fichiers de configuration, les
répertoires de l'utilisateur et la recherche récursive de sous-répertoires. Nous
disons alors que \KPS{} \emph{étend} un élément, c'est-à-dire que \KPS{}
transforme toutes ces spécifications en noms de répertoires de base. Cette
opération est décrite dans les sections suivantes.

Notons que si le nom de fichier cherché est absolu ou explicitement relatif,
c'est-à-dire commençant par \samp{/}, \samp{./} ou \samp{../}, \KPS{} ne vérifie
que l'existence de ce fichier.  \ifSingleColumn \else
\begin{figure*}
  \centering \verbatiminput{examples/ex5.tex} \setlength{\abovecaptionskip}{0pt}
  \caption{Un exemple de fichier de configuration}
  \label{fig:config-sample}
\end{figure*}
\fi

\subsubsection{Les différentes sources}
\label{sec:path-sources}

Un chemin de recherche peut provenir de plusieurs sources. Voici l'ordre dans
lequel \KPS\ les utilise.

\begin{enumerate}
\item Une variable d'environnement définie par l'utilisateur, par exemple
  \envname{TEXINPUTS}. Les variables d'environnement avec une extension attachée
  (nom de programme) sont d'abord prises en compte : par exemple, si
  \samp{latex} est le nom du programme exécuté, \envname{TEXINPUTS.latex}
  passera avant \envname{TEXINPUTS}.
\item Un fichier de configuration de programme spécifique, par exemple une ligne
  « \texttt{S /a:/b} » dans le fichier \file{config.ps} de \cmdname{dvips}.
\item Un fichier de configuration \file{texmf.cnf} de \KPS{} contenant une ligne
  telle que \\ \samp{TEXINPUTS=/c:/d} (voir ci-dessous).
\item La valeur par défaut obtenue à la compilation.
\end{enumerate}
\noindent On peut voir chacune de ces valeurs pour un chemin de recherche donné
en utilisant l'option de débogage (voir page~\pageref{sec:debugging}).

\subsubsection{Fichiers de configuration}
\label{sec:configfiles}

\KPS{} lit les chemins de recherche et d'autres définitions dans des
\emph{fichiers de configuration à l'exécution} nommés \file{texmf.cnf}. Le
chemin \envname{TEXMFCNF} est utilisé pour rechercher ces fichiers, mais nous ne
recommandons pas de définir cette variable d'environnement (ou toute autre) pour
passer outre les répertoires du système.

À la place, un fichier \file{\dots/2023/texmf.cnf} est créé à la suite d'une
installation normale. Si vous devez changer les valeurs par défaut de certaines
variables (ce qui n'est en général pas nécessaire), c'est le bon endroit pour le
faire. Le fichier de configuration principal est dans
\file{\dots/2023/texmf-dist/web2c/texmf.cnf}. Vous ne devriez jamais éditer ce
fichier car vos changements seraient alors perdus lors de futures mises à jour
de la distribution.

En passant, si vous souhaitez simplement ajouter un répertoire personnel à un
chemin de recherche particulier, une méthode raisonnable consiste à définir une
variable d'environnement :
\begin{verbatim}
  TEXINPUTS=.:/my/macro/dir:
\end{verbatim}
Pour que cette configuration puisse être maintenue et portable au fil des ans,
utilisez un \samp{:} final (\samp{;} sur Windows) pour insérer les chemins
d'accès système, au lieu d'essayer de les écrire tous explicitement (voir
section~\ref{sec:default-expansion}). Une autre option consiste à utiliser
l'arborescence \dirname{TEXMFHOME} (voir section~\ref{sec:directories}).

Il est important de noter que \emph{tous} les fichiers \file{texmf.cnf} trouvés
seront lus et que, en cas de conflit, la première définition lue l'emporte. Par
exemple, si les fichiers \file{texmf.cnf} sont cherchés dans le chemin
\verb|.:$TEXMF|, les valeurs de \file{./texmf.cnf} l'emportent sur celles de
\verb|$TEXMF/texmf.cnf|.

\begin{itemize*}
\item Les commentaires sont signalés par un \code{\%}, soit au début d'une
  ligne, soit précédés d'un espace, et se terminent à la fin de la ligne.
\item Les lignes vides sont ignorées.
\item Un \bs{} à la fin d'une ligne joue le rôle d'un lien entre deux lignes,
  c'est-à-dire que la ligne courante se poursuit à la ligne suivante. Dans ce
  cas, les espaces présents au début de la ligne suivante ne sont pas ignorés.
\item Toutes les autres lignes sont de la forme :\\
  \hspace*{2em}\texttt{\var{variable} \textrm{[}.\var{progname}\textrm{]}
    \textrm{[}=\textrm{]} \var{value}}\\[1pt]
  où le \samp{=} et les espaces autour sont optionnels. (Mais si \var{valeur}
  commence par \samp{.}, il est plus simple d'utiliser le \samp{=} pour éviter
  que le point ne soit interprété comme le qualificatif du nom du programme).
\item Le nom de la \ttvar{variable} peut contenir n'importe quel caractère autre
  que les espaces, \samp{=}, ou \samp{.} mais on recommande d'utiliser
  \samp{A-Za-z\_} pour éviter les problèmes.
\item Si \samp{.\var{progname}} est présent, sa définition s'applique seulement
  si le programme exécuté se nomme \texttt{\var{progname}} ou
  \texttt{\var{progname}.exe}. Ceci permet par exemple à différentes variantes
  de \TeX{} d'avoir des chemins de recherche différents.
\item \var{value} peut contenir n'importe quel caractère excepté « \code{\%} »
  et \samp{@}.  L'option \code{\$\var{var}.\var{prog}} n'est pas disponible
  à droite du signe \samp{=} ; à la place, on doit utiliser une variable
  supplémentaire. Un \samp{;}\ dans \var{value} est compris comme un \samp{:}\
  si on travaille sous Unix ; ceci est très utile et permet d'avoir un seul
  \file{texmf.cnf} pour les systèmes Unix, MS-DOS et Windows.
\item Considérée comme une chaîne de caractères, \var{valeur} peut contenir
  n'importe quel caractère.  Toutefois, dans la pratique, la plupart des valeurs
  \file{texmf.cnf} sont liées l'expansion du chemin et, comme divers caractères
  spéciaux sont utilisés dans l'expansion (voir
  section~\ref{sec:cnf-special-chars}), tels les accolades et les virgules, ils
  ne peuvent pas être utilisés dans les noms de répertoires.

  Un \samp{;} dans \var{value} est traduit par \samp{:} s'il est exécuté sous
  Unix, afin de disposer d'un seul \file{texmf.cnf} qui puisse prendre en charge à la
  fois les systèmes Unix et Windows. Cette traduction se produit pour toute
  valeur, et pas seulement pour les chemins de recherche mais, heureusement, en
  pratique, le \samp{;} n'est pas nécessaire dans d'autres valeurs.

  La fonction \code{\$\var{var}.\var{prog}} n'est pas disponible du côté droit ;
  vous devez à la place utiliser une variable supplémentaire.
\item Avant tout désarchivage ou décompactage, toutes les définitions sont lues
  de telle façon que les variables peuvent être référencées avant d'être
  définies.
\end{itemize*}
Voici un fichier de configuration illustrant les points précédents :
\ifSingleColumn

\begin{verbatim}
TEXMF              = {$TEXMFLOCAL,!!$TEXMFMAIN}
TEXINPUTS.latex    = .;$TEXMF/tex/{latex,generic;}//
TEXINPUTS.fontinst = .;$TEXMF/tex//;$TEXMF/fonts/afm//
% e-TeX related files
TEXINPUTS.elatex   = .;$TEXMF/{etex,tex}/{latex,generic;}//
TEXINPUTS.etex     = .;$TEXMF/{etex,tex}/{eplain,plain,generic;}//
\end{verbatim}

\else dans la figure~\ref{fig:config-sample}.  \fi

\subsubsection{Expansion d'un chemin de recherche}
\label{sec:path-expansion}

\KPS{} reconnaît certains caractères et constructions spéciales dans les chemins
de recherche, semblables à ceux disponibles dans les \textit{shells}
Unix. Ainsi, le chemin \verb+~$USER/{foo,bar}//baz+ %$
étend la recherche vers tous les sous-répertoires situés sous les répertoires
\file{foo} et \file{bar} dans le répertoire utilisateur \texttt{\$USER}
contenant un répertoire ou un fichier appelé %$
\file{baz}. Ces expansions sont explicitées dans les sections suivantes.

\subsubsection{Expansion par défaut}
\label{sec:default-expansion}

Si le chemin de recherche le plus prioritaire (voir
section~\ref{sec:path-sources}) contient un \samp{:} \emph{supplémentaire}
(\mbox{c.-à-d.} en début ou fin de ligne ou double), \KPS{} insère à cet endroit
le chemin suivant dont la priorité définie est immédiatement inférieure. Si ce
chemin inséré possède un \samp{:} supplémentaire, le même processus se répète
pour le chemin prioritaire suivant.  Par exemple, étant donné une variable
d'environnement définie ainsi :

\begin{alltt}
> \Ucom{setenv TEXINPUTS /home/karl:}
\end{alltt}
la valeur de \code{TEXINPUTS} d'après le fichier \file{texmf.cnf} étant :

\begin{alltt}
  .:\$TEXMF//tex
\end{alltt}
alors la valeur finale utilisée pour la recherche sera :

\begin{alltt}
  /home/karl:.:\$TEXMF//tex
\end{alltt}

Comme il est inutile d'insérer la valeur par défaut en plusieurs endroits,
\KPS{} applique la substitution à seulement un \samp{:} supplémentaire et laisse
les autres inchangés : il cherche d'abord un \samp{:} en début de ligne, puis en
fin de ligne et enfin un double \samp{:}.

\subsubsection{Expansion spécifiée par les accolades}
\label{sec:brace-expansion}

Option utile, l'expansion par le biais des accolades signifie, par exemple, que
\verb+v{a,b}w+ va permettre la recherche dans \verb+vaw:vbw+. Les définitions
emboîtées sont autorisées. Ceci peut être utilisé pour établir des hiérarchies
\TeX{} multiples en attribuant une liste entre accolades à \code{\$TEXMF}. Dans
le fichier \file{texmf.cnf} fourni, on trouve une définition qui ressemble (il
y a en fait plus de répertoires) à la suivante.
\begin{verbatim}
  TEXMF = {$TEXMFVAR,$TEXMFHOME,!!$TEXMFLOCAL,!!$TEXMFDIST}
\end{verbatim}
Nous utilisons ensuite ceci pour définir, par exemple, le chemin d'accès \TeX\ :
\begin{verbatim}
  TEXINPUTS = .;$TEXMF/tex//
\end{verbatim}
% ce qui signifie que, après avoir cherché dans le répertoire courant, les
% arborescences complètes \code{\$TEXMFVAR/tex}, \code{\$TEXMFHOME/tex},
% \code{\$TEXMFLOCAL/tex} (sur le disque) et ensuite les arborescences
% \code{!!\$TEXMFVAR/tex} et \code{!!\$TEXMFMAIN/tex} (en utilisant le fichier
% de référence \file{ls-R} \emph{seulement}) seront inspectées. C'est un moyen
% pratique permettant d'utiliser en parallèle deux distributions \TeX{}, une
% « figée » (sur un \CD, par exemple) et une autre régulièrement mise à jour
% avec de nouvelles versions quand elles deviennent disponibles. En utilisant la
% variable \code{\$TEXMF} dans toutes les définitions, on est toujours sûr
% d'inspecter d'abord l'arborescence la plus récente.
ce qui signifie que, après avoir cherché dans le répertoire courant, les
arborescences complètes \code{\$TEXMFVAR/tex}, \code{\$TEXMFHOME/tex},
\code{\$TEXMFLOCAL/tex} et \code{\$TEXMFDIST/tex} seront inspectées (les deux
derniers utilisant les fichiers de référence \file{ls-R}).

\subsubsection{Expansion des sous-répertoires}
\label{sec:subdirectory-expansion}

Deux barres \samp{//} ou plus consécutives dans une partie d'un chemin suivant
un répertoire \var{d} sont remplacées par tous les sous-répertoires de \var{d} :
d'abord les sous-répertoires directement présents dans \var{d}, ensuite les
sous-répertoires de ceux-ci et ainsi de suite. À chaque niveau, l'ordre dans
lequel les répertoires sont inspectés est \emph{non déterminé}.

Dans le cas où l'on spécifie une partie de nom de fichier après le \samp{//},
seuls sont inclus les sous-répertoires auxquels le nom correspond. Par exemple,
\samp{/a//b} va correspondre aux répertoires \file{/a/1/b}, \file{/a/2/b},
\file{/a/1/1/b} et ainsi de suite, mais pas à \file{/a/b/c} ni \file{/a/1}.

Des \samp{//} multiples et successifs dans un chemin sont possibles, mais
\samp{//} au début d'un chemin est ignoré.

\subsubsection{Récapitulatif des caractères spéciaux dans les fichiers
  \file{texmf.cnf}}
\label{sec:cnf-special-chars}

La liste suivante récapitule les caractères spéciaux et les constructions dans
les fichiers de configuration de \KPS{}.

\newcommand{\CODE}[1]{\makebox[3em][l]{\code{#1}}}

\begin{ttdescription}

\item[\CODE{:}] Séparateur dans un chemin de recherche ; au début ou à la fin
  d'un chemin, ou doublé au milieu, il remplace le chemin par défaut.

\item[\CODE{;}] Séparateur dans les systèmes non Unix (joue le rôle de
  \code{:}).

\item[\CODE{\$}] Substitue le contenu d'une variable.

\item[\CODE{\string~}] Représente le répertoire racine de l'utilisateur.

\item[\CODE{\char`\{...\char`\}}] Expansion par les accolades, par exemple
  \verb+a{1,2}b+ devient \verb+a1b:a2b+.

\item[\CODE{,}] Sépare les éléments dans une expansion spécifiée par accolades.

\item[\CODE{//}] La recherche concernera aussi les sous-répertoires (peut être
  inséré n'importe où dans un chemin sauf au début).

\item[\CODE{\%{\rm\ et }\#}]  Début d'un commentaire.

\item[\CODE{\bs}] À la fin d'une ligne, caractère de continuation pour permettre
  les entrées à plusieurs lignes.

\item[\CODE{!!}] Cherche \emph{seulement} dans la base de données pour localiser
  le fichier et \emph{ne cherche pas} sur le disque.

\end{ttdescription}

Quant à savoir quand un caractère sera considéré comme spécial ou agira comme
tel, cela dépend du contexte dans lequel il est utilisé. Les règles sont
inhérentes aux multiples niveaux d'interprétation de la configuration (analyse,
expansion, recherche, \ldots)\ et ne peuvent donc malheureusement pas être
énoncées de manière concise. Il n'y a pas de mécanisme général d'échappement ;
en particulier, \samp{\bs} n'est pas un \enquote{caractère d'échappement} dans
les fichiers \file{texmf.cnf}.

Lorsqu'il s'agit de choisir les noms de répertoires pour l'installation, il est
plus sûr d'éviter tous les caractères spéciaux.

\subsection{Bases de données}
\label{sec:filename-database}

\KPS{} a une certaine profondeur d'investigation pour minimiser les accès disque
durant les recherches. Néanmoins, dans le cas de la \TL{} ou de distributions
comprenant beaucoup de répertoires, inspecter tous les répertoires possibles
pour un fichier donné durera excessivement longtemps (ceci est typiquement le
cas quand plusieurs centaines de répertoires de polices de caractères doivent
être parcourus). En conséquence, \KPS{} peut utiliser un fichier texte appelé
\file{ls-R} --- en fait une base de données construite au préalable --- qui fait
correspondre les fichiers à leur répertoire, ce qui permet d'éviter une
recherche exhaustive sur le disque.

Un deuxième fichier appelé \file{aliases} (qui est également une base de
données) permet de donner des noms différents aux fichiers listés dans
\mbox{\file{ls-R}}.

\subsubsection{Le fichier base de données}
\label{sec:ls-R}

Comme nous l'avons expliqué ci-dessus, le nom du principal fichier-base de
données doit être \mbox{\file{ls-R}}.  Dans votre installation, vous pouvez en
mettre un à la racine de chaque arborescence \TeX{} que vous désirez voir
inspectée (\code{\$TEXMF} par défaut) ; la plupart des sites ont une seule
arborescence \TeX{}.  \KPS{} cherche les fichiers \file{ls-R} dans le chemin
spécifié dans la variable \code{TEXMFDBS}.

La meilleure façon de créer et mettre à jour le fichier \file{ls-R} est
d'exécuter le script \cmdname{mktexlsr} inclus dans la distribution. Il est
appelé par les divers scripts \cmdname{mktex}\dots\ En principe, ce script
exécute uniquement la commande
% NdT% le \ls est correct (ça sert à ignorer les alias)
\begin{alltt}
cd \var{/your/texmf/root} && \boi{}ls -1LAR ./ >ls-R
\end{alltt}
en supposant que la commande \code{ls} de votre système produise le bon format
de sortie (le \code{ls} de \GNU convient parfaitement). Pour s'assurer que la
base de données est toujours à jour, le meilleur moyen est de la reconstruire en
utilisant la table des \code{cron}, de telle façon que le fichier \file{ls-R}
prenne automatiquement en compte les changements dans les fichiers installés,
par exemple après une installation ou une mise à jour d'un composant \LaTeX{}.

Si un fichier n'est pas trouvé dans la base de données, par défaut \KPS{} décide
de le chercher sur le disque. En revanche, si un élément du chemin commence par
\samp{!!}, \emph{seule} la base de données sera inspectée pour cet élément,
jamais le disque.

\subsubsection{kpsewhich : programme de recherche dans une arborescence}
\label{sec:invoking-kpsewhich}

Le programme \texttt{kpsewhich} effectue une recherche dans une arborescence
indépendamment de toute application. On peut le considérer comme une sorte de
\code{find} pour localiser des fichiers dans les arborescences \TeX{} (ceci est
largement utilisé dans les scripts \cmdname{mktex}\dots\ de la distribution).

\begin{alltt}
> \Ucom{kpsewhich \var{option}\dots{} \var{filename}\dots{}}
\end{alltt}
Les options spécifiées dans \ttvar{option} peuvent commencer soit par \samp{-}
soit par \samp{-{}-} ; n'importe quelle abréviation claire est acceptée.

\KPS{} considère tout argument non optionnel dans la ligne de commande comme un
nom de fichier et renvoie la première occurrence trouvée. Il n'y a pas d'option
pour renvoyer tous les fichiers ayant un nom particulier (vous pouvez utiliser
le \cmdname{find} d'Unix pour cela).

Les options les plus importantes sont décrites ci-après.

\begin{ttdescription}
\item[\texttt{-{}-dpi=\var{num}}]\mbox{} \\
  Définit la résolution à \ttvar{num} ; ceci affecte seulement la recherche des
  fichiers \samp{gf} et \samp{pk}. \samp{-D} est un synonyme pour assurer la
  compatibilité avec \cmdname{dvips}. Le défaut est 600.
\item[\texttt{-{}-format=\var{name}}]\mbox{}\\
  Définit le format pour la recherche à \ttvar{name}.  Par défaut, le format est
  estimé en fonction du nom de fichier.  Pour les formats qui n'ont pas de
  suffixe clair associé, comme les fichiers de support \MP{} et les fichiers de
  configuration \cmdname{dvips}, vous devez spécifier le nom connu de \KPS,
  comme \texttt{tex} ou \texttt{enc files}. Exécutez la commande
  \texttt{kpsewhich -{}-help-formats} pour en obtenir la liste précise.

\item[\texttt{-{}-mode=\var{string}}]\mbox{}\\
  Définit le nom du mode comme étant \ttvar{string} ; ceci affecte seulement la
  recherche des \samp{gf} et des \samp{pk}.  Pas d'option par défaut, n'importe
  quel mode sera trouvé.
\item[\texttt{-{}-must-exist}]\mbox{}\\
  Fait tout ce qui est possible pour trouver les fichiers, ce qui inclut une
  recherche sur le disque. Par défaut, seule la base de données \file{ls-R} est
  inspectée, dans un souci d'efficacité.
\item[\texttt{-{}-path=\var{string}}]\mbox{}\\
  Recherche dans le chemin \ttvar{string} (séparé par deux-points comme
  d'habitude), au lieu de prendre le chemin à partir du nom de
  fichier. \samp{//} et toutes les expansions habituelles sont prises en charge.
  Les options \samp{-{}-path} et \samp{-{}-format} s'excluent mutuellement.
\item[\texttt{-{}-progname=\var{name}}]\mbox{}\\
  Définit le nom de programme comme étant \ttvar{name}.  Ceci peut affecter les
  chemins de recherche via l'option \texttt{.\var{progname}} dans les fichiers
  de configuration.  Le défaut est \cmdname{kpsewhich}.
\item[\texttt{-{}-show-path=\var{name}}]\mbox{}\\
  Montre le chemin utilisé pour la recherche des fichiers de type \ttvar{name}.
  On peut utiliser soit une extension de fichier (\code{.pk}, \code{.vf}, etc.),
  soit un nom de fichier, comme avec l'option \samp{-{}-format}.
\item[\texttt{-{}-debug=\var{num}}]\mbox{}\\
  Définit les options de débogage comme étant \ttvar{num}.
\end{ttdescription}

\subsubsection{Exemples d'utilisation}
\label{sec:examples-of-use}

Jetons un coup d'œil à \KPS{} en action ; voici une recherche toute simple :

\begin{alltt}
> \Ucom{kpsewhich  article.cls}
/usr/local/texmf-dist/tex/latex/base/article.cls
\end{alltt}
Nous recherchons le fichier \file{article.cls}. Puisque le suffixe \file{.cls}
est non ambigu, nous n'avons pas besoin de spécifier que nous voulons rechercher
un fichier de type \optname{tex} (répertoires des fichiers sources de
\TeX{}). Nous le trouvons dans le sous-répertoire \filename{tex/latex/base} du
répertoire racine \samp{TEXMF}.  De même, le suffixe non ambigu permet de
trouver facilement les autres fichiers.
\begin{alltt}
> \Ucom{kpsewhich array.sty}
   /usr/local/texmf-dist/tex/latex/tools/array.sty
> \Ucom{kpsewhich latin1.def}
   /usr/local/texmf-dist/tex/latex/base/latin1.def
> \Ucom{kpsewhich size10.clo}
   /usr/local/texmf-dist/tex/latex/base/size10.clo
> \Ucom{kpsewhich small2e.tex}
   /usr/local/texmf-dist/tex/latex/base/small2e.tex
> \Ucom{kpsewhich tugboat.bib}
   /usr/local/texmf-dist/bibtex/bib/beebe/tugboat.bib
\end{alltt}

Le dernier exemple est une base de données bibliographiques pour \BibTeX{}
servant aux articles de \textsl{TUGboat}.

\begin{alltt}
> \Ucom{kpsewhich cmr10.pk}
\end{alltt}
Les fichiers de glyphes de fontes bitmaps, de type \file{.pk}, sont utilisés
pour l'affichage par des programmes comme \cmdname{dvips} et \cmdname{xdvi}.
Rien n'est renvoyé dans ce cas puisqu'il n'y a pas de fichier Computer Modern
\file{.pk} créé en amont sur nos systèmes (nous utilisons les versions type~1).
\begin{alltt}
> \Ucom{kpsewhich wsuipa10.pk}
\ifSingleColumn
  /usr/local/texmf-var/fonts/pk/ljfour/public/wsuipa/wsuipa10.600pk
\else /usr/local/texmf-var/fonts/pk/ljfour/public/
...                         wsuipa/wsuipa10.600pk
\fi\end{alltt}
Pour ces fontes (alphabet phonétique de l'université de Washington),
nous avons dû créer les fichiers \file{.pk} et, puisque le mode \MF{} par
défaut sur notre installation est \texttt{ljfour} avec une résolution de
base de 600\dpi{} \textit{(dots per inch)}, cette instance est trouvée.
\begin{alltt}
> \Ucom{kpsewhich -dpi=300 wsuipa10.pk}
\end{alltt}
Dans ce cas, lorsque l'on spécifie que nous recherchons une résolution de
300\dpi{} (\texttt{-dpi=300}) nous voyons qu'aucune fonte pour cette résolution
n'est disponible dans le système. En fait, un programme comme \cmdname{dvips} ou
\cmdname{xdvi} ne s'en préoccuperait pas et créerait les fichiers \file{.pk}
à la résolution demandée en utilisant le script \cmdname{mktexpk}.

Intéressons-nous à présent aux fichiers d'en-tête et de configuration pour
\cmdname{dvips}.  Regardons en premier le fichier \file{tex.pro} communément
utilisé pour le support de \TeX{} avant de regarder le fichier de configuration
générique (\file{config.ps}) et la liste des fontes \PS{} \file{psfonts.map}.
Depuis l'édition 2004, les fichiers \file{.map} et les fichiers de codage ont
changé de place dans l'arborescence \dirname{texmf}. Comme le suffixe \file{.ps}
est ambigu, nous devons spécifier quel type particulier du fichier
\texttt{config.ps} nous considérons (\optname{dvips config}).
\begin{alltt}
> \Ucom{kpsewhich tex.pro}
   /usr/local/texmf/dvips/base/tex.pro
> \Ucom{kpsewhich -{}-format="dvips config" config.ps}
   /usr/local/texmf-var/dvips/config/config.ps
> \Ucom{kpsewhich psfonts.map}
   /usr/local/texmf-var/fonts/map/dvips/updmap/psfonts.map
\end{alltt}

Regardons plus en détail les fichiers de support Times \PS{} d'URW. Leur nom
standard dans le schéma de nommage des fontes est \samp{utm}. Le premier fichier
que nous voyons est le fichier de configuration, qui contient le nom du fichier
de la liste :
\begin{alltt}
> \Ucom{kpsewhich -{}-format="dvips config" config.utm}
   /usr/local/texmf-dist/dvips/psnfss/config.utm
\end{alltt}
Le contenu de ce fichier est
\begin{alltt}
  p +utm.map
\end{alltt}
qui pointe vers le fichier \file{utm.map}, que nous cherchons à localiser
ensuite.
\begin{alltt}
> \Ucom{kpsewhich utm.map}
  /usr/local/texmf-dist/fonts/map/dvips/times/utm.map
\end{alltt}
Ce fichier liste les noms des fichiers des fontes \PS{} de type~1 dans la
collection URW. Son contenu ressemble à (nous ne montrons qu'une partie des
lignes) :
\begin{alltt}
utmb8r  NimbusRomNo9L-Medi    ... <utmb8a.pfb
utmbi8r NimbusRomNo9L-MediItal... <utmbi8a.pfb
utmr8r  NimbusRomNo9L-Regu    ... <utmr8a.pfb
utmri8r NimbusRomNo9L-ReguItal... <utmri8a.pfb
utmbo8r NimbusRomNo9L-Medi    ... <utmb8a.pfb
utmro8r NimbusRomNo9L-Regu    ... <utmr8a.pfb
\end{alltt}
Prenons par exemple le cas de Times Roman \file{utmr8a.pfb} et trouvons sa
position dans l'arborescence \file{texmf} en utilisant une recherche applicable
aux fichiers de fontes de type~1 :
\begin{alltt}
> \Ucom{kpsewhich utmr8a.pfb}
\ifSingleColumn   /usr/local/texmf-dist/fonts/type1/urw/times/utmr8a.pfb
\else   /usr/local/texmf-dist/fonts/type1/
... urw/utm/utmr8a.pfb
\fi\end{alltt}

Il devrait être clair, d'après ces quelques exemples, qu'il est facile de
trouver l'endroit où se cache un fichier donné. C'est particulièrement important
si vous suspectez que c'est, pour une raison quelconque, une mauvaise version du
fichier qui est utilisée, puisque \cmdname{kpsewhich} va vous montrer le premier
fichier trouvé.

\subsubsection{Opérations de débogage}
\label{sec:debugging}

Il est quelquefois nécessaire de savoir comment un programme référence les
fichiers. Pour permettre cela, \KPS{} offre plusieurs niveaux de débogage :
\begin{ttdescription}
\item[\texttt{\ 1}] Appels à \texttt{stat} (test d'existence de fichier). Lors
  d'une exécution utilisant une base de données \file{ls-R} à jour, ce niveau ne
  devrait donner presque aucune information en sortie.
\item[\texttt{\ 2}] Références aux différentes tables (comme la base de données
  \file{ls-R}, les fichiers de correspondance de fontes, les fichiers de
  configuration).
\item[\texttt{\ 4}] Opérations d'ouverture et de fermeture des fichiers.
\item[\texttt{\ 8}] Information globale sur la localisation des types de
  fichiers recherchés par \KPS. Ceci est utile pour trouver où a été défini le
  chemin particulier pour un fichier.
\item[\texttt{16}] Liste des répertoires pour chaque élément du chemin (utilisé
  uniquement en cas de recherche sur le disque).
\item[\texttt{32}] Recherche de fichiers.
\item[\texttt{64}] Valeur des variables.
\end{ttdescription}
Une valeur de \texttt{-1} activera toutes les options ci-dessus ; en pratique,
c'est habituellement la valeur la plus adaptée.

De la même façon, avec le programme \cmdname{dvips}, en utilisant une
combinaison d'options de débogage, on peut suivre en détail la localisation des
différents fichiers. De plus, lorsqu'un fichier n'est pas trouvé, la trace du
débogage montre les différents répertoires dans lesquels le programme va
chercher tel ou tel fichier, donnant ainsi des indices sur le problème.

Généralement, comme la plupart des programmes appellent la bibliothèque \KPS{}
en interne, on peut sélectionner une option de débogage en utilisant la variable
d'environnement \envname{KPATHSEA\_DEBUG} et en la définissant égale à une
valeur (ou à une combinaison de valeurs) décrite(s) dans la liste ci-dessus.

Note à l'intention des utilisateurs de Windows : il n'est pas facile de
rediriger les messages d'erreur vers un fichier sur ces systèmes. À des fins de
diagnostic, vous pouvez temporairement affecter
\texttt{KPATHSEA\_DEBUG\_OUTPUT=err.log} pour capturer le flux standard d'erreur
dans le fichier \texttt{err.log}.

Considérons comme exemple un petit fichier source \LaTeX{},
\file{hello-world.tex}, dont le contenu est le suivant.
\begin{verbatim}
\documentclass{article}
\begin{document}
Hello World!
\end{document}
\end{verbatim}
Ce petit fichier utilise simplement la fonte \file{cmr10}, aussi allons voir
comment \cmdname{dvips} prépare le fichier \PS{} (nous voulons utiliser la
version type~1 des fontes Computer Modern, d'où l'option \texttt{-Pcms}).
\begin{alltt}
> \Ucom{dvips -d4100 hello-world -Pcms -o}
\end{alltt}
Dans ce cas, nous avons combiné le niveau 4 de débogage de \cmdname{dvips}
(chemins des fontes) avec l'option d'expansion des éléments du chemin de \KPS\
(voir le manuel de \cmdname{dvips}).  La sortie (légèrement modifiée) apparaît
dans la figure~\ref{fig:dvipsdbga}.
\begin{figure*}[tp]
  \centering \input{examples/ex6a.tex}
  \caption{Recherche des fichiers de configuration}\label{fig:dvipsdbga}
\end{figure*}

\cmdname{dvips} commence par localiser ses fichiers de fonctionnement. D'abord,
\file{texmf.cnf} est trouvé, ce qui donne les définitions pour les chemins de
recherche servant à localiser les autres fichiers, ensuite le fichier base de
données \file{ls-R} (pour optimiser la recherche des fichiers) et le fichier
\file{aliases}, qui permet de déclarer plusieurs noms (par exemple un nom DOS de
type 8.3 court et une version longue plus naturelle) pour le même
fichier. Ensuite \cmdname{dvips} continue en cherchant le fichier de
configuration générique \file{config.ps} avant de rechercher le fichier de
paramétrisation \file{.dvipsrc} (qui, dans notre cas, \emph{n'est pas
  trouvé}). Enfin, \cmdname{dvips} localise le fichier de configuration pour les
fontes \PS{} Computer Modern \file{config.cms} (ceci est lancé par l'option
\mbox{\texttt{-Pcms}} de la commande \cmdname{dvips}). Ce fichier contient la
liste des fichiers qui définissent la relation entre les noms des fontes selon
\TeX{}, selon \PS{} et dans le système de fichiers.
\begin{alltt}
> \Ucom{more /usr/local/texmf/dvips/cms/config.cms}
   p +ams.map
   p +cms.map
   p +cmbkm.map
   p +amsbkm.map
\end{alltt}
\cmdname{dvips} veut chercher tous ces fichiers, y compris le fichier générique
d'association \file{psfonts.map}, qui est toujours chargé (il contient des
déclarations pour les fontes \PS{} les plus communément utilisées ; voir la
dernière partie de la section \ref{sec:examples-of-use} pour plus de détails sur
la gestion du fichier d'association \PS{}).

Arrivé là, \cmdname{dvips} s'identifie à l'utilisateur :
\begin{alltt}
 This is dvips(k) 5.92b Copyright 2002 Radical Eye Software (www.radicaleye.com)
 \end{alltt}

 \ifSingleColumn pour continuer ensuite en cherchant le fichier prologue
 \file{texc.pro},
 \begin{alltt}\small
kdebug:start search(file=texc.pro, must\_exist=0, find\_all=0,
  path=.:~/tex/dvips//:!!/usr/local/texmf/dvips//:
       ~/tex/fonts/type1//:!!/usr/local/texmf/fonts/type1//).
kdebug:search(texc.pro) => /usr/local/texmf/dvips/base/texc.pro
\end{alltt}
\else pour continuer ensuite en cherchant le fichier prologue \file{texc.pro}
(voir la figure~\ref{fig:dvipsdbgb}).  \fi

Après avoir trouvé ce fichier, \cmdname{dvips} affiche la date et l'heure, nous
informe qu'il va générer le fichier \file{hello-world.ps} puis qu'il a besoin du
fichier de fonte \file{cmr10} et que ce dernier est déclaré comme « résident»
(pas besoin de bitmaps) :
\begin{alltt}\small
TeX output 1998.02.26:1204' -> hello-world.ps
Defining font () cmr10 at 10.0pt
Font cmr10 <CMR10> is resident.
\end{alltt}
Maintenant la recherche concerne le fichier \file{cmr10.tfm}, qui est trouvé,
puis quelques fichiers de prologue de plus (non montrés) sont référencés ;
finalement le fichier de la fonte type~1 \file{cmr10.pfb} est localisé et inclus
dans le fichier de sortie (voir la dernière ligne).
\begin{alltt}\small
kdebug:start search(file=cmr10.tfm, must\_exist=1, find\_all=0,
  path=.:~/tex/fonts/tfm//:!!/usr/local/texmf/fonts/tfm//:
       /var/tex/fonts/tfm//).
kdebug:search(cmr10.tfm) => /usr/local/texmf/fonts/tfm/public/cm/cmr10.tfm
kdebug:start search(file=texps.pro, must\_exist=0, find\_all=0,
   ...
<texps.pro>
kdebug:start search(file=cmr10.pfb, must\_exist=0, find\_all=0,
  path=.:~/tex/dvips//:!!/usr/local/texmf/dvips//:
       ~/tex/fonts/type1//:!!/usr/local/texmf/fonts/type1//).
kdebug:search(cmr10.pfb) => /usr/local/texmf/fonts/type1/public/cm/cmr10.pfb
<cmr10.pfb>[1]
\end{alltt}

\subsection{Options à l'exécution}

\Webc{} offre la possibilité de contrôler à l'exécution bon nombre de paramètres
concernant la mémoire (en particulier la taille des tableaux utilisés) à partir
du fichier \file{texmf.cnf} qui est lu par \KPS.  Les paramètres en question se
trouvent dans la troisième partie du fichier inclus dans la distribution \TL{}.
Les variables les plus importantes sont :
\begin{ttdescription}
\item[\texttt{main\_memory}] Nombre total de mots mémoire disponibles pour
  \TeX{}, \MF{} et \MP.  Vous devez générer un nouveau fichier de format pour
  chaque nouveau paramétrage. Par exemple, vous pouvez générer une version large
  de \TeX{} et appeler le fichier de format \texttt{hugetex.fmt}. En utilisant
  la méthode prise en charge par \KPS{} qui consiste à suffixer la variable par
  le nom du programme, la valeur particulière de la variable
  \texttt{main\_memory} destinée à ce fichier de format sera lue dans le fichier
  \file{texmf.cnf}.
\item[\texttt{extra\_mem\_bot}] Espace supplémentaire pour certaines structures
  de données de \TeX{} : boîtes, \textit{glue}, points d'arrêt\dots{} Surtout
  utile si vous utilisez \PiCTeX{} par exemple.
\item[\texttt{font\_mem\_size}] Nombre de mots mémoire disponibles pour décrire
  les polices. C'est plus ou moins l'espace occupé par les fichiers TFM lus.
\item[\texttt{hash\_extra}] Espace supplémentaire pour la table de hachage des
  noms de séquences de contrôle ; sa valeur par défaut est \texttt{600000}.
\end{ttdescription}

Cette possibilité ne remplace pas une véritable allocation dynamique des
tableaux et de la mémoire mais, puisque c'est complexe à implémenter dans le
présent source \TeX{}, ces paramètres lus à l'exécution fournissent un compromis
pratique qui procure une certaine souplesse.

\htmlanchor{texmfdotdir}
\subsection{\texttt{\$TEXMFDOTDIR}}
\label{sec:texmfdotdir}

À divers endroits ci-dessus, nous avons indiqué différents chemins de recherche
commençant par \code{.} (pour rechercher d'abord dans le répertoire actuel),
comme dans
\begin{alltt}\small
TEXINPUTS=.;$TEXMF/tex//
\end{alltt}

Il s'agit d'une simplification. Au lieu de simplement \samp{.}, le fichier
\code{texmf.cnf} que nous distribuons dans \TL{} utilise
\filename{$TEXMFDOTDIR}
comme dans :
\begin{alltt}\small
TEXINPUTS=$TEXMFDOTDIR;$TEXMF/tex//
\end{alltt}
(Dans le fichier distribué, le deuxième élément du chemin d'accès est également
légèrement plus compliqué que \filename{$TEXMF/tex//}.
Mais c'est mineur ; ici nous voulons pour discuter de la fonctionnalité
\filename{$TEXMFDOTDIR}).

La raison pure et simple de l'utilisation de la variable
\filename{$TEXMFDOTDIR}
dans le chemin au lieu d'un simple \samp{.} est qu'il puisse être surchargé. Par
exemple, un document complexe peut avoir de nombreux fichiers sources disposés
dans de nombreux sous-répertoires. Pour gérer cela, vous pouvez définir
\filename{$TEXMFDOTDIR}
comme étant \filename{.//} (par exemple, dans le cas où vous compilez le
document) et ces sous-répertoires seront tous fouillés. (Attention : ne pas
utiliser \filename{.//} par défaut ; il est en général hautement non désirable,
et potentiellement peu sûr, de rechercher un document arbitraire dans tous les
sous-répertoires.)

Autre exemple : vous pouvez ne pas du tout chercher dans le répertoire actuel,
par exemple si vous avez fait en sorte que tous les dossiers soient trouvés via
des chemins explicites. Vous pouvez pour cela par exemple définir
\filename{$TEXMFDOTDIR}
comme \filename{/riendetel} ou tout répertoire inexistant.

La valeur par défaut de \filename{$TEXMFDOTDIR}
est juste \samp{.}, comme défini dans notre \filename{texmf.cnf}.

\htmlanchor{ack}
\section{Remerciements}

\TL{} est le résultat des efforts collectifs de pratiquement tous les groupes
d'utilisateurs de \TeX.  La présente édition de \TL{} a été coordonnée par Karl
Berry.  Voici la liste des principaux contributeurs :

\begin{itemize*}

  % NdT% on considère que ce sont les nationalités des associations :
  % elles regroupent en pratique des adhérents d'originies diverses
\item Les associations d'utilisateurs anglaise, allemande, néerlandaise et
  polonaise (TUG, DANTE e.V., NTG, et GUST, respectivement) qui contribuent
  ensemble à l'infrastructure technique et administrative. Soutenez votre
  association locale (voir la liste \url{https://tug.org/usergroups.html}) !

\item L'équipe du CTAN (\url{https://ctan.org}) qui distribue les images des
  distributions \TL{} et fournit les sites d'hébergement pour le stockage et la
  mise à jour des extensions qui sont la base de \TL.

\item Nelson Beebe, pour avoir permis l'accès à de nombreuses plateformes aux
  développeurs \TL, avoir lui-même participé aux tests de façon étendue et pour
  ses efforts bibliographiques sans pareil.

\item John Bowman, pour avoir effectué de nombreux changements dans son
  programme Asymptote pour le faire fonctionner dans \TL.

\item Peter Breitenlohner et toute l'équipe \eTeX{} qui construisent les bases
  des successeurs de \TeX, Peter tout particulièrement pour des années d'aide
  concernant l'usage des autotools de GNU partout dans \TL\ et pour avoir
  conservé à jour les sources. Peter est décédé en octobre 2015 et nous dédions
  à sa mémoire la poursuite de notre travail.

\item Jin-Hwan Cho et toute l'équipe de DVIPDFM$x$ pour leur excellent programme
  et leur réactivité face aux problèmes de configuration.

\item Thomas Esser et sa merveilleuse distribution \teTeX{} sans laquelle \TL{}
  n'aurait jamais vu le jour.

\item Michel Goossens, en tant que coauteur de la documentation initiale.

\item Eitan Gurari, dont le programme \TeX4ht est utilisé pour créer la version
  \HTML{} de cette documentation et qui travaillait inlassablement
  à l'améliorer, ce dans des délais très courts. Eitan nous a quittés
  prématurément en juin 2009, et nous dédions la présente documentation à sa
  mémoire.

\item Hans Hagen qui, outre sa participation active aux tests, a adapté
  l'extension \ConTeXt\ (\url{https://pragma-ade.com/}) aux besoins de \TL{}, et
  qui est un moteur permanent du développement de \TeX.

\item \Thanh, Martin Schröder et toute l'équipe pdf\TeX\ qui continuent
  inlassablement à étendre les performances de \TeX.

\item Hartmut Henkel, pour ses contributions au développement de pdf\TeX{} et
  Lua\TeX{} entre autres.

\item Shunshaku Hirata, pour son travail original and permanent sur DVIPDFM$x$.

\item Taco Hoekwater, pour son travail important et incessant sur le
  développement de \MP{} et de (Lua)\TeX\ (\url{http://luatex.org}) lui-même,
  pour l'intégration de \ConTeXt\ dans \TL, pour les nouvelles fonctionnalités
  \emph{multi-thread} de Kpathsea, et bien plus encore.

\item Khaled Hosny, pour son travail substantiel sur \XeTeX et DVIPDFM$x$ et ses
  efforts concernant les fontes arabes et autres.

\item Pawe{\l} Jackowski pour l'installateur Windows \cmdname{tlpm} et Tomasz
  {\L}uczak pour la version graphique \cmdname{tlpmgui} utilisée dans l'édition
  précédente.

\item Akira Kakuto, pour son aide précieuse qui nous a permis d'intégrer dans
  \TL{} les binaires Windows de ses distributions W32TEX et W64TEX
  (\url{https://www.w32tex.org/}).

\item Jonathan Kew, pour avoir produit Xe\TeX{}, pour l'avoir intégré dans
  \TL{}, pour avoir créé la première version de l'installateur Mac\TeX et enfin
  pour l'éditeur \TeX{}works que nous recommandons.

\item Hironori Kitagawa, pour la maintenance de (e)p\TeX{} et le support afférent.

\item Dick Koch, pour la maintenance de Mac\TeX\ (\url{https://tug.org/mactex})
  faite en symbiose avec \TL.

\item Reinhard Kotucha, pour ses contributions majeures à l'infrastructure de
  \TL{} 2008 et à son programme d'installation, pour ses contributions sous
  Windows et pour son script \texttt{getnonfreefonts} en particulier.

\item Siep Kroonenberg, également pour ses contributions majeures
  à l'infrastructure de \TL{} 2008 et à son programme d'installation ainsi que
  pour la réécriture de cette documentation concernant ces fonctionnalités.

\item Clerk Ma, pour les corrections de bugs de moteurs et leurs extensions.

\item Mojca Miklavec, pour son aide précieuse concernant \ConTeXt, ses nombreux
  exécutables et autres.

\item Heiko Oberdiek, pour le paquet \pkgname{epstopdf} et bien d'autres, pour
  avoir compressé l'énorme \pkgname{pst-geo} de façon à ce que nous puissions
  l'inclure, et, par-dessus tout, son travail remarquable sur
  \pkgname{hyperref}.

\item Phelype Oleinik, pour le \cs{input} délimité par des groupes sur plusieurs
  moteurs en 2020, et plus encore.

\item Petr Ol\v{s}ak, qui coordonna et vérifia minutieusement toute la partie
  tchèque et slovaque.

\item Toshio Oshima, pour le visualisateur \cmdname{dviout} pour Windows.

\item Manuel Pégourié-Gonnard, pour son aide concernant le programme de mise
  à jour de la distribution, la documentation et pour le développement de
  \cmdname{texdoc}.

\item Fabrice Popineau, pionnier du développement de \TL{} sous Windows et pour
  son travail sur la documentation française.

\item Norbert Preining, principal architecte de la présente infrastructure \TL{}
  et de son installateur, coordinateur (avec Frank K\"uster) de la version
  Debian de \TL{}, et important contributeur de longue date.

\item Sebastian Rahtz, qui a créé la distribution \TL{} et en a assuré la
  maintenance pendant de nombreuses années. Sebastian est décédé en mars 2016 et
  nous dédions à sa mémoire la poursuite de notre travail.

\item Luigi Scarso, pour continuer le développement de \MP{}, Lua\TeX, et bien
  plus encore.

\item Andreas Scherer, pour \texttt{cwebbin}, l'implémentation de CWEB utilisée
  dans \TL{}.

\item Takuji Tanaka, pour la maintenance de (e)(u)p\TeX{} et le support
  afférent.

\item Tomasz Trzeciak, pour son aide générale concernant Windows.

\item Vladimir Volovich, pour son aide substantielle, en particulier pour avoir
  rendu possible l'intégration de \cmdname{xindy} dans \TL.

\item Staszek Wawrykiewicz, un des principaux testeurs de \TL{} et coordinateur
  des contributions polonaises (fontes, installation Windows, etc.).  Staszek
  est décédé en février 2018 et nous dédions à sa mémoire la poursuite de notre
  travail.

\item Olaf Weber, pour son patient assemblage de \Webc\ les années précédentes.

\item Gerben Wierda, qui a créé et maintenu initialement la partie \macOS{}.

\item Graham Williams, l'auteur original du catalogue \TeX.

\item Joseph Wright, pour son travail conséquent sur la mise à disposition de la
  même fonctionnalité primitive sur tous les moteurs.

\item Hironobu Yamashita, pour son travail sur p\TeX\ et le support associé.

\end{itemize*}

Les binaires ont été compilés par :
\begin{itemize}
\item Marc Baudoin (\pkgname{amd64-netbsd}, \pkgname{i386-netbsd}) ;
\item Karl Berry (\pkgname{i386-linux}) ;
\item Ken Brown (\pkgname{x86\_64-cygwin}) ;
\item Simon Dales (\pkgname{armhf-linux}) ;
\item Johannes Hielscher (\pkgname{aarch64-linux}) ;
\item Akira Kakuto (\pkgname{windows}) ;
\item Dick Koch (\pkgname{universal-darwin}) ;
\item Mojca Miklavec (\pkgname{amd64-freebsd}, \pkgname{armhf-linux},
  \pkgname{i386-freebsd}, \pkgname{x86\_64-darwinlegacy},
  \pkgname{i386-solaris}, \pkgname{x86\_64-solaris}, \pkgname{sparc-solaris}), ;
\item Norbert Preining (\pkgname{i386-linux}, \pkgname{x86\_64-linux},
  \pkgname{x86\_64-linuxmusl}).
\end{itemize}
Pour des informations concernant la compilation de \TL, cf.
\url{https://tug.org/texlive/build.html}.

Traducteurs de ce manuel :
\begin{itemize}
\item Takuto Asakura (japonais),
\item Denis Bitouzé \& Patrick Bideault (français) ;
\item Carlos Enriquez Figueras (castillan) ;
\item Jjgod Jiang, Jinsong Zhao, Yue Wang, \& Helin Gai (mandarin) ;
\item Marco Pallante \& Carla Maggi (italien) ;
\item Nikola Le\v{c}i\'c (serbe) ;
\item Petr Sojka \& Jan Busa (tchèque et slovaque) ;
\item Boris Veytsman (russe) ;
\item Zofia Walczak (polonais) ;
\item Uwe Ziegenhagen (allemand).
\end{itemize}
La page d'accueil de la documentation \TL{} est
\url{https://tug.org/texlive/doc.html}.

Bien sûr, notre gratitude va en premier lieu à Donald Knuth pour avoir inventé
\TeX{} et l'avoir offert au monde entier.

\section{Historique des versions successives}
\label{sec:history}

\subsection{Éditions précédentes}

La discussion commença à la fin de 1993 quand le Groupe des utilisateurs
néerlandais de \TeX{} commençait à travailler à son \CD{} 4All\TeX{} pour les
utilisateurs de MS-DOS et on espérait à ce moment sortir un \CD{} unique pour
tous les systèmes.  C'était un objectif beaucoup trop ambitieux, mais il permit
la naissance du \CD{} 4All\TeX{}, projet couronné de succès, et aussi d'un
groupe de travail « TUG Technical Council » pour mettre en place TDS
(\emph{\TeX{} Directory Structure} : \url{https://tug.org/tds}), qui spécifiait
la gestion des fichiers \TeX{} sous une forme logique. La mouture finale de
\TDS{} fut publiée dans le numéro de décembre 1995 de \textsl{TUGboat} et il
était clair depuis un certain temps qu'il fallait proposer un produit contenant
une structure modèle sur \CD{}.  La distribution que vous possédez est le
résultat direct des délibérations de ce groupe de travail.  Il était également
clair que le succès des \CD{} 4All\TeX{} démontrait que les utilisateurs d'Unix
trouveraient leur bonheur avec une distribution aussi simple et ceci a été
l'autre objectif de \TL.

Nous avons d'abord entrepris de créer un nouveau \CD{} \TDS{} Unix à l'automne
1995 et nous avons rapidement choisi \teTeX{} de Thomas Esser comme étant la
configuration idéale, car il supportait déjà plusieurs plateformes et avait
été construit en gardant à l'esprit la portabilité entre systèmes.  Thomas
accepta de nous aider et commença à travailler sérieusement au début de 1996.
La première édition sortit en mai 1996.  Au début de 1997, Karl Berry acheva une
nouvelle distribution de \Webc{}, qui incluait presque toutes les
caractéristiques que Thomas Esser avait ajoutées dans \teTeX{} et il fut décidé
de baser la deuxième édition du \CD{} sur le standard \Webc, en y ajoutant le
script \texttt{texconfig} de \teTeX.  La troisième édition du \CD{} était basée
sur une version majeure de \Webc, 7.2, par Olaf Weber ; en même temps, une
nouvelle version révisée de \teTeX{} était achevée dont \TL{} partageait presque
toutes les caractéristiques. La quatrième édition a suivi le même schéma, en
utilisant une nouvelle version de \teTeX{} et une nouvelle version de \Webc{}
(7.3). Le système incluait dorénavant un programme complet d'installation pour
Windows grâce à Fabrice Popineau.

Pour la cinquième édition (mars 2000), de nombreuses parties du \CD{} ont été
vérifiées et révisées, des centaines de composants mis à jour.  Le contenu
détaillé des composants était décrit par des fichiers XML. Mais le changement
majeur de cette cinquième édition a été la suppression de tout logiciel non
libre de droits.  Tout ce qui se trouve dans \TL{} devait être compatible avec
la licence Debian (\emph{Debian Free Software Guidelines} :
\url{https://debian.org/intro/free}) ; nous avons fait de notre mieux pour
vérifier les termes des licences de chaque composant et nous souhaiterions que
toute erreur nous soit signalée.

La sixième édition (juillet 2001) contient un grand nombre de mises à jour.  Le
changement majeur de cette version réside dans la refonte du processus
d'installation : l'utilisateur peut désormais choisir les collections de manière
plus précise. Les collections concernant les langues ont été entièrement
réorganisées, aussi le choix d'une langue installe non seulement les macros, les
fontes, etc., mais prépare également un fichier \file{language.dat} adéquat.

La septième édition (mai 2002) a comme ajout majeur une installation pour
\macOS{} et l'habituelle myriade de mises à jour de composants et de
programmes. Un objectif important a été de fusionner à nouveau les sources avec
ceux de \teTeX{}, alors que les versions 5 et 6 s'en étaient éloignées.

\subsubsection{2003}

En 2003, le flot de mises à jour et d'additions a continué, mais nous avons
constaté que \TL{} était devenu si volumineux qu'il ne pouvait plus tenir sur un
seul \CD, aussi l'avons-nous divisé en trois distributions distinctes (voir
section~\ref{sec:tl-coll-dists}, page~\pageref{sec:tl-coll-dists}). Par
ailleurs :
\begin{itemize}
\item À la demande de l'équipe \LaTeX{}, nous avons modifié les commandes
  standard \cmdname{latex} et \cmdname{pdflatex} pour qu'elles utilisent \eTeX{}
  (voir page~\pageref{text:etex}).
\item Les nouvelles fontes Latin Modern sont disponibles (et recommandées).
\item Le support pour Alpha OSF a été supprimé (celui pour HPUX l'avait été
  auparavant), car personne disposant des machines nécessaires ne s'est proposé
  pour compiler les nouveaux binaires.
\item L'installation pour Windows a été largement modifiée ; un environnement de
  travail intégré basé sur \cmdname{XEmacs} a été introduit.
\item Des programmes supplémentaires importants pour Windows (Perl,
  Ghost\-script, Image\-Magick, Ispell) sont maintenant installés dans le
  répertoire d'installation de \TL{}.
\item Les fichiers \emph{font map} utilisés par \cmdname{dvips},
  \cmdname{dvipdfm} et \cmdname{pdftex} sont maintenant générés par le nouveau
  programme \cmdname{updmap} et installés dans \dirname{texmf/fonts/map}.
\item Dorénavant, \TeX{}, \MF{} et \MP{} écrivent les caractères 8 bit
  présentés en entrée sans modification et non pas avec la notation \verb|^^|,
  que ce soit dans des fichiers (par la commande \verb|write|), dans les
  fichiers de trace (\verb|.log|) ou sur le terminal.  Dans le \TL{}~7,
  l'écriture de ces caractères 8 bit était influencée par les paramètres de
  localisation du système; maintenant ces paramètres n'influent plus du tout sur
  le comportement des programmes \TeX{}. Si pour quelque raison que ce soit,
  vous avez besoin de la notation \verb|^^| en sortie, renommez le fichier
  \verb|texmf-dist/web2c/cp8bit.tcx|.  Les prochaines versions disposeront d'un
  moyen plus propre pour contrôler cette sortie.
\item La documentation de \TL{} a été largement révisée.
\item Enfin, comme la numérotation séquentielle des versions devenait peu
  maniable, il a été décidé d'identifier désormais la version de \TL{} par
  l'année : \TL{}~2003 au lieu de \TL{}~8.
\end{itemize}


\subsubsection{2004}

2004 a apporté beaucoup de changements (et quelques incompatibilités avec les
versions précédentes) :

\begin{itemize}

\item Si vous avez installé des fontes supplémentaires qui ont leur propre
  fichier \file{.map} ou des fichiers \file{.enc} spécifiques, vous devrez
  vraisemblablement déplacer ces fichiers.

  Les fichiers \file{.map} sont désormais recherchés uniquement dans les
  sous-répertoires \dirname{fonts/map} (dans chaque arborescence
  \filename{texmf}), leur chemin de recherche est donné par la variable
  \envname{TEXFONTMAPS} de \filename{texmf.cnf}.  De même, les fichiers
  \file{.enc} sont désormais recherchés uniquement dans les sous-répertoires
  \dirname{fonts/enc}, leur chemin de recherche est donné par la variable
  \envname{ENCFONTS} de \filename{texmf.cnf}.  Le script \cmdname{updmap}
  devrait émettre des messages d'avertissement pour les fichiers \file{.map} et
  \file{.enc} mal placés.

  Sur les différentes façons de traiter le problème, consulter
  \url{https://tug.org/texlive/mapenc.html}.

\item La distribution pour Windows a changé cette année : l'installation de la
  distribution \fpTeX{} (basée sur \Webc{}) de Fabrice Popineau n'est plus
  proposée. À la place, vous pouvez tester et installer la distribution
  \ProTeXt{} basée sur \MIKTEX{} (indépendante de \Webc{}), voir
  section~\ref{sec:overview-tl}, page~\pageref{sec:overview-tl}.

\item L'ancien répertoire \dirname{texmf} a été éclaté en trois parties :
  \dirname{texmf}, \dirname{texmf-dist} et \dirname{texmf-doc}.  Voir
  section~\ref{sec:tld}, page~\pageref{sec:tld}.

\item Tous les fichiers relatifs aux différents avatars de \TeX{} sont désormais
  regroupés dans le même sous-répertoire \dirname{tex} des arborescences
  \dirname{texmf*}, plutôt qu'avoir des répertoires séparés apparentés
  \dirname{tex}, \dirname{etex}, \dirname{pdftex}, \dirname{pdfetex}, etc. Voir
  \CDref{texmf-dist/doc/generic/tds/tds.html\#Extensions}
  {\texttt{texmf-dist/doc/generic/tds/tds.html\#Extensions}}.

\item Les scripts auxiliaires, normalement pas exécutés directement par les
  utilisateurs, sont regroupés dans des sous-répertoires \dirname{scripts} des
  arborescences \dirname{texmf*}. On les localise grâce à la commande
  \verb|kpsewhich -format=texmfscripts|. Au cas où vous en auriez, les
  programmes utilisant ces scripts nécessiteront une adaptation.  Voir
  \CDref{texmf-dist/doc/generic/tds/tds.html\#Scripts}
  {\texttt{texmf-dist/doc/generic/tds/tds.html\#Scripts}}.

\item La plupart des formats affichent en clair (dans les fichiers \file{.log}
  en particulier) les caractères imprimables au lieu de les transcrire en
  notation hexadécimale \verb|^^|. Ceci se fait grâce au fichier
  \filename{cp227.tcx}, qui considère comme imprimables les caractères 32 à 256,
  ainsi que les tabulations et les changements de page (caractères 9 à 11). Les
  formats faisant exception sont plain \TeX\ (seuls les caractères 32 à 127 sont
  déclarés imprimables), \ConTeXt{} (0 à 255 imprimables) et les formats basés
  sur \OMEGA.  Il y a peu de différence avec le comportement de \TL\ 2003, mais
  la mise en œuvre est plus propre et plus facilement configurable.  Voir
  \OnCD{texmf-dist/doc/web2c/web2c.html\#TCX-files}.  Noter que l'utilisation du
  codage Unicode en entrée peut provoquer des affichages défectueux en sortie
  (TeX code sur un seul octet).

\item Tous les formats, sauf plain \TeX, font appel au moteur \pkgname{pdfetex}
  (qui produit bien sûr par défaut des fichiers DVI lorsque le format choisi est
  \LaTeX). Ceci permet aux formats \LaTeX, \ConTeXt, etc., d'avoir accès aux
  fonctionnalités microtypographiques de \pkgname{pdftex} (alignement optique des
  marges par exemple) et aux fonctionnalités de \eTeX{}
  (\OnCD{texmf-dist/doc/etex/base/}).

  Ceci rend \emph{indispensable} le recours à l'extension \pkgname{ifpdf} (qui
  fonctionne aussi bien avec plain que \LaTeX) pour déterminer si le format de
  sortie est DVI ou PDF. Tester si la commande \cs{pdfoutput} est définie ou non
  \emph{n'est pas} un moyen fiable de le faire.

\item pdf\TeX\ (\url{https://tug.org/applications/pdftex/}) offre de nouvelles
  fonctionnalités :

  \begin{itemize}

  \item Les commandes \cs{pdfmapfile} et \cs{pdfmapline} permettent de spécifier
    le choix des fichiers \file{.map} à utiliser pour le document en cours.

  \item L'amélioration du gris typographique par variation (infime) de la
    largeur des caractères (\textit{font expansion}) est plus facile à mettre en
    œuvre, voir
    \url{https://www.ntg.nl/pipermail/ntg-pdftex/2004-May/000504.html}

  \item Le fichier \filename{pdftex.cfg} n'est plus utilisé.  Toutes les
    affectations de paramètres doivent désormais être faites dans le préambule
    en utilisant les primitives ad~hoc.

  \item Pour plus d'informations, consulter le manuel de pdf\TeX\ :
    \OnCD{texmf-dist/doc/pdftex/manual}.

  \end{itemize}

\item La primitive \cs{input} de \cmdname{tex}, \cmdname{mf} et \cmdname{mpost},
  accepte désormais les espaces et autres caractères spéciaux dans les noms de
  fichiers à condition d'utiliser des \textit{double quotes}, en voici deux
  exemples typiques :
\begin{verbatim}
\input "filename with spaces"   % plain
\input{"filename with spaces"}  % latex
\end{verbatim}
  Consulter le manuel \Webc{} pour plus d'informations :
  \OnCD{texmf-dist/doc/web2c}.

\item Les fonctionnalités de enc\TeX\ sont désormais incluses dans \Webc{}.
  Ainsi, tous les formats \emph{construits avec l'option \optname{-enc}} y ont
  accès.  enc\TeX\ permet le transcodage en entrée et en sortie et l'utilisation
  transparente du codage Unicode UTF-8. Voir
  \OnCD{texmf-dist/doc/generic/enctex/} et \url{https://olsak.net/enctex.html}.

\item Un nouveau moteur combinant les fonctionnalités de \eTeX\ et d'\OMEGA,
  appelé Aleph, est disponible. \OnCD{texmf-dist/doc/aleph/base} et
  \url{https://texfaq.org/FAQ-enginedev} fournissent une information
  succincte. Le format \LaTeX{} utilisant Aleph s'appelle \pkgname{lamed}.

\item La licence LPPL de \LaTeX\ (version décembre 2003) a changé, elle est
  désormais compatible avec les prescriptions Debian.  Les autres changements
  sont décrits dans le fichier \filename{ltnews}, voir
  \OnCD{texmf-dist/doc/latex/base}.

\item Un nouveau programme, \cmdname{dvipng}, qui convertit les fichiers DVI en
  images PNG a été ajouté. Voir \url{https://ctan.org/pkg/dvipng}.

\item Nous avons dû réduire le nombre de fontes incluses dans l'extension
  \pkgname{cbgreek}, ceci a été fait avec l'accord et l'aide de l'auteur
  (Claudio Beccari). Les fontes exclues (invisibles, transparentes, contours)
  sont rarement utilisées et la place nous manquait. La collection complète des
  fontes \pkgname{cbgreek} est disponible sur CTAN
  (\url{https://ctan.org/pkg/cbgreek-complete}).

\item La commande \cmdname{oxdvi} a été supprimée, il suffit d'utiliser
  \cmdname{xdvi} à la place.

\item Les commandes \cmdname{initex}, \cmdname{virtex} et leurs homologues pour
  \cmdname{mf} et \cmdname{mpost} ont disparu.  Vous pouvez les recréer si
  nécessaire, mais elles sont avantageusement remplacées, depuis des années
  maintenant, par l'option \optname{-ini} (\cmdname{tex} \optname{-ini} pour
  \cmdname{initex} et \cmdname{virtex}).

\item Les binaires pour l'architecture \pkgname{i386-openbsd} ont été supprimés
  par manque de volontaires pour les compiler.

  Sur \pkgname{sparc-solaris} (au moins) il sera probablement nécessaire de
  positionner la variable d'environnement \envname{LD\_LIBRARY\_PATH} pour
  utiliser les programmes de la famille \pkgname{t1utils}.  Ceci vient du fait
  qu'ils sont compilés en C++ et que l'emplacement des bibliothèques dynamiques
  est variable.  Ce n'est pas une nouveauté 2004, mais ce point n'était pas
  documenté précédemment. De même, sur \pkgname{mips-irix}, les bibliothèques
  dynamiques MIPSpro 7.4 sont nécessaires.

\end{itemize}

\subsubsection{2005}

2005 a apporté son lot habituel d'innombrables mises à jour d'extensions et de
programmes. L'infrastructure est restée relativement stable par rapport à 2004,
à quelques changements inévitables près :

\begin{itemize}

\item Trois nouveaux scripts \cmdname{texconfig-sys}, \cmdname{updmap-sys} et
  \cmdname{fmtutil-sys} ont été introduits ; ils agissent sur la configuration
  générale de la machine, comme le faisaient les scripts \cmdname{texconfig},
  \cmdname{updmap} et \cmdname{fmtutil} jusqu'à l'an dernier. Les nouveaux
  scripts \cmdname{texconfig}, \cmdname{updmap} et \cmdname{fmtutil} modifient
  maintenant la configuration \emph{personnelle} de l'utilisateur qui les
  lance ; le résultat est placé dans le répertoire personnel de l'utilisateur
  (sous \dirname{$HOME/.texlive2005}). \iffalse$\fi

\item De nouvelles variables, \envname{TEXMFCONFIG} (resp.
  \envname{TEXMFSYSCONFIG}) ont été introduites ; elles définissent les
  répertoires où doivent se trouver les fichiers de configuration de
  l'utilisateur (resp. de la machine), comme \filename{fmtutil.cnf} et
  \filename{updmap.cfg}. Les utilisateurs de fichiers de configuration locaux
  \filename{fmtutil.cnf} ou \filename{updmap.cfg} devront probablement les
  déplacer ; une autre possibilité est de modifier la définition des variables
  \envname{TEXMFCONFIG} ou \envname{TEXMFSYSCONFIG} dans
  \filename{texmf.cnf}. L'important est de s'assurer de la cohérence entre les
  définitions de ces variables dans \filename{texmf.cnf} et l'emplacement réel
  de ces fichiers.  Voir section~\ref{sec:texmftrees},
  page~\pageref{sec:texmftrees} pour la description des différentes
  arborescences texmf utilisées.

\item L'an dernier, nous avions rendu « \texttt{undefined} » certaines
  primitives propres à \cmdname{pdftex} (comme \verb|\pdfoutput|) dans les
  formats créés à partir de \cmdname{pdfetex} (\cmdname{latex} \cmdname{amstex},
  \cmdname{context} par exemple)).  C'était à titre transitoire, aussi cette
  année, ces primitives sont de nouveau définies dans tous les formats à base
  \cmdname{pdf(e)tex}, \emph{même lorsqu'ils sont utilisés pour produire des
    fichiers \file{.dvi}}. Ceci implique qu'il vous faudra modifier vos
  documents qui utilisent le test \verb|\ifx\pdfoutput\undefined| pour
  déterminer si la sortie est en DVI ou en PDF. Le mieux est d'utiliser
  l'extension \pkgname{ifpdf.sty} et son test \verb|\ifpdf| qui fonctionne même
  en plain \TeX.

\item L'an dernier, nous avions fait en sorte que la plupart des formats
  impriment dans les sorties \file{.log} des caractères 8 bit lisibles à la
  place des notations hexadécimales \verb|^^| (voir section précédente).  Le
  nouveau fichier TCX \filename{empty.tcx} permet de revenir facilement à la
  notation traditionnelle \verb|^^|, il suffit de coder :
\begin{verbatim}
latex --translate-file=empty.tcx fichier.tex
\end{verbatim}

\item Le nouveau programme \cmdname{dvipdfmx} est disponible pour convertir des
  fichiers DVI en PDF ; ce programme remplace \cmdname{dvipdfm}, toujours
  disponible, mais dont l'usage est maintenant déconseillé.

\item Les nouveaux programmes \cmdname{pdfopen} et \cmdname{pdfclose} sont
  inclus pour permettre de relancer les fichiers PDF dans Adobe Acrobat Reader
  sans devoir relancer le programme (d'autres afficheurs PDF, notamment
  \cmdname{xpdf}, \cmdname{gv} et \cmdname{gsview}, n'ont jamais souffert de ce
  problème).

\item Les variables \envname{HOMETEXMF} et \envname{VARTEXMF} ont été renommées
  en \dirname{TEXMFHOME} et \envname{TEXMFSYSVAR} respectivement pour raisons de
  cohérence avec les autres noms de variables.  Il y a aussi \envname{TEXMFVAR}
  qui désigne maintenant un répertoire personnel de l'utilisateur (cf. le
  premier point de la présente liste).
\end{itemize}

\subsubsection{2006--2007}

En 2006--2007, la nouveauté majeure a été l'arrivée dans \TL{} de Xe\TeX{}
disponible sous forme de deux programmes \texttt{xetex} et \texttt{xelatex},
voir \url{https://scripts.sil.org/xetex}.

\MP{} a subi une mise à jour importante et d'autres améliorations sont prévues,
voir \url{https://tug.org/metapost/articles}. Il en va de même pour pdf\TeX{},
voir \url{https://tug.org/applications/pdftex}.

Le format \filename{tex.fmt} et les formats pour \MP{} et \MF\ ne se trouvent
plus dans \dirname{texmf-dist/web2c} mais dans des sous-répertoires de
\dirname{texmf-dist/web2c} (la recherche de fichiers \filename{.fmt} est
néanmoins faite aussi dans \dirname{texmf-dist/web2c}).  Ces sous-répertoires
portent le nom du moteur utilisé pour construire le format, par exemple
\filename{tex}, \filename{pdftex} ou \filename{xetex}.  Ce changement ne devrait
pas avoir d'effet visible pour les utilisateurs.

Le programme (plain) \texttt{tex} ignore désormais les lignes commençant par
\texttt{\%\&} qui permettent de déterminer le format à utiliser ; c'est un vrai
\TeX{} « à la Knuth » !  \LaTeX\ et tous les autres prennent toujours en compte
les lignes commençant par \texttt{\%\&}.

Comme chaque année des centaines d'extensions et de programmes ont été mis
à jour, voir le CTAN (\url{https://ctan.org}).

L'arborescence utilisée en interne a été réorganisée avec de nouveaux outils qui
devraient fournir une meilleure base de travail pour les développements futurs.

Enfin, en mai 2006 Thomas Esser a annoncé qu'il renonçait à poursuivre le
développement de te\TeX{} (\url{https://tug.org/tetex}).  Sa décision a relancé
l'intérêt pour \TL, en particulier chez les distributeurs de solutions
\GNU/Linux (un nouveau schéma d'installation \texttt{tetex} a été ajouté dans le
script d'installation de \TL{} pour produire une distribution proche de
l'ancienne te\TeX). La distribution \TL{} existe déjà sous forme de paquets
Debian, espérons que les autres acteurs du monde Linux (RedHat, SUSE, etc.)
suivront et que les utilisateurs se verront proposer à l'avenir des
distributions \TeX{} riches et plus faciles à installer.

\subsubsection{2008}

En 2008, l'infrastructure de la distribution \TL{} a été entièrement remaniée.
Un nouveau fichier texte, \filename{tlpkg/texlive.tlpdb}, regroupe toutes les
informations concernant la configuration \TL{} de la machine.

Ce fichier permet entre autres choses de procéder à des mises à jour par le
réseau après l'installation initiale.  Cette possibilité était offerte depuis
des années par MiK\TeX.  Nous espérons pouvoir fournir des mises à jour
régulières du contenu des archives \CTAN.

Le nouveau moteur Lua\TeX\ (\url{http://luatex.org}) a été intégré ; il offre de
nouvelles fonctionnalités typographiques et repose sur l'excellent langage de
commande Lua qui peut aussi être utilisé en dehors de \TeX.

Les versions Windows et Unix sont beaucoup plus proches que par le passé.  En
particulier les scripts en Perl et en Lua sont communs aux deux architectures.

\TL{} dispose d'une nouvelle interface pour la maintenance (\cmdname{tlmgr} voir
section~\ref{sec:tlmgr}), elle permet les ajouts, mises à jour et suppressions
de composants et prend en charge la régénération des bases \texttt{ls-R}, des
formats et des fichiers \filename{.map} lorsque c'est nécessaire.

Les fonctionnalités de \cmdname{tlmgr} englobent toutes les tâches dévolues
auparavant à \cmdname{texconfig} qui ne devrait plus être utilisé (il est
conservé, mais avec un champ d'action réduit).

Le programme d'indexation \cmdname{xindy} (\url{http://xindy.sourceforge.net/})
est maintenant disponible pour la plupart des plateformes.

L'utilitaire \cmdname{kpsewhich} dispose de deux options nouvelles :
\optname{-{}-all} qui retourne toutes les occurrences du fichier recherché et
\optname{-{}-subdir} qui limite la recherche à un sous-répertoire donné.

Le programme \cmdname{dvipdfmx} permet maintenant d'extraire les informations
concernant la \textit{bounding box} par le biais de la commande
\cmdname{extractbb}; c'était une des dernières fonctionnalités de
\cmdname{dvipdfm} qui manquaient à \cmdname{dvipdfmx}.

Les alias de polices \filename{Times-Roman}, \filename{Helvetica}, etc.  ont été
supprimés, les conflits de codage qu'ils induisaient n'ayant pas pu être
résolus.

Le format \pkgname{platex} a été supprimé afin de résoudre un conflit de nom
avec son homonyme japonais ; le support pour le polonais est assuré maintenant
par l'extension \pkgname{polski}.

Les fichiers \web{} d'extension \filename{.pool} sont maintenant inclus dans les
binaires afin de faciliter les mises à jour.

Enfin, les changements décrits par Donald Knuth dans « \TeX\ tuneup of 2008 »
(voir \url{https://tug.org/TUGboat/Articles/tb29-2/tb92knut.pdf}) sont inclus
dans la présente édition.

\subsubsection{2009}
\label{sec:2009news} % keep with 2009

En 2009, le format de sortie par défaut de Lua\AllTeX\ est maintenant le PDF
afin de profiter de la prise en charge des polices OpenType et autres
fonctionnalités de Lua\TeX. De nouveaux exécutables nommés \code{dviluatex} et
\code{dvilualatex} fournissent une sortie DVI. La page d'accueil de Lua\TeX{}
est \url{http://luatex.org}.

Le moteur Omega ainsi que le format Lambda ont été retirés, après discussion
avec les auteurs d'Omega. Les versions à jour d'Aleph et de Lamed ont été
conservées, ainsi que les utilitaires Omega.

\TL fournit la nouvelle version des polices \TypeI{} de l'AMS, y compris
Computer Modern : les quelques changements de forme faits au cours des ans par
Knuth dans les sources \MF{} ont été intégrés et le \eng{hinting} a été
amélioré. Les polices Euler ont été en grande partie redessinées par Hermann
Zapf (voir \url{https://tug.org/TUGboat/Articles/tb29-2/tb92hagen-euler.pdf}).
Dans tous les cas, les métriques n'ont pas changé. La page d'accueil des polices
de l'AMS est \url{https://ams.org/tex/amsfonts.html}.

Le nouvel éditeur intégré \TeX{}works est fourni pour Windows, ainsi que dans
Mac\TeX. Pour les autres plateformes, ainsi que d'autres informations, voir la
page de \TeX{}works : \url{https://tug.org/texworks}. Cet éditeur fonctionnant
sur de nombreuses plateformes, inspiré par l'éditeur TeXShop de \macOS, vise
à faciliter l'utilisation de \TeX.

Le programme de création de graphiques Asymptote est fourni pour plusieurs
plateformes. Il fournit un langage de description de graphiques en mode texte
plus ou moins du même genre que \MP{}, mais avec, entre autres, des
fonctionnalités 3D avancées. Sa page d'accueil est
\url{https://asymptote.sourceforge.io}.

Le programme \code{dvipdfm} séparé a été remplacé par \code{dvipdfmx} qui,
lorsqu'il est appelé par ce nom, travaille dans un mode de compatibilité
spécifique. Le programme \code{dvipdfmx} fournit des fonctionnalités pour les
écritures CJK et inclut de nombreux correctifs accumulés au fil des ans depuis
la dernière sortie de \code{dvipdfm}.

Des exécutables pour les plateformes \pkgname{cygwin} et \pkgname{i386-netbsd}
sont maintenant fournis, tandis que les autres distributions BSD ont été
abandonnées ; nous avons entendu dire que OpenBSD et FreeBSD fournissent \TeX\
via leurs propres systèmes de gestion de paquets, et par ailleurs il était
difficile sur ces plateformes de fabriquer des binaires qui aient une chance
de fonctionner sur plus d'une version.

Quelques autres changements en vrac : nous utilisons maintenant \pkgname{xz},
qui remplace \pkgname{lzma} (\url{https://tukaani.org/xz/}), pour comprimer nos
archives ; un |$|\iffalse|$|\fi littéral est autorisé dans les noms de fichiers
s'il n'est pas suivi du nom d'une variable connue ; la bibliothèque Kpathsea est
maintenant \emph{multi-threadée} (ce qui sert dans \MP{}) ; le processus complet
de compilation de \TL est maintenant basé sur Automake.

Remarque finale concernant le passé : toutes les anciennes distributions \TL{}
ainsi que les jaquettes des \CD{} correspondants sont disponibles ici :
\url{ftp://tug.org/historic/systems/texlive}.

% 
\subsubsection{2010}
\label{sec:2010news} % keep with 2010

En 2010, les PDF générés utilisent par défaut la version 1.5 du format PDF, ce
qui permet une plus grande compression. Ceci concerne tous les moteurs \TeX{}
produisant directement du PDF, ainsi qu'à \code{dvipdfmx}. Pour revenir au
format PDF 1.4, vous pouvez charger le paquet \pkgname{pdf14} sous \LaTeX{}, ou
régler manuellement |\pdfminorversion=4| (sous pdf\TeX).

pdf\AllTeX\ convertit maintenant \emph{automatiquement} les fichiers EPS
utilisés au format PDF, en utilisant le paquet \pkgname{epstopdf}, dans tous les
cas où le fichier de configuration \LaTeX{} \code{graphics.cfg} est chargé, et
que le format de sortie est le PDF. Les options par défaut sont choisies pour
éviter autant que possible tout risque d'écrasement d'un fichier PDF créé
manuellement, mais vous pouvez aussi empêcher le chargement d'\pkgname{epstopdf}
en plaçant |\newcommand{\DoNotLoadEpstopdf}{}| (ou |\def...| avant la
déclaration \cs{documentclass}. Le paquet n'est pas chargé non plus si
\pkgname{pst-pdf} est utilisé. Pour plus de détails, reportez-vous à la
documentation du paquet \pkgname{epstopdf}
(\url{https://ctan.org/pkg/epstopdf-pkg}).

Un autre changement, relié au précédent, est que l'exécution d'un tout petit
nombre de commandes externes depuis \TeX{} (avec le commande |\write18|) est
désormais autorisée par défaut.  Ces commandes sont \code{repstopdf},
\code{makeindex}, \code{kpsewhich}, \code{bibtex}, et \code{bibtex8} ; cette
liste est définie dans \file{texmf.cnf}. Si vous souhaitez désactiver cette
fonctionnalité, vous pouvez désélectionner cette option à l'installation
(section~\ref{sec:options}), ou modifier la valeur après installation en
exécutant |tlmgr conf texmf shell_escape 0|.

Un autre changement est encore relié aux deux derniers : \BibTeX\ et Makeindex
refusent désormais par défaut d'écrire leur fichier de sortie dans n'importe
quel répertoire (ce que \TeX\ refusait déjà). Ceci a pour but de permettre leur
inclusion dans la liste ci-dessus. Pour modifier ce comportement, vous pouvez
utiliser la variable d'environnement \envname{TEXMFOUTPUT}, ou modifier la
valeur de |openout_any|.

\XeTeX{} offre désormais un ajustement optique des marges similaire à celui de
pdf\TeX (sans la dilatation des fontes, non disponible actuellement).

Par défaut, \prog{tlmgr} garde maintenant une copie de sauvegarde de tous les
paquets mis à jour (\code{tlmgr option autobackup 1}) pour permettre de revenir
facilement à l'ancienne version au cas où la nouvelle soit cassée, en utilisant
\code{tlmgr restore}. Si vous faites des mises à jour après l'installation, et
n'avez pas la place de garder ces copies de sauvegarde, exécutez \code{tlmgr
  option autobackup 0}.

De nouveaux programmes sont inclus : le moteur p\TeX{} et les utilitaires reliés
pour la composition du japonais ; le programme \BibTeX{}U qui est une version de
\BibTeX\ gérant Unicode ; l'utilitaire \prog{chktex} (originellement disponible
à \url{http://baruch.ev-en.org/proj/chktex}) pour vérifier les documents
\AllTeX ; le programme de conversion DVI vers SVG \prog{dvisvgm}
(\url{https://dvisvgm.de}).

Nous fournissons des exécutables pour les nouvelles plateformes suivantes :
\code{amd64-freebsd}, \code{amd64-kfreebsd}, \code{i386-freebsd},
\code{i386-kfreebsd}, \code{x86\_64-darwin} et \code{x86\_64-solaris}.

Un changement dans \TL que nous avions oublié de noter : de nombreux exécutables
relatifs à \TeX4ht ont été retirés de la liste des exécutables. La commande
générique \code{mk4ht} permet de lancer les nombreuses variantes de
\code{tex4ht}.

Enfin, \TL telle qu'elle est présentée dans le \DVD\ \TK{} ne peut plus être
exécutée de façon portable (en \emph{live}) --- contrairement à ce qu'indique le
nom. Il n'y a désormais plus assez de place sur un seul \DVD. Un effet
secondaire agréable de cette nouvelle disposition est que l'installation depuis
un \DVD\ physique est maintenant beaucoup plus rapide.

\subsubsection{2011}

Les binaires pour \macOS\ (\code{universal-darwin} et \code{x86\_64-darwin}) ne
fonctionnent désormais que sous Leopard ou une version ultérieure ; Panther et
Tiger ne sont plus pris en charge.

Le programme \code{biber} pour le traitement des bibliographies est inclus pour
les plateformes les plus courantes. Son développement suit de près celui du
paquet \code{biblatex} qui réimplémente totalement la gestion des bibliographies
sous \LaTeX.

Le programme \MP{} (\code{mpost}) ne crée plus et n'utilise plus de fichiers
\code{.mem}. Les fichiers utiles, comme \code{plain.mp}, sont simplement traités
à chaque exécution. Ceci est relié au support de \MP{} en tant que bibliothèque,
qui est un autre changement important bien que peu visible par les utilisateurs.

L'implémentation en Perl de \code{updmap}, qui n'était utilisée que sous
Windows, a été remaniée et est désormais utilisée sur toutes les plateformes.
Il ne devrait pas y avoir de changement visible par les utilisateurs, mis à part
une exécution bien plus rapide.

Les commandes \cmdname{initex} et \cmdname{inimf} ont été réintroduites (mais
aucune autre variante \cmdname{ini*}).

\subsubsection{2012}

\code{tlmgr} permet d'utiliser plusieurs dépôts de paquets simultanément pour
les mises à jour. Pour plus de détails, consulter la section sur les dépôts
multiples dans l'aide de \code{tlmgr} (utiliser \code{tlmgr help} par exemple).

Le paramètre \cs{XeTeXdashbreakstate} est réglé à~1 par défaut, pour
\code{xetex} et \code{xelatex}. Ceci autorise les coupures de lignes après les
tirets cadratin et demi-cadratins, ce qui a toujours été le comportement de
\TeX, \LaTeX, Lua\TeX, etc. Si des documents \XeTeX\ existants doivent conserver
une parfaite compatibilité au niveau des coupures de lignes, il devront mettre
\cs{XeTeXdashbreakstate} à~0 explicitement.

Les fichiers de sortie générés par \code{pdftex} et \code{dvips}, entre autres,
peuvent maintenant dépasser les 2~gigabytes.

Les 35 polices Postscript standard sont incluses par défaut dans la sortie de
\code{dvips}, car il y en a trop de versions différentes dans la nature.

Dans le mode d'exécution par \cs{write18} limité (le mode par défaut), la
commande \code{mpost} est maintenant autorisée.

Un fichier \code{texmf.cnf} placé dans \filename{../texmf-local} (par exemple :
\filename{/usr/local/texlive/texmf-local/web2c/texmf.cnf}) sera trouvé et
utilisé s'il existe.

Le script \code{updmap} lit maintenant un fichier \code{updmap.cfg} par
arborescence au lieu d'un seul fichier global. Ce changement devrait être
invisible à moins que vous n'ayez édité vos fichiers \code{updmap.cfg}
directement. Voir la sortie de \code{updmap --help} pour plus de détails.

Plateformes : \pkgname{armel-linux} et \pkgname{mipsel-linux} ont été
ajoutés ; \pkgname{sparc-linux} et \pkgname{i386-netbsd} ne font plus partie de
la distribution principale.

\subsubsection{2013}

Disposition des fichiers : le répertoire de premier niveau \code{texmf/} a été
fusionné avec \code{texmf-dist} par souci de simplicité. Les variables Kpathsea
\code{TEXMFMAIN} et \code{TEXMFDIST} pointent désormais toutes les deux sur
\code{texmf-dist}.

Plusieurs petites collections de langues ont été fusionnées entre elles afin de
simplifier l'installation.

\begin{description}
\item[\MP{} :] la gestion native des sorties en PNG, ainsi que des calculs en
  virgule flottante (IEEE double-précision) ont été ajoutés.
\item[Lua\TeX{} :] mis à jour vers Lua 5.2 et inclusion d'une nouvelle
  bibliothèque (\code{pdfscanner}) pour traiter le contenu de fichiers PDF
  externes, ainsi que bien d'autres choses (voir ses pages web).
\item[\XeTeX{} :] (voir également ses pages web pour plus de détails) :
  \begin{itemize*}
  \item la bibliothèque HarfBuzz est maintenant utilisée pour la composition des
    fontes au lieu d'ICU (qui reste utilisée pour la gestion des encodages
    d'entrée, la bidirectionnalité, et l'option de coupure de ligne d'Unicode).
  \item Graphite2 et HarfBuzz sont utilisés à la place de SilGraphite pour la
    composition Graphite.
  \item Sur Mac, Core Text est utilisé en remplacement d'ATSUI qui est déprécié.
  \item Quand deux polices portent le même nom, la version OpenType/TrueType est
    choisie de préférence à la version Type 1.
  \item Une différence possible entre les recherches de fontes de \XeTeX\ et de
    \code{dvipdfmx} a été corrigée.
  \item Gestion des incrustations (cut-ins) OpenType Math.
  \end{itemize*}
\item[\cmdname{xdvi} :] utilise désormais FreeType plutôt que \code{t1lib} pour
  le rendu.
\item[\pkgname{microtype.sty} :] entre autres améliorations, plus de
  fonctionnalités sous \XeTeX{} (protrusion) et Lua\TeX\ (protrusion, expansion
  de fontes, \emph{traking}).
\item[\cmdname{tlmgr} :] nouvelle action \code{pinning} pour faciliter la
  gestion de plusieurs dépôts de paquets; voir la section correspondante de
  \verb|tlmgr --help|, par exemple en ligne :
  \url{https://tug.org/texlive/doc/tlmgr.html#MULTIPLE-REPOSITORIES}.
\item[Plateformes :] \pkgname{armhf-linux}, \pkgname{mips-irix},
  \pkgname{i386-netbsd}, et \pkgname{amd64-netbsd} ajoutées ou ressuscitées,
  \pkgname{powerpc-aix} enlevé.
\end{description}

% 
\subsubsection{2014}

2014 a vu un autre réglage \TeX{} de la part de Knuth ; cela affecte tous les
moteurs, mais le seul changement visible est probablement la restauration de la
chaîne \code{preloaded format} sur la ligne bannière. Pour Knuth, elle reflète
maintenant le format qui \emph{devrait} être chargé par défaut, plutôt qu'un
format non compilé réellement préchargé par le binaire. Cela peut être modifié
de multiples façons.

\begin{description}
\item[pdf\TeX :] nouveau paramètre d'avertissement de suppression
  \cs{pdfsuppresswarningpagegroup}; nouvelles primitives pour des espaces
  intermots factices aidant à la mise en forme du texte PDF:
  \cs{pdfinterwordspaceon}, \cs{pdfinterwordspaceoff}, \cs{pdffakespace}.
\item[Lua\TeX :] des modifications et corrections notables ont été faites
  concernant le chargement des fontes et les césures. L'ajout le plus important
  est la nouvelle variante de moteur, \code{luajittex} et ses sœurs
  \code{texluajit} et \code{texluajitc}. Cela utilise un compilateur Lua à la
  volée (détaillé dans l'article du \textsl{TUGboat}
  \url{https://tug.org/TUGboat/tb34-1/tb106scarso.pdf}).  \code{luajittex} est
  encore en développement, n'est pas disponible pour toutes les plateformes et
  est considérablement moins stable que \code{luatex}. Ni nous ni ses
  développeurs ne recommandons de l'utiliser, sauf à des fins de tests avec jit
  sur du code Lua.
\item[\XeTeX :] les mêmes formats d'images sont acceptés sur toutes les
  plateformes (dont Mac) ; éviter en Unicode les décompositions de
  compatibilité (mais pas les autres variantes) ; préférer les fontes OpenType
  aux fontes Graphite, par souci de compatibilité avec les versions précédentes
  de \XeTeX.
\item[\MP :] un nouveau système numérique \code{decimal} est accepté,
  parallèlement à un compagnon interne \code{numberprecision} ; une nouvelle
  définition de \code{drawdot} dans \filename{plain.mp}, par Knuth; corrections
  de bugs dans les sorties SVG et PNG output, entre autres.
\item[\cmdname{pstopdf} :] cet utilitaire \ConTeXt{} sera retiré en tant que
  commande autonome à un certain moment après la sortie de la 2014, du fait de
  conflits avec des utilitaires de systèmes d'exploitation de même nom. Il peut
  encore (et doit désormais) être invoqué via \code{mtxrun --script pstopdf}.
\item[\cmdname{psutils} :] a été substantiellement révisé par un nouveau
  mainteneur. De ce fait, plusieurs utilitaires rarement utilisés (\code{fix*},
  \code{getafm}, \code{psmerge}, \code{showchar}) se trouvent désormais
  uniquement dans le répertoire \dirname{scripts/} plutôt qu'au niveau des
  exécutables utilisateurs (cela est réversible si ça s'avérait
  problématique). Un nouveau script, \code{psjoin}, a été ajouté.
\item[Mac\TeX :] cette redistribution de la \TL{} (section~\ref{sec:macosx}) ne
  fournit plus les paquets optionnels propres à Mac pour Latin Modern et les
  fontes \TeX\ Gyre puisqu'il est assez aisé pour les utilisateurs de les rendre
  accessibles au système. Le programme \cmdname{convert} de ImageMagick a aussi
  été retiré, puisque \TeX4ht (spécifiquement \code{tex4ht.env}) utilise
  maintenant Ghostscript directement.
\item[\pkgname{langcjk} :] cette collection pour le support du chinois, du
  japonais et du coréen a été scindée en collections de langues individuelles
  par souci de modération de tailles.
\item[Plateformes :] \pkgname{x86\_64-cygwin} ajoutée, \pkgname{mips-irix}
  supprimée; Microsoft ne maintient plus Windows XP donc nos programmes peuvent
  à tout moment commencer à être défaillants sur ce système.
\end{description}

\subsubsection{2015}

\LaTeXe\ incorpore maintenant, par défaut, les changements jusqu'ici inclus
uniquement en chargeant explicitement le paquet \pkgname{fixltx2e}, qui devient
non opérationnel. Un nouveau paquet \pkgname{latexrelease} et d'autres
mécanismes permettent de contrôler ce qui se passe. Des détails se trouvent dans
les documents inclus \emph{\LaTeX\ News \#22} et « \LaTeX\ changes ».
Incidemment, les paquets \pkgname{babel} et \pkgname{psnfss}, quoique parties
intégrantes de \LaTeX, sont maintenus séparément et ne sont pas affectés par ces
changements (et doivent continuer à fonctionner).

En interne, \LaTeXe\ maintenant inclut une configuration de moteur concernant
Unicode (quels caractères sont des lettres, noms des primitives, etc.) qui était
auparavant une composante de la \TL. Ce changement est supposé invisible pour
les utilisateurs ; quelques commandes internes de bas niveau ont été renommées
ou supprimées, mais le résultat devrait être le même.

\begin{description}
\item[pdf\TeX :] prend en charge Exif de JPEG ainsi que JFIF ; n'émet même plus
  d'avertissement si \cs{pdfinclusionerrorlevel} est négatif; synchronisation
  avec \prog{xpdf}~3.04.
\item[Lua\TeX :] nouvelle librairie \pkgname{newtokenlib} pour scanner les
  lexèmes ; corrections de bugs dans le générateur de nombres aléatoires
  \code{normal} et à d'autres endroits.
\item[\XeTeX :] corrections dans la manipulation d'images ; binaire
  \prog{xdvipdfmx} recherché en premier comme programme frère de \prog{xetex} ;
  code opérationnel interne \code{XDV} modifié.
\item[\MP{} :] nouveau système de nombres \code{binary}; nouvelle activation
  japonaise des programmes \prog{upmpost} et \prog{updvitomp}, analogues
  à \prog{up*tex}.
\item[Mac\TeX :] mises à jour du paquet Ghostscript inclus pour le support CJK.
  Le panneau de préférence de la distribution \TeX\ fonctionne maintenant sur
  Yosemite (\macOS~10.10). Les valises de polices du type « ressources »
  (« \emph{resource-fork font suitcases} »), qui ont des noms généralement sans
  extension, ne sont plus prises en charge par \XeTeX; les valises de polices du
  type « data » (extension \code{.dfont}) continuent elles à l'être.
\item[Infrastructure :] le script \prog{fmtutil} a été réimplementé pour lire
  \filename{fmtutil.cnf} sur une base « par arborescence », de façon analogue
  à \prog{updmap}. Les scripts Web2C \prog{mktex*} (dont \prog{mktexlsr},
  \prog{mktextfm}, \prog{mktexpk}) préfèrent maintenant les programmes dans
  leurs répertoires propres, plutôt que recourir systématiquement au
  \envname{PATH} existant.
\item[Plateformes :] \pkgname{*-kfreebsd} supprimé, puisque \TL{} est
  maintenant facilement disponible au travers des mécanismes de plateformes
  systèmes. Support pour quelques plateformes additionnelles disponibles en
  tant que binaires personnalisés
  (\url{https://tug.org/texlive/custom-bin.html}).  De plus, quelques
  plateformes sont désormais omises du \DVD{} (simplement pour gagner de la
  place), mais peuvent être installées normalement depuis le réseau.
\end{description}

\subsubsection{2016}

\begin{description}
\item[Lua\TeX :] modifications radicales concernant les primitives, à la fois
  renommées et supprimées, parallèlement à des réorganisations dans la structure
  des nœuds. Les changements sont résumés dans un article de Hans Hagen,
  « Lua\TeX\ 0.90 backend changes for PDF and more »
  (\url{https://tug.org/TUGboat/tb37-1/tb115hagen-pdf.pdf}). Les détails se
  trouvent dans le manuel de Lua\TeX,
  \OnCD{texmf-dist/doc/luatex/base/luatex.pdf}.
\item[\METAFONT{} :] MFlua et MFluajit, nouveaux programmes hautement
  expérimentaux qui intègrent Lua à \MF (à fins de tests).
\item[\MP{} :] corrections de bugs et préparations internes pour \MP{} 2.0.
\item[pdf\TeX :] utilisation de la variable \code{SOURCE\_DATE\_EPOCH} (si
  configurée) pour l'horodatage ; nouvelles primitives \cs{pdfinfoomitdate},
  \cs{pdftrailerid}, \cs{pdfsuppressptexinfo}, pour contrôler les valeurs
  apparaissant dans la sortie et normalement modifiées à chaque compilation. Ces
  fonctionnalités ne concernent que la sortie PDF, pas la sortie DVI.
\item[Xe\TeX :] nouvelles primitives
  \begin{itemize}
  \item \cs{XeTeXhyphenatablelength},
  \item \cs{XeTeXgenerateactualtext},
  \item \cs{XeTeXinterwordspaceshaping},
  \item \cs{mdfivesum};
  \end{itemize}
  augmentation de la limite de caractères de classe à 4096; identifiant byte du
  DVI incrémenté.
\item[Autres utilitaires :]\
  \begin{itemize*}
  \item \code{gregorio} est un nouveau programme faisant partie du package
    \code{gregoriotex} dédié à la composition de partitions de chants
    grégoriens ; il est par défaut inclus dans \code{shell\_escape\_commands}.
  \item \code{upmendex} est un programme de création d'index, presque
    complètement compatible avec \code{makeindex}, proposant un support pour le
    classement Unicode, entre autres différences.
  \item \code{afm2tfm} désormais ne fait des ajustements à la hausse que sur la
    base de la hauteur des accents ; une nouvelle option \code{-a} omet tous les
    ajustements.
  \item \code{ps2pk} peut traiter les fontes PK/GF étendues.
  \end{itemize*}
\item[Mac\TeX :] le panneau de préférence de la distribution a disparu ; sa
  fonctionnalité se trouve désormais dans la « \TL{} Utility » ; l'interface
  graphique utilisateur a été mise à jour ; un nouveau script
  \code{cjk-gs-integrate} à lancer par les utilisateurs qui souhaitent
  incorporer diverses fontes CJK dans Ghostscript.
\item[Infrastructure :] fichier de configuration au niveau du système pris en
  charge ; vérification des sommes de contrôle des packages ; si GPG est
  disponible, vérification de la signature des mises à jour réseau (dans le cas
  contraire, les mises à jour se font comme par le passé).
\item[Plateformes :] \code{alpha-linux} et \code{mipsel-linux} supprimées.
\end{description}

%
\subsubsection{2017}

\begin{description}
\item[Lua\TeX :] davantage de \enquote{callbacks}\footnote{N.d.T. : fonctions de
    rappel}, de contrôle de composition, d'accès aux fonctions internes ;
  bibliothèque \code{ffi} pour un chargement dynamique de code ajoutée
  à certaines plateformes.
\item[pdf\TeX :] variable d'environnement |SOURCE_DATE_EPOCH_TEX_PRIMITIVES| de
  l'an passé renommée en |FORCE_SOURCE_DATE| sans changement de fonctionnalité ;
  si la liste de lexèmes \cs{pdfpageattr} contient la chaîne \code{/MediaBox},
  omission de la sortie par défaut \code{/MediaBox}.
\item[Xe\TeX :] mathématiques Unicode/OpenType maintenant basées sur le support
  de la table MATH HarfBuzz's ; quelques corrections de bogues.
\item[Dvips :] le \enquote{special} de taille de papier pris en compte est le
  dernier stipulé, par cohérence avec \code{dvipdfmx} et avec ce qu'attendent
  les extensions ; l'option \code{-L0} (configuration \code{L0}) restaure le
  comportement précédent en ce qui concerne le premier \enquote{special} pris en
  compte.
\item[ep\TeX, eup\TeX :] nouvelles primitives \cs{pdfuniformdeviate},
  \cs{pdfnormaldeviate}, \cs{pdfrandomseed}, \cs{pdfsetrandomseed},
  \cs{pdfelapsedtime}, \cs{pdfresettimer}, provenant de pdf\TeX.
\item[Mac\TeX :] à compter de cette année, seules les versions de \macOS\ pour
  lesquelles Apple fournit encore des mises à jour de sécurité seront prises en
  charge dans la Mac\TeX, sous le nom de plateforme |x86_64-darwin|;
  actuellement, cela recouvre les versions Yosemite, El~Capitan et Sierra (10.10
  et suivantes). Les binaires pour les versions plus anciennes de \macOS\ ne
  sont pas incluses dans la Mac\TeX{} mais sont toujours disponibles dans la
  \TL{} (|x86_64-darwinlegacy|, \code{i386-darwin}, \code{powerpc-darwin}).
\item[Infrastructure :] par défaut, l'arborescence \dirname{TEXMFLOCAL} est
  maintenant cherchée avant les arborescences \envname{TEXMFSYSCONFIG} et
  \envname{TEXMFSYSVAR} ; le but est de correspondre davantage à ce qui est
  attendu par les fichiers locaux destinés à prendre le pas sur les fichiers
  système. De plus, \code{tlmgr} a un nouveau mode \code{shell} pour un usage
  interactif et via des scripts, et une nouvelle action \code{conf auxtrees}
  permettant de facilement ajouter et supprimer des arborescences
  supplémentaires.
\item[\code{updmap} et \code{fmtutil} :] ces scripts émettent désormais un
  avertissement quand ils sont invoqués sans que soient spécifiés
  \begin{itemize}
  \item soit le mode dit \enquote{système} (\code{updmap-sys},
    \code{fmtutil-sys} ou option \code{-sys}) ;
  \item soit le mode \enquote{utilisateur} (\code{updmap-user},
    \code{fmtutil-user} ou option \code{-user}).
  \end{itemize}
  Le but est de réduire le problème récurrent consistant à invoquer par accident
  le mode utilisateur et ainsi perdre les mises à jour système
  ultérieures. Cf. \url{https://tug.org/texlive/scripts-sys-user.html} pour plus
  de détails.
\item[\code{install-tl} :] les arborescences personnelles telles que
  \dirname{TEXMFHOME} sont désormais fixées sur Mac aux valeurs par défaut
  (|~/Library/...|).  Nouvelle option \code{-init-from-profile} pour démarrer
  une installation avec des valeurs fixées par un profil donné ; nouvelle
  commande \code{P} pour explicitement sauvegarder un profil ; nouveaux noms de
  variable de profil (mais les précédents sont toujours acceptés).
\item[Sync\TeX :] le nom du fichier temporaire est désormais de la forme
  \code{foo.synctex(busy)}, et non plus \code{foo.synctex.gz(busy)} (absence
  de~\code{.gz}). Les interfaces graphiques et systèmes de compilation qui
  veulent supprimer les fichiers temporaires peuvent devoir ajuster en
  conséquence.
\item[Autres utilitaires :] \code{texosquery-jre8} est un nouveau programme
  multi-plateforme permettant de retrouver la \enquote{locale} et autres
  informations du système d'exploitation depuis un document \TeX{}; il est par
  défaut inclus dans les |shell_escape_commands| autorisant les exécutions shell
  restreintes. (Les anciennes versions de JRE sont supportées par texosquery,
  mais ne peuvent être activées en mode restreint car elles ne sont plus
  supportées par Oracle, même pour des problèmes de sécurité.)
\item[Plateformes :] cf. entrée Mac\TeX\ ci-dessus; pas d'autres changements.
\end{description}

\subsubsection{2018}

\begin{description}
\item[Kpathsea :] par défaut, la recherche est désormais insensible à la casse,
  et ce pour \emph{tous} les répertoires.  Pour désactiver cette fonctionnalité,
  il faut changer \code{texmf.cnf} ou donner à la variable d'environnement
  \code{texmf\_casefold\_search} la valeur \code{0}.  Vous trouverez tous les
  détails nécessaires dans le manuel de Kpathsea
  (\url{https://tug.org/kpathsea}).
\item[ep\TeX, eup\TeX :] nouvelle primitive \cs{epTeXversion}.
\item[Lua\TeX :] préparation de la migration vers Lua\TeX\ 5.3 en 2019 : un
  exécutable \code{luatex53} est désormais disponible pour la plupart des
  architectures, mais doit être renommé en \code{luatex} pour être utilisé.  Il
  est également possible d'utiliser les fichiers de \ConTeXt\ Garden
  (\url{https://wiki.contextgarden.net}) en suivant les informations fournies
  par ledit site.
\item[\MP{} :] rustines corrigeant les chemins erronés, sorties TFM et PNG.
\item[pdf\TeX :] possibilité de coder des vecteurs pour fontes bitmap ; le titre
  des fichiers PDF ne fait plus mention de leur répertoire ; rustines pour
  \cs{pdfprimitive} et fichiers y afférant.
\item[Xe\TeX :] la fonction \code{/Rotate} est désormais utilisable lors de
  l'inclusion d'un fichier PDF ; code de retour d'erreur non nul si le pilote de
  sortie échoue ; plusieurs rustines subtiles concernant l'UTF-8 et autres
  primitives.
\item[Mac\TeX :] voir les changements de support de version ci-dessous.  De
  surcroît, et pour plus de limpidité, les fichiers installés par Mac\TeX\ dans
  le dossier \code{/Applications/TeX/} ont subi une réorganisation ; ce dossier
  contient désormais, au premier niveau, quatre applications graphiques
  (BibDesk, LaTeXiT, \TL{} Utility, et TeXShop) ainsi que des dossiers
  comprenant les utilitaires additionnels et la documentation.
\item[\code{tlmgr} :] nouvelles applications graphiques \code{tlshell} (rédigée
  en Tcl/Tk) et \code{tlcockpit} (rédigée en Java) ; sortie JSON ;
  \code{uninstall} est désormais synonyme de \code{remove}; nouvelle
  commande/option \code{print-platform-info}.
\item[Architectures :]\
  \begin{itemize*}
  \item deux suppressions : \code{armel-linux} et \code{powerpc-linux}.
  \item Mac :
    \begin{itemize*}
    \item \code{x86\_64-darwin} prend en charge 10.10--10.13 (Yosemite,
      El~Capitan, Sierra et High~Sierra).
    \item \code{x86\_64-darwinlegacy} prend en charge 10.6--10.10 (mais
      \code{x86\_64-darwin} est préférable pour 10.10).  10.5 (Leopard) n'est
      plus pris en charge, ce qui implique que \code{powerpc-darwin} et les
      architectures \code{i386-darwin} ne le sont également plus.
    \end{itemize*}
  \item Windows: XP n'est désormais plus pris en charge.
  \end{itemize*}
\end{description}

\subsubsection{2019}

\begin{description}
\item[Kpathsea :] expansion d'accolades et découpage de chemin plus cohérent ;
  nouvelle variable \code{TEXMFDOTDIR} au lieu du \code{.}  codé en dur dans les
  chemins : permet de facilement rechercher des répertoires additionnels ou
  inclus (cf. commentaires dans le fichier \code{texmf.cnf}).
\item[ep\TeX, eup\TeX :] nouvelles primitives \cs{readpapersizespecial} et
  \cs{expanded}.
\item[Lua\TeX :] Lua 5.3 désormais utilisé, avec les changements concomitants
  de l'arithmétique et de l'interface. La librairie propre pplib est utilisée
  pour lire les fichiers PDF, éliminant ainsi la dépendance de poppler (et la
  nécessité de C++) ; interface Lua modifiée en conséquence.
\item[MetaPost :] nom de commande \code{r-mpost} reconnu comme un alias pour
  l'invocation avec l'option \code{-{}-restricted}, et ajout à la liste des
  commandes restreintes disponibles par défaut. Précision minimale désormais~2
  pour les modes décimal et binaire. Mode binaire désormais indisponible avec
  MPlib, mais toujours disponible avec MetaPost autonome.
\item[pdf\TeX :] nouvelle primitive \cs{expanded}; si le nouveau paramètre de
  primitive \cs{pdfomitcharset} est fixé à 1, la chaîne \code{/CharSet} est
  supprimée de la sortie PDF, car il est impossible de garantir qu'elle soit
  correcte, ce qui est requis par PDF/A-2 et PDF/A-3.
\item[Xe\TeX :] nouvelles primitives \cs{expanded}, \cs{creationdate},
  \cs{elapsedtime}, \cs{filedump}, \cs{filemoddate}, \cs{filesize},
  \cs{resettimer}, \cs{normaldeviate}, \cs{uniformdeviate}, \cs{randomseed};
  \cs{Ucharcat} étendu pour produire des caractères actifs.
\item[\code{tlmgr} :] support de \code{curl} comme programme de téléchargement ;
  si disponibles, utilise \code{lz4} et gzip avant \code{xz} pour les
  sauvegardes locales ; en ce qui concerne les programmes de compression et de
  téléchargement, priorité aux exécutables fournis par le système d'exploitation
  sur ceux fournis par la \TL{}, à moins que la variable d'environnement
  \code{TEXLIVE\_PREFER\_OWN} soit configurée.
\item[\code{install-tl} :] la nouvelle option \code{-gui} (sans argument) est
  celle par défaut sous Windows et Mac, et invoque une nouvelle interface
  graphical Tcl/TK GUI (voir sections~\ref{sec:basic}
  et~\ref{sec:graphical-inst}).
\item[Utilitaires :] \
  \begin{description}
  \item[\code{cwebbin} (\url{https://ctan.org/pkg/cwebbin})] est désormais
    l'implémentation CWEB dans la \TL{}, avec un support élargi des dialectes,
    et incluant le programme \code{ctwill} permettant de créer des mini-index.
  \item[\code{chkdvifont} :] rapporte les informations de fontes fournies par les
    fichiers \dvi{}, y compris les fontes tfm/ofm, vf, gf, pk.
  \item[\code{dvispc} :] crée un fichier \dvi{} indépendant des pages vis-à-vis
    des « specials ».
  \end{description}
\item[Mac\TeX{} :] \code{x86\_64-darwin} prend en charge désormais les systèmes 10.12
  et plus (Sierra, High Sierra, Mojave) ; \code{x86\_64-darwinlegacy} prend en charge
  encore les systèmes 10.6 et plus. Le correcteur orthographique Excalibur n'est
  plus inclus puisqu'il requiert le support des architectures 32~bit.
\item[Architectures :] \code{sparc-solaris} supprimée.
\end{description}

\subsubsection{2020}

\begin{description}
\item[Général :]\
  \begin{itemize}
  \item Dans tous les moteurs \TeX, y compris \texttt{tex}, la primitive
    \cs{input} accepte désormais également un argument de nom de fichier
    délimité par groupe, en tant que extension dépendant du
    système. L'utilisation avec un nom de fichier délimité par un espace/token
    est totalement inchangée. La délimitation par groupe était auparavant
    implémentée dans Lua\TeX ; maintenant, elle est disponible dans tous les
    moteurs.  Les doubles guillemets ASCII (\texttt{"}) sont supprimés du nom de
    fichier, mais ils restent inchangés après tokenisation. Cela n'affecte
    actuellement pas la commande \cs{input} de \LaTeX, car c'est une
    redéfinition de macro de la primitive standard \cs{input}.
  \item Nouvelle option \texttt{-{}-cnf-line} pour \texttt{kpsewhich},
    \texttt{tex}, \texttt{mf}, et tous les autres moteurs, pour prendre en charge une
    configuration arbitraire en ligne de commande.
  \item L'ajout de diverses primitives à divers moteurs cette année et les
    précédentes est destiné à aboutir à un ensemble commun de fonctionnalités
    disponible sur tous les moteurs (\textsl{\LaTeX\ News \#31},
    \url{https://latex-project.org/news}).
  \end{itemize}
\item[ep\TeX, eup\TeX :] nouvelles primitives \cs{Uchar}, \cs{Ucharcat},
  \cs{current(x)spacingmode}, \cs{ifincsname} ; révision de \cs{fontchar??} et
  \cs{iffontchar}. Pour eup\TeX\ seulement : \cs{currentcjktoken}.
\item[Lua\TeX :] intégration avec la bibliothèque HarfBuzz, disponible sous
  forme de nouveaux moteurs \texttt{luahbtex} (utilisé pour \texttt{lualatex})
  et \texttt{luajithbtex}.  Nouvelles primitives : \cs{eTeXgluestretch},
  \cs{eTeXglueshrink}.
\item[pdf\TeX :] nouvelle primitive \cs{pdfmajorversion} ; cela ne fait que
  changer le numéro de version dans la sortie PDF ; ça n'a aucun effet sur le
  contenu du PDF.  \cs{pdfximage} et similaires recherchent maintenant les
  fichiers images de la même manière que \cs{openin}.
\item[p\TeX :] nouvelles primitives : \cs{ifjfont}, \cs{iftfont}. Aussi dans
  ep\TeX, up\TeX, eup\TeX.
\item[Xe\TeX :] corrections pour \cs{Umathchardef}, \cs{XeTeXinterchartoks},
  \cs{pdfsavepos}.
\item[Dvips :] encodages de sortie pour les polices bitmap, pour de meilleures
  capacités de copier/coller
  (\url{https://tug.org/TUGboat/tb40-2/tb125rokicki-type3search.pdf}).
\item[Mac\TeX :] Mac\TeX\ et \texttt{x86\_64-darwin} requièrent désormais la
  version 10.13 ou plus élevé de Mac OS (High~Sierra, Mojave et Catalina) ;
  \texttt{x86\_64-darwinlegacy} prend en charge 10.6 et plus récent. Mac\TeX\ est
  certifié et les programmes en ligne de commande ont des \enquote{lancements
    renforcés} (\emph{hardened runtimes}), comme maintenant requis par Apple
  pour l'installation des paquets. BibDesk et \TL{} Utility ne sont pas
  dans Mac\TeX\ parce qu'ils ne sont pas certifiés, mais un fichier
  \filename{README} donne la liste des URL où ils peuvent être obtenus.
\item[\code{tlmgr} et infrastructure :]\
  \begin{itemize*}
  \item Réessayer automatiquement (une fois) les téléchargements de paquets qui
    ont échoué.
  \item Nouvelle option \texttt{tlmgr check texmfdbs}, pour vérifier pour chaque
    arborescence la cohérence des fichiers \texttt{ls-R} et \texttt{!!}.
  \item Utiliser des noms de fichiers versionnés pour les conteneurs de paquets,
    comme dans \texttt{tlnet/archive/\textsl{pkgname}.rNNN.tar.xz} ; devrait
    être invisible pour les utilisateurs, mais constitue un changement notable
    dans la distribution.
  \item Les informations \texttt{catalogue-date} ne sont plus propagées à partir
    du catalogue \TeX{} car il n'était souvent pas lié aux mises à jour des
    paquets.
  \end{itemize*}
\end{description}

\subsubsection{2021}

\begin{description}
\item[Général :]\
  \begin{itemize}
  \item Les modifications apportées par Donald Knuth pour la mise au point 2021
    de \TeX{} et Metafont sont incorporées
    (\url{https://tug.org/TUGboat/tb42-1/tb130knuth-tuneup21.pdf}). Elles sont
    égalements disponibles sur le CTAN sous les packages \code{knuth-dist} et
    \code{knuth-local}. Comme prévu, les corrections concernent des cas obscurs
    et n'affectent rien en pratique.

  \item (Sauf dans l'original \TeX{}.) Si \cs{tracinglostchars} est réglé sur
    3 ou de plus, les caractères manquants n'entraîneront pas seulement un
    message dans le fichier journal, mais une erreur et le code de caractère
    manquant sera affiché en hexadécimal.

  \item (Sauf dans l'original \TeX{}.)  Un nouveau paramètre entier
    \cs{tracingstacklevels}, si positif et si \cs{tracingmacros} est également
    positif, provoque un préfixe indiquant la profondeur de la macro-expansion
    à sortir sur chaque ligne de journal pertinente (par exemple, |~..| à la
    profondeur 2).  En outre, l'enregistrement des macros est tronqué à une
    profondeur $\geqslant$ la valeur du paramètre.

  \end{itemize}

\item[Aleph :] Le format \LaTeX\ basé sur Aleph, appelé \code{lamed}, a été
  supprimé. Le binaire \code{aleph} lui-même est toujours inclus et supporté.

\item[Lua\TeX :]\
  \begin{itemize}
  \item Lua 5.3.6.
  \item Callback pour les niveaux imbriqués dans \cs{tracingmacros}, en tant que
    variante généralisée des nouveaux \cs{tracingstacklevels}.
  \item Marque les glyphes mathématiques comme étant protégés pour éviter qu'ils
    ne soient traités comme du texte.
  \item Suppression de la compensation width/ic pour le chemin de code
    mathématique traditionnel.
  \end{itemize}

\item[MetaPost :]\
  \begin{itemize}
  \item |SOURCE_DATE_EPOCH| : support de variable d'environnement pour une
    sortie reproductible.
  \item Évite un mauvais \texttt{\%} final dans \texttt{mpto}.
  \item Documente l'option \texttt{-T} ; autres corrections au manuel.
  \item Valeur de \texttt{epsilon} modifiée en modes binaire et décimal, si bien
    que |mp_solve_rising_cubic| fonctionne comme prévu.
  \end{itemize}

\item[pdf\TeX{} :]\
  \begin{itemize}
  \item Nouvelles primitives \cs{pdfrunninglinkoff} et \cs{pdfrunninglinkon} ;
    par exemple pour désactiver la génération de liens dans les en-têtes et les
    pieds de page.
  \item Avertir au lieu d'interrompre quand \foreignquote{english}{cs{pdfendlink} ended up in
    different nesting level than \cs{pdfstartlink}}.
  \item Dump des assignations \cs{pdfglyphtounicode} dans le fichier
    \texttt{fmt}.
  \item Source : support de \texttt{poppler} supprimé car il était trop
    difficile de rester en phase avec l'original. Dans la TL native, pdf\TeX\
    a toujours utilisé \texttt{libs/xpdf}, qui est un code réduit et adapté de
    \texttt{xpdf}.
  \end{itemize}

\item[Xe\TeX{} :] Corrections pour le crénage mathématique.

\item[Dvipdfmx :]\
  \begin{itemize}
  \item Ghostscript est maintenant par défaut invoqué en toute sécurité ; pour
    outrepasser cela (en supposant que tous les fichiers d'entrée sont fiables),
    utilisez \verb|-i dvipdfmx-unsafe.cfg|.

    Attention ! Pour utiliser PSTricks avec \XeTeX, cette dernière option est
    nécessaire. Ainsi, avec un fichier \filename{foo.tex} qui contient du code
    PSTricks, vous devez lancer :

    \verb|xetex -output-driver="xdvipdfmx -i dvipdfmx-unsafe.cfg -q -E" foo|
   \item Si un fichier image n'est pas trouvé, interruption avec un statut
    \enquote{mauvais}.
  \item Syntaxe spéciale étendue pour le support des couleurs.
  \item \emph{Specials} pour la manipulation de |ExtGState|.
  \item Compatibilité des \emph{specials} \code{pdfcolorstack} et \code{pdffontattr}.
  \item Support expérimental pour les |fnt_def| étendus de \code{dviluatex}.
  \item Prise en charge d'une nouvelle fonctionnalité de la police virtuelle
    pour la définition de la police japonaise.
  \end{itemize}

\item[Dvips :]\
  \begin{itemize}
  \item Le titre du document PostScript par défaut est maintenant le nom de base
    du fichier d'entrée et peut être remplacé par la nouvelle option
    \texttt{-title}.
  \item Si un \texttt{.eps} ou un autre fichier image n'est pas trouvé,
    interruption avec un statut \enquote{mauvais}.
  \item Prise en charge la nouvelle fonctionnalité de la police virtuelle comme
    solution de repli sur la définition de la police japonaise.
  \end{itemize}

\item[Mac\TeX{} :] Mac\TeX{} et son nouveau dossier binaire
  \texttt{universal-darwin} nécessitent maintenant \macOS{} 10.14 ou plus (Mojave,
  Catalina, et Big~Sur) ; le dossier binaire |x86_64-darwin| n'est plus
  présent. Le dossier binaire |x86_64-darwinlegacy|, disponible uniquement avec
  l'installateur Unix \texttt{install-tl}, prend en charge les versions 10.6 et plus
  récentes.

  C'est une année importante pour le Macintosh car Apple, qui a introduit les
  processeurs ARM en novembre, vendra et prendra en charge pendant de nombreuses
  années des processeurs à la fois ARM et Intel. Tous les programmes dans
  \texttt{universal-darwin} ont un code exécutable pour ARM et Intel. Les deux
  binaires sont compilés à partir du même code source.

  Les programmes supplémentaires Ghostscript, LaTeXiT, \TL{} Utility et
  TeXShop, tous universels et signés avec un \enquote{lancement renforcé}
  (\emph{hardened runtime}), sont inclus dans Mac\TeX{} cette année.

\item[\code{tlmgr} et l'infrastructure :]\
  \begin{itemize}
  \item Une seule sauvegarde du dépôt principal \texttt{texlive.tlpdb} est
    conservée.
  \item Davantage de portabilité entre les systèmes et les versions Perl.
  \item \texttt{tlmgr info} signale les nouveaux \texttt{lcat-*} et
    \texttt{rcat-*} pour les données du catalogue local ou distant.
  \item Enregistrement complet des sous-commandes transféré dans un nouveau
    fichier journal \texttt{texmf-var/web2c/tlmgr-commands.log}.
  \end{itemize}

\end{description}

\subsubsection{2022}

\begin{description}
\item[Général :]\leavevmode
  \begin{itemize}
  \item Nouveau moteur \code{hitex}, qui produit son propre format HINT, conçu
    spécialement pour la lecture de documents techniques sur des appareils
    mobiles.  Les visionneurs HINT pour GNU/Linux, Windows et Android sont
    disponibles séparément de \TL.
  \item \code{tangle}, \code{weave} : support d'un troisième argument facultatif
    pour spécifier le fichier de sortie.
  \item \code{twill}, le programme de Knuth permettant de créer des mini-index
    pour les programmes \texttt{WEB} originaux, est maintenant inclus.
  \end{itemize}

\item[Extensions inter-moteurs :] (sauf dans les versions originales \TeX{},
  Aleph et hi\TeX{})
  \begin{itemize}
  \item Nouvelle primitive \cs{showstream} pour rediriger la sortie \cs{show}
    vers un fichier.
  \item Les nouvelles primitives \cs{partokenname} et \cs{partokencontext}
    permettent de modifier le nom du lexème \cs{par} émis aux lignes vides, à la
    fin des « vboxes », etc.
  \end{itemize}

\item[ep\TeX{}, eup\TeX{} :]\leavevmode
  \begin{itemize*}\raggedright
  \item Nouvelles primitives : \cs{lastnodefont}, \cs{suppresslongerror},
    \cs{suppressoutererror}, \cs{suppressmathparerror}.
  \item extension pdf\TeX{} \cs{vadjust pre} maintenant disponible.
  \end{itemize*}

\item[Lua\TeX{} :]\leavevmode
  \begin{itemize*}
  \item Prise en charge des destinations structurées à partir de PDF 2.0.
  \item PNG /Smask pour PDF 2.0.
  \item Interface de police variable pour \code{luahbtex}.
  \item Différents styles de radicaux par défaut dans mathdefaultsmode.
  \item Blocage optionnel de la création discrétionnaire sélectionnée.
  \item Améliorations de l'implémentation des polices TrueType.
  \item Allocation plus efficace des \cs{fontdimen}.
  \item Paragraphes comportant uniquement un nœud \code{par} local suivi de
    nœuds de synchronisation de direction ignorés.
  \end{itemize*}

\item[MetaPost :] Correction d'un bogue relatif à l'expansion infinie des
  macros.

\item[pdf\TeX{} :]\leavevmode
  \begin{itemize*}\raggedright
  \item Prise en charge des destinations structurées à partir de PDF 2.0.
  \item Pour les polices à espacement de lettres, usage explicite de
    \cs{fontdimen}6 si spécifié.
  \item Tous les avertissements commencent en début de ligne.
  \item Pour les caractères avec autokern (\cs{pdfappendkern} et
    \cs{pdfprependkern}), faites toujours la protrusion ; de même, autokern
    implicite et explicite des traits d'union.
  \end{itemize*}

\item[p\TeX{}\ et al. :]\leavevmode
  \begin{itemize*}
  \item Mise à jour majeure de p\TeX{} vers la version 4.0.0 pour mieux
    prendre en charge la version actuelle de \LaTeX{}.
  \item Nouvelles primitives \cs{ptexlineendmode} et \cs{toucs}.
  \item \cs{ucs} (auparavant disponible dans uptex, euptex) désormais disponible
    également dans p\TeX\ et ep\TeX.
  \item Distinction des caractères 8 bit et des caractères japonais, comme
    indiqué dans un article du TUGboat rédigé par Hironori Kitagawa
    (\url{https://tug.org/TUGboat/tb41-3/tb129kitagawa-char.pdf}).
  \end{itemize*}

\item[Xe\TeX{} :] Nouveaux scripts enveloppants (« wrappers »)
  \texttt{xetex-unsafe} et \texttt{xelatex-unsafe} pour une invocation plus
  simple des documents nécessitant à la fois \XeTeX{} et les opérateurs de
  transparence PSTricks, ce qui est intrinsèquement dangereux (jusqu'à ce que la
  réimplémentation dans Ghostscript ait lieu). Par mesure de sécurité, utilisez
  plutôt Lua\AllTeX{}.

\item[Dvipdfmx :]\leavevmode
  \begin{itemize*}
  \item Prise en charge de PSTricks sans nécessité de recourir
    à \texttt{-dNOSAFER}, sauf pour la transparence.
  \item Option \texttt{-r} permettant de définir la résolution des polices
    bitmap à nouveau fonctionnelle.
  \end{itemize*}

\item[Dvips :] Par défaut, pas de tentative d'ajustement automatique du support
  pour les formats de papier pivotés ; la nouvelle option
  \texttt{--landscaperotate} l'active à nouveau.

\item[\code{upmendex} :] Prise en charge expérimentale des écritures arabe et
  hébraïque ; amélioration de la classification des caractères et de la prise en
  charge des langues.

\item[Kpathsea :] Le premier chemin renvoyé par \texttt{kpsewhich -all} est
  maintenant le même qu'avec une recherche régulière (non « -all »).

\item[\code{tlmgr} et infrastructure :]\leavevmode
  \begin{itemize*}
  \item Utilisation par défaut de https pour \code{mirror.ctan.org}.
  \item utilisation de \code{TEXMFROOT} à la place de \code{SELFAUTOPARENT} pour
    faciliter la relocalisation.
  \item \code{install-tl} : si le téléchargement ou l'installation échoue pour
    un package donné, poursuite automatique de l'installation avec nouvel essai
    ultérieur (unique).
  \end{itemize*}

\item[Mac\TeX{} :] Mac\TeX{} et son dossier binaire \texttt{universal-darwin}
  nécessitent \macOS{} 10.14 ou supérieur (Mojave, Catalina, Big~Sur, Monterey). Le
  dossier binaire |x86_64-darwinlegacy|, disponible uniquement avec le
  \texttt{install-tl} d'Unix, prend en charge les versions 10.6 (Snow~Leopard)
  et plus récentes.

\item[Plateformes :] Aucun changement dans la prise en charge des plateformes
  pour cette année (2022). Cependant, pour la version de l'année prochaine
  (2023), nous prévoyons de faire passer les binaires Windows de 32 à 64
  bit. Malheureusement, nous ne pouvons pas prendre en charge les deux
  simultanément.

\end{description}

\htmlanchor{news}
\subsection{Présent : 2023}
\label{sec:tlcurrent}

\begin{description}
\item[Windows :] Comme annoncé précédemment, \TL{} contient maintenant des
  binaires Windows 64~bit au lieu de 32~bit. Le nouveau nom du répertoire est
  \texttt{bin/windows} (il ne semblait pas correct de mettre les binaires 64~bit
  dans un répertoire nommé « 32 »). Nous savons que cela entraînera un surcroît
  de travail supplémentaire pour les utilisateurs de Windows, mais il ne
  semblait pas y avoir de meilleure alternative. Voir la page web séparée \TL{}
  Windows (\url{https://tug.org/texlive/windows.html}).

\item[Extensions inter-moteurs\footnote{Sauf dans les versions originales \TeX{}
    et e-\TeX{}.} :] \cs{special} suivi d'un nouveau mot-clé
  « \texttt{shipout} » retarde l'expansion des tokens de l'argument jusqu'au
  moment \cs{shipout}, comme avec un
  non-\verb|\immediate\write|.{\raggedright\par}

\item[ep\TeX{}, eup\TeX{} :]\leavevmode{}
  \begin{itemize*}
  \item « Raw » (u)ptex n'est plus construit ; (u)ptex fonctionne maintenant
    dans le mode de compatibilité d'e(u)ptex. Idem pour les outils p\TeX{},
    listés ci-dessous.
  \item Nouvelles primitives : \cs{tojis}, \cs{ptextracingfonts},
    \cs{ptexfontname}.
  \item Pour \cs{font}, la nouvelle syntaxe pour JIS/UCS est prise en charge.
  \end{itemize*}

\item[Lua\TeX :]\leavevmode{}
  \begin{itemize*}
  \item nouvelle primitive \cs{variablefam} pour permettre aux caractères
    mathématiques de conserver leur classe tout en laissant la famille
    s'adapter.
  \item amélioration des zones d'annotation r2l
  \item « \cs{special} retardé » inter-moteurs décrit ci-dessus.
  \end{itemize*}

\item[MetaPost :] Correction de bogues. \texttt{svg->dx} et \texttt{svg->dy}
  sont maintenant \texttt{double}, pour une meilleure précision ;
  \verb|mp_begin_iteration| mis à jour ; fuite de mémoire dans \texttt{mplib}
  corrigée.

\item[pdf\TeX{} :] \leavevmode{}
\begin{itemize*}
\item nouvelle primitive \cs{pdfomitinfodict} pour omettre complètement le
  dictionnaire \texttt{/Info}.
\item nouvelle primitive \cs{pdfomitprocset} pour contrôler l'omission du
  tableau \texttt{/ProcSet} : \texttt{/ProcSet} est inclus si ce paramètre est
  négatif, ou si ce paramètre est zéro et que \prog{pdftex} génère une sortie
  PDF~1.x.
\item avec \cs{pdfinterwordspaceon}, si l'encodage de la police actuelle possède
  un caractère \texttt{/space} à l'emplacement 32, il est utilisé ; sinon, le
  \texttt{/space} de la (nouvelle) police par défaut \texttt{pdftexspace} est
  utilisé. Cette police par défaut peut être remplacée par la nouvelle primitive
  \cs{pdfspacefont}. Cette même nouvelle procédure est utilisée pour
  \cs{pdffakespace}.
 \end{itemize*}

\item[p\TeX{} et al. :]\leavevmode{}
  \begin{itemize*}
  \item Comme mentionné ci-dessus, \texttt{ptex} exécute maintenant
    \texttt{eptex} en mode de compatibilité au lieu d'être construit séparément.
  \item Les outils p\TeX{} (pbibtex, pdvitype, ppltotf, ptftopl) ont été
    fusionnés en versions up\TeX{} correspondantes, fonctionnant en mode de
    compatibilité.
  \end{itemize*}

\item[Xe\TeX :]\leavevmode{}
  \begin{itemize}
  \item correction d'un bogue dans le calcul de \cs{topskip} et
    \cs{splittopskip} lorsque \cs{XeTeXupwardsmode} est actif ;
  \item « \cs{special} retardé » décrit ci-dessus.
  \end{itemize}

\item[Dvipdfmx :] nouvelle option \texttt{--pdfm-str-utf8} pour créer un
  « pdfmark » et/ou un signet.

\item[\BibTeX{}u :]\leavevmode{}
  \begin{itemize*}
  \item Cette variante de Bib\TeX{} est principalement compatible avec \BibTeX,
    avec un bien meilleur support multilingue (basé sur Unicode). Elle est
    présente dans \TL{} depuis quelques années.
  \item Cette année, des fonctionnalités supplémentaires ont été ajoutées pour
    la prise en charge des langues CJK, dont certaines ont été étendues à partir
    du programme japonais (u)pbibtex et d'autres programmes.
 \end{itemize*}

\item[Kpathsea :] Prise en charge de l'encodage des fichiers d'entrée pour les
  plates-formes Unix, comme sous Windows ; activé pour
  (\texttt{e})\texttt{p}(\texttt{la})\texttt{tex}, \texttt{pbibtex},
  \texttt{mendex}.

\item[\code{tlmgr} et infrastructure :]\leavevmode{}
 \begin{itemize*}
 \item l'interface texte par défaut  sur \macOS ;
 \item installer les paquets de base en premier, réessayer les autres paquets
   une fois ;
 \item vérification simpliste effectuée pour savoir si l'espace disque est
   suffisant.
 \end{itemize*}

\item[Mac\TeX{} :] \leavevmode{}
  \begin{itemize*}
 \item Mac\TeX{} et son dossier binaire \texttt{universal-darwin} nécessitent
   \macOS{} 10.14 ou supérieur (Mojave, Catalina, Big~Sur, Monterey, Ventura). Le
   dossier binaire \texttt{x86\_64-darwinlegacy}, disponible uniquement avec le
   programme Unix \texttt{install-tl}, prend en charge les versions 10.6
   (Snow~Leopard) et ultérieures.
 \item Le package \GUI{} de Mac\TeX{} contient désormais \texttt{hintview}, un
   visualisateur \macOS{} pour les documents HINT (créés par les programmes
   \texttt{hitex} et \texttt{hilatex} pour les appareils mobiles ; voir la page
   Web Hi\TeX{}, \url{https://hint.userweb.mwn.de/hint/hitex.html}). Le
   package \GUI{} n'installe plus de dossier de documents, il les
   remplace par un bref \texttt{READ~ME} pour les nouveaux utilisateurs et une
   page sur \texttt{hintview}.
 \item Le dossier \texttt{Extras} de logiciels \TeX{} supplémentaires sur le DVD
   a été remplacé par un document contenant des liens vers des sites de
   téléchargement.
 \end{itemize*}

\item[Plates-formes :] \leavevmode{}
  \begin{itemize*}
  \item Comme mentionné ci-dessus, le nouveau répertoire binaire
    \texttt{windows} contient les binaires Windows 64~bit, et
  \item le répertoire binaire \texttt{bin/win32} a disparu, puisque nous ne
    pouvons pas prendre en charge Windows 32~bit et 64~bit simultanément.
  \item Le répertoire binaire \texttt{i386-cygwin} a disparu, car Cygwin ne
    ne prend plus en charge i386.
 \end{itemize*}
\end{description}

\subsection{Versions futures}

% \TL{} n'est pas un produit parfait et ne le sera jamais.
Nous prévoyons de continuer à produire de nouvelles versions et aimerions
fournir plus de documentation, plus d'applications, une arborescence améliorée
et vérifiée de macros et de fontes\dots{} et tout ce qui concerne \TeX{}.  Ce
travail est effectué par des volontaires sur leur temps libre, et il
y a toujours plus à faire. Si vous pouvez nous aider, n'hésitez pas à nous
contacter (cf. \url{https://tug.org/texlive/contribute.html}).

Corrections, suggestions et propositions d'aide doivent être envoyées à :
\begin{quote}
  \email{tex-live@tug.org} \\
  \url{https://tug.org/texlive}
\end{quote}

\medskip
\noindent \textsl{Bon travail avec \TeX{} !}

\end{document}

%%% Local Variables:
%%% mode: latex
%%% mode: flyspell
%%% ispell-local-dictionary: "francais"
%%% TeX-master: t
%%% coding: utf-8
%%% End:

%  LocalWords:  Alpeh
