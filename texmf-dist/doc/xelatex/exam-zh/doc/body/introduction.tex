% !TeX root = ../exam-zh-doc.tex

\section{介绍}

试卷排版是中小学教师经常遇到的需求,目前在网上可以找到的试卷排版相关文类或宏包有:
\begin{itemize}
  \item Philip Hirschhorn:\href{https://www.ctan.org/pkg/exam}{exam}
  \item 吕荐瑞:\href{https://www.ctan.org/pkg/jnuexam}{jnuexam}
  \item 胡振震:\href{https://github.com/hushidong/simplexam}{simplexam}
  \item 鲍宏昌:\href{https://github.com/mathedu4all/bhcexam}{BHCexam}
  \item htharoldht:\href{https://github.com/htharoldht/USTBExam}{USTBExam}
  \item 唐绍东:\href{https://github.com/shaodongtang/gaokao_exam}{GEEexam}
  \item 唐绍东:\href{https://github.com/shaodongtang/CMC}{CMC}
  \item sd44:\href{https://github.com/sd44/DANexam}{DANexam}
\end{itemize}

但是大部分没有经过系统设计以及后续进一步的维护,\href{https://www.ctan.org/pkg/exam}{exam} 大部分设置与国内习惯不同,调试配置起来增加用户的使用成本 \href{https://www.ctan.org/pkg/jnuexam}{jnuexam}、\href{https://github.com/shaodongtang/CMC}{CMC} 是比较“定制化”的,也无法顺利地进行迁移使用。

但是上述前人所做的工作值得参考,比如 \cls{exam-zh} 的 A4 和 A3 页面切换就参考了 \href{https://www.ctan.org/pkg/jnuexam}{jnuexam} 项目。

本模板将借鉴前辈经验,重新设计,并使用 \LaTeX3 编写,以适应 \TeX 技术发展潮流; 同时还将构建一套简洁的接口,方便用户使用。