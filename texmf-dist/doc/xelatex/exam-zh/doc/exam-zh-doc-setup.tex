\usepackage{xeCJKfntef, xpinyin}
\usepackage{graphicx}
\usepackage{zhlipsum}
\usepackage{tabularray}
\usepackage{../exam-zh-choices}
\usepackage{../exam-zh-question}
\usepackage{../exam-zh-symbols}
\usepackage{../exam-zh-chinese-english}
\usepackage{../exam-zh-textfigure}
\usepackage{../exam-zh-math}

\ExplSyntaxOn
\NewDocumentCommand \examsetup { m }
  { \keys_set:nn { exam-zh } {#1} }
\ExplSyntaxOff
% \usepackage[
%   backend = biber,
%   style = gb7714-2015
% ]{biblatex}
% \addbibresource{exam-zh.bib}
\graphicspath{{figures}}

\hypersetup{
  pdftitle  = {exam-zh: 中国试卷 LaTeX 模板},
  pdfauthor = {夏康玮}
}
% 全角标点放在引号中,需要改成半角式,否则间距过大,不好看
\def\FSID{“{\xeCJKsetup{PunctStyle=banjiao}。}”} % U+3002
\def\FSFW{“{\xeCJKsetup{PunctStyle=banjiao}.}”} % U+FF0E
\def\COFW{“{\xeCJKsetup{PunctStyle=banjiao}:}”} % U+FF1A
\def\SCFW{“{\xeCJKsetup{PunctStyle=banjiao};}”} % U+FF1B


\title{\textcolor{MaterialIndigo800}{%
  \textbf{exam-zh: 中国试卷 \LaTeX \xpinyin[font=\sffamily,format=\color{MaterialIndigo800}]{模}{mu2}板}}}
\author{%
  夏康玮\thanks{%
    李泽平构建了 \cls{exam-zh} 的最初的基本框架;张庭瑄开发 \cls{exam-zh-font} 模块;郭李军开发了连线题环境}
}
\date{\DocDate\quad \DocVersion%
  \thanks{%
    \url{https://gitee.com/xkwxdyy/exam-zh} \\
    \hspace*{1.5em} QQ 用户交流群:652500180
  }
}

\ExplSyntaxOn
\NewDocumentCommand { \scoringbox } { s }
  {
    \IfBooleanTF {#1}
      { \__examzh_scoringbox_onecolumn: }
      { \__examzh_scoringbox_twocolumn: }
  }
\cs_new_protected:Nn \__examzh_scoringbox_twocolumn:
  {
    \begin{tabular}{|c|c|}
      \hline 
      得分 & \rule{3em}{0pt}\rule[-0.7em]{0pt}{2em} \\\hline
      阅卷人 & \rule{3em}{0pt}\rule[-0.7em]{0pt}{2em} \\\hline
    \end{tabular}
  }
\cs_new_protected:Nn \__examzh_scoringbox_onecolumn:
  {
    \begin{tabular}{|c|}
      \hline 
      得分\rule[-0.7em]{0pt}{2em} \\\hline
      \rule[-0.7em]{0pt}{2em} \\\hline
    \end{tabular}
  }
\ExplSyntaxOff

\AddToHook{env/latexexample/after}
  {%
    \examsetup{
      question/index=1
    }
  }