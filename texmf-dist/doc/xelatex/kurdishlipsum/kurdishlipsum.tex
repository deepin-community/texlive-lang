\documentclass[a4paper,12pt]{article}   
\usepackage[left=2cm,right=2cm]{geometry}   
\usepackage{fancyvrb}
\usepackage{hyperref,url}  
\usepackage{xepersian}
\settextfont{XB Niloofar}
\title{به‌سته‌ی
\textsf{kurdishlipsum}\RTLthanks{ڤێرژن 
$1,1$}
}\author{اسعد ابوزید جند%
\LTRthanks{\href{mailto:abozeid84@gmail.com}{\texttt{asaad.jund@soran.edu.iq}}}}
\date{\latintoday}
\begin{document}
 \renewcommand{\abstractname}{پوخته‌ی باسه‌كه‌}
\baselineskip=.7cm
\maketitle
\begin{abstract}
بەستە (Package) ـی 
\verb|\kurdishlipsum|،
  بە کەڵک ‌وەرگرتن لە بەستەی
  \verb!\ptext! 
 نووسراوە. 
کاری ئەم بەستەیە، چاککردنی بەند‌ (Paragraph)‌ ـه له‌ شیعر و غەزەل و قەسیدەی شاعیرانی بە وێنەی وه‌فایی، نالی، مەحوی و خانی و ...، هەروەها بۆ لێنووسینی لاپەڕەکان بەکار دێ.
\end{abstract}
\section{شێوازی به‌كارهێنان}
بەشێوەیەکی گشتی دەتوانین ئەم بەستەیە بە 
\verb|\usepackage{...}|
بنووسین.

 وەکوو بەستە (Package) ـەکانی تر، کە لەگەڵ
\verb|xepersian|
به‌كارده‌هێنرێن، وا چاکترە پێش ئەم package ـە بەکاری بێنین. 
وەک دەیبینن چوار کۆدی گرنگ لەم بەستەیەدا هەن،  گرنگترینیان كۆدی،
\verb!\kurdishlipsum!
 ـه. ئەم کۆدە، بەند (Paragraph) ـەکانی ۱ تا ۷ ـی ئەم به‌سته‌یه‌مان بۆ چاپ دەکا، کە پتر لە لاپەڕەیەکی $A4$ ،
\verb!a4paper!،
 دەبێ. بۆ گۆڕین و دەستکاریکردنی ژمارەی بەند (Paragraph) ـەکان، لە 
 \texttt{n} تا \texttt{m}  
  سوود لە کۆدی
\LRE{\verb!\setkurdishlipsumdefault{n-m}!}
 وەردەگرین و پێی چاپ دەکەین.
  دەتوانین بە جۆرێکی تریش کۆدی
\verb!\kurdishlipsum!
بەکار بێنین، بە نموونە:

\LRE{\verb|\kurdishlipsum[4-|\verb|57]|}،
به‌نده‌كانی‌ $4$ تاكو $57$  چاپ ده‌كات، یا كۆدی  
\verb!\kurdishlipsum[23]!،
پاراگرافی $23$ـی پێ چاپ ده‌كرێ.

ئه‌و به‌ند (Paragraph) ـانەی کە بە كۆدی
\verb!\kurdishlipsum!
چاك ده‌كرێن، به‌ كۆدی
\verb!\par!
له‌ یه‌ك جیا ده‌كرێنه‌وه‌. به‌ مانایه‌كی تر، هه‌ر به‌ند(Paragraph) ـێك به‌ كۆدی
\verb!\par!
كۆتایی دێ. ڕەنگە هەندێ جار ئەم کارە، کێشەی لابه‌لا درووست بکا، بۆیە ئەم بەستە (Package) ـە، بژاره‌ی
\textsf{nopar}
ـی هه‌یه‌. بۆ ئه‌م مه‌به‌سته‌، به‌ شێوه‌ی 
\begin{Verbatim}
\usepackage[nopar]{kurdishlipsum}
\end{Verbatim}
به‌كار ده‌هێنرێ.
سه‌ره‌ڕای ئه‌م كۆدانه‌ی كه‌ باسمان كردن، ‌  كۆدی 
\verb!\kurdishlipsum!،
پێناسه‌یه‌كی تریشی هه‌یه،‌ ‌ ئه‌گه‌ر بێتو ئه‌ستێره * ـی له‌گه‌ڵ دابنێین (واتا به‌م شێوه‌یه‌:
\verb!\kurdishlipsum*!)،
وەکوو ئەوە وایه کە کۆدی 
\textsf{nopar}
ـمان پێناسه‌ كردبێ. به‌ واتا‌یه‌كی تر، ئه‌گه‌ر كۆدی
\textsf{nopar}
به‌كار نه‌هێنرێ، ئه‌وا كۆدی
\verb!\kurdishlipsum*!،
به‌نده‌كان به‌ بێ كۆدی
\verb!\par!
چاپ ده‌كا.
كۆدی
\verb!\par!
به‌و مانایه‌ دێت كه‌ به‌نده‌كان جیا بن و بەسەریەکەوە نەنووسرێن.


\section{سوپاس و پێزانین‌}

سوپاس و پێزانین بۆ هەردوو بەڕێز: وەفا خەلیقی\LTRfootnote{\url{persian-tex@tug.org}}، کە هەریەک لە بەستەکانی 
\verb!\xepersian!
 و
\verb!\bidi!
   و وەحید دامەن ئەفشان\LTRfootnote{\url{www.damanafshan.ir}}، کە بەستەی
\verb!\ptext!
 ی درووست کردووە.
 
 سوپاسی به‌رێز: زیاد اسعد عەلی ده‌كه‌ین بۆ پێداچوونه‌وه‌ی ده‌قه‌ كوردییه‌كه‌.
\end{document}