% !TEX TS-program = XeLaTeX


\documentclass[12pt]{exam}
\usepackage{wexam}
%\setlength{\headsep}{20pt}
%------------------------
\newcommand{\lycee}{\sffamily ثانوية الدكتور أحمد عروة}
\newcommand{\annee}{2018-2017}
\newcommand{\examnum}{ امتحان الفصل الأول مادة الرياضيات}
\newcommand{\examdate}{\date}
\newcommand{ \duree }{ساعتان \yagding[ifsymclock]{151}}
\newcommand{ \niveau }{سنة ثالثة تسيير واقتصاد }
%------------------------------------------------------
\begin{document} 
%------------------------------------
{ \lycee}
\hfill
{\sffamily 
 السنة الدراسية: 
 \annee
}


$\rule{\textwidth}{1pt}$

\vspace{7pt}
%-------------------------------------
\centerline{\sffamily\large  \examnum}
%----------------------------------
$\rule{\textwidth}{1pt}$
%---------------------------------

{\sffamily
الشعبة:
 \niveau  
 %--------
\hfill
%------------
   المدة:
    \duree 
}
%------------------
\vspace{1cm}
%-------------------------------------------------
\begin{questions}
%\pointsinrightmargin     % جعل التنقيط على اليسار
\question [6]    %التمرين الأول
المتتالية العددية 
$(U_n)$
 معرفة كما يلي: 
 $U_0=6$
   و من أجل كل عدد طبيعي  
  $n$ 
   فإن :
   $U_{n+1}=\dfrac{1}{4}U_n +3$.
\begin{parts}   %  بداية الأسئلة الخاصة بالتمرين الأول
\part[2]         % عليه نقطتين السؤال 1
أحسب
$U_1$ , 
$ U_2$
و
$U_3$
 .
\part      %   غير منقط السؤال الثاني
\begin{subparts}
\subpart[1]   % `\textarabic{السؤال الفرعي الأول الخاص بالسؤال رقم 1}` 
أثبت بالتراجع أنه من أجل كل عدد طبيعي $n$ : 
$U_n \geq 4$ .

\subpart         % السؤال الفرعي غير منقط
بيّن أنّ المتتالية 
$(U_n)$
متناقصة . هل 
$(U_n)$
متقاربة؟
عيّن نهايتها.
\end{subparts}   % نهاية الأسئلة الفرعية الخاصة بالسؤال 2

\part      % السؤال الثالث
$(V_n)$
العددية المعرفة من أجل كل عدد طبيعي $n$ كما يلي :
$V_n=U_n-4$. 
\begin{subparts}
\subpart[1]   % `\textarabic{السؤال الفرعي الأول الخاص بالسؤال رقم 3}` 
بيّن أن المتتالية 
$(V_n)$
متتالية هندسية أساسها 
$q=\dfrac{1}{4}$
وحدها الأول  
$V_0$
، ثم أكتب عبارة حدها العام. 
\subpart
بيّن أنّه من أجل كل عدد طبيعي $n$ لدينا :
$U_n=2\left(\dfrac{1}{4}\right)^n +4$.
ثم أحسب
$\lim\limits_{n \rightarrow + \infty} U_n$
.
\subpart
أحسب بدلالة $n$ المجموع 
$S_n$
حيث :
$S_n=U_0 + U_1 + U_2 + \ldots + U_n$.
\end{subparts}   % 

\end{parts}
 %  نهاية التمرين الأول

\vspace{2cm}


\question [7]      % بداية التمرين الثاني عليه 7 نقاط مثلا
الجدول  التالي يعطي مسافة التوقف بالأمتار عند الضغط على المكبح لسيارة ما حسب السرعة المستعملة\\
 و المقدرة بـ :
$\mathtt{Km/h}$.
\begin{table}[h]
\centering
\begin{tabular}{|c|c|c|c|c|c|c|c|}
\hline 
$100$ & $90$ & $80$ & $70$ & $60$ & $50$ & $40$ & السرعة $x_i$ \\ 
\hline 
$85.4$ & $70.7$ & $57.5$ & $46$ & $35.7$ & $26.5$ & $18.6$ & المسافة $y_i$ \\ 
\hline 
\end{tabular} 
\end{table}
\begin{parts}     % بداية الأسئلة
\part[2]     %السؤال الأول  له علامتان
مثل سحابة النقط في معلم متعامد و متجانس 
$(O;\vec{i};\vec{j})$
،
الوحدة :
\quad
$
\begin{cases}
1 \mathtt{Cm} &\rightarrow 10 \, \mathtt{ Km/h}\\
1 \mathtt{Cm} &\rightarrow 10 \,\mathtt{ m}
\end{cases}
$
\part
\begin{subparts}
\subpart[1]
عين احداثيا النقطة المتوسطة 
$G$ 
،
 ثم مثلها في نفس المعلم . 
\subpart[2]
بيّن أنّ معامل توجيه مستقيم الانحدار بالمربعات الدنيا هو  
$a=1.11$
 ،
 انشئ هذا المستقيم.
\end{subparts}
\part     \hfill     %  \hfill اذا اردنا السؤال الفرعي يكون في سطر مستقل نستعمل 
\begin{subparts}
\subpart[1]
كم ستكون مسافة التوقف عند استعمال السرعة 
$160\, \mathtt{Km/h}$؟
\subpart[1]
 أوقفت المصالح المختصة أحد السائقين و بعد تسببه في حادث مرور و بعد حساب المسافة\\
  وجدوها 
 $230\,\mathtt{m}$
\\
$-$ 
\,
باستعمال التعديل السابق أوجد السرعة التي كان يسوق بها السائق         
\quad
                  ( تدور القيم إلى  $10^{-2}$ ).         
\end{subparts}
\end{parts}


\newpage   % التمرين الثالث يكون في صفحة جديدة مثلا

\question[7]     % التمرين الثالث 7 نقاط
لتكن الدالة 
$f$
المعرفة على    
$\left] 1,+\infty \right[ $
  بالعبارة: 
  $$f(x)=\dfrac{-x^2+4x-1}{x-1}$$
  و ليكن $(C_{f})$ تمثيلها البياني في مستوي منسوب الى معلم متعامد و متجانس
$(O;\vec{i};\vec{j})$ 
  .
 \begin{parts} 
\part
عين نهايتي الدالة  $f$ عند أطراف مجال تعريفها .
\part
عين الاعداد الحقيقية 
$b$ , $a$
و 
$c$
بحيث يكون من أجل كل 
$x$
من
$] 1,+\infty [ $
:
  $$f(x)=ax+b+\dfrac{c}{x-1}$$
   \part
   بين أنه من أجل كل $x$ 
   من
$\left] 1,+\infty \right[ $ :
$$f'(x)=\dfrac{-x^2+2x-3}{(x-1)^2}$$
\part   
  أعط جدول تغيرات الدالة 
  $f$.
   \part
    أثبت أنّ المستقيم $(D)$ ذو المعادلة 
   $y=-x+3$
   مستقيم مقارب لمنحني الدالة   
    $f$
بجوار 
$+\infty$    
    .
    \part
أدرس الوضع النسبي للمنحني $(C_{f})$  و المستقيم  $(D)$ 
.
       \part
  بين أن المعادلة :
  $f(x)=0$
   تقبل حلاً  وحيدًا
   $\alpha$ 
  في المجال 
  $\left[ 3.5;4 \right] $.
        \part
     أرسم المستقيم
     $ (D)$
   و  المنحني 
     $(C_{f})$ 
. 
  \end{parts}
\end{questions}     % نهاية التمارين
\end{document}