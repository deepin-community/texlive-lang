% Copyright for this work included in README.
%
% This work may be distributed and/or modified under the
% conditions of the LaTeX Project Public License, either version 1.3c
% of this license or (at your option) any later version.
% The latest version of this license is in
%   https://www.latex-project.org/lppl.txt
% and version 1.3 or later is part of all distributions of LaTeX
% version 2005/12/01 or later.
%
% This work has the LPPL maintenance status `maintained'.
%
% The Current Maintainer of this work is Will Robertson.
%
% This work consists of the files xetex-reference.tex,
% README.txt, and the derived file xetex-reference.pdf.

\documentclass[12pt]{article}

\title{The \xetex reference guide\\\url{https://ctan.org/pkg/xetexref}}
\author{Will Robertson \and Khaled Hosny \and Karl Berry}
\date{\today}

\suppressfontnotfounderror=1
\tracinglostchars=1

\makeatletter
\def\@dotsep{999}
\makeatother

\usepackage{refstyle} % must load before unicode-math with 2023 LaTeX.
% https://tex.stackexchange.com/questions/675145

\usepackage{fontspec,unicode-math}
\setmainfont{texgyrepagella}[
    Extension      = .otf ,
    UprightFont    = *-regular ,
    ItalicFont     = *-italic ,
    BoldFont       = *-bold ,
    BoldItalicFont = *-bolditalic ,
    Scale=MatchLowercase
  ]
\setsansfont{texgyreheros}[
    Extension      = .otf ,
    UprightFont    = *-regular ,
    ItalicFont     = *-italic ,
    BoldFont       = *-bold ,
    BoldItalicFont = *-bolditalic ,
    Scale=MatchLowercase
  ]
\iffalse \setmonofont{inconsolata}[
  Scale=MatchLowercase,
  StylisticSet=3, % <- straight quotes
  AutoFakeSlant=.2]\fi
\setmathfont{texgyrepagella-math.otf}

\usepackage{calc,fancyvrb,hyperref,varioref,xcolor,hologo,xspace}
\usepackage{geometry}
%\geometry{screen,margin=3cm}

\newcommand\tex    {\hologo{TeX}\xspace}
\newcommand\xetex  {\hologo{XeTeX}\xspace}
\newcommand\pdftex {\hologo{pdfTeX}\xspace}
\newcommand\luatex {\hologo{LuaTeX}\xspace}
\newcommand\latex  {\hologo{LaTeX}\xspace}
\newcommand\context{\hologo{ConTeXt}\xspace}

\hypersetup{
  colorlinks,
  linkcolor=black,
  urlcolor=black,
  pdfsubject={The XeTeX reference guide},
  pdfauthor={Will Robertson \& Khaled Hosny \& Karl Berry},
  pdfkeywords={xetex, tex, typesetting, unicode, math, opentype, graphite, aat}
}

\usepackage[it]{titlesec}
\usepackage{enumitem}
\setlist{nolistsep}

\newenvironment{optdesc}
  {\begin{description}[font=\ttfamily,style=nextline,leftmargin=1.5cm]}
  {\end{description}}

\newcommand\cmd{%
  \noindent
  \begin{trivlist}\item[]
  \SaveVerb[%
    aftersave={%
      \begin{minipage}{\linewidth}
        \parindent=2em\relax\noindent
        \UseVerb{CMD}}]{CMD}}
\edef\|{|}
\DefineShortVerb{\|}
\newcommand\xarg[1]{$\langle\hbox{\rmfamily\itshape #1}\rangle$}
\def\<#1>{\xarg{#1}}
\newcommand\oarg[1]{\texttt{[\,#1\,]}}
\newcommand\barg[1]{\mbox{{\tt\char`\{\,}#1\,{\tt\char`\}}}}% braced arg
\newcommand\desc[1]{\par\noindent\ignorespaces#1\par}
\def\endcmd{%
  \end{minipage}
  \end{trivlist}}

\def\cs#1{\texttt{\textbackslash#1}}

\newsavebox\verbatimbox
\edef\examplefilename{\jobname.example}
\newlength\exampleindent
\setlength\exampleindent{1em}

\newenvironment{examplenooutput}
  {\wlog{^^Jstarting examplenooutput at line \the\inputlineno}
   \VerbatimEnvironment
   \begin{VerbatimOut}{\examplefilename}}
  {\end{VerbatimOut}
   \def\typesetoutput{[No output for the example, since no AAT or
                       Graphite fonts are available.}
   \typesetexample}

\newenvironment{example}
  {\wlog{^^Jstarting example at line \the\inputlineno}
   \VerbatimEnvironment
   \begin{VerbatimOut}{\examplefilename}}
  {\end{VerbatimOut}
   \def\typesetoutput{\color[rgb]{0.7,0,0}\input\examplefilename\relax}%
   \typesetexample}


\newcommand\typesetexample{%
  \begin{trivlist}\item[]
  \vrule
  \hspace{\exampleindent}
  \begin{minipage}{\linewidth-\exampleindent-\exampleindent}
    \hfuzz=24pt % italic small caps example has overfull line
    \textit{Example:}\par
    \vspace{0.4\baselineskip}
    \BVerbatimInput[fontsize=\small]{\examplefilename}\par
    \vspace{0.4\baselineskip}
    \typesetoutput
  \end{minipage}\par
  \end{trivlist}}


\let\strong\textbf
\newcommand\hlink[2]{\href{#1}{#2}\footnote{\url{#1}}}

\let\latin\relax % let existing font continue
\def\eg{\latin{e.g.}}
\def\ie{\latin{i.e.}}
\def\Eg{\latin{E.g.}}
\def\Ie{\latin{I.e.}}
\def\etc{\@ifnextchar.{\latin{etc}}{\latin{etc.}\@}}

\def\opteq{\unskip\,\textcolor{gray}{[\textcolor{black}{=}]}\,}

\setlength\parskip{0pt}
\setlength\parindent{2em}
\raggedbottom

\begin{document}
\maketitle

\vfill

\section*{Introduction}

This document serves as a reference for additional features of \xetex.
It is not a users' guide. Much of the functionality addressed here is
provided in abstracted form in various \latex packages and \context
modules.

The descriptions here are intended to be a reasonably complete, though
terse, list of the new primitives and features of \xetex.

For contributions (very welcome!), bug reports, etc., see links from
the CTAN package page: \url{https://ctan.org/pkg/xetexref}.

\section*{License}

This work, is distributed under the terms of the LaTeX Project Public License
(\url{https://www.latex-project.org/lppl.txt}).

This basically means you are free to re-distribute this file as you
wish; you may also make changes to this file or use its contents for
another purpose, in which case you should make it clear, by way of a
name-change or some other means, that your changed version is a modified
version of the original. Please read the license text for more detailed
information.

\vfill\vfill\vfill\null

\newpage
\tableofcontents

\section{The \cs{font} command}

Traditionally, fonts were selected in \tex like this:
|\font\1=|\xarg{tfm name} \xarg{\tex font options}, where the \xarg{\tex
font options} included possibilities such as
\ ‘| at 10pt|’ \
or \ ‘| scaled 1200|’,
with evident meaning. This syntax still works, but it has been greatly
extended in \xetex.

The extended syntax looks schematically like this:

\begin{quote}
 |\font\1="|\xarg{font identifier}
              \xarg{font options}%
              |:|\xarg{font features}%
         |"|\ \xarg{\tex font options}
\end{quote}

\noindent Each part of this command is discussed in the following, but
here is an overview, starting from the end:

\begin{itemize}
\item The \xarg{\tex font options} are unchanged from traditional \tex.

\item The ASCII double quote character |"| (hex 0x22) surrounds any font
definition using the extended syntax.

\item The \xarg{font features} are an optional list of names,
comma or semi-colon separated, preceded by an ASCII colon |:| (0x3a).

\item The \xarg{font options} are an optional list of specifiers which
can only be used with system fonts (see next).

\item The \xarg{font identifier} is the only mandatory part of the
above syntax.  If it is given in square brackets, (\eg,
|\font\1="[lmroman10-regular]"|), it is taken as a font file name.
Without brackets, the name is looked up as both a file name and a system
font name.

\end{itemize}

This distinction between file name lookups and system font name lookups
is crucial to understanding \xetex's behavior and to writing portable
documents (in short: don't use system fonts).

System font name lookups use (except on Mac~OS~X) the
|fontconfig| library; running |fc-list| should show you the font names
available.  \Eg,
\begin{quote}\small
|\font\1="Liberation Serif"| \hfill \emph{look for OS-installed font}
\end{quote}

Fonts have many internal names, and XeTeX does the system font lookups
in the following order:
\vskip1ex
\begin{itemize}
  \item Full Name;
  \item if the name has a hyphen, it is split into Family-Style pair then matched;
  \item PostScript Name;
  \item Family Name, if there is more than one match;
  \begin{itemize}
    \item look for font with “regular” bit set in OS/2 table, if no match;
    \item look for font with style “Regular”, “Plain”, “Normal” or “Roman”, in that order.
  \end{itemize}
\end{itemize}
\bigskip

When using a file name, the |xdvipdfmx| driver must be used (this is the
default). The current directory and the |texmf| trees are searched for
files matching the name, or the path may be embedded in the font
declaration, as usual with |kpathsea|. \Eg,
\begin{quote}\small
|\font\1="[lmroman10-regular]"| \hfill
  {\em find |lmroman10-regular.otf| in any tree}
|\font\2="[/myfonts/fp9r8a]"| \hfill
  {\em look for |fp9r8a| only in |/myfonts/|}
\end{quote}

A file with either an |.otf|, |.ttf| or |.pfb| extension (in that order) will be found.  The
extension can also be specified explicitly.

\subsection{Font collection files}

One more special case about the \xarg{font identifier}: if the file is a
font collection (\eg, |.ttc| or |.dfont|), the index of the font can be
specified using a colon followed by zero-based font index inside the
square brackets (in contrast to the font features we'll describe soon,
which follow a colon outside any square brackets). \Eg,
\begin{quote}\small
|\font\2="[myfont.ttc:1]"| \hfill {\em load the second font from |myfont.ttc| file}
\end{quote}


\subsection{Font options: For OS-selected fonts only}

The following \xarg{font options} are only applicable when the font is selected
through the operating system (\ie, without square brackets):
\begin{optdesc}
\item[/B] Use the bold version of the selected font.
\item[/I] Use the italic version of the selected font.
\item[/BI] Use the bold italic version of the selected font.
\item[/IB] Same as \texttt{/BI}.
\item[/S=$x$] Use the version of the selected font corresponding to the
optical size $x$\,pt.
\end{optdesc}

The following \xarg{font options} control which `renderer' \xetex uses to interface with the font:
\begin{optdesc}
\item[/AAT] Explicitly use the AAT renderer (Mac~OS~X only).
\item[/OT] Explicitly use the OpenType renderer (new in 0.9999).
\item[/GR] Explicitly use the Graphite renderer.%
           \footnote{\raggedright \url{https://scripts.sil.org/cms/scripts/page.php?site_id=projects&item_id=graphite_home}}
\item[/ICU] Explicitly use the OpenType renderer (deprecated since 0.9999).
\end{optdesc}


\subsection{Font features: For all fonts}

The \xarg{font features} is a comma or semi-colon separated list
activating or deactivating various OpenType, Graphite, or AAT font
features; these vary by font.  In contrast to font options,
features work whether the font is selected by file name or through the
operating system.

The \xetex utility files \path{aat-info.tex} and
\path{opentype-info.tex} provide lists of supported features for a given
font (see comments at the top of each).

\subsubsection{Font features specific to OpenType, Graphite, or AAT}

OpenType font features are usually specified with
\hlink{https://www.microsoft.com/typography/otspec/featuretags.htm}{standard
tags} in the \cs{font} command. They may be either comma- or
semicolon-separated, and prefixed with a |+| to turn them on and a |-|
to turn them off, optionally followed by |=| and a 0-based index for
selecting alternates from multiple alternates features (ignored for |-|
prefixed tags).

\begin{example}
\font\liber="[LinLibertine_RI.otf]/I=5:+smcp" at 12pt
\liber This is the OpenType font Linux Libertine in italic with small caps.
\end{example}

OpenType features can be activated by default:

\begin{example}
\font\antt="[AntykwaTorunska-Regular.otf]"         at 12pt \antt 0
\font\antt="[AntykwaTorunska-Regular.otf]:+aalt=0" at 12pt \antt 0
\font\antt="[AntykwaTorunska-Regular.otf]:+aalt=1" at 12pt \antt 0
\font\antt="[AntykwaTorunska-Regular.otf]:+aalt=2" at 12pt \antt 0
\font\antt="[AntykwaTorunska-Regular.otf]:+aalt=3" at 12pt \antt 0
\font\antt="[AntykwaTorunska-Regular.otf]:+aalt=4" at 12pt \antt 0
\end{example}

AAT font features and Graphite font features are specified by strings
within each font rather than standardised tags. Therefore, even
equivalent features can and do have different names in different fonts.

\begin{examplenooutput}
\font\gra="[CharisSIL-Regular.ttf]/GR:Small Caps=True" at 12pt
\gra This is the Graphite font Charis SIL with small caps.
\end{examplenooutput}

\subsubsection{Features for all fonts}

Some font features may be applied for any font. These are
\begin{optdesc}

\item[color={\slshape RRGGBB}{[{\slshape TT}]}]
Triple pair of hex values to specify the colour in RGB space, with an
optional value for the transparency.

\begin{example}
\font\9="[lmsans10-regular.otf]:color=0000FF,mapping=tex-text"
\9 A sans blue quoted em-dash: ``---''.
\end{example}

\item[embolden=$x$]
Increase the envelope of each glyph by the set amount (this makes the
letters look ‘more bold’). $x=0$ corresponds to no change; $x=1.5$ is a
good default value.

\item[extend=$x$]
Stretch each glyph horizontally by a factor of $x$ (\ie, $x=1$
corresponds to no change).

\item[letterspace=$x$]
Adds $x/S$ space between letters in words, where $S$ is the font size.

\item[mapping=\textsl{<font map>}]
Uses the specified font mapping for this font. This uses the TECKit
engine to transform Unicode characters in the last-minute processing
stage of the source. For example, |mapping=tex-text| will enable the
classical mappings from ASCII |``---''| to proper typographical
glyphs “—”, and so on.

\item[slant=$x$]
Slant each glyph by the set amount. $x=0$ corresponds to no change;
$x=0.2$ is a good default value. The slant is given by $x=R/S$ where $R$
is the displacement of the top edge of each glyph and $S$ is the point
size.

\end{optdesc}

\subsubsection{OpenType script and language support}\seclabel{script}

OpenType font features (and font behavior) can vary by
\hlink{https://www.microsoft.com/typography/otspec/scripttags.htm}{script}
(‘alphabet’) and by
\hlink{https://www.microsoft.com/typography/otspec/languagetags.htm}{language}.
These are selected with four and three letter tags, respectively.

\begin{optdesc}
\item[script=\textsl{<script tag>}] Selects the font script.
\item[language=\textsl{<lang tag>}] Selects the font language.
\end{optdesc}

\subsubsection{Multiple Master and Variable Axes AAT font support}

\begin{optdesc}
\item[weight=$x$] Selects the normalised font weight, $x$.
\item[width=$x$] Selects the normalised font width, $x$.
\item[optical size=$x$] Selects the optical size, $x$\,pt. Note the
difference between the \texttt{/S} font option, which selects discrete
fonts.
\end{optdesc}

\subsubsection{Vertical typesetting}
\begin{optdesc}
\item[vertical]
Enables glyph rotation in the output so vertical typesetting can be performed.
\end{optdesc}

\section{Font primitives}

\cmd|\XeTeXtracingfonts|
\desc{If nonzero, reports where fonts are found in the log file.}
\endcmd

\cmd|\XeTeXfonttype|
\xarg{font}
\desc{
  Expands to a number corresponding to which renderer is used for a
  \xarg{font}:
  \begin{optdesc}
    \item [0] for \tex (a legacy TFM-based font);
    \item [1] for AAT;
    \item [2] for OpenType;
    \item [3] for Graphite.
  \end{optdesc}
}
\endcmd

\begin{example}
\newcommand\whattype[1]{%
  \texttt{\fontname#1} is rendered by
  \ifcase\XeTeXfonttype#1\TeX\or AAT\or OpenType\or Graphite\fi.\par}
\font\1="cmr10"
\font\2="[CharisSIL-Regular.ttf]"
\font\3="[CharisSIL-Regular.ttf]/OT"
\whattype\1 \whattype\2 \whattype\3
\end{example}

\cmd|\XeTeXfirstfontchar|
\xarg{font}
\desc{Expands to the code of the first character in \xarg{font}.}
\endcmd

\cmd|\XeTeXlastfontchar|
\xarg{font}
\desc{Expands to the code of the last character in \xarg{font}.}
\endcmd

\begin{example}
\font\1="[CharisSIL-Regular.ttf]"\1
The first character in Charis SIL is: "\char\XeTeXfirstfontchar\1"
and the last character is: "\char\XeTeXlastfontchar\1".
\end{example}

\cmd|\XeTeXglyph|
\xarg{glyph slot}
\desc{Inserts the glyph in \xarg{glyph slot} of the current font. \strong{Font
specific}, so will give different output for different fonts and
possibly even different versions of the same font.}
\endcmd

\cmd|\XeTeXcountglyphs|
\xarg{font}
\desc{The count of the number of glyphs in the specified \xarg{font}.}
\endcmd

\cmd|\XeTeXglyphname|
\xarg{font}
\xarg{glyph slot}
\desc{Expands to the name of the glyph in \xarg{glyph slot} of \xarg{font}.
\strong{Font specific}, so will give different output for different fonts and
possibly even different versions of the same font.}
\endcmd

\cmd|\XeTeXglyphindex|
|"|\xarg{glyph name}|"| \xarg{space} \emph{or} \cs{relax}
\desc{Expands to the glyph slot corresponding to the (possibly
font specific) \xarg{glyph name} in the currently selected font. Only
works for TrueType fonts (or TrueType-based OpenType fonts) at
present. Use \texttt{fontforge} or similar to discover glyph names.}
\endcmd

\cmd|\XeTeXcharglyph|
\xarg{char code}
\desc{Expands to the default glyph number of character \xarg{char code}
in the current font, or 0 if the character is not available in the
font.}
\endcmd

\begin{example}
\font\1="[CharisSIL-Regular.ttf]"\1
The glyph slot in Charis SIL for the Yen symbol is:
    \the\XeTeXglyphindex"yen" . % font-specific glyph name
Or: \the\XeTeXcharglyph"00A5.   % unicode character slot

This glyph may be typeset with the font-specific glyph slot:
\XeTeXglyph150,
or the Unicode character slot:
\char"00A5.
\end{example}

\cmd|\XeTeXglyphbounds|
\xarg{edge} \xarg{glyph slot}
\desc{Expands to a dimension corresponding to one of the bounds of a
glyph, where \xarg{edge} is an integer from 1 to~4 indicating the
left/top/right/bottom edge respectively, and \xarg{glyph slot} is an
integer glyph index in the current font (only valid for non TFM-based
fonts).

The left and right measurements are the glyph sidebearings, measured
‘inwards’ from the origin and advance respectively, so for a glyph that
fits completely within its ‘cell’ they will both be positive; for a
glyph that ‘overhangs’ to the left or right, they will be negative. The
actual width of the glyph’s bounding box, therefore, is the character
width (advance) minus both these sidebearings.

The top and bottom measurements are measured from the baseline, like
\tex’s height and depth; the height of the bounding box is the sum of
these two dimensions.}
\endcmd

\begin{example}
\def\shadebbox#1{%
\leavevmode\rlap{%
  \dimen0=\fontcharwd\font`#1%
  \edef\gid{\the\XeTeXcharglyph`#1}%
  \advance\dimen0 by -\XeTeXglyphbounds1 \gid
  \advance\dimen0 by -\XeTeXglyphbounds3 \gid
  \kern\XeTeXglyphbounds1 \gid
  \special{color push rgb 1 1 0.66667}%
  \vrule width \dimen0
         height \XeTeXglyphbounds2 \gid
         depth \XeTeXglyphbounds4 \gid
  \special{color pop}%
  \kern\XeTeXglyphbounds3 \gid}%
  #1}

\noindent
\font\x="[CharisSIL-Italic.ttf]" at 24pt \x
\shadebbox{A} \shadebbox{W} \shadebbox{a} \shadebbox{f}
\shadebbox{;} \shadebbox{*} \shadebbox{=}
\end{example}

\cmd|\XeTeXuseglyphmetrics|
\desc{Counter to specify if the height and depth of characters are taken
into account while typesetting ($\ge\mathtt1$). Otherwise ($<\mathtt1$),
a single height and depth for the entire alphabet is used. Gives better
output but is slower. Activated ($\ge\mathtt1$) by default.}
\endcmd

\begin{example}
\XeTeXuseglyphmetrics=0 \fbox{a}\fbox{A}\fbox{j}\fbox{J} vs.
\XeTeXuseglyphmetrics=1 \fbox{a}\fbox{A}\fbox{j}\fbox{J}
\end{example}

\cmd|\XeTeXgenerateactualtext|
\xarg{integer}
\desc{Controls the output of \texttt{/ActualText} entry. Default is 0.
When set to 1, the \texttt{/ActualText} entry is added to the output PDF
for better copy/paste and search in PDF viewers.}
\endcmd

\subsection{OpenType fonts}

\cmd|\XeTeXOTcountscripts|
\xarg{font}
\desc{Expands to the number of scripts in the \xarg{font}.}
\endcmd

\cmd|\XeTeXOTscripttag|
\xarg{font}
\xarg{integer, $n$}
\desc{Expands to the $n$-th script tag of \xarg{font}.}
\endcmd

\cmd|\XeTeXOTcountlanguages|
\xarg{font}
\xarg{script tag}
\desc{Expands to the number of languages in the script of \xarg{font}.}
\endcmd

\cmd|\XeTeXOTlanguagetag|
\xarg{font}
\xarg{script tag}
\xarg{integer, $n$}
\desc{Expands to the $n$-th language tag in the script of \xarg{font}.}
\endcmd

\cmd|\XeTeXOTcountfeatures|
\xarg{font}
\xarg{script tag}
\xarg{language tag}
\desc{Expands to the number of features in the language of a script of \xarg{font}.}
\endcmd

\cmd|\XeTeXOTfeaturetag|
\xarg{font}
\xarg{script tag}
\xarg{language tag}
\xarg{integer, $n$}
\desc{Expands to the $n$-th feature tag in the language of a script of \xarg{font}.}
\endcmd


\subsection{AAT and Graphite fonts}

These commands apply to AAT and Graphite fonts; for other kinds of
fonts, they produce an error.

\subsubsection{Features}

\cmd|\XeTeXcountfeatures|
\xarg{font}
\desc{Expands to the number of features in the \xarg{font}.}
\endcmd

\cmd|\XeTeXfeaturecode|
\xarg{font}
\xarg{integer, $n$}
\desc{Expands to the feature code for the $n$-th feature in the \xarg{font}.}
\endcmd

\cmd|\XeTeXfeaturename|
\xarg{font}
\xarg{feature code}
\desc{Expands to the name corresponding to the \xarg{feature code} in
the \xarg{font}.}
\endcmd

\cmd|\XeTeXisexclusivefeature|
\xarg{font}
\xarg{feature code}
\desc{Expands to a number greater than zero if the feature of a font is
exclusive (can only take a single selector).}
\endcmd

\cmd|\XeTeXfindfeaturebyname|
\xarg{font}
\xarg{feature name}
\desc{This command provides a method to query whether a feature name
corresponds to a feature contained in the font. It returns an integer
corresponding to the feature number used to access the feature
numerically. If the feature does not exist, the integer is
\texttt{-1}. Also see \cs{XeTeXfindselectorbyname}.}
\endcmd

\begin{examplenooutput}
\font\1="[CharisSIL-Regular.ttf]/GR" at 10pt
\def\featname{Uppercase Eng alternates}
The feature ‘\featname’ has index
\the\XeTeXfindfeaturebyname\1 "\featname"\relax
\end{examplenooutput}

\subsubsection{Feature selectors}

\cmd|\XeTeXcountselectors|
\xarg{font}
\xarg{feature}
\desc{Expands to the number of selectors in a \xarg{feature} of a \xarg{font}.}
\endcmd

\cmd|\XeTeXselectorcode|
\xarg{font}
\xarg{feature code}
\xarg{integer, $n$}
\desc{Expands to the selector code for the $n$-th selector in a
\xarg{feature} of a \xarg{font}.}
\endcmd

\cmd|\XeTeXselectorname|
\xarg{font}
\xarg{feature code}
\xarg{selector code}

\desc{Expands to the name corresponding to the \xarg{selector code} of a
feature of a \xarg{font}.}
\endcmd

\cmd|\XeTeXisdefaultselector|
\xarg{font}
\xarg{feature code}
\xarg{selector code}
\desc{Expands to a number greater than zero if the selector of a feature
of a font is on by default.}
\endcmd

\cmd|\XeTeXfindselectorbyname|
\xarg{font}
\xarg{feature code}
\xarg{selector name}
\desc{This command provides a method to query whether a feature selector
name corresponds to a selector of a specific feature contained in the
font. It returns an integer corresponding to the selector number used
to access the feature selector numerically. If the feature selector does
not exist, the integer is \texttt{-1}.

The indices given by this command and by \cs{XeTeXfindfeaturebyname} can
be used in Graphite fonts to select font features directly (see example
below). Alternatively, they can be used as a means of checking whether a
feature/selector exists before attempting to use it.}
\endcmd

\begin{examplenooutput}
\font\1="[CharisSIL-Regular.ttf]/GR" at 10pt
\def\featname{Uppercase Eng alternates}
\newcount\featcount
\featcount=\XeTeXfindfeaturebyname\1 "\featname"\relax

\def\selecname{Large eng on baseline}
\newcount\seleccount
\seleccount=\XeTeXfindselectorbyname\1 \featcount "\selecname"\relax
The feature selector ‘\selecname’ has index \the\seleccount

\font\2="[CharisSIL-Regular.ttf]/GR:\featname=\selecname" at 10pt
\font\3="[CharisSIL-Regular.ttf]/GR:\the\featcount=\the\seleccount" at 10pt

Activating the feature: \1 Ŋ \2 Ŋ \3 Ŋ
\end{examplenooutput}

\subsubsection{Variation axes}

\cmd|\XeTeXcountvariations|
\xarg{font}
\desc{Expands to the number of variation axes in the \xarg{font}.}
\endcmd

\cmd|\XeTeXvariation|
\xarg{font}
\xarg{integer, $n$}
\desc{Expands to the variation code for the $n$-th feature in the \xarg{font}.}
\endcmd

\cmd|\XeTeXvariationname|
\xarg{font}
\xarg{variation code}
\desc{Expands to the name corresponding to the \xarg{feature code} in
the \xarg{font}.}
\endcmd

\cmd|\XeTeXvariationmin|
\xarg{font}
\xarg{variation code}
\desc{Expands to the minimum value of the variation corresponding to the
\xarg{variation code} in the \xarg{font}.}
\endcmd

\cmd|\XeTeXvariationmax|
\xarg{font}
\xarg{variation code}
\desc{Expands to the maximum value of the variation corresponding to the
\xarg{variation code} in the \xarg{font}.}
\endcmd

\cmd|\XeTeXvariationdefault|
\xarg{font}
\xarg{variation code}
\desc{Expands to the default value of the variation corresponding to the
\xarg{variation code} in the \xarg{font}.}
\endcmd

\cmd|\XeTeXfindvariationbyname|
\xarg{font}
\xarg{variation name}
\desc{An integer corresponding to the internal index corresponding to
the \xarg{variation name}. This index cannot be used directly but may be
used to error-check that a specified variation name exists before
attempting to use it.}
\endcmd

\subsection{Contextual adjustment space handling}
\cmd|\XeTeXinterwordspaceshaping|
\xarg{integer}
\desc{Controls the shaping of space characters in context when using
native fonts. Defaults to 0 (disabled). Set to 1 for contextual
adjustment of space width only, and 2 for full cross-space shaping
(e.g., multi-word ligatures).

When set to 2, each run of text (words and intervening spaces) in a
single font will be re-shaped as a unit as a page is shipped out, so
setting this register to 2 affects the whole page, not respecting \TeX\
grouping.}
\endcmd

\subsection{Maths fonts}

The primitives described following are extensions of \tex’s 8-bit primitives.

In the following commands, \xarg{fam.} is a number (0–255) representing
font to use in maths. \xarg{math type} is the 0–7 number corresponding
to the type of math symbol; see a \tex reference for details.

Before version 0.9999.0 the following primitives had |\XeTeX| prefix instead of
|\U|, the old names are deprecated and will be removed in the future.

\cmd|\Umathcode|
\xarg{char slot}
\opteq
\xarg{math type}
\xarg{fam.}
\xarg{glyph slot}
\desc{Defines a maths glyph accessible via an input character. Note that
the input takes \emph{three} arguments unlike \tex’s \cs{mathcode}.}
\endcmd

\cmd|\Umathcodenum|
\xarg{char slot}
\opteq
\xarg{math type/fam./glyph slot}
\desc{Pure extension of \cs{mathcode} that uses a ‘bit-packed’ single
number argument. Can also be used to extract the bit-packed mathcode
number of the \xarg{char slot} if no assignment is given.}
\endcmd

\cmd|\Umathchar|
\xarg{math type}
\xarg{fam.}
\xarg{glyph slot}
\desc{Typesets the math character in the \xarg{glyph slot} in the family
specified.}
\endcmd

\cmd|\Umathcharnum|
\xarg{type/fam./glyph slot}
\desc{Pure extension of \cs{mathchar} that uses a ‘bit-packed’ single
number argument. Can also be used to extract the bit-packed mathcode
number of the \xarg{char slot} if no assignment is given.}
\endcmd

\cmd|\Umathchardef|
\xarg{control sequence}
\opteq
\xarg{math type}
\xarg{fam.}
\xarg{glyph slot}
\desc{Defines a maths glyph accessible via a control sequence.}
\endcmd

\cmd|\Umathcharnumdef|
\xarg{control sequence}
\opteq
\xarg{type/fam./glyph slot}
\desc{Defines a control sequence for accessing a maths glyph using the
‘bit-packed’ number output by, \eg, \cs{Umathcodenum}. This would
be used to replace legacy code such as
\cs{mathchardef}\cs{foo}\texttt{=}\cs{mathcode}\texttt{`}\cs{\-}.}
\endcmd

\cmd|\Udelcode|
\xarg{char slot}
\opteq
\xarg{fam.}
\xarg{glyph slot}
\desc{Defines a delimiter glyph accessible via an input character.}
\endcmd

\cmd|\Udelcodenum|
\xarg{char slot}
\opteq
\xarg{fam./glyph slot}
\desc{Pure extension of \cs{delcode} that uses a ‘bit-packed’ single
number argument. Can also be used to extract the bit-packed delcode
number of the \xarg{char slot} if no assignment is given.}
\endcmd

\cmd|\Udelimiter|
\xarg{math type}
\xarg{fam.}
\xarg{glyph slot}
\desc{Typesets the delimiter in the \xarg{glyph slot} in the family
specified of either \xarg{math type} 4 (opening) or 5 (closing).}
\endcmd

\cmd|\Umathaccent|
\oarg{keyword}
\xarg{math type}
\xarg{fam.}
\xarg{glyph slot}
\desc{Typesets the math accent character in the \xarg{glyph slot} in the
family specified. Starting from version 0.9998, \cs{Umathaccent} accepts
optional keyword:\medskip
\begin{optdesc}
\item[fixed] Don’t stretch the accent, the default is to stretch it:
             $\widehat{M}$ vs $\hat{M}$.
\item[bottom] Place the accent below its base. Can be followed by the
             \texttt{fixed} keyword.
\end{optdesc}}
\endcmd

\cmd|\Uradical|
\xarg{fam.}
\xarg{glyph slot}
\desc{Typesets the radical in the \xarg{glyph slot} in the family specified.}
\endcmd

\section{Characters}

\cmd|\char|
\xarg{number}
\desc{This and related \TeX\ primitives have been extended to take any
16-bit number, that is, up to 65536. As usual, hex digits must be
specific in uppercase: \cs{char"FF}, not \cs{char"ff}.}
\endcmd

\cmd|\Uchar|
\xarg{number}
\desc{Expands to a character token with specified slot \xarg{number}
(range 0 to 1,114,111) with category code 12. While it looks
superficially like the \TeX\ primitive \cs{char}, \cs{Uchar} is an
expandable operation.}
\endcmd

\begin{example}
\edef\x{\Uchar`\$}
\ttfamily\meaning\x\par
\expandafter\ifcat\x.[`other']\fi
\end{example}

\cmd|\Ucharcat|
\xarg{number}
\xarg{catcode}
\desc{Expands to a character token with slot \xarg{number} and
\xarg{catcode} specified. The values allowed for \xarg{catcode} are:
1--4, 6--8 and 10--13.}
\endcmd

\begin{example}
\edef\x{\Ucharcat`\# 3}
\ttfamily\meaning\x\par
\expandafter\ifcat\x$[`mathshift']\fi
\end{example}

\subsection{Character classes}

The idea behind character classes is to define a boundary where tokens
can be added to the input stream without explicit markup. It was
originally intended to add glue around punctuation to effect correct
Japanese typesetting. This feature can also be used to adjust space
around punctuation for European traditions. The general nature of this
feature, however, lends it to several other useful applications
including automatic font switching when small amounts of another
language (in another script) is present in the text.

\cmd|\XeTeXinterchartokenstate|
\desc{Counter. If positive, enables the character classes functionality.}
\endcmd

\cmd|\newXeTeXintercharclass|
\xarg{control sequence}
\desc{Allocates a new interchar class and assigns it to the
\xarg{control sequence} argument.}
\endcmd

\cmd|\XeTeXcharclass|
\xarg{char slot}
\opteq
\xarg{interchar class}
\desc{Assigns a class corresponding to \xarg{interchar class} (range
0–4095) to a \xarg{char slot}. Most characters are class 0 by
default. Class 1 is for CJK ideographs, classes 2 and 3 are CJK
punctuation. The boundary of a text string is considered class 4095,
wherever there is a boundary between a ‘run’ of characters and something
else — glue, kern, math, box, etc. Special case class 4096 is ignored;
useful for diacritics so I’m told.}
\endcmd

\cmd|\XeTeXinterchartoks|
\xarg{interchar class 1}
\xarg{interchar class 2}
\opteq
|{|\xarg{token list}|}|
\desc{Defines tokens to be inserted between
\xarg{interchar class 1} and \xarg{interchar class 2} (in that order).}
\endcmd

\begin{example}
\XeTeXinterchartokenstate = 1
\newXeTeXintercharclass \mycharclassa
\newXeTeXintercharclass \mycharclassA
\newXeTeXintercharclass \mycharclassB
\XeTeXcharclass `\a \mycharclassa
\XeTeXcharclass `\A \mycharclassA
\XeTeXcharclass `\B \mycharclassB

% between "a" and "A":
\XeTeXinterchartoks \mycharclassa \mycharclassA = {[\itshape}
\XeTeXinterchartoks \mycharclassA \mycharclassa = {\upshape]}

% between " " and "B":
\XeTeXinterchartoks 4095 \mycharclassB = {\bgroup\color{blue}}
\XeTeXinterchartoks \mycharclassB 4095 = {\egroup}

% between "B" and "B":
\XeTeXinterchartoks \mycharclassB \mycharclassB = {.}

aAa A a B aBa BB
\end{example}

\noindent In the above example the input text is typeset as\par
{\centering
 \verb|a[\itshape A\upshape]a A a \bgroup\color{blue}B\egroup aBa B.B|\par}

\newpage
\section{Encodings}

\cmd|\XeTeXinputnormalization|
\xarg{integer}
\desc{Specify whether \xetex is to perform normalisation on the input
text and, if so, what type of normalisation to use. See
\url{https://unicode.org/reports/tr15/} for a description of Unicode
normalisation. The \<Integer> value can be:\medskip
\begin{optdesc}
\item[0] (default) do not perform normalisation.
\item[1] normalise to NFC form, using precomposed characters where possible
         instead base characters with combining marks.
\item[2] normalise to NFD form, using base characters with combining marks
         instead of precomposed characters.
\end{optdesc}}
\endcmd

\cmd|\XeTeXinputencoding|
\xarg{charset name}
\desc{Defines the input encoding of the following text.}
\endcmd

\cmd|\XeTeXdefaultencoding|
\xarg{charset name}
\desc{Defines the input encoding of subsequent files to be read.}
\endcmd

\section{Line breaking}

\cmd|\XeTeXdashbreakstate|
\xarg{integer}
\desc{Specify whether line breaks after en- and em-dashes are
allowed. Off, 0, by default.}
\endcmd

\cmd|\XeTeXlinebreaklocale|
\xarg{locale-id}
\desc{Defines how to break lines for multilingual text.}
\endcmd

\cmd|\XeTeXlinebreakskip|
\xarg{glue}
\desc{Inter-character linebreak stretch}
\endcmd

\cmd|\XeTeXlinebreakpenalty|
\xarg{integer}
\desc{Inter-character linebreak penalty}
\endcmd

\cmd|\XeTeXupwardsmode|
\xarg{integer}
\desc{If greater than zero, successive lines of text (and rules, boxes,
etc.) will be stacked upwards instead of downwards.}
\endcmd


\cmd|\XeTeXhyphenatablelength|
\xarg{integer}
\desc{Sets the maximum hyphenatable word length;
     the default is 63, for compatibility with classic \TeX.}
\endcmd

\section{Graphics}

Thanks to Heiko Oberdiek, Paul Isambert, and William Adams for their
help with the documentation in this section.

\cmd|\XeTeXpicfile|
\xarg{filename}
\oarg{ scaled \xarg{int} \|
  xscaled \xarg{int} \|
  yscaled \xarg{int} \| \\\hspace*{2em}
  width \xarg{dimen} \|
  height \xarg{dimen} \|
  rotated \xarg{decimal} }
\desc{Insert an image. See below for explanation of optional arguments.}
\endcmd

\cmd|\XeTeXpdffile|
\xarg{filename}
\oarg{page \xarg{int}}
\oarg{ crop \| media \| bleed \| trim \| art }\\\hspace*{2em}
\oarg{scaled \xarg{int} \|
  xscaled \xarg{int} \|
  yscaled \xarg{int} \|
  width \xarg{dimen} \|\\\hspace*{2em}~
  height \xarg{dimen} \|
  rotated \xarg{decimal}}
\desc{Insert (pages of) a PDF. See below for explanation of optional
arguments.}
\endcmd

\noindent In the graphic/PDF commands above, \xarg{filename} is the
usual file name argument of \cs{input}, \cs{openin}, \etc.  It must not
terminated by \cs{relax} if options are given.  \xarg{int} and
\xarg{dimen} are the usual integer or dimen specifications of regular
\tex.

The rotation is specified in degrees (\ie, an input of ‘|360|’ is full
circle) and the rotation is counterclockwise. The syntax of
\xarg{decimal} require some explanation:
\begin{quote}
\<decimal> $\to$ \<optional signs>\<unsigned decimal>\\
\<unsigned decimal> $\to$ \<normal decimal>
  \| \<coerced dimen> \| \<internal dimen>\\
\<normal decimal> $\to$ \<normal integer> \| \<decimal constant>
\end{quote}
A \xarg{coerced dimen} or \xarg{internal dimen} is interpreted as a number
with unit ‘|pt|’. For example, for a rotation specified with a dimension
\cs{testdim},
\begin{itemize}
\item \verb|\testdim=45pt  | results in a rotation of 45\textdegree,
\item \verb|\testdim=1in    | is 72.27\textdegree, and
\item \verb|\testdim=100sp| is (100/65536)\textdegree.
\end{itemize}
In all cases, the final decimal number for rotation $x$ must be
within the limits $-16384 < x < 16384$.

The \cs{XeTeXpdffile} command takes one more optional argument for
specifying to which ‘box’ the PDF should be cropped before inserting
it (the second optional argument listed in the syntax of
\cs{XeTeXpdffile} above). The PDF standard defines a number of
(rectangular) bounding boxes that may be specified for various
purposes. These are described in the PDF Standard\footnote{Adobe
Systems Incorporated, 2008:\\
\url{https://adobe.com/devnet/acrobat/pdfs/PDF32000_2008.pdf}} and
summarised below.
\begin{quote}
\begin{description}[style=nextline,leftmargin=1.5cm]
\item [media] the box defining the physical page size.
\item [crop] the box of the page contents for display/printing purposes.
\item [bleed] the box containing the page contents plus whatever extra
space required for printing purposes.
\item [trim] the box of the finished page after trimming the printed
‘bleed box’.
\item [art] the box containing the ‘meaningful content’ of the
page. This could be the crop box with boilerplate text/logos trimmed
off.
\end{description}
\end{quote}
When not specified in the PDF to be inserted, the crop box defaults
to the media box, and the bleed, trim, and art boxes default to the crop
box.

\cmd|\XeTeXpdfpagecount|
\xarg{filename}
\desc{Expands to the number of pages in a PDF file.}
\endcmd


\section{Character protrusion}

\cmd|\XeTeXprotrudechars|
\xarg{integer}
\desc{Equivalent to \cs{pdfprotrudechars} in \pdftex\ for controlling
character protrusion or ‘margin kerning’. When set to zero (default), character
protrusion is turned off. When set to one, it is activated but will not affect
line-breaking. When set to two, line-breaking decisions will change as a result
of the character protrusion.}
\begin{example}
\XeTeXprotrudechars=2
\font\rm="[texgyrepagella-regular.otf]"\relax
\rm
\rpcode\font\XeTeXcharglyph\hyphenchar\font=250
\hsize=20mm
a a a a a a a a a a abbabbabb aabbabbabb abbabb
\end{example}
See the \pdftex\ documentation for further details.
\endcmd

\cmd|\rpcode| \xarg{font} \xarg{char slot} (integer, $n$)
\desc{Sets the right-side character protrusion value of the \xarg{char slot} in
the specified \xarg{font} to $n/1000$\,em. $n$ is clipped to $\pm1000$.}
\endcmd

\cmd|\lpcode| \xarg{font} \xarg{char slot} (integer, $n$)
\desc{Sets the left-side character protrusion value of the \xarg{char slot} in
the specified \xarg{font} to $n/1000$\,em. $n$ is clipped to $\pm1000$.}
\endcmd

\cmd|\rightmarginkern| \xarg{box number}
\desc{Expands to the width of the margin kerning at the right side of
the horizontal list stored in \cs{box} \xarg{box number}.}
\endcmd

\cmd|\leftmarginkern| \xarg{box number}
\desc{Expands to the width of the margin kerning at the left side of
the horizontal list stored in \cs{box} \xarg{box number}.}
\endcmd


\section{Cross-compatibility with other \TeX\ engines}

All of the major (non-frozen) \TeX\ engines support most of the
functionality described in this section, in one way or another, so that
macro formats (e.g., \LaTeX) can make use of it. Please see the
\pdftex\ manual for details (in \pdftex, all these primitive names
begin with \verb|\pdf...|).

\subsection{Geometry}

\cmd|\pdfpageheight|
\xarg{dimension}
\desc{The height of the PDF page.}
\endcmd

\cmd|\pdfpagewidth|
\xarg{dimension}
\desc{The width of the PDF page.}
\endcmd

\cmd|\pdfsavepos|
\desc{Saves the current location of the page in the typesetting stream.}
\endcmd

\cmd|\pdflastxpos|
\desc{Retrieves the horizontal position saved by \texttt{\char`\\pdfsavepos}.}
\endcmd

\cmd|\pdflastypos|
\desc{Retrieves the vertical position saved by \texttt{\char`\\pdfsavepos}.}
\endcmd

\subsection{Programming}

\cmd|\expanded|
\xarg{general text}
\desc{Carries out full expansion of a token list like \texttt{\char`\\message},
but it is still expandable.}
\endcmd

\cmd|\ifincsname...[\else...]\fi|
\desc{\tex conditional, true if the expansion occurs within
\texttt{\char`\\csname ... \char`\\endcsname}.}
\endcmd

\begin{example}
\def\x{\ifincsname y\else hello\fi}
\def\y{goodbye}
\x/\csname\x\endcsname
\end{example}

\cmd|\ifprimitive| \xarg{control sequence} |...[\else...]\fi|
\desc{\tex conditional to test if a control sequence is a primitive
which still has its primitive meaning, \ie, has not been redefined or
undefined.}
\endcmd

\cmd|\partokencontext|
\xarg{integer}
\desc{Specifies the cases in which \cs{par}, or the given
\cs{partokenname} (see next) is internally emitted. The default is zero,
yielding the standard behavior. See the \pdftex\ manual for more.}
\endcmd

\cmd|\partokenname|
\xarg{control sequence}
\desc{By default, \TeX\ emits a control sequence \cs{par} at blank lines
and other situations. This command changes the name of what is emitted
to the given control sequence. See the \pdftex\ manual for more.}
\endcmd

\cmd|\primitive|
\xarg{control sequence}
\desc{This command executes the primitive meaning of the following control
sequence, regardless of whether the control sequence has been redefined
or made undefined.  If the primitive was expandable, \cs{primitive} expands
also.  On the other hand, if the control sequence never was a primitive,
nothing happens and no error is raised.}
\endcmd

\cmd|\shellescape|
\desc{Read-only status indicating the level of ‘shell escape’
allowed. That is, whether commands are allowed to be executed through
\texttt{\char`\\write18\char`\{...\char`\}}. Expands to zero for off;
one for on (allowed); two is ‘restricted’ (default in TeX Live 2009 and
greater) in which a subset of commands only are allowed.}
\endcmd

\begin{example}
Shell escape \ifnum\shellescape>0 is \else is not \fi enabled.
\end{example}

\cmd|\showstream|
\xarg{integer}
\desc{If this primitive parameter has a value corresponding to an open output
stream (which has been opened with \cs{openout}), then any
\cs{show}, \cs{showthe}, \cs{showbox} or \cs{showlists}
commands do not write output to the terminal, but instead write to only
the referenced output stream, as if they were written with
\cs{immediate}\cs{write}.}
\endcmd

\cmd|\special shipout |
\xarg{token list}
\desc{With the additional ``keyword'' \texttt{shipout}, expansion of the
\xarg{token listr} is delayed until \cs{shipout}, like
non-\cs{immediate}\cs{write}.}
\endcmd

\cmd|\strcmp|
\xarg{arg one}
\xarg{arg two}
\desc{Compares the full expansion of the two token list
arguments. Expands to zero if they are the same, $-1$ if the
first argument sorts lower (lexicographically) than the second argument,
and~$1$ if vice versa.}
\endcmd

\begin{example}
‘a’ is less than ‘z’: \strcmp{a}{z}

\def\z{a}
The tokens expand before being compared: \strcmp{a}{\z}

\def\a{z}
Therefore, |\a| is greater than |\z|: \strcmp{\a}{\z}

\edef\b{\string b}
Also note that catcodes are ignored: \strcmp{b}{\b}
\end{example}

\cmd|\tracingstacklevels|
\xarg{integer}
\desc{If this integer parameter is positive, and \cs{tracingmacros}
is also positive, a prefix indicating the macro expansion depth
is output on each relevant log line (e.g., \texttt{\char`\~..}\ at depth~2).
Also, macro logging is truncated at a depth $\ge$ the parameter value.}
\endcmd

\subsection{Randomness}

\cmd|\normaldeviate|
\xarg{number}
\desc{Generate a normally-distributed random integer with a mean of $0$ and
standard deviation $65\,536$. That is, about $68$\% of the time, the result
will be between $−65\,536$ and $65\,536$ (one standard
deviation away from
the mean). About $95\%$ of results will be within two standard deviations, and
$99.7$\% within three.}
\endcmd

\cmd|\randomseed|
\desc{A read-only integer which allows querying of the random seed.}
\endcmd

\cmd|\setrandomseed|
\xarg{number}
\desc{Set the random seed to a specific value, allowing you to replay
sequences of semi-randoms at a later moment.}
\endcmd

\cmd|\uniformdeviate|
\xarg{number}
\desc{Generate a uniformly-distributed random integer value between $0$
(inclusive) and ⟨number⟩ (exclusive).}
\endcmd

\cmd|SOURCE_DATE_EPOCH|
(environment variable)
\desc{If this environment variable is set, its value is used as the current
time in seconds since the usual Unix ``epoch'': the beginning of
1970-01-01, UTC. Negative or non-integer values cause a fatal error.
Thus, a value of \texttt{32} would result in a \texttt{/CreationDate} and
\texttt{/ModDate} values of \texttt{19700101000032Z}. This is useful for
reproducible builds of documents.}
\endcmd

\cmd|FORCE_SOURCE_DATE|
(environment variable)
\desc{If this is set to~\texttt{1}, \xetex's time-related primitives are also
initialized from the value of \texttt{SOURCE\_DATE\_EPOCH}. These primitives
are \cs{year}, \cs{month}, \cs{day}, and \cs{time}. If
\texttt{SOURCE\_DATE\_EPOCH} is not set, setting \texttt{FORCE\_SOURCE\_DATE}
has no effect. (See the \pdftex\ manual for pathological cases.)}
\endcmd

\subsection{Timing}

\cmd|\elapsedtime|
\desc{Return a number that represents the time elapsed from the moment of
the start of the run or the last point at which the timer was reset.
The elapsed time is returned in `scaled seconds', meaning seconds divided by
$65\,536$. If the time exceeds $32\,767$ seconds, the constant value
$2^{31} − 1$ will be returned.}
\endcmd

\cmd|\resettimer|
\desc{Reset the internal timer to $0$.}
\endcmd

\subsection{File handling}

\cmd|\creationdate|
\desc{Expands to the date string \xetex\ uses in the \texttt{/Info}
dictionary of the output PDF document. This corresponds to the date and
time when \xetex\ engine started, \eg\ \creationdate\ for this file. If
\texttt{SOURCE\_DATE\_EPOCH} is set, its value is used for
\cs{creationdate}.}
\endcmd

\cmd|\filedump|
\texttt{[offset]} \xarg{number}
\texttt{[length]} \xarg{number}
\barg{\xarg{filename}}
\desc{Expands to a hexadecimal dump of the contents of the file, starting
from position zero and extending to the end of the file unless either an
offset or length are given.}
\endcmd

\cmd|\filemoddate|
\barg{\xarg{filename}}
\desc{Expands to the date the file was last modified in the same format
as for \cs{creationdate} as a string. If both \texttt{SOURCE\_DATE\_EPOCH}
and \texttt{FORCE\_SOURCE\_DATE} are set, \cs{filemoddate} returns a value
in UTC, ending in \texttt{Z}, regardless of the values of environment
variables.}
\endcmd

\cmd|\filesize|
\barg{\xarg{filename}}
\desc{Expands to the size of the file in bytes.}
\endcmd

\cmd|\mdfivesum|
\oarg{file}
\barg{\xarg{text}}
\desc{Expands to the MD5 sum of the \xarg{text}, or if the \texttt{file}
keyword is given, the MD5 sum of the contents of the file named \xarg{text}.}
\endcmd

\cmd|\input|
\barg{\xarg{filename}}
\desc{As of \TeX~Live 2020, the \cs{input} primitive in all \TeX\
engines, including \xetex, now also accepts a group-delimited filename
argument, as a system-dependent extension, as in
\cs{input\barg{foo.tex}}. The usage of \cs{input} with a standard
space/token-delimited filename is completely unchanged.

This group-delimited argument was previously implemented in \luatex; now
it is available in all engines. ASCII double quote characters
(\texttt{"}) are removed from the filename, but it is otherwise left
unchanged after tokenization.

This extension is unlike most others in that it affects a primitive in
standard \TeX\ (\cs{input}), rather than being related to a new
primitive, command line option, etc. This is allowed because additional
methods of recognizing filenames are explicitly mentioned in
\texttt{tex.web} as acceptable system-dependent extensions.

Incidentally, this does not directly affect \LaTeX's \cs{input} command
(which takes a group-delimited argument), as that is a macro
redefinition of the \cs{input} primitive.}
\endcmd


\subsection{Fonts}

\cmd|\pdfmapfile|
\barg{\xarg{map file}}
\desc{Defined in \texttt{xe(la)tex.ini} to place a whatsit special for \texttt{xdvipdfmx} to set a font map file. See the \pdftex user manual for full details.}
\endcmd

\cmd|\pdfmapline|
\barg{\xarg{line from map file}}
\desc{Defined in \texttt{xe(la)tex.ini} to place a whatsit special for \texttt{xdvipdfmx} to set up a font map. See the \pdftex user manual for full details.}
\endcmd

\cmd|\tracinglostchars|
\xarg{integer}
\desc{When set to 3 or larger, a character missing from a font provokes
a full error, not just a diagnostic message (perhaps only in the log
file, depending on \texttt{\char`\\tracingonline}. Also, the missing
character code is always reported in hex, two digits for normal \TeX\
fonts or \texttt{U+....} format for system fonts, in addition to its ASCII
character, if printable.}
\endcmd

\cmd|\suppressfontnotfounderror|
\xarg{integer}
\desc{When set to zero (default) if a font is loaded that cannot be
located by \xetex, an error message results and typesetting is
halted. When set to one, this error message is
suppressed and the font control sequence being defined is set to
\cs{nullfont}.}
\begin{example}
\suppressfontnotfounderror=1
\font\x="ImpossibleFont" at 10pt
\ifx\x\nullfont
  \font\x="[texgyrepagella-regular.otf]" at 10pt
\fi
\x This would be ‘ImpossibleFont’, if it existed.
\end{example}
\endcmd

\subsection{Invoking \xetex}

\xetex\ has many command line options. They are almost all inherited
from the common framework for \TeX\ engines as implemented in Web2C
(its manual is available at \url{https://tug.org/web2c}).

The main exceptions are \verb|-no-pdf|, which tells \xetex\ to output
its XDV (extended DVI) without further processing, and
\verb|-output-driver=|\xarg{cmd}, which processes the XDV output with
\xarg{cmd}; the default is \texttt{xdvipdfmx}.


\section{Engine version}

\cmd|\XeTeXversion|
\desc{Expands to a number corresponding to the \xetex version:
      \the\XeTeXversion.}
\endcmd

\cmd|\XeTeXrevision|
\desc{Expands to a string corresponding to the \xetex revision number:
      \XeTeXrevision.}
\endcmd

\begin{example}
The \xetex version used to typeset this document is:
\the\XeTeXversion\XeTeXrevision
\end{example}

\end{document}
