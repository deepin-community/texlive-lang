% \iffalse meta-comment
%
% File `germanb.dtx'
% 
% Copyright 1989--2024
%     Johannes L. Braams
%     Bernd Raichle
%     Walter Schmidt
%     Juergen Spitzmueller
% All rights reserved.
% 
% This file is part of the babel-german bundle,
% an extension to the Babel system.
% ----------------------------------------------
% 
% It may be distributed and/or modified under the
% conditions of the LaTeX Project Public License, either version 1.3
% of this license or (at your option) any later version.
% The latest version of this license is in
%   http://www.latex-project.org/lppl.txt
% and version 1.3 or later is part of all distributions of LaTeX
% version 2003/12/01 or later.
% 
% This work has the LPPL maintenance status "maintained".
% 
% The Current Maintainer of this work is Juergen Spitzmueller.
% \fi
% \CheckSum{538}
%
% \iffalse
%    Tell the \LaTeX\ system who we are and write an entry on the
%    transcript.
%<*dtx>
\ProvidesFile{germanb.dtx}
%</dtx>
%<austrian>\ProvidesLanguage{austrian}
%<german>\ProvidesLanguage{german}
%<swissgerman>\ProvidesLanguage{swissgerman}
%<germanb>\ProvidesLanguage{germanb}
%\fi
%\ProvidesFile{germanb.dtx}
        [2024/01/19 v2.14 German support for babel (pre-1996 orthography)]
%\iffalse
%
%<*filedriver>
\documentclass{ltxdoc}
\usepackage[T1]{fontenc}
\usepackage[osf]{libertine}
\usepackage[scaled=0.76]{beramono}
\usepackage{url}
\usepackage{array}
\usepackage{booktabs}
\usepackage[tableposition=top,skip=5pt]{caption}
\usepackage{marginnote}
\usepackage[svgnames]{xcolor}
\usepackage[pdfusetitle]{hyperref}
\hypersetup{%
  colorlinks,
  linkcolor=black,
  filecolor=Maroon,
  urlcolor=Maroon,
  citecolor=black
}
\newcommand*\babel{\textsf{babel}}
\newcommand*\Babel{\textsf{Babel}}
\newcommand*\langvar{$\langle \it lang \rangle$}
\newcommand*\graph[1]{$\langle$#1$\rangle$}
\newcommand*\note[1]{}
\newcommand*\Lopt[1]{\texttt{#1}}
\newcommand*\file[1]{\texttt{#1}}
\newcommand*\newfeature[1]{\marginnote{\footnotesize New feature\\in v.\,#1!}}
\GlossaryMin = .33\textheight

\begin{document}
 \DocInput{germanb.dtx}
\end{document}
%</filedriver>
%\fi
% \GetFileInfo{germanb.dtx}
%
%\RecordChanges
%
% \changes{Version 2.9a=Version 2.10}{2016/11/07}{Improvements to the manual}
%
% \changes{Version 2.7}{2013/12/13}{Revised documentation: Turn the \babel{}
%          manual chapter into a self-enclosed manual.}
% \changes{Version 2.7}{2013/12/13}{Added support for variety \Lopt{swissgerman}.}
% \changes{Version 2.7}{2013/12/13}{Revised \Lopt{austrian} support.}
%
% \changes{Version 2.6d}{1996/07/10}{Replaced \cs{undefined} with
%    \cs{@undefined} and \cs{empty} with \cs{@empty} for consistency
%    with \LaTeX} 
% \changes{Version 2.6d}{1996/10/10}{Moved the definition of
%    \cs{atcatcode} right to the beginning.}
%
% \changes{Version 2.6a}{1995/02/15}{Moved the identification to the
%    top of the file}
% \changes{Version 2.6a}{1995/02/15}{Rewrote the code that handles the
%    active double quote character}
%
% \changes{Version 2.5c}{1994/06/26}{Removed the use of \cs{filedate}
%    and moved the identification after the loading of
%    \file{babel.def}}
%
% \changes{Version 2.5}{1994/02/08}{Update or \LaTeXe}
%
% \changes{Version 2.3e}{1991/11/10}{Brought up-to-date with
%    \file{german.tex} v2.3e (plus some bug fixes) [br]}
%
% \changes{Version 2.2a}{1991/07/15}{Renamed \file{babel.sty} in
%    \file{babel.com}}
%
% \changes{Version 2.3}{1991/11/05}{Rewritten parts of the code to use
%    the new features of babel version 3.1}
%
% \changes{Version 2.2}{1991/06/11}{Removed global assignments,
%    brought uptodate with \file{german.tex} v2.3d}
%
% \changes{Version 2.1}{1991/05/29}{Removed bug found by van der Meer}
%
% \changes{Version 2.0a}{1991/05/25}{Removed some problems in change
%    log}
%
% \changes{Version 2.0}{1991/04/23}{Modified for babel 3.0}
%
% \changes{Version 1.1a}{1990/08/27}{Modified the documentation
%    somewhat}
%
% \changes{Version 1.1}{1990/07/30}{When using PostScript fonts with
%    the Adobe fontencoding, the dieresis-accent is located elsewhere,
%    modified germanb}
%
% \changes{Version 1.0c}{1990/07/16}{Fixed some typos}
%
% \changes{Version 1.0b}{1990/05/22}{fixed typo in definition for
%    austrian language found by Werenfried Spit
%    \texttt{nspit@fys.ruu.nl}}
%
% \changes{Version 1.0a}{1990/05/14}{Incorporated Nico's comments}
%
%  \title{Babel support for the German language (pre-1996~orthography)}
%  \author{Johannes Braams \and Bernd Raichle \and Walter Schmidt \and J\"urgen Spitzm\"uller\thanks{%
%         Current maintainer. Please report issues via \protect\url{https://github.com/jspitz/babel-german}.}}
%  \date{\fileversion\ (\filedate)}
%  \maketitle
%
%    \begin{abstract}
%      \noindent This manual documents \babel\ language support for German (pre-1996 orthography),
%      including support for the Austrian and Swiss (standard) varieties of German.
%      The manual is part of the \textsf{babel-german} bundle.
%    \end{abstract}
%
%    \section{Aim and usage}
%
%    The \babel\ `language definition file' \file{germanb.ldf} documented in this manual provides
%    the \babel\ package with all language specific strings, settings and commands needed for writing
%    German texts, including texts in the Austrian and Swiss (standard) varieties of German, in \emph{traditional
%    (1901--1996) spelling}.\footnote{%
%              The file \file{germanb.ldf} started as a re-implementation of the
%              package \file{german.sty} (v.\,2.5b), which was originally
%              developed by Hubert Partl (cf.\,\cite{HP}) and later maintained by Bernd Raichle
%              (cf.\,\cite{gerdoc}). Johannes Braams did the initial re-implementation.}
%    As for support for contemporary (`reformed', i.\,e., post-1996) German orthography, please refer to
%    the complementary \href{file:ngermanb.pdf}{manual for the \file{ngermanb.ldf} language definition file}.
%    The `language definition file' also assures that the correct hyphenation patterns for the respective
%    language or variety are used (see sec.~\ref{sec:patterns} for details).
%
%    In order to use the language definitions provided here, you need to use
%    the \babel\ package and pass the respective language\slash variety name as an
%    option, either of
%
%    \begin{itemize}
%       \setlength{\itemsep}{0pt}
%       \item |\usepackage[german]{babel}|
%       \item |\usepackage[austrian]{babel}|
%       \item |\usepackage[swissgerman]{babel}|
%       \item |\usepackage[swissgerman.toss]{babel}|\newfeature{2.10}\footnote{%
%              See sec.~\ref{sec:toss} on the \Lopt{toss} modifier.}
%    \end{itemize}
%    Using multiple varieties in parallel is possible;
%    consult the \babel\ manual \cite{babel} for details.
%
%    \section{Shorthands}
%
%    For all three varieties of German, the character |"| is made active
%    in order to provide some shorthand macros.
%    Some of these shorthands address a peculiarity of pre-1996 German
%    spelling: consonantial character combinations that change in the context
%    of hyphenations.
%    Other shorthands are provided for frequently used
%    special characters as well as for better control of hyphenation,
%    line breaks and ligatures.
%
%    Table~\ref{tab:german-quote} provides an overview of the shorthands
%    that are provided by \textsf{babel-german} for \Lopt{german}, \Lopt{austrian}
%    and \Lopt{swissgerman}.
%
%    \begin{table}[htb]
%     \begin{center}
%     \small
%     \caption{Shorthands provided by \file{germanb.ldf}}\label{tab:german-quote}
%     \begin{tabular}{l>{\raggedright}p{.9\textwidth}}
%      \toprule
%      |"a|	& 	Umlaut \graph{\"a} (shorthand for |\"a|). Similar shorthands are
%                  	available for all other lower- and uppercase
%                  	vowels (umlauts: |"a|, |"o|, |"u|, |"A|, |"O|,
%                  	|"U|; tremata: |"e|, |"i|, |"E|, |"I|).			\tabularnewline
%      |"s|	& 	German \graph{\ss} (shorthand for |\ss{}|);
%                       but cf. sec.~\ref{sec:toss}.                             \tabularnewline
%      |"z|	& 	German \graph{\ss} (shorthand for |\ss{}|). 
% 			The difference to |"s| is the uppercase version;
%                       but cf. sec.~\ref{sec:toss}.                            \tabularnewline
%      |"ck|	& 	\graph{ck}, hyphenated as \graph{k-k}.			\tabularnewline
%      |"ff|	& 	\graph{ff}, hyphenated as \graph{ff-f};
%                  	this is also implemented for \graph{l}, \graph{m}, 
%                 	\graph{n}, \graph{p}, \graph{r} and \graph{t}. 		
%                       Please refer to sec.~\ref{sec:toss} for why this does not
%                       include \graph{s}.                                      \tabularnewline
%      |"S|	& 	|\uppercase{"s}|, typeset as \graph{SS} (\graph{\ss}
%			must be written as \graph{SS} [or \graph{SZ}, see below] 
%			in uppercase writing).					\tabularnewline
%      |"Z|	& 	|\uppercase{"z}|, typeset as \graph{SZ} (\graph{\ss}
%			must be written as \graph{SZ} [or \graph{SS}, see above]
%			in uppercase writing).					\tabularnewline
%     \verb="|= & 	Disable ligature at this position (e.\,g., at morpheme
%			boundaries, as in \verb=Auf"|lage=).                 	\tabularnewline
%      |"-|	& 	An additional breakpoint that does still
%             		allow for hyphenation at the breakpoints preset in
%             		the hyphenation patterns (as opposed to |\-|).       	\tabularnewline
%      |"=|	& 	An explicit hyphen with a breakpoint, allowing
%             		for hyphenation at the other points preset in the
%             		hyphenation patterns (as opposed to plain |-|);
%			useful for long compounds such as
%			|IT"=Dienstleisterinnen|.      				\tabularnewline
%      |"~|	& 	An explicit hyphen without a breakpoint. Useful for
%              		cases where the hyphen should stick at the following
%			syllable, e.\,g., |bergauf und "~ab| .			\tabularnewline
%      |""|	& 	A breakpoint that does not output a hyphen if the line 
%             		break is performed (consider parenthetical extensions
%			as in |(pseudo"~)""wissenschaftlich|).           	\tabularnewline
%      |"/|	& 	\newfeature{2.9}A slash that allows for a linebreak. 
%			As opposed to |\slash{}|, hyphenation at the breakpoints
%             		preset in the hyphenation patterns is still allowed.	\tabularnewline
%      |"*|	& 	\newfeature{2.14}An asterisk which assures the word can
%                       still be hyphenated at its defined breakpoints.
%                       Useful if you want to employ gender-sensitive writing
%                       (,gender star`).					
%                       Similar shorthands are available for the alternative
%                       gender-sensitive spellings, |":| and |"_|.		\tabularnewline
%      |"x|	& 	\newfeature{2.14}Inserts a gender mark which assures
%                       the word can still be hyphenated at its defined breakpoints.
%                       This is predefined to |*| but can be globally redefined
%                       by redefining the macro |\def\mkgender{*}|.		\tabularnewline
%      |"`|	& 	German left double quotes \graph{,,}.               	\tabularnewline
%      |"'|	& 	German right double quotes \graph{``}.              	\tabularnewline
%      |"<|	& 	French/Swiss left double quotes \graph{<<}.   		\tabularnewline
%      |">|	& 	French/Swiss right double quotes \graph{>>}.  		\tabularnewline
%     \bottomrule
%     \end{tabular}
%     \end{center}
%    \end{table}
%
%    \noindent Table~\ref{tab:more-quote} lists some \babel\ macros for quotation marks
%    that might be used as an alternative to the quotation mark shorthands listed above.
%    \begin{table}[!h]
%     \begin{center}
%     \small
%     \caption{Alternative commands for quotation marks (provided by \babel)}\label{tab:more-quote}
%     \begin{tabular}{lp{.88\textwidth}}
%      \toprule
%      |\glqq|	&	German left double quotes \graph{,,}.				\tabularnewline
%      |\grqq|	&	German right double quotes \graph{``}.				\tabularnewline
%      |\glq|	&	German left single quotes \graph{,}.				\tabularnewline
%      |\grq|	&	German right single quotes \graph{`}.				\tabularnewline
%      |\flqq|	&	French/Swiss left double quotes \graph{<<}.			\tabularnewline
%      |\frqq|	&	French/Swiss right double quotes \graph{>>}.			\tabularnewline
%      |\flq|	&	French/Swiss left single quotes \graph{\guilsinglleft}.		\tabularnewline
%      |\frq|	&	French/Swiss right single quotes \graph{\guilsinglright}.	\tabularnewline
%      |\dq|	&	The straight quotation mark character \graph{\textquotedbl}.	\tabularnewline
%      \bottomrule
%     \end{tabular}
%     \end{center}
%    \end{table}
%
%   \section{Hyphenation patterns}\label{sec:patterns}
%
%   The question which hyphenation patterns are used by \Babel\ in case of the varieties of German needs some elaboration.
%   There is a set of established hyphenation patterns for pre- and post-1996 German orthography that has been available
%   with \TeX\ distributions for a long time (currently, these are shipped in form of the \texttt{dehypht} and \texttt{dehyphn} files).
%   These patterns, though, have many flaws (they produce wrong hyphenations, and not much is known about their construction). Therefore, a group
%   of people developed completely new patterns that do much better, the so-called `experimental' new hyphenation patterns of German,
%   distributed in the \textsf{dehyph-exptl} package \cite{exptl}.
%   As opposed to the old patterns, the new ones undergo constant improvement. The price for this, however, is that hyphenation and thus
%   the typeset document is subject to change with, and only due to, pattern updates.
%
%   Modern engines (i.\,e., \texttt{xetex} and \texttt{luatex}) have already embraced those new patterns, i.\,e., they are activated on these engines by default.
%   The classic \TeX\ engines (\texttt{tex}\slash\texttt{pdftex}), however, haven't: they continue to use the old patterns. The reason for this is one
%   of \TeX's quality standards: refrain, if ever possible, from changing the output of user's documents in the wake of software updates.
%   An exception is (pre-1996) Swiss Standard German: here, the classic engines use the `experimental' patterns by default (since there were no patterns available
%   previously anyway).
%
%   So you need to explicitly activate the new patterns for a given document (except for \Lopt{swissgerman}) with the classic engines, should you
%   want to use them instead of the old ones.
%   With \Babel, this can be done easily by means of the \cs{babelprovide} command and the \textsf{hyphenrules} option:
%   \begin{verbatim}
%     \babelprovide[hyphenrules=german-x-latest]{german}
%   \end{verbatim}\vspace{-\baselineskip}
%   For \Lopt{austrian}, use a respectively adapted version of the above.
%
%
%   \section{Variety-specific options}\label{sec:toss}
%
%   In\newfeature{2.10} Swiss (and Liechtensteinian) German writing, the use of \graph{\ss} is rather uncommon.
%   Swiss writers would normally use \graph{ss} where German or Austrian writers use the \graph{\ss} character
%   (e.\,g., \emph{Bu\ss e} vs.\,\emph{Busse}). When texts (or names) from other German speaking areas are quoted,
%   however, the spelling and hence the \graph{\ss} is often maintained  (particularly in scholarly writing where
%   the spelling of quoted text is not supposed to be touched).
%
%   We assume that Swiss writers will normally input \graph{ss} directly when they mean \graph{ss}, and we assume furthermore
%   that the \graph{\ss}-related shorthands |"s| and |"z| are useful also for Swiss writers when they actually need \graph{\ss},
%   the more so since the \graph{\ss} is not as directly accessible on Swiss keyboards as it is on German and Austrian ones.
%   On the other hand, there might be occasions where writers want to transfer a text from German or Austrian Standard into Swiss
%   Standard German and adapt the spelling on the fly, i.\,e., transform all \graph{\ss} into \graph{ss}.
%
%   For this special case, we provide an option to make the \graph{\ss}-related shorthands |"s| and |"z| expand to the
%   respective digraphs\footnote{In graphematics, the term \emph{digraph} denotes two characters that make a functional pair
%   (which means, depending on the theoretical assumptions: they represent a single sound or they are semantically distinctive).},
%   \graph{ss} and \graph{sz}, rather than to \graph{\ss}. This is not the default behavior with
%   \Lopt{swissgerman} since, as mentioned, there are situations when the \graph{\ss} is (and has to be) used in Swiss writing,
%   and normally, no shorthand is needed to input (or output) two simple \graph{s} characters. You can opt-in (and out) digraphical
%   expansion of |"s| and |"z| on a global and local level:
%   \begin{itemize}
%   \item To globally switch on the digraphical expansion, use the \Babel\ modifier \Lopt{toss} (read: `to \graph{ss}') with
%         \Lopt{swissgerman}. I.\,e., pass \Lopt{swissgerman.toss} (rather than \Lopt{swissgerman}) as \babel\ option.
%
%   \item To switch on the digraphical expansion only locally, you can use the boolean switch |\tosstrue|. Likewise,
%         |\tossfalse| switches off (both locally and globally set) digraphical expansion.
%   \end{itemize}
%   Both these changes result in the following deviant behavior of two shorthands:
%    \begin{table}[!h]
%     \begin{tabular}{l>{\raggedright}p{.9\textwidth}}
%      |"s|	& 	Expands to digraph \graph{ss}             \tabularnewline
%      |"z|	& 	Expands to digraph \graph{sz}             \tabularnewline
%     \end{tabular}
%    \end{table}
%
%    \noindent One further note related to the use of \graph{ss} in Swiss Standard German. As opposed to other consonantial letters, the \graph{s}
%    is excluded from the three consonant rule (\emph{Dreikonsonantenregel}) of traditional German spelling which prescribes
%    that one of three identical consonants has to be omitted if a vowel follows the three consonants (i.\,e., \emph{Schiffahrt},
%    not \emph{Schifffahrt}), except if the word is hyphenated (\emph{Schiff-fahrt}); the shorthands |"ff| etc. account for that.
%    This does not apply to \graph{s}! In that case, always all three consonants are spelled out (e.\,g., \emph{Kongresssaal},
%    not \emph{Kongressaal}). This is why we don't provide a shorthand for the \graph{sss} case.
%
%
% \StopEventually{}
%
%    \section{Implementation}
%
%    \subsection{General settings}
%
%
%    If \file{germanb.ldf} is read via the deprecated  \babel\ option
%    \Lopt{germanb}, we make it behave as if \Lopt{german} was specified.
% \changes{Version 2.6m}{2008/06/01}{Corrected a typo}
%    \begin{macrocode}
\def\tmpa{germanb}
\let\SaveCurrentOption\CurrentOption
\ifx\CurrentOption\tmpa
  \def\CurrentOption{german}
\fi
%    \end{macrocode}
%
%    The macro |\LdfInit| takes care of preventing that this file is
%    loaded more than once with the same option, checking the category
%    code of the \texttt{@} sign, etc.
% \changes{Version 2.6d}{1996/11/02}{Now use \cs{LdfInit} to perform
%    initial checks} 
%    \begin{macrocode}
\LdfInit\CurrentOption{captions\CurrentOption}
%    \end{macrocode}
%
%    We define some helper macros that help us to identify later on
%    which variety of German we are currently dealing with.
% \changes{Version 2.9a=Version 2.10}{2016/11/03}{Add helper macros to identify the current option.}
% \changes{Version 2.9d=Version 2.13}{2021/02/27}{Move option helper macros after \cs{LdfInit} to fix plain tex usage.}
% \iffalse
%<*germanb>
% \fi
%    \begin{macrocode}
\def\bbl@opt@german{german}
\def\bbl@opt@germanb{germanb}
\def\bbl@opt@austrian{austrian}
\def\bbl@opt@swissgerman{swissgerman}
%    \end{macrocode}
%
% Some more work to make germanb behave like german.
% \changes{Version 2.6l}{2008/03/17}{Making germanb behave like german
%    needs some more work besides defining \cs{CurrentOption}}
%    \begin{macrocode}
\ifx\SaveCurrentOption\bbl@opt@germanb
  \ifx\l@german\@undefined
    \@nopatterns{German (trad. orthography)}
    \adddialect\l@german0
  \fi
  \let\l@germanb\l@german
  \AtBeginDocument{%
    \let\captionsgermanb\captionsgerman
    \let\dategermanb\dategerman
    \let\extrasgermanb\extrasgerman
    \let\noextrasgermanb\noextrasgerman
  }
\fi
%    \end{macrocode}
%
%    If \file{germanb.ldf} is read as an option, i.e. via |\usepackage|
%    command, \Lopt{german} could be an `unknown' language, so we
%    have to make it known.  We check for the existence of
%    |\l@german| and issue a warning if it is unknown.
%
% \changes{Version 2.0}{1991/04/23}{Now use \cs{adddialect} if
%    language undefined}
% \changes{Version 2.2d}{1991/10/27}{Removed use of \cs{@ifundefined}}
% \changes{Version 2.3e}{1991/11/10}{Added warning, if no german
%    patterns loaded}
% \changes{Version 2.5c}{1994/06/26}{Now use \cs{@nopatterns} to
%    produce the warning}
%    \begin{macrocode}
\ifx\l@german\@undefined
  \@nopatterns{German (trad. orthography)}
  \adddialect\l@german0
\fi
%    \end{macrocode}
%
%    We set \Lopt{austrian} as a dialect of \Lopt{german},
%    since the Austrian variety uses the same hyphenation patterns as
%    Germany's Standard German. If no German patterns are found, we issue
%    a warning.
% \changes{Version 2.0}{1991/04/23}{Now use \cs{adddialect} for
%    austrian}
% \changes{Version 2.8}{2016/11/01}{Only add Austrian dialect if \Lopt{austrian} is loaded}
% \changes{Version 2.9}{2016/11/02}{Do not attempt to load \cs{l@austrian}, which does not exist}
%    \begin{macrocode}
\ifx\CurrentOption\bbl@opt@austrian
  \ifx\l@german\@undefined
    \@nopatterns{German (trad. orthography), needed by Austrian (trad. orthography)}
    \adddialect\l@austrian0
  \else
    \adddialect\l@austrian\l@german
  \fi
\fi
%    \end{macrocode}
%
%    For the Swiss variety, we attempt to load the specific
%    \Lopt{swissgerman} hyphenation patterns and fall back to
%    \Lopt{german} if those are not available.
%    If no patterns are found, we issue a warning.
% \changes{Version 2.7b}{2014/10/10}{Do not warn about missing swissgerman patterns
%    if \Lopt{swissgerman} is not loaded}
%    \begin{macrocode}
\ifx\CurrentOption\bbl@opt@swissgerman
  \ifx\l@swissgerman\@undefined
    \ifx\l@german\@undefined
      \@nopatterns{Swiss Standard German (trad. orthography) and German (trad. orthography)}
      \adddialect\l@swissgerman0
    \else
      \@nopatterns{Swiss Standard German (trad. orthography)}
      \adddialect\l@swissgerman\l@german
    \fi
  \fi
\fi
%    \end{macrocode}
%
%    \subsection{Language-specific strings (captions)}
%
%    The next step consists of defining macros that provide language specific strings
%    and settings.
%
%  \begin{macro}{\@captionsgerman}
%    The macro |\@captionsgerman| defines all strings used in the four
%    standard document classes provided with \LaTeX\ for German.
%    This is an internal macro that is inherited and modified by the following
%    macros for the respective language varieties.
%
% \changes{Version 2.2}{1991/06/06}{Removed \cs{global} definitions}
% \changes{Version 2.2}{1991/06/06}{\cs{pagename} should be
%    \cs{headpagename}}
% \changes{Version 2.3e}{1991/11/10}{Added \cs{prefacename},
%    \cs{seename} and \cs{alsoname}}
% \changes{Version 2.4}{1993/07/15}{\cs{headpagename} should be
%    \cs{pagename}}
% \changes{Version 2.6b}{1995/07/04}{Added \cs{proofname} for
%    AMS-\LaTeX}
% \changes{Version 2.6d}{1996/07/10}{Construct control sequence on the
%    fly}
% \changes{Version 2.6j}{2000/09/20}{Added \cs{glossaryname}}
% \changes{Version 2.7}{2013/12/13}{Split \cs{captionsgerman} from
%                                    \cs{captionsaustrian} and
%                                    \cs{captionsswissgerman}.}
% \changes{Version 2.7}{2013/12/13}{Changed \cs{enclname} in
%                                    \Lopt{austrian} to
%                                    \emph{Beilage(n)}.}
% \changes{Version 2.8}{2016/11/01}{Define trans-variational base captions
%                                   which are loaded and modified by the varieties}
%    \begin{macrocode}
\@namedef{@captionsgerman}{%
  \def\prefacename{Vorwort}%
  \def\refname{Literatur}%
  \def\abstractname{Zusammenfassung}%
  \def\bibname{Literaturverzeichnis}%
  \def\chaptername{Kapitel}%
  \def\appendixname{Anhang}%
  \def\contentsname{Inhaltsverzeichnis}%
  \def\listfigurename{Abbildungsverzeichnis}%
  \def\listtablename{Tabellenverzeichnis}%
  \def\indexname{Index}%
  \def\figurename{Abbildung}%
  \def\tablename{Tabelle}%
  \def\partname{Teil}%
  \def\enclname{Anlage(n)}%
  \def\ccname{Verteiler}%
  \def\headtoname{An}%
  \def\pagename{Seite}%
  \def\seename{siehe}%
  \def\alsoname{siehe auch}%
  \def\proofname{Beweis}%
  \def\glossaryname{Glossar}%
}
%    \end{macrocode}
%  \end{macro}
%  \begin{macro}{\captionsgerman}
%    The macro |\captionsgerman| is identical to |\@captionsgerman|,
%    but only defined if \Lopt{german} is requested.
%  \changes{Version 2.8}{2016/11/01}{Only define \cs{captionsgerman} if
%                                    \Lopt{german} is requested.}
%    \begin{macrocode}
\ifx\CurrentOption\bbl@opt@german
  \@namedef{captionsgerman}{%
    \@nameuse{@captionsgerman}%
  }
\fi
%    \end{macrocode}
%  \end{macro}
%  \begin{macro}{\captionsaustrian}
%    The macro |\captionsaustrian| builds on |\@captionsgerman|, but
%    redefines some strings following Austrian conventions (for the
%    respective variants, cf. \cite{vwb}). It is only defined if
%    \Lopt{austrian} is requested.
%  \changes{Version 2.8}{2016/11/01}{Only define \cs{captionsaustrian} if
%                                    \Lopt{austrian} is requested.}
%    \begin{macrocode}
\ifx\CurrentOption\bbl@opt@austrian
  \@namedef{captionsaustrian}{%
    \@nameuse{@captionsgerman}%
    \def\enclname{Beilage(n)}%
  }
\fi
%    \end{macrocode}
%  \end{macro}
%  \begin{macro}{\captionsswissgerman}
%    The macro |\captionsswissgerman| builds on |\@captionsgerman|, but
%    redefines some strings following Swiss conventions (for the
%    respective variants, cf. \cite{vwb}). It is only defined if
%    \Lopt{swissgerman} is requested.
%  \changes{Version 2.8}{2016/11/01}{Only define \cs{captionsswissgerman} if
%                                    \Lopt{swissgerman} is requested.}
%    \begin{macrocode}
\ifx\CurrentOption\bbl@opt@swissgerman
  \@namedef{captionsswissgerman}{%
    \@nameuse{@captionsgerman}%
    \def\enclname{Beilage(n)}%
  }
\fi
%    \end{macrocode}
%  \end{macro}
%
%  \subsection{Date localizations}
%
%  \begin{macro}{\month@german}
%    The macro |\month@german| defines German month names for all varieties.
% \changes{Version 2.3e}{1991/11/10}{Added \cs{month@german}}
%    \begin{macrocode}
\def\month@german{\ifcase\month\or
  Januar\or Februar\or M\"arz\or April\or Mai\or Juni\or
  Juli\or August\or September\or Oktober\or November\or Dezember\fi}
%    \end{macrocode}
%  \end{macro}
%
%  \begin{macro}{\dategerman}
%    The macro |\dategerman| redefines the command
%    |\today| to produce German dates. It is only defined if
%    \Lopt{german} is requested.
% \changes{Version 2.6f}{1997/10/01}{Use \cs{edef} to define
%    \cs{today} to save memory}
% \changes{Version 2.6f}{1998/03/28}{use \cs{def} instead of
%    \cs{edef}}
%  \changes{Version 2.8}{2016/11/01}{Only define \cs{dategerman} if
%                                    \Lopt{german} is requested.}
%    \begin{macrocode}
\ifx\CurrentOption\bbl@opt@german
  \def\dategerman{\def\today{\number\day.~\month@german
      \space\number\year}}
\fi
%    \end{macrocode}
%  \end{macro}
%
%  \begin{macro}{\dateswissgerman}
%    \changes{Version 2.7}{2013/12/13}{Added \cs{dateswissgerman}.}
%    The macro |\dateswissgerman| does the same for Swiss Standard German dates. It
%    is only defined if \Lopt{swissgerman} is requested.
%    The result is identical to German.
%    \changes{Version 2.8}{2016/11/01}{Only define \cs{dateswissgerman} if
%                                      \Lopt{swissgerman} is requested.}
%    \begin{macrocode}
\ifx\CurrentOption\bbl@opt@swissgerman
  \def\dateswissgerman{\def\today{\number\day.~\month@german
      \space\number\year}}
\fi
%    \end{macrocode}
%  \end{macro}
%
%  \begin{macro}{\dateaustrian}
%    The macro |\dateaustrian| redefines the command
%    |\today| to produce Austrian versions of the German dates.
%    Here, the naming of January (,,J\"anner``) differs from the
%    other German varieties. The macro is only defined if
%    \Lopt{austrian} is requested.
% \changes{Version 2.6f}{1997/10/01}{Use \cs{edef} to define
%    \cs{today} to save memory}
% \changes{Version 2.6f}{1998/03/28}{use \cs{def} instead of
%    \cs{edef}}
%  \changes{Version 2.8}{2016/11/01}{Only define \cs{dateaustrian} if
%                                    \Lopt{austrian} is requested.}
%    \begin{macrocode}
\ifx\CurrentOption\bbl@opt@austrian
  \def\dateaustrian{\def\today{\number\day.~\ifnum1=\month
    J\"anner\else \month@german\fi \space\number\year}}
\fi
%    \end{macrocode}
%  \end{macro}
%
%  \subsection{Extras}
%
%  \begin{macro}{\extrasgerman}
% \changes{Version 2.0b}{1991/05/29}{added some comment chars to
%    prevent white space}
% \changes{Version 2.2}{1991/06/11}{Save all redefined macros}
%  \begin{macro}{\extrasaustrian}
%  \begin{macro}{\extrasswissgerman}
%    \changes{Version 2.7}{2013/12/13}{Added \cs{extrasswissgerman}.}
%  \begin{macro}{\noextrasaustrian}
%  \begin{macro}{\noextrasswissgerman}
%    \changes{Version 2.7}{2013/12/13}{Added \cs{noextrasswissgerman}.}
%  \begin{macro}{\noextrasgerman}
% \changes{Version 1.1}{1990/07/30}{Added \cs{dieresis}}
% \changes{Version 2.0b}{1991/05/29}{added some comment chars to
%    prevent white space}
% \changes{Version 2.2}{1991/06/11}{Try to restore everything to its
%    former state}
% \changes{Version 2.6d}{1996/07/10}{Construct control sequence
%    \cs{extrasgerman} or \cs{extrasaustrian} on the fly}
%
%    The macros |\extrasgerman|, |\extrasaustrian|
%    and |\extrasswissgerman|, respectively, will perform all the extra 
%    definitions needed for the German language or the respective
%    variety. The macro |\noextrasgerman| is used to cancel the
%    actions of |\extrasgerman|. |\noextrasaustrian| and
%    |\noextrasswissgerman| behave analoguously.
%
%    First, the character \texttt{"} is declared active for all German
%    varieties. This is done once, later on its definition may vary.
%    \begin{macrocode}
\initiate@active@char{"}
%    \end{macrocode}
%
%    Depending on the option with which the language definition file
%    has been loaded, the macro |\extrasgerman|, |\extrasaustrian|
%    or |\extrasswissgerman| is defined. Each of those is identical:
%    they load the shorthands defined below and activate the \texttt{"}
%    character.
%    \begin{macrocode}
\@namedef{extras\CurrentOption}{%
  \languageshorthands{german}}
\expandafter\addto\csname extras\CurrentOption\endcsname{%
  \bbl@activate{"}}
%    \end{macrocode}
%
%   \changes{Version 2.9a=Version 2.10}{2018/03/28}{Implement boolean switch
%            \cs{tosstrue}\slash\cs{tossfalse} to customize \graph{\ss}-related
%            shorthands in Swiss Standard German context.}
%   \changes{Version 2.9a=Version 2.10}{2018/03/28}{Implement modifier \Lopt{toss}
%            to customize \graph{\ss}-related shorthands in Swiss Standard German context.}
%   \begin{macro}{toss}
%   \begin{macro}{\tosstrue}
%   \begin{macro}{\tossfalse}
%    For Swiss Standard German, we allow optionally to expand the \graph{\ss}-related
%    shorthands the Swiss way, i.\,e. as \graph{ss} (globally, if the modifier \Lopt{toss}
%    is used or locally if |\tosstrue|.).
%    \begin{macrocode}
\newif\iftoss\tossfalse
\newif\ifbbl@toss\bbl@tossfalse
\ifx\bbl@mod@swissgerman\@undefined\else
  \@expandtwoargs\in@{,toss,}{,\bbl@mod@swissgerman,}
  \ifin@
    \tosstrue
  \fi
  \addto\extrasswissgerman{%
    \iftoss\bbl@tosstrue\else\bbl@tossfalse\fi}
\fi
%    \end{macrocode}
%   \end{macro}
%   \end{macro}
%   \end{macro}
%
%    Next, again depending on the option with which the language definition
%    file has been loaded, the macro |\noextrasgerman|, |\noextrasaustrian|
%    or |\noextrasswissgerman| is defined. 
%    These deactivate the \texttt{"} character and thus turn the shorthands
%    off again outside of the respective variety.
% \changes{Version 2.6i}{1999/12/16}{Deactivate shorthands outside of
%    German.}
% \changes{Version 2.7}{2013/12/13}{Deactivate shorthands also outside of
%    \Lopt{austrian} and \Lopt{swissgerman}.}
%    \begin{macrocode}
\expandafter\addto\csname noextras\CurrentOption\endcsname{%
  \bbl@deactivate{"}}
\ifx\CurrentOption\bbl@opt@swissgerman
  \addto\noextrasswissgerman{\bbl@tossfalse}
\fi
%    \end{macrocode}
%
% \changes{Version 2.6a}{1995/02/15}{All the code to handle the active
%    double quote has been moved to \file{babel.def}}
%
%    In order for \TeX\ to be able to hyphenate German words which
%    contain `\ss' (in the \texttt{OT1} position |^^Y|) we have to
%    give the character a nonzero |\lccode| (see Appendix H, the \TeX
%    book).
% \changes{Version 2.6c}{1996/04/08}{Use decimal number instead of
%    hat-notation as the hat may be activated}
%    \begin{macrocode}
\expandafter\addto\csname extras\CurrentOption\endcsname{%
  \babel@savevariable{\lccode25}%
  \lccode25=25}
%    \end{macrocode}
% \changes{Version 2.6a}{1995/02/15}{Removeed \cs{3} as it is no
%    longer in \file{germanb.ldf}}
%
%    The umlaut accent macro |\"| is changed to lower the umlaut dots.
%    The redefinition is done with the help of |\umlautlow|.
%    \begin{macrocode}
\expandafter\addto\csname extras\CurrentOption\endcsname{%
  \babel@save\"\umlautlow}
%    \end{macrocode}
% \changes{Version 2.7}{2013/12/13}{Do not use \cs{@namedef} when
%    \cs{noextras} is already defined and should not be overwritten.}
%    \begin{macrocode}
\expandafter\addto\csname noextras\CurrentOption\endcsname{%
  \umlauthigh}
%    \end{macrocode}
%    The German hyphenation patterns can be used with |\lefthyphenmin|
%    and |\righthyphenmin| set to~2.
% \changes{Version 2.6a}{1995/05/13}{use \cs{germanhyphenmins} to store
%    the correct values}
% \changes{Version 2.6j}{2000/09/22}{Now use \cs{providehyphenmins} to
%    provide a default value}
%    \begin{macrocode}
\providehyphenmins{\CurrentOption}{\tw@\tw@}
%    \end{macrocode}
%    For German texts we need to assure that |\frenchspacing| is
%    turned on.
% \changes{Version 2.6k}{2001/01/26}{Turn frenchspacing on, as in
%    \file{german.sty}}
%    \begin{macrocode}
\expandafter\addto\csname extras\CurrentOption\endcsname{%
  \bbl@frenchspacing}
\expandafter\addto\csname noextras\CurrentOption\endcsname{%
  \bbl@nonfrenchspacing}
%    \end{macrocode}
%  \end{macro}
%  \end{macro}
%  \end{macro}
%  \end{macro}
%  \end{macro}
%  \end{macro}
%
% \changes{Version 2.6a}{1995/02/15}{\cs{umlautlow} and
%    \cs{umlauthigh} moved to \file{glyphs.dtx}, as well as
%    \cs{newumlaut} (now \cs{lower@umlaut}}
%
%    \subsection{Active characters, macros \& shorthands}
%
%    The following code is necessary because we need an extra active
%    character. This character is then used as indicated in
%    table~\ref{tab:german-quote}.
%
%    In order to be able to define the function of |"|, we first define a
%    couple of `support' macros.
%
% \changes{Version 2.3e}{1991/11/10}{Added \cs{save@sf@q} macro and
%    rewrote all quote macros to use it}
% \changes{Version 2.3h}{1991/02/16}{moved definition of
%    \cs{allowhyphens}, \cs{set@low@box} and \cs{save@sf@q} to
%    \file{babel.com}}
% \changes{Version 2.6a}{1995/02/15}{Moved all quotation characters to
%    \file{glyphs.dtx}}
%
%  \begin{macro}{\dq}
%    We save the original double quotation mark character in |\dq| to keep
%    it available, the math accent |\"| can now be typed as |"|.
%
%    Furthermore, we define some helper macros for contextual \graph{\ss} handling.
%    \begin{macrocode}
\begingroup \catcode`\"12
\def\x{\endgroup
  \def\dq{"}
  \def\@SS{\mathchar"7019 }
  \def\bbl@ss{\ifbbl@toss ss\else\textormath{\ss}{\@SS{}}\fi}
  \def\bbl@SS{SS}
  \def\bbl@sz{\ifbbl@toss sz\else\textormath{\ss}{\@SS{}}\fi}
  \def\bbl@SZ{SZ}
}
\x
%    \end{macrocode}
%  \end{macro}
% \changes{Version 2.6c}{1996/01/24}{Moved \cs{german@dq@disc} to
%    babel.def, calling it \cs{bbl@disc}}
%
% \changes{Version 2.6a}{1995/02/15}{Use \cs{ddot} instead of
%    \cs{@MATHUMLAUT}}
%
%    Since we need to add special cases for hyperref which needs hyperref's |\texorpdfstring|,
%    we provide a dummy command for the case that hyperref is not loaded.
% \changes{Version 2.9c=Version 2.12}{2020/07/21}{Properly handle shorthands in hyperref pdf strings}
%    \begin{macrocode}
\providecommand\texorpdfstring[2]{#1}
%    \end{macrocode}
%
%   Now we can define the doublequote shorthands: the umlauts,
% \changes{Version 2.6c}{1996/05/30}{added the \cs{allowhyphens}}
%    \begin{macrocode}
\declare@shorthand{german}{"a}{\textormath{\"{a}\bbl@allowhyphens}{\ddot a}}
\declare@shorthand{german}{"o}{\textormath{\"{o}\bbl@allowhyphens}{\ddot o}}
\declare@shorthand{german}{"u}{\textormath{\"{u}\bbl@allowhyphens}{\ddot u}}
\declare@shorthand{german}{"A}{\textormath{\"{A}\bbl@allowhyphens}{\ddot A}}
\declare@shorthand{german}{"O}{\textormath{\"{O}\bbl@allowhyphens}{\ddot O}}
\declare@shorthand{german}{"U}{\textormath{\"{U}\bbl@allowhyphens}{\ddot U}}
%    \end{macrocode}
%    tremata,
%    \begin{macrocode}
\declare@shorthand{german}{"e}{\textormath{\"{e}}{\ddot e}}
\declare@shorthand{german}{"E}{\textormath{\"{E}}{\ddot E}}
\declare@shorthand{german}{"i}{\textormath{\"{\i}}%
                              {\ddot\imath}}
\declare@shorthand{german}{"I}{\textormath{\"{I}}{\ddot I}}
%    \end{macrocode}
%    German \ss{},
% \changes{Version 2.6f}{1997/05/08}{use \cs{SS} instead of
%    \texttt{SS}, removed braces after \cs{ss}} 
%    \begin{macrocode}
\declare@shorthand{german}{"s}{\bbl@ss}
\declare@shorthand{german}{"S}{\bbl@SS}
\declare@shorthand{german}{"z}{\bbl@sz}
\declare@shorthand{german}{"Z}{\bbl@SZ}
%    \end{macrocode}
%    German and French/Swiss quotation marks,
%    \begin{macrocode}
\declare@shorthand{german}{"`}{\glqq}
\declare@shorthand{german}{"'}{\grqq}
\declare@shorthand{german}{"<}{\flqq}
\declare@shorthand{german}{">}{\frqq}
%    \end{macrocode}
%    discretionary commands
%    \begin{macrocode}
\declare@shorthand{german}{"c}{\textormath{\bbl@disc ck}{c}}
\declare@shorthand{german}{"C}{\textormath{\bbl@disc CK}{C}}
\declare@shorthand{german}{"F}{\textormath{\bbl@disc F{FF}}{F}}
\declare@shorthand{german}{"l}{\textormath{\bbl@disc l{ll}}{l}}
\declare@shorthand{german}{"L}{\textormath{\bbl@disc L{LL}}{L}}
\declare@shorthand{german}{"m}{\textormath{\bbl@disc m{mm}}{m}}
\declare@shorthand{german}{"M}{\textormath{\bbl@disc M{MM}}{M}}
\declare@shorthand{german}{"n}{\textormath{\bbl@disc n{nn}}{n}}
\declare@shorthand{german}{"N}{\textormath{\bbl@disc N{NN}}{N}}
\declare@shorthand{german}{"p}{\textormath{\bbl@disc p{pp}}{p}}
\declare@shorthand{german}{"P}{\textormath{\bbl@disc P{PP}}{P}}
\declare@shorthand{german}{"r}{\textormath{\bbl@disc r{rr}}{r}}
\declare@shorthand{german}{"R}{\textormath{\bbl@disc R{RR}}{R}}
\declare@shorthand{german}{"t}{\textormath{\bbl@disc t{tt}}{t}}
\declare@shorthand{german}{"T}{\textormath{\bbl@disc T{TT}}{T}}
%    \end{macrocode}
%    (we need to treat |"f| a bit differently in order to preserve the
%    ff-ligature)
% \changes{Version 2.6f}{1998/06/15}{Copied the coding for \texttt{"f}
%    from german.dtx version 2.5d} 
%    \begin{macrocode}
\declare@shorthand{german}{"f}{%
  \texorpdfstring{\textormath{\bbl@discff}{f}}% TeX string
                 {f}% PDF string
}
\def\bbl@discff{\penalty\@M
  \afterassignment\bbl@insertff \let\bbl@nextff= }
\def\bbl@insertff{%
  \if f\bbl@nextff
    \expandafter\@firstoftwo\else\expandafter\@secondoftwo\fi
  {\relax\discretionary{ff-}{f}{ff}\bbl@allowhyphens}{f\bbl@nextff}}
\let\bbl@nextff=f
%    \end{macrocode}
%    and some additional commands (hyphenation, line breaking and ligature control):
%  \changes{Version 2.9}{2016/11/02}{Add \texttt{"/} shortcut for breakable slash
%                                    (taken from \file{dutch.ldf})}
%  \changes{Version 2.9b=Version 2.11}{2018/12/08}{Fix old hyphenation regression introduced with \babel\ 3.7
%                                     (2002) in a number of shorthands (change of meaning of \cs{allowhyphens}
%                                      vs. \cs{bbl@allowhyphens})}
%    \begin{macrocode}
\declare@shorthand{german}{"-}{\nobreak\-\bbl@allowhyphens}
\declare@shorthand{german}{"|}{%
  \texorpdfstring{\textormath{\penalty\@M\discretionary{-}{}{\kern.03em}\bbl@allowhyphens}{}}% TeX string
                 {}% PDF string
}
\declare@shorthand{german}{""}{\hskip\z@skip}
\declare@shorthand{german}{"~}{%
  \texorpdfstring{\textormath{\leavevmode\hbox{-}}{-}}% tex string
                 {-}% PDF string
}
\declare@shorthand{german}{"=}{\penalty\@M-\hskip\z@skip}
\declare@shorthand{german}{"/}{\textormath
  {\bbl@allowhyphens\discretionary{/}{}{/}\bbl@allowhyphens}{}}
%    \end{macrocode}
%  and some shorthands to support gender-sensitive spelling:
%  \changes{Version 2.9e=Version 2.14}{2024/01/19}{Add \texttt{"*}, \texttt{":}, \texttt{"\_}, and \texttt{"x}
%                                     shorthands to support gender-sensitive writing}
%    \begin{macrocode}
\def\mkgender{*}
\declare@shorthand{german}{":}{\nobreak:\bbl@allowhyphens}
\declare@shorthand{german}{"*}{\nobreak*\bbl@allowhyphens}
\declare@shorthand{german}{"_}{\nobreak\_\bbl@allowhyphens}
\declare@shorthand{german}{"x}{\nobreak\mkgender\bbl@allowhyphens}
%    \end{macrocode}
%
%  \begin{macro}{\mdqon}
%  \begin{macro}{\mdqoff}
%  \begin{macro}{\ck}
%    All that's left to do now is to  define a couple of commands
%    for reasons of compatibility with \file{german.sty}.
% \changes{Version 2.6f}{1998/06/07}{Now use \cs{shorthandon} and
%    \cs{shorthandoff}} 
%    \begin{macrocode}
\def\mdqon{\shorthandon{"}}
\def\mdqoff{\shorthandoff{"}}
\def\ck{\bbl@allowhyphens\discretionary{k-}{k}{ck}\bbl@allowhyphens}
%    \end{macrocode}
%  \end{macro}
%  \end{macro}
%  \end{macro}
%
%    The macro |\ldf@finish| takes care of looking for a
%    configuration file, setting the main language to be switched on
%    at |\begin{document}| and resetting the category code of
%    \texttt{@} to its original value.
% \changes{Version 2.6d}{1996/11/02}{Now use \cs{ldf@finish} to wrap
%    up} 
%    \begin{macrocode}
\ldf@finish\CurrentOption
%    \end{macrocode}
% \iffalse
%</germanb>
% \fi
%
%
%  \subsection{\file{austrian.ldf}, \file{german.ldf} and \file{swissgerman.ldf}}
%
% \changes{Version 2.7}{2013/12/13}{Generate portmanteau files \file{austrian.ldf},
%            \file{german.ldf} and \file{swissgerman.ldf}.}
%
%  \Babel\ expects a \file{\langvar{}.ldf} file for each \langvar. So we create portmanteau
%    ldf files for \Lopt{austrian}, \Lopt{german} and \Lopt{swissgerman}.\footnote{%
%    For \Lopt{austrian} and \Lopt{german}, this is not strictly necessary,
%    since \babel\ provides aliases for these languages (pointing to \Lopt{germanb}).
%    However, since \babel\ does not officially support these aliases anymore after
%    the language definition files have been separated from the core, we provide
%    the whole range of ldf files for the sake of completeness.} These files themselves
%    only load \file{germanb.ldf}, which does the real work:
%
% \iffalse
%<*austrian|german|swissgerman>
% \fi
%    \begin{macrocode}
\input germanb.ldf\relax
%    \end{macrocode}
% \iffalse
%</austrian|german|swissgerman>
% \fi
%
%
%\PrintChanges
%
%  \begin{thebibliography}{9}
%    \bibitem{vwb} Ammon, Ulrich et al.:
%       \emph{Variantenw\"orterbuch des Deutschen. Die Standardsprache in \"Osterreich, der Schweiz
%        und Deutschland sowie in Liechtenstein, Luxemburg, Ostbelgien und S\"udtirol.}
%        Berlin, New York: De Gruyter.
%    \bibitem{babel} Braams, Johannes and Bezos, Javier:
%       \emph{Babel}.
%       \url{http://www.ctan.org/pkg/babel}.
%    \bibitem{exptl} Deutschsprachige Trennmustermannschaft:
%       \emph{dehyph-exptl -- Experimental hyphenation patterns for the German language}.
%       \url{https://ctan.org/pkg/dehyph-exptl}.
%    \bibitem{HP} Partl, Hubert:
%      \emph{German \TeX}, \emph{TUGboat} 9/1 (1988), p.~70--72.
%    \bibitem{gerdoc} Raichle, Bernd:
%       \emph{German}.
%       \url{http://www.ctan.org/pkg/german}.
%    \bibitem{hyph-utf8} Manuel P\'egouri\'e-Gonnard et al.:
%      \emph{hyph-utf8 -- Hyphenation patterns expressed in UTF-8}.
%       \url{https://ctan.org/pkg/hyph-utf8}.
%  \end{thebibliography}
%
% \Finale
%%
%% \CharacterTable
%%  {Upper-case    \A\B\C\D\E\F\G\H\I\J\K\L\M\N\O\P\Q\R\S\T\U\V\W\X\Y\Z
%%   Lower-case    \a\b\c\d\e\f\g\h\i\j\k\l\m\n\o\p\q\r\s\t\u\v\w\x\y\z
%%   Digits        \0\1\2\3\4\5\6\7\8\9
%%   Exclamation   \!     Double quote  \"     Hash (number) \#
%%   Dollar        \$     Percent       \%     Ampersand     \&
%%   Acute accent  \'     Left paren    \(     Right paren   \)
%%   Asterisk      \*     Plus          \+     Comma         \,
%%   Minus         \-     Point         \.     Solidus       \/
%%   Colon         \:     Semicolon     \;     Less than     \<
%%   Equals        \=     Greater than  \>     Question mark \?
%%   Commercial at \@     Left bracket  \[     Backslash     \\
%%   Right bracket \]     Circumflex    \^     Underscore    \_
%%   Grave accent  \`     Left brace    \{     Vertical bar  \|
%%   Right brace   \}     Tilde         \~}
%%
\endinput
