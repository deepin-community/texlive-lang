% \iffalse meta-comment
%
% File: chhaya.dtx
% ---------------------------------------------------------------------------
% आज्ञासंच:   chhaya
% लेखक:     निरंजन
% आवृत्ती:     ०.४ (१६ डिसेंबर, २०२१)
% माहिती:     भाषावैज्ञानिक छायांगांचे संक्षेप पुरवणारा आज्ञासंच
% संग्राहिका:   https://git.gnu.org.ua/chhaya.git
% अडचणी:    https://puszcza.gnu.org.ua/bugs/?group=chhaya
% परवाना:     आज्ञालेखांकरिता आलोक नित्यमुक्त परवाना (आवृत्ती १.०+) आणि ग्नू पब्लिक परवाना
%           (आवृत्ती ३.०+) व केवळ हस्तपुस्तिकेकरिता ग्नू फ्री डॉक्युमेन्टेशन परवाना (१.३+)
%           दुवे:
%           https://varnamudra.com/aalok/parwana
%           https://www.gnu.org/licenses/gpl-3.0.txt
%           https://www.gnu.org/licenses/fdl-1.3.html
% ---------------------------------------------------------------------------
% The LaTeX package `chhaya'
% Copyright © 2020, 2021 निरंजन
% This program is free software: you can redistribute it and/or modify
% it under the terms of the GNU General Public License as published by
% the Free Software Foundation, either version 3 of the License, or
% (at your option) any later version.
% 
% This program is distributed in the hope that it will be useful,
% but WITHOUT ANY WARRANTY; without even the implied warranty of
% MERCHANTABILITY or FITNESS FOR A PARTICULAR PURPOSE. See the
% GNU General Public License for more details.
% 
% You should have received a copy of the GNU General Public License
% along with this program. If not, see <https://www.gnu.org/licenses/>.
%
% हे काम आलोक नित्यमुक्त परवान्याच्या (आ. १.०+) अटींचे पालन करून वितरित केले जाऊ शकते
% तसेच सुधारले जाऊ शकते. ह्या परवान्याची नवीनतम प्रत खालील दुव्यावर उपलब्ध आहे.
% 
% https://varnamudra.com/aalok/parwana
% --------------------------------------------------------------------
%
% \fi
% \iffalse
%<*internal>
\iffalse
%</internal>
%<*readme>
----------------------------------------------------------------------
आज्ञासंच:       chhaya
लेखक:         निरंजन
आवृत्ती:         ०.४  (१६ डिसेंबर, २०२१)
माहिती:         भाषावैज्ञानिक छायांगांचे संक्षेप पुरवणारा आज्ञासंच
दुवा:           https://git.gnu.org.ua/chhaya.git
अडचणी:        https://puszcza.gnu.org.ua/bugs/?group=chhaya
परवाना:         आज्ञालेखांकरिता आलोक नित्यमुक्त परवाना (आवृत्ती १.०+) आणि ग्नू पब्लिक परवाना
              (आवृत्ती ३.०+) व केवळ हस्तपुस्तिकेकरिता ग्नू फ्री डॉक्युमेन्टेशन परवाना (१.३+)
              दुवे:
              https://varnamudra.com/aalok/parwana
              https://www.gnu.org/licenses/gpl-3.0.txt
              https://www.gnu.org/licenses/fdl-1.3.html
----------------------------------------------------------------------
Package:      chhaya
Author:       निरंजन
Version:      0.4  (16 December, 2021)
Description:  For linguistic glossing in Marathi language.
Repository:   https://git.gnu.org.ua/chhaya.git
Bug tracker:  https://puszcza.gnu.org.ua/bugs/?group=chhaya
License:      `आलोक' copyleft license v1.0+ and GPL v3.0+ for the code.
              GFDL v1.3+ only for the documentation
              Links:
              https://varnamudra.com/aalok/parwana
              https://www.gnu.org/licenses/gpl-3.0.txt
              https://www.gnu.org/licenses/fdl-1.3.html
----------------------------------------------------------------------
%</readme>
%<*internal>
\fi
%</internal>
%<*driver>
\documentclass[10pt]{l3doc}
\usepackage{marathi}
\usepackage{xcolor}
\usepackage{longtable}
\usepackage{biblatex}
\addbibresource{ref.bib}
\usepackage{hyperref}
\hypersetup{%
  colorlinks,%
  unicode,%
  pdftitle={छाया आज्ञासंच (आ. ०.४)},%
  pdfauthor={निरंजन},%
  pdfsubject={छायांगलेखनाकरिता लाटेक् आज्ञासंच},%
  pdfkeywords={छायालेखन, मुंबई विद्यापीठ, भाषाविज्ञान विभाग},%
  pdfproducer={हायपर्रेफ्-सह झीलाटेक्},%
  pdfcreator={हायपर्रेफ्-सह झीलाटेक्},%
  linkcolor={red!50!black},%
  citecolor={blue!50!black},%
  urlcolor={blue!80!black}%
}
\usepackage{fontawesome5}
\usepackage{expl3}
\ExplSyntaxOn
\pdf_version_gset:n{2.0}
\ExplSyntaxOff
\usepackage{booktabs}
\renewcommand{\theCodelineNo}{{\scriptsize\arabic{CodelineNo}}\quad}
\setmonofont[Script=Devanagari]{Mukta}
\newfontfamily{\mukta}[Script=Devanagari]{Mukta}
\newfontfamily{\sho}{Shobhika}
\renewcommand{\abstractname}{सारांश}
\renewcommand{\contentsname}{अनुक्रमणिका}

\begin{document}
\DocInput{chhaya.dtx}
\end{document}
%</driver>
% \fi
% \title{छाया}
% \author{निरंजन}
% \date^^A
%   {^^A
%     आवृत्ती ०.४ \textemdash\ ०८ डिसेंबर, २०२१\\[1ex]^^A
%     {^^A
%       \small\faIcon{globe}\quad
%       \url{https://ctan.org/pkg/chhaya}\\^^A
%       \small\faIcon{bug}\quad
%       \url{https://puszcza.gnu.org.ua/bugs/?group=chhaya}^^A
%     }^^A
%   }
%
% \maketitle
%
% \begin{abstract}
% भाषावैज्ञानिक लेखनात अपरिचित भाषांतील उदाहरणांची छाया देणे ही एक अनिवार्य गोष्ट
% आहे. त्यासाठीच्या संक्षेपांचा संग्रह ह्या आज्ञासंचात करण्यात आला आहे. इंग्रजी छायांगांचे संक्षेप
% \href{https://ctan.org/pkg/leipzig?lang=en}{leipzig} ह्या आज्ञासंचामार्फत पुरवले
% जातात. लायप्चिश् विद्यापीठाच्या नियमावलीनुसार लागणाऱ्या अनेक निकषांची पूर्तता ह्या
% आज्ञासंचातर्फे केली जाते. मराठी भाषावैज्ञानिक लेखनाकरिता छायांगलेखनाचे नवे नियम मुंबई
% विद्यापीठाच्या संकेतस्थळावर
% \href{https://www.mumbailinguisticcircle.com/resources/}{येथे} देण्यात आले आहेत
% \cite{मुंबई}. त्यांचा विचार करून हा आज्ञासंच घडवण्यात आला आहे. ह्या आज्ञासंचात काही
% छायांगांच्या तयार आज्ञा आहेतच, शिवाय लेखकांना गरजेनुसार नवी छायांगे निर्माण करण्यासाठी एक
% आज्ञादेखील आहे. \href{https://ctan.org/pkg/hyperref?lang=en}{hyperref} हा आज्ञासंच
% वापरत असाल, तर छाया हा आज्ञासंच त्यानंतर वापरा.
% \end{abstract}
% 
% \tableofcontents
% \clearpage
% \hspace{0pt}
% \vfill
% \section{परवाना}
% 
% {%
% \setlength{\parindent}{0pt}
% प्रतिमुद्राधिकार © २०२०, २०२१ निरंजन.
%
% ह्या सामग्रीच्या वितरणाचे व प्रतिमुद्रणाचे अधिकार आलोक नित्यमुक्त परवान्यासह मुक्त करण्यात येत
% आहेत. ह्या सामग्रीची यथामूल अथवा परिवर्तित प्रतिमुद्रणे व्यावसायिक अथवा अव्यावसायिक
% स्वरूपात वितरित करण्यास प्रतिमुद्राधिकारधारक संमती देत आहे, परंतु असे करताना वितरकाने
% प्रतिमुद्राधिकारांचा योग्य श्रेयनिर्देश करणे व सामग्री परिवर्तित असल्यास ती ह्याच अटींसह
% वितरित करणे बंधनकारक आहे. ही सामग्री जशी आहे तशी पुरवण्यात येत आहे, पुरवणारा/पुरवणारी
% हिच्याबाबत कोणतीही हमी देत नाही. ह्या (व ह्यावर आधारित) सामग्रीचे अमुक्त वितरण
% बेकायदेशीर मानले जाईल. आलोक नित्यमुक्त परवान्याचा संपूर्ण मसुदा पुढील दुव्यावर वाचता
% येईल.
%
% \url{https://gitlab.com/aalok/nityamukta-parwana}
% 
% \rule{\linewidth}{0.5mm}
%
% \fontfamily{lmr}\selectfont
% Copyright © $2020$, $2021$ {\normalfont निरंजन}.
%
% Permission is granted to copy, distribute and/or modify this document
% under the terms of the GNU Free Documentation License, Version 1.3
% or any later version published by the Free Software Foundation;
% with no Invariant Sections, no Front-Cover Texts, and no Back-Cover Texts.
% A copy of the license is included in the section entitled ``GNU
% Free Documentation License''.
%
% \url{https://www.gnu.org/licenses/fdl}
% }%
%
% \vfill
% \clearpage\pagebreak
% \begin{documentation}
% \begin{function}{\छायांग}
% \begin{syntax}
% \cs{छायांग}\marg{संक्षिप्त रूप}\marg{विस्तृत वर्णन}
% \end{syntax}
% ह्या आज्ञेचा पहिला कार्यघटक छायांगाचे संक्षिप्त रूप हा आहे व दुसरा कार्यघटक त्या छायांगाचे
% स्पष्टीकरणात्मक वर्णन. उदा. एकवचनासाठीचे \textbf{एव} हे छायांग पुढीलप्रमाणे घडवता येते.
% 
% \noindent\verb|\छायांग{एव}{एकवचन}|
% \end{function}
% 
% \begin{function}{\छायांगसूची}
% ही आज्ञा वापरल्यास दस्तऐवजातील सर्व छायांगांची यादी छापली जाते. आज्ञासंचातर्फे ह्या यादीचे
% नाव छायांगसूची असे ठेवले आहे. हे जर बदलायचे असेल तर ह्या आज्ञेस पुढीलप्रमाणे वैकल्पिक कार्यघटक
% देता येतो व त्यात ह्या सूचीचे नाव बदलता येते.
% \begin{verbatim}
% \छायांगसूची[संक्षेपसूची]
% \end{verbatim}
% ह्यामुळे छापल्या जाणाऱ्या यादीचे नाव छायांगसूचीऐवजी संक्षेपसूची ठेवले जाईल.
% \end{function}
% 
% \begin{function}{समरेखा}
% छायालेखनाच्या नियमावलीतील तिसऱ्या नियमानुसार मजकूर पारंपरिक टंकात असेल, तर छायांगांकरिता
% समरेखा टंक वापरण्यात यावेत व मजकूर समरेखा टंकात असेल, तर छायांगांकरिता पारंपरिक टंक
% वापरावेत. पारंपरिक टंक देवनागरी लिहिताना जास्त वापरले जात असल्यामुळे ह्या आज्ञासंचाद्वारे
% मुक्त हा समरेखा टंक छायांगांसाठी निवडण्यात आला आहे. हा टंक तुमच्याकडे नसेल, तर
% \href{https://ctan.org/pkg/ektype-tanka}{एक-टाईप टंक} हा आज्ञासंच तुमच्या संगणकावर
% बसवून घ्या. मजकूर समरेखा टंकात लिहीत असाल, तर आज्ञासंचासह \textbf{समरेखा} हे प्राचल
% वापरा. त्यामुळे छायांगांकरिता शोभिका हा पारंपरिक टंक निवडला जाईल. ह्या प्राचलास किंमत
% देता येते. \verb|समरेखा=<टंकाचे नाव>| अशा प्रकारे हे प्राचल लिहिल्यास छायांगांचा टंक आपल्या
% पसंतीनुसार निवडता येतो.
% \end{function}
% 
% \bigskip
% \paragraph{महत्त्वाची सूचना:} दोन छायांगांमध्ये मोकळी जागा हवी असल्यास छायांगानंतर
% महिरपी कंस टाकण्यात यावेत. उदा. \verb|\एव{}|.
%
% \section{योगदान}
%
% ह्या आज्ञासंचात संक्षेपांची भर घालण्याकरिता गिट-प्रकल्पावर जोड-विनंत्या सादर केल्या जाऊ
% शकतात. आवृत्तिक्रमांक ०.३मध्ये सुशान्त देवळेकर ह्यांनी मराठी व्याकरणात सामान्यपणे वापरले
% जाणारे संक्षेप ह्या आज्ञासंचात समाविष्ट करण्याची जोड-विनंती २०२१/०६/१३ रोजी सादर केली व
% ह्या आज्ञासंचात काही नव्या संक्षेपांची भर पडली. अशीच व ह्या प्रकारची कोणत्याही स्वरूपातील
% जोड-विनंती ह्या प्रकल्पाकरिता सादर करण्यात येऊ शकते. उपयुक्तता व आवश्यकता पाहून प्रकल्पात
% तिचा समावेश केला जाईल.
%
% \section{आज्ञासंचातील छायांगे}
% \begin{longtable}{lll}
%   \toprule
%   छायांग & वर्णन \\
%   \midrule
%   पुं & पुल्लिंग\\
%   स्त्री & स्त्रीलिंग\\
%   नपुं & नपुंसकलिंग\\
%   १ & प्रथम व्यक्ती\\
%   २ & द्वितीय व्यक्ती\\
%   ३ & तृतीय व्यक्ती\\
%   एव & एकवचन\\ 
%   द्विव & द्विवचन\\ 
%   त्रिव & त्रिव\\
%   अव & अल्पवचन\\
%   बव & बहुवचन\\
%   अवि & अभिधानपर विभक्ती\\
%   कर्मवि & कर्मपर विभक्ती\\
%   सा & साधनपर विभक्ती\\
%   दावि & दानपर विभक्ती\\
%   वियो & वियोगपर विभक्ती\\
%   संयो & संबंधयोजक विभक्ती\\
%   अधि & अधिकरण विभक्ती\\
%   संबो & संबोधन विभक्ती\\
%   साह & साहचर्यदर्शक विभक्ती\\
%   कवि & कर्तृत्वपर विभक्ती\\
%   आवि & आगत विभक्ती\\
%   साक्रि & साहाय्यक क्रियापद\\
%   गणक & गणक\\
%   भूत & भूतकाळ\\
%   वर्त & वर्तमान काळ\\
%   भवि & भविष्यकाळ\\
%   पू & पूर्ण\\
%   अपू & अपूर्ण\\
%   नि & नित्य\\
%   अखं & अखंडित\\
%   क्र & क्रमिक\\
%   अक्र & अक्रमिक\\
%   नामि & नामिक\\
%   ना & नाम\\
%   प्राति & प्रातिपदिक\\
%   प्रत्य & प्रत्यय\\
%   सारू & सामान्य रूप\\
%   आब & आदरार्थी बहुवचन\\
%   प्र & प्रथमा\\
%   द्वि & द्वितीया\\
%   तृ & तृतीया\\
%   चतु & चतुर्थी\\
%   पं & पंचमी\\
%   ष & षष्ठी\\
%   सप्त & सप्तमी\\
%   वि & विशेषण\\
%   गोवि & गोड-गणातील विशेषण\\
%   पांवि & पांढर-गणातील विशेषण\\
%   विवि & विकारी विशेषण\\
%   अविवि & अविकारी विशेषण\\
%   धा & धातु\\
%   कृ & कृदन्त\\
%   धासा & धातुसाधित\\
%   क्रि & क्रियापद\\
%   कर्त & कर्तरी\\
%   कर्म & कर्मणि\\
%   भा & भावे\\
%   शक्य & शक्यार्थक\\
%   प्रयो & प्रयोजक\\
%   क्रिवि & क्रियाविशेषण\\
%   के & केवलप्रयोगी\\
%   शयो & शब्दयोगी\\
%   उद्गा & उद्गारवाचक\\
%   अव्य & अव्यय\\
%   \bottomrule
% \end{longtable}
% \printbibliography
% \end{documentation}
% 
% \begin{implementation}
% \section{आज्ञासंचाची घडण}
% आज्ञासंचाकरिता आवश्यक सामग्री पुढील आज्ञांद्वारे पुरवली आहे.
%    \begin{macrocode}
%<*package>
\ProvidesPackage{chhaya}[2021-12-08 v0.4 भाषावैज्ञानिक छायांगे पुरवणारा आज्ञासंच]
\RequirePackage{marathi}
\RequirePackage[acronym]{glossaries}
\RequirePackage{xkeyval}
\RequirePackage{iftex}
%    \end{macrocode}
% मराठी आज्ञासंचातर्फे मूलभूत निवडल्या जाणाऱ्या शोभिका ह्या टंकासह छापल्या जाणाऱ्या
% छायांगांकरिता एक-टाईप संस्थेचा मुक्त हा टंक पुढील आज्ञेने निवडण्यात येतो.
%    \begin{macrocode}
\दुसराटंक{\छायांगांचाटंक}{Mukta}
%    \end{macrocode}
% समरेखा हे प्राचल पुढील आज्ञावलीमार्फत पुरवले जाते.
%    \begin{macrocode}
\DeclareOptionX{समरेखा}[Shobhika]{%
  \renewfontfamily{\छायांगांचाटंक}[%
    Script=Devanagari,%
    \ifluatex
      Renderer=Harfbuzz,%
    \else
      Mapping=devanagarinumerals%
    \fi
  ]%
  {#1}%
}
\ProcessOptionsX
%    \end{macrocode}
% मजकुराकरिता टेक्-शैलीतील कार्यघटकयुक्त आज्ञा घडवण्याकरिता पुढील आज्ञा वापरण्यात येते.
%    \begin{macrocode}
\DeclareTextFontCommand{\छायांगांच्याटंकाचीआज्ञा}{\छायांगांचाटंक}
%    \end{macrocode}
% नव्या छायांगांकरिता आज्ञेची निर्मिती पुढील आज्ञेने होते.
%    \begin{macrocode}
\newcommand{\छायांग}[2]%
{%
  \newacronym{#1}{\छायांगांच्याटंकाचीआज्ञा{#1}}{#2}%
  \expandafter\newcommand\csname#1\endcsname{\acrshort{#1}}%
}
%    \end{macrocode}
% \verb|sankshep.tex| ह्या धारिकेत काही छायांगे पुरवली आहेत. त्यांना पुढील आज्ञांमुळे वापरता
% येते. 
%    \begin{macrocode}
\makeglossaries
% -----------------------------------------------------------
% Glossaries file for LaTeX package `chhaya'.
% Copyright © 2020, 2021 निरंजन
%
% This program is free software: you can redistribute it
% and/or modify it under the terms of the GNU General Public
% License as published by the Free Software Foundation,
% either version 3 of the License, or (at your option) any
% later version.
%
% This program is distributed in the hope that it will be
% useful, but WITHOUT ANY WARRANTY; without even the implied
% warranty of MERCHANTABILITY or FITNESS FOR A PARTICULAR
% PURPOSE. See the GNU General Public License for more
% details.
%
% You should have received a copy of the GNU General Public
% License along with this program. If not, see
% <https://www.gnu.org/licenses/>.
% -----------------------------------------------------------
\छायांग{पुं}{पुल्लिंग}% Masculine
\छायांग{स्त्री}{स्त्रीलिंग}% Feminine
\छायांग{नपुं}{नपुंसकलिंग}% Neuter
\छायांग{१}{प्रथम व्यक्ती}% First person
\छायांग{२}{द्वितीय व्यक्ती}% Second person
\छायांग{३}{तृतीय व्यक्ती}% Third person
\छायांग{एव}{एकवचन}% Singular
\छायांग{द्विव}{द्विवचन}% Dual
\छायांग{त्रिव}{त्रिव}% Trial 
\छायांग{अव}{अल्पवचन}% Paucal
\छायांग{बव}{बहुवचन}% Plural
\छायांग{अवि}{अभिधानपर विभक्ती}% Nominative
\छायांग{कर्मवि}{कर्मपर विभक्ती}% Accusative
\छायांग{सा}{साधनपर विभक्ती}% Instrumental
\छायांग{दावि}{दानपर विभक्ती}% Dative
\छायांग{वियो}{वियोगपर विभक्ती}% Ablative
\छायांग{संयो}{संबंधयोजक विभक्ती}% Genitive
\छायांग{अधि}{अधिकरण विभक्ती}% Locative
\छायांग{संबो}{संबोधन विभक्ती}% Vocative
\छायांग{साह}{साहचर्यदर्शक विभक्ती}% Associative
\छायांग{कवि}{कर्तृत्वपर विभक्ती}% Ergative
\छायांग{आवि}{आगत विभक्ती}% Oblique
\छायांग{साक्रि}{साहाय्यक क्रियापद}% Auxiliary
\छायांग{गणक}{गणक}% Counter
\छायांग{भूत}{भूतकाळ}% Past
\छायांग{वर्त}{वर्तमान काळ}% Present
\छायांग{भवि}{भविष्यकाळ}% Future
\छायांग{पू}{पूर्ण}% Perfective
\छायांग{अपू}{अपूर्ण}% Imperfective
\छायांग{नि}{नित्य}% Habitual
\छायांग{अखं}{अखंडित}% Continuous
\छायांग{क्र}{क्रमिक}% Progressive
\छायांग{अक्र}{अक्रमिक}% Non-progressive
% 
% मराठी व्याकरणात वापरण्यात येणाऱ्या संज्ञांची छायांगसूची. ह्यात वरच्या यादीत उपस्थित असलेली
% छायांगे टाळली आहेत. मराठी व्याकरणातील पुढील संक्षेपांची यादी २०२१/०६/१३ रोजी आज्ञासंचाच्या
% आवृत्तिक्रमांक ०.३मध्ये सुशान्त देवळेकर ह्यांनी जोडली.
% 
% नामिक ह्या गटासंदर्भातील संज्ञा
\छायांग{नामि}{नामिक}
\छायांग{ना}{नाम}
\छायांग{प्राति}{प्रातिपदिक}
\छायांग{प्रत्य}{प्रत्यय}
\छायांग{सारू}{सामान्य रूप}% Oblique form
\छायांग{आब}{आदरार्थी बहुवचन}
\छायांग{प्र}{प्रथमा}
\छायांग{द्वि}{द्वितीया}
\छायांग{तृ}{तृतीया}
\छायांग{चतु}{चतुर्थी}
\छायांग{पं}{पंचमी}
\छायांग{ष}{षष्ठी}
\छायांग{सप्त}{सप्तमी}
% 
\छायांग{वि}{विशेषण}
\छायांग{गोवि}{गोड-गणातील विशेषण}
\छायांग{पांवि}{पांढर-गणातील विशेषण}
\छायांग{विवि}{विकारी विशेषण}
\छायांग{अविवि}{अविकारी विशेषण}
% 
% 
% धातु ह्या गटाशी संबंधित संज्ञा
\छायांग{धा}{धातु}
\छायांग{कृ}{कृदन्त}
\छायांग{धासा}{धातुसाधित}
\छायांग{क्रि}{क्रियापद}
\छायांग{कर्त}{कर्तरी}
\छायांग{कर्म}{कर्मणि}
\छायांग{भा}{भावे}
\छायांग{शक्य}{शक्यार्थक}
\छायांग{प्रयो}{प्रयोजक}
% 
% क्रियाविशेषणादी गटातील संज्ञा
\छायांग{क्रिवि}{क्रियाविशेषण}
\छायांग{के}{केवलप्रयोगी}
\छायांग{शयो}{शब्दयोगी}
\छायांग{उद्गा}{उद्गारवाचक}
\छायांग{अव्य}{अव्यय}
%    \end{macrocode}
% छायांगसूची छापण्यासाठी पुढील आज्ञा समाविष्ट केली आहे.
%    \begin{macrocode}
\providecommand{\छायांगसूची}[1][छायांगसूची]{%
  \printglossary[type=\acronymtype,title={#1}]%
}
%    \end{macrocode}
%    \begin{macrocode}
%</package>
%    \end{macrocode}
% \end{implementation}
% \begin{otherlanguage*}{english}
%   {%
%     \fontfamily{lmr}\selectfont
%     ^^A -------------------------------------------------------
^^A TeX-code of the GFDL
^^A Copyright © 2020, 2021 निरंजन
^^A
^^A This file is a dependency of the documentation of the
^^A LaTeX package `chhaya'. The source file for the
^^A documentation is `chhaya.dtx'.
^^A
^^A Note that this copyright notice just applies to the
^^A TeX-code. The actual license text is a copyright of the
^^A FSF as can be viewed on the following package.
^^A
^^A https://www.gnu.org/licenses/fdl-1.3
^^A -------------------------------------------------------
% \section{\centering GNU Free Documentation License}
% \begin{center}
% 
% Version 1.3, 3 November 2008
% 
% Copyright \copyright{} 2000, 2001, 2002, 2007, 2008  Free Software Foundation, Inc.
% 
% \bigskip
% 
% \url{https://fsf.org/}
% 
% \bigskip
% 
% Everyone is permitted to copy and distribute verbatim copies
% of this license document, but changing it is not allowed.
% \end{center}
% 
% \begin{center}
%   {\bf\large Preamble}
% \end{center}
% 
% The purpose of this License is to make a manual, textbook, or other
% functional and useful document ``free'' in the sense of freedom: to
% assure everyone the effective freedom to copy and redistribute it,
% with or without modifying it, either commercially or noncommercially.
% Secondarily, this License preserves for the author and publisher a way
% to get credit for their work, while not being considered responsible
% for modifications made by others.
% 
% This License is a kind of ``copyleft'', which means that derivative
% works of the document must themselves be free in the same sense.  It
% complements the GNU General Public License, which is a copyleft
% license designed for free software.
% 
% We have designed this License in order to use it for manuals for free
% software, because free software needs free documentation: a free
% program should come with manuals providing the same freedoms that the
% software does.  But this License is not limited to software manuals;
% it can be used for any textual work, regardless of subject matter or
% whether it is published as a printed book.  We recommend this License
% principally for works whose purpose is instruction or reference.
% 
% \begin{center}
%   {\Large\bf 1. APPLICABILITY AND DEFINITIONS\par}
% \end{center}
% 
% This License applies to any manual or other work, in any medium, that
% contains a notice placed by the copyright holder saying it can be
% distributed under the terms of this License.  Such a notice grants a
% world-wide, royalty-free license, unlimited in duration, to use that
% work under the conditions stated herein.  The ``\textbf{Document}'', below,
% refers to any such manual or work.  Any member of the public is a
% licensee, and is addressed as ``\textbf{you}''.  You accept the license if you
% copy, modify or distribute the work in a way requiring permission
% under copyright law.
% 
% A ``\textbf{Modified Version}'' of the Document means any work containing the
% Document or a portion of it, either copied verbatim, or with
% modifications and/or translated into another language.
% 
% A ``\textbf{Secondary Section}'' is a named appendix or a front-matter section of
% the Document that deals exclusively with the relationship of the
% publishers or authors of the Document to the Document's overall subject
% (or to related matters) and contains nothing that could fall directly
% within that overall subject.  (Thus, if the Document is in part a
% textbook of mathematics, a Secondary Section may not explain any
% mathematics.)  The relationship could be a matter of historical
% connection with the subject or with related matters, or of legal,
% commercial, philosophical, ethical or political position regarding
% them.
% 
% The ``\textbf{Invariant Sections}'' are certain Secondary Sections whose titles
% are designated, as being those of Invariant Sections, in the notice
% that says that the Document is released under this License.  If a
% section does not fit the above definition of Secondary then it is not
% allowed to be designated as Invariant.  The Document may contain zero
% Invariant Sections.  If the Document does not identify any Invariant
% Sections then there are none.
% 
% The ``\textbf{Cover Texts}'' are certain short passages of text that are listed,
% as Front-Cover Texts or Back-Cover Texts, in the notice that says that
% the Document is released under this License.  A Front-Cover Text may
% be at most 5 words, and a Back-Cover Text may be at most 25 words.
% 
% A ``\textbf{Transparent}'' copy of the Document means a machine-readable copy,
% represented in a format whose specification is available to the
% general public, that is suitable for revising the document
% straightforwardly with generic text editors or (for images composed of
% pixels) generic paint programs or (for drawings) some widely available
% drawing editor, and that is suitable for input to text formatters or
% for automatic translation to a variety of formats suitable for input
% to text formatters.  A copy made in an otherwise Transparent file
% format whose markup, or absence of markup, has been arranged to thwart
% or discourage subsequent modification by readers is not Transparent.
% An image format is not Transparent if used for any substantial amount
% of text.  A copy that is not ``Transparent'' is called ``\textbf{Opaque}''.
% 
% Examples of suitable formats for Transparent copies include plain
% ASCII without markup, Texinfo input format, LaTeX input format, SGML
% or XML using a publicly available DTD, and standard-conforming simple
% HTML, PostScript or PDF designed for human modification.  Examples of
% transparent image formats include PNG, XCF and JPG.  Opaque formats
% include proprietary formats that can be read and edited only by
% proprietary word processors, SGML or XML for which the DTD and/or
% processing tools are not generally available, and the
% machine-generated HTML, PostScript or PDF produced by some word
% processors for output purposes only.
% 
% The ``\textbf{Title Page}'' means, for a printed book, the title page itself,
% plus such following pages as are needed to hold, legibly, the material
% this License requires to appear in the title page.  For works in
% formats which do not have any title page as such, ``Title Page'' means
% the text near the most prominent appearance of the work's title,
% preceding the beginning of the body of the text.
% 
% The ``\textbf{publisher}'' means any person or entity that distributes
% copies of the Document to the public.
% 
% A section ``\textbf{Entitled XYZ}'' means a named subunit of the Document whose
% title either is precisely XYZ or contains XYZ in parentheses following
% text that translates XYZ in another language.  (Here XYZ stands for a
% specific section name mentioned below, such as ``\textbf{Acknowledgements}'',
% ``\textbf{Dedications}'', ``\textbf{Endorsements}'', or ``\textbf{History}''.)  
% To ``\textbf{Preserve the Title}''
% of such a section when you modify the Document means that it remains a
% section ``Entitled XYZ'' according to this definition.
% 
% The Document may include Warranty Disclaimers next to the notice which
% states that this License applies to the Document.  These Warranty
% Disclaimers are considered to be included by reference in this
% License, but only as regards disclaiming warranties: any other
% implication that these Warranty Disclaimers may have is void and has
% no effect on the meaning of this License.
% 
% \begin{center}
%   {\Large\bf 2. VERBATIM COPYING\par}
% \end{center}
% 
% You may copy and distribute the Document in any medium, either
% commercially or noncommercially, provided that this License, the
% copyright notices, and the license notice saying this License applies
% to the Document are reproduced in all copies, and that you add no other
% conditions whatsoever to those of this License.  You may not use
% technical measures to obstruct or control the reading or further
% copying of the copies you make or distribute.  However, you may accept
% compensation in exchange for copies.  If you distribute a large enough
% number of copies you must also follow the conditions in section~3.
% 
% You may also lend copies, under the same conditions stated above, and
% you may publicly display copies.
% 
% \begin{center}
%   {\Large\bf 3. COPYING IN QUANTITY\par}
% \end{center}
% 
% If you publish printed copies (or copies in media that commonly have
% printed covers) of the Document, numbering more than 100, and the
% Document's license notice requires Cover Texts, you must enclose the
% copies in covers that carry, clearly and legibly, all these Cover
% Texts: Front-Cover Texts on the front cover, and Back-Cover Texts on
% the back cover.  Both covers must also clearly and legibly identify
% you as the publisher of these copies.  The front cover must present
% the full title with all words of the title equally prominent and
% visible.  You may add other material on the covers in addition.
% Copying with changes limited to the covers, as long as they preserve
% the title of the Document and satisfy these conditions, can be treated
% as verbatim copying in other respects.
% 
% If the required texts for either cover are too voluminous to fit
% legibly, you should put the first ones listed (as many as fit
% reasonably) on the actual cover, and continue the rest onto adjacent
% pages.
% 
% If you publish or distribute Opaque copies of the Document numbering
% more than 100, you must either include a machine-readable Transparent
% copy along with each Opaque copy, or state in or with each Opaque copy
% a computer-network location from which the general network-using
% public has access to download using public-standard network protocols
% a complete Transparent copy of the Document, free of added material.
% If you use the latter option, you must take reasonably prudent steps,
% when you begin distribution of Opaque copies in quantity, to ensure
% that this Transparent copy will remain thus accessible at the stated
% location until at least one year after the last time you distribute an
% Opaque copy (directly or through your agents or retailers) of that
% edition to the public.
% 
% It is requested, but not required, that you contact the authors of the
% Document well before redistributing any large number of copies, to give
% them a chance to provide you with an updated version of the Document.
% 
% 
% \begin{center}
%   {\Large\bf 4. MODIFICATIONS\par}
% \end{center}
% 
% You may copy and distribute a Modified Version of the Document under
% the conditions of sections 2 and 3 above, provided that you release
% the Modified Version under precisely this License, with the Modified
% Version filling the role of the Document, thus licensing distribution
% and modification of the Modified Version to whoever possesses a copy
% of it.  In addition, you must do these things in the Modified Version:
% 
% \begin{itemize}
% \item[A.] 
%   Use in the Title Page (and on the covers, if any) a title distinct
%   from that of the Document, and from those of previous versions
%   (which should, if there were any, be listed in the History section
%   of the Document).  You may use the same title as a previous version
%   if the original publisher of that version gives permission.
%   
% \item[B.]
%   List on the Title Page, as authors, one or more persons or entities
%   responsible for authorship of the modifications in the Modified
%   Version, together with at least five of the principal authors of the
%   Document (all of its principal authors, if it has fewer than five),
%   unless they release you from this requirement.
%   
% \item[C.]
%   State on the Title page the name of the publisher of the
%   Modified Version, as the publisher.
%   
% \item[D.]
%   Preserve all the copyright notices of the Document.
%   
% \item[E.]
%   Add an appropriate copyright notice for your modifications
%   adjacent to the other copyright notices.
%   
% \item[F.]
%   Include, immediately after the copyright notices, a license notice
%   giving the public permission to use the Modified Version under the
%   terms of this License, in the form shown in the Addendum below.
%   
% \item[G.]
%   Preserve in that license notice the full lists of Invariant Sections
%   and required Cover Texts given in the Document's license notice.
%   
% \item[H.]
%   Include an unaltered copy of this License.
%   
% \item[I.]
%   Preserve the section Entitled ``History'', Preserve its Title, and add
%   to it an item stating at least the title, year, new authors, and
%   publisher of the Modified Version as given on the Title Page.  If
%   there is no section Entitled ``History'' in the Document, create one
%   stating the title, year, authors, and publisher of the Document as
%   given on its Title Page, then add an item describing the Modified
%   Version as stated in the previous sentence.
%   
% \item[J.]
%   Preserve the network location, if any, given in the Document for
%   public access to a Transparent copy of the Document, and likewise
%   the network locations given in the Document for previous versions
%   it was based on.  These may be placed in the ``History'' section.
%   You may omit a network location for a work that was published at
%   least four years before the Document itself, or if the original
%   publisher of the version it refers to gives permission.
%   
% \item[K.]
%   For any section Entitled ``Acknowledgements'' or ``Dedications'',
%   Preserve the Title of the section, and preserve in the section all
%   the substance and tone of each of the contributor acknowledgements
%   and/or dedications given therein.
%   
% \item[L.]
%   Preserve all the Invariant Sections of the Document,
%   unaltered in their text and in their titles.  Section numbers
%   or the equivalent are not considered part of the section titles.
%   
% \item[M.]
%   Delete any section Entitled ``Endorsements''.  Such a section
%   may not be included in the Modified Version.
%   
% \item[N.]
%   Do not retitle any existing section to be Entitled ``Endorsements''
%   or to conflict in title with any Invariant Section.
%   
% \item[O.]
%   Preserve any Warranty Disclaimers.
% \end{itemize}
% 
% If the Modified Version includes new front-matter sections or
% appendices that qualify as Secondary Sections and contain no material
% copied from the Document, you may at your option designate some or all
% of these sections as invariant.  To do this, add their titles to the
% list of Invariant Sections in the Modified Version's license notice.
% These titles must be distinct from any other section titles.
% 
% You may add a section Entitled ``Endorsements'', provided it contains
% nothing but endorsements of your Modified Version by various
% parties---for example, statements of peer review or that the text has
% been approved by an organization as the authoritative definition of a
% standard.
% 
% You may add a passage of up to five words as a Front-Cover Text, and a
% passage of up to 25 words as a Back-Cover Text, to the end of the list
% of Cover Texts in the Modified Version.  Only one passage of
% Front-Cover Text and one of Back-Cover Text may be added by (or
% through arrangements made by) any one entity.  If the Document already
% includes a cover text for the same cover, previously added by you or
% by arrangement made by the same entity you are acting on behalf of,
% you may not add another; but you may replace the old one, on explicit
% permission from the previous publisher that added the old one.
% 
% The author(s) and publisher(s) of the Document do not by this License
% give permission to use their names for publicity for or to assert or
% imply endorsement of any Modified Version.
% 
% \begin{center}
%   {\Large\bf 5. COMBINING DOCUMENTS\par}
% \end{center}
% 
% You may combine the Document with other documents released under this
% License, under the terms defined in section~4 above for modified
% versions, provided that you include in the combination all of the
% Invariant Sections of all of the original documents, unmodified, and
% list them all as Invariant Sections of your combined work in its
% license notice, and that you preserve all their Warranty Disclaimers.
% 
% The combined work need only contain one copy of this License, and
% multiple identical Invariant Sections may be replaced with a single
% copy.  If there are multiple Invariant Sections with the same name but
% different contents, make the title of each such section unique by
% adding at the end of it, in parentheses, the name of the original
% author or publisher of that section if known, or else a unique number.
% Make the same adjustment to the section titles in the list of
% Invariant Sections in the license notice of the combined work.
% 
% In the combination, you must combine any sections Entitled ``History''
% in the various original documents, forming one section Entitled
% ``History''; likewise combine any sections Entitled ``Acknowledgements'',
% and any sections Entitled ``Dedications''.  You must delete all sections
% Entitled ``Endorsements''.
% 
% \begin{center}
%   {\Large\bf 6. COLLECTIONS OF DOCUMENTS\par}
% \end{center}
% 
% You may make a collection consisting of the Document and other documents
% released under this License, and replace the individual copies of this
% License in the various documents with a single copy that is included in
% the collection, provided that you follow the rules of this License for
% verbatim copying of each of the documents in all other respects.
% 
% You may extract a single document from such a collection, and distribute
% it individually under this License, provided you insert a copy of this
% License into the extracted document, and follow this License in all
% other respects regarding verbatim copying of that document.
% 
% \begin{center}
%   {\Large\bf 7. AGGREGATION WITH INDEPENDENT WORKS\par}
% \end{center}
% 
% A compilation of the Document or its derivatives with other separate
% and independent documents or works, in or on a volume of a storage or
% distribution medium, is called an ``aggregate'' if the copyright
% resulting from the compilation is not used to limit the legal rights
% of the compilation's users beyond what the individual works permit.
% When the Document is included in an aggregate, this License does not
% apply to the other works in the aggregate which are not themselves
% derivative works of the Document.
% 
% If the Cover Text requirement of section~3 is applicable to these
% copies of the Document, then if the Document is less than one half of
% the entire aggregate, the Document's Cover Texts may be placed on
% covers that bracket the Document within the aggregate, or the
% electronic equivalent of covers if the Document is in electronic form.
% Otherwise they must appear on printed covers that bracket the whole
% aggregate.
% 
% \begin{center}
%   {\Large\bf 8. TRANSLATION\par}
% \end{center}
% 
% Translation is considered a kind of modification, so you may
% distribute translations of the Document under the terms of section~4.
% Replacing Invariant Sections with translations requires special
% permission from their copyright holders, but you may include
% translations of some or all Invariant Sections in addition to the
% original versions of these Invariant Sections.  You may include a
% translation of this License, and all the license notices in the
% Document, and any Warranty Disclaimers, provided that you also include
% the original English version of this License and the original versions
% of those notices and disclaimers.  In case of a disagreement between
% the translation and the original version of this License or a notice
% or disclaimer, the original version will prevail.
% 
% If a section in the Document is Entitled ``Acknowledgements'',
% ``Dedications'', or ``History'', the requirement (section~4) to Preserve
% its Title (section~1) will typically require changing the actual
% title.
% 
% \begin{center}
%   {\Large\bf 9. TERMINATION\par}
% \end{center}
% 
% You may not copy, modify, sublicense, or distribute the Document
% except as expressly provided under this License.  Any attempt
% otherwise to copy, modify, sublicense, or distribute it is void, and
% will automatically terminate your rights under this License.
% 
% However, if you cease all violation of this License, then your license
% from a particular copyright holder is reinstated (a) provisionally,
% unless and until the copyright holder explicitly and finally
% terminates your license, and (b) permanently, if the copyright holder
% fails to notify you of the violation by some reasonable means prior to
% 60 days after the cessation.
% 
% Moreover, your license from a particular copyright holder is
% reinstated permanently if the copyright holder notifies you of the
% violation by some reasonable means, this is the first time you have
% received notice of violation of this License (for any work) from that
% copyright holder, and you cure the violation prior to 30 days after
% your receipt of the notice.
% 
% Termination of your rights under this section does not terminate the
% licenses of parties who have received copies or rights from you under
% this License.  If your rights have been terminated and not permanently
% reinstated, receipt of a copy of some or all of the same material does
% not give you any rights to use it.
% 
% \begin{center}
%   {\Large\bf 10. FUTURE REVISIONS OF THIS LICENSE\par}
% \end{center}
% 
% The Free Software Foundation may publish new, revised versions
% of the GNU Free Documentation License from time to time.  Such new
% versions will be similar in spirit to the present version, but may
% differ in detail to address new problems or concerns.  See
% \texttt{https://www.gnu.org/licenses/}.
% 
% Each version of the License is given a distinguishing version number.
% If the Document specifies that a particular numbered version of this
% License ``or any later version'' applies to it, you have the option of
% following the terms and conditions either of that specified version or
% of any later version that has been published (not as a draft) by the
% Free Software Foundation.  If the Document does not specify a version
% number of this License, you may choose any version ever published (not
% as a draft) by the Free Software Foundation.  If the Document
% specifies that a proxy can decide which future versions of this
% License can be used, that proxy's public statement of acceptance of a
% version permanently authorizes you to choose that version for the
% Document.
% 
% \begin{center}
%   {\Large\bf 11. RELICENSING\par}
% \end{center}
% 
% ``Massive Multiauthor Collaboration Site'' (or ``MMC Site'') means any
% World Wide Web server that publishes copyrightable works and also
% provides prominent facilities for anybody to edit those works.  A
% public wiki that anybody can edit is an example of such a server.  A
% ``Massive Multiauthor Collaboration'' (or ``MMC'') contained in the
% site means any set of copyrightable works thus published on the MMC
% site.
% 
% ``CC-BY-SA'' means the Creative Commons Attribution-Share Alike 3.0
% license published by Creative Commons Corporation, a not-for-profit
% corporation with a principal place of business in San Francisco,
% California, as well as future copyleft versions of that license
% published by that same organization.
% 
% ``Incorporate'' means to publish or republish a Document, in whole or
% in part, as part of another Document.
% 
% An MMC is ``eligible for relicensing'' if it is licensed under this
% License, and if all works that were first published under this License
% somewhere other than this MMC, and subsequently incorporated in whole
% or in part into the MMC, (1) had no cover texts or invariant sections,
% and (2) were thus incorporated prior to November 1, 2008.
% 
% The operator of an MMC Site may republish an MMC contained in the site
% under CC-BY-SA on the same site at any time before August 1, 2009,
% provided the MMC is eligible for relicensing.
% 
% \begin{center}
%   {\Large\bf ADDENDUM: How to use this License for your documents\par}
% \end{center}
% 
% To use this License in a document you have written, include a copy of
% the License in the document and put the following copyright and
% license notices just after the title page:
% 
% \bigskip
% \begin{quote}
%   Copyright \copyright{}  YEAR  YOUR NAME.
%   Permission is granted to copy, distribute and/or modify this document
%   under the terms of the GNU Free Documentation License, Version 1.3
%   or any later version published by the Free Software Foundation;
%   with no Invariant Sections, no Front-Cover Texts, and no Back-Cover Texts.
%   A copy of the license is included in the section entitled ``GNU
%   Free Documentation License''.
% \end{quote}
% \bigskip
% 
% If you have Invariant Sections, Front-Cover Texts and Back-Cover Texts,
% replace the ``with \dots\ Texts.''\ line with this:
% 
% \bigskip
% \begin{quote}
%   with the Invariant Sections being LIST THEIR TITLES, with the
%   Front-Cover Texts being LIST, and with the Back-Cover Texts being LIST.
% \end{quote}
% \bigskip
% 
% If you have Invariant Sections without Cover Texts, or some other
% combination of the three, merge those two alternatives to suit the
% situation.
% 
% If your document contains nontrivial examples of program code, we
% recommend releasing these examples in parallel under your choice of
% free software license, such as the GNU General Public License,
% to permit their use in free software.%
%   }
% \end{otherlanguage*}
% \Finale