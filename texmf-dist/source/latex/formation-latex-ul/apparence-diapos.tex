%%% Copyright (C) 2015-2024 Vincent Goulet
%%%
%%% Ce fichier fait partie du projet
%%% «Rédaction avec LaTeX»
%%% https://gitlab.com/vigou3/formation-latex-ul
%%%
%%% Cette création est mise à disposition sous licence
%%% Attribution-Partage dans les mêmes conditions 4.0
%%% International de Creative Commons.
%%% https://creativecommons.org/licenses/by-sa/4.0/

\section{Apparence et disposition du texte}

\begin{frame}
  \frametitle{Police de caractères}

  Par défaut, {\LaTeX} compose les documents dans la police
  {\CM Computer Modern}.

  \begin{itemize}
  \item Dans un premier temps, n'essayez pas de changer la police de
    caractère du document
  \item Commandes pour modifier les \alert{attributs} de la police
    (famille, forme, graisse)
    \medskip

    par ex.:
    {
      \small
      \begin{tabular}[t]{l}
        \cs{rmfamily} \\ \CM romain
      \end{tabular} \quad
      \begin{tabular}[t]{l}
        \cs{ttfamily} \\ \CMtt largeur fixe
      \end{tabular} \quad
      \begin{tabular}[t]{l}
        \cs{itshape} \\ \CM\itshape italique
      \end{tabular} \quad
      \begin{tabular}[t]{l}
        \cs{bfseries} \\ \CM\bfseries gras
      \end{tabular}
    }
  \item Commandes pour modifier la \alert{taille} du texte
    \medskip

    par ex.:
    {
      \small
      \begin{tabular}[t]{l}
        \cs{footnotesize} \\ \CM\footnotesize très petit
      \end{tabular} \quad
      \begin{tabular}[t]{l}
        \cs{small} \\ \CM\small petit
      \end{tabular} \quad
      \begin{tabular}[t]{l}
        \cs{large} \\ \CM\large grand
      \end{tabular} \quad
      \begin{tabular}[t]{l}
        \cs{Large} \\ \CM\Large très grand
      \end{tabular}
    }
  \end{itemize}
\end{frame}

\begin{frame}[fragile=singleslide]
  \frametitle{Italique}
  \begin{itemize}
  \item Une des propriétés les plus utilisées dans le texte
  \item Commande sémantique:
\begin{lstlisting}
\emph`\marg{texte}'
\end{lstlisting}
  \item Pas de commande pour souligner en {\LaTeX\dots} et ce n'est
    pas une omission!
  \end{itemize}
\end{frame}

\begin{frame}[fragile]
  \frametitle{Listes}
  \begin{itemize}
  \item Deux principales sortes de listes:
    \begin{enumerate}
    \item \alert{à puce} avec environnement \code{itemize}
    \item \alert{numérotée} avec environnement \code{enumerate}
    \end{enumerate}
  \item Possible de les imbriquer les unes dans les autres
  \item Marqueurs adaptés automatiquement jusqu'à 4 niveaux
  \pause

\begin{lstlisting}
\begin{itemize}
\item Deux principales sortes de listes:
  \begin{enumerate}
  \item à puce avec environnement \texttt{itemize}
  \item numérotée avec environnement \texttt{enumerate}
  \end{enumerate}
\item Possible de les imbriquer les unes dans les autres
\item Marqueurs adaptés automatiquement jusqu'à 4 niveaux
\end{itemize}
\end{lstlisting}
  \end{itemize}
\end{frame}

\begin{frame}[fragile]
  \frametitle{Notes de bas de page}
  \begin{itemize}
  \item Note de bas de page insérée avec la commande
\begin{lstlisting}
\footnote`\marg{texte de la note}'
\end{lstlisting}
  \item Commande doit suivre immédiatement le texte à annoter
  \item Numérotation et disposition automatiques
  \end{itemize}
\end{frame}

\begin{frame}[fragile=singleslide]
  \frametitle{Code source}
  \begin{itemize}
  \item Environnement \code{verbatim}
\begin{lstlisting}
\begin{verbatim}
Texte disposé exactement tel qu'il est saisi
dans une police à largeur fixe
\end{verbatim}
\end{lstlisting}
  \item Pour usage plus intensif, utiliser les paquetages
    \pkg{fancyvrb} ou \pkg{listings}
  \end{itemize}
\end{frame}

\begin{frame}[plain]
  \tipbox{Il est aujourd'hui beaucoup plus facile d'utiliser d'autres
    polices de caractères pour vos documents, surtout avec {\XeLaTeX}.


    Attention, toutefois: peu de polices sont adaptées pour les
    mathématiques.

    Excellents choix modernes: %
    \link{https://ctan.org/pkg/stix2-otf/}{\stixtwo STIX~Two}, %
    \link{https://ctan.org/pkg/fira}{Fira Sans}.}
\end{frame}

\begin{exercice}
  Utiliser le fichier \fichier{exercice-complet.tex}.

  \begin{enumerate}
  \item Étudier le code source du fichier, puis le compiler.
  \item Supprimer l'option \code{article} au chargement de la classe
    et compiler de nouveau le document. Observer l'effet de cette
    option.
  \item Effectuer les modifications suivantes au document.
    \begin{enumerate}[a)]
    \item Dernier paragraphe de la première section, placer toute la
      phrase débutant par \code{«De simple dérivé»} à l'intérieur
      d'une commande \cs{emph}.
    \item Changer la puce des listes pour le symbole \code{\$>\$} en
      activant la commande \cs{frenchbsetup\{ItemLabeli=\$>\$\}} dans
      le préambule.
    \end{enumerate}
  \end{enumerate}
\end{exercice}

%%% Local Variables:
%%% TeX-master: "formation-latex-ul-diapos"
%%% TeX-engine: xetex
%%% coding: utf-8
%%% End:
