%%% Copyright (C) 2015-2023 Vincent Goulet
%%%
%%% Ce fichier fait partie du projet
%%% «Rédaction avec LaTeX»
%%% https://gitlab.com/vigou3/formation-latex-ul
%%%
%%% Cette création est mise à disposition sous licence
%%% Attribution-Partage dans les mêmes conditions 4.0
%%% International de Creative Commons.
%%% https://creativecommons.org/licenses/by-sa/4.0/

\chapter{Principes de base}
\label{chap:bases}

Un système de mise en page tel que {\LaTeX} repose sur une logique de
séparation entre l'apparence d'un document et sa structure. Si vous
avez l'habitude d'utiliser un traitement de texte, vous devrez fort
probablement vous défaire d'une vilaine habitude: vous préoccuper sans
cesse, au moment de la rédaction, de la disposition du texte.

Ce principe accepté, il vous faudra néanmoins indiquer au logiciel la
structure du document. Avec {\LaTeX} cela s'effectue par le biais de
diverses instructions que l'on insère au fil du texte. À la base, les
logiciels de traitements de texte n'opèrent pas différemment, sauf
qu'ils cachent les codes aux utilisateurs\footnote{%
  Le leader du traitement de texte jusqu'au milieu des années 1990,
  Word~Perfect, offrait l'option d'afficher l'ensemble des codes de
  mise en page. C'est malheureusement une caractéristique brillante
  que Microsoft Word et les autres progiciels développés depuis ont
  choisi d'omettre.}. %
Sous prétexte de simplicité d'utilisation, ils causent en fait bien
des mots de tête. Qui ne s'est pas déjà demandé, en utilisant un
traitement de texte moderne: «Pourquoi donc mon texte est-il
soudainement en gras?»

Ce chapitre explique comment aborder la rédaction d'un document avec
{\LaTeX} ainsi que la syntaxe de base des différents types
d'instructions que l'on peut insérer dans un texte pour en spécifier
la structure et la mise en forme.


\section{Séparation du contenu et de l'apparence}
\label{sec:bases:separation}

Lors de la rédaction avec un système de mise en page tel que {\LaTeX},
on se concentre sur le contenu et la \emph{structure} du document, et
non pas sur son \emph{apparence}. Par exemple:
\begin{itemize}
\item au lieu de prescrire qu'un titre de section doit être en gras
  14~points, on indique simplement à {\LaTeX} que le texte doit être
  traité comme un titre de section;
  \begin{demo}
    \begin{minipage}{0.45\linewidth}
\begin{lstlisting}
\textbf{\large Titre}
\end{lstlisting}
    \end{minipage}
    \hfill \faArrowRight \hfill
    \begin{minipage}{0.45\linewidth}
\begin{lstlisting}
\section{Titre}
\end{lstlisting}
    \end{minipage}
  \end{demo}
\item au lieu de décider qu'un mot sur lequel l'on souhaite insister
  sera en italique, on indique à {\LaTeX} de mettre de l'emphase sur
  ce mot sans se soucier de la mise en forme.
  \begin{demo}
    \begin{minipage}{0.45\linewidth}
\begin{lstlisting}
\textit{texte}
\end{lstlisting}
    \end{minipage}
    \hfill \faArrowRight \hfill
    \begin{minipage}{0.45\linewidth}
\begin{lstlisting}
\emph{texte}
\end{lstlisting}
    \end{minipage}
  \end{demo}
\end{itemize}

L'apparence du texte sera prise en charge par {\LaTeX}. Comme les
gabarits sont l'{\oe}uvre de spécialistes en typographie, il est
généralement préférable de ne pas les modifier. À titre d'exemple,
{\LaTeX} détermine automatiquement la largeur des marges en fonction
de la taille de la police de manière à ce que les lignes de texte
comptent approximativement 70~caractères. La raison: lorsqu'une ligne
de texte est trop longue, notre {\oe}il a de la difficulté à la suivre
sur toute sa longueur. Il a tendance à passer à la ligne inférieure,
ce qui rend la lecture plus difficile.


\section{Règles de saisie}
\label{sec:bases:saisie}

Une fois le principe de séparation du contenu et de l'apparence
compris et accepté, veillez à respecter les règles simples suivantes
lors de la saisie du texte.
\begin{enumerate}
\item On sépare les mots par une ou plusieurs \emph{espaces}. Qu'il y
  en ait une ou un millier, seule la première compte et la mise en
  page sera la même.
  \begin{demo}
    \begin{texample}
\begin{lstlisting}
Les espaces délimitent les
mots. Leur nombre n'a pas
d'importance.
\end{lstlisting}
      \producing
      Les espaces délimitent les
      mots. Leur nombre n'a pas
      d'importance.
    \end{texample}
    \begin{texample}
\begin{lstlisting}[showstringspaces=true]
Les  espaces   délimitent
les                 mots.
Leur    nombre  n'a pas
d'importance.
\end{lstlisting}
      \producing
      Les  espaces   délimitent
      les                 mots.
      Leur    nombre  n'a pas
      d'importance.
    \end{texample}
  \end{demo}
%
\item On sépare les paragraphes par une ou plusieurs lignes blanches.
  Celles-ci n'apparaitront pas nécessairement dans le texte final; les
  gabarits standards identifient les paragraphes par un retrait de
  première ligne.
  \begin{demo}
    \begin{texample}
\begin{lstlisting}
Les lignes blanches
délimitent les paragraphes.

Une ou plusieurs, ça ne fait
aucune différence!
\end{lstlisting}
      \producing
Les lignes blanches
délimitent les
paragraphes.

        Une ou plusieurs, ça ne fait aucune différence!
    \end{texample}
\begin{texample}
\begin{lstlisting}
Les lignes blanches
délimitent les paragraphes.



Une ou plusieurs, ça ne fait
aucune différence!
\end{lstlisting}
      \producing
        Les lignes blanches  délimitent
        les paragraphes.



        Une ou plusieurs, ça ne fait aucune différence!
    \end{texample}
  \end{demo}
%
\item On utilise des \emph{commandes} pour indiquer la structure du
  texte dans le texte. Celles-ci débutent presque toujours par
  le symbole «{\bs}». À la différence des logiciels de traitement de
  texte, les instructions de mise en forme du document sont donc
  toujours visibles et, par conséquent, modifiables facilement et sans
  surprise (on ne se demande donc jamais où se termine le gras).
  \begin{demo}
    \begin{texample}
\begin{lstlisting}
Les commandes sont visibles
dans le \textbf{texte},
mais évidemment pas dans le
\emph{document} fini.
\end{lstlisting}
      \producing
      Les commandes sont visibles
      dans le \textbf{texte},
      mais évidemment pas dans le
      \emph{document} fini.
    \end{texample}
  \end{demo}
\end{enumerate}


\section{Structure d'un fichier}
\label{sec:bases:structure}

Un fichier source {\LaTeX} --- dont vous trouverez un exemple simple à la
\autoref{fig:bases:parties} --- est toujours composé de deux parties:
le préambule et le corps du document.

\begin{figure}
  \centering
  \begin{minipage}{0.75\linewidth}
\begin{lstlisting}[numbers=left, numberstyle=\tiny]
\documentclass[11pt,french]{article} `\label{lst:bases:preambule_debut}'
  \usepackage{babel}
  \usepackage[autolanguage]{numprint}
  \usepackage[utf8]{inputenc}
  \usepackage[T1]{fontenc} `\label{lst:bases:preambule_fin}'

\begin{document} `\label{lst:bases:corps_debut}'

Lorem ipsum dolor sit amet, consectetur
adipiscing elit. Donec quam nulla, bibendum
vitae ipsum vel, fermentum pellentesque
orci.

\end{document} `\label{lst:bases:corps_fin}'
\end{lstlisting}
  \end{minipage}
  \caption[Fichier source {\LaTeX} simple comportant les deux parties
  obligatoires: le préambule et le corps du document]{%
    Fichier source {\LaTeX} simple comportant les deux parties
    obligatoires: le préambule (lignes
    \ref*{lst:bases:preambule_debut}--\ref*{lst:bases:preambule_fin})
    et le corps du document (lignes
    \ref*{lst:bases:corps_debut}--\ref*{lst:bases:corps_fin}).}
  \label{fig:bases:parties}
\end{figure}

\begin{description}
\item[Préambule] Suite de commandes spécifiant la mise en forme
  globale du document (format du papier, marges, entête et pied de
  page, etc.). Il contient au minimum la commande
  \cmd{\documentclass}. Les commandes contenues dans le préambule ont
  un effet global sur le document. Les lignes
  \ref*{lst:bases:preambule_debut}--\ref*{lst:bases:preambule_fin}
  forment le préambule dans l'exemple de la
  \autoref{fig:bases:parties}.
\item[Corps du document] Contenu du document en tant que tel. Il
  débute par \verb=\begin{document}= et se termine par
    \verb=\end{document}=. Le corps du document peut aussi contenir
  des commandes, mais l'effet de celles-ci demeure presque toujours
  local. Les lignes
  \ref*{lst:bases:corps_debut}--\ref*{lst:bases:corps_fin} du code de
  la \autoref{fig:bases:parties} forment le corps du document.
\end{description}


\section{Classes et paquetages}
\label{sec:bases:classes}

La première commande du préambule est normalement la déclaration de la
\emph{classe} du document. La forme de la déclaration est la suivante:
\begin{lstlisting}
\documentclass`\oarg{options}\marg{classe}'
\end{lstlisting}
Les classes standards de {\LaTeX} sont \class{article},
\class{report}, \class{book}, \class{letter} et \class{slides}. La
\autoref{sec:organisation:classe} traite des différences entre les
diverses classes et des \meta{options} disponibles.

Les \emph{paquetages} permettent de modifier des commandes ou
d'ajouter des fonctionnalités à {\LaTeX}. On charge les paquetages
dans le préambule avec des commandes de la forme
\begin{lstlisting}
\usepackage`\marg{paquetage}'
\usepackage`\oarg{options}\marg{paquetage}'
\usepackage`\marg{paquetage1,paquetage2, ...}'
\end{lstlisting}
La première et la troisième forme permettent de charger un ou
plusieurs paquetages sans options. La seconde permet de spécifier des
\meta{options} au chargement du paquetage. Il n'est évidemment pas
possible de préciser des options avec la troisième forme puisque
{\LaTeX} ne saurait à quel paquetage celles-ci se rapportent.

Certains paquetages permettent que leurs options apparaissent parmi
les \meta{options} de la commande \cmdprint{\documentclass}. Elles
sont ainsi plus «visibles» pour d'autres paquetages. Par exemple,
l'option \code{french} que l'on retrouve dans la déclaration de la
classe à la \autoref{fig:bases:parties} est en fait une option du
paquetage \pkg{babel}.


\section{Commandes}
\label{sec:bases:commandes}

J'ai déjà fait référence à quelques reprises au concept de commande
{\LaTeX}. Cette section se penche sur leur syntaxe.

Les formes générales des commandes {\LaTeX} sont:
\begin{lstlisting}
\`\meta{nomcommande}\oarg{arg\_optionnel}\marg{arg\_obligatoire}'
\`\meta{nomcommande}'*`\oarg{arg\_optionnel}\marg{arg\_obligatoire}'
\end{lstlisting}
Ici, \meta{nomcommande} est le nom de la commande. Il débute par le
symbole «{\bs}» et il est exclusivement formé de lettres,
habituellement des minuscules ({\LaTeX} est sensible à la casse). La
forme étoilée d'une commande réalise généralement une action
légèrement différente de la version sans étoile. Par exemple, la
commande \cmd{\section} crée une nouvelle section numérotée, alors que
\cmd{\section*} n'insère aucune numérotation.

Lorsque la commande accepte des arguments, les arguments obligatoires
sont placés entre accolades \verb={ }= et les arguments optionnels
sont placés entre crochets \verb=[ ]=.

Certaines commandes n'ont aucun argument. Leur forme est alors
\begin{lstlisting}
\`\meta{nomcommande}
\end{lstlisting}
Dans ce cas, le nom de la commande se termine par tout symbole qui
n'est pas une lettre --- y compris l'espace! Cette règle fait en sorte
qu'une espace après le nom d'une commande est considérée comme un
marqueur de la fin du nom de la commande. Cette règle joue parfois de
vilains tours en «avalant» l'espace entre une commande et le mot qui
suit; voir l'\autoref{exemple:base:commandes} et
l'\autoref{ex:base:commandes}.

La portée d'une commande est limitée à la zone entre accolades
\verb={ }=, le cas échéant.

\begin{exemple}
  \label{exemple:base:commandes}
  \enlargethispage{5mm}
  Voici trois exemples de commandes {\LaTeX}: une sans argument, une
  avec un seul argument obligatoire et une commande avec deux
  arguments obligatoires et un argument optionnel.
  \begin{enumerate}
  \item La commande \cmd{\LaTeX} permet de composer de logo {\LaTeX}.
    \begin{demo}
      \begin{texample}
\begin{lstlisting}
Apprendre \LaTeX c'est
formidable!
\end{lstlisting}
        \producing
        Apprendre \LaTeX c'est
        formidable!
      \end{texample}
    \end{demo}
    Vous pouvez constater ici que l'espace suivant le nom de la
    commande a été interprétée par {\LaTeX} comme un marqueur de la
    fin de la commande et qu'elle a été supprimée du texte. Deux
    possibilités pour contourner cette particularité du langage:
    fournir un argument vide à la commande, ou placer celle-ci entre
    accolades pour limiter sa portée à elle-même:
    \begin{demo}
      \begin{texample}
\begin{lstlisting}
Apprendre \LaTeX{} c'est
formidable!
\end{lstlisting}
        \producing
        Apprendre \LaTeX{} c'est
        formidable!
      \end{texample}
      \begin{texample}
\begin{lstlisting}
Apprendre {\LaTeX} c'est
formidable!
\end{lstlisting}
        \producing
        Apprendre {\LaTeX} c'est
        formidable!
      \end{texample}
    \end{demo}
    %
  \item La commande \cmd{\emph} met de l'emphase (en général sous
    forme d'italique) sur le ou les mots en argument.
    \begin{demo}
      \begin{texample}
\begin{lstlisting}
Il est \emph{essentiel} de
connaitre la syntaxe de
{\LaTeX}.
\end{lstlisting}
        \producing
        Il est \emph{essentiel} de
        connaitre la syntaxe de
        {\LaTeX}.
      \end{texample}
    \end{demo}
    %
  \item La commande \cmd{\rule} produit un rectangle plein. Elle a
    deux arguments obligatoires: la longueur et la hauteur du
    rectangle, dans l'ordre. Un argument optionnel permet de surélever
    le rectangle au-dessus de la ligne de base (voir le
    \autoref{chap:boites} pour plus de détails).
    \begin{demo}
      \begin{texample}
\begin{lstlisting}
Réglure de 1~cm de long et
3~mm d'épais surélevée de
2~points au-dessus de la
ligne de base:
\rule[2pt]{1cm}{3mm}.
\end{lstlisting}
        \producing
        Réglure de $1$~cm de long et
        $3$~mm d'épais surélevée de
        $2$~points au-dessus de la
        ligne de base: \rule[2pt]{1cm}{3mm}.
      \end{texample}
    \end{demo}
  \end{enumerate}
  \qed
\end{exemple}

\begin{exemple}
  La commande \cmd{\bfseries} sélectionne une police
  grasse pour tout le texte qui suit.
  \begin{demo}
    \begin{texample}
\begin{lstlisting}
En typographie, la \bfseries
graisse est l'épaisseur d'un
trait ou d'un caractère.
\end{lstlisting}
      \producing
      En typographie, la \bfseries
      graisse est l'épaisseur d'un
      trait ou d'un caractère.
    \end{texample}
  \end{demo}
  Pour limiter le changement à une zone de texte, il faut la délimiter
  par des accolades.
\begin{demo}
    \begin{texample}
\begin{lstlisting}
En typographie, la {\bfseries
graisse est l'épaisseur d'un
trait} ou d'un caractère.
\end{lstlisting}
      \producing
      En typographie, la {\bfseries
      graisse est l'épaisseur d’un
      trait} ou d’un caractère.
    \end{texample}
  \end{demo}
  \qed
\end{exemple}

Vous pouvez définir des nouvelles commandes {\LaTeX} à loisir. Ceci
est expliqué au \autoref{chap:commandes}.


\section{Environnements}
\label{sec:bases:environnements}

Un environnement {\LaTeX} est une zone de texte délimitée par une
construction du type
\begin{lstlisting}
\begin`\marg{environnement}'
   ...
\end`\marg{environnement}'
\end{lstlisting}
Le contenu d'un environnement est traité différemment du reste du
texte en fonction des paramètres de l'environnement. Par exemple, le
texte à l'intérieur d'un environnement \Ie{center} est centré sur la
page.

Les changements induits par un environnement s'appliquent uniquement à
l'intérieur de celui-ci. Il en va de même des commandes utilisées à
l'intérieur d'un environnement.

\begin{exemple}
  L'environnement \Ie{quote} sert à composer des citations. Le texte à
  l'intérieur de l'environnement sera placé dans un bloc séparé du
  texte principal et en retrait des marges gauche et droite.
  \begin{demo}
    \begin{texample}
\begin{lstlisting}
La phrase
\begin{quote}
  Attention aux bogues dans
  le code ci-dessus; je ne
  l'ai pas testé, j'ai
  seulement prouvé qu'il
  était correct.
\end{quote}
est une citation célèbre du
créateur de {\TeX}, Donald
Knuth.
\end{lstlisting}
      \producing
      La phrase
      \begin{quote}
        Attention aux bogues dans le code ci-dessus; je ne l'ai pas testé,
        j'ai seulement prouvé qu'il était correct.
      \end{quote}
      est une citation célèbre du créateur de {\TeX}, Donald Knuth.
    \end{texample}
  \end{demo}
  Si la citation est dans la langue originale, il est préférable de la
  composer en italique.
  \begin{demo}
    \begin{texample}
\begin{lstlisting}
La phrase
\begin{quote}
  \itshape
  Beware of bugs in the above
  code; I have only proved it
  correct, not tried it.
\end{quote}
est une citation célèbre du
créateur de {\TeX}, Donald
Knuth.
\end{lstlisting}
      \producing
      La phrase
      \begin{quote}
        \itshape
        Beware of bugs in the above code; I have only proved it
        correct, not tried it.
      \end{quote}
      est une citation célèbre du créateur de {\TeX}, Donald Knuth.
    \end{texample}
  \end{demo}
  On remarque que l'effet de la commande \cmdprint{\itshape} s'est
  limité à l'environnement. %
  \qed
\end{exemple}


\section{Longueurs}
\label{sec:bases:longueurs}

Plusieurs commandes {\LaTeX} requièrent en argument une mesure de
largeur ou de hauteur. Dans la terminologie de {\LaTeX}, on parle plus
généralement de longueur\index{longueur} (\emph{length}).

Une longueur est un nombre positif, négatif ou nul
\emph{obligatoirement} et \emph{immédiatement} suivi d'un symbole
d'unité de mesure. Le \autoref{tab:bases:longueurs} présente les
principales unités de mesure utilisées par {\LaTeX} et le symbole
correspondant.

\begin{table}
  \caption{Principales unités de mesure pour les longueurs dans
    {\LaTeX}}
  \label{tab:bases:longueurs}
  \centering
  \begin{tabular}{lcl}
    \toprule
    Nom ou description & Symbole & Longueur équivalente \\
    \midrule
    millimètre & \texttt{mm} \\
    centimètre & \texttt{cm} & $10$~mm \\
    pouce      & \texttt{in} & $2,54$~cm \\
    point      & \texttt{pt} & $1/72,27$~pouce \\
    point PostScript     & \texttt{bp}   & $1/72$~pouce \\
    largeur de la lettre M & \texttt{em} & fonction de la police \\
    hauteur de la lettre x & \texttt{ex} & fonction de la police \\
    \bottomrule
  \end{tabular}
\end{table}

Il existe un certain nombre de longueurs prédéfinies. Les plus utiles
sont \cmd{\linewidth}, qui contient la largeur de la ligne de texte
courante, et \cmd{\textwidth}, qui contient la largeur de la page
courante. Dans du texte normal, les deux mesures sont habituellement
égales.


\section{Commentaires}
\label{sec:bases:commentaires}

Le symbole «\verb=%=» indique un commentaire dans le code source: tout
le texte après le symbole jusqu'à la fin de la ligne est ignoré par
{\LaTeX}.
\begin{demo}
  \begin{texample}
\begin{lstlisting}
texte % ignoré par LaTeX
\end{lstlisting}
    \producing
    texte % ignoré par LaTeX
  \end{texample}
\end{demo}


\section{Caractères spéciaux}
\label{sec:bases:caracteres}

Les claviers d'ordinateur mettent à la disposition des auteurs toutes
les lettres de l'alphabet (en versions minuscule et majuscule), les
chiffres de 0 à 9, un certain nombre de symboles et, selon le clavier,
des versions accentuées de certaines lettres. L'entrée des lettres et
des chiffres ne pose pas de problème particulier pour {\LaTeX}, mais
certains symboles sont réservés. De plus, certains symboles d'usage
courant ne sont pas disponibles sur les claviers.

\subsection{Espaces et retours à la ligne}
\label{sec:bases:caracteres:espaces}

J'ai déjà abordé à la \autoref{sec:bases:saisie} le traitement spécial
réservé par {\LaTeX} aux espaces et aux retours à la ligne dans le
code source. Les précisions suivantes s'imposent:
\begin{itemize}
\item seule la première espace entre deux éléments compte;
\item les espaces en début de ligne sont ignorées;
\item un retour à la ligne simple est traité comme une espace;
\item il faut deux retours à la ligne consécutifs (ce qui résulte en
  une ligne blanche dans le code source) pour identifier un changement
  de paragraphe.
\end{itemize}

Pour forcer une espace à un endroit où {\LaTeX} la supprimerait
normalement, utiliser la commande \verb*=\ = (le symbole «\bs» suivi
d'une espace, représentée ici par le symbole \verb*| |.
\begin{demo}
  \begin{texample}
\begin{lstlisting}
Apprendre \LaTeX\ c'est
formidable!
\end{lstlisting}
    \producing
    Apprendre \LaTeX\ c'est
    formidable!
  \end{texample}
\end{demo}

Dans le même ordre d'idées, le symbole «\verb=~=» insère une espace
insécable entre deux mots, de telle sorte que {\LaTeX} ne pourra
placer les mots sur des lignes différentes. Vous devriez insérer ce
symbole dans les noms ou entre une quantité et son unité.
\begin{demo}
  \begin{texample}
\begin{lstlisting}
M.~Tremblay me doit au
moins 200~\$.
\end{lstlisting}
    \producing
    M.~Tremblay me doit au moins 200~\$.
  \end{texample}
\end{demo}

Il peut arriver que l'espace générée par un retour à la ligne
\emph{simple} s'avère indésirable. Dans de tels cas, placez un symbole
de commentaire «\verb=%=» à la fin de la ligne.
\begin{demo}
  \begin{texample}
\begin{lstlisting}
Donald Knuth est un dieu
\textsuperscript{[citation]}
.
\end{lstlisting}
    \producing
    Donald Knuth est un dieu
    \textsuperscript{[citation]}
    .
  \end{texample}
  \begin{texample}
\begin{lstlisting}
Donald Knuth est un dieu%
\textsuperscript{[citation]}%
.
\end{lstlisting}
    \producing
    Donald Knuth est un dieu%
    \textsuperscript{[citation]}%
    .
  \end{texample}
\end{demo}

\subsection{Caractères réservés}
\label{sec:bases:caracteres:reserves}

Comme à peu près tous les langages de programmation, {\TeX} réserve
certains caractères pour son usage interne. Les caractères suivants sont
interprétés comme des commandes:
\begin{center}
  \verb=# $ & ~ _ ^ % { }=
\end{center}

Pour utiliser les symboles ci-dessus tels quels dans le texte, il faut
les précéder par le symbole «\bs»:
\index{#@\cs{}\texttt{\#}}
\index{$@\cs{}\texttt{\$}} %$
\index{&@\cs{}\texttt{\&}}
\index{_@\cs{}\texttt{\_}}
\index}
\begin{demo}
  \begin{minipage}{0.15\linewidth}
    \begin{texample}
\begin{lstlisting}
\#
\end{lstlisting}
      \producing\ \#
    \end{texample}
  \end{minipage}
  \hfill
  \begin{minipage}{0.15\linewidth}
    \begin{texample}
\begin{lstlisting}
\$
\end{lstlisting}
      \producing\ \$
    \end{texample}
  \end{minipage}
  \hfill
  \begin{minipage}{0.15\linewidth}
    \begin{texample}
\begin{lstlisting}
\&
\end{lstlisting}
      \producing\ \&
    \end{texample}
  \end{minipage}
  \hfill
  \begin{minipage}{0.15\linewidth}
    \begin{texample}
\begin{lstlisting}
\_
\end{lstlisting}
      \producing\rule{0pt}{1em}\ \_
    \end{texample}
  \end{minipage}
  \\
  \begin{minipage}{0.15\linewidth}
    \begin{texample}
\begin{lstlisting}[commentstyle=\mdseries]
\%
\end{lstlisting}
      \producing\ \%
    \end{texample}
  \end{minipage}
  \hfill
  \begin{minipage}{0.15\linewidth}
    \begin{texample}
\begin{lstlisting}
\{
\end{lstlisting}
      \producing\ \{
    \end{texample}
  \end{minipage}
  \hfill
  \begin{minipage}{0.15\linewidth}
    \begin{texample}
\begin{lstlisting}
\}
\end{lstlisting}
      \producing\ \}
    \end{texample}
  \end{minipage}
  \hfill
  \begin{minipage}{0.15\linewidth}
    \mbox{}
  \end{minipage}
  \hfill
\end{demo}
(Les commandes \verb=\^= et \verb=\~= servent à créer des accents;
voir la \autoref{sec:bases:caracteres:accents}.)

\subsection{Guillemets}
\label{sec:bases:caracteres:guillemets}

L'usage des guillemets nécessite une attention particulière dans
{\LaTeX}. Vous ne devriez pas utiliser dans le code source les
guillemets doubles «\verb="=» qui se trouvent couramment sur les
claviers d'ordinateurs.

Pour obtenir les guillemets anglais, il faut utiliser deux symboles
d'accent grave côte à côte pour les guillemets ouvrants «``» et deux
apostrophes côte à côte pour les guillemets fermants «''».
\begin{demo}
  \begin{texample}
\begin{lstlisting}[escapeinside={}]
``guillemets anglais''
\end{lstlisting}
    \producing
    ``guillemets anglais''
  \end{texample}
\end{demo}

En typographie française, il convient d'utiliser les chevrons («\ »)
comme guillemets. Avec la configuration appropriée de \pkg{babel}
(voir la \autoref{sec:bases:francais}), vous pourrez utiliser
directement dans le code source les symboles \,\verb=«=\, et
\,\verb=»=\, disponibles sur un clavier français. L'espacement requis
autour des symboles est géré automatiquement.
\begin{demo}
  \begin{texample}
\begin{lstlisting}
«guillemets français»
\end{lstlisting}
    \producing
    «guillemets français»
  \end{texample}
  \begin{texample}
\begin{lstlisting}
« guillemets français »
\end{lstlisting}
    \producing
    « guillemets français »
  \end{texample}
\end{demo}
En français, les guillemets anglais ne servent que pour les citations
incluses, c'est-à-dire pour les citations à l'intérieur d'une
citation.
\begin{demo}
  \begin{texample}
\begin{lstlisting}[escapeinside={}]
Elle m'a dit: «Laurent fait
dire ``non merci''.»
\end{lstlisting}
    \producing
    Elle m'a dit: «Laurent fait
    dire ``non merci''.»
  \end{texample}
\end{demo}

Les bons éditeurs de texte adaptés pour {\LaTeX} redéfinissent
l'action de la touche %
%% simili touche de clavier faite main à partir; utilisation de
%% \ooalign tirée de comprehensive.pdf, section 10.3
\raisebox{-2pt}{%
  \ooalign{\hfil\Large\faSquare[regular]\hfil\cr\hfil\ttfamily\textquotedbl\hfil}} %
du clavier pour insérer les symboles appropriés selon la langue du
document.

Le paquetage \pkg{csquotes} \citep{csquotes} propose une autre approche
pour la saisie des guillemets. Il fournit une commande
\cmdprint{\enquote} qui entoure son argument des guillemets appropriés
selon le contexte et la langue du texte. Cette solution peut s'avérer
intéressante pour les nouveaux venus à {\LaTeX} qui n'ont pas encore
pris l'habitude d'entrer directement les bons guillemets.

\subsection{Trait d'union et tirets}
\label{sec:bases:caracteres:tirets}

On trouve en typographie soignée trois sortes de tirets de longueurs
différentes:
\begin{enumerate}
\item le trait d'union «-» qui sert à relier des mots entre eux;
\item le tiret demi-cadratin «--» qui sert à joindre deux éléments qui
  comportent déjà des traits d'union (\emph{Trois-Rivières--Québec});
\item le tiret cadratin «---» qui sert à identifier le changement
  d'interlocuteur dans les dialogues ou qui remplacent parfois les
  parenthèses dans du texte normal.
\end{enumerate}

S'il est simple d'entrer un trait d'union à l'ordinateur, il est rare
que les claviers comportent des touches pour entrer les tirets
demi-cadratin et cadratin (ou alors cela se fait via une obscure
combinaison de touches). {\LaTeX} permet de créer facilement tous les
tirets ci-dessus en répétant simplement le trait d'union deux ou trois
fois.
\begin{demo}
  \begin{minipage}{0.2\linewidth}
    \begin{texample}
\begin{lstlisting}
-
\end{lstlisting}
      \producing\ -
    \end{texample}
  \end{minipage}
  \hfill
  \begin{minipage}{0.2\linewidth}
    \begin{texample}
\begin{lstlisting}
--
\end{lstlisting}
      \producing\ --
    \end{texample}
  \end{minipage}
  \hfill
  \begin{minipage}{0.2\linewidth}
    \begin{texample}
\begin{lstlisting}
---
\end{lstlisting}
      \producing\ ---
    \end{texample}
  \end{minipage}
\end{demo}

\subsection{Accents et ligatures}
\label{sec:bases:caracteres:accents}

À la base, {\LaTeX} ne reconnait pas les lettres accentuées ni les
ligatures comme {\ae} et {\oe} qui ne sont pas courantes en anglais.
Néanmoins, il est possible de créer ces symboles avec des commandes.
\begin{demo}
  \begin{minipage}{0.2\linewidth}
    \begin{texample}
\begin{lstlisting}
\'{o}
\end{lstlisting}
      \producing\ \'{o}
    \end{texample}
  \end{minipage}
  \hfill
  \begin{minipage}{0.2\linewidth}
    \begin{texample}
\begin{lstlisting}[escapeinside={}]
\`{o}
\end{lstlisting}
      \producing\ \`{o}
    \end{texample}
  \end{minipage}
  \hfill
  \begin{minipage}{0.2\linewidth}
    \begin{texample}
\begin{lstlisting}
\^{o}
\end{lstlisting}
      \producing\ \^{o}
    \end{texample}
  \end{minipage}
  \hfill
  \begin{minipage}{0.2\linewidth}
    \begin{texample}
\begin{lstlisting}
\"{o}
\end{lstlisting}
      \producing\ \"{o}
    \end{texample}
  \end{minipage} \\
  \hfill
  \begin{minipage}{0.2\linewidth}
    \begin{texample}
\begin{lstlisting}
{\ae}
\end{lstlisting}
      \producing\ \ae
    \end{texample}
  \end{minipage}
  \hfill
  \begin{minipage}{0.2\linewidth}
    \begin{texample}
\begin{lstlisting}
{\AE}
\end{lstlisting}
      \producing\ \AE
    \end{texample}
  \end{minipage}
  \hfill
  \begin{minipage}{0.2\linewidth}
    \begin{texample}
\begin{lstlisting}
{\oe}
\end{lstlisting}
      \producing\ \oe
    \end{texample}
  \end{minipage}
  \hfill
  \begin{minipage}{0.2\linewidth}
    \begin{texample}
\begin{lstlisting}
{\OE}
\end{lstlisting}
      \producing\ \OE
    \end{texample}
  \end{minipage}
\end{demo}

Évidemment, entrer tous les accents d'un texte en français avec les
commandes ci-dessus se révélerait un véritable cauchemar.
Heureusement, avec la configuration appropriée, il est possible
d'entrer directement les lettres accentuées que l'on trouve couramment
sur un clavier, voire même certaines ligatures si l'on utilise
l'encodage UTF-8. La \autoref{sec:bases:francais} fournit les détails.

Les commandes pour les accents demeurent utiles pour composer les
lettres accentuées des langues européennes autres que le français.
Vous trouverez la liste de ces commandes, ainsi que celles pour
composer une multitude d'autres symboles, à la section~2 de la très
utile %
\doc[\emph{Comprehensive {\LaTeX} Symbol
  List}]{comprehensive}{https://texdoc.net/pkg/comprehensive} %
\citep{comprehensive}.


\section{{\LaTeX} en français}
\label{sec:bases:francais}

Historiquement, {\LaTeX} était prévu pour la rédaction de documents en
anglais. Fort heureusement, il est aujourd'hui beaucoup plus aisé de
rédiger des documents dans d'autres langues.

\subsection{Approche moderne recommandée}
\label{sec:bases:francais:xetex}

La manière de loin la plus simple, et celle que je recommande
vivement, pour rédiger des documents en français consiste à
enregistrer le code source dans le codage de caractères %
\href{https://fr.wikipedia.org/wiki/UTF-8}{UTF-8} %
et à compiler avec {\XeLaTeX}. Vous pourrez ainsi entrer les lettres
accentuées directement au clavier («\code{é}») sans devoir passer par
les commandes {\TeX} de la section précédente («\code{{\bs}'e}»).

Vous devrez aussi charger le paquetage \pkg{fontspec} \citep{fontspec}
dans le préambule pour obtenir les lettres accentuées dans le fichier
PDF. Cela ne représente toutefois pas véritablement une étape de
configuration additionnelle puisque ce paquetage est fréquemment
employé avec {\XeLaTeX} pour contrôler le chargement des polices de
caractères (plus de détails à la \autoref{sec:trucs:polices}).

\subsection{Approche tradionnelle}
\label{sec:bases:francais:pdftex}

L'approche plus traditionnelle, basée sur pdf{\LaTeX}, de rédaction de
documents dans une langue autre que l'anglais nécessite le chargement
de divers paquetages.

Tout d'abord, il faut charger le paquetage \pkg{inputenc}
\citep{inputenc} pour faire en sorte que les lettres accentuées seront
reconnues par {\LaTeX}. Il faut spécifier le type de codage utilisé.
Tel que mentionné ci-dessus, vous devriez aujourd'hui utiliser
l'UTF-8.
\begin{lstlisting}
\usepackage[utf8]{inputenc}
\end{lstlisting}

Après avoir configuré l'entrée des caractères, il faut également
configurer la sortie. Pour que la coupure automatique des mots
contenant des lettres accentuées fonctionne correctement avec
pdf{\LaTeX}, il faut activer le codage de sortie des caractères nommé
«T1» à l'aide du paquetage \pkg{fontenc} \citep{fontenc}.
\begin{lstlisting}
\usepackage[T1]{fontenc}
\end{lstlisting}

\subsection{Typographie et mots clés français}
\label{sec:bases:francais:babel}

Entrer du texte en français dans le code source n'est pas tout. Il
faut adapter {\LaTeX} au français, qu'il s'agisse des mots clés (tels
«Table des matières» ou «Bibliographie»), de la typographie ou de la
coupure des mots. La solution standard provient du paquetage
\pkg{babel} \citep{babel}. Celui-ci permet de combiner plusieurs
langues dans un même document et de passer de l'une à l'autre
facilement. Consultez la %
\doc{frenchb}{https://texdoc.net/pkg/babel-french} %
du paquetage \pkg{babel-french} pour connaitre toutes les adaptations
au français offertes par ce paquetage incontournable.

\subsection{Guillemets français}
\label{sec:bases:francais:guillemets}

Une espace fine devrait séparer les guillemets du ou des mots qu'ils
entourent. Pour que l'espace autour des symboles soit géré
automatiquement par \pkg{babel}, il suffit d'ajouter dans le préambule
la commande
\begin{lstlisting}
\frenchbsetup{og=«, fg=»}
\end{lstlisting}

\subsection{Séparateur décimal}
\label{sec:bases:francais:virgule}

Le séparateur décimal en français est la virgule et non le point. Or,
en mode mathématique, {\LaTeX} ajoute automatiquement une espace fine
après une virgule comme s'il s'agissait d'une énumération.

Afin de pouvoir utiliser de manière conviviale la virgule comme
séparateur décimal en mode mathématique, chargez dans vos documents le
très pratique paquetage \pkg{icomma} \citep{icomma}. Avec ce
paquetage, la virgule agira comme séparateur décimal en mode
mathématique seulement lorsqu'elle est suivie d'un caractère autre que
l'espace.
\begin{demo}
  \begin{minipage}{0.5\linewidth}
    \begin{texample}[0.6\linewidth]
\begin{lstlisting}
% nombre décimal
$x = 1,2$
\end{lstlisting}
      \producing
      $x = 1,2$
    \end{texample}
  \end{minipage}
  \hfill
  \begin{minipage}{0.5\linewidth}
    \begin{texample}[0.6\linewidth]
\begin{lstlisting}
% énumération
$x = 1, 2, 3$
\end{lstlisting}
      \producing
      $x = 1, 2, 3$
    \end{texample}
  \end{minipage}
\end{demo}

\importantbox{Le paquetage \pkg{icomma} peut interagir avec d'autres.
  Aussi est-il préférable de le charger parmi les derniers dans le
  préambule, et certainement après les paquetages \pkg{fontspec} et
  \pkg{unicode-math} avec {\XeLaTeX}.}

\subsection{Séparateur des milliers}
\label{sec:bases:francais:milliers}

Le séparateur des milliers en français est l'espace. Lorsque le
paquetage \pkg{numprint} \citep{numprint} est chargé, \pkg{babel}
fournit la commande \cmd{\nombre} pour formater automatiquement les
nombres.
\begin{demo}
  \begin{texample}
\begin{lstlisting}
\nombre{123456789}
\end{lstlisting}
    \producing
    \nombre{123456789}
  \end{texample}
\end{demo}

Le \autoref{tab:bases:francais} résume les divers enjeux liés à la
rédaction en français avec {\LaTeX} et les paquetages qui offrent des
solutions.

\begin{table}
  \centering
  \caption{Principaux enjeux de rédaction en français et paquetages
    offrant des solutions}
  \label{tab:bases:francais}
  \begin{tabularx}{1.0\linewidth}{Xl}
    \toprule
    Enjeu & Solution \\
    \midrule
    \addlinespace[0.5\normalbaselineskip]
    Traduction des mots clés prédéfinis & \pkg{babel} \\
    \addlinespace[0.5\normalbaselineskip]
    Coupure de mots & \pkg{babel} \\
    \addlinespace[0.5\normalbaselineskip]
    Typographie française & \pkg{babel} \\
    \addlinespace[0.5\normalbaselineskip]
    Lettres accentuées dans source & source en UTF-8 (\XeLaTeX) \\
                                   & \pkg{inputenc} (\LaTeX) \\
    \addlinespace[0.5\normalbaselineskip]
    Lettres accentuées dans sortie & \pkg{fontspec} (\XeLaTeX) \\
                                   & \pkg{fontenc} (\LaTeX) \\
    \addlinespace[0.5\normalbaselineskip]
    Virgule comme séparateur décimal & \pkg{icomma} \\
    \addlinespace[0.5\normalbaselineskip]
    Espace comme séparateur des milliers & \pkg{numprint} et
                                           \pkg{babel} \\
    \bottomrule
  \end{tabularx}
\end{table}


%%%
%%% Exercices
%%%

\section{Exercices}
\label{sec:bases:exercices}

\Opensolutionfile{solutions}[solutions-bases]

\begin{Filesave}{solutions}
\section*{Chapitre \ref*{chap:bases}}
\addcontentsline{toc}{section}{Chapitre \protect\ref*{chap:bases}}

\end{Filesave}

\begin{exercice}[nosol]
  Utiliser le fichier \fichier{exercice-minimal.tex}.
  \begin{enumerate}
  \item Compiler le document avec la classe \class{article}, puis avec
    la classe \class{book}. Observer le résultat.
  \item Ajouter du texte en français (avec accents), puis compiler et
    observer le résultat.
  \end{enumerate}
\end{exercice}

\begin{exercice}
  \label{ex:base:commandes}
  Modifier le fichier \fichier{exercice-commandes.tex} afin
  de produire le texte ci-dessous.
  \begin{demo}
    \fbox{\includegraphics[viewport=108 551 502 665,clip=true,width=0.98\linewidth]{auxdoc/exercice_commandes-solution}}
  \end{demo}
  \begin{sol}
    Il y a trois modifications principales à apporter au fichier.
    \begin{enumerate}[1.]
    \item La commande {\LaTeX} qui imprime le logo de LaTeX n'ayant
      pas d'argument, il faut l'écrire sous la forme
      \verb={\LaTeX}= ou \verb=\LaTeX{}= pour éviter que l'espace
      qui suit ne soit avalée par la commande.
    \item Il faut délimiter par des accolades la zone à laquelle la
      commande \cmdprint{\bfseries} doit s'appliquer:
\begin{lstlisting}
{\bfseries limitée}
\end{lstlisting}
    \item Il faut utiliser l'environnement \Pe{enumerate} pour les
      listes numérotées. L'environnement \Pe{itemize} sert pour les
      listes à puces.
    \end{enumerate}
  \end{sol}
\end{exercice}

\begin{exercice}[nosol]
  \begin{enumerate}
  \item Compiler le fichier \fichier{exercice-classe+paquetages.tex}.
  \item Changer la police du document pour 11~points, puis 12~points.
    Observer l'effet sur les marges et sur la coupure automatique des
    mots dans le document compilé.
  \item Activer le paquetage \pkg{icomma} en supprimant le symbole
    \% au début de la ligne dans le préambule. Compiler le document
    et observer l'effet sur la formule mathématique.
  \item Charger le paquetage \pkg{numprint} avec l'option
    \verb=autolanguage= (\emph{après} le paquetage \pkg{babel}). Dans
    le code source de la formule mathématique, changer
\begin{lstlisting}
10 000
\end{lstlisting}
    pour
\begin{lstlisting}
\nombre{10000}
\end{lstlisting}
    Compiler et observer le résultat.
  \end{enumerate}
\end{exercice}

\Closesolutionfile{solutions}

%%% Local Variables:
%%% mode: latex
%%% TeX-master: "formation-latex-ul"
%%% TeX-engine: xetex
%%% coding: utf-8
%%% End:
