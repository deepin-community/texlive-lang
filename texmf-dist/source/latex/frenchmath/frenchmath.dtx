% \iffalse meta-comment
%
% Copyright (C) 2019-2024 by Antoine Missier <antoine.missier@ac-toulouse.fr>
%
% This file may be distributed and/or modified under the conditions of
% the LaTeX Project Public License, either version 1.3 of this license
% or (at your option) any later version. The latest version of this
% license is in:
%
%   http://www.latex-project.org/lppl.txt
%
% and version 1.3 or later is part of all distributions of LaTeX version
% 2005/12/01 or later.
% \fi
%
% \iffalse
%<*driver>
\ProvidesFile{frenchmath.dtx}
%</driver>
%<*package> 
\NeedsTeXFormat{LaTeX2e}[2005/12/01]
\ProvidesPackage{frenchmath}
    [2024/01/22 v2.8 .dtx frenchmath file]
%</package>
%<*driver>
\documentclass{ltxdoc}
\usepackage[utf8]{inputenc}
\usepackage[T1]{fontenc}
\usepackage[french]{babel}
\usepackage{lmodern}
\usepackage[lgrmath]{frenchmath}
\usepackage[Euler]{upgreek}
\usepackage{textalpha}
\usepackage{metalogo} % pour les logos XeLaTeX et LuaLaTeX
\usepackage{sectsty}
\usepackage{xcolor}

\definecolor{darkbrown}{rgb}{0.5,0.1,0.1}
\allsectionsfont{\color{darkbrown}}
\DeclareSymbolFont{ugrfm}{U}{psy}{m}{n} % pour les lettres de upgreek avec l'option Symbol
\DeclareMathSymbol{\upalphasym}{\mathord}{ugrfm}{`a}
\DeclareMathSymbol{\upbetasym}{\mathord}{ugrfm}{`b}
\DeclareMathSymbol{\uppisym}{\mathord}{ugrfm}{`p}
\DeclareSymbolFont{cmsymbols}{OMS}{cmsy}{m}{n} % symbole \emptyset de computer modern
\DeclareMathSymbol{\cmemptyset}{\mathord}{cmsymbols}{'73} % code octal dans cmsy
\DeclareTextSymbol{\textmicro}{TS1}{181} % mu de micro
\DeclareTextSymbolDefault{\textmicro}{TS1}
\DisableCrossrefs
%\CodelineIndex
%\RecordChanges
\usepackage{hyperref}
\hypersetup{%
    colorlinks,
    linkcolor=blue,
    citecolor=red,
    pdftitle={frenchmath},
    pdfsubject={LaTeX package},
    pdfauthor={Antoine Missier}
}
\begin{document}
\DocInput{frenchmath.dtx}
%\PrintChanges
%\PrintIndex
\end{document}
%</driver>
% \fi
%
% \CheckSum{519}
%
% \CharacterTable
%  {Upper-case    \A\B\C\D\E\F\G\H\I\J\K\L\M\N\O\P\Q\R\S\T\U\V\W\X\Y\Z
%   Lower-case    \a\b\c\d\e\f\g\h\i\j\k\l\m\n\o\p\q\r\s\t\u\v\w\x\y\z
%   Digits        \0\1\2\3\4\5\6\7\8\9
%   Exclamation   \!     Double quote  \"     Hash (number) \#
%   Dollar        \$     Percent       \%     Ampersand     \&
%   Acute accent  \'     Left paren    \(     Right paren   \)
%   Asterisk      \*     Plus          \+     Comma         \,
%   Minus         \-     Point         \.     Solidus       \/
%   Colon         \:     Semicolon     \;     Less than     \<
%   Equals        \=     Greater than  \>     Question mark \?
%   Commercial at \@     Left bracket  \[     Backslash     \\
%   Right bracket \]     Circumflex    \^     Underscore    \_
%   Grave accent  \`     Left brace    \{     Vertical bar  \|
%   Right brace   \}     Tilde         \~}
%
%
% \changes{v0.1}{27/12/2011}{Version personnelle préliminaire}
%
% \changes{v1.0}{15/01/2019}{
%    - Première version publiée, création des fichiers dtx et ins.}
%
% \changes{v1.1}{07/04/2019}{
%    - Nouvelles macros pour les repères (Oij, Oijk),
%    - ajouté ensuremath dans curs.}
% \changes{v1.1}{15/04/2019}{
%    - Changements mineurs dans la documentation.}
%
% \changes{v1.2}{25/04/2019}{
%    - L'option capsrm fonctionne à présent avec beamer, 
%    - ajouté Ouv, modifications du fichier README.md.}
% \changes{v1.2}{27/04/2019}{
%    - capsrm -> capsup.}
%
% \changes{v1.3}{15/05/2019}{
%    - Intégration de icomma et psset{comma=true}, 
%    - changements dans la documentation.}
%
% \changes{v1.4}{2019/05/22}{
%    - Changement de la définition de fonte up -> operators, 
%        car incompatibilité avec l'extension unicode-math.}
%
% \changes{v1.5}{2020/11/02}{
%    - Ajout des macros étoilées pour les repères (O;i,j) et (O;i,j,k)
%    - ajout des macros pour les bases (i,j) et (i,j,k).}
%
% \changes{v1.6}{2022/10/16}{
%    - Remplacement de icomma par ncccomma, grâce à une proposition de J. F. Burnol,
%    - amélioration du code redéfinissant les majuscules mathématiques,
%    - suppression de l'option capsup (capsit à false suffit),
%    - remplacement de tgh par th pour la tangente hyperbolique.}
%
% \changes{v2.0}{2022/10/24}{
%    - Nouvelle option permettant de définir les lettres grecques minuscules en forme droite, 
%    - correction de bug avec la commande \bslash qui n'était pas définie, 
%    - compatibilité avec mathdesign.}
%
% \changes{v2.1}{11/11/2022}{
%    - Reprise du doc pour les lettres grecques, les crochets et les références ;
%    - nouvelle commande paral*, 
%    - '\mathop{\operatorfont th}' remplacé par '\operatorname{th}'.}
%
% \changes{v2.2}{15/11/2022}{
%    - Quelques changements dans la doc, police Palatino avec mathpazo,
%    - commande paral redéfinie, la commande paral* devient inutile.}
% 
% \changes{v2.3}{16/11/2022}{
%    - Bug corrigé dans la commande Vect.}
% \changes{v2.3}{20/11/2022}{
%    - Ajout dans la doc : frenchmath  doit être chargé après babel.}
% \changes{v2.3}{19/12/2022}{
%    - Suppression des warnings systématiques, 
%    - utilisation de \string à la place de \bslash.}
% \changes{v2.3}{25/12/2022}{
%    - Nouvelle option lgrmath, 
%    - utilisation de l'extension ibrackets,
%    - définition de cosec et cosech.}
%
% \changes{v2.4}{22/01/2023}{
%    - Option noibrackets car incompatibilité avec DeclarePairedDelimiter,
%    - warning si compilation avec LuaLaTeX et babel-french non chargé.}
%
% \changes{v2.5}{24/02/2023}{
%    - Mention des macros MathUp et MathIt de mismath dans la doc,
%    - remplacement des ifthenelse par des if... fi,
%    - simplification du code pour la mise en majuscule,
%    - chargement conditionnel de dotlessj pour mathptmx.}
%
% \changes{v2.6}{08/03/2023}{
%    - La macro curs ne doit et ne peut plus être utilisée en mode texte.}
%
% \changes{v2.7}{06/12/2023}{
%    - Nouveau package decimalcomma pour la virgule en mode math.}
% \changes{v2.7}{08/12/2023}{
%    - Changements dans la doc : lettres grecques en dernier, retour à la police lmodern.}
% \changes{v2.7}{12/12/2023}{
%    - upgreek charge l'option Symbol et non plus Euler,
%    - sauvegarde des lettres grecques d'origine avec lgrmath comme pour upgreek,
%    - réécriture de l'introduction, mention de unicode-math avec XeLaTeX ou LuaLaTeX,
%    - PackageWarning remplacé par PackageInfo,
%    - optimisation du code pour les majuscules avec la macro apply.}
%
% \changes{v2.8}{22/01/2024}{
%    - Prise en compte de l'extension mathalpha qui fournit d'autres lettres cursives.}
%
% \GetFileInfo{frenchmath.sty}
%
% \title{L'extension \textsf{frenchmath}\thanks{Ce document
%     correspond à \textsf{frenchmath}~\fileversion, dernière modification le 22/01/2024.}}
% \author{Antoine Missier \\ \texttt{antoine.missier@ac-toulouse.fr}}
% \date{22 janvier 2024}
% \maketitle
%
% \section{Introduction}
% Cette extension, inspirée de \textsf{mafr} de Christian Obrecht~\cite{MAFR},
% permet le respect des règles typographiques des mathématiques françaises, 
% en particulier elle permet d'obtenir automatiquement
% les majuscules en romain (lettres droites) plutôt qu'en italique 
% (voir~\cite{RTIN} et~\cite{IGEN}),
% et elle gère correctement les espacements
% pour les virgules, point-virgules et crochets.
%
% L'extension fournit en outre diverses macros francisées.
% Mais contrairement à \textsf{mafr} nous avons choisi de ne pas
% conserver le même nom de commande pour substituer 
% des symboles français aux symboles anglais.
%
% D'autres solutions existent pour composer les majuscules mathématiques en romain,
% par exemple l'extension \textsf{unicode-math} mais qui doit être compilée
% avec \XeLaTeX\ ou \LuaLaTeX  ;
% avec pdf\LaTeX\, nous avons les extensions \textsf{fourier} de Michel Bovani~\cite{FOUR} 
% (avec la famille des fontes Adobe Utopia)
% ou encore \textsf{mathdesign} de Paul Pichaureau~\cite{DESIGN} 
% (avec les polices Adobe Utopia, URW Garamond ou Bitstream Charter). 
% Mais \textsf{frenchmath} fournit une solution générique 
% s'adaptant à n'importe quelle police de caractères.
%
% Certaines préconisations, telles que composer en lettre droite
% et non en italique le symbole différentiel, 
% les constantes mathématiques i et e~\cite{IGEN}, 
% sont des règles internationales~\cite{TYPMA}~\cite{NIST}~\cite{ICTNS}.
% Elles ne sont donc pas implémentées dans \textsf{frenchmath}
% \footnote{Nous proposons pour cela l'extension \textsf{mismath}~\cite{MIS}
% qui fournit diverses macros pour les mathématiques internationales.},
% tout comme diverses commandes que l'on trouve dans \textsf{mafr} 
% et qui ne sont pas spécifiques aux mathématiques françaises :
% c'est le cas de |\vect|
% \footnote{Pour de jolis vecteurs on dispose de l'extension \textsf{esvect}~\cite{VECT}
% d'Eddie Saudrais.},
% des ensembles de nombres |\R|, |\N|\ldots (pour $\mathbf{R}, \mathbf{N}\ldots$)
% ainsi que celles relatives à la réalisation de feuilles d'exercices.
%
% Mentionnons par ailleurs l'extension \textsf{tablvar}~\cite{TABL} qui permet de réaliser 
% de jolis tableaux de variations, spécificité des mathématiques françaises,
% ou encore \textsf{tdsfrmath} de Yvon Henel~\cite{FRM}
% qui fournit également beaucoup de commandes francisées.
%
% Depuis la version 2.0, \textsf{frenchmath} propose deux options permettant
% de composer les minuscules grecques du mode mathématique en forme droite.
%
% \section{Utilisation}
%
% \subsection{Majuscules mathématiques}
% Dans les mathématiques françaises, pour l'alphabet latin,
% \og les lettres majuscules sont toujours
% composées en romain \fg ($A, B, C\ldots$) et non en italique 
% (cf. \cite{RTIN} p.107, voir aussi~\cite{IGEN}).
% En utilisant \XeLaTeX\ ou \LuaLaTeX\ avec des polices mathématiques OpenType, 
% cette convention est assez commode à mettre en œuvre
% \footnote{Voir les options \texttt{math-style=upright} 
% ou \texttt{math-style=french} de l'extension \textsf{unicode-math}.} ;
% par contre, avec \LaTeX\ ou pdf\LaTeX, elle est peu respectée
% et les extensions précitées ne fonctionnent qu'avec des polices particulières.
% Par défaut \textsf{frenchmath} compose automatiquement les majuscules
% mathématiques latines en romain,
% quelle que soit la fonte utilisée.
% Par exemple |\[ P(X)=\sum_{i=0}^{n} a_i X^i \]| donne avec \textsf{frenchmath}
% \[ P(X)=\sum_{i=0}^{n}a_i X^i. \]
%
% \DescribeEnv{[capsit]}
% L'option \texttt{capsit} de \textsf{frenchmath} 
% permet de désactiver la composition des majuscules du mode mathématique
% en romain pour conserver la composition par défaut (en italique) :
% |\usepackage[capsit]{frenchmath}|.
%
% \medskip
% Que l'option soit activée ou pas, il est toujours possible de changer ponctuellement 
% l'aspect d'une lettre particulière, avec les macros \LaTeX\ |\mathrm| et |\mathit|.
% D'autre part l'extension \textsf{mismath}~\cite{MIS} fournit deux bascules puissantes
% |\MathUp| et |\MathIt| qui agissent de manière globale (ou locale dans un environnement)
% et permettent à tout moment 
% de changer la \og famille \fg d'une lettre particulière ;
% une commande générique |\apply| permet y compris d'appliquer ces macros sur une liste.
% Ainsi |\apply\MathIt{F,G,X}| remettra en italique les lettres $\mathit{F,G}$ et $\mathit{X}$.
%
% \subsection{Virgule, point-virgule et crochets}
%
% \DescribeMacro{virgule}
% Dans le mode mathématique de \LaTeX, la virgule est toujours, par défaut, 
% un symbole de ponctuation et sera donc suivie d'une espace.
% Ceci est légitime dans une liste ou un intervalle :
% |$[a,b]$| donne $[a,b]$. Mais, en français, la virgule sert aussi de séparateur décimal
% pour les nombres et ne doit, dans ce cas, pas être suivie d'espace ; or |$12,5$| donne 
% $12, 5$ au lieu de $12,5$.
% L'extension \textsf{babel}, avec l'option |french|~\cite{BABEL}, fournit deux bascules :
% |\DecimalMathComma| et |\StandardMathComma|, qui permettent d'adapter
% le comportement de la virgule du mode mathématique.
%
% Deux autres extensions bien commodes permettent néanmoins de se passer de ces bascules
% \footnote{Dans ce cas il ne faut pas utiliser les bascules, 
% au risque de rendre ces extensions inopérantes.}.
% En mode mathématique :
% \begin{itemize}
% \item avec \textsf{icomma} (intelligent comma) de Walter Schmidt~\cite{ICOMMA},
% la virgule se comporte comme un caractère de ponctuation si elle est suivie d'une espace,
% sinon c'est un caractère ordinaire ;
% \item avec \textsf{ncccomma} de Alexander I.~Rozhenko~\cite{NCC},
% la virgule se comporte comme un caractère ordinaire si elle est suivie d'un chiffre 
% (sans espace), sinon c'est un caractère de ponctuation.
% \end{itemize}
%
% Cette deuxième approche parait meilleure, néanmoins \textsf{ncccomma}
% ne fonctionne pas avec avec \textsf{babel-french} utilisé conjointement avec
% l'option \texttt{autolanguage}
% \footnote{L'option \texttt{autolanguage} de \textsf{numprint} utilisée 
% conjointement avec l'option \texttt{french} de \textsf{babel} garantit un espacement
% correct entre les groupes de trois chiffres dans les grands nombres,
% qui doit être une espace insécable et non dilatable~\cite{RTIN},
% légèrement plus grande que l'espace que l'on obtient sans cette option.}
% de l'extension \textsf{numprint}.
% En outre \textsf{ncccomma} ne fonctionne pas non plus
% avec l'extension \textsf{unicode-math} (et bugue à l'appel |\setmathfont|).
% Dans son article \emph{Intelligent commas}~\cite{BECC},
% Claudio Beccari propose une autre solution,
% voisine de \textsf{ncccomma}, mais qui produit le même type d'incompatibilités.
% Le code a donc été revu et simplifié afin de régler proprement ces incompatibilités.
% Comme bien des pays utilisent la virgule comme
% séparateur décimal, il fait l'objet d'une extension séparée,
% \textsf{decimalcomma}~\cite{DECC}, qui est chargée par \textsf{frenchmath}
% (depuis la version 2.7).
% Mais il est alors impératif de charger \textsf{unicode-math} \emph{avant} 
% \textsf{frenchmath} 
% \footnote{L'extension \textsf{icomma} présente la même limitation et doit être chargé
% après \textsf{unicode-math}.}.
%
% \medskip
% Lorsque l'on utilise l'extension \textsf{pstricks-add} de \textsf{PSTricks}
% pour tracer des axes de coordonnées, l'appel |\psset{comma=true}|
% permet d'avoir les graduations avec une virgule au lieu du point décimal.
% Ce réglage est effectué par défaut ici.
%
% \medskip
% \DescribeMacro{point-virgule}
% Le symbole \og;\fg\ a été redéfini pour le mode mathématique
% car l'espace précédant le point-virgule est incorrecte en français
% |$x \in [0,25 ; 3,75]$| donne
% $x\in [0,25 \mathpunct; 3,75 ]$ sans \textsf{frenchmath} et $x\in [0,25 ; 3,75]$ 
% avec \textsf{frenchmath} ;
% le comportement de \og ;\fg devient identique à celui de \og:\fg.
% 
% \medskip
% \DescribeMacro{crochets}
% Alors que les Anglais utilisent généralement les parenthèses 
% pour les intervalles ouverts $(0, +\infty)$, l'usage en français est d'utiliser
% les crochets $]0, +\infty[$. Mais comme cela n'est pas prévu par \LaTeX, 
% les espaces seront souvent incorrectes.
% Nous avons redéfini les crochets dans l'extension \textsf{ibrackets}~\cite{BRACKETS}
% qui est chargée par \textsf{frenchmath},
% \DescribeMacro{[noibrackets]}
% sauf si l'on active l'option \texttt{noibrackets}
% \footnote{D'autres solutions existent,
% par exemple avec l'extension \textsf{interval}
% ou encore avec la macro \texttt{\string\DeclarePairedDelimiter}
% de \textsf{mathtools}, mais utilisée avec des crochets,
% cette dernière est incompatible avec \textsf{ibrackets}, 
% d'où la possibilité de désactiver \textsf{ibrackets}.}.
% Le code 
% |$x\in ]-\pi, 0[ \cup ]2\pi, 3\pi[$|
% produira
% \begin{eqnarray*}
%  & x\in ]-\pi, 0[ \cup ]2\pi, 3\pi[ \mbox{\quad avec \textsf{ibrackets}},\\ 
% \mbox{au lieu de} &  x\in \mathclose{]}-\pi, 0 \mathopen{[} 
% \cup \mathclose{]} 2\pi, 3\pi \mathopen{[} 
% \mbox{\quad  sans \textsf{ibrackets}}.
% \end{eqnarray*}
%
% Avec \textsf{ibrackets}, un crochet devient un caractère ordinaire,
% sauf s'il est immédiatement suivi par un signe + ou - (sans espace), auquel cas 
% c'est un délimiteur ouvrant. 
% Si la borne de gauche possède un signe - (ou +),
% \emph{il ne faut pas laisser d'espace entre le premier crochet 
% et le signe}: par exemple |$x \in ] -\infty, 0]$| produit 
% $x \in ] -\infty, 0]$ au lieu de $x \in ]-\infty, 0]$.
% Mais au contraire lorsque l'on veut faire de l'algèbre sur les intervalles,
% \emph{il faut laisser une espace entre le second crochet et l'opération} + ou -,
% par exemple, |$[a,b] + [c,d]$| produit $[a,b] + [c,d]$
% mais |$[a,b]+ [c,d]$| produit $[a,b]+ [c,d]$.
%
% En cas de comportement problématique, par exemple si
% une coupure de ligne se produit entre les deux crochets d'un intervalle,
% il est toujours possible de transformer alors ces crochets en délimiteurs
% avec |\left| et |\right|.
%
% \subsection{Quelques macros et alias utiles}
%
% \DescribeMacro{\curs}
% Les lettres cursives ($\curs{A}, \curs{B}, \curs{C}, \curs{D}\ldots$),
% sont composées avec |\mathscr|, ou son alias |\curs|,
% et sont différentes de celles obtenues avec |\mathcal| 
% \footnote{L'extension \textsf{calrsfs} fournit les mêmes cursives mais en redéfinissant
% la commande \texttt{\string\mathcal}.}
% ($\mathcal{A}, \mathcal{B}, \mathcal{C}, \mathcal{D}\ldots$).
% En principe \textsf{frenchmath} charge l'extension \textsf{mathrsfs}
% qui fournit ces lettres cursives, 
% sauf si la commande |\mathscr| est déjà définie par ailleurs,
% en particulier si on utilise l'extension \textsf{mathdesign}~\cite{DESIGN}
% ou l'option \texttt{scr} de \textsf{mathalpha}~\cite{MATAL}
% \footnote{L'extension \textsf{mathalpha} de Michael Sharpe
% permet d'accéder à différentes variantes élégantes de lettres calligraphiques,
% par exemple avec les options \texttt{scr=boondox}, \texttt{scr=rsfso} ou \texttt{scr=kp}.}.
%
% \medskip
% \DescribeMacro{\infeg} \DescribeMacro{\supeg}
% Les relations $\infeg$ et $\supeg$ s'obtiennent avec les commandes |\infeg| et |\supeg|
% et diffèrent des versions anglaises de |\leq| ($\leq$) et |\geq| ($\geq$).
% Ce sont des alias de |\leqslant| et |\geqslant| de l'extension \textsf{amssymb},
% chargée par \textsf{frenchmath}.
%
% \medskip
% \DescribeMacro{\vide}
% Le symbole de l'ensemble vide $\vide$ 
% s'obtient avec |\vide| (alias de la commande |\varnothing| de l'extension \textsf{amssymb}) ;
% il diffère de celui obtenu avec |\emptyset|,
% particulièrement laid dans la fonte classique Latin Modern : $\cmemptyset$.
%
% \medskip
% \DescribeMacro{\paral}
% La commande |\paral| fournit la \emph{relation} 
% \footnote{Pour noter que deux objets sont perpendiculaires, on utilise 
% \texttt{\string\perp} : $\curs{D}\perp\curs{D}'$, 
% défini comme une \emph{relation} mathématique plutôt que 
% \texttt{\string\bot} défini comme un \emph{symbole} (les espacements diffèrent).}
% du parallélisme : $\curs{D}\paral\curs{D}'$,
% plutôt que sa version anglaise |\parallel| : $\curs{D}\parallel\curs{D}'$.
%
% \medskip
% \DescribeMacro{\ssi}
% La commande |\ssi| produit le texte \og \ssi \fg.
%
% \medskip
% \DescribeMacro{\cmod}
% Le modulo se compose normalement entre parenthèses, avec |\pmod|,
% mais on rencontre aussi, en français, le modulo entre crochets,
% ce que permet la commande |\cmod| en respectant le bon espacement
% propre au modulo : $ 53 \equiv 5 \cmod{12}$.
%
% \subsection{Identifiants de \og fonctions\fg classiques}
%
% \DescribeMacro{\pgcd} \DescribeMacro{\ppcm} 
% En arithmétique, nous avons les classiques |\pgcd| et |\ppcm|, 
% qui diffèrent de leur version anglaise |\gcd| et |\lcm|
% \footnote{Cette dernière n'est pas implémentée en standard dans \LaTeX\ 
% (mais dans \textsf{mismath}~\cite{MIS}).}.
%
% \medskip
% \DescribeMacro{\card} \DescribeMacro{\Card}
% Pour le cardinal d'un ensemble, nous proposons |\card|, 
% cité dans~\cite{RTIN} et~\cite{AA}, 
% ou |\Card|, qui est aussi d'usage courant (cf. Wikipedia).
%
% \medskip
% \DescribeMacro{\Ker} \DescribeMacro{\Hom}
% \LaTeX\ fournit les macros
% |\ker| et |\hom|, alors que l'usage français est souvent
% de commencer ces noms par une majuscule pour obtenir $\Ker$
% \footnote{La commande \texttt{\string\Im} existe déjà pour la
% partie imaginaire des nombres complexes et produit $\Im$ ; 
% elle est redéfinie en Im par l'extension \textsf{mismath} 
% et peut aussi être utilisée pour l'image.}
% et $\Hom$.
%
% \medskip
% \DescribeMacro{\rg} \DescribeMacro{\Vect}
% Le rang d'une application linéaire ou d'une matrice ($\rg$) s'obtient avec la commande |\rg|
% et l'espace vectoriel engendré par une famille de vecteurs avec |\Vect|.
%
% \medskip
% \DescribeMacro{\ch} \DescribeMacro{\sh} \DescribeMacro{\th}
% En principe, les fonctions hyperboliques s'écrivent en français 
% avec les macros \LaTeX\ standard |\cosh|, |\sinh|, |\tanh|.
% Néanmoins les écritures $\ch x$, $\sh x$ et $\th x$, qui sont la norme
% avec les langues d'Europe de l'Est (voir \cite{COMP}), 
% sont aussi utilisées en français~\cite{RTIN}. 
% On les obtient avec les commandes |\ch|, |\sh| et |\th|
% \footnote{La commande \texttt{\string\th} existe déjà, pour le mode texte uniquement,
% et produit \th ;
% elle a été redéfinie, uniquement pour le mode mathématique.}.
%
% \medskip
% \DescribeMacro{\cosec} \DescribeMacro{\cosech}
% La fonction cosécante (inverse du sinus) s'obtient avec la macro |\csc|, mais en français,
% on utilise aussi |\cosec|~\cite{RTIN} et |\cosech| pour la cosécante hyperbolique
% \footnote{La fonction sécante est définie en standard par \LaTeX\ 
% avec \texttt{\textbackslash sec} et la sécante hyperbolique \texttt{\textbackslash sech}
% est définie par \textsf{mismath}~\cite{MIS}.}.
%
% \subsection{Bases et repères}
%
% \DescribeMacro{\Oij} \DescribeMacro{\Oijk}
% Les repères classiques du plan ou de l'espace seront composés 
% avec des hauteurs de flèches homogénéisées :
% |\Oij| compose \Oij, |\Oijk| compose \Oijk et |\Ouv| compose \Ouv
% (utilisé dans le plan complexe). \DescribeMacro{\Ouv}
% On peut écrire ces commandes en mode texte, sans les délimiteurs du mode mathématique.
%
% \DescribeMacro{\Oij*} \DescribeMacro{\Oijk*} \DescribeMacro{\Ouv*}
% Les versions étoilées utilisent le point-virgule et non la virgule
% comme séparateur après le point O, comme mentionné dans~\cite{RTIN}.
% On obtient \Oij*, \Oijk*, \Ouv*.
%
% \medskip
% \DescribeMacro{\ij} \DescribeMacro{\ijk}
% Enfin les macros |\ij|
% \footnote{Notons que la macro \texttt{\string\ij} existait déjà 
% (ligature entre i et j pour le hollandais) et a été redéfinie.}
% et |\ijk| composent les bases du plan et de l'espace, \ij
% et \ijk, en homogénéisant la hauteur des flèches.
%
% Signalons que, pour l'extension \textsf{mathptmx} (basée sur la police de texte Times),
% |\jmath| n'est pas disponible, mais \textsf{frenchmath} contourne ce problème
% en chargeant alors \textsf{dotlessj}~\cite{DLJ} de David Carlisle,
% ce qui permet aux macros ci-dessus de fonctionner normalement.
%
% \subsection{Lettres grecques}
% La norme concernant l'usage des lettres grecques en italique ou en forme droite
% pour les mathématiques françaises ne semble pas aussi claire que pour les lettres romaines
% et les auteurs 
% divergent sur ce point. Plusieurs recommandent l'usage des lettres grecques
% minuscules en forme droite~\cite{FOUR}~\cite{DESIGN}~\cite{PMISO}, mais d'autres
% préconisent l'italique,
% comme pour toutes les variables mathématiques~\cite{AA}.
% Le lexique des règles typographiques en usage à l’Imprimerie Nationale~\cite{RTIN}
% les compose en forme droite et relativement grasses (p.108) 
% sans préciser s'il s'agit vraiment d'une règle
% s'appliquant aux variables, au même titre que celles énoncées pour l'alphabet latin.
%
% \medskip
% Pour les physiciens (et chimistes) l'affaire est plus claire puisque 
% les quantités doivent toujours être écrites en italique et les unités ou les constantes en 
% romain (forme droite), conformément à la norme ISO~\cite{TYPMA}~\cite{NIST}~\cite{ICTNS}. 
% Ainsi la constante $\pi \approx 3,14$ ne s'écrit pas de la même manière
% qu'une variable $\itpi$.
% Dans la section \og How to get upright small Greek letters \fg,
% la documentation de \textsf{isomath} de Günter Milde~\cite{ISOM} 
% expose différentes méthodes pour obtenir les lettres grecques
% en forme droite.
% Par exemple les extensions \textsf{mathdesign}~\cite{DESIGN},
% \textsf{fourier}~\cite{FOUR}
% ou \textsf{kpfonts}~\cite{KPF} disposent d'options permettant
% l'écriture automatique des lettres grecques minuscules en forme droite 
% (ou des majuscules en italique).
% Citons également \textsf{newpxmath}, \textsf{newtxmath} 
% \footnote{L'extension \textsf{newtxmath} doit être chargée
% après \textsf{frenchmath} qui utilise \textsf{amssymb} car la compilation
% produit sinon un message d'erreur pour la commande
% \texttt{\string\Bbbk}.}
% et \textsf{libertinust1math} 
% de Michael Sharpe,
% \textsf{pxgreeks}, \textsf{txgreeks} 
% \footnote{Si on utilise \textsf{amsmath} (ou \textsf{mismath}),
% \textsf{pxgreeks} ou \textsf{txgreeks} doit être chargée 
% \emph{après} \textsf{amsmath} (ou \textsf{mismath}),
% pour éviter une erreur de compilation due à la redéfinition des commandes 
% \texttt{\string\iint}, \texttt{\string\iiint}, \texttt{\string\iiiint}, 
% \texttt{\string\idotsint}.}
% et \textsf{libgreek} de Jean-François Burnol,
% qui donnent de beaux résultats pour une utilisation avec 
% respectivement les polices Palatino, Times et Libertinus.
%
% Enfin, comme pour les majuscules, \XeLaTeX\ ou \LuaLaTeX\ réalisent cette tâche
% automatiquement avec l'option \texttt{math-style=french} de l'extension \textsf{unicode-math}. 
%
% \medskip
% Jean-François Burnol a également développé l'extension \textsf{lgrmath}~\cite{LGR}
% qui permet d'utiliser, en mode mathématique, les différentes fontes de lettres
% grecques accessibles par \LaTeX\ avec l'encodage LGR. La documentation
% de l'extension indique comment consulter et utiliser les fontes accessibles 
% sur votre distribution. 
%
% \medskip
% \DescribeEnv{[lgrmath]}
% En activant l'option \texttt{lgrmath}, \textsf{frenchmath}  
% charge cette extension avec son option \texttt{style=french} 
% et la fonte fcm (de l'extension \textsf{cm-lgc})
% \footnote{Évidemment il faut que \textsf{cm-lgc} soit installée
% sur votre distribution sans quoi la fonte de substitution LGR/cmr/m/n sera utilisée.}.
% Celle-ci se marie particulièrement bien avec la police usuelle Latin Modern.
% Les commandes |\alpha|, |\beta|, etc.
% produisent alors les lettres en forme droite 
% $\alphaup$, $\betaup$, \ldots, $\piup$, etc.,
% tandis que |\alphait|, |\betait|, etc. produisent des formes italiques
% $\alphait$, $\betait$, \ldots, $\piit$, etc.
% Ces dernières sont peu à notre goût, mais on conserve les lettres
% d'origine avec |\italpha|, |\itbeta| : $\italpha$, $\itbeta$, \ldots, $\itpi$, etc.
% On peut aussi choisir d'autres fontes en chargeant l'extension
% \textsf{lgrmath} indépendamment de \textsf{frenchmath}
% (voir par exemple avec l'option \texttt{font=Alegreya-LF} ou \texttt{font=Cochineal-LF}).
%
% \medskip
% \DescribeEnv{[upgreek]}
% Avec la même philosophie, \textsf{frenchmath} fournit aussi l'option \texttt{upgreek}
% basée sur l'extension \textsf{upgreek} de Walter Schmidt~\cite{UPGREEK} qui donne
% accès à d'autres fontes de lettres grecques minuscules en forme droite avec 
% |\upalpha|, |\upbeta|, etc. 
% L'extension \textsf{upgreek} sera chargée avec son option \texttt{Symbol}
% \footnote{L'option	\texttt{Symbol} de \textsf{upgreek} se marie bien 
% avec une police comme Times par exemple.}
% utilisant la police Adobe Symbol et qui produit des lettres grecques assez grasses : 
% $\upalphasym$, $\upbetasym$, \ldots, $\uppisym$, etc.
% Si l'on veut, par contre, utiliser l'extension \textsf{upgreek}
% avec l'une des deux autres options disponibles, \texttt{Euler}
% (qui produit $\upalpha, \upbeta, \ldots, \uppi$, etc.)
% ou \texttt{Symbolsmallscale},
% il faut charger l'extension \textsf{upgreek} avec l'option souhaitée 
% indépendamment de \textsf{frenchmath}.
% Mais en conservant toutefois l'option \texttt{upgreek}, \textsf{frenchmath} 
% redéfinira les commandes |\alpha|, |\beta|, etc. 
% pour composer automatiquement les lettres en forme droite ; 
% les formes italiques d'origine, 
% restant toujours disponibles avec les commandes 
% |\italpha|, |\itbeta|, \ldots, |\itpi|, etc.
%
% \medskip
% \DescribeEnv{[Upgreek]}
% Avec \LaTeX, les lettres grecques majuscules sont automatiquement composées
% en forme droite et l'option \texttt{upgreek} ne concerne que les minuscules.
% Cependant l'extension \textsf{upgreek}
% propose aussi |\Upgamma|, \ldots, |Upomega| : $\Upgamma, \ldots, \Upomega$.
% Afin de conserver majuscules et minuscules dans le même style, 
% \textsf{frenchmath} fournit l'option \texttt{Upgreek} qui redéfinit les majuscules 
% |\Gamma|, \ldots, |\Omega| pour correspondre à ces variantes.
% Par contre l'on n'a alors plus accès aux caractères d'origine : $\Gamma, \ldots, \Omega$.
% L'option \texttt{Upgreek} reprend aussi
% les minuscules grecques de l'option \texttt{upgreek}, qu'il est donc 
% inutile d'invoquer simultanément.
%
% \medskip
% Signalons aussi l'extension \textsf{textalpha} de Günter Milde~\cite{ALPHA} qui 
% donne accès aux lettres en forme droite \textalpha, \textbeta, \ldots,
% \textpi, etc.,
% mais en mode texte avec |\textalpha|, |\textbeta|, etc.
% Ces glyphes, voisines de la fonte fcm accessible avec \textsf{lgrmath},
% se marient également bien avec la police Latin Modern,
% par contre le thêta produit, \texttheta, n'est pas vraiment 
% celui qui est d'usage en mathématiques.
%
% \medskip
% Mentionnons pour finir ce commentaire de Walter Schmidt~\cite{UPGREEK} que le mu
% utilisé pour le préfixe des unités physiques, \textmicro, doit se composer avec |\textmu|
% \footnote{L'extension \textsf{textalpha} fournit à la place \texttt{\string\textmicro}
% (depuis 2020) car elle redéfinit \texttt{\string\textmu}.},
% disponible en mode texte dans beaucoup de fontes (ou avec \textsf{textcomp}) ;
% il diffère du $\mu$ droit mathématique.
% \StopEventually{}
%
% \section{Le code}
%
%    \begin{macrocode}
\newif\ifcapsit
\DeclareOption{capsit}{\capsittrue}
\newif\iffrenchmathgreek % pour sauvegarder les lettres d'origine
\newif\iflgrmath
\DeclareOption{lgrmath}{\lgrmathtrue\frenchmathgreektrue}
\newif\ifupgreek
\DeclareOption{upgreek}{\upgreektrue\frenchmathgreektrue}
\newif\ifUpgreek
\DeclareOption{Upgreek}{\Upgreektrue\upgreektrue\frenchmathgreektrue}
\newif\ifnoibrackets
\DeclareOption{noibrackets}{\noibracketstrue}
\ProcessOptions \relax

\AtBeginDocument{
    \@ifpackageloaded{mathdesign}{
        \PackageInfo{frenchmath}{Package mathdesign 
            is loaded, \MessageBreak
            I don't load mathrsfs and amssymb packages}
    }{
        \RequirePackage{amssymb} % \leqslant, \geqslant, \varnothing
        \@ifundefined{mathscr}{\RequirePackage{mathrsfs}}{}
    }
}
\RequirePackage{amsopn} % fournit \DeclareMathOperator
\@ifpackageloaded{mathptmx}{\RequirePackage{dotlessj}}{}

%    \end{macrocode}
% Ce n'est qu'à la fin du préambule, donc avec |\AtBeginDocument|, que l'on examine
% si les extensions \textsf{lgrmath} ou \textsf{upgreek} sont déjà chargées,
% pour laisser à l'utilisateur la possibilité de les charger après \textsf{frenchmath}
% et éviter un conflit d'option.
%    \begin{macrocode}
\AtBeginDocument{
  \iflgrmath
    \@ifpackageloaded{lgrmath}{}{
        \RequirePackage[font=fcm,style=french]{lgrmath}}
  \fi
  \ifupgreek
    \@ifpackageloaded{upgreek}{}{
        \RequirePackage[Symbol]{upgreek}}
  \fi
}
\RequirePackage{xspace} % utile pour les commandes \ssi, \Oij
\ifnoibrackets\else\RequirePackage{ibrackets}\fi % intelligent brackets

\RequirePackage{decimalcomma} %  depuis frenchmath 2.7

%    \end{macrocode}
%
% \DescribeMacro{\DeclareMathUp}
% Sauf si l'option \texttt{capsit} est activée, on redéfinit toutes les lettres majuscules
% du mode mathématique grâce à la commande |\DeclareMathUp|.
% Contrairement aux bascules |\MathUp| et |\MathIt| de l'extension \textsf{mismath},
% |\DeclareMathUp| ne fonctionne que dans le préambule, mais son code est bien plus simple
% et suffit à nos besoins ici.
% Pour balayer toutes les lettres majuscules de l'alphabet,
% les boucles usuelles |\@for| ou |\foreach| 
% ne fonctionnent pas et produisent un message d'erreur 
% \og\texttt{!~Improper alphabetic constant}\fg.
% Ceci est dû au fait qu'une \emph{séquence de contrôle} comme par exemple |\@letter| 
% ne peut être utilisée comme argument de la macro |\DeclareMathUp|
% et ne sera pas gérée de manière identique à un simple caractère.
% \DescribeMacro{\apply}
% Mais la macro |\apply| ci-dessous va faire le travail
% \footnote{Cette puissante macro, postée le 26/02/2021 sous le pseudonyme de \emph{wipet},
% se trouve sur le forum de discussion \texttt{iterate190.rssing.com},
% en réponse à \og TeX How to iterate over a comma separated list$?$\fg.
% Que son auteur, Petr Olšák, soit remercié.
% Cette macro a été définie de manière identique dans \textsf{mismath}, 
% mais cela ne génère pas d'incompatibilité.}.
% |\AtBeginDocument| est nécessaire pour que ces définitions 
% soient prises en compte avec la classe \textsf{beamer} par exemple.
% \smallskip
%    \begin{macrocode}
\newcommand*\DeclareMathUp[1]{
    \DeclareMathSymbol{#1}{\mathalpha}{operators}{`#1}}
    
\def\apply#1#2{\apply@#1#2,\apply@,}
\def\apply@#1#2,{\ifx\apply@#2\empty
    \else #1{#2}\afterfi@{\apply@#1}\fi}
\def\afterfi@#1#2\fi{\fi#1}

\ifcapsit\else
  \AtBeginDocument{
    \apply\DeclareMathUp{A,B,C,D,E,F,G,H,I,J,K,L,M,N,O,P,Q,R,S,T,U,V,W,X,Y,Z}
  }
\fi    

\AtBeginDocument{\@ifpackageloaded{pstricks-add}{\psset{comma=true}}{}}
\DeclareMathSymbol{;}{\mathbin}{operators}{'73} % \mathpunct à l'origine

%    \end{macrocode}
% Passons aux alias et identifiants de fonctions classiques.
%    \begin{macrocode}
\newcommand\curs{\mathscr}
\newcommand\infeg{\leqslant}
\newcommand\supeg{\geqslant}
\newcommand\vide{\varnothing}
%    \end{macrocode}
% La définition de |\paral| remplace, depuis la version 2.2,
% l'ancienne définition plus simple |\mathrel{/\!\!/}|, mais qui donnait des barres 
% trop serrées avec \textsf{mathastext} + \textsf{times} ou avec \textsf{libertinust1math}.
% Merci à Jean-François Burnol de me l'avoir fait remarquer 
% et pour ses suggestions dans la mise au point d'une macro plus efficace.
%    \begin{macrocode}
\newcommand\paral{\mathrel{\ooalign{$\mkern-1.75mu/\mkern1.75mu$\cr%
    $\mkern1.75mu/\mkern-1.75mu$}}}
\newcommand\ssi{si, et seulement si,\xspace}
\newcommand*\cmod[1]{\quad[#1]}

\DeclareMathOperator{\pgcd}{pgcd}
\DeclareMathOperator{\ppcm}{ppcm}
\DeclareMathOperator{\card}{card}
\DeclareMathOperator{\Card}{Card}
\DeclareMathOperator{\Ker}{Ker}
\DeclareMathOperator{\Hom}{Hom}
\DeclareMathOperator{\rg}{rg}
\DeclareMathOperator{\Vect}{Vect}
\DeclareMathOperator{\ch}{ch}
\DeclareMathOperator{\sh}{sh}
\AtBeginDocument{\let\oldth\th %\th existe déjà (mode texte)
    \renewcommand\th{\TextOrMath{\oldth}{\operatorname{th}}}}
\DeclareMathOperator{\cosec}{cosec}
\DeclareMathOperator{\cosech}{cosech}

%    \end{macrocode}
% Présentons les commandes pour les bases et repères, qui peuvent être utilisées
% en mode texte grâce à |\ensuremath| (et |\xspace| qui garantit le bon espacement).
%    \begin{macrocode}
\newcommand\@Oij{%
    \ensuremath{\left(O, \vec{\imath}, \vec{\jmath}\,\right)}\xspace}
\newcommand\@@Oij{%
    \ensuremath{\left(O ; \vec{\imath}, \vec{\jmath}\,\right)}\xspace}
\newcommand\Oij{\@ifstar{\@@Oij}{\@Oij}}

\newcommand\@Oijk{%
    \ensuremath{%
        \left(O, \vec{\vphantom{t}\imath}, \vec{\vphantom{t}\jmath},
        \vec{\vphantom{t}\smash{k}}\,\right)}%
    \xspace}
\newcommand\@@Oijk{%
    \ensuremath{%
        \left(O ; \vec{\vphantom{t}\imath}, \vec{\vphantom{t}\jmath},
        \vec{\vphantom{t}\smash{k}}\,\right)}%
    \xspace}
\newcommand\Oijk{\@ifstar{\@@Oijk}{\@Oijk}}

\newcommand\@Ouv{%
    \ensuremath{\left(O, \vec{u}, \vec{v}\,\right)}\xspace}
\newcommand\@@Ouv{%
    \ensuremath{\left(O ; \vec{u}, \vec{v}\,\right)}\xspace}
\newcommand\Ouv{\@ifstar{\@@Ouv}{\@Ouv}}

\AtBeginDocument{
    \renewcommand\ij{%
        \ensuremath{\left(\vec{\imath}, \vec{\jmath}\,\right)}\xspace}}
\newcommand\ijk{%
    \ensuremath{%
        \left(\vec{\vphantom{t}\imath}, \vec{\vphantom{t}\jmath},
        \vec{\vphantom{t}\smash{k}}\,\right)}%
    \xspace}

%    \end{macrocode}
% On sauvegarde les lettres grecques d'origine dans les macros |\italpha|, \ldots, |\itomega|.
% Le booléen |frenchmathgreek| a été activé par les options de lettres grecques
% \texttt{lgrmath}, \texttt{upgreek} ou \texttt{Upgreek}.
%    \begin{macrocode}
\iffrenchmathgreek
    \@ifundefined{italpha}{\let\italpha\alpha}{
        \PackageWarningNoLine{frenchmath}{Command
            \string\italpha\space already exist \MessageBreak
            and will not be redefined, \MessageBreak
            no more warning for the other Greek letters, \MessageBreak
            except pi}
    }
    \@ifundefined{itbeta}{\let\itbeta\beta}{}
    \@ifundefined{itgamma}{\let\itgamma\gamma}{}
    \@ifundefined{itdelta}{\let\itdelta\delta}{}
    \@ifundefined{itepsilon}{\let\itepsilon\epsilon}{}
    \@ifundefined{itzeta}{\let\itzeta\zeta}{}
    \@ifundefined{iteta}{\let\iteta\eta}{}
    \@ifundefined{ittheta}{\let\ittheta\theta}{}
    \@ifundefined{itiota}{\let\itiota\iota}{}
    \@ifundefined{itkappa}{\let\itkappa\kappa}{}
    \@ifundefined{itlambda}{\let\itlambda\lambda}{}
    \@ifundefined{itmu}{\let\itmu\mu}{}
    \@ifundefined{itnu}{\let\itnu\nu}{}
    \@ifundefined{itxi}{\let\itxi\xi}{}
    \@ifundefined{itpi}{\let\itpi\pi}{
       \PackageWarningNoLine{frenchmath}{Command
            \string\itpi\space already exist \MessageBreak
            and will not be redefined}
    }
    \@ifundefined{itrho}{\let\itrho\rho}{}
    \@ifundefined{itsigma}{\let\itsigma\sigma}{}
    \@ifundefined{ittau}{\let\ittau\tau}{}
    \@ifundefined{itupsilon}{\let\itupsilon\upsilon}{}
    \@ifundefined{itphi}{\let\itphi\phi}{}
    \@ifundefined{itchi}{\let\itchi\chi}{}
    \@ifundefined{itpsi}{\let\itpsi\psi}{}
    \@ifundefined{itomega}{\let\itomega\omega}{}
    \@ifundefined{itvarepsilon}{\let\itvarepsilon\varepsilon}{}
    \@ifundefined{itvartheta}{\let\itvartheta\vartheta}{}
    \@ifundefined{itvarpi}{\let\itvarpi\varpi}{}
    \@ifundefined{itvarsigma}{\let\itvarsigma\varsigma}{}
    \@ifundefined{itvarphi}{\let\itvarphi\varphi}{}
\fi

\AtBeginDocument{
  \ifupgreek
    \renewcommand\alpha{\upalpha}
    \renewcommand\beta{\upbeta}
    \renewcommand\gamma{\upgamma}
    \renewcommand\delta{\updelta}
    \renewcommand\epsilon{\upepsilon}
    \renewcommand\zeta{\upzeta}
    \renewcommand\eta{\upeta}
    \renewcommand\theta{\uptheta}
    \renewcommand\iota{\upiota}
    \renewcommand\kappa{\upkappa}
    \renewcommand\lambda{\uplambda}
    \renewcommand\mu{\upmu}
    \renewcommand\nu{\upnu}
    \renewcommand\xi{\upxi}
    \renewcommand\pi{\uppi}
    \renewcommand\rho{\uprho}
    \renewcommand\sigma{\upsigma}
    \renewcommand\tau{\uptau}
    \renewcommand\upsilon{\upupsilon}
    \renewcommand\phi{\upphi}
    \renewcommand\chi{\upchi}
    \renewcommand\psi{\uppsi}
    \renewcommand\omega{\upomega}
    \renewcommand\varepsilon{\upvarepsilon}
    \renewcommand\vartheta{\upvartheta}
    \renewcommand\varpi{\upvarpi}
    \renewcommand\varrho{\upvarrho}
    \renewcommand\varsigma{\upvarsigma}
    \renewcommand\varphi{\upvarphi}
  \fi

  \ifUpgreek
    \renewcommand\Gamma{\Upgamma}
    \renewcommand\Delta{\Updelta}
    \renewcommand\Theta{\Uptheta}
    \renewcommand\Lambda{\Uplambda}
    \renewcommand\Xi{\Upxi}
    \renewcommand\Pi{\Uppi}
    \renewcommand\Sigma{\Upsigma}
    \renewcommand\Upsilon{\Upupsilon}
    \renewcommand\Phi{\Upphi}
    \renewcommand\Psi{\Uppsi}
    \renewcommand\Omega{\Upomega}
  \fi
}
%    \end{macrocode}
%
% \medskip
% \begin{thebibliography}{29}
% \begin{raggedright}
% \bibitem{RTIN} \emph{Lexique des règles typographiques en usage à l’Imprimerie Nationale},
% édition du 26/08/2002.
% \bibitem{IGEN} \emph{Composition des textes scientifiques},
% Inspection Générale de mathématiques (IGEN-DESCO), 06/12/2001.
% \url{http://mslp.ac-dijon.fr/IMG/pdf/typo_txt_sci.pdf}\\
% \url{https://euler.ac-versailles.fr/IMG/pdf/typo2.pdf}
% \bibitem{AA} \emph{Règles françaises de typographie mathématique},
% Alexandre André, 02/09/2015.
% \url{http://sgalex.free.fr/typo-maths_fr.pdf}
% \bibitem{ES} \emph{Le petit typographe rationnel}, Eddie Saudrais, 20/03/2000.
% \url{https://www.gutenberg-asso.fr/IMG/pdf/saudrais-typo.pdf}
% \bibitem{TYPMA} \emph{Typesetting mathematics for science and technology according 
% to ISO 31/XI}, Claudio Beccari, TUGboat Volume 18 (1997), \No1.
% \url{http://www.tug.org/TUGboat/tb18-1/tb54becc.pdf}
% \bibitem{NIST} \emph{Typefaces for Symbols in Scientific Manuscripts}.
% \url{https://www.physics.nist.gov/cuu/pdf/typefaces.pdf}
% \bibitem{ICTNS} \emph{On the Use of Italic and up Fonts for Symbols in Scientific Text},
% I.M.~Mills and W.V.~Metanomski, ICTNS (Interdivisional Committee on Terminology,
% Nomenclature and Symbols), dec 1999.
%\url{https://old.iupac.org/standing/idcns/italic-roman_dec99.pdf}
% \bibitem{ISOM} \emph{\textsf{isomath} -- Mathematical style for science and technology},
% Günter Milde, CTAN, v0.6.1 2012/09/04.
% \bibitem{PMISO} \emph{\textsf{PM-ISOmath} -- The Poor Man ISO math bundle}, 
% Claudio Beccari, CTAN, v1.2.00 2021/08/04.
% \bibitem{MAFR} \emph{La distribution \textsf{mafr}}, Christian Obrecht, 
% CTAN, v1.0 17/09/2006.
% \bibitem{FOUR} \textsf{Fourier-GUT\hspace{-0.1em}\emph{enberg}},
% Michel Bovani, CTAN, v1.3 30/01/2005.
% \bibitem{DESIGN} \emph{The \textsf{mathdesign} package},
% Paul Pichaureau, CTAN, v2.31 29/08/2013.
% \bibitem{MIS} \emph{\textsf{mismath} -- Miscellaneous mathematical macros},
% Antoine Missier, CTAN, v2.8 27/07/2023.
% \bibitem{VECT} \emph{\textsf{esvect}  -- Typesetting vectors with beautiful 
% arrow with \LaTeXe}, Eddie Saudrais, CTAN, v1.3 11/07/2013.
% \bibitem{TABL} \emph{L'extension \textsf{tablvar}}, Antoine Missier,
% CTAN, v2.0 23/12/2023.
% \bibitem{FRM} \emph{L'extension \textsf{tdsfrmath}}, Yvon Henel, 
% CTAN, v1.3 22/06/2009.
% \bibitem{BABEL} \emph{A Babel language definition file for French}, extension \LaTeX\ 
% \textsf{babel-french} de Daniel Flipo, CTAN, v3.5c 14/09/2018.
% \bibitem{ICOMMA} \emph{The \textsf{icomma} package for \LaTeXe}. 
% Walter Schmidt, CTAN, v2.0 10/03/2002.
% \bibitem{NCC} \emph{The \textsf{ncccomma} package}. Alexander I.~Rozhenko, 
% CTAN, v1.0 10/02/2005.
% \bibitem{BECC} \emph{Intelligent commas}. Claudio Beccari, The Prac\TeX\ Journal, 
% 2011, No.\@1.
% \url{https://tug.org/pracjourn/2011-1/beccari/Intcomma.pdf}
% \bibitem{DECC} \emph{The \textsf{decimalcomma} package}. Antoine Missier,
% CTAN, v1.1 19/12/2023.
% \bibitem{BRACKETS} \emph{Intelligent brackets -- The \textsf{ibrackets} package},
% Antoine Missier, v1.1 26/12/2022.
% \bibitem{MATAL} \emph{The \textsf{mathalpha}, \textsc{aka} \textsf{mathalfa} package}, 
% Michael Sharpe, CTAN, v1.143 18/11/2021.
% \bibitem{DLJ} \emph{\textsf{dotlessj}}, David Carlisle, CTAN, v0.03 09/12/1998.
% \bibitem{KPF} \emph{\textsf{Kp-Fonts} -- The Johannes Kepler project},
% Christophe Caignaert, CTAN, v3.34 20/09/2022.
% \bibitem{LGR} \emph{The \textsf{lgrmath} package}, Jean-François B., CTAN, 
% v1.0 16/11/2022.
% \bibitem{UPGREEK} \emph{The \textsf{upgreek} package for \LaTeXe}, Walter Schmidt, CTAN, 
% v2.0 12/03/2003.
% \bibitem{ALPHA} \emph{The \textsf{textalpha} package}
% (partie de l'extension \textsf{greek-fontenc}), Günter Milde, CTAN, v2.1 14/06/2022.
% \bibitem{COMP} \emph{\LaTeX\ Companion}, Frank Mittelbach, Michel Goossens,
% 2\ieme édition, Pearson Education France, 2005.
% \end{raggedright}
% \end{thebibliography}

% \Finale
\endinput

