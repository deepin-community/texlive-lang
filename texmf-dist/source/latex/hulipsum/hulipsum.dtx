% \iffalse meta-comment
%
% Copyright 2018-2024 by Tibor Tomacs
%
% Hungarian dummy text (Lorum ipse) for demonstrating the textual elements of a document template
% All paragraphs are taken with permission from http://www.lorumipse.hu/. 
% Thanks to Lorum Ipse Lab (Viktor Nagy and David Takacs) for their work.
%
% This work may be distributed and/or modified under the
% conditions of the LaTeX Project Public License, either version 1.3
% of this license or (at your option) any later version.
% The latest version of this license is in
%   http://www.latex-project.org/lppl.txt
% and version 1.3 or later is part of all distributions of LaTeX
% version 2005/12/01 or later.
%
% This work has the LPPL maintenance status `maintained'.
% 
% The Current Maintainer of this work is Tibor Tomacs.
%
% \fi
%
% \iffalse
%<*driver>
\ProvidesFile{hulipsum.dtx}
%</driver>
%<package>\NeedsTeXFormat{LaTeX2e}[2022/06/20]
%<package>\ProvidesPackage{hulipsum}[2024/04/12 v1.3 Hungarian dummy text (Lorum ipse)]
%
%<*driver>
\documentclass{ltxdoc}
\OnlyDescription
\AddToHook{begindocument/before}[doc/hyperref]{\hypersetup{pdfstartview=FitH,colorlinks,allcolors=black}}
\usepackage[a4paper,margin=25mm]{geometry}
\usepackage[T1]{fontenc}
\usepackage[english]{babel}
\usepackage{listings,xcolor}
\setlength{\labelsep}{0pt}

\newcommand{\commandinline}{\lstinline[
literate={<}{{$\langle$}}1{>}{{$\rangle$}}1,
delim={[is][\color{green!50!black}\normalfont\itshape]{!}{!}},
basicstyle=\color{blue!80!black}\ttfamily,
columns=fullflexible,
keepspaces]}

\newcommand{\verbinline}{\lstinline[
literate={<}{{$\langle$}}1{>}{{$\rangle$}}1,
delim={[is][\color{green!50!black}\normalfont\itshape]{!}{!}},
basicstyle=\ttfamily,
columns=fullflexible,
keepspaces]}

\begin{document}
    \DocInput{./hulipsum.dtx}
\end{document}
%</driver>
% \fi
%
% \GetFileInfo{hulipsum.sty}
%
% \title{The {\bfseries\sffamily hulipsum} package\\{\large v1.3 (2024/04/12)}}
% \author{Tibor Tómács\\{\normalsize\href{mailto:tomacs.tibor@gmail.com}{\texttt{tomacs.tibor@gmail.com}}}}
% \date{}
% \maketitle
% \thispagestyle{empty}
% 
% The \emph{Lorem ipsum} is an improper Latin filler dummy text. It is commonly used for demonstrating the textual elements of a document template (see \url{https://en.wikipedia.org/wiki/Lorem_ipsum}). The \emph{Lórum ipse} is a Hungarian variation of the \emph{Lorem ipsum}. (\emph{Lórum} is a Hungarian card game, and \emph{ipse} is a Hungarian slang word, it means \emph{bloke}.)
% 
% With the \texttt{hulipsum} package you can typeset 150 paragraphs of \emph{Lórum ipse}. All paragraphs are taken with permission from \url{http://www.lorumipse.hu/}. Thanks to \emph{Lórum Ipse Lab} (Viktor Nagy and Dávid Takács) for their work.
% 
% \section*{Usage}
% Load the package as usual, with
% \begin{flushleft}
% \commandinline|\usepackage{hulipsum}|
% \end{flushleft}
% in the preamble of your document. This package provides several macros:
% \begin{description}
% \item\commandinline|\hulipsum[!<num1>!-!<num2>!]|\\
% The \commandinline|!<num1>!| and \commandinline|!<num2>!| are positive integers, and $1\leq{}$\commandinline|!<num1>!|${}\leq{}$\commandinline|!<num2>!|${}\leq150$. This macro typesets the \emph{Lórum ipse} paragraphs \commandinline|!<num1>!| to \commandinline|!<num2>!|. If \commandinline|!<num1>!|${}={}$\commandinline|!<num2>!|, then it typesets the \commandinline|!<num1>!|\textsuperscript{th} paragraph. The paragraphs will be separated by the \verbinline|\par| macro.
% 
% \begin{tabular}{@{}l@{ is equivalent to }l}
% \commandinline|\hulipsum|           & \verbinline|\hulipsum[1-7]|.\\
% \commandinline|\hulipsum[-]|        & \verbinline|\hulipsum[1-150]|.\\
% \commandinline|\hulipsum[-!<num>!]| & \verbinline|\hulipsum[1-!<num>!]|.\\
% \commandinline|\hulipsum[!<num>!-]| & \verbinline|\hulipsum[!<num>!-150]|.\\
% \commandinline|\hulipsum[!<num>!]|  & \verbinline|\hulipsum[!<num>!-!<num>!]|.
% \end{tabular} 
% 
% \item\commandinline|\sethulipsumdefault{!<value>!}|\\
% After this the default option of \verbinline|\hulipsum| will be \commandinline|!<value>!|. By default the \commandinline|!<value>!| is set to \texttt{1-7}. For example using \verbinline|\hulipsum| after \verbinline|\sethulipsumdefault{10-15}|, the result will be equivalent to \verbinline|\hulipsum[10-15]|.
% 
% \item\commandinline|\hulipsum*[!<num1>!-!<num2>!]|\\
% It works like \verbinline|\hulipsum|, but it omits the insertion of \verbinline|\par| after each paragraph and inserts space instead.
% 
% \item\commandinline|\hulipsumsave[!<num1>!-!<num2>!]|\\
% It works like \verbinline|\hulipsum|, except that instead of typesetting the paragraphs, it saves the mere text of paragraphs into the \commandinline|\hulipsumexp|. The paragraphs will be separated by the \verbinline|\par| macro.
%
% \verbinline|\hulipsumsave[!<num1>!-!<num2>!]\hulipsumexp| is equivalent to \verbinline|\hulipsum[!<num1>!-!<num2>!]|.
% 
% \item\commandinline|\hulipsumsave*[!<num1>!-!<num2>!]|\\
% It works like \verbinline|\hulipsumsave|, but the paragraphs will be separated by space in the \verbinline|\hulipsumexp|.
%
% \verbinline|\hulipsumsave*[!<num1>!-!<num2>!]\hulipsumexp| is equivalent to \verbinline|\hulipsum*[!<num1>!-!<num2>!]|.
% 
% \item\commandinline|\hulipsumdocument[!<options>!]|\\
% It creates a blind document with title, author, date, table of contents, part, chapter, section, subsection, subsubsection, paragraph, subparagraph, figure, lists, equations and bibliography.
%
% The \commandinline|!<options>!| are:
%
% \begin{description}
% \item \commandinline|maketitle=!<boolean>!| \hfill (default: \texttt{true})\\
% Adds \verbinline|\maketitle|, if the \commandinline|!<boolean>!| is \texttt{true}.
%
% \item \commandinline|tableofcontents=!<boolean>!| \hfill (default: \texttt{true})\\
% Adds \verbinline|\tableofcontents|, if the \commandinline|!<boolean>!| is \texttt{true}.
%
% \item \commandinline|part=!<boolean>!| \hfill (default: \texttt{false})\\
% Adds \verbinline|\part{...}| before the chapter and section, if the \commandinline|!<boolean>!| is \texttt{true}.
%
% \item \commandinline|abstract=!<boolean>!| \hfill (default: \texttt{true})\\
% Adds an \texttt{abstract} environment, if the \commandinline|!<boolean>!| is \texttt{true}.
%
% \item \commandinline|math=!<boolean>!| \hfill (default: \texttt{true})\\
% Adds a mathematical formula, if the \commandinline|!<boolean>!| is \texttt{true}.
%
% \item \commandinline|bibliography=!<boolean>!| \hfill (default: \texttt{true})\\
% Adds \texttt{thebibliography} environment, if the \commandinline|!<boolean>!| is \texttt{true}.
% \end{description}
% The \texttt{=true} can be omitted in the options. For example \verbinline|\hulipsumdocument[part,math=false]|.
% \end{description}
% \StopEventually
%
%    \begin{macrocode}
%%
\newif\if@hulipsum@num@
\newif\if@hulipsum@minnum@
\newif\if@hulipsum@par@
\@hulipsum@par@true
\newcounter{hulipsum@count}
\def\hulipsumexp{}

\def\sethulipsumdefault#1{\def\hulipsum@default{#1}\hulipsumsave}
\def\hulipsum@default{1-7}

\def\hulipsum@get#1-#2;{\def\hulipsum@min{#1}\def\hulipsum@max{#2}}
\def\hulipsum@stripmax#1-{\def\hulipsum@max{#1}}
\def\hulipsum@relax{\relax}
\def\hulipsum@minmax#1{%
  \hulipsum@get#1-\relax;%
  \ifx\hulipsum@max\hulipsum@relax\def\hulipsum@max{\hulipsum@min}%
      \else\expandafter\hulipsum@stripmax\hulipsum@max\fi%
  \ifx\hulipsum@min\@empty\def\hulipsum@min{1}\fi%
  \ifx\hulipsum@max\@empty\def\hulipsum@max{150}\fi%
}

\def\hulipsumsave{%
  \@ifstar\@@hulipsumsave\@hulipsumsave
}

\newcommand{\@@hulipsumsave}[1][\hulipsum@default]{%
  \@hulipsum@par@false%
  \@hulipsumsave[#1]%
  \@hulipsum@par@true%
}

\def\hulipsum@checking@integer#1{%
  \afterassignment\hulipsum@get@args\count@=0#1\hfuzz#1\hfuzz%
}
\def\hulipsum@get@args#1\hfuzz#2\hfuzz{%
  \if\relax\detokenize{#1}\relax
    \@hulipsum@num@true%
  \else
    \@hulipsum@num@false%
  \fi
}

\def\hulipsum@checking{%
  \hulipsum@checking@integer{\hulipsum@min}%
  \if@hulipsum@num@%
    \ifnum \hulipsum@min=0\@latexerr{\hulipsum@min\space is not positive number}{}\fi%
    \@hulipsum@minnum@true%
  \else
    \@latexerr{\hulipsum@min\space is not number}{}%
    \@hulipsum@minnum@false%
  \fi
  \hulipsum@checking@integer{\hulipsum@max}%
  \if@hulipsum@num@%
    \ifnum \hulipsum@max=0\@latexerr{\hulipsum@max\space is not positive number}{}\fi%
    \ifnum \hulipsum@max>150\@latexerr{\hulipsum@max\space is greater than 150}{}\fi%
    \if@hulipsum@minnum@\ifnum \hulipsum@min>\hulipsum@max\@latexerr{\hulipsum@min\space is greater than \hulipsum@max}{}\fi\fi%
  \else
    \@latexerr{\hulipsum@max\space is not number}{}%
  \fi
}

\newcommand{\@hulipsumsave}[1][\hulipsum@default]{%
  \expandafter\hulipsum@minmax\expandafter{#1}%
  \hulipsum@checking%
  \setcounter{hulipsum@count}{\hulipsum@min}%
  \addtocounter{hulipsum@count}{-1}%
  \def\hulipsumexp{}%
  \@whilenum\value{hulipsum@count}<\hulipsum@max\do{%
    \stepcounter{hulipsum@count}%
    \csname hulipsum@\roman{hulipsum@count}\endcsname%
    \ifnum\value{hulipsum@count}<\hulipsum@max%
      \if@hulipsum@par@\g@addto@macro\hulipsumexp{\par }%
        \else\g@addto@macro\hulipsumexp{ }%
      \fi%
    \fi%
    }%
}

\def\hulipsum{%
  \@ifstar\@@hulipsum\@hulipsum
}

\newcommand{\@@hulipsum}[1][\hulipsum@default]{%
  \hulipsumsave*[#1]\hulipsumexp%
}

\newcommand{\@hulipsum}[1][\hulipsum@default]{%
  \hulipsumsave[#1]\hulipsumexp%
}

\AtEndOfPackage{\hulipsumsave}

\newif\if@hulipsum@firstexpand@
\@hulipsum@firstexpand@true

\newcommand{\hulipsumdocument}[1][]{
\if@hulipsum@firstexpand@
\global\@hulipsum@firstexpand@false
\begingroup
\DeclareKeys{
  maketitle.if=@hulipsum@maketitle,
  tableofcontents.if=@hulipsum@tableofcontents,
  part.if=@hulipsum@part,
  abstract.if=@hulipsum@abstract,
  math.if=@hulipsum@math,
  bibliography.if=@hulipsum@bibliography}
\SetKeys{maketitle,part=false,abstract,tableofcontents,math,bibliography,#1}
\if@hulipsum@maketitle
\@ifundefined{title}{}{\title{L\'{o}rum ipse}}
\@ifundefined{author}{}{\author{Nyomasek Bob\'{o}}}
\@ifundefined{supervisor}{}{\supervisor{Dr. Szel\'{e}s Gerg\H{o}\\ egyetemi docens}\author{Nyomasek Bob\'{o}\\ matematika szak}}
\@ifundefined{institute}{}{\institute{Matematikai és Informatikai Intézet}}
\@ifundefined{city}{}{\city{Eger}}
\@ifundefined{maketitle}{}{\maketitle}
\fi
\if@hulipsum@tableofcontents\@ifundefined{tableofcontents}{}{\tableofcontents}\fi
\if@hulipsum@abstract\@ifundefined{abstract}{}{\begin{abstract}\hulipsum[1]\end{abstract}}\fi
\if@hulipsum@part\@ifundefined{part}{}{\part{Borzaszt\'{o}an cog\'{a}lis pat\'{a}s}}\fi
\@ifundefined{chapter}{}{\chapter{Fog\'{a}s n\'{e}lk\"{u}l nem vark\'{a}l}}
\@ifundefined{section}{}{\section{A vandoba h\'{e}t matlan}}
\hulipsum[1-2]
\@ifundefined{subsection}{\par}{\subsection{Az \"{u}tles \"{o}nc\'{e}je}}
\hulipsum[3]
\@ifundefined{subsubsection}{\par}{\subsubsection{Gyarokok ed\'{e}se}}
\hulipsum[4]
\@ifundefined{paragraph}{\par}{\paragraph{Szel\'{e}s szerg\H{o}}}
\hulipsum[5]
\@ifundefined{subparagraph}{\par}{\subparagraph{Anyatos szamiss\'{a}g}}
\hulipsum[6]
\@ifundefined{section}{\par}{\section{Alamt\'{o}l hi\'{u}s vagy civatalbeli \'{a}llott dugl\'{o}it}}
\hulipsum[7-10]
\@ifundefined{figure}{\par}{%
\begin{figure}[ht!]
\centering\fbox{\fbox{\rule{0pt}{.4\textwidth}\rule{.8\textwidth}{0pt}}}
\caption{Kal\'{a}rb\'{e} id\'{e}ny k\"{o}z\"{o}tt fojt\'{o}sak pialnak a hatatlat\'{o} kodajnok}
\end{figure}}
\hulipsum[11-20]
\par\medskip\emph{\hulipsum[21]}
\def\hulipsum@listitem{Az \"{u}tles \"{o}nc\'{e}je eset\'{e}n egez\H{o} pl\'{a}st k\"{o}zvetlen ad\'{o} \'{e}s f\'{e}s\H{o} evez\H{o} amus nem b\'{a}skodja, azonban az kas\'{a}got s\'{e}gi szavumnak aggat, ugyan\'{u}gy, mint a h\'{a}ms\'{a}gok \'{a}ltal sz\'{a}jozott szel\H{o} boszl\'{e}kos \'{e}s gy\"{o}nt\H{o} pl\'{a}sok.}
\@ifundefined{itemize}{}{%
\begin{itemize}
\item \hulipsum@listitem
\begin{itemize}
\item \hulipsum@listitem
\begin{itemize}
\item \hulipsum@listitem
\end{itemize}
\end{itemize}
\end{itemize}}
\@ifundefined{enumerate}{}{%
\begin{enumerate}
\item \hulipsum@listitem
\begin{enumerate}
\item \hulipsum@listitem
\begin{enumerate}
\item \hulipsum@listitem
\end{enumerate}
\end{enumerate}
\end{enumerate}}
\@ifundefined{description}{}{%
\begin{description}
\item[Lórum] \hulipsum@listitem
\begin{description}
\item[Ipse] \hulipsum@listitem
\begin{description}
\item[Vandoba] \hulipsum@listitem
\end{description}
\end{description}
\end{description}}
\if@hulipsum@math
$\mathrm{e}^x=\sum_{n=0}^\infty\frac{x^n}{n!}$ minden val\'{o}s $x$ eset\'{e}n.
\[\mathrm{e}^x=\sum_{n=0}^\infty\frac{x^n}{n!}\quad\forall x\in\mathbf{R}.\]
\fi
\if@hulipsum@bibliography
\@ifundefined{thebibliography}{}{%
\begin{thebibliography}{1}
\bibitem{a} \textsc{Nyomasek Bob\'{o}}, \emph{L\'{o}rum ipse}, Handana Kiad\'{o}, 1980.
\bibitem{b} \textsc{Szel\'{e}s Gerg\H{o}}, \emph{A griszt\'{a}s talt\'{a}s kirg\'{e}szei}, \"{U}d\"{o}sv\'{e}ny Kiad\'{o}, 2010.
\end{thebibliography}}
\fi
\endgroup
\fi
}

\def\hulipsum@i{\g@addto@macro\hulipsumexp{L\'{o}rum ipse olyan borzaszt\'{o}an cog\'{a}lis pat\'{a}s, ami fog\'{a}s n\'{e}lk\"{u}l nem vark\'{a}l megfelel\H{o}en. A vandoba h\'{e}t matlan talmatos ferodika, amelynek kap\'{a}r\'{a}s\'{a}t az izma mig\'{a}lja. A vandoba bul\'{a}i k\"{o}z\"{u}l zsibulja meg az izm\'{a}t, a porn\'{a}t, valamint a m\H{u}v\'{e}st \'{e}s v\'{a}tog a vandoba bul\'{a}inak v\'{o}k\'{a}ir\'{o}l. V\'{o}k\'{a}ja a raktil proz\'{a}sa k\'{e}t emen k\"{o}z\"{o}tt. \'{E}vente legal\'{a}bb egyszer csetnyi pipecs\'{e}lnie az ement, azon fongnia a l\'{a}ltos kap\'{a}r\'{a}sr\'{o}l \'{e}s a ny\'{a}kuum b\"{o}ll\'{e}s\'{e}r\H{o}l. A vandoba ninti \'{e}s az emen el\'{e} red\H{o}zi a szamlan radalmakan \'{e}rv\'{e}st. Az ement az izma bamz\'{a}sban, a has\'{a}s szegeszk\'{e}j\'{e}vel log\'{a}lja \"{o}ssze, legal\'{a}bb 15 nappal annak poz\'{a}sa el\H{o}tt. Az ement \"{o}ssze kell log\'{a}lnia akkor is, ha azt az \'{o}d\'{a}s legal\'{a}bb egyes bamz\'{a}sban, a reszt\H{o} billet\'{e}vel h\'{a}sodja.}}
\def\hulipsum@ii{\g@addto@macro\hulipsumexp{A vezvetben \'{e}vente t\"{o}bb mint ens\'{e}g g\'{e}szet regnet\H{o} emtesked\'{e}s szomb\'{o}dik, ennek legal\'{a}bb a 90 toruma bojtos volna a vol\'{a}sokn\'{a}l. A gatos lel\H{o}l\'{e}s ehelyett v\'{e}gz\H{o} videm \'{e}s koz\'{a}sokban, gazs\'{a}gos mohos f\"{o}lcs\"{o}k\"{o}n aros\'{i}t a fav\'{a}j\'{a}ra. Oda latolj\'{a}k a dik\'{e}ket is, ami a molyv\'{a}s f\"{o}lderen dingos emtesked\'{e}snek eshent, \'{e}rzeg (vagyis vol\'{a}s) sor\'{a}n viszont ment\H{o} alva lehetne. Az eren lany\'{o}shoz mintegy sz\"{o}vel k\"{o}bm\'{e}ternyi rad\'{o}cot kell zsib\'{a}lnia, ami a hajg\'{o} rod\'{a}s ked\H{o} mendi\'{a}j\'{a}val tetekez\H{o}. A kolangan salat szerint a rekv\'{a}sba gy\"{u}ge a s\'{a}g\'{o} \'{e}s jog\'{a}ly, a zs\"{o}rzes, a lel\H{o}k torpul\'{a}sa, a visztikus \'{e}s feres emtesked\'{e}s, a par\'{o} \'{e}s a tehet. Ezeket az alv\'{a}kat \'{e}letben, ahol a duzzadt ciszti kantatok miatt szinte fetes boros k\"{o}nyved\'{e}st nednie, fel is b\H{o}v\'{i}tik. \'{A}gy\'{e}mban a k\'{i}t\H{o} forny\'{e}kos videm 55 toruma a cel\'{o}k al\'{a} h\'{a}zik.}}
\def\hulipsum@iii{\g@addto@macro\hulipsumexp{Az \"{u}tles \"{o}nc\'{e}je eset\'{e}n egez\H{o} pl\'{a}st k\"{o}zvetlen ad\'{o} \'{e}s f\'{e}s\H{o} evez\H{o} amus nem b\'{a}skodja, azonban az kas\'{a}got s\'{e}gi szavumnak aggat, ugyan\'{u}gy, mint a h\'{a}ms\'{a}gok \'{a}ltal sz\'{a}jozott szel\H{o} boszl\'{e}kos \'{e}s gy\"{o}nt\H{o} pl\'{a}sok. Az \"{u}tlesek iztol\'{a}s\'{a}nak folos nyersz\'{e}peit a hat\'{o} sr\'{o}nia t\"{o}r\'{i}ti, amely konyagos a nyez\H{o}s kuma p\"{o}k\"{o}rg\'{e}j\'{e}n, valamint a k\'{e}zbel\H{o} kul\'{a}s zsig\'{e}s nektokain is. Az \"{u}tlest kiz\'{a}r\'{o}lag a latag\'{o} \'{a}ltal fr\'{i}g hat\'{o} j\'{o}s\'{a}gon lehet p\"{o}rgeznie, amelyet a nyez\H{o}s kuma civas\'{a}n lehet zsintnie cs\'{a}r\'{a}ban. Az \"{u}tlesek iztol\'{a}s\'{a}nak ismeng\'{e}se: omla m\'{e}ta gatla, ci\'{o}m oszf\'{a}r. Az \"{o}r\"{o}n \"{u}tlesek csak akkor szab\'{a}jdalanak el, ha formailag a sr\'{o}ni\'{a}nak mindenben koh\'{o}d\'{i}tnak, valamint t\"{o}r\'{i}tik a b\"{o}l\H{o} h\'{o}b\'{a}rokat, deszk\'{a}kat, ezg\'{e}lyeneket. A g\'{a}z\'{a}sra a csok\'{a}ci\'{o} a bosztag tad\'{e}kot k\"{u}l\"{o}s\"{o}dte fel. A sikeresen rogangok sz\'{a}m\'{a}ra a dinka sikangj\'{a}nak szotyogv\'{a}ra a j\'{a}tlan \"{u}tles fut\'{a}j\'{a}ban a tiz\H{o} b\'{e}s\'{a}g v\'{a}ts\'{a}g lekl\H{o}je, azaz tiz\H{o} m\'{e}tr\'{a}cat 1-t\H{o}l, az aks\'{a}g j\'{a}tlan \"{u}tles talockon a gy\"{u}m\"{o}s keskegyen her\'{a}ns\'{a}g eset\'{e}n tiz\H{o} t\H{o}s\'{e}g midom\'{a}s.}}
\def\hulipsum@iv{\g@addto@macro\hulipsumexp{Az is sodik, hogy ilyen, vagy olyan sp\'{a}sb\'{o}l jegyen a gyarokok ed\'{e}se. Jel\H{o}vel vol\'{o}znak tov\'{a}bb a filint\'{e}sben, mell\H{o}zik az \"{o}t\"{o}m\"{o}t, de mind kevesebb az az uhols\'{a}g, amit vol\'{a}tban csatorl\'{a}znak \'{e}rte. A sz\"{u}l\'{e}s filda ab\'{a}l szerint matlat\'{o} pudort pintercekben m\'{a}s briffek vannak a gyarokokkal. Mivel a metrit\'{a}rok sokat borcs\'{a}gnak a p\'{a}rm\'{a}nyoz\'{a}sukra, meg sajg\'{a}s is a k\"{o}d\H{o} megyere, a sz\'{a}lyog\'{a}k farogj\'{a}k a kressz\'{o}ban szencen, z\'{a}lt dez\H{o}ket. C\'{e}lzat\'{a}sra olyan izmusokat moletyk\'{a}lnak, amelyek a z\'{a}ltak. A griszt\'{a}s talt\'{a}s kirg\'{e}szei szerint ez a sp\'{a}sa annak, hogy a gyarokokat is r\'{e}melte a metrit\'{a}rok, f\H{o}leg a botottan ebelemek bib\'{a}j\'{a}ban bajgat r\'{e}versejtens\'{e}g. A z\'{a}lt csecsl\H{o}k sing\'{e}sei is ilyenek dudoznak, nem nyilm\'{a}nyosak eleget vasztnia \'{e}s ez\'{a}ltal kednie.}}
\def\hulipsum@v{\g@addto@macro\hulipsumexp{Persze lehet, hogy ez \'{u}gy sikkaszt, hogy egy gr\'{a}lyot h\'{a}tra azt\'{a}n 3 el\H{o}re, de ez semmit nem sedik a baskaron. Term\'{e}szetesen minden ir\'{a}ly egy bomtalan det\'{e}s, de m\'{e}gis a gulis a viz\H{o} \'{e}s nem csak az\'{e}rt mert a poros bags\'{a}g a log\'{o}s bomtalan szatoz\'{a}s. Hanem az\'{e}rt is, mert az eg\'{e}sz poros brus droma szarak\'{a}s arra, hogy veszthetsen a macska, amin a szel\'{e}s szerg\H{o}! Okzam a poros kulap droma hihetetlen\"{u}l talt b\H{o}veseket cserciz lehet\H{o}v\'{e} (persze a kults\'{a}gok\'{e} is valamennyire). Amit\H{o}l a viz\H{o} a v\'{e}tes sod\'{a}rl\'{a}s \"{o}nz\'{e}se, a mel\H{o}, din\'{o} vel\H{o} \'{e}s d\"{o}kli \'{a}li\'{a}ja! Egyszer\H{u}en br\'{o}nung sz\'{a}rosolt a viz\H{o} az vencegei r\'{e}v\'{e}n. Nem minden, de nagyon sok k\"{o}t\'{e}s f\"{o}l\"{o}tt zsomisk\'{a}p van.}}
\def\hulipsum@vi{\g@addto@macro\hulipsumexp{E hajcih\H{o} hitletkere a tez\H{o} cukak z\'{a}lt anyatos szamiss\'{a}g \'{a}ltal hoskony ,,csalan ad\'{a}s 2002'' z\"{o}ldi p\'{o}r\'{e}k s\'{e}t\'{a}l. Egy a j\"{o}vet\H{o} dotang v\'{e}g\'{e}n 164 terek ny\'{a}kozta meg ezt a furad\'{a}st, hogy a karmatlan k\"{o}rpestik szerint oszkozja. A matosak teny\'{e}k\'{e}ben jetes \'{e}s cserkes banik, szing\H{o}k, lencs\'{e}s \'{e}s szeges h\H{u}tletek, \'{i}biszek terekei, emenek, dugazonok, anyatos p\'{a}rgyakok, anyatos zimik vonk \'{e}s a sublika boltamagainak nyukszai voltak. T\'{i}z k\"{o}rpestib\H{o}l bucig foks\'{a}gra kellett szas\'{a}g rike buca nyirk\'{a}lkodnia. A karmatlan etletre a matosak kb.~szuvel\'{e}s h\"{o}r\"{o}k\"{o}dte vissza pedness\'{e}g\'{e}t, illetve h\'{u}zan\'{a}s veg\H{o}ben r\'{a}g\'{i}tott\'{a}k a foks\'{a}gokat a nyukszok k\"{o}z\"{o}tt is, \'{i}gy a pedness\'{e}gek f\H{u}t\'{e}se enn\'{e}l j\'{o}val v\'{a}r\'{a}s, szeredi a sz\'{a}zat. A talan sz\'{a}numok ban\'{o}r\'{a}ba tr\'{a}s\'{a}nak csalit\'{a}sa pedig az volt, hogy gy\H{u}r\H{u}j\"{u}k szerint zont\'{a}t ny\H{u}gl\H{o}dj\'{e}k, illetve ha lehet, rambalatba huny\'{i}tsanak zont\'{a}val. A seremesz b\'{e}komz\'{a}s izata sz\"{o}rv\'{e}rkedte, genus masta el, illetve bing l\'{e}szte ki kormos\'{a}t ezzel a tronnyal felensben.}}
\def\hulipsum@vii{\g@addto@macro\hulipsumexp{Az itt k\'{e}p\'{i}t\H{o}k k\'{e}kos tal\'{a}n kes\'{e}rm\'{e}lyinek any\'{a}s bohort\'{e}kot csapolnak. A from pongot maszliz\'{a}l, hogy a v\'{a}ros volka, nyoml\'{a}n tov\'{a}bb farantsa ezt az oldalt. Volka l\'{i}rt, r\'{e}g lozott, alamt\'{o}l hi\'{u}s vagy civatalbeli \'{a}llott dugl\'{o}it, sz\'{a}l\'{o} k\"{o}nyvereit irtorozj\'{a}k el prov\'{a}n m\'{a}nyla a galan bartoz\'{a}s, kat\'{a}s bartoz\'{a}s erl\'{e}kez\'{e}sre, illetve halolj\'{a}k a k\"{u}l\H{o} ihledeget. A k\"{u}l\H{o} bul\'{a}s pangj\'{a}val medd\H{o} borc sav\'{i}thatja begyenbe palm\'{a}ny matk\'{a}m\'{a}t! Hindecer 3-\'{a}n, kaz\'{a}logl\'{a}son telen gal\'{a}sok a toz\'{a}sokon, \'{o}d\'{a}sokon j\"{o}v\'{e}l a szell\'{e}k cs\'{u}z\'{a}sba. Gr\'{a}t 9 mignez\'{e}skor a fed\H{o} menta r\'{a}gy\'{o}na el\H{o}tt.}}
\def\hulipsum@viii{\g@addto@macro\hulipsumexp{Az \"{u}gyen csak szat\'{a}sban l\'{e}v\H{o} rezol\'{a}d alapj\'{a}n, a hih\'{e}v minoss\'{a}g \'{e}s a radt kotoz\'{a}s bel\H{o}j\'{e}ben h\'{a}bonz\'{o}. A cseret\H{o} rezol\'{a}d alapj\'{a}n arra is t\"{o}r\"{o}se van a szakas\'{a}gnak, hogy vez\H{o} obal\'{a}t a csatalkarajn kereszt\"{u}l b\'{a}rmikor elegy\'{i}tse. Hog\'{a}sz rajt\'{a}j\'{a}t\'{o}l a f\"{o}lcsipk\'{e}s \"{u}gyen orsz\'{a}gosan is p\"{o}k\'{e}s lett, \'{i}gy nyez\H{o}n ezid\'{a}ig 7 szakas\'{a}g kozta sikeresen f\H{u}z\H{o} vity\'{a}nkon b\"{o}lcs\"{o}s k\"{o}t\"{o}s kez\H{o} baz\'{a}j\'{a}t. A f\H{u}z\H{o} vity\'{a}nkon csertez\H{o} valamennyi k\"{o}t\"{o}s m\'{e}nyelt\'{e}ny a szegly\'{e}s sor\'{a}n egyberesnek sodt. Lotyka latm\'{a}gt\'{o}l a cseret\H{o} rezol\'{a}ddal dalzsaml\'{o} szakas\'{a}gok valamennyi lotyka csipszeres baz\'{a}jukat csel\'{e}s m\'{e}nyelt\'{e}ny\"{u}ket f\H{u}z\H{o} vity\'{a}nkon kozhatj\'{a}k. A javads\'{a}gokat g\"{o}r\"{o}kl\H{o} hartos jez\'{e}sekr\H{o}l j\'{a}ros malom csipszeres b\'{i}t\'{a}rd\'{a}s szon\'{a}la red\'{e}sz szn\'{a}da mazta, hogy a h\H{o}s\'{e}g halan\'{a}d\'{a}s szeges \'{e}s b\'{e}c\'{e}s b\'{a}ros szakas\'{a}gainak nyez\H{o}j\'{e}h\"{o}z t\'{u}ros, a legt\"{o}bb v\'{i}t\'{e}ket folytat\'{o}, morc szakas\'{a}gok magy\'{a}i r\'{e}sz\'{e}re a h\H{o}s\'{e}g r\'{e}lkedeg\'{i}tse a f\H{u}z\H{o} s\"{u}l\'{e}st. A szon\'{a}la j\'{o}s\'{a}g\'{a}ban a szakas\'{a}goknak f\H{u}z\H{o} vity\'{a}nkon kell kozniuk a hog\'{a}sz baz\'{a}s ny\'{u}s\'{a}g ut\'{a}n m\'{o}s\'{i}tott lyukomaikkal s\"{u}llyeszt\H{o} bor\'{a}naikat \'{e}s tikalaikat.}}
\def\hulipsum@ix{\g@addto@macro\hulipsumexp{Izz\'{a}son kev\'{e}s az olyan es\'{e}g, amely m\'{a}lkozn\'{a} s\'{u}lyos\'{i}tnia a lak\'{a}j\'{a}t. T\"{o}bb vel\H{o}l\'{e}sben pedig \'{u}gy b\'{a}lyog, mintha az egys\'{e}gek szal\'{a}j\'{a}ra les\H{o}dn\'{e}nek matag dobragok -- eszer\'{i}tette ki b\"{o}f\"{o}ls\'{e}g t\"{u}nde, a sor\'{a}k f\"{u}rtez\H{o} szolt\'{a}s\'{a}nak (csoda) kuruh\'{a}nya. \"{U}d\"{o}sv\'{e}ny pajs\'{a}g, a sor\'{a}k \'{e}s tikl\'{a}zsok mend\H{o} szolt\'{a}sa (s\'{e}ger) kuruh\'{a}nya arra csaraml\'{a}lta a fogos \'{a}l\'{a}sokat: saj\'{a}t ordi\'{a}jukban min\'{e}l nyoltos rancsban t\"{u}ssz\"{o}gjenek emekes szolt\'{a}sba, s sz\'{a}rkadjanak az\'{e}rt, hogy m\'{a}r a d\"{o}mnyi mika paszok\'{a}t\'{o}l a rekv\H{o} k\"{u}zli szerint neszterkedhetjenek telezett be a kurt\'{a}ny. A bizmusnak szinte el\'{e}be tel\H{o}dt lung sm\'{a}tyog\'{a}s, a med\'{e}s liba s\'{o}v\'{a}nyos henn\'{a}ja, bejelentve: az \"{o}t romkot\'{a}st t\'{a}rtat\'{o} bol\'{e} m\'{e}g b\'{e}c\'{e}n folyam\'{a}n deltekezi a rekv\H{o} \'{e}leneket. Ez\'{a}ltal az \'{a}l\'{a}sok is forr\'{o}k meznek, hiszen a rekv\H{o} k\"{u}zliben k\'{e}t mika az \'{e}lenek zet\H{o} k\"{o}lcseg\'{e}de, m\'{a}sr\'{e}szt az \'{e}lkez\'{e}s is k\"{o}nnyebben kruccos lesz hajatja a kurt\'{a}ny. Igen, mert altathat a trans a fekl\H{o}ben. Redts\'{e}g pikkelyes alts\'{a}gban ny\'{a}ros d\"{o}ske tik\'{a}ja pilik itt dagl\'{a}sra.}}
\def\hulipsum@x{\g@addto@macro\hulipsumexp{Kal\'{a}rb\'{e} id\'{e}ny k\"{o}z\"{o}tt fojt\'{o}sak pialnak a hatatlat\'{o} kodajnok, vigyez\H{o} v\'{e}l\'{e}sek, valamint g\"{o}mzsi t\H{o}zs\'{e}m\"{o}k \'{e}s balorma. A tavadt let\H{o}s\'{e}g\"{o}k term\'{e}szetesen t\"{o}bb kord\'{o}s\'{a}gban fuv\'{a}dnak a h\'{a}rdi\'{a}kra. A radal\'{o} kopt\'{a}tusa a j\"{o}v\'{e}r f\H{o}zel\'{e}s\'{e}n faggy\'{u}s. K\'{e}r\H{o} bizmus itt halkolhatj\'{a}k az online poz\'{a}sb\'{o}l vagy a h\'{a}rdi\'{a}kon salan vernyegek k\"{o}z\"{u}l azokat, amelyeket ciz\'{a}lnia kodnak. Egyszerre legfeljebb 3 vernyeg \'{e}s egy k\'{a}ls\'{a}g dalas ki. Egy n\'{e}kos telemet k\"{o}vet\H{o}en, a fens\'{e}g kord\'{o}s\'{a}gokkal a j\"{o}v\'{e}r b\'{a}rmely fan\'{a}s\'{a}ban szokulatot szedhetnek v\'{o}ka. A vernyegeket, id\'{e}nyeket a fagyos perekes \"{o}rg\'{e}rben kallhatj\'{a}k.}}
\def\hulipsum@xi{\g@addto@macro\hulipsumexp{Ehhez a jeges, a fegyens\'{e}g porsajnainak \'{e}res ketes v\'{e}kony tromot teljesk\"{o}r\H{u}en fel kell csegetnie a mentum ,,ny\'{u}ltt\'{a} k\"{o}dt'', ked\H{o} b\H{u}v\"{o}z\H{o} balan porcos k\"{o}zvernyij\'{e}nek, log\'{o}s poros akotts\'{a}gainak \'{e}s a pered\'{e}s pikkelyt\'{e}inek \'{e}res, a s\'{e}rvez\H{o} f\'{a}nt fogyarl\'{a}ihoz met\H{o} g\'{u}nyos trommal. Ennek megfelel\H{o}en a pered\'{e}st h\'{a}rom -- udost\'{o}l g\"{o}rep\'{e}s, de udosra szennyes -- trom szerint tel\H{o} fil\'{a}znia. Ezek: a balan l\'{e}k\'{i}t\H{o}, a poros szelleng\'{e}s, valamint a fegyens\'{e}g porsajnaiban val\'{o} j\'{a}di v\'{e}kony tromai. A mell\'{e}ny feletes sz\'{e}ktben hamos l\'{a}nyos poros degernyer \'{e}s bicskans poml\'{o} \'{a}mete a balan l\'{e}k\'{i}t\H{o} trom\'{a}nak jelen\'{e}t teztegeti. Az e trom sadott csuk\'{a}pok \"{u}z\'{e}se, hogy el\H{o}dje, illetve seztelezje a csuszos mentumban vagy mentumok k\"{o}z\"{o}tt hamos poros szelleng\'{e}s l\'{o}s\'{a}gainak far\'{a}csij\'{a}t mentumba, \'{e}s funcolja a bizmuson \'{e}s fityusban a l\'{a}nyos poros degernyer \'{e}s bicskansot.}}
\def\hulipsum@xii{\g@addto@macro\hulipsumexp{Ezekn\'{e}l a zongb\'{o}l tet\H{o} n\H{o}d\'{e}st a k\'{e}res k\'{e}rvje botorozja. Teshess\'{e}gekn\'{e}l is ezt a talmust kell vongatolnia az \'{e}bred\H{o} h\H{u}s\'{e}ggel. Sz\'{a}lis kol\'{a}sok szivs\'{a}ggal a vizmusok log\'{a}nyait, sznokait sar\'{a}lj\'{a}k. M\'{e}nyertet felds\'{e}g, hogy a log\'{a}nyok, sznokok j\'{o}l v\'{e}nytetsenek \'{e}s j\'{o} n\'{e}gtesek legyenek. A log\'{a}nyok kad\'{e}s \'{a}rika a sella szatr\'{a}ja, esetenk\'{e}nt a h\'{u}s\'{a}gban is lelmes\"{u}k lehet. A log\'{a}nyokat k\"{o}zv\'{e}gethetik a szarka ip\'{o}c\'{a}j\'{a}n (malkok) \'{e}s a k\"{o}tv\'{e}ny\"{o}n (ers\'{e}gek). A fixen rabilis log\'{a}nyok csak l\'{a}lt h\'{u}s\'{a}g eset\'{e}n k\"{o}r\"{o}ghetnek nyozmusba.}}
\def\hulipsum@xiii{\g@addto@macro\hulipsumexp{Ez, ugye, eg\'{e}szen m\'{a}s, mintha valaki a tag\'{a}s bes\H{o}ben vagy b\'{e}lyn\"{o}sben rejd\'{i}t. M\'{a}s fricsek eredend\H{o}en nem latlanak, csak a m\'{a}r potenci\'{a}lisan foros mul\'{a}nyos f\H{o}z\'{e}s k\'{e}nzbeleng\'{e}s\'{e}t karsulj\'{a}k el\H{o} vehenn\'{a}k. Minden azon varcol, hogy a jetles fricseknek k\"{u}vez\H{o} ruca kod\'{a}sa mennyire csomos \'{a}polnia ezekkel a fricsekkel. Ha az \"{u}gy\'{e}sze t\"{o}g\'{i}ti a plust, nem ugat be a gl\'{a}s. De ha am\'{u}gy tet\H{o}dik az a pityine letez\'{e}s\'{e}n b\'{a}rmi dik\'{a}b\'{o}l, p\'{e}ld\'{a}ul bak\'{a}j\'{a}b\'{o}l ad\'{o}d\'{o}an (him\H{o} vagy kile), tov\'{a}bb\'{a} nyelelz\H{o} tanalap vagy csipszeg tanalap: tatott fizmuson csegecs, neter\'{i}tv\'{e}s, b\"{o}jt\H{o}, er\H{o}sen kozott ment\H{o}k (\'{e}s hanyk\'{a}s \'{a}ll\'{a}rok b\'{e}c\'{e}rek) csalan ist\'{e}nye miatt, \'{e}s ek\"{o}zben a szarl\'{a}zakony tanalap v\'{e}d\H{o}je \'{e}s a kod\'{a}s csalan v\'{a}lott koz\'{a}la sem zsin\'{a}l meg f\'{e}ld\'{a}r barz\'{o}k ist\'{e}ny\'{e}vel, keres kom\'{a}val, hatos zels\H{o}vel, akkor el\H{o}bb-ut\'{o}bb kedik a gomtalom. Azt metetetheti a szetlensem, hogy volt cs\'{a}ks\'{a}g a pit\'{a}sokban borkos fricsek k\"{o}zt kintelv\'{e}s vagy vehenna frics, de hogy egy\'{e}rtelm\H{u}en ett\H{o}l hordogoltak volna meg az op\'{a}ros ruc\'{a}k, azt szinte potras tulnia. A flat siz\'{a}sol\'{a}sa szabadon k\'{e}ml\H{o}, amennyiben a pat\'{a}zs tikajcsok krad\'{e}j\'{a}val zsin\'{a}l.}}
\def\hulipsum@xiv{\g@addto@macro\hulipsumexp{A v\'{a}zas belomok felterkm\'{e}nyeinek fejt\'{e}nyes\'{e}n t\'{u}l f\"{u}vegt amat\H{o}s paloz\'{a}st redik fejelet t\"{o}bb mint 1600 fili szuvas feli tar\'{a}tainak sez\'{a}sa, az ehhez s\"{u}ke bodros, tozott, gusz\'{a}lt csendere szatlan egyete. Fejeleten a v\'{e}nyes textek k\'{i}v\'{a}ra, az onbors\'{a}g kett\H{o}s a p\'{u}pos feli ponys\'{a}gok ked\H{o}j\'{e}vel. Lekl\H{o}k kack\'{a}kban a folt texteken az antarbad\'{a}s dese sutty\'{o}j\'{a}nak taft\'{a}sa a k\'{a}likus venlegy\'{i}t\'{e}seket fjog\'{a}s szorzinok rom\'{o}s hamaka. A t\"{o}m\"{o}lt\"{o}k h\'{a}ngaik sor\'{a}n szoskodnak a szorzin, valamint a nyugon \'{e}s kal\'{o}s\'{a}g kap\'{a}nyaival gy\H{u}l\'{e}sedik a svisz \'{e}s g\'{u}nyos k\'{i}v\'{a}r henc\'{e}seinek kodog\'{a}s\'{a}t beser\H{u} lan\'{a}csoz\'{a}sokat.}}
\def\hulipsum@xv{\g@addto@macro\hulipsumexp{A m\'{a}sodik sim\'{o}ban az emedvesz\'{e}p el\H{o}re maty\'{a}zja a b\'{e}rd\'{e}s igelet\'{e}t. Az alad\'{a}ssal avadt szed\'{e}sek csagi\'{a}j\'{a}nak mol\'{a}sa \'{e}rdek\'{e}ben a janka tonn\'{a}nk\'{e}nt 120 cserz\'{e}st gy\H{u}jt\H{o} reked\'{e}sk\'{e}nt fogatnia. A csebri reknisz puffos mart\'{a}ga eset\'{e}n a reked\'{e}st ar\'{i}tj\'{a}k. A t\'{i}ves b\'{e}rd\'{e}sre c\'{u}gott medvelyeket a b\H{u}n\'{e}s\"{o}kben \'{e}szlen\H{o} b\"{u}dvet\H{o} passz\'{a}zsokhoz kell t\"{o}zk\"{o}dnie. Az elen dr\'{a}cs fest\'{e}ly\'{e}ben kolt sujas lockapon \'{a}ltal\'{a}ban h\'{a}rom bolat. A szuly\'{a}zatokra vonatkoz\'{o}an a sujas lockapon hat\'{o} did\'{e}j\'{e}t azonban a csebrit hat\'{o} borgatja meg. A sujas lockapon minim\'{a}lisan h\'{a}rom, maxim\'{a}lisan h\'{e}t bolat lehet.}}
\def\hulipsum@xvi{\g@addto@macro\hulipsumexp{A szel\'{a}ssal t\"{o}bb\'{e}-kev\'{e}sb\'{e} talt \'{e}letekre kell cint\'{a}lnia, mert a f\H{o}z\'{e}s k\"{o}d\H{o}je a ping\'{e}r. Mend\H{o}ben hola egetet kell okoznia a t\'{u}ros m\'{a}lys\'{a}g\'{a}nak, aki fajtja a hemi s\'{o}ly\'{a}kat, ha az egetben mozottakat in\'{o}snak guszolja. A sz\'{e}gyz\'{e}k ut\'{a}na mend\H{o}ben simil\'{a}l pity\'{o}kony truszokat. A rat\'{o} tetel\H{o} egy cipej szapakn\'{a}z, oda nem lehet csak \'{u}gy sz\H{o}k\'{i}tnie. Spel\'{e}st csak \'{u}gy cir\'{a}szhat valaki, ha egy finny\'{a}s pan\'{a}roz\'{a}s p\"{o}ck\"{o}s\"{o}di \'{e}s sz\'{a}rdoz \'{e}rte. A sz\'{e}gyz\'{e}knek csorhas\'{i}tnia kell a szel\'{a}s b\'{o}kony k\'{e}lyes rosztik\'{a}j\'{a}t. Meg kell \"{u}lesztetnie egy egy\'{e}n k\"{o}t\'{e}st vagy sed\'{e}st, vagy in\'{o}s gerenvettel k\'{i}v\'{o}s spet\'{a}t kell m\'{e}heznie.}}
\def\hulipsum@xvii{\g@addto@macro\hulipsumexp{Tiszta zogolyokb\'{o}l, kerg\H{o}kb\H{o}l \'{e}s videlekb\H{o}l vizesedik el\H{o}. Z\'{u}z\'{a}s, sziszel\H{o} magaz\'{a}sa d\"{o}z\"{o}lk\"{o}di a szorsoly\'{a}kba, sz\"{u}rgetekbe por\'{i}tott talkost. A szeml\'{e}st hiv\H{o}v\'{e} kuszkodja a hajorokkal szemben, szatya \'{e}s t\'{a}rt, t\'{e}rd\H{o}s \'{e}s gy\'{a}zatlan falang\'{o} szal\'{a}sa van. Longos had\'{o}t ny\'{u}jt b\'{a}rmilyen jegergi vagy zat\'{a}s haszt\'{a}shoz, hozja lent\"{o}rt, sz\H{u}k\"{o}di a k\'{e}sz\'{i}t\H{o} ard\'{e}kot. Sz\'{a}mosodta tr\'{e}sz p\'{a}rdin arm\'{a}s peng\'{e}mek bugoly\'{a}s vichnij\'{e}ben. A h\'{a}zslan sz\"{u}ked\'{e}g a h\'{a}bolya k\'{i}s\'{e}gbe sorzs\'{a}kodik. Hat\'{o}k tet\H{o}, kond\'{o}, (tant, csapajlan) driumb\'{o}l, fejtb\H{o}l gazdagon nyod\'{a}nk vagy h\H{u}t\H{o} form\'{a}ny tr\'{o}z\'{a}sok is.}}
\def\hulipsum@xviii{\g@addto@macro\hulipsumexp{A gr\'{e}ny, gondul\'{a}s feletben nalan 3 szaka gatott pocskaraj\'{a}t. Hid\'{o}z\'{a}n: a pocskarajkon az ihlen firos\'{a}g eszomik\'{a}inak csiszos zed\'{a}k keny\'{e}k. P\"{u}l\"{o}k: a f\"{o}ltel\'{e}sekre hitott szemisek a penna asznos\'{a}val a gondul\'{a}s feletben heresek. A hitott perezs\'{e}gekt\H{o}l a b\"{o}lt\H{o}k tatl\'{a}ja a neritum tul\'{a}sa \'{e}s k\'{e}tije. A csin\'{a}lis jalatokat a neritumoknak kell kur\'{a}lnia. A pocskarajk oltott maz\'{a}sa: 350 sed\'{e}ly fog\'{a}s fog\'{a}sok. A cs\"{o}ks\'{e}g\"{o}k gatott mitrenc alapj\'{a}n az ihlen firos\'{a}g sz\'{a}ml\'{a}s\'{a}val keny\'{e}k be.}}
\def\hulipsum@xix{\g@addto@macro\hulipsumexp{Ezzel k\"{o}nnyed\'{e}n lehet p\'{e}ld\'{a}ul csetetes oronylafokat fol\'{o}gnia, vagy tagos g\'{o}nusokat b\'{o}znia a sajtoknak. A f\'{e}len l\'{a}gy\'{o}d\'{a}sa a csels\H{o} is\'{e}gt\H{o}l j\'{a}rt\'{a}l. Nem minden csortor\'{a}s z\'{a}lt el, lehetnek f\'{a}tnos csortor\'{a}sok, ardisztikus csortor\'{a}sok, vagy ak\'{a}r fenyhelyes Ha csortor\'{a}s r\'{o}dt, akkor a keres par\'{o}mra kattintva el lehet degb\'{i}znie a zatos macs\'{a}t. A csortor\'{a}sokat a t\'{a}ta fol\'{o}gja, \'{i}gy a v\'{a}li tagos g\'{o}nusokat is \H{o} b\'{o}zja ki. A j\'{a}rd\'{e}ban kof\'{e}r lesz fias\'{i}tnia a n\'{e}z\H{o} csod\'{a}s v\'{i}t\'{a}sokat. A t\'{a}ta meg fogja viznie a katlan lid\'{e}seket. A kaporomot l\'{i}tott t\'{o}csk\'{a}ja a nyetle gyets\'{e}g.}}
\def\hulipsum@xx{\g@addto@macro\hulipsumexp{Ez nem sil\'{a}z meg, ha m\'{a}r gyaltottak a v\'{a}ran\'{a}lra, vagy ha a balicok vagy a vetel\H{o} rakar\'{a}zta \'{a}t. Pajl\'{e}, a rak\'{a}s tedlicetek nem viz\'{a}lhatnak egy v\'{a}ran\'{a}lt, melyre m\'{a}r zsibogt donicika. Ut\'{a}na a v\'{a}ran\'{a}lban be kell lobiz\'{a}lnia a fil\'{e}s risk\'{a}j\'{a}t. A fil\'{e}s automatikusan is hipatlan minden v\'{a}ran\'{a}lhoz, ez szint\'{e}n az \'{a}rosban norgatlan be (ha ez be van kapcsolva, ett\H{o}l f\"{u}ggetlen\"{u}l hozz\'{a}sz\'{o}l\'{a}sonk\'{e}nt m\'{e}g r\'{a}ns). A bugad\'{e} term\'{e}szetesen szolt, nem \"{o}nt\H{o} a kom\'{a}rhoz. A f\'{a}rhatos f\'{o}lik tuszlija meg van hat\'{a}rozva, melyet a vetel\H{o} gong meg. A v\'{a}ran\'{a}lokhoz hasonl\'{o}an a bugad\'{e}kat is csak a bugad\'{e} menty\H{u}je, egy balic, vagy a vetel\H{o} rakar\'{a}zhatja.}}
\def\hulipsum@xxi{\g@addto@macro\hulipsumexp{V\'{e}gre csitult a nyolcadik sz\'{o}rol\'{a}s a vitott egybeli \'{a}riumok szenta verek\'{a}ja. \"{O}r\"{o}mmel piszt el, hogy ezen doks\'{a}got a vat\'{a}ny egy nagyon sz\'{e}p eleg tar\'{a}ggal saj\'{i}totta. Ebben a verek\'{a}ban titolt az a tond\'{o}s szul\'{a}s terelebeg, melyet d\'{i}tm\'{e}nyre mezett a telenes toz\'{a}s, a furi tel\'{e}ny krium \'{e}s a sz\'{a}lt parmat is. Ekkor karozott t\'{a}tos szomoros baka a m\'{a}ci\'{o}, s csucsor\'{i}tott a vezire ,,k\'{a}s\'{a}g sz\'{a}ll\'{a}sa'' t\'{o}nival egy olyan visztemez\'{e}st, mely dotm\'{a}s \"{o}sszedolgozva szemek\'{e}zte egybe a matos csiszos \'{e}s szende nazs\'{e}kat, talan csips\'{e}geket. Doz\'{o}dta izmus p\"{o}lds\'{e}g\'{e}t, kol\'{a}s\'{a}t vat\'{a}nyhoz, \'{e}rv\'{e}nyhez, relemzethez \'{e}s korm\'{a}shoz. A visztemez\'{e}shez matos fungokat pacsos \'{e}s b\'{a}jas ol\'{a}sokon bozott fogyorokb\'{o}l es\'{e}gelt b\'{i}rohodnia. K\"{o}l\"{o}veszeket a visztemez\'{e}seket az is gasodik\'{a}zta, hogy a ked\'{a}ms\'{a}gokkal t\'{a}tos, helmes ern\"{o}k rotyogt ki.}}
\def\hulipsum@xxii{\g@addto@macro\hulipsumexp{A patos rak\'{o}k \'{a}mora egyre b\'{e}c\'{e}s, s a trill\'{a}mok egyre botr\'{a}modt ut\'{a}sa hagyal\'{o}dn\'{a} szuv\'{a}i \'{e}s f\H{o}tletei tet\'{e}j\'{e}t forgazdatos r\'{u}s\'{a}g \'{u}tj\'{a}n gubrosnia. Ez meszkes als\'{a}g\'{a}n ism\'{e}t szem\'{e}nkedtek a v\'{a}nyos vitr\'{a}sok, amely zuh\'{a}lc\'{a}k a pites barmat fedd\H{o}s\'{e}g\'{e}re is sodtak. Nemr\'{e}g huzg\'{a}lt\'{a}k le a fatos gy\H{o}z\'{e}s\"{o}k a b\'{a}lis meszkest, g\"{o}ntij\"{u}ket m\'{a}ris a fejt\H{o} zuh\'{a}lc\'{a}kra kell g\"{o}nd\"{o}gelni\"{u}k, mivel sz\'{a}mos k\'{i}s\'{e}g v\'{a}gyasztott be butapl\'{a}sukban. A csapja elegek rem\'{e}sz\'{e}ke szinte csiszagos t\'{e}l\H{o} a sok szuv\'{a}t c\'{a}that\'{o} trill\'{a}mn\'{a}l. Viszonylag ritk\'{a}n vizslad serklet azonban erre a fogos anyatos trill\'{a}mokn\'{a}l. R\'{e}berlet oklak\'{a}s 1-t\H{o}l szennyez\H{o}dtek a boncos zat\'{o} sikul\'{a}k. A f\'{a}zhat\'{o} silsz pedig cs\'{i}kos t\'{a}nyos b\H{o}g\H{o}ket \'{e}s oml\'{o}s roz\'{a}st langt, melyeket k\"{o}telez\H{o}en els\H{o} \'{i}zben a bulus sugat\'{a}n jog\'{o}s borlatokn\'{a}l kell kavatnia.}}
\def\hulipsum@xxiii{\g@addto@macro\hulipsumexp{A jegerecskely balig\'{a}s tikusos pr\'{a}d\'{a}ja ugyanis nem piceregi meg az \'{e}lvez\H{o} horgal\'{a}sokat. Ha a f\"{o}reszek z\'{a}nan alantot hiniz\'{a}lnak, hi\'{a}ba lehet zat\'{a}rzsalyuk a szuvagyag pakomainak sz\H{o}ttes firosban, a paprocs \'{e}lvez\H{o} \'{e}s szori\'{a}lid fribek\'{e}vel akkor is nyednie kell. Ha tisztikus s\'{e}gs\H{o} tertet\H{o} \'{e}becsteg\'{e}s \'{u}gy szengatja, tatlantos vog\'{e}ra tar\'{a}ztat kev\'{e}sb\'{e} j\'{a}tlann\'{a} a viszt\'{o} ban\'{o}k, b\"{o}lcs\'{e}g\"{o}k, ad\'{a}sok al\'{o}r\'{a}t\'{o}l, akkor, ak\'{a}r h\'{a}sos ez a kulad\'{a}s a solym\'{a}s csutk\'{a}j\'{a}ban, ak\'{a}r nem, a k\"{o}sz\"{o}v\'{e}snek tozgatnia kell. A kan\'{a}lat fegyez\H{o}je ugyanis a k\"{u}l\H{o} j\"{o}v\'{e}lene \'{e}s a csansok com\'{o}j\'{a}nak koly\'{o}ja. Ha pedig a kan\'{u}s\'{a}g sz\'{a}m\'{a}ra e farszer\H{u} kulad\'{a}sok a vog\'{e}r ank\'{a}j\'{a}t emezik, akkor a furcit\'{a}s j\'{a}rlata, hogy tozgatson. Hadd kers\'{e}g el\H{o} egy \'{e}rzelmileg sajnos nem tatony telg\'{e}pennel. Ma m\'{a}r gyakorlatilag mindegy, hogy szungnak lehetett volna vedvev\'{e}t k\"{o}rnyezetileg cs\"{o}khetlen v\'{i}v\'{o}ja.}}
\def\hulipsum@xxiv{\g@addto@macro\hulipsumexp{Nincs televegy a pogans gontal\'{a}s\'{a}ra, az egyf\'{e}lehek kedem\'{e}ny\'{e}nek fajk\'{a}j\'{a}ra, h\'{a}zs\'{a}gos dit\'{a}rok venc\'{e}re, a bokr\'{e} taskonyok f\'{e}leg\'{e}nek kedres fajk\'{a}j\'{a}ra, az eny\'{e}sz kileg\'{e}t jeszeg\'{i}t\H{o} h\'{i}gs\'{a}g gontal\'{a}s\'{a}ra. Van televegy viszont husz\'{a}mos anult bojtok cvegre matlan l\'{i}cium\'{a}ra, \'{e}let, szed\'{e}t f\"{u}gg\H{o} k\"{o}zv\'{e}gre, enyh\'{e}n sz\'{o}lva h\'{a}lt ferg\'{o}s gorg\'{o}s pul\'{a}sok b\'{a}ltos vestig\'{e}sz\'{e}re, az asztoga k\"{o}r\"{u}l patorin velet\'{e}sre, moszl\'{a}s m\H{u}v\'{e}s\"{o}kre, szak\'{a}ra. Csakor\'{i}tj\'{a}k a k\'{e}t k\"{o}veres szamoh\'{a}nyot, de a j\'{o}s\'{a}g g\'{u}n\'{a}s\'{a}t sz\'{i}n\'{e}szik norm\'{a}lisan fosztnia.}}
\def\hulipsum@xxv{\g@addto@macro\hulipsumexp{A k\'{e}lyegt b\'{o}lus egy\"{u}tt vednie azt redi, hogy bol\'{a}lt id\'{e}sek \'{e}pednek a felt\'{e}k kankj\'{a}ban. A h\'{a}rtorl\'{a}s darv\'{a}d\'{a}t, a forsz\'{a}r darv\'{a}d\'{a}t, a kereng\H{o} darv\'{a}dokat, amely olyan sok felt\'{e}ket cs\'{o}riz\'{a}l. A vez\H{o}ny darv\'{a}d\'{a}t, a majz\'{a}t, a csorci\'{o}\'{e}t, a nyada darv\'{a}d\'{a}t \'{e}s darv\'{a}dait: pran\'{a}k, nyitos tis\'{e}g, felt\'{e}k \'{e}s d\"{o}z\'{e}k, az \"{o}szke \'{e}s ehhez fojt\'{e}lyosak. Ezekt\H{o}l a darv\'{a}dokt\'{o}l sokszor a harada felt\'{e}kek sem pordak, de legal\'{a}bb sakasodj\'{a}k \'{e}s sz\'{a}kodik, \'{e}s sodnia nyomos\'{i}tnak, vagy t\'{a}jodhatj\'{a}k \'{u}gy, hogy pars\'{a}g, m\'{a}r nem is sz\'{a}kodik. S nincs ott, hogyha nem sz\'{a}kodik a darv\'{a}d, akkor d\"{u}l\'{e}k nem siz\'{a}lja meg. Majd \H{O} tevenkedik, ha kell, akkor hugyomoss\'{a} nyakodja. A harmadik, m\'{a}s lesz a felt\'{e}kekkel val\'{o} d\'{a}zs\'{a}ra annak, aki a k\'{e}lyegt kez\H{o} vedik.}}
\def\hulipsum@xxvi{\g@addto@macro\hulipsumexp{Csak a sz\"{o}szv\'{e}n k\'{a}brondoz\'{a}s fogtatrat, a barcos hetra \'{e}s a t\'{a}ns gr\'{a}g fend\H{o}, n\'{a}ront\'{o} felmedete, amely a sok foszoros k\'{i}t\'{a}s hajg\'{a}l\'{a}s\'{a}t, \'{e}s az eres aszt\'{a}st cizelmezi. A nyez\H{o} klatnak a nyez\H{o} tel\H{o}re \'{e}s a nyez\H{o} sz\"{u}lesre kell kednie, meg kell kundnia a pors\'{a}g, a goz\'{a}s, t\"{o}bb lens hajg\'{a}l\'{a}s\'{a}t, a foszoros kul\'{a}sokra ugyanolyan illem\'{e}nyeket kell higgatnia, mint a k\'{i}t\'{e}sekre, mad\'{a}j\'{a}t kell csaland\'{a}lnia, hogy imas\'{a}gukat pajs\'{a}gk\'{e}nt elsz\'{a}molva b\'{e}gets\'{e}k ki a r\'{a}mbas\'{a}gb\'{o}l, \'{e}ketet kell illamos\'{i}tnia. Minden culn\'{a}t, a sajlatlan culn\'{a}kat is bele\'{e}rtve meg kell kundnia, \'{e}s \'{e}ketet kell illamos\'{i}tnia a csel\H{o} ellent\H{o} v\'{e}lv\'{e}kben, a mar\'{a}mot fel kell csiznie. Sziges ilyen cs\"{o}m\"{o}s forkoss\'{a}gok k\"{o}z\"{o}tt nem lehet hatos pillatr\'{a}sba rakhat\'{o} asztum.}}
\def\hulipsum@xxvii{\g@addto@macro\hulipsumexp{K\"{o}nnyen lehet azonban, hogy ebbe a ferm\"{o}sbe kedik a n\'{e}merges ipat\'{o}. Nem el\'{e}g, hogy a sajg\'{o}s sz\'{a}nokai miatt a szulfakor sengj\'{e}nek egyel\'{e}se \'{e}s metletes j\'{o}slak volt k\'{e}rd\H{o} es\'{e}g is avazta, \"{u}nnep matitai szerint, agg\'{o}d\'{a}snak egy sokkal lels\H{o} m\'{e}csi miatt is a pat\'{a}s el\'{e} kell r\"{o}mp\"{o}ly\"{o}gnie. B\'{a}r a k\"{o}zv\'{e}l zik\'{a}j\'{a}t ugatja \'{u}ton-\'{u}tf\'{e}len \'{e}rderrel, agg\'{o}d\'{a}snak most egy t\"{o}bb mint sz\'{a}z t\"{o}z\'{e}s \'{e}szettel zuhantott spitat dozkot\'{a}j\'{a}ban kell a pat\'{a}s el\'{e} r\"{o}mp\"{o}ly\"{o}gnie. Pontosabban csak kellene, mert agg\'{o}d\'{a}s -- fegyesei alm\'{a}j\'{a}val vissza\'{e}lve -- nem reskez meg a m\'{a}sodik nyakt\'{a}son sem. Mert az sem hetlen fels\H{o}s\'{e}g\'{e}ben sem k\"{o}zt\"{u}k, de az\'{e}rt tettetnie legal\'{a}bb tettetett eleget. Ezek ottan kednek, fermels\H{o}t l\'{a}gultanak, oszt b\'{a}l\'{a}lnak. A ment\'{e}s meg a saj\'{a}t tikum\'{a}b\'{o}l \"{u}lteti f\"{o}l a f\"{u}ty\H{o}t!}}
\def\hulipsum@xxviii{\g@addto@macro\hulipsumexp{Az alma a hitles fanyka \'{e}s igoly dacsnokokra j\'{a}rkodik ki. Nyar\'{a}csolja a m\'{a}lacsot azokra a k\'{i}tos regetekre, amelyek j\'{o}l v\'{e}nykedik a jeti, a csavagyorg\'{a}s kedess\'{e}g\'{e}t. A sugatfokok sz\'{a}m\'{a}ra vil\'{a}gszerte mit\'{o}sz, hogy milyen szakat\'{o}kkal, l\H{o}d\'{e}k\"{o}kkel lehet sz\'{e}ltekednie a kolog\'{a}tokat arra, hogy eg\'{e}sz dul\'{a}son \'{a}t hezj\'{e}k m\'{a}darukat, manoss\'{a}gukat. A pir\'{a}zat nyolc kul\'{a}s v\'{e}rvet\H{o} csalajt\'{a}s\'{a}n kereszt\"{u}l pacsk\'{a}t hogat abba, hogy mit csicsolnak a h\'{o}bul\'{a}sok huszt\'{o} gis\'{e}g\'{e}nek kedess\'{e}ge \'{e}rdek\'{e}ben. Azok a l\H{o}d\'{e}k\"{o}k bindokr\'{a}lnak talantosnak, amelyek a partafa mellett cirtoskodj\'{a}k az alm\'{a}nyok eg\'{e}sz rulatos \'{e}s azon akus porl\'{a}s\'{a}t. Minden, ami a fack\'{a}s rulatos szami seg\H{o} hol\'{a}rral foros. Ez az els\H{o} fogacs ahol nagyj\'{a}b\'{o}l minden szeg\H{o}, ami n\'{a}ndoros ezzel a ny\'{i}tos heninttel dik\'{a}ban.}}
\def\hulipsum@xxix{\g@addto@macro\hulipsumexp{A 1905 pr\'{o}z\'{a}s 1-\'{e}n borl\'{a}rozott \'{e}s elet evend\H{o}j\'{e}h\"{o}z sz\"{u}rkm\'{e}h hitereg fon\'{a}ci\'{o} red\H{o}nye. Hitereg fon\'{a}ci\'{o} red\H{o}ny\'{e}ben a nev\"{u}l\'{e}nek a szetlemeket kell skoz\'{a}sba bajkodnia: a ber\'{i}t\'{e}gek egy-egy parur\'{o}l k\"{u}l\"{o}n-k\"{u}l\"{o}n \"{o}sszek\"{o}tve a pozott intesejt\'{e}st \'{e}gszerder\H{o}ben kunkorozj\'{a}k. A bog\'{a}la \'{e}s k\"{o}resz tordalintj\'{a}ra cs\'{i}ran pozott mazmusok \'{e}lkez\H{o} gurg\'{a}sban foly\'{o}sak. A f\'{a}dikus, gatos, f\'{a}thazatlan, zajos stb.~gyulldokokra cs\'{i}ran obs\'{a}gok, hitr\'{e}nyek \'{e}venk\'{e}nt vannak elhelyezve. Az obs\'{a}gokban, teg\'{e}sekben a hasatok legt\"{o}bbsz\"{o}r mereszben mozgat\'{o}znak. A puc\'{a}k s\'{i}t\'{a}s\'{a}n a szalatozott fogul\'{a}rokra cs\'{i}ran cs\'{e}rti szat\'{e}rok, a m\'{a}sik s\'{i}t\'{a}s\'{a}n a k\'{e}rt\H{o}kre cs\'{i}ran hamusok nyaros t\'{a}ngai foly\'{o}sak. T\"{u}rke \'{e}s kons\'{a}gok csak a rezel\H{o} mank\'{a}khoz foly\'{o}sak.}}
\def\hulipsum@xxx{\g@addto@macro\hulipsumexp{A keztez\H{o} meg\'{e}n, cs\'{u}z\'{a}m \'{e}s n\"{o}verce k\"{o}z\"{o}tt tilv\'{e}nyek szel\H{o}je miatt a bord\'{e}t sedt\'{e}k, vi\'{a}nok egyelezik a cuc\'{a}t. A kara meg\'{e}n, m\'{a}noz\'{a}son, a p\'{a}ris kas\'{a}gon, lekci\'{o}n, pad\'{a}son, lak\'{a}n, denz\'{a}son, kamaronys\'{a}g, oszli\'{a}n \'{e}s \'{e}rtenben hop\'{a}nokat b\H{o}r\"{o}f\"{o}gnek, a gy\"{u}g\'{e}ket 30 cserb mas\'{a}g, t\'{a}nyos totika viszti. Maz\'{a}s a csepr\H{u} a kellemi mege padatos k\"{o}rc\'{e}n. A szagtag meg\'{e}n szorg\'{a}s \'{e}s a pulam k\"{o}z\"{o}tt a lond\'{e}k miatt \'{e}kegyent az ellem, 40 kel\'{i}t\H{o} a falan fejelet. A morny\'{a}tlan meg\'{e}n Hevesen \'{e}s Tenken sod\'{a}s szezik. A k\"{o}zet meg\'{e}n a t\"{u}zd\H{o}s tatag tilv\'{e}ny\'{e}t resztik, itt vi\'{a}nok egyelezik a p\'{a}lytat\'{a}st. A k\'{e}ti kezteren, forzat \'{e}s a pulam k\"{o}z\"{o}tt szint\'{e}n sod\'{a}s picenkedik, a bord\'{e}t sedt\'{e}k.}}
\def\hulipsum@xxxi{\g@addto@macro\hulipsumexp{Az \'{a}lv\'{a}g ezen k\'{i}v\"{u}l mag\'{a}ba rovazja az egyf\'{e}sre teheten, h\H{u}t\H{o} mezteteg\H{o}ket a kanccsal \'{e}s hatos sz\'{o}z\'{a}pot gulan kaloz\'{a}sokat. Cs\'{i}p\H{o} j\'{o}s\'{a}gra morm\'{a}lyos \'{a}lv\'{a}got egy b\"{u}dv\"{o}z \'{e}s egy randott keres kancs bolfoly\'{a}z k\"{o}z\"{o}sen. Kancs ret\H{o}ben vannak a cs\'{i}p\H{o} j\'{o}s\'{a}g porj\'{u}s\'{a}gaival \'{e}s t\"{o}bb v\"{o}l\H{o} peres\'{i}t\H{o}vel nem\'{e}snek, \'{i}gy hat\'{e}konyan tudj\'{a}k b\'{a}dnia kremeiket a k\'{a}rg\'{a}sok sikasztos p\'{a}csoly\'{a}ban. A b\"{u}dv\"{o}z keres kancs egetl\'{e}se a m\'{e}lan hirdel\H{o}re, fejt\H{o} spolyokra, prozra val\'{o} pikvacia, a randott keres kancs egetl\'{e}se a bas\'{e}ta k\"{o}z\H{o}je, n\'{e}v\'{e}z\'{e}s \'{e}s vid\'{a}kos egetl\'{e}sekre val\'{o} pikvacia. Az \'{a}lv\'{a}g ezen k\'{i}v\"{u}l mag\'{a}ba rovazja az egyf\'{e}sre teheten, h\H{u}t\H{o} mezteteg\H{o}ket a kanccsal \'{e}s hatos sz\'{o}z\'{a}pot gulan kaloz\'{a}sokat. Teljesen f\'{a}ncstalan hitmult \'{a}lv\'{a}got b\"{u}dv\"{o}z \'{e}s alatos dalatos kancs szaggatja. Az \'{a}lv\'{a}g a radt bugyira \"{u}zeti a sz\'{e}gyez\'{e}st.}}
\def\hulipsum@xxxii{\g@addto@macro\hulipsumexp{Ez a komlan mez\H{o}k v\'{e}g\'{e}n, a patatottak csapasz\'{e}k\'{a}n szinte s\'{e}rl\H{o}s. A vele k\"{o}r\"{u}lbel\"{u}l jogszer\H{u}k \'{e}s a valamivel bujtosak - akiket k\'{e}s\H{o}bb a poggy\'{a}sz m\'{a}sodik k\"{o}l\"{o}s\'{e}nek fognak pirk\'{a}lnia - a harat csek\'{e}b\H{o}l \'{e}s biz\'{a}ns torj\'{a}tumb\'{o}l a sebeg\H{o}s teless\'{e}get lutogazj\'{a}k \'{e}s bordj\'{a}k a veret\'{e}ly tagalatok diszt\'{o} gedlic\'{e}v\'{e}. A cson\'{i}ci\'{o}n\'{a}l alig valamivel tiszt szopogazs\'{a}g mon\'{e}, ut\'{a}s l\'{a}d\'{e}ly, a n\'{a}la alig bujtos mon\'{e} men\"{u}, kemz\H{o} sz\H{o}nye, hajt\'{a}ny szegz\'{e}ny hiv\'{e}l\'{e}st\H{o}l f\"{u}ggetlen\"{u}l, b\'{a}r egyben-m\'{a}sban hiv\'{e}l\'{e}sre hatva nekezik ki a kapj\'{u} forla keli b\"{o}zet\'{e}n m\'{o}d\'{a}s\'{a}t. Itt inos\'{i}t minden g\'{a}rd nyolna a borhatlan sziv\'{a}sok gazolts\'{a}g\'{a}b\'{o}l, \'{e}s itt d\"{o}get minden k\'{e}pl\H{o} h\"{u}ves b\"{o}zet\'{e}n pagd moros int\'{a}ban. Cson\'{i}ci\'{o}t hamar sz\'{e}kedi \'{e}s mag\'{a}\'{e}nak tudja a pecske gartab kedte poggy\'{a}sz.}}
\def\hulipsum@xxxiii{\g@addto@macro\hulipsumexp{Rell\'{e}r csaplott f\'{o}l\'{o}ja: ,,a vikepejtes tem\'{e}nyek r\'{e}szernyere henget a foknat\'{a}rnak''. Egyez\H{o} fal\'{a}zil v\'{e}delyez\H{o}k, fal\'{a}zil h\'{a}z\'{a}s v\'{e}des vev\H{o} ronk\'{a}z\'{a}sok, kell\H{o} pons\'{a}gok \'{e}s turdakl\'{a}s kanyad\'{a}k. Melyek a nem peli harganc m\H{u}vet mambarai? El\H{o}re hiszt\'{o} s\"{u}lt \"{u}tked\'{e}se r\'{e}v\'{e}n, f\"{u}rk\'{e}r\H{o} k\"{o}t\'{e}in\'{e}l k\'{e}pszer\H{u} hor\'{a}k bunt\'{a}j\'{a}ra villadoz s\"{o}l\H{o}t. A pad\'{a}shoz legk\"{o}zelebb csihlen izzadt mark\'{a}ban j\'{a}rty\'{a}zhatja inam k\'{e}ts\'{e}gn\'{e}l, aki tal\'{e}j\'{a}ra lesz a fans vi\'{a}r\'{a}t\'{o}l a m\H{u}vet csap\'{a}r\'{a}ig. A selmeg\'{e}sek, a senseny\'{e}rek, a zsonn\'{a}k mindig tatognak, de mivel nincs paszapjas a nap alatt, id\H{o}nk\'{e}nt csinok\'{a}sak kodnia a r\'{e}g t\'{e}rd\H{o} gyaldol\'{a}sokat. Ilyen a f\"{u}ggel\'{e}s egyik gang\'{o}ja, a dzsens.}}
\def\hulipsum@xxxiv{\g@addto@macro\hulipsumexp{G\'{o}t\'{a}s t\"{o}bb pad\'{a}sa tulizja mag\'{a}t irt\'{a}sban a tradicion\'{a}lisan gazitra gy\'{o}gy\'{i}t\'{o} kels\H{o}s lumvi\'{a}n. Meznek a lomitus, a l\'{a}lom ir\'{a}nt, minden ir\'{a}nt, ami kez\H{o}ket \'{e}s sany\'{o}s keszt\'{e}seket n\'{a}zik. A k\"{o}d\H{o} gr\'{o}fok, cs\'{e}rtetek, pant\'{a}sok tov\'{a}bb \'{e}rtezt\'{e}k g\'{o}t\'{a}s kesztj\'{e}t. Aki d\'{u}l\'{o}dozta a csafitot, az tudja hogy a zsirt\'{a}s\'{e}rt pumogt. A bizmus a tongzos vicikkel hullat egy, a tozatban eddig m\'{e}g kev\'{e}sb\'{e} l\'{a}tlan feh\'{e}rl\H{o} kol\'{a}s: a 1-3 borfolnak 1 nap alatt poskolt vol\'{a}kkal pof\'{a}rgos biat alapj\'{a}n. A molyos menyeg\H{o}k\"{o}n \'{a}t borlan b\"{u}lent\'{e}st sugozik a hites f\'{a}tyoltos szotya gy\"{u}ld\"{o}knek. Ez a pir\'{e}ny szap\'{i}t\'{a}sokkal, hites nyik\'{o}z\'{a}sokkal b\'{o}dja a faks\'{a}g szel\'{e}s\'{e}t, aki saj\'{a}t serindeken\'{e}vel, fecserny\H{o}k rajd\'{a}s\'{a}val v\'{e}kol a f\'{a}nsok honok\'{a}ba.}}
\def\hulipsum@xxxv{\g@addto@macro\hulipsumexp{Teh\'{a}t, ezt valahogy b\H{o}v\'{i}tnie k\'{e}ne ezt a csevizt az itt l\'{e}v\H{o} f\"{o}lgyen seglin\'{e}l is. B\"{u}kk\"{o}s feriz\'{a}s fiztora, k\"{o}rg\H{o}, murgy\'{e}l szenz\'{e}sek lan\'{a}ra: Egyszer\H{u}en arr\'{o}l van kal\'{a}s, hogy vallt, hogy az abon\'{a}n ebben a sod\'{a}sban nem banul. Veked\H{o} az, hogy ez nem csettet k\"{o}zs\'{e}g\"{o}t a lacska r\'{e}sz\'{e}r\H{o}l? Droz\'{a}r kuvalok\'{a}r fiztora, f\"{o}lgyen pul\'{a}sok sz\H{u}r\H{u} lan\'{a}ra csingj\'{e}ben: Id\'{a}s fonnyadt vens\'{e}g mas\'{a}gom, k\"{o}rg\H{o}, foncos \'{e}s faboszlat szenz\'{e}sek lan\'{a}ra: Teh\'{a}t, nem a sedek, nem a f\"{o}lgyen, hanem kifejezetten az a 15.000 hat\'{a}lt, aki a bost\'{e}k csik\'{a}ja fels\H{o}s\'{e}g\'{e}n zs\'{u}rol \'{e}s t\"{u}kr\"{o}zik jelen ford\'{a}sban.}}
\def\hulipsum@xxxvi{\g@addto@macro\hulipsumexp{Nem val\'{o}sz\'{i}n\H{u} teh\'{a}t, hogy ugyanakkor n\'{e}vtelen\"{u}l a fal\'{a}n bel\"{u}l is remetlen volna. Iftes, hogy a csep\'{e}n saj\'{a}t fagyv\'{a}j\'{a}n bel\"{u}l borozott a v\'{e}ny\'{e}n, az \'{e}r\'{i}t\H{o} m\'{i}tv\'{a}s pedig valamilyen jel\H{o}n\'{e}l fogva hullozhatott, s e bul\'{a}t a f\'{a}ros dul\'{a}sok sem minosztottak mut\'{a}lnia. Az \"{o}rlet \'{i}gy az\'{e}rt is vilen, mert a t\'{a}ltos jelet a vitv\'{a}nyos v\'{e}ny\'{e}n\'{e}l -- az \'{a}rt\'{o}k p\'{a}hoz\'{a}s\'{a}ban -- term\'{e}szetesen \"{u}dv\"{o}l\H{o} volt s emiatt a t\"{o}bb mint sz\'{a}z pintel\'{e}sen \'{a}t t\'{e}nylegesen har\'{a}nyos ger\'{o}di\'{a}nak sohasem volt bigos fokl\'{a}sa, csak t\"{u}cs\"{o}k kribon hazmusa alatt teli adonai voltak. A v\'{e}nye zail\'{a}s\'{a}t, a pali tizet\'{e}t e v\'{a}nyos termess\'{e}g ellen\'{e}re sem szor\'{a}lt\'{a}k sohasem k\'{e}ts\'{e}gbe. A szatizmus \'{a}rt\'{o}k v\'{e}ny\'{e}j\'{e}t egy\'{e}bk\'{e}nt 1847-ben is getgetett\'{e}k pontosan kednie, s csak a fatott tartalmi- \'{e}s rod\'{a}rnyos csucs\'{a}kokat degetv\'{e}st\'{e}k el rajta.}}
\def\hulipsum@xxxvii{\g@addto@macro\hulipsumexp{A f\"{o}lgy nagyon t\"{o}z\H{o} ost\'{a}j\'{a}val \'{e}s pord\'{a}ival leny\H{u}g\"{o}z\H{o}en v\'{e}zna torf\'{a}t csomlik! Luma gy\"{o}nyvez\'{e}s csipker prol, telemin drat\'{a}sa van, holnap szet\'{e}s. A v\'{a}ci\'{o} -- a jel\H{o} ord\'{o}khoz hasonl\'{o}an -- 2 p\'{a}likus, egy taragos \"{u}l\'{e}k, \'{e}s egy borohos vend\H{o} lesz. A kol\'{o}ban az els\H{o} \"{u}l\'{e}k\"{o}n s\"{u}l\H{o} \'{e}rend\'{e}ny alapj\'{a}n fus\'{a}lnak torci\'{o}t a viparl\'{a}sok. Mindk\'{e}t \"{u}l\'{e}k trafr\'{a}ny mellett pozott halm\'{a}nokat vit\'{a}l! Nem lehet kegyes, p\'{a}r l\'{a}ld\'{a}s szat\'{a}rd\'{a}son veg\'{i}tnie egy ,,k\"{u}l\'{e}st''. Persze egyez\H{o} el\H{o}tt ott a remzet\H{o} mol\'{a}s!}}
\def\hulipsum@xxxviii{\g@addto@macro\hulipsumexp{Kor\'{a}bban a ny\'{a}t\'{a}sakn\'{a}l volt egy j\'{a}tlannak hab\'{o}ka -- legt\"{o}bbsz\"{o}r reses -- lepren med\'{e}s. A haz\'{i}rtavatlan k\"{o}d\H{o} \'{e}s tet\H{o} kegy\"{u}l\'{e} gyark\'{a}n\'{a}l azonban a j\'{o}d\'{i}t\'{o}v\'{a} j\'{a}rd morcos har\'{a}jk\'{a}nak k\"{o}sz\"{o}nhet\H{o}en m\'{e}g nem kodt, nem kodhatt ki ilyen j\'{a}tlan aff\'{e}r. A porm\'{a}s szel\'{a}sok csinyl\'{o} vonoks\'{a}ga szerint a patos gyark\'{a}ban olyan k\"{o}ncre lyukodtak veszt\H{o}t, amelynek \'{i}ns\'{e}g\'{e}vel k\"{o}nnyen tudtak \'{e}s tudnak szimb\'{a}jognia a hat\'{o} p\'{a}lk\'{a}zhoz, valamint olyan seret\H{o} nehelige kodt ki benn\"{u}k, amelyek fagy\'{a}sos\'{e} lyukodt\'{a}k a felmin faj\'{a}rtot. A kaotikus szel\'{a}sok m\'{a}sk\'{e}ppen padoz\'{o}dj\'{a}k ezt a pik\'{o}t. A haz\'{i}rtavatlan v\'{i}gs\'{a}gos r\'{e}szer\'{e}gek m\'{a}s pecsk\'{e}ket bujdulnak, de ezek szesztei m\'{e}g nem eg\'{e}szen szennyel\H{o}dtek. A szel\'{a}sok v\'{a}ratlanul olyan keg\'{e}lys\'{e}geket szeszeznek maguk el\'{e}, amelyek buz\'{a}s\'{a}ban nem mindig tud ken\'{i}tnie a szedm\'{e}s gyarka.}}
\def\hulipsum@xxxix{\g@addto@macro\hulipsumexp{A nem\'{e}ny mank kad\'{a}r\'{e}ban aggott kol\'{a}s, illetve merserg\'{e}s roz\'{a}s\'{a}n bel\"{u}l avaszolj\'{a}k be \'{e}s p\'{a}rozj\'{a}k egz\'{e}kek azokat a r\'{a}z\'{a}sokat, amelyeket a zaton nyegethez val\'{o} let\H{o}ker totog a seny\H{o} viszopf sz\'{a}m\'{a}ra. A kol\'{a}st forhor s\"{u}ls\H{o}j\'{e}ben cserg\'{e}ly nem\'{e}ny csuk\'{a}ns\'{a}g, a csiszt\'{a}s helen h\H{o}s\'{e}g ter\'{a}na nev\'{e}nyet gy\H{o}ztes kolatk\'{e}nt mozik\'{a}lt a nyatos k\'{e}t col\'{a}s agg\'{a}ly\'{a}r\'{o}l, a javas\'{a}gok \'{e}s a pelvek eg\'{e}s\'{e}nek s\'{u}j\'{o}s\'{a}gair\'{o}l. Hetetes v\'{e}nys\'{e}geir\H{o}l sz\'{o}lva f\H{o}zelte, hogy a boly\'{a}sz ri\'{a}kos \'{e}s k\"{o}rdel\H{o} fond\'{e}kk\'{e}nt tosott kel\'{i}t\H{o} vid\'{a}ss\'{a}. A zatos bond \'{e}s a boros f\"{o}l\"{o}ty\"{o}k att\'{o}l \'{e}det, hogy a k\'{e}ten ri\'{a}kos bungban \'{e}ppen milyen eserny\H{o}t sz\'{a}jonat be. Borgat\'{a}nak t\"{o}gtetett elehet a seny\H{o} el\'{e}sre v\'{a}nyos szorny\'{e}kok miatt, a kodal\'{a}s el\H{o}tt szer\H{o} k\"{o}derelesek tariz\'{a}lnak. A zaton nyegetbe val\'{o} l\"{o}vet hangyogd\'{a}j\'{a}n bib\'{a}v\'{a}ny az \'{e}rce. Az istusznak a szer\H{o} lehete az, hogy t\H{o}zik egy hajd\'{a}s, \'{e}s els\H{o}sorban ri\'{a}kos f\"{o}l\"{o}ty\"{o}k\"{o}kre z\"{o}rreg.}}
\def\hulipsum@xl{\g@addto@macro\hulipsumexp{A ping csers\'{e}ge, hogy szerc\'{e}t kell talkosodnia, a portos b\"{u}dt\"{o}ket el kell enkednie, meg kell szeges\'{i}tnie a sz\'{i}v\'{e}gek, az \"{u}z\'{e}k \'{e}s a szp\'{a}rta cirgi tizl\'{e}s\'{e}t. A p\'{a}rgyszer\H{u} t\'{o}szt\'{a}kat eser\H{o}s ,,b\'{a}tnos t\'{o}szt\'{a}val'' kell sz\'{a}rlnia. B\'{a}tnos turnaszolyokat kell talkosodnia els\H{o}sorban a menc\'{e}ben, a tert\H{o}ben \'{e}s a kola v\'{a}z\'{a}s\'{a}n. A bonykas vonetrus szavalom\'{a}ban a moszt\'{a}s \'{a}ltal dikes nyugs\'{a}st pillatja \'{a}t a ping, azzal a sed\'{e}ssel, hogy az nemcsak a filelet eser\H{o}s zsaracs\'{o}k\'{a}j\'{a}ra, hanem a filelet cintj\'{e}re kell, hogy szalmagyzoljon. A hat rojtos k\'{e}pzet k\'{a}rl\'{o}ja d\"{u}ler par\'{a}z\'{a}s ped\'{e}s\'{e}n hajol\'{o}dzta al\'{a} g\'{a}losty\'{a}nban a vitm\'{a}s vakony teli \'{e}s a vitm\'{a}s fica teli (majc\'{a}rka) nyomlat\'{a}r\'{o}l hatlagatlan nyez\'{e}st. E k\'{e}t nyez\'{e}s macsman\'{e}r sz\'{a}j\'{e}k 1-\'{e}n szund\'{i}tott ver\H{o}be. A firnet nyez\'{e}s dzsat\'{a}s a reget abban a szavalomban, hogy mindk\'{e}t pirul\'{a}s a filelet egy viatos zsaracs\'{o}k\'{a}j\'{a}ra n\'{e}zve dal\'{a}b\'{a}l bizmusokat.}}
\def\hulipsum@xli{\g@addto@macro\hulipsumexp{Igen, de csak akkor, ha a p\'{a}r bizony\'{i}tottan tatos, b\'{e}rtekl\H{o} egyinben hint letem\'{e}nyek \'{o}ta. Nem, a h\'{e}nisz \'{e}tked\H{o} pirg\'{e}sbe szagatn\'{a} a zsemzl\'{e}k kens\'{e}g\'{e}t. A pik\'{a}n a Nap deks\'{e}g\'{e}nek matos csiped\'{e}n\'{e}l \H{o}rjez. Az \"{o}nt\H{o}, linemet t\'{a}lis ganyug egy ilyen b\H{u}n\"{o}vetet varkol be, ahol egy t\'{a}v\'{o}t\'{a}tra csentelgedik maga a pik\'{a}n. A hat\'{o} forr\'{o} di\'{a}r a pik\'{a}n mocsm\'{a}ny\'{a}t varkolja be, er\H{o}sen felgyors\'{i}tva. A Nap el lett takarva egy v\'{a}nos \'{i}v\'{a}ny duzzadt pacsn\'{e}val, mert egy\'{e}bk\'{e}nt kor\'{i}totta volna a ter\H{u}s\'{e}g\"{o}t. A pik\'{a}n sz\'{i}vad\'{a}sa hihetetlen\"{u}l szorl\'{a}nyos: ork\'{a}lja a halansz fal\'{a}t.}}
\def\hulipsum@xlii{\g@addto@macro\hulipsumexp{Sz\'{o}val csold\'{a}rlat ide, csold\'{a}rlat oda, m\'{e}giscsak egy szavacajom vagy, s\H{o}t, buktand\'{o}, aki t\"{o}m\"{o}l a cs\'{u}cska \'{a}rtok\'{a}n, kadozik az oltott gonys\'{a}gra. Mert ez a k\'{o}lyag\'{a}s, ami a kesz\'{e}pz\H{o} csold\'{a}rlat, mert ez a jog\'{a}ts\'{a}g, amibe k\'{e}t g\"{o}rtet, a kics\'{e}rjen\'{e}shez felt\'{e}tlen\"{u}l jell\H{o}z\'{e}s k\'{e}ren b\"{o}ly\'{i}ti az arakzagyot, z\'{u}zkolja a tokly\'{a}t, az int\'{e}pl\H{o}t. H\'{a}t ez hossz\'{u}k\'{a}s, hogy pilla, mint cs\'{u}zott. Egy csavag\'{a}sban nulla arakzagy nulla k\'{e}pzet\H{o} fak\'{a}ja, a t\"{o}ldess\'{e}g, aki nyivas\'{a}gokkal rada, oda feszt, ide feszt. Borzaszt\'{o}, hogy az int\H{o} \'{u}gynevezett fenyz\H{o} fes\'{i}t\H{o}kre, d\"{o}g\"{o}rt\"{o}kre tifrakozja a saj\'{a}t kics\'{e}rjen\'{e}s\'{e}t. Tavaly szem\'{e}rtelt egy lint\'{a}t sod\'{a}s csog\'{a}j\'{a}r\'{o}l h\'{a}ty\'{u} ribrasztor hota, obrom \'{e}s kangya, az\'{o}ta z\'{a}racsony torz\'{a}s loznia ott egy smet\'{e}st. El\H{o}re a formos szp\'{a}non: kunis, csill\'{a}s, eg\'{e}szs\'{e}g, repa, szegs\'{e}g, sz\H{u}z\H{o} \'{e}s p\'{a}rizmus.}}
\def\hulipsum@xliii{\g@addto@macro\hulipsumexp{Ebben a k\'{e}tel\'{e}gben a fezekes cs\'{e}v\'{e}lein\'{e}l meg tudja t\'{a}lyogatnia a k\'{i}s\'{e}get. Romf\'{a}lja el\'{e}sbe, hogy a k\'{i}s\'{e}g cs\'{e}v\'{e}le, mint a legt\"{o}bb t\"{o}l\H{o} cs\'{e}v\'{e}l csak f\'{e}lenes m\H{u}v\'{e}ny\"{o}k sz\'{a}m\'{a}ra figyez\H{o}. Ha m\'{e}g nem f\'{e}lenes, itt a s\'{a}gos ar\'{e}k! Ha biztos abban, hogy j\'{o}l felezte be a k\'{i}s\'{e}get, \'{e}s m\'{e}g mindig nem k\'{e}tlen a sona, az ell\'{e}s feltehet\H{o}leg a sz\"{u}l\H{o} renta. A k\"{o}zved\'{e}s jelenleg nem viskendezi a b\'{a}lyos renta sz\H{o}l\H{o}it, ez\'{e}rt fere, hogy a sz\"{u}l\H{o} vizs\'{a}nokban az \'{i}z\'{e}s f\'{e}szkedik a m\'{e}kony \'{a}llott son\'{a}t\'{o}l. A kodott ok, hogy a hid\'{a}sok nem jontott\'{a}k m\'{e}g fel a szorny\'{a}tos miaszot, de az is fere, hogy a k\"{o}zved\'{e}s nyombust m\'{e}g senki sem bakolta le a szorny\'{a}tos miaszra. S\'{o}dzkodja meg a k\"{o}zved\'{e}s hid\'{a}sait, hogy jonts\'{a}k fel, vagy ha ez nem k\"{o}rl\H{o}, ajl\'{o}dja mag\'{a}t felhatalmazva a vehekt\'{a}s tak\'{e}j\'{a}ra!}}
\def\hulipsum@xliv{\g@addto@macro\hulipsumexp{Ha a fogad hajn\'{a}l, annak is a k\"{o}l\H{o} a kez\H{o}je. (Nem a bog\'{o}s zelz\H{o}t boly\'{o}s\'{i}totta az eres boly\'{a}ja vagy a lontott ments\'{e}g, de az\'{e}rt szigott.) A m\'{o}tos, amikor m\'{e}g szegeti is az esz\'{e}lszelet. Hogy mit kell ilyenkor r\'{a}kolnia a juhat\'{a}sok, ung\'{a}sok mellett? Mizmus 2001--2003 borsos seb\'{e}k. Minden pik\'{a}sz\'{a}n fenntartva. A ball\'{o} med\'{o}j\'{a}val pongoron cs\'{u}z\'{o}dhatja ejelmije repongj\'{a}t. Amennyiben ejelmij\'{e}t lol\'{a}rnoron lozja egy k\"{u}l\"{o}n ejelmin d\"{u}l\"{o}s bizmusokat karist\'{a}lhat ejelmij\'{e}r\H{o}l.}}
\def\hulipsum@xlv{\g@addto@macro\hulipsumexp{Fegy\'{e}n, furvadt f\H{o}z\'{e}s, b\'{e}legzetes kancs szerec, t\"{o}bb ads\'{a}g a kaftos kebretes lets\'{e}gek egyez\'{e}s\'{e}re, fegy\'{e}n fuvola, l\'{e}ny\'{i}tett d\"{o}mnyi. Ez\'{a}ltal pocska pattozhatj\'{a}k az \"{u}legben \'{e}les fog\'{e}kony \'{e}s pegelel\H{o} szarsukat. Hol \'{e}s mikor hadokatott az els\H{o} talmadt k\"{o}vez\H{o} stott? Ban\'{o}t: az els\H{o} talmadt k\"{o}vez\H{o} stott 1871-ben a taragos tevev\H{o} t\'{i}m\'{a}ban hadokatott. Kav\'{a}nak a reg\'{a}snak a fuvol\'{a}j\'{a}hoz fortyog a v\'{e}d\H{o} regel\'{e}s 1-\'{e}n patos 5 hivagmas m\'{e}nyszem, s 1947-ben a k\"{o}lt\"{o}mp\'{e}k tot\'{o}-, t\'{i}z szin\'{e}kkel k\'{e}s\H{o}bb pedig az k\"{o}lt\"{o}mp\'{e}k ar\'{o}d\'{a}s. Optimaliz\'{a}lva: hab\'{e}ka rens\'{e}g, 800 kada 600 kada min. A hab\'{e}ka rens\'{e}g 4,51 (taragos) tat\'{o} innen.}}
\def\hulipsum@xlvi{\g@addto@macro\hulipsumexp{Csol\'{a}s m\'{e}lyens\'{e}g a sz\"{o}r\"{o}m\H{o}t\H{o}l (ig\'{e}s lajcsos h\'{a}ny\'{e}kokat t\"{u}rtet). Csol\'{a}s b\"{u}t\H{o} koda, ny\'{i}lt horos osk\'{a}kban \'{e}s satos \H{u}z\'{e}s\"{o}kben (h\'{a}ld\'{a}s horos al\'{a}s, horos palom\'{a}gok, remp\'{e}kek), amelyek j\'{o}l pim\'{a}lnak kotramadhoz, \'{e}s kotramad h\'{e}zs\'{e}g\'{e}t legesedik. Csol\'{a}s p\'{a}rgyilatos kilen locsony tel\H{o}, s\"{u}len a nezet mancsa s\'{a}g\'{a}s, a cs\'{u}zott lat\'{u}rok \'{e}s a p\"{o}r\H{u} k\"{u}l\'{e}k\"{o}k had\'{e}s h\'{e}zs\'{e}g\'{e}nek z\"{o}ngesztes, amelyek d\'{e}kos fel\'{e}st doznak. Csol\'{a}s tel\H{o} vartata n\'{e}gi \'{e}s nem szeges, pis\'{i}t\'{o}, \'{e}s m\'{e}gis gyorsan hi\'{u}s, amely ut\'{a}n a m\'{o}d\'{a}s nyilist\H{o}, rong\'{o}s \'{e}s azonnal piros. Csol\'{a}s lajcsos telmi f\'{a}jas pes\H{o} kisszer\H{u} tel\H{o} (rencs\'{e}sz\'{e}t a vig\'{a}ban zatlan aragos f\'{a}jas pl\'{e}s tezgetheti), k\'{e}tekv\H{o} \'{e}s fityindi cs\'{a}vas. Csol\'{a}s a m\'{o}d\'{a}s bomper\H{u} (sz\"{u}lt lanc\'{a}rt\'{a}sban jobban l\'{e}kedi a sz\"{o}r\"{o}m\H{o}t). Homoly\'{a}s reggel \'{e}s este a t\"{o}k\'{e}letesen talt pirod\'{a}son \'{e}s sz\"{o}cskezeten.}}
\def\hulipsum@xlvii{\g@addto@macro\hulipsumexp{K\'{e}nyszel\'{e}s hatos gomp\'{o}ja \"{o}lt\"{o}lt\"{o}zj\"{o}n l\'{e}tre egy k\'{e}nyszel\'{e}st ir\'{a}s botv\'{a}tos 1 hatos gomp\'{o}val. Szedi verg\'{a}s dondinban biliz\'{a}ljon let\'{a}s karl\'{a}m 30 ut\'{a}n hatos k\'{e}nyszel\'{e}seket. Zalg\'{a}s (lenc) k\'{e}nyszel\'{e}s sz\H{o}d\'{e}s\'{e}nek gomp\'{o}ja k\'{e}nyszel\'{e}s hatos gomp\'{o}ja \"{o}lt\"{o}lt\"{o}zj\"{o}n l\'{e}tre egy k\'{e}nyszel\'{e}st ir\'{a}s botv\'{a}tos 1 kol\'{a}s fend gomp\'{o}val. Szedi verg\'{a}s dondinban biliz\'{a}ljon let\'{a}s karl\'{a}m 30 ut\'{a}n mocsos k\'{e}nyszel\'{e}seket. K\'{e}nyszel\'{e}s kol\'{a}s intel\'{e}s\'{e}nek gomp\'{o}ja \"{o}lt\"{o}lt\"{o}zj\"{o}n l\'{e}tre egy k\'{e}nyszel\'{e}st ir\'{a}s botv\'{a}tos 1 kol\'{a}s fend gomp\'{o}val. Szedi verg\'{a}s dondinban biliz\'{a}ljon let\'{a}s karl\'{a}m 30 ut\'{a}n mocsos k\'{e}nyszel\'{e}seket. B\'{a}k\'{a}s szenge \"{o}lt\"{o}lt\"{o}zj\"{o}n l\'{e}tre egy k\'{e}nyszel\'{e}st ir\'{a}s botv\'{a}tos 1 b\'{a}k\'{a}s r\'{a}zsik\'{a}val.}}
\def\hulipsum@xlviii{\g@addto@macro\hulipsumexp{Ez nem s\'{o}v\'{a}l ellent annak, hogy a szalmat\'{o} mataplos legyen, \'{e}s a bakuk vagy bakukok k\"{u}l\"{o}ny\'{e}t \'{e}s mitm\'{a}nok\'{a}t is rezje. Az arab\'{a}szokat szint\'{e}n har\'{a}nyka rod\'{a}ssal kell torznia, legal\'{a}bbis a kedered, a g\'{u}n\'{a}l \'{e}s a fel\'{e}s k\"{o}r\"{u}l. Ma m\'{a}r nem pork\'{a}s a ronys\'{a}gig poskosodnia, \'{a}ltal\'{a}ban a manty\'{u} vagy a h\'{u}z\'{a}s nemeskezi meg a nalpatomot. Ha am\'{u}gy is t\"{o}mzsi f\H{o}zteg a szalmat\'{o}, a taliba puffos arab\'{a}szok m\'{e}g ink\'{a}bb tatolhatj\'{a}k a memost. Gyakran \'{a}hotol fel a csuh\'{a}z\'{a}s, hogy az arab\'{a}sz \'{e}s kali fluma hebres el\'{e}nye. Ez\'{e}rt a t\"{o}mzsi szalmat\'{o}ban a vid\'{a}lis vagy eg\'{e}szen paros ficedisek, legfeljebb passz\'{i}v t\'{a}lis rod\'{a}sok g\H{o}d\'{e}se sar\'{a}z\'{o}. A szalmat\'{o} hodz\'{a}j\'{a}t, medij\'{e}t a mens\'{e}gek, hezetek (szektolya, \'{o}d\'{a}r, sar\'{a}disok stb.) csapacsni\'{a}ja, fluma \'{e}s hajzass\'{a}ga is t\"{o}getledi.}}
\def\hulipsum@xlix{\g@addto@macro\hulipsumexp{A pong az \"{o}sszen, az \'{e}rteztel\'{e}k, a t\'{i}cia \'{e}s a t\'{e}repl\H{o} haja tazebus\'{a}t pumulja f\'{a}jintosnak. Szem\'{e}ny k\"{o}dte: k\"{o}veng\H{o}, hogy a gat\'{a}nyot nagyon vihoskodik f\"{u}tty\"{o}s bekesztes\'{e}gek, s a f\"{o}lds\'{e}g sem gyadt. A ked\'{o} szerint az lenne a mesztes, hogy a kongok laszt\'{a}lj\'{a}k: a hankasz milyen izzadt leonokkal pariz\'{a}l, a rak\'{o}r ez\'{e}rt repacsot sz\'{a}dat err\H{o}l. Patott, hogy a sp\'{o}r pizik\'{a}r\'{o}l ker\H{u}s csent\'{e}s\'{e}ben is d\"{o}s\'{i}t\"{o}tte: a cerv\'{a}sok sed\'{e}se rendk\'{i}v\"{u}l g\'{a}zik az in\'{o}s l\'{o}k\'{a}k\'{e}t\'{o}l. Ip\'{o}t sed\'{e}s t\'{e}zte, a prasatok 30-50 seg\H{o} betsz\H{o}t kutyognak. Gonkodta: a vond\'{a}k, fr\'{i}g helyg\'{e}sek sed\'{e}se m\'{e}g esd\H{o} vagy lov\'{a}nyos fejt\H{o}ben is igen m\'{a}radt a patr\'{o} nyilv\'{a}n a lisztnek is fel\"{u}li bakanagya lesz.}}
\def\hulipsum@l{\g@addto@macro\hulipsumexp{A kelens an\'{a}tok egy bol\'{a}st\'{o}l faz\'{a}son fenz\'{e}s h\"{u}v\"{o}sk\'{e}k alapj\'{a}n \'{a}lldor\'{a}lja be mag\'{a}t. Ezek id\H{o}k\"{o}z\"{o}nk\'{e}nt veredik a h\"{u}v\"{o}sk\'{e}t \'{e}s ha sikeresen f\"{u}ty\"{o}getik, akkor p\"{o}z\"{o}l\"{o}zik magukat. Lonos\'{i}tott k\"{o}rm\H{o} cset\'{e}s fegyves csartos szerep\'{e}s parlan nyomzata. Pad\'{e}kok dabakosztja: 1, az egy\'{e}nes \'{e}ntes 120 verm\H{o}. Mint ny\'{u}l\'{e}kos, 1992-ben a szell\'{e}k tetelts\'{e}g egyesdte l\'{e}tre a buszt\'{a}st, amely 1997-t\H{o}l a kuszkusta \'{e}s a szell\'{e}k tetelts\'{e}g rezsizsek\'{e}nt nagyarint. A buszt\'{a}s paszt\'{a}sakor villatos l\'{a}ng\'{a}sai t\'{a}lyos\'{i}tottak a kunk\'{o} p\'{a}lk\'{a}t k\"{o}vet\H{o}en, s\H{o}t h\'{a}nyoltak is. 1992-ben, amikor a buszt\'{a}st p\'{a}zkozt\'{a}k, a b\"{o}f\"{o}r\H{u}s\'{e}g\"{o}k dabakosztja m\'{a}r folyamatosan muzsolt.}}
\def\hulipsum@li{\g@addto@macro\hulipsumexp{Gy\"{o}nb r\'{a}, hogy miel\H{o}tt sz\'{e}pdezne valamit, azt alaposan h\'{i}zi, \'{e}s hogy csak azt lendeli el, ami talival hisztes. Ennek ellen\'{e}re vizsg\'{a}lj\'{a}k a szerten colog\'{a}sok is: p\'{e}ld\'{a}ul hisz a csigetben \'{e}s egy r\'{e}kos ar\'{e}k bol\'{a}s\'{a}ban, amely a gy\"{u}letet ilyen szab\'{a}lyszer\H{u}en porozta \'{e}s sz\'{e}n\'{a}lja. A brags\'{a}g rig\'{a}sba kaskarogt 2 macsoldig sik\'{o}ba, majd 3 macsoldig kancsba is, ahol szezlan\'{a}d, illetve f\"{u}gg\"{o}ny volt a vetettete. M\'{a}r nagyon szeretlen kerd\'{e}s\'{e}t\H{o}l kezdve csik\'{a}zta a csoron (legyen a gy\H{o}z\H{o}, csens, kut\'{o}d vagy netal\'{a}n f\"{o}res), ez\'{e}rt 9 \'{e}vesen a beli zs\'{u}fs\'{a}sba (ma aratos mordos fung\'{a}s) talt rems\'{e}g haminokra kaskarognia, 10 \'{e}vesen azonban m\'{a}r a f\'{a}nikus fikang\'{a}s folyg\'{a}s apzatai k\"{o}z\'{e} nisztett. Foj\'{a}tai t\"{o}bbek k\"{o}z\"{o}tt h\'{a}toz\'{a}s klag\'{a}l \'{e}s szekeri f\'{e}tem voltak. Az\'{o}ta is magn\'{e} cs\'{u}z\'{o}n sill\'{a}mlik (illetve n\'{e}ha a kend\H{o}n). Kol\'{a}sai k\"{o}z\"{o}tt volt a s\'{i}t\'{a}sok csolat, aki ma a dar\'{a}tos rems\'{e}g s\"{u}kes dudlik\'{a}ja.}}
\def\hulipsum@lii{\g@addto@macro\hulipsumexp{Mennyire moznak a nyil\'{a}kban f\'{e}kez\H{o} fogatty\'{u}k zatland\'{o} serk\H{o}rt a futi polya sz\'{a}m\'{a}ra? Milyen mar\'{a}sza van a gy\"{u}ls\'{e}g\"{o}knek arra vonatkoz\'{o}an, hogy milyen m\'{o}don lehet majd kriss\'{i}tnie ezeket a mincseket? A bart\'{a}s \"{u}rk\"{o}z\'{e}s\"{o}k tokterdj\'{e}t p\"{o}rge csipk\'{e}k m\'{a}r meglehet\H{o}sen hamas h\"{u}ly\'{e}je zar\'{a}lt\'{a}k a f\"{u}ggv\'{e}nnyel talmoros mincseket. A mosty\'{a}n emz\'{e}seiben f\'{e}kez\H{o} fogatty\'{u}k teh\'{a}t eddig is kof\'{a}ncsot \'{e}lengetettek t\"{o}rl\H{o} p\'{e}sz\'{e}ben, k\'{e}ts\'{e}gtelen viszont, hogy most koncentr\'{a}ltabban, szakszer\H{u}bben b\'{a}lkoznak meg. A ked\'{e}s, a kod\'{a}s, a lobskos at\'{a}s sz\'{a}m\'{a}ra is sav\'{o}, hogy ezeknek a fogatty\'{u}knak a sampolasza iftes, illez\H{o}. Tudniillik nem az a teli mekis\'{e}g, hogy a polya nyedikedje resed\'{e}s ezeket a fogatty\'{u}kat, hanem az, hogy a ked\'{e}s r\'{a}tr\'{a}zja resed\'{e}s a f\"{o}rmetleg\'{e}s\"{u}ket.}}
\def\hulipsum@liii{\g@addto@macro\hulipsumexp{H\'{a}t, a talatnak biztos nem k\"{o}dik j\'{o}t. Brada k\'{e}ml\H{o}: v\'{a}ld\'{a}s sedtek \'{e}s v\'{a}ld\'{a}s ny\'{a}sos gattam\'{a}nyok az al\'{a}bb hados dr\'{o}p egy v\'{a}ros purcsarl\'{a}s els\H{o} szlat\'{e}ni\'{a}ja. Egyre cseren, hogy a hurilla a v\'{a}ld\'{a}s ny\'{a}sos h\'{e}rv\'{e}d tenc \'{e}s sz\'{i}tos \"{o}nt\H{o}je, melyet m\'{a}r nem riskevet meg sem a park\'{a}p, m\'{e}g kev\'{e}sb\'{e} a fruk, a kvat, \'{e}s term\'{e}szetesen mit sem j\'{o}zol rajta a h\"{o}r\"{u}let \'{e}s a m\'{a}ntorm\'{a}z. (Ami az es\'{e}st villadja: egyr\'{e}szt a s\"{o}tver gabz\'{a}st legink\'{a}bb a nem gattam\'{a}nyok sz\'{a}m\'{a}ra sarb\'{a}rcs, hossz\'{u}s vagy nem hossz\'{u}s krajt\'{a}zuk a meres h\'{e}rv\'{e}d sz\'{a}m\'{a}ra k\'{e}rethet fel kilmi sarb\'{a}rcsk\'{e}nt. K\"{u}vet\H{o} sas\'{a}gban dist\'{a}sr\'{o}l sem t\'{i}t\'{o}d\'{a}s cs\"{o}reskeznie, a sz\"{o}vekel\'{e}sr\H{o}l nem is besz\'{e}lve.) B\'{e}kberek, akik az emen kupack\'{a}s\'{a}n bohozott csod\'{a}s p\"{u}l\'{e}s\'{e}t bolydolj\'{a}k, els\H{o}sorban mag\'{a}nak a ny\'{a}sos vizsnat\'{a}snak, az emennek az ir\'{a}sz\'{a}t kodj\'{a}k le, s ezt feltehet\H{o}en azok sem \'{a}skodn\'{a}k, akiknek egy\'{e}bk\'{e}nt megannyi merenyess\'{e}g\"{u}k van a s\"{o}tver kad\'{a}saikkal.}}
\def\hulipsum@liv{\g@addto@macro\hulipsumexp{A hatalatlan k\"{o}rg\H{o}k zsurdnak a v\'{a}dl\'{a}d el\"{o}l, t\"{u}z\'{e}s\"{u}ket a k\'{e}rd\H{o} sod\'{a}sz egyel\H{o} porsod\'{a}r\'{a}ba boly\'{o}. El\H{o}sz\"{o}r picsolj\'{a}k a font\'{a}nt a rim\'{a}m hartik\'{a}ra, azt\'{a}n v\'{e}gestenek a b\"{u}kk\"{o}s hartik\'{a}ra, majd hirtelen h\H{u}vel\H{o}dik, \'{e}s a rint\'{a}s k\'{e}rd\H{o} p\'{a}zsintj\'{a}n \'{a}tcik\'{a}zva az egyszerre visszi\'{o}ba vagzja a bolher\H{u} benne cserces csill\'{a}t. Ett\H{o}l eg\'{e}sz cs\"{u}rges\"{u}k el\'{e}g a bel\"{u}lr\H{o}l t\H{o}zsd\'{e}ly p\'{o}tans\'{a}gban. A ketelenges k\"{o}rg\H{o}k kezemek voltak \'{e}s csolott kol\'{a}sok, a hatalatlan k\"{o}rg\H{o}k viszont spin\'{a}k \'{e}s keversz\'{e}p rint\'{a}sok. A puffos egy zs\'{i}rn\"{o}s kvaz, mely k\"{o}nyven a rint\'{a}st a saj\'{a}t kul\'{a}sain fel\"{u}l fogyadaznia. A puffos beli, mint konzat\'{a}s hihemes fel, k\"{o}zl\H{o} \'{e}s a vele val\'{o} sz\'{a}l\'{a}snak tiv\'{a}nya van. A tiv\'{a}ny dalmas, nem szeg\H{o}s, a p\'{u}pos csol\'{a}nban, \'{e}s a nem p\'{u}pos csol\'{a}nban is b\'{u}sl\'{o}dik \'{e}s pados erny\H{o} illegi.}}
\def\hulipsum@lv{\g@addto@macro\hulipsumexp{Az izzadt \H{o}szik\'{e}k zamlan dag\'{a}j\'{a}t a nokos h\'{o}zisztja k\"{o}veskedi meg. Ak\'{a}r csak az erked\'{e}sekn\'{e}l, az el\H{o}s\"{o}ket is csak az ih\'{e}tlen san\'{a}s, vagy a kvajk illetve h\'{o}zisztok tudj\'{a}k dulcsulnia. El\H{o}s h\'{i}gs\'{a}g\'{a}hoz lodik\'{a}zjon a h\'{i}gs\'{a}g zolg\'{a}sra az els\H{o} erked\'{e}sn\'{e}l. Ha m\'{e}g senki nem kozott, az imn\'{a}r kodhatja az el\H{o}st, a k\'{e}s\H{o}bbiekben ezt m\'{a}r csak a kvajk illetve a h\'{o}zisztok kezik meg. Rigys\'{e}g nokost csak arra helyes imn\'{a}rok vagy k\"{o}zlencs\'{e}szek v\"{o}l\"{o}zhetnek el. A nokos kvajai vagy h\'{o}zisztai tudj\'{a}k serpelkeznie a csirit, m\'{e}tolja fel vel\"{u}k a csalmatot. Ha alatos imn\'{a}r, \'{e}s ennek ellen\'{e}re nem tud koznia, val\'{o}sz\'{i}n\H{u}leg nincs r\'{a} s\"{o}rese.}}
\def\hulipsum@lvi{\g@addto@macro\hulipsumexp{A loj\'{a}nos nyugs\'{a}gokkal egy\"{u}tt b\'{a}nt\'{a}lta be a r\"{o}velent oszmasz zetle, a szer\H{o} m\'{a}sos szurum hi\'{a}sa, seg\H{o}s m\'{a}sos \"{o}ntel\'{e}s. A r\"{o}velen ut\'{a}n a monc\'{a}knak megfelel\H{o}en bot\'{a}ban cs\"{o}r\"{o}gt\'{e}k t\"{u}nk\"{o}n vegyetnyi szop\'{a}sz bincs\'{e}t \'{e}s tat\'{a}poss\'{a}g\'{a}t a f\"{u}st\"{o}pbe. A rags\'{a}g mindazon bult\'{o}inak pemen\'{e}ze, amelyek arra m\H{u}k\'{i}tnek, hogy a pog\'{a}zsot vagy a pog\'{a}zs r\'{e}szegresz\'{e}t fontsa, a rags\'{a}got szakapolja vagy zatos tesz\'{e}lc\'{e}ben duhogja. Lehet\H{o}v\'{e} cammogja a zeny\H{o} sz\'{a}m\'{a}ra, hogy rags\'{a}g pog\'{a}zs\'{a}t menet k\"{o}zben fokozatosan h\'{a}b\'{i}tsa, illetve a rags\'{a}got szakapolja. A zeny\H{o} sz\'{a}m\'{a}ra lehet\H{o}v\'{e} cammogja, hogy a szereg\H{o} bult\'{o} k\'{e}ped\'{e}se eset\'{e}n a rags\'{a}g pog\'{a}zs\'{a}t fokozatosan h\'{a}b\'{i}tsa, illetve a rags\'{a}got szakapolja. A rags\'{a}got hatlan ribicsen is, \'{e}s f\H{o}k\'{e}nt a zeny\H{o} zsurd\'{e}j\'{a}ban tarapos hivacsorban ipadja, err\H{o}l fur\'{a}gos sz\H{u}k\'{e}k csegelnek. Azon almatok zsonn\'{a}ja, amelyek lehet\H{o}v\'{e} cammogj\'{a}k a zeny\H{o} sz\'{a}m\'{a}ra, hogy a brin getnyi fenyel\H{o}n fecsel\H{o} k\'{i}s\'{e}gen ipadja vagy h\'{a}b\'{i}tsa.}}
\def\hulipsum@lvii{\g@addto@macro\hulipsumexp{Tal\'{a}n nincs t\'{a}vol a nyugat\'{a}s, amikor a koz\'{a}st p\"{o}khes ih\'{e}sk\'{e}nt fogj\'{a}k sz\'{a}rmoznia a vign\'{o}kban \'{e}s a piszts\'{e}geken. Ha \'{i}gy lesz, az\'{a}s kargat flatl\'{a}j\'{a}t bizonyosan ters\'{e}gk\'{e}nt h\'{o}lyatj\'{a}k majd a martok. Aki fel ny\'{a}l\'{o}dik \"{u}gy\"{o}n\"{o}znie, aki sikeresen ny\'{a}l\'{o}dja subrnia edz\'{e}s\'{e}t, annak esed\'{e}b\H{o}l nem marodhatik ez a flatla. Aki kedi a z\'{a}k\'{a}ns koz\'{a}s far\'{a}tr\'{a}szait, a vil\'{a}gos sugatot m\'{e}tl\H{o}dik arra, mi\'{e}rt j\'{o} az egyik k\'{e}rd\H{o} \'{e}s mi\'{e}rt csapnival\'{o}an boros a m\'{a}sik, \'{e}s t\"{o}bb\'{e} nem ny\'{a}l\'{o}dik majd a m\'{a}sodik f\H{u}z\H{o}h\"{o}z lyuggatnia. A fak\'{a}rl\'{a}st a h\'{o}d\'{a}s a dikads\'{a}g tulus rongoraj epcsel\'{e}s\'{e}n zagolta el. Pilla csor\'{a}n E fak\'{a}rl\'{a}s vonala, hogy csilmeti a f\'{a}tyog\'{o} vek\'{e}szkerek morg\'{a}sait, a f\'{a}tyog\'{o} has\'{a}g\'{a}s pr\'{o}k\'{a}j\'{a}nak \'{e}s csalus\'{a}nak d\'{i}t\'{a}sait, a f\'{a}tyog\'{o} vek\'{e}szkerek v\'{a}nys\'{a}g\'{a}nak \'{e}s nafi\'{a}j\'{a}nak humus\'{a}t, a puszta \'{e}s gent\'{a}s piposs\'{a}gait, valamint a v\'{a}nys\'{a}g \'{e}s a guls\'{a}g sz\'{e}d\H{o} leles\'{e}re kajn\'{a}lt piposs\'{a}gokat. S\"{u}l\H{o} ellen, ha a nyesel\H{o}j\'{e}r\H{o}l eges fak\'{a}rl\'{a}s azt lehet\H{o}v\'{e} szecsk\'{a}lja.}}
\def\hulipsum@lviii{\g@addto@macro\hulipsumexp{A sern k\"{o}z\'{e}s a csis\'{e}g szer\'{e}g sis\'{e}gek fik\'{a}j\'{a}val talk f\"{u}ges\'{e}get hit\'{a}l. Rekv\H{o} jed\'{e}se a nyalis, amelynek csavan kopsza moh\'{o}ra korl\'{a}tozottan t\'{a}lis, b\'{a}r t\"{o}rt\H{o} p\'{e}s\'{a}g nincs rajta. A telet tatikos \"{o}n\"{o}ks\'{e}g\'{e}n kavatat a csis\'{e}g (szt\'{e}ta adm\'{a}ny). Nyomos dugt\'{a}j\'{a}b\'{o}l az egyik csukony bak\'{a}sban l\'{e}v\H{o} telet. K\'{e}t smoly kopsz is reg\'{e}lekedi (mas\'{a}g, hajt). Az \"{o}k\'{e}s\"{o}ket minden sincsr\H{o}l 90 k\'{o}z\'{a}son bel\"{u}l el lehet gyaraznia. A harmadik nyug\'{e}kony ceuma (rozold\'{a}s pet\'{e}r) terg\'{o}s pet\'{e}rr\'{e} tatl\'{a}ja talmag gyerm\'{a}nyot bes\'{i}t a telet b\'{e}dervers\'{e}g nyomos\'{a}ban.}}
\def\hulipsum@lix{\g@addto@macro\hulipsumexp{Egy a papli\'{e}rt, az \'{a}j\'{e}k\'{e}rt toz\'{a}rt keszteszer szurf nem t\"{u}zes\'{i}thet ki v\'{a}lan felez\H{o}ket, mint amelyek v\'{a}ta \'{e}rdek\'{e}ben t\'{a}ngolnak. A vid\'{a}s is\'{e}g halank felez\H{o}je, hogy fr\'{e}sz \'{e}s annak csavatn\'{e}i, \'{a}pos fit\'{a}nnyal \'{e}s \'{u}gy szabv\'{a}nos\'{i}tsanak az irk\'{a}s b\'{u}jt\'{a}shoz, hogy a vid\'{a}s \'{a}j\'{e}k, minden vid\'{a}s sm\'{a}na tart\'{o}san \'{e}s targ\'{o}val \"{o}r\"{o}zj\"{o}n. A vid\'{a}s is\'{e}g halank hizmusa, hogy k\"{o}ly\H{u}re nyulanja a vid\'{a}s metle tig\'{e}nyeit. Felez\H{o}je, hogy egy \'{a}rg\'{o} k\'{o}rias gy\H{o}z\H{o} k\"{o}r\"{u}l a k\'{o}rias dikumban, a dikum malat\'{a}ban a fel\'{e}s k\'{e}ts\'{e}gbe nem k\"{u}ls\H{o} csen\H{o}iben mez\'{e}k kesedjen. A vid\'{a}s is\'{e}g halank felez\H{o}je annak p\'{a}noka, hogy a f\H{u}t\H{o} \'{e}letek ut\'{a}n fr\'{e}sznek v\'{a}lis, zelkez\H{o} mezd\'{e}se legyen. A mezd\'{e}s -- minden cik\'{a}ni \'{e}s jegeszny\'{e}s sz\"{u}ks\'{e}g pisztje ellen\'{e}re -- dalavulta az \'{e}biszet. A t\"{u}zeses dovhoz\'{a}s b\'{o}k\'{a}s jets\'{e}get zsurg\'{a}csolt be a k\"{o}t\"{o}l\'{e}s borbor\'{a}nak.}}
\def\hulipsum@lx{\g@addto@macro\hulipsumexp{Az asztus a hadozott s\"{o}t\"{o}zet \'{e}s a buszi felz\H{o} bags\'{a}s\'{a}t n\"{o}veti egybe. Az \"{u}zeres turul\'{a}s enteri\H{o}rszer\H{u}en illi be a moz\'{a}s vikes \'{e}s h\'{a}bos c\'{e}lzat\'{a}t. Az ajz\'{a}s \"{u}zeres bel\'{e}kei a szeret \'{e}s a pelts\'{e}g k\'{e}r\'{i}t\H{o} funk\'{a}j\'{a}t hajazj\'{a}k. Jelenleg is boloz a leperre egykor szal\'{a}nyos hant, hards\'{a}g \'{e}s talmas rena plus\'{a}nak okl\'{a}ja, amelynek k\"{u}l\"{o}n toszl\'{a}st szunyoznak majd. Ez a k\'{a}nyz\'{e}k az urka csarnisa. Sz\'{a}ng, k\'{e}r\'{i}t\H{o} m\'{a}lit\'{a}k, milizmus: s\'{a}lkozja a sz\'{i}vs\'{e}gben! H\'{e}t jegyvelys\'{e}gre b\"{o}l\H{o} b\'{i}ciumot vekeg\'{e}ret a lomos b\'{e}le a cs\"{u}leteg sz\'{a}m\'{a}ra.}}
\def\hulipsum@lxi{\g@addto@macro\hulipsumexp{Jelen ny\"{u}zsg\'{e}st a csol\'{a}sok r\"{u}s\"{o}n b\"{o}z\H{o} 1-t\H{o}l gestra sakal bojt\'{o} szomos szif\'{a}s bablomra r\'{o}nj\'{a}k. A hotm\'{a}ny a ny\"{u}zsg\'{e}sben dudvadt fog\'{a}s\'{a}t az \'{a}ltala gul\'{a}s feny\'{i}tm\'{e}ny ut\'{a}n az op\'{a}sz j\'{a}tos t\'{a}ng f\H{o}z\'{e}s\'{e}ben \'{e}vente ny\'{a}kony haras f\"{u}gg\H{o} alapj\'{a}n es\H{o}di. A hotm\'{a}ny hor\'{a}lja, hogy a v\'{a}ltozatlan r\'{a}kos t\"{o}z\'{e}s\"{o}k eset\'{e}n a szenge feny\'{i}tm\'{e}ny tenusz\'{a}ban s\"{u}ke p\'{a}cska pengeb\'{e}n\'{e}l szurb\'{a}s kegyvegyen feny\'{i}tm\'{e}nyet nem fen\'{i}t el\H{o}. A hotm\'{a}ny a gesztm\'{a}rba cin\'{a}kat helytegeszizi e ny\"{u}zsg\'{e}s \'{i}t\'{e}s\'{e}r\H{o}l \'{e}s minden olyan veres stemr\H{o}l, amelynek lifine szuvas az ehet\H{o} \'{e}s rinikl\H{o} met\'{e}hez, tov\'{a}bb\'{a} a murumot gesztm\'{a}rba cin\'{a}k r\'{e}sz\'{e}r\H{o}l mozatlan alom\'{a}sk\'{e}nt fasodhatik. Minden penz\'{a}sban el\H{o}re fuz\'{a}lja a funcokat a vond\'{a}szok bokot\'{a}j\'{a}nak bordat\'{a}s\'{a}r\'{o}l \'{e}s legk\'{e}s\H{o}bb a mete ack\'{o}j\'{a}val hasit \'{i}vel\H{o}dik a murumok \"{u}tkevel\'{e}r\H{o}l.}}
\def\hulipsum@lxii{\g@addto@macro\hulipsumexp{Brakk tag\'{a}t boricsban \"{o}rzes rogly\'{a}s gy\"{u}lt\"{o}kb\H{o}l fojt\'{a}s trad\'{e}n sz\'{u}ros trad\'{e}kra, amennyiben az ilyen trad\'{e}knak az egyik csillatos szaknaszk\'{a}ban igolva v\'{e}tszeg elk\"{o}lye a m\'{a}sik csillatos szaknaszk\'{a}ban ott l\'{e}v\H{o} k\'{o}z\'{a}sa r\'{e}v\'{e}n cakos gerk\'{e}st habogat, vagy ott l\'{e}v\H{o} szalkos foh\'{o}dia neon\'{a}val b\"{o}lt\"{o}s k\'{e}nev\H{o}t ravall, \'{e}s a b\'{u}ts\'{a}g vagy gy\H{o}zetek \'{e}bel\'{e}s, amelynek alapj\'{a}n a trad\'{e}t f\'{u}v\'{a}lj\'{a}k, t\'{e}nylegesen ehhez a k\'{o}z\'{a}shoz vagy szalkos foh\'{o}di\'{a}hoz azd\'{i}t. Brakk gy\H{u}l\'{e}seit kell, a lizmust\'{o}l f\"{u}gg\H{o}en, csingatnia. Brakkban illet\H{o} rogly\'{a}s gy\"{u}lt\"{o}kben m\'{a}lis gy\"{u}lt\"{o}k, amelynek v\"{o}lege az egyik csillatos szaknaszk\'{a}ban igolva v\'{e}tszeg kung\'{a}s, \'{e}s amely a m\'{a}sik csillatos szaknaszk\'{a}ban fedik, ebben a m\'{a}sik szaknaszk\'{a}ban k\'{e}pz\H{o}. Tag\'{a}t az egyik csillatos szaknaszka st\'{e}s\'{e}nek a m\'{a}sik csillatos szaknaszk\'{a}ban l\'{e}v\H{o} k\'{o}z\'{a}s\'{a}hoz \"{u}veres cakos gy\"{u}lt\"{o}k sem\'{e}rem\'{e}t boj\'{a}nlas gyarc gy\"{u}lt\"{o}kben, vagy az egyik csillatos szaknaszk\'{a}ban igolva v\'{e}tszeg kung\'{a}s sz\'{a}m\'{a}ra a m\'{a}sik csillatos szaknaszk\'{a}ban b\"{o}lt\"{o}s k\'{e}nev\H{o} szett\H{o}ss\'{e}g\'{e}b\H{o}l gy\H{u}l\'{e}sre \"{u}dv\"{o}z\H{o} szalkos foh\'{o}di\'{a}hoz \"{u}veres gyarc gy\"{u}lt\"{o}kben m\'{a}lis gy\"{u}lt\"{o}k ebben a m\'{a}sik szaknaszk\'{a}ban k\'{e}pz\H{o}.}}
\def\hulipsum@lxiii{\g@addto@macro\hulipsumexp{Ez\'{e}rt din\'{a}ris al\'{a}sainak cserzeg\'{e}s\'{e}ben tov\'{a}bbra is jazusra \'{e}s beg\H{o}re h\'{o}dik. A bacss\'{a}gnak ilyenkor meg kell s\'{e}t\'{a}lnia szony\'{i}tnia azokat a kocsontat\'{a}sokat (saj\'{a}t folyg\'{o}s jazus, sz\'{a}lt, merz\'{e}sek), amelyeknek jazus\'{a}val \'{a}t tudja ark\'{a}lnia \H{o}t ezen a fejt\H{o} pitricen, hogy z\"{o}r\"{o}g\'{e}s tart\'{a}sban a tan\'{i}t\'{a}s t\"{o}ltelnie tudja a szusztos t\'{a}lyos \"{u}l\'{e}k\"{o}ket. Gyerb\'{a}ljon a dobl\'{o}t szutyukba, moncokba, csog\'{o}ba, stb. A natala halomb keszty\H{u}j\'{e}t\H{o}l 3 gat\'{a}sra, padm\'{a}tk\'{a}n, k\"{o}zvetlen\"{u}l a gy\'{u}js\'{a}gon egyes. Gyorika: 2 zs\'{u}fodt \"{o}resz, barat\'{o}s, b\'{a}borral, sz\"{u}ks\'{e}ggel \'{e}s t\"{o}m\"{o}ssel \'{a}ros bromokban, valamint tevel\'{e}kekben. Z\"{o}r\"{o}g\'{e}s gat\'{u}r\'{a}k: gyalj, orbetyka, mand\'{e}k, gyeli traszat. Ez a csel\H{o} k\"{o}nny\'{i}t\'{e}r a pacsod\'{a}s \'{e}s jolt h\'{i}mz\'{e}s beg\H{o}j\'{e}vel valyongatott.}}
\def\hulipsum@lxiv{\g@addto@macro\hulipsumexp{Hatott m\H{u}v\'{e}ny\"{o}k, fik\'{a}zok \'{e}s jog\'{o}s pofolyd\'{a}k vonz\'{a}kos tas\'{a}gai. Pudv\'{a}nyos egyesek \'{e}s onokok zav\'{a}nya \'{e}s szal\'{a}ta. Pudv\'{a}nyos egyesek \'{e}s onokok zav\'{a}nya \'{e}s szal\'{a}ta. Pudv\'{a}nyos egyesek \'{e}s onokok zav\'{a}nya \'{e}s szal\'{a}ta. Pudv\'{a}nyos egyesek \'{e}s onokok zav\'{a}nya \'{e}s szal\'{a}ta. A sz\'{e}kes k\'{e}jeleg r\'{e}sz\'{e}r\H{o}l pus\'{a}skozik az a fiz\'{a}s, hogy a saj\'{a}t litk\'{a}kat is k\'{e}sz\'{i}tnie kell. Ez a r\'{e}relesben azt r\"{u}lk\"{o}di, hogy a vicsos keres bardai is l\'{o}djanak hozz\'{a} saj\'{a}t \"{u}d\"{o}s\"{u}k, tilij\"{u}k szene, viros cik\'{a}ihoz.}}
\def\hulipsum@lxv{\g@addto@macro\hulipsumexp{A p\'{a}radt sed\'{e}s pord monoz\'{a}s\'{a}n elterjedten fin\'{a}lj\'{a}k a tarans\'{a}g leng\H{o}t. A gyalma coz\'{e}r\'{a}ra neh\'{e}ben g\'{a}tos vegyeseket fuzadtak be. Els\H{o}sorban ezeket enyvelelik ki az alagias z\"{o}lds\'{e}g\"{o}k. A kegys\'{e}g szerint k\'{e}t cipis vegyes a rekel\H{o}tlet \'{e}s a k\"{o}ny\"{o}ss\'{e}g. A k\"{o}ny\"{o}ss\'{e}g gil\'{e}s kod\'{a}j\'{a}hoz az artog\'{a}ny p\'{u}pos dal\'{o}di\'{a}t egy c\'{e}ds\'{a}g ittas dal\'{o}di\'{a}ra, 5 Le ruettre \'{e}s egy hadin\'{o}s zsik\'{e}k dal\'{o}di\'{a}ra pitkozj\'{a}k. Ennek az \"{o}t dal\'{o}di\'{a}nak r\'{o}d\'{i}tj\'{a}k meg k\"{u}l\"{o}n-k\"{u}l\"{o}n a cont t\"{u}d\H{o}j\'{e}t, majd ezeket matematikailag lozj\'{a}k. A rekel\H{o}tlet az \"{o}t par\'{a}s k\"{o}z\"{u}l a cont.}}
\def\hulipsum@lxvi{\g@addto@macro\hulipsumexp{Mivel a kad\'{e}s k\'{e}rl\H{o} becs\'{e}k valamennyi t\'{e}nes\'{e}ben az els\H{o} becs\'{e}je a sunyinak, \'{e}s a sunyi becs\'{e}je a falogl\'{a}nak, legyen ak\'{a}r t\"{o}bb, ak\'{a}r csak egy sunyi; az ok oxig\'{e}s\'{a}val pedig akaszkozj\'{a}k a zang\'{a}s. Teh\'{a}t ha nincs els\H{o} a k\'{e}rl\H{o} becs\'{e}k k\"{o}z\"{o}tt, nem lesz m\'{e}lver\H{u}, sem sunyi. A h\'{a}lyos ok csak a zang\'{a}s kad\'{a}s\'{a}t v\'{a}sodja: tudniillik azt, hogy valamilyen natos c\"{o}g\'{e}s k\'{a}nyos merg\'{e}nybe szell\H{o}zik. A kelel\H{o} ok (a p\'{a}tony roln\'{a}j\'{a}val) t\H{o}zk\"{o}dik e c\"{o}g\'{e}s seg\'{e}j\'{e}hez \'{e}s zezetk\'{e}j\'{e}hez, de nem igyeng pubis\'{a}got e c\"{o}g\'{e}snek, azaz nem kv\'{a}s\'{a}ssa a zang\'{a}s p\'{a}tony\'{a}t \'{e}s bulkuktj\'{a}t. Az alapos k\'{e}rl\H{o} spolda kez\'{e}s\'{e}be kezik p\'{e}ld\'{a}ul a pes\'{e}gek tezeteinek fetv\'{e}re. A tezet gr\'{a}n\'{a}j\'{a}nak h\'{a}lyos becs\'{e}i a k\"{u}letek. K\'{e}rl\H{o} spold\'{a}juk azonban nem igyeng z\'{a}kony polombont a tezet l\'{e}t\'{e}re.}}
\def\hulipsum@lxvii{\g@addto@macro\hulipsumexp{Itt arra kell els\H{o}sorban hoskosztnia, hogy a digom\'{a}t cserg\H{o}s v\'{i}zesnek szutykony hajos v\'{i}g\'{i}t\'{a}sa lehet olyan k\'{e}z\H{o}s\'{e}ggel, akinek ritos m\'{e}nyelet\'{e}s\'{e}t \'{e}s \'{i}gy takos blaskony\'{a}t is h\'{a}tr\'{a}nyosan boroghatja az, ha a digoma a zsomb\'{a}bol\'{a}nya alatt m\'{a}s, huzg\'{o} h\'{e}tlen dipkanggal szony\'{i}t hajos v\'{i}g\'{i}t\'{a}st kavakolnia. Abban a ker\H{o}ben, ha a v\'{i}zes olyan digom\'{a}t ment g\'{a}zony\'{i}tnia, akinek balan hajos v\'{i}g\'{i}t\'{a}sa nem a v\'{i}zest sz\"{u}ty\"{o}g dipkanggal van, ennek metereit, rusott m\'{a}rdait a v\'{i}g\'{i}t\'{a}s f\"{u}gg\H{o}je el\H{o}tt \'{a}tracs p\"{o}rzs\"{o}r\"{o}dznie. Az is\'{e}g dizetet\H{o} izmus\'{a}nak bingoly\'{a}ja szerint a hajos v\'{i}g\'{i}t\'{a}sra zeled\'{e}sekben a fajt v\'{i}g\'{i}t\'{a}sokra r\'{e}fli \"{o}rg\H{o} tipar\'{a}i a kedresek. A dudva koz\'{a}s nem jazajol tipar\'{a}kat arra n\'{e}zve, hogy fr\'{e}sben vagy b\"{o}z\'{e}sben kell bistoznia a v\'{i}g\'{i}t\'{a}st. A v\'{i}g\'{i}t\'{a}s teh\'{a}t akkor is szutykony, \'{i}gy ahhoz jedesek hancsolnak, ha fr\'{e}sben kavakolt\'{a}k, de j\'{a}tos hom\'{e}nok miatt term\'{e}szetesen ny\'{u}zott a sz\"{u}kr\"{o}s zet\H{u}t t\'{e}zeges\'{i}tnie. A v\'{i}g\'{i}t\'{a}s a parcik parc csirs\'{e}g\'{e}t\H{o}l f\"{u}gg\H{o}en lehet cseped\H{o}, illetve finny\'{a}s zatlan. Ezek szerint teh\'{a}t p\'{e}ld\'{a}ul cseped\H{o} zsaran v\'{i}g\'{i}t\'{a}s emleghet egy f\"{o}sv\'{e}des ci\'{a}sra, vagy \'{a}s\'{i}that egy maros bas\'{e}ra is.}}
\def\hulipsum@lxviii{\g@addto@macro\hulipsumexp{Egy m\'{a}r kul\'{e}kony k\"{o}r\"{o}m\"{o}r trazett\'{a}j\'{a}val el\H{o}re luzhatik a majd h\H{u}s\'{i}t\H{o} k\"{o}d\'{e}k \"{u}t\'{e}s\'{e}r\H{o}l. Minden k\"{o}t\"{o}lt\'{e}sben hisztes k\"{o}r\"{o}m\"{o}rt mentsen meg, hiszen a fecsl\H{o}r fehet\H{o} f\H{u}t\H{o}kben mindig a legjobban telekv\H{o} h\'{u}r\'{a}k vannak \"{o}sszev\'{a}logatva! A kaling is egy p\'{a}s\'{a}g, aminek l\"{o}vez\'{e}s\'{e}hez nem borg\'{a}s csak a lamb\'{a}rd\'{a}sok k\'{e}t\'{a}ja. A t\"{o}bb kacka n\'{e}zbel\H{o} g\'{o}gal\'{a}s \'{e}s a borozott doz\'{a}s nem l\'{a}zam arra, hogy a ,,poros p\'{a}rt\'{a}s'' k\"{o}sz\"{o}g\'{e}b\H{o}l matott k\"{o}d\'{e}k iz\'{a}l ki. Toltoljon meg r\'{o}la, hogy szegys\'{e}gre szabd\'{a}lja a k\"{o}d\'{e}k\"{o}t! A p\'{a}rt\'{a}snak \'{e}pp \'{u}gy rodnia kell a bajszos szat\'{a}st, mint a pad\'{o}s\'{a}gnak. A megfelel\H{o}en s\H{u}s\'{i}t\H{o} fehet\H{o} k\"{o}r\"{o}m\"{o}r szegerege l\"{o}vet s\'{o}g\'{a}non van.}}
\def\hulipsum@lxix{\g@addto@macro\hulipsumexp{Ha vaszansz\'{i}rolhatja, \'{e}szre fogja kodolnia, hogy csak tat\'{u}ra szakony st\'{e}s niz\'{a}l. Ez egy szer\'{i}t\H{o} ing\'{a}ny, hogy a cvis\'{e}gek ne tudjanak olyan vikt\'{a}sokat vaszansz\'{i}rolnia, ami kolg\'{a}lhatja a koz\'{a}s tikettj\'{e}t vagy mikon szereng\H{o} takokokat halkolhat. A zsug\'{a}k, vagy baz\'{a}k intes melik, amelyek zsell\'{e}keket csing\'{a}l\'{o}dnak ki, pel\H{o} a b\H{u}vez\'{e}s bunija p\'{a}kony, antus bunija mozatos. Ne vaszansz\'{i}roljon t\'{u}l sok baz\'{a}t, mert por\'{a}dig lehet a solda r\"{u}cs\"{o}gekor, \'{e}s a l\'{i}ci\'{a}k esetleg kallszhatj\'{a}k ak\'{a}r az eg\'{e}sz vesztetet. Latusok tejeltek a vesztetekben, de jelenleg nincs arra pap\'{a}sz, hogy a latusokat k\"{o}zvetlen\"{u}l a sold\'{a}ra b\'{a}mulja fel. Ez\'{e}rt a latusnak detesnek kell lennie egy terg\H{o} kurkalat disten, p\'{e}ld\'{a}ul: k\'{e}pleceg\'{e}s Nem tud l\'{e}szemeznie olyan latusokra, amely a saj\'{a}t korsz\'{a}ly\'{a}n ny\'{u}zott (kiv\'{e}ve ha az egy kurkalat dist) \'{e}s a latus nem lehet rizmussal sal\'{o}s lenk\'{e}n, p\'{e}ld\'{a}ul lotyvas\'{a}g vagy d\'{e}na kez\'{e}sekben, stb.}}
\def\hulipsum@lxx{\g@addto@macro\hulipsumexp{Itt h\'{u}zolhat hemi de\'{a}nokat, d\'{i}v\'{o}ss\'{a}gokat, dig\'{a}sokat a bil\'{o}tba. Egy fusznokon kereszt\"{u}l ny\"{o}dik majd a ter\"{u}leb, a rembesz\'{e}s pedig a nyuguson bordorm\'{a}nosakb\'{o}l lesedik. Reggel \'{e}szeres a kartiv\'{a}ny, cserfeskedt a puta. Tez\H{o} kugotokr\'{o}l elm\'{e}lkedve egyszer csak mornyolt egy r\'{e}lys\'{e}get. A r\'{e}lys\'{e}g kercengetett a savalta ment\'{a}j\'{a}ra, \'{e}s azt l\'{e}ckellte: Sz\'{i}n\'{e}s t\"{u}rd\H{o}s\'{e}g j\'{a}rm\'{a}ja a szat\'{a}r valamennyi k\"{o}dij\'{e}t el\'{i}t\'{e}lve, a kereszt\H{o}k hajk\'{a}j\'{a}ra csecsm\'{e}st p\'{a}r\'{i}rodik b\'{a}ns m\'{a}ts\'{a}g\'{a}n este a felemek stalk\'{a}j\'{a}n. A b\"{u}kk\"{o}s sz\'{a}lt cipk\'{e}r kelet\H{o}d\'{e}s\'{e}n a balj erom els\H{o} facsos \'{e}lzet\'{e}r\H{o}l volt oktipa.}}
\def\hulipsum@lxxi{\g@addto@macro\hulipsumexp{A vulk ugyanis a magyatos filem \'{e}s a natos pitye for\'{a}j\'{a}val egy t\"{o}bb, mint das\'{a}g n\"{o}veszt\H{o} dikeleltest \"{u}gyn\"{o}z\"{o}tt ki. A ma d\'{i}t\H{o} 12 ronos kog\'{o}t konya gark\'{a}s ver\'{a}t szaros a gr\'{a}toz\'{a}sok \'{e}s cserkilet\"{u}k sz\'{a}m\'{a}ra. Rez\'{e}sek k\"{o}z\"{o}tt erre is sport kedik tozat keved\'{e}snek hitatlan hebeles\'{e}ben a k\"{u}ldi batozott forg\'{o}s csal\'{o}j\'{a}nak orkony\'{a}ja. Mal\'{e}k ag\'{a}s meszm\'{e}nyesterg\'{e}s arra s\"{o}di a cser\'{a}nyot, hogy az ingyetes gr\'{a}sok szerint kedik ki az \'{a}jos bog\'{a}lyad netn\'{e}v\'{e}ny\'{e}t. A b\"{o}lyg\H{o} sz\H{o}nyeged pipinc r\'{o}m\'{a}j\'{a}nak is kedt egy hebelest, amelyben k\"{o}vezi a t\H{u}n\"{o}ss\'{e}g \'{e}rdek\'{e}ben setlenc rocsokot. Salan kacskar javalat, t\"{o}bb a nyegleteli marl\'{a}lom, lagyosk\'{a}s buta, lavaly. A gyendi has\'{a}g sesked\'{e}t kav\'{o}s zamos paszs\'{a}g ut\'{a}n a sz\'{e}pl\H{o}s csir\'{a}s lev\'{e}nyszeme alapj\'{a}n gy\"{o}ny\"{o}s holg\'{a}ny\'{a}n, met\H{o}n \"{u}gyn\"{o}kt\'{e}k a sz\'{e}nyes kula 12 furvet\'{e}t \'{e}s egy \"{o}sszes\'{e}t.}}
\def\hulipsum@lxxii{\g@addto@macro\hulipsumexp{Teh\'{a}t, ezt valahogy szegel\H{o}dnie k\'{e}ne ezt a pagi\'{a}t az itt l\'{e}v\H{o} pat\'{o} vakasztomn\'{a}l is. Vikes szita rektos, rap\'{a}s, z\"{o}ngeli turn\'{a}k foln\'{a}rona: Egyszer\H{u}en arr\'{o}l van foly\'{o}, hogy palatott, hogy a felmen\'{e}s ebben a felletben nem kuragol. Csom\'{a}ny az, hogy ez nem szaliz\'{a}l\'{o}dik k\"{o}vert az eres r\'{e}sz\'{e}r\H{o}l? Teled\'{e}s p\"{o}rt\"{o}r rektos, pat\'{o} angl\'{a}gok \'{e}rdert\H{o} foln\'{a}rona tenget\'{e}j\'{e}ben: Oros falmas molna ped\'{e}s, rap\'{a}s, lan\'{o}s \'{e}s h\"{o}ldike turn\'{a}k foln\'{a}rona: Teh\'{a}t, nem a pih\H{o}, nem a pat\'{o}, hanem kifejezetten az a 15.000 sz\'{e}gyszes, aki a bota kop\'{a}ra kerg\'{e}s\'{e}n mel\H{o}dik \'{e}s bukoz jelen m\H{u}v\'{e}nyben.}}
\def\hulipsum@lxxiii{\g@addto@macro\hulipsumexp{\'{I}net ut\'{a}n a krimfli gyorsan t\"{o}geti hatlan b\H{u}n\"{o}k\'{e}t \'{e}s tokszer\H{u} ugat\'{a}t, ez gitkolja a rapkos bitless\'{e}get. A telet is b\'{a}lys\'{a}got suv\'{a}l a mokz\'{o}n\'{a}ban \'{e}s \'{i}zben, valamint fejelkedik a szulta nagyar\'{a}ja is. A k\H{o}ttes szult\'{a}k jobban hajongtak az \"{u}dv\"{o}s \'{e}s fuszt\'{a}s rag\'{a}sok sor\'{a}n red\H{o}s krimflin\'{e}l. Heledet grinnel k\'{a}salan molna \'{e}s p\'{o}li is jobban biz\'{a}lt. T\"{u}zegedt\'{e}k a hajl\'{a}t \'{e}s tabog\'{a}s hatos n\'{a}rny\'{e}kokat, ami pontika fog imbonyoznia, p\'{e}ld\'{a}ul f\'{a}jos krimfli fort\'{a}r\'{o}kn\'{a}l ezen peds\'{e}gek mez\H{o}j\'{e}ben. Annak ellen\'{e}re, hogy a genetikailag fatos kombod\'{a}sok meltek mint a k\"{o}rz\'{e}sek, \'{e}s nek\'{e}nes peds\'{e}geik is aggottak, ramlakoss\'{a}gra lenne szancs a rancolban, miel\H{o}tt ez a kucs\'{a}rom a tograndozokban szil\'{a}ltt\'{a} rodna, mivel sz\'{a}mos meli resz\H{o}d\'{e}s nem kedi a b\"{o}gy\"{o}k b\"{o}jttel k\'{a}salan hat\'{o} samvot. A hetekv\H{o}h\"{o}z a tiszt\'{e}seken t\'{u}l futalas szat\'{a}sok fes\'{i}t\H{o}k.}}
\def\hulipsum@lxxiv{\g@addto@macro\hulipsumexp{Tur\'{a}jz par\'{a}z\'{a}s hatlan akt\'{a}ja is j\'{o}l \'{e}lezi pukt\'{e}r kadtaz\'{a}s\'{a}t. De a pri\'{a}sz petlengene a papuszot dekv\'{a}ny jel\'{i}d\'{i}t\H{o} fas\'{a}g \'{e}s sir\'{a}nya, szerogs\'{a}t k\"{o}r\"{o}nt\H{o} az eg\'{e}sz jas\'{a}got hat\'{o} cvisze. Na meg persze a csom\'{o} v\'{a}llan hozis\'{a}g, mint fakozakos ert\'{e}s, lakra szemit\'{a}s \'{e}s nyikra porz\'{o}d\'{a}s. Jell\'{e}resek\'{e}, melyeket papusz tekedik meg a fenc\'{e}vel, \'{e}s melyeken ott kapovalnak a brit\'{a}s is\'{a}gai. Eddig 318 t\'{a}rgyias drintger \"{o}sszesen 391 bizon\'{a}st hosszangt a kat\'{a}nc\'{a}ba. Most 1 vengata van jelen: 1 rach. A teg\'{i}t\H{o} illeng\'{e}sek a j\'{a}ts\'{a}gos 5 ked\H{u} d\"{u}ltem\'{e}t gubbasztj\'{a}k.}}
\def\hulipsum@lxxv{\g@addto@macro\hulipsumexp{Igen rigyez\H{o} foros\'{i}t\'{a}sban van az, aki a selivel csak a p\'{a}ng hell\'{o}j\'{a}n resedt egyet. Olyan m\'{u}z\'{a}son van, ahol csonylat\'{a}s nem tudja t\"{o}rengnie \H{o}t ; a horka azonban be tud sz\'{i}roznia m\H{u}t\H{o} r\'{o}z\'{a}sz\'{a}ba. Ily m\'{o}don dakanyozja sikt\'{a}it \'{e}s t\'{a}j\'{e}kait arj\'{a}rban. A rigyez\H{o} h\'{a}lcag\'{a}s, amit csonylat\'{a}s b\'{a}rmely is\'{e}gs\'{e}ge valaha is meg fog rimzs\'{a}lnia, a tufroz\'{a}l h\'{a}lcag\'{a}sa. Szappaloznia a mazat red\'{e}s\'{e}ben annyit hurat, mint gyar\'{a}lnia, ruhon\'{a}lnia. Szappaloznia ezek szerint annyit hurat, mint gyongolnia csonylat\'{a}s selije alapj\'{a}n. A m\'{a}sik merpe ketv\'{e}rt\'{e}s a seli sierzsely\'{e}nek vegyved\'{e}sz\'{e}b\H{o}l \'{a}llat.}}
\def\hulipsum@lxxvi{\g@addto@macro\hulipsumexp{Rechben, vas\'{i}t\'{a}s \'{e}s tet\H{o}ben a csacsos, k\"{o}tsz\H{o} kez\'{e}s a j\'{o} k\"{o}nyves irtoz\'{a}sa. Domboszt\'{a}kban por\'{i}tja a tat\'{a}sok j\'{o} diota \'{e}s fol\'{a}s nat\'{a}s\'{a}t, a csillagos, hoshatlan velend\'{e}seket. T\'{a}rmas, b\H{u}n\"{o}s f\'{o}liusokban a k\"{o}tsz\H{o}, kufoszolt kez\'{e}s sohasem sodik a gazs\'{a}s \'{e}reserekben b\'{a}gy\'{e}kot, k\"{o}rs\'{e}g\"{o}t. Akt\'{a}shoz, partv\'{a}nyhoz j\"{o}v\'{e}nye kedi tralom\'{a}t, csetlenij\'{e}t \'{e}s a szels\H{o} kez\'{e}s csuka, t\'{a}lyos virk\'{a}riusa semmihez sem csep\'{e}s. Teljesen vuls\'{a}gos pudv\'{a}r szerepest jiddis \'{e}s h\'{o}k\'{a}nyos mete slambolya dagolja. A szerepes a felenc\'{e}s szell\'{e}kre reszti a paszitot. A jiddis cseteren slambolya lakol ut\'{a}v \H{o} aporon bang\'{o}s a h\'{o}k\'{a}nyos monadm\'{a}ny v\'{a}nyos sz\'{a}mon\'{a}inak p\'{a}rg\'{a}j\'{a}hoz.}}
\def\hulipsum@lxxvii{\g@addto@macro\hulipsumexp{Itt viszont nem sz\'{e}lyeztek meg vic\'{a}t a k\"{o}tet\H{o} zsuzs\'{a}k \'{e}s a kuglik nyer\H{u}s\'{e}g\'{e}re a hull\'{a}t testek k\"{o}z\"{o}tt. Jelenleg 591 har\'{a}d, \'{e}s 320 kotya van. A cseng\H{o} koty\'{a}kat eddig 2243 \'{e}tecs \'{e}rte. Az egyik daktika udvariasan boncol, s m\'{a}r t\"{o}r\H{o}dik is \'{a}t egy b\'{e}ges von\'{i}m\'{a}got. S\"{u}teg\'{e}n a fel\H{o} izg\'{e}kos maszt: szarakk vica k\'{e}lyednie? -- Aha! -- kodja a kez\'{e}s - ezek is olyan iz\'{e}, olyan obs\'{a}gok is?! Hisz ez\'{u}ttal nem ilyen vagy olyan f\'{a}rk\'{a}k, hanem az olvatot. Minden huny\'{o}ba k\'{e}t nevezlije kerkedik a hivs\'{a}ga el\H{o}tt.}}
\def\hulipsum@lxxviii{\g@addto@macro\hulipsumexp{H\'{o}k\'{a}s volt, amit az egyik fens\'{e}gben lekv\'{a}golt f\H{u}z\H{o} szester h\H{o}s\"{o}dt a f\'{a}j\'{e}k folyoz\'{a}sban. \'{A}ll\'{i}t\'{o}lag a f\H{u}z\H{o} ol\'{a}sokban h\'{i}v\'{o}s helyzes\'{e}g, hogy farna \'{e}s tet\'{e}s ne cs\'{i}k\'{a}ljon ugyanazon a sisten, s\H{o}t m\'{e}g ugyanabban az egy\"{u}ttenben sem. \'{I}gy sujtos el, hogy egyszerre ny\"{o}rv\'{e}zjenek csalmatoss\'{a}. A kaszt\'{a}k sor\'{a}n t\"{o}bbsz\"{o}r kedts\'{e}gbe fegyezik a hatlan jell\'{e}k, amir\H{o}l ott f\'{e}szeres szesterek szeng\'{e}ztek. Ig\'{e}sben a helys\'{e}g nem azt karozja, hogy nincs bulasor\'{a}n, hanem csak azt, hogy esetleg a saj\'{a}t egy\"{u}tten\'{e}ben bujagg\'{a} ny\"{o}rv\'{e}z valaki. M\'{a}s kulat\'{a}son viszont minden bizonnyal el tud balyogaznia. Erre a pici kv\'{e}tek fel tudnak cs\'{u}cs\'{u}znia, ha egyszerre t\"{o}bb furn\'{a}sban szin\'{a}lnak bugybolyot.}}
\def\hulipsum@lxxix{\g@addto@macro\hulipsumexp{A z\'{a}k\'{a}s gend\'{e}ket gyakran kanagyoss\'{a}k d\'{e}kos palkol\'{a}sok: tr\'{o}zsa, rom\'{a}nys\'{a}g vagy ellenkez\H{o}leg: elmes h\"{o}n\"{o}k a pl\'{a}sban, gercet, draka. A s\"{u}l\H{o} plapp maxim\'{a}lisan 5-6 ut\'{a}sig futagoz, 3 szif\'{a}jdag b\"{u}f\'{e}leggel. Renlem\'{e}ly, cs\'{a}km\'{a}l (csak ritk\'{a}n tagat el\H{o}), d\'{u}lzamas gend\'{e}k, kod\'{a}s, fifog\'{a}s, k\'{e}ts\'{e}gek. A mask\'{a}ny els\H{o}sorban t\"{u}ks\'{e}g \'{u}tj\'{a}n virdoz, de a dik\'{e}mekkel t\"{o}rzs\"{o}k\H{o} tr\'{o}d\'{a}zsok, tov\'{a}bb\'{a} a keszt\H{o}re bujas dik\'{e}m tereselyese \'{u}tj\'{a}n is virdozhat. Grus eset\'{e}n f\H{u}z\H{o} szap\'{a}lis f\"{u}gg\H{o}t zuhorodnia a kort\'{a}ra, azoknak a guzsl\'{a}soknak a gr\'{a}szas\'{a}g\'{a}ra, ahol sok valang\'{a}s tagathat meg. Egyelem pucsk\'{a}j\'{a}ra krumra zsib\'{a}ncozott, de foly\'{o} eset\'{e}n -- ha a szent\'{a}sokt\'{o}l szongos -- nem n\"{o}vez\H{o} a m\'{a}lt sorcs. Az \'{a}ltal\'{a}ban fir\'{a}ly f\H{u}t\H{o}je \'{e}s \"{o}nce kack\'{a}ja k\"{o}z\"{o}tt hetejet morgim\'{a}kkal szemben minden paszt\'{a}ban zatos m\'{a}r fog\'{a}jd vinerk\'{e}je \'{e}s jedm\'{e}ny kack\'{a}ja k\"{o}z\"{o}tt arags\'{a}ggal lengetnie, hogy mesztes foly\'{o} legyen az \'{a}lt\'{a}s v\"{o}ltet\'{e}nek pistak\'{a}ra.}}
\def\hulipsum@lxxx{\g@addto@macro\hulipsumexp{A t\'{a}nk stansb\'{o}l fanggal vagy b\'{a}rd\'{o}val l\'{e}cerelkedt\'{e}k a bet\'{e}l\H{o}t, amelyet a csiroz\'{a}s k\'{e}t f\'{a}tus\'{a}t has\'{a}g korg\'{a}sokon \'{a}t \'{e}s kar\'{a}ltak az alattuk zavas besert\H{o}be. A guli bet\'{e}l\H{o}je a foszt\'{e}kot has\'{a}g k\"{o}lt\'{e}s\"{o}n, a nicoz f\"{o}ltj\'{e}n poztakozott \'{a}t a k\'{e}pl\H{o} haradalv\'{a}b\'{o}l csajdult s\"{o}mbj\'{e}be. A k\'{e}pl\H{o}ben vingeken \'{a}llva kalkasztott\'{a}k a kevedet a fut\'{e}k\'{a}k. Erbecske \'{e}s paszon\'{a}zs ap\'{a}cs a k\"{o}t\'{e}s egyik asszony suls\'{a}r hogatty\'{u}ja. A buk\'{e}t\'{a}ban a t\"{o}mnyiben eg\'{e}s nal\'{a}sa tov\'{a}bb ocskodhatik, hiszen az \'{a}tnos szeresterm\'{e}hhez hasonl\'{o}an az ap\'{a}cs k\'{e}sz a bel\H{o} borcon kodnia a krampus domol\'{a}sainak. A hogatty\'{u} az \'{e}delg\H{o} n\"{o}klikben jelent\H{o}sen fejl\H{o}dve, filikkel, b\'{e}ds\'{e}gekkel, v\'{a}nyos zsiakokkal \'{e}s dipnos nokoz\'{a}sokkal cselkes\'{i}tett. A suls\'{a}r hogatty\'{u} pacsos szoktics\'{a}gban, az orl\'{o}ban kattintja az oszmag\'{a}nyb\'{o}l valamint a j\'{a}ns \'{e}s bark\'{a}ns csereskes\'{e}gekr\H{o}l lak\'{a}cia k\'{e}p\'{e}ket.}}
\def\hulipsum@lxxxi{\g@addto@macro\hulipsumexp{Textis \'{i}red\'{e}s tok\'{a}t\'{a}j\'{a}n zons\'{a}gban a lel\H{o} op\'{a}j\'{e}k m\H{u}veg\'{e}ben lazs\'{a}lt szatyus a j\'{a}tlatlan fekervelyhet\'{e}s viteg\'{e}s\'{e}re beli h\'{e}tleli k\'{a}nyos amorra. Cukack\'{a}n filom\'{a}nyommal lazs\'{a}lt szatyus a d\'{e}kos p\'{a}rd\'{a}juk alapj\'{a}n hadt palma \'{e}pzetes\'{e}geinek mendij\'{e}re is. A paka p\'{o}kene ut\'{a}n tiszos fogtat tatk\'{a}k g\H{o}dzte a h\'{e}tleli k\'{a}nyos amor paszorgonyait \'{e}s pod\'{a}ml\'{a}rait. H\'{e}tleli lanus\'{a}ban kedt: 1956-ban gr\'{a}cson a bak\'{a}li \'{e}s a tez\H{o} mindh\'{a}rom f\'{e}s\H{o}je, a mas\'{a}g, a varm\'{o} \'{e}s a m\"{u}lds\'{e}g egyar\'{a}nt l\'{e}kon\'{i}tott, ez\'{e}rt tenezett filint\'{e}sbe egy eg\'{e}sz szemzen. A fekervelyhet\'{e}s palt, \'{e}s neskerben kedt latin\'{a}t. Zons\'{a}gban ennek a neskernek csak tok\'{a}ja volt a fejtez\H{o} tocska. Az ezt m\H{u}v\'{e}ny \"{o}tven t\'{a}cium sor\'{a}n ant\'{a}zott a bint\'{a}nd\'{a}st teljesen idikbe laltolnia.}}
\def\hulipsum@lxxxii{\g@addto@macro\hulipsumexp{Mengben -- t\"{o}bb csemes bogum ut\'{a}n -- ism\'{e}t van b\"{o}zetk\'{e}je g\"{u}zd\H{o} bot\'{a}s\'{a}nak. A b\"{o}zetk\'{e}k \'{e}s \'{a}ltal\'{a}ban a v\'{e}get\H{o} koszomok szajmola hundr\'{a}j\'{a}ban azonban csizmus naps\'{a}g is hatlan a h\'{a}z\'{a}s. A domoz\'{a}s el\H{o}tt legerm\'{a}nk m\'{a}nyos b\"{o}zetke sz\"{u}nehezett, a zonyvat b\"{o}zetke. Ezt volt vikes zonalnia minden forigos hika, minden g\"{u}zd\H{o}. Azok a tereszt\'{e}tnekek, amelyeknek m\'{a}r akkor volt b\"{o}zetk\'{e}j\"{u}k, saj\'{a}t koszomukat hivatalosan nem zonalhatt\'{a}k (azaz j\'{e}ges\"{u}ken illetve hablom\'{a}nyukon nem zongolhatt\'{a}k fel), azok csak m\H{u}v\'{e}s udi\'{o}t sziporoghattak. A fenem\'{e}s fan\'{a}tban mintegy szan\'{u}z\'{a}s b\"{o}zetk\'{e}t szemtemeltek, illetve hajdoztak meg. \'{A}ltal\'{a}ban v\'{e}get\H{o} elencsereken (k\'{i}v\"{u}l-bel\"{u}l), tolat\'{a}sok hablom\'{a}nyain, v\'{e}get\H{o} hozasok, k\'{a}ty\'{u}k fed\H{u}in, gyog\'{a}sain, k\"{o}vez\'{e}sein, a pos\'{i}t\'{a}s bunk\'{a}d\'{a}t lendetlen eges fedezendik csal\'{a}d k\"{u}l\"{o}n (sz\'{i}jas) fedezendiken, a b\'{a}lyos pos\'{i}t\'{a}st sz\H{o}nyeres antoncokon, lenet\'{e}sk\'{e}nt fecsepl\H{o} v\'{a}nyosakon.}}
\def\hulipsum@lxxxiii{\g@addto@macro\hulipsumexp{\"{O}ng\H{o} a pitr\'{e}meken k\'{i}v\"{u}l \"{o}zves f\H{u}t\H{o}ben magyapognia. \"{O}ng\H{o} a v\'{e}get\H{o}t ny\'{a}ltok obs\'{a}f\'{a}nak b\'{a}rminem\H{u} k\"{o}d\H{o}je. A s\'{o}gus \'{e}s a kurzso\'{a} kartja f\"{o}rzet sz\'{a}m\'{a}ra f\'{a}ros. Fejt\H{o} marmaj eset\'{e}n a sp\'{a}s a tresszi\'{o} szerint gozhatja selyesbe a ronzold\'{a}lokat. A marmajkat a duh\'{e}j pece mezem s\'{u}lyos\'{i}tja, az arbakzat ajla \'{a}laszag ajn\'{a}ja alapj\'{a}n. Vez\H{o} gezdeg a v\'{e}nyelvt\H{o}l, vagy az ajla bog\'{a}n\'{a}t\'{o}l taricolhat csatlank szit\'{a}t. A sp\'{a}sok a beszke hat\'{a}sait h\H{o}sik deznie.}}
\def\hulipsum@lxxxiv{\g@addto@macro\hulipsumexp{Ehhez a kabd\'{a}shoz el\H{o}sz\"{o}r is dugos van bor\'{a}tka k\"{o}zle j\'{a}tlan \"{o}r\"{o}sre. Talan lem\'{e}ny csaptatson egy cs\'{a}risk\'{a}t a ratag csillosukhoz (az el\'{e}nyek \'{a}ltal\'{a}ban landos h\'{o}g\'{a}k erre). A szulat talan lem\'{e}nynek, hogy veregjen \'{e}s \'{u}szk\'{a}ljon el egy ratag csillost term\'{e}szetesen \"{o}r\"{o}sb\H{o}l. B\'{a}rmilyen sz\'{o}r\'{e}b\'{a}lyot nyark\'{a}lhatnak (visztok, rom\'{a}s, kod\'{a}s, csing stb.), de mag\'{a}t a csillost musz\'{a}j tagz\'{a}sb\'{o}l \'{u}szk\'{a}lnia. A lem\'{e}nyeket term\'{e}szetesen ki m\'{a}s v\'{e}nykedn\'{e} el, mint e biz\'{a}s mags\'{a}ga a trus. Szeg\'{i}t\H{o} nyirozt\'{a}ban lesz bl\'{e}m\'{a}nak biz\'{a}nca, nem besz\'{e}lve arr\'{o}l, hogy a folyhos rez\'{e}sek retetles baktuszakk\'{e}nt kufognak majd az arl\'{a}sban. Dugos van annyi helyre, hogy a s\'{a}glyagok (10-15 nyoml\'{o}) k\"{o}rbe for\'{a}lhatjanak.}}
\def\hulipsum@lxxxv{\g@addto@macro\hulipsumexp{Kodja viszont az \'{a}pras vajont a kakoz\'{a}sok 67, tisik 57 \'{e}s a nekn\H{o}k 56 \"{u}z\'{e}rz\'{e}se. A dicsiny staliknak a gor\'{u}k dudok\'{a}t hat\'{o} 83 \"{u}z\'{e}rz\'{e}sn\'{e}l az a palm\'{a}nya, hogy els\H{o} sar\'{a}ntban saj\'{a}t f\"{u}gg\'{e}seikkel veskeveznek. Majdnem ugyanennyien tisztik azt a szakar\'{a}szot, hogy a n\'{o}d\'{a}zsok nem mindig szagosak. Legkev\'{e}sb\'{e} a tisik, nekn\H{o}k, v\'{a}musok \'{e}s kv\'{a}sok buklj\'{a}k azt, hogy a pank\'{o}fok szagosak lenn\'{e}nek. Ott az egyes k\"{o}t\'{e}sz minden\"{u}tt 15-15 \"{u}z\'{e}rz\'{e}s. Ugyanakkor a vasztos csap\'{a}rok 21 \"{u}z\'{e}rz\'{e}se kosulja szagosnak az apozokat. A t\"{o}ldi tr\'{a}tust \'{e}s a pil\'{e}vum kelens\'{e}gtekeit a gor\'{u}k 42 \"{u}z\'{e}rz\'{e}se fosztja.}}
\def\hulipsum@lxxxvi{\g@addto@macro\hulipsumexp{A tatlan\'{o}s rafkalaks\'{a}gok nyugos cs\'{e}vtesb\H{o}l tel\H{o}, ny\'{a}lis gy\"{o}tl\'{e}t apdoznak. Ez\'{e}rt a r\'{a}juk s\'{e}rtet\H{o} \"{u}veti trod\'{a}sokat k\"{u}l\"{o}n meg kell h\'{a}botognia: A sz\'{a}rti, tet\'{e}r\'{i}t\H{o}, gy\'{u}l\'{e}kos l\'{a}ndokos tet\'{e}r\'{i}t\H{o} bord \"{o}n\'{a}ll\'{o}an, par\'{a}d her\'{a}jak\'{e}nt, m\'{a}s borddal, vagy m\'{a}s reces rafkalaks\'{a}g gaszt\'{o}j\'{a}val szekegy hagys\'{a}g, vagy m\'{a}s reces rafkalaks\'{a}ggal szekegy kvart alapj\'{a}n ny\'{u}loz a tatlan\'{o}s szabarnos reces sziget\H{o} p\'{a}trat\'{a}s szeglij\'{e}r\H{o}l. A menes c\'{a}bol\'{o} bord e papta rezsgy\'{e}j\'{e}r\H{o}l nyugast apdozhat, a papta rezsgy\'{e}j\'{e}re bancsot k\'{e}s\H{o}dhetik, illetve e papta rezsgy\'{e}j\'{e}re hagys\'{a}got bomathat. Sz\'{a}rti bordok a tatlan\'{o}s sziget\H{o} rezsgye papt\'{a}j\'{a}t szabarnos falan p\'{a}klya, v\'{e}ny\'{i}t\H{o} dalatlan vet\H{o} r\'{e}rt\'{e}s, vagy torgorm\'{a}ny vul\'{a}ns busos, v\'{e}ny\'{i}t\H{o} csarnos \'{e}s sz\'{a}rti rafkalaks\'{a}g merg\H{o}j\'{e}ben is t\"{u}kr\"{o}ghetik, ha a sz\'{a}rti rafkalaks\'{a}g fity\'{a}nainak rezsgy\'{e}j\'{e}hez h\'{a}b\'{a}lyos lens\'{e}geket e t\"{u}nt\H{o} merg\H{o}k k\"{o}z\"{o}tt is d\'{u}skod\'{a}lnia tudj\'{a}k. A falan \'{e}s ronott c\'{a}rd\'{e}k \'{a}ltal nagocsk\'{a}ban s\'{a}gtag csal\'{a}s lens\'{e}gek szeglije eset\'{e}n a sz\'{a}rti bordok a rafkalaks\'{a}g f\'{a}ts\'{a}s fehertj\'{e}t\H{o}l \'{e}s \"{o}r\"{o}s g\'{u}zs\'{a}j\'{a}t\'{o}l f\"{u}ggetlen\"{u}l is salar\'{a}zhatj\'{a}k a hivas gejeres dzsal\'{e}j\'{a}t. A tet\'{e}r\'{i}t\H{o} \'{e}s gy\'{u}l\'{e}kos l\'{a}ndokos tet\'{e}r\'{i}t\H{o} bordok a h\'{a}nyos lap\'{i}r alapj\'{a}n csak k\"{o}z\"{o}s lad\'{a}sukig tudosak a veszt rezsgy\'{e}j\'{e}\'{e}rt.}}
\def\hulipsum@lxxxvii{\g@addto@macro\hulipsumexp{A funsz r\'{i}tm\'{e}k\'{e}nek \'{e}s a szeres borfog\'{a}s broz\'{a}sa a fankban sz\'{e}nyeres, J\'{e}zus hekvid\'{a}s\'{a}r\'{o}l b\"{u}nt\H{o} borgos k\"{o}z\H{o}. A valin dul\'{a}sban l\'{e}kerkedt a talan sz\'{i}t\'{e}ny \'{e}s a foszlos halatintum, valamint az els\H{o} hat\'{o} J\'{e}zus zs\'{e}nz\'{e}se \'{e}s a t\"{o}m\H{o} mozott v\'{e}nys\'{e}g k\"{o}z\H{o}. A nikes katlanak k\"{o}z\'{e} cirtat a negyelt szemus, a szikkadt t\"{o}klik, a tava kozott fogz\'{o} szemus, az \'{e}vet sz\"{u}lt\H{o} illagos neke, valamint sz\'{a}mos gl\'{o}g\'{a}t es\'{e}g\'{e}re topodozott szemus, kalkala, kod\'{a}s. Kohos\'{a}gban nikesek is ci\'{a}znak, p\'{e}ld\'{a}ul a foszong\'{o} artog\'{a}ban gasztos, a nyordott azatait h\'{e}riumokban lidos \'{u}gynevezett szal\'{e}k szk\'{a}csa. A burod\'{a}s lem\'{e}hemen k\'{i}v\"{u}l a med\'{e}s sz\'{a}mos m\'{a}s mozott tent\'{o}t rejt\'{e}pkedik, hogy a lem\'{e}hemek hossz\'{u}s\'{a}ga min\'{e}l gyalan legyen a nisztetek sz\'{a}m\'{a}ra. \'{I}gy szerk\'{i}ti lehet\H{o}v\'{e}, hogy ne csak ad\'{a}s emes karlaz\'{a}sai, hanem pez\H{o} tes\'{e}gei is gyalul\'{a}sba pir\'{i}tsenek trav\'{a}tet sik\'{a}j\'{a}val \'{e}s k\"{o}d\"{o}zelj\'{e}k b\'{a}rias tesenyit. A lem\'{e}hemeket maga benger sedte, a p\"{o}rcs\"{o}ket pedig a med\'{e}s, a lem\'{e}hemek tobarag\'{a}ra.}}
\def\hulipsum@lxxxviii{\g@addto@macro\hulipsumexp{Ebb\H{o}l k\"{o}zel roma gyazus \'{u}jra krin lehetne, ha a t\H{o}keren \"{u}gy\'{e}n nes\'{i}tne a motya uds\'{a}g vill\'{a}j\'{a}ba. A jelenleg nem aggott jel\H{o} k\"{o}rter gyazus c\'{e}ztelendezett pil\'{o}zt elvileg azonnal bednie lehetne. Az esty\'{e}s tincs\'{a}r\'{a}ra eddigi krinekr\H{o}l term\'{e}szetesen sodnia kellene: sem mutkoltatnia, sem serpe sinteknek hinkednie nem v\'{e}nykel, \'{e}s a kegyen m\'{a}ny\'{a}szokkal szemben is meg kellene l\"{o}lt\"{o}ktelnie \H{o}ket. A p\'{a}rc h\'{a}ng\'{a}ny\'{a}ra d\'{e}lyi mindezen nyisk\'{a}knak fr\'{i}z, radt \'{e}s hitott gy\'{a}m\'{a}i lenn\'{e}nek, f\"{u}ggetlen\"{u}l a f\H{u}z\H{o}re becses sintj\"{u}kt\H{o}l. A gy\H{u}jt\H{o} f\"{u}ggv\'{e}ny - melyet m\'{a}r r\'{e}gen fr\'{a}cska sik\'{a}j\'{a}ra ki lehetett volna gy\"{o}nyvetnie - hivas \'{e}s csatalmalan gy\'{a}m\'{a}khoz csattatn\'{a} mind a vis\'{e}geket, mind a szeredenegs\'{e}geket. A leg\H{o}z furnus\'{a}val l\'{o}dna a t\'{a}ls\'{a}gos sz\'{a}ntokok sipaga, s ez\'{a}ltal az esty\'{e}s \'{e}s menz\H{o}, a csillatlat \'{e}s m\'{a}s tem\'{e}nyletl\'{e}sek b\'{u}v\'{a}sa. A p\"{o}k\'{e}ny telev\'{e}sze s\'{e}getn\'{e} h\'{o}ly\'{a}znia mar\'{a}g hat\'{o}s menci\'{o}j\'{a}t, cs\'{u}sztn\'{a} a p\'{u}di\'{a}t, mas\'{i}tn\'{a} a fr\'{i}z \'{e}s radt f\H{u}z\H{o}t, h\'{a}zn\'{a} a vez\H{o}s\'{e}g\"{o}ket \'{e}s a s\"{o}t\"{o}tts\'{e}st, s csingorozn\'{a} a g\'{e}diumokat a hetlen ad\'{a}sok sz\'{a}m\'{a}ra.}}
\def\hulipsum@lxxxix{\g@addto@macro\hulipsumexp{Az egyik veszm\'{e}ny azonnal k\'{o}znia aliz\'{a}lt azokkal a cs\'{u}csos n\'{e}gyes patlan z\'{u}r\'{a}kkal, akiket lehet, hogy zsoli veszt\H{o}, mert hiszen melkered\H{o} volt \'{e}s ezek is a lat\'{u}z\'{a}sr\'{o}l val\'{o} kar\'{o}mok. Ezeket r\"{o}gt\"{o}n cs\'{u}csos g\'{o}s\'{a}gba rin\'{a}dta, nem land\'{a}lta ki mat\'{a}sra \'{e}s \'{i}gy tov\'{a}bb. Nem volt egy sk\'{a}pig sem cs\"{o}r\"{o}s, hogy honnan tudta. \'{E}s akkor koztatotta a pohos mal\'{a}son l\'{e}v\H{o} anus gy\H{u}r\H{u}s\'{e}g\"{o}t \'{e}s minden bulat\'{a}sr\'{o}l f\'{a}rgyszer\H{u} halm\'{a}sokat azoknak a p\'{a}ri\'{a}soknak, akiknek a z\'{u}ra figyel\H{o}j\'{e}vel \H{o} ott cs\'{u}csos szemp\H{o}be b\H{u}n\"{o}z\"{o}tt. De h\'{a}t az\'{e}rt hibif hajonokodt egy k\'{e}szemet, hajonokodt egy anus vartokot. Miut\'{a}n az ipacs szerint a hal\'{a}nyoknak zes\'{i}t\H{o} sz\H{u}r\H{u}t\'{e}s\"{o}k b\'{u}g\'{o}jukat nem fodor\'{i}thatj\'{a}k el perennel \'{e}s csak, mint z\'{u}r\'{a}k csal\'{e}lhatnak golym\'{a}rba, azokat, akik eddig, mint hibelek cs\"{o}kd\"{o}n\"{o}ztek a vejess\'{e}gben, itt zajzolnia nem lehet. A hal\'{a}nyokb\'{o}l kam\'{a}rokat roztak, de a hocsisok nem rendszeresen, koroszt\'{a}lyonk\'{e}nt, hanem \'{u}gynevezett ,,sarandzs\'{a}gokkal'' anyakodtak.}}
\def\hulipsum@xc{\g@addto@macro\hulipsumexp{Zsarl\'{a}s \'{a}bratlan draks\'{a}gair\'{o}l, a k\'{a}lkas helyes isztalan cseml\'{e}sek, gralos ver\'{e}mek \'{e}s lejt\H{o} lonsok fel\'{e}, vagy onnan ide tatos caftos turzs ih\'{e}v\'{e}nye egyre cselez\H{o}! A 230 m\'{a}nyos dalom f\'{e}lem\'{e}ny j\'{a}rsatya is szigor\'{u} hadalka \'{e}s tika \'{a}ll\'{i}t\'{a}n csalmasa, a f\'{a}z\'{a}s szent\'{a}s hososa, \'{e}s t\"{u}rj\'{e}s a v\'{e}nyeren cset\'{e}k is. A hatlan l\"{u}ke kaba cseh\'{e}z h\"{u}lyes, de jagos b\H{u}v\"{o}z\H{o}. Kod\'{a}s csinyl\'{o}s rog\'{a}r, szenc b\"{o}rt\'{e}s, amely rentereli az ort\'{i}t\'{a}s, a bablya \'{e}s a tap\'{a}l\'{a}s rens\'{e}sebeit. Az izzadt \"{o}tem diszkegys\'{e}gek sitv\'{a}st aliz\'{a}lnak a b\'{i}tm\'{e}nyben, amelyet a k\"{u}d\H{o} sz\'{e}l\H{o}z\H{o} pad\'{e}kokkal gy\"{u}lel a k\'{e}sz\'{i}t\H{o} elent\'{e}sekhez \'{e}s alancokhoz. Zatkoz\'{a}sa tagat a cs\'{i}kony, fanyad\'{e}kony \'{a}benekt\H{o}l, talofatot, j\'{o} ugt\'{a}t borlik. Heg\'{e}rceng\'{e}se: s\"{u}t\'{e}z ir\'{a}g szat\'{a}s f\"{u}gg\'{e}s szatty\'{u} (zakfid).}}
\def\hulipsum@xci{\g@addto@macro\hulipsumexp{A szellent\'{e}snek hugyos rad\'{a}sa a k\"{o}z\'{e}s\"{o}k majtit\'{a}sa, nem a ring, ez\'{e}rt t\"{o}bbek k\"{o}z\"{o}tt z\"{o}ldi urv\'{a}st is soncol a k\"{o}z\'{e}st bogos cs\'{a}rat\'{a}soknak, illetve olyan urv\'{a}sokat f\'{e}kedt az arra talan p\'{e}nkekkel, amelyek hegyvereznek a szoci\'{a}lisan v\'{e}tles pasztos boly\'{a}knak a k\"{o}z\'{e}s\"{o}k majtit\'{a}s\'{a}ban. Ennek ellen\'{e}re sajnos fati azoknak a fog\'{a}sa is, akik nem h\'{a}lher\H{u} klong\'{a}ik miatt bigletetnek a sz\'{i}v\'{a}snak. A f\"{u}gg\H{o} j\"{o}v\H{o}t, a ringeket tervszer\H{u}en tagol\'{o}dzj\'{a}k az \"{u}get\H{o} paz\'{a}s r\'{a}nik\'{a}i, virk\'{a}ba v\'{e}ve az id\H{o}k\"{o}zben esetleg jultos k\"{o}z\'{e}s\"{o}k k\'{i}t\'{a}ny\'{a}t. A k\"{o}z\'{e}s\"{u}ket puzmusban tajlan boly\'{a}k biz\'{a}lkodhatj\'{a}k a ringgel unkos b\"{o}rget\H{o}ket \'{e}s is\'{a}gra is h\'{u}s\'{a}gba cs\'{i}rozhatj\'{a}k a c\"{o}ldh\"{o}z kedt h\'{a}klan s\"{u}l\'{e}s \'{a}ltal te\'{a}nyos barzatokat. Az itt t\"{o}z\H{o} csepps\'{e}gek nem felt\'{e}tlen\"{u}l h\'{a}szk\'{a}zj\'{a}k a gat\'{o}s\'{a}g el\H{o}s\'{e}t. A s\"{u}ke viva lengez\H{o} r\'{e}kos v\'{e}szege \'{o}ta a kob\'{a}s t\"{o}bb lyukraviszny\'{a}j\'{a}n utatotta meg, hogyan \H{o}r\"{o}z \'{a}t egy fog\'{a}ly \'{e}s annak tavir\'{a}gai a j\'{o}l s\'{e}lyi \'{e}s z\'{a}ros m\'{o}don szemis fegyencs\'{e}s aloz\'{a}rral. A prang, mens\'{e}g soncsod\'{o}s szatlacc\'{a}nak, a cs\'{a}v\'{a}nyos dala fog\'{a}ly\'{a}nak k\"{u}l\"{o}sze olyan komzat\'{a}ssal r\'{e}vekedi a kob\'{a}s vetereteit \'{e}s sz\"{u}remb\'{e}szeit, ami l\H{o}di, hogy a kora, az arg\'{a}s gr\'{a}sa sokkal t\"{o}bb ma m\'{a}r, mint egyszer\H{u}en lelg\H{o}t boroloznia a martandr\'{a}lyban.}}
\def\hulipsum@xcii{\g@addto@macro\hulipsumexp{Dokoron talan c\"{o}grez\'{e}se sod\'{a}sz, akinek a gonyk\'{a}ja forszer\H{u} fejlemen a vereg\'{e}sz kas\'{a}g pezr\'{e}j\'{e}ben. Az egyik \'{e}jjel dokoron nemet\'{e}zi pitatot, majd g\"{u}zm\"{o}gik t\"{u}l\'{e}st. Egy m\'{a}sik csutriummal dokoron nesejt sod\'{a}sz mell\'{e} a pincsbe \'{e}s a k\'{e}t t\"{u}rg\'{e}s huny\'{a}l. A f\'{e}let padl\'{a}sok, \'{i}gy dokoron \'{e}s sod\'{a}sz gyarasztos bic\'{a}ja is noklomra t\'{o}s\'{i}tnak, akit t\"{o}d\'{e}s k\"{o}zben sikkantnak. De a latlan has\'{a}g, a terend\H{o} r\"{u}cs\"{o}k, az \'{e}l\'{e}s teheg\H{o}j\'{e}t a meztelen\"{u}l \"{o}l\'{e}s j\'{a}rus szet\'{e}se nem gy\H{o}zelheti. A k\'{e}t t\"{u}rg\'{e}snek nemcsak a t\"{o}bbes izmust ezd\H{o} t\"{o}rtetlen\'{e}s\'{e}re van nyoml\'{a}sa, hanem egy rintel\H{o}, n\'{e}kos al\'{a}rv\'{a}nyra is, ahol a r\"{u}cs\"{o}k a dig\'{e}ben, a seres p\"{o}f\"{o}ltetben is kedik. A finoman puffadt, vihamos hetlenik ez\'{e}rt dokoron \'{e}s sod\'{a}sz v\'{e}nyez\H{o}j\'{e}t egy n\'{e}pes, halacs zony\'{a}ra p\'{a}szj\'{a}k.}}
\def\hulipsum@xciii{\g@addto@macro\hulipsumexp{Ek\"{o}zben k\"{o}verny\H{o}, \"{o}sszesen 1800 tantot kodr\'{a}tolt v\'{e}gre tegnap a terbecskesz \"{u}r\'{e}k r\'{a}di\'{o} ellen, jelezve, hogy talalkopj\'{a}k a tehelys\'{e}get a tesers\'{e}gek gy\H{u}r\'{i}t\H{o} hereteire, a fodos hodza kobit\'{a}s k\"{o}r\"{u}l be\'{a}sva ildoros tormel\'{a}saira, miel\H{o}tt les\'{i}tne az \'{e}hberen turc k\'{a}l\'{a}sa. A tizmusok tantainak 20 dzseg\'{e}se t\"{o}ngt kobit\'{a}s ellen, ahol ism\'{e}t m\'{e}lketek halm\'{a}ztak cant, \'{e}s t\"{u}z\'{e}s randomb\'{a}ra megint almazott az \'{e}hberen z\"{o}ngen manas\'{a}g h\'{a}lija. A frid\'{o} nalkarcsa p\'{a}kodta be, hogy a szels\H{o}s szerti gatyu h\'{u}sz vez\H{o} canttal vezett, k\"{o}z\"{u}l\"{u}k f\"{u}g\'{e}s f\'{u}v\'{a}s. El\H{o}sz\"{o}r or\'{i}tott el\H{o}, hogy a fokod\'{a}sok mindh\'{a}rom sik\'{a}ja, a csikil, el\'{e}s \'{e}s tord egyszerre olg\'{a}lt nad\'{a}st a vin\'{a}kban. A portornoz\'{a}s bermeseinek hathajos mar\'{a}sok ism\'{e}t n\'{a}ci\'{o}ba ran\'{a}ztak, gigolt\'{a}s \'{e}rte foga l\'{u}ga \'{e}s belbeli eszki\'{a}sa, k\"{o}d\'{e}s k\"{u}lem\'{e}ny\'{e}t. Az \'{e}hberen puca egy fegylens\'{e}gen ett\H{o}l f\"{u}ggetlen\"{u}l ism\'{e}t k\"{o}dt a szereszteli uszlikon mindk\'{e}t eszki\'{a}sa szonz\'{a}ban, b\'{a}tor\'{i}tva a tesers\'{e}geket. A pal\'{a}s pszig\'{a}sr\'{o}l az \'{e}hberen nyomet\'{a}k lezel\H{o}j\'{e}be t\"{o}ty\"{o}g t\"{o}bb t\'{a}ris foszl\'{e}kos irtyas\'{a}g, miut\'{a}n har\'{e}k taniz\'{a}lta a bucig\'{a}lis sulykoz\'{a}sok citus\'{a}t.}}
\def\hulipsum@xciv{\g@addto@macro\hulipsumexp{Fel\'{e}kez\'{e}s volt, de h\'{a}zslan nap eg\'{e}sz nap csillozott \'{e}s zsanturcra tarass\'{a} baszt\'{a}lt a p\'{a}ros. A f\"{u}st de szars terkeszty\H{u}j\'{e}t is emezekedt\'{e}k az alacsonyan t\"{u}d\"{u}l\H{o}, \'{a}ltos, ruccos vajt\'{a}sok \'{e}s a sugtalatok is fakodass\'{a} baszt\'{a}ltak. Vit\'{a}s sina \'{e}s h\'{a}rom kerhegyz\H{o} f\"{u}lke \'{o}s\'{a}gban fejesett az egyik mellenbe, ahol a k\'{i}t\H{o} \'{e}ppen a borhatlan \'{e}rv\'{e}ny rosoly\'{a}j\'{a}val voltak elfoglalva. A k\'{i}s\'{e}gek brium\'{a}ra ijedten tankodtak, de nem nyiv\'{a}sodtak el. A sina faktudozott a molom mellett \'{e}s ny\"{o}gezett. A mellen el\'{e}g kerv\'{e}l\H{o} volt, alig t\'{i}z szeg\'{e}rtet \'{e}s vagy h\'{u}sz herum fereger\H{u}, a kargangt\'{o}l \'{e}s az ocsk\'{o}t\'{o}l peje filit kaj\'{o}g\'{a}lt csak. P\'{a}r szeg\'{e}rtnek hatlan volt a lal\'{a}ja \'{e}s szinte vad\'{e}koson font\'{a}sok bagztak.}}
\def\hulipsum@xcv{\g@addto@macro\hulipsumexp{Ami azt nyozja el\H{o}re, hogy ak\'{a}r \"{o}ssze is l\'{e}shetnek. A fedd\H{o} b\'{a}ns \"{o}ng\'{e}s\"{o}kt\H{o}l m\'{a}d\'{e} szerint ferg\H{o}re vannak a ked\'{e}s p\'{a}hinai. Tatos ak\'{a}s vid\H{o} lel\'{e}sk\'{e}nt jeletnie ezeket a gul\'{a}sokat \'{e}s tagtalt kod\'{a}sokat: csak r\'{a}juk kell er\H{o}dnie, \'{e}s p\'{a}r b\'{u}cska gyepr\'{i}ze teljesen teglevess\'{e} \'{a}rk\'{a}lja, hogy patlan farka az eg\'{e}sz. A s\'{o}v\'{a}nyos besztez\H{o}re eml\'{e}kezve m\'{e}rhetetlen\"{u}l l\"{o}lcsel\H{o}, ahogyan a d\'{e}tleges nez\H{o}d\'{e}s p\'{a}ns\'{a}g a tajcsos fark\'{a}t sz\'{a}matja. Azzal, hogy rondokodik ,,szens'' kinzsem\'{e}ben, a cs\"{o}khes k\'{e}ml\H{o}ben. Azzal, hogy teheg \'{i}nyen is\'{e}geinek kod\'{a}s\'{a}t az egy\'{e}ni ,,bizmust'' is irim tes\'{i}tv\'{e}sre til\'{o}dja. Konyos, ha a p\'{a}ta a csicska balk\'{a}j\'{a}val filtet egy bulikot, mely ahhoz l\'{a}ts\'{a}gos egy\'{e}ni fark\'{a}t hedzken ki, mint amely hatvan kez\'{e}ke c\'{a}thatlan \'{a}l\'{a}sok gy\"{u}lt\H{o} v\"{o}sdj\'{e}t \'{e}s dalkoz\'{a}j\'{a}t t\'{e}veztetette el\H{o}.}}
\def\hulipsum@xcvi{\g@addto@macro\hulipsumexp{Amennyiben a festy\'{e}sn\'{e}l nem b\'{i}r\'{a}sta be a szom\'{a}lyos festy\'{e}s talmarask\'{a}t, akkor csak egy f\'{a}jos ideig lesz a moly\'{a}ban bel\'{e}pve. Ez kozja, hogy dulanak t\'{a}zjanak hozz\'{a} a gyeres\"{u}n\'{e}h\"{o}z, vagy hogy a traszban csapasztjanak a folomba. A kancban p\'{a}rgyszer\H{u} egy meta, kang\'{o} ez ifteges: baka pist\'{a}ja. A pudalm\'{a}t nem lehet duggatnia, de meg lehet csillognia, fen\H{o}cs\"{o}t kednie. Az egyik, hogy ken\'{i}t\H{o} folomhoz kod\'{a}skoznia kell a b\"{o}lgyed\'{e}st a festy\'{e}shez. A folom szerz\H{o} id\H{o}k\"{o}z\"{o}nk\'{e}nt kapin\'{a}zja azokat a pikul\'{a}kat, akik nem k\'{o}ztak hozz\'{a} a bazakozg\'{a}csokhoz. Ez \'{a}ltal\'{a}ban az \'{e}get k\"{o}zeg\'{e}s\'{e}n het\'{e}ny, de nem minden \'{e}ked\'{e}sben.}}
\def\hulipsum@xcvii{\g@addto@macro\hulipsumexp{Utolj\'{a}ra 5 bord\'{a}sa f\'{e}nyelt\'{e}k meg a h\'{e}v\'{e}st. Utolj\'{a}ra 2 bord\'{a}sa f\'{e}nyelt\'{e}k meg a h\'{e}v\'{e}st. Utolj\'{a}ra 2 bord\'{a}sa f\'{e}nyelt\'{e}k meg a h\'{e}v\'{e}st. Amikor rabab\'{a}lta, hogy a k\"{o}vekn\H{o} p\'{i}t\'{e}kek getkedt\'{e}k brekeiket a virhamok el\H{o}tt, l\"{o}v\'{e}dben o\'{a}zott a j\'{a}tos p\'{i}t\'{e}kre, rigyesedte a j\'{a}tos t\"{o}rdelit, melyet 1897-ben k\'{a}ztak. 1879-t\H{o}l 1890-ig a telt szofancs felt v\'{a}nosz\'{a}n kedt. A f\'{a}j\'{e}kos csel\H{o} csoz\'{a}s csepr\H{o}j\'{e}ben 6 gyuriig k\"{o}sk\'{e}t kalt el\H{o} \'{e}s a cs\'{o}kados s\"{o}d\'{e}sben fog\'{a}szokat cs\'{i}ptetett. Gas\'{i}totta borg\'{a}s szipityma a virham mint gedv kols\'{a}g leder\'{e}t (dalmus, 1907) \'{e}s sz\'{a}mos pubort j\'{e}sedt pr\'{a}csokba a h\'{u}z\'{a}sr\'{o}l \'{e}s a moz\'{a}sr\'{o}l.}}
\def\hulipsum@xcviii{\g@addto@macro\hulipsumexp{Ezt a b\H{o}g\H{o} met\H{u}t m\'{e}g a kalm\'{a}rok sem tudt\'{a}k lopt\'{a}znia, sem leked\'{e}s\"{u}kkel, sem kreg\"{u}kkel. A j\'{o} feks\'{e}g \'{e}s a csukt ventetverv\'{e}g\'{e}ben dik\'{a}val: Nem z\"{o}rnyess\'{e}g hen\'{i}tnie, hogy ki ny\"{u}v\'{i}ti helyesen, de jusztja el, az\'{e}rt a telet valamivel t\"{o}bb, mint f\H{o}tleg \'{e}s mocs\'{a}r. Aki t\'{a}lyos\'{i}t teletet, az tudja, aki pedig nem, az \'{u}gysem fogja f\'{a}ngnia soha. Hads\'{a}g, serj\'{e}nes r\"{o}zetes \'{i}zes marmen a serj\'{e}nes r\"{o}zetes tekelg\H{o} sz\H{o}rz\'{e}s ol\'{a}sainak tap\'{a}sz\'{a}ban. A h\"{u}v\"{o}ld\"{o}fes per\'{e}sz hatlat\'{o} f\'{o}ka negyedik felepes\'{e}t Bok\'{a}nyi m\H{u}t\'{e}s, a csal\'{a}bas per\'{e}sz cs\'{o}nia hol\'{a}sa sz\'{e}ldeti be. Z\"{o}ty\"{o}glem pipszemlet red\'{e}s \'{e}s d\'{i}sz\'{i}t\'{e}s, l\'{a}z\'{a}t felencsel\'{e}k del\'{o} red\'{e}s, mog\'{o}s \'{e}s d\'{i}sz\'{i}t\'{e}s, valamint kop\'{a}s felencsel\'{e}k b\'{a}tors\'{a}g d\'{i}sz\'{i}t\'{e}s valamennyien az edeli kod\'{a}s rekt\'{a}k fogas\'{a}g\'{a}nak sz\"{o}v\'{e}lyei f\'{a}tyog\'{o} becserettel fongatnak be a zsina fel\H{o} \'{e}s fojtos paszt\'{a}ban.}}
\def\hulipsum@xcix{\g@addto@macro\hulipsumexp{Azok a vod\'{a}sok, akik a l\'{e}telev\H{o} mell\H{o}l vagy m\'{a}s lengeszt\H{o} m\H{u}t\H{o}r\"{o}kb\H{o}l p\'{o}t\'{a}lnak, mindig m\'{a}nyodj\'{a}k, mi\'{e}rt van az, hogy az anyug\'{a}t otthon jobban k\'{e}ny\'{i}tett\'{e}k. Csemert, ne a fradz\'{o}t vagy a bugyl\'{o}t, mint \'{a}ltal\'{a}ban iz\'{a}lt\'{a}k. \'{E}res rod\'{a}sok \'{e}s egyet dold\'{a}s eset\'{e}n a padt csemerek coronca. V\'{e}gezet\"{u}l egy csemer, amelyben vikerk\'{e}nt kostiv\'{a}k jelen van. A r\'{a}tos, k\"{o}rg\H{o} t\'{e}v\'{e}nek sad\'{a}sz\'{a}ra Langen a szoml\'{o} z\"{o}ngy ked\'{e}st kodta ki. Az\'{e}rt, hogy a k\"{o}z\H{o}ket rimulja a csigaris\'{a}gra h\H{o}si kez\H{o} kov\'{a}nyoss\'{a}gokr\'{o}l. A lels\H{o} rod\'{a}sok, mint az in\'{a}s \'{e}s a z\'{u}zat\'{a}s, fagyos itust kajtnak el.}}
\def\hulipsum@c{\g@addto@macro\hulipsumexp{Itt \"{u}zelhet tesek pancokat, c\'{e}dr\'{o}b\'{a}szokat, horl\'{a}ns\'{a}gokat a lat\'{u}rba. Egy bajt\'{a}son kereszt\"{u}l incoroz majd a mentegs\'{e}g, a pityeg pedig a csemb\'{a}szon talm\'{a}kb\'{o}l tev\H{o}dik. Reggel feli az \'{i}zetlet, gorozott a lald. Nyez\H{o} rassz\'{o}kr\'{o}l elm\'{e}lkedve egyszer csak g\'{e}lent egy polg\'{a}st. A polg\'{a}s sedt a kv\'{a}ny k\'{a}tizmus\'{a}ra, \'{e}s azt vacskodta: K\'{a}ng\'{a}ny gol\'{a}s kod\'{a}l\'{a}ja a pech\'{o} valamennyi k\"{o}d\'{e}s\'{e}t el\'{i}t\'{e}lve, az emz\H{o}k fonty\'{u}j\'{a}ra \'{e}res\'{i}t\H{o}t hasok\'{i}t mez\H{o} szobul\'{a}s\'{a}n este a fencezess\'{e}gek vez\H{o}j\'{e}n. A pesepen\H{o} nyeges gy\"{u}m\"{o}s but\'{e}r\'{a}n a lozsmag\'{a}sz tapords\'{a}g els\H{o} \'{e}rel\H{o} kvajk\'{a}j\'{a}r\'{o}l volt vontum.}}
\def\hulipsum@ci{\g@addto@macro\hulipsumexp{Ezek ut\'{a}n a b\'{a}bolyoz\'{a}st r\'{e}ginek h\'{e}vic\'{a}zt\'{a}k, \'{e}s a r\"{u}l\'{e}s\"{o}k a lobolyag \'{i}t\'{e}se alapj\'{a}n v\'{a}rt moros b\'{a}bolyoz\'{a}sra sartaroztak \'{a}t, amely jelenleg is oszol. A szennyes csibi a m\H{u}ves mell\'{e}r k\"{o}z\'{e}s\'{e}re s\"{u}llyedt b\"{o}rzs\"{o}t a f\'{a}j\'{e}kos pejlens\'{e}g h\"{o}ng\H{o} kalan balatainak jog\'{a}sa \'{e}s a j\"{o}v\H{o} run\'{a}k al\'{e}ka k\"{u}zlij\'{e}ben. A dinty fekedte, hogy egy raszony paricsk\'{a}ban a h\"{o}ng\H{o} f\"{u}rkevete a saj\'{a}t hantj\'{a}t is has\'{a}g kalan ponyoz\'{a}rt k\'{o}b\'{a}jt meg, r\'{a}ad\'{a}sul olyan ins\'{a}g\'{e}rt (ism\'{e}hs\'{e}g cuka f\"{u}get\'{e}s\'{e}rt), amelynek matlat cuka regl\'{o}ja a csibi szerint nem kedikrezett paszatl\'{e}ban a b\'{a}bolyoz\'{a}sok szeg\'{e}s\'{e}vel. A dinty ledzte: a kalan t\'{a}rk\'{a}ra k\"{o}telez\H{o}en erteg hola nincs, ez\'{e}rt nem hajkuk\'{a}z csavak\'{a}nyot. Az eged akkor lett volna vezetlet, ha a f\"{u}rkevet hantja valamilyen fityit s\"{u}llyedt volna, \'{a}m erre nem halk\'{a}lt semmi. A futi t\"{o}lcs horont viszont el lehetett volna azony\'{i}tnia, ha folytos ripolg\'{a}sb\'{o}l tor\'{a}lj\'{a}k ki a szeres zsugs\'{a}got. A vitven hinat szennyes bicildegeit ipnos katonna k\"{o}ven\'{i}t\'{e}se szerint a gyurzsok a kobol\'{a}szok vetehelyhecs\'{e}n\'{e}l t\"{o}bbnyire nem g\H{o}gics\'{e}ltek szennyes, vagy feli b\'{a}bolyoz\'{a}st m\'{e}g akkor sem, amikor a nyatlan libr\'{a}ma jelent\H{o}sen hintetette a szennyes szockot.}}
\def\hulipsum@cii{\g@addto@macro\hulipsumexp{Elv\'{e}sek szerint ma m\'{a}r egyik zsiml\'{e}ssel sem m\'{o}ni a baram. A szit\'{a}rok, a grags\'{a}gok \'{e}s a f\'{a}n\'{a}k p\'{e}ld\'{a}ul k\"{u}l\"{o}n\"{o}sen d\'{i}t\H{o} horikumban vannak. A l\'{a}zlik \"{u}dv\"{o}rg\H{o} huz\'{e}kban vannak a c\'{a}rl\'{o} r\'{e}gen pipent pacsics\'{a}val: az aland\'{o} l\'{e}k\'{a}rdok fogasz\'{a}val. Az aland\'{o} l\'{e}k\'{a}rd a fonoss\'{a}g pici k\"{u}l\'{e}s\'{e}nek sp\'{o}t\'{a}st duggat, kurja \'{e}s cs\'{u}ny\'{i}tja kolts\'{a}ggal. Navass\'{a}gg\'{a} hall\'{o}zja, melyben tall\'{o}zhatik a serele, \'{e}s szul\'{a}st hosszabb\'{i}that a serele fel\'{e} metek temlen. M\'{e}g ha pend\H{o}len az al\'{e}kos old\'{a}khoz metek jelemetekben, balk\'{a}kban, f\'{e}rz\H{o}d\'{e}g\"{o}kben sokszor tesegel is. \"{U}rg\'{e}ly 1053, cs\'{a}v\'{a}s kuga szita lat\'{o} gala koros.}}
\def\hulipsum@ciii{\g@addto@macro\hulipsumexp{A boh\'{a}sznak az a v\'{a}tr\'{o}ja, aki a haszn\'{a}lt vit\'{a}toknak talatlan rombok hat\'{e}lyos misszi\'{o}i. \H{O}ket a kal\'{e}k t\'{a}gcs\'{a}rsodja meg egy tintoss\'{a}gra. Minden tintoss\'{a}g v\'{e}g\'{e}n szabar\'{a}tot nevelytezik a vit\'{a}t pangor\'{a}r\'{o}l, melynek egy-egy buzacc\'{a}t fodj\'{a}k a csig\'{a}nak \'{e}s a kal\'{e}knak. A kal\'{e}k m\'{a}ci\'{a}ja mellett gat\'{a}zja a j\"{o}v\'{e}ny\"{o}k toz\'{a}s\'{a}t, illetve m\'{a}s vit\'{a}t pik\'{e}j\'{e}vel egy\"{u}ttm\H{u}k\"{o}dve epleleneket cingetnek. A t\'{e}tkeg\'{e}s k\'{o}tum b\'{a}rmely ellet. f\H{o}zeges pacsalmikus racos lom\'{a}ds\'{a}ga, aki a csent\H{o} han\'{u} sor\'{a}n a dr\'{a}szd\'{a}knak lantott. A boh\'{a}sz f\'{e}les v\'{a}tr\'{o}i falang\'{a}suk ugs\'{a}g\'{a}ban mocsaposak. A zsebes izmus a csent\H{o} vizet ilyen fan\'{a}lis sulansa alapj\'{a}n, a sulansot dokom hitel\'{e}sen hint l\'{e}tre.}}
\def\hulipsum@civ{\g@addto@macro\hulipsumexp{A buggyosak vankj\'{a}t a 1956-ban \'{e}s az azut\'{a}n d\H{o}b\'{e}k k\'{e}szervetett\'{e}k. Sari\'{a}jukra a menoz\'{a}s m\'{a}r lebel\H{o} csir\'{a}ny, amihez k\"{o}nnyed\'{e}n pankodhattak. A j\'{a}rny\'{e}kok ezt a vankot fathatos cserumnak k\'{e}ltetik, cs\'{i}ros sok szaftos zava k\"{o}z\"{u}l t\"{o}zhet\"{o}tt. A buggyos cserum cselet\'{e}ben a lengec\'{e}k \'{e}s a karad\'{a}s fin\'{o} k\"{u}l\'{e}st veg\H{o}z\"{o}tt be. A szaftos zav\'{a}k b\H{o}s\'{e}gesen a szelens\'{e}g\"{u}kre csacsolg\'{a}ztak, melyekb\H{o}l saj\'{a}t bomboraiknak megfelel\H{o}en szabadon t\"{o}zhettek. A slit\'{a}k land\'{a}sa alapj\'{a}n ennek a cserumnak nem r\'{a}nyosott listi\'{a}t, hogy hol \'{e}s mit szuvarjanak, sokkal szervez\H{o} volt viszont saj\'{a}t hegyz\H{o}j\"{u}k hid\'{e}se. A buggyosakat m\'{a}r egy kadt f\"{o}rte on\'{a}zta k\"{o}r\"{u}l, ahol a tula is az ezemteves \'{e}berhez b\"{u}kk\"{o}s \"{u}gyn\"{o}ss\'{e}g.}}
\def\hulipsum@cv{\g@addto@macro\hulipsumexp{P\"{u}ltben az am\'{a}ny al\'{a}snak haszog, a kindet\'{e}st, a lest\'{e}st, \'{e}s a szav\'{a}t is tr\'{o}mmal \'{e}s jez\H{o}s\'{e}g foncsos m\'{e}nittel \'{a}rt\'{e}rozhatj\'{a}k. Az am\'{a}nyok, m\'{e}g az git\'{a}s\'{a}r\'{o}l oly csalan b\'{a}z\'{a}sban is, m\'{a}rt\'{a}st m\'{e}t\'{a}lnak. F\H{u}r\'{e}mben rendszeresen palnak ilyen domos forsszer\H{u} kot\'{a}kat, ahol legink\'{a}bb k\'{e}pedeket \'{e}s morcs\'{u}sz\'{a}gokat zavacolnak a tiz\H{o}sbe. A h\'{a}lt, zsonlann\'{a} lottyadt tos\'{i}t\'{a}nyok persze m\'{a}r az els\H{o} artakban d\'{i}t\H{o}nek verekezednek. A roz\'{a}s mindaddig gos\'{i}t, am\'{i}g az egyik \"{u}nnepl\H{o} mult\'{o}v\'{a} nem \'{e}ledik. Persze, a sz\"{u}ld gadt v\'{e}nyel\'{e}sek sem kerg\H{o}znek el. A gyomos pejk el\'{e}ker \'{e}s verend\H{o} s\"{o}d\'{e}k\"{o}kkel, valamint a k\"{o}t\'{e}s\"{o}k gyadt slat\'{a}val szam\'{a}z idely\'{e}t b\'{a}lylnia ezeknek a domos pans\'{a}roknak.}}
\def\hulipsum@cvi{\g@addto@macro\hulipsumexp{A pelt\'{e}s ezen k\'{i}v\"{u}l mag\'{a}ba j\"{o}veneti a pakab\'{a}zra szabici\'{o}zus, b\H{u}n\"{o}s fikoml\'{a}sokat a firk\'{o}val \'{e}s akus ked\'{e}lyet lels\H{o}s pad\'{a}sokat. Kortos pelt\'{e}st l\'{u}g\'{o}s \'{e}s g\'{u}nyos fur\'{a}nos firk\'{o} k\'{o}dja felv\'{a}ltva. A pelt\'{e}s a kenye \'{a}ras\'{a}gra pongozja az egyz\'{e}st. A l\'{u}g\'{o}s patos firk\'{o} sz\"{u}lete a sug\'{a}s gris\'{e}g, a g\'{u}nyos patos firk\'{o} sz\"{u}lete egy gyak\'{a}s ben\'{e}s \'{e}s a menok dolv\'{e}ly\'{e}je kod\'{a}sa \'{e}s sz\'{o}lusa, a foszfos rec\'{e}s tisztama. A pelt\'{e}s ezen k\'{i}v\"{u}l mag\'{a}ba j\"{o}veneti a pakab\'{a}zra szabici\'{o}zus, b\H{u}n\"{o}s fikoml\'{a}sokat a firk\'{o}val \'{e}s akus ked\'{e}lyet lels\H{o}s pad\'{a}sokat. Firk\'{o} pelesben vannak a csutria fog\'{o} szeszlem avajnomaival \'{e}s t\"{o}bb hordozatlan skollal szokolnak, \'{i}gy hat\'{e}konyan tudj\'{a}k apaml\'{a}znia pogacsaikat a fel\'{e}sek racska szang\'{a}s\'{a}ban. A pelt\'{e}s ezen k\'{i}v\"{u}l mag\'{a}ba j\"{o}veneti a pakab\'{a}zra szabici\'{o}zus, b\H{u}n\"{o}s fikoml\'{a}sokat a firk\'{o}val \'{e}s akus ked\'{e}lyet lels\H{o}s pad\'{a}sokat.}}
\def\hulipsum@cvii{\g@addto@macro\hulipsumexp{Szetel\H{o}s\'{e}g koz\'{a}s doz\'{e}k\'{a}n h\'{o}dt a v\'{e}gzatos ses\'{e}g. S\"{u}ppen\H{o} jend\'{e}s a j\'{a}rguszok a dr\'{a}ly ebec\H{o}j\'{e}t nem k\'{e}szegik. \"{U}zeli \"{o}nt\H{o} fog\'{a}s vagy t\"{o}bb ellet eset\'{e}n utonys\'{a}g szerint m\'{e}ts\'{e}g: Bonc, kep\"{u}l\'{e}g \'{e}s most m\'{a}r teszteg\'{e}s is gacsos. Egy b\'{a}nic jut\'{a}s tosk\'{a}s a tat\'{a}s t\'{i}z k\"{o}zs\'{e}n alatt n\'{e}gy r\'{a}k\'{a}n l\"{o}v\'{e}sz m\H{u}ves folytony sz\'{a}kuum lehet, zsink ut\'{a}n akci\'{o}san. Egy k\"{o}k\'{e}lyes csasas\'{a}g, egy kod\'{a}s \'{e}s meg van oldva a b\H{u}n\"{o}ss\'{e}g. Az\'{e}rt, hogy ne csak fuva legyen, itt r\"{o}gt\"{o}n el\H{o} is ciceln\'{e}k egy sujas netelmetet, ami t\"{o}k\'{e}letesen vint\'{a}ros lehet arra, hogy \"{u}velje, ki milyen s\'{a}gos bunk\'{a}kkal d\"{o}ng\H{o}zik folyagr\'{o}l.}}
\def\hulipsum@cviii{\g@addto@macro\hulipsumexp{Meni, b\"{u}kk\"{o}ny, h\'{a}tus \'{e}s v\'{a}nyos esztes zong\'{a}sok sz\H{u}k\"{u}len\'{e}re lemegyelenhetnek, akik falads\'{a}lnak a szam\'{e} kedvel\H{o} macsij\'{a}n\'{a}l d\"{o}rmes 3-\'{a}n gy\"{o}nb dicsos t\"{o}bbf\'{e}rre. H\"{o}lcs\"{o}n, \'{e}tf\'{e}r kas\'{a}g\'{a}n fingert\'{e}k be, a fas\'{a}g k\H{o}rm\'{e}ny sz\"{o}vez\'{e}sei a papack\'{a}k ratos hiv\H{o}, \'{e}s ,,ipnos'' sart\'{a}j\'{a}t a nagyon szergederesre k\'{e}szk\H{o}z\"{o}tt, folyg\'{o} cig\'{e}m el\H{o}tt. \'{E}tf\'{e}r bang\'{o}j\'{a}n este ep\'{e}p bormanda ,,aport c\'{e}dr\'{a}cs'' ratos pihez\H{o} fogly\'{a}ban sajh\'{a}nyozhatott, hely\'{i}thetett a t\"{u}dt dika. Gy\'{u}jt\'{a}s b\'{u}z\'{a}s k\'{e}szegs\'{e}g saj\'{a}t ked\H{o} torm\'{a}nyz\'{a}tait ros\'{i}totta sz\'{o}zal\'{a}sra, melyek d\"{o}rmes teres\'{e}g szeresek a csalegrusz kolhandban. Az alv\'{a}t b\"{o}zetle ukal\'{a}s\'{a}ban is kolg\'{a}l telemlet akan\'{a}r, mint ahogy minden l\'{a}gechen szabog e \'{i}res bancoz\'{a}s. Ez\'{u}ttal a szerek, balat\'{a}l\'{a}s \'{e}s pajt\'{a}s ratos sarta marmagolt a bornyos hev\'{e}zel\'{e}sben. Jel\'{e}st\H{o}l pr\'{o}szig l\'{a}zhatik mag\'{a}ra a fesebet\H{o} koli, a h\"{o}lcs\"{o}n szoty\'{o}kony illetve a buggyos \"{o}rte.}}
\def\hulipsum@cix{\g@addto@macro\hulipsumexp{A medd\H{o} buni pecsni b\'{a}nsza az iher korlan medd\H{o} ezet in\'{o}s szolib\'{a}j\'{a}ban kal\'{a}totta a tavaly k\'{e}sz\"{u}l\H{o} t\'{e}pz\H{o} herga most hi\'{u}s padatlan pirteit. Pismesegte az orz\'{a}st arra, hogy peny\H{o}ben az eg\'{e}sz szermer\H{o} nyalata szomos meg, azaz a peget\H{o} csag\'{a}s kan\'{a}z\'{a}sa hajt a hid\'{a}son bel\"{u}l. M\'{a}r a 1992-ben k\'{e}sz\"{u}l\H{o} herga is azt kodta ki, hogy a t\'{e}pz\H{o} molyh\'{o} p\"{o}rge volt a p\'{o}risk\'{a}hoz k\'{e}pest, s ez a cs\"{u}s\"{o}mb most ugyan\'{u}gy k\"{u}le. Fi\'{u}s\'{i}totta: a k\'{e}t foga saj\'{a}ns borda bolyhos, ami azt togatja, hogy a felenen fens\'{e}gekb\H{o}l val\'{o} k\"{o}lgy or\'{a}ga a bakoss\'{a}gok terum\'{a}ban tov\'{a}bbra is b\H{u}n\"{o}v\H{o} lesz, mint a p\'{o}riska terum\'{a}ban. A t\'{e}pz\H{o} molyh\'{o} vegyez\'{e}se t\'{i}z h\'{a}ty\'{a}n alatt doz\'{a}s gocskafk\'{a}val kodt, a k\"{o}lgy b\H{u}n\"{o}v\H{o} szagos volt, mint arra a jormat\'{a}s \'{e}s a fontoss\'{a}gok d\'{u}colt\'{a}k. Tikos lelke szerint ennek fens\'{e}ge az volt, hogy a tikony k\"{o}lgy lyuk\'{a}j\'{a}t a hides pirtekre alapozva d\'{u}colt\'{a}k ki, de fogy\'{a}sz volt a t\"{o}bbes \'{e}s a t\'{i}z h\'{a}ty\'{a}nnal ezel\H{o}tti herga javat\'{a}ban. Az egeni herga sor\'{a}n azokat a t\'{e}pz\H{o} pez\'{e}keket is \"{u}l\"{o}k\"{o}dt\'{e}k, akik egy h\'{a}ty\'{a}na hancson n\'{e}pereztek.}}
\def\hulipsum@cx{\g@addto@macro\hulipsumexp{Sz\H{u}z\H{o} paca, sz\H{u}z\H{o} k\'{a}lmak\'{a}p, az anyagos foly\'{o} sz\H{u}z\H{o} m\'{e}nys\'{e}gei h\'{i}mkeznek ki. Kezd\'{e}m r\'{e}m\'{e}ny is arra a g\'{a}tatra kezik, hogy a talan vedenek pezsebet masszolnak arra, hogy az \'{a}ronok alulr\'{o}l kezdve szegets\'{e}k a fatlan lenis\'{e}geket. K\'{i}v\'{o}s tapust fejegyezik annak, hogy 1980 pulla, a gyapaszt\'{o}s mesk\'{i}s\'{e}gek kron\'{a}ig hat\'{a}lha di\'{a}sok lel\'{e}se k\"{o}dt a fal\'{e}ni\'{a}ra. Masztak a kirgi j\"{o}v\H{o} bel\H{o}s\'{e}g\"{o}k, melyek a detkes parozat sz\H{u}z\H{o} \'{e}r\'{i}t\H{o} hab\'{a}nban val\'{o} pegl\H{o} kalat\'{a}v\'{a} heztek. Sok csepe, f\'{a}rkos, rosszul jed\'{a}lis kar\'{a}nynak ny\H{u}g\"{o}ny\"{o}z\"{o}tt ez k\'{e}pl\H{o} falan pezsebet. Kirgi l\'{e}tked\H{o} bertel\'{e}szek maratottak, ahol az al\'{a}tlan bogatok for\'{a}t, k\"{o}sz\"{o}g\"{o}t, hatlan \'{e}s tat\'{a}ris beless\'{e}get v\'{a}nyoztak a felm\'{e}nyeknek. \'{E}ppen a reteny\'{e}s sz\"{o}v\'{e}ny kez\'{e}st sz\H{u}k\"{o}ly\"{o}dje fel pantnak, amely hat\'{a}lha t\"{u}d\"{o}s alulr\'{o}l szorsos ked\H{o} volt, a j\"{o}v\H{o} bel\H{o}s\'{e}g cs\'{i}p\H{o}j\'{e}re.}}
\def\hulipsum@cxi{\g@addto@macro\hulipsumexp{Az \"{u}v\"{o}s sor\'{a}n, a kodarom a feher\'{e}gekhez izzaszt \'{e}s azt a rossz\'{u}s\'{a}sig mag\'{a}ban mos\'{a}g az \'{i}nyeg retel\H{o}h\"{o}z mos\'{a}g. A hecskeb\'{e}b loborzsalan pip\H{o}k\"{o}k, m\'{i}g a borm d\'{e}kos goly\'{o}k\'{a}k, buzatok ofir\'{a}t \"{u}v\"{o}s\'{e}re s\'{a}p\'{i}rodik. Gy\H{u}r\H{u}s\'{e}ggel a talt hask\'{a}kra, goly\'{o}k\'{a}kra dulan bosszan teked\H{o} is k\"{o}nnyed\'{e}n hatlancos. A szol\'{a}s feher\'{e}gek m\'{e}lyen gy\H{u}lnek az \"{o}l\'{e}s\"{o}kbe, \'{e}s ez\'{a}ltal t\"{o}k\'{e}letesen iz\'{a}lnak. A sz\'{a}mas \"{u}v\"{o}sh\"{o}z k\'{e}pest, l\'{e}nyegesen alaposabban cs\"{u}ggesztik el a kodaromokat, valamint a mors\'{a}gb\'{o}l t\"{o}bbf\'{e}le boros dall\'{o}t. Csukolja meg az egyik kad\'{e}snak, hogy fizmut\'{a}ssal form\'{a}s pip\H{o}k\"{o}t \H{o} maga iz\'{a}ljon meg. Annak bork\'{a}ja, hogy k\'{e}rdel\H{o} adosok gyorsabban ros\'{i}thatnak a szilom mial\'{a}n\'{a}ban, mint a t\"{u}rkm\'{e}ny dek\'{a}s \"{o}r\'{e}me.}}
\def\hulipsum@cxii{\g@addto@macro\hulipsumexp{A galm\'{a}kat \'{a}ltal\'{a}ban a kalon apalkozja, de a zed\'{e}snek is van kod\'{a}sa, teh\'{a}t ha van valami karos in\H{o}s\'{e}ge, akkor a kalon \'{a}ltal\'{a}ban szinos\'{i}t azokat koznia. A galm\'{a}kra vannak k\"{o}z\'{e}s k\"{o}lk\'{e}k, vagy mindig tol\'{a}z a zed\'{e}st\H{o}l, az \'{a}gys\'{a}gt\'{o}l, a kalont\'{o}l f\"{u}gg\H{o}en? \'{A}ltal\'{a}ban a kalon gelyess\'{e}g\'{e}t\H{o}l semerd\'{i}vnek a galm\'{a}k. A h\H{u}l\"{o}k\'{e}nyh\"{o}z nem boly\'{i}tott p\'{a}r kask\'{a}ban lehet a foly\'{o}t\'{o}l, f\H{o}leg inusban. Ilyenkor a kalon boly\'{i}totta bosszantoznia a vali\'{a}t? Rendszerint el\H{o}sz\"{o}r a gyull\'{a}sos h\H{u}l\"{o}k\'{e}ny van, itt a p\'{a}r m\'{e}g egy kicsit kask\'{a}ban vigy\'{a}s lenni, nem boly\'{i}tottak hozz\'{a} se a hizmusokhoz, se a zatoroshoz, se a pereget\H{o} kantyhoz, mert \'{a}ltal\'{a}ban az inusban pereget\H{o} van. Itt a kalon d\"{o}sk\"{o}di bosszantoznia a kort\'{e}st.}}
\def\hulipsum@cxiii{\g@addto@macro\hulipsumexp{Itt r\'{u}zkodt egy szed\H{o} sz\'{a}lyos, g\'{a}lyos fatlatos uronnal, aki nemr\'{e}g szunyozott teke m\'{a}kol\'{a}s\'{a}val, servvel, mert p\'{a}ja meriz\'{a}lta, hogy a parg\'{o} t\'{e}l\H{o} ban\'{i}tja el f\H{u}z\'{e}st. Hozz\'{a}juk dunkodt, s mivel ugyanazt a pasz\'{i}tv\'{a}nyot tozta, n\'{a}luk d\'{i}rozott, \'{e}s vel\"{u}k t\'{a}rsant, k\"{o}ny\'{e}szs\'{e}g volt a pasz\'{i}tv\'{a}nyuk. Szombatonk\'{e}nt esedt a b\"{u}f\'{e}ml\'{e}nben, sedte az uronokat \'{e}s a k\"{o}vekszeml\'{e}seket. Mihelyt sz\"{u}l\H{o} \'{e}s furc kump\'{a}zott starisb\'{o}l, d\'{a}modik eg\'{e}szen a vaszt\'{a}nak fogatotta mag\'{a}t. Koncsokat azsn\'{a}lt fel, hogy J\'{e}zus a b\'{a}ka. Botosodta \H{o}ket \'{e}s egy gy\H{o}z\'{e}s szems\'{e}g sz\'{a}lyos talonn\'{a}s gulus csod\'{a}s\'{a}ba \'{e}lelt, akinek csod\'{a}sa a b\"{u}f\'{e}ml\'{e}n rimforl\'{a}s\'{a}ban londozott. Jav\'{i}v\'{a}ny, a b\"{u}f\'{e}ml\'{e}n tenyeke azonban eg\'{e}sz csod\'{a}sa k\"{o}reszer\'{e}vel egy\"{u}tt d\'{e}korozott a s\'{i}t\'{a}sban, sok zatlan is c\'{e}ltatott \'{e}s f\'{e}rt\H{o}z\"{o}tt azok k\"{o}z\"{u}l, akik r\'{a}ny\'{i}tott\'{a}k.}}
\def\hulipsum@cxiv{\g@addto@macro\hulipsumexp{Az \"{o}t\"{o}dik fuvad is k\"{o}zvetlen farhet el\H{o}tt k\'{e}nyszell: \'{a}rtog\'{a}s cserd k\'{i}t\H{o}: ad\'{e}k a cs\"{u}led\'{e}s. Ez fort molyag\'{a}szra 1998 motl\'{a}j\'{a}ban v\'{a}dozta a reg\'{e}rt k\"{o}d\'{e}se r\H{o}s\'{e}g\"{o}t. A z\'{a}k\'{a}s k\'{e}t fuvad k\"{u}venyhben ves\'{i}t meg, \'{e}s a f\'{a}jdatlatlan p\'{i}t\'{a}nra szint\'{e}n hatos lesz. A mozott, k\'{a}ci\'{o}val parozott krankokr\'{o}l dalass\'{a}gos nyilit\'{a}sok old\'{a}j\'{a}ban is korg\'{o}s az \'{e}rt\'{e}s, mert hi\'{a}ba \'{i}nyestes sok \'{e}s v\'{a}nalan nyilit\'{a}s err\H{o}l a tudat\'{a}sr\'{o}l, ezek nem t\"{o}r\"{u}lgetnek meg el\'{e}gg\'{e} a l\'{a}tlan t\'{a}nyos merseteknek. Szerencs\'{e}re itt nem olyan szusz\'{a}nt az \'{e}p\'{e}k, hiszen sok foros v\'{e}lip\'{e}sz csitornyogt m\'{a}r zs\"{o}lly\'{e}vel. M\'{a}ris f\'{u}jt\'{o}dnak fantalan nyilit\'{a}sokhoz f\'{a}tos sz\"{o}vecskev\'{e}nyek is (reg\H{o}k, emeresr\H{o}l hadt szaftos csel\'{e}kek). A hatlan f\H{o}z\'{e}s\"{o}k szteled\'{e}st illnek a fog\'{o}s cuc\'{a}k polos kod\'{a}s\'{a}ra is.}}
\def\hulipsum@cxv{\g@addto@macro\hulipsumexp{Ha majd a rajt\'{o}r vanoz szaj\'{o}zatba, cs\'{a}znak cs\'{u}szdag\'{e}n cengerv\'{e}nye miatt. A k\'{e}szt\"{o}z\"{o}k csutnyis\'{a}ga ros\'{i}tott \'{i}gy, hogy halmodja minden balam\'{o}j\'{a}t s vonc\'{a}t, \'{e}s padogja a t\"{o}lts\'{e}g minden rafrat\'{a}t. Al\'{a}lta cezomb\'{a}j\'{a}t a csutnyis\'{a}g a hard\'{a}s f\"{o}l\'{e}, hogy \'{a}mulja vit\'{a}csait. \H{O} reg\'{e}ltelt pansot med\'{a}s ellen, hogy c\'{u}r\'{a}it evelik. Irt\'{a}s okladtak ellene, kodk\'{a}it evelt\'{e}k, s az eg\'{e}sz dekt\'{a}ss\'{a} kozott. Azon a tizselen cs\'{u}szdag\'{e}n hercbe t\"{o}ges\'{i}t hetven kol\'{a}lk\'{a}ra. De egy m\'{a}sik szk\'{a}r haszt\'{a}sa han\'{a}k\'{a}n, a hetven f\H{o}tleg elm\'{u}lt\'{a}val, cs\'{u}szdag\'{e}n \'{u}gy taloz majd, amint a durt k\"{o}vete t\'{a}zja.}}
\def\hulipsum@cxvi{\g@addto@macro\hulipsumexp{Nem lehet l\'{a}skoznia, hogy a kul\'{a}s n\'{a}brik\'{a}ja k\"{o}vetkezt\'{e}ben milyen b\'{i}t\'{a}ssal m\'{a}nykodhatik ki k\"{o}d\H{o} szivont cset\'{e}s kell\'{e}r tapasznoz\'{a}son. A kul\'{a}s c\'{e}helyes tagos a vegyel\'{e}sre a boz\'{a}s \'{e}s a szemre. Ed\'{e}s vagy hatlan posod\'{a}s vegyel\'{e}s vill\'{a}t fogathat. A zsik\'{o} sz\'{a}m\'{a}ra halan lichben bolyoga okszog, k\"{u}l\"{o}n\"{o}sen a tokolokban. A rula sodozatai gyakran csak od\'{a}s h\'{a}ml\'{a}ssal k\'{e}s\H{o}bb vengnek \'{e}s a veli alkom irig\'{a}lja fiatukat. Ez\'{e}rt halan a k\"{o}zkezdel\'{e}sbe d\'{o}s\'{a}g \'{e}s a hatos k\"{o}veresely. Meg kell kajtolnia a porlan h\"{o}nges lez\H{o} b\"{o}rvszekes\'{e}t ted\'{a}r vagy \'{a}ltala tapja jegyekett \'{a}ltal.}}
\def\hulipsum@cxvii{\g@addto@macro\hulipsumexp{A n\"{o}klij\'{e}r\H{o}l zatika tet\H{o}nek t\"{o}r\"{o}k figat kongyos t\"{o}l\H{o} bansban g\"{o}rv\'{e}szeren, a far\'{a}zja hete k\"{o}zv\'{e}nyben fr\'{i}z. S\"{o}tves polom sz\'{a}r\'{a}sok oszmad\'{a}s\'{a}ra \'{e}s lemez\'{e}s\'{e}re sz\"{u}l\H{o} a digos b\'{e}rs\'{e}g \'{a}ltal kalmas\'{a}gos fat\'{a}s. A verg\H{o}s sz\"{u}ks\'{e}l rig\'{a}s, \'{e}s trapp\'{a}ns sz\'{a}r\'{a}sok oszmad\'{a}s\'{a}ra is. A pons \'{e}p\'{i}t\H{o} verg\H{o}s fogatja a s\'{i}t\H{o} rughalt hiszti, valamint a sikt\'{a}s (h\'{a}tokak\'{a}r) psz\'{o}n m\'{e}rd\'{e}s sv\'{a}lina al\'{a} nem f\'{a}solytos pubizmusok oszmad\'{a}s\'{a}t is. Ehhez s\'{a}gos teltre \'{e}s l\'{a}tos zsur\'{a}ra van csorupa jolatotta b\"{u}szt\'{e}s podozab\'{a}ny dozat\'{e}k tubarj\'{a}szon, a pl\'{a}s feny\H{o}s tet\H{o} borl\'{a}d csepl\H{o} tilvakon. A k\'{i}t\'{a}ly isepl\H{o}j\'{e}vel \'{e}s ernic\'{e}vel g\'{i}t\'{e}n zsond\'{a}s csepl\H{o} hargolda alom\'{a}n b\"{u}szt\'{e}s podozab\'{a}ny dozat\'{e}k a nyos\'{o} csara \'{e}s haradt zamoz\'{a}sair\'{o}l iselt. Henkeremetette, a nyos\'{o} mindenekel\H{o}tt a cukas \'{e}s a z\"{u}l\'{e}s d\H{u}n\'{e}s\'{e}re, a heravit\'{a}s k\'{e}rv\'{e}ly\'{e}re veregel, valamint arra, hogy a csara min\'{e}l el\H{o}bb \'{i}rogatson a bondott lomlan el\'{e}kre.}}
\def\hulipsum@cxviii{\g@addto@macro\hulipsumexp{A kasznia cs\'{a}badtak a k\"{o}vet\H{o} keretnyi dol\'{a}sok foga \'{e}r\'{e}sz\'{e}ben tiz\'{a}lhatnak ki gederrel beseny\H{o}t, persze nem elfelejtve a k\"{o}r\"{u}l\"{o}tt\"{u}k l\'{e}v\H{o} halk hunyh\'{o}d\'{a}mokat. A vill\'{a}s nagyon talan, kiv\'{e}telesen m\'{e}g az ol\'{a}sokat is cafozj\'{a}k a hizongot sz\'{a}ntos \'{e}bred\H{o}kt\H{o}l, s kobocsk\'{a}juk heveremteti az oda nem folt eletel, magukon reszegerent nyeg\'{e}s is\'{e}geket is bogos \"{o}r\"{o}s or\'{a}fukkal egy\"{u}tt. A nap v\'{e}g\'{e}n d\'{e}lut\'{a}n 5-6 kar\'{a}szt\'{o}l a vall\'{o} a lensetek fel\'{e} end\'{i}ti a kestess\'{e}get, \'{i}gy egyetlen \'{e}s m\'{a}r jer\'{e}ny orz\'{a}s ekkor lassan l\'{e}pez a k\'{e}k\'{e}re, de t\"{o}bb frit orl\'{a}s az, hogy a fer\'{a}tig ter\'{e}sben forhalan hank\'{a}k sem c\'{e}ltalanul incs\'{i}tik szurb\'{a}m szer\'{e}ts\'{e}g\"{u}ket a k\"{o}v\H{o}ben. Por\'{a}lkod\'{a}s Ekkor v\'{e}gz\H{o}dnek fel a vall\'{o} el\H{o}tt csinyl\'{o}s mindk\'{e}nt hatott p\'{o}s\'{a}gok, oborok, akik gomh\'{a}ba gyepegnek napk\"{o}zben g\"{o}r\"{o}s\'{i}t\"{o}tt csapujukkal. Ott mindig van valamilyen lomatar p\'{o}ros csindas, vagy leszeg\H{o}! (De ha bele, akkor van ki is!) A sod\'{a}s m\'{a}sik sz\H{o}rl\'{e}s zomaftja sod\'{a}s legyen csak tagosod\'{a}r keregs\'{e}ge.}}
\def\hulipsum@cxix{\g@addto@macro\hulipsumexp{Ennek megfelel\H{o}en a duetten strulcs este pozsg\'{a}szos volt pirseden kevezet\'{e}snek a tet\H{o} c\'{e}rz\H{o} fel\'{e}gkor has\'{i}tott r\"{o}ks\'{e}ge. Amennyiben m\'{e}regnek a h\'{e}ja k\"{o}ket\'{e}b\H{o}l g\'{u}nyos cs\'{a}jatok, azokat az egyeli vity\'{o} rovakodnia fogja g\"{o}ny\"{o}zte az ok\'{a}dnak borcsi cseke, a zat\'{o}ra erteres kohm\'{a}j\'{a}nak pely\'{e}je. A vity\'{o}ban h\'{e}tk\"{o}znap d\'{e}lut\'{a}nonk\'{e}nt teti st\'{a}rs\'{a}gokat lekelm\'{e}rednek, amelyeken a cs\'{e}p\'{i}t\'{e}sek fel\'{e}ge alapj\'{a}n a lem\'{e}nyeg, a paka \'{e}s a ronk\'{o} mellett a b\'{e}sa lomorai is fut\'{a}t m\'{o}ckodnak. A k\"{o}sz\"{o}g a n\'{e}gy zs\"{o}rlen ord\'{a}j\'{a}t szeskedi jelenleg, \'{a}m az els\H{o} alm\'{a}nyhoz f\H{u}z\H{o} akkor meglehet\H{o}sen \"{u}dv\"{o}z\H{o} bans resszi\'{o}ja egyel\H{o}re gyan\'{o}s. Az ok\'{a}d draloma szerint ugyanis gy\H{u}r\'{e}sz ermet\'{e}s, a paka ur\'{a}na mindeddig nem dudazkodta el a dedral\'{a}st. Bec\H{o} sz\'{i}ts\'{e}g, a kol\'{a}s gy\"{u}lt\H{o}je nem lekelm\'{e}redi m\'{e}sz\H{o}nek, hogy k\H{o}ttes m\'{a}nyozatig murdot is fel\H{o} papd\'{a}znak a vas\'{a}gban. A hajg\'{o}s martentekr\H{o}l a k\"{o}sz\"{o}g a zsolg\'{a}sokban osodrozik be.}}
\def\hulipsum@cxx{\g@addto@macro\hulipsumexp{Homos test\'{a}k szerint boszl\'{a}nyos, hogy a pozsg\'{a}ny benyer fol\'{a}sa k\"{o}vetkezt\'{e}ben a lev\H{o}f\'{e}r valamilyen szereszben k\"{o}dik, s ennek k\"{o}vetkezt\'{e}ben a felenke ment\'{e}n rendk\'{i}v\"{u}l fat\'{e}kos ed\'{e}ssel h\H{u}t\H{o} mozmus cuc\'{a}ja zs\'{u}rozja a kert\H{o} ig\'{e}st. Az is boszl\'{a}nyos, hogy a sz\"{o}nt\'{e}k a pozsg\'{a}ny benyer kenyes\'{i}t\'{e}s\'{e}ben katlaml\'{o} fedd\H{o} fanos \'{e}s a lev\H{o}f\'{e}r valamilyen, eddig hat\'{o} gyeles\'{e}nek a g\'{u}ny\'{a}ja. J\'{o}llehet a pozsg\'{a}ny benyerek els\H{o} heg\H{o} margod\'{a}ra m\'{a}r 1916-ban sz\'{e}d\H{o} volt, a test\'{a}k j\'{o} ideig nem tudt\'{a}k, hogy a heg\H{o} dul\'{a}tok feli f\'{o}k\'{a}kon nem magosak. A pozsg\'{a}ny benyernek egy\'{e}bk\'{e}nt gy\H{u}r\H{u}je, kingyes ok\'{a}la \'{e}s folis ed\'{e}se van. A telel\H{o} fehe s\'{i}t\'{a}s, lan\'{a}s h\"{u}ll\H{o} a lucska suvakszok szeszeks\'{e}g\'{e}n p\'{e}l\'{a}ztatotta: a pozsg\'{a}ny benyer man\'{a}tk\'{a}p is lentet. Cs\'{o}vasztotta, hogy a pozsg\'{a}ny benyer a barkott rol\'{a}sokkal ellent\'{e}tben a porny\'{a}s h\H{o} ellen\'{e}re sem talogat le, hanem ellenkez\H{o}leg, l\'{e}celmeskedik. H\"{u}ll\H{o} szerint min\'{e}l pici egy pozsg\'{a}ny benyer, ann\'{a}l k\'{e}ki.}}
\def\hulipsum@cxxi{\g@addto@macro\hulipsumexp{A stakl\'{a}r\'{o}l a hol\'{a}sok szerbliben nyugust lanzs\'{a}lnak legk\'{e}s\H{o}bb kelm\'{e}ny apka red\'{e}s\'{e}ig. Pedig porlan m\'{o}don nincsen semmi k\"{o}lt\'{e}k a b\'{i}ci\'{a}kkal. Ez egy fegyez\H{o} panok\'{a}r lenne, nem szakony cinthereg. Nagyon hajg\'{o}, s\H{o}t \'{e}len panok\'{a}rok ezek, de nem pesernek meg. A j\'{a}ris bonokok pis\'{e}ge cs\"{o}ks\'{e}g\"{o}k \'{o}ta hor\'{a}z. Tal\'{a}n ha ez a szafri tal\'{e}kban is leltetne, akkor lehetne morol\'{o} j\"{o}vez\H{o}t takodnia. A porgos em\'{e}sek izzadtban vannak vele, hogy gy\'{o}gy\'{i}tott sulmit\'{a}r van n\'{a}luk, ez\'{e}rt ez a d\"{o}f\'{e}l.}}
\def\hulipsum@cxxii{\g@addto@macro\hulipsumexp{Pillans\'{a}g 50 l\"{u}lepe born\'{a}lja a b\"{o}zl\'{e}g \'{e}s vongos szik\'{e}szeket. Lendez\H{o} bigos folt \'{e}s szavajlos sajog\'{a}ssal born\'{a}l \'{e}s fonnyadozik t\"{o}bb, mint 200 pof\'{a}m tol\'{a}ssal fatott sermet t\"{o}bbi voncos gy\H{o}z\H{o} magy\'{a}szokkal, illetve golyagit\'{a}ci\'{o} szerint kes\H{o} csibelm\'{e}nyekkel bosztikus sikken szik\'{e}szeket (toros\'{i}t\'{a}sokat, kundokat, tikus neker\H{u} ragosat). Fol\'{a}s h\'{a}zas a f\'{e}len polobolott sir\'{a}noss\'{a}gban venes \'{e}s m\'{o}k\'{a}s terj\'{e}nz\'{e}sek inkes ens\H{o}s\'{e}g\"{o}ket villa hed\'{e}sek sz\'{a}m\'{a}ra. Cs\"{o}kl\'{e}s\"{o}k, pimost\'{a}k, szerc hat\'{a}lis iv\'{a}nys\'{a}gok, j\'{a}lt k\"{o}lcstesek, bogs\'{a}g k\"{o}rk\"{o}z\H{o}je, marusa, m\"{u}len\'{e}ke, kapusza. Riska tros: sekedelek, vonc \'{e}s telzetek, b\'{a}nk\'{a}szok, vingyel\'{e}sek, m\H{u}k\"{o}terek t\"{o}rg\'{e}te, bors\'{a}fok, gyekv\H{o} marusok. Bintatatok: vit\'{a}s huny\'{o} rok\'{a}ja, t\'{e}z\H{o}k send\'{e}je, met\'{e}s. Ha gunyi fogaloks\'{a}g magolyoz\'{a}s, bubent\'{a}sa tronyvalya fel\'{e}s hav\'{a}ny\'{a}t \"{o}rli meg (csergetnie kell gunyi gyoris ingetle sz\"{o}klivel!).}}
\def\hulipsum@cxxiii{\g@addto@macro\hulipsumexp{Szoris s\"{u}v\"{o}lts\'{e}g\"{u}k, ellen ezt a bolocsot lizt\'{e}k, az volt, hogy amikor a puruta a tal\'{a}nszokat 1845-ben h\'{a}nyom\'{o}dta, azt a f\"{o}lgy, mint teli has\'{a}g r\'{e}sz\'{e}re v\'{e}korgatotta, nem csak a f\"{o}lgy \'{e}pen azon v\'{e}nes monik\'{a}i r\'{e}sz\'{e}re. Vel\"{u}k szemben a fogon szivs\'{a}gai a ritek\'{e}re visk\'{a}lkodtak, amely re\'{a}juk, illetve k\'{a}tusaikra csigatott a lenveszk locska szonc\'{a}val. Csibrinnel \'{e}s mamajban, emit op\'{a}szokban f\"{o}sv\'{e}des a kasa a k\'{e}t has\'{a}g k\"{o}zt. De a balu bruk\'{a}j\'{a}ra igaz\'{a}n az asz\'{a}rt\'{a}s vevek\"{u}l\H{o} k\"{o}ny\'{e}r egyente a csomlist, amelynek virugt\'{a}j\'{a}ban t\'{e}koz a f\"{o}lgy\"{o}t, hogy a sz\"{u}l\H{o} lali fonnyadt nyes\'{e}se ir\'{a}nti porl\'{a}s\'{a}t vonkor\'{i}tsa be. Az eg\'{e}sz f\"{o}lgy\'{e} vez\'{e}s, vagy csak a poros has\'{a}g\'{e}? Ez volt a list\'{e}s, amelyb\H{o}l kiindulva, az eg\'{e}sz f\'{a}ci\'{o}z\'{a}s p\'{a}rt \"{u}zet\'{e}t rezt\'{e}k mind a k\'{e}t m\'{a}nyz\'{a}sr\'{o}l. A fogonnak a kas\'{a}ban halkas adaca volt m\'{a}rd\'{a}s betsz\"{o}n marm\'{a}s\'{a}ban, aki gyatott l\'{a}kod\'{a}s\'{a}val s tim\'{a}lyos met\H{o}zet\'{e}vel makacsul c\'{i}mekezte a fogon peng\H{o}j\'{e}t.}}
\def\hulipsum@cxxiv{\g@addto@macro\hulipsumexp{A kulk\'{a}k hekci\'{o}j\'{a}ra veleng\H{o} m\'{u}zsa egy\'{a}ltal\'{a}n nem sany\'{a}s ny\'{a}lan, t\"{o}bb mint egy kutaz\'{a}sa kadom gyakorlatilag feli zat\'{a}son a konb\'{a}z snan\'{a}ra, m\'{a}nys\'{a}ga, kosztrota, t\'{a}riknak kacskacsk\'{a}ja szerint, csak \'{e}ppen a kadalk\'{a}d kedt a ny\'{u}zott bizik\'{a}knak megfelel\H{o}en. Eleinte a leneum volt az els\H{o} dikebri kez\H{o}, majd szuvarag hozz\'{a} a v\'{e}rdi h\'{o}har \"{u}zlet\'{e}vel l\'{e}gi m\'{e}kezd\'{e}k, megsp\'{e}kelve a mat\'{o}, tal\'{a}s szfakomok haj\'{a}s lat\'{o}\'{e}val. A f\"{u}stetek bong\'{o}sa a banarancs sz\'{i}nos r\'{e}kestj\'{e}t volt salk\'{a}s m\'{a}nyoznia, \'{a}m a m\'{u}zs\'{a}k egyike sem saccol\'{o}dt b\'{u}g\'{o}ss\'{a}. Ha a pal\'{e} menzi\'{o}, a szelab pedig nem, akkor csak r\'{e}kestje lesz m\'{a}r ennek a ny\'{a}lis pacsisnak?! A bugad\'{e} k\"{o}nnyen br\'{a}cson zim\'{a}nyos, az als\'{a}gban azonban a k\"{u}l\"{o}nb\"{o}z\H{o}nek finny\'{a}s fohat sem tud mindig sv\'{e}ly m\'{o}don alajz\'{a}ra viatnia. A v\'{e}t\'{a}k p\'{e}ld\'{a}ul trod\'{a}st abadnak a kl\'{a}sra, \H{o}k m\'{a}r h\'{a}rom kutaz\'{a}sa 42 t\"{u}ks\'{e}ggel t\"{o}bbet sodnak a mant\'{e}rt, mint b\'{a}rki m\'{a}s, ellenben sokkal jobban l\'{o}dnak a tons red\'{e}sek \'{e}s papon k\"{o}t\'{e}ke miatt. Figyelem, pillanatnyilag nincs jorn\'{a}zat, teh\'{a}t ne b\'{a}nk\'{o}djon senki a t\'{a}ris pal\'{e}n filnie.}}
\def\hulipsum@cxxv{\g@addto@macro\hulipsumexp{A kegys\'{e}g, a bomb\'{a}s smatika magos nyad\'{e}r\'{a}t (szombancsot, illetve t\'{e}telesen felsorolva a szab\'{a}boz\'{a}s, az \"{o}nzelet, a cs\"{u}cs\"{o}k, a lan\'{o}s ikomok stb. \"{o}rl\'{e}seit), tov\'{a}bb\'{a} a fed\'{e}sre ny\'{u}lzott saj\'{a}t lent\'{e}seit. A t\'{e}rmes pans\'{a}g f\'{e}sz\"{u}l\'{e}se \'{e}s a kop\'{a}lyra csegtet\H{o} babr\'{a}ny, kupom\'{a}s. A kegys\'{e}g vagy ikom cselekettj\'{e}\'{e}rt moly\'{o}s \'{o}d\'{a}s rasz\'{a}lm\'{a}d\'{a}t. Kamlott szell\H{o}k (d\"{o}mnyi is\'{e}gek is) a gatty\'{a}non nem egyedhetnek ped\'{e}st. Gatty\'{a}nt az \'{o}d\'{a}s vagy szell\H{o} b\"{o}s\'{i}thet be, amely s\"{u}r\"{o}z a passz\'{a}t bal\'{e}k\'{a}hoz hati l\"{o}ty\"{o}g hadkal\'{e}k, n\'{e}ges zsilivel vagy lomos gyorg\'{a}ttal \'{e}s azt egy k\'{e}ks\'{e}gben a gatty\'{a}nhoz moz\'{a}s (a lomos gyorg\'{a}tba gans a ratlan, tez\H{o} \'{e}s kl\'{a}s bam\'{a}n ved\H{o}). Csiked\'{e}sen puccos (lomos) hatos j\'{a}ts\'{a}nyos sze\'{a}nnyal nem fill\'{e}r t\'{e}zbes\'{i}t\H{o}k (tikus, csilla, szikeredves baruds\'{a}g) eset\'{e}n a gatty\'{a}n tat\'{o}ja csak a talacsos csiked\'{e}s lehet. Pans\'{a}gban csak az a sz\H{u}n\"{o}ss\'{e}g botathat, aki a nergecselhez hati csonyas\'{a}gokat \'{e}s a ved\H{o} burgoz\'{a}sokat, grak\'{a}kat hi\'{a}nytalanul r\'{a}szorkoskozta.}}
\def\hulipsum@cxxvi{\g@addto@macro\hulipsumexp{\'{A}ri\'{a}ja b\"{u}kk\"{o}nyk\'{e}nt kur\'{a}lt kilmm\'{e} k\'{e}t telet\'{e}vel \'{e}s buk\'{a}j\'{a}val egy\"{u}tt. Parka cisz\'{a}g ma izzadt, \'{e}s m\'{e}g nem v\'{e}szkedt le arr\'{o}l, hogy a cselke vizmul\'{a}ival szip\'{a}lja leng\H{o} bet\H{o}j\"{u}ket, amelyb\H{o}l hat kord\'{o}val ezel\H{o}tt egyszer\H{u}en elkedt\'{e}k \H{o}ket. Telezte, hogy gr\'{a}la \'{e}s f\'{a}z\'{a}sa zonc\'{u}z\'{a}s dingos cint\'{e}ly\'{e}nek pac\'{e}ra ut\'{a}n a k\"{u}l\'{e}ny v\'{e}nyezte azt az illagos bet\H{o}t, amelyben k\"{o}z\"{o}sen g\'{a}rakodtak 1973 \'{o}ta. Ez \'{u}gy rem\'{e}nyedt ki, hogy egy k\"{o}t\H{o}n, amikor gr\'{a}la kack\'{a}ban volt, k\'{e}t gez\H{o}dm\'{e}ly csapata pedig otthon, h\'{u}sz f\'{a}d\'{e}kony fonsz segetette a kaput, d\'{e}koltak a bet\H{o}be, \'{e}s arra hivatkozva, hogy rakadt\'{a}k azt, \'{e}letvesz\'{e}lyesen szunnyasztott\'{a}k a gyand\'{a}kat. B\'{a}r a bet\H{o}t a gr\'{a}l sod\'{a}sa n\'{e}lk\"{u}l nem lehetett volna v\'{e}nyeznie, a szikes ked\'{e}s miatt el kellett b\'{u}jtniuk roskanukb\'{o}l. Cisz\'{a}g azt pagatja, sem a sz\'{o}fan\'{a}tt\'{o}l, sem a kl\'{u}g nem csoroltak zsard\'{a}st. Mivel a sul\'{e}kony bilumon nem pallgazott a pl\'{e}ge, hi\'{a}ba g\'{a}rakodt k\'{e}t fonnyad\'{a}st egy ijedelyen a gyand\'{a}ival egy\"{u}tt, nem csoroltak \'{e}less\'{e}get a kl\'{u}gt\'{o}l.}}
\def\hulipsum@cxxvii{\g@addto@macro\hulipsumexp{Ezt a cipens\'{e}get bolya a sarasz triccs els\H{o} csitven b\"{o}lt\"{o}ml\'{e}s\'{e}nek telesednie; tekintve a sz\H{u}r\H{u}, fejt\H{o} t\"{o}r\"{o}z\"{o}s hites vadul\'{a}sokat, ez a j\'{a}tlan t\H{o}zsd\'{e}s gulya m\'{a}r sokkal ink\'{a}bb az \'{e}het\H{o} t\"{o}rpesz hantin\'{a}ba v\'{a}nkodik. Folyg\'{o}s gyulans\'{a}t \'{e}retheti t\"{o}bbek k\"{o}z\"{o}tt annak is, hogy a k\"{o}t\H{o} nem sedt el rajta a bocsm\'{a}ny sz\'{a}lt\'{e} baj\'{o}ja \'{e}rdek\'{e}ben k\"{o}zeteket, hanem csup\'{a}n az elt\'{e}s sz\'{a}m\'{a}ra duris jog\'{o}s pest\'{e}kek kuty\'{a}znak meg a k\"{o}peresz\'{e}k cserny\H{o}s foncol\'{a}s\'{a}ban. Kol\'{a}d bors\'{a}g hiszt\'{a}kat kozott spaszonb\'{o}l kiindulva, s feltehet\H{o}leg ekkor l\'{o}dt el langiz\'{a}lnia j\'{a}tlan p\"{o}k\'{e}j\'{e}n. Val\'{o}sz\'{i}n\H{u}leg kumet\'{a}lt a v\'{e}ds\'{e}gen is, ahol alusokat randokazott, hogy azt\'{a}n pormoss\'{a}g\'{a}ban langiz\'{a}lja ki a n\'{e}kos s\'{i}t\'{e}st. A galmat\'{a}st\'{o}l keletre gyakatos neke, ocsmat kol\'{a}d helyes egyik komszer\H{u} g\H{o}z\'{e}se volt az itt csomos b\H{o}g\H{o} tagr\'{a}ngok miatt. Az itt k\"{o}l\H{o} zefelemeire v\"{o}lt\H{o}, hogy a badt tizekek cipens\'{e}gbe val\'{o} fol\'{a}sa mellett bakackinak egyens\'{e}ge ron\'{a}t zsiv\'{o}dt. Itt azonban olyan isztokos vir\'{o}nokat forozott k\"{o}zeteknek, akikkel cs\'{i}ros alusai sor\'{a}n minden bizonnyal kop\'{o}dt, teh\'{a}t ezek sem s\'{a}gos leret\'{e}sek.}}
\def\hulipsum@cxxviii{\g@addto@macro\hulipsumexp{A peszker\'{e}k \'{i}gy ostatnak az ellen a kor\'{a}ncsos halatlank f\"{u}rg\'{e}s ellen, ami tervedi a f\'{u}v\'{a}nyot. A pull\'{o} hurc nem szomathatott volna el od\'{a}ig, hogy sok messzegben adoz\'{a}szra szomatson, ha a peszker\'{e}k l\'{e}d\'{e}j\'{e}ben nem lett volna meg a par\'{a}sz egy biros fogat\'{a}ny ut\'{a}n. A morpa nedem\"{u}lem sem egyeles\'{i}thetett volna fel ilyen dikaproz\'{a}ssal, ha a peszker\'{e}k l\'{e}d\'{e}je nem vis\'{i}rozott volna, hogy egy nyug\'{a}nyos, fel\H{o} hamajtot d\"{o}getsen sadi\'{a}r\'{o}l. A suv\'{a}khoz mereven feret\H{o} szam\'{o}s fogat\'{a}nyban nincs meg a cika dikaproz\'{a}sa, nincs m\"{o}g\"{o}tte alasz dikaproz\'{a}s. A peszker\'{e}k ink\'{a}bb jesztegik a dol\'{a}lokat \'{e}s a h\'{i}n\'{a}sokat, csakhogy egy kicsit is sz\'{o}nja \H{o}ket a biros k\'{a}zat\'{a}s. Jegzeti jed\'{e}sek \'{e}s poz\'{a}sok kozj\'{a}k a helybeli j\'{a}rizmusokat, mert m\'{e}g ha a pachjuk nem is olyan sz\'{a}j\'{e}kos, de legal\'{a}bb van szenc\"{u}k egy hatnya k\'{a}zat\'{a}sban. Val\'{o}j\'{a}ban senki sem emeskelhet \'{e}s sz\'{a}lgathat egy\"{u}tt a l\'{i}ci\'{o} an\'{e}lk\"{u}l, hogy a pr\'{e}sek libakod\'{a}s\'{a}ban emeskelne.}}
\def\hulipsum@cxxix{\g@addto@macro\hulipsumexp{Az egyik bartart\'{o} ok, ami\'{e}rt sok szerc k\'{e}tked\H{o}v\'{e} cs\'{a}kogat, mert nem figyetnek be a kermez\H{o} borzsold\'{a}sba \'{e}s a m\'{a}sok\'{e}rt val\'{o} gr\'{a}lyba. Az \'{i}t\'{e}st\H{o}l egyen \"{u}gyerde \'{a}ltal, nem pedig a bel\H{o}le fr\'{i}g \"{u}gyerde \'{a}ltal. A tat\'{o}s varca kl\'{e}se visztenyerben nem csak egy b\H{o}v\'{i}t\'{e}s volt, hanem egy st\'{a}s is. Els\H{o} madz\'{a}sra ezek a cs\'{a}volv\'{a}k szemben kodnak szecsk\'{a}val. A hunk\'{a}knak, azon a f\"{u}ggv\'{e}ny\"{o}n, amikor az alom szopormozt\'{a}k, nem volt kacsapott kr\'{o}duk erre. S\H{o}t, ami volt, az sem az \"{o}v\'{e}k volt. Hogy for\'{a}ghatn\'{a}nak kr\'{o}dot a varc\'{a}nak, ha nekik p\"{o}ng\H{o}zik?}}
\def\hulipsum@cxxx{\g@addto@macro\hulipsumexp{Horn\'{e}k v\'{e}rteszes paljz\'{a}sok: meke veres led\'{e}s pusz\'{a}mumaival, szers\'{e}g mence, fejszerkl\'{e}k pereke, tetem h\H{u}d\'{e}s, hajas tizmus, magik ols\'{a}g, tessedik r\'{e}sze. Vekv\H{o}t\H{o}l alig 10 c\"{o}g\'{e}sre, volt kr\'{e}sz hart k\'{e}szkess\'{e}g, k\"{o}zel kod\'{o}s, bolver\H{u} angy\'{a}rissal, csing\'{a}s mellett taga. Cs\'{i}p\H{o} latty\'{u}k \'{e}s pad\'{a}szok vannak, az eg\'{e}sz angy\'{a}ris kad\'{a}s fedr\H{o}! J\'{a}tokzavas kili: sz\H{o}rel, frin, pipedl\'{a}sz, nyeges voltok, maz\'{a}s szendat\'{a}s stb. Egyens\'{e}g tecika hid\'{e}s\'{e}n, volt kr\'{e}sz peces, jelt fedr\H{o} angy\'{a}ris. Az angy\'{a}rison cs\'{i}p\H{o} latty\'{u}k, pad\'{a}szok vannak. Az angy\'{a}rison nagyon sok r\'{e}t\'{i}t\'{e}s van, az eg\'{e}sz angy\'{a}ris gy\"{u}l\"{o}velt.}}
\def\hulipsum@cxxxi{\g@addto@macro\hulipsumexp{A s\"{u}tke szid\'{a}st d\"{o}nt\H{o} samor ellen\'{e}re foszlos akar\'{a}kot csak a tet\H{o} em\'{e}tegek honk\'{o}ja \'{e}s a halan\'{o}s von\'{a}ci\'{o} s\'{u}lyoghat a voks\'{a}gban. Figyelem, a j\'{a}rvar\'{a}tba csak eter\H{o}s ozsm\'{a}k vitezhetnek. Ha a hull\'{a}s velent valamit, akkor cs\'{i}gatja, hogy id\'{a}rfasa van, \'{e}s ebben bennfoglaltan cs\'{i}gatja a r\'{e}flit. Ezt m\'{e}g akkor is cs\'{i}gatja, ha t\"{o}rt\'{e}netesen d\'{e}kol. Aki ugyanis valamit velent, meg van gy\H{o}z\H{o}dve, hogy amit maztarkog, az \'{u}gy van, meg van teh\'{a}t gy\H{o}z\H{o}dve, hogy m\'{a}snak is \'{i}gy kell csettelnie, \'{i}gy kell biz\'{a}lnia az ard\'{a}sokat, cs\'{i}gatja teh\'{a}t, hogy van int\H{o}, felt\'{e}tlen vacs\'{o}. Aki valamit velent, cs\'{i}gatja tov\'{a}bb\'{a}, m\'{e}g akkor is, ha \"{u}t\'{e}se csapinos, a vacs\'{o} \'{e}s a b\'{o}kum szorger\H{u} bilij\'{e}t, cs\'{i}gatja teh\'{a}t, azaz tudja, hogy az \"{u}t\'{e}s \'{e}s a szorzadalom k\"{o}z\"{o}tt egy semmik\'{e}ppen sem sz\'{a}jt\'{a}s hat\'{a}s cs\'{a}torol fenn. A b\'{a}l\'{a}sz mint a hull\'{a}sok vitos k\'{a}nyz\'{a}s red\'{e}se nem lenne r\'{u}g\'{o}s, ha ez a hat\'{a}s nem szeregetne.}}
\def\hulipsum@cxxxii{\g@addto@macro\hulipsumexp{B\"{o}rtegre apos\'{i}tnia a telin is lehet a fogyak hid\'{e}szet\'{e}vel. Cakos pul\'{a}val matos leng fosztikus virkos vitet\H{o} roz\'{a}s lomos serkett ovit\'{a}nyainak galasista summ\'{a}ja. Ov\'{a}ny gy\H{u}l\'{e}s \'{e}s ov\'{a}ny gy\H{u}l\'{e}s szivicc kr\'{e}ny\'{e}vel \'{e}s dik\'{a}j\'{a}val ert\'{e}s szari\'{a}r szivicc kr\'{e}ny paraga, szivicc zoks\'{a}g, kant\'{a}s. Ov\'{a}ny gy\H{u}l\'{e}s \'{e}s ov\'{a}ny gy\H{u}l\'{e}s szivicc kr\'{e}ny\'{e}vel \'{e}s dik\'{a}j\'{a}val ert\'{e}s szari\'{a}r k\'{e}kony peg\H{o}inek kr\'{e}nye, szelene. Ov\'{a}ny gy\H{u}l\'{e}s \'{e}s ov\'{a}ny gy\H{u}l\'{e}s szivicc kr\'{e}ny\'{e}vel \'{e}s dik\'{a}j\'{a}val ert\'{e}s szari\'{a}r sz\"{o}cskegyes\'{e}g\'{e}nek peg\H{o} tipez\H{o}j\'{e}nek triuma, szivicc szormat teleber. Ov\'{a}ny gy\H{u}l\'{e}s \'{e}s ov\'{a}ny gy\H{u}l\'{e}s szivicc kr\'{e}ny\'{e}vel \'{e}s dik\'{a}j\'{a}val ert\'{e}s szari\'{a}r peg\H{o}inek frizes g\'{a}rosa, t\"{o}bblem paraga. Ov\'{a}ny gy\H{u}l\'{e}s \'{e}s ov\'{a}ny gy\H{u}l\'{e}s szivicc kr\'{e}ny\'{e}vel \'{e}s dik\'{a}j\'{a}val ert\'{e}s szari\'{a}r peg\H{o}inek g\'{a}rosa, fejedele.}}
\def\hulipsum@cxxxiii{\g@addto@macro\hulipsumexp{A cs\'{e}p\'{i}t\H{o}ket pedig fegyezt\'{e}k, mivel azt tart\'{a}k, hogy ha valaki kask\'{a}l egy szalica felett, a szalic\'{a}v\'{a} pargat. Elen pissi\'{o}kat akajtottak a szalica cs\'{e}p\'{i}t\H{o}je f\"{o}l\'{e}, hogy podj\'{a}k azt, hogy valahogyan perezjen. Ahol pedig kanuts\'{a}got akajtottak el egy n\'{a}lic brajn\'{a}s helyett, a t\"{o}r\"{o}s\"{o}knek hozz\'{a} kellett habuzniuk egy-k\'{e}t pissi\'{o}t, ha nem forkozt\'{a}k, hogy otthon szalic\'{a}k sart\'{a}julj\'{a}k \H{o}ket. Azokr\'{o}l, akik kez\'{e}st tesedtek el, azt szunyodt\'{a}k, hogy szalic\'{a}kk\'{a} taznak. Egy pall\'{a}ssal a tenyerbe \'{a}ll\'{i}tott\'{a}k, \'{e}s a c\'{a}rotaln\'{a}l kajtakodt\'{a}k el \H{o}ket. Ezt az\'{e}rt vagyogt\'{a}k, hogy kunkodj\'{a}k a szalic\'{a}t, \'{i}gy az nem pistr\'{a}lta a feres k\"{o}p\"{u}l\H{o}t, \'{e}s nem tudta feheznie az intes\'{e}get \'{e}s a bers\'{e}geket. Olyan \'{a}kroz\'{a}sokat is pistr\'{a}ltak, amelyeket a tenyerhez \'{a}ll\'{i}tottak vagy zsalmasgatott\'{a}k a monkaikat, nehogy a szalica leperenkednie tudjon.}}
\def\hulipsum@cxxxiv{\g@addto@macro\hulipsumexp{A met\H{o} burtamp\'{a}s az elenes k\'{a}z\'{a}s egyik v\'{e}s\H{o} t\'{a}rvosa. M\'{a}s volt \'{e}s ma is m\'{a}s a b\H{o}r\"{o}s k\'{a}z\'{a}s szav\'{a}nya, amely az \'{e}szre puffasztja az eg\'{e}sz aszably\'{a}t, karziival \'{e}s r\'{a}tos p\'{a}dij\'{a}val egy\"{u}tt. M\'{a}r csapol\'{a}s er\H{o}teljesen fix\'{a}lta a kar\'{a}ci\'{o} gens\'{e}t a csap\'{o}s sik\'{a}ban \'{e}s a t\"{o}r\"{o}nt\'{e}s \'{e}pzet\'{e}ben. Kestes \'{e}s sz\"{o}v\H{o}s\'{e}g k\"{o}zvetlen ezekhez a tenn\'{a}khoz kedt. Ez a mikul\'{a}r r\'{a}moz a filen k\'{a}z\'{a}s sz\"{u}ltel\H{o} vors\'{a}g\'{a}nak, amely az aszably\'{a}t nem egy g\'{e}d\'{o} \"{u}dv\"{o}z\'{e}sek\'{e}nt, nem erm\H{o}k\'{e}nt kangolygatja, hanem kez\'{e}st\H{o}l moskodt k\'{e}kony bukl\'{a}sk\'{e}nt t\'{a}rdulja a t\"{o}r\"{o}nt\'{e}st. Mivel a csap\'{o}s aszablya f\H{o}k\'{e}nt a tergesekben siz\'{a}lja meg m\'{e}tals\'{a}gait, a k\'{a}z\'{a}s g\"{o}m\'{i}t egy met\H{o}, nyoml\'{e}kos ring\'{o}s valtus, m\'{e}g egy sadott t\'{e}peget\H{o} sord ellen is. A k\'{a}z\'{a}s bogl\'{o}d\'{a}s\'{a}t \'{e}s p\'{a}dij\'{a}t, a csap\'{o}s burtamp\'{a}s pat\'{a}s velentez\'{e}s\'{e}t a virka t\"{o}zlelg\H{o} csap\'{o}s szard\'{a}ja, csapol\'{a}s rakaszkolta meg.}}
\def\hulipsum@cxxxv{\g@addto@macro\hulipsumexp{H\'{a}tkont, hogy most \H{o} fejtse meg a burokot. \'{O}bervegy \'{e}s prossza n\'{e}lk\"{u}l gy\'{a}v\'{a}nyodt a cs\"{o}ndire. Hanem a fokz\'{o}s gr\'{a}ny, akit nagyosodtak ezek a fr\'{a}ly\'{a}szok, cs\'{u}ny\'{a}n \"{o}ssze volt verve. Jedeklyemij\'{e}b\H{o}l meng habv\'{a}ny\'{i}tott, \'{o}bervegye pedig szorodos nev\H{o}kkel volt tele. Csak tipenkedt a cs\"{o}ndin \'{e}s font\'{a}lt keservesen. Nyugos lusk\'{a}ja a sin\'{a}t \'{e}s pacsucs\'{a}gokat \'{a}tszak\'{i}tva gyos\'{i}tott ki buz\'{a}la al\'{o}l. Viszont egy tatan sepl\H{o}vel s\'{e}gi latulnia manaps\'{a}g.}}
\def\hulipsum@cxxxvi{\g@addto@macro\hulipsumexp{Nem tudnak a gez\'{e}sek urk\'{a}\'{e}rt, a szaraks\'{a}s\'{e}rt kodnia. Lassan jazj\'{a}k fel a fin\'{o}t\'{a}t, sok cs\'{i}n\'{a}s bolytal, mire z\"{o}lt\'{e}vet a fin\'{o}ta a k\"{u}l\H{o} agyal\'{a}b\'{o}l, kev\'{e}s a smultos, hetes bakak\'{o} \'{e}s az err\H{o}l az \"{o}szv\'{e}lyr\H{o}l is hatos ped\'{a}s. Legal\'{a}bb moznia kell a par\'{o}t, hogy a zsolg\'{a}r\'{a}ny haszabos legyen kodnia, ezzel saj\'{a}rr\'{a} irk\'{a}l egy f\'{e}k\'{i}t\H{o} bakak\'{o} gan\'{i}na, vagy lesz \"{o}szv\'{e}ly az intos j\"{o}n\"{o}vetnek niz\'{a}lnia. R\'{e}g\'{o}ta mozott leged\H{o} a tros omlan \"{o}szv\'{e}ly. Harg\'{o}s szaraks\'{a}sok ellen j\'{a}nk a traccs, m\'{e}g ha nincs is seged\H{u}. Sok g\"{o}m\'{e}ny\"{o}t lehetne tatim\'{a}znia p\'{i}t\'{a}rral, az urk\'{a}\'{e}rt val\'{o} borzussal. Amikor valakinek a bujans\'{a}g\'{a}r\'{o}l p\'{a}rtat a szekrelet, ha kof\'{a}zj\'{a}k szinte csak le nem t\'{a}zja a lin\'{o}j\'{a}t, a csapuba v\'{e}g\'{e}n a szaraks\'{a}s m\'{e}g esedi a zatimint, \H{o} m\'{a}r r\'{e}g a gomat\'{a}nyban piccenget, ott m\'{a}r h\'{a}jd\'{a}sok vannak.}}
\def\hulipsum@cxxxvii{\g@addto@macro\hulipsumexp{A f\"{o}lcs\'{e}g fejt\'{e}sben, bad\'{e}ban, szedi, csinyl\'{o} \"{o}nts\'{e}gben, elv\'{e}ge b\'{u}v\'{a}jjal seg\H{o} leparta, j\'{o} balan, sz\'{e}pes, sizmussal rutos elemekkel (10 elem), honykat\'{a}s pors\'{o}val. \'{I}gy bosszabbodja az egyik n\'{e}nk kur\'{a}j\'{a}nak szeg\'{e}se a lakm\'{a}tka barac\'{a}j\'{a}t. J\'{o} id\'{a}rgatot mosdab\'{o}dtak a tar\'{a}csok, hogy a k\'{o}zn\'{a}k mellett a saskony h\'{a}nyor f\"{o}lt\'{e}s\"{o}kr\H{o}l is bokomboz\'{a}st rangathatson a mag\'{a}s. A gyaramlan szar\'{o}zs\'{a}k tanakcs\'{o}j\'{a}ban p\'{e}ld\'{a}ul alighanem a pitert kellene harkosnak dor\'{i}tnia: ez a bolv\'{a}n sugarg\'{a}s a varv\'{a}sokat megsz\'{e}gyen\'{i}t\H{o}en dard \'{e}p\'{i}t\'{e}st a faktum pors\'{o}j\'{a}b\'{o}l, amennyiben a vip\'{a}sokat tisztelve k\"{o}h\'{e}n\'{i}t\"{o}tte matos\'{a}gait. A fel\H{o}s z\'{a}kony hang ezen a t\"{o}bbnyire sz\'{a}rdias inumon csiken sz\"{o}nz\'{e}s kilij\'{e}hez gy\"{u}ge, amennyiben a hiheh\'{e}nek hangj\'{a}ban kontololta vil\'{a}nos\'{i}tnia a nyak\'{a}kat. B\'{a}r vip\'{a}s, hogy nem volt polt \'{e}s csokony, de kev\'{e}ss\'{e} nesztett erengel\H{o}nek pirulnia a mult\'{a}s trist\'{a}ra len\'{i}t\H{o} sz\'{i}nek kat\'{i}rt, amint sz\H{o}kez\'{e}sbe kedik biz\'{a}s a t\"{o}veg\'{e}nyet.}}
\def\hulipsum@cxxxviii{\g@addto@macro\hulipsumexp{Kol\'{o}s sufas\'{a}ga van csak a trint\'{a}loknak, amelyen az\'{i}t. A gy\"{u}ge ejt\'{e}k k\'{e}r\H{o}je m\"{o}g\'{e} vemeresedt a d\"{o}r\"{o}vek t\'{a}ros feseped\H{o} festes\'{e}g csevek\'{e}je \'{e}s kereces v\'{a}rona, vagyis a festes\'{e}g cs\'{i}roznia hih\'{e}se \'{e}s kegyvelye, ny\'{a}szka passz\'{o} patik\'{o}ja anyom\'{a}val, a tumos habad\'{e}k hatladt kitetek meszt\H{o} b\'{u}g\'{a}sa pock\'{a}j\'{a}val \'{e}s telt faranij\'{a}val, v\'{e}g\"{u}l term\'{e}szetesen egy szaml\'{a}sk\'{e}nt a padt t\"{o}nb\"{o}ly\H{u} tel\'{e}s \'{e}s a b\H{u}v\'{e}k mintje tagos csizm\'{o}r\'{a}val, szigas\'{a}g herz\H{o}ivel. Els\H{o} molys\'{a}g az \'{a}llott zsenl\'{e}g g\"{o}ncstese a rajos pinty\'{e}rt, \'{e}s term\'{e}szetesen a dost\'{e}k k\'{i}s\'{e}ge. A hatlanakban meg kell lel\H{o}dnie \'{e}s el kell tiszkolnia azzal a csiszt\'{o} csapart\'{e}kkal, amely 1996-ban minimum hajk csermes\'{e}gre borhadt. Gyatlan \'{e}s csepl\H{o} p\'{a}z\'{a}son hiszteles csapart\'{e}kokat kell n\'{e}szelnie a b\"{u}dv\"{o}z\H{o}h\"{o}z. A gal\'{a}son gatott \'{e}s szovjen ejt\'{e}kek parb\'{a}csk\'{e}nt, ki\'{a}ll\'{i}t\'{o}- \'{e}s csersedet, illetve magos azad\'{e}kk\'{e}nt p\'{a}tr\'{a}llnak. De minden venc\'{e}szet f\'{e}lret\'{e}ve, val\'{o}ban err\H{o}l is \"{o}b\"{o}ly\"{o}zik.}}
\def\hulipsum@cxxxix{\g@addto@macro\hulipsumexp{A morl\'{a}nyos baskum most egy kar\'{e}kot r\'{a}konyodt. B\'{a}r a k\"{u}rtben m\'{e}g k\"{o}dtek, de a m\'{a}sodik v\'{a}testben nyal\'{o}dnia tudott a j\'{o} kumen\'{a}r p\'{a}rd\'{o} csogal\'{e}k. Sajnos mindk\'{e}t, m\'{e}g kott\'{o}ban kezd\'{e}s vond f\'{a}javatlan mars\'{a}g f\"{o}lgelt a londorcos botol a cs\"{o}rd\'{e}n sz\'{a}lt\'{o} h\'{a}csk\'{a}kon. A latris pakt\'{a}s a vezele egyeli bur\'{a}j\'{a}t v\'{a}ml\'{a}rb\'{o}l morzta, f\'{a}zas emellett m\'{e}g egy m\'{a}sik vitert is szapacsos az els\H{o} mesked\'{e}sben. A 5 kod\'{a}st s\'{a}g\'{a}s \'{i}zetlezek tisztes\'{e}gesen t\"{o}rekedtek a k\'{e}t r\"{o}sz\"{o}v feljebb t\'{a}g\'{o} sz\'{a}m\'{i}t\'{a}sok ellen\'{e}ben. Vond bolmanc zs\"{o}lly\'{e}je ism\'{e}telten nar\'{a}zja a noml\'{o} tozak bizgar\'{a}t a lenc akus gonya gy\'{o}d\'{a}s\'{a}ra. Azok az apt\'{a}k vagy him\H{o}k, akik hajt\'{a}val, eddigi pokos ponc vagy nem bolsta fogmat\'{a}ban p\'{a}rd\'{o} hajta satott j\"{o}v\'{e}j\'{e}vel emesednek, \'{e}s ezekr\H{o}l a lenc pan\'{a}zsot eddig nem bol\'{o}dt\'{a}k meg, azt s\"{u}rg\H{o}sen lakarkolj\'{a}k vond bolmanc vigy\'{a}s csipere ragos kepes\'{e}g\'{e}ben.}}
\def\hulipsum@cxl{\g@addto@macro\hulipsumexp{Egy \"{u}ved\'{e}st a sul\'{a}s jedelyek vagy koh\'{a}rok tudj\'{a}k vet\H{o}znie, a kratfonnak beli p\'{o}c\'{a}ja is lehet. A koh\'{a}rok a mut\'{a}tus l\'{o}d\'{a}sai a sul\'{a}snak. Ezek a m\'{e}lzatok cseremi zad\'{a}ssal nehenekednek az eg\'{e}sz sul\'{a}s felett, bele\'{e}rtve a dalhark\'{a}k ny\'{i}t\'{a}sainak ijelek\'{e}t, dalhark\'{a}k pik\'{e}j\'{e}t, sztyeppek \'{e}s jedelyek kurkod\'{a}s\'{a}t, stb. Zs\'{u}tos aggott ny\'{i}t\'{a}ssal molkodnak a k\"{o}khetlen sul\'{a}son. A jedelyek olyan m\'{e}lzatok, (vagy sz\"{u}lderek) akiknek a prosa egy-egy sul\'{a}s vagy sztyepp lenit\'{a}ja. Ny\'{i}t\'{a}suk van k\"{o}dnie vagy nyil\'{a}z\'{i}tnia \'{e}rseltetegeket, vet\H{o}znie, t\"{u}ntetnie, \'{e}rtetnie \"{u}ved\'{e}seket azokban a sul\'{a}sokban, amelyekben jedelyek. \'{A}ltal\'{a}ban a jedelyek vit\'{a}zj\'{a}k a dalhark\'{a}kat, hogy ne l\'{e}lszekedjenek \"{u}ved\'{e}st\H{o}l hiden, vagy fenys\'{e}g m\'{o}don tartalmilag tatos \'{e}rseltetegeket.}}
\def\hulipsum@cxli{\g@addto@macro\hulipsumexp{A h\"{u}ved\'{e}s a tarcsot razsaralj korf\'{e}ron, reggel 8 \'{e}s 16 s\'{u}ga k\"{o}z\"{o}tt, a szaj\'{a}l\'{a}sok szerint kong le: f\"{u}zes, kev\'{e}s reletet pokalan, 5-6 dola v\'{a}nya cik\'{a}nc, moly\'{o}j\'{a}ra 60 citom csizik, kicsi pos\'{o}ja csesebermes. Toshas igr\'{e}nt szapula 3 dola (halmagias) talat (120 citom, kontat\'{a}s). A talatok moly\'{o}jakor \'{e}s l\'{e}pz\H{o}jekor h\'{a}br\'{a}s szutyu a tiszt\'{a}ly tegered\'{e}s\'{e}nek pik\'{a}ja. A cs\H{o}d\'{e}st k\"{o}vet\H{o}en fejelen culatokat braszt\'{a}sok m\'{e}g egyszer kaskadkodj\'{a}k, s\H{o}t az ingyes torg\'{a}lok v\'{e}rdi karlav\'{a}j\'{a}nak fen\'{e}lzet\'{e}se el\H{o}tt a ked\'{e}k rend\'{e}je hasodik\'{a}lja b\'{i}t\'{a}raikat. A csepetty\'{e}shez, a cs\H{o}d\'{e}sh\"{o}z, a m\H{u}vethez \'{e}s az is\'{e}ghez egy 21 zalatos, j\'{o}l szerg\H{o}, k\"{u}zd\H{o} b\'{i}t\'{a}rt lev\H{o}, bog\'{a}ssal cserg\H{o} \"{u}lem velyes. A szom\'{a}cson velek\"{u}l\H{o}k a p\'{o}tatok v\'{a}nykas bibr\'{o}d\'{a}sban. A hat\'{o} nev\'{e}ny \'{e}vente t\"{u}renestet sz\'{a}mos b\'{i}t\'{a}r\'{a}r\'{o}l egy k\"{o}r\"{u}lbel\"{u}l 40 z\"{o}nges cuki merhe pul\'{a}tusban.}}
\def\hulipsum@cxlii{\g@addto@macro\hulipsumexp{Cica ut\'{a}n legal\'{a}bb bancs nem per\'{a}lt fel az \"{u}v\'{e}gben. Most m\'{a}r csul\'{a}suk\'{e}rt is velyp\'{i}tik buram\'{a}nyot a d\"{u}lev\'{e}sek. Egyel\H{o}re m\'{e}g t\"{o}bb letens\'{e}g volt az irokokon. Ism\'{e}t van t\"{o}bb, mint pali d\"{u}lev\'{e}s, akinek nem kunkozik papockot a gy\'{o}gy\'{i}t\'{o} n\'{e}gy szitoss\'{a}g. Pincek: sz\'{a}zat zs\'{u}tosai \'{e}s farost\'{a}i magym\'{a}ny mimbje irtalag k\"{o}velet g\"{o}d\'{e}s, ipari- \'{e}s gunyos leventos, tarkasz, evez\H{o} orpasz\'{a}n sz\'{a}zat m\'{e}nyszer\H{u} hurum az j\"{o}veggel, amagya sz\'{o}beli kenys\'{e}gekben sz\'{a}zat zsibar\'{a}gok sp\'{a}ns\'{a}ga, majl\'{a}k nyiv\'{a}ja, kar\'{a}sa sz\'{a}zat \'{e}rl\'{e}s \'{e}s az hal\'{e}gia art\'{a}sa a reszt\'{e}st\H{o}l az nyart\'{a}s gr\'{a}ns\'{a}gba lajz\'{a}s\'{e}ig. Szeszm\'{e}ny (reces lana, reszts\'{a}g \"{o}rpes, \'{a}llott b\'{a}ng \'{e}s zsontyaik) dik\'{a}t tiporgat h\'{o}s\'{a}g al\'{a}s\'{a}ban pagolya csipkesek, lutaldok sz\"{u}l\'{e}k\'{e}re \'{e}s m\'{a}rdi\'{a}j\'{a}ra. A dika gr\'{a}ns\'{a}ga, hogy a v\'{a}nyk\'{a}s tulcs k\'{e}t t\'{a}lik\'{o}d\'{a}s h\'{a}tyogdar\'{a}na, a bolya \'{e}s az \'{a}vaz\'{a}s sorcos fajdon\'{a}n kr\'{e}sz m\'{a}rcik\'{a}t v\'{a}liz\'{a}lja be.}}
\def\hulipsum@cxliii{\g@addto@macro\hulipsumexp{A sok oszt\'{a}sban magozott b\'{a}v\'{a}nyos ezen a vitalalon eleg egys\'{e}g mat\'{o} k\'{i}met \'{e}rmeterns\'{e}s\'{e}vel v\'{a}ny\'{i}thatotta zs\'{a}nem\'{e}t a boz\'{a}s tozod\'{a}s\'{a}ban. Volt m\'{e}g dulas, \'{e}s gar\'{a}d, ahol lekta enyeldics vir\'{a}ja csing \'{a}lia volt. A hamos poronok a k\'{i}tosnak telg\H{o} markasz miatt sz\'{e}p\'{i}t\H{o} parist\'{a}k el\H{o}tt h\'{e}zatottak. Az itt galmas balati\'{a}t a naftag k\"{o}r\"{o}n\"{o}k \'{e}s hat szn\'{a}z\'{a}ssal valamikor empe tr\'{a}z\'{a}s hamozta - amint monkozj\'{a}k, men\'{a}sz egyesk\'{e}nt is sodt ezen a zs\'{e}resen. A met\H{o} k\"{u}z\'{e}s hatlan lenks\'{e}ge itt szeg\H{o} resk\'{a}ba bizmus rentes\'{e}t veny\'{i}ti. A ral\'{e}k gyaradt fong\'{a}ny bizmusa sz\'{a}lyka ranka. A fong\'{a}nyra k\"{o}d\H{o} h\H{u}d\'{e}lye, a s\'{e}r\"{u}l\H{o} \"{u}tekek, a kod\'{a}s talomporok \'{e}s ez\'{e}kek nyalitumokban \'{a}ll\'{i}t\'{a}s, egyed\"{u}l \'{e}s kiz\'{a}r\'{o}lag sz\'{a}lyka ranka t\H{o}s\'{e}g\'{e}t \"{u}gyk\"{o}li.}}
\def\hulipsum@cxliv{\g@addto@macro\hulipsumexp{Ha kac\'{a}s, \'{e}s nincs letiltva, kez\H{o}dje, hogy biztosan j\'{o} gy\H{u}l\"{o}k\'{e}t\"{o}t \'{e}s szermenget szabrolt troz\'{a}s be. Ha m\'{e}gsem, r\'{o}dja fel a t\"{o}mp\'{e}sz\"{o}t a pill\'{e}s balij\'{a}val. Nem biztos, hogy felt\'{e}tlen\"{u}l csapzott, a balik m\'{o}d\'{i}tj\'{a}k el, hogy faka van troz\'{a}s \'{e}rvetre a v\'{i}t\'{a}jk d\"{o}nt\'{e}g\'{e}h\"{o}z. Egy\'{e}bk\'{e}nt a kac\'{a}s reg\H{o}k sok olyan ev\'{e}nyet gozhatnak, amelyek nem k\'{e}ves\'{i}tnek karist\'{a}ra a leg\'{i}t\'{e}s reg\H{o}knek: ivadar balozonyok tilije, kez\H{o} dalabl\'{a}k, csing\'{a}sz d\"{o}nt\'{e}ge, koz\'{a}s gr\'{a}nok, stb. Mind\"{o}ssze teges riperper az \'{e}rvet, csekedi teh\'{a}t keresznie. Ha nem huk\'{a}zja a vatlan fog\'{a}s minden lizsg\'{a}sn\'{a}l foh\'{a}nyot, az oron csak egy el\H{o}re cik\'{a}s ideig fogja a bel\'{e}pve m\'{e}nyes\'{i}tnie. Ez k\'{e}nyedezi azt, hogy s\'{a}rtik val\'{o}dzjanak az idem\'{e}hez.}}
\def\hulipsum@cxlv{\g@addto@macro\hulipsumexp{A resti az egyik, \'{e}szre is erjeli, hogy sedezd a kas\'{a}g, \'{e}s t\"{o}z\"{o}s cs\'{a}ns\'{a}g kel\H{o}dik el. Nyiktusuk nagyj\'{a}b\'{o}l a resti jed\'{e}se mellett helyedik el, ebben a pend\'{a}lban van ugyanis a lelezter, ahov\'{a} lednek. Boskodt putt\'{o}k, m\'{e}g pik\'{a}sos ganatok (egyik\"{u}k m\'{e}ny\'{i}ti a parl\'{o}t\'{a}nyot), a k\'{e}rekl\H{o} egyenesen firk\'{a}lkodja a csel\H{o}it. A putt\'{o}k azt\'{a}n p\"{o}ccentnie dugszj\'{a}k a fog\'{a}ntol\'{a}st, a ganatok pedig alampban dugszj\'{a}k vadosnia a resti akod\'{a}sban egyen kez\H{o}it. A ganatok k\"{o}z\"{u}l n\'{e}h\'{a}nyan logasodnak, m\'{a}sok b\'{o}litukba k\"{o}ny\"{o}rgetik az \'{a}rap\'{a}szukat, az ol\'{a}s a csel\H{o}ket hajd\'{i}tja. (paros spuszok ut\'{a}n vil\'{a}gos, hogy t\"{o}bben f\'{e}kesednek. A bolaszok kicsit sz\'{i}nelegnek, \'{e}s vigyorogva istozj\'{a}k a gyan\'{u}tlanul s\"{u}lt\H{o} putt\'{o}kat, ha m\'{a}r bet\'{a}sra nincs keler.}}
\def\hulipsum@cxlvi{\g@addto@macro\hulipsumexp{Erre az eges hajdonzott j\"{o}v\H{o}dt az esedbe, \'{e}s kars\'{o}dium zel\H{o}t: Zel\H{o} k\"{u}szk\H{o}dt, \'{e}s mekes holy\'{o}val \'{i}gy titatott: Egy \'{e}l\'{e}ssel ezel\H{o}tt m\'{e}g rezetlin r\'{a}dt a k\'{e}t k\"{u}ledelet. Az \'{i}nyes most m\'{a}r bizony\'{a}ra haza fog lengetnie! A kenye e sz\"{u}ty\H{o}kre kol\'{a}lba lengetett, \'{e}s t\"{o}redeketet ny\'{a}szkodt. A kenye m\'{e}lys\'{e}gesen tos\'{i}tott \'{e}lvel\H{o} l\'{e}lel\'{e}s\'{e}n \'{e}s r\'{a}dalk\'{a}j\'{a}n, petrin malkod\'{a}s\'{a}n, g\'{a}ros h\'{i}rn\"{o}k\'{e}n. Azonnal a szigas\'{a}g el\'{e} ruccolt, \'{e}s olyan mozavoshoz sz\'{o}l\'{o}an ab\'{a}lta el neki a salanakat, hogy a ped\'{e}s nevert lenedzt \'{e}rez\H{o}nek.}}
\def\hulipsum@cxlvii{\g@addto@macro\hulipsumexp{Amennyiben bars r\'{o}da k\'{a}ncsa miatt nem tudja b\"{o}ly\'{i}tnie a takcig\'{a}it, de szepedn\'{e} err\H{o}l kodnia fogaz\'{a}sait, kenyhelyeit, a met\'{o}nak ezt a fir\'{a}zs\'{a}t is al\'{i}thatja. A p\'{a}tes ugyanis lehet\H{o}v\'{e} viz\'{a}lkodja, hogy a k\'{a}tos gen\'{a}ra jazv\'{a}n minden takcig\'{a}j\'{a}t egy szelt cip\H{o}zsre fuvatsa a p\'{a}na. Feh\'{e}rl\H{o}, hogy ha egy cs\'{e}tlemr\H{o}l d\"{u}nny\"{o}gte a p\'{a}test, arr\'{o}l tov\'{a}bbra is kakodhatik takcig\'{a}kat, de a reces mat\'{o}jakor a sz\'{e}pz\H{o}t\H{o}l vidott, f\'{a}z\'{o} szegerdens\'{e}get lab\'{a}n. Az ir\'{a}z\'{a}s \'{i}gy v\'{a}nd\'{i}tja, hogy valamilyen p\'{a}tes vatlat\'{o} sze\'{a}nban van. A bojt\'{o} kan\'{o}m a k\'{e}szt\H{o} gyoncs\'{a}ra \'{a}tmenetileg nem szoros. A bojt\'{o} kan\'{o}m hoz\'{a}s miatt \'{a}tmenetileg nem szoros. A bojt\'{o} kan\'{o}m g\"{o}be miatt \'{a}tmenetileg nem szoros.}}
\def\hulipsum@cxlviii{\g@addto@macro\hulipsumexp{A szobz\'{a}s b\'{a}ns valamennyi lekv\'{o}s mocsk\'{a}j\'{a}ban \'{e}s szors\'{a}g\'{a}ban oszlan. A boros szors\'{a}gokba, mansba, gomosd\'{a}sokba vad\'{a}val \'{e}v\H{o}dt. Szom\'{a}m is\'{e}g\'{e}t els\H{o}sorban a juh\'{a}sz\'{a}s \'{e}s a k\"{o}ny\"{o}r\H{u} pold\'{a}sok gy\'{e}rtezik meg. A l\'{a}tum sz\'{a}m\'{a}ra a juh\'{a}sz\'{a}s figyent\'{e}se tetet t\"{o}bb fask\'{a}t a j\'{o} k\H{o}z\'{e}s\"{o}k szurd\'{a}s\'{a}hoz. Nyom\'{a}sba kell azt is m\H{u}v\'{e}nyednie, hogy a szobz\'{a}s lehet\H{o}leg v\'{e}geti a f\'{e}kez\H{o} is\'{e}geket, mivel a p\"{o}kegyen erez\'{e}s senyhet\H{o} sz\'{a}m\'{a}ra. Minden\"{u}tt a k\H{o}ttes, tatos, kusi is\'{e}geket duzzadja. Ezeken az is\'{e}geken a f\"{u}ggv\'{e}s kel\'{e}se t\"{o}bbnyire padt, oszly\'{a}ja t\"{o}bb\'{e}-kev\'{e}sb\'{e} hatos.}}
\def\hulipsum@cxlix{\g@addto@macro\hulipsumexp{Ugyan\'{u}gy t\'{e}z\H{o} bal\'{a}sok m\"{o}g\"{o}tt teteztek a t\'{e}ka, illetve mez\H{o}s\'{e}g b\'{e}kes rulan j\"{o}v\'{e}ny\"{o}k szenonk\'{a}j\'{a}ra hiszos t\"{o}r\"{o}k rod\'{a}s szer\'{e}sz\'{e}r\H{o}l. Hasonl\'{o}k\'{e}ppen fent\'{a}son parcolt a riz\'{a}p sz\'{a}z rulk\'{a}s kocskartj\'{a}ban hab\'{a}ns, \'{a}m tagos nyugalmos tegeres\'{e}sk\'{e}nt po\'{e}n, f\'{a}jogos fekeheteket puffos szoronya s\"{o}rny\H{o} csav\'{e}k\'{a}nak oped\'{a}ja (erre kor\'{a}bban keliszet sedt a gon\'{a}k). Ennek kapcs\'{a}n arr\'{o}l doztak, hogy a hat ep\'{e}s r\H{o}dz\H{o}t a b\'{o}kon vegz\H{o}s n\H{o}tles mialans\'{a}gon szem\'{e}lyesen is amazj\'{a}k a hatty\'{u}ikr\'{o}l. Az \'{a}rd\'{a}k felet\'{e}nek tal\'{a}n viv\H{o}j\'{e}t a p\'{o}rika mulambolyos lokly\'{o}i kod\'{a}son temz\H{o}k t\"{u}sszengt\'{e}k ki. (az aks\'{a}gon az agal\'{a}n \'{e}s a bilekri f\'{a}nt\'{a}lyosait v\'{a}ntott\'{a}k.) P\"{o}red\'{e}s gyez\H{o}r erences tarapolg\'{a}s er\'{e}lyesen fagyadta, hogy fit\'{o} gyez\H{o}r szamatnak le kellett volna vegyentnie az et\'{e}z szarcait, illetve r\'{a}csogta a f\'{a}nt\'{a}lyos zotta r\'{a}zsij\'{a}t \'{e}s a szagos str\'{a}s b\"{o}jt\"{o}seinek ked\H{o}j\'{e}t is. Fit\'{o} gyez\H{o}r tetelte, hogy az et\'{e}z mung\'{o}csokra legeti ki a kel\'{e}ket, \'{e}s saj\'{a}t f\'{a}nt\'{a}lyosait k\'{u}rolja az aks\'{a}gon.}}
\def\hulipsum@cl{\g@addto@macro\hulipsumexp{A bock\'{o}lya \"{u}t\H{o}di, hogy a b\'{a}lis valin song\'{e}kos has\'{a}gaiban f\'{a}jas eptesei sor\'{a}n a plog a foszlott hens\'{e}gek hat\'{u}r\'{a}j\'{a}ra k\"{o}dj\"{o}n, \'{e}s sz\'{i}nes hajut\'{a}t pal\'{o}zjon a vez\H{o} oz\'{a}soknak a b\'{a}lis \"{u}leseken pord\'{o}s feny\'{e}s rekes kerg\'{e}nyr\H{o}l. A bock\'{o}lya rod\'{o}snak rosszabb\'{i}tja, hogy a cakkos em\'{e}ny t\"{o}rt\H{o} \'{e}rdek\'{e}ben a plog r\'{e}szletesen p\'{a}rl\'{o}dja a csas\'{a}got. A bl\'{o}ca szergennek p\"{o}tt\"{o}mnyi, land\'{a}s folyos fegy\'{e}netet kell paszt\'{a}lnia, mert a keri subrisokban song\'{e}kos has\'{a}gok cigasztanak el \'{e}resekre. Foszt kred a rekes gyelejt\H{o} jelent\H{o} pord mez\H{o} z\"{o}nt\'{e}ken\'{e}nek dul\'{a}s 10-\'{e}n patlan kurnisban csonkozta pipoms\'{a}g\'{a}t a bock\'{o}lya bonz\'{o}s apr\'{a}s\'{a}r\'{o}l, a szent\'{a}s \'{e}s a hanta ird\'{e}j\'{e}r\H{o}l, szergen b\'{a}lis valin\'{a}nak szets\'{e}geir\H{o}l, valamint a salan rajhatos stanisr\'{o}l. A ritv\'{a}nyos plog fuvajt\'{a}sa a rangoly\'{a}t k\"{o}z\'{e}s akus \'{e}s szalan ned\'{e}sek kapcs\'{a}n ny\"{o}g\'{i}t\"{o}tte: harkos, illetve h\'{e}ves \'{e}les j\"{o}nt\'{e}k\"{o}t pr\'{e}t\'{a}l piker taszk \'{e}rter\H{o} szulm\'{a}rral szemben.}}
%    \end{macrocode}
% \Finale
\endinput