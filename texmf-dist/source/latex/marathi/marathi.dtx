% \iffalse meta-comment
%
% File: marathi.dtx
% ---------------------------------------------------------------------------
% आज्ञासंच:     marathi
% लेखक:       निरंजन
% माहिती:       लुआलाटेक् व झीलाटेक् ह्यांच्यासह मराठीचा सुलभ वापर करण्यासाठी.
% संग्राहिका:     https://git.gnu.org.ua/marathi.git
% अडचणी:      https://puszcza.gnu.org.ua/bugs/?group=marathi
% परवाना:      आज्ञालेखांकरिता आलोक नित्यमुक्त परवाना (आवृत्ती १.०+) व ग्नू पब्लिक परवाना
%            (आवृत्ती ३.०+). हस्तपुस्तिकेकरिता ग्नू फ्री डॉक्युमेन्टेशन परवाना (आवृत्ती १.३+)
% ---------------------------------------------------------------------------
% LaTeX Package marathi v1.7
% Copyright © 2020, 2021 निरंजन
%
% This program is free software: you can redistribute it
% and/or modify it under the terms of the GNU General Public
% License as published by the Free Software Foundation,
% either version 3 of the License, or (at your option) any
% later version.
%
% This program is distributed in the hope that it will be
% useful, but WITHOUT ANY WARRANTY; without even the implied
% warranty of MERCHANTABILITY or FITNESS FOR A PARTICULAR
% PURPOSE. See the GNU General Public License for more
% details.
%
% You should have received a copy of the GNU General Public
% License along with this program. If not, see
% <https://www.gnu.org/licenses/>.
%
% The current maintainer of this work is निरंजन.
% हे काम आलोक नित्यमुक्त परवाना तसेच ग्नू पब्लिक परवान्याच्या अटींचे
% पालन करून वितरित केले जाऊ शकते तसेच सुधारले जाऊ शकते.
% 
% ह्या परवान्यांची नवीनतम आवृत्ती पुढील दुव्यांवर पाहता येऊ शकते.
% 
% https://varnamudra.com/aalok/parwana
% https://www.gnu.org/licenses/gpl-3.0.txt
% https://www.gnu.org/licenses/fdl-1.3.html
% 
% ह्या आज्ञासंचाची लाटेक् प्रकल्पाच्या नियमांनुसार देखरेख केली जात आहे.
% 
% ह्या आज्ञासंचाचा लेखक व पालक निरंजन आहे.
% 
% ह्या कामात marathi.dtx, marathi.ins तसेच त्यांपासून निर्माण केलेली
% marathi.sty ही धारिका समाविष्ट आहे.
% ---------------------------------------------------------------------------
%
% \fi
% \iffalse
%<*internal>
\iffalse
%</internal>
%<*readme>
--------------------------------------------------------------------------
Note: This README file contains information about LaTeX package <marathi> in
Marathi & English. As this package is targeted more towards the speakers of
Marathi language, the description is initially given in Marathi, but for general
readers an English description is provided at the end.
--------------------------------------------------------------------------
आज्ञासंच:      marathi
लेखक:       निरंजन
आवृत्ती:       १.७ (१६ डिसेंबर, २०२१)
माहिती:       लुआलाटेक् व झीलाटेक् ह्यांच्यासह मराठीचा सुलभ वापर करण्यासाठी. इथून पुढे लाटेक्-वरील
            मराठीच्या स्थानिकीकरणाचे काम ह्या आज्ञासंचामार्फत केले जाईल. तूर्त expex व
            blindtext ह्या आज्ञासंचांचे स्थानिकीकरण ह्या आज्ञासंचामार्फत पुरवले आहे.
संग्राहिका:     https://git.gnu.org.ua/marathi.git
अडचणी:     https://puszcza.gnu.org.ua/bugs/?group=marathi
परवाने:      आज्ञालेखांकरिता आलोक नित्यमुक्त परवाना (आवृत्ती १.०+) आणि ग्नू पब्लिक परवाना
           (आवृत्ती ३.०+). केवळ हस्तपुस्तिकेकरिता ग्नू फ्री डॉक्युमेन्टेशन परवाना (१.३+)
           दुवे:
           https://varnamudra.com/aalok/parwana
           https://www.gnu.org/licenses/gpl-3.0.txt
           https://www.gnu.org/licenses/fdl-1.3.html
--------------------------------------------------------------------------
काही सूचना -
पुढील धारिका texmf-dist/tex/latex/marathi ह्या पत्त्यावर उतरवून घ्या.
१) namuna-article.tex
२) namuna-book.tex
३) namuna-report.tex
४) namuna-beamer.tex
५) namuna-letter.tex
६) namuna-para.tex
--------------------------------------------------------------------------
Package:        marathi
Author:         Niranjan
Version:        1.7 (16 December, 2021)
Description:    For conveniently typesetting Marathi
                language with LuaLaTeX and XeLaTeX. This package
                will now onwards host localizations needed for Marathi
                language. Currently the package localizes package
                blindtext and package expex.
Repository:     https://git.gnu.org.ua/marathi.git
Bug tracker:    https://puszcza.gnu.org.ua/bugs/?group=marathi
License:        `आलोक' copyleft license v1.0+ and GPL v3.0+ for the code.
                GFDL v1.3+ only for the documentation.
                Links:
                https://varnamudra.com/aalok/parwana
                https://www.gnu.org/licenses/gpl-3.0.txt
                https://www.gnu.org/licenses/fdl-1.3.html
--------------------------------------------------------------------------
Installation instructions -
Copy the following files in the following path texmf-dist/tex/latex/marathi
1) namuna-article.tex
2) namuna-book.tex
3) namuna-report.tex
4) namuna-beamer.tex
5) namuna-letter.tex
6) namuna-para.tex
%</readme>
%<*internal>
\fi
%</internal>
%<*driver>
\documentclass[10pt]{l3doc}
\usepackage{fontspec}
\usepackage{xcolor}
\usepackage{hyperref}
\hypersetup{%
  unicode,%
  pdfauthor={निरंजन},%
  pdftitle={मराठी (आवृत्ती १.७, डिसेंबर २०२१)},%
  pdfsubject={लुआलाटेक् व झीलाटेक्-सह मराठीत अक्षरजुळणी.},%
  pdfkeywords={मराठी, लुआलाटेक्, झीलाटेक्, बेबल, पॉलिग्लॉसिया},%
  pdfproducer={हायपर्रेफ्-सह झीलाटेक्},%
  pdfcreator={हायपर्रेफ्-सह झीलाटेक्},%
  colorlinks,%
  linkcolor={red!50!black},%
  citecolor={blue!50!black},%
  urlcolor={blue!80!black}%
}
\usepackage{devanagaridigits}
\usepackage{capt-of}
\renewcommand{\tablename}{कोष्टक}
\renewcommand{\arraystretch}{1.2}
\usepackage{standalone}
\makeatletter
\RequirePackage{devanagaridigits}
\def\@arabic#1{\expandafter\devanagaridigits\expandafter{\number#1}}
\makeatother
\renewcommand{\theCodelineNo}{{\scriptsize\arabic{CodelineNo}}\quad}
\setmainfont[%
  Script=Devanagari%
]{Shobhika}
\setmonofont[%
  Script=Devanagari%
]{Mukta}
\usepackage{fontawesome5}
\renewcommand{\glossaryname}{संज्ञासूची}
\renewcommand{\abstractname}{सारांश}
\renewcommand{\contentsname}{अनुक्रमणिका}
\renewcommand{\baselinestretch}{1.5}
\newcommand{\होय}{%
  \textcolor{green!70!black}{\faIcon{thumbs-up}}
}%
\newcommand{\नाही}{%
  \textcolor{red!70!black}{\faIcon{thumbs-down}}
}%
\usepackage[sort=use]{glossaries}
\makenoidxglossaries
\input{glossaries.gls}
\begin{document}
\DocInput{marathi.dtx}
\end{document}
%</driver>
% \fi
%
%
%
% \title{मराठी}
% \author{निरंजन}
% \date^^A
%   {^^A
%     आवृत्ती १.७ \textemdash\ ०८ डिसेंबर, २०२१\\[1ex]^^A
%     {^^A
%       \small\faIcon{globe}\quad
%       \url{https://ctan.org/pkg/marathi}\\^^A
%       \small\faIcon{bug}\quad
%       \url{https://puszcza.gnu.org.ua/bugs/?group=marathi}^^A
%     }^^A
%   }^^A
%     
% \maketitle
%
% \begin{abstract}
% युनिकोडाचा वापर करून मराठीत \gls{अक्षरजुळणी}\footnote{ह्या दस्तऐवजात 
% लाटेक्-मध्ये प्रचलित असणाऱ्या अनेक इंग्रजी संज्ञांसाठी पर्यायी मराठी 
% संज्ञा वापरण्यात आल्या आहेत, हे सर्व प्रयोग तुलनेने नवे असल्याने ह्याच 
% दस्तऐवजात शेवटी दिलेल्या संज्ञासूचीत सर्व संज्ञांचे इंग्रजी अर्थ दिले 
% आहेत.} शक्य आहे, परंतु सुलभ नाही. लाटेक्-सह मराठीत अक्षरजुळणी करताना आपली 
% \gls{बीजधारिका} अनेक आज्ञांनी भरून जाते. हा \gls{आज्ञासंच} अशा सर्व 
% उपयुक्त \gls{आज्ञा} आधीच लिहून ठेवतो. त्यामुळे नव्या वापरकर्त्यांना त्या 
% नव्याने शिकाव्या लागत नाहीत. शिवाय \pkg{blindtext} ह्या आज्ञासंचाप्रमाणे 
% मराठीकरिता नमुना मजकूर उत्पन्न करण्यासाठीची सोय ह्या आज्ञासंचात करण्यात 
% आली आहे.
% \end{abstract}
% 
% \tableofcontents
% \clearpage
% \hspace{0pt}
% \vfill
% \section*{परवाना}
% {%
% \setlength{\parindent}{0pt}
% प्रतिमुद्राधिकार © २०२०, २०२१ निरंजन.
%
% ह्या सामग्रीच्या वितरणाचे व प्रतिमुद्रणाचे अधिकार आलोक नित्यमुक्त परवान्यासह मुक्त करण्यात येत
% आहेत. ह्या सामग्रीची यथामूल अथवा परिवर्तित प्रतिमुद्रणे व्यावसायिक अथवा अव्यावसायिक
% स्वरूपात वितरित करण्यास प्रतिमुद्राधिकारधारक संमती देत आहे, परंतु असे करताना वितरकाने
% प्रतिमुद्राधिकारांचा योग्य श्रेयनिर्देश करणे व सामग्री परिवर्तित असल्यास ती ह्याच अटींसह
% वितरित करणे बंधनकारक आहे. ही सामग्री जशी आहे तशी पुरवण्यात येत आहे, पुरवणारा/पुरवणारी
% हिच्याबाबत कोणतीही हमी देत नाही. ह्या (व ह्यावर आधारित) सामग्रीचे अमुक्त वितरण
% बेकायदेशीर मानले जाईल. आलोक नित्यमुक्त परवान्याचा संपूर्ण मसुदा पुढील दुव्यावर वाचता
% येईल.
%
% \url{https://gitlab.com/aalok/nityamukta-parwana}
% 
% \rule{\linewidth}{0.5mm}
%
% \fontfamily{lmr}\selectfont
% Copyright © 2020, 2021 {\normalfont निरंजन}.
%
% Permission is granted to copy, distribute and/or modify this document
% under the terms of the GNU Free Documentation License, Version 1.3
% or any later version published by the Free Software Foundation;
% with no Invariant Sections, no Front-Cover Texts, and no Back-Cover Texts.
% A copy of the license is included in the section entitled ``GNU
% Free Documentation License''.
%
% \url{https://www.gnu.org/licenses/fdl}
% }%
%
% \vfill
% \begin{documentation}
% \section{प्रस्तावना}
% \subsection*{देवनागरी दिसण्यासाठी टाकावयाच्या आज्ञा}
% लाटेक्-मध्ये मराठीचा वापर करताना सर्वात महत्त्वाची गरज म्हणजे देवनागरी लिपी नीट दिसणे. लाटेक्
% विकसित झाले तेव्हा युनिकोड ही प्रणाली अस्तित्वात नसल्याने टेक्, लाटेक् ह्यांसारख्या चालकांसह
% युनिकोड अक्षरे वापरता येत नाहीत, परंतु लवकरच युनिकोड अक्षरांचा वापर लाटेक्-मध्ये करता यावा
% ह्याकरिता झीलाटेक् तसेच लुआलाटेक् ह्या नव्या चालकांचा विकास झाला. \pkg{fontspec} ह्या
% आज्ञासंचासह एखादा युनिकोड-आधारित \gls{टंक} वापरणे व युनिकोड-मजकूर थेट झीलाटेक् अथवा
% लुआलाटेक्-सोबत चालवणे हे ह्या नव्या चालकांचे प्रमुख उद्दिष्ट होते, परंतु हे काम अतिशय गुंतागुंतीचे
% होते.  २०२० चे \gls{टेक्-वितरण} येईपर्यंत व त्यात हर्फ़बझ नावाची नवी \gls{आज्ञावली} येेईपर्यंत
% लुआलाटेक् देवनागरी हाताळू शकलेच नाही. झीलाटेक्-सह मात्र देवनागरी व्यवस्थित दिसणे शक्य होत
% होते. देवनागरी योग्य तऱ्हेने दिसण्यासाठी काही आज्ञांचा वापर करणे अनिवार्य आहे. सध्याच्या
% व्यवस्थेनुसार जोडाक्षरे तुटक दिसतात (उदा. नमस्‌कार). त्यासाठी \verb|\setmainfont| ह्या आज्ञेस
% \verb|Script=Devanagari| असे \gls{प्राचल} द्यावे लागते. लाटेक्-तर्फे आपोआप पुरवले जाणारे
% आकडे (उदा. पृष्ठक्रमांक, तळटिपांचे क्रमांक) देवनागरीत येत नाहीत. त्याकरिता ह्याच आज्ञेस
% \verb|Mapping=devanagarinumerals| असे आणखी एक प्राचल द्यावे लागते, परंतु हे प्राचल केवळ
% झीलाटेक्-सह काम करते. शिवाय लाटेक्-ला इंग्रजीव्यतिरिक्त इतर कोणतीही भाषा कळत नसल्यामुळे
% पूर्वस्थित इंग्रजी शब्दांची भाषांतरे पुरवणारे बेबल अथवा पॉलिग्लॉसिया ह्यांसारखे आज्ञासंच वापरून
% भाषा निवडावी लागते. रोहित होळकरांच्या
% \href{https://ctan.org/pkg/latex-mr?lang=en}{\pkg{latex-mr}} ह्या पुस्तिकेत ह्या
% सर्व अडचणींची गोष्टींची तपशीलवार चर्चा झाली आहे.
% 
% सद्यपरिस्थितीत लाटेक्-चे किमान ज्ञान असलेल्या नव्या वापरकर्त्याला मराठी लिहिण्यासाठी एवढा
% सगळा प्रपंच करायला लावणे म्हणजे ज्या फांदीवर आपण बसलो आहोत तीच तोडण्यासारखे आहे. त्यामुळे
% कोणत्याही वापरकर्त्याला केवळ लाटेक्-च्या किमान ज्ञानासह \verb|\usepackage{marathi}| एवढी
% एक आज्ञा लिहून उत्तम देवनागरी अक्षरजुळणी करता यावी हा ह्या आज्ञासंचाचा उद्देश आहे. ह्या
% आज्ञासंचात पुढील आज्ञांचा समावेश आहे.
%
% \begin{function}{\परिच्छेद}
%   ही आज्ञा केवळ एक नमुना परिच्छेद निर्माण करते. ही आज्ञा एकामागोमाग एक अनेकदा
%   टाकल्याने तितके परिच्छेद निर्माण केले जाऊ शकतात.
% \end{function}
%
% \begin{function}{\नमुना}
%   नमुना ही आज्ञा सर्वप्रथम वापरला गेलेला \gls{लाटेक्-वर्ग} पाहते व त्यानुसार त्या
%   वर्गाच्या किमान क्षमता दर्शवणारी एक \gls{फलित-धारिका} निर्माण करते. उदाहरणादाखल
%   \verb|article, book, report, beamer| व \verb|letter| ह्या लाटेक्-वर्गांसह
%   \verb|\नमुना| ही आज्ञा चालवून पाहा. एकाच बीजधारिकेत केवळ लाटेक्-वर्ग बदलत असाल,
%   तर लाटेक्-ने तयार केलेल्या \gls{साहाय्यक धारिका} काही वेळा अनपेक्षित \gls{अडचण}
%   दाखवतात. जुनी माहिती शिल्लक असल्याने त्या दिल्या जातात. घाबरून न जाता, बीजधारिका
%   दोनदा चालवावी. सर्व अडचणी सुटतात.
% \end{function}
%
% \begin{function}{\टंक}
%   \begin{syntax}
%     \cs{टंक} \marg{टंकाचे नाव}
%   \end{syntax}
%   ह्या आज्ञासंचात शोभिका हा \gls{मूलटंक} म्हणून निवडून ठेवला आहे. तो बदलायचा असेल तर
%   \verb|\टंक| ह्या आज्ञेची सोय करण्यात आली आहे. ह्या आज्ञेसह देवनागरी टंकांसाठी आवश्यक
%   असणारी  \verb|Renderer=Harfbuzz, Script=Devanagari| (लुआलाटेक्),
%   \verb|Script=Devanagari, Mapping=devanagarinumerals| (झीलाटेक्) ही प्राचले
%   आपोआप लिहून ठेवली आहेत. शिवाय \verb|\setmainfont|{\meta{टंकाचे नाव}} ही आज्ञा
%   नेहमीप्रमाणे चालतेच. टंकाचे नाव हा ह्या आज्ञेचा \gls{कार्यघटक} आहे.
% \end{function}
%
% \begin{function}{\दुसराटंक}
%   \begin{syntax}
%     \cs{दुसराटंक}\marg{टंकासाठीची आज्ञा}\marg{टंकाचे नाव}
%   \end{syntax}
%   दस्तऐवजात मूलटंकाखेरीज आणखी एखादा टंक वापरण्याची गरजदेखील पडू शकते. त्याकरिता
%   \verb|\दुसराटंक| ही आज्ञा पुढीलप्रमाणे वापरावी.
% \begin{verbatim}
% \documentclass{article}
% \usepackage{marathi}
% \दुसराटंक{\नवाटंक}{Mukta}
% 
% \begin{document}
% नमस्कार, हा मजकूर शोभिका टंकात छापला गेला आहे, परंतु {\नवाटंक हा मजकूर मुक्त टंकात छापला
% गेला आहे.}
% \end{document}
% \end{verbatim}
% \end{function}
%
% \begin{function}{अंतर}
% हे प्राचल वापरल्यास आज्ञासंचातर्फे दोन ओळींमधील अंतरात कोणताही हस्तक्षेप केला जात
% नाही. अधिक माहितीसाठी \ref{अंतर} वाचा. ह्या प्राचलाला किंमतदेखील देता
% येते. उदा. \verb|\usepackage[अंतर=2]{marathi}| अशा प्रकारे आज्ञासंच वापरल्यास
% ओळींमधले अंतर दुप्पट होते. कोणत्याही प्राचलाशिवाय वापरल्यास आज्ञासंचातर्फे मूलभूत
% अंतराच्या दीडपट अंतर पुरवले जाते. पुढील कोष्टकाने अंतर ह्या प्राचलाचा वापर अधिक स्पष्ट
% होईल.\label{प्राचल}
%
% \begin{center}
%   \begin{tabular}{ll}
%     \hline
%     आज्ञासंचा वापर & ओळींमधले अंतर\\
%     \hline
%     \verb|\usepackage{marathi}| & मूळ अंतराच्या दीडपट\\
%     \verb|\usepackage[अंतर]{marathi}| & मूळ अंतरात कोणताही फरक नाही.\\
%     \verb|\usepackage[अंतर=2]{marathi}| & मूळ अंतराच्या दुप्पट\\
%   \end{tabular}
%   \captionof{table}{अंतर}
% \end{center}
% \end{function}
% \begin{function}{\अंतरबदल}
%   \begin{syntax}
%     \cs{अंतरबदल}\marg{किंमत}
%   \end{syntax}
%   दस्तऐवजात कुठेही अंतर बदलावयाचे असल्यास ह्या आज्ञेने ते बदलता येते. ह्या आज्ञेला एक
%   कार्यघटक आहे. त्यात ओळींमधले मूळ अंतर जितक्या पटींनी बदलायचे आहे ती किंमत
%   टाकावी. पुढील उदाहरण पाहा.
%   \begin{verbatim}
% \documentclass{article}
% \usepackage{marathi}
% 
% \begin{document}
% \परिच्छेद
% \अंतरबदल{5}
% \परिच्छेद
% \end{document}    
% \end{verbatim}
% \end{function}
% \end{documentation}
% \begin{implementation}
%   \section{आज्ञासंचाची घडण}
% आता आपण आज्ञासंचाची घडण व त्यातील आज्ञांचा उपयोग लक्षात घेऊयात.
%    \begin{macrocode}
%<*package>
%    \end{macrocode}
%    \begin{macrocode}
\ProvidesPackage{marathi}[2021-12-16 v1.7 झीलाटेक् व लुआलाटेक्-सह मराठीचा सुलभ वापर]
%    \end{macrocode}
% ह्या आज्ञांसह आज्ञासंचाची पायाभूत माहिती पुरवली. डेव्हिड कार्लआयल ह्यांनी गिट
% प्रकल्पावर नोंदवलेल्या
% \href{https://gitlab.com/niruvt/marathi/-/issues/1}{ह्या}
% तक्रारीमुळे आज्ञासंचाची संक्षिप्त माहिती पुरवण्याची पद्धत मला वापरता आली.
% 
% \subsection{अडचण दाखवणे}
% ह्या आज्ञासंचाच्या गिट प्रकल्पावर नोंदवलेल्या
% \href{https://gitlab.com/niruvt/marathi/-/issues/4}{तक्रारीचा}
% विचार करून झी-लाटेक् अथवा लुआलाटेक् ह्या चालकांव्यतिरिक्त एखादा चालक वापरला गेल्यास
% वापरकर्त्यास चालक बदलण्याची सूचना दिली जावी ह्याकरिता पुढील आज्ञावली लिहिण्यात
% आली आहे.
%    \begin{macrocode}
\RequirePackage{iftex}
\iftutex
\else
\PackageError{marathi}{%
  चालक बदला | Unsupported engine%
}%
{%
  लुआलाटेक् अथवा झीलाटेक् वापरा.\MessageBreak
  Only LuaLaTeX or XeLaTeX supported%
}%
\expandafter\endinput
\fi
%    \end{macrocode}

% \subsection{अंतर}\label{अंतर}
% इंग्रजीमधली g, j, y अशी अक्षरे सोडली तर ह्याहून जास्त खोल असणारा मजकूर त्या लिपीत
% आढळत नाही. देवनागरीचे तसे नाही. क ह्या अक्षराहून क्क थोडे अधिक खोल. ट्ट त्याहून थोडे
% अधिक व ट्टू त्याहून. अशा असमान उंचीच्या अक्षरांमुळे लाटेक् आपोआप दोन ओळींमधले अंतर बदलते
% व त्यामुळे ओळींची उंची असमान दिसू लागते. साध्या मजकुरातील ओळींप्रमाणेच कोष्टकेदेखील कुरूप
% दिसू लागतात. ह्यावर तोडगा काय? ह्याची दोन उत्तरे आहेत. पहिले उत्तर मराठीच्या
% आजवरच्या छपाईच्या इतिहासाचा अभ्यास करून, मराठी लिहिताना दोन ओळींमध्ये सोडले गेलेले
% सरासरी अंतर किती ह्याचे संशोधन करणे व त्यानुसार सर्व ठिकाणी ते अंतर लागू करणे. हा
% मार्ग सहज नाही. त्याकरिता टंकाच्या आकाराचे व त्यानुसार बदलणाऱ्या ओळींच्या अंतराचे
% प्रमाण काळजीपूर्वक अभ्यासावे लागेल. हे सर्व करणे जरी इष्ट व आवश्यक असले, तरी हे संशोधन
% पूर्ण होईस्तोवर सामान्य वापरकर्त्याला असमान उंचीच्या ओळींचा जो त्रास सहन करावा
% लागतो, तो कसा टाळावा? त्यासाठी तात्पुरता तोडगा काढावा लागतो. तो असा की
% ओळींमधले अंतर दस्तऐवजाच्या सुरुवातीलाच (अंदाजे) वाढवून ठेवायचे.
%
% आमच्या काही सहकाऱ्यांच्या मते हा निर्णय अतिशय उग्र आहे. त्यामुळे दस्तऐवजातील इतर काही
% ठिकाणच्या अंतरांवर सूक्ष्म परिणाम घडतात, जे टाळणे अधिक इष्ट. त्यांच्या ह्या मताचा
% विचार करून ह्या आज्ञासंचाकरिता  \textbf{अंतर} नावाचे प्राचल विकसित केले आहे. त्याचा
% वापर कसा केला जावा ही माहिती आपण \ref{प्राचल} मध्ये वाचलीच आहे. हे प्राचल
% वापरल्यास आज्ञासंचातर्फे ओळींमधल्या अंतरात कोणताही हस्तक्षेप केला जात नाही. ह्या
% गुंतागुंतीच्या रचनेकरिता पुढील आज्ञा समाविष्ट केल्या आहेत. ह्या आज्ञांसाठी जोनाथन स्प्राट
% व जूल वोऊ ह्यांच्या टॉप-आन्सर्सवरील उत्तरांची मदत झाली.
%    \begin{macrocode}
\RequirePackage{setspace}
\RequirePackage{pgfkeys}
\def\बदल#1{\pgfqkeys{/marathi}{#1}}
\newcommand*\मराठीअंतर{1.5}
\pgfkeys
{%
  marathi/.is family, marathi/.cd,%
  अंतर/.store in=\मराठीअंतर,%
  अंतर/.default=1%
}
\AddToHook{begindocument/before}
{%
  \@ifclassloaded{memoir}%
  {%
    \setSingleSpace{\मराठीअंतर}\SingleSpacing
    \pgfkeys{marathi/अंतर/.code=\setSingleSpace{#1}\SingleSpacing}%
  }%
  {%
    \setstretch{\मराठीअंतर}%
    \pgfkeys{marathi/अंतर/.code=\setstretch{#1}}%
  }%
  \@ifpackageloaded{csquotes}{}{%
    \RequirePackage[style=british]{csquotes}%
  }%
}
\DeclareOption*{\expandafter\बदल\expandafter{\CurrentOption}}
\providecommand{\अंतरबदल}[1]{\बदल{अंतर=#1}}
\ProcessOptions\relax
\def\arraystretch{1.2}
%    \end{macrocode}
% \subsection{भाषा व टंक}
% लाटेक्-ला भाषा पुरवणारे \pkg{babel} व \pkg{polyglossia} हे दोन आज्ञासंच
% आहेत. \pkg{babel} आज्ञासंच लुआलाटेक् व झीलाटेक्-सह चांगले फलित देतो, त्यामुळे ह्या
% आज्ञासंचाला मूलभूत भाषांतरे पुरवणारा आज्ञासंच मानून \pkg{polyglossia} हा आज्ञासंच
% वैकल्पिक ठेवला आहे. \pkg{polyglossia} आज्ञासंच केवळ झीलाटेक्-सह योग्य फलित
% देतो. डेव्हिड कार्लआयल ह्यांनी नोंदवलेल्या
% \href{https://gitlab.com/niruvt/marathi/-/issues/2}{ह्या}
% तक्रारीमुळे भाषांतरे पुरवणारा आज्ञासंच निवडण्यात खूप मदत झाली. शोभिका हा उत्कृष्ट
% देवनागरी टंक टेक्-वितरणासह येत असल्याने त्याची निवड मूलटंक म्हणून करण्यात आली
% आहे. \verb|\टंक| ह्या आज्ञेसह मूलटंक बदलता येतो. \verb|\दुसराटंक| ह्या आज्ञेसह अधिकचे
% टंक निवडता येतात. सुशान्त देवळेकर ह्यांनी गिटपृष्ठावर नोंदवलेल्या
% \href{https://gitlab.com/niruvt/marathi/-/issues/5}{ह्या} सुविधा-विनंतीद्वारे
% टंकांमध्ये मराठी भाषेची निवड करून ठेवण्याचा सल्ला दिला. आवृत्ती १.६.३पासून ही सुविधा
% आज्ञासंचामार्फत पुरवण्यात येत आहे.
% 
%    \begin{macrocode}
\RequirePackage{fontspec}
\ifluatex
\setmainfont[%
  Script=Devanagari,%
  Renderer=Harfbuzz,%
  Language=Marathi%
]%
{Shobhika}
\providecommand{\टंक}[1]{%
  \setmainfont[%
    Script=Devanagari,%
    Renderer=Harfbuzz,%
    Language=Marathi%
  ]%
  {#1}%
}
\providecommand{\दुसराटंक}[2]{%
  \newfontfamily{#1}[%
    Renderer=Harfbuzz,%
    Script=Devanagari,%
    Language=Marathi%
  ]%
  {#2}%
}%
\else
\setmainfont[%
  Script=Devanagari,%
  Mapping=devanagarinumerals,%
  Language=Marathi%
]%
{Shobhika}
\providecommand{\टंक}[1]{%
  \setmainfont[%
    Script=Devanagari,%
    Mapping=devanagarinumerals,%
    Language=Marathi%
  ]%
  {#1}%
}
\providecommand{\दुसराटंक}[2]{%
  \newfontfamily{#1}[%
    Script=Devanagari,%
    Mapping=devanagarinumerals,%
    Language=Marathi%
  ]%
  {#2}%
}
\fi
\@ifpackageloaded{polyglossia}{\setdefaultlanguage{marathi}}
{%
  \usepackage{babel}%
  \babelprovide[%
    import,main,mapdigits,%
    counters/स्वर = अ आ इ ई उ ऊ ए ऐ ओ औ अं अः ॲ ऋ ऌ ऑ,%
    alph=स्वर,%
    counters/anka = एक दोन तीन चार पाच सहा सात आठ नऊ दहा अकरा बारा%
    तेरा चौदा पंधरा सोळा सतरा अठरा एकोणीस वीस एकवीस बावीस तेवीस%
    चोवीस पंचवीस सव्वीस सत्तावीस अठ्ठावीस एकोणतीस तीस एकतीस बत्तीस%
    तेहतीस चौतीस पस्तीस छत्तीस सदतीस अडतीस एकोणचाळीस चाळीस%
    एकेचाळीस बेचाळीस त्रेचाळीस चव्वेचाळीस पंचेचाळीस शेहेचाळीस सत्तेचाळीस%
    अठ्ठेचाळीस एकोणपन्नास पन्नास एकावन्न बावन्न त्रेपन्न चौपन्न पंचावन्न छप्पन्न%
    सत्तावन्न अठ्ठावन्न एकोणसाठ साठ एकसष्ट बासष्ट त्रेसष्ट चौसष्ट पासष्ट सहासष्ट%
    सदुष्ट अडुसष्ट एकोणसत्तर सत्तर एकाहत्तर बाहत्तर त्र्याहत्तर चौऱ्याहत्तर पंचाहत्तर%
    शाहत्तर सत्त्याहत्तर अठ्ठ्याहत्तर एकोणऐंशी ऐंशी एक्याऐंशी ब्याऐंशी त्र्याऐंशी%
    चौऱ्याऐंशी पंच्याऐंशी श्याऐंशी सत्त्याऐंशी अठ्ठ्याऐंशी एकोणनव्वद नव्वद एक्याण्णव%
    ब्याण्णव त्र्याण्णव चौऱ्याण्णव पंचाण्णव शहाण्णव सत्त्याण्णव अठ्याण्णव नव्याण्णव शंभर,%
    counters/व्यंजन = क ख ग घ ङ%
    च छ ज झ ञ%
    ट ठ ड ढ ण%
    त थ द ध न%
    प फ ब भ म%
    य र ल व श ष स ह ळ,%
    Alph=व्यंजन%
  ]%
  {marathi}
  \renewcommand\thepart{\localecounter{स्वर}{part}}%
  \renewcommand\theenumiii{\localecounter{व्यंजन}{enumiii}}%
}
%    \end{macrocode}
% ह्या आज्ञांमुळे धारिका लुआलाटेक् अथवा झीलाटेक् ह्यांपैकी कोणत्याही चालकासह चालवता
% येते. \pkg{polyglossia} हा आज्ञासंच शक्यतो वापरू नये व वापरल्यास केवळ झीलाटेक् हाच
% चालक वापरावा असा सल्ला मी देईन. कारण सध्या चालक व आज्ञासंचांची क्षमता पाहता पुढील
% गोष्टी मराठी आज्ञासंचासह पुरवल्या जातात.
%
% \begin{center}
% \begin{tabular}{|c|ccc|}
% \hline
% चालक & अचूक अक्षरे & अचूक भाषांतरे & देवनागरी अंक \\
% \hline
% \pkg{babel} + झीलाटेक् & \होय & \होय & \होय \\
% \pkg{babel} + लुआलाटेक् & \होय & \होय & \होय \\
% \pkg{polyglossia} + झीलाटेक् & \होय & \होय & \होय \\
% \pkg{polyglossia} + लुआलाटेक् & \होय & \होय & \नाही \\
% \hline
% \end{tabular}
% \captionof{table}{चालक व आज्ञासंचांची तुलना}
% \end{center}
% \subsubsection{गणित}
% लाटेक्-मधील \gls{गणित-क्षेत्र} हे केवळ इंग्रजी अक्षरांची अक्षरजुळणी करू शकते, त्यात देवनागरी
% अक्षरे व अंक ह्यांचा समावेश करण्याकरिता काही आज्ञांचा समावेश करणे आवश्यक
% ठरते. \url{https://tex.stackexchange.com/a/535738/174620} ह्या उत्तरातील
% आज्ञावलीचा वापर करून, त्यात देवनागरी युनिकोड क्रमांकांचा समावेश केल्यामुळे आता लाटेक्-च्या
% गणित-क्षेत्रात देवनागरी आकडे व अक्षरे समाविष्ट करण्यात आली आहेत.
%    \begin{macrocode}
\DeclareSymbolFont{devletters}{\encodingdefault}{Shobhika(0)}{}{}
\ExplSyntaxOn
\int_step_inline:nnn { "0900 } { "097F }
{
  \Umathcode #1 = "0 ~ \use:c{ symdevletters } ~ #1
}
\ExplSyntaxOff
%    \end{macrocode}
% \subsection{नमुना मजकूर}
% \subsubsection{परिच्छेद}
% परिच्छेद ही आज्ञा दस्तऐवजात कुठेही वापरली तरी एक लहानसा परिच्छेद आपोआप छापला
% जातो. त्याकरिता एक मजकूर धारिका आज्ञासंचासोबत येते. ती आज्ञासंचात पुढील आज्ञांनी
% समाविष्ट करून घेतली आहे.
%    \begin{macrocode}
\providecommand{\परिच्छेद}{% ----------------------------------------------------------
% Sample file for LaTeX package `marathi'.
% Copyright © 2020, 2021 निरंजन
% This program is free software: you can redistribute it
% and/or modify it under the terms of the GNU General Public
% License as published by the Free Software Foundation,
% either version 3 of the License, or (at your option) any
% later version.
%
% This program is distributed in the hope that it will be
% useful, but WITHOUT ANY WARRANTY; without even the implied
% warranty of MERCHANTABILITY or FITNESS FOR A PARTICULAR
% PURPOSE. See the GNU General Public License for more
% details.
%
% You should have received a copy of the GNU General Public
% License along with this program. If not, see
% <https://www.gnu.org/licenses/>.
% ----------------------------------------------------------
नमस्कार! हा मजकूर अर्थशून्य आहे. ह्या ठिकाणी काय व कसे छापले जाईल ह्याचा हा केवळ एक नमुना
आहे. जर तुम्ही हे वाचले, तर तुम्हाला कोणतीच माहिती मिळणार नाही. खरेच? ह्यात कोणतीच माहिती
नाही काय? ह्या मजकुरात व `पिढ्ढ करढपाखू' अशा निरर्थक शब्दांमध्ये काही फरक आहे का? हो!
ह्याला आंधळा मजकूर असे म्हणतात. हा मजकूर तुम्हाला निवडलेला टंक कोणता आहे, अक्षरे कशी दिसतात
ह्या सगळ्याबाबत माहिती देतो. ह्यासाठी विशिष्ट शब्दांची गरज नाही, परंतु शब्द वापरल्या गेलेल्या
भाषेशी जुळायला हवेत. \par
}
%    \end{macrocode}
% \subsubsection{दस्तऐवज}
% मागे म्हटल्याप्रमाणे नमुना मजकूर तयार करण्यासाठी ह्या आज्ञासंचाचा वापर करता येतो, परंतु
% त्याकरिता लाटेक्-ला थोडी माहिती पुरवावी लागते. उदा. दस्तऐवजाचा/ची लेखक/लेखिका,
% दस्तऐवजाचं शीर्षक इत्यादी. ही माहिती पुरवण्याचे विशिष्ट स्थान आहे. लाटेक्-मध्ये मूळ
% दस्तऐवज सुरू होण्यापूर्वी ही माहिती पुरविण्याकरिता \gls{आज्ञापीठ} असते, तिथे ही
% माहिती पुरवली जाते, परंतु ह्यामुळे फलित-धारिकेच्या \gls{पायाभूत माहिती}त ती नावे दिसू
% लागतात. ह्यासाठी आज्ञासंचात ही माहिती पुरवली गेली नाही आहे, ह्याउलट सोबत जोडलेल्या
% वेगवेगळ्या दस्तऐवजांमध्ये ती माहिती पुरवली गेली आहे. त्या धारिका केवळ नमुना मजकूर
% पुरवण्यासाठी आहेत. निरनिराळ्या लाटेक्-वर्गांसाठी संबंधित धारिका निवडणे व दस्तऐवजात
% लिहिलेला लाटेक्-वर्ग कोणता आहे हे पाहून त्यानुसार नमुना मजकूर छापणे ह्यासाठीच्या आज्ञा
% पुढीलप्रमाणे.
%    \begin{macrocode}
\newcounter{क्र}
\@ifclassloaded{article}{\setcounter{क्र}{1}}{}
\@ifclassloaded{book}{\setcounter{क्र}{2}}{}
\@ifclassloaded{report}{\setcounter{क्र}{3}}{}
\@ifclassloaded{beamer}{\setcounter{क्र}{4}}{}
\@ifclassloaded{letter}{\setcounter{क्र}{5}}{}
\providecommand{\नमुना}{%
  \ifnum\value{क्र}=1\input{namuna-article}\else
  \ifnum\value{क्र}=2\input{namuna-book}\else
  \ifnum\value{क्र}=3\input{namuna-report}\else
  \ifnum\value{क्र}=4% ----------------------------------------------------------
% Sample file for LaTeX package `marathi'.
% Copyright © 2020, 2021 निरंजन
% This program is free software: you can redistribute it
% and/or modify it under the terms of the GNU General Public
% License as published by the Free Software Foundation,
% either version 3 of the License, or (at your option) any
% later version.
%
% This program is distributed in the hope that it will be
% useful, but WITHOUT ANY WARRANTY; without even the implied
% warranty of MERCHANTABILITY or FITNESS FOR A PARTICULAR
% PURPOSE. See the GNU General Public License for more
% details.
%
% You should have received a copy of the GNU General Public
% License along with this program. If not, see
% <https://www.gnu.org/licenses/>.
% ----------------------------------------------------------
\maketitle
\begin{frame}{शीर्षक}{उपशीर्षक}
  \begin{itemize}
  \item पहिला मुद्दा
  \item दुसरा मुद्दा
  \item तिसरा मुद्दा
  \item चौथा मुद्दा
  \end{itemize}
\end{frame}
\begin{frame}{शीर्षक}{उपशीर्षक}
  \begin{enumerate}
  \item पहिला मुद्दा
  \item दुसरा मुद्दा
  \item तिसरा मुद्दा
  \item चौथा मुद्दा
  \end{enumerate}
\end{frame}
\begin{frame}{शीर्षक}{उपशीर्षक}
  \begin{itemize}
  \item पहिला मुद्दा
  \item दुसरा मुद्दा
  \item तिसरा मुद्दा
  \item चौथा मुद्दा
  \end{itemize}
\end{frame}
\begin{frame}{शीर्षक}{उपशीर्षक}
  \begin{description}
  \item[अ] पहिला मुद्दा
  \item[आ] दुसरा मुद्दा
  \item[इ] तिसरा मुद्दा
  \item[ई] चौथा मुद्दा
  \end{description}
\end{frame}
\begin{frame}{शीर्षक}{उपशीर्षक}
  \begin{itemize}
  \item पहिला मुद्दा
  \item दुसरा मुद्दा
  \item तिसरा मुद्दा
  \item चौथा मुद्दा
  \end{itemize}
\end{frame}
\begin{frame}{थांबे असलेली चौकट}{ह्या चौकटीतील सर्व मुद्दे थांब्यांसह आहेत.}
  \begin{enumerate}
  \item<+-> पहिला मुद्दा
  \item<+-> दुसरा मुद्दा
  \item<+-> तिसरा मुद्दा
  \item<+-> चौथा मुद्दा
  \end{enumerate}
\end{frame}
\else
  \input{namuna-letter}\relax\fi\fi\fi\fi
}
%    \end{macrocode}
% \verb|beamer| लाटेक्-वर्ग वापरताना \verb|serif| ही टंकछटा निवडावी लागते,
% त्याशिवाय देवनागरी लिपी दिसत नाही. त्यासाठीची आज्ञा पुढीलप्रमाणे.
%    \begin{macrocode}
\@ifclassloaded{beamer}{\usefonttheme{serif}}{}
%    \end{macrocode}
% \verb|expex| आज्ञासंचाचे स्थानिकीकरणदेखील पुढील आज्ञांमध्ये समाविष्ट आहे.
%    \begin{macrocode}
\@ifpackageloaded{expex}{%
  \definelabeltype{devanagari}%
  {%
    labelgen=list,%
    labellist={अ,आ,इ,ई,उ,ऊ,ए,ऐ,ओ,औ,अं,अः},%
    labelformat=A.,%
    fullrefformat=XA,%
    labelalign=left,%
    labelwidth=1.5em%
  }%
  \lingset{labeltype=devanagari}%
}%
{}
%    \end{macrocode}
%    \begin{macrocode}
%</package>
%    \end{macrocode}
% \end{implementation}
% \pagebreak
% {%
% \fontfamily{lmr}\selectfont
% \phantomsection
% \addcontentsline{toc}{section}{GNU Free Documentation License}
% \section*{\centering GNU Free Documentation License}
% \begin{center}
% 
% Version 1.3, 3 November 2008
% 
% 
% Copyright \copyright{} 2000, 2001, 2002, 2007, 2008  Free Software Foundation, Inc.
% 
% \bigskip
% 
% \texttt{<https://fsf.org/>}
% 
% \bigskip
% 
% Everyone is permitted to copy and distribute verbatim copies
% of this license document, but changing it is not allowed.
% \end{center}
% 
% 
% \begin{center}
%   {\bf\large Preamble}
% \end{center}
% 
% The purpose of this License is to make a manual, textbook, or other
% functional and useful document ``free'' in the sense of freedom: to
% assure everyone the effective freedom to copy and redistribute it,
% with or without modifying it, either commercially or noncommercially.
% Secondarily, this License preserves for the author and publisher a way
% to get credit for their work, while not being considered responsible
% for modifications made by others.
% 
% This License is a kind of ``copyleft'', which means that derivative
% works of the document must themselves be free in the same sense.  It
% complements the GNU General Public License, which is a copyleft
% license designed for free software.
% 
% We have designed this License in order to use it for manuals for free
% software, because free software needs free documentation: a free
% program should come with manuals providing the same freedoms that the
% software does.  But this License is not limited to software manuals;
% it can be used for any textual work, regardless of subject matter or
% whether it is published as a printed book.  We recommend this License
% principally for works whose purpose is instruction or reference.
% 
% 
% \begin{center}
%   {\Large\bf 1. APPLICABILITY AND DEFINITIONS\par}
% \end{center}
% 
% This License applies to any manual or other work, in any medium, that
% contains a notice placed by the copyright holder saying it can be
% distributed under the terms of this License.  Such a notice grants a
% world-wide, royalty-free license, unlimited in duration, to use that
% work under the conditions stated herein.  The ``\textbf{Document}'', below,
% refers to any such manual or work.  Any member of the public is a
% licensee, and is addressed as ``\textbf{you}''.  You accept the license if you
% copy, modify or distribute the work in a way requiring permission
% under copyright law.
% 
% A ``\textbf{Modified Version}'' of the Document means any work containing the
% Document or a portion of it, either copied verbatim, or with
% modifications and/or translated into another language.
% 
% A ``\textbf{Secondary Section}'' is a named appendix or a front-matter section of
% the Document that deals exclusively with the relationship of the
% publishers or authors of the Document to the Document's overall subject
% (or to related matters) and contains nothing that could fall directly
% within that overall subject.  (Thus, if the Document is in part a
% textbook of mathematics, a Secondary Section may not explain any
% mathematics.)  The relationship could be a matter of historical
% connection with the subject or with related matters, or of legal,
% commercial, philosophical, ethical or political position regarding
% them.
% 
% The ``\textbf{Invariant Sections}'' are certain Secondary Sections whose titles
% are designated, as being those of Invariant Sections, in the notice
% that says that the Document is released under this License.  If a
% section does not fit the above definition of Secondary then it is not
% allowed to be designated as Invariant.  The Document may contain zero
% Invariant Sections.  If the Document does not identify any Invariant
% Sections then there are none.
% 
% The ``\textbf{Cover Texts}'' are certain short passages of text that are listed,
% as Front-Cover Texts or Back-Cover Texts, in the notice that says that
% the Document is released under this License.  A Front-Cover Text may
% be at most 5 words, and a Back-Cover Text may be at most 25 words.
% 
% A ``\textbf{Transparent}'' copy of the Document means a machine-readable copy,
% represented in a format whose specification is available to the
% general public, that is suitable for revising the document
% straightforwardly with generic text editors or (for images composed of
% pixels) generic paint programs or (for drawings) some widely available
% drawing editor, and that is suitable for input to text formatters or
% for automatic translation to a variety of formats suitable for input
% to text formatters.  A copy made in an otherwise Transparent file
% format whose markup, or absence of markup, has been arranged to thwart
% or discourage subsequent modification by readers is not Transparent.
% An image format is not Transparent if used for any substantial amount
% of text.  A copy that is not ``Transparent'' is called ``\textbf{Opaque}''.
% 
% Examples of suitable formats for Transparent copies include plain
% ASCII without markup, Texinfo input format, LaTeX input format, SGML
% or XML using a publicly available DTD, and standard-conforming simple
% HTML, PostScript or PDF designed for human modification.  Examples of
% transparent image formats include PNG, XCF and JPG.  Opaque formats
% include proprietary formats that can be read and edited only by
% proprietary word processors, SGML or XML for which the DTD and/or
% processing tools are not generally available, and the
% machine-generated HTML, PostScript or PDF produced by some word
% processors for output purposes only.
% 
% The ``\textbf{Title Page}'' means, for a printed book, the title page itself,
% plus such following pages as are needed to hold, legibly, the material
% this License requires to appear in the title page.  For works in
% formats which do not have any title page as such, ``Title Page'' means
% the text near the most prominent appearance of the work's title,
% preceding the beginning of the body of the text.
% 
% The ``\textbf{publisher}'' means any person or entity that distributes
% copies of the Document to the public.
% 
% A section ``\textbf{Entitled XYZ}'' means a named subunit of the Document whose
% title either is precisely XYZ or contains XYZ in parentheses following
% text that translates XYZ in another language.  (Here XYZ stands for a
% specific section name mentioned below, such as ``\textbf{Acknowledgements}'',
% ``\textbf{Dedications}'', ``\textbf{Endorsements}'', or ``\textbf{History}''.)  
% To ``\textbf{Preserve the Title}''
% of such a section when you modify the Document means that it remains a
% section ``Entitled XYZ'' according to this definition.
% 
% The Document may include Warranty Disclaimers next to the notice which
% states that this License applies to the Document.  These Warranty
% Disclaimers are considered to be included by reference in this
% License, but only as regards disclaiming warranties: any other
% implication that these Warranty Disclaimers may have is void and has
% no effect on the meaning of this License.
% 
% 
% \begin{center}
%   {\Large\bf 2. VERBATIM COPYING\par}
% \end{center}
% 
% You may copy and distribute the Document in any medium, either
% commercially or noncommercially, provided that this License, the
% copyright notices, and the license notice saying this License applies
% to the Document are reproduced in all copies, and that you add no other
% conditions whatsoever to those of this License.  You may not use
% technical measures to obstruct or control the reading or further
% copying of the copies you make or distribute.  However, you may accept
% compensation in exchange for copies.  If you distribute a large enough
% number of copies you must also follow the conditions in section~3.
% 
% You may also lend copies, under the same conditions stated above, and
% you may publicly display copies.
% 
% 
% \begin{center}
%   {\Large\bf 3. COPYING IN QUANTITY\par}
% \end{center}
% 
% 
% If you publish printed copies (or copies in media that commonly have
% printed covers) of the Document, numbering more than 100, and the
% Document's license notice requires Cover Texts, you must enclose the
% copies in covers that carry, clearly and legibly, all these Cover
% Texts: Front-Cover Texts on the front cover, and Back-Cover Texts on
% the back cover.  Both covers must also clearly and legibly identify
% you as the publisher of these copies.  The front cover must present
% the full title with all words of the title equally prominent and
% visible.  You may add other material on the covers in addition.
% Copying with changes limited to the covers, as long as they preserve
% the title of the Document and satisfy these conditions, can be treated
% as verbatim copying in other respects.
% 
% If the required texts for either cover are too voluminous to fit
% legibly, you should put the first ones listed (as many as fit
% reasonably) on the actual cover, and continue the rest onto adjacent
% pages.
% 
% If you publish or distribute Opaque copies of the Document numbering
% more than 100, you must either include a machine-readable Transparent
% copy along with each Opaque copy, or state in or with each Opaque copy
% a computer-network location from which the general network-using
% public has access to download using public-standard network protocols
% a complete Transparent copy of the Document, free of added material.
% If you use the latter option, you must take reasonably prudent steps,
% when you begin distribution of Opaque copies in quantity, to ensure
% that this Transparent copy will remain thus accessible at the stated
% location until at least one year after the last time you distribute an
% Opaque copy (directly or through your agents or retailers) of that
% edition to the public.
% 
% It is requested, but not required, that you contact the authors of the
% Document well before redistributing any large number of copies, to give
% them a chance to provide you with an updated version of the Document.
% 
% 
% \begin{center}
%   {\Large\bf 4. MODIFICATIONS\par}
% \end{center}
% 
% You may copy and distribute a Modified Version of the Document under
% the conditions of sections 2 and 3 above, provided that you release
% the Modified Version under precisely this License, with the Modified
% Version filling the role of the Document, thus licensing distribution
% and modification of the Modified Version to whoever possesses a copy
% of it.  In addition, you must do these things in the Modified Version:
% 
% \begin{itemize}
% \item[A.] 
%   Use in the Title Page (and on the covers, if any) a title distinct
%   from that of the Document, and from those of previous versions
%   (which should, if there were any, be listed in the History section
%   of the Document).  You may use the same title as a previous version
%   if the original publisher of that version gives permission.
%   
% \item[B.]
%   List on the Title Page, as authors, one or more persons or entities
%   responsible for authorship of the modifications in the Modified
%   Version, together with at least five of the principal authors of the
%   Document (all of its principal authors, if it has fewer than five),
%   unless they release you from this requirement.
%   
% \item[C.]
%   State on the Title page the name of the publisher of the
%   Modified Version, as the publisher.
%   
% \item[D.]
%   Preserve all the copyright notices of the Document.
%   
% \item[E.]
%   Add an appropriate copyright notice for your modifications
%   adjacent to the other copyright notices.
%   
% \item[F.]
%   Include, immediately after the copyright notices, a license notice
%   giving the public permission to use the Modified Version under the
%   terms of this License, in the form shown in the Addendum below.
%   
% \item[G.]
%   Preserve in that license notice the full lists of Invariant Sections
%   and required Cover Texts given in the Document's license notice.
%   
% \item[H.]
%   Include an unaltered copy of this License.
%   
% \item[I.]
%   Preserve the section Entitled ``History'', Preserve its Title, and add
%   to it an item stating at least the title, year, new authors, and
%   publisher of the Modified Version as given on the Title Page.  If
%   there is no section Entitled ``History'' in the Document, create one
%   stating the title, year, authors, and publisher of the Document as
%   given on its Title Page, then add an item describing the Modified
%   Version as stated in the previous sentence.
%   
% \item[J.]
%   Preserve the network location, if any, given in the Document for
%   public access to a Transparent copy of the Document, and likewise
%   the network locations given in the Document for previous versions
%   it was based on.  These may be placed in the ``History'' section.
%   You may omit a network location for a work that was published at
%   least four years before the Document itself, or if the original
%   publisher of the version it refers to gives permission.
%   
% \item[K.]
%   For any section Entitled ``Acknowledgements'' or ``Dedications'',
%   Preserve the Title of the section, and preserve in the section all
%   the substance and tone of each of the contributor acknowledgements
%   and/or dedications given therein.
%   
% \item[L.]
%   Preserve all the Invariant Sections of the Document,
%   unaltered in their text and in their titles.  Section numbers
%   or the equivalent are not considered part of the section titles.
%   
% \item[M.]
%   Delete any section Entitled ``Endorsements''.  Such a section
%   may not be included in the Modified Version.
%   
% \item[N.]
%   Do not retitle any existing section to be Entitled ``Endorsements''
%   or to conflict in title with any Invariant Section.
%   
% \item[O.]
%   Preserve any Warranty Disclaimers.
% \end{itemize}
% 
% If the Modified Version includes new front-matter sections or
% appendices that qualify as Secondary Sections and contain no material
% copied from the Document, you may at your option designate some or all
% of these sections as invariant.  To do this, add their titles to the
% list of Invariant Sections in the Modified Version's license notice.
% These titles must be distinct from any other section titles.
% 
% You may add a section Entitled ``Endorsements'', provided it contains
% nothing but endorsements of your Modified Version by various
% parties---for example, statements of peer review or that the text has
% been approved by an organization as the authoritative definition of a
% standard.
% 
% You may add a passage of up to five words as a Front-Cover Text, and a
% passage of up to 25 words as a Back-Cover Text, to the end of the list
% of Cover Texts in the Modified Version.  Only one passage of
% Front-Cover Text and one of Back-Cover Text may be added by (or
% through arrangements made by) any one entity.  If the Document already
% includes a cover text for the same cover, previously added by you or
% by arrangement made by the same entity you are acting on behalf of,
% you may not add another; but you may replace the old one, on explicit
% permission from the previous publisher that added the old one.
% 
% The author(s) and publisher(s) of the Document do not by this License
% give permission to use their names for publicity for or to assert or
% imply endorsement of any Modified Version.
% 
% 
% \begin{center}
%   {\Large\bf 5. COMBINING DOCUMENTS\par}
% \end{center}
% 
% 
% You may combine the Document with other documents released under this
% License, under the terms defined in section~4 above for modified
% versions, provided that you include in the combination all of the
% Invariant Sections of all of the original documents, unmodified, and
% list them all as Invariant Sections of your combined work in its
% license notice, and that you preserve all their Warranty Disclaimers.
% 
% The combined work need only contain one copy of this License, and
% multiple identical Invariant Sections may be replaced with a single
% copy.  If there are multiple Invariant Sections with the same name but
% different contents, make the title of each such section unique by
% adding at the end of it, in parentheses, the name of the original
% author or publisher of that section if known, or else a unique number.
% Make the same adjustment to the section titles in the list of
% Invariant Sections in the license notice of the combined work.
% 
% In the combination, you must combine any sections Entitled ``History''
% in the various original documents, forming one section Entitled
% ``History''; likewise combine any sections Entitled ``Acknowledgements'',
% and any sections Entitled ``Dedications''.  You must delete all sections
% Entitled ``Endorsements''.
% 
% \begin{center}
%   {\Large\bf 6. COLLECTIONS OF DOCUMENTS\par}
% \end{center}
% 
% You may make a collection consisting of the Document and other documents
% released under this License, and replace the individual copies of this
% License in the various documents with a single copy that is included in
% the collection, provided that you follow the rules of this License for
% verbatim copying of each of the documents in all other respects.
% 
% You may extract a single document from such a collection, and distribute
% it individually under this License, provided you insert a copy of this
% License into the extracted document, and follow this License in all
% other respects regarding verbatim copying of that document.
% 
% 
% \begin{center}
%   {\Large\bf 7. AGGREGATION WITH INDEPENDENT WORKS\par}
% \end{center}
% 
% 
% A compilation of the Document or its derivatives with other separate
% and independent documents or works, in or on a volume of a storage or
% distribution medium, is called an ``aggregate'' if the copyright
% resulting from the compilation is not used to limit the legal rights
% of the compilation's users beyond what the individual works permit.
% When the Document is included in an aggregate, this License does not
% apply to the other works in the aggregate which are not themselves
% derivative works of the Document.
% 
% If the Cover Text requirement of section~3 is applicable to these
% copies of the Document, then if the Document is less than one half of
% the entire aggregate, the Document's Cover Texts may be placed on
% covers that bracket the Document within the aggregate, or the
% electronic equivalent of covers if the Document is in electronic form.
% Otherwise they must appear on printed covers that bracket the whole
% aggregate.
% 
% 
% \begin{center}
%   {\Large\bf 8. TRANSLATION\par}
% \end{center}
% 
% 
% Translation is considered a kind of modification, so you may
% distribute translations of the Document under the terms of section~4.
% Replacing Invariant Sections with translations requires special
% permission from their copyright holders, but you may include
% translations of some or all Invariant Sections in addition to the
% original versions of these Invariant Sections.  You may include a
% translation of this License, and all the license notices in the
% Document, and any Warranty Disclaimers, provided that you also include
% the original English version of this License and the original versions
% of those notices and disclaimers.  In case of a disagreement between
% the translation and the original version of this License or a notice
% or disclaimer, the original version will prevail.
% 
% If a section in the Document is Entitled ``Acknowledgements'',
% ``Dedications'', or ``History'', the requirement (section~4) to Preserve
% its Title (section~1) will typically require changing the actual
% title.
% 
% 
% \begin{center}
%   {\Large\bf 9. TERMINATION\par}
% \end{center}
% 
% 
% You may not copy, modify, sublicense, or distribute the Document
% except as expressly provided under this License.  Any attempt
% otherwise to copy, modify, sublicense, or distribute it is void, and
% will automatically terminate your rights under this License.
% 
% However, if you cease all violation of this License, then your license
% from a particular copyright holder is reinstated (a) provisionally,
% unless and until the copyright holder explicitly and finally
% terminates your license, and (b) permanently, if the copyright holder
% fails to notify you of the violation by some reasonable means prior to
% 60 days after the cessation.
% 
% Moreover, your license from a particular copyright holder is
% reinstated permanently if the copyright holder notifies you of the
% violation by some reasonable means, this is the first time you have
% received notice of violation of this License (for any work) from that
% copyright holder, and you cure the violation prior to 30 days after
% your receipt of the notice.
% 
% Termination of your rights under this section does not terminate the
% licenses of parties who have received copies or rights from you under
% this License.  If your rights have been terminated and not permanently
% reinstated, receipt of a copy of some or all of the same material does
% not give you any rights to use it.
% 
% 
% \begin{center}
%   {\Large\bf 10. FUTURE REVISIONS OF THIS LICENSE\par}
% \end{center}
% 
% 
% The Free Software Foundation may publish new, revised versions
% of the GNU Free Documentation License from time to time.  Such new
% versions will be similar in spirit to the present version, but may
% differ in detail to address new problems or concerns.  See
% \texttt{https://www.gnu.org/licenses/}.
% 
% Each version of the License is given a distinguishing version number.
% If the Document specifies that a particular numbered version of this
% License ``or any later version'' applies to it, you have the option of
% following the terms and conditions either of that specified version or
% of any later version that has been published (not as a draft) by the
% Free Software Foundation.  If the Document does not specify a version
% number of this License, you may choose any version ever published (not
% as a draft) by the Free Software Foundation.  If the Document
% specifies that a proxy can decide which future versions of this
% License can be used, that proxy's public statement of acceptance of a
% version permanently authorizes you to choose that version for the
% Document.
% 
% 
% \begin{center}
%   {\Large\bf 11. RELICENSING\par}
% \end{center}
% 
% 
% ``Massive Multiauthor Collaboration Site'' (or ``MMC Site'') means any
% World Wide Web server that publishes copyrightable works and also
% provides prominent facilities for anybody to edit those works.  A
% public wiki that anybody can edit is an example of such a server.  A
% ``Massive Multiauthor Collaboration'' (or ``MMC'') contained in the
% site means any set of copyrightable works thus published on the MMC
% site.
% 
% ``CC-BY-SA'' means the Creative Commons Attribution-Share Alike 3.0
% license published by Creative Commons Corporation, a not-for-profit
% corporation with a principal place of business in San Francisco,
% California, as well as future copyleft versions of that license
% published by that same organization.
% 
% ``Incorporate'' means to publish or republish a Document, in whole or
% in part, as part of another Document.
% 
% An MMC is ``eligible for relicensing'' if it is licensed under this
% License, and if all works that were first published under this License
% somewhere other than this MMC, and subsequently incorporated in whole
% or in part into the MMC, (1) had no cover texts or invariant sections,
% and (2) were thus incorporated prior to November 1, 2008.
% 
% The operator of an MMC Site may republish an MMC contained in the site
% under CC-BY-SA on the same site at any time before August 1, 2009,
% provided the MMC is eligible for relicensing.
% 
% 
% \begin{center}
%   {\Large\bf ADDENDUM: How to use this License for your documents\par}
% \end{center}
% 
% To use this License in a document you have written, include a copy of
% the License in the document and put the following copyright and
% license notices just after the title page:
% 
% \bigskip
% \begin{quote}
%   Copyright \copyright{}  YEAR  YOUR NAME.
%   Permission is granted to copy, distribute and/or modify this document
%   under the terms of the GNU Free Documentation License, Version 1.3
%   or any later version published by the Free Software Foundation;
%   with no Invariant Sections, no Front-Cover Texts, and no Back-Cover Texts.
%   A copy of the license is included in the section entitled ``GNU
%   Free Documentation License''.
% \end{quote}
% \bigskip
% 
% If you have Invariant Sections, Front-Cover Texts and Back-Cover Texts,
% replace the ``with \dots\ Texts.''\ line with this:
% 
% \bigskip
% \begin{quote}
%   with the Invariant Sections being LIST THEIR TITLES, with the
%   Front-Cover Texts being LIST, and with the Back-Cover Texts being LIST.
% \end{quote}
% \bigskip
% 
% If you have Invariant Sections without Cover Texts, or some other
% combination of the three, merge those two alternatives to suit the
% situation.
% 
% If your document contains nontrivial examples of program code, we
% recommend releasing these examples in parallel under your choice of
% free software license, such as the GNU General Public License,
% to permit their use in free software.
% }%
% \pagebreak
% \printnoidxglossaries