%% \CharacterTable
%%  {Upper-case    \A\B\C\D\E\F\G\H\I\J\K\L\M\N\O\P\Q\R\S\T\U\V\W\X\Y\Z
%%   Lower-case    \a\b\c\d\e\f\g\h\i\j\k\l\m\n\o\p\q\r\s\t\u\v\w\x\y\z
%%   Digits        \0\1\2\3\4\5\6\7\8\9
%%   Exclamation   \!     Double quote  \"     Hash (number) \#
%%   Dollar        \$     Percent       \%     Ampersand     \&
%%   Acute accent  \'     Left paren    \(     Right paren   \)
%%   Asterisk      \*     Plus          \+     Comma         \,
%%   Minus         \-     Point         \.     Solidus       \/
%%   Colon         \:     Semicolon     \;     Less than     \<
%%   Equals        \=     Greater than  \>     Question mark \?
%%   Commercial at \@     Left bracket  \[     Backslash     \\
%%   Right bracket \]     Circumflex    \^     Underscore    \_
%%   Grave accent  \`     Left brace    \{     Vertical bar  \|
%%   Right brace   \}     Tilde         \~}
%%
%\iffalse
%
% (c) Copyright 2007-2024 Apostolos Syropoulos 
% This program can be redistributed and/or modified under the 
% terms of the LaTeX Project Public License Distributed from 
% http://www.latex-project.org/lppl.txt; either
% version 1.3c of the License, or any later version.
%  
% This work has the LPPL maintenance status `maintained'.
%
% Please report errors or suggestions for improvement to
%
%    Apostolos Syropoulos  (asyropoulos@yahoo.com)
%
%\fi
% \CheckSum{802}
% \iffalse This is a Metacommentxel	
%
%<xgreek, >\ProvidesFile{xgreek.sty}
%<xelistings, >\ProvidesFile{xelistings}
%
%<xgreek, > [2024/01/30 v3.4.0 Package `xgreek.sty']
%<xelistings, > [2022/09/04 v1.0 Package `xelistings.sty']
%
%    \begin{macrocode}
%<*driver>
\documentclass{ltxdoc}
\GetFileInfo{xgreek.drv}
\usepackage{xltxtra}
\usepackage{fullpage}
\begin{document}
%\newopentypefeature{Contextuals}{Alts}{+calt}
\setmainfont[Mapping=tex-text,Script=Greek,
             SmallCapsFeatures={Contextuals=Alternate}]{Universal Modern}
\setmonofont{UM Typewriter}
\setsansfont[Mapping=tex-text]{GFS Neohellenic}
   \DocInput{xgreek.dtx}
\end{document}
%</driver>
%    \end{macrocode}
% \fi
%\StopEventually{}
%\title{Greek Language Support for\\ \XeLaTeX\ and Lua\LaTeX}
%\author{Apostolos Syropoulos\\
%        Xanthi, Greece\\
%        \texttt{asyropoulos@yahoo.com}}
% \date{2009/11/23\\ Last Updated 2024/01/30}
%\maketitle
% \begin{abstract}
% The \textsf{xgreek} package provides rudimentary support for Greek language 
% typesetting with \XeLaTeX and Lua\LaTeX. In particular, it provides support for modern 
% Greek (either monotonic or polytonic) and ancient Greek.  
%\end{abstract}
%
%\section{Introduction}
%
% The \textsf{xgreek} package provides rudimentary support for Greek language
% typesetting with \XeLaTeX and Lua\LaTeX. Users will be able to typeset documents in 
% either modern Greek (monotonic or polytonic) or ancient Greek by selecting the appropriate
% package option. The default ``language'' is monotonic Greek.  
%
% Support for Lua\LaTeX\ was provided by Javier Bezos. 
%
% \section{The Source Code of \textsf{xgreek}}
% According to the Unicode standard 
%\begin{center}
%|http://www.unicode.org/Public/UNIDATA/UnicodeData.txt|
%\end{center}
% the uppercase form of the letter GREEK SMALL LETTER EPSILON WITH TONOS is the letter
% GREEK CAPITAL LETTER EPSILON WITH TONOS. According to the grammar of the Greek
% language this is wrong. When text is tranfformed into uppercase, all letters lose 
% accents and when a letter has GREEK DIALYTIKA TONOS, then only the DIALYTIKA remain.
% Thus, the uppercase form of the letter GREEK SMALL LETTER IOTA WITH DIALYTIKA AND TONOS is
% the letter GREEK CAPITAL LETTER IOTA WITH DIALYTIKA. Also, the uppercase form of the letter
% GREEK SMALL LETTER EPSILON WITH TONOS is the letter GREEK CAPITAL LETTER EPSILON. For ancient
% Greek something similar holds---All accents and breathings disappear and only the letters having
% DIALYTIKA keep them. For example, the uppercase form of the letter GREEK SMALL LETTER 
% UPSILON WITH DIALYTIKA AND PERISPOMENI is the letter GREEK CAPITAL LETTER UPSILON WITH DIALYTIKA.
% In addition, for any lowercase letter that has a GREEK YPOGEGRAMMENI the equivalent uppercase letter
% is the one with a PROSGEGRAMMENI. For example, the uppercase form of the letter GREEK SMALL LETTER ETA 
% WITH OXIA AND YPOGEGRAMMENI is the letter GREEK CAPITAL LETTER ETA WITH PROSGEGRAMMENI. However, there is
% one exception: the word GREEK SMALL LETTER ETA WITH PSILI AND OXIA (GREEK SMALL LETTER ETA WITH TONOS
% in Modern Greek), which is the equiavalent of the English word ``or.'' In order not to confuse it with
% the feminine article in the nominative case (i.e., the letter GREEK SMALL LETTER ETA WITH DASIA), 
% this article keeps the accent in an uppercase letters only text. Unfortunately, this feature cannot be
% easily implemented since it would require a rule of the form
%\begin{center}
% ``SPACE'' ``GREEK SMALL LETTER ETA WITH PSILI AND OXIA'' ``SPACE''\\ --->\\ ``SPACE'' ``GREEK CAPITAL LETTER ETA WITH OXIA'' ``SPACE''
%\end{center}
% 
% It is a fact that \LaTeX\ {\em blindly} follows the Unicode standard and so the command |\MakeUppercase| produces wrong output. 
% For this reason, in previous versions of this package, I introduced many pairs of |\uccode| and |\lccode| commands to fix
% this problem. Quite recently, I realize that these commands do not produce the expected result as the new implementation of
% the command |\MakeUppercase| completely ignores these commands. Fortunately, the person who did this re-implementation,
% defined the new command |\DeclareUppercaseMapping| which maps the character code of a lowercase letter to the actual
% uppercase character. The commands that follow implement the grammatical rules of the Greek language except of course the
% rule for the Greek disjunctive conjunction.
%    \begin{macrocode}
%<*xgreek>
\message{Package 'xgreek' version 3.4.0 by Apostolos Syropoulos}
\DeclareUppercaseMapping{"0390}{Ϊ} \DeclareLowercaseMapping{"0390}{ΐ}
\DeclareUppercaseMapping{"03AC}{Α} \DeclareLowercaseMapping{"03AC}{ά}
\DeclareUppercaseMapping{"03AD}{Ε} \DeclareLowercaseMapping{"03AD}{έ}
\DeclareUppercaseMapping{"03AE}{Η} \DeclareLowercaseMapping{"03AE}{ή}
\DeclareUppercaseMapping{"03AF}{Ι} \DeclareLowercaseMapping{"03AF}{ί}
\DeclareUppercaseMapping{"03B0}{Ϋ} \DeclareLowercaseMapping{"03B0}{ΰ}
\DeclareUppercaseMapping{"03CA}{Ϊ} \DeclareLowercaseMapping{"03CA}{ϊ}
\DeclareUppercaseMapping{"03CB}{Ϋ} \DeclareLowercaseMapping{"03CB}{ϋ}
\DeclareUppercaseMapping{"03CC}{Ο} \DeclareLowercaseMapping{"03CC}{ό}
\DeclareUppercaseMapping{"03CD}{Υ} \DeclareLowercaseMapping{"03CD}{ύ}
\DeclareUppercaseMapping{"03CE}{Ω} \DeclareLowercaseMapping{"03CE}{ώ}
\DeclareUppercaseMapping{"1F00}{Α} \DeclareUppercaseMapping{"0386}{Α}
\DeclareUppercaseMapping{"1F01}{Α} \DeclareUppercaseMapping{"0388}{Ε}
\DeclareUppercaseMapping{"1F02}{Α} \DeclareUppercaseMapping{"0389}{Η}
\DeclareUppercaseMapping{"1F03}{Α} \DeclareUppercaseMapping{"038A}{Ι}
\DeclareUppercaseMapping{"1F04}{Α} \DeclareUppercaseMapping{"038C}{Ο}
\DeclareUppercaseMapping{"1F05}{Α} \DeclareUppercaseMapping{"038E}{Υ}
\DeclareUppercaseMapping{"1F06}{Α} \DeclareUppercaseMapping{"038F}{Ω}
\DeclareUppercaseMapping{"1F07}{Α}
\DeclareUppercaseMapping{"1F10}{Ε}
\DeclareUppercaseMapping{"1F11}{Ε}
\DeclareUppercaseMapping{"1F12}{Ε}
\DeclareUppercaseMapping{"1F13}{Ε}
\DeclareUppercaseMapping{"1F14}{Ε}
\DeclareUppercaseMapping{"1F15}{Ε}
\DeclareUppercaseMapping{"1F20}{Η}
\DeclareUppercaseMapping{"1F21}{Η}
\DeclareUppercaseMapping{"1F22}{Η}
\DeclareUppercaseMapping{"1F23}{Η}
\DeclareUppercaseMapping{"1F24}{Η}
\DeclareUppercaseMapping{"1F25}{Η}
\DeclareUppercaseMapping{"1F26}{Η}
\DeclareUppercaseMapping{"1F27}{Η}
\DeclareUppercaseMapping{"1F30}{Ι}
\DeclareUppercaseMapping{"1F31}{Ι}
\DeclareUppercaseMapping{"1F32}{Ι}
\DeclareUppercaseMapping{"1F33}{Ι}
\DeclareUppercaseMapping{"1F34}{Ι}
\DeclareUppercaseMapping{"1F35}{Ι}
\DeclareUppercaseMapping{"1F36}{Ι}
\DeclareUppercaseMapping{"1F37}{Ι}
\DeclareUppercaseMapping{"1F40}{Ο}
\DeclareUppercaseMapping{"1F41}{Ο}
\DeclareUppercaseMapping{"1F42}{Ο}
\DeclareUppercaseMapping{"1F43}{Ο}
\DeclareUppercaseMapping{"1F44}{Ο}
\DeclareUppercaseMapping{"1F45}{Ο}
\DeclareUppercaseMapping{"1F50}{Υ}
\DeclareUppercaseMapping{"1F51}{Υ}
\DeclareUppercaseMapping{"1F52}{Υ}
\DeclareUppercaseMapping{"1F53}{Υ}
\DeclareUppercaseMapping{"1F54}{Υ}
\DeclareUppercaseMapping{"1F55}{Υ}
\DeclareUppercaseMapping{"1F56}{Υ}
\DeclareUppercaseMapping{"1F57}{Υ}
\DeclareUppercaseMapping{"1F60}{Ω}
\DeclareUppercaseMapping{"1F61}{Ω}
\DeclareUppercaseMapping{"1F62}{Ω}
\DeclareUppercaseMapping{"1F63}{Ω}
\DeclareUppercaseMapping{"1F64}{Ω}
\DeclareUppercaseMapping{"1F65}{Ω}
\DeclareUppercaseMapping{"1F66}{Ω}
\DeclareUppercaseMapping{"1F67}{Ω}
\DeclareUppercaseMapping{"1F70}{Α} \DeclareUppercaseMapping{"1F71}{Α}
\DeclareUppercaseMapping{"1F72}{Ε} \DeclareUppercaseMapping{"1F73}{Ε}
\DeclareUppercaseMapping{"1F74}{Η} \DeclareUppercaseMapping{"1F75}{Η}
\DeclareUppercaseMapping{"1F76}{Ι} \DeclareUppercaseMapping{"1F77}{Ι}
\DeclareUppercaseMapping{"1F78}{Ο} \DeclareUppercaseMapping{"1F79}{Ο}
\DeclareUppercaseMapping{"1F7A}{Υ} \DeclareUppercaseMapping{"1F7B}{Υ}
\DeclareUppercaseMapping{"1F7C}{Ω} \DeclareUppercaseMapping{"1F7D}{Ω}
\DeclareUppercaseMapping{"1F80}{ᾼ}
\DeclareUppercaseMapping{"1F81}{ᾼ}
\DeclareUppercaseMapping{"1F82}{ᾼ}
\DeclareUppercaseMapping{"1F83}{ᾼ}
\DeclareUppercaseMapping{"1F84}{ᾼ}
\DeclareUppercaseMapping{"1F85}{ᾼ}
\DeclareUppercaseMapping{"1F86}{ᾼ}
\DeclareUppercaseMapping{"1F87}{ᾼ}
\DeclareUppercaseMapping{"1F90}{ῌ}
\DeclareUppercaseMapping{"1F91}{ῌ}
\DeclareUppercaseMapping{"1F92}{ῌ}
\DeclareUppercaseMapping{"1F93}{ῌ}
\DeclareUppercaseMapping{"1F94}{ῌ}
\DeclareUppercaseMapping{"1F95}{ῌ}
\DeclareUppercaseMapping{"1F96}{ῌ}
\DeclareUppercaseMapping{"1F97}{ῌ}
\DeclareUppercaseMapping{"1FA0}{ῼ}
\DeclareUppercaseMapping{"1FA1}{ῼ}
\DeclareUppercaseMapping{"1FA2}{ῼ}
\DeclareUppercaseMapping{"1FA3}{ῼ}
\DeclareUppercaseMapping{"1FA4}{ῼ}
\DeclareUppercaseMapping{"1FA5}{ῼ}
\DeclareUppercaseMapping{"1FA6}{ῼ}
\DeclareUppercaseMapping{"1FA7}{ῼ}
\DeclareUppercaseMapping{"1FB2}{ᾼ}
\DeclareUppercaseMapping{"1FB3}{ᾼ}
\DeclareUppercaseMapping{"1FB4}{ᾼ}
\DeclareUppercaseMapping{"1FB6}{Α}
\DeclareUppercaseMapping{"1FB7}{ᾼ}
\DeclareUppercaseMapping{"1FC2}{ῌ}
\DeclareUppercaseMapping{"1FC3}{ῌ}
\DeclareUppercaseMapping{"1FC4}{ῌ}
\DeclareUppercaseMapping{"1FC6}{Η}
\DeclareUppercaseMapping{"1FC7}{ῌ}
\DeclareUppercaseMapping{"1FD2}{Ϊ}
\DeclareUppercaseMapping{"1FD3}{Ϊ}
\DeclareUppercaseMapping{"1FD6}{Ι}
\DeclareUppercaseMapping{"1FD7}{Ϊ}
\DeclareUppercaseMapping{"1FE2}{Ϋ}
\DeclareUppercaseMapping{"1FE3}{Ϋ}
\DeclareUppercaseMapping{"1FE4}{Ρ}
\DeclareUppercaseMapping{"1FE5}{Ρ}
\DeclareUppercaseMapping{"1FE6}{Υ}
\DeclareUppercaseMapping{"1FE7}{Ϋ}
\DeclareUppercaseMapping{"1FF2}{ῼ}
\DeclareUppercaseMapping{"1FF3}{ῼ}
\DeclareUppercaseMapping{"1FF4}{ῼ}
\DeclareUppercaseMapping{"1FF6}{Ω}
\DeclareUppercaseMapping{"1FF7}{ῼ}
%    \end{macrocode}
% Next I define the various strings that correspond to the standard \LaTeX\ captions.
% I first define the strings for monotonic Greek. 
%    \begin{macrocode}
\def\prefacename{Πρόλογος}%
\def\refname{Αναφορές}%
\def\abstractname{Περίληψη}%
\def\bibname{Βιβλιογραφία}%
\def\chaptername{Κεφάλαιο}%
\def\appendixname{Παράρτημα}%
\def\contentsname{Περιεχόμενα}%
\def\listfigurename{Κατάλογος σχημάτων}%
\def\listtablename{Κατάλογος πινάκων}%
\def\indexname{Ευρετήριο}%
\def\figurename{Σχήμα}%
\def\tablename{Πίνακας}%
\def\partname{Μέρος}%
\def\enclname{Συνημμένα}%
\def\ccname{Κοινοποίηση}%
\def\headtoname{Προς}%
\def\pagename{Σελίδα}%
\def\seename{βλέπε}%
\def\alsoname{βλέπε επίσης}%
\def\proofname{Απόδειξη}%
\def\glossaryname{Γλωσσάρι}%
%    \end{macrocode}
% Macro |\polytonicn@mes| is invoked when polytonic Greek is the main language of the document.
%    \begin{macrocode} 
\def\polytonicn@mes{%
  \def\refname{Ἀναφορὲς}%
  \def\indexname{Εὑρετήριο}%
  \def\figurename{Σχῆμα}%
  \def\headtoname{Πρὸς}%
  \def\alsoname{βλέπε ἐπίσης}%
  \def\proofname{Ἀπόδειξη}%
}
%    \end{macrocode}
% Macro |\@ncientn@mes| is invoked when ancient Greek is the main language of the document (Dinitrios Filippou
% spotted a couple of spelling errors in the list below).
%    \begin{macrocode} 
\def\@ncientn@mes{%
  \def\prefacename{Προοίμιον}%
  \def\abstractname{Περίληψις}%
  \def\bibname{Βιβλιογραφία}%
  \def\chaptername{Κεφάλαιον}%
  \def\appendixname{Παράρτημα}%
  \def\contentsname{Περιεχόμενα}%
  \def\listfigurename{Κατάλογος σχημάτων}%
  \def\listtablename{Κατάλογος πινάκων}%
  \def\indexname{Εὑρετήριον}%
  \def\tablename{Πίναξ}%
  \def\partname{Μέρος}%
  \def\enclname{Συνημμένως}%
  \def\ccname{Κοινοποίησις}%
  \def\headtoname{Πρὸς}%
  \def\pagename{Σελὶς}%
  \def\seename{ὅρα}%
  \def\alsoname{ὅρα ὡσαύτως}%
  \def\proofname{Ἀπόδειξις}%
  \def\glossaryname{Γλωσσάριον}%
  \def\refname{Ἀναφοραὶ}%
  \def\figurename{Σχῆμα}%
  \def\headtoname{Πρὸς}%
}
%    \end{macrocode}
% I redefine |\today| so as to produce dates in Greek. The
% names of months are defined by the macro |\gr@month|. 
%    \begin{macrocode}
\def\gr@month{%
  \ifcase\month\or Ιανουαρίου\or Φεβρουαρίου\or Μαρτίου\or Απριλίου\or
    Μαΐου\or Ιουνίου\or Ιουλίου\or Αυγούστου\or
     Σεπτεμβρίου\or Οκτωβρίου\or Νοεμβρίου\or Δεκεμβρίου\fi}
\def\today{\number\day \space \gr@month\space \number\year}
%    \end{macrocode}
% When either polytonic Greek or ancient Greek is the main language of the document,
% then the macro |\gr@c@month| becomes active.  
%    \begin{macrocode}
\def\gr@c@month{%
  \ifcase\month\or Ἰανουαρίου\or Φεβρουαρίου\or Μαρτίου\or Ἀπριλίου\or 
   Μαΐου\or Ἰουνίου\or Ἰουλίου\or Αὐγούστου\or Σεπτεμβρίου\or
  Ὀκτωβρίου\or Νοεμβρίου\or Δεκεμβρίου\fi}
%    \end{macrocode}
% Next, I define a few macros that allow one to access characters
% that are not usually easily accessible from the keyboard (e.g., the sampi or the
% koppa symbol). The list includes a command for the Unicode symbol GREEK ANO TELEIA,
% which, in some systems, is confused with MIDDLE DOT. The use of command |\numer@lsign|
% will be explained later.
%    \begin{macrocode}
\def\anwtonos{ʹ} %GREEK NUMERAl SIGN
\let\numer@lsign\anwtonos
\def\katwtonos{͵} %GREEK LOWER NUMERAL SIGN
\def\koppa{\char"03DF\relax}
\def\sampi{\char"03E1\relax}
\def\Digamma{\char"03DC\relax}
\def\ddigamma{\char"03DD\relax}
\def\anoteleia{\char"0387\relax}
\def\euro{\char"20AC\relax}
\def\permill{\char"2030\relax}
%    \end{macrocode}
% Many users prefer the use of the letters sigma and tau instead of the stigma symbol in
% Greek numerals, therefore, by default the |\stigma| command expands to ``στ''. 
%    \begin{macrocode}
\def\stigma{στ\relax}
%    \end{macrocode}
% The following commands take care of the basic rules of typography. Note that the first command
% changes the way space is added after punctuation symbols and the last two commands force \LaTeX\
% to add indentation space to the first paragraph after a header. Since a number of users need, for 
% their own reasons, to be able to disable this particular feature I have introduced a new package
% option, namely |noindentfirst|, which restores the default behavior. In order to be able
% to do this I need the original value of the boolean variable |\@afterindentfalse|.
%    \begin{macrocode}
\frenchspacing
\let\@saveafterindentfalse\@afterindentfalse
\let\@afterindentfalse\@afterindenttrue
\@afterindenttrue
%    \end{macrocode}
% Lua\LaTeX\ and \XeLaTeX\ have different ways to load hyphenation patterns. Package 
% \textsf{luahyphenrules} by Javier Bezos facilitates this process for people who
% want to use Lua\LaTeX\ and the ``traditional'' way to load hyphenation patterns. 
% To ensure proper inclusion of LuaTeX staff, I use the following ``idiom'':
% \begin{center}
% |\ifx\directlua\undefined |\texttt{\textit{non Lua\LaTeX\ code}}|\else |%
% \texttt{\textit{Lua\LaTeX\ code}}|\fi|
% \end{center} 
%    \begin{macrocode}
\ifx\directlua\undefined\else\RequirePackage{luahyphenrules}\fi
%    \end{macrocode}
% The code that follows specifies which hyphenation patterns will be active. The \XeLaTeX\ code
% is quite standard and depends on the \textsf{babel} pattern loading mechanism, while the
% Lua\LaTeX\ code uses the |\HyphenRules| macro, which has essentially the functionality 
% of the |\selectlanguage| macro.
%    \begin{macrocode}
\DeclareOption{monogreek}{%
   \ifx\directlua\undefined%
   \language\l@monogreek\else\HyphenRules{monogreek}\fi%
}
\DeclareOption{polygreek}{%
   \ifx\directlua\undefined%
   \language\l@polygreek\else\HyphenRules{polygreek}\fi%
   \polytonicn@mes%
   \let\gr@month\gr@c@month%
}
\DeclareOption{ancientgreek}{%
   \ifx\directlua\undefined%
   \language\l@ancientgreek\else\HyphenRules{ancientgreek}\fi%
   \@ncientn@mes%
   \let\gr@month\gr@c@month%
}
%    \end{macrocode}
% If a user wants to use the stigma symbol in Greek numerals, she should use the
% |stigma| option.
%    \begin{macrocode}
\DeclareOption{stigma}{%
   \def\stigma{\char"03DB\relax}
}
%    \end{macrocode}
% As noted above, the new option |noindentfirst| restores the default \LaTeX\ behavior of adding no 
% indentation to the first paragraph after any header. 
%    \begin{macrocode}
\DeclareOption{noindentfirst}{%
   \let\@afterindentfalse\@saveafterindentfalse
}
%    \end{macrocode}
% Nowadays it is customary in Greece to use Greek numerals without the GREEK NUMERAL SIGN at the end
% of a numeral. Thus, the |nonumeralsign| option disables the typesetting of the GREEK NUMERAL SIGN
% at the end of Greek numerals. 
%    \begin{macrocode}
\DeclareOption{nonumeralsign}{%
   \let\numer@lsign\relax
}
%    \end{macrocode}
%  Package \textsf{listings} does not work properly with UTF-8 encoded files. So this
% option should be used whenever one wants to use this package and see Greek text come out
% correctly. In version 3.1.0 of this package, I included code that modified the source code of 
% package \textsf{listings}. However, this decision was wrong. In particular, when one did not use 
% the corresponding \texttt{listings} option, processing of the input file stopped with an error message 
% about a text line that contains an invalid character. So the best way to solve this problem was to move 
% the code to a different file and create essentially a new package. This package is automatically loaded 
% when the user pecifies the \texttt{listings} option. To make this possible, I used a boolean 
% variable.
%    \begin{macrocode}
\newif\if@mylistings
\@mylistingsfalse
\DeclareOption{listings}{\@mylistingstrue}
%    \end{macrocode}
% By default the |monogreek| option is activated. 
%    \begin{macrocode}
\ExecuteOptions{monogreek}
\ProcessOptions
%    \end{macrocode}
% If the user has enabled the |listings| option, then the package loads the package \textsf{xelistings}.
%    \begin{macrocode}
\if@mylistings
\RequirePackage{xelistings}
\fi
%    \end{macrocode}
% Now I am going to define the macros that typeset alphabetic Greek numerals. The code 
% is borrowed from the Greek option for the babel package.
%  \begin{macro}{\gr@ill@value}
%    When the argument of |\greeknumeral| has a value outside of the
%    acceptable bounds ($0 < x < 999999$) a warning will be issued
%    (and nothing will be printed).
%    \begin{macrocode}
\def\gr@ill@value#1{%
  \PackageWarning{xgreek}{Illegal value (#1) for greeknumeral}}
%    \end{macrocode}
%  \end{macro}
%  \begin{macro}{\anw@true}
%  \begin{macro}{\anw@false}
%  \begin{macro}{\anw@print}
%    When a large number with three \emph{trailing} zeros is to be
%    printed those zeros \emph{and} the numeric mark need to be
%    discarded. As each `digit' is processed by a separate macro
%    \emph{and} because the processing needs to be expandable we need
%    some helper macros that help remember to \emph{not} print the
%    numeric mark (|\numer@lsign|).
%
%    The command |\anw@false| switches the printing of the numeric
%    mark off by making |\anw@print| expand to nothing. The command
%    |\anw@true| (re)enables the printing of the numeric marc. These
%    macro's need to be robust in order to prevent improper expansion
%    during writing to files or during |\uppercase|.
%    \begin{macrocode}
\DeclareRobustCommand\anw@false{%
  \DeclareRobustCommand\anw@print{}}
\DeclareRobustCommand\anw@true{%
  \DeclareRobustCommand\anw@print{\numer@lsign}}
\anw@true
%    \end{macrocode}
%  \end{macro}
%  \end{macro}
%  \end{macro}
%
%  \begin{macro}{\@greeknumeral}
%    This command is used to get Greek numerals. The command uses
%    |\numer@lsign| to get the NUMERAL SIGN. Obviously, when the
%    user has specified the \texttt{no\-numeral\-sign} option, then numeral
%    comes out without the trailing NUMERAL SIGN. However, when a user
%    wants to typeset a Greek numeral, the numeral must come out correctly,
%    regardless of what appears in headers, etc. And that is exactly the
%    reason why this command is inaccessible to users.
%    The command |\@greeknumeral| needs to be \emph{fully} expandable
%    in order to get the right information in auxiliary
%    files. Therefore we use a big |\if|-construction to check the
%    value of the argument and start the parsing at the right level.
%    \begin{macrocode}
\def\@greeknumeral#1{%
%    \end{macrocode}
%    If the value is negative or zero nothing is printed and a warning
%    is issued.
%    \begin{macrocode}
  \ifnum#1<\@ne\space\gr@ill@value{#1}%
  \else
    \ifnum#1<10\expandafter\gr@num@i\number#1%
    \else
      \ifnum#1<100\expandafter\gr@num@ii\number#1%
      \else
%    \end{macrocode}
%    The available shorthands for 1.000 (|\@m|) and 10.000
%    (|\@M|) are used to save a few tokens.
%    \begin{macrocode}
        \ifnum#1<\@m\expandafter\gr@num@iii\number#1%
        \else
          \ifnum#1<\@M\expandafter\gr@num@iv\number#1%
          \else
            \ifnum#1<100000\expandafter\gr@num@v\number#1%
            \else
              \ifnum#1<1000000\expandafter\gr@num@vi\number#1%
              \else
%    \end{macrocode}
%    If the value is too large, nothing is printed and a warning
%    is issued.
%    \begin{macrocode}
                \space\gr@ill@value{#1}%
              \fi
            \fi
          \fi
        \fi
      \fi
    \fi
  \fi
}
%    \end{macrocode}
%  \end{macro}
%
%    What is left to make complete the definition of command |\greeknumeral| is a set of macros to produce
%    the various digits.
%  \begin{macro}{\gr@num@i}
%  \begin{macro}{\gr@num@ii}
%  \begin{macro}{\gr@num@iii}
%    As there is no ``digit'' representing $0$ in this system, the zeros
%    are simply discarded. When there is a large number with three
%    \emph{trailing} zeros also the numeric mark is discarded. 
%    Therefore these macros need to pass the information to each other
%    about the (non-)translation of a zero.
%    \begin{macrocode}
\def\gr@num@i#1{%
  \ifcase#1\or α\or β\or γ\or δ\or ε\or \stigma\or ζ\or η\or θ\fi
  \ifnum#1=\z@\else\anw@true\fi\anw@print}
\def\gr@num@ii#1{%
  \ifcase#1\or ι\or κ\or λ\or μ\or ν\or ξ\or ο\or π\or \koppa\fi
  \ifnum#1=\z@\else\anw@true\fi\gr@num@i}
\def\gr@num@iii#1{%
  \ifcase#1\or ρ\or σ\or τ\or υ\or φ\or χ\or ψ\or ω\or \sampi\fi
  \ifnum#1=\z@\anw@false\else\anw@true\fi\gr@num@ii}
%    \end{macrocode}
%  \end{macro}
%  \end{macro}
%  \end{macro}
%
%  \begin{macro}{\gr@num@iv}
%  \begin{macro}{\gr@num@v}
%  \begin{macro}{\gr@num@vi}
%    The first three ``digits'' always have the numeric mark, except
%    when one is discarded because it's value is zero.
%    \begin{macrocode}
\def\gr@num@iv#1{%
  \ifnum#1=\z@\else\katwtonos\fi
  \ifcase#1\or α\or β\or γ\or δ\or ε\or \stigma\or ζ\or η\or θ\fi
  \gr@num@iii}
\def\gr@num@v#1{%
  \ifnum#1=\z@\else\katwtonos\fi
  \ifcase#1\or ι\or κ\or λ\or μ\or ν\or ξ\or ο\or π\or \koppa\fi
  \gr@num@iv}
\def\gr@num@vi#1{%
  \katwtonos
  \ifcase#1\or ρ\or σ\or τ\or υ\or φ\or χ\or ψ\or ω\or \sampi\fi
  \gr@num@v}
%    \end{macrocode}
%  \end{macro}
%  \end{macro}
%  \end{macro}
%  \begin{macro}{\@Greeknumeral}
%    The command |\@Greeknumeral| prints uppercase Greek numerals. 
%    The parsing is performed by the macro |\@greeknumeral|. The printing
%    of the NUMERAL SIGN depends on the value of |\numer@lsign|.
%    \begin{macrocode}
\def\@Greeknumeral#1{%
  \expandafter\MakeUppercase\expandafter{\@greeknumeral{#1}}}
%    \end{macrocode}
%  \end{macro}
%  \begin{macro}{\greeknumeral}
%   This command prints lowercase Greek numerals and the NUMERAL SIGN
%   is always printed.
%    \begin{macrocode}
\def\greeknumeral#1{%
  \let\@numer@lsign\numer@lsign%
  \let\numer@lsign\anwtonos%
  \@greeknumeral{#1}
  \let\numer@lsign\@numer@lsign}
%    \end{macrocode}
%  \end{macro}
%  \begin{macro}{\Greeknumeral}
%   This command prints uppercase Greek numerals and the NUMERAL SIGN
%   is always printed.
%    \begin{macrocode}
\def\Greeknumeral#1{%
  \let\@numer@lsign\numer@lsign%
  \let\numer@lsign\anwtonos%
  \@Greeknumeral{#1}
  \let\numer@lsign\@numer@lsign}
%    \end{macrocode}
%  \end{macro}
%
% The alphabetic numbering system is not the only numbering system employed by Greeks.
% In fact, Greeks used various systems that are now known as {\em acrophonic} numbering
% systems. Many scholars are familiar with the acrophonic Attic numbering system and the
% the command |\atticnum| can be used to generate acrophonic Attic numerals.  
% The acrophonic Attic numbering system, like the Roman one, employs
% letters to denote important numbers. Multiple occurrence of a letter denote
% a multiple of the ``important'' number, e.g., the letter Ι denotes 1, so
% ΙΙΙ denotes 3. Here are the basic digits used in the acrophonic Attic numbering
% system:
% \begin{itemize}
%  \item Ι denotes the number one (1)
%  \item Π denotes the number five (5)
%  \item Δ denotes the number ten (10)
%  \item Η denotes the number one hundred (100)
%  \item Χ denotes the number one thousand (1000)
%  \item Μ denotes the number ten thousands (10000)
%\end{itemize}
% Moreover,  the letters Δ, Η, Χ, and Μ under the letter ^^^^^^010143 (a form of Π)
% denote five times their original value. In particular, the symbol ^^^^^10144, denotes 
% the number 50, the symbol ^^^^^10145 denotes the number 500, the symbol ^^^^^10146
% denotes the number 5000, and the symbol ^^^^^10147 denotes the number 50,000. It 
% must be noted that the numbering system does not provide negative numerals or a symbol for
% zero. 
%\begin{macro}{\@@atticnum}            
% Now, let me definite the macro 
% |\@@atticnum|. This macro uses one integer variable (or counter in 
% \TeX's jargon.)
%    \begin{macrocode}
\newcount\@attic@num
%    \end{macrocode}
% The macro |\@@atticnum| is also defined as a robust command.
%    \begin{macrocode}
\DeclareRobustCommand*{\@@atticnum}[1]{%
%    \end{macrocode}
% After assigning to variable |\@attic@num| the value of the macro's argument, 
% we make sure that the argument is in the expected range, i.e., it is greater
% than zero, and less or equal to $249999$.  In case it is not, it simply 
% produces a |\space|, warns the user about it and quits. Although, the
% |\atticnum| macro is capable to produce an Athenian numeral for even greater
% intergers, the following argument by Claudio Beccari convinced me to place
% this upper limit:
% \begin{quote} 
% According to psychological perception studies (that ancient Athenians
% and Romans perfectly knew without needing to study Freud and Jung)
% living beings (which includes at least all vertebrates, not only
% humans) can perceive up to four randomly set objects of the same kind   
% without the need of counting, the latter activity being a specific
% acquired ability of human kind; the biquinary numbering notation
% used by the Athenians and the Romans exploits this natural
% characteristic of human beings.
% \end{quote}
%    \begin{macrocode}
        \@attic@num#1\relax
        \ifnum\@attic@num<\@ne%
          \space%
          \PackageWarning{xgreek}{%
          Illegal value (\the\@attic@num) for acrophonic 
          Attic numeral}%
        \else\ifnum\@attic@num>249999%
          \space%
          \PackageWarning{xgreek}{%
          Value too large (\the\@attic@num) for acrophonic 
          Attic numeral}%
        \else
%    \end{macrocode}
% Having done all the necessary checks, it is possible to proceed with the actual
% computation. If the number is greater than $49999$, then it certainly
% has at least one ^^^^^10147 ``digit''. The macro finds all such digits by continuously
% subtracting $50000$ from |\@attic@num|, until |\@attic@num| becomes less than
% $50000$. 
%    \begin{macrocode}
            \@whilenum\@attic@num>49999\do{%
               ^^^^^^010147\advance\@attic@num-50000}%
%    \end{macrocode}
% Next the macro checks for tens of thousands.
%    \begin{macrocode}
            \@whilenum\@attic@num>9999\do{%
               M\advance\@attic@num-\@M}%
%    \end{macrocode}
% Since a number can have only one ^^^^^10146 ``digit'' (equivalent to 5000), it 
% is easy to check whether is should have one and produce the corresponding numeral when
% it does have one.
%    \begin{macrocode}
            \ifnum\@attic@num>4999%
               ^^^^^^010146\advance\@attic@num-5000%
            \fi\relax
%    \end{macrocode}
% The macro should also check for thousands, the same way it checked for tens of thousands.
%    \begin{macrocode}
            \@whilenum\@attic@num>999\do{%
               Χ\advance\@attic@num-\@m}%
%    \end{macrocode} 
% Since a numeral can have at most one ^^^^^10145 ``digit'' (equivalent to 500), this should be
% handled the way the macro handled the case of the five thousands ``digit''.
%    \begin{macrocode}
            \ifnum\@attic@num>499%
               ^^^^^^010145\advance\@attic@num-500%
            \fi\relax
%    \end{macrocode}
% It is time to check hundreds, which follow the same pattern as thousands.
%    \begin{macrocode}
            \@whilenum\@attic@num>99\do{%
               Η\advance\@attic@num-100}%
%    \end{macrocode}
% A numeral can have only one ^^^^^10144 ``digit'' (equivalent to 50).
%    \begin{macrocode} 
            \ifnum\@attic@num>49%
               ^^^^^^010144\advance\@attic@num-50%
            \fi\relax
%    \end{macrocode}
% The macro now checks now for tens digit.
%    \begin{macrocode}         
            \@whilenum\@attic@num>9\do{%
               Δ\advance\@attic@num by-10}%
%    \end{macrocode}
% Finally, it has to check for fives and the digits 1, 2, 3, and 4.
%    \begin{macrocode}
            \@whilenum\@attic@num>4\do{%
               Π\advance\@attic@num-5}%
            \ifcase\@attic@num\or Ι\or ΙΙ\or ΙΙΙ\or ΙΙΙΙ\fi%
   \fi\fi}
%    \end{macrocode}
%\end{macro}
% 
%\begin{macro}{\@atticnum}
% The command |\@atticnum| has one argument, which
% is a counter. It calls the command |\@@atticnum| to process the value of
% the counter.
%    \begin{macrocode}
\def\@atticnum#1{%
     \expandafter\@@atticnum\expandafter{\the#1}}
%    \end{macrocode}
%\end{macro}
%\begin{macro}{\atticnum}
% The command |\atticnum| is a wrapper that declares
% a new counter in a local scope, assigns to it the value of the argument of the command
% and calls the macro |\@atticnum|. This way the command can process correctly
% either a number or a counter. 
%    \begin{macrocode}
\def\atticnum#1{%
     \@attic@num#1\relax
     \@atticnum{\@attic@num}}
%    \end{macrocode}
%\end{macro}
%
%  \begin{macro}{\greek@alph}
%  \begin{macro}{\greek@Alph}
% Here I redefine the macros |\@alph| and |\@Alph|. First, I define some placeholders
%    \begin{macrocode}
\let\latin@alph\@alph
\let\latin@Alph\@Alph
%    \end{macrocode}
%    Then I define the Greek versions; the additional |\expandafter|s
%    are needed in order to make sure the table of contents will be
%    correct (e.g., when there are appendices). 
%    \begin{macrocode}
\def\greek@alph#1{\expandafter\@greeknumeral\expandafter{\the#1}}
\def\greek@Alph#1{\expandafter\@Greeknumeral\expandafter{\the#1}}
%    \end{macrocode}
% By default, Greek alphabetic numerals instead of Latin numerals are used to enumerate items in an
% enumeration environment.
%    \begin{macrocode}
  \let\@alph\greek@alph
  \let\@Alph\greek@Alph
%    \end{macrocode}
% If for some reason, one needs to have the Latin numerals back, then she has to invoke command
% |\nogreekalph|. And if she wants to switch back, then she has to use the |\greekalph|
% command:
%    \begin{macrocode}   
\def\nogreekalph{%
  \let\@alph\latin@alph
  \let\@Alph\latin@Alph}
\def\greekalph{%
  \let\@alph\greek@alph
  \let\@Alph\greek@Alph}
%    \end{macrocode}
%  \end{macro}
%  \end{macro}
%
%  \begin{macro}{\setlanguage}
% We provide the |\setlanguage| command which 
% activates the hypehnation patterns of some other language. It is similar
% to babel's |\selectlanguage|, but we opted to use a new name to avoid possible name conflicts. 
% Valid arguments include |monogreek|, |polygreek|, |ancientgreek|, and |american|. As was noted
% previously, package \textsf{luahyphenrules} provides the command |\HyphenRules| which has exactly
% the same functionality as this command. So when using Lua\LaTeX\ users will actually use the
% |\HyphenRules| command.
%    \begin{macrocode}
\ifx\directlua\undefined%
  \def\setlanguage#1{%
     \expandafter\ifx\csname l@#1\endcsname\relax%
     \typeout{^^J Error: No hyphenation pattern for}%
     \typeout{ language #1 are loaded,}%
     \typeout{ default hyphenation patterns are used.^^J}%
     \language=0%
     \else\language=\csname l@#1\endcsname\fi}
\else
  \let\setlanguage\HyphenRules
\fi
%    \end{macrocode}
%  \end{macro}
%    The macros |\grtoday| and |\Grtoday| produces the current date, only that the
%    month and the day are shown as Greek numerals instead of Arabic
%    as it is usually the case. In addition, the two commands differ in that the
%    later produces the Greek numerals in uppercase.
%    \begin{macrocode}
\def\grtoday{%
  \expandafter\greeknumeral\expandafter{\the\day}\space
  \gr@c@month\space
  \expandafter\greeknumeral\expandafter{\the\year}}
\def\Grtoday{%
  \expandafter\Greeknumeral\expandafter{\the\day}\space
  \gr@c@month\space
  \expandafter\Greeknumeral\expandafter{\the\year}}
%</xgreek>
%    \end{macrocode}
% \section{The Source Code of \textsf{xelistings}}
%
% If the user has enabled the |listings| option, then the package loads the rudimentary package
% \textsf{xelistings}. This package loads the \textsf{listings} package and makes accessible to
% it all characters in the range 128--255 plus all Greek letters that belong to the 
% Greek and Coptic Unicode block. This is achieved by redefining the command |\lst@DefEC|.
%    \begin{macrocode}
%<*xelistings>
\RequirePackage{listings}
\lstset{inputencoding=utf8}
\lst@InputCatcodes
\gdef\lst@DefEC{%
 \lst@CCECUse \lst@ProcessLetter
  ^^80^^81^^82^^83^^84^^85^^86^^87^^88^^89^^8a^^8b^^8c^^8d^^8e^^8f%
  ^^90^^91^^92^^93^^94^^95^^96^^97^^98^^99^^9a^^9b^^9c^^9d^^9e^^9f%
  ^^a0^^a1^^a2^^a3^^a4^^a5^^a6^^a7^^a8^^a9^^aa^^ab^^ac^^ad^^ae^^af%
  ^^b0^^b1^^b2^^b3^^b4^^b5^^b6^^b7^^b8^^b9^^ba^^bb^^bc^^bd^^be^^bf%
  ^^c0^^c1^^c2^^c3^^c4^^c5^^c6^^c7^^c8^^c9^^ca^^cb^^cc^^cd^^ce^^cf%
  ^^d0^^d1^^d2^^d3^^d4^^d5^^d6^^d7^^d8^^d9^^da^^db^^dc^^dd^^de^^df%
  ^^e0^^e1^^e2^^e3^^e4^^e5^^e6^^e7^^e8^^e9^^ea^^eb^^ec^^ed^^ee^^ef%
  ^^f0^^f1^^f2^^f3^^f4^^f5^^f6^^f7^^f8^^f9^^fa^^fb^^fc^^fd^^fe^^ff%
  ^^^^0396^^^^0388^^^^0389^^^^038a^^^^038c% <--- Begin of Greek Letters
  ^^^^038e^^^^038f^^^^0390^^^^0391^^^^0392%
  ^^^^0393^^^^0394^^^^0395^^^^0396^^^^0397%
  ^^^^0398^^^^0399^^^^039a^^^^039b^^^^039c%
  ^^^^039d^^^^039e^^^^039f^^^^03a0^^^^03a1%
  ^^^^03a3^^^^03a4^^^^03a5^^^^03a6^^^^03a7%
  ^^^^03a8^^^^03a9^^^^03aa^^^^03ab^^^^03ac%
  ^^^^03ad^^^^03ae^^^^03af^^^^03b0^^^^03b1%
  ^^^^03b2^^^^03b3^^^^03b4^^^^03b5^^^^03b6%
  ^^^^03b7^^^^03b8^^^^03b9^^^^03ba^^^^03bb%
  ^^^^03bc^^^^03bd^^^^03be^^^^03bf^^^^03c0%
  ^^^^03c1^^^^03c2^^^^03c3^^^^03c4^^^^03c5%
  ^^^^03c6^^^^03c7^^^^03c8^^^^03c9^^^^03ca%
  ^^^^03cb^^^^03cc^^^^03cd^^^^03ce% <--- End of Greek Letters
  ^^00}%
\lst@RestoreCatcodes
%</xelistings>
%    \end{macrocode} 
% \Finale
