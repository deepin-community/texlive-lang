% ^^A -*- japanese-latex -*-
%
% \ifx\epTeXinputencoding\undefined\else
%   \epTeXinputencoding utf8 % ^^A added (2017-10-04)
% \fi
%
% \iffalse meta-comment
%
%<*driver>
\ifx\epTeXinputencoding\undefined\else
  \epTeXinputencoding utf8
\fi
%</driver>
%
%  LaTeX 関連のロゴ(pLaTeX2ε新ドキュメントクラス付録)
%
%  Copyright 1999-2016 Haruhiko Okumura
%  Copyright 2016-2022 Japanese TeX Development Community
%
% \fi
%
% \iffalse
%
%    \begin{macrocode}
%<package>\NeedsTeXFormat{LaTeX2e}
%<package>\ProvidesPackage{jslogo}
%<*driver>
\ProvidesFile{jslogo.dtx}
%</driver>
  [2019/07/25 okumura, texjporg]
%<*driver>
\RequirePackage{plautopatch}
\documentclass[dvipdfmx]{jsarticle}
\usepackage{doc}
\xspcode"5C=1 %% \
\xspcode"22=1 %% "
\addtolength{\textwidth}{-1in}
\addtolength{\evensidemargin}{1in}
\addtolength{\oddsidemargin}{1in}
\addtolength{\marginparwidth}{1in}
\setlength\marginparpush{0pt}
% \OnlyDescription
\CodelineNumbered
\DisableCrossrefs
\setcounter{StandardModuleDepth}{1}
\GetFileInfo{jslogo.dtx}
\begin{document}
  \DocInput{jslogo.dtx}
\end{document}
%</driver>
%    \end{macrocode}
%
% \fi
%
% \title{\pLaTeXe 新ドキュメントクラス付録 ロゴパッケージ}
% \author{奥村晴彦,日本語\TeX 開発コミュニティ}
% \date{\filedate}
% \maketitle
%
% \MakeShortVerb{\|}
%
% \section{はじめに}
%
% これはもともと奥村晴彦氏による\texttt{jsclasses.dtx}で定義され
% ていた\LaTeX 関連のロゴを,独立のパッケージに抽出したものです。
% \texttt{jsclasses}ではデフォルトで読み込まれます。
% 現在は日本語\TeX 開発コミュニティによりGitHubで管理されています。
% \begin{quote}
% |https://github.com/texjporg/jsclasses|
% \end{quote}
% ライセンスはmodified BSDです。
%
% \StopEventually{}
%
% 以下では実際のコードに即して説明します。
%
%
% \section{いろいろなロゴ}
%
% \LaTeX 関連のロゴを作り直します。
%
% \begin{macro}{\jslg@small}
% \begin{macro}{\jslg@uppersmall}
%
% 文字を小さめに出したり上寄りに小さめに出したりする命令です。
%
%    \begin{macrocode}
\def\jslg@small#1{\hbox{$\m@th$%
  \csname S@\f@size\endcsname
  \fontsize\sf@size\z@
  \math@fontsfalse\selectfont
  #1}}
\def\jslg@uppersmall#1{{\sbox\z@ T\vbox to\ht0{\jslg@small{#1}\vss}}}
%    \end{macrocode}
% \end{macro}
% \end{macro}
%
% \begin{macro}{\TeX}
% \begin{macro}{\LaTeX}
%
% これらは \texttt{ltlogos.dtx} で定義されていますが,
% TimesやHelveticaでも見栄えがするように若干変更しました。
%
% Computer Modern Romanです(ファミリ名\texttt{cmr})。
% 元の定義に最も近いので,特別に定義しない場合に用います。
%
%    \begin{macrocode}
\def\cmrTeX{%
  \ifdim \fontdimen\@ne\font >\z@
    T\kern-.25em\lower.5ex\hbox{E}\kern-.125emX\@
  \else
    T\kern-.1667em\lower.5ex\hbox{E}\kern-.125emX\@
  \fi}
\def\cmrLaTeX{%
  \ifdim \fontdimen\@ne\font >\z@
    L\kern-.32em\jslg@uppersmall{A}\kern-.22em\cmrTeX
  \else
    L\kern-.36em\jslg@uppersmall{A}\kern-.15em\cmrTeX
  \fi}
%    \end{macrocode}
%
% Computer Modern Sans Serif(ファミリ名\texttt{cmss})や
% Helvetica(ファミリ名\texttt{phv})などのサンセリフ体です。
%
%    \begin{macrocode}
\def\sfTeX{T\kern-.1em\lower.4ex\hbox{E}\kern-.07emX\@}
\def\sfLaTeX{L\kern-.25em\jslg@uppersmall{A}\kern-.08em\sfTeX}
%    \end{macrocode}
%
% Timesです(ファミリ名\texttt{ptm})。
% TXフォント(ファミリ名\texttt{txr})でも用います。
%
%    \begin{macrocode}
\def\ptmTeX{%
  \ifdim \fontdimen\@ne\font >\z@
    T\kern-.12em\lower.37ex\hbox{E}\kern-.02emX\@
  \else
    T\kern-.07em\lower.37ex\hbox{E}\kern-.05emX\@
  \fi}
\def\ptmLaTeX{%
  \ifdim \fontdimen\@ne\font >\z@
    L\kern-.2em\jslg@uppersmall{A}\kern-.1em\ptmTeX
  \else
    L\kern-.3em\jslg@uppersmall{A}\kern-.1em\ptmTeX
  \fi}
%    \end{macrocode}
%
% New Century Schoolbookです(ファミリ名\texttt{pnc})。
%
%    \begin{macrocode}
\def\pncTeX{%
  \ifdim \fontdimen\@ne\font >\z@
    T\kern-.2em\lower.5ex\hbox{E}\kern-.08emX\@
  \else
    T\kern-.13em\lower.5ex\hbox{E}\kern-.13emX\@
  \fi}
\def\pncLaTeX{%
  \ifdim \fontdimen\@ne\font >\z@
    L\kern-.3em\jslg@uppersmall{A}\kern-.1em\pncTeX
  \else
    L\kern-.3em\jslg@uppersmall{A}\kern-.1em\pncTeX
  \fi}
%    \end{macrocode}
%
% Palatinoです(ファミリ名\texttt{ppl})。
% PXフォント(ファミリ名\texttt{pxr})でも用います。
%
%    \begin{macrocode}
\def\pplTeX{%
  \ifdim \fontdimen\@ne\font >\z@
    T\kern-.17em\lower.32ex\hbox{E}\kern-.15emX\@
  \else
    T\kern-.12em\lower.34ex\hbox{E}\kern-.1emX\@
  \fi}
\def\pplLaTeX{%
  \ifdim \fontdimen\@ne\font >\z@
    L\kern-.27em\jslg@uppersmall{A}\kern-.12em\pplTeX
  \else
    L\kern-.3em\jslg@uppersmall{A}\kern-.15em\pplTeX
  \fi}
%    \end{macrocode}
%
% URW Garamondです(ファミリ名\texttt{ugm})。
%
%    \begin{macrocode}
\def\ugmTeX{%
  \ifdim \fontdimen\@ne\font >\z@
    T\kern-.1em\lower.32ex\hbox{E}\kern-.06emX\@
  \else
    T\kern-.12em\lower.34ex\hbox{E}\kern-.1emX\@
  \fi}
\def\ugmLaTeX{%
  \ifdim \fontdimen\@ne\font >\z@
    L\kern-.2em\jslg@uppersmall{A}\kern-.13em\ugmTeX
  \else
    L\kern-.3em\jslg@uppersmall{A}\kern-.13em\ugmTeX
  \fi}
%    \end{macrocode}
%
% 以上の定義を,ファミリによって使い分けます。
%
% [2003-06-12] Palatinoも加えました(要調整)。
%
%    \begin{macrocode}
\DeclareRobustCommand{\TeX}{%
  \def\@tempa{cmss}%
  \ifx\f@family\@tempa\sfTeX
  \else
  \def\@tempa{phv}%
  \ifx\f@family\@tempa\sfTeX
  \else
  \def\@tempa{qhv}%
  \ifx\f@family\@tempa\sfTeX
  \else
    \def\@tempa{ptm}%
    \ifx\f@family\@tempa\ptmTeX
    \else
    \def\@tempa{qtm}%
    \ifx\f@family\@tempa\ptmTeX
    \else
      \def\@tempa{txr}%
      \ifx\f@family\@tempa\ptmTeX
      \else
      \def\@tempa{ntxtlf}%
      \ifx\f@family\@tempa\ptmTeX
      \else
        \def\@tempa{pnc}%
        \ifx\f@family\@tempa\pncTeX
        \else
          \def\@tempa{ppl}%
          \ifx\f@family\@tempa\pplTeX
          \else
          \def\@tempa{qpl}%
          \ifx\f@family\@tempa\pplTeX
          \else
          \def\@tempa{pplx}%
          \ifx\f@family\@tempa\pplTeX
          \else
          \def\@tempa{pplj}%
          \ifx\f@family\@tempa\pplTeX
          \else
            \def\@tempa{pxr}%
            \ifx\f@family\@tempa\pplTeX
            \else
            \def\@tempa{zpltlf}%
            \ifx\f@family\@tempa\pplTeX
            \else
              \def\@tempa{ugm}%
              \ifx\f@family\@tempa\ugmTeX
              \else\cmrTeX
              \fi
            \fi\fi
          \fi\fi\fi\fi
        \fi
      \fi\fi
    \fi\fi
  \fi\fi\fi}
\DeclareRobustCommand{\LaTeX}{%
  \def\@tempa{cmss}%
  \ifx\f@family\@tempa\sfLaTeX
  \else
  \def\@tempa{phv}%
  \ifx\f@family\@tempa\sfLaTeX
  \else
  \def\@tempa{qhv}%
  \ifx\f@family\@tempa\sfLaTeX
  \else
    \def\@tempa{ptm}%
    \ifx\f@family\@tempa\ptmLaTeX
    \else
    \def\@tempa{qtm}%
    \ifx\f@family\@tempa\ptmLaTeX
    \else
      \def\@tempa{txr}%
      \ifx\f@family\@tempa\ptmLaTeX
      \else
      \def\@tempa{ntxtlf}%
      \ifx\f@family\@tempa\ptmLaTeX
      \else
        \def\@tempa{pnc}%
        \ifx\f@family\@tempa\pncLaTeX
        \else
          \def\@tempa{ppl}%
          \ifx\f@family\@tempa\pplLaTeX
          \else
          \def\@tempa{qpl}%
          \ifx\f@family\@tempa\pplLaTeX
          \else
          \def\@tempa{pplx}%
          \ifx\f@family\@tempa\pplLaTeX
          \else
          \def\@tempa{pplj}%
          \ifx\f@family\@tempa\pplLaTeX
          \else
            \def\@tempa{pxr}%
            \ifx\f@family\@tempa\pplLaTeX
            \else
            \def\@tempa{zpltlf}%
            \ifx\f@family\@tempa\pplLaTeX
            \else
              \def\@tempa{ugm}%
              \ifx\f@family\@tempa\ugmLaTeX
              \else\cmrLaTeX
              \fi
            \fi\fi
          \fi\fi\fi\fi
        \fi
      \fi\fi
    \fi\fi
  \fi\fi\fi}
%    \end{macrocode}
%
% \end{macro}
% \end{macro}
%
% \begin{macro}{\LaTeXe}
%
% |\LaTeXe| コマンドの |\mbox{\m@th ...| で始まる新しい定義では
% 直後の和文との間に |\xkanjiskip| が入りません。また,
% |mathptmx| パッケージなどと併用すると,最後の $\varepsilon$ が
% 下がりすぎてしまいます。そのため,ちょっと手を加えました。
%
%    \begin{macrocode}
\DeclareRobustCommand{\LaTeXe}{$\mbox{%
  \if b\expandafter\@car\f@series\@nil\boldmath\fi
  \LaTeX\kern.15em2\raisebox{-.37ex}{$\textstyle\varepsilon$}}$}
%    \end{macrocode}
% \end{macro}
%
% \begin{macro}{\pTeX}
% \begin{macro}{\pLaTeX}
% \begin{macro}{\pLaTeXe}
%
% \pTeX ,\pLaTeXe のロゴを出す命令です。
%
%    \begin{macrocode}
\def\pTeX{p\kern-.05em\TeX}
\def\pLaTeX{p\LaTeX}
\def\pLaTeXe{p\LaTeXe}
%    \end{macrocode}
%
% \end{macro}
% \end{macro}
% \end{macro}
%
% \begin{macro}{\upTeX}
% \begin{macro}{\upLaTeX}
% \begin{macro}{\upLaTeXe}
%
% [2016-07-14] \upTeX ,\upLaTeXe のロゴを出す命令も定義します。
%
%    \begin{macrocode}
\def\upTeX{u\pTeX}
\def\upLaTeX{u\pLaTeX}
\def\upLaTeXe{u\pLaTeXe}
%    \end{macrocode}
%
% \end{macro}
% \end{macro}
% \end{macro}
%
% \begin{macro}{\AmSTeX}
%
% \texttt{amstex.sty} で定義されています。
%
%    \begin{macrocode}
\def\AmSTeX{\protect\AmS-\protect\TeX{}}
%    \end{macrocode}
%
% \end{macro}
%
% \begin{macro}{\BibTeX}
% \begin{macro}{\SliTeX}
%
% これらは \texttt{doc.dtx} から取ったものです。
% ただし,|\BibTeX| だけはちょっと修正しました。
%
%    \begin{macrocode}
% \@ifundefined{BibTeX}
%    {\def\BibTeX{{\rmfamily B\kern-.05em%
%     \textsc{i\kern-.025em b}\kern-.08em%
%     T\kern-.1667em\lower.7ex\hbox{E}\kern-.125emX}}}{}
\DeclareRobustCommand{\BibTeX}{%
  B\kern-.05em\jslg@small{I\kern-.025em B}\kern-.15em\TeX}
\DeclareRobustCommand{\SliTeX}{%
  S\kern-.06emL\kern-.18em\jslg@uppersmall{I}\kern -.03em\TeX}
%    \end{macrocode}
% \end{macro}
% \end{macro}
%
%
% \section{PDF文字列中でロゴが使われた場合への対策}
%
% 以上でロゴの定義は終わりですが,|hyperref|を読み込んだ状態で |jslogo| の命令を
% PDF文字列として解釈される箇所(例 |\section| など)で使うと警告が出てしまいます。
%\begin{verbatim}
%  Package hyperref Warning: Token not allowed in a PDF string (PDFDocEncoding):
%  (hyperref)                \kern -0.50003pt
%  (hyperref)                removed on input line 5.
%\end{verbatim}
% そこで,この警告が出ないように対策します(Issue \#51)。
%
% まず,|\pdfstringdefPreHook| を(なければ)用意します。
%    \begin{macrocode}
\providecommand*\pdfstringdefPreHook{}
%    \end{macrocode}
%
% そして,警告が出てしまうロゴに対するPDF文字列中での定義を追加します。
%    \begin{macrocode}
\g@addto@macro\pdfstringdefPreHook{%
  \def\pTeX{p\TeX}%
  \def\AmS{AMS}%
  \def\SliTeX{Sli\TeX}%
  \def\BibTeX{Bib\TeX}%
}
%    \end{macrocode}
%
% |hyperref| 読込時に |\pdfstringdefPreHook| が定義済ならそれはそのまま保持され,
% 警告が出なくなります。
%
% 以上です。
%
% \Finale
%
\endinput
