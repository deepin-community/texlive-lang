% \iffalse meta-comment
% !TEX program  = XeLaTeX
%<*internal>
\iffalse
%</internal>
%<*readme>
Boxed Chinese characters with Pinyin above and translation below based LaTeX3
=======

`hanzibox` is a LaTeX package developed in LaTeX3, which provides `\hanzibox`, `\hanzidialog` and `\writegrid` macros. These macros are used for typesetting Chinese character with or without a background grid such as a cross grid or star grid for Chinese character writting learning.

And you can display the Hanyu Pinyin above the Chinese character and the translation below them as needed.

The `\hanzibox` macro can automatically invoke the `\xpinyin*` macro of the `xpinyin` package to get Hanyu Pinyin according to the Chinese characters.  And the `\hanzibox` macro provides an asterisk version of `\hanzibox*`, whose function is achieved by automatically invoking the `xpinyin` package's `\pinyin` macro to get Hanyu Pinyin. The `\hanzidialog` macro is implemented by manually inserting the `\pinyin` macro of the `xpinyin` package in it's pinyin option. The `\writegrid` macro is for write a composition anwser grid.


`hanzibox` is a utility for learning to write and pronounce Chinese characters, and can be used for Chinese character learning plans, presentations, exercise booklets and other documentation work.

The development of this package was inspired by [Jan Vorisek's hanzibox package](https://github.com/janvorisek/chinese-latex-utilities).

However, the new package redesigned the `\hanzibox` and `\hanzidialog` macros and the background grid is redesigned with the `l3draw` package by reference to [zitie package](https://www.ctan.org/pkg/zitie).  Also, the new package provides more options and arguments in order to get better results.

The `hanzibox.sty` package currently only supports the XeTeX engine and only supports UTF-8 encoded LaTeX source files.

You can read the manual (in Chinese) for more details and examples.

Contributing
------------

1. github repository:
    1. repository: [hanzibox-l3](https://github.com/registor/hanzibox-l3)
    2. Issues and pull requests are welcome. [issue](https://github.com/registor/hanzibox-l3/issues) or [pull request](https://github.com/registor/hanzibox-l3/pulls).

2. gitee repository:
    1. repository: [hanzibox-l3](https://gitee.com/nwafu_nan/hanzibox-l3)
    2. Issues and pull requests are welcome. [issue](https://gitee.com/nwafu_nan/hanzibox-l3/issues) or [pull request](https://gitee.com/nwafu_nan/hanzibox-l3/pulls).

基于LaTeX3的带注音和译文的汉字练习宏包
=======

`hanzibox`是一个用LaTeX3开发的LaTeX宏包,它提供了`\hanzibox`、`\hanzidialog` 和`\writegrid`三个个命令。这三个命令分别用于排版汉字学习中带有或不带田字格、米字格等背景格子的汉字,并可以根据需要在汉字正上方显示拼音,在正下方显示译文。其中,`\hanzibox`命令能够根据汉字自动调用`xpinyin`宏包的`\xpinyin*`命令实现汉字注音。并且`\hanzibox`命令提供了`\hanzibox*`星号版本,其注音功能是通过自动调用`xpinyin`宏包的`\pinyin`命令实现的。`\hanzidialog`命令的注音功能是通过在拼音选项中手动插入`xpinyin`宏包`\pinyin`命令实现的。`\writegrid`命令用于排版作文题目的答题格子纸。

`hanzibox` 是一个用于学习汉字书写与发音的工具,可以用于汉字学习教案、演示文稿、习题册等文档工作。

该宏包的开发灵感源自[Jan Vorisek的hanzibox宏包](https://github.com/janvorisek/chinese-latex-utilities)。但对`\hanzibox`和`\hanzidialog`命令重新进行了设计,并参考[zitie字贴宏包](https://www.ctan.org/pkg/zitie)重新用`l3draw`宏包设计了背景格子。同时,新的宏包也提供了更多命令选项和参数,以期更好地控制排版结果。

目前,`hanzibox.sty`宏包仅支持XeTeX编译引擎,并且只支持UTF-8编码的LaTeX源文件。

可以通过阅读宏包手册(中文)以也解该宏包更多的使用细节和使用样例。

###  参与贡献
---------------------

1. github仓库:
    (1). 仓库地址: [hanzibox-l3](https://github.com/registor/hanzibox-l3)
    (2). Issues和PR: [issue](https://github.com/registor/hanzibox-l3/issues) or [pull request](https://github.com/registor/hanzibox-l3/pulls).

2. gitee仓库:
    (1). 仓库地址: [hanzibox-l3](https://gitee.com/nwafu_nan/hanzibox-l3)
    (2). Issues and PR: [issue](https://gitee.com/nwafu_nan/hanzibox-l3/issues) or [pull request](https://gitee.com/nwafu_nan/hanzibox-l3/pulls).

Copyright and Licence
---------------------

    Copyright (C) 2020-2022 by Nan Geng <nangeng@nwafu.edu.cn>
    ----------------------------------------------------------------------

    This work may be distributed and/or modified under the
    conditions of the LaTeX Project Public License, either
    version 1.3c of this license or (at your option) any later
    version. This version of this license is in
       http://www.latex-project.org/lppl/lppl-1-3c.txt
    and the latest version of this license is in
       http://www.latex-project.org/lppl.txt
    and version 1.3 or later is part of all distributions of
    LaTeX version 2005/12/01 or later.

    This work has the LPPL maintenance status "maintained".

    The Current Maintainer of this work is Nan Geng.

    This package consists of the file  hanzibox.dtx,
                 and the derived files hanzibox.sty,
                                       hanzibox.pdf,
                                       hanzibox.ins,
                                       README.md (this file).

%</readme>
%<*internal>
\fi
\begingroup
  \def\temp{LaTeX2e}
\expandafter\endgroup\ifx\temp\fmtname\else
\csname fi\endcsname
%</internal>
%<*install>

\input ctxdocstrip %

\let\MetaPrefix\relax

\preamble

    Copyright (C) 2020-2022 by Nan Geng <nangeng@nwafu.edu.cn>
--------------------------------------------------------------------------

    This work may be distributed and/or modified under the
    conditions of the LaTeX Project Public License, either
    version 1.3c of this license or (at your option) any later
    version. This version of this license is in
       http://www.latex-project.org/lppl/lppl-1-3c.txt
    and the latest version of this license is in
       http://www.latex-project.org/lppl.txt
    and version 1.3 or later is part of all distributions of
    LaTeX version 2005/12/01 or later.

    This work has the LPPL maintenance status "maintained".

    The Current Maintainer of this work is Nan Geng.

--------------------------------------------------------------------------

\endpreamble

\postamble

    This package consists of the file  hanzibox.dtx,
                 and the derived files hanzibox.sty,
                                       hanzibox.pdf,
                                       hanzibox.ins,
                                       README.md.
\endpostamble

\declarepostamble\emptypostamble
\endpostamble

\def\MetaPrefix{-- }


\let\MetaPrefix\DoubleperCent

\generate
  {
%</install>
%<*internal>
    \usedir{source/xelatex/hanzibox}
    \file{hanzibox.ins} {\from{\jobname.dtx}{install}}
%</internal>
%<*install>
    \usedir{xetex/xelatex/hanzibox}
    \file{hanzibox.sty} {\from{\jobname.dtx}{package}}
    \nopreamble\nopostamble
    \usedir{doc/xelatex/hanzibox}
    \file{README.md}   {\from{\jobname.dtx}{readme}}
  }

\endbatchfile
%</install>
%<*internal>
\fi
%</internal>
%<package>\NeedsTeXFormat{LaTeX2e}[2020/10/01]
%<package>\RequirePackage{expl3}
%<+package>\GetIdInfo$Id: hanzibox.dtx 2.3.0 2022-04-17 08:00:00 +0800 Nan Geng <nangeng@nwafu.edu.cn> $
%<package>  {Boxed Chinese characters with Pinyin above and translation below.}
%<package>\ProvidesExplPackage{\ExplFileName}
%<package>  {\ExplFileDate}{\ExplFileVersion}{\ExplFileDescription}
%<*driver>
\documentclass{ctxdoc}
\usepackage{listings}
\usepackage{xcolor}
\usepackage{xcolor-material}
\usepackage{hanzibox}
\renewcommand*\marg[1]{\{\meta{#1}\}}
\renewcommand*\oarg[1]{[\meta{#1}]}
\renewcommand*\parg[1]{(\meta{#1})}
\NewDocumentCommand{\init}{+v}{\hspace{\fill}初始值~=~\textcolor{blue}{\bfseries#1}}
\DeclareDocumentCommand\opt{m}{\texttt{#1}}
\DeclareDocumentCommand\kvopt{mm}
  {\texttt{#1\breakablethinspace=\breakablethinspace#2}}
\def\breakablethinspace{\hskip 0.16667em\relax}
\begin{document}
%<!--CODEDOC-->  \DisableImplementation
  \EnableImplementation
  \DocInput{\jobname.dtx}
  \IndexLayout
  \PrintChanges
  \PrintIndex
\end{document}
%</driver>
% \fi
%
% \changes{v1.0.0}{2021/09/18}{开始编写模板}
% \changes{v2.0.0}{2021/09/25}{基于l3draw重新设计}
% \changes{v2.0.0}{2021/09/28}{重新编写README.md内容}
%
% \CheckSum{0}
% \GetFileId{hanzibox.sty}
%
% \CharacterTable
%  {Upper-case    \A\B\C\D\E\F\G\H\I\J\K\L\M\N\O\P\Q\R\S\T\U\V\W\X\Y\Z
%   Lower-case    \a\b\c\d\e\f\g\h\i\j\k\l\m\n\o\p\q\r\s\t\u\v\w\x\y\z
%   Digits        \0\1\2\3\4\5\6\7\8\9
%   Exclamation   \!     Double quote  \"     Hash (number) \#
%   Dollar        \$     Percent       \%     Ampersand     \&
%   Acute accent  \'     Left paren    \(     Right paren   \)
%   Asterisk      \*     Plus          \+     Comma         \,
%   Minus         \-     Point         \.     Solidus       \/
%   Colon         \:     Semicolon     \;     Less than     \<
%   Equals        \=     Greater than  \>     Question mark \?
%   Commercial at \@     Left bracket  \[     Backslash     \\
%   Right bracket \]     Circumflex    \^     Underscore    \_
%   Grave accent  \`     Left brace    \{     Vertical bar  \|
%   Right brace   \}     Tilde         \~}
%
%
% \title{\bfseries\pkg{hanzibox}:田字格-米字格汉字练习宏包}
% \author{耿楠\\ \path{nangeng@nwafu.edu.cn}}
% \date{\filedate\qquad\fileversion\thanks{\url{https://github.com/registor/hanzibox-l3}}\thanks{\url{https://gitee.com/nwafu_nan/hanzibox-l3}}}
% \maketitle
%
% \changes{v2.0.0}{2021/09/28}{修订说明文档}
% \changes{v2.1.2}{2021/10/07}{为说明文档添加目录}
%
% \begin{documentation}
%
% \begin{abstract}
%
% \pkg{hanzibox} 是一个用\LaTeX3开发的 \LaTeX 宏包,它提供
% 了\tn{hanzibox}、\tn{hanzidialog}和\tn{writegrid}三
% 个命令。这三个命令用于输出汉字学习中带有或不带田字格、米字格等背景格子的汉字,
% 并可以根据需要在汉字正上方显示拼音,在正下方显示译文。
% 其中,\tn{hanzibox}命令能够根据汉字利用\pkg{xpinyin}宏包自动实现汉字注音。
% 同时,\tn{hanzibox}命令还提供了\tn{hanzibox*}星号版本,以实现汉字的手动注音功能。
% \tn{hanzidialog}命令的注音功能则是通过在其拼音选项中手动插入
% \pkg{xpinyin}宏包的\tn{pinyin}命令实现。\tn{writegrid}命令用于排版
% 作文题目的答题格子纸。
%
% \end{abstract}
%
% \tableofcontents
%
% \section{简要说明}
%
% \pkg{hanzibox} 是一个用于输出汉字学习中的田字格、米字格等背景,并在汉字正上方显示拼音,
% 在正下方显示译文。当然,也可以根据需要隐藏拼音、汉字或译文,还可以选择性地隐藏拼音中的
% 声母、韵母或音调,从而有效实现汉字学习中的素材准备。
%
% 使用\pkg{hanzibox}宏包的\LaTeX 源文件需采用 \texttt{UTF-8}编码,并且需使用\XeLaTeX 进行编译。
%
% \pkg{hanzibox} 依赖 \package{l3kernel} 、\package{l3packages}、\package{l3draw}和\package{xpinyin}宏包。
%
% \section{用户接口}
%
% \subsection{\tn{hanzibox}命令}
%
% \begin{function}[added=2021-09-18,updated=2021-10-07]{\hanzibox,\hanzibox*}
%   \begin{syntax}
%     \tn{hanzibox}  \oarg{外观选项} \Arg{汉字} \oarg{拼音选项} \oarg{译文选项}
%     \tn{hanzibox*} \oarg{外观选项} \Arg{汉字} \oarg{拼音选项} \oarg{译文选项}
%   \end{syntax}
%   排版汉字,并根据选项内容在顶部排版拼音,在底部排版译文。\\
%   其中,\Arg{汉字}可以留空,\oarg{外观选项}用于设置盒子外观;
%   \tn{hanzibox}命令中的\oarg{拼音选项}无效,可以省略,也可以留空;
%   \oarg{译文选项}可以是任意文本,需要注意的是,当需要\oarg{译文选项}时,
%   \oarg{拼音选项}可以留空,但不能省略。
%   注音由\pkg{xpinyin}宏包自动根据汉字获得,此时,可能会存在多音字等问题,
%   其调整详情请参阅\pkg{xpinyin}宏包说明。星号命令\tn{hanzibox*}用于手动添加注音。
%
%   排版样式可通过\tn{hanziboxset}命令或\tn{hanzibox}\oarg{外观选项}的key-value进行设置。
%
%   \begin{SideBySideExample}[frame=single,numbers=left,xrightmargin=.35\linewidth,gobble=5]
%     \centering
%     \hanziboxset{fillcolor=yellow!30, charcolor=red,
%       xscale=1.5,yscale=1.5,resize=real,framecolor=red}
%     \hanzibox[frametype=none]{我}\\[1ex]
%     \hanzibox[frametype=十  ]{我}[wo3][俺]
%     \hanzibox[frametype=×   ]{我}[wo2][爷]
%     \hanzibox[frametype=米  ]{我}[ni3][奴]\\[1ex]
%     \hanzibox[frametype=口,pinyinline=true]{我}[][吾]
%     \hanzibox[frametype=田,pinyinline=true]{我}[wo1][愚]
%     \hanzibox[frametype=咪,pinyinline=true]{我}[ta1][山人]
%   \end{SideBySideExample}
%   \begin{SideBySideExample}[frame=single,numbers=left,xrightmargin=.35\linewidth,gobble=5]
%     \centering
%     \hanziboxset{xscale=1.5,yscale=1.5,resize=real}
%     \hanzibox*[frametype=none]{我}[wo3][me]\\[1ex]
%     \hanzibox*[frametype=十  ]{我}[wo3][俺]
%     \hanzibox*[frametype=×   ]{我}[wo3][爷]
%     \hanzibox*[frametype=米  ]{我}[ni3][奴]\\[1ex]
%     \hanzibox*[frametype=口,pinyinline=true]{我}[wo3][吾]
%     \hanzibox*[frametype=田,pinyinline=true]{我}[wo2][愚]
%     \hanzibox*[frametype=咪,pinyinline=true]{我}[ta5][山人]
%   \end{SideBySideExample}
% \end{function}
%
% \subsection{\tn{hanzidialog}命令}
%
% \begin{function}[added=2021-09-18,updated=2021-10-07]{\hanzidialog}
%   \begin{syntax}
%     \tn{hanzidialog} \oarg{外观选项} \Arg{汉字} \oarg{拼音选项} \oarg{译文选项}
%   \end{syntax}
%   排版汉字,并根据选项内容在顶部排版拼音,在底部排版译文。\\
%   其中,\Arg{汉字}可以留空,\oarg{拼音选项}可以是任意文本;\oarg{译文选项}可以是任意文本。
%   \oarg{拼音选项}和\oarg{译文选项}都可以留空,也可以省略,
%   但当需要\oarg{译文选项}时,\oarg{拼音选项}可以留空,但不能省略。
%   如果是汉语拼音则需要手动使用\pkg{xpinyin}宏包的\tn{pinyin}命令添加注音,
%   排版样式可通过\tn{hanziboxset}命令或\tn{hanzidialog}\oarg{外观选项}设置。
%   \begin{SideBySideExample}[frame=single,numbers=left,xrightmargin=.25\linewidth,gobble=5]
%     \centering
%     \hanzidialog{王老师}[Wang \pinyin{lao3shi1}][teacher wang]
%
%     \hanzidialog[frametype=田]{王老师}[Wang \pinyin{lao3shi1}]
%                 [王先生]
%
%     \hanzidialog[frametype=咪,framecolor=red,pinyinline=true,
%                  height=1cm,resize=real,pinyincolor=blue]
%                 {王老师}[Wang \pinyin{lao3shi1}][王先生]
%   \end{SideBySideExample}
% \end{function}
%
% \subsection{\tn{writegrid}命令}
%
% \begin{function}[added=2022-04-17,updated=2022-04-17]{\writegrid}
%   \begin{syntax}
%     \tn{writegrid} \oarg{外观选项} \Arg{行数}
%   \end{syntax}
%   根据指定的\Arg{行数},用\oarg{外观选项}指定的外观参数和行间距及
%   列数排版作文题目中的写作格子纸。\\
%   其中,\oarg{外观选项}用于单个格子盒子外观,但要注意此时,
%   \oarg{拼音选项}、\oarg{译文选项}及其相关选项无效。与作文
%   直接相关的选项有\oarg{gridsepv}用于设置不同格子行的间距,
%   \oarg{gridcols}用于设置每行的格子数
%
%   排版样式可通过\tn{hanziboxset}命令或\tn{writegrid}\oarg{外观选项}的key-value进行设置。
%
%   \begin{SideBySideExample}[frame=single,numbers=left,xrightmargin=.65\linewidth,gobble=5]
%     \centering
%     \hanziboxset{framecolor=red,
%             fillcolor=yellow!30}
%     \writegrid{5}
%   \end{SideBySideExample}
% \end{function}
%
% \subsection{\tn{hanziboxset}命令}
%
% \begin{function}[added=2021-09-20,updated=2021-09-24]{\hanziboxset}
%   \begin{syntax}
%     \tn{hanziboxset} \marg{键值列表}
%   \end{syntax}
%   \tn{hanziboxset} 的参数是一组由(英文)逗号隔开的选项列表,
%   列表中的选项通常是 \kvopt{\meta{key}}{\meta{value}} 形式。
%   部分选项的\meta{value} 可以省略。对于同一选项,后续设置会覆盖以前的设置。
%   多数选项都设有默认值。
%
%   \tn{hanziboxset} 采用 \LaTeX3 风格的键值设置,支持不同类型以及多种
%   层次的选项设定。键值列表中,“|=|”左右的空格不影响设置;但需注意,
%   参数列表中\textbf{不可以出现空行}。
%
%   布尔型的参数 \kvopt{\meta{选项}}{true} 中的“|= true|”可以省略。
% \end{function}
%
% \section{选项说明}
%
%   本宏包提供了一系列选项,以实现汉字盒子外观样式设置。
%   载入\pkg{hanzibox}宏包后,以下选项均可通过用户接口命令\tn{hanziboxset}进行设置。
%   同时,这些选项也可以通过\tn{hanzibox}或\tn{hanzidialog}命令的\oarg{外观选项}进行设置。
%
% \subsection{基础字符和字号}
%
% \begin{function}[added=2021-09-24,updated=2021-09-24]{basechar,zihao}
%   \begin{syntax}
%     basechar = \meta{CJK char} \init{好}
%     zihao    = \meta{字号}     \init{4}
%   \end{syntax}
%   \opt{basechar} 设置基字符,用于计算缩放比例及留空汉字占位处理,
%   基字符不同时,即使给定相同的缩放比例,其实际缩放比例也可能不同。
%
%   \opt{zihao} 设置基字符的字号。
% \end{function}
%
% \subsection{拼音、汉字和译文格式}
%
% \begin{function}[added=2021-09-27,updated=2021-10-08]{pinyinf,charf,tranf}
%   \begin{syntax}
%     pinyinf = \meta{格式命令}   \init{\normalsize}
%     charf   = \meta{格式命令}   \init{\tiny}
%     tranf   = \meta{格式命令}   \init{\tiny}
%   \end{syntax}
%   分别用于设置拼音、汉字、译文的排版格式,主要用于设置字体、字号、粗细等格式。
%
%   为了分解拼音,本宏包截获了原\pkg{xpinyin}宏包中的拼音输出,因此若需要修改拼音字体,
%   请在\verb!pinyinf!选项中进行设置。
%
%   \textcolor[HTML]{AD1457}{强烈建议将单个汉字宽度设置为大于其拼音或译文总宽度,
%   以免在汉字间形成缝隙}。
% \end{function}
%
% \subsection{外框类型和缩放方式}
%
% \begin{function}[added=2021-09-24,updated=2021-10-08]{frametype,resize}
%  \begin{syntax}
%    frametype = <none|十|×|米|口|田|咪> \init{none}
%    resize    = <none|real|base> \init{none}
%  \end{syntax}
%  \opt{frametype} 设置汉字盒子样式。可用值的效果与选项值文字形状类似:
%  \verb|十|--仅中间的横线和竖线,\verb|×|--仅中间的两条对角线,
%  \verb|米|--十字格再加上斜的两条对角线,
%  \verb|口|--仅方框,\verb|田|--常见的田字格,\verb|咪|--常见的米字格。
% \end{function}
%
%   \begin{SideBySideExample}[frame=single,numbers=left,xrightmargin=.50\linewidth,gobble=5]
%     \centering
%     \hanzibox[frametype=none]{无}\\[1ex]
%     \hanzibox[frametype=十  ]{十}
%     \hanzibox[frametype=×   ]{义}
%     \hanzibox[frametype=米  ]{米}\\[1ex]
%     \hanzibox[frametype=口  ]{口}
%     \hanzibox[frametype=田  ]{田}
%     \hanzibox[frametype=咪  ]{咪}
%   \end{SideBySideExample}
%
%  \opt{resize} 设置缩放方式,\opt{real}--使用字符实际宽高缩放,
%  \opt{base}--使用 \opt{basechar} 字符的宽高缩放,
%
%  以下为宽度设置为 1cm 时的缩放情况。
%   \begin{SideBySideExample}[frame=single,numbers=left,xrightmargin=.45\linewidth,gobble=5]
%     \centering
%     \hanziboxset{width=1cm,frametype=咪,
%                  framecolor=black}
%     \hanzibox[resize=none  ]{无}
%     \hanzibox[resize=real  ]{实}
%     \hanzibox[resize=base  ]{基}
%   \end{SideBySideExample}
%
% \subsection{缩放比例及尺寸}
%
% \begin{function}[added=2021-09-24,updated=2021-09-24]{
%   xscale,yscale,scale,
%   width,height}
%   \begin{syntax}
%     xscale = \meta{scale ratio} \init{1}
%     yscale = \meta{scale ratio} \init{1}
%     scale  = \meta{scale ratio}
%     width  = \meta{dim}
%     height = \meta{dim}
%   \end{syntax}
%   设置缩放比例和盒子宽高。
%
%   宽高具有更高的优先级,即若比例和宽高都设置了,则使用宽高来计算。
%   宽高都为0cm视为未设置,二者有一大于0cm,视为设置了宽高。
% \end{function}
%
% \subsection{盒子样式}
%
% \begin{function}[added=2021-09-24,updated=2021-09-24]{
%   linewidth,
%   dashpattern,
%   framearc,
%   framearc*,
%   framelinewidth,
%   pinyinlinewidth,
%   crosslinewidth
%   }
%   \begin{syntax}
%     linewidth       = \meta{dim} \init{0.4pt}
%     dashpattern     = \meta{ dim1, dim2, ... }
%     framearc        = \meta{dim}
%     framearc*       = \{ \marg{dim1} \marg{dim2} \}
%     framelinewidth  = \meta{dim} \init{0.4pt}
%     pinyinlinewidth = \meta{dim} \init{0.4pt}
%     crosslinewidth  = \meta{dim} \init{0.4pt}
%   \end{syntax}
%   设置边框线宽、线型、转角样式及拼音四线格和内格线线宽。
%
%   \textcolor[HTML]{AD1457}{注:目前\textbf{linewidth}与
%     \textbf{framelinewidth}选项的作用相同,都是设置边框线宽,
%     在下一个版本中,会删除\textbf{linewidth}选项}。
%   \begin{SideBySideExample}[frame=single,numbers=left,xrightmargin=.28\linewidth,gobble=5]
%     \centering
%     \hanziboxset{width=1.0cm,resize=real,framecolor=red}
%     \hanzibox[linewidth=1pt,frametype=口]{好}
%     \hanzibox[dashpattern={1.5mm,1mm,2mm,1.5mm},frametype=田]{好}
%     \hanzibox[framelinewidth=1pt,frametype=咪]{好} \\
%     \hanzibox[pinyinline,framearc=1mm,frametype=口]{好}
%     \hanzibox[framearc*={1mm}{2mm},frametype=田]{好}
%     \hanzibox[framearc*={2mm}{1mm},frametype=咪]{好}\\
%     \hanzibox[pinyinline,pinyinlinewidth=0.8pt,frametype=口]{好}
%     \hanzibox[crosslinewidth=1pt,frametype=田]{好}
%     \hanzibox[crosslinewidth=1pt,frametype=咪]{好}
%   \end{SideBySideExample}
% \end{function}
%
% \subsection{颜色设置}
%
%
% \begin{function}[added=2021-10-07,updated=2021-10-07]{crosscolorratio}
%   \begin{syntax}
%     crosscolorratio  = \meta{integer} \init{20}
%   \end{syntax}
%   格子内部十字线或米字线颜色占边框颜色的比例(0$\sim$100\%)。
%
%   \textcolor[HTML]{AD1457}{注:\textbf{crosscolorratio}选项须在设置了\textbf{framecolor}选项后才能生效}。
%
%   \begin{SideBySideExample}[frame=single,numbers=left,xrightmargin=.25\linewidth,gobble=5]
%     \centering
%     \hanzibox[frametype=咪,framecolor=red]{十}
%     \hanzibox[frametype=咪,crosscolorratio=30,framecolor=red]{田}
%     \hanzibox[frametype=咪,crosscolorratio=50,framecolor=red]{米}
%     \hanzibox[frametype=咪,crosscolorratio=80,framecolor=red]{咪}
%   \end{SideBySideExample}
% \end{function}
%
% \begin{function}[added=2021-09-24,updated=2021-10-07]{
%   framecolor,
%   framecolor*,
%   charcolor,
%   charcolor*,
%   pinyincolor,
%   pinyincolor*,
%   trancolor,
%   trancolor*,
%   fillcolor,
%   fillcolor*
%   }
%   \begin{syntax}
%     framecolor   = \meta{color expr} \init{black}
%     framecolor*  = \meta{model(s)} \meta{value(s)}
%     charcolor    = \meta{color expr} \init{black}
%     charcolor*   = \meta{model(s)} \meta{value(s)}
%     pinyincolor  = \meta{color expr} \init{black}
%     pinyincolor* = \meta{model(s)} \meta{value(s)}
%     trancolor    = \meta{color expr} \init{black}
%     trancolor*   = \meta{model(s)} \meta{value(s)}
%     fillcolor    = \meta{color expr}
%     fillcolor*   = \meta{model(s)} \meta{value(s)}
%   \end{syntax}
%   分别设置格子外框、字符、拼音、译文和填充颜色。
%   颜色名称仅支持 \LaTeX3 定义的 black, white, red, green, blue, cyan, magenta 和 yellow。
%   颜色模型和表达式也应使用 \LaTeX3 支持的模型和表达式,详见 \pkg{interface3.pdf} 文档。
%
%   若要去掉 \opt{fillcolor},应将其置为空(\verb|fillcolor={}|),而不是将其设置为white(白色)。
% \end{function}
%
% \subsection{字符轮廓类型}
%
% \begin{function}[added=2021-10-08,updated=2021-10-08]{charstroke}
%   \begin{syntax}
%     charstroke = <none|solid|dashed|invisible> \init{none}
%   \end{syntax}
%   设置字符外轮廓样式。
%
%   初始值 \opt{none} 按原样输出。\opt{solid} 设置外轮廓为 0.10bp 的实线,
%   \opt{dashed} 设置外轮廓为 0.10bp 的虚线。同时,不填充轮廓内部,显示为背景颜色。
%   \opt{invisible} 将字符设置为不可见,但不影响背景和网格的显示,隐藏的字仍然可被复制。
%
%   \begin{SideBySideExample}[frame=single,numbers=left,xrightmargin=.30\linewidth,gobble=5]
%     \centering
%     \hanziboxset{width=1.0cm,resize=real,frametype=咪,
%                  framecolor=red}
%     \hanzibox[charstroke=none                           ]{我}
%     \hanzibox[charstroke=solid,charcolor=red            ]{我}\\
%     \hanzibox[charstroke=dashed,charcolor=green!40!black]{我}
%     \hanzibox[charstroke=invisible                      ]{我}
%   \end{SideBySideExample}
% \end{function}
%
% \subsection{声母、韵母和声调开关}
%
% \begin{function}[added=2021-09-24,updated=2021-09-24]{initial,vowel,tone}
%   \begin{syntax}
%     initial = <\TTF> \init{true}
%     vowel   = <\TTF> \init{true}
%     tone    = <\TTF> \init{true}
%   \end{syntax}
%   分别用于设置是否输出拼音的声母、韵母和声调,默认值为\textbf{true}。
%
%   该选项对\tn{hanzidialog}命令无效。
%   \begin{SideBySideExample}[frame=single,numbers=left,xrightmargin=.28\linewidth,gobble=5]
%
%     \centering
%     \hanziboxset{frametype=咪,framecolor=red,
%                  fillcolor=yellow!40,resize=real}
%     \hanzibox{明月几时有}[]
%       [How long will the full moon appear?]\\[0.5ex]
%     \hanzibox[initial=false]{明月几时有}[]
%       [How long will the full moon appear?]\\[0.5ex]
%     \hanzibox[vowel=false]{明月几时有}[]
%       [How long will the full moon appear?]\\[0.5ex]
%     \hanzibox[tone=false]{明月几时有}[]
%       [How long will the full moon appear?]\\[0.5ex]
%     \hanzibox[initial=false,vowel=false]{明有几时有}
%       [][How long will the full moon appear?]
%
%   \end{SideBySideExample}
%   \begin{SideBySideExample}[frame=single,numbers=left,xrightmargin=.28\linewidth,gobble=5]
%     \centering
%     \hanziboxset{frametype=咪,framecolor=red,fillcolor=yellow!40,
%       charf=\Huge,pinyinf=\small,tranf=\small,resize=real}
%     \hanzibox*[tone=false]{我}[wo3][吾]
%     \hanzibox*[vowel=false]{我}[wo3][吾]
%     \hanzibox*[initial=false]{我}[wo3][不才]
%   \end{SideBySideExample}
% \end{function}
%
% \subsection{拼音四线格开关}
%
% \begin{function}[added=2021-10-07,updated=2021-10-08]{pinyinline}
%   \begin{syntax}
%     pinyinline = <\TFF> \init{false}
%   \end{syntax}
%   用于设置是否输出拼音四线格,默认值为\textbf{false}。
%   为保持拼音对齐一致性,\verb!pinyinline=false!时,
%   仅不输出拼音四线格,但拼音四线格的空间占位仍然存在。
%   \begin{SideBySideExample}[frame=single,numbers=left,xrightmargin=.33\linewidth,gobble=5]
%     \centering
%     \hanziboxset{frametype=咪,framecolor=red,charf=\large,
%                  fillcolor=yellow!40}
%     \hanzibox[pinyinline=false]{明月几时有}[]
%       [How long will the full moon appear?]\\[0.5ex]
%     \hanzibox[pinyinline=true]{明月几时有}[]
%   \end{SideBySideExample}
% \end{function}
%
% \subsection{拼音、汉字和译文开关}
%
% \begin{function}[added=2021-09-21,updated=2021-09-26]{pinyin,hanzi,tran}
%   \begin{syntax}
%     pinyin = <\TTF> \init{true}
%     hanzi  = <\TTF>  \init{true}
%     tran   = <\TTF>   \init{true}
%   \end{syntax}
%   分别用于设置是否输出拼音、汉字和译文,默认值为\textbf{true}。
%   该选项对\tn{hanzidialog}命令无效。
%   \begin{SideBySideExample}[frame=single,numbers=left,xrightmargin=.33\linewidth,gobble=5]
%     \centering
%     \hanziboxset{frametype=咪,framecolor=red,charf=\large,
%                  fillcolor=yellow!40}
%     \hanzibox[pinyin=false]{明月几时有}[]
%       [How long will the full moon appear?]\\[0.5ex]
%     \hanzibox[hanzi=false]{明月几时有}[]
%       [How long will the full moon appear?]\\[0.5ex]
%     \hanzibox[tran=false]{明月几时有}[]
%       [How long will the full moon appear?]\\[0.5ex]
%   \end{SideBySideExample}
% \end{function}
%
% \subsection{作文格式选项}
%
% \begin{function}[added=2022-04-17,updated=2022-04-17]{gridsepv,gridcols}
%   \begin{syntax}
%     gridsepv  = \meta{number} \init{4}
%     gridcols  = \meta{integer} \init{20}
%   \end{syntax}
%   \oarg{gridsepv}用于设置作文格子行间间距,
%   $\text{行间距}=\text{单个盒子高度}\times\frac{1}{number}$,
%   默认值取4。
%
%   \oarg{gridcols}用于设置格子每行的列数,默认值取20。
%
%   这两个选项对\tn{hanzibox}和\tn{hanzidialog}命令无效。
%   \begin{SideBySideExample}[frame=single,numbers=left,xrightmargin=.50\linewidth,gobble=5]
%     \centering
%     \hanziboxset{frametype=咪,framecolor=red,
%        charf=\large,fillcolor=yellow!40}
%     \writegrid{3}\\
%     \writegrid[gridsepv=0.5,gridcols=10]{3}\\
%     \writegrid[gridsepv=1.0,gridcols=15]{3}
%   \end{SideBySideExample}
% \end{function}
%
% \changes{v2.0.0}{2021/09/28}{在说明文档中添加应用实例}
% \changes{v2.2.0}{2021/10/07}{为部分实例添加拼音四线格}
%
% \section{应用实例}
%
% \pkg{hanzibox}宏包可以广泛用于汉字学习的练习中。
%
% \subsection{拼一拼---写一写练习}
% 利用各种选项的有效组合,可以实现汉字拼一拼---写一写练习。
%
% \textcolor[HTML]{AD1457}{注意:\tn{hspace*}命令中的
% 参数\textbf{0.4pt}是边框线条宽度,请根据实际情况调整}。
%   \begin{SideBySideExample}[frame=single,numbers=left,xrightmargin=.32\linewidth,gobble=5]
%     \centering
%     \hanziboxset{frametype=田,framecolor=blue,
%                  charf=\Huge,pinyinf=\small}
%     \hanzibox{门}\hspace*{-0.4pt}\hanzibox[hanzi=false]{口}
%     \hfill
%     \hanzibox{生}\hspace*{-0.4pt}\hanzibox[hanzi=false]{日}\\
%     \hanzibox{题}\hspace*{-0.4pt}\hanzibox[hanzi=false]{目}
%     \hfill
%     \hanzibox[hanzi=false]{田}\hspace*{-0.4pt}\hanzibox{野}
%   \end{SideBySideExample}
%
% \subsection{标注声母练习}
%
%   \begin{SideBySideExample}[frame=single,numbers=left,xrightmargin=.32\linewidth,gobble=5]
%     \centering
%     \hanziboxset{frametype=咪,framecolor=red,charcolor=red,
%      pinyinline=true,charf=\Huge,pinyinf=\small,initial=false}
%     \hanzibox{门口}\hfill
%     \hanzibox{生日}\\
%     \hanzibox{题目}\hfill
%     \hanzibox{田野}
%   \end{SideBySideExample}
%
% \subsection{标注韵母练习}
%
%   \begin{SideBySideExample}[frame=single,numbers=left,xrightmargin=.32\linewidth,gobble=5]
%     \centering
%     \hanziboxset{frametype=咪,framecolor=red,charcolor=red,
%       pinyinline=true,charf=\Huge,pinyinf=\small,vowel=false}
%     \hanzibox{门口}\hfill
%     \hanzibox{生日}\\
%     \hanzibox{题目}\hfill
%     \hanzibox{田野}
%   \end{SideBySideExample}
%
% \subsection{标注声调练习}
%
%   \begin{SideBySideExample}[frame=single,numbers=left,xrightmargin=.32\linewidth,gobble=5]
%     \centering
%     \hanziboxset{frametype=咪,framecolor=red,charcolor=red,
%       pinyinline=true,charf=\Huge,pinyinf=\small,tone=false}
%     \hanzibox{门口}\hfill
%     \hanzibox{生日}\\
%     \hanzibox{题目}\hfill
%     \hanzibox{田野}
%   \end{SideBySideExample}
%
% \subsection{随机生成生词练习}
%
% 假设提前准备了生词表\verb!\clist_set:Nn \l__words_clist!,则可以使用
% \LaTeX3的随机函数随机生成生词练习题(每次编译可以得到不同的结果)。
%
%   \begin{SideBySideExample}[frame=single,numbers=left,xrightmargin=.30\linewidth,gobble=5]
%     \ExplSyntaxOn
%     \clist_set:Nn \l__words_clist
%       {
%         {铅笔} , {橡皮} , {报纸} ,
%         {头发} , {耳朵} , {眼睛} ,
%         {大象} , {蚂蚁} , {松鼠} ,
%         {男孩} , {同学} , {兄弟} ,
%         {学生} , {医生} , {护士} ,
%         {老师} , {警察} , {羊肉} ,
%         {窗户} , {镜子} , {沙发}
%       }
%     \hanziboxset{frametype=咪,framecolor=red,charcolor=red,
%                  charf=\huge,pinyinf=\footnotesize,hanzi=false}
%     \centering
%     \int_step_inline:nn {6}
%       {
%         \hanzibox{\clist_rand_item:N \l__words_clist}\\
%       }
%     \ExplSyntaxOff
%   \end{SideBySideExample}
%
% \subsection{随机生成拼音练习}
%
% 假设提前准备了声母和韵母表,则可以使用
% \LaTeX3的随机函数随机生成拼音练习题(每次编译可以得到不同的结果)。
% 此时,若生成的拼音不正确,可让学生填写“无”。
%
%   \begin{SideBySideExample}[frame=single,numbers=left,xrightmargin=.25\linewidth,gobble=5]
%     \ExplSyntaxOn
%     \int_new:N \l__tone_int
%     \clist_set:Nn \l__initials_clist
%       {
%         {zh} , {ch} , {sh} , {b} , {p} , {m} , {f} ,
%         {d}  , {t}  , {n}  , {l} , {g} , {k} , {h} ,
%         {j}  , {q}  , {x}  , {r} , {z} , {c} , {s} ,
%         {y}  , {w}
%       }
%     \clist_set:Nn \l__vowel_clist
%       {
%         {iang} , {iong} , {uang} , {ueng} , {ang} , {eng} , {ing} ,
%         {ong}  , {uai}  , {uan}  , {uai}  , {uei} , {iao} , {iou} ,
%         {ian}  , {van}  , {uen}  , {ai}   , {ei}  , {ua}  , {uo}  ,
%         {ui}   , {ao}   , {ou}   , {iu}   , {ie}  , {ve}  , {er}  ,
%         {an}   , {en}   , {in}   , {un}   , {vn}  , {a}   , {e}   ,
%         {i}    , {o}    , {u}    , {v}
%       }
%     \hanziboxset{frametype=咪,framecolor=red,charcolor=red,
%                  charf=\huge,pinyinf=\footnotesize,hanzi=false}
%     \centering
%     \int_step_inline:nn {10}
%       {
%         \int_zero:N \l__tone_int
%         \int_set:Nn \l__tone_int {\int_rand:n {5}}
%         \hanzibox*{好}[
%           \clist_rand_item:N \l__initials_clist
%           \clist_rand_item:N \l__vowel_clist
%           \int_use:N \l__tone_int
%          ]\\
%       }
%     \ExplSyntaxOff
%   \end{SideBySideExample}
%
% \changes{v2.1.1}{2021/10/07}{在格子纸示例中用coffin实现每行格子以提升编译速度}
%
% \subsection{生成汉字字帖格子纸}
%
% 可以通过将\tn{hanzibox}命令的\Arg{汉字}参数留空,并设置\verb!tran=false!,
% 或将\tn{hanzibox}、\tn{hanzibox*}命令的\verb!hanzi!选项置为
% \verb!false!(\verb!hanzi=false!),从而生成空白背景格子,
% 再根据需要通过循环的方式生成指定行数和列数的
% 汉字书写练习用格子纸。
%
% \textcolor[HTML]{AD1457}{注意:参数中的\textbf{0.4pt}是边框线条宽度,
% 请根据实际情况调整}。
%   \begin{SideBySideExample}[frame=single,numbers=left,xrightmargin=.37\linewidth,gobble=5]
%     \hanziboxset{frametype=咪,framecolor=red,
%          tran=false,charcolor=red,charf=\huge}
%     \centering
%     \ExplSyntaxOn
%     \hcoffin_set:Nn \l_tmpa_coffin
%       {
%         \int_step_inline:nn {6}
%           {
%             \hanzibox{}
%             \hspace*{-0.40pt}
%           }
%       }
%     \hcoffin_set:Nn \l_tmpb_coffin
%       {}
%     \int_step_inline:nn {8}
%       {
%         \coffin_join:NnnNnnnn \l_tmpb_coffin { hc } { b }
%           \l_tmpa_coffin { hc } { t } { 0pt } { 0.4pt }
%       }
%     \coffin_typeset:Nnnnn
%       \l_tmpb_coffin { l } { b } { 0pt } { 0pt }
%     \ExplSyntaxOff
%   \end{SideBySideExample}
%
%   \begin{SideBySideExample}[frame=single,numbers=left,xrightmargin=.37\linewidth,gobble=5]
%     \hanziboxset{frametype=咪,framecolor=red,
%       pinyinline=true,charf=\huge,hanzi=false}
%     \centering
%     \ExplSyntaxOn
%     \hcoffin_set:Nn \l_tmpa_coffin
%       {
%         \int_step_inline:nn {6}
%           {
%             \hanzibox*{国}
%             \hspace*{-0.40pt}
%           }
%       }
%     \int_step_inline:nn {8}
%       {
%         \coffin_typeset:Nnnnn
%           \l_tmpa_coffin { l } { b } { 0pt } { 0pt }
%         \par\nointerlineskip
%      }
%     \ExplSyntaxOff
%   \end{SideBySideExample}
%
% \subsection{生成作文题目格子纸}
%
% 可以使用本宏包提供的\tn{writegrid}命令生成作文题目中用于答题的
% 格式纸。
%
%   \begin{SideBySideExample}[frame=single,numbers=left,xrightmargin=.48\linewidth,gobble=5]
%     \hanziboxset{frametype=口,framecolor=black,
%                  charcolor=red,charf=\huge}
%     \centering
%     \writegrid[gridcols=8]{10}
%   \end{SideBySideExample}
%
% \subsection{生成诗词注音}
%
% 可以通过自动注音生成带有注音的诗词排版,
% 但当有多音字时,需要使用\pkg{xpinyin}宏包的\tn{setpinyin}命令
% 为多音字设置正确的读音。
%
%   \begin{SideBySideExample}[frame=single,numbers=left,xrightmargin=.50\linewidth,gobble=5]
%     \setpinyin{长}{chang2}
%     \setpinyin{尽}{jin4}
%     \hanziboxset{frametype=咪,framecolor=red,
%       charf=\huge,pinyinf=\footnotesize,
%       charcolor=green!40!black,
%       pinyincolor=green!40!black,
%       trancolor=green!40!black}
%     \centering
%     \hanzibox{故人西辞黄鹤楼}
%     \hanzibox{烟花三月下扬州}
%     \hanzibox{孤帆远影碧空尽}
%     \hanzibox{唯见长江天际流}
%   \end{SideBySideExample}
%
% \subsection{诗词手动注音}
%
% 也可以使用\tn{hanzibox*}命令实现诗词手动注音,
% 此时,可以通过留空拼音或文字构成注音或根据拼音写汉字练习。
% 但需要注意,目前只能在一行文本的尾部实现留空练习。
%
%   \begin{SideBySideExample}[frame=single,numbers=left,xrightmargin=.33\linewidth,gobble=5]
%     \hanziboxset{frametype=咪,framecolor=red,charf=\Large,
%       pinyinline=true,charcolor=green!40!black,
%       pinyincolor=red!20!black,trancolor=blue!40!black}
%     \centering
%     \hanzibox*{故人西辞黄鹤楼}[gu4ren2xi1ci2huang2he2lou2]
%     \hanzibox*{烟花三月下扬州}[yan1hua1san1yue4]
%     \hanzibox*{孤帆远影      }[gu1fan1yuan3ying3bi4kong1jin4]
%     \hanzibox*{}[wei2jian4chang2jiang1tian1ji4liu2]
%   \end{SideBySideExample}
%
% \subsection{生成描红练习}
%
% 合理的设置汉字的颜色浓淡或通过\verb!charstroke!选项设置
% 汉字轮廓选项,可以生成用于描红练习的格子纸。
%
% 若设置\verb!charstroke=invisible!,则会使汉字隐藏不可见,
% 但隐藏的汉字仍然可被复制。
%
%   \begin{SideBySideExample}[frame=single,numbers=left,xrightmargin=.25\linewidth,gobble=5]
%     \hanziboxset{frametype=咪,framecolor=red,pinyinline=true,
%       pinyincolor=green!30!black,charf=\Large,pinyinf=\footnotesize}
%     \centering
%     \hanzibox[charcolor=red!30,pinyincolor=red]{讲普通话}
%     \hanzibox[charcolor=black!30,pinyinline=false]{写规范字}
%     \hanzibox[charcolor=red,charstroke=solid]{讲普通话}
%     \hanzibox[charcolor=black,charstroke=dashed]{写规范字}
%     \hanzibox[charcolor=red,charstroke=invisible]{讲普通话}
%   \end{SideBySideExample}
%
% \end{documentation}
%
% \StopEventually{}
%
% \begin{implementation}
%
% \section{代码实现}
%
% 本宏包使用 \LaTeX3 语法编写,依赖 \pkg{expl3} 环境,
% 并需调用 \pkg{l3packages}、\pkg{l3draw}、\pkg{xpinyin}等宏包。
%
% 按照 \LaTeX3 语法,代码中的空格、换行、回车与制表符会完全被忽略,
% 而下划线“|_|”和冒号“|:|”则可作为一般字母使用。
% 正常的空格可以使用“|~|”代替;至于 |~| 原来所表示的“带子”,
% 则要用 \LaTeXe{} 的原始命令 \tn{nobreakspace} 代替。
%
% 以下代码中有一些形如 \textcolor[HTML]{2E3191}{\textsf{<*package>}}
% 的标记,这是 \pkg{DocStrip} 中的“guard”,用来选择性地提取文件。
% “\textsf{*}”和“\textsf{/}”分别表示该部分的开始和结束。不含
% “\textsf{*}”和“\textsf{/}”的 guard 出现在行号右侧,它们用来确定
% 单独一行代码的归属。这些 guard 的颜色深浅不一,用以明确嵌套关系。
%
% 另有若干形如 \textcolor[HTML]{AD1457}{\textsf{<@@=hanzibox>}} 的 guard,
% 它们由 \pkg{l3docstrip} 定义,用来标识名字空间(模块)。
%
% \subsection{环境检测与准备}
%
%    \begin{macrocode}
%<*package>
%<@@=hanzibox>
%    \end{macrocode}
%
% 载入必要的宏包
%
%    \begin{macrocode}

\RequirePackage { xtemplate, l3keys2e, l3draw, xparse }

%    \end{macrocode}
%
% 检查LaTeX3宏包版本
%
%    \begin{macrocode}
% \clist_map_inline:nn { xtemplate, l3keys2e }
%   {
%     \@ifpackagelater {#1} { 2020/07/17 }
%       { } { \msg_error:nnn { hanzibox } { l3-too-old } {#1} }
%   }
% \msg_new:nnn { hanzibox } { l3-too-old }
%   {
%     Package~ "#1"~ is~ too~ old. \\\\
%     Please~ update~ an~ up-to-date~ version~ of~ the~ bundles \\
%     "l3kernel"~ and~ "l3packages"~ using~ your~ TeX~ package \\
%     manager~ or~ from~ CTAN.
%   }
%
%    \end{macrocode}
%
% \changes{v2.1.1}{2021/10/03}{解决盒子高度计算函数的expl3版本兼容问题}
%
% 判断\tn{box_ht_plus_dp:N}函数是否存在,若不存在,则定义该函数。
% 为了解决与expl3的旧版本兼容问题(
% 摘录于\url{https://ask.latexstudio.net/ask/question/3773.html})。
%
%    \begin{macrocode}
\cs_if_free:NT \box_ht_plus_dp:N
  {
    \cs_new_protected:Npn \box_ht_plus_dp:N #1
      { \tex_dimexpr:D \box_ht:N #1 + \box_dp:N #1 \scan_stop: }
  }
%    \end{macrocode}
%
% 检查编译引擎,目前仅支持xetex引擎。
%
%    \begin{macrocode}
\sys_if_engine_xetex:F
  {
    \msg_fatal:nnx { hanzibox } { unsupported-engine }
      { \c_sys_engine_str }
  }
\msg_new:nnn { hanzibox } { unsupported-engine }
  {
    The~ hanzibox~ packages~ requires~ XeTeX. \\\\
    "#1"~ is~ not~ supported~ at~ present.~ You~ must~ change \\
    your~ typesetting~ engine~ to~ "xelatex"~ or~ "lualatex".
  }

%    \end{macrocode}
%
% 载入xpinyin宏包
%
%    \begin{macrocode}
\RequirePackage { xpinyin }

%    \end{macrocode}
%
% \subsection{用户接口}
%
% \changes{v1.1.0}{2021/09/22}{为用户命令添加组限制}
% \changes{v2.0.0}{2021/09/26}{调整星号命令为手动拼音注音}
% 背景盒子由l3draw实现,
% 其设计思路和部分源码来自\LaTeX 的\pkg{zitie}宏包(\url{https://www.ctan.org/pkg/zitie})。
%
% \begin{macro}{\hanzibox}
% 自动注音汉字盒子命令。
%    \begin{macrocode}
\NewDocumentCommand{\hanzibox}{ s O{} m O{} O{} }
  {
    \IfBooleanTF{#1}
      {
        \bool_set_false:N \l_@@_autopinyin_bool
      }{
        \bool_set_true:N  \l_@@_autopinyin_bool
      }
      \group_begin:
        \@@_handle:nnnn { #2 } { #3 } { #4 } { #5 }
      \group_end:
  }

%    \end{macrocode}
% \end{macro}
%
% \begin{macro}{\hanzidialog}
% 手动注音汉字盒子命令。
%    \begin{macrocode}
\NewDocumentCommand{\hanzidialog}{O{} m O{} O{} }
  {
    \group_begin:
      \bool_set_false:N \l_@@_autopinyin_bool
      \@@_dialog:nnnn { #1 } { #2 } { #3 } { #4 }
    \group_end:
  }
%    \end{macrocode}
% \end{macro}
%
% \changes{v2.3.0}{2022/04/17}{添加\tn{writegrid}作文格子输出命令。}
%
% \begin{macro}{\writegrid}
% 自动注音汉字盒子命令。
%    \begin{macrocode}
\NewDocumentCommand{\writegrid}{ O{} m }
  {
    \group_begin:
      \@@_writegrid:nn { #1 } { #2 }
    \group_end:
  }
%    \end{macrocode}
% \end{macro}
%
% \subsection{内部变量声明}
%
% \changes{v1.1.0}{2021/09/21}{添加缩放比例,前景/背景色,隐藏控制变量。}
% \changes{v2.0.0}{2021/09/24}{根据l3draw的需要重新设计各个变量。}
% \changes{v2.1.0}{2021/09/30}{添加记录拼音返回结果的clist变量。}
% \changes{v2.2.0}{2021/10/07}{添加拼音线开关及拼音和译文颜色变量}
% \changes{v2.2.0}{2021/10/08}{添加字符轮廓类型选择变量}
% \changes{v2.2.0}{2021/10/09}{添加拼音汉字coffin变量}
% \changes{v2.3.0}{2022/04/17}{添加作文格子垂直间距系数变量及coffin变量。}
%
% \begin{variable}{\l_@@_autopinyin_bool,
% \l_@@_withinitial_bool,
% \l_@@_withvowel_bool,
% \l_@@_withtone_bool,
% \l_@@_withpinyin_bool,
% \l_@@_withpinyinlines_bool,
% \l_@@_withhanzi_bool,
% \l_@@_withtran_bool,
% \l_@@_basebox_box,
% \l_@@_frame_type_tl,
% \g_@@_frame_list_clist,
% \l_@@_resize_method_tl,
% \g_@@_resize_method_clist,
% \l_@@_frame_size_dim,
% \l_@@_char_width_dim,
% \l_@@_char_height_dim,
% \l_@@_pinyin_height_i_dim,
% \l_@@_pinyin_height_ii_dim,
% \l_@@_pinyin_height_iii_dim,
% \l_@@_box_width_dim,
% \l_@@_box_height_dim,
% \l_@@_frame_linewidth_dim,
% \l_@@_pinyin_linewidth_dim,
% \l_@@_cross_linewidth_dim,
% \l_@@_tone_pinyin_clist,
% \l_@@_str_box_coffin,
% \l_@@_box_coffin,
% \l_@@_pinyin_box_coffin,
% \l_@@_tran_box_coffin,
% \l_@@_pinyin_hanzi_coffin,
% \l_@@_tmpa_coffin,
% \l_@@_tmpb_coffin,
% \l_@@_grid_coffin,
% \l_@@_grid_tmpa_coffin,
% \l_@@_grid_tmpb_coffin,
% \hanziboxwidth,
% \hanziboxheight,
% \l_@@_pinyin_tl,
% \l_@@_character_tl,
% \l_@@_translation_tl,
% \l_@@_pinyin_format_tl,
% \l_@@_character_format_tl,
% \l_@@_translation_format_tl,
% \l_@@_cross_color_ratio_int,
% \l_@@_pinyin_int,
% \l_@@_character_int,
% \l_@@_translation_int,
% \l_@@_charstroke_type_int,
% \l_@@_grid_cols_int,
% \l_@@_tone_int,
% \l_@@_pinyin_str,
% \l_@@_initial_tl,
% \l_@@_vowel_tl,
% \l_@@_grid_sep_v_tl,
%  }
% 定义变量。
%
%    \begin{macrocode}
\bool_new:N   \l_@@_autopinyin_bool
\bool_new:N   \l_@@_withinitial_bool
\bool_new:N   \l_@@_withvowel_bool
\bool_new:N   \l_@@_withtone_bool
\bool_new:N   \l_@@_withpinyin_bool
\bool_new:N   \l_@@_withpinyinlines_bool
\bool_new:N   \l_@@_withhanzi_bool
\bool_new:N   \l_@@_withtran_bool

\box_new:N    \l_@@_basebox_box
\tl_new:N     \l_@@_frame_type_tl
\clist_new:N  \g_@@_frame_list_clist
\tl_new:N     \l_@@_resize_method_tl
\clist_new:N  \g_@@_resize_method_clist
\dim_new:N    \l_@@_frame_size_dim
\dim_new:N    \l_@@_char_width_dim
\dim_new:N    \l_@@_char_height_dim
\dim_new:N    \l_@@_pinyin_height_i_dim
\dim_new:N    \l_@@_pinyin_height_ii_dim
\dim_new:N    \l_@@_pinyin_height_iii_dim
\dim_new:N    \l_@@_box_width_dim
\dim_new:N    \l_@@_box_height_dim
\dim_new:N    \l_@@_frame_linewidth_dim
\dim_new:N    \l_@@_pinyin_linewidth_dim
\dim_new:N    \l_@@_cross_linewidth_dim

\clist_new:N  \l_@@_tone_pinyin_clist

\coffin_new:N \l_@@_str_box_coffin
\coffin_new:N \l_@@_box_coffin
\coffin_new:N \l_@@_pinyin_box_coffin
\coffin_new:N \l_@@_tran_box_coffin
\coffin_new:N \l_@@_pinyin_hanzi_coffin
\coffin_new:N \l_@@_tmpa_coffin
\coffin_new:N \l_@@_tmpb_coffin
\coffin_new:N \l_@@_grid_coffin
\coffin_new:N \l_@@_grid_tmpa_coffin
\coffin_new:N \l_@@_grid_tmpb_coffin

\dim_new:N    \hanziboxwidth
\dim_new:N    \hanziboxheight

\tl_new:N     \l_@@_pinyin_tl
\tl_new:N     \l_@@_character_tl
\tl_new:N     \l_@@_translation_tl
\tl_new:N     \l_@@_pinyin_format_tl
\tl_new:N     \l_@@_character_format_tl
\tl_new:N     \l_@@_translation_format_tl
\int_new:N    \l_@@_cross_color_ratio_int
\int_new:N    \l_@@_pinyin_int
\int_new:N    \l_@@_character_int
\int_new:N    \l_@@_translation_int
\int_new:N    \l_@@_charstroke_type_int
\int_new:N    \l_@@_grid_cols_int

\int_new:N    \l_@@_tone_int
\str_new:N    \l_@@_pinyin_str
\tl_new:N     \l_@@_initial_tl
\tl_new:N     \l_@@_vowel_tl
\tl_new:N     \l_@@_grid_sep_v_tl

%    \end{macrocode}
% \end{variable}
%
% \subsection{辅助函数}
%
% \subsubsection{设置填充色}
%
% \begin{macro}{\@@_aux_color_fill:}
% 设置空白填充色
%    \begin{macrocode}
\cs_new_nopar:Nn \@@_aux_color_fill:
  { }
%    \end{macrocode}
% \end{macro}
%
% \changes{v2.2.0}{2021/10/07}{添加拼音高度计算函数}
%
% \subsubsection{计算拼音高度}
%
% \begin{macro}{\@@_calc_pinyin_h:}
% 计算拼音线基础调试(通过字母"a"的高度计算)
%    \begin{macrocode}
\cs_new:Npn \@@_calc_pinyin_h:
  {
    \hbox_set:Nn \l_tmpa_box
      {
        \tl_use:N \l_@@_pinyin_format_tl
        a
      }
    \dim_set:Nn \l_@@_pinyin_height_i_dim
      {
        \box_ht:N \l_tmpa_box
      }
    \dim_set:Nn \l_@@_pinyin_height_ii_dim
      {
        \l_@@_pinyin_height_i_dim + \l_@@_pinyin_height_i_dim
      }
    \dim_set:Nn \l_@@_pinyin_height_iii_dim
      {
        \l_@@_pinyin_height_i_dim + \l_@@_pinyin_height_i_dim + \l_@@_pinyin_height_i_dim
      }
  }
%    \end{macrocode}
% \end{macro}
%
% \subsubsection{计算盒子尺寸}
%
% \begin{macro}{\@@_calc_basechar_w_h:}
% 计算基字符盒子宽和高
%    \begin{macrocode}
\cs_new:Npn \@@_calc_basechar_w_h:
  {
    \dim_set:Nn \l_@@_char_width_dim
      {
        \box_wd:N \l_@@_basebox_box
      }
    \dim_set:Nn \l_@@_char_height_dim
      {
        \box_ht_plus_dp:N \l_@@_basebox_box
      }
  }
%    \end{macrocode}
% \end{macro}
%
% \begin{macro}{\@@_coffin_ht_plus_dp:N}
% 获取coffin盒子总高度
%    \begin{macrocode}
\cs_new_nopar:Npn \@@_coffin_ht_plus_dp:N #1
  {
    \coffin_ht:N #1 + \coffin_dp:N #1
  }
%    \end{macrocode}
% \end{macro}
%
% \changes{v2.0.0}{2021/09/28}{添加根据基字符格式计算盒子尺寸函数}
%
% \begin{macro}{\@@_calc_frame_size:}
% 计算外框长度(正方形,由基字符按charf选项设定的格式构造的盒子确定)
%    \begin{macrocode}
\cs_new:Npn \@@_calc_frame_size:
  {
    \hbox_set:Nn \l_tmpa_box
      {
        \tl_use:N \l_@@_character_format_tl
        \tl_use:N \c_@@_basechar_tl
      }

    \dim_set:Nn \l_tmpa_dim
      {
        \box_wd:N \l_tmpa_box
      }
    \dim_set:Nn \l_tmpb_dim
      {
        \box_ht_plus_dp:N \l_tmpa_box
      }

    \dim_compare:nNnTF \l_tmpa_dim > \l_tmpb_dim
      {
        \dim_gset_eq:NN \l_@@_frame_size_dim \l_tmpa_dim
      }
      {
        \dim_gset_eq:NN \l_@@_frame_size_dim \l_tmpb_dim
      }

    \dim_gadd:Nn \l_@@_frame_size_dim { 1pt }

    \dim_gset_eq:NN \hanziboxwidth \l_@@_frame_size_dim
    \dim_gset_eq:NN \hanziboxheight \l_@@_frame_size_dim

  }
%    \end{macrocode}
% \end{macro}
%
% \subsubsection{定义边框样式}
%
% \begin{macro}{\@@_frame_type:n,\@@_frame_type_c:n}
% 生成边框样式函数名称
%    \begin{macrocode}
\cs_new_nopar:Npn \@@_frame_type:n #1
  {
    @@_frame_construct_type_ #1 :nnnnnn
  }
\cs_new_nopar:Npn \@@_frame_type_c:n #1
  {
    \use:c
      {
        @@_frame_construct_type_ #1 :nnnnnn
      }
  }
%    \end{macrocode}
% \end{macro}
%
% \begin{macro}{\@@_new_frame_construct:nn}
% 边框样式函数的定义函数。
%    \begin{macrocode}
\cs_new:Npn \@@_new_frame_construct:nn #1
  {
    \clist_put_right:Nn \g_@@_frame_list_clist {#1}
    \cs_new:cn { \@@_frame_type:n {#1} }
  }
\@@_new_frame_construct:nn { none } { }
%    \end{macrocode}
% \end{macro}
%
% \changes{v2.2.0}{2021/10/07}{添加拼音四线格绘制函数}
%
%    \begin{macrocode}
\@@_new_frame_construct:nn { pinyinlines }
  {
    \bool_if:NTF \l_@@_withpinyinlines_bool
      {
        \draw_scope_begin:
          \draw_path_moveto:n { #1, 0 }
          \draw_path_lineto:n { #3, 0 }
          \draw_path_moveto:n { #1, \l_@@_pinyin_height_i_dim }
          \draw_path_lineto:n { #3, \l_@@_pinyin_height_i_dim }
          \draw_path_moveto:n { #1, \l_@@_pinyin_height_ii_dim }
          \draw_path_lineto:n { #3, \l_@@_pinyin_height_ii_dim }
          \draw_path_moveto:n { #1, \l_@@_pinyin_height_iii_dim }
          \draw_path_lineto:n { #3, \l_@@_pinyin_height_iii_dim }
          \draw_path_use_clear:n { stroke }
        \draw_scope_end:
      }
      {
        \draw_scope_begin:
          \hcoffin_set:Nn \l_tmpa_coffin
            {
              \tl_use:N \l_@@_pinyin_format_tl
              \phantom{a}
            }

          \coffin_resize:Nnn \l_tmpa_coffin
            { #3 } { \l_@@_pinyin_height_iii_dim }

          \draw_coffin_use:Nnn \l_tmpa_coffin { l } { b }
        \draw_scope_end:
      }
  }
%    \end{macrocode}
%
% \changes{v2.1.2}{2021/10/07}{分离汉字外框与填充绘制函数}
%
%    \begin{macrocode}
\@@_new_frame_construct:nn { filledbox }
  {
    \cs_if_eq:NNF \@@_aux_color_fill: \c_empty_tl
      {
        \color_stroke:n { hanziboxframecolor }
        \draw_path_rectangle_corners:nn { #1, #2} { #3, #4}

        \draw_path_use_clear:n { stroke, fill }
      }
  }
%    \end{macrocode}
%
%    \begin{macrocode}
\@@_new_frame_construct:nn { framebox }
  {
    \draw_scope_begin:
      \color_stroke:n { hanziboxframecolor }
      \draw_path_rectangle_corners:nn { #1, #2} { #3, #4}
      \draw_path_use_clear:n { stroke }
    \draw_scope_end:
  }
%    \end{macrocode}
%
% \changes{v2.1.1}{2021/10/07}{内格子线颜色独立设置}
%
%    \begin{macrocode}
\@@_new_frame_construct:nn { 十 }
  {
    \draw_scope_begin:
      \tl_if_empty:NF \l_@@_dash_pattern_tl
        {
          \exp_args:No \draw_dash_pattern:nn { \l_@@_dash_pattern_tl } { 0pt }
        }
      \draw_linewidth:n{ \l_@@_cross_linewidth_dim }
      \color_stroke:n { hanziboxcrosscolor }
      \draw_path_moveto:n { (#3)/2, #2 }
      \draw_path_lineto:n { #3/2, #4 }
      \draw_path_moveto:n { #1, (#4)/2 }
      \draw_path_lineto:n { #3, (#4)/2 }
      \draw_path_use_clear:n { stroke }
    \draw_scope_end:
  }

\@@_new_frame_construct:nn { × }
  {
    \draw_scope_begin:
      \tl_if_empty:NF \l_@@_dash_pattern_tl
        {
          \exp_args:No \draw_dash_pattern:nn { \l_@@_dash_pattern_tl } { 0pt }
        }
      \draw_linewidth:n{ \l_@@_cross_linewidth_dim }
      \color_stroke:n { hanziboxcrosscolor }
      \draw_path_moveto:n { #1, #2 }
      \draw_path_lineto:n { #3, #4 }
      \draw_path_moveto:n { #1, #4 }
      \draw_path_lineto:n { #3, #2 }
      \draw_path_use_clear:n { stroke }
    \draw_scope_end:
  }

\@@_new_frame_construct:nn { 米 }
  {
    \@@_frame_type_c:n { × } {#1} {#2} {#3} {#4} {#5} {#6}
    \@@_frame_type_c:n { 十 } {#1} {#2} {#3} {#4} {#5} {#6}
  }

\@@_new_frame_construct:nn { 口 }
  {
    \@@_frame_type_c:n { filledbox } {#1} {#2} {#3} {#4} {#5} {#6}
    \@@_frame_type_c:n { framebox } {#1} {#2} {#3} {#4} {#5} {#6}
  }

\@@_new_frame_construct:nn { 田 }
  {
    \@@_frame_type_c:n { filledbox } {#1} {#2} {#3} {#4} {#5} {#6}
    \@@_frame_type_c:n { 十 } {#1} {#2} {#3} {#4} {#5} {#6}
    \@@_frame_type_c:n { framebox } {#1} {#2} {#3} {#4} {#5} {#6}
  }

\@@_new_frame_construct:nn { 咪 }
  {
    \@@_frame_type_c:n { filledbox } {#1} {#2} {#3} {#4} {#5} {#6}
    \@@_frame_type_c:n { × } {#1} {#2} {#3} {#4} {#5} {#6}
    \@@_frame_type_c:n { 十 } {#1} {#2} {#3} {#4} {#5} {#6}
    \@@_frame_type_c:n { framebox } {#1} {#2} {#3} {#4} {#5} {#6}
  }
%    \end{macrocode}
%
% \subsubsection{定义边框类型错误提示信息}
%
%    \begin{macrocode}
\msg_new:nnn { hanzibox } { frame-exists } { The~ frame~ type~ `#1~ not~ exists. }
%    \end{macrocode}
%
% \subsubsection{定义缩放方式}
%
% \begin{macro}{\@@_resize:n,\@@_resize_c:n}
% 生成缩放方式函数名称
%    \begin{macrocode}
\cs_new_nopar:Npn \@@_resize:n #1
  {
    @@_processor_resize_ #1 :w
  }
\cs_new_nopar:Npn \@@_resize_c:n #1
  {
    \use:c
      {
        @@_processor_resize_ #1 :w
      }
  }
%    \end{macrocode}
% \end{macro}
%
% \begin{macro}{\@@_dim_gezero_dispatch:NNnnn,\@@_dim_gezero_dispatch:NNnnn,
%               \@@_force_size_dispatch:nnn,\@@_force_size_dispatch:nnnn}
% 定义缩放方式函数需要的辅助函数。
%    \begin{macrocode}
\cs_new:Npn \@@_dim_gezero_dispatch:NNnnn #1#2 #3#4#5
  {
    \dim_compare:nNnTF #1 > \c_zero_dim
      { #3 }
      {
        \dim_compare:nNnTF #2 > \c_zero_dim
          { #4 } { #5 }
      }
  }
\cs_new:Npn \@@_dim_gezero_dispatch:NNnnnn #1#2 #3#4#5#6
  {
    \dim_compare:nNnTF #1 > \c_zero_dim
      {
        \dim_compare:nNnTF #2 > \c_zero_dim
          { #3 } { #4 }
      }
      {
        \dim_compare:nNnTF #2 > \c_zero_dim
          { #5 } { #6 }
      }
  }
\cs_new:Npn \@@_force_size_dispatch:nnn % height, width, none
  {
    \@@_dim_gezero_dispatch:NNnnn \l_@@_height_dim \l_@@_width_dim
  }
\cs_new:Npn \@@_force_size_dispatch:nnnn % both, height, width, none
  {
    \@@_dim_gezero_dispatch:NNnnnn \l_@@_box_height_dim \l_@@_box_width_dim
  }
%    \end{macrocode}
% \end{macro}
%
% \begin{macro}{\@@_new_resize_method:nn}
% 构建缩放方式列表及函数。
%    \begin{macrocode}
\cs_new:Npn \@@_new_resize_method:nn #1
  {
    \clist_put_right:Nn \g_@@_resize_method_clist {#1}
    \cs_new:cpn { \@@_resize:n {#1} }
  }
%    \end{macrocode}
% \end{macro}
%
% \changes{v2.0.0}{2021/09/28}{删除square缩放样式}
%
% 定义缩放方式函数。
%    \begin{macrocode}
\@@_new_resize_method:nn { none } { }

\@@_new_resize_method:nn { real }
  {
    \@@_force_size_dispatch:nnnn
      {
        \coffin_resize:Nnn \l_@@_box_coffin
                           \l_@@_box_width_dim
                           \l_@@_box_height_dim
      }
      {
        \coffin_scale:Nnn \l_@@_box_coffin
          {
            \dim_ratio:nn { \l_@@_box_height_dim }
                          { \@@_coffin_ht_plus_dp:N \l_@@_box_coffin }
          }
          {
            \dim_ratio:nn { \l_@@_box_height_dim }
                          { \@@_coffin_ht_plus_dp:N \l_@@_box_coffin }
          }
      }
      {
        \coffin_scale:Nnn \l_@@_box_coffin
          {
            \dim_ratio:nn { \l_@@_box_width_dim }
                          { \coffin_wd:N \l_@@_box_coffin }
          }
          {
            \dim_ratio:nn { \l_@@_box_width_dim }
                          { \coffin_wd:N \l_@@_box_coffin }
          }
      }
      {
        \coffin_scale:Nnn \l_@@_box_coffin
                          { \l_@@_x_scale_tl }
                          { \l_@@_y_scale_tl }
      }
  }

\@@_new_resize_method:nn { base }
  {
    \@@_force_size_dispatch:nnnn
      {
        \coffin_resize:Nnn \l_@@_box_coffin
                           \l_@@_box_width_dim
                           \l_@@_box_height_dim
      }
      {
        \coffin_resize:Nnn \l_@@_box_coffin
           {
             \l_@@_char_width_dim * \dim_ratio:nn { \l_@@_box_height_dim }
               { \@@_coffin_ht_plus_dp:N \l_@@_box_coffin }
           }
           {
             \l_@@_box_height_dim
           }
      }
      {
        \coffin_resize:Nnn \l_@@_box_coffin
           {
             \l_@@_box_width_dim
           }
           {
             \l_@@_char_height_dim * \dim_ratio:nn { \l_@@_box_width_dim }
               { \coffin_wd:N \l_@@_box_coffin }
           }
      }
      {
        \coffin_resize:Nnn \l_@@_box_coffin
           {
             \l_@@_x_scale_tl \l_@@_char_width_dim
           }
           {
             \l_@@_y_scale_tl \l_@@_char_height_dim
           }
      }
  }

\msg_new:nnn { hanzibox } { frame-type } { using~ `#1'~ frame. }
%    \end{macrocode}
%
% \subsubsection{设置字号}
%
% \begin{macro}{\@@_zihao:n}
% 设置字号
%    \begin{macrocode}
\cs_new_nopar:Npn \@@_zihao:n #1 { \zihao {#1} }
%    \end{macrocode}
% \end{macro}
%
% \changes{v2.2.0}{2021/10/08}{添加字符轮廓处理函数}
%
% \subsubsection{字符轮廓处理函数}
% 源码改自\LaTeX 的\pkg{zitie}宏包(\url{https://www.ctan.org/pkg/zitie})。
%
% \begin{macro}{\@@_chars_stroke:nn}
% 设置字符轮廓函数
%    \begin{macrocode}
\cs_new:Npn \@@_chars_stroke:nn #1#2
  {
    \special { pdf:code ~ q ~ #1 } #2 \special { pdf:code ~ Q }
  }
%    \end{macrocode}
% \end{macro}
%
% \begin{macro}{\@@_chars_stroke:nn}
% 字符轮廓选择函数
%    \begin{macrocode}
\cs_new_protected:Npn \@@_chars_stroke_construct:n #1
  {
    \int_case:nn {\l_@@_charstroke_type_int}
      {
        {1}{ #1 }
        {2}{
          \@@_chars_stroke:nn { 1 ~ Tr ~ 0.10 ~ w ~ [] ~ 0 ~ d ~ 1 ~ J } {#1}
        }
        {3}{
          \@@_chars_stroke:nn { 1 ~ Tr ~ 0.10 ~ w ~ [1~1] ~ 0 ~ d ~ 1 ~ J } {#1}
        }
        {4}{
          \@@_chars_stroke:nn { 3 ~ Tr } {#1}
        }

      }
  }
\cs_generate_variant:Nn  \@@_chars_stroke_construct:n { V }
\cs_generate_variant:Nn  \@@_chars_stroke_construct:n { x }
%    \end{macrocode}
% \end{macro}
%
% \subsubsection{命名颜色}
%
% \begin{macro}{\@@_color_select:nn,\@@_color_select:nnn}
% 颜色命名函数(使用l3语法)
%    \begin{macrocode}
\cs_set_nopar:Npn \@@_color_select:nn #1#2
  {
    \color_set:nn {#1} {#2}
  }
\cs_generate_variant:Nn \@@_color_select:nn {nx}
\cs_set_nopar:Npn \@@_color_select:nnn #1#2#3
  {
    \color_set:nnn {#1} {#2} {#3}
  }
\cs_generate_variant:Nn \@@_color_select:nnn {nnx}
%    \end{macrocode}
% \end{macro}
%
% \subsubsection{设置Debug状态}
%
% \begin{macro}{\@@_debug:n}
% 设置debug状态
%    \begin{macrocode}
\cs_new:Npn \@@_debug:n
  {
    \bool_if:NTF \l_@@_debug_bool
      { \use:n } { \use_none:n }
  }
%    \end{macrocode}
% \end{macro}
%
% \subsection{选项处理}
%
% \changes{v2.0.0}{2021/09/25}{参考zitie宏包,重新设计选项,
%                              仅保留1.1.0版本中的拼音、汉字、译文显示控制选项。}
%
% 定义 |hanzibox| 键值类。
%    \begin{macrocode}
\keys_define:nn { hanzibox }
  {
%    \end{macrocode}
%
% \begin{macro}{basechar}
% 设置基字符。
%    \begin{macrocode}
    basechar  .code:n = { \tl_gset:Nx \c_@@_basechar_tl {#1}
                          \@@_calc_basechar_w_h:
                        },
%    \end{macrocode}
% \end{macro}
%
% \begin{macro}{zihao}
% 设置字号。
%    \begin{macrocode}
    zihao     .code:n = { \hbox_gset:Nn \l_@@_basebox_box
                            {
                              \@@_zihao:n {#1} \c_@@_basechar_tl
                            }
                          \@@_calc_basechar_w_h:
                        },
%    \end{macrocode}
% \end{macro}
%
% \changes{v2.2.0}{2021/10/07}{为pinyinf选项增加计算拼音高度功能}
%
% \begin{macro}{pinyinf}
% 拼音格式
%    \begin{macrocode}
    pinyinf .code:n = { \tl_set:Nn \l_@@_pinyin_format_tl { #1 }
                        \@@_calc_pinyin_h:
                      },
    pinyinf .initial:n = \tiny ,
%    \end{macrocode}
% \end{macro}
%
% \begin{macro}{charf}
% 汉字格式
%    \begin{macrocode}
    charf .code:n = { \tl_gset:Nn \l_@@_character_format_tl {#1}
                      \@@_calc_frame_size:
                    },
%    \end{macrocode}
% \end{macro}
%
% \begin{macro}{tranf}
% 译文格式
%    \begin{macrocode}
    tranf .tl_set:N = \l_@@_translation_format_tl ,
    tranf .initial:n = \tiny ,
%    \end{macrocode}
% \end{macro}
%
% \begin{macro}{frametype}
% 边框类型
%    \begin{macrocode}
    frametype .code:n = { \exp_args:NNx \clist_if_in:NnTF \g_@@_frame_list_clist {#1}
                            { \tl_set:Nx \l_@@_frame_type_tl {#1} }
                            { \msg_error:nnx { hanzibox } { frame-exists } {#1} }
                        },
%    \end{macrocode}
% \end{macro}
%
% \begin{macro}{resize}
% 缩放方式
%    \begin{macrocode}
    resize    .code:n = { \exp_args:NNx \clist_if_in:NnTF \g_@@_resize_method_clist {#1}
                            { \tl_set:Nx \l_@@_resize_method_tl {#1} }
                            { \msg_error:nnx { hanzibox } { resize-method } {#1} }
                        },
%    \end{macrocode}
% \end{macro}
%
% \begin{macro}{xscale}
% x方向缩放比例
%    \begin{macrocode}
    xscale .tl_set:N = \l_@@_x_scale_tl ,
    xscale .initial:n = 1 ,
%    \end{macrocode}
% \end{macro}
%
% \begin{macro}{yscale}
% y方向缩放比例
%    \begin{macrocode}
    yscale .tl_set:N = \l_@@_y_scale_tl ,
    yscale .initial:n = 1 ,
%    \end{macrocode}
% \end{macro}
%
% \begin{macro}{scale}
% x,y方向缩放比例
%    \begin{macrocode}
    scale  .meta:n = { xscale = #1 , yscale = #1 } ,
%    \end{macrocode}
% \end{macro}
%
% \begin{macro}{width}
% 盒子宽度
%    \begin{macrocode}
    width  .dim_set:N = \l_@@_box_width_dim ,
%    \end{macrocode}
% \end{macro}
%
% \begin{macro}{height}
% 盒子高度
%    \begin{macrocode}
    height .dim_set:N = \l_@@_box_height_dim ,
%    \end{macrocode}
% \end{macro}
%
% \begin{macro}{linewidth}
% 外框线条宽度
%    \begin{macrocode}
    linewidth .dim_set:N = \l_@@_frame_linewidth_dim ,
    linewidth .initial:n = 0.4pt ,
%    \end{macrocode}
% \end{macro}
%
% \changes{v2.2.0}{2021/10/10}{添加边框线线宽\opt{framelinewidth}选项}
%
% \begin{macro}{framelinewidth}
% 外框线条宽度
%    \begin{macrocode}
    framelinewidth .dim_set:N = \l_@@_frame_linewidth_dim ,
    framelinewidth .initial:n = 0.4pt ,
%    \end{macrocode}
% \end{macro}
%
% \changes{v2.2.0}{2021/10/10}{添加拼音四线格线宽pinyinlinewidth选项}
%
% \begin{macro}{pinyinlinewidth}
% 拼音四线格线条宽度
%    \begin{macrocode}
    pinyinlinewidth .dim_set:N = \l_@@_pinyin_linewidth_dim ,
    pinyinlinewidth .initial:n = 0.4pt ,
%    \end{macrocode}
% \end{macro}
%
% \changes{v2.2.0}{2021/10/10}{添加内格十字和米字线线宽crosslinewidth选项}
%
% \begin{macro}{crosslinewidth}
% 内格十字和米字线线条宽度
%    \begin{macrocode}
    crosslinewidth .dim_set:N = \l_@@_cross_linewidth_dim ,
    crosslinewidth .initial:n = 0.3pt ,
%    \end{macrocode}
% \end{macro}
%
% \changes{v2.1.1}{2021/10/07}{添加颜色比例选项corsscolorratio}
%
% \begin{macro}{crosscolorratio}
% 盒子内部线条颜色占边框颜色的百分比
%    \begin{macrocode}
    crosscolorratio  .int_set:N = \l_@@_cross_color_ratio_int,
    crosscolorratio  .initial:n = 20,
%    \end{macrocode}
% \end{macro}
%
% \begin{macro}{framecolor}
% 边框颜色
%    \begin{macrocode}
    framecolor  .code:n = { \tl_set:Nx \l_tmpa_tl { #1 ! \int_use:N \l_@@_cross_color_ratio_int }
                            \@@_color_select:nn { hanziboxframecolor } {#1}
                            \@@_color_select:nx{ hanziboxcrosscolor } { \l_tmpa_tl } } ,
    framecolor  .initial:n = black ,
    framecolor* .code:n = { \tl_set:Nx \l_tmpa_tl { #1 ! \int_use:N \l_@@_cross_color_ratio_int }
                            \@@_color_select:nnn { hanziboxframecolor } #1
                            \@@_color_select:nnx { hanziboxcrosscolor } \l_tmpa_tl } ,
%    \end{macrocode}
% \end{macro}
%
% \begin{macro}{charcolor}
% 字符颜色
%    \begin{macrocode}
    charcolor  .code:n = { \@@_color_select:nn { hanziboxcharcolor } {#1} } ,
    charcolor  .initial:n = black ,
    charcolor* .code:n = { \@@_color_select:nnn { hanziboxcharcolor } #1 } ,
%    \end{macrocode}
% \end{macro}
%
% \changes{v2.2.0}{2021/10/07}{添加拼音颜色\opt{pinyincolor}选项}
%
% \begin{macro}{pinyincolor}
% 拼音颜色
%    \begin{macrocode}
    pinyincolor  .code:n = { \@@_color_select:nn { hanziboxpinyincolor } {#1} } ,
    pinyincolor  .initial:n = black ,
    pinyincolor* .code:n = { \@@_color_select:nnn { hanziboxpinyincolor } #1 } ,
%    \end{macrocode}
% \end{macro}
%
% \changes{v2.2.0}{2021/10/07}{添加译文颜色\opt{trancolor}选项}
%
% \begin{macro}{trancolor}
% 译文颜色
%    \begin{macrocode}
    trancolor  .code:n = { \@@_color_select:nn { hanziboxtrancolor } {#1} } ,
    trancolor  .initial:n = black ,
    trancolor* .code:n = { \@@_color_select:nnn { hanziboxtrancolor } #1 } ,
%    \end{macrocode}
% \end{macro}
%
% \begin{macro}{color}
% 同时设置边框、字符、拼音和译文颜色
%    \begin{macrocode}
    color  .meta:n = { framecolor = #1, crosscolor = #1,  charcolor = #1,
                       pinyincolor = #1, trancolor = #1 } ,
    color* .meta:n = { framecolor* = #1, crosscolor = #1,  charcolor* = #1,
                       pinyincolor* = #1, trancolor* = #1 } ,
%    \end{macrocode}
% \end{macro}
%
% \begin{macro}{fillcolor}
% 填充色
%    \begin{macrocode}
    fillcolor  .code:n = { \exp_args:Nx \tl_if_empty:nTF {#1}
                            {  \@@_color_select:nn { hanziboxfillcolor } { white }
                               \cs_set_nopar:Npn \@@_aux_color_fill: { }
                            }{ \@@_color_select:nn { hanziboxfillcolor } {#1}
                               \cs_set_nopar:Npn \@@_aux_color_fill: { \color_fill:n {#1} }
                            }
                        } ,
    fillcolor* .code:n = { \@@_color_select:nnn { hanziboxfillcolor } #1
                           \cs_set_nopar:Npn \@@_aux_color_fill: { \color_fill:nn #1 }
                         } ,
%    \end{macrocode}
% \end{macro}
%
% \changes{v2.2.0}{2021/10/08}{添加字符轮廓类型charstroke选项}
%
% \begin{macro}{charstroke}
% 设置汉字轮廓类型
%    \begin{macrocode}
    charstroke .choice:,
    charstroke .value_required:n = true,
    charstroke .choices:nn =
      { none, solid, dashed, invisible }
      { \int_set_eq:NN \l_@@_charstroke_type_int \l_keys_choice_int },
    charstroke .initial:n = none,
%    \end{macrocode}
% \end{macro}
%
% \begin{macro}{dashpattern}
% 虚线样式
%    \begin{macrocode}
    dashpattern .tl_set:N = \l_@@_dash_pattern_tl ,
    dashpattern .initial:n = { } ,
%    \end{macrocode}
% \end{macro}
%
% \begin{macro}{framearc}
% 矩形外框转角半径
%    \begin{macrocode}
    framearc  .code:n = { \tl_set:Nn \l_@@_frame_arc_tl { {#1}{#1} } } ,
    framearc* .tl_set:N = \l_@@_frame_arc_tl ,
    framearc* .initial:n = { { 0cm }{ 0cm } } ,
%    \end{macrocode}
% \end{macro}
%
% \begin{macro}{debug}
% Debug状态
%    \begin{macrocode}
    debug .bool_set:N = \l_@@_debug_bool ,
    debug .initial:n = false ,
    debug .default:n = true ,
%    \end{macrocode}
% \end{macro}
%
% \begin{macro}{autopinyin}
% 是否通过汉字自动获取拼音,默认为true。
%    \begin{macrocode}
    autopinyin .bool_set:N = \l_@@_autopinyin_bool,
    autopinyin .default:n = true,
    autopinyin .initial:n = true,

%    \end{macrocode}
% \end{macro}
%
% \begin{macro}{initial}
% 是否输出声母,默认为true。
%    \begin{macrocode}
    initial .bool_set:N = \l_@@_withinitial_bool,
    initial .default:n = true,
    initial .initial:n = true,

%    \end{macrocode}
% \end{macro}
%
% \begin{macro}{vowel}
% 是否输出韵母,默认为true。
%    \begin{macrocode}
    vowel .bool_set:N = \l_@@_withvowel_bool,
    vowel .default:n = true,
    vowel .initial:n = true,

%    \end{macrocode}
% \end{macro}
%
% \begin{macro}{tone}
% 是否输出声调,默认为true。
%    \begin{macrocode}
    tone .bool_set:N = \l_@@_withtone_bool,
    tone .default:n = true,
    tone .initial:n = true,

%    \end{macrocode}
% \end{macro}
%
% \changes{v2.2.0}{2021/10/07}{添加是否显示拼音线\opt{pinyinline}选项}
%
% \begin{macro}{pinyinline}
% 是否绘制拼音四线格,默认为true。
%    \begin{macrocode}
    pinyinline .bool_set:N = \l_@@_withpinyinlines_bool,
    pinyinline .default:n = true,
    pinyinline .initial:n = false,

%    \end{macrocode}
% \end{macro}
%
% \changes{v1.1.0}{2021/09/21}{添加隐藏拼音、汉字和译文选项}
%
% \begin{macro}{pinyin}
% 是否显示拼音,默认为true。
%    \begin{macrocode}
    pinyin .bool_set:N = \l_@@_withpinyin_bool,
    pinyin .default:n = true,
    pinyin .initial:n = true,

%    \end{macrocode}
% \end{macro}
%
% \begin{macro}{hanzi}
% 是否显示汉字,默认为true。
%    \begin{macrocode}
    hanzi .bool_set:N = \l_@@_withhanzi_bool,
    hanzi .default:n = true,
    hanzi .initial:n = true,

%    \end{macrocode}
% \end{macro}
%
% \begin{macro}{tran}
% 是否显示译文,默认为true。
%    \begin{macrocode}
    tran .bool_set:N = \l_@@_withtran_bool,
    tran .default:n = true,
    tran .initial:n = true,

%    \end{macrocode}
% \end{macro}
%
% \changes{v2.3.0}{2022/04/17}{添加作文格子垂直间距gridsepv选项}
%
% \begin{macro}{gridsepv}
% 作文格子垂直间距
%    \begin{macrocode}
    gridsepv .tl_set:N  = \l_@@_grid_sep_v_tl ,
    gridsepv .initial:n = 4.0 ,
%    \end{macrocode}
% \end{macro}
%
% \changes{v2.3.0}{2022/04/17}{添加作文格子列数gridcols选项}
%
% \begin{macro}{gridsepv}
% 作文格子每行列数
%    \begin{macrocode}
    gridcols .int_set:N  = \l_@@_grid_cols_int ,
    gridcols .initial:n = 20 ,
%    \end{macrocode}
% \end{macro}
%
% 处理未知选项。
%    \begin{macrocode}
    unknown .code:n = { \@@_error:n { unknown-option } }
  }
\msg_new:nnn { hanzibox } { unknown-option }
  { package~ option~ "\l_keys_key_tl"~ is~ unknown. }

%    \end{macrocode}
%
% 参数默认值
%    \begin{macrocode}
\keys_set:nn { hanzibox }
  {
    basechar = 好 ,
    zihao = 4 ,
    pinyinf = \tiny ,
    charf = \normalsize ,
    tranf = \tiny ,
    frametype = none ,
    resize = none ,
  }

%    \end{macrocode}
%
% \subsection{选项用户接口}
%
% \begin{macro}{\hanziboxset}
% 选项设置用户接口。
%    \begin{macrocode}
\NewDocumentCommand \hanziboxset { m }
  { \keys_set:nn { hanzibox } {#1} }
%    \end{macrocode}
% \end{macro}
%
% \subsection{内部函数}
%
% \begin{macro}{\@@_dialog:nnnn}
% 手动汉字盒子排版命令。
%    \begin{macrocode}
\cs_new:Npn \@@_dialog:nnnn #1#2#3#4
  {
    \group_begin:
      \keys_set:nn { hanzibox } { #1 }

      \tl_set:Nx \l_@@_character_tl {#2}
      \tl_set:Nx \l_@@_pinyin_tl {#3}
      \tl_set:Nx \l_@@_translation_tl {#4}

      \hcoffin_set:Nn \l_@@_str_box_coffin
        {
          \tl_map_inline:Nn \l_@@_character_tl
            {
              \@@_single_handle:N ##1
            }
        }
      \hcoffin_set:Nn \l_tmpa_coffin
        {
          \hcoffin_set:Nn \l_@@_pinyin_box_coffin
            {
              \color_select:n { hanziboxpinyincolor }
              \tl_use:N \l_@@_pinyin_format_tl
              \tl_use:N \l_@@_pinyin_tl
            }
          \dim_set:Nn \l_tmpa_dim { \coffin_wd:N \l_@@_pinyin_box_coffin }
          \draw_begin:
            \draw_linewidth:n { \l_@@_frame_linewidth_dim }
            \color_stroke:n { hanziboxframecolor!50 }

            \draw_path_scope_begin:
              \@@_frame_type_c:n { pinyinlines }
                { 0 } { 0 } { \l_tmpa_dim } { \hanziboxheight } { 1.0 } { 1.0 }
              \draw_transform_shift:n {\l_tmpa_dim / 2.0, \l_@@_pinyin_height_i_dim }
              \draw_coffin_use:Nnn \l_@@_pinyin_box_coffin { hc } { H }
            \draw_path_scope_end:
          \draw_end:
        }
      \hcoffin_set:Nn \l_@@_tran_box_coffin
        {
          \tl_use:N \l_@@_translation_format_tl
          \tl_use:N \l_@@_translation_tl
        }
      \coffin_join:NnnNnnnn \l_tmpa_coffin { hc } { b }
        \l_@@_str_box_coffin { hc } { t } { 0pt } { \l_@@_frame_linewidth_dim }
      \coffin_join:NnnNnnnn \l_tmpa_coffin
        { hc } { b } \l_@@_tran_box_coffin { hc } { t } { 0pt } { -2pt }

      \coffin_set_eq:NN \l_@@_box_coffin \l_tmpa_coffin

      \@@_resize_c:n { \l_@@_resize_method_tl }

      \coffin_typeset:Nnnnn \l_@@_box_coffin
        { l } { b } { 0pt } { 0pt }
      \allowbreak
    \group_end:
  }
%    \end{macrocode}
% \end{macro}
%
% \changes{v2.0.1}{2021/09/28}{分离多字、单字、0个字的盒子构造代码}
% \changes{v2.1.0}{2021/09/30}{更新拼音获取方式}
% \changes{v2.2.0}{2021/10/09}{分离汉字拼音coffin构造函数}
%
% \begin{macro}{\@@_single_pinyin_hanzi_construct:NN}
% 构造单个拼音+汉字coffin。
%    \begin{macrocode}
\cs_new:Npn \@@_single_pinyin_hanzi_construct:NN #1#2
  {
    \tl_if_empty:NTF #1
      {
        \hcoffin_set:Nn \l_tmpa_coffin
          {
            \@@_single_handle:N \c_@@_basechar_tl
          }
      }
      {
        \hcoffin_set:Nn \l_tmpa_coffin
          {
            \@@_single_handle:N #1
          }
      }

    \tl_if_empty:NTF #2
      {
        \hcoffin_set:Nn \l_@@_pinyin_hanzi_coffin
          {
            \@@_single_pinyin_lines:
          }

        \coffin_join:NnnNnnnn \l_@@_pinyin_hanzi_coffin
          { hc } { b } \l_tmpa_coffin { hc } { t } { 0pt } { \l_@@_pinyin_linewidth_dim }
      }
      {
        \bool_if:NTF \l_@@_withpinyin_bool
          {
            \hcoffin_set:Nn \l_@@_pinyin_hanzi_coffin
              {
                \@@_single_pinyin:V #2
              }

            \coffin_join:NnnNnnnn \l_@@_pinyin_hanzi_coffin
              { hc } { b } \l_tmpa_coffin { hc } { t } { 0pt } { \l_@@_pinyin_linewidth_dim }
          }
          {
            \coffin_set_eq:NN \l_@@_pinyin_hanzi_coffin \l_tmpa_coffin
          }
      }
  }
%    \end{macrocode}
% \end{macro}
%
% \begin{macro}{\@@_multi_str_coffin_construct:}
% 构造多汉字带拼音字符串盒子。
%    \begin{macrocode}
\cs_new:Npn \@@_multi_str_coffin_construct:
  {
    \hcoffin_set:Nn \l_@@_str_box_coffin
      {
      }
    \bool_if:NTF \l_@@_autopinyin_bool
      {
        \tl_map_inline:Nn \l_@@_character_tl
          {
            \@@_get_hanzi_pinyin:n { ##1 }

            \@@_single_pinyin_hanzi_construct:NN ##1 \l_@@_hanzi_pinyin_tl

            \coffin_join:NnnNnnnn \l_@@_str_box_coffin { r } { b }
              \l_@@_pinyin_hanzi_coffin { l } { b }
                { -\l_@@_frame_linewidth_dim } { 0pt }
          }
      }
      {
        \@@_get_tone_pinyin:V \l_@@_pinyin_tl
        \clist_clear:N \l_@@_tone_pinyin_clist
        \clist_set:NV \l_@@_tone_pinyin_clist \l_@@_tone_pinyin_tl
        \int_set:Nn \l_tmpa_int {\clist_count:N \l_@@_tone_pinyin_clist}
       \int_compare:nNnTF { \l_@@_character_int } = { \l_tmpa_int }
         {
           \tl_map_inline:Nn \l_@@_character_tl
             {
               \clist_pop:NN \l_@@_tone_pinyin_clist \l_tmpb_tl
               \@@_single_pinyin_hanzi_construct:NN ##1 \l_tmpb_tl

               \coffin_join:NnnNnnnn \l_@@_str_box_coffin { r } { b }
                 \l_@@_pinyin_hanzi_coffin { l } { b }
                   { -\l_@@_frame_linewidth_dim } { 0pt }
             }
         }
         {
           \int_compare:nNnTF { \l_@@_character_int } > { \l_tmpa_int }
             {
               \int_step_inline:nn { \l_tmpa_int }
                 {
                   \tl_set:Nx \l_tmpa_tl {\tl_item:Nn \l_@@_character_tl { ##1 }}
                   \clist_pop:NN \l_@@_tone_pinyin_clist \l_tmpb_tl

                   \@@_single_pinyin_hanzi_construct:NN \l_tmpa_tl \l_tmpb_tl
                   \coffin_join:NnnNnnnn \l_@@_str_box_coffin { r } { b }
                     \l_@@_pinyin_hanzi_coffin { l } { b }
                       { -\l_@@_frame_linewidth_dim } { 0pt }
                 }
               \int_step_inline:nnn { \l_tmpa_int + 1 } { \l_@@_character_int }
                 {
                   \tl_set:Nx \l_tmpa_tl {\tl_item:Nn \l_@@_character_tl { ##1 }}
                   \tl_clear:N \l_tmpb_tl

                   \@@_single_pinyin_hanzi_construct:NN \l_tmpa_tl \l_tmpb_tl
                   \coffin_join:NnnNnnnn \l_@@_str_box_coffin { r } { b }
                     \l_@@_pinyin_hanzi_coffin { l } { b }
                       { -\l_@@_frame_linewidth_dim } { 0pt }
                 }
             }
             {
               \int_step_inline:nn { \l_@@_character_int }
                 {
                   \tl_set:Nx \l_tmpa_tl {\tl_item:Nn \l_@@_character_tl { ##1 }}
                   \clist_pop:NN \l_@@_tone_pinyin_clist \l_tmpb_tl

                   \@@_single_pinyin_hanzi_construct:NN \l_tmpa_tl \l_tmpb_tl
                   \coffin_join:NnnNnnnn \l_@@_str_box_coffin { r } { b }
                     \l_@@_pinyin_hanzi_coffin { l } { b }
                       { -\l_@@_frame_linewidth_dim } { 0pt }
                 }

               \bool_set_eq:NN \l_tmpa_bool \l_@@_withhanzi_bool
               \bool_set_false:N \l_@@_withhanzi_bool
               \int_step_inline:nnn { \l_@@_character_int + 1 } { \l_tmpa_int }
                 {
                   \tl_clear:N \l_tmpa_tl
                   \clist_pop:NN \l_@@_tone_pinyin_clist \l_tmpb_tl

                   \@@_single_pinyin_hanzi_construct:NN \l_tmpa_tl \l_tmpb_tl
                   \coffin_join:NnnNnnnn \l_@@_str_box_coffin { r } { b }
                     \l_@@_pinyin_hanzi_coffin { l } { b }
                       { -\l_@@_frame_linewidth_dim } { 0pt }
                 }
               \bool_set_eq:NN \l_@@_withhanzi_bool \l_tmpa_bool
             }
         }
      }
  }
%    \end{macrocode}
% \end{macro}
%
% \begin{macro}{\@@_single_str_coffin_construct:}
% 构造单个汉字带拼音字符串盒子。
%    \begin{macrocode}
\cs_new:Npn \@@_single_str_coffin_construct:
  {
    \bool_if:NTF \l_@@_autopinyin_bool
      {
        \hcoffin_set:Nn \l_@@_str_box_coffin
          {
            \@@_get_hanzi_pinyin:V \l_@@_character_tl

            \@@_single_pinyin_hanzi_construct:NN
              \l_@@_character_tl \l_@@_hanzi_pinyin_tl
            \coffin_typeset:Nnnnn \l_@@_pinyin_hanzi_coffin
              { l } { b } { 0pt } { 0pt }
          }
      }
      {
        \hcoffin_set:Nn \l_@@_str_box_coffin
          {
            \@@_get_tone_pinyin:V \l_@@_pinyin_tl
            \clist_clear:N \l_@@_tone_pinyin_clist
            \clist_set:NV \l_@@_tone_pinyin_clist \l_@@_tone_pinyin_tl
            \tl_set:Nx \l_tmpb_tl { \clist_use:Nn \l_@@_tone_pinyin_clist { } }

            \@@_single_pinyin_hanzi_construct:NN \l_@@_character_tl \l_tmpb_tl
            \coffin_typeset:Nnnnn \l_@@_pinyin_hanzi_coffin
              { l } { b } { 0pt } { 0pt }
          }
      }
  }
%    \end{macrocode}
% \end{macro}
%
% \changes{v2.1.1}{2021/10/02}{修复无汉字手动拼音分割问题}
%
% \begin{macro}{\@@_null_str_coffin_construct:}
% 构造空白汉字(0个汉字)带拼音字符串盒子。
%    \begin{macrocode}
\cs_new:Npn \@@_null_str_coffin_construct:
  {
    \bool_set_eq:NN \l_tmpa_bool \l_@@_withhanzi_bool
    \bool_set_false:N \l_@@_withhanzi_bool
    \bool_if:NTF \l_@@_autopinyin_bool
      {
        \hcoffin_set:Nn \l_@@_str_box_coffin
          {
            \@@_single_handle:N \c_@@_basechar_tl
          }
      }
      {
        \hcoffin_set:Nn \l_@@_str_box_coffin
          {
          }
        \bool_if:NTF \l_@@_withpinyin_bool
          {
            \@@_get_tone_pinyin:V \l_@@_pinyin_tl
            \clist_clear:N \l_@@_tone_pinyin_clist
            \clist_set:NV \l_@@_tone_pinyin_clist \l_@@_tone_pinyin_tl
            \int_set:Nn \l_tmpa_int {\clist_count:N \l_@@_tone_pinyin_clist}

            \int_step_inline:nn { \l_tmpa_int }
              {
                \tl_clear:N \l_tmpa_tl
                \clist_pop:NN \l_@@_tone_pinyin_clist \l_tmpb_tl

                \@@_single_pinyin_hanzi_construct:NN \l_tmpa_tl \l_tmpb_tl
                \coffin_join:NnnNnnnn \l_@@_str_box_coffin { r } { b }
                  \l_@@_pinyin_hanzi_coffin { l } { b }
                    { -\l_@@_frame_linewidth_dim } { 0pt }
              }
          }
          {
            \bool_set_false:N \l_@@_withhanzi_bool
            \@@_single_handle:N \c_@@_basechar_tl
          }
      }
    \bool_set_eq:NN \l_@@_withhanzi_bool \l_tmpa_bool
  }
%    \end{macrocode}
% \end{macro}
%
% \changes{v2.0.0}{2021/09/28}{将拼音处理调整为在handle函数中直接实现}
% \changes{v2.0.0}{2021/09/28}{区分了单个汉字和空白汉字的处理}
% \changes{v2.0.1}{2021/09/29}{将汉字盒子处理过程拆解为函数实现}
%
% \begin{macro}{\@@_handle:nnnn}
% 构造汉字盒子入口
%    \begin{macrocode}
\cs_new:Npn \@@_handle:nnnn #1#2#3#4
  {
    \group_begin:
      \keys_set:nn { hanzibox } { #1 }

      \tl_gset:Nx \l_@@_character_tl {#2}
      \tl_gset:Nx \l_@@_pinyin_tl {#3}
      \tl_gset:Nx \l_@@_translation_tl {#4}

      \int_set:Nn \l_@@_character_int
        {
          \tl_count:V \l_@@_character_tl
        }
      \int_set:Nn \l_@@_translation_int
        {
          \tl_count:V \l_@@_translation_tl
        }
      \int_set:Nn \l_@@_pinyin_int
        {
          \tl_count:V \l_@@_pinyin_tl
        }

      \int_compare:nNnTF { \l_@@_character_int } > { 1 }
        {
          \@@_multi_str_coffin_construct:
        }
        {
          \int_compare:nNnTF { \l_@@_character_int } = { 1 }
            {
              \@@_single_str_coffin_construct:
            }
            {
              \@@_null_str_coffin_construct:
            }
        }
      \bool_if:NT \l_@@_withtran_bool
        {
          \hcoffin_set:Nn \l_@@_tran_box_coffin
            {
              \color_select:n { hanziboxtrancolor }
              \tl_use:N \l_@@_translation_format_tl
              \tl_use:N \l_@@_translation_tl
            }
        }
      \coffin_join:NnnNnnnn \l_@@_str_box_coffin
        { hc } { b } \l_@@_tran_box_coffin { hc } { t } { 0pt } { -3pt }

      \coffin_set_eq:NN \l_@@_box_coffin \l_@@_str_box_coffin

      \@@_resize_c:n { \l_@@_resize_method_tl }

      \coffin_typeset:Nnnnn \l_@@_box_coffin
        { l } { b } { 0pt } { 0pt }
      \allowbreak
    \group_end:
  }
%    \end{macrocode}
% \end{macro}
%
% \changes{v2.3.0}{2022/04/17}{添加作文格子内部命令。}
%
% \begin{macro}{\@@_writegrid:nnn}
% 构造作文格子入口
%    \begin{macrocode}
\cs_new:Npn \@@_writegrid:nn #1#2
  {
    \group_begin:
      \keys_set:nn { hanzibox } { #1 }

      \tl_if_eq:NnT \l_@@_frame_type_tl { none }
        { \tl_set:Nn \l_@@_frame_type_tl { 口 } }

      \hcoffin_set:Nn \l_@@_grid_tmpa_coffin
        {
          \draw_begin:
            \draw_linewidth:n { \l_@@_frame_linewidth_dim }
            \@@_aux_color_fill:
            \color_stroke:n { hanziboxframecolor }

            \draw_path_scope_begin:
              \@@_frame_type_c:n { \l_@@_frame_type_tl }
                { 0 } { 0 } { \hanziboxwidth } { \hanziboxheight } { 1.0 } { 1.0 }
              \int_decr:N \l_@@_grid_cols_int
              \int_step_inline:nn  { \l_@@_grid_cols_int }
                {
                  \draw_transform_shift:n {\hanziboxwidth, 0.0 }
                  \@@_frame_type_c:n { \l_@@_frame_type_tl }
                    { 0 } { 0 } { \hanziboxwidth } { \hanziboxheight } { 1.0 } { 1.0 }
                }
            \draw_path_scope_end:
          \draw_end:
        }

      \hcoffin_set:Nn \l_@@_grid_tmpb_coffin
        {
          \coffin_typeset:Nnnnn \l_@@_grid_tmpa_coffin
            { l } { b } { 0pt } { 0pt }
        }

      \int_step_inline:nn { #2 - 1 }
        {
          \coffin_join:NnnNnnnn \l_@@_grid_tmpb_coffin { hc } { b }
            \l_@@_grid_tmpa_coffin { hc } { t } { 0pt }
            { -\hanziboxheight * \dim_ratio:nn { 1 pt }{ \l_@@_grid_sep_v_tl pt } }
        }

      \dim_set:Nn \l_tmpa_dim
        {
          \coffin_wd:N \l_@@_grid_tmpb_coffin
        }
      \dim_set:Nn \l_tmpb_dim
        {
          \tex_dimexpr:D \coffin_ht:N \l_@@_grid_tmpb_coffin +
                          \coffin_dp:N \l_@@_grid_tmpb_coffin \scan_stop:
        }

      \draw_begin:
        \draw_linewidth:n { \l_@@_frame_linewidth_dim * 4 }
        \@@_aux_color_fill:
        \color_stroke:n { hanziboxframecolor }

        \draw_path_scope_begin:
          \draw_path_rectangle_corners:nn { 0cm , 0cm } { \l_tmpa_dim, \l_tmpb_dim }
          \draw_path_use_clear:n { draw }
          \draw_transform_shift:n {\l_tmpa_dim / 2.0, \l_tmpb_dim / 2.0 }
          \draw_coffin_use:Nnn \l_@@_grid_tmpb_coffin { hc } { vc }
        \draw_path_scope_end:
      \draw_end:

    \group_end:
  }
%    \end{macrocode}
% \end{macro}
%
% \changes{v2.0.0}{2021/09/27}{对于无需分解声韵母的情况,
%     直接使用xpinyin宏包的结果排版拼音。}
%
% \begin{macro}{\@@_single_pinyin:n}
% 构造单个汉字的拼音盒子
%    \begin{macrocode}
\cs_new:Npn \@@_single_pinyin:n #1
  {
    \bool_if:NTF \l_@@_withtone_bool
      {
        \bool_if:nTF { !(\l_@@_withinitial_bool) || !(\l_@@_withvowel_bool) }
          {
            \@@_split_pinyin_withtone:n { #1 }
            \hcoffin_set:Nn \l_@@_pinyin_box_coffin
              {
                \color_select:n { hanziboxpinyincolor }
                \tl_use:N \l_@@_pinyin_format_tl

                \bool_if:NTF \l_@@_withinitial_bool
                  {
                    \bool_if:NTF \l_@@_withvowel_bool
                    {
                      \tl_use:N \l_@@_initial_tl
                      \tl_use:N \l_@@_vowel_tl
                    }
                    {
                      \tl_use:N \l_@@_initial_tl
                      \phantom{ \tl_use:N \l_@@_vowel_tl }
                    }
                  }
                  {
                    \bool_if:NTF \l_@@_withvowel_bool
                    {
                      \phantom{ \tl_use:N \l_@@_initial_tl }
                      \tl_use:N \l_@@_vowel_tl
                    }
                    {
                      \phantom{ \tl_use:N \l_@@_initial_tl }
                      \phantom{ \tl_use:N \l_@@_vowel_tl }
                    }
                  }
              }
          }
          {
            \hcoffin_set:Nn \l_@@_pinyin_box_coffin
              {
                \color_select:n { hanziboxpinyincolor }
                \tl_use:N \l_@@_pinyin_format_tl
                #1
              }
          }
      }
      {
        \@@_split_pinyin_withouttone:n { #1 }
        \hcoffin_set:Nn \l_@@_pinyin_box_coffin
        {
          \color_select:n { hanziboxtrancolor }
          \tl_use:N \l_@@_pinyin_format_tl

          \tl_use:N \l_@@_pinyin_tl
        }
      }
      \@@_single_pinyin_lines_construct:
  }
\cs_generate_variant:Nn  \@@_single_pinyin:n { V }
\cs_generate_variant:Nn  \@@_single_pinyin:n { x }
\cs_set:Npn \@@_single_pinyin_o:n
  { \exp_after:wN \@@_single_pinyin:n }
\cs_set:Npn \@@_single_pinyin_f:n
  { \exp_args:Nf \@@_single_pinyin:n }
%    \end{macrocode}
% \end{macro}
%
% \begin{macro}{\@@_single_handle:nN,\@@_single_handle:N}
% 构造单个汉字盒子入口
%    \begin{macrocode}
\cs_new:Npn \@@_single_handle:nN #1#2
  {
    \group_begin:
      \tl_if_empty:nF {#1} { \keys_set:nn { hanzibox } {#1} }

      \tl_set:Nf \l_@@_curr_char_tl {#2}

      \@@_single_construct_o:N \l_@@_curr_char_tl
    \group_end:
  }
\cs_new:Npn \@@_single_handle:N #1
  {
    \group_begin:
      \tl_set:Nf \l_@@_curr_char_tl {#1}
      \@@_single_construct_o:N \l_@@_curr_char_tl
    \group_end:
  }
%    \end{macrocode}
% \end{macro}
%
% \changes{v2.0.0}{2021/09/27}{删除单个汉字构造中添加拼音的功能}
% \changes{v2.2.0}{2021/10/08}{为汉字添加字符轮廓处理}
%
% \begin{macro}{\@@_single_construct:N}
% 构造单个汉字盒子
%    \begin{macrocode}
\cs_new:Npn \@@_single_construct:N #1
  {
    \bool_if:NTF \l_@@_withhanzi_bool
      {
        \hcoffin_set:Nn \l_@@_box_coffin
          {
            \color_select:n { hanziboxcharcolor }
            \tl_use:N \l_@@_character_format_tl
            \@@_chars_stroke_construct:n { #1 }
          }
      }
      {
        \hcoffin_set:Nn \l_@@_box_coffin
          {
            \color_select:n { hanziboxcharcolor }
            \tl_use:N \l_@@_character_format_tl
            \phantom{#1}
          }
      }

    \@@_single_frame_construct:
  }
\cs_set:Npn \@@_single_construct_o:N
  { \exp_after:wN \@@_single_construct:N }
\cs_set:Npn \@@_single_construct_f:N
  { \exp_args:Nf \@@_single_construct:N }
%    \end{macrocode}
% \end{macro}
%
% \changes{v2.0.0}{2021/09/28}{将汉字盒子与边框按中心对齐,以使汉字居中}
%
% \begin{macro}{\@@_single_frame_construct:}
% 构造单个汉字盒子边框
%    \begin{macrocode}
\cs_new:Npn \@@_single_frame_construct:
  {
    \draw_begin:
      \draw_linewidth:n { \l_@@_frame_linewidth_dim }
      \@@_aux_color_fill:
      \color_stroke:n { hanziboxframecolor }

      \exp_after:wN \draw_path_corner_arc:nn \l_@@_frame_arc_tl

      \draw_path_scope_begin:
        \@@_frame_type_c:n { \l_@@_frame_type_tl }
          { 0 } { 0 } { \hanziboxwidth } { \hanziboxheight } { 1.0 } { 1.0 }
        \draw_transform_shift:n {\hanziboxwidth / 2.0, \hanziboxheight / 2.0 }
        \draw_coffin_use:Nnn \l_@@_box_coffin { hc } { vc }
      \draw_path_scope_end:
    \draw_end:
  }
%    \end{macrocode}
% \end{macro}
%
% \changes{v2.2.0}{2021/10/07}{添加构造单个拼音盒子函数}
%
% \begin{macro}{\@@_single_pinyin_lines_construct:}
% 构造单个拼音盒子
%    \begin{macrocode}
\cs_new:Npn \@@_single_pinyin_lines_construct:
  {
    \draw_begin:
      \draw_linewidth:n { \l_@@_pinyin_linewidth_dim }
      \color_stroke:n { hanziboxframecolor!50 }

      \draw_path_scope_begin:
        \@@_frame_type_c:n { pinyinlines }
          { 0 } { 0 } { \hanziboxwidth } { \hanziboxheight } { 1.0 } { 1.0 }
        \draw_transform_shift:n {\hanziboxwidth / 2.0, \l_@@_pinyin_height_i_dim }
        \draw_coffin_use:Nnn \l_@@_pinyin_box_coffin { hc } { H }
      \draw_path_scope_end:
    \draw_end:
  }
%    \end{macrocode}
% \end{macro}
%
% \changes{v2.2.0}{2021/10/07}{添加构造单个拼音线函数}
%
% \begin{macro}{\@@_single_pinyin_lines:}
% 构造单个拼音线
%    \begin{macrocode}
\cs_new:Npn \@@_single_pinyin_lines:
  {
    \draw_begin:
      \draw_linewidth:n { \l_@@_pinyin_linewidth_dim }
      \color_stroke:n { hanziboxframecolor!50 }

      \@@_frame_type_c:n { pinyinlines }
        { 0 } { 0 } { \hanziboxwidth } { \hanziboxheight } { 1.0 } { 1.0 }
    \draw_end:
  }
%    \end{macrocode}
% \end{macro}
%
% \subsection{\pkg{xpinyin}宏包拼音后处理函数}
%
% \changes{v2.1.0}{2021/09/30}{添加从xpinyin宏包中提取拼音串函数。}
% 摘录自LaTeX工作室问答:如何得到xpinyin拼音宏包得到的拼音文本?
% (\url{https://ask.latexstudio.net/ask/question/3768.html})
%
% 变量定义
%    \begin{macrocode}
\tl_new:N \l_@@_save_tl
\tl_new:N \l_@@_hanzi_pinyin_tl
\tl_new:N \l_@@_tone_pinyin_tl
%    \end{macrocode}
%
% 构造声调表
%    \begin{macrocode}
\clist_const:Nn \c_@@_tone_a_clist { ā,á,ǎ,à,a }
\clist_const:Nn \c_@@_tone_o_clist { ō,ó,ǒ,ò,o }
\clist_const:Nn \c_@@_tone_e_clist { ē,é,ě,è,e }
\clist_const:Nn \c_@@_tone_u_clist { ū,ú,ǔ,ù,u }
\clist_const:Nn \c_@@_tone_i_clist { ī,í,ǐ,ì,i }
\clist_const:Nn \c_@@_tone_v_clist { ǖ,ǘ,ǚ,ǜ,ü }
%    \end{macrocode}
%
% \begin{macro}{\@@_pinyin_aux:n}
% 拼音生成辅助函数(改自xpinyin宏包的\verb!\__xpinyin_pinyin_aux:n #1!函数)
%    \begin{macrocode}
\cs_new_protected:Npn \@@_pinyin_aux:n #1
  {
    \quark_if_recursion_tail_stop_do:nn {#1}
      {
        \bool_if:NT \l__xpinyin_first_bool
          { \tl_set:NV \l_@@_tone_pinyin_tl \l__xpinyin_item_tl }
      }
    \__xpinyin_if_number:nTF {#1}
      {
        \bool_if:NT \l__xpinyin_first_bool
          { \bool_set_false:N \l__xpinyin_first_bool }
        \tl_put_right:NV \l_@@_tone_pinyin_tl \l__xpinyin_pre_tl
        \tl_put_right:Nx \l_@@_tone_pinyin_tl
          { \clist_item:cn { c_@@_tone_ \l__xpinyin_tone_tl _clist } {#1} }
        \tl_put_right:NV \l_@@_tone_pinyin_tl \l__xpinyin_post_tl
        \bool_if:NF \l_@@_autopinyin_bool
          {
            \tl_put_right:Nn \l_@@_tone_pinyin_tl {,}
          }
        \__xpinyin_pinyin_init:
      }
      {
        \int_compare:nNnTF
          { 0 \cs_if_exist_use:c { c__xpinyin_ \tl_to_str:N \l__xpinyin_tone_tl _tl } } >
          { 0 \cs_if_exist_use:c { c__xpinyin_ \tl_to_str:n {#1} _tl } }
          { \tl_put_right:Nn \l__xpinyin_post_tl {#1} }
          {
            \tl_set:Nn \l__xpinyin_tone_tl {#1}
            \tl_set_eq:NN \l__xpinyin_pre_tl \l__xpinyin_item_tl
            \tl_clear:N \l__xpinyin_post_tl
          }
        \tl_put_right:Nx \l__xpinyin_item_tl { \__xpinyin_replace_v:n {#1} }
      }
    \@@_pinyin_aux:n
  }
%    \end{macrocode}
% \end{macro}
%
% \begin{macro}{\@@_get_tone_pinyin:n}
% 根据手动拼音得到拼音
%    \begin{macrocode}
\cs_new:Npn \@@_get_tone_pinyin:n #1
  {
    \tl_clear:N \l_@@_tone_pinyin_tl
    \__xpinyin_pinyin_init:
    \tl_set:Nn \l_@@_save_tl {#1}
    \bool_set_true:N \l__xpinyin_first_bool
    \@@_pinyin_aux:n #1 \q_recursion_tail \q_recursion_stop
  }
\cs_generate_variant:Nn  \@@_get_tone_pinyin:n { V }
%    \end{macrocode}
% \end{macro}
%
% \begin{macro}{\@@_get_hanzi_pinyin:n}
% 自动拼音
%    \begin{macrocode}
\cs_new:Npn \@@_get_hanzi_pinyin:n #1
  {
    \tl_set_eq:Nc \l_tmpa_tl { c__xpinyin_ \__xpinyin_char_to_unicode:n {#1} _tl }
    \exp_args:No \tl_if_head_eq_meaning:nNTF { \l_tmpa_tl } \__xpinyin_pinyin:n
      {
        \exp_args:Nf \@@_get_tone_pinyin:n { \exp_after:wN \use_ii:nn \l_tmpa_tl }
        \tl_set_eq:NN \l_@@_hanzi_pinyin_tl \l_@@_tone_pinyin_tl
      }
      { \tl_set_eq:NN \l_@@_hanzi_pinyin_tl \l_tmpa_tl }
  }
\cs_generate_variant:Nn  \@@_get_hanzi_pinyin:n { V }
%    \end{macrocode}
% \end{macro}
%
% \changes{v2.0.0}{2021/09/24}{添加从xpinyin宏包中提取拼音函数。}
% 代码摘录自LaTeX工作室:基于xpinyin宏包获取汉字的声母,韵母,声调
% (\url{https://www.latexstudio.net/index/details/index/mid/1994.html})
%
% 从需要的声母、韵母、读音表。
%
% 声母表
%    \begin{macrocode}
\clist_set:Nn \l_@@_initials_clist
  {
    {zh} , {ch} , {sh} , {b} , {p} , {m} , {f} ,
    {d}  , {t}  , {l}  , {k} , {h} , {j} , {q} ,
    {x}  , {r}  , {z}  , {c} , {s} , {y} , {w} ,
    {g}  , {n}
  }
%    \end{macrocode}
%
% \changes{v2.0.0}{2021/09/27}{修订部分错误带音调韵母表}
%
% 带声音调韵母表
%    \begin{macrocode}
\clist_set:Nn \l_@@_vowel_tone_clist
  {
    {iāng} , {iáng} , {iǎng} , {iàng} , {iang} ,
    {iōng} , {ióng} , {iǒng} , {iòng} , {iong} ,
    {uāng} , {uáng} , {uǎng} , {uàng} , {uang} ,
    {uēng} , {uéng} , {uěng} , {uèng} , {ueng} ,
    {āng}  , {áng}  , {ǎng}  , {àng}  , {ang}  ,
    {ēng}  , {éng}  , {ěng}  , {èng}  , {eng}  ,
    {īng}  , {íng}  , {ǐng}  , {ìng}  , {ing}  ,
    {ōng}  , {óng}  , {ǒng}  , {òng}  , {ong}  ,
    {uāi}  , {uái}  , {uǎi}  , {uài}  , {uai}  ,
    {uān}  , {uán}  , {uǎn}  , {uàn}  , {uan}  ,
    {uēi}  , {uéi}  , {uěi}  , {uèi}  , {uei}  ,
    {uāo}  , {uáo}  , {uǎo}  , {uào}  , {uao}  ,
    {iōu}  , {ióu}  , {iǒu}  , {iòu}  , {iou}  ,
    {iān}  , {ián}  , {iǎn}  , {iàn}  , {ian}  ,
    {üān}  , {üán}  , {üǎn}  , {üàn}  , {üan}  ,
    {uēn}  , {uén}  , {uěn}  , {uèn}  , {uen}  ,
    {āi}   , {ái}   , {ǎi}   , {ài}   , {ai}   ,
    {ēi}   , {éi}   , {ěi}   , {èi}   , {ei}   ,
    {uā}   , {uá}   , {uǎ}   , {uà}   , {ua}   ,
    {uō}   , {uó}   , {uǒ}   , {uò}   , {uo}   ,
    {uī}   , {uí}   , {uǐ}   , {uì}   , {ui}   ,
    {āo}   , {áo}   , {ǎo}   , {ào}   , {ao}   ,
    {ōu}   , {óu}   , {ǒu}   , {òu}   , {ou}   ,
    {iū}   , {iú}   , {iǔ}   , {iù}   , {iu}   ,
    {iā}   , {iá}   , {iǎ}   , {ià}   , {ia}   ,
    {iē}   , {ié}   , {iě}   , {iè}   , {ie}   ,
    {uē}   , {ué}   , {uě}   , {uè}   , {ue}   ,
    {üē}   , {üé}   , {üě}   , {üè}   , {üe}   ,
    {ēr}   , {ér}   , {ěr}   , {èr}   , {er}   ,
    {ān}   , {án}   , {ǎn}   , {àn}   , {an}   ,
    {ēn}   , {én}   , {ěn}   , {èn}   , {en}   ,
    {īn}   , {ín}   , {ǐn}   , {ìn}   , {in}   ,
    {ūn}   , {ún}   , {ǔn}   , {ùn}   , {un}   ,
    {ǖn}   , {ǘn}   , {ǚn}   , {ǜn}   , {ün}   ,
    {ā}    , {á}    , {ǎ}    , {à}    , {a}    ,
    {ē}    , {é}    , {ě}    , {è}    , {e}    ,
    {ī}    , {í}    , {ǐ}    , {ì}    , {i}    ,
    {ō}    , {ó}    , {ǒ}    , {ò}    , {o}    ,
    {ū}    , {ú}    , {ǔ}    , {ù}    , {u}    ,
    {ǖ}    , {ǘ}    , {ǚ}    , {ǜ}    , {ü}
  }
%    \end{macrocode}
%
% 韵母表
%    \begin{macrocode}
\clist_set:Nn \l_@@_vowel_clist
  {
    {iang} , {iong} , {uang} , {ueng} , {ang} , {eng} , {ing} ,
    {ong}  , {uai}  , {uan}  , {uai}  , {uei} , {iao} , {iou} ,
    {ian}  , {üan}  , {uen}  , {ai}   , {ei}  , {ua}  , {uo}  ,
    {ui}   , {ao}   , {ou}   , {iu}   , {ie}  , {üe}  , {er}  ,
    {an}   , {en}   , {in}   , {un}   , {ün}  , {a}   , {e}   ,
    {i}    , {o}    , {ü}    , {u}
  }
%    \end{macrocode}
%
% 声调表
%    \begin{macrocode}
\clist_set:Nn \l_@@_tone_num_clist
  {
    {ā} {a1} , {á} {a2} , {ǎ} {a3} , {à} {a4} ,
    {ō} {o1} , {ó} {o2} , {ǒ} {o3} , {ò} {o4} ,
    {ē} {e1} , {é} {e2} , {ě} {e3} , {è} {e4} ,
    {ū} {u1} , {ú} {u2} , {ǔ} {u3} , {ù} {u4} ,
    {ḿ} {m2} ,
    {ń} {n2} , {ň} {n3} , {ǹ} {n4} ,
    {ī} {i1} , {í} {i2} , {ǐ} {i3} , {ì} {i4} ,
    {ǖ} {v1} , {ǘ} {v2} , {ǚ} {v3} , {ǜ} {v4}
  }
%    \end{macrocode}
%
% 去声调表
%    \begin{macrocode}
\clist_set:Nn \l_@@_nonetone_clist
  {
    {ā} {a} , {á} {a} , {ǎ} {a} , {à} {a} ,
    {ō} {o} , {ó} {o} , {ǒ} {o} , {ò} {o} ,
    {ē} {e} , {é} {e} , {ě} {e} , {è} {e} ,
    {ū} {u} , {ú} {u} , {ǔ} {u} , {ù} {u} ,
    {ḿ} {m} ,
    {ń} {n} , {ň} {n} , {ǹ} {n} ,
    {ī} {i} , {í} {i} , {ǐ} {i} , {ì} {i} ,
    {ǖ} {ü} , {ǘ} {ü} , {ǚ} {ü} , {ǜ} {ü}
  }
%    \end{macrocode}
%
% \begin{macro}{\@@_split_pinyin_withtone:n}
% 分离拼音中的声母和带声调的韵母。
%    \begin{macrocode}
\cs_new_protected:Npn \@@_split_pinyin_withtone:n #1
  {
    \int_zero:N  \l_@@_tone_int
    \str_clear:N \l_@@_pinyin_str
    \tl_clear:N  \l_@@_pinyin_tl
    \tl_clear:N  \l_@@_initial_tl
    \tl_clear:N  \l_@@_vowel_tl

    \tl_set:Nn \l_@@_pinyin_tl {#1}

    \tl_map_inline:Nn \l_@@_pinyin_tl
      {
        \str_put_right:Nn \l_@@_pinyin_str {##1}
      }

    \clist_map_inline:Nn \l_@@_initials_clist
      {
        \str_if_in:NnT { \l_@@_pinyin_str } {##1}
          {
            \tl_set:Nn \l_@@_initial_tl {##1}
            \clist_map_break:
          }
      }

    \clist_map_inline:Nn \l_@@_vowel_tone_clist
      {
        \str_if_in:NnT { \l_@@_pinyin_str } { ##1 }
          {
            \tl_set:Nn \l_@@_vowel_tl {##1}
            \clist_map_break:
          }
      }
  }
%    \end{macrocode}
% \end{macro}
%
% \begin{macro}{\@@_split_pinyin_withouttone:n}
% 分离拼音中的声母和不带声调的韵母。
%    \begin{macrocode}
\cs_new_protected:Npn \@@_split_pinyin_withouttone:n #1
  {
    \int_zero:N  \l_@@_tone_int
    \str_clear:N \l_@@_pinyin_str
    \tl_clear:N  \l_@@_pinyin_tl
    \tl_clear:N  \l_@@_initial_tl
    \tl_clear:N  \l_@@_vowel_tl

    \tl_set:Nn \l_@@_pinyin_tl {#1}

    \clist_map_inline:Nn \l_@@_nonetone_clist
      {
        \tl_replace_all:Nnn \l_@@_pinyin_tl ##1
      }

    \tl_map_inline:Nn \l_@@_pinyin_tl
      {
        \str_put_right:Nn \l_@@_pinyin_str {##1}
      }

    \clist_map_inline:Nn \l_@@_initials_clist
      {
        \str_if_in:NnT {\l_@@_pinyin_str} {##1}
          {
            \tl_set:Nn \l_@@_initial_tl {##1}
            \clist_map_break:
          }
      }

    \clist_map_inline:Nn \l_@@_vowel_clist
      {
        \str_if_in:NnT { \l_@@_pinyin_str } { ##1 }
          {
            \tl_set:Nn \l_@@_vowel_tl {##1}
            \clist_map_break:
          }
      }
  }
%    \end{macrocode}
% \end{macro}
%
%    \begin{macrocode}
%</package>
%    \end{macrocode}
%
% \end{implementation}
%
% \Finale
%
\endinput
