% ----------------------------------------------------------
% Sample file for LaTeX package `marathi'.
% Copyright © 2020, 2021 निरंजन
% This program is free software: you can redistribute it
% and/or modify it under the terms of the GNU General Public
% License as published by the Free Software Foundation,
% either version 3 of the License, or (at your option) any
% later version.
%
% This program is distributed in the hope that it will be
% useful, but WITHOUT ANY WARRANTY; without even the implied
% warranty of MERCHANTABILITY or FITNESS FOR A PARTICULAR
% PURPOSE. See the GNU General Public License for more
% details.
%
% You should have received a copy of the GNU General Public
% License along with this program. If not, see
% <https://www.gnu.org/licenses/>.
% ----------------------------------------------------------
\title{नमुना}
\author{लेखक}

\maketitle

\begin{abstract}
नमस्कार! हा मजकूर अर्थशून्य आहे. ह्या ठिकाणी काय व कसे छापले जाईल ह्याचा हा केवळ एक नमुना
आहे. जर तुम्ही हे वाचले, तर तुम्हाला कोणतीच माहिती मिळणार नाही. खरेच? ह्यात कोणतीच माहिती
नाही काय? ह्या मजकुरात व `पिढ्ढ करढपाखू' अशा निरर्थक शब्दांमध्ये काही फरक आहे का? हो!
ह्याला आंधळा मजकूर असे म्हणतात. हा मजकूर तुम्हाला निवडलेला टंक कोणता आहे, अक्षरे कशी दिसतात
ह्या सगळ्याबाबत माहिती देतो. ह्यासाठी विशिष्ट शब्दांची गरज नाही, परंतु शब्द वापरल्या गेलेल्या
भाषेशी जुळायला हवेत.
\end{abstract}

\tableofcontents

\section{पहिल्या स्तरावरील शीर्षक (विभाग)}
नमस्कार! हा मजकूर अर्थशून्य आहे. ह्या ठिकाणी काय व कसे छापले जाईल ह्याचा हा केवळ एक नमुना
आहे. जर तुम्ही हे वाचले, तर तुम्हाला कोणतीच माहिती मिळणार नाही. खरेच? ह्यात कोणतीच माहिती
नाही काय? ह्या मजकुरात व `पिढ्ढ करढपाखू' अशा निरर्थक शब्दांमध्ये काही फरक आहे का? हो!
ह्याला आंधळा मजकूर असे म्हणतात. हा मजकूर तुम्हाला निवडलेला टंक कोणता आहे, अक्षरे कशी दिसतात
ह्या सगळ्याबाबत माहिती देतो. ह्यासाठी विशिष्ट शब्दांची गरज नाही, परंतु शब्द वापरल्या गेलेल्या
भाषेशी जुळायला हवेत.

\subsection{दुसऱ्या स्तरावरील शीर्षक (उपविभाग)}
नमस्कार! हा मजकूर अर्थशून्य आहे. ह्या ठिकाणी काय व कसे छापले जाईल ह्याचा हा केवळ एक नमुना
आहे. जर तुम्ही हे वाचले, तर तुम्हाला कोणतीच माहिती मिळणार नाही. खरेच? ह्यात कोणतीच माहिती
नाही काय? ह्या मजकुरात व `पिढ्ढ करढपाखू' अशा निरर्थक शब्दांमध्ये काही फरक आहे का? हो!
ह्याला आंधळा मजकूर असे म्हणतात. हा मजकूर तुम्हाला निवडलेला टंक कोणता आहे, अक्षरे कशी दिसतात
ह्या सगळ्याबाबत माहिती देतो. ह्यासाठी विशिष्ट शब्दांची गरज नाही, परंतु शब्द वापरल्या गेलेल्या
भाषेशी जुळायला हवेत.

\subsubsection{तिसऱ्या स्तरावरील शीर्षक (उपउपविभाग)}
नमस्कार! हा मजकूर अर्थशून्य आहे. ह्या ठिकाणी काय व कसे छापले जाईल ह्याचा हा केवळ एक नमुना
आहे. जर तुम्ही हे वाचले, तर तुम्हाला कोणतीच माहिती मिळणार नाही. खरेच? ह्यात कोणतीच माहिती
नाही काय? ह्या मजकुरात व `पिढ्ढ करढपाखू' अशा निरर्थक शब्दांमध्ये काही फरक आहे का? हो!
ह्याला आंधळा मजकूर असे म्हणतात. हा मजकूर तुम्हाला निवडलेला टंक कोणता आहे, अक्षरे कशी दिसतात
ह्या सगळ्याबाबत माहिती देतो. ह्यासाठी विशिष्ट शब्दांची गरज नाही, परंतु शब्द वापरल्या गेलेल्या
भाषेशी जुळायला हवेत.

\paragraph{चौथ्या स्तरावरील शीर्षक (परिच्छेद)}
नमस्कार! हा मजकूर अर्थशून्य आहे. ह्या ठिकाणी काय व कसे छापले जाईल ह्याचा हा केवळ एक नमुना
आहे. जर तुम्ही हे वाचले, तर तुम्हाला कोणतीच माहिती मिळणार नाही. खरेच? ह्यात कोणतीच माहिती
नाही काय? ह्या मजकुरात व `पिढ्ढ करढपाखू' अशा निरर्थक शब्दांमध्ये काही फरक आहे का? हो!
ह्याला आंधळा मजकूर असे म्हणतात. हा मजकूर तुम्हाला निवडलेला टंक कोणता आहे, अक्षरे कशी दिसतात
ह्या सगळ्याबाबत माहिती देतो. ह्यासाठी विशिष्ट शब्दांची गरज नाही, परंतु शब्द वापरल्या गेलेल्या
भाषेशी जुळायला हवेत.

\section{याद्या}

\subsection{बिंदुक्रमित यादीचे उदाहरण}

\begin{itemize}
\item पहिला मुद्दा
\item दुसरा मुद्दा
\item तिसरा मुद्दा
\item चौथा मुद्दा
\item पाचवा मुद्दा
\end{itemize}

\subsection*{बिंदुक्रमित यादीचे दुसरे उदाहरण}

\begin{itemize}
\item पहिला मुद्दा
  \begin{itemize}
  \item पहिला मुद्दा
    \begin{itemize}
    \item पहिला मुद्दा
      \begin{itemize}
      \item पहिला मुद्दा
      \item दुसरा मुद्दा
      \end{itemize}
    \item दुसरा मुद्दा
    \end{itemize}
  \item दुसरा मुद्दा
  \end{itemize}
\item दुसरा मुद्दा
\end{itemize}

\subsection{अनुक्रमित यादीचे उदाहरण}

\begin{enumerate}
\item पहिला मुद्दा
\item दुसरा मुद्दा
\item तिसरा मुद्दा
\item चौथा मुद्दा
\item पाचवा मुद्दा
\end{enumerate}

\subsection*{अनुक्रमित यादीचे दुसरे उदाहरण}

\begin{enumerate}
\item पहिला मुद्दा
  \begin{enumerate}
  \item पहिला मुद्दा
    \begin{enumerate}
    \item पहिला मुद्दा
      \begin{enumerate}
      \item पहिला मुद्दा
      \item दुसरा मुद्दा
      \end{enumerate}
    \item दुसरा मुद्दा
    \end{enumerate}
  \item दुसरा मुद्दा
  \end{enumerate}
\item दुसरा मुद्दा
\end{enumerate}

\subsection{वर्णनक्रमित यादीचे उदाहरण}

\begin{description}
\item[पहिला] मुद्दा
\item[दुसरा] मुद्दा
\item[तिसरा] मुद्दा
\item[चौथा] मुद्दा
\item[पाचवा] मुद्दा
\end{description}

\subsection*{वर्णनक्रमित यादीचे दुसरे उदाहरण}

\begin{description}
\item[पहिला] मुद्दा
  \begin{description}
  \item[पहिला] मुद्दा
    \begin{description}
    \item[पहिला] मुद्दा
      \begin{description}
      \item[पहिला] मुद्दा
      \item[दुसरा] मुद्दा
      \end{description}
    \item[दुसरा] मुद्दा
    \end{description}
  \item[दुसरा] मुद्दा
  \end{description}
\item[दुसरा] मुद्दा
\end{description}
