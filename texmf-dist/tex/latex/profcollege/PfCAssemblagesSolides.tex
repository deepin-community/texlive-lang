%%%
% Assemblages solides
%%%
\def\filedateAssemblageS{2024/08/04}%
\def\fileversionAssemblageS{0.1}%
\message{-- \filedateAssemblageS\space v\fileversionAssemblageS}%
%
\setKVdefault[Assemblage]{Phi=30,Theta=20,Distance=50,Anglez=0,Anglex=0,Largeur=1.5,Profondeur=0.75,Hauteur=1,RayonBoule=1,RayonCone=1,RayonCylindre=1,HauteurPrisme=1,HauteurPyramideA=1,HauteurPyramideB=1,HauteurConeA=1,HauteurConeB=1,HauteurCylindre=1,Cotes=3,Trace=false,Type=Pyramides}%
\defKV[Assemblage]{Traces=\setKV[Assemblage]{Trace}}%

\NewDocumentCommand\AssemblageSolides{o}{%
  \useKVdefault[Assemblage]%
  \setKV[Assemblage]{#1}%
  \IfStrEqCase{\useKV[Assemblage]{Type}}{%
    {Pyramides}{\MPAssemblagePyramides{\useKV[Assemblage]{Cotes}}}%
    {PavePyramide}{\MPAssemblagePavePyramide}%
    {PyramidePave}{\MPAssemblagePavePyramide}%
    {PavePyramides}{\MPAssemblagePavePyramides}%
    {PyramidesPave}{\MPAssemblagePavePyramides}%
    {PrismePyramide}{\MPAssemblagePrismePyramide{\useKV[Assemblage]{Cotes}}}%
    {PyramidePrisme}{\MPAssemblagePrismePyramide{\useKV[Assemblage]{Cotes}}}%
    {CylindreCone}{\MPAssemblageCylindreCone}%
    {ConeCylindre}{\MPAssemblageCylindreCone}%
    {CylindreCones}{\MPAssemblageCylindreCones}%
    {ConesCylindre}{\MPAssemblageCylindreCones}%
    {CylindreConeBoule}{\MPAssemblageCylindreConeBoule}%
    {CylindreBouleCone}{\MPAssemblageCylindreConeBoule}%
    {ConeCylindreBoule}{\MPAssemblageCylindreConeBoule}%
    {ConeBouleCylindre}{\MPAssemblageCylindreConeBoule}%
    {BouleCylindreCone}{\MPAssemblageCylindreConeBoule}%
    {BouleConeCylindre}{\MPAssemblageCylindreConeBoule}%
    {CylindreBoule}{\MPAssemblageCylindreBoule}%
    {BouleCylindre}{\MPAssemblageCylindreBoule}%
    {CylindreBoules}{\MPAssemblageCylindreBoules}%
    {BoulesCylindre}{\MPAssemblageCylindreBoules}%
    {ConeBoule}{\MPAssemblageConeBoule}%
    {BouleCone}{\MPAssemblageConeBoule}%
  }%
}%

\NewDocumentCommand\MPAssemblagePyramides{m}{%
  \ifluatex
    \mplibforcehmode
    \begin{mplibcode}
      input PfCSolid;

      Initialisation(1500,\useKV[Assemblage]{Phi},\useKV[Assemblage]{Theta},\useKV[Assemblage]{Distance});
      angx:=\useKV[Assemblage]{Anglex};
      angz:=\useKV[Assemblage]{Anglez};

      HautA=\useKV[Assemblage]{HauteurPyramideA};
      HautB=\useKV[Assemblage]{HauteurPyramideB};

      boolean Trace;
      Trace=\useKV[Assemblage]{Trace};
      
      color Sommet[];
      
      NbS:=#1+2;
      angbase=360/#1;
      
      for k=0 upto #1-1:
      Sommet[k]:=(cosd(k*angbase),sind(k*angbase),0);
      endfor;
      Sommet[NbS-1]:=(0,0,HautA);
      Sommet[NbS]:=(0,0,-HautB);
      for k=0 upto NbS:
      Sommet[k]:=Image(Sommet[k]);
      endfor;
      %% Faces
      NF:=2*#1;
      for k=1 upto #1:
      Fc[k*100]:=3;Fc[k*100+1]:=(k+1) mod #1;Fc[k*100+3]:=k mod #1;Fc[k*100+2]:=NbS-1;
      endfor;
      for k=#1+1 upto 2*#1:
      Fc[k*100]:=3;Fc[k*100+1]:=k mod #1;Fc[k*100+3]:=(k+1) mod #1;Fc[k*100+2]:=NbS;
      endfor;
      %
      DessineObjet;
      %
      if Trace:
      \useKV[Assemblage]{Traces};
      fi;
    \end{mplibcode}
  \fi
}%

\NewDocumentCommand\MPAssemblagePavePyramide{}{%
  \ifluatex
    \mplibforcehmode
    \begin{mplibcode}
      input PfCSolid;

      Initialisation(1500,\useKV[Assemblage]{Phi},\useKV[Assemblage]{Theta},\useKV[Assemblage]{Distance});
      angx:=\useKV[Assemblage]{Anglex};
      angz:=\useKV[Assemblage]{Anglez};

      Larg=\useKV[Assemblage]{Hauteur};
      Long=\useKV[Assemblage]{Largeur};
      Prof=\useKV[Assemblage]{Profondeur};
      HautA=\useKV[Assemblage]{HauteurPyramideA};
      
      boolean Trace;
      Trace=\useKV[Assemblage]{Trace};
      
      color Sommet[];
      NbS:=9;

      Sommet1:=(Long,0,0);
      Sommet2:=(Long,Prof,0);
      Sommet3:=(0,Prof,0);
      Sommet4:=(0,0,0);
      Sommet5:=(0,0,Larg);
      Sommet6:=(Long,0,Larg);
      Sommet7:=(Long,Prof,Larg);
      Sommet8:=(0,Prof,Larg);
      Sommet9:=(Sommet5+Sommet6+Sommet7+Sommet8)/4+(0,0,HautA);
      for k=1 upto 9:
      Sommet[k]:=Image(Sommet[k]);
      endfor;
      %% Faces
      NF:=9;
      Fc[100]:=4;Fc[101]:=1;Fc[102]:=4;Fc[103]:=3;Fc[104]:=2;
      Fc[200]:=4;Fc[201]:=4;Fc[202]:=5;Fc[203]:=8;Fc[204]:=3;
      Fc[300]:=4;Fc[301]:=1;Fc[302]:=6;Fc[303]:=5;Fc[304]:=4;
      Fc[400]:=4;Fc[401]:=2;Fc[402]:=3;Fc[403]:=8;Fc[404]:=7;
      Fc[500]:=4;Fc[501]:=1;Fc[502]:=2;Fc[503]:=7;Fc[504]:=6;
      Fc[600]:=3;Fc[601]:=8;Fc[602]:=5;Fc[603]:=9;
      Fc[700]:=3;Fc[701]:=5;Fc[702]:=6;Fc[703]:=9;
      Fc[800]:=3;Fc[801]:=6;Fc[802]:=7;Fc[803]:=9;
      Fc[900]:=3;Fc[901]:=7;Fc[902]:=8;Fc[903]:=9;
      %
      DessineObjet;
      %
      %
      if Trace:
      \useKV[Assemblage]{Traces};
      fi;
    \end{mplibcode}
  \fi
}%

\NewDocumentCommand\MPAssemblagePavePyramides{}{%
  \ifluatex
    \mplibforcehmode
    \begin{mplibcode}
      input PfCSolid;

      Initialisation(1500,\useKV[Assemblage]{Phi},\useKV[Assemblage]{Theta},\useKV[Assemblage]{Distance});
      angx:=\useKV[Assemblage]{Anglex};
      angz:=\useKV[Assemblage]{Anglez};

      Larg=\useKV[Assemblage]{Hauteur};
      Long=\useKV[Assemblage]{Largeur};
      Prof=\useKV[Assemblage]{Profondeur};
      HautA=\useKV[Assemblage]{HauteurPyramideA};
      HautB=\useKV[Assemblage]{HauteurPyramideB};
      
      boolean Trace;
      Trace=\useKV[Assemblage]{Trace};
      
      color Sommet[];
      NbS:=10;
      
      Sommet1:=(Long,0,0);
      Sommet2:=(Long,Prof,0);
      Sommet3:=(0,Prof,0);
      Sommet4:=(0,0,0);
      Sommet5:=(0,0,Larg);
      Sommet6:=(Long,0,Larg);
      Sommet7:=(Long,Prof,Larg);
      Sommet8:=(0,Prof,Larg);
      Sommet9:=(Sommet5+Sommet6+Sommet7+Sommet8)/4+(0,0,HautA);
      Sommet10:=(Sommet1+Sommet2+Sommet3+Sommet4)/4+(0,0,-HautB);
      for k=1 upto 10:
      Sommet[k]:=Image(Sommet[k]);
      endfor;
      %% Faces
      NF:=12;
      Fc[100]:=4;Fc[101]:=4;Fc[102]:=5;Fc[103]:=8;Fc[104]:=3;
      Fc[200]:=4;Fc[201]:=1;Fc[202]:=6;Fc[203]:=5;Fc[204]:=4;
      Fc[300]:=4;Fc[301]:=2;Fc[302]:=3;Fc[303]:=8;Fc[304]:=7;
      Fc[400]:=4;Fc[401]:=1;Fc[402]:=2;Fc[403]:=7;Fc[404]:=6;
      %
      Fc[500]:=3;Fc[501]:=8;Fc[502]:=5;Fc[503]:=9;
      Fc[600]:=3;Fc[601]:=5;Fc[602]:=6;Fc[603]:=9;
      Fc[700]:=3;Fc[701]:=6;Fc[702]:=7;Fc[703]:=9;
      Fc[800]:=3;Fc[801]:=7;Fc[802]:=8;Fc[803]:=9;
      %
      Fc[900]:=3;Fc[901]:=4;Fc[902]:=3;Fc[903]:=10;
      Fc[1000]:=3;Fc[1001]:=1;Fc[1002]:=4;Fc[1003]:=10;
      Fc[1100]:=3;Fc[1101]:=2;Fc[1102]:=1;Fc[1103]:=10;
      Fc[1200]:=3;Fc[1201]:=3;Fc[1202]:=2;Fc[1203]:=10;
      %
      DessineObjet;
      %
      if Trace:
      \useKV[Assemblage]{Traces};
      fi;
    \end{mplibcode}
  \fi
}%

\NewDocumentCommand\MPAssemblagePrismePyramide{m}{%
  \ifluatex
    \mplibforcehmode
    \begin{mplibcode}
      input PfCSolid;

      Initialisation(1500,\useKV[Assemblage]{Phi},\useKV[Assemblage]{Theta},\useKV[Assemblage]{Distance});
      angx:=\useKV[Assemblage]{Anglex};
      angz:=\useKV[Assemblage]{Anglez};

      HautP=\useKV[Assemblage]{HauteurPrisme};
      HautA=\useKV[Assemblage]{HauteurPyramideA};
      
      boolean Trace;
      Trace=\useKV[Assemblage]{Trace};
      
      color Sommet[];
      
      NbS:=2*#1+1;
      angbase=360/#1;
      
      for k=0 upto #1-1:
      Sommet[k]:=(cosd(k*angbase),sind(k*angbase),0);
      endfor;

      for k=#1 upto 2*#1-1:
      Sommet[k]:=Sommet[k-#1]+(0,0,-HautP);
      endfor;
      Sommet[NbS-1]:=(0,0,HautA);
      for k=0 upto NbS-1:
      Sommet[k]:=Image(Sommet[k]);
      endfor;
      %% Faces
      NF:=2*#1+1;
      %latérale pyramide
      for k=1 upto #1:
      Fc[k*100]:=3;Fc[k*100+1]:=k mod #1;Fc[k*100+2]:=(k+1) mod #1;Fc[k*100+3]:=NbS-1;
      endfor;
      % face latérales prisme
      for k=1 upto #1:
      rang:=k+#1;
      Fc[rang*100]:=4;Fc[rang*100+1]:=(k mod #1)+#1;Fc[rang*100+2]:=((k+1) mod #1)+#1;Fc[rang*100+3]:=(k+1) mod #1;Fc[rang*100+4]:=(k) mod #1;
      endfor;
      %face base du prisme
      Fc[(2*#1+1)*100]:=#1;
      for k=1 upto #1:
      Fc[(2*#1+1)*100+k]:=((#1-k) mod #1)+#1;
      endfor;
      %
      DessineObjet;
      % 
      if Trace:
      \useKV[Assemblage]{Traces};
      fi;
    \end{mplibcode}
  \fi
}%

\NewDocumentCommand\MPAssemblageCylindreCone{}{%
  \ifluatex
    \mplibforcehmode
    \begin{mplibcode}
      input PfCSolid;

      Initialisation(1500,\useKV[Assemblage]{Phi},\useKV[Assemblage]{Theta},\useKV[Assemblage]{Distance});

      angx:=\useKV[Assemblage]{Anglex};
      angz:=\useKV[Assemblage]{Anglez};

      Rayon=\useKV[Assemblage]{RayonCylindre};
      HautC=\useKV[Assemblage]{HauteurCylindre};
      HautCo=\useKV[Assemblage]{HauteurConeA};
      
      boolean Trace;
      Trace=\useKV[Assemblage]{Trace};
      
      nb:=36;
      subh:=1;
      incolor:=white;
      outcolor:=0.8white;
      Ferme1:=true;
      Objetcylindre1("r="&"Rayon","h="&"HautC");
      TR:=(0,-HautC*sind(angx),HautC*cosd(angx));
      nb:=72;
      Objetcone2("r="&"Rayon","h="&"HautCo");
      AffichageObjetCylindre1;
      AffichageObjetCone2;
      if Trace:
      \useKV[Assemblage]{Traces};
      fi;
    \end{mplibcode}
  \fi
}%

\NewDocumentCommand\MPAssemblageCylindreCones{}{%
  % à faire
  \ifluatex
    \mplibforcehmode
    \begin{mplibcode}
      input PfCSolid;

      Initialisation(1500,\useKV[Assemblage]{Phi},\useKV[Assemblage]{Theta},\useKV[Assemblage]{Distance});

      angx:=\useKV[Assemblage]{Anglex};
      angz:=\useKV[Assemblage]{Anglez};

      Rayon=\useKV[Assemblage]{RayonCylindre};
      HautC=\useKV[Assemblage]{HauteurCylindre};
      HautCo=\useKV[Assemblage]{HauteurConeA};
      HautCob=\useKV[Assemblage]{HauteurConeB};

      boolean Trace;
      Trace=\useKV[Assemblage]{Trace};
      
      nb:=36;
      subh:=1;
      incolor:=white;
      outcolor:=0.8white;
      Ferme1:=true;
      Objetcylindre1("r="&"Rayon","h="&"HautC");
      nb:=72;
      TR:=(0,-HautC*sind(angx),HautC*cosd(angx));
      Objetcone2("r="&"Rayon","h="&"HautCo");
      TR:=(0,0,0);
      angx:=180+angx;
      Objetcone3("r="&"Rayon","h="&"HautCob");
      angx:=0;
      AffichageObjetCylindreSansBase1;
      AffichageObjetCone2;
      AffichageObjetCone3;
      if Trace:
      \useKV[Assemblage]{Traces};
      fi;
    \end{mplibcode}
  \fi
}%

\NewDocumentCommand\MPAssemblageCylindreConeBoule{}{%
  % à faire
  \ifluatex
    \mplibforcehmode
    \begin{mplibcode}
      input PfCSolid;

      Initialisation(1500,\useKV[Assemblage]{Phi},\useKV[Assemblage]{Theta},\useKV[Assemblage]{Distance});

      angx:=\useKV[Assemblage]{Anglex};
      angz:=\useKV[Assemblage]{Anglez};

      Rayon=\useKV[Assemblage]{RayonCylindre};
      HautC=\useKV[Assemblage]{HauteurCylindre};
      HautCo=\useKV[Assemblage]{HauteurConeA};

      boolean Trace;
      Trace=\useKV[Assemblage]{Trace};
      
      nb:=36;%36
      subh:=1;%
      incolor:=white;
      outcolor:=0.8white;
      Ferme1:=true;
      Objetcylindre1("r="&"Rayon","h="&"HautC");
      TR:=(0,-HautC*sind(angx),HautC*cosd(angx));
      Objetcone2("r="&"Rayon","h="&"HautCo");
      TR:=(0,0,0);
      nb:=12;
      subh:=72;
      angx:=180+angx;
      Objetcalotte3("R="&"Rayon","phib="&"-pi/36","phih="&"pi/2");
      angx:=0;
      AffichageObjetCylindreSansBase1;
      AffichageObjetCone2;
      AffichageObjetCalotte3;
      %
      if Trace:
      \useKV[Assemblage]{Traces};
      fi;
    \end{mplibcode}
  \fi
}%

\NewDocumentCommand\MPAssemblageCylindreBoule{}{%
  % à faire
  \ifluatex
    \mplibforcehmode
    \begin{mplibcode}
      input PfCSolid;

      Initialisation(1500,\useKV[Assemblage]{Phi},\useKV[Assemblage]{Theta},\useKV[Assemblage]{Distance});

      angx:=\useKV[Assemblage]{Anglex};
      angz:=\useKV[Assemblage]{Anglez};

      Rayon=\useKV[Assemblage]{RayonCylindre};
      HautC=\useKV[Assemblage]{HauteurCylindre};

      boolean Trace;
      Trace=\useKV[Assemblage]{Trace};
      
      nb:=36;%36
      subh:=1;%
      incolor:=white;
      outcolor:=0.8white;
      Ferme1:=true;
      Objetcylindre1("r="&"Rayon","h="&"HautC");
      TR:=(0,-HautC*sind(angx),HautC*cosd(angx));
      nb:=12;
      subh:=72;%
      Objetcalotte2("R="&"Rayon","phib="&"-pi/36","phih="&"pi/2");
      TR:=(0,0,0);
      AffichageObjetCylindreSansBase1;
      AffichageObjetCalotte2;
      % 
      if Trace:
      \useKV[Assemblage]{Traces};
      fi;
    \end{mplibcode}
  \fi
}%

\NewDocumentCommand\MPAssemblageCylindreBoules{}{%
  % à faire
  \ifluatex
    \mplibforcehmode
    \begin{mplibcode}
      input PfCSolid;

      Initialisation(1500,\useKV[Assemblage]{Phi},\useKV[Assemblage]{Theta},\useKV[Assemblage]{Distance});
      
      angx:=\useKV[Assemblage]{Anglex};
      angz:=\useKV[Assemblage]{Anglez};
      
      Rayon=\useKV[Assemblage]{RayonCylindre};
      HautC=\useKV[Assemblage]{HauteurCylindre};
      
      boolean Trace;
      Trace=\useKV[Assemblage]{Trace};
      
      nb:=36;
      subh:=1;
      incolor:=white;
      outcolor:=0.8white;
      Ferme1:=true;
      Objetcylindre1("r="&"Rayon","h="&"HautC");
      TR:=(0,-HautC*sind(angx),HautC*cosd(angx));
      nb:=12;
      subh:=72;%
      Objetcalotte2("R="&"Rayon","phib="&"0","phih="&"pi/2");
      TR:=(0,0,0);
      angx:=180+angx;
      TR:=(0,0,0);
      Objetcalotte3("R="&"Rayon","phib="&"0","phih="&"pi/2");
      TR:=(0,0,0);
      AffichageObjetCylindreSansBase1;
      AffichageObjetCalotte2;
      AffichageObjetCalotte3;
      % 
      if Trace:
      \useKV[Assemblage]{Traces};
      fi;
    \end{mplibcode}
  \fi
}%

\NewDocumentCommand\MPAssemblageConeBoule{}{%
  % à faire
  \ifluatex
    \mplibforcehmode
    \begin{mplibcode}
      input PfCSolid;

      Initialisation(1500,\useKV[Assemblage]{Phi},\useKV[Assemblage]{Theta},\useKV[Assemblage]{Distance});
      
      angx:=\useKV[Assemblage]{Anglex};
      angz:=\useKV[Assemblage]{Anglez};
      
      Rayon=\useKV[Assemblage]{RayonCone};
      HautCa=\useKV[Assemblage]{HauteurConeA};
      
      boolean Trace;
      Trace=\useKV[Assemblage]{Trace};
      
      nb:=72;%36
      subh:=1;%
      incolor:=white;
      outcolor:=0.8white;
      Ferme1:=true;
      Objetcone1("r="&"Rayon","h="&"HautCa");
      angx:=180+angx;
      nb:=12;
      subh:=72;%
      Objetcalotte2("R="&"Rayon","phib="&"-pi/144","phih="&"pi/2");
      TR:=(0,0,0);
      AffichageObjetConeAvecBase1;
      AffichageObjetCalotte2;%Cone3;
      % 
      if Trace:
      \useKV[Assemblage]{Traces};
      fi;
    \end{mplibcode}
  \fi
}%