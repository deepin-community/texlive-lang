%%%
% CAN
%
\def\filedateCAN{2024/08/04}%
\def\fileversionCAN{0.1}%
\message{-- \filedateCAN\space v\fileversionCAN}%
%
% Gestion de l'aléatoire (pour la CAN)
\newcommand\ChoixAlea[4][]{%
  \ifx\bla#1\bla%
  \xdef#4{\fpeval{randint(#2,#3)}}%
  \else%
  \xdef#4{\fpeval{round(randint(#2,#3)+rand(),#1)}}%
  \fi%
}%

\def\VariableAlea#1#2{%
  \xdef#1{\fpeval{#2}}%
}%

% On bloque la graine de fabrication des nombres aléatoires.
\ExplSyntaxOn
\cs_new_eq:NN \PfCGraineAlea \sys_gset_rand_seed:n
\ExplSyntaxOff

%%%
% Course aux nombres
%%%

% https://tex.stackexchange.com/questions/642775/when-displaying-a-datatool-database-with-dtldisplaylongdb-how-to-obtain-a-hor
% Patch éventuel à ajouter
%\makeatletter
%\expandafter\patchcmd\expandafter{\csname\string\DTLdisplaylongdb\endcsname}%
%         {\@dtl@resetdoamp\dtldisplaystarttab}%
%         {\dtldisplaystarttab\@dtl@resetdoamp}%
%         {\message{Patching succeeded.}}%
%         {\message{Patching failed.}}%
%\makeatother

\setKVdefault[ClesMathAlea]{NbQ=5,ChoixTables=2-9,ValeurMax=20,Classique,Trou=false,Melange=false,Multiple=2,Theme=false}%
\defKV[ClesMathAlea]{Trous=\setKV[ClesMathAlea]{Classique=false}\setKV[ClesMathAlea]{Trou}}%
\defKV[ClesMathAlea]{Melanges=\setKV[ClesMathAlea]{Classique=false}\setKV[ClesMathAlea]{Melange}}%

\newcommand\MathAlea[2][]{%
  \useKVdefault[ClesMathAlea]%
  \setKV[ClesMathAlea]{#1}%
  \input{#2}%
}%

\newcommand\CourseNombreTotalQuestions[1]{%
  \setKV[ClesCN]{TotalQ=#1}%
  \xdef\CNfoo{}%
  \xintFor* ##1 in{\xintSeq{1}{#1}}\do{%
    \xdef\CNfoo{\CNfoo ##1,}%
  }%
}%

\setKVdefault[ClesCN]{%
  Stretch=2.5,%Elasticité du tableau
  NbQ=5,%Nb de questions à poser
  RAZ=false,%Pour remettre à zéro le compteur des questions
  Ordre=false,%Pour mettre toutes les questions dans l'ordre des numéros
  Exercice=false,%Pour utiliser l'aléatoire des questions mais sans habillage particulier, autre que des \item
  Perso=false,%Pour utiliser un choix personnel des questions
  Lecture=false,%Pour différencier avec une lecture automatisée sur les numéros de fichiers
  Maitre=false,%Pour indiquer qu'on prend en compte ou pas les sous-dossiers
  Nom=false,%Pour afficher le nom des fichiers utilisés.
  Theme=false,%Pour afficher le theme travaillé dans les fichiers utilisés (à retravailler).
  AMC=false,%
  Multi=false,%
  CAN=false,%Pour compléter la case réponse.
  CoefQ=0.4,%Coefficient multiplication de \linewidth pour la colonne question
  CoefR=0.35,%Coefficient multiplication de \linewidth pour la colonne réponse
  CoefJ=0.15,%Coefficient multiplication de \linewidth pour la colonne jury
}%
\defKV[ClesCN]{Liste=\setKV[ClesCN]{Perso=true}\setKV[ClesCN]{Lecture}}%
\defKV[ClesCN]{Dossier=\setKV[ClesCN]{Lecture}}%
\defKV[ClesCN]{Debut=\setKV[ClesCN]{RAZ}}%
\setsepchar[*]{,*/}%
\readlist*\ListeMotsCAN{La moitié de /2,Le double de /1,Le triple de /1, Le tiers de /3,Le nombre dix fois plus grand que /1,Le nombre cent fois plus grand que /1,Le nombre mille fois plus grand que /1,Le nombre dix fois plus petit que /10,Le nombre cent fois plus petit que /100,Le nombre mille fois plus petit que /1000}%
\readlist*\ListeMulAstucieuxCAN{50/2,25/4,20/5}%
\readlist*\ListeLieuxObjetsCAN{boulangerie/pains au chocolat/un pain au chocolat,boulangerie/cookies/un cookie,boulangerie/brioches/une brioche,piscine/entrées/une entrée,boucherie/saucisses/une saucisse,boucherie/cuisses de poulet/une cuisse de poulet}%
\readlist*\ListeNomsCAN{Aude/Elle/d'Aude,Bernard/Il/de Bernard,Céline/Elle/de Céline,Daniel/Il/de Daniel,\'Elise/Elle/d'\'Elise,Fabien/Il/de Fabien,Gérard/Il/de Gérard,Hélène/Elle/d'Hélène,Ilies/Il/d'Ilies,Jasmine/Elle/de Jasmine,Kylian/Il/de Kylian,Laurent/Il/de Laurent,Mathilde/Elle/de Mathilde,Nina/Elle/de Nina,Octave/Il/d'Octave,Philippe/Il/de Philippe,Joachim/Il/de Joachim,Thérèse/Elle/de Thérèse,Nawel/Elle/de Nawel,Alexandre/Il/d'Alexandre,Maxence/Il/de Maxence,Sophie/Elle/de Sophie,Christophe/Il/de Christophe,Myriam/Elle/de Myriam,Nathalie/Elle/de Nathalie}%
\readlist*\ListeComposantStatCAN{voitures/un garage/Cross-over/Utilitaires/Berlines,fruits/une corbeille/Pommes/Oranges/Bananes,vêtements/une armoire/Pulls/T-shirts/Chemises,couverts/un tiroir/Fourchettes/Couteaux/Cuillères}%
\readlist*\ListeObjetsSymbolesCAN{fleurs/96,ciseaux/36,crayons/47,enveloppes/41}%uniquement avec pifont
\setsepchar{,}%
\readlist*\ListeObjetsCAN{bonbons,billes,jouets,fruits,biscuits,gâteaux,pommes,poires,pains au chocolat,cookies,croissants,muffins,brioches,saucisses,cuisses de poulet}%
\readlist*\ListePetitsObjetsCAN{bonbons,billes,biscuits,gommes,clous,vis}
\readlist*\ListeFruitsCAN{pommes,poires,abricots,cerises,fraises,framboises,noix,pêches,nectarines,myrtilles,oranges,mandarines}
\readlist*\ListeSommetsCAN{A,B,C,D,E,F,G,H,I,J,K,L,M,N,O,P,Q,R,S,T,U,V,W,X,Y,Z}
\newcommand\ExtraitPetitsObjets[1]{%
  \xdef\CANSGFoo{}%
  \foreachitem\compteur\in\ListePetitsObjetsCAN{%
    \xdef\CANSGFoo{\CANSGFoo \ListePetitsObjetsCAN[\compteurcnt],}%
  }%
  \MelangeListe{\CANSGFoo}{#1}%
  \readlist*\PetitsObjetsMelanges{\faa}%
}%

\newcommand\ExtraitObjets[1]{%
  \xdef\CANSGFoo{}%
  \foreachitem\compteur\in\ListeObjetsCAN{%
    \xdef\CANSGFoo{\CANSGFoo \ListeObjetsCAN[\compteurcnt],}%
  }%
  \MelangeListe{\CANSGFoo}{#1}%
  \readlist*\ObjetsMelanges{\faa}%
}%

\newcommand\ExtraitSymboles[1]{%
  \xdef\CANSGFoo{}%
  \foreachitem\compteur\in\ListeObjetsSymbolesCAN{%
    \xdef\CANSGFoo{\CANSGFoo \ListeObjetsSymbolesCAN[\compteurcnt],}%
  }%
  \MelangeListe{\CANSGFoo}{#1}%
  \setsepchar[*]{,*/}%
  \readlist*\SymbolesMelanges{\faa}%
  \setsepchar{,}%
}%

\newcommand\ExtraitSommet[1]{%
  \xdef\CANSGFoo{}%
  \foreachitem\compteur\in\ListeSommetsCAN{%
    \xdef\CANSGFoo{\CANSGFoo \ListeSommetsCAN[\compteurcnt],}%
  }%
  \MelangeListe{\CANSGFoo}{#1}%
  \readlist*\SommetsMelanges{\faa}%
  \readlist*\PtAlea{\faa}%
}%

\newcommand\ExtraitFruit[1]{%
  \xdef\CANSGFoo{}%
  \foreachitem\compteur\in\ListeFruitsCAN{%
    \xdef\CANSGFoo{\CANSGFoo \ListeFruitsCAN[\compteurcnt],}%
  }%
  \MelangeListe{\CANSGFoo}{#1}%
  \readlist*\FruitsMelanges{\faa}%
}%

\newcommand\ExtraitNom[1]{%
  \xdef\CANSGFoo{}%
  \foreachitem\compteur\in\ListeNomsCAN{%
    \xdef\CANSGFoo{\CANSGFoo \ListeNomsCAN[\compteurcnt],}%
  }%
  \MelangeListe{\CANSGFoo}{#1}%
  \setsepchar[*]{,*/}%
  \readlist*\NomsMelanges{\faa}%
  \setsepchar{,}%
}%

\newcommand\ExtraitElements[1]{%
  \xdef\CANSGFoo{}%
  \foreachitem\compteur\in\ListeComposantStatCAN{%
    \xdef\CANSGFoo{\CANSGFoo \ListeComposantStatCAN[\compteurcnt],}%
  }%
  \MelangeListe{\CANSGFoo}{#1}%
  \setsepchar[*]{,*/}%
  \readlist*\ElementsMelanges{\faa}%
  \setsepchar{,}%
}%

\newcounter{CNNumQ}%
\setcounter{CNNumQ}{1}%

\newcommand\CANNew{%
  \textcolor{Crimson}{\faCopy}%
}%

\newread\RecupNbFichiers
\newread\RecupSSDossiers
\newread\zzz
\def\zzpar{\par}%

\makeatletter
\def\app@exe{\write18}
% Recuperer les sous-dossiers
\def\RecupererSousDossiers#1{%#1 est le dossier contenant les fichiers can classés en sous-dossier
  \app@exe{rm -f RecapSSDossier.tmp ; ls -1 -I "*.tex" #1 > RecapSSDossier.tmp}%
  \immediate\openin\zzz=RecapSSDossier.tmp
  \loop%
  \read\zzz to \tmp%
  \ifx\tmp\zzpar\else%
  \xdef\CNFOOSSDossiers{\CNFOOSSDossiers \expandafter\detokenize\expandafter{\tmp},}%
  \fi%
  \ifeof\zzz\else%
  \repeat%
  \setsepchar{,}\ignoreemptyitems
  \readlist*\ListeSSDossiers{\CNFOOSSDossiers}
}

\def\RecupererFichierTeXSSDossier#1#2{%
  \openin\zzz="|ls #1/#2/*.tex"%
  \loop%
  \read\zzz to \tmp%
  \ifx\tmp\zzpar\else%
  \xdef\CNFOO{\CNFOO \expandafter\detokenize\expandafter{\tmp},}%
  \fi%
  \ifeof\zzz\else%
  \repeat%
}

\def\RecupererFichierTeXDossier#1{%
  \xdef\CNFOO{}%
  \openin\zzz="|ls #1/*.tex"
  \loop%
  \read\zzz to \tmp%
  \ifx\tmp\zzpar\else%
  \xdef\CNFOO{\CNFOO \expandafter\detokenize\expandafter{\tmp},}%
  \fi%
  \ifeof\zzz\else%
  \repeat%
}

\def\DefinirListeFichiers{%
  \setsepchar{,}\ignoreemptyitems%
  \readlist*\ListeFichiers{\CNFOO}%
}

\def\RecupererFichierTeXAllSSDossier#1{%
  \RecupererSousDossiers{#1}%
  \foreachitem\compteur\in\ListeSSDossiers{\RecupererFichierTeXSSDossier{#1}{\ListeSSDossiers[\compteurcnt]}}%
  \DefinirListeFichiers%
}

\def\RecupererFichierTeXAllDossier#1{%
  \RecupererFichierTeXDossier{#1}%
  \DefinirListeFichiers%
}

\def\RecupererFichierTeXComplet#1{%
  \RecupererFichierTeXDossier{#1}%
  \RecupererSousDossiers{#1}%
  \foreachitem\compteur\in\ListeSSDossiers{\RecupererFichierTeXSSDossier{#1}{\ListeSSDossiers[\compteurcnt]}}%
  \DefinirListeFichiers%
}

\def\RecupererFichierTeXSSDossierPerso#1{%
  \openin\zzz="|ls #1/*.tex"
  \loop%
  \read\zzz to \tmp%
  \ifx\tmp\zzpar\else%
  \xdef\CNFOO{\CNFOO \expandafter\detokenize\expandafter{\tmp},}%
  \fi%
  \ifeof\zzz\else%
  \repeat%
}

\makeatother

\newcommand\PfCJury{Jury}%
\newcommand\PfCEnonce{Énoncé}%
\newcommand\PfCReponse{Reponse}%
\newcommand\CNTheme[1]{}%
\newcommand\CNReponse{}%

\newlength{\PfCLargeurQuestion}%
\newlength{\PfCLargeurReponse}%
\newlength{\PfCLargeurJury}%

\NewDocumentCommand\CourseNombre{o m}{%
  \useKVdefault[ClesCN]%
  \setKV[ClesCN]{#1}%
  \ifboolKV[ClesCN]{Theme}{\renewcommand\CNTheme[1]{{\ttfamily Thème : ##1}\par}}{}%
  \ifboolKV[ClesCN]{RAZ}{%
    \setcounter{CNNumQ}{\useKV[ClesCN]{Debut}}%
  }{}%
  \setKV[ClesMathAlea]{NbQ=1}%
  \xdef\CNFOOSSDossiers{}%
  \xdef\CNFOO{}%
  \ifboolKV[ClesCN]{Lecture}{%
    \ifboolKV[ClesCN]{Perso}{%
      %      % Ici \useKV[ClesCN]{Liste}
      \xdef\CNFOOListePerso{\useKV[ClesCN]{Liste}}%
      %      % OK -> Là, normalement, c'est la liste des dossiers choisis par l'utilisateur : \CNFOOListePerso
      \setsepchar{,}%
      \readlist*\ListeDesSSDossiersPerso{\CNFOOListePerso}%
      %      % \par OK -> Là, normalement, c'est la liste des dossiers perso sous forme de liste : \showitems\ListeDesSSDossiersPerso[]%
      \foreachitem\compteur\in\ListeDesSSDossiersPerso{\RecupererFichierTeXSSDossierPerso{\ListeDesSSDossiersPerso[\compteurcnt]}}%
      \DefinirListeFichiers%
      %      % \par OK -> Là, enfin, normalement, c'est la liste des fichiers contenus dans les dossiers perso :
      %      % \showitems\ListeFichiers[]
    }{%
      \ifboolKV[ClesCN]{Maitre}{%
        %        % On récupère tous les fichiers can dans les sous-dossiers
        %        % et on construit une liste \ListeFichiers contenant les chemins des fichiers considérés.
        \RecupererFichierTeXComplet{\useKV[ClesCN]{Dossier}}%
      }{%
        %        % Là
        %        % On récupère seulement les fichiers can de ce dossier
        %        % et on consruit une liste \ListeFichiers contenant les chemins des fichiers considérés.
        \RecupererFichierTeXAllDossier{\useKV[ClesCN]{Dossier}}%
      }%
    }%
    %    % On crée la liste du nombre total de questions
    \xdef\CNfooListe{}%
    \xintFor* ##1 in{\xintSeq{1}{\ListeFichierslen}}\do{%
      \xdef\CNfooListe{\CNfooListe ##1,}%
    }%
    %    % \par OK -> normalement, on a la liste du nombre de questions : \CNfooListe
    % On règle maintenant les dimensions nécessaires.
    % \ifboolKV[ClesCN]{CAN}{
    \setlength{\PfCLargeurQuestion}{\useKV[ClesCN]{CoefQ}\linewidth}\setlength{\PfCLargeurReponse}{\useKV[ClesCN]{CoefR}\linewidth}\setlength{\PfCLargeurJury}{\useKV[ClesCN]{CoefJ}\linewidth}%
    %    % On affiche les questions :
    \ifboolKV[ClesCN]{Ordre}{%
      %        % soit dans un exercice
      \ifboolKV[ClesCN]{Exercice}{%
        %          % Attention le style de l'environnement est à choisir par l'utilisateur
        \xintFor* ##1 in {\xintSeq{1}{\ListeFichierslen}}\do{%
        \item\input{\ListeFichiers[##1]}%
        }%
      }{% Soit toutes les questions au format course
        \renewcommand{\arraystretch}{\useKV[ClesCN]{Stretch}}%
        \begin{longtable}{|c|p{\PfCLargeurQuestion}|>{\centering\arraybackslash}p{\PfCLargeurReponse}|p{\PfCLargeurJury}|}%
          \hhline{~---}%
          \multicolumn{1}{c|}{}&\cellcolor{gray!15}\centering\PfCEnonce&\cellcolor{gray!15}\centering\PfCReponse&\cellcolor{gray!15}\centering\arraybackslash\PfCJury\\%
          \hline%
          \endhead%
          \xintFor* ##1 in {\xintSeq{1}{\ListeFichierslen}}\do{%
            \ttfamily{\theCNNumQ}\stepcounter{CNNumQ}&\xdef\CNReponse{}\input{\ListeFichiers[##1]}\ifboolKV[ClesCN]{Nom}{\par\hfill\footnotesize\ttfamily\ListeFichiers[##1]}{}&\CNReponse%\ifboolKV[ClesCN]{CAN}{\ifboolKV[ClesCN]{Nom}{\multirow{2}{*}{\CNReponse}}{\multirow{1}{*}{\CNReponse}}}{}
            &\\%
            \hline%
          }%
        \end{longtable}%
        \renewcommand{\arraystretch}{1}%
      }%
    }{%
      % On prévient l'utilisateur si le nb de questions est inférieur à NbQ
      \xintifboolexpr{\ListeFichierslen<\useKV[ClesCN]{NbQ}}{%
        \faExclamationCircle~Le nombre maximal de questions disponibles est inférieur au nombre de questions à afficher. Modifier la clé {\ttfamily NbQ} ou ajouter des questions dans le(s) répertoire(s).%
      }{%%On choisit NbQ questions parmi la liste de fichiers
        \MelangeListe{\CNfooListe}{\useKV[ClesCN]{NbQ}}%
        \setsepchar{,}\ignoreemptyitems%
        \readlist*\ListeCNQuestions{\faa}
        \ifboolKV[ClesCN]{Exercice}{%
          \xintFor* ##1 in {\xintSeq{1}{\useKV[ClesCN]{NbQ}}}\do{%
          \item\xdef\Numerodelaquestionaposer{\ListeCNQuestions[##1]}\input{\ListeFichiers[\Numerodelaquestionaposer]}%
          }%
        }{% On les affiche
          \renewcommand{\arraystretch}{\useKV[ClesCN]{Stretch}}%
          \begin{longtable}{|c|p{\PfCLargeurQuestion}|>{\centering\arraybackslash}p{\PfCLargeurReponse}|p{\PfCLargeurJury}|}%
            \hhline{~---}%
            \multicolumn{1}{c|}{}&\cellcolor{gray!15}\centering\PfCEnonce&\cellcolor{gray!15}\centering\PfCReponse&\cellcolor{gray!15}\centering\arraybackslash\PfCJury\\%
            \hline%
            \endhead%
            \xintFor* ##1 in {\xintSeq{1}{\useKV[ClesCN]{NbQ}}}\do{%
              \ttfamily{\theCNNumQ}\stepcounter{CNNumQ}&\xdef\CNReponse{}\xdef\Numerodelaquestionaposer{\ListeCNQuestions[##1]}\input{\ListeFichiers[\Numerodelaquestionaposer]}\ifboolKV[ClesCN]{Nom}{\par\hfill\footnotesize\ttfamily\ListeFichiers[\Numerodelaquestionaposer]}{}&\ifboolKV[ClesCN]{CAN}{\CNReponse}{}&\\
              \hline
            }%
          \end{longtable}%
          \renewcommand{\arraystretch}{1}%
        }%
      }%
    }%
  }{%
    \input{#2}
  }%
}%
% fin course aux nombres