%%%
% Criteres de divisibilite
%%%
\def\filedateCritere{2024/08/04}%
\def\fileversionCritere{0.1}%
\message{-- \filedateCritere\space v\fileversionCritere}%
%
\setKVdefault[Critere]{Div=2,Remarque=false,All=false,RemarqueAll=false}

\NewDocumentCommand\Critere{om}{%
  \useKVdefault[Critere]%
  \setKV[Critere]{#1}%
  \xdef\PfCListeARetenirCriteres{}%
  \ifboolKV[Critere]{All}{%
      \begin{itemize}
      \item \CritereDeux{#2}%
      \item \CritereTrois{#2}
      \item \CritereQuatre{#2}
      \item \CritereCinq{#2}
      \item \CritereNeuf{#2}
      \item \CritereDix{#2}
      \end{itemize}
    \ifx\bla\PfCListeARetenirCriteres\bla%
    \else
      \setsepchar{,}\ignoreemptyitems%
      \readlist*\PfCAfficheDivCriteres{\PfCListeARetenirCriteres}%
      \reademptyitems%
      \xdef\PfCNumListe{}%
      En appliquant les critères de divisibilité connus, \num{#2} est donc divisible par \num{\PfCAfficheDivCriteres[1]}%
      \xintifboolexpr{\PfCAfficheDivCritereslen>1}{\xintFor* ##1 in{\xintSeq{2}{\PfCAfficheDivCritereslen}}\do{%
          \xintifForLast{ et }{ ; }\num{\PfCAfficheDivCriteres[##1]}%
        }%
      }{}.%
      \ifboolKV[Critere]{RemarqueAll}{\\\textbf{Remarque} : La liste des diviseurs de \num{#2} est \ListeDiviseur{#2}.}{}%
    \fi%
  }{%
    \IfStrEqCase{\useKV[Critere]{Div}}{%
      {2}{\CritereDeux{#2}}%
      {3}{\CritereTrois{#2}}%
      {4}{\CritereQuatre{#2}}%
      {5}{\CritereCinq{#2}}%
      {9}{\CritereNeuf{#2}}%
      {10}{\CritereDix{#2}}%
    }%
  }%
}%

\NewDocumentCommand\CritereDeux{m}{%
  \StrRight{#1}{1}[\PfCCriDeuxUnite]%
  Le chiffre des unités de \num{#1} est \num{\PfCCriDeuxUnite}. Donc \num{#1} %
  \modulo{\PfCCriDeuxUnite}{2}%
  \ifnum\remainder=0\relax%
  est divisible par 2.%
  \ifboolKV[Critere]{Remarque}{\\\textbf{Remarque} : $\num{#1}=\num{\fpeval{#1/2}}\times2$.}{}%
  \xdef\PfCListeARetenirCriteres{\PfCListeARetenirCriteres,2}%
  \else%
  n'est pas divisible par 2.%
  \fi%
}%

\NewDocumentCommand\CritereTrois{m}{%
  La somme des chiffres de \num{#1} est \PfCSommeChiffres{#1}. Comme \num{\PfCSMSomme} %
  \modulo{\PfCSMSomme}{3}%
  \ifnum\remainder=0\relax%
  est divisible par 3, alors \num{#1} est aussi divisible par 3.%
  \ifboolKV[Critere]{Remarque}{\\\textbf{Remarque} : $\num{#1}=\num{\fpeval{#1/3}}\times3$.}{}%
  \xdef\PfCListeARetenirCriteres{\PfCListeARetenirCriteres,3}%
  \else%
  n'est pas divisible par 3, alors \num{#1} n'est pas divisible par 3.%
  \fi%
}%

\NewDocumentCommand\CritereQuatre{m}{%
  \ifnum#1<100%
    \quotient{#1}{4}%
    \modulo{#1}{4}%
    \ifnum\remainder=0\relax
      Comme $\num{#1}=\num{\the\intquotient}\times4$ alors \num{#1} est divisible par 4.
      \xdef\PfCListeARetenirCriteres{\PfCListeARetenirCriteres,4}%
    \else
      Comme $\num{#1}=\num{\the\intquotient}\times4+\num{\the\remainder}$ alors \num{#1} n'est pas divisible par 4.
    \fi
  \else
    \StrRight{#1}{2}[\PfCCriQuatreUnite]%
    \num{#1} se termine par \num{\PfCCriQuatreUnite}. %
    \modulo{\PfCCriQuatreUnite}{4}%
    \ifnum\remainder=0\relax%
      Comme \num{\PfCCriQuatreUnite} est divisible par 4, alors \num{#1} est divisible par 4.%
      \ifboolKV[Critere]{Remarque}{\\\textbf{Remarque} : $\num{#1}=\num{\fpeval{#1/4}}\times4$.}{}%
      \xdef\PfCListeARetenirCriteres{\PfCListeARetenirCriteres,4}%
    \else%
      Comme \num{\PfCCriQuatreUnite} n'est pas divisible par 4, alors \num{#1} n'est pas divisible par 4.%
    \fi%
 \fi
}%

\NewDocumentCommand\CritereCinq{m}{%
  \StrRight{#1}{1}[\PfCCriCinqUnite]%
  Le chiffre des unités de \num{#1} est \num{\PfCCriCinqUnite}. Donc \num{#1} %
  \modulo{\PfCCriCinqUnite}{5}%
  \ifnum\remainder=0\relax%
  est divisible par 5.%
  \ifboolKV[Critere]{Remarque}{\\\textbf{Remarque} : $\num{#1}=\num{\fpeval{#1/5}}\times5$.}{}%
    \xdef\PfCListeARetenirCriteres{\PfCListeARetenirCriteres,5}%
  \else%
  n'est pas divisible par 5.%
  \fi%
}%

\NewDocumentCommand\CritereNeuf{m}{%
  La somme des chiffres de \num{#1} est \PfCSommeChiffres{#1}. Comme \num{\PfCSMSomme} %
  \modulo{\PfCSMSomme}{9}%
  \ifnum\remainder=0\relax%
  est divisible par 9, alors \num{#1} est aussi divisible par 9.%
  \ifboolKV[Critere]{Remarque}{\\\textbf{Remarque} : $\num{#1}=\num{\fpeval{#1/9}}\times9$.}{}%
  \xdef\PfCListeARetenirCriteres{\PfCListeARetenirCriteres,9}%
  \else%
  n'est pas divisible par 9, alors \num{#1} n'est pas divisible par 9.%
  \fi%
}%

\NewDocumentCommand\CritereDix{m}{%
  \StrRight{#1}{1}[\PfCCriDixUnite]%
  Le chiffre des unités de \num{#1} est \num{\PfCCriDixUnite}. Donc \num{#1} %
  \ifnum\PfCCriDixUnite=0\relax
  est divisible par 10.%
  \ifboolKV[Critere]{Remarque}{\\\textbf{Remarque} : $\num{#1}=\num{\fpeval{#1/10}}\times10$.}{}%
  \xdef\PfCListeARetenirCriteres{\PfCListeARetenirCriteres,10}%
  \else%
  n'est pas divisible par 10.%
  \fi%
}%