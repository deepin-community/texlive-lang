%%%
% Lien : ratio
%%%
\setKVdefault[ClesRatio]{FigureCours=false,Figure=false,Longueur=5cm,TexteTotal=quantit\'e,TextePart=part,Tableau=false,GrandeurA=Grandeur A,GrandeurB=Part(s),Largeur=1cm,Stretch=1,Nom=false,CouleurUn=gris,CouleurDeux=0.5gris+0.5blanc,CouleurTrois=white,CouleurTab=gray!15,ModeleCouleur=5}

\newcommand\MPTest[9][]{%
  % #2 : Longueur de la barre unit\'e
  % #3 : premier nombre
  % #4 : deuxi\`eme nombre
  % #5 : troisi\`eme nombre
  % #6 : ne sert à rien. Laissé en attente.
  % #7 \`a #9: Couleurs de remplissage
  \ifluatex
   \mplibforcehmode
   \begin{mplibcode}
     defaultcolormodel := \useKV[ClesRatio]{ModeleCouleur};
    vardef RatioTrois(expr long)(text t)=%longueur de la barre / quantit\'e \`a partager / textepart :) / t le ratio
    pair A,B,C,D;
    A=u*(1,1);
    B-A=(long,0);
    C-B=u*(0,0.5);
    D-C=A-B;
    n:=0;%n pour savoir si le ratio est a:b ou a:b:c
    numeric N[];%Pour sauvegarder les \'el\'ements du ratio
    for p_=t:
    n:=n+1;
    N[n]=p_;
    endfor;
    % on fait la somme totale "du ratio"
    somme=0;
    somme:=somme for k=1 upto n:+N[k] endfor;
    %Figure(0,0,long+2u,3u);
    remplis polygone(A,(N[1]/somme)[A,B],(N[1]/somme)[D,C],D)
    withcolor #7;
    remplis polygone(B,(N[1]/somme)[A,B],(N[1]/somme)[D,C],C)
    withcolor #8;
    if n>2:
    remplis
    polygone((N[1]/somme)[A,B],((N[1]+N[2])/somme)[A,B],((N[1]+N[2])/somme)[D,C],(N[1]/somme)[D,C])
    withcolor #8;
    remplis
    polygone(B,((N[1]+N[2])/somme)[A,B],((N[1]+N[2])/somme)[D,C],C)
    withcolor #9;
    fi;
    drawoptions(withpen pencircle scaled1.5bp);
    draw polygone(A,B,C,D);
    for k=1 upto somme-1:
    draw segment((k/somme)[A,B],(k/somme)[D,C]);
    endfor;
    drawoptions();
    % accolades
    labeloffset:=labeloffset/2;
    label.top(TEX("\footnotesize$\overbrace{\hbox
      to"&decimal(abs(A-B))&"pt{}}$"),iso(D,C));
    labeloffset:=labeloffset*2;
    label.bot(TEX("\footnotesize$\underbrace{\hbox
      to"&decimal(abs((N[1]/somme)[A,B]-A))&"pt{}}$"),iso(A,(N[1]/somme)[A,B]));
    label.bot(TEX("\footnotesize$\underbrace{\hbox
      to"&decimal(abs((N[1]/somme)[A,B]-((N[1]+N[2])/somme)[A,B]))&"pt{}}$"),iso(((N[1]+N[2])/somme)[A,B],(N[1]/somme)[A,B]));
    if n>2:
    label.bot(TEX("\footnotesize$\underbrace{\hbox
      to"&decimal(abs(((N[1]+N[2])/somme)[A,B]-B))&"pt{}}$"),iso(B,((N[1]+N[2])/somme)[A,B]));
    fi;
    enddef;
    RatioTrois(#2)(#3,#4,#5);
    %etiquettage
    labeloffset:=labeloffset*3;
    label.top(btex \useKV[ClesRatio]{TexteTotal} etex,iso(D,C));
    if #3>1:
    label.bot(btex #3~\useKV[ClesRatio]{TextePart}s
    etex,iso(A,(N[1]/somme)[A,B]));
    else:
    label.bot(btex #3~\useKV[ClesRatio]{TextePart} etex,iso(A,(N[1]/somme)[A,B]));
    fi;
    if #4>1:
    label.bot(btex #4~\useKV[ClesRatio]{TextePart}s etex,iso(((N[1]+N[2])/somme)[A,B],(N[1]/somme)[A,B]));
    else:
    label.bot(btex #4~\useKV[ClesRatio]{TextePart}
    etex,iso(((N[1]+N[2])/somme)[A,B],(N[1]/somme)[A,B]));
    fi;
    if n>2:
    if #5>1:
    label.bot(btex #5~\useKV[ClesRatio]{TextePart}s etex,iso(B,((N[1]+N[2])/somme)[A,B]));
    else:
    label.bot(btex #5~\useKV[ClesRatio]{TextePart} etex,iso(B,((N[1]+N[2])/somme)[A,B]));
    fi;
    fi;
  \end{mplibcode}
  \else
  \mpxcommands{%
    \usepackage{ProfCollege}
    \setKVdefault[ClesRatio]{TexteTotal=quantit\'e,TextePart=part}
    \setKV[ClesRatio]{#1}
  }
  \begin{mpost}
    vardef RatioTrois(expr long)(text t)=%longueur de la barre / quantit\'e \`a partager / textepart :) / t le ratio
    pair A,B,C,D;
    A=u*(1,1);
    B-A=(long,0);
    C-B=u*(0,0.5);
    D-C=A-B;
    n:=0;%n pour savoir si le ratio est a:b ou a:b:c
    numeric N[];%Pour sauvegarder les \'el\'ements du ratio
    for p_=t:
    n:=n+1;
    N[n]=p_;
    endfor;
    % on fait la somme totale "du ratio"
    somme=0;
    somme:=somme for k=1 upto n:+N[k] endfor;
    Figure(0,0,long+2u,3u);
    remplis polygone(A,(N[1]/somme)[A,B],(N[1]/somme)[D,C],D)
    withcolor #7;
    remplis polygone(B,(N[1]/somme)[A,B],(N[1]/somme)[D,C],C)
    withcolor #8;
    if n>2:
    remplis
    polygone((N[1]/somme)[A,B],((N[1]+N[2])/somme)[A,B],((N[1]+N[2])/somme)[D,C],(N[1]/somme)[D,C])
    withcolor #8;
    remplis
    polygone(B,((N[1]+N[2])/somme)[A,B],((N[1]+N[2])/somme)[D,C],C)
    withcolor #9;
    fi;
    drawoptions(withpen pencircle scaled1.5bp);
    draw polygone(A,B,C,D);
    for k=1 upto somme-1:
    draw segment((k/somme)[A,B],(k/somme)[D,C]);
    endfor;
    drawoptions();
    %accolades
    label.top(LATEX("\noexpand\footnotesize$\noexpand\overbrace{\noexpand\hbox
      to"&decimal(abs(A-B))&"pt{}}$"),iso(D,C));
    label.bot(LATEX("\noexpand\footnotesize$\noexpand\underbrace{\noexpand\hbox
      to"&decimal(abs((N[1]/somme)[A,B]-A))&"pt{}}$"),iso(A,(N[1]/somme)[A,B]));
    label.bot(LATEX("\noexpand\footnotesize$\noexpand\underbrace{\noexpand\hbox
      to"&decimal(abs((N[1]/somme)[A,B]-((N[1]+N[2])/somme)[A,B]))&"pt{}}$"),iso(((N[1]+N[2])/somme)[A,B],(N[1]/somme)[A,B]));
    if n>2:
    label.bot(LATEX("\noexpand\footnotesize$\noexpand\underbrace{\noexpand\hbox
      to"&decimal(abs(((N[1]+N[2])/somme)[A,B]-B))&"pt{}}$"),iso(B,((N[1]+N[2])/somme)[A,B]));
    fi;
    enddef;
    RatioTrois(#2)(#3,#4,#5);
    %etiquettage
    labeloffset:=labeloffset*3;
    label.top(\btex \useKV[ClesRatio]{TexteTotal} etex,iso(D,C));
    if #3>1:
    label.bot(btex #3\unexpanded{~\useKV[ClesRatio]{TextePart}}s etex,iso(A,(N[1]/somme)[A,B]));
    else:
    label.bot(btex #3\unexpanded{~\useKV[ClesRatio]{TextePart}} etex,iso(A,(N[1]/somme)[A,B]));
    fi;
    if #4>1:
    label.bot(btex #4\unexpanded{~\useKV[ClesRatio]{TextePart}}s etex,iso(((N[1]+N[2])/somme)[A,B],(N[1]/somme)[A,B]));
    else:
    label.bot(btex #4\unexpanded{~\useKV[ClesRatio]{TextePart}} etex,iso(((N[1]+N[2])/somme)[A,B],(N[1]/somme)[A,B]));
    fi;
    if n>2:
    if #5>1:
    label.bot(btex #5\unexpanded{~\useKV[ClesRatio]{TextePart}}s etex,iso(B,((N[1]+N[2])/somme)[A,B]));
    else:
    label.bot(btex #5\unexpanded{~\useKV[ClesRatio]{TextePart}} etex,iso(B,((N[1]+N[2])/somme)[A,B]));
    fi;
    fi;
  \end{mpost}
  \fi
}%

\newcommand\MPTestCours[9][]{%
  % #2 : Longueur de la barre unit\'e
  % #3 : premier nombre
  % #4 : deuxi\`eme nombre
  % #5 : troisi\`eme nombre
  % #6 : Valeurs du ratio
  % #7 \`a #9: Couleurs de remplissage
  \ifluatex
   \mplibforcehmode
   \begin{mplibcode}
     defaultcolormodel := \useKV[ClesRatio]{ModeleCouleur};
    vardef RatioTrois(expr long)(text t)=%longueur de la barre / quantit\'e \`a partager / textepart :) / t le ratio
    pair A,B,C,D;
    A=u*(1,1);
    B-A=(long,0);
    C-B=u*(0,0.5);
    D-C=A-B;
    n:=0;%n pour savoir si le ratio est a:b ou a:b:c
    numeric N[];%Pour sauvegarder les \'el\'ements du ratio
    for p_=t:
    n:=n+1;
    N[n]=p_;
    endfor;
    % on fait la somme totale "du ratio"
    somme=0;
    somme:=somme for k=1 upto n:+N[k] endfor;
    %%Figure(0,0,long+2u,3u);
    remplis polygone(A,(N[1]/somme)[A,B],(N[1]/somme)[D,C],D)
    withcolor #7;
    remplis polygone(B,(N[1]/somme)[A,B],(N[1]/somme)[D,C],C)
    withcolor #8;
    if n>2:
    remplis
    polygone((N[1]/somme)[A,B],((N[1]+N[2])/somme)[A,B],((N[1]+N[2])/somme)[D,C],(N[1]/somme)[D,C])
    withcolor #8;
    remplis
    polygone(B,((N[1]+N[2])/somme)[A,B],((N[1]+N[2])/somme)[D,C],C)
    withcolor #9;
    fi;
    drawoptions(withpen pencircle scaled1.5bp);
    draw polygone(A,B,C,D);
    for k=1 upto somme-1:
    draw segment((k/somme)[A,B],(k/somme)[D,C]);
    endfor;
    drawoptions();
    % accolades
    label.bot(TEX("\footnotesize$\underbrace{\hbox
      to"&decimal(abs((N[1]/somme)[A,B]-A))&"pt{}}$"),iso(A,(N[1]/somme)[A,B]));
    label.bot(TEX("\footnotesize$\underbrace{\hbox
      to"&decimal(abs((N[1]/somme)[A,B]-((N[1]+N[2])/somme)[A,B]))&"pt{}}$"),iso(((N[1]+N[2])/somme)[A,B],(N[1]/somme)[A,B]));
    if n>2:
    label.bot(TEX("\footnotesize$\underbrace{\hbox
      to"&decimal(abs(((N[1]+N[2])/somme)[A,B]-B))&"pt{}}$"),iso(B,((N[1]+N[2])/somme)[A,B]));
    fi;
    enddef;
    RatioTrois(#2)(#3,#4,#5);
    %etiquettage
    labeloffset:=labeloffset*3;
    label.bot(btex $a$ etex,iso(A,(N[1]/somme)[A,B]));
    label.bot(btex $b$ etex,iso(((N[1]+N[2])/somme)[A,B],(N[1]/somme)[A,B]));
    if n>2:
    label.bot(btex $c$ etex,iso(B,((N[1]+N[2])/somme)[A,B]));
    fi;
  \end{mplibcode}
  \else
  \begin{mpost}
    vardef RatioTrois(expr long)(text t)=%longueur de la barre / quantit\'e \`a partager / textepart :) / t le ratio
    pair A,B,C,D;
    A=u*(1,1);
    B-A=(long,0);
    C-B=u*(0,0.5);
    D-C=A-B;
    n:=0;%n pour savoir si le ratio est a:b ou a:b:c
    numeric N[];%Pour sauvegarder les \'el\'ements du ratio
    for p_=t:
    n:=n+1;
    N[n]=p_;
    endfor;
    % on fait la somme totale "du ratio"
    somme=0;
    somme:=somme for k=1 upto n:+N[k] endfor;
    %Figure(0,0,long+2u,3u);
    remplis polygone(A,(N[1]/somme)[A,B],(N[1]/somme)[D,C],D)
    withcolor #7;
    remplis polygone(B,(N[1]/somme)[A,B],(N[1]/somme)[D,C],C)
    withcolor #8;
    if n>2:
    remplis
    polygone((N[1]/somme)[A,B],((N[1]+N[2])/somme)[A,B],((N[1]+N[2])/somme)[D,C],(N[1]/somme)[D,C])
    withcolor #8;
    remplis
    polygone(B,((N[1]+N[2])/somme)[A,B],((N[1]+N[2])/somme)[D,C],C)
    withcolor #9;
    fi;
    drawoptions(withpen pencircle scaled1.5bp);
    draw polygone(A,B,C,D);
    for k=1 upto somme-1:
    draw segment((k/somme)[A,B],(k/somme)[D,C]);
    endfor;
    drawoptions();
    % accolades
    label.bot(LATEX("\noexpand\footnotesize$\noexpand\underbrace{\noexpand\hbox
      to"&decimal(abs((N[1]/somme)[A,B]-A))&"pt{}}$"),iso(A,(N[1]/somme)[A,B]));
    label.bot(LATEX("\noexpand\footnotesize$\noexpand\underbrace{\noexpand\hbox
      to"&decimal(abs((N[1]/somme)[A,B]-((N[1]+N[2])/somme)[A,B]))&"pt{}}$"),iso(((N[1]+N[2])/somme)[A,B],(N[1]/somme)[A,B]));
    if n>2:
    label.bot(LATEX("\noexpand\footnotesize$\noexpand\underbrace{\noexpand\hbox
      to"&decimal(abs(((N[1]+N[2])/somme)[A,B]-B))&"pt{}}$"),iso(B,((N[1]+N[2])/somme)[A,B]));
    fi;
    enddef;
    RatioTrois(#2)(#3,#4,#5);
    %etiquettage
    labeloffset:=labeloffset*3;
    label.bot(btex $a$ etex,iso(A,(N[1]/somme)[A,B]));
    label.bot(btex $b$ etex,iso(((N[1]+N[2])/somme)[A,B],(N[1]/somme)[A,B]));
    if n>2:
    label.bot(btex $c$ etex,iso(B,((N[1]+N[2])/somme)[A,B]));
    fi;
  \end{mpost}
  \fi
}

\newtoks\toklisteratio
\def\UpdateRatio#1\nil{\addtotok\toklisteratio{#1,}}

\def\updateratiotoks#1/#2/#3\nil{\addtotok\tabtoksa{&\ifx\bla#2\bla\else\num{#2}\fi}\addtotok\tabtoksb{&\ifx\bla#3\bla\else\num{#3}\fi}\addtotok\tabtoksc{&#1}}

\def\buildtabratio{%
  \tabtoksa{}\tabtoksb{}\tabtoksc{}%
  \tabtoksa{\useKV[ClesRatio]{GrandeurA}}\tabtoksb{\useKV[ClesRatio]{GrandeurB}}
  \foreachitem\compteur\in\ListeRatio{\expandafter\updateratiotoks\compteur\nil}%
  \xdef\LongListe{\ListeRatiolen}%
  \renewcommand{\arraystretch}{\useKV[ClesRatio]{Stretch}}%
  \begin{tabular}{|>{\columncolor{\useKV[ClesRatio]{CouleurTab}}}c|*{\number\numexpr\ListeRatiolen}{>{\centering\arraybackslash}p{\useKV[ClesRatio]{Largeur}}|}l}
    \ifboolKV[ClesRatio]{Nom}{%
    \hhline{~*{\number\numexpr\ListeRatiolen}{-}}
    \multicolumn{1}{c|}{}\the\tabtoksc\\
    }{}
    \hhline{*{\number\numexpr\ListeRatiolen+1}{-}}%
    \the\tabtoksa&\setcounter{NbPropor}{1}\TikzRH\\%
    \hhline{*{\number\numexpr\ListeRatiolen+1}{-}}%
    \the\tabtoksb&\setcounter{NbPropor}{1}\TikzRB\\%
    \hhline{*{\number\numexpr\ListeRatiolen+1}{-}}%
  \end{tabular}%
}%

\newcommand\TikzRH{%
   \tikz[remember picture,overlay]{%
   \coordinate[name=ProporH-\theNbPropor,yshift=\getstrut\ht*\arraystretch];}%
   \stepcounter{NbPropor}%
 }%

 \newcommand\TikzRB{%
   \tikz[remember picture, overlay]{%
     \coordinate[name=ProporB-\theNbPropor,yshift=-\getstrut\dp*\arraystretch];}%
   \stepcounter{NbPropor}%
 }%
 
\newcommand\FlecheRatio[2][\EcartLargeur]{%
  \begin{tikzpicture}[remember picture, overlay]%
    \node[] (Point1) at ($(ProporH-1)!0.1!(ProporB-1)$) {};%
    \node[] (Point2) at ($(ProporH-1)!0.9!(ProporB-1)$) {};%
    \coordinate[right of=Point1,node distance=0*#1-\tabcolsep] (point1);%
    \coordinate[right of=Point2,node distance=0*#1-\tabcolsep] (point2);%
    \draw[-stealth,out=-20,in=20] (point1) to node[midway,right,inner sep=1pt]{#2}(point2);%
\end{tikzpicture}%
}%

\newcommand\FlecheInvRatio[2][\EcartLargeur]{%
  \begin{tikzpicture}[remember picture, overlay]%
    \node[] (Point1) at ($(ProporH-1)!0.1!(ProporB-1)$) {};%
    \node[] (Point2) at ($(ProporH-1)!0.9!(ProporB-1)$) {};%
    \coordinate[right of=Point1,node distance=0*#1-\tabcolsep] (point1);%
    \coordinate[right of=Point2,node distance=0*#1-\tabcolsep] (point2);%
    \draw[-stealth,out=20,in=-20] (point2) to node[midway,right,inner sep=1pt]{#2}(point1);%
\end{tikzpicture}%
}%

\newcommand\Ratio[2][]{%
  \useKVdefault[ClesRatio]%
  \setKV[ClesRatio]{#1}%
  \xdef\EcartLargeur{\useKV[ClesRatio]{Largeur}}%
  \ifboolKV[ClesRatio]{FigureCours}{%
    \setsepchar{,}\ignoreemptyitems%
    \readlist*\ListeRatio{#2}%
    \toklisteratio{}%
    \foreachitem\compteur\in\ListeRatio{\expandafter\UpdateRatio\compteur\nil}%
    \itemtomacro\ListeRatio[1]\NbUn%
    \itemtomacro\ListeRatio[2]\NbDeux%
    \xintifboolexpr{\ListeRatiolen>2}{\itemtomacro\ListeRatio[3]\NbTrois}{\xdef\NbTrois{}}%
    \MPTestCours[#1]{\useKV[ClesRatio]{Longueur}}{\NbUn}{\NbDeux}{\NbTrois}{\the\toklisteratio}{\useKV[ClesRatio]{CouleurUn}}{\useKV[ClesRatio]{CouleurDeux}}{\useKV[ClesRatio]{CouleurTrois}}%
  }{%
    \ifboolKV[ClesRatio]{Figure}{%
      \setsepchar{,}\ignoreemptyitems%
      \readlist*\ListeRatio{#2}%
      \toklisteratio{}%
      \foreachitem\compteur\in\ListeRatio{\expandafter\UpdateRatio\compteur\nil}%
      \itemtomacro\ListeRatio[1]\NbUn%
      \itemtomacro\ListeRatio[2]\NbDeux%
      \xintifboolexpr{\ListeRatiolen>2}{\itemtomacro\ListeRatio[3]\NbTrois}{\xdef\NbTrois{}}%
      \MPTest[#1]{\useKV[ClesRatio]{Longueur}}{\NbUn}{\NbDeux}{\NbTrois}{\the\toklisteratio}{\useKV[ClesRatio]{CouleurUn}}{\useKV[ClesRatio]{CouleurDeux}}{\useKV[ClesRatio]{CouleurTrois}}%
    }{%
      \ifboolKV[ClesRatio]{Tableau}{%
        \setsepchar[*]{,*/}\ignoreemptyitems%
        \readlist*\ListeRatio{#2}%
        \buildtabratio%
      }{}%
    }%
  }%
}%