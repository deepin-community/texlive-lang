%%%
% Enquête
%%%
\def\filedateEnquete{2024/08/04}%
\def\fileversionEnquete{0.1}%
\message{-- \filedateEnquete\space v\fileversionEnquete}%
%
\setKVdefault[ClesEnquete]{Perso=Qui ?,Objet=Quoi ?,Lieu=Où ?,Largeur=4.5cm}

\newcommand\Enquete[1][]{%
  \useKVdefault[ClesEnquete]%
  \setKV[ClesEnquete]{#1}%
}

\newcommand\ListePersonnages[1]{%
  \setsepchar{§}%
  \readlist*\RetiensListePersonnages{#1}%
  \setsepchar{,}%
}

\newcommand\ListeObjets[1]{%
  \setsepchar{§}%
  \readlist*\RetiensListeObjets{#1}%
  \setsepchar{,}%
}

\newcommand\ListeLieux[1]{%
  \setsepchar{§}%
  \readlist*\RetiensListeLieux{#1}%
  \setsepchar{,}%
}

\newcommand\ListeQuestions[1]{%
  \setsepchar[*]{§*/}%
  \readlist*\RetiensListeQuestions{#1}
  % On crée la liste des numéros des questions
  \xdef\foo{}%
  \xintFor* ##1 in {\xintSeq{1}{\RetiensListeQuestionslen}}\do{%
    \xdef\foo{\foo ##1,}%
  }%
  % Mélange des questions
  % on crée les 6 premieres solutions.
  \MelangeListe{\foo}{\fpeval{\RetiensListePersonnageslen}}%
  \readlist*\ListeMelangeePersonnages{\faa}%
  \readlist*\ListeRestante{\fii}%\ListeRestante[]
  \MelangeListe{\faa}{\fpeval{\RetiensListePersonnageslen-1}}%
  \readlist*\ListefuuPerso{\faa}%
  % on crée les 6 solutions suivantes.
  \xdef\fuuu{}%
  \foreachitem\compteur\in\ListeRestante{\xdef\fuuu{\fuuu,\ListeRestante[\compteurcnt]}}%
  \MelangeListe{\fuuu}{\fpeval{\RetiensListeObjetslen}}%
  \readlist*\ListeMelangeeObjets{\faa}%
  \readlist*\ListeRestante{\fii}%
  \MelangeListe{\faa}{\fpeval{\RetiensListeObjetslen-1}}%
  \readlist*\ListefuuObjets{\faa}%
  % on crée les dernieres solutions.
  \xdef\fuuu{}%
  \foreachitem\compteur\in\ListeRestante{\xdef\fuuu{\fuuu,\ListeRestante[\compteurcnt]}}%
  \MelangeListe{\fuuu}{\fpeval{\RetiensListeLieuxlen}}
  \readlist*\ListeMelangeeLieux{\faa}%
  %% \readlist*\ListeRestante{\fii}%
  \MelangeListe{\faa}{\fpeval{\RetiensListeLieuxlen-1}}%
  \readlist*\ListefuuLieux{\faa}%
  % --------------
  % Liste des questions sauvegardees
  \xdef\fuu{}%
  \foreachitem\compteur\in\ListefuuPerso{\xdef\fuu{\fuu,\ListefuuPerso[\compteurcnt]}}%
  \foreachitem\compteur\in\ListefuuObjets{\xdef\fuu{\fuu,\ListefuuObjets[\compteurcnt]}}%
  \foreachitem\compteur\in\ListefuuLieux{\xdef\fuu{\fuu,\ListefuuLieux[\compteurcnt]}}%
  \ignoreemptyitems\setsepchar{,}%
  \MelangeListe{\fuu}{\fpeval{\RetiensListePersonnageslen+\RetiensListeObjetslen+\RetiensListeLieuxlen-3}}%
  \readlist*\ListeMelangeeQuestions{\faa}%
  \reademptyitems%
}%

\newcommand\AffichageQuestions{%
  \begin{enumerate}
    \xintFor* ##1 in {\xintSeq{1}{\fpeval{\RetiensListePersonnageslen+\RetiensListeObjetslen+\RetiensListeLieuxlen-3}}}\do{%
    \item\xdef\Compteur{\ListeMelangeeQuestions[##1]}\RetiensListeQuestions[\Compteur,1]
    }
  \end{enumerate}
}

\newcommand\AffichageTableau{%
  \begin{tabular}{|>{\PfCTBstrut$\square$~}m{\useKV[ClesEnquete]{Largeur}}|}
    \hline
    \rowcolor{gray!15}\multicolumn{1}{|c|}{\PfCTBstrut\useKV[ClesEnquete]{Perso}}\\
    \hline
    \xintFor* ##1 in {\xintSeq{1}{\RetiensListePersonnageslen}}\do{%
    \RetiensListePersonnages[##1]\xdef\PfCCompteurMelange{\ListeMelangeePersonnages[##1]} $\star$ \RetiensListeQuestions[\PfCCompteurMelange,2]\\
    }
    % 
    \hline
    \rowcolor{gray!15}\multicolumn{1}{|c|}{\PfCTBstrut\useKV[ClesEnquete]{Objet}}\\
    \hline
    \xintFor* ##1 in {\xintSeq{1}{\RetiensListeObjetslen}}\do{%
    \RetiensListeObjets[##1]\xdef\PfCCompteurMelange{\ListeMelangeeObjets[##1]} $\star$ \RetiensListeQuestions[\PfCCompteurMelange,2]\\
    }
    % 
    \hline
    \rowcolor{gray!15}\multicolumn{1}{|c|}{\PfCTBstrut\useKV[ClesEnquete]{Lieu}}\\
    \hline
    \xintFor* ##1 in {\xintSeq{1}{\RetiensListeLieuxlen}}\do{%
    \RetiensListeLieux[##1]\xdef\PfCCompteurMelange{\ListeMelangeeLieux[##1]} $\star$ \RetiensListeQuestions[\PfCCompteurMelange,2]
    \\
    }
    \hline
  \end{tabular}
}%