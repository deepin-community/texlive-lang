% Licence    : Released under the LaTeX Project Public License v1.3c
% or later, see http://www.latex-project.org/lppl.txtf
\newtoks\TestEquaMBarreH%
\newtoks\TestEquaMBarreB%

\def\UpdateToksBarreH#1\nil{\addtotok\TestEquaMBarreH{#1}}%
\def\UpdateToksBarreB#1\nil{\addtotok\TestEquaMBarreB{#1}}%

\newcommand\EquaBaseMBarre[5][]{%type ax=d ou b=cx
  \useKVdefault[ClesEquation]%
  \setKV[ClesEquation]{#1}%
  \ifnum\fpeval{#2}=0\relax
    \ifnum\fpeval{#4}=0\relax
      \textbf{??}%% il manque un paramètre
    \else
      \xintifboolexpr{\fpeval{#4}==0}{%
        \xintifboolexpr{\fpeval{#3}==0}{%
          L'équation $0\times\useKV[ClesEquation]{Lettre}=0$ a une infinité de solutions.}{L'équation $0\times\useKV[ClesEquation]{Lettre}=\num{\fpeval{#3}}$ n'a aucune solution.}%
      }{%\else
        \xintifboolexpr{\fpeval{#3}==0}{L'équation $0=\num{\fpeval{#4}}\times\useKV[ClesEquation]{Lettre}$ a une unique solution : $\useKV[ClesEquation]{Lettre}=0$.}{%\else
          \TestEquaMBarreB{white}%
          \TestEquaMBarreH{}%
          \xintFor* ##1 in{\xintSeq{1}{\fpeval{#4}}}\do{%
            \expandafter\UpdateToksBarreB{,1}\nil%
            \expandafter\UpdateToksBarreB{,"$x$"}\nil%
          }%
          \expandafter\UpdateToksBarreH{white,#4,"\num{\fpeval{#3}}"}\nil%
          %          % Le tok H est \the\TestEquaMBarreH.%
          %          % \\Le tok B est \the\TestEquaMBarreB.%
          \PfCMPDessineModelBarreNonHomogene{\the\TestEquaMBarreH}{\the\TestEquaMBarreB}{}%
          \ifnum\fpeval{#4}=1\relax
          \else
            \bigskip%
          
            \PfCMPDessineModelBarreNonHomogene{white,1,"$\num{\fpeval{#3}}/\num{\fpeval{#4}}$"}{white,1,"$x$"}{}%
            \SSimpliTest{\fpeval{#3}}{\fpeval{#4}}%
            \ifthenelse{\boolean{Simplification}}{\bigskip%
              
              \PfCMPDessineModelBarreNonHomogene{white,1,"\PGCD{\fpeval{#3}}{\fpeval{#4}}\ifnum\fpeval{abs(\pgcd)}=\fpeval{#4}\relax{}\num{\fpeval{#3/#4}}\else\num{\fpeval{#3/\pgcd}}/\num{\fpeval{#4/\pgcd}}\fi"}{white,1,"$x$"}{}%
            }{}%
          \fi
        }%
      }%
    \fi%
  \else%
    % si non, on est dans le cas ax=d
    \xintifboolexpr{\fpeval{#2}==0}{%
      \xintifboolexpr{\fpeval{#5}==0}{%
        L'équation $0\times\useKV[ClesEquation]{Lettre}=0$ a une infinité de solutions.}{L'équation $0\times\useKV[ClesEquation]{Lettre}=\num{\fpeval{#5}}$ n'a aucune solution.}%
    }{%\else
      \xintifboolexpr{\fpeval{#5}==0}{L'équation $\num{\fpeval{#2}}\times\useKV[ClesEquation]{Lettre}=0$ a une unique solution : $\useKV[ClesEquation]{Lettre}=0$.}{%\else
        % \begin{center}
        \TestEquaMBarreH{white}%
        \TestEquaMBarreB{}%
        \xintFor* ##1 in{\xintSeq{1}{\fpeval{#2}}}\do{%
          \expandafter\UpdateToksBarreH{,1}\nil%
          \expandafter\UpdateToksBarreH{,"$x$"}\nil%
        }%
        \expandafter\UpdateToksBarreB{white,#2,"\num{\fpeval{#5}}"}\nil%
        % Le tok H est \the\TestEquaMBarreH.%
        % \\Le tok B est \the\TestEquaMBarreB.%
        \PfCMPDessineModelBarreNonHomogene{\the\TestEquaMBarreH}{\the\TestEquaMBarreB}{}%
        \ifnum\fpeval{#2}=1\relax%
        \else
          \bigskip%
          
          \PfCMPDessineModelBarreNonHomogene{white,1,"$x$"}{white,1,"$\num{\fpeval{#5}}/\num{\fpeval{#2}}$"}{}%
          \SSimpliTest{\fpeval{#5}}{\fpeval{#2}}%
          \ifthenelse{\boolean{Simplification}}{\bigskip
            
            \PfCMPDessineModelBarreNonHomogene{white,1,"$x$"}{white,1,"\PGCD{\fpeval{#5}}{\fpeval{#2}}\ifnum\fpeval{abs(\pgcd)}=\fpeval{#2}\relax{}\num{\fpeval{#5/#2}}\else\num{\fpeval{#5/\pgcd}}/\num{\fpeval{#2/\pgcd}}\fi"}{}%
          }{}%
        \fi
      }%
    }%
  \fi%
}%

\newcommand\EquaDeuxMBarre[5][]{%type ax+b=d ou b=cx+d$
  \useKVdefault[ClesEquation]%
  \setKV[ClesEquation]{#1}%
  \setKV[ClesEquation]{Fleches=false,FlecheDiv=false,Terme=false,Decomposition=false}
  \ifnum\fpeval{#2}=0\relax%On échange en faisant attention à ne pas boucler : c doit être non vide
    % cas b=cx+d
    \xintifboolexpr{\fpeval{#4}==0}{%
      \xintifboolexpr{\fpeval{#3}==\fpeval{#5}}{%b=d
        L'équation $\num{\fpeval{#3}}=\num{\fpeval{#5}}$ a une infinité de solutions.%
      }{%b<>d
        L'équation $\num{\fpeval{#3}}=\num{\fpeval{#5}}$ n'a aucune solution.%
      }%
    }{%ELSE
      \xintifboolexpr{\fpeval{#3}==0}{%ax+b=d
        \EquaBaseMBarre[#1]{0}{\fpeval{-#5}}{\fpeval{#4}}{0}%
      }{%ax+b=d$ Ici
        % \begin{center}
        \TestEquaMBarreB{white}%
        \TestEquaMBarreH{}%
        \xintFor* ##1 in{\xintSeq{1}{#4}}\do{%
          \expandafter\UpdateToksBarreB{,1}\nil%
          \expandafter\UpdateToksBarreB{,"$x$"}\nil%
        }%
        \expandafter\UpdateToksBarreB{,white,2.5,"\num{\fpeval{#5}}"}\nil%
        \expandafter\UpdateToksBarreH{white,#4+2.5,"\num{\fpeval{#3}}"}\nil%
        % Le tok H est \the\TestEquaMBarreH.%
        % \\Le tok B est \the\TestEquaMBarreB.%
        \PfCMPDessineModelBarreNonHomogene{\the\TestEquaMBarreH}{\the\TestEquaMBarreB}{}%
        
        \bigskip%
        
        \PfCMPDessineModelBarreNonHomogene{white,#4,"\num{\fpeval{#3-#5}}",white,2.5,"\num{\fpeval{#5}}"}{\the\TestEquaMBarreB}{#4}%
        
        \bigskip
        
        \EquaBaseMBarre{0}{(#3-(#5))}{#4}{0}%
      }%
    }%
  \else%cas ax+b=d
    \xintifboolexpr{\fpeval{#2}==0}{%
      \xintifboolexpr{\fpeval{#3}==\fpeval{#5}}{%b=d
        L'équation $\num{\fpeval{#3}}=\num{\fpeval{#5}}$ a une infinité de solutions.%
      }{%b<>d
        L'équation $\num{\fpeval{#3}}=\num{\fpeval{#5}}$ n'a aucune solution.%
      }%
    }{%ELSE
      \xintifboolexpr{\fpeval{#3}==0}{%ax+b=d
        \EquaBaseMBarre[#1]{#2}{0}{0}{#5}%
      }{%ax+b=d$ Ici
%        % \begin{center}
        \TestEquaMBarreH{white}%
        \TestEquaMBarreB{}%
        \xintFor* ##1 in{\xintSeq{1}{\fpeval{#2}}}\do{%
          \expandafter\UpdateToksBarreH{,1}\nil%
          \expandafter\UpdateToksBarreH{,"$x$"}\nil%
        }%
        \expandafter\UpdateToksBarreH{,white,2.5,"\num{\fpeval{#3}}"}\nil%
        \expandafter\UpdateToksBarreB{white,#2+2.5,"\num{\fpeval{#5}}"}\nil%
%        % Le tok H est \the\TestEquaMBarreH.%
%        % \\Le tok B est \the\TestEquaMBarreB.%
        \PfCMPDessineModelBarreNonHomogene{\the\TestEquaMBarreH}{\the\TestEquaMBarreB}{}%
        
        \bigskip%
        
        \PfCMPDessineModelBarreNonHomogene{\the\TestEquaMBarreH}{white,#2,"\num{\fpeval{#5-#3}}",white,2.5,"\num{\fpeval{#3}}"}{#2}%
%        
        \bigskip
        
       \EquaBaseMBarre{#2}{0}{0}{(#5-(#3))}%
      }%
    }%
  \fi%
}%

\newcommand\EquaTroisMBarre[5][]{%ax+b=cx ou ax=cx+d
  \useKVdefault[ClesEquation]%
  \setKV[ClesEquation]{#1}%
  \ifnum\fpeval{#3}=0\relax%on inverse en faisant attention à la boucle #3<->#5
    \ifnum\fpeval{#5}=0\relax%
      %% paramètre oublié
    \else
      \EquaTroisMBarre[#1]{#4}{#5}{#2}{0}%
    \fi
  \else
    \xintifboolexpr{\fpeval{#2}==0}{%b=cx
      \EquaBaseMBarre[#1]{#4}{0}{0}{#3}
    }{%
      \xintifboolexpr{\fpeval{#4}==0}{%ax+b=0
        \EquaDeuxMBarre[#1]{#2}{#3}{0}{0}
      }{%ax+b=cx
        \xintifboolexpr{\fpeval{#2}==\fpeval{#4}}{%
          \xintifboolexpr{\fpeval{#3}==0}{%ax=ax
            L'équation $\xintifboolexpr{#2==1}{}{\num{#2}}\useKV[ClesEquation]{Lettre}=\xintifboolexpr{#4==1}{}{\num{#4}}\useKV[ClesEquation]{Lettre}$ a une infinité de solutions.%
          }{%ax+b=ax
            L'équation $\xintifboolexpr{#2==1}{}{\num{#2}}\useKV[ClesEquation]{Lettre}\xintifboolexpr{#3>0}{+\num{#3}}{-\num{\fpeval{0-#3}}}=\xintifboolexpr{#4==1}{}{\num{#4}}\useKV[ClesEquation]{Lettre}$ n'a aucune solution.%
          }
        }{%% Cas délicat
          \xintifboolexpr{\fpeval{#2}>\fpeval{#4}}{%ax+b=cx avec a>c
            \TestEquaMBarreH{white}%
            \TestEquaMBarreB{white}%
            \xintFor* ##1 in{\xintSeq{1}{\fpeval{#2}}}\do{%
              \expandafter\UpdateToksBarreH{,1}\nil%
              \expandafter\UpdateToksBarreH{,"$x$"}\nil%
            }%
            \expandafter\UpdateToksBarreH{,white,2.5,"\num{\fpeval{#3}}"}\nil%
            \xintFor* ##1 in{\xintSeq{1}{\fpeval{#4}}}\do{%
              \expandafter\UpdateToksBarreB{,1}\nil%
              \expandafter\UpdateToksBarreB{,"$x$"}\nil%
            }%
            \expandafter\UpdateToksBarreB{,white,#2-#4+2.5,"0"}\nil%
            \PfCMPDessineModelBarreNonHomogene{\the\TestEquaMBarreH}{\the\TestEquaMBarreB}{}%
            
            \PfCMPDessineModelBarreNonHomogene{\the\TestEquaMBarreH}{\the\TestEquaMBarreB}{#4}%      
          
            \EquaDeuxMBarre{(#2-#4)}{#3}{0}{0}
          }{%ax+b=cx avec a<c              % Autre cas délicat
            \TestEquaMBarreH{white}%
            \TestEquaMBarreB{white}%
            \xintFor* ##1 in{\xintSeq{1}{\fpeval{#2}}}\do{%
              \expandafter\UpdateToksBarreH{,1}\nil%
              \expandafter\UpdateToksBarreH{,"$x$"}\nil%
            }%
            \expandafter\UpdateToksBarreH{,white,#4-#2,"\num{\fpeval{#3}}"}\nil%
            \xintFor* ##1 in{\xintSeq{1}{\fpeval{#4}}}\do{%
              \expandafter\UpdateToksBarreB{,1}\nil%
              \expandafter\UpdateToksBarreB{,"$x$"}\nil%
            }%
%            \expandafter\UpdateToksBarreB{,white,#2-#4+2.5,"0"}\nil%
            \PfCMPDessineModelBarreNonHomogene{\the\TestEquaMBarreH}{\the\TestEquaMBarreB}{}%
            
            \PfCMPDessineModelBarreNonHomogene{\the\TestEquaMBarreH}{\the\TestEquaMBarreB}{#2}%
            
            \EquaDeuxMBarre{0}{#3}{(#4-#2)}{0}
          }%
        }%
      }%
    }%
  \fi
}%

\newcommand\ResolEquationMBarre[5][]{%
  \useKVdefault[ClesEquation]%
  \setKV[ClesEquation]{#1}%
  \setKV[ClesEquation]{Fleches=false,FlecheDiv=false,Terme=false,Decomposition=false}
  \xintifboolexpr{#2==0}{%
    \xintifboolexpr{#4==0}{%
      \xintifboolexpr{#3==#5}{%b=d
        L'équation $\num{\fpeval{#3}}=\num{\fpeval{#5}}$ a une infinité de solutions.}%
      {%b<>d
        L'équation $\num{\fpeval{#3}}=\num{\fpeval{#5}}$ n'a aucune solution.%
      }%
    }%
    {%0x+b=cx+d$
      \EquaDeuxMBarre[#1]{#4}{#5}{#2}{#3}%
    }%
  }{%
    \xintifboolexpr{#4==0}{%ax+b=0x+d
      \EquaDeuxMBarre[#1]{#2}{#3}{0}{#5}%
    }
    {%ax+b=cx+d$
      \xintifboolexpr{#3==0}{%
        \xintifboolexpr{#5==0}{%ax=cx
          \EquaTroisMBarre[#1]{#2}{0}{#4}{0}%
        }%
        {%ax=cx+d
          \EquaTroisMBarre[#1]{#4}{#5}{#2}{0}%
        }%
      }%
      {\xintifboolexpr{#5==0}{%ax+b=cx
          \EquaTroisMBarre[#1]{#2}{#3}{#4}{0}%
        }%
        {%ax+b=cx+d -- ici
          \xintifboolexpr{#2==#4}{%
            \xintifboolexpr{#3==#5}{%b=d
                            L'équation $\xintifboolexpr{#2==1}{}{\num{#2}}\useKV[ClesEquation]{Lettre}\xintifboolexpr{#3>0}{+\num{#3}}{-\num{\fpeval{0-#3}}}=\xintifboolexpr{#4==1}{}{\num{#4}}\useKV[ClesEquation]{Lettre}\xintifboolexpr{#5>0}{+\num{#5}}{-\num{\fpeval{0-#5}}}$ a une infinité de solutions.}%
            {%b<>d
              L'équation $\xintifboolexpr{#2==1}{}{\num{#2}}\useKV[ClesEquation]{Lettre}\xintifboolexpr{#3>0}{+\num{#3}}{-\num{\fpeval{0-#3}}}=\xintifboolexpr{#4==1}{}{\num{#4}}\useKV[ClesEquation]{Lettre}\xintifboolexpr{#5>0}{+\num{#5}}{-\num{\fpeval{0-#5}}}$ n'a aucune solution.%
            }%
          }{%
            %% Cas délicat
            \xintifboolexpr{#2>#4}{%ax+b=cx+d avec a>c
              \TestEquaMBarreH{white}%
              \TestEquaMBarreB{white}%
              \xintFor* ##1 in{\xintSeq{1}{\fpeval{#2}}}\do{%
                \expandafter\UpdateToksBarreH{,1}\nil%
                \expandafter\UpdateToksBarreH{,"$x$"}\nil%
              }%
              \expandafter\UpdateToksBarreH{,white,2.5,"\num{\fpeval{#3}}"}\nil%
              \xintFor* ##1 in{\xintSeq{1}{\fpeval{#4}}}\do{%
                \expandafter\UpdateToksBarreB{,1}\nil%
                \expandafter\UpdateToksBarreB{,"$x$"}\nil%
              }%
              \expandafter\UpdateToksBarreB{,white,#2-#4+2.5,"\num{\fpeval{#5}}"}\nil%
              \PfCMPDessineModelBarreNonHomogene{\the\TestEquaMBarreH}{\the\TestEquaMBarreB}{}%
            
              \PfCMPDessineModelBarreNonHomogene{\the\TestEquaMBarreH}{\the\TestEquaMBarreB}{#4}%      
          
              \EquaDeuxMBarre{(#2-(#4))}{#3}{0}{#5}
            }{%ax+b=cx+d avec a<c              % Autre cas délicat
              \TestEquaMBarreH{white}%
              \TestEquaMBarreB{white}%
              \xintFor* ##1 in{\xintSeq{1}{\fpeval{#4}}}\do{%
                \expandafter\UpdateToksBarreB{,1}\nil%
                \expandafter\UpdateToksBarreB{,"$x$"}\nil%
              }%
              \expandafter\UpdateToksBarreB{,white,2.5,"\num{\fpeval{#5}}"}\nil%
              \xintFor* ##1 in{\xintSeq{1}{\fpeval{#2}}}\do{%
                \expandafter\UpdateToksBarreH{,1}\nil%
                \expandafter\UpdateToksBarreH{,"$x$"}\nil%
              }%
              \expandafter\UpdateToksBarreH{,white,#4-#2+2.5,"\num{\fpeval{#3}}"}\nil%
              \PfCMPDessineModelBarreNonHomogene{\the\TestEquaMBarreH}{\the\TestEquaMBarreB}{}%
            
              \PfCMPDessineModelBarreNonHomogene{\the\TestEquaMBarreH}{\the\TestEquaMBarreB}{#2}%
          
              \EquaDeuxMBarre{0}{#3}{(#4-(#2))}{#5}
            }%
          }%
        }%
      }%
    }%
  }%
}%