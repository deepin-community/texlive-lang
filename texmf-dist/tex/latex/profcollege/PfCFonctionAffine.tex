%%%
% Fonction Affine
%%%
\def\filedateAffine{2024/08/04}%
\def\fileversionAffine{0.1}%
\message{-- \filedateAffine\space v\fileversionAffine}%
%
\setKVdefault[ClesAffine]{Nom=f,Variable=x,CoefDir=a,OrdoOrig=b,Ligne=false,Image=false,Antecedent=false,Graphique=false,Retrouve=false,ProgCalcul=false,Unitex=1,Unitey=1,VoirCoef=false,ACoef=0,Redaction=false,Ecriture=false,Definition=false,Centre=false}%ACoefficient=false%: inutile ?

\newcommand\FonctionAffine[5][]{%
  % #1 nombre ou abscisse premier point
  % #2 a ou ordonn\'ee premier point
  % #3 b ou abscisse deuxi\`eme point
  % #4 {} ou ordonn\'ee deuxi\`eme point
  \useKVdefault[ClesAffine]%A supprimer car appel r\'ecursif avec Redaction
  \setKV[ClesAffine]{#1}%
  \ifboolKV[ClesAffine]{Image}{%
    \ifboolKV[ClesAffine]{Ligne}{%
      \ensuremath{\useKV[ClesAffine]{Nom}(\num{#2})=\num{#3}\times\xintifboolexpr{#2<0}{(\num{#2})}{\num{#2}}\xintifboolexpr{#4==0}{}{\xintifboolexpr{#4>0}{+\num{#4}}{-\num{\fpeval{0-#4}}}}=\num{\fpeval{#2*#3}}\xintifboolexpr{#4==0}{}{\xintifboolexpr{#4>0}{+\num{#4}}{-\num{\fpeval{0-#4}}}}\xintifboolexpr{#4==0}{}{=\num{\fpeval{#2*#3+#4}}}}%
    }{%
      \ifboolKV[ClesAffine]{ProgCalcul}{%
        \begin{align*}
          \useKV[ClesAffine]{Nom}&:\useKV[ClesAffine]{Variable}\stackrel{\times\xintifboolexpr{#3<0}{(\num{#3})}{\num{#3}}}{\longrightarrow}\num{#3}\useKV[ClesAffine]{Variable}\xintifboolexpr{#4==0}{}{\xintifboolexpr{#4>0}{\stackrel{+\num{#4}}{\longrightarrow}}{\stackrel{\num{#4}}{\longrightarrow}}\num{#3}\useKV[ClesAffine]{Variable}\xintifboolexpr{#4==0}{}{\xintifboolexpr{#4>0}{+\num{#4}}{\num{#4}}}}\\
          \useKV[ClesAffine]{Nom}&:\num{#2}\stackrel{\times\xintifboolexpr{#3<0}{(\num{#3})}{\num{#3}}}{\longrightarrow}\num{\fpeval{#3*#2}}\xintifboolexpr{#4==0}{}{\xintifboolexpr{#4>0}{\stackrel{+\num{#4}}{\longrightarrow}}{\stackrel{\num{#4}}{\longrightarrow}}\num{\fpeval{#3*#2+#4}}}
        \end{align*}    
      }{%
        \begin{align*}
          \useKV[ClesAffine]{Nom}(\num{#2})&=\num{#3}\times\xintifboolexpr{#2<0}{(\num{#2})}{\num{#2}}\xintifboolexpr{#4==0}{}{\xintifboolexpr{#4>0}{+\num{#4}}{-\num{\fpeval{0-#4}}}}\\
          \useKV[ClesAffine]{Nom}(\num{#2})&=\num{\fpeval{#3*#2}}\xintifboolexpr{#4==0}{}{\xintifboolexpr{#4>0}{+\num{#4}}{-\num{\fpeval{0-#4}}}}%\\
          \xintifboolexpr{#4==0}{}{\\
          \useKV[ClesAffine]{Nom}(\num{#2})&=\num{\fpeval{#3*#2+#4}}%\\
          }
        \end{align*}
      }%
    }%
  }{\ifboolKV[ClesAffine]{Antecedent}{%
      \ifboolKV[ClesAffine]{ProgCalcul}{%
        La fonction affine $\useKV[ClesAffine]{Nom}$ est d\'efinie par :
        \begin{align*}
          \useKV[ClesAffine]{Nom}&:\useKV[ClesAffine]{Variable}\stackrel{\times\xintifboolexpr{#3<0}{(\num{#3})}{\num{#3}}}{\longrightarrow}\num{#3}\useKV[ClesAffine]{Variable}\xintifboolexpr{#4==0}{}{\xintifboolexpr{#4>0}{\stackrel{+\num{#4}}{\longrightarrow}}{\stackrel{\num{#4}}{\longrightarrow}}\num{#3}\useKV[ClesAffine]{Variable}\xintifboolexpr{#4==0}{}{\xintifboolexpr{#4>0}{+\num{#4}}{\num{#4}}}}.
        \end{align*}
        Nous cherchons le nombre $\useKV[ClesAffine]{Variable}$ tel que son image par la fonction $\useKV[ClesAffine]{Nom}$ soit $\num{#2}$. Donc on obtient :
        \begin{align*}
          \useKV[ClesAffine]{Nom}&:\frac{\num{\fpeval{#2-#4}}}{\num{#3}}\stackrel{\div\xintifboolexpr{#3<0}{(\num{#3})}{\num{#3}}}{\longleftarrow}\num{\fpeval{#2-#4}}\xintifboolexpr{#4==0}{}{\xintifboolexpr{#4>0}{\stackrel{-\num{#4}}{\longleftarrow}}{\stackrel{+\num{\fpeval{0-#4}}}{\longleftarrow}}\num{#2}}
        \end{align*}    
      }{%
        On cherche l'ant\'ec\'edent de $\num{#2}$ par la fonction
        $\useKV[ClesAffine]{Nom}$, c'est-\`a-dire le nombre
        $\useKV[ClesAffine]{Variable}$ tel que
        $\useKV[ClesAffine]{Nom}(\useKV[ClesAffine]{Variable})=\num{#2}$. Or,
        la fonction $\useKV[ClesAffine]{Nom}$ est d\'efinie par : \[%
          \useKV[ClesAffine]{Nom}(\useKV[ClesAffine]{Variable})=\xintifboolexpr{#3==0}{}{\num{#3}\useKV[ClesAffine]{Variable}}\xintifboolexpr{#3==0}{\num{#4}}{\xintifboolexpr{#4==0}{}{\xintifboolexpr{#4>0}{+\num{#4}}{-\num{\fpeval{0-#4}}}}}.
        \]
        Par cons\'equent, on a :
        \begin{align*}
            \num{#3}\useKV[ClesAffine]{Variable}\xintifboolexpr{#4==0}{}{\xintifboolexpr{#4>0}{+\num{#4}}{-\num{\fpeval{0-#4}}}}&=\num{#2}\\
          \xintifboolexpr{#4==0}{\useKV[ClesAffine]{Variable}\uppercase{&}=\frac{\num{#2}}{\num{#3}}%\\
          }{\num{#3}\useKV[ClesAffine]{Variable}&=\num{\fpeval{#2-#4}}\\
          \useKV[ClesAffine]{Variable}&=\frac{\num{\fpeval{#2-#4}}}{\num{#3}}%\\
          }
        \end{align*}
      }%
    }{%
      \ifboolKV[ClesAffine]{Retrouve}{%
          \xdef\OrdOrigine{\fpeval{#3-(#3-#5)*#2/(#2-#4)}}%%
        On sait que $\useKV[ClesAffine]{Nom}$ est une fonction affine. Donc elle s'\'ecrit sous la forme : \[\useKV[ClesAffine]{Nom}(\useKV[ClesAffine]{Variable})=\useKV[ClesAffine]{CoefDir}\useKV[ClesAffine]{Variable}+\useKV[ClesAffine]{OrdoOrig}.\]
        Or, $\useKV[ClesAffine]{Nom}(\num{#2})=\num{#3}$ et $\useKV[ClesAffine]{Nom}(\num{#4})=\num{#5}$. \ifnum\fpeval{#3-#5}=0
          Les images de \num{#2} et \num{#4} sont égales à \num{#3}. La fonction affine cherchée est donc constante et s'écrit :
          \[\useKV[ClesAffine]{Nom}:\useKV[ClesAffine]{Variable}\mapsto\num{\OrdOrigine}.\]
        \else
          Par cons\'equent, d'apr\`es la propri\'et\'e des accroissements :
        \begin{align*}
          \useKV[ClesAffine]{CoefDir}&=\frac{\useKV[ClesAffine]{Nom}(\num{#2})-\useKV[ClesAffine]{Nom}(\num{#4})}{\num{#2}-\xintifboolexpr{#4<0}{(\num{#4})}{\num{#4}}}\\
          \useKV[ClesAffine]{CoefDir}&=\frac{\num{#3}-\xintifboolexpr{#5<0}{(\num{#5})}{\num{#5}}}{\num{\fpeval{#2-#4}}}\\
          \useKV[ClesAffine]{CoefDir}&=\frac{\num{\fpeval{#3-#5}}}{\num{\fpeval{#2-#4}}}%\\
          \SSimpliTest{\fpeval{#3-#5}}{\fpeval{#2-#4}}\ifthenelse{\boolean{Simplification}}{\\\useKV[ClesAffine]{CoefDir}&=\SSimplifie{\fpeval{#3-#5}}{\fpeval{#2-#4}}}{}%
        \end{align*}
        \def\PfCQuotientAffine{\fpeval{(#3-#5)/(#2-#4)}}%
        La fonction $\useKV[ClesAffine]{Nom}$ s'\'ecrit alors sous la forme $\displaystyle\useKV[ClesAffine]{Nom}(\useKV[ClesAffine]{Variable})=\ifnum\fpeval{abs(\PfCQuotientAffine)}=1\relax\Affichage[Lettre={\useKV[ClesAffine]{Variable}}]{0}{\PfCQuotientAffine}{0}\else\SSimplifie{\fpeval{#3-#5}}{\fpeval{#2-#4}}\useKV[ClesAffine]{Variable}\fi+\useKV[ClesAffine]{OrdoOrig}$.
        \\De plus, comme $\useKV[ClesAffine]{Nom}(\num{#2})=\num{#3}$, alors :
        \begin{align*}
          \ifnum\PfCQuotientAffine=1\relax\else\SSimplifie{\fpeval{#3-#5}}{\fpeval{#2-#4}}\times\xintifboolexpr{#2<0}{(\num{#2})}{\num{#2}}+\useKV[ClesAffine]{OrdoOrig}\uppercase{&}=\num{#3}\\\fi
          \SSimplifie{\fpeval{(#3-#5)*#2}}{\fpeval{#2-#4}}+\useKV[ClesAffine]{OrdoOrig}&=\num{#3}\\
          \useKV[ClesAffine]{OrdoOrig}&=\num{\fpeval{#3-(#3-#5)*#2/(#2-#4)}}
        \end{align*}
        La fonction affine $\useKV[ClesAffine]{Nom}$ cherch\'ee est :
        \[\useKV[ClesAffine]{Nom}:\useKV[ClesAffine]{Variable}\mapsto\ifnum\fpeval{abs(\PfCQuotientAffine)}=1\relax\Affichage[Lettre={\useKV[ClesAffine]{Variable}}]{0}{\PfCQuotientAffine}{\OrdOrigine}\else\SSimplifie{\fpeval{#3-#5}}{\fpeval{#2-#4}}\useKV[ClesAffine]{Variable}\xintifboolexpr{\OrdOrigine==0}{}{\xintifboolexpr{\OrdOrigine>0}{+\num{\OrdOrigine}}{-\num{\fpeval{0-\OrdOrigine}}}}\fi.\]
        \fi
      }{%
        %
      }%
    }%
  }%
  \ifboolKV[ClesAffine]{Graphique}{%
    \ifboolKV[ClesAffine]{VoirCoef}{%
      \MPFonctionAffine{\useKV[ClesAffine]{Unitex}}{\useKV[ClesAffine]{Unitey}}{#2}{#3}{#4}{#5}{\useKV[ClesAffine]{ACoef}}%
    }{%
      \MPFonctionAffine{\useKV[ClesAffine]{Unitex}}{\useKV[ClesAffine]{Unitey}}{#2}{#3}{#4}{#5}{""}}{}%
  }{}%
  \ifboolKV[ClesAffine]{Redaction}{%
    \xintifboolexpr{#2==0}{Comme la fonction $\useKV[ClesAffine]{Nom}$
      est une fonction constante, alors sa repr\'esentation graphique est une droite parall\`ele \`a l'axe des abscisses passant par le point de coordonn\'ees $(0;\num{#3})$.}%
    {\xintifboolexpr{#3==0}{Comme la fonction $\useKV[ClesAffine]{Nom}$ est une fonction lin\'eaire, alors sa repr\'esentation graphique est une droite passant par l'origine du rep\`ere.\\Je choisis $\useKV[ClesAffine]{Variable}=\num{#4}$. Son image est \xdef\NomFonctionA{\useKV[ClesAffine]{Nom}}\FonctionAffine[Nom=\NomFonctionA,Image,Ligne]{#4}{#2}{#3}{#5}. On place le point de coordonn\'ees $(\num{#4}\,;\,\num{\fpeval{#2*#4+#3}})$.
      }{%
        Comme $\useKV[ClesAffine]{Nom}$ est une fonction affine, alors sa repr\'esentation graphique est une droite.\\Je choisis $\useKV[ClesAffine]{Variable}=\num{#4}$. Son image est %
        \xdef\NomVariable{\useKV[ClesAffine]{Variable}}\xdef\NomFonction{\useKV[ClesAffine]{Nom}}%
        \ifboolKV[ClesAffine]{Centre}{: \[
           \FonctionAffine[Nom=\NomFonction,Image,Ligne]{#4}{#2}{#3}{#5}.
         \]}{%
         \FonctionAffine[Nom=\NomFonction,Image,Ligne]{#4}{#2}{#3}{#5}. %
       }%
       On place le point de coordonn\'ees $(\num{#4}\,;\,\num{\fpeval{#2*#4+#3}})$.\\Je choisis \setKV[ClesAffine]{Variable=\NomVariable}$\useKV[ClesAffine]{Variable}=\num{#5}$. Son image est %
       \ifboolKV[ClesAffine]{Centre}{: \[
           \FonctionAffine[Nom=\NomFonction,Image,Ligne]{#5}{#2}{#3}{#5}.
         \]}{%
         \FonctionAffine[Nom=\NomFonction,Image,Ligne]{#5}{#2}{#3}{#5}. %
       }%
       On place le point de coordonn\'ees $(\num{#5}\,;\,\num{\fpeval{#2*#5+#3}})$.%        
      }%
    }%
  }%
  {}%
  \ifboolKV[ClesAffine]{Ecriture}{\ensuremath{\useKV[ClesAffine]{Nom}(\useKV[ClesAffine]{Variable})=\xintifboolexpr{#2==0}{}{\num{#2}\useKV[ClesAffine]{Variable}}\xintifboolexpr{#2==0}{\num{#3}}{\xintifboolexpr{#3==0}{}{\xintifboolexpr{#3>0}{+\num{#3}}{-\num{\fpeval{0-#3}}}}}}}{}%
  \ifboolKV[ClesAffine]{Definition}{\ensuremath{\useKV[ClesAffine]{Nom}:\useKV[ClesAffine]{Variable}\mapsto\xintifboolexpr{#2==0}{}{\num{#2}\useKV[ClesAffine]{Variable}}\xintifboolexpr{#2==0}{\num{#3}}{\xintifboolexpr{#3==0}{}{\xintifboolexpr{#3>0}{+\num{#3}}{-\num{\fpeval{0-#3}}}}}}}{}%
}%

\def\MPFonctionAffine#1#2#3#4#5#6#7{%
  % #1 Unitex #2 Unitey
  % #3 a pour f1 - #4 b pour f1
  % #5 abscisse du premier point
  % #6 abscisse du deuxi\`eme point
  % #7 on affiche ou pas la lecture du coef directeur
  \ifluatex
   \mplibforcehmode
  \begin{mplibcode}
    XMin=-2;
    XMax=2;
    if #5<XMin:
    XMin:=#5;
    fi;
    if #6<XMin:
    XMin:=#6;
    fi;
    if #5>XMax:
    XMax:=#5;
    fi;
    if #6>XMax:
    XMax:=#6;
    fi;
    YMax=2;
    YMin=-2;
    if (#5*#3+(#4))>YMax:
    YMax:=(#5*#3+(#4));
    fi;
    if (#6*#3+(#4))>YMax:
    YMax:=(#6*#3+(#4));
    fi;
    if (#5*#3+(#4))<YMin:
    YMin:=(#5*#3+(#4));
    fi;
    if (#6*#3+(#4))<YMin:
    YMin:=(#6*#3+(#4));
    fi;
    unitex:=#1*cm;
    unitey:=#2*cm;
    XMax:=XMax+2;
    XMin:=XMin-2;
    YMax:=YMax+2;
    YMin:=YMin-2;
    %On trace la grille
    drawoptions(withcolor 0.95white);
    for k=0 upto (XMax-XMin):
    draw ((XMin+k)*unitex,YMin*unitey)--((XMin+k)*unitex,YMax*unitey);
    endfor;
    for k=0 upto (YMax-YMin):
    draw (XMin*unitex,(YMin+k)*unitey)--(XMax*unitex,(YMin+k)*unitey);
    endfor;
    drawoptions();
    %On trace les axes
    drawarrow (XMin*unitex,0)--(XMax*unitex,0);
    drawarrow (0,YMin*unitey)--(0,YMax*unitey);
    label.llft(TEX("\footnotesize O"),(0,0));
    dotlabel.lrt(TEX("1"),(unitex,0));
    dotlabel.ulft(TEX("1"),(0,unitey));
    % On trace la droite
    pair A[];
    A1=(#5*unitex,(#5*#3+(#4))*unitey);
    A2=(#6*unitex,(#6*#3+(#4))*unitey);
    draw 10[A1,A2]--10[A2,A1];
    clip currentpicture to ((XMin*unitex,YMin*unitey)--(XMax*unitex,YMin*unitey)--(XMax*unitex,YMax*unitey)--(XMin*unitex,YMax*unitey)--cycle);
    %On labellise les points
    fill (fullcircle scaled 1mm) shifted A1;
    fill (fullcircle scaled 1mm) shifted A2;
    draw (xpart(A1),0)--A1--(0,ypart(A1)) dashed evenly;
    draw (xpart(A2),0)--A2--(0,ypart(A2)) dashed evenly;
    if (#5*#3+(#4))=0:
    else:
    if #5<>1:
    if (#5*#3+(#4))<0:
    label.top(TEX("\num{"&decimal(#5)&"}"),(xpart(A1),0));
    else:
    label.bot(TEX("\num{"&decimal(#5)&"}"),(xpart(A1),0));
    fi;
    fi;
    fi;
    if (#6*#3+(#4))=0:
    else:
    if (#6*#3+(#4))<0:
    if #6<>1:
    label.top(TEX("\num{"&decimal(#6)&"}"),(xpart(A2),0));
    fi;
    else:
    if #6<>1:
    label.bot(TEX("\num{"&decimal(#6)&"}"),(xpart(A2),0));
    fi;
    fi;
    fi;
    if #3=0:
    label.urt(TEX("\num{"&decimal(#5*#3+(#4))&"}"),(0,ypart(A1)));
    else:
    if #3>0:
    if (#5*#3+(#4))=0:
    else:
    if (#5*#3+(#4))<0:
    if #5<0:
    label.rt(TEX("\num{"&decimal(#5*#3+(#4))&"}"),(0,ypart(A1)));
    else:
    label.lft(TEX("\num{"&decimal(#5*#3+(#4))&"}"),(0,ypart(A1)));
    fi;
    fi;
    fi;
    if (#6*#3+(#4))=0:
    else:
    if (#6*#3+(#4))<0:
    if #6<0:
    label.rt(TEX("\num{"&decimal(#6*#3+(#4))&"}"),(0,ypart(A2)));
    else:
    label.lft(TEX("\num{"&decimal(#6*#3+(#4))&"}"),(0,ypart(A2)));
    fi;
    fi;
    fi;
    else:
    if (#5*#3+(#4))=0:
    else:
    if (#5*#3+(#4))<0:
    label.lft(TEX("\num{"&decimal(#5*#3+(#4))&"}"),(0,ypart(A1)));
    else:
    label.rt(TEX("\num{"&decimal(#5*#3+(#4))&"}"),(0,ypart(A1)));
    fi;
    fi;
    if (#6*#3+(#4))=0:
    else:
    if (#6*#3+(#4))<0:
    label.lft(TEX("\num{"&decimal(#6*#3+(#4))&"}"),(0,ypart(A2)));
    else:
    label.rt(TEX("\num{"&decimal(#6*#3+(#4))&"}"),(0,ypart(A2)));
    fi;
    fi;
    fi;
    fi;
    % On affiche ou pas "la marche" du coef directeur
    for p_=#7:
    if numeric p_:
    draw ((#7*unitex,(#7*#3+(#4))*unitey)--((#7+1)*unitex,(#7*#3+(#4))*unitey)--((#7+1)*unitex,((#7+1)*#3+(#4))*unitey)) withcolor red;
    fi;
    endfor;
  \end{mplibcode}
  \else
  \begin{mpost}
    % On d\'efinit les constantes
    XMin=-2;
    XMax=2;
    if #5<XMin:
    XMin:=#5;
    fi;
    if #6<XMin:
    XMin:=#6;
    fi;
    if #5>XMax:
    XMax:=#5;
    fi;
    if #6>XMax:
    XMax:=#6;
    fi;
    YMax=2;
    YMin=-2;
    if (#5*#3+(#4))>YMax:
    YMax:=(#5*#3+(#4));
    fi;
    if (#6*#3+(#4))>YMax:
    YMax:=(#6*#3+(#4));
    fi;
    if (#5*#3+(#4))<YMin:
    YMin:=(#5*#3+(#4));
    fi;
    if (#6*#3+(#4))<YMin:
    YMin:=(#6*#3+(#4));
    fi;
    unitex:=#1*cm;
    unitey:=#2*cm;
    XMax:=XMax+2;
    XMin:=XMin-2;
    YMax:=YMax+2;
    YMin:=YMin-2;
    %On trace la grille
    drawoptions(withcolor 0.95white);
    for k=0 upto (XMax-XMin):
    draw ((XMin+k)*unitex,YMin*unitey)--((XMin+k)*unitex,YMax*unitey);
    endfor;
    for k=0 upto (YMax-YMin):
    draw (XMin*unitex,(YMin+k)*unitey)--(XMax*unitex,(YMin+k)*unitey);
    endfor;
    drawoptions();
    %On trace les axes
    drawarrow (XMin*unitex,0)--(XMax*unitex,0);
    drawarrow (0,YMin*unitey)--(0,YMax*unitey);
    label.llft(btex O etex,(0,0));
    dotlabel.bot(btex 1 etex,(unitex,0));
    dotlabel.lft(btex 1 etex,(0,unitey));
    % On trace la droite
    pair A[];
    A1=(#5*unitex,(#5*#3+(#4))*unitey);
    A2=(#6*unitex,(#6*#3+(#4))*unitey);
    draw 10[A1,A2]--10[A2,A1];
    clip currentpicture to ((XMin*unitex,YMin*unitey)--(XMax*unitex,YMin*unitey)--(XMax*unitex,YMax*unitey)--(XMin*unitex,YMax*unitey)--cycle);
    %On labellise les points
    fill (fullcircle scaled 1mm) shifted A1;
    fill (fullcircle scaled 1mm) shifted A2;
    draw (xpart(A1),0)--A1--(0,ypart(A1)) dashed evenly;
    draw (xpart(A2),0)--A2--(0,ypart(A2)) dashed evenly;
    if (#5*#3+(#4))=0:
    else:
    if (#5*#3+(#4))<0:
    label.top(LATEX("\num{"&decimal(#5)&"}"),(xpart(A1),0));
    else:
    label.bot(LATEX("\num{"&decimal(#5)&"}"),(xpart(A1),0));
    fi;
    fi;
    if (#6*#3+(#4))=0:
    else:
    if (#6*#3+(#4))<0:
    label.top(LATEX("\num{"&decimal(#6)&"}"),(xpart(A2),0));
    else:
    label.bot(LATEX("\num{"&decimal(#6)&"}"),(xpart(A2),0));
    fi;
    fi;
    if #3=0:
    label.urt(LATEX("\num{"&decimal(#5*#3+(#4))&"}"),(0,ypart(A1)));
    else:
    if #3>0:
    if (#5*#3+(#4))=0:
    else:
    if (#5*#3+(#4))<0:
    label.rt(LATEX("\num{"&decimal(#5*#3+(#4))&"}"),(0,ypart(A1)));
    else:
    label.lft(LATEX("\num{"&decimal(#5*#3+(#4))&"}"),(0,ypart(A1)));
    fi;
    fi;
    if (#6*#3+(#4))=0:
    else:
    if (#6*#3+(#4))<0:
    label.rt(LATEX("\num{"&decimal(#6*#3+(#4))&"}"),(0,ypart(A2)));
    else:
    label.lft(LATEX("\num{"&decimal(#6*#3+(#4))&"}"),(0,ypart(A2)));
    fi;
    fi;
    else:
    if (#5*#3+(#4))=0:
    else:
    if (#5*#3+(#4))<0:
    label.lft(LATEX("\num{"&decimal(#5*#3+(#4))&"}"),(0,ypart(A1)));
    else:
    label.rt(LATEX("\num{"&decimal(#5*#3+(#4))&"}"),(0,ypart(A1)));
    fi;
    fi;
    if (#6*#3+(#4))=0:
    else:
    if (#6*#3+(#4))<0:
    label.lft(LATEX("\num{"&decimal(#6*#3+(#4))&"}"),(0,ypart(A2)));
    else:
    label.rt(LATEX("\num{"&decimal(#6*#3+(#4))&"}"),(0,ypart(A2)));
    fi;
    fi;
    fi;
    fi;
    % On affiche ou pas "la marche" du coef directeur
    for p_=#7:
    if numeric p_:
    draw ((#7*unitex,(#7*#3+(#4))*unitey)--((#7+1)*unitex,(#7*#3+(#4))*unitey)--((#7+1)*unitex,((#7+1)*#3+(#4))*unitey)) withcolor red;
    fi;
    endfor;
  \end{mpost}
  \fi
}%