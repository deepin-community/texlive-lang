%%%
% Fraction Nombre
%%%
\newtoks\tokFractionNombrePerso%
\def\UpdatetoksFN#1\nil{\addtotok\tokFractionNombrePerso{"#1",}}%

\setKVdefault[ClesFN]{Unite=5cm,Aire=false,Solution=false,Quadrillage=false,PasQuad=5mm,HautQuad=3,LargQuad=8,Vide=false,Creation=false,Couleur=blue,UniteAire=1x1,Pointilles=false,Tentatives=10}%
\defKV[ClesFN]{Perso=\setKV[ClesFN]{Creation}\xdef\PfCFooCheminFN{#1}}%

\NewDocumentCommand\FractionNombre{om}{%
  \useKVdefault[ClesFN]%
  \setKV[ClesFN]{#1}%
  \tokFractionNombrePerso{}%
  \ifboolKV[ClesFN]{Creation}{%
    \setsepchar{,}\ignoreemptyitems%
    \readlist*\PfCListeDeplacements{\PfCFooCheminFN}%
    \reademptyitems%
    \foreachitem\compteur\in\PfCListeDeplacements{\expandafter\UpdatetoksFN\compteur\nil}%
  }{}%
  \StrCut{#2}{/}\PfCNumA\PfCNumB%
  \ifboolKV[ClesFN]{Aire}{%
    \StrCut{\useKV[ClesFN]{UniteAire}}{x}\PfCAirex\PfCAirey%
    \ifboolKV[ClesFN]{Creation}{%
        \BuildFractionNombreAirePerso{\PfCNumA}{\PfCNumB}{\PfCAirex}{\PfCAirey}{\the\tokFractionNombrePerso}%
      }{%
      \BuildFractionNombreAire{\PfCNumA}{\PfCNumB}{\PfCAirex}{\PfCAirey}%
    }%
  }{%
    \ifboolKV[ClesFN]{Quadrillage}{%
      \BuildFractionNombreQuadrillage{\PfCNumA}{\PfCNumB}{\the\tokFractionNombrePerso}%
    }{%
      \BuildFractionNombreSegment{\PfCNumA}{\PfCNumB}%
    }%
  }%
}%

\NewDocumentCommand\BuildFractionNombreAirePerso{mmmmm}{%
  \ifluatex%
    \mplibforcehmode%
    \begin{mplibcode}
      vardef LireMotif(text t)=
      lmotif=0;
      hmotif=0;
      for p_=t:
      hmotif:=hmotif+1;
      if lmotif<(length p_):
      lmotif:=length p_;
      fi;
      endfor;
      enddef;
      %
      vardef PlacerMotif(text t)=
      ChoixLigneD=hmotif+ceiling(uniformdeviate(HautQuad-1-hmotif));
      ChoixColonneD=ceiling(uniformdeviate(LargQuad-#3-1-lmotif));
      Colonne:=ChoixColonneD;
      Ligne=ChoixLigneD;
      for p_=t:
      Colonne:=ChoixColonneD;
      for k=0 upto ((length p_)-1):
      if (substring(k,k+1) of p_)="x":
      fill ((unitsquare scaled 5mm) shifted (5mm*(Colonne-1,Ligne-1))) withcolor LightSteelBlue;
      fi;
      Colonne:=Colonne+1;
      endfor;
      Ligne:=Ligne-1;
      endfor;
      enddef;
      %
      PasQuad=\useKV[ClesFN]{PasQuad};
      HautQuad=\useKV[ClesFN]{HautQuad};
      LargQuad=\useKV[ClesFN]{LargQuad};
      pair A,B,C,D,U[];
      A=(0,0);
      B-A=PasQuad*(LargQuad,0);
      C-B=PasQuad*(0,HautQuad);
      D-C=A-B;
      % l'unité
      U0=C;
      U1-U0=#3*PasQuad*(-1,0);
      U3-U0=#4*PasQuad*(0,-1);
      U2-U3=U1-U0;
      %
      LireMotif(#5);
      PlacerMotif(#5);
      % Tracage
      fill polygone(U0,U1,U2,U3) withcolor 0.7white;
      label.top(TEX("\textbf{1 ua}"),iso(U1,U0));
      %Le quadrillage
      trace polygone(A,B,C,D);
      for k=1 upto LargQuad-1:
      trace (k/LargQuad)[A,B]--(k/LargQuad)[D,C];
      endfor;
      for k=1 upto HautQuad-1:
      trace (k/HautQuad)[A,D]--(k/HautQuad)[B,C];
      endfor;
    \end{mplibcode}
  \fi%
}%

\NewDocumentCommand\BuildFractionNombreAire{mmmm}{%
  \ifluatex%
    \mplibforcehmode%
    \begin{mplibcode}
      boolean Impossible[][];
      %
      vardef InitialisationZone=
      for k=-1 upto HautQuad+1:
      for l=-1 upto LargQuad+1:
      Impossible[k][l]=true;
      endfor;
      endfor;
      for k=1 upto HautQuad-1:
      for l=1 upto LargQuad-#3-1:
      Impossible[k][l]:=false;
      endfor;
      endfor;
      enddef;
      %
      vardef RAZZone=
      pair PileChemin[];
      indiceChemin:=0;
      enddef;
      % 
      def PushChemin(expr tt)=
      if indiceChemin<1:
      PileChemin[1]:=tt;
      indiceChemin:=1;
      else:
      PileChemin[indiceChemin+1]:=tt;
      indiceChemin:=indiceChemin+1;
      fi;
      enddef;
      %
      def PopChemin=
      tentative:=tentative+1;
      currentpicture:=nullpicture;
      RAZZone;
      for k=-1 upto HautQuad:
      for l=-1 upto LargQuad:
      Impossible[k][l]:=true;
      endfor;
      endfor;
      for k=1 upto HautQuad-1:
      for l=1 upto LargQuad-#3-1:
      Impossible[k][l]:=false;
      endfor;
      endfor;
      %Redemarrage
      nbcaseschoisies:=1;
      PushChemin((ChoixLigneD,ChoixColonneD));
      fill ((unitsquare scaled 5mm) shifted (5mm*(ChoixColonneD-1,ChoixLigneD-1))) withcolor LightSteelBlue;
      Impossible[ChoixLigneD][ChoixColonneD]:=true;
      VoisinDispo(ChoixLigneD,ChoixColonneD);
      enddef;
      %
      %Pile des cases voisines de la case parcourue
      pair PileVoisin[];
      numeric indiceVoisin;
      indiceVoisin=0;
      % 
      vardef RAZPileVoisin=
      indiceVoisin:=0;
      enddef;
      %
      def PushVoisin(expr tt)=
      if indiceVoisin<1:
      PileVoisin[1]:=tt;
      indiceVoisin:=1;
      else:
      PileVoisin[indiceVoisin+1]:=tt;
      indiceVoisin:=indiceVoisin+1;
      fi;
      enddef;
      % 
      vardef VoisinDispo(expr la,lo)=
      RAZPileVoisin;
      numeric nbvoisin;
      nbvoisin=0;
      if Impossible[la+1][lo]=false:
      nbvoisin:=nbvoisin+1;
      PushVoisin((la+1,lo));
      fi;
      if Impossible[la-1][lo]=false:
      nbvoisin:=nbvoisin+1;
      PushVoisin((la-1,lo));
      fi;
      if Impossible[la][lo+1]=false:
      nbvoisin:=nbvoisin+1;
      PushVoisin((la,lo+1));
      fi;
      if Impossible[la][lo-1]=false:
      nbvoisin:=nbvoisin+1;
      PushVoisin((la,lo-1));
      fi;
      enddef;
      %
      boolean Vide;
      Vide=\useKV[ClesFN]{Vide};
      PasQuad=\useKV[ClesFN]{PasQuad};
      HautQuad=\useKV[ClesFN]{HautQuad};
      LargQuad=\useKV[ClesFN]{LargQuad};
      Tentatives=\useKV[ClesFN]{Tentatives};
      pair A,B,C,D,U[];
      A=(0,0);
      B-A=PasQuad*(LargQuad,0);
      C-B=PasQuad*(0,HautQuad);
      D-C=A-B;
      % l'unité
      U0=C;
      U1-U0=#3*PasQuad*(-1,0);
      U3-U0=#4*PasQuad*(0,-1);
      U2-U3=U1-U0;
      % Le nombres total de cases à colorier
      NbTotalCasesAColorier=((#1*#3*#4)/#2);
      NbTotalCasesRestantes=HautQuad*(LargQuad-#3-1);
      %
      InitialisationZone;
      RAZZone;
      %
      ChoixLigneD=ceiling(uniformdeviate(HautQuad-1));
      ChoixColonneD=ceiling(uniformdeviate(LargQuad-#3-1));
      %
      tentative:=1;
      if Vide=false:
      PushChemin((ChoixLigneD,ChoixColonneD));
      nbcaseschoisies=1;
      fill ((unitsquare scaled 5mm) shifted (5mm*(ChoixColonneD-1,ChoixLigneD-1))) withcolor LightSteelBlue;
      Impossible[ChoixLigneD][ChoixColonneD]:=true;
      VoisinDispo(ChoixLigneD,ChoixColonneD);
      % Le "parcours"
      forever: exitif (nbcaseschoisies>NbTotalCasesAColorier-1) or (tentative>Tentatives);
      nb:=ceiling(uniformdeviate(nbvoisin));
      if nb>0:
      for k=1 upto nbvoisin:
      Impossible[xpart(PileVoisin[nb])][ypart(PileVoisin[nb])]:=true;
      endfor;
      PushChemin((xpart(PileVoisin[nb]),ypart(PileVoisin[nb])));
      nbcaseschoisies:=nbcaseschoisies+1;
      fill ((unitsquare scaled 5mm) shifted (5mm*(ypart(PileVoisin[nb])-1,xpart(PileVoisin[nb])-1))) withcolor LightSteelBlue;
      VoisinDispo(xpart(PileChemin[indiceChemin]),ypart(PileChemin[indiceChemin]));
      else:
      PopChemin;
      fi;
      endfor;
      fi;
      % Tracage
      if tentative>Tentatives:
      currentpicture:=nullpicture;
      picture MiseEnGarde;
      MiseEnGarde=image(
      label(TEX("\begin{minipage}{0.5\linewidth}
        La construction automatique a échouée. Relancez en augmentant le nombre de tentatives ou en augmentant les dimensions du quadrillage (Hauteur,Largeur)
      \end{minipage}"),(0,0));
      );
      draw (((llcorner MiseEnGarde)--(lrcorner MiseEnGarde)--(urcorner MiseEnGarde)--(ulcorner MiseEnGarde)--cycle) scaled 1.1) withpen pensquare scaled 2;
      draw MiseEnGarde;
      else:
      fill polygone(U0,U1,U2,U3) withcolor 0.7white;
      label.top(TEX("\textbf{1 ua}"),iso(U1,U0));
      %Le quadrillage
      trace polygone(A,B,C,D);
      for k=1 upto LargQuad-1:
      trace (k/LargQuad)[A,B]--(k/LargQuad)[D,C];
      endfor;
      for k=1 upto HautQuad-1:
      trace (k/HautQuad)[A,D]--(k/HautQuad)[B,C];
      endfor;
      fi;
    \end{mplibcode}
  \fi
}%

\NewDocumentCommand\BuildFractionNombreSegment{mm}{%
  \ifluatex%
    \mplibforcehmode%
    \begin{mplibcode}
      Unite:=\useKV[ClesFN]{Unite};
      boolean Solution,Vide;
      Solution=\useKV[ClesFN]{Solution};
      Vide=\useKV[ClesFN]{Vide};
      color CoulFN;
      CoulFN=\useKV[ClesFN]{Couleur};
      Num:=#1;
      Pas:=#2;
      pair A,B,C,D;
      A=(0,0);
      B=Unite*(1,0);
      if Num>Pas:
      C=((Num+1)/Pas)[A,B];
      Total=Num+1;
      else:
      C=B;
      Total=Pas;
      fi;
      D=(Num/Pas)[A,B];      
      trace segment(A,C);
      for k=0 upto Total:
      trace ((0,-1mm)--(0,1mm)) shifted ((k/Pas)[A,B]);
      endfor;
      trace cotationmil(A,B,5mm,15,TEX("1 ul"));
      if Solution:
      draw (A--D) withpen pencircle scaled 2 withcolor CoulFN;
      trace cotationmil(A,D,-5mm,25,TEX("$?=\dfrac{"&decimal(#1)&"}{"&decimal(#2)&"}$~ul"));
      else:
      if Vide:
      else:
      trace cotationmil(A,D,-5mm,25,TEX("$?=\dots$~ul"));
      fi;
      fi;
    \end{mplibcode}
  \fi%
}%

\NewDocumentCommand\BuildFractionNombreQuadrillage{mmm}{%
  \ifluatex%
    \mplibforcehmode%
    \begin{mplibcode}
      Unite:=#2;%\useKV[ClesFN]{Unite};%en carreaux
      HautQuad:=\useKV[ClesFN]{HautQuad};
      Largeur:=Unite;%en carreaux
      boolean Solution,Vide,Creation,Pointilles;
      Solution=\useKV[ClesFN]{Solution};
      Vide=\useKV[ClesFN]{Vide};
      Creation=\useKV[ClesFN]{Creation};
      Pointilles=\useKV[ClesFN]{Pointilles};
      color CoulFN;
      CoulFN=\useKV[ClesFN]{Couleur};
      Num:=#1;
      Pas:=#2;
      PasQuad:=\useKV[ClesFN]{PasQuad};
      pair A,B,C,D;
      A=(0,0);
      B=(((#1 div #2)+1)*Unite+1)*PasQuad*(1,0);
      C-B=(0,HautQuad*PasQuad);
      D-C=A-B;
      trace polygone(A,B,C,D);
      TotalLargeur=(((#1 div #2)+1)*Unite+1);
      for k=1 upto TotalLargeur-1:
      trace (k/TotalLargeur)[A,B]--(k/TotalLargeur)[D,C];
      endfor;
      for k=1 upto HautQuad-1:
      trace (k/HautQuad)[A,D]--(k/HautQuad)[B,C];
      endfor;
      % Macro pour le chemin perso
      vardef RemplirChemin(text t)=
      nbpoints:=0;
      % On compte le dénivelé
      denivele=0;
      denivmax:=0;
      denivmin:=0;
      for p_=t:
      if p_="+":
      denivmax:=denivmax+1;
      elseif p_="-":
      denivmin:=denivmin-1;
      fi;
      endfor;
      depart:=-(denivmin)+floor(uniformdeviate(HautQuad-2-abs(denivmax-abs(denivmin))));
      ch[0]=PasQuad*(0,depart);
      for p_=t:
      nbpoints:=nbpoints+1;
      if p_=">":
      ch[nbpoints]=ch[nbpoints-1]+PasQuad*(1,0);
      elseif p_="<":
      ch[nbpoints]=ch[nbpoints-1]+PasQuad*(-1,0);
      elseif p_="+":
      ch[nbpoints]=ch[nbpoints-1]+PasQuad*(0,1);
      elseif p_="-":
      ch[nbpoints]=ch[nbpoints-1]+PasQuad*(0,-1);
      fi;
      endfor;
      enddef;
      % On choisit le départ.
      pair ch[];
      numeric xe[],ye[];
      if Creation:
      RemplirChemin(#3);
      else:
      depart:=floor(uniformdeviate(HautQuad-1));
      ch[0]=PasQuad*(0,depart);
      xe0=0;
      ye0=depart;
      for k=1 upto #1:
      if (k mod 2)=1:
      ch[k]-ch[k-1]=PasQuad*(1,0);
      xe[k]=xe[k-1]+1;
      ye[k]=ye[k-1];
      else:
      if ye[k-1]=0:
      alea:=uniformdeviate(2);
      if alea<1:
      ch[k]-ch[k-1]=PasQuad*(0,1);
      xe[k]=xe[k-1];
      ye[k]=ye[k-1]+1;
      else:
      ch[k]-ch[k-1]=PasQuad*(1,0);
      xe[k]=xe[k-1]+1;
      ye[k]=ye[k-1];
      fi;
      elseif ye[k-1]=2:
      alea:=uniformdeviate(2);
      if alea<1:
      ch[k]-ch[k-1]=PasQuad*(0,-1);
      xe[k]=xe[k-1];
      ye[k]=ye[k-1]-1;
      else:
      ch[k]-ch[k-1]=PasQuad*(1,0);
      xe[k]=xe[k-1]+1;
      ye[k]=ye[k-1];
      fi;
      else:
      alea:=uniformdeviate(3);
      if alea<1:
      ch[k]-ch[k-1]=PasQuad*(0,1);
      xe[k]=xe[k-1];
      ye[k]=ye[k-1]+1;
      elseif alea<2:
      ch[k]-ch[k-1]=PasQuad*(1,0);
      xe[k]=xe[k-1]+1;
      ye[k]=ye[k-1];
      else:
      ch[k]-ch[k-1]=PasQuad*(0,-1);
      xe[k]=xe[k-1];
      ye[k]=ye[k-1]-1;
      fi;
      fi;      
      fi;
      endfor;
      fi;
      if Vide=false:
      trace ch[0] for k=1 upto #1:--ch[k] endfor withpen pencircle scaled 2 if Pointilles: dashed evenly else: withcolor CoulFN fi;
      fi;
      trace cotationmil(D,(D+#2*(PasQuad,0)),5mm,15,TEX("1 ul"));
      trace segment(D,D+#2*(PasQuad,0)) withpen pencircle scaled 1.5;
    \end{mplibcode}
  \fi
}%
