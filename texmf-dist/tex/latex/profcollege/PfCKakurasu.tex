%%%
% Kakurasu
%%%
\def\filedateKakurasu{2024/08/04}%
\def\fileversionKakurasu{0.1}%
\message{-- \filedateKakurasu\space v\fileversionKakurasu}%
%
\setKVdefault[Kakurasu]{Solution=false,Longueur=6,Largeur=6,Echelle=8mm,Graines=false,Trace=false}
\defKV[Kakurasu]{Graine=\setKV[Kakurasu]{Graines}}
\defKV[Kakurasu]{Traces=\setKV[Kakurasu]{Trace}}

\NewDocumentCommand\Kakurasu{o}{%
  \useKVdefault[Kakurasu]%
  \setKVdefault[Kakurasu]{#1}%
  \BuildKakurasu%
}

\NewDocumentCommand\BuildKakurasu{}{%
  \ifluatex
  \mplibforcehmode
  \begin{mplibcode}
    boolean Graines,Solution,Trace;
    Graines=\useKV[Kakurasu]{Graines};
    Solution=\useKV[Kakurasu]{Solution};
    Trace=\useKV[Kakurasu]{Trace};
    
    if Graines:
    randomseed:=\useKV[Kakurasu]{Graine};
    fi;
    
    numeric Longueur,Largeur;
    Longueur=\useKV[Kakurasu]{Longueur};
    Largeur=\useKV[Kakurasu]{Largeur};
    
    u:=\useKV[Kakurasu]{Echelle};
    
    pair A[][];%centre des carrés.
    boolean Allume[][];
    numeric NombreLigne[][],NombreColonne[][];
    
    for k=-1 upto Largeur+1:
    for l=-1 upto Longueur+1:
    Allume[k][l]=false;
    endfor;
    endfor;
    
    for k=0 upto Largeur-1:
    for l=0 upto Longueur-1:
    A[k][l]=u*(l,-k);
    trace (unitsquare scaled u) shifted A[k][l];
    endfor;
    endfor;

    for k=0 upto Largeur-1:
    for l=0 upto Longueur-1:
    if Allume[k][l]:
    else:
    Allume[k][l]:=true;
    m:=uniformdeviate(1);
    if m<0.5:
    % on ne met pas de case noire
    else:
    % on met une case noire
    NombreLigne[k][l]=l+1;
    NombreColonne[k][l]=k+1;
    if Solution:
    fill (unitsquare scaled 0.9u) shifted (A[k][l]+0.05*u*(1,1));
    fi;
    fi;
    fi;
    endfor;
    endfor;
    
    % on compte le nombre de chiffres dans les lignes et on les ajoute
    for k=0 upto Largeur-1:
    Somme:=0;
    for l=0 upto Longueur-1:
    if unknown NombreLigne[k][l]:
    else:
    Somme:=Somme+NombreLigne[k][l];
    fi;
    endfor;
    label(TEX(decimal(Somme)),u*(-0.5,-k+0.5));
    endfor;

    % on compte le nombre de chiffres dans les colonnes et on les ajoute
    for k=0 upto Longueur-1:
    Somme:=0;
    for l=0 upto Largeur-1:
    if unknown NombreColonne[l][k]:
    else:
    Somme:=Somme+NombreColonne[l][k];
    fi;
    endfor;
    label(TEX(decimal(Somme)),u*(k+0.5,1.5));
    endfor;

    % On affiche les valeurs des cases noires.
    drawoptions(withcolor 0.5white);
    for k=0 upto Longueur-1:
    label(TEX("\footnotesize"&decimal(k+1)),u*(k+0.5,-Largeur+0.5));
    endfor;
    for k=0 upto Largeur-1:
    label(TEX("\footnotesize"&decimal(k+1)),u*(Longueur+0.5,-k+0.5));
    endfor;
    drawoptions();
    if Trace:
    \useKV[Kakurasu]{Traces};
    fi;
  \end{mplibcode}
 \fi
}%