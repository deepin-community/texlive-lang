%%%
% Labyrinthe Nombre
%%%
\setKVdefault[ClesLabyNb]{Multiple=5,XDepart=0,YDepart=0,Longueur=7,Largeur=4,Solution=false,Echelle=1,Angle=0,Couleur=red,Murs=false,Nom=Ex1,CouleurChemin=LightSteelBlue,DoubleOperation=false,Multiplication=false,Chemin=false,EntreeSortie=false,Entree="Nord",Sortie="Sud",SortieUnique=true,Graines=false,XArrivees=false,YArrivees=false,ArriveeX=5,ArriveeY=5}
\defKV[ClesLabyNb]{Graine=\setKV[ClesLabyNb]{Graines}}
\defKV[ClesLabyNb]{XArrivee=\setKV[ClesLabyNb]{XArrivees}\setKV[ClesLabyNb]{ArriveeX=#1}}
\defKV[ClesLabyNb]{YArrivee=\setKV[ClesLabyNb]{YArrivees}\setKV[ClesLabyNb]{ArriveeY=#1}}

\NewDocumentCommand\LabyNombre{o}{%
  \useKVdefault[ClesLabyNb]%
  \setKV[ClesLabyNb]{#1}%
  \ifboolKV[ClesLabyNb]{XArrivees}{}{\setKV[ClesLabyNb]{ArriveeX=\fpeval{\useKV[ClesLabyNb]{Longueur}-1}}}
  \ifboolKV[ClesLabyNb]{YArrivees}{}{\setKV[ClesLabyNb]{ArriveeY=\fpeval{\useKV[ClesLabyNb]{Largeur}-1}}}
  \ifboolKV[ClesLabyNb]{Multiplication}{%
    \TraceLabyMul{\useKV[ClesLabyNb]{Multiple}}{\useKV[ClesLabyNb]{Longueur}}{\useKV[ClesLabyNb]{Largeur}}{(\useKV[ClesLabyNb]{YDepart},\useKV[ClesLabyNb]{XDepart})}{(\useKV[ClesLabyNb]{ArriveeY},\useKV[ClesLabyNb]{ArriveeX})}%
  }{%
    \ifboolKV[ClesLabyNb]{Solution}{%
      \TraceLabyFactoSolution{\useKV[ClesLabyNb]{Multiple}}{\useKV[ClesLabyNb]{Longueur}}{\useKV[ClesLabyNb]{Largeur}}{(\useKV[ClesLabyNb]{YDepart},\useKV[ClesLabyNb]{XDepart})}{(\useKV[ClesLabyNb]{ArriveeY},\useKV[ClesLabyNb]{ArriveeX})}{\useKV[ClesLabyNb]{Solution}}%
    }{%
      \TraceLabyFacto{\useKV[ClesLabyNb]{Multiple}}{\useKV[ClesLabyNb]{Longueur}}{\useKV[ClesLabyNb]{Largeur}}{(\useKV[ClesLabyNb]{YDepart},\useKV[ClesLabyNb]{XDepart})}{(\useKV[ClesLabyNb]{ArriveeY},\useKV[ClesLabyNb]{ArriveeX})}{\useKV[ClesLabyNb]{Solution}}%
    }%
  }%
}%

\newcommand\TraceLabyMul[5]{
  \ifluatex
  \mplibnumbersystem{double}
  \mplibforcehmode
  \begin{mplibcode}
  input PfCLabyNombre;
  
  boolean Murs,Solution,Graines;
  Murs=\useKV[ClesLabyNb]{Murs};
  Solution:=\useKV[ClesLabyNb]{Solution};
  Graines:=\useKV[ClesLabyNb]{Graines};
  if Graines:
  randomseed:=\useKV[ClesLabyNb]{Graine}
  fi;

  path Fullcircle;
  Fullcircle=fullcircle shifted((0.5,0.5));
  
  numeric Multiple;
  Multiple=#1;
  % Initialisation du labyrinthe
  InitialisationLabyrintheMul(#2,#3);
  % On initialise les paramètres du parcours
  numeric choixligneD,choixligneA,choixcolonneD,choixcolonneA;
  choixligneD=xpart(#4);
  choixcolonneD=ypart(#4);
  choixligneA=xpart(#5);
  choixcolonneA=ypart(#5);%
  pair Depart;
  Depart=N[choixligneD][choixcolonneD];
  pair Arrivee;
  Arrivee=N[choixligneA][choixcolonneA];
  pair Mobile;
  Mobile=Depart;
  RAZPileChemin;
  % Exploration du labyrinthe
  PushChemin((choixligneD,choixcolonneD));
  CaseExploree[choixligneD][choixcolonneD]:=true;
  VoisinDispo(choixligneD,choixcolonneD);
  forever: exitif Mobile=Arrivee;%nb=0;
  nb:=ceiling(uniformdeviate(nbvoisin));
  if nb>0:
  for k=1 upto nbvoisin:
  CaseExploree[xpart(PileVoisin[k])][ypart(PileVoisin[k])]:=true;
  endfor;
  PushChemin((xpart(PileVoisin[nb]),ypart(PileVoisin[nb])));
  Mobile:=N[xpart(PileChemin[indiceChemin])][ypart(PileChemin[indiceChemin])];
  VoisinDispo(xpart(PileChemin[indiceChemin]),ypart(PileChemin[indiceChemin]));
  else:
  PopChemin;
  fi;
  endfor;

  vardef PremierInfCent=
    Totalpremier:=25;

    numeric p[]; boolean n_is_prime; p[1]=2; kis:=1;
    for nis=3 step 2 until infinity:
    n_is_prime := true;
    for jis=2 upto kis:
    if nis mod p[jis]=0: n_is_prime := false; fi
    exitif nis/p[jis] < p[jis];
    endfor
    if n_is_prime: p[incr kis] := nis; exitif kis=Totalpremier; fi
    endfor fi
    
    numeric PR[],pretiens[];
    
    for ki=1 upto Totalpremier-1:
    pretiens[ki]:=p[ki+1];
    endfor;
    
    Kk:=0;
    
    for ki=Totalpremier-1 downto 1:
    alea:=1+floor(uniformdeviate(ki));
    PR[incr Kk]:=pretiens[alea];
    Kkl:=0;
    for li=1 upto ki:
    if li<>alea:
      pretiens[incr Kkl]:=pretiens[li];
      fi;
    endfor;
    endfor;
    enddef;
  
    vardef ListeDiviseurs(expr nbtest)=
    numeric diviseurs[],nondiviseurs[],pas;
    nbdiv:=0;
    nbnondiv:=0;
    pas=2;
    forever:
    if (nbtest mod pas)=0:
    nbdiv:=nbdiv+1;
    diviseurs[nbdiv]=pas;
    else:
    nbnondiv:=nbnondiv+1;
    nondiviseurs[nbnondiv]=pas;
    fi;
    pas:=pas+1;
    exitif pas>10;
    endfor;
    enddef;    
    
  % Affichagefinal
  % on sauvegarde les nombres aléatoires
    numeric NbAffiche[];
    numeric NbSol[];
    % on écrit des nombres premiers au hasard >10
    PremierInfCent;
    numeric nbaffiche;
    nbaffiche=0;
    for k=0 upto LargeurLaby-1:
      for l=0 upto LongueurLaby-1:
      nbaffiche:=nbaffiche+1;
      if (nbaffiche mod 2)=0:
      nbpremier:=3+ceiling(uniformdeviate(Totalpremier-4));
      NbAffiche[nbaffiche]:=PR[nbpremier];
      else:
      NbAffiche[nbaffiche]:=2+floor(uniformdeviate(8));
      fi;
      endfor;
    endfor;
    % On crée des multiples du nombre choisi
    NbSol[1]=Multiple;
    for k=2 upto indiceChemin:
    if (k mod 2)=0:
    NbSol[k]=NbSol[k-1]*(2+ceiling(uniformdeviate(8)));
    else:
    % Diviseurs du précédent sauf 1 mais inférieur ou égal à 10
    % choix aléatoire du diviseur
    ListeDiviseurs(NbSol[k-1]);
    choixdiv:=ceiling(uniformdeviate(nbdiv));
    NbSol[k]=diviseurs[choixdiv];
    fi;
    endfor;
  % On affiche
  picture Corps;
  Corps=image(
  drawoptions(withcolor 0.7white);
  for k=0 upto LargeurLaby-1:
  trace (N[k][0]--N[k][LongueurLaby-1]) shifted((5mm,0.25*uv));
  endfor;
  for k=0 upto LongueurLaby-1:
  trace (N[0][k]--N[LargeurLaby-1][k]) shifted((5mm,0.25*uv));
  endfor;
  drawoptions();
  nbaffiche:=0;
  for k=0 upto LargeurLaby-1:
  for l=0 upto LongueurLaby-1:
  nbaffiche:=nbaffiche+1;
    remplis ((Fullcircle scaled 10mm) shifted N[k][l]) withcolor white;
  label(TEX("\num{"&decimal(NbAffiche[nbaffiche])&"}"),M[k][l]);
  endfor;
  endfor;
  if Solution:
  trace ((center Fullcircle scaled 10mm) shifted N[choixligneD][choixcolonneD])
  for k=2 upto indiceChemin-1:
  -- ((center Fullcircle scaled 10mm) shifted N[xpart(PileChemin[k])][ypart(PileChemin[k])])
  endfor
  --((center Fullcircle scaled 10mm) shifted Arrivee) dashed evenly withpen pencircle scaled 1.5 withcolor Brown;
  fi;
  remplis ((Fullcircle scaled 10mm) shifted N[choixligneD][choixcolonneD]) withcolor \useKV[ClesLabyNb]{Couleur};
  if (LargeurLaby mod 2)=0:
  remplis ((unitsquare scaled 10mm) shifted Arrivee) withcolor \useKV[ClesLabyNb]{Couleur};
  else:
  remplis ((Fullcircle scaled 10mm) shifted Arrivee) withcolor \useKV[ClesLabyNb]{Couleur};
  fi;
  for k=2 upto indiceChemin-1:
  remplis ((Fullcircle scaled 10mm) shifted N[xpart(PileChemin[k])][ypart(PileChemin[k])]) withcolor white;
  label(TEX("\num{"&decimal(NbSol[k])&"}"),M[xpart(PileChemin[k])][ypart(PileChemin[k])]);
  endfor;
  k:=1;
  label(TEX("\num{"&decimal(NbSol[k])&"}"),M[xpart(PileChemin[k])][ypart(PileChemin[k])]);
  k:=indiceChemin;
  label(TEX("\num{"&decimal(NbSol[k])&"}"),M[xpart(PileChemin[k])][ypart(PileChemin[k])]);
  trace TraceLabyrintheMul;
  );
  Corps:=(Corps scaled \useKV[ClesLabyNb]{Echelle}) rotated \useKV[ClesLabyNb]{Angle};
  trace Corps;
\end{mplibcode}
\mplibnumbersystem{scaled}
\fi
}

\newcommand\TraceLabyFacto[6]{%
  \mplibforcehmode%
  \xdef\PfCNomLabyrinthe{\useKV[ClesLabyNb]{Nom}}%
  \begin{mplibcode}[\PfCNomLabyrinthe]
    input PfCLabyNombre;

    boolean Murs;
    Murs=\useKV[ClesLabyNb]{Murs};

    string Entree,Sortie;
    Entree=\useKV[ClesLabyNb]{Entree};
    Sortie=\useKV[ClesLabyNb]{Sortie};
    boolean SortieUnique,EntreeSortie;
    SortieUnique:=\useKV[ClesLabyNb]{SortieUnique};
    EntreeSortie:=\useKV[ClesLabyNb]{EntreeSortie};
    picture Square;
    Square=image(
    path Test;
    Test=unitsquare scaled 10mm;
    trace subpath(0.85,1.15) of Test;
    trace subpath(1.85,2.15) of Test;
    trace subpath(2.85,3.15) of Test;
    trace subpath(3.85,4.15) of Test;
    );

    vardef RemplissagePremier=
    Totalpremier:=if indiceChemin<62:62 else: indiceChemin fi;

    numeric p[]; boolean n_is_prime; p[1]=2; kis:=1;
    for nis=3 step 2 until infinity:
    n_is_prime := true;
    for jis=2 upto kis:
    if nis mod p[jis]=0: n_is_prime := false; fi
    exitif nis/p[jis] < p[jis];
    endfor
    if n_is_prime: p[incr kis] := nis; exitif kis=Totalpremier; fi
    endfor fi
    
    numeric PR[],pretiens[];
    
    for ki=1 upto Totalpremier-1:
    pretiens[ki]:=p[ki+1];
    endfor;
    
    Kk:=0;
    
    for ki=Totalpremier-1 downto 1:
    alea:=1+floor(uniformdeviate(ki));
    PR[incr Kk]:=pretiens[alea];
    % message("k="&decimal(k)&" alea="&decimal(alea)&" Kk="&decimal(Kk)&" PR[Kk]="&decimal(PR[Kk]));
    Kkl:=0;
    for li=1 upto ki:
    if li<>alea:
      pretiens[incr Kkl]:=pretiens[li];
      % message("Kkl="&decimal(Kkl));
      fi;
    endfor;
    endfor;
    enddef;
    
    numeric Multiple;
    Multiple=#1;
    % Initialisation du labyrinthe
    InitialisationLabyrinthe(#2,#3);
    % On initialise les paramètres du parcours
    numeric choixligneD,choixligneA,choixcolonneD,choixcolonneA;
    choixligneD=xpart(#4);
    choixcolonneD=ypart(#4);
    choixligneA=xpart(#5);
    choixcolonneA=ypart(#5);%
    pair Depart;
    Depart=N[choixligneD][choixcolonneD];
    pair Arrivee;
    Arrivee=N[choixligneA][choixcolonneA];
    pair Mobile;
    Mobile=Depart;
    %
    RAZPileChemin;

    if \useKV[ClesLabyNb]{SortieUnique}:
    if (Sortie="Est") or (Sortie="Ouest"):
    for k=0 upto LargeurLaby-1:
    if k<>choixligneA:
    CaseExploree[k][choixcolonneA]:=true;
    fi;
    endfor;
    elseif (Sortie="Nord") or (Sortie="Sud"):
    for k=0 upto LongueurLaby-1:
    if k<>choixcolonneA:
    CaseExploree[choixligneA][k]:=true;
    fi;
    endfor;
    fi;
    fi;
    % Exploration du labyrinthe
    if Entree="Nord":
    CaseExploree[choixligneD-1][choixcolonneD]:=true;
    elseif Entree="Sud":
    CaseExploree[choixligneD+1][choixcolonneD]:=true;
    elseif Entree="Est":
    CaseExploree[choixligneD][choixcolonneD+1]:=true;
    elseif Entree="Ouest":
    CaseExploree[choixligneD][choixcolonneD-1]:=true;
    fi;
    %
    PushChemin((choixligneD,choixcolonneD));
    CaseExploree[choixligneD][choixcolonneD]:=true;
    VoisinDispo(choixligneD,choixcolonneD);
    forever: exitif Mobile=Arrivee;%nb=0;
    nb:=ceiling(uniformdeviate(nbvoisin));
    if nb>0:
      for k=1 upto nbvoisin:
        CaseExploree[xpart(PileVoisin[k])][ypart(PileVoisin[k])]:=true;
      endfor;
      PushChemin((xpart(PileVoisin[nb]),ypart(PileVoisin[nb])));
      Mobile:=N[xpart(PileChemin[indiceChemin])][ypart(PileChemin[indiceChemin])];
      VoisinDispo(xpart(PileChemin[indiceChemin]),ypart(PileChemin[indiceChemin]));
    else:
      PopCheminMathAlea;
    fi;
    endfor;
    % Affichagefinal
    % on sauvegarde les nombres aléatoires
    numeric NbAffiche[];
    numeric NbSol[];
    % on écrit des nombres au hasard, mais sans être multiple du nombre choisi
    numeric nbaffiche;
    nbaffiche=0;
    for k=0 upto LargeurLaby-1:
      for l=0 upto LongueurLaby-1:
      nbaffiche:=nbaffiche+1;
      if Multiple=1:%On affichera que des multiples de 2/3/5/9/10 pour afficher un chemin de nombre premiers
      mathalea:=uniformdeviate(5);
      if mathalea<1:
      NbAffiche[nbaffiche]:=(10+ceiling(uniformdeviate(100)))*2;
      elseif mathalea<2:
      NbAffiche[nbaffiche]:=(10+ceiling(uniformdeviate(100)))*3;
      elseif mathalea<3:
      NbAffiche[nbaffiche]:=(10+ceiling(uniformdeviate(100)))*5;
      elseif mathalea<4:
      NbAffiche[nbaffiche]:=(10+ceiling(uniformdeviate(100)))*9;
      else:  NbAffiche[nbaffiche]:=(10+ceiling(uniformdeviate(100)))*10;
      fi;
      else:
      NbAffiche[nbaffiche]:=(50+ceiling(uniformdeviate(100)))*Multiple+ceiling(uniformdeviate(Multiple-1));
      fi;
      endfor;
    endfor;
    % On crée des multiples du nombre choisi
    RemplissagePremier;
    
    for k=1 upto indiceChemin:
    if Multiple>1:
    NbSol[k]=(50+ceiling(uniformdeviate(100)))*Multiple;
    else:
    NbSol[k]=PR[k];
    fi;
    endfor;
    % On affiche
    picture Corps;
    Corps=image(
      nbaffiche:=0;
      for k=0 upto LargeurLaby-1:
        for l=0 upto LongueurLaby-1:
	  nbaffiche:=nbaffiche+1;
          label(TEX("\num{"&decimal(NbAffiche[nbaffiche])&"}"),M[k][l]);
        endfor;
      endfor;
      for k=1 upto indiceChemin:
      remplis ((unitsquare scaled 10mm) shifted N[xpart(PileChemin[k])][ypart(PileChemin[k])]) withcolor white;
      label(TEX("\num{"&decimal(NbSol[k])&"}"),M[xpart(PileChemin[k])][ypart(PileChemin[k])]);
      endfor;
      if EntreeSortie=false:
      remplis ((unitsquare scaled 10mm) shifted N[choixligneD][choixcolonneD]) withcolor \useKV[ClesLabyNb]{Couleur};
      remplis ((unitsquare scaled 10mm) shifted Arrivee) withcolor \useKV[ClesLabyNb]{Couleur};
      trace TraceLabyrinthe;
      else:
      for k=0 upto LargeurLaby-1:
      for l=0 upto LongueurLaby-1:
      trace Square shifted N[k][l];
      endfor;
      endfor;
      trace u*(0,0)--u*(LongueurLaby,0)--u*(LongueurLaby,-LargeurLaby)--u*(0,-LargeurLaby)--cycle;
      % On dessine l'entrée.
      if Entree="Nord":
      N[-1][-1]:=Depart+(0,10mm);
      fill ((unitsquare scaled 10mm) shifted N[-1][-1]) withcolor white;
      draw (subpath(1,4) of unitsquare scaled 10mm) shifted N[-1][-1];
      undraw (subpath(0.18,0.82) of unitsquare scaled 10mm) shifted N[-1][-1] withpen pencircle scaled 1.25;
      label(btex \footnotesize Départ etex,(center Square) shifted N[-1][-1]);
      elseif Entree="Ouest":
      N[-1][-1]:=Depart+(-10mm,0);
      fill ((unitsquare scaled 10mm) shifted N[-1][-1]) withcolor white;
      draw (subpath(2,5) of unitsquare scaled 10mm) shifted N[-1][-1];
      undraw (subpath(1.15,1.85) of unitsquare scaled 10mm) shifted N[-1][-1] withpen pencircle scaled 1.25;
      label(btex \footnotesize Départ etex,(center Square) shifted N[-1][-1]);
      elseif Entree="Sud":
      N[-1][-1]:=Depart+(0,-10mm);
      fill ((unitsquare scaled 10mm) shifted N[-1][-1]) withcolor white;
      draw (subpath(3,6) of unitsquare scaled 10mm) shifted N[-1][-1];
      undraw (subpath(2.15,2.85) of unitsquare scaled 10mm) shifted N[-1][-1] withpen pencircle scaled 1.25;
      label(btex \footnotesize Départ etex,(center Square) shifted N[-1][-1]);
      elseif Entree="Est":
      N[-1][-1]:=Depart+(10mm,0);
      fill ((unitsquare scaled 10mm) shifted N[-1][-1]) withcolor white;
      draw (subpath(0,3) of unitsquare scaled 10mm) shifted N[-1][-1];
      undraw (subpath(3.18,3.82) of unitsquare scaled 10mm) shifted N[-1][-1] withpen pencircle scaled 1.25;
      label(btex \footnotesize Départ etex,(center Square) shifted N[-1][-1]);
      fi;
      % On dessine les sorties
      if Sortie="Nord":
      N[-2][-2]=Arrivee+(5mm,20mm);
      for k=0 upto LongueurLaby-1:
      draw ((point(0.15) of unitsquare scaled 10mm)--0.5[(point(0.15) of unitsquare scaled 10mm),(point(2.85) of unitsquare scaled 10mm)]) shifted (N[0][k]+(0,10mm));
      draw ((point(0.85) of unitsquare scaled 10mm)--0.5[(point(0.85) of unitsquare scaled 10mm),(point(2.15) of unitsquare scaled 10mm)]) shifted (N[0][k]+(0,10mm));
      undraw (subpath(0.18,0.82) of unitsquare scaled 10mm) shifted (N[0][k]+(0,10mm)) withpen pencircle scaled 1.25;
      label(TEX("\footnotesize S"&decimal(k)),center (Square) shifted (N[0][k]+(0,10mm)));
      endfor;
      elseif Sortie="Ouest":
      N[-2][-2]=Arrivee+(-10mm,5mm);
      for k=0 upto LargeurLaby-1:
      draw ((point(1.15) of unitsquare scaled 10mm)--0.5[(point(1.15) of unitsquare scaled 10mm),(point(3.85) of unitsquare scaled 10mm)]) shifted (N[k][0]+(-10mm,0));
      draw ((point(1.85) of unitsquare scaled 10mm)--0.5[(point(1.85) of unitsquare scaled 10mm),(point(3.15) of unitsquare scaled 10mm)]) shifted (N[k][0]+(-10mm,0));
      undraw (subpath(1.18,1.82) of unitsquare scaled 10mm) shifted (N[k][0]+(-10mm,0)) withpen pencircle scaled 1.25;
      label(TEX("\footnotesize S"&decimal(k)),center (Square) shifted (N[k][0]+(-10mm,0)));
      endfor;
      elseif Sortie="Sud":
      N[-2][-2]=Arrivee+(5mm,-10mm);
      for k=0 upto LongueurLaby-1:
      draw ((point(2.85) of unitsquare scaled 10mm)--0.5[(point(0.15) of unitsquare scaled 10mm),(point(2.85) of unitsquare scaled 10mm)]) shifted (N[LargeurLaby-1][k]+(0,-10mm));
      draw ((point(2.15) of unitsquare scaled 10mm)--0.5[(point(0.85) of unitsquare scaled 10mm),(point(2.15) of unitsquare scaled 10mm)]) shifted (N[LargeurLaby-1][k]+(0,-10mm));
      undraw (subpath(2.18,2.82) of unitsquare scaled 10mm) shifted (N[LargeurLaby-1][k]+(0,-10mm)) withpen pencircle scaled 1.25;
      label(TEX("\footnotesize S"&decimal(k)),center (Square) shifted (N[LargeurLaby-1][k]+(0,-10mm)));
      endfor;
      elseif Sortie="Est":
      N[-2][-2]=Arrivee+(20mm,5mm);
      for k=0 upto LargeurLaby-1:
      draw ((point(3.15) of unitsquare scaled 10mm)--0.5[(point(3.15) of unitsquare scaled 10mm),(point(1.85) of unitsquare scaled 10mm)]) shifted (N[k][LongueurLaby-1]+(10mm,0));
      draw ((point(3.85) of unitsquare scaled 10mm)--0.5[(point(3.85) of unitsquare scaled 10mm),(point(1.15) of unitsquare scaled 10mm)]) shifted (N[k][LongueurLaby-1]+(10mm,0));
      undraw (subpath(3.18,3.8) of unitsquare scaled 10mm) shifted (N[k][LongueurLaby-1]+(10mm,0)) withpen pencircle scaled 1.25;
      label(TEX("\footnotesize S"&decimal(k)),center (Square) shifted (N[k][LongueurLaby-1]+(10mm,0)));
      endfor;
      fi;
      fi;
      );
    Corps:=(Corps scaled \useKV[ClesLabyNb]{Echelle}) rotated \useKV[ClesLabyNb]{Angle};
    trace Corps;
  \end{mplibcode}
}

\newcommand\TraceLabyFactoSolution[6]{%
  \mplibforcehmode%
  \xdef\PfCNomLabyrinthe{\useKV[ClesLabyNb]{Nom}}%
  \begin{mplibcode}[\PfCNomLabyrinthe]
    color CouleurChemin;
    CouleurChemin:=\useKV[ClesLabyNb]{CouleurChemin};
    picture CorpsSolution;
    CorpsSolution=image(
    nbaffiche:=0;
      for k=0 upto LargeurLaby-1:
        for l=0 upto LongueurLaby-1:
          nbaffiche:=nbaffiche+1;
          label(TEX("\num{"&decimal(NbAffiche[nbaffiche])&"}"),M[k][l]);
        endfor;
      endfor;
      if EntreeSortie=false:
            remplis ((unitsquare scaled 10mm) shifted N[choixligneD][choixcolonneD]) withcolor \useKV[ClesLabyNb]{Couleur};
      remplis ((unitsquare scaled 10mm) shifted Arrivee) withcolor \useKV[ClesLabyNb]{Couleur};
      for k=2 upto indiceChemin-1:
        remplis ((unitsquare scaled 10mm) shifted N[xpart(PileChemin[k])][ypart(PileChemin[k])]) withcolor \useKV[ClesLabyNb]{CouleurChemin};
        label(TEX("\num{"&decimal(NbSol[k])&"}"),M[xpart(PileChemin[k])][ypart(PileChemin[k])]);
        endfor;
        else:
      for k=1 upto indiceChemin:
      remplis ((unitsquare scaled 10mm) shifted N[xpart(PileChemin[k])][ypart(PileChemin[k])]) withcolor white;
      label(TEX("\num{"&decimal(NbSol[k])&"}"),M[xpart(PileChemin[k])][ypart(PileChemin[k])]);
      endfor;
      fi;
      if EntreeSortie=false:
      trace TraceLabyrinthe;
      else:
      for k=0 upto LargeurLaby-1:
      for l=0 upto LongueurLaby-1:
      trace Square shifted N[k][l];
      endfor;
      endfor;
      trace u*(0,0)--u*(LongueurLaby,0)--u*(LongueurLaby,-LargeurLaby)--u*(0,-LargeurLaby)--cycle;
      % On dessine l'entrée.
      if Entree="Nord":
      N[-1][-1]:=Depart+(0,10mm);
      fill ((unitsquare scaled 10mm) shifted N[-1][-1]) withcolor white;
      draw (subpath(1,4) of unitsquare scaled 10mm) shifted N[-1][-1];
      undraw (subpath(0.18,0.82) of unitsquare scaled 10mm) shifted N[-1][-1] withpen pencircle scaled 1.25;
      label(btex \footnotesize Départ etex,(center Square) shifted N[-1][-1]);
      elseif Entree="Ouest":
      N[-1][-1]:=Depart+(-10mm,0);
      fill ((unitsquare scaled 10mm) shifted N[-1][-1]) withcolor white;
      draw (subpath(2,5) of unitsquare scaled 10mm) shifted N[-1][-1];
      undraw (subpath(1.15,1.85) of unitsquare scaled 10mm) shifted N[-1][-1] withpen pencircle scaled 1.25;
      label(btex \footnotesize Départ etex,(center Square) shifted N[-1][-1]);
      elseif Entree="Sud":
      N[-1][-1]:=Depart+(0,-10mm);
      fill ((unitsquare scaled 10mm) shifted N[-1][-1]) withcolor white;
      draw (subpath(3,6) of unitsquare scaled 10mm) shifted N[-1][-1];
      undraw (subpath(2.15,2.85) of unitsquare scaled 10mm) shifted N[-1][-1] withpen pencircle scaled 1.25;
      label(btex \footnotesize Départ etex,(center Square) shifted N[-1][-1]);
      elseif Entree="Est":
      N[-1][-1]:=Depart+(10mm,0);
      fill ((unitsquare scaled 10mm) shifted N[-1][-1]) withcolor white;
      draw (subpath(0,3) of unitsquare scaled 10mm) shifted N[-1][-1];
      undraw (subpath(3.18,3.82) of unitsquare scaled 10mm) shifted N[-1][-1] withpen pencircle scaled 1.25;
      label(btex \footnotesize Départ etex,(center Square) shifted N[-1][-1]);
      fi;
      % On dessine les sorties
      if Sortie="Nord":
      N[-2][-2]=Arrivee+(5mm,20mm);
      for k=0 upto LongueurLaby-1:
      draw ((point(0.15) of unitsquare scaled 10mm)--0.5[(point(0.15) of unitsquare scaled 10mm),(point(2.85) of unitsquare scaled 10mm)]) shifted (N[0][k]+(0,10mm));
      draw ((point(0.85) of unitsquare scaled 10mm)--0.5[(point(0.85) of unitsquare scaled 10mm),(point(2.15) of unitsquare scaled 10mm)]) shifted (N[0][k]+(0,10mm));
      undraw (subpath(0.18,0.82) of unitsquare scaled 10mm) shifted (N[0][k]+(0,10mm)) withpen pencircle scaled 1.25;
      label(TEX("\footnotesize S"&decimal(k)),center (Square) shifted (N[0][k]+(0,10mm)));
      endfor;
      elseif Sortie="Ouest":
      N[-2][-2]=Arrivee+(-10mm,5mm);
      for k=0 upto LargeurLaby-1:
      draw ((point(1.15) of unitsquare scaled 10mm)--0.5[(point(1.15) of unitsquare scaled 10mm),(point(3.85) of unitsquare scaled 10mm)]) shifted (N[k][0]+(-10mm,0));
      draw ((point(1.85) of unitsquare scaled 10mm)--0.5[(point(1.85) of unitsquare scaled 10mm),(point(3.15) of unitsquare scaled 10mm)]) shifted (N[k][0]+(-10mm,0));
      undraw (subpath(1.18,1.82) of unitsquare scaled 10mm) shifted (N[k][0]+(-10mm,0)) withpen pencircle scaled 1.25;
      label(TEX("\footnotesize S"&decimal(k)),center (Square) shifted (N[k][0]+(-10mm,0)));
      endfor;
      elseif Sortie="Sud":
      N[-2][-2]=Arrivee+(5mm,-10mm);
      for k=0 upto LongueurLaby-1:
      % draw (Square) shifted (N[LargeurLaby-1][k]+(0,-10mm)) withcolor red;
      draw ((point(2.85) of unitsquare scaled 10mm)--0.5[(point(0.15) of unitsquare scaled 10mm),(point(2.85) of unitsquare scaled 10mm)]) shifted (N[LargeurLaby-1][k]+(0,-10mm));
      draw ((point(2.15) of unitsquare scaled 10mm)--0.5[(point(0.85) of unitsquare scaled 10mm),(point(2.15) of unitsquare scaled 10mm)]) shifted (N[LargeurLaby-1][k]+(0,-10mm));
      undraw (subpath(2.18,2.82) of unitsquare scaled 10mm) shifted (N[LargeurLaby-1][k]+(0,-10mm)) withpen pencircle scaled 1.25;
      label(TEX("\footnotesize S"&decimal(k)),center (Square) shifted (N[LargeurLaby-1][k]+(0,-10mm)));
      endfor;
      elseif Sortie="Est":
      N[-2][-2]=Arrivee+(20mm,5mm);
      for k=0 upto LargeurLaby-1:
      draw ((point(3.15) of unitsquare scaled 10mm)--0.5[(point(3.15) of unitsquare scaled 10mm),(point(1.85) of unitsquare scaled 10mm)]) shifted (N[k][LongueurLaby-1]+(10mm,0));
      draw ((point(3.85) of unitsquare scaled 10mm)--0.5[(point(3.85) of unitsquare scaled 10mm),(point(1.15) of unitsquare scaled 10mm)]) shifted (N[k][LongueurLaby-1]+(10mm,0));
      undraw (subpath(3.18,3.8) of unitsquare scaled 10mm) shifted (N[k][LongueurLaby-1]+(10mm,0)) withpen pencircle scaled 1.25;
      label(TEX("\footnotesize S"&decimal(k)),center (Square) shifted (N[k][LongueurLaby-1]+(10mm,0)));
      endfor;
      fi;
      path CheminSolution;
      CheminSolution=(N[-1][-1] shifted (5mm,5mm))--((center unitsquare scaled 10mm) shifted N[choixligneD][choixcolonneD])
      for k=2 upto indiceChemin-1:
      -- ((center unitsquare scaled 10mm) shifted N[xpart(PileChemin[k])][ypart(PileChemin[k])])
      endfor
      --((center unitsquare scaled 10mm) shifted Arrivee)--N[-2][-2];
      trace CheminSolution dashed evenly withcolor CouleurChemin;
      fi;
      );
      CorpsSolution:=(CorpsSolution scaled \useKV[ClesLabyNb]{Echelle}) rotated \useKV[ClesLabyNb]{Angle};
      trace CorpsSolution;
  \end{mplibcode}
}

\newcommand\TraceLabyNombreDouble[6]{%
  \mplibforcehmode%
  \xdef\PfCNomLabyrinthe{\useKV[ClesLabyNb]{Nom}}%
  \begin{mplibcode}[\PfCNomLabyrinthe]
    input PfCLabyNombre;

    boolean Murs;
    Murs=\useKV[ClesLabyNb]{Murs};
    
    numeric Multiple;
    Multiple=#1;
    % Initialisation du labyrinthe
    InitialisationLabyrinthe(#2,#3);
    % On initialise les paramètres du parcours
    numeric choixligneD,choixligneA,choixcolonneD,choixcolonneA;
    choixligneD=xpart(#4);
    choixcolonneD=ypart(#4);
    choixligneA=xpart(#5);
    choixcolonneA=ypart(#5);%
    pair Depart;
    Depart=N[choixligneD][choixcolonneD];
    pair Arrivee;
    Arrivee=N[choixligneA][choixcolonneA];
    pair Mobile;
    Mobile=Depart;
    RAZPileChemin;
    % Exploration du labyrinthe
    PushChemin((choixligneD,choixcolonneD));
    CaseExploree[choixligneD][choixcolonneD]:=true;
    VoisinDispo(choixligneD,choixcolonneD);
    forever: exitif Mobile=Arrivee;%nb=0;
    nb:=ceiling(uniformdeviate(nbvoisin));
    if nb>0:
      for k=1 upto nbvoisin:
        CaseExploree[xpart(PileVoisin[k])][ypart(PileVoisin[k])]:=true;
      endfor;
      PushChemin((xpart(PileVoisin[nb]),ypart(PileVoisin[nb])));
      Mobile:=N[xpart(PileChemin[indiceChemin])][ypart(PileChemin[indiceChemin])];
      VoisinDispo(xpart(PileChemin[indiceChemin]),ypart(PileChemin[indiceChemin]));
    else:
      PopChemin;
    fi;
    endfor;
    % Affichagefinal
    % on sauvegarde les nombres aléatoires
    numeric NbAffiche[];
    numeric NbSol[];
    % on écrit des nombres au hasard, mais sans être multiple du nombre choisi
    numeric nbaffiche;
    nbaffiche=0;
    NbAffiche[0]=5+ceiling(uniformdeviate(7));
    for k=0 upto LargeurLaby-1:
      for l=0 upto LongueurLaby-1:
      nbaffiche:=nbaffiche+1;
      mathalea:=uniformdeviate(1);
      if mathalea<0.95:
      NbAffiche[nbaffiche]=NbAffiche[nbaffiche-1]+Multiple+1;
      else:
      NbAffiche[nbaffiche]=NbAffiche[nbaffiche-1]*(Multiple-1);
      fi;
      endfor;
    endfor;
    % On crée des multiples du nombre choisi
    NbSol[1]=5+ceiling(uniformdeviate(7));
    for k=2 upto indiceChemin-1:
    mathalea:=uniformdeviate(1);
    if mathalea<0.9:
    NbSol[k]=NbSol[k-1]+Multiple;
    else:
    NbSol[k]=NbSol[k-1]*Multiple;
    fi;
    endfor;
    % On affiche
    picture Corps;
    Corps=image(
      nbaffiche:=0;
      for k=0 upto LargeurLaby-1:
        for l=0 upto LongueurLaby-1:
	  nbaffiche:=nbaffiche+1;
          label(TEX("\num{"&decimal(NbAffiche[nbaffiche])&"}"),M[k][l]);
        endfor;
      endfor;
      remplis ((unitsquare scaled 10mm) shifted N[choixligneD][choixcolonneD]) withcolor \useKV[ClesLabyNb]{Couleur};
      remplis ((unitsquare scaled 10mm) shifted Arrivee) withcolor \useKV[ClesLabyNb]{Couleur};
      for k=2 upto indiceChemin-1:
        remplis ((unitsquare scaled 10mm) shifted N[xpart(PileChemin[k])][ypart(PileChemin[k])]) withcolor white;
        label(TEX("\num{"&decimal(NbSol[k])&"}"),M[xpart(PileChemin[k])][ypart(PileChemin[k])]);
      endfor;
      trace TraceLabyrinthe;
    );
    Corps:=(Corps scaled \useKV[ClesLabyNb]{Echelle}) rotated \useKV[ClesLabyNb]{Angle};
    trace Corps;
  \end{mplibcode}
}

\newcommand\TraceDoubleSolution[6]{%
  \mplibforcehmode%
  \xdef\PfCNomLabyrinthe{\useKV[ClesLabyNb]{Nom}}%
  \begin{mplibcode}[\PfCNomLabyrinthe]
    picture CorpsSolution;
    CorpsSolution=image(
      nbaffiche:=0;
      for k=0 upto #3-1:
        for l=0 upto #2-1:
          nbaffiche:=nbaffiche+1;
          label(TEX("\num{"&decimal(NbAffiche[nbaffiche])&"}"),M[k][l]);
        endfor;
        endfor;
      remplis ((unitsquare scaled 10mm) shifted N[choixligneD][choixcolonneD]) withcolor \useKV[ClesLabyNb]{Couleur};
      remplis ((unitsquare scaled 10mm) shifted Arrivee) withcolor \useKV[ClesLabyNb]{Couleur};
      if \useKV[ClesLabyNb]{Chemin}:
      for k=2 upto indiceChemin-1:
      remplis ((unitsquare scaled 10mm) shifted N[xpart(PileChemin[k])][ypart(PileChemin[k])]) withcolor white;
      endfor;
      trace ((center unitsquare scaled 10mm) shifted N[choixligneD][choixcolonneD])
      for k=2 upto indiceChemin-1:
      -- ((center unitsquare scaled 10mm) shifted N[xpart(PileChemin[k])][ypart(PileChemin[k])])
      endfor
      --((center unitsquare scaled 10mm) shifted Arrivee) dashed evenly withcolor Brown;
      for k=2 upto indiceChemin-1:
      label(TEX("\num{"&decimal(NbSol[k])&"}"),M[xpart(PileChemin[k])][ypart(PileChemin[k])]);
      endfor;
      else:
      for k=2 upto indiceChemin-1:
      remplis ((unitsquare scaled 10mm) shifted N[xpart(PileChemin[k])][ypart(PileChemin[k])]) withcolor \useKV[ClesLabyNb]{CouleurChemin};
      label(TEX("\num{"&decimal(NbSol[k])&"}"),M[xpart(PileChemin[k])][ypart(PileChemin[k])]);
      endfor;
      fi;
      trace TraceLabyrinthe;
      );
      CorpsSolution:=(CorpsSolution scaled \useKV[ClesLabyNb]{Echelle}) rotated \useKV[ClesLabyNb]{Angle};
      trace CorpsSolution;
  \end{mplibcode}
}%