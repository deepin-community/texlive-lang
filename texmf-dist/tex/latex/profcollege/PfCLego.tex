%%%
% Lego
%%%
\setKVdefault[Lego]{Echelle=1,Couleur=LightSteelBlue,Unite=8,Diviseur=2,Perso=false,Plaque=false,Largeura=2,Largeurb=2,Unique=false}%
\defKV[Lego]{ListePerso=\setKV[Lego]{Perso}\xdef\PfCFooLegoListePerso{#1}}%

\NewDocumentCommand\ChoixCouleur{m}{%
  \xdef\ListeCouleurs{Maroon,PeachPuff,Yellow,red,Gray,Crimson,Cornsilk,Purple,Cornsilk,Salmon}%
  \readlist*\PfCListeCouleurs{\ListeCouleurs}%
  \xintifboolexpr{\fpeval{#1+1}>\PfCListeCouleurslen}{\textbf{! Vous demandez trop de couleurs !}}{%
    \MelangeListe{1,2,3,4,5,6,7,8,9,10}{\fpeval{#1+1}}%
    \setsepchar{,}\ignoreemptyitems%
    \readlist*\ListeRotation{\faa}%
    \reademptyitems%
    \xintFor* ##2 in{\xintSeq{1}{#1}}\do{%
      \xdef\RetiensCouleur{\ListeRotation[##2]}%
      \xintifForFirst{\xdef\FooCouleur{\PfCListeCouleurs[\RetiensCouleur]}}{\xdef\FooCouleur{\FooCouleur,\PfCListeCouleurs[\RetiensCouleur]}}%
    }%
  }%
}%

\NewDocumentCommand\Lego{o}{%
  \useKVdefault[Lego]%
  \setKV[Lego]{#1}%
  \xintifboolexpr{\useKV[Lego]{Unite}==1}{%
    \xdef\PfCNbCouleurs{1}%
  }{%
    \xdef\PfCNbCouleurs{\fpeval{floor(\useKV[Lego]{Unite}/\useKV[Lego]{Diviseur})}}%
  }%
  \ChoixCouleur{\PfCNbCouleurs}%
  \ifboolKV[Lego]{Perso}{%
    \BuildBarreLego{\FooCouleur}{\PfCFooLegoListePerso}%
  }{%
    \BuildBarreLego{\FooCouleur}{}%
  }%
}%

\NewDocumentCommand\BuildBarreLego{mm}{%
  \ifluatex%
    \mplibforcehmode%
    \begin{mplibcode}
      boolean Plaque,Unique;
      Plaque=\useKV[Lego]{Plaque};
      Unique=\useKV[Lego]{Unique};

      Echelle=\useKV[Lego]{Echelle};
      
      input PfCLego;
      %
      vardef ChoixCouleur(text t)=
      color Brique[];
      nbc:=0;
      for p_=t:
      Brique[nbc]=p_;
      nbc:=nbc+1;
      endfor;
      enddef;
      %
      vardef ListePerso(text t)=
      nbperso=0;
      for p_=t:
      Diviseurs[nbperso]=p_;
      nbperso:=nbperso+1;
      endfor;
      enddef;
      %
      ChoixCouleur(#1);
      %
      Longueur=\useKV[Lego]{Unite};
      Largeura=\useKV[Lego]{Largeura};
      Largeurb=\useKV[Lego]{Largeurb};
      Diviseur=\useKV[Lego]{Diviseur};
      color BriqueUnite;
      BriqueUnite=\useKV[Lego]{Couleur};
      boolean Perso;
      Perso=\useKV[Lego]{Perso};
      %
      euler(-45,0,0);
      if Plaque:
      lego_height_old:=lego_height;
      lego_height:=lego_thickness;
      TR:=(0,1,0);
      Lego_box(2*Longueur,4*Largeura,Gold);
      TR:=(0,0,0);
      lego_height:=lego_height_old;
      fi;      
      Lego_box(Longueur,Largeura,BriqueUnite);
      if Unique=false:
      if Perso:
      Somme=0;
      ListePerso(#2);
      for k=0 upto (nbperso-1):
      TR:=(0.5*(Longueur-Diviseurs[k]-Somme),Largeura+0.5*(Largeurb-Largeura),0);
      Lego_box(Diviseurs[k],Largeurb,Brique[k]);
      Somme:=Somme+2*Diviseurs[k];
      endfor;
      else:
      for k=0 upto ((Longueur div Diviseur)-1):
      TR:=(0.5*(Longueur-Diviseur)-k*Diviseur,Largeura+0.5*(Largeurb-Largeura),0);
      Lego_box(Diviseur,Largeurb,Brique[k]);
      endfor;
      fi;
      fi;
      picture RetiensLego;
      RetiensLego=currentpicture;
      currentpicture:=nullpicture;
      draw RetiensLego scaled Echelle;
    \end{mplibcode}
  \else
    \begin{mpost}
      input PfCLego;
      %
      vardef ChoixCouleur(text t)=
      color Brique[];
      nbc:=0;
      for p_=t:
      Brique[nbc]=p_;
      nbc:=nbc+1;
      endfor;
      enddef;
      %
      vardef ListePerso(text t)=
      nbperso=0;
      for p_=t:
      Diviseurs[nbperso]=p_;
      nbperso:=nbperso+1;
      endfor;
      enddef;
      %
      ChoixCouleur(#1);
      %
      Longueur=\useKV[Lego]{Unite};
      Largeur=2;
      Diviseur=\useKV[Lego]{Diviseur};
      color BriqueUnite;
      BriqueUnite=\useKV[Lego]{Couleur};
      boolean Perso;
      Perso=\useKV[Lego]{Perso};
      %
      euler(-45,0,0);
      Lego_box(Longueur,Largeur,BriqueUnite);
      if Perso:
      Somme=0;
      ListePerso(#2);
      for k=0 upto (nbperso-1):
      TR:=(0.5*(Longueur-Diviseurs[k]-Somme),2,0);
      Lego_box(Diviseurs[k],2,Brique[k]);
      Somme:=Somme+2*Diviseurs[k];
      endfor;
      else:
      for k=0 upto ((Longueur div Diviseur)-1):
      TR:=(0.5*(Longueur-Diviseur)-k*Diviseur,2,0);
      Lego_box(Diviseur,2,Brique[k]);
      endfor;
      fi;
    \end{mpost}
  \fi%
}%