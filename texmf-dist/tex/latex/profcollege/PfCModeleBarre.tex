%%%
% Modele en barre
%%%
\def\filedateModeleBarre{2024/08/04}%
\def\fileversionModeleBarre{0.1}%
\message{-- \filedateModeleBarre\space v\fileversionModeleBarre}%
%
\setKVdefault[ClesModeleBarre]{Hauteur=0.5cm,Largeur=1cm,Separation={0},Homogene=false,Multiplicatif=false,MultiplicatifH=false,MultiplicatifB=false}%Separation pour conserver le lien avec les équations.
\defKV[ClesModeleBarre]{Longueur=\setKV[ClesModeleBarre]{Homogene}}%
\defKV[ClesModeleBarre]{AccoladesH=\setKV[ClesModeleBarre]{MultiplicatifH}\setKV[ClesModeleBarre]{Multiplicatif}}%
\defKV[ClesModeleBarre]{AccoladesB=\setKV[ClesModeleBarre]{MultiplicatifB}\setKV[ClesModeleBarre]{Multiplicatif}}%

\newtoks\toklistemodelbarresup%
\def\UpdateListeModelBarreSup#1\nil{\addtotok\toklistemodelbarresup{#1,}}%
\newtoks\toklistemodelbarreinf%
\def\UpdateListeModelBarreInf#1\nil{\addtotok\toklistemodelbarreinf{#1,}}%
\newtoks\toklistemodelbarresep%
\def\UpdateListeModelBarreSep#1\nil{\addtotok\toklistemodelbarresep{#1,}}%
\newtoks\toklistemodelbarreaccosup%
\def\UpdateListeModelBarreAccoSup#1\nil{\addtotok\toklistemodelbarreaccosup{#1,}}%
\newtoks\toklistemodelbarreaccoinf%
\def\UpdateListeModelBarreAccoInf#1\nil{\addtotok\toklistemodelbarreaccoinf{#1,}}%

\newcommand\ModeleBarre[3][]{%
  % #1 options
  % #2 éléments de la barre supérieure C1 2 $x$ 1 3/C2 1 1x/
  % #3 éléments de la barre inférieure (même lecture)
  \useKVdefault[ClesModeleBarre]%
  \setKV[ClesModeleBarre]{#1}%
  \setsepchar{\\/ }%
  \readlist\ListeModelBarreSup{#2}%
  \readlist\ListeModelBarreInf{#3}%
  \setsepchar{,}%
  \xdef\fooSeparation{\useKV[ClesModeleBarre]{Separation}}
  \readlist\ListeModelBarreSep{\fooSeparation}%
  \toklistemodelbarresup{}%
  \foreachitem\compteur\in\ListeModelBarreSup{%
    \foreachitem\valeur\in\ListeModelBarreSup[\compteurcnt]{%
      \expandafter\UpdateListeModelBarreSup\valeur\nil%
    }%
  }%
  \toklistemodelbarreinf{}%
  \foreachitem\compteur\in\ListeModelBarreInf{%
    \foreachitem\valeur\in\ListeModelBarreInf[\compteurcnt]{%
      \expandafter\UpdateListeModelBarreInf\valeur\nil%
    }%
  }%
  \toklistemodelbarresep{}%
  \foreachitem\compteur\in\ListeModelBarreSep{%
    \expandafter\UpdateListeModelBarreSep\compteur\nil%
  }%
  \ifboolKV[ClesModeleBarre]{Homogene}{%
    \PfCMPDessineModelBarre{\the\toklistemodelbarresup}{\the\toklistemodelbarreinf}%
  }{%
    \ifboolKV[ClesModeleBarre]{Multiplicatif}{%
      %
      \toklistemodelbarreaccosup{}%
      \toklistemodelbarreaccoinf{}%
      \ifboolKV[ClesModeleBarre]{MultiplicatifH}{%
        \xdef\PfCListeAccoladesH{\useKV[ClesModeleBarre]{AccoladesH}}%
        \setsepchar{\\/ }%
        \readlist\ListeModeleBarreAccoSup{\PfCListeAccoladesH}%
        \foreachitem\compteur\in\ListeModeleBarreAccoSup{%
          \foreachitem\valeur\in\ListeModeleBarreAccoSup[\compteurcnt]{%
            \expandafter\UpdateListeModelBarreAccoSup\valeur\nil%
          }%
        }%
      %  Les accolades hautes sont \the\toklistemodelbarreaccosup.\\
      }{}% 
      \ifboolKV[ClesModeleBarre]{MultiplicatifB}{%
        \xdef\PfCListeAccoladesB{\useKV[ClesModeleBarre]{AccoladesB}}%
        \setsepchar{\\/ }%
        \readlist\ListeModeleBarreAccoInf{\PfCListeAccoladesB}%
        \foreachitem\compteur\in\ListeModeleBarreAccoInf{%
          \foreachitem\valeur\in\ListeModeleBarreAccoInf[\compteurcnt]{%
            \expandafter\UpdateListeModelBarreAccoInf\valeur\nil%
          }%
        }%
      %  Les accolades basses sont \the\toklistemodelbarreaccoinf.\\
      }{}%
      \PfCMPDessineModelBarreNonHomogeneMultiplicatif{\the\toklistemodelbarresup}{\the\toklistemodelbarreinf}{\the\toklistemodelbarresep}{\the\toklistemodelbarreaccosup}{\the\toklistemodelbarreaccoinf}%
    }{%
      \PfCMPDessineModelBarreNonHomogene{\the\toklistemodelbarresup}{\the\toklistemodelbarreinf}{\the\toklistemodelbarresep}%
    }%
  }%
}%

\newcommand\PfCMPDessineModelBarre[2]{%
  \begin{mplibcode}%
    Longueur:=\useKV[ClesModeleBarre]{Longueur};
    Hauteur:=\useKV[ClesModeleBarre]{Hauteur};

    vardef Briquesup(expr col)(text t)=
    save $;
    picture $;
    $=image(
    u:=Longueur/MBnbcasessuptotal;
    remplis polygone((0,0),(u,0),(u,Hauteur),(0,Hauteur)) withcolor col;
    draw polygone((0,0),(u,0),(u,Hauteur),(0,Hauteur));
    label(TEX(t),iso((0,0),(u,Hauteur)));
    u:=1cm;
    );
    $
    enddef;

    vardef Briqueinf(expr col)(text t)=
    save $;
    picture $;
    $=image(
    u:=Longueur/MBnbcasesinftotal;
    remplis polygone((0,0),(u,0),(u,Hauteur),(0,Hauteur)) withcolor col;
    draw polygone((0,0),(u,0),(u,Hauteur),(0,Hauteur));
    label(TEX(t),iso((0,0),(u,Hauteur)));
    u:=1cm;
    );
    $
    enddef;
    
    numeric MBnbcolsup;%compte combien de couleurs
    MBnbcolsup=0;
    color MBColSup[];%Sauvegarde les couleurs de la barre sup
    numeric MBnbinfosup[];%compte combien d'infos par couleurs.
    numeric MBnbcasessup[][];
    numeric MBnbcasessuptotal;
    MBnbcasessuptotal=0;
    string MBTextcasessup[][];
    vardef toto(text t)=
    for p_=t:
    if color p_:
    MBnbcolsup:=MBnbcolsup+1;% on incrémente le nombre de couleurs
    MBColSup[MBnbcolsup]:=p_;% on définit la couleur de la série de cases
    MBnbinfosup[MBnbcolsup]:=0;
    else:
    if numeric p_:
    MBnbinfosup[MBnbcolsup]:=MBnbinfosup[MBnbcolsup]+1;
    MBnbcasessuptotal:=MBnbcasessuptotal+p_;
    MBnbcasessup[MBnbcolsup][MBnbinfosup[MBnbcolsup]]:=p_;
    else:
    MBTextcasessup[MBnbcolsup][MBnbinfosup[MBnbcolsup]]:=p_;
    fi;
    fi;
    endfor;
    enddef;
    
    toto(#1);

    numeric MBnbcolinf;%compte combien de couleurs
    MBnbcolinf=0;
    color MBColInf[];%Sauvegarde les couleurs de la barre inf
    numeric MBnbinfoinf[];%compte combien d'infos par couleurs.
    numeric MBnbcasesinf[][];
    numeric MBnbcasesinftotal;
    MBnbcasesinftotal=0;
    string MBTextcasesinf[][];
    vardef tata(text t)=
    for p_=t:
    if color p_:
    MBnbcolinf:=MBnbcolinf+1;% on incrémente le nombre de couleurs
    MBColInf[MBnbcolinf]:=p_;% on définit la couleur de la série de cases
    MBnbinfoinf[MBnbcolinf]:=0;
    else:
    if numeric p_:
    MBnbinfoinf[MBnbcolinf]:=MBnbinfoinf[MBnbcolinf]+1;
    MBnbcasesinftotal:=MBnbcasesinftotal+p_;
    MBnbcasesinf[MBnbcolinf][MBnbinfoinf[MBnbcolinf]]:=p_;
    else:
    MBTextcasesinf[MBnbcolinf][MBnbinfoinf[MBnbcolinf]]:=p_;
    fi;
    fi;
    endfor;
    enddef;
    
    tata(#2);
    
    pair VecteurDeplasup;
    VecteurDeplasup=(Longueur/MBnbcasessuptotal,0);

    numeric MBnbdepla;
    MBnbdepla=0;
    for k=1 upto MBnbcolsup:
    for l=1 upto MBnbinfosup[k]:
    for j=1 upto MBnbcasessup[k][l]:
    trace Briquesup(MBColSup[k])(MBTextcasessup[k][l]) shifted(MBnbdepla*VecteurDeplasup);
    MBnbdepla:=MBnbdepla+1;
    endfor;
    endfor;
    endfor;

    pair VecteurDeplainf;
    VecteurDeplainf=(Longueur/MBnbcasesinftotal,0);

    numeric MBnbdepla;
    MBnbdepla=0;
    for k=1 upto MBnbcolinf:
    for l=1 upto MBnbinfoinf[k]:
    for j=1 upto MBnbcasesinf[k][l]:
    trace Briqueinf(MBColInf[k])(MBTextcasesinf[k][l]) shifted(MBnbdepla*VecteurDeplainf+(0,-Hauteur));
    MBnbdepla:=MBnbdepla+1;
    endfor;
    endfor;
    endfor;
  \end{mplibcode}
}%

\def\PfCMPDessineModelBarreAccoladeCode{%
}%

\newcommand\PfCMPDessineModelBarreNonHomogeneMultiplicatif[5]{%
  \ifluatex%
    \mplibforcehmode%
    \begin{mplibcode}
      \PfCMPDessineModelBarreNHCode
      \PfCMPDessineModelBarreBriqueCode
      % \PfCMPDessineModelBarreAccoladeCode

        vardef TraceAccoH(text t)=
  pair PfCAccoPtH[];
  numeric etape,ProporHauteur;
  etape:=0;
  ProporHauteur:=1;
  numeric decaetape[][];
  string toto;
  numeric n;
  n=0;
  for p_=t:
  n:=n+1;
  if (n mod 3)=1:
  etape:=etape+1;
  decaetape[etape][1]=p_;
  fi;
  if (n mod 3)=2:
  decaetape[etape][2]=p_;
  fi;
  if (n mod 3)=0:
    toto:=p_;
  if substring(1,2) of toto="*":
  ltoto=length toto;
  ProporHauteur:=scantokens(substring(0,1) of toto)+1;
  toto:=substring(2,ltoto) of toto;
  else:
  ProporHauteur:=1;
  fi;
  PfCAccoPtH[1]:=(0,ProporHauteur*Hauteur) shifted(VecteurDepla*(decaetape[etape][1]-1));
  PfCAccoPtH[2]:=PfCAccoPtH[1] shifted(VecteurDepla*decaetape[etape][2]);
  if ProporHauteur>1:
  draw PfCAccoPtH[1]--(PfCAccoPtH[1] shifted(0,-(ProporHauteur-1)*Hauteur)) dashed evenly;
  draw PfCAccoPtH[2]--(PfCAccoPtH[2] shifted(0,-(ProporHauteur-1)*Hauteur)) dashed evenly;
  fi;
  label(TEX("$\overbrace{\hbox to"&decimal(abs(PfCAccoPtH[1]-PfCAccoPtH[2]))&"pt{\PfCTBstrut}}^{"&toto&"}$"),1/2[PfCAccoPtH[1],PfCAccoPtH[2]]);
  fi;      
  endfor;
  enddef;
  % 
  vardef TraceAccoB(text t)=
  pair PfCAccoPtB[];
  numeric etape,ProporHauteur;
  etape:=0;
  ProporHauteur:=1;
  numeric decaetape[][];
  string toto;
  numeric n;
  n=0;
  for p_=t:
  n:=n+1;
  if (n mod 3)=1:
  etape:=etape+1;
  decaetape[etape][1]=p_;
  fi;
  if (n mod 3)=2:
  decaetape[etape][2]=p_;
  fi;
  if (n mod 3)=0:
  toto:=p_;
  if substring(1,2) of toto="*":
  ltoto=length toto;
  ProporHauteur:=scantokens(substring(0,1) of toto)+1;
  toto:=substring(2,ltoto) of toto;
  else:
  ProporHauteur:=1;
  fi;
  PfCAccoPtB[1]:=(0,-ProporHauteur*Hauteur) shifted(VecteurDepla*(decaetape[etape][1]-1));
  PfCAccoPtB[2]:=PfCAccoPtB[1] shifted(VecteurDepla*decaetape[etape][2]);
  if ProporHauteur>1:
  draw PfCAccoPtB[1]--(PfCAccoPtB[1] shifted(0,(ProporHauteur-1)*Hauteur)) dashed evenly;
  draw PfCAccoPtB[2]--(PfCAccoPtB[2] shifted(0,(ProporHauteur-1)*Hauteur)) dashed evenly;
  fi;
  label(TEX("$\underbrace{\hbox to"&decimal(abs(PfCAccoPtB[1]-PfCAccoPtB[2]))&"pt{\PfCTBstrut}}_{"&toto&"}$"),1/2[PfCAccoPtB[1],PfCAccoPtB[2]]);
  fi;      
  endfor;
  enddef;
      
      Longueur:=\useKV[ClesModeleBarre]{Largeur};
      Hauteur:=\useKV[ClesModeleBarre]{Hauteur};
      u:=Longueur;
      
      toto(#1);
      tata(#2);
      
      pair VecteurDepla;
      VecteurDepla=(Longueur,0);
      
      numeric MBnbdepla;
      MBnbdepla=0;
      for k=1 upto MBnbcolsup:
      for l=1 upto MBnbinfosup[k]:
      trace Brique(MBColSup[k],MBnbcasessup[k][l])(MBTextcasessup[k][l]) shifted(MBnbdepla*VecteurDepla);
      MBnbdepla:=MBnbdepla+MBnbcasessup[k][l];
      endfor;
      endfor;
      
      MBnbdepla:=0;
      for k=1 upto MBnbcolinf:
      for l=1 upto MBnbinfoinf[k]:
      trace Brique(MBColInf[k],MBnbcasesinf[k][l])(MBTextcasesinf[k][l]) shifted(MBnbdepla*VecteurDepla+(0,-Hauteur));
      if MBnbcasesinf[k][l]<0:
      for j=1 upto (abs(MBnbcasesinf[k][l])-1):
      trace ((0,0)--(0,Hauteur)) shifted((MBnbdepla+j)*VecteurDepla+(0,-Hauteur)) withcolor black;
      endfor;
      fi;
      MBnbdepla:=MBnbdepla+abs(MBnbcasesinf[k][l]);
      endfor;
      endfor;
      
      Separation(#3);
      
      TraceAccoH(#4);
      TraceAccoB(#5);
    \end{mplibcode}%
  \fi
}

\def\PfCMPDessineModelBarreNHCode{%
  numeric MBnbcolinf;%compte combien de couleurs
  MBnbcolinf=0;
  color MBColInf[];%Sauvegarde les couleurs de la barre inf
  numeric MBnbinfoinf[];%compte combien d'infos par couleurs.
  numeric MBnbcasesinf[][];
  numeric MBnbcasesinftotal;
  MBnbcasesinftotal=0;
  string MBTextcasesinf[][];
  vardef tata(text t)=
  for p_=t:
  if color p_:
  MBnbcolinf:=MBnbcolinf+1;% on incrémente le nombre de couleurs
  MBColInf[MBnbcolinf]:=p_;% on définit la couleur de la série de cases
  MBnbinfoinf[MBnbcolinf]:=0;
  else:
  if numeric p_:
  MBnbinfoinf[MBnbcolinf]:=MBnbinfoinf[MBnbcolinf]+1;
  MBnbcasesinftotal:=MBnbcasesinftotal+p_;
  MBnbcasesinf[MBnbcolinf][MBnbinfoinf[MBnbcolinf]]:=p_;
  else:
  MBTextcasesinf[MBnbcolinf][MBnbinfoinf[MBnbcolinf]]:=p_;
  fi;
  fi;
  endfor;
  enddef;
  % 
  vardef Separation(text t)=
  for p_=t:
  if p_>0:
  nbdecal:=p_;
  draw (u*nbdecal,-1.5Hauteur)--(u*nbdecal,1.5Hauteur) withpen pencircle scaled 1.5 dashed evenly withcolor red;
  fi;
  endfor;
  enddef;
  numeric MBnbcolsup;%compte combien de couleurs
    MBnbcolsup=0;
    color MBColSup[];%Sauvegarde les couleurs de la barre sup
    numeric MBnbinfosup[];%compte combien d'infos par couleurs.
    numeric MBnbcasessup[][];
    numeric MBnbcasessuptotal;
    MBnbcasessuptotal=0;
    string MBTextcasessup[][];
    vardef toto(text t)=
    for p_=t:
    if color p_:
    MBnbcolsup:=MBnbcolsup+1;% on incrémente le nombre de couleurs
    MBColSup[MBnbcolsup]:=p_;% on définit la couleur de la série de cases
    MBnbinfosup[MBnbcolsup]:=0;
    else:
    if numeric p_:
    MBnbinfosup[MBnbcolsup]:=MBnbinfosup[MBnbcolsup]+1;
    MBnbcasessuptotal:=MBnbcasessuptotal+p_;
    MBnbcasessup[MBnbcolsup][MBnbinfosup[MBnbcolsup]]:=p_;
    else:
    MBTextcasessup[MBnbcolsup][MBnbinfosup[MBnbcolsup]]:=p_;
    fi;
    fi;
    endfor;
    enddef;
}%

\def\PfCMPDessineModelBarreBriqueCode{%
  vardef Brique(expr col,nbfois)(text t)=
  save $;
  picture $;
  $=image(
  if t="?":
  draw polygone((0,0),(u*abs(nbfois),0),(u*abs(nbfois),Hauteur),(0,Hauteur)) dashed evenly;
  else:
  remplis polygone((0,0),(u*abs(nbfois),0),(u*abs(nbfois),Hauteur),(0,Hauteur)) withcolor col;
  draw polygone((0,0),(u*abs(nbfois),0),(u*abs(nbfois),Hauteur),(0,Hauteur));
  fi;
  if nbfois>0:
  label(TEX(t),iso((0,0),(u*abs(nbfois),Hauteur)));
  else:
  label.bot(TEX(t),iso((0,0),(u*abs(nbfois),-Hauteur)));
  drawdblarrow (0,-0.5*Hauteur)--(u*abs(nbfois),-0.5*Hauteur);
  fi;
  );
  $
  enddef;
}%

\def\PfCMPDessineModelBarreBriqueCodePDF{%
  vardef Brique(expr col,nbfois)(text t)=
  save $;
  picture $;
  $=image(
  if t="?":
  draw polygone((0,0),(u*abs(nbfois),0),(u*abs(nbfois),Hauteur),(0,Hauteur)) dashed evenly;
  else:
  remplis polygone((0,0),(u*abs(nbfois),0),(u*abs(nbfois),Hauteur),(0,Hauteur)) withcolor col;
  draw polygone((0,0),(u*abs(nbfois),0),(u*abs(nbfois),Hauteur),(0,Hauteur));
  fi;
  if nbfois>0:
  label(LATEX(t),iso((0,0),(u*abs(nbfois),Hauteur)));
  else:
  label.bot(LATEX(t),iso((0,0),(u*abs(nbfois),-Hauteur)));
  drawdblarrow (0,-0.5*Hauteur)--(u*abs(nbfois),-0.5*Hauteur);
  fi;
  );
  $
  enddef;
}%
    
\newcommand\PfCMPDessineModelBarreNonHomogene[3]{%
  \ifluatex%
  \mplibforcehmode%
  \begin{mplibcode}
    \PfCMPDessineModelBarreNHCode
    \PfCMPDessineModelBarreBriqueCode
    
    Longueur:=\useKV[ClesModeleBarre]{Largeur};
    Hauteur:=\useKV[ClesModeleBarre]{Hauteur};
    u:=Longueur;
        
    toto(#1);
    tata(#2);
    
    pair VecteurDepla;
    VecteurDepla=(Longueur,0);

    numeric MBnbdepla;
    MBnbdepla=0;
    for k=1 upto MBnbcolsup:
    for l=1 upto MBnbinfosup[k]:
    trace Brique(MBColSup[k],MBnbcasessup[k][l])(MBTextcasessup[k][l]) shifted(MBnbdepla*VecteurDepla);
    MBnbdepla:=MBnbdepla+MBnbcasessup[k][l];
    endfor;
    endfor;

    MBnbdepla:=0;
    for k=1 upto MBnbcolinf:
    for l=1 upto MBnbinfoinf[k]:
    trace Brique(MBColInf[k],MBnbcasesinf[k][l])(MBTextcasesinf[k][l]) shifted(MBnbdepla*VecteurDepla+(0,-Hauteur));
    if MBnbcasesinf[k][l]<0:
    for j=1 upto (abs(MBnbcasesinf[k][l])-1):
    trace ((0,0)--(0,Hauteur)) shifted((MBnbdepla+j)*VecteurDepla+(0,-Hauteur)) withcolor black;
    endfor;
    fi;
    MBnbdepla:=MBnbdepla+abs(MBnbcasesinf[k][l]);
    endfor;
    endfor;

    Separation(#3);
  \end{mplibcode}%
  \else
  \begin{mpost}[mpsettings={\PfCMPDessineModelBarreNHCode;\PfCMPDessineModelBarreBriqueCodePDF;Longueur:=\useKV[ClesModeleBarre]{Largeur};Hauteur:=\useKV[ClesModeleBarre]{Hauteur};u:=Longueur;}]
    toto(#1);
    tata(#2);
    
    pair VecteurDepla;
    VecteurDepla=(Longueur,0);

    numeric MBnbdepla;
    MBnbdepla=0;
    for k=1 upto MBnbcolsup:
    for l=1 upto MBnbinfosup[k]:
    trace Brique(MBColSup[k],MBnbcasessup[k][l])(MBTextcasessup[k][l]) shifted(MBnbdepla*VecteurDepla);
    MBnbdepla:=MBnbdepla+MBnbcasessup[k][l];
    endfor;
    endfor;

    MBnbdepla:=0;
    for k=1 upto MBnbcolinf:
    for l=1 upto MBnbinfoinf[k]:
    trace Brique(MBColInf[k],MBnbcasesinf[k][l])(MBTextcasesinf[k][l]) shifted(MBnbdepla*VecteurDepla+u*(0,-0.5));
    if MBnbcasesinf[k][l]<0:
    for j=1 upto (abs(MBnbcasesinf[k][l])-1):
    trace ((0,0)--(0,Hauteur)) shifted((MBnbdepla+j)*VecteurDepla+u*(0,-0.5)) withcolor black;
    endfor;
    fi;
    MBnbdepla:=MBnbdepla+abs(MBnbcasesinf[k][l]);
    endfor;
    endfor;

    Separation(#3);
  \end{mpost}
  \fi
}%