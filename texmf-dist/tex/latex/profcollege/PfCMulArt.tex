%%%
% Mulart
%%%
\setKVdefault[TabMul]{Multiple=2,Cibles=false,Largeur=20pt,Couleurs=false,Graines=false,Operations=false,Enonce=false,SchemaEnonce=false}
\defKV[TabMul]{Couleur=\setKV[TabMul]{Couleurs}}
\defKV[TabMul]{Graine=\setKV[TabMul]{Graines}\PfCGraineAlea{#1}}
\defKV[TabMul]{Cible=\setKV[TabMul]{Cibles}}%

\newtoks\tokstabmul
\newtoks\tokstabmulpdts
\newtoks\tokstabmulnonpdts
\newtoks\tokstabmulpdtscible
\def\UpdatetoksTabMul#1\nil{\addtotok\tokstabmul{"#1",}}%
\def\UpdatetoksTabMulPdts#1\nil{\addtotok\tokstabmulpdts{"#1",}}%
\def\UpdatetoksTabMulNonPdts#1\nil{\addtotok\tokstabmulnonpdts{"#1",}}%
\def\UpdatetoksTabMulPdtsCible#1\nil{\addtotok\tokstabmulpdtscible{"#1",}}%

\NewDocumentCommand\TableauMultiplicatif{om}{%
  \useKVdefault[TabMul]%
  \setKV[TabMul]{#1}%
  \setsepchar{,}\ignoreemptyitems%
  \readlist*\PfCListeTabMul{#2}%
  \reademptyitems%
  \StrLen{\PfCListeTabMul[1]}[\TailleTabMul]%
  \tokstabmul{}%
  \foreachitem\compteur\in\PfCListeTabMul{\expandafter\UpdatetoksTabMul\compteur\nil}%
  \savecomparemode%
  \comparestrict%
  \ifboolKV[TabMul]{Cibles}{%
    \tokstabmulpdtscible{}%
    \xdef\PfCCompteurX{0}%
    \xdef\PfCCompteurO{0}%
    \xintFor* ##1 in {\xintSeq{1}{\PfCListeTabMullen}}\do{%
      \xintFor* ##2 in {\xintSeq{1}{\TailleTabMul}}\do{%
        \StrChar{\PfCListeTabMul[##1]}{##2}[\PfCLettre]%
        \IfStrEq{\PfCLettre}{x}{%
          \xdef\PfCCompteurX{\fpeval{\PfCCompteurX+1}}%
        }{\IfStrEq{\PfCLettre}{o}{%
            \xdef\PfCCompteurO{\fpeval{\PfCCompteurO+1}}%
          }{}%
        }%
      }%
    }%
    \xdef\PfCFooRetiensCible{\useKV[TabMul]{Cible}}%
    \readlist*\PfCListeDesNombresCibles{\PfCFooRetiensCible}%
    \xintFor* ##2 in{\xintSeq{1}{\PfCListeDesNombresCibleslen}}\do{%
      \PfCListeDiviseursInfDix{\PfCListeDesNombresCibles[##2]}%
      \setsepchar{,}\ignoreemptyitems%
      \readlist*\PfCTabMulDiviseursCible{\PfCListeDiviseursInfDixRetour}%
      \xdef\PfCFooListePdtCible{$1\times\num{\PfCListeDesNombresCibles[##2]}$,$\num{\PfCListeDesNombresCibles[##2]}\times1$}%
      \xintFor* ##1 in {\xintSeq{1}{\PfCNbDiviseurs}}\do{%
        \xdef\PfCFooListePdtCible{\PfCFooListePdtCible,$\num{\PfCTabMulDiviseursCible[##1]}\times\num{\fpeval{\PfCListeDesNombresCibles[##2]/\PfCTabMulDiviseursCible[##1]}}$,$\num{\fpeval{\PfCListeDesNombresCibles[##2]/\PfCTabMulDiviseursCible[##1]}}\times\num{\PfCTabMulDiviseursCible[##1]}$}%
      }%
    }%
    \ignoreemptyitems%
    \readlist*\PfCListeProduitsCible{\PfCFooListePdtCible}%
    \foreachitem\compteur\in\PfCListeProduitsCible{\expandafter\UpdatetoksTabMulPdtsCible\compteur\nil}%
    \BuildTabMulCible{\the\tokstabmul}{\the\tokstabmulpdtscible}{\PfCFooRetiensCible}%
  }{%
    \ifboolKV[TabMul]{Operations}{%
      \xdef\PfCTabMulListePdts{4,6,8,9,10,12,14,15,16,18,20,21,24,25,27,28,30,32,35,36,40,42,45,48,49,50,54,56,60,63,64,70,72,80,81,90,100}%37
      \xdef\PfCTabMulListeNonPdts{2,3,5,7,11,13,17,19,22,23,26,29,31,33,34,37,38,39,41,43,44,46,47,51,52,53,55,57,58,59,61,62,65,66,67,68,69,71,73,74,75,76,77,78,79,82,83,84,85,86,87,88,89,91,92,93,94,95,96,97,98,99}%62
      \xdef\PfCCompteurX{0}
      \xdef\PfCCompteurO{0}
      \xintFor* ##1 in {\xintSeq{1}{\PfCListeTabMullen}}\do{%
        \xintFor* ##2 in {\xintSeq{1}{\TailleTabMul}}\do{%
          \StrChar{\PfCListeTabMul[##1]}{##2}[\PfCLettre]%
          \IfStrEq{\PfCLettre}{x}{%
            \xdef\PfCCompteurX{\fpeval{\PfCCompteurX+1}}%
          }{\IfStrEq{\PfCLettre}{o}{%
              \xdef\PfCCompteurO{\fpeval{\PfCCompteurO+1}}%
            }{}%
          }%
        }%
      }%
      \xdef\NombreRepetX{\fpeval{ceil(\PfCCompteurX/37)}}%
      \xdef\NombreRepetO{\fpeval{ceil(\PfCCompteurO/62)}}%
      \xdef\PfCTabMulListeFinalePdts{}
      \xdef\PfCTabMulListeFinaleNonPdts{}
      \xintFor* ##1 in {\xintSeq{1}{\NombreRepetX}}\do{%
        \xdef\PfCTabMulListeFinalePdts{\PfCTabMulListeFinalePdts,\PfCTabMulListePdts}
      }%
      \xintFor* ##1 in {\xintSeq{1}{\NombreRepetO}}\do{%
        \xdef\PfCTabMulListeFinaleNonPdts{\PfCTabMulListeFinaleNonPdts,\PfCTabMulListeNonPdts}
      }%
      \MelangeListe{\PfCTabMulListeFinalePdts}{\PfCCompteurX}
      \ignoreemptyitems
      \readlist*\PfCListeProduits{\faa}
      \MelangeListe{\PfCTabMulListeFinaleNonPdts}{\PfCCompteurO}
      \ignoreemptyitems
      \readlist*\PfCListeNonProduits{\faa}
      \ifboolKV[TabMul]{Enonce}{%
        \begin{enumerate}
          \xintFor* ##1 in {\xintSeq{1}{\PfCCompteurX}}\do{%
            \PfCListeDiviseursInfDix{\PfCListeProduits[##1]}%
            \ifnum\PfCNbDiviseurs>1\relax
              \MelangeListe{\PfCListeDiviseursInfDixRetour}{1}%
            \else
              \xdef\faa{\PfCListeDiviseursInfDixRetour}
            \fi
            \ignoreemptyitems
            \readlist*\PfCListeDiviseurChoisi{\faa}%
          \item $\PfCListeDiviseurChoisi[1]\times\fpeval{\PfCListeProduits[##1]/\PfCListeDiviseurChoisi[1]}=$
          }
        \end{enumerate}
      }{}
      \ifboolKV[TabMul]{SchemaEnonce}{%
        \tokstabmulpdts{}
        \tokstabmulnonpdts{}
        \xdef\PfCFoo{\PfCListeProduits[1]}
        \xintFor* ##3 in {\xintSeq{2}{\PfCCompteurX}}\do{%
          \xdef\PfCFoo{\PfCFoo,\PfCListeProduits[##3]}%
        }%
        \MelangeListe{\PfCFoo}{\PfCCompteurX}%
        \ignoreemptyitems
        \readlist*\PfCListeProduits{\faa}%
        \foreachitem\compteur\in\PfCListeProduits{\expandafter\UpdatetoksTabMulPdts\compteur\nil}%
        \foreachitem\compteur\in\PfCListeNonProduits{\expandafter\UpdatetoksTabMulNonPdts\compteur\nil}%
        \BuildTabMulEnonce{\the\tokstabmul}{\the\tokstabmulpdts}{\the\tokstabmulnonpdts}
      }{}%%
    }{%
      \BuildTabMul{\the\tokstabmul}%
    }
  }
  \restorecomparemode%
  \reademptyitems%
}%

\NewDocumentCommand\PfCListeDiviseursInfDix{m}{%
  \ifnum#1<10\relax
    \xdef\PfCLimiteDiv{#1}
  \else
    \xdef\PfCLimiteDiv{10}
  \fi
  \xdef\PfCListeDiviseursInfDixRetour{}
  \xdef\PfCNbDiviseurs{}
  \xintFor* ##2 in {\xintSeq{2}{\PfCLimiteDiv}}\do{%
    \modulo{#1}{##2}\relax
    \ifnum\remainder=0\relax
      \xdef\PfCListeDiviseursInfDixRetour{\PfCListeDiviseursInfDixRetour ##2,}%
      \xdef\PfCNbDiviseurs{\fpeval{\PfCNbDiviseurs+1}}%
    \fi
  }%
}%

\def\TabMulCibleCode{%
  NbCasesLarg=\TailleTabMul;
  Largeur=\useKV[TabMul]{Largeur};
  color CoulFond;
  boolean Colore,Stop,Graines;
  Colore=\useKV[TabMul]{Couleurs};
  if Colore:
    CoulFond=\useKV[TabMul]{Couleur};
  fi;
  Graines=\useKV[TabMul]{Graines};
  if Graines:
    randomseed:=\useKV[TabMul]{Graine};
  fi;
  vardef NombresCibles(text t)=
    nbc=0;
    for p_=t:
      nbc:=nbc+1;
    Cible[nbc]=p_;
    endfor;
  enddef;
  % 
  vardef LectureDesProduits(text t)=
    string Sproduits[];
    nbp=0;
    for p_=t:
      nbp:=nbp+1;
      Sproduits[nbp]=p_;
    endfor;
  enddef;
  %
  vardef Remplissage(text t)=
    pair M[][];
    n=0;
    for p_=t:
      for k=0 upto NbCasesLarg-1:
        l:=n mod NbCasesLarg;
        h:=n div NbCasesLarg;
        M[l][h]=Largeur*(l,-h);
        if (substring(k,k+1) of p_)="x":
          if Colore:
            fill ((unitsquare scaled Largeur) shifted M[l][h]) withcolor CoulFond;
          fi;
          label(TEX(Sproduits[ceiling(uniformdeviate(nbp))]),M[l][h] shifted (Largeur*(0.5,0.5)));
        elseif (substring(k,k+1) of p_)="X":
          fill ((unitsquare scaled Largeur) shifted M[l][h]);
        else:
          Stop:=false;
          forever:
            basea:=ceiling(1+uniformdeviate(9));
            baseb:=ceiling(1+uniformdeviate(9));
            basec:=0;
            for r=1 upto nbc:
              if (basea*baseb)<>Cible[r]:
                basec:=basec+1;
              fi;
            endfor;
            if basec=nbc:
              Stop:=true;
            fi;
          exitif Stop;
          endfor;
          label(TEX("$\num{"&decimal(basea)&"}\times\num{"&decimal(baseb)&"}$"),M[l][h] shifted (Largeur*(0.5,0.5)));
        fi;
        trace (unitsquare scaled Largeur) shifted M[l][h];
        n:=n+1;
      endfor;
    endfor;
  enddef;
  %
  vardef RemplissagePDF(text t)=
    pair M[][];
    n=0;
    for p_=t:
      for k=0 upto NbCasesLarg-1:
        l:=n mod NbCasesLarg;
        h:=n div NbCasesLarg;
        M[l][h]=Largeur*(l,-h);
        if (substring(k,k+1) of p_)="x":
          if Colore:
            fill ((unitsquare scaled Largeur) shifted M[l][h]) withcolor CoulFond;
          fi;
          label(LATEX(Sproduits[ceiling(uniformdeviate(nbp))]),M[l][h] shifted (Largeur*(0.5,0.5)));
        elseif (substring(k,k+1) of p_)="X":
          fill ((unitsquare scaled Largeur) shifted M[l][h]);
        else:
          Stop:=false;
          forever:
            basea:=ceiling(1+uniformdeviate(9));
            baseb:=ceiling(1+uniformdeviate(9));
            basec:=0;
            for r=1 upto nbc:
              if (basea*baseb)<>Cible[r]:
                basec:=basec+1;
              fi;
            endfor;
            if basec=nbc:
              Stop:=true;
            fi;
          exitif Stop;
          endfor;
          label(LATEX("$\num{"&decimal(basea)&"}\times\num{"&decimal(baseb)&"}$"),M[l][h] shifted (Largeur*(0.5,0.5)));
        fi;
        trace (unitsquare scaled Largeur) shifted M[l][h];
        n:=n+1;
      endfor;
    endfor;
  enddef;
}%

\NewDocumentCommand\BuildTabMulCible{mmm}{%
  \ifluatex
    \mplibforcehmode
    \mplibnumbersystem{double}
    \begin{mplibcode}
      \TabMulCibleCode
      NombresCibles(#3);
      LectureDesProduits(#2);
      Remplissage(#1);
    \end{mplibcode}
    \mplibnumbersystem{scaled}
  \else
    \begin{mpost}[mpsettings={\TabMulCibleCode}]
      NombresCibles(#3);
      LectureDesProduits(#2);
      RemplissagePDF(#1);
    \end{mpost}
  \fi
}

\def\TabMulCode{%
  NbCasesLarg=\TailleTabMul;
      Multiple=\useKV[TabMul]{Multiple};
      Largeur=\useKV[TabMul]{Largeur};
      color CoulFond;
      boolean Colore,Stop,Graines;
      Colore=\useKV[TabMul]{Couleurs};
      if Colore:
      CoulFond=\useKV[TabMul]{Couleur};
      fi;
      Graines=\useKV[TabMul]{Graines};
      if Graines:
      randomseed:=\useKV[TabMul]{Graine};
      fi;
      vardef Remplissage(text t)=
      pair M[][];
      n=0;
      for p_=t:
      for k=0 upto NbCasesLarg-1:
      l:=n mod NbCasesLarg;
      h:=n div NbCasesLarg;
      M[l][h]=Largeur*(l,-h);
      if (substring(k,k+1) of p_)="x":
      if Colore:
      fill ((unitsquare scaled Largeur) shifted M[l][h]) withcolor CoulFond;
      fi;
      label(TEX(decimal(Multiple*ceiling(uniformdeviate(9)+1))),M[l][h] shifted (Largeur*(0.5,0.5)));
      elseif (substring(k,k+1) of p_)="X":
      fill ((unitsquare scaled Largeur) shifted M[l][h]);
      else:
      Stop:=false;
      forever:
      base:=ceiling((1+uniformdeviate(9))*(1+uniformdeviate(9)));
      if (base mod Multiple)>0:
      Stop:=true;
      fi;
      exitif Stop;
      endfor;
      label(TEX(decimal(base)),M[l][h] shifted (Largeur*(0.5,0.5)));
      fi;
      trace (unitsquare scaled Largeur) shifted M[l][h];
      n:=n+1;
      endfor;
      endfor;
      enddef;
      %
      vardef RemplissagePDF(text t)=
      pair M[][];
      n=0;
      for p_=t:
      for k=0 upto NbCasesLarg-1:
      l:=n mod NbCasesLarg;
      h:=n div NbCasesLarg;
      M[l][h]=Largeur*(l,-h);
      if (substring(k,k+1) of p_)="x":
      if Colore:
      fill ((unitsquare scaled Largeur) shifted M[l][h]) withcolor CoulFond;
      fi;
      label(LATEX(decimal(Multiple*ceiling(uniformdeviate(9)+1))),M[l][h] shifted (Largeur*(0.5,0.5)));
      elseif (substring(k,k+1) of p_)="X":
      fill ((unitsquare scaled Largeur) shifted M[l][h]);
      else:
      Stop:=false;
      forever:
      base:=ceiling((1+uniformdeviate(9))*(1+uniformdeviate(9)));
      if (base mod Multiple)>0:
      Stop:=true;
      fi;
      exitif Stop;
      endfor;
      label(LATEX(decimal(base)),M[l][h] shifted (Largeur*(0.5,0.5)));
      fi;
      trace (unitsquare scaled Largeur) shifted M[l][h];
      n:=n+1;
      endfor;
      endfor;
      enddef;
    }
    
\NewDocumentCommand\BuildTabMul{m}{%
  \ifluatex
    \mplibforcehmode
    \mplibnumbersystem{double}
    \begin{mplibcode}
      \TabMulCode
      Remplissage(#1);
    \end{mplibcode}
    \mplibnumbersystem{scaled}
  \else
    \begin{mpost}[mpsettings={\TabMulCode}]
      Remplissage(#1);
    \end{mpost}
  \fi
}

\def\TabMulEnonceCode{
  NbCasesLarg=\TailleTabMul;
      Multiple=\useKV[TabMul]{Multiple};
      Largeur=\useKV[TabMul]{Largeur};
      color CoulFond;
      boolean Colore,Stop;%,Graines;
      Colore=\useKV[TabMul]{Couleurs};
      if Colore:
      CoulFond=\useKV[TabMul]{Couleur};
      fi;
      vardef Remplissage(text t)(text listepdt)(text listenonpdt)=
      string pdt[],nonpdt[];
      n:=0;
      for p_=listepdt:
      pdt[n]=p_;
      n:=n+1;
      endfor;
      n:=0;
      for p_=listenonpdt:
      nonpdt[n]=p_;
      n:=n+1;
      endfor;
      pair M[][];
      n:=0;
      nx=0;
      no=0;
      for p_=t:
      for k=0 upto NbCasesLarg-1:
      l:=n mod NbCasesLarg;
      h:=n div NbCasesLarg;
      M[l][h]=Largeur*(l,-h);
      if (substring(k,k+1) of p_)="x":
      if Colore:
      fill ((unitsquare scaled Largeur) shifted M[l][h]) withcolor CoulFond;
      fi;
      label(TEX(pdt[nx]),M[l][h] shifted (Largeur*(0.5,0.5)));
      nx:=nx+1;
      elseif (substring(k,k+1) of p_)="X":
      fill ((unitsquare scaled Largeur) shifted M[l][h]);
      else:
      label(TEX(nonpdt[no]),M[l][h] shifted (Largeur*(0.5,0.5)));
      no:=no+1;
      fi;
      trace (unitsquare scaled Largeur) shifted M[l][h];
      n:=n+1;
      endfor;
      endfor;
      enddef;
      vardef RemplissagePDF(text t)(text listepdt)(text listenonpdt)=
      string pdt[],nonpdt[];
      n:=0;
      for p_=listepdt:
      pdt[n]=p_;
      n:=n+1;
      endfor;
      n:=0;
      for p_=listenonpdt:
      nonpdt[n]=p_;
      n:=n+1;
      endfor;
      pair M[][];
      n:=0;
      nx=0;
      no=0;
      for p_=t:
      for k=0 upto NbCasesLarg-1:
      l:=n mod NbCasesLarg;
      h:=n div NbCasesLarg;
      M[l][h]=Largeur*(l,-h);
      if (substring(k,k+1) of p_)="x":
      if Colore:
      fill ((unitsquare scaled Largeur) shifted M[l][h]) withcolor CoulFond;
      fi;
      label(TEX(pdt[nx]),M[l][h] shifted (Largeur*(0.5,0.5)));
      nx:=nx+1;
      elseif (substring(k,k+1) of p_)="X":
      fill ((unitsquare scaled Largeur) shifted M[l][h]);
      else:
      label(TEX(nonpdt[no]),M[l][h] shifted (Largeur*(0.5,0.5)));
      no:=no+1;
      fi;
      trace (unitsquare scaled Largeur) shifted M[l][h];
      n:=n+1;
      endfor;
      endfor;
      enddef;
    }

\NewDocumentCommand\BuildTabMulEnonce{mmm}{%
  \ifluatex
    \mplibforcehmode
    \mplibnumbersystem{double}
    \begin{mplibcode}
      \TabMulEnonceCode
      Remplissage(#1)(#2)(#3);
    \end{mplibcode}
    \mplibnumbersystem{scaled}
  \else
    \begin{mpost}[mpsettings={\TabMulEnonceCode}]
      RemplissagePDF(#1)(#2)(#3);
    \end{mpost}
  \fi
}