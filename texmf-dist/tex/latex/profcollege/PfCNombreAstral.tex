%%%
% Nombre Astral
%%%
\setKVdefault[NombreAstral]{Solution=false,Graines=false,Echelle=1,Couleur=LightSteelBlue,Horizontal=false}
\defKV[NombreAstral]{Graine=\setKV[NombreAstral]{Graines}}%

\newtoks\toklisteNANombres%
\newtoks\toklisteNAMelange%
\def\UpdatetoksNANombres#1\nil{\addtotok\toklisteNANombres{"#1",}}%
\def\UpdatetoksNAMelange#1\nil{\addtotok\toklisteNAMelange{#1,}}%

\NewDocumentCommand\NombreAstral{o}{%
  \toklisteNANombres{}%
  \toklisteNAMelange{}%
  \useKVdefault[NombreAstral]%
  \setKV[NombreAstral]{#1}%
  \ifboolKV[NombreAstral]{Graines}{\PfCGraineAlea{\useKV[NombreAstral]{Graine}}}{}%
  \xdef\PfCNACible{\fpeval{2*randint(25,45)}}%
  \xdef\PfCNAListeNombres{}%
  \xintFor* ##1 in{\xintSeq{11}{\fpeval{floor(\PfCNACible/3)}}}\do{%
    \xdef\PfCNAListeNombres{\PfCNAListeNombres ##1,}%
  }%
  \MelangeListe{\PfCNAListeNombres}{5}%
  \setsepchar{,}\ignoreemptyitems%
  \readlist*\PfCNAListeNombresBase{\faa}%
  \reademptyitems
  \readlist*\PfCNAListeNombres{%
    \PfCNAListeNombresBase[1],%
    \PfCNAListeNombresBase[2],%
    \fpeval{0.5*\PfCNACible+\PfCNAListeNombresBase[5]-\PfCNAListeNombresBase[4]-\PfCNAListeNombresBase[3]},%
    \fpeval{0.5*\PfCNACible+\PfCNAListeNombresBase[4]-\PfCNAListeNombresBase[2]-\PfCNAListeNombresBase[5]},%
    \PfCNAListeNombresBase[3],%
    \fpeval{0.5*\PfCNACible+\PfCNAListeNombresBase[3]-\PfCNAListeNombresBase[5]-\PfCNAListeNombresBase[1]},%
    \fpeval{0.5*\PfCNACible+\PfCNAListeNombresBase[2]-\PfCNAListeNombresBase[4]-\PfCNAListeNombresBase[1]},%
    \fpeval{0.5*\PfCNACible+\PfCNAListeNombresBase[1]-\PfCNAListeNombresBase[2]-\PfCNAListeNombresBase[3]},%
    \PfCNAListeNombresBase[4],%
    \PfCNAListeNombresBase[5]%
  }%
  \foreachitem\compteur\in\PfCNAListeNombres{\expandafter\UpdatetoksNANombres\compteur\nil}%
  \xdef\PfCNAListeAEffacer{1,2,3,4,5,6,7,8,9,10}%
  \MelangeListe{\PfCNAListeAEffacer}{6}%
  \ignoreemptyitems%
  \readlist*\PfCNAListeMelange{\faa}%
  \reademptyitems%
  \foreachitem\compteur\in\PfCNAListeMelange{\expandafter\UpdatetoksNAMelange\compteur\nil}%
  \BuildNombreAstral{\the\toklisteNANombres}{\the\toklisteNAMelange}%
}%

\NewDocumentCommand\BuildNombreAstral{m m}{%
  \ifluatex
  \mplibforcehmode
  \begin{mplibcode}
    Cible=\PfCNACible;

    color MiseEnValeurSol;
    MiseEnValeurSol=\useKV[NombreAstral]{Couleur};
    
    boolean Solution,Horizontal;
    Solution=\useKV[NombreAstral]{Solution};
    Horizontal=\useKV[NombreAstral]{Horizontal};

    Echelle:=\useKV[NombreAstral]{Echelle};
    
    string Rappel[];
    Rayon:=4cm;
    
    path cc,ca[];
    cc=cercles((0,0),Rayon);
    
    pair A[],C[],D[];
    for k=1 upto 5:
    C[k]=pointarc(cc,90+(k-1)*72);
    endfor;
    
    D1=C2+30*unitvector(C2-C4);
    D2=C2+30*unitvector(C2-C5);
    D3=C3+30*unitvector(C3-C1);
    D4=C4+30*unitvector(C4-C1);
    D5=C5+30*unitvector(C5-C3);
    
    drawarrow C2--D1;
    drawarrow C2--D2;
    drawarrow C3--D3;
    drawarrow C4--D4;
    drawarrow C5--D5;
    
    label.ulft(TEX("\bfseries\Large"&decimal(Cible)),D1);
    label.lft(TEX("\bfseries\Large"&decimal(Cible)),D2);
    label.llft(TEX("\bfseries\Large"&decimal(Cible)),D3);
    label.lrt(TEX("\bfseries\Large"&decimal(Cible)),D4);
    label.urt(TEX("\bfseries\Large"&decimal(Cible)),D5);
    
    trace polygone(C1,C3,C5,C2,C4) withpen (pencircle scaled 8);
    A1=C1;
    A2=C2;
    A9=C3;
    A10=C4;
    A5=C5;
    A3=segment(C1,C3) intersectionpoint segment(C2,C5);
    A4=segment(C1,C4) intersectionpoint segment(C2,C5);
    A6=segment(C1,C3) intersectionpoint segment(C2,C4);
    A7=segment(C1,C4) intersectionpoint segment(C3,C5);
    A8=segment(C2,C4) intersectionpoint segment(C3,C5);
    for k=1 upto 10:
    ca[k]=cercles(A[k],0.6u);
    fill ca[k] withcolor white;
    trace ca[k];
    endfor;

    vardef DefinirRappel(text t)=
    n:=0;
    for p_=t:
    n:=n+1;
    Rappel[n]=p_;
    endfor;
    enddef;
    
    vardef Affichage(text t)=
    n:=0;
    for p_=t:
    n:=n+1;
    label(TEX("\Large"&p_),A[n]);
    endfor;
    enddef;
    
    vardef Suppression(text t)=
    k:=0;
    for p_=t:
    k:=k+1;
    fill ca[p_] withcolor if Solution=false: white else: MiseEnValeurSol fi;
    trace ca[p_];
    if Horizontal:
    label(TEX("\Large"&Rappel[p_]),(xpart(C2)+1.5*(k-1)*u,ypart(C2)-7.5u));
    trace cercles((xpart(C2)+1.5*(k-1)*u,ypart(C2)-7.5u),0.6u);
    else:
    label(TEX("\Large"&Rappel[p_]),u*(-7,5.6-1.5k));
    trace cercles(u*(-7,5.6-1.5k),0.6u);
    fi;
    endfor;
    enddef;

    DefinirRappel(#1);
    if Solution=false:
    Affichage(#1);
    Suppression(#2);
    else:
    Suppression(#2);
    Affichage(#1);
    fi;
    picture recap;%
    recap:=currentpicture scaled Echelle;%
    currentpicture:=nullpicture;%
    draw recap;%
  \end{mplibcode}
  \else
  \begin{mpost}[mpsettings={%
      Echelle:=\useKV[NombreAstral]{Echelle};
      color MiseEnValeurSol;
      MiseEnValeurSol=\useKV[NombreAstral]{Couleur};
      Cible=\PfCNACible;boolean Solution;Solution:=\useKV[NombreAstral]{Solution};
      string Rappel[];
      n:=0;
      for p_=#1:
      n:=n+1;
      Rappel[n]=p_;
      endfor;
    }]
    
    Rayon:=4cm;
    
    path cc,ca[];
    cc=cercles((0,0),Rayon);
    
    pair A[],C[],D[];
    for k=1 upto 5:
    C[k]=pointarc(cc,90+(k-1)*72);
    endfor;
    
    D1=C2+30*unitvector(C2-C4);
    D2=C2+30*unitvector(C2-C5);
    D3=C3+30*unitvector(C3-C1);
    D4=C4+30*unitvector(C4-C1);
    D5=C5+30*unitvector(C5-C3);
    
    drawarrow C2--D1;
    drawarrow C2--D2;
    drawarrow C3--D3;
    drawarrow C4--D4;
    drawarrow C5--D5;
    
    label.ulft(LATEX("\noexpand\bfseries\noexpand\Large"&decimal(Cible)&""),D1);
    label.lft(LATEX("\noexpand\bfseries\noexpand\Large"&decimal(Cible)),D2);
    label.llft(LATEX("\noexpand\bfseries\noexpand\Large"&decimal(Cible)),D3);
    label.lrt(LATEX("\noexpand\bfseries\noexpand\Large"&decimal(Cible)),D4);
    label.urt(LATEX("\noexpand\bfseries\noexpand\Large"&decimal(Cible)),D5);
    
    trace polygone(C1,C3,C5,C2,C4) withpen (pencircle scaled 8);
    A1=C1;
    A2=C2;
    A9=C3;
    A10=C4;
    A5=C5;
    A3=segment(C1,C3) intersectionpoint segment(C2,C5);
    A4=segment(C1,C4) intersectionpoint segment(C2,C5);
    A6=segment(C1,C3) intersectionpoint segment(C2,C4);
    A7=segment(C1,C4) intersectionpoint segment(C3,C5);
    A8=segment(C2,C4) intersectionpoint segment(C3,C5);
    for k=1 upto 10:
    ca[k]=cercles(A[k],0.6u);
    fill ca[k] withcolor white;
    trace ca[k];
    endfor;

    vardef DefinirRappel(text t)=
    n:=0;
    for p_=t:
    n:=n+1;
    Rappel[n]=p_;
    endfor;
    enddef;
    
    vardef Affichage(text t)=
    n:=0;
    for p_=t:
    n:=n+1;
    label(LATEX("\noexpand\Large"&p_),A[n]);
    endfor;
    enddef;
    
    vardef Suppression(text t)=
    k:=0;
    for p_=t:
    k:=k+1;
    fill ca[p_] withcolor if Solution=false: white else: MiseEnValeurSol fi;
    trace ca[p_];
    label(LATEX("\noexpand\Large"&Rappel[p_]),u*(-7,5.6-1.5k));
    trace cercles(u*(-7,5.6-1.5k),0.6u);
    endfor;
    enddef;

    DefinirRappel(#1);
    if Solution=false:
    Affichage(#1);
    Suppression(#2);
    else:
    Suppression(#2);
    Affichage(#1);
    fi;
    picture recap;%
    recap:=currentpicture scaled Echelle;%
    currentpicture:=nullpicture;%
    draw recap;%
  \end{mpost}
  \fi
}%