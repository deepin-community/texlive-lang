%%%
% Fonction
%%%
\def\filedateFonction{2024/08/04}%
\def\fileversionFonction{0.1}%
\message{-- \filedateFonction\space v\fileversionFonction}%
%
\setKVdefault[ClesFonction]{Nom=f,Variable=x,Calcul=x,Tableau=false,Largeur=5mm,Ecriture=false,Definition=false,Points=false,Tangentes=false,PasX=1,PasY=1,UniteX=1,UniteY=1,Prolonge=false,Trace=false,Catmull=false,Splines=false,Lagrange=false,Epaisseur=1,Couleur=black,CouleurTrace=black,PointsCourbe=true,Codes=false,Origine={(0,0)},Vide=false}
\defKV[ClesFonction]{Traces=\setKV[ClesFonction]{Codes}}%

\newtoks\toklistePtsFn%pour la discipline
\newtoks\tokPfCListeDesFonctions%
\newtoks\tokPfCListeDesNomsFonctions%
\newtoks\tokPfCLabelX%
\newtoks\tokPfCLabelY%
\def\UpdatePfCListeDesFonctions#1\nil{\addtotok\tokPfCListeDesFonctions{"#1",}}%
\def\UpdatePfCListeDesNomsFonctions#1\nil{\addtotok\tokPfCListeDesNomsFonctions{"#1",}}%
\def\UpdatePtsFn#1/#2/#3/#4\nil{\addtotok\toklistePtsFn{#1,(#2,#3),#4,}}%
\def\UpdatePtsFN#1/#2/#3/#4\nil{\addtotok\toklistePtsFn{(#2,#3),}}%

\NewDocumentCommand\Fonction{o m}{%
  \useKVdefault[ClesFonction]%
  \setKV[ClesFonction]{#1}%
  \ifboolKV[ClesFonction]{Trace}{%
    \useKVdefault[TraceG]%
    \setKV[TraceG]{#1}%
    \tokPfCListeDesFonctions{}%
    \tokPfCListeDesNomsFonctions{}%
    \xdef\PfCFooFn{\useKV[ClesFonction]{Calcul}}%
    \xdef\PfCFooNomFn{\useKV[TraceG]{NomCourbe}}%
    \ifx\bla\PfCFooFn\bla%
    \else%
      \setsepchar{,}\ignoreemptyitems
      \readlist*\ListeDesFonctions{\PfCFooFn}%
      \reademptyitems
    \foreachitem\compteur\in\ListeDesFonctions{\expandafter\UpdatePfCListeDesFonctions\compteur\nil}%
    \fi%
    \ifx\bla\PfCFooNomFn\bla%
    \else%
    \readlist*\ListeDesNomsFonctions{\PfCFooNomFn}%
    \foreachitem\compteur\in\ListeDesNomsFonctions{\expandafter\UpdatePfCListeDesNomsFonctions\compteur\nil}%
  \fi%
  \tokPfCLabelX{}%
  \tokPfCLabelY{}%
  \addtotok\tokPfCLabelX{"\useKV[TraceG]{LabelX}"}%
  \addtotok\tokPfCLabelY{"\useKV[TraceG]{LabelY}"}%
  \TraceMultiGraphique{\the\tokPfCListeDesFonctions}{\the\tokPfCListeDesNomsFonctions}%
  }{%
    \ifboolKV[ClesFonction]{Points}{%
      \toklistePtsFn{}%
      \setsepchar[*]{§*/}%\ignoreemptyitems%
      \readlist*\ListePoints{#2}%
      \ifboolKV[ClesFonction]{Lagrange}{
          \xdef\PfCFooListePoints{}%
          \foreachitem\compteur\in\ListePoints{%
            \xdef\PfCFooListePoints{\PfCFooListePoints (\ListePoints[\compteurcnt,2],\ListePoints[\compteurcnt,3]),}%
          }%
          \MPLagrange{\PfCFooListePoints}{\useKV[ClesFonction]{PasX}}{\useKV[ClesFonction]{PasY}}{\useKV[ClesFonction]{UniteX}}{\useKV[ClesFonction]{UniteY}}%
        }{%
      \ifboolKV[ClesFonction]{Splines}{%
        \xdef\PfCFooListePoints{}%
          \foreachitem\compteur\in\ListePoints{%
            \xdef\PfCFooListePoints{\PfCFooListePoints (\ListePoints[\compteurcnt,2],\ListePoints[\compteurcnt,3]),(\ListePoints[\compteurcnt,1],\ListePoints[\compteurcnt,4]),}%
          }%
          \MPSpline{\PfCFooListePoints}{\useKV[ClesFonction]{PasX}}{\useKV[ClesFonction]{PasY}}{\useKV[ClesFonction]{UniteX}}{\useKV[ClesFonction]{UniteY}}%
      }{%
        \ifboolKV[ClesFonction]{Catmull}{%
          \xdef\PfCFooListePoints{(\ListePoints[1,2],\ListePoints[1,3]),}%
          \foreachitem\compteur\in\ListePoints{%
            \xdef\PfCFooListePoints{\PfCFooListePoints (\ListePoints[\compteurcnt,2],\ListePoints[\compteurcnt,3]),}%
          }%
          \xdef\PfCFooListePoints{\PfCFooListePoints (\ListePoints[\ListePointslen,2],\ListePoints[\ListePointslen,3]),}%
          \MPCatmull{\PfCFooListePoints}{\useKV[ClesFonction]{PasX}}{\useKV[ClesFonction]{PasY}}{\useKV[ClesFonction]{UniteX}}{\useKV[ClesFonction]{UniteY}}%
        }{%
          \xdef\PfCFooListePoints{}
          \ifboolKV[ClesFonction]{Tangentes}{%
            \foreachitem\compteur\in\ListePoints{%
              \xdef\PfCFooListePoints{\PfCFooListePoints \ListePoints[\compteurcnt,1],(\ListePoints[\compteurcnt,2],\ListePoints[\compteurcnt,3]),\ListePoints[\compteurcnt,4],}%
            }%
          }{%
            \foreachitem\compteur\in\ListePoints{%
              \xdef\PfCFooListePoints{\PfCFooListePoints (\ListePoints[\compteurcnt,2],\ListePoints[\compteurcnt,3]),}%
            }%
          }%
          \MPCourbeNew{\PfCFooListePoints}%
        }%
      }%
    }%
    }{%
      \setsepchar{,}\ignoreemptyitems%
      \readlist*\ListeFonction{#2}%
      \StrSubstitute{\useKV[ClesFonction]{Calcul}}{\useKV[ClesFonction]{Variable}}{\i}[\temp]%
      \StrSubstitute{\useKV[ClesFonction]{Calcul}}{**}{^}[\tempa]%
      \StrSubstitute{\tempa}{*}{}[\tempab]%
      \ifboolKV[ClesFonction]{Ecriture}{%
        \ensuremath{\useKV[ClesFonction]{Nom}(\useKV[ClesFonction]{Variable})=\tempab}%
      }{}%
      \ifboolKV[ClesFonction]{Definition}{%
        \ensuremath{\useKV[ClesFonction]{Nom}:\useKV[ClesFonction]{Variable}\mapsto\tempab}%
      }{}%
      \ifboolKV[ClesFonction]{Tableau}{%
        \buildtabfonction%
      }{}%
      \reademptyitems%
    }%
  }%
}%

\setKVdefault[LecGra]{NbPoints=7,Image=false,Antecedent=false,Equation=false,UniteX=1,UniteY=1,Epaisseur=1,CouleurTrace=black,Points,Tableau=false,SolutionTableau=false,Prolonge=false}%
\defKV[LecGra]{Graine=\PfCGraineAlea{#1}}%

\newread\myread

\NewDocumentCommand\LectureGraphique{o}{%
  \useKVdefault[LecGra]%
  \setKV[LecGra]{#1}%
  % Partie Trace
  % On choisit les abscisses
  \xdef\FooAbscisses{\fpeval{randint(-9,-5)}}%
  \xdef\RappelA{\FooAbscisses}%
  \xintFor*##1 in{\xintSeq{2}{\useKV[LecGra]{NbPoints}}}\do{%
    \xdef\NewAbscisse{\fpeval{\RappelA+randint(2,3)}}%
    \xdef\RappelA{\NewAbscisse}%
    \xdef\FooAbscisses{\FooAbscisses,\NewAbscisse}%
  }%
  % on choisit les ordonnées.
  \xdef\FooOrdonnees{\fpeval{randint(-3,3)}}%
  \xdef\RappelO{\FooOrdonnees}%
  \xdef\ListeDesEcartsO{1,2,3}%
  \xdef\FooListeDesOperations{+,-}%
  \MelangeListe{\FooListeDesOperations}{2}%
  \setsepchar{,}\ignoreemptyitems%
  \readlist*\PfCListeChoixOperations{\faa}%
  \xintFor*##1 in{\xintSeq{2}{\useKV[LecGra]{NbPoints}}}\do{%
    \MelangeListe{\ListeDesEcartsO}{1}%
    \setsepchar{,}\ignoreemptyitems%
    \readlist*\PfCListeEcartSeul{\faa}%
    \reademptyitems%
    \modulo{##1}{2}\relax%
    \ifnum\remainder=0\relax%
      \xdef\NewOrdonnee{\fpeval{\RappelO\PfCListeChoixOperations[1]\PfCListeEcartSeul[1]}}%
    \else
      \xdef\NewOrdonnee{\fpeval{\RappelO\PfCListeChoixOperations[2]\PfCListeEcartSeul[1]}}%
    \fi
    \xdef\RappelO{\NewOrdonnee}%
    \xdef\FooOrdonnees{\FooOrdonnees,\NewOrdonnee}%
  }%
  \setsepchar{,}%
  \readlist*\ListeAbscisses{\FooAbscisses}%
  \readlist*\ListeOrdonnees{\FooOrdonnees}%
  \ifboolKV[LecGra]{Tableau}{%
      \ifboolKV[LecGra]{Prolonge}{\xdef\LecGraDepartTab{2}\xdef\LecGraArriveeTab{\fpeval{\ListeAbscisseslen-1}}}{\xdef\LecGraDepartTab{1}\xdef\LecGraArriveeTab{\ListeAbscisseslen}}
    \begin{center}
      \begin{NiceTabular}{c*{\ListeAbscisseslen}{m{12pt}}}[hvlines]
        $x$\xintFor* ##1 in{\xintSeq{\LecGraDepartTab}{\LecGraArriveeTab}}\do{
          &\Block{}{\num{\ListeAbscisses[##1]}}%
        }\\
        $f(x)$\ifboolKV[LecGra]{SolutionTableau}{\xintFor* ##1 in{\xintSeq{\LecGraDepartTab}{\LecGraArriveeTab}}\do{
            &\num{\ListeOrdonnees[##1]}%
          }%
        }{}\\
      \end{NiceTabular}
    \end{center}
  }{%
  \xdef\FooPoints{0/\ListeAbscisses[1]/\ListeOrdonnees[1]/0}%
  \xintFor*##1 in{\xintSeq{2}{\useKV[LecGra]{NbPoints}}}\do{%
    \xdef\FooPoints{\FooPoints§0/\ListeAbscisses[##1]/\ListeOrdonnees[##1]/0}%
  }%
  \xdef\PfCOrdonneeMin{\fpeval{min(\FooOrdonnees)}}%
  \xdef\PfCOrdonneeMax{\fpeval{max(\FooOrdonnees)}}%
  \xdef\PfCAbscisseMin{\fpeval{min(\FooAbscisses)}}%
  \xdef\PfCAbscisseMax{\fpeval{max(\FooAbscisses)}}%
  \xdef\PfCChoixImage{\fpeval{randint(\PfCAbscisseMin,\PfCAbscisseMax)}}%
  \xdef\PfCChoixAntecedent{\fpeval{randint(\PfCOrdonneeMin,\PfCOrdonneeMax)}}%
  \ifboolKV[LecGra]{Image}{On effectue la lecture graphique à partir du point $(\num{\PfCChoixImage};0)$.}{}%
  \ifboolKV[LecGra]{Antecedent}{On effectue la lecture graphique en cherchant tous les points de la courbe ayant pour ordonnée \num{\PfCChoixAntecedent}.}{}%
  \begin{center}
    \Fonction[
    Points,
    Splines,
    Prolonge=\useKV[LecGra]{Prolonge},
    PointsCourbe=\useKV[LecGra]{Points},
    Epaisseur=\useKV[LecGra]{Epaisseur},
    CouleurTrace=\useKV[LecGra]{CouleurTrace},
    UniteX=\useKV[LecGra]{UniteX},
    UniteY=\useKV[LecGra]{UniteY},
    Traces={%
      boolean PdtImage,PdtAntecedent,PdtEquation;
      PdtImage:=\useKV[LecGra]{Image};
      PdtAntecedent:=\useKV[LecGra]{Antecedent};
      PdtEquation:=\useKV[LecGra]{Equation};
      % Image
      PfCepsilon:=0.001;
      pair A,B,C,D,Ant[];
      C=placepoint(\PfCChoixImage,0);
      path droiteimage,AxeO,droiteantecedent;
      droiteimage=placepoint(\PfCChoixImage,-100)--placepoint(\PfCChoixImage,100);
      droiteantecedent=placepoint(-100,\PfCChoixAntecedent)--placepoint(100,\PfCChoixAntecedent);
      AxeO=placepoint(0,-100)--placepoint(0,100);
      intersectioncourbes(CourbeTotale,droiteimage);
      A=pti[1];
      write  decimal(ypart(retrouvecoord(A))) to "PfCValeurImage.tex";%
      write EOF to "PfCValeurImage.tex";
      intersectioncourbes(CourbeTotale,AxeO);
      if n>0:
      B=pti[1];
      else:
      B=placepoint(0,\PfCOrdonneeMax);
      fi;
      D=(xpart(B),ypart(A));
      if PdtImage:
      drawoptions(dashed evenly);
      drawarrow C--1/4[C,A];
      drawarrow A--3/4[A,D];
      trace (C--A--D) withpen pencircle scaled 1.25;
      if ypart(A)>ypart(C):
      label.bot(TEX("\num{\fpeval{\PfCChoixImage}}"),C);
      else:
      label.top(TEX("\num{\fpeval{\PfCChoixImage}}"),C);
      fi;
      drawoptions();
      fi;
      %antecedent
      intersectioncourbes(CourbeTotale,droiteantecedent);
      for k=1 upto n:
      Ant[k]=pti[k];
      endfor;
      NbAntecedents:=1;
      for k=2 upto n:
      if abs(xpart(retrouvecoord(Ant[k]))-xpart(retrouvecoord(Ant[k-1])))>PfCepsilon:
      NbAntecedents:=NbAntecedents+1;
      Ant[NbAntecedents]:=pti[k];
      fi;
      endfor;
      write decimal(NbAntecedents) to "PfCNombreAntecedents.tex";
      write EOF to "PfCNombreAntecedents.tex";
      if NbAntecedents>0:
      for k=1 upto NbAntecedents:
      if PdtAntecedent or PdtEquation:
      draw ((fullcircle scaled 1mm) shifted Ant[k]) withcolor red;
      draw ((fullcircle scaled 1mm) shifted (xpart(Ant[k]),ypart(C))) withcolor red;
      draw Ant[k]--(xpart(Ant[k]),ypart(C)) withcolor red;
      drawarrow Ant[k]--1/4[Ant[k],(xpart(Ant[k]),ypart(C))] withcolor red;
      fi;
      write decimal(xpart(retrouvecoord(Ant[k]))) to "PfCListeAntecedents.tex";
      endfor;
      fi;
      write EOF to "PfCListeAntecedents.tex";
      if PdtAntecedent or PdtEquation:
      draw droiteantecedent withcolor red;
      if ypart(placepoint(0,\PfCChoixAntecedent))<ypart(B):
      dotlabel.lrt(TEX("\num{\fpeval{\PfCChoixAntecedent}}"),placepoint(0,\PfCChoixAntecedent));
      else:
      dotlabel.urt(TEX("\num{\fpeval{\PfCChoixAntecedent}}"),placepoint(0,\PfCChoixAntecedent));
      fi;
      fi;
      clip currentpicture to cadreexterieur;
    }]{%
      \FooPoints
    }
  \end{center}
  \ifboolKV[LecGra]{Image}{%
  \openin\myread=PfCValeurImage.tex
  \read\myread to \LectureImage
  L'image de \num{\PfCChoixImage} par la fonction $f$ est \num{\fpeval{round(\LectureImage,1)}}.%
  }{}%
  \ifboolKV[LecGra]{Antecedent}{%
      \openin\myread=PfCNombreAntecedents.tex
  \read\myread to \NombreAntecedents
  \ifnum\NombreAntecedents=0\relax
    Le nombre \num{\PfCChoixAntecedent} n'a pas d'antécédent par la fonction $f$.%
  \else
    Le nombre \num{\PfCChoixAntecedent} a \numberstringnum{\fpeval{\NombreAntecedents}} antécédent\ifnum\NombreAntecedents>1\relax s\fi{}  par la fonction $f$. % Titia vaut \Titia
    \openin\myread=PfCListeAntecedents.tex
    Par lecture graphique, \ifnum\NombreAntecedents>1\relax ce sont \xintFor* ##1 in{\xintSeq{1}{\NombreAntecedents}}\do{%
      \read\myread to \ListeNombreAntecedents%
      \xintifForLast{ et }{\xintifForFirst{}{;}}\num{\fpeval{round(\ListeNombreAntecedents,1)}}%
    }\else c'est \read\myread to \ListeNombreAntecedents\num{\fpeval{round(\ListeNombreAntecedents,1)}}\fi 
  \fi.%
}{}%
}
}%

\def\MPLagrangeCode{%
  path Cb[];
  %
  boolean PointsCourbe,Code,Prolonge;
  PointsCourbe=\useKV[ClesFonction]{PointsCourbe};
  Code=\useKV[ClesFonction]{Codes};
  Prolonge:=\useKV[ClesFonction]{Prolonge};
  color CoulTrace;
  CoulTrace:=\useKV[ClesFonction]{CouleurTrace};
  Epaisseur:=\useKV[ClesFonction]{Epaisseur};
  %
  vardef EffectuerTraces=
  \useKV[ClesFonction]{Traces};
  enddef;
%
  def enplace=
  xscaled X.u yscaled Y.u
  enddef;
  %
  vardef placepoint(expr q,r)=
  (q,r)*cm enplace
  enddef;
  %
  def replace=
  xscaled (1/X.u) yscaled (1/Y.u)
  enddef;
%
  def retrouvecoord(expr p)=%
  (p/cm) replace
  enddef;
  %
%  nbpoints:=100;
  %
  vardef intersectioncourbes(expr aa,bb)=
  pair pti[];
  nbpoints:=100;
  n:=0;
  for k=0 upto nbpoints-1:
  if ((subpath(k/nbpoints*length aa,(k+1)/nbpoints*length aa) of aa) intersectiontimes bb)<>(-1,-1):
      n:=n+1;
      pti[n]:=(subpath(k/nbpoints*length aa,(k+1)/nbpoints*length aa) of aa) intersectionpoint bb;
    fi;
  endfor;
  enddef;
  %
  vardef IntersectionCourbes(expr aa,bb)=
  pair pti[];
  numeric vali[];
  n:=0;
  for k=0 upto 100:
    if ((subpath(k/100*length aa,(k+1)/100*length aa) of aa) intersectiontimes bb)<>(-1,-1):
      n:=n+1;
      vali[n]:=k;
      message("vali : "&decimal(vali[n]));
      pti[n]:=(subpath(k/100*length aa,(k+1)/100*length aa) of aa) intersectionpoint bb;
    fi;
  endfor;
  enddef;
  %
  vardef courbe[](expr a,b,nb)(text texte)=
 path PfCCb;
 for i:=0 upto nb :
 x@[i]:=(a+i*(b-a)/nb);
 x:=x@[i];
 y@[i]:=texte;
% if y@[i]>MaxY:
% MaxY:=y@[i];
%    fi;
%    if y@[i]<MinY:
%    MinY:=y@[i];
%    fi;
 endfor ;
  Cb@:=placepoint(x@.0,y@.0)
 for i:=1 upto nb :
   ..placepoint(x@[i],y@[i])
 endfor;  
 PfCCb=Cb@;
 PfCCb
enddef;
}

\def\MPLagrange#1#2#3#4#5{
  % #1 la liste des points
  % #2: pas en x
  % #3: pas en y
  % #4: unit\'e en x
  % #5: unit\'e en y
  \ifluatex
    \mplibforcehmode
    \mplibnumbersystem{double}
    \begin{mplibcode}
      \MPLagrangeCode

      pair M[];%Les points de passage
      pair N[][];%Les points conservés pour les calculs.
      numeric nbm;%Le nombre de points de passage;
      
vardef Lagrange(expr valeur)(text t)=
  save Lagpt;
  numeric Lagpt;
  nbm:=0;
  for p_=t:
    nbm:=nbm+1;
  endfor;
  nbpoints:=0;
  for p_=t:
    nbpoints:=nbpoints+1;
    M[nbpoints]:=p_;
  endfor;
  for k=1 upto nbm:
    nbpoints:=0;
    for l=1 upto nbm:
      if l<>k:
	nbpoints:=nbpoints+1;
	N[k][nbpoints]:=M[l];
      fi;
    endfor;
  endfor;
  Lagpt=0;
  for k=1 upto nbm:
    ImagePoly:=for l=1 upto (nbm-1):(xpart(M[k])-xpart(N[k][l]))* endfor 1;
    Lagpt:=Lagpt+((ypart(M[k])/ImagePoly)*for l=1 upto (nbm-1):(valeur-xpart(N[k][l]))* endfor 1);
    endfor;
    Lagpt
    enddef;

x.u:=#2;
    y.u:=#3;
    X.u:=#4;
    Y.u:=#5;
    %
    MinX=999;
    MaxX=-999;
    MinY=999;
    MaxY=-999;
    %
    nbpointsLg=0;
    for p_=#1:
    nbpointsLg:=nbpointsLg+1;
    M[nbpointsLg]:=p_;
    endfor;
    for k=1 upto nbpointsLg:
    if xpart(M[k])<MinX:
    MinX:=xpart(M[k]);
    fi;
    if xpart(M[k])>MaxX:
    MaxX:=xpart(M[k]);
    fi;
    if ypart(M[k])<MinY:
    MinY:=ypart(M[k]);
    fi;
    if ypart(M[k])>MaxY:
    MaxY:=ypart(M[k]);
    fi;
    endfor;
    if MaxY<1:
    MaxY:=2;
    fi;
    if MinY>-1:
    MinY:=-2;
    fi;
    if MinY<0:
    if MinY<>ceiling(MinY):
    MinY:=ceiling(MinY)-1;
    fi;
    fi;
    if MinX<0:
    if MinX<>ceiling(MinX):
    MinX:=ceiling(MinX)-1;
    fi;
    fi;
    if MaxY<>ceiling(MaxY):
    MaxY:=ceiling(MaxY);
    fi;
    if MaxX<>ceiling(MaxX):
    MaxX:=ceiling(MaxX);
    fi;
    if MaxX<1:
    MaxX:=1;
    fi;
    if MaxY<1:
    MaxY:=1;
    fi;
    if Prolonge:
    draw courbe1(MinX-1,MaxX+1,25,Lagrange(x)(#1)) withpen pencircle scaled Epaisseur withcolor CoulTrace;
    else:
    draw courbe1(MinX,MaxX,25,Lagrange(x)(#1)) withpen pencircle scaled Epaisseur withcolor CoulTrace;
    fi;
    % Détermination des MinY et MaxY
%    MinY:=Lagrange(MinX)(#1);
    for k=0 upto 100:
    if Lagrange(MinX+(k/100)*(MaxX-MinX))(#1)<MinY:
    MinY:=Lagrange(MinX+(k/100)*(MaxX-MinX))(#1);
    fi;
    endfor;
    %for k=0 upto 100:
    %if Lagrange(MinX+(k/100)*(MaxX-MinX))(#2)<MinY:
    %MinY:=Lagrange(MinX+(k/100)*(MaxX-MinX))(#2);
    %fi;
    %endfor;
%    MaxY:=Lagrange(MinX)(#1);
    for k=0 upto 100:
    if Lagrange(MinX+(k/100)*(MaxX-MinX))(#1)>MaxY:
    MaxY:=Lagrange(MinX+(k/100)*(MaxX-MinX))(#1);
    fi;
    endfor;
    %for k=0 upto 100:
    %if Lagrange(MinX+(k/100)*(MaxX-MinX))(#2)>MaxY:
    %MaxY:=Lagrange(MinX+(k/100)*(MaxX-MinX))(#2);
    %fi;
    %endfor;
    %%%
    path cadreexterieur;
    cadreexterieur=(((MinX-1)*X.u*cm,(floor(MinY)-1)*Y.u*cm)--((MaxX+1)*X.u*cm,(floor(MinY)-1)*Y.u*cm)--((MaxX+1)*X.u*cm,(floor(MaxY)+1)*Y.u*cm)--((MinX-1)*X.u*cm,(floor(MaxY)+1)*Y.u*cm)--cycle);
    clip currentpicture to cadreexterieur;
    picture RetiensCourbe;
    RetiensCourbe=currentpicture;
    currentpicture:=nullpicture;
    for k=(floor(MinY)-1) step y.u until (floor(MaxY)+1):
    draw cm*((MinX-1)*X.u,k*Y.u)--cm*((MaxX+1)*X.u,k*Y.u) withcolor 0.75white;
    endfor;
    for k=MinX-1 step x.u until MaxX+1:
    draw cm*(k*X.u,(floor(MinY)-1)*Y.u)--cm*(k*X.u,(floor(MaxY)+1)*Y.u) withcolor 0.75white;
    endfor;
    draw RetiensCourbe;
    if PointsCourbe:
    for k=1 upto nbpointsLg:
    fill cercles(placepoint(xpart(M[k]),ypart(M[k])),0.5mm);
    endfor;
    fi;
    drawarrow (0,(floor(MinY)-1)*Y.u*cm)--(0,(floor(MaxY)+1)*Y.u*cm);
    drawarrow ((MinX-1)*X.u*cm,0)--((MaxX+1)*X.u*cm,0);
    %
    label.llft(TEX("O"),(0,0));
    dotlabel.bot(TEX("1"),cm*X.u*(1,0));
    dotlabel.lft(TEX("1"),cm*Y.u*(0,1));
    if Code:
    EffectuerTraces;
    fi;
  \end{mplibcode}
  \mplibnumbersystem{scaled}
  \fi
}%

\def\MPSplineCode{%
  boolean PointsCourbe,Code,Prolonge;
  PointsCourbe=\useKV[ClesFonction]{PointsCourbe};
  Code=\useKV[ClesFonction]{Codes};
  Prolonge:=\useKV[ClesFonction]{Prolonge};
  color CoulTrace;
  CoulTrace:=\useKV[ClesFonction]{CouleurTrace};
  Epaisseur:=\useKV[ClesFonction]{Epaisseur};
  %
  vardef EffectuerTraces=
  \useKV[ClesFonction]{Traces};
  enddef;
%
  def enplace=
  xscaled X.u yscaled Y.u
  enddef;
  %
  vardef placepoint(expr q,r)=
  (q,r)*cm enplace
  enddef;
  %
  def replace=
  xscaled (1/X.u) yscaled (1/Y.u)
  enddef;
%
  def retrouvecoord(expr p)=%
  (p/cm) replace
  enddef;
%
  vardef intersectioncourbes(expr aa,bb)=
  pair pti[];
  n:=0;
  for k=0 upto nbpoints-1:
  if ((subpath(k/nbpoints*length aa,(k+1)/nbpoints*length aa) of aa) intersectiontimes bb)<>(-1,-1):
      n:=n+1;
      pti[n]:=(subpath(k/nbpoints*length aa,(k+1)/nbpoints*length aa) of aa) intersectionpoint bb;
    fi;
  endfor;
  enddef;
  %
  nbspline=0;
  %
  nbpoints=50;
  %
  vardef SplineCubique(expr xzero,yzero,xuno,yuno,dzero,duno)=
  nbspline:=nbspline+1;
  a[nbspline]*(xzero**3)+b[nbspline]*(xzero**2)+c[nbspline]*xzero+d[nbspline]=yzero;
  a[nbspline]*(xuno**3)+b[nbspline]*(xuno**2)+c[nbspline]*xuno+d[nbspline]=yuno;
  a[nbspline]*(3*(xzero**2))+b[nbspline]*(2*xzero)+c[nbspline]=dzero;
  a[nbspline]*(3*(xuno**2))+b[nbspline]*(2*xuno)+c[nbspline]=duno;
  % 
  numeric xx[],yy[];
  for k=1 upto nbpoints-1:
  xx[k]=xzero+(k/nbpoints)*(xuno-xzero);
  yy[k]=a[nbspline]*(xx[k]**3)+b[nbspline]*(xx[k]**2)+c[nbspline]*xx[k]+d[nbspline];
  endfor;
  save Spline;
  path Spline;
  Spline=placepoint(xzero,yzero) for k=1 upto nbpoints-1:
  --placepoint(xx[k],yy[k])
  endfor --placepoint(xuno,yuno);
  Spline
  enddef;
  %
  path Cb[];
  vardef courbe[](expr a,b,nb)(text texte)=
  path PfCCb;
  for i:=0 upto nb :
  x@[i]:=(a+i*(b-a)/nb);
  x:=x@[i];
  y@[i]:=texte;
%  if y@[i]>MaxY:
%  MaxY:=y@[i];
%  fi;
%  if (y@[i]/(Y.u*cm))<MinY:
%  MinY:=y@[i];
%  fi;
  endfor ;
  Cb@:=placepoint(x@.0,y@.0)
  for i:=1 upto nb :
  ..placepoint(x@[i],y@[i])
  endfor;  
  PfCCb=Cb@;
  PfCCb
  enddef;
  %
  pair M[];%Les points de passage
      pair N[][];%Les points conservés pour les calculs.
      numeric nbm;%Le nombre de points de passage;
  %    
vardef Lagrange(expr valeur)(text t)=
  save Lagpt;
  numeric Lagpt;
  nbm:=0;
  for p_=t:
    nbm:=nbm+1;
  endfor;
  nbpoints:=0;
  for p_=t:
    nbpoints:=nbpoints+1;
    M[nbpoints]:=p_;
  endfor;
  for k=1 upto nbm:
    nbpoints:=0;
    for l=1 upto nbm:
      if l<>k:
	nbpoints:=nbpoints+1;
	N[k][nbpoints]:=M[l];
      fi;
    endfor;
  endfor;
  Lagpt=0;
  for k=1 upto nbm:
    ImagePoly:=for l=1 upto (nbm-1):(xpart(M[k])-xpart(N[k][l]))* endfor 1;
    Lagpt:=Lagpt+((ypart(M[k])/ImagePoly)*for l=1 upto (nbm-1):(valeur-xpart(N[k][l]))* endfor 1);
    endfor;
    Lagpt
    enddef;
    vardef SplinePointsDeux(text t)=
    pair Xxn[],Ffn[],Ddn[];%Xn abscisse,ordonnée D : dérivé g,dérivée d
    nbsplined=0;
    for p_=t:
    nbsplined:=nbsplined+1;
    if (nbsplined mod 2)=1:
    Xxn[(nbsplined div 2)+1]=p_;
    Ffn[(nbsplined div 2)+1]=cm*(X.u*xpart(p_),Y.u*ypart(p_));
    else:
    Ddn[(nbsplined div 2)]=p_
    fi;
    endfor;
    enddef;    
}

\def\MPLagrangeDouble#1#2#3#4#5#6{
  % #1 la liste des points
  % #2 la deuxième liste de points
  % #3: pas en x
  % #4: pas en y
  % #5: unit\'e en x
  % #6: unit\'e en y
  \ifluatex
    \mplibforcehmode
    \mplibnumbersystem{double}
    \begin{mplibcode}
      \MPLagrangeCode

      pair M[];%Les points de passage
      pair N[][];%Les points conservés pour les calculs.
      numeric nbm;%Le nombre de points de passage;
      
      vardef Lagrange(expr valeur)(text t)=
      save Lagpt;
      numeric Lagpt;
      nbm:=0;
      for p_=t:
      nbm:=nbm+1;
      endfor;
      nbpoints:=0;
      for p_=t:
      nbpoints:=nbpoints+1;
      M[nbpoints]:=p_;
      endfor;
      for k=1 upto nbm:
      nbpoints:=0;
      for l=1 upto nbm:
      if l<>k:
      nbpoints:=nbpoints+1;
      N[k][nbpoints]:=M[l];
      fi;
      endfor;
      endfor;
      Lagpt=0;
      for k=1 upto nbm:
      ImagePoly:=for l=1 upto (nbm-1):(xpart(M[k])-xpart(N[k][l]))* endfor 1;
      Lagpt:=Lagpt+((ypart(M[k])/ImagePoly)*for l=1 upto (nbm-1):(valeur-xpart(N[k][l]))* endfor 1);
      endfor;
      Lagpt
      enddef;
      
      x.u:=#3;
      y.u:=#4;
      X.u:=#5;
      Y.u:=#6;
      % 
      MinY=-10;%999;
      MaxY=10;%-999;
      % 
      MaxX=7;
      MinX=-7;
      %
    draw courbe1(-7,7,100,Lagrange(x)(#1)) withpen pencircle scaled Epaisseur withcolor CoulTrace;
    draw courbe2(-7,7,100,Lagrange(x)(#2)) withpen pencircle scaled Epaisseur dashed dashpattern(on4 off2) withcolor CoulTrace;
    % Détermination des MinY et MaxY
    MinY:=Lagrange(MinX)(#1);
    for k=0 upto 100:
    if Lagrange(MinX+(k/100)*(MaxX-MinX))(#1)<MinY:
    MinY:=Lagrange(MinX+(k/100)*(MaxX-MinX))(#1);
    fi;
    endfor;
    for k=0 upto 100:
    if Lagrange(MinX+(k/100)*(MaxX-MinX))(#2)<MinY:
    MinY:=Lagrange(MinX+(k/100)*(MaxX-MinX))(#2);
    fi;
    endfor;
    MaxY:=Lagrange(MinX)(#1);
    for k=0 upto 100:
    if Lagrange(MinX+(k/100)*(MaxX-MinX))(#1)>MaxY:
    MaxY:=Lagrange(MinX+(k/100)*(MaxX-MinX))(#1);
    fi;
    endfor;
    for k=0 upto 100:
    if Lagrange(MinX+(k/100)*(MaxX-MinX))(#2)>MaxY:
    MaxY:=Lagrange(MinX+(k/100)*(MaxX-MinX))(#2);
    fi;
    endfor;
    if MinY>0:
    MinY:=-1;
    fi;
    if MaxY<0:
    MaxY:=1;
    fi;
    path cadreexterieur;
    % cadreexterieur=(((MinX)*X.u*cm,(floor(MinY)-1)*Y.u*cm)--((MaxX)*X.u*cm,(floor(MinY)-1)*Y.u*cm)--((MaxX)*X.u*cm,(floor(MaxY)+1)*Y.u*cm)--((MinX)*X.u*cm,(floor(MaxY)+1)*Y.u*cm)--cycle);
    cadreexterieur=(((MinX)*X.u*cm,(floor(MinY))*Y.u*cm)--((MaxX)*X.u*cm,(floor(MinY))*Y.u*cm)--((MaxX)*X.u*cm,(ceiling(MaxY))*Y.u*cm)--((MinX)*X.u*cm,(ceiling(MaxY))*Y.u*cm)--cycle);
    clip currentpicture to cadreexterieur;
    picture RetiensCourbe;
    RetiensCourbe=currentpicture;
    currentpicture:=nullpicture;
    for k=(floor(MinY)) step y.u until (ceiling(MaxY)):
    draw cm*((MinX)*X.u,k*Y.u)--cm*((MaxX)*X.u,k*Y.u) withcolor 0.75white;
    endfor;
    for k=MinX step x.u until MaxX:
    draw cm*(k*X.u,(floor(MinY))*Y.u)--cm*(k*X.u,(ceiling(MaxY))*Y.u) withcolor 0.75white;
    endfor;
    draw RetiensCourbe;
%    if PointsCourbe:
%    for k=1 upto nbpointsLg:
%    fill cercles(placepoint(xpart(M[k]),ypart(M[k])),0.5mm);
%    endfor;
%    fi;
    drawarrow (0,(floor(MinY))*Y.u*cm)--(0,(ceiling(MaxY))*Y.u*cm);
    drawarrow ((MinX)*X.u*cm,0)--((MaxX)*X.u*cm,0);
    %
    label.llft(TEX("O"),(0,0));
    dotlabel.bot(TEX("1"),cm*X.u*(1,0));
    dotlabel.lft(TEX("1"),cm*Y.u*(0,1));
    if Code:
    EffectuerTraces;
    fi;
    %draw cadreexterieur withpen pencircle scaled 2;
  \end{mplibcode}
  \mplibnumbersystem{scaled}
  \fi
}%

\def\MPSplineDouble#1#2#3#4#5#6{%
  % #1 la liste des points
  % #2 la deuxième liste de points
  % #3: pas en x
  % #4: pas en y
  % #5: unit\'e en x
  % #6: unit\'e en y
  \ifluatex
    \mplibforcehmode
  \begin{mplibcode}
    \MPSplineCode;
    
    x.u:=#3;
    y.u:=#4;
    X.u:=#5;
    Y.u:=#6;

    pair Xn[][],Fn[][],Dn[][];%Xn abscisse,ordonnée D : dérivé g,dérivée d

    numeric N[];
    
    NbTotalSpline=0;

    vardef PointsSplines(text t)=
    n:=0;
    NbTotalSpline:=NbTotalSpline+1;
    for p_=t:
    n:=n+1;
    if (n mod 2)=1:
    Xn[NbTotalSpline][(n div 2)+1]=p_;
    Fn[NbTotalSpline][(n div 2)+1]=cm*(X.u*xpart(p_),Y.u*ypart(p_));
    else:
    Dn[NbTotalSpline][(n div 2)]=p_
    fi;
    endfor;
    N[NbTotalSpline]:=n div 2;
    enddef;

    PointsSplines(#1);
    PointsSplines(#2);
    
    if Prolonge:
    debutbalai=2;
    else:
    debutbalai=1;
    fi;

    vardef DeterminationMinMax(expr nbs)=
    finbalai:=N[nbs];
    Minx[nbs]=999;
    Maxx[nbs]=-999;
    Miny[nbs]=999;
    Maxy[nbs]=-999;
    for k=debutbalai upto finbalai:
    if xpart(Xn[nbs][k])<Minx[nbs]:
    Minx[nbs]:=xpart(Xn[nbs][k]);
    fi;
    if xpart(Xn[nbs][k])>Maxx[nbs]:
    Maxx[nbs]:=xpart(Xn[nbs][k]);
    fi;
    if ypart(Xn[nbs][k])<Miny[nbs]:
    Miny[nbs]:=ypart(Xn[nbs][k]);
    fi;
    if ypart(Xn[nbs][k])>Maxy[nbs]:
    Maxy[nbs]:=ypart(Xn[nbs][k]);
    fi;
    endfor;
    enddef;

    DeterminationMinMax(1);
    DeterminationMinMax(2);

    if Minx[1]<Minx[2]:
    MinX:=Minx[2];
    else:
    MinX:=Minx[1];
    fi;

    if Maxx[1]<Maxx[2]:
    MaxX:=Maxx[1];
    else:
    MaxX:=Maxx[2];
    fi;

    if Miny[1]<Miny[2]:
    MinY:=Miny[1];
    else:
    MinY:=Miny[2];
    fi;

    if Maxy[1]<Maxy[2]:
    MaxY:=Maxy[2];
    else:
    MaxY:=Maxy[1];
    fi;
    
    if MaxY<1:
    MaxY:=2;
    fi;
    if MinY>-1:
    MinY:=-2;
    fi;
    if MinY<0:
    if MinY<>ceiling(MinY):
    MinY:=ceiling(MinY)-1;
    fi;
    fi;
    if MinX<0:
    if MinX<>ceiling(MinX):
    MinX:=ceiling(MinX)-1;
    fi;
    fi;
    if MaxY<>ceiling(MaxY):
    MaxY:=ceiling(MaxY);
    fi;
    if MaxX<>ceiling(MaxX):
    MaxX:=ceiling(MaxX);
    fi;
    if MinX<0:
    else:
    MinX:=-1;
    fi;
    if MaxX<1:
    MaxX:=1;
    fi;
    if MaxY<1:
    MaxY:=1;
    fi;
    %
    path cadreexterieur;
    cadreexterieur=(((MinX)*X.u*cm,(MinY-1)*Y.u*cm)--((MaxX)*X.u*cm,(MinY-1)*Y.u*cm)--((MaxX)*X.u*cm,(MaxY+1)*Y.u*cm)--((MinX)*X.u*cm,(MaxY+1)*Y.u*cm)--cycle);
    path SplineRecap[][],CourbeTotale[];
    for r=1 upto 2:
    for l=1 upto N[r]-1:
    SplineRecap[r][l]=SplineCubique(xpart(Xn[r][l]),ypart(Xn[r][l]),xpart(Xn[r][l+1]),ypart(Xn[r][l+1]),ypart(Dn[r][l]),xpart(Dn[r][l+1]));
    endfor;
    CourbeTotale[r]=SplineRecap[r][1] for l=2 upto N[r]-1:..SplineRecap[r][l] endfor;
    endfor;
    draw CourbeTotale[1] withpen pencircle scaled Epaisseur withcolor CoulTrace;
    draw CourbeTotale[2] withpen pencircle scaled Epaisseur dashed dashpattern(on4 off 2) withcolor CoulTrace;
    %
    if PointsCourbe:
    for k=debutbalai upto finbalai:
    fill cercles(Fn[1][k],0.5mm);
    fill cercles(Fn[2][k],0.5mm);
    endfor;
    fi;
    %
    clip currentpicture to cadreexterieur;
    picture RetiensCourbe;
    RetiensCourbe=currentpicture;
    currentpicture:=nullpicture;
    for k=MinY step y.u until MaxY:
    draw cm*((MinX)*X.u,k*Y.u)--cm*((MaxX)*X.u,k*Y.u) withcolor 0.75white;
    endfor;
    for k=MinX step x.u until MaxX:
    draw cm*(k*X.u,(MinY-1)*Y.u)--cm*(k*X.u,(MaxY+1)*Y.u) withcolor 0.75white;
    endfor;
    drawarrow (0,(MinY-1)*Y.u*cm)--(0,(MaxY+1)*Y.u*cm);
    drawarrow ((MinX)*X.u*cm,0)--((MaxX)*X.u*cm,0);
    %
    draw RetiensCourbe;
    label.llft(btex O etex,(0,0));
    dotlabel.bot(btex 1 etex,cm*X.u*(1,0));
    dotlabel.lft(btex 1 etex,cm*Y.u*(0,1));
    if Code:
    EffectuerTraces;
    fi;
    trace cadreexterieur withcolor 0.75white;
  \end{mplibcode}
\fi
}

\def\MPSpline#1#2#3#4#5{%
  % #1 la liste des points
  % #2: pas en x
  % #3: pas en y
  % #4: unit\'e en x
  % #5: unit\'e en y
  \ifluatex
    \mplibforcehmode
  \begin{mplibcode}
    \MPSplineCode;
    
    x.u:=#2;
    y.u:=#3;
    X.u:=#4;
    Y.u:=#5;

    pair Xn[],Fn[],Dn[];%Xn abscisse,ordonnée D : dérivé g,dérivée d
    n=0;
    for p_=#1:
    n:=n+1;
    if (n mod 2)=1:
    Xn[(n div 2)+1]=p_;
    Fn[(n div 2)+1]=cm*(X.u*xpart(p_),Y.u*ypart(p_));
    else:
    Dn[(n div 2)]=p_
    fi;
    endfor;
    N:=n div 2;
    MinX=999;
    MaxX=-999;
    MinY=999;
    MaxY=-999;
    if Prolonge:
    debutbalai=2;
    finbalai=N-1;
    else:
    debutbalai=1;
    finbalai=N;
    fi;
    for k=debutbalai upto finbalai:
    if xpart(Xn[k])<MinX:
    MinX:=xpart(Xn[k]);
    fi;
    if xpart(Xn[k])>MaxX:
    MaxX:=xpart(Xn[k]);
    fi;
    if ypart(Xn[k])<MinY:
    MinY:=ypart(Xn[k]);
    fi;
    if ypart(Xn[k])>MaxY:
    MaxY:=ypart(Xn[k]);
    fi;
    endfor;
    if MaxY<1:
    MaxY:=2;
    fi;
    if MinY>-1:
    MinY:=-2;
    fi;
    if MinY<0:
    if MinY<>ceiling(MinY):
    MinY:=ceiling(MinY)-1;
    fi;
    fi;
    if MinX<0:
    if MinX<>ceiling(MinX):
    MinX:=ceiling(MinX)-1;
    fi;
    fi;
    if MaxY<>ceiling(MaxY):
    MaxY:=ceiling(MaxY);
    fi;
    if MaxX<>ceiling(MaxX):
    MaxX:=ceiling(MaxX);
    fi;
    if MinX<0:
    else:
    MinX:=-1;
    fi;
    if MaxX<1:
    MaxX:=1;
    fi;
    if MaxY<1:
    MaxY:=1;
    fi;
    %
    path cadreexterieur;
    cadreexterieur=(((MinX-1)*X.u*cm,(MinY-1)*Y.u*cm)--((MaxX+1)*X.u*cm,(MinY-1)*Y.u*cm)--((MaxX+1)*X.u*cm,(MaxY+1)*Y.u*cm)--((MinX-1)*X.u*cm,(MaxY+1)*Y.u*cm)--cycle);
    path SplineRecap[],CourbeTotale;
    for l=1 upto N-1:
    SplineRecap[l]=SplineCubique(xpart(Xn[l]),ypart(Xn[l]),xpart(Xn[l+1]),ypart(Xn[l+1]),ypart(Dn[l]),xpart(Dn[l+1]));
    endfor;
    CourbeTotale=SplineRecap[1] for l=2 upto N-1:..SplineRecap[l] endfor;
    draw CourbeTotale withpen pencircle scaled Epaisseur withcolor CoulTrace;
    clip currentpicture to cadreexterieur;
    picture RetiensCourbe;
    RetiensCourbe=currentpicture;
    currentpicture:=nullpicture;
    for k=MinY-1 step y.u until MaxY+1:
    draw cm*((MinX-1)*X.u,k*Y.u)--cm*((MaxX+1)*X.u,k*Y.u) withcolor 0.75white;
    endfor;
    for k=MinX-1 step x.u until MaxX+1:
    draw cm*(k*X.u,(MinY-1)*Y.u)--cm*(k*X.u,(MaxY+1)*Y.u) withcolor 0.75white;
    endfor;
    if PointsCourbe:
    for k=debutbalai upto finbalai:
    fill cercles(Fn[k],0.5mm);
    endfor;
    fi;
    drawarrow (0,(MinY-1)*Y.u*cm)--(0,(MaxY+1)*Y.u*cm);
    drawarrow ((MinX-1)*X.u*cm,0)--((MaxX+1)*X.u*cm,0);
    %
    draw RetiensCourbe;
    label.llft(btex O etex,(0,0));
    dotlabel.bot(btex 1 etex,cm*X.u*(1,0));
    dotlabel.lft(btex 1 etex,cm*Y.u*(0,1));
    if Code:
    EffectuerTraces;
    fi;
  \end{mplibcode}
  \else
  \begin{mpost}[mpsettings={\MPSplineCode;}]    
    x.u:=#2;
    y.u:=#3;
    X.u:=#4;
    Y.u:=#5;

    pair Xn[],Fn[],Dn[];%Xn abscisse,ordonnée D : dérivé g,dérivée d
    n=0;
    for p_=#1:
    n:=n+1;
    if (n mod 2)=1:
    Xn[(n div 2)+1]=p_;
    Fn[(n div 2)+1]=cm*(X.u*xpart(p_),Y.u*ypart(p_));
    else:
    Dn[(n div 2)]=p_
    fi;
    endfor;
    N:=n div 2;
    MinX=999;
    MaxX=-999;
    MinY=999;
    MaxY=-999;
    if Prolonge:
    debutbalai=2;
    finbalai=N-1;
    else:
    debutbalai=1;
    finbalai=N;
    fi;
    for k=debutbalai upto finbalai:
    if xpart(Xn[k])<MinX:
    MinX:=xpart(Xn[k]);
    fi;
    if xpart(Xn[k])>MaxX:
    MaxX:=xpart(Xn[k]);
    fi;
    if ypart(Xn[k])<MinY:
    MinY:=ypart(Xn[k]);
    fi;
    if ypart(Xn[k])>MaxY:
    MaxY:=ypart(Xn[k]);
    fi;
    endfor;
    if MaxY<1:
    MaxY:=2;
    fi;
    if MinY>-1:
    MinY:=-2;
    fi;
    if MinY<0:
    if MinY<>ceiling(MinY):
    MinY:=ceiling(MinY)-1;
    fi;
    fi;
    if MinX<0:
    if MinX<>ceiling(MinX):
    MinX:=ceiling(MinX)-1;
    fi;
    fi;
    if MaxY<>ceiling(MaxY):
    MaxY:=ceiling(MaxY);
    fi;
    if MaxX<>ceiling(MaxX):
    MaxX:=ceiling(MaxX);
    fi;
    if MaxX<1:
    MaxX:=1;
    fi;
    if MaxY<1:
    MaxY:=1;
    fi;
    %
    path cadreexterieur;
    cadreexterieur=(((MinX-1)*X.u*cm,(MinY-1)*Y.u*cm)--((MaxX+1)*X.u*cm,(MinY-1)*Y.u*cm)--((MaxX+1)*X.u*cm,(MaxY+1)*Y.u*cm)--((MinX-1)*X.u*cm,(MaxY+1)*Y.u*cm)--cycle);
    for l=1 upto N-1:
    draw SplineCubique(xpart(Xn[l]),ypart(Xn[l]),xpart(Xn[l+1]),ypart(Xn[l+1]),ypart(Dn[l]),xpart(Dn[l+1])) withpen pencircle scaled Epaisseur withcolor CoulTrace;
    endfor;
    clip currentpicture to cadreexterieur;
    picture RetiensCourbe;
    RetiensCourbe=currentpicture;
    currentpicture:=nullpicture;
    for k=MinY-1 step y.u until MaxY+1:
    draw cm*((MinX-1)*X.u,k*Y.u)--cm*((MaxX+1)*X.u,k*Y.u) withcolor 0.75white;
    endfor;
    for k=MinX-1 step x.u until MaxX+1:
    draw cm*(k*X.u,(MinY-1)*Y.u)--cm*(k*X.u,(MaxY+1)*Y.u) withcolor 0.75white;
    endfor;
    if PointsCourbe:
    for k=debutbalai upto finbalai:
    fill cercles(Fn[k],0.5mm);
    endfor;
    fi;
    drawarrow (0,(MinY-1)*Y.u*cm)--(0,(MaxY+1)*Y.u*cm);
    drawarrow ((MinX-1)*X.u*cm,0)--((MaxX+1)*X.u*cm,0);
    %
    draw RetiensCourbe;
    label.llft(btex O etex,(0,0));
    dotlabel.bot(btex 1 etex,cm*X.u*(1,0));
    dotlabel.lft(btex 1 etex,cm*Y.u*(0,1));
    if Code:
    EffectuerTraces;
    fi;    
  \end{mpost}
  \fi
}

\def\PfCTraceMGCode{%
  nbcourbe=\ListeDesFonctionslen;
  % 
  numeric borneinf[],bornesup[],pos[];
  % 
  vardef RetiensPos(text t)=
  n:=0;
  for p_=t:
  n:=n+1;
  pos[n]=p_;
  endfor;
  enddef;
  % 
  RetiensPos(\useKV[TraceG]{LabelC});
  % 
  vardef RetiensBorneInf(text t)=
  borneinfmin=infinity;
  n:=0;
  for p_=t:
  n:=n+1;
  borneinf[n]=p_;
  if borneinf[n]<borneinfmin:
  borneinfmin:=borneinf[n]
  fi;
  endfor;
  enddef;
  % 
  RetiensBorneInf(\useKV[TraceG]{Bornea});
  % 
  vardef RetiensBorneSup(text t)=
  bornesupmax=-infinity;
  n:=0;
  for p_=t:
  n:=n+1;
  bornesup[n]=p_;
  if bornesup[n]>bornesupmax:
  bornesupmax:=bornesup[n]
  fi;
  endfor;
  enddef;%
  % 
  RetiensBorneSup(\useKV[TraceG]{Borneb});
  % 
  nbpointsCourbe:=\useKV[TraceG]{NbPointsCourbe};
  xmin=\useKV[TraceG]{Xmin};
  xmax=\useKV[TraceG]{Xmax};
  ymin=\useKV[TraceG]{Ymin};
  ymax=\useKV[TraceG]{Ymax};
  pasx=\useKV[TraceG]{Xstep};
  pasy=\useKV[TraceG]{Ystep};
  xu=1cm/\useKV[TraceG]{Xstep};
  yu=1cm/\useKV[TraceG]{Ystep};
  grillex=\useKV[TraceG]{PasGrilleX};
  grilley=\useKV[TraceG]{PasGrilleY};
  gradx=\useKV[TraceG]{PasGradX};
  grady=\useKV[TraceG]{PasGradY};
  Epaisseur:=\useKV[TraceG]{Epaisseur};
  %
  string LabelX,LabelY;
  LabelX=\the\tokPfCLabelX;
  LabelY=\the\tokPfCLabelY;
  %
  color colortrace[];
  vardef RetiensCouleur(text t)=
  n:=0;
  for p_=t:
  n:=n+1;
  colortrace[n]=p_;
  endfor;
  enddef;
  % 
  RetiensCouleur(\useKV[TraceG]{CouleurTrace});
  %
  numeric styletrace[];
  vardef RetiensStyle(text t)=
  n:=0;
  for p_=t:
  n:=n+1;
  styletrace[n]=p_;
  endfor;
  enddef;
  %
  RetiensStyle(\useKV[TraceG]{StyleTrace});
  %
  pair Origine,DecalageOrigine;
  Origine=(xmin,ymin)+\useKV[TraceG]{Origine};
  DecalageOrigine=\useKV[TraceG]{Origine};
  %
  boolean Grille,Vide,Graduations,Code,LegendeX,LegendeY,Millimetre,Tiret;
  Grille=\useKV[TraceG]{Grille};
  Tiret=\useKV[TraceG]{Tiret};
  Millimetre=\useKV[TraceG]{Millimetre};
  Vide=\useKV[TraceG]{Vide};
  Graduations=\useKV[TraceG]{Graduations};
  Code=\useKV[TraceG]{Code};
  LegendeX=\useKV[TraceG]{LegendeX};
  LegendeY=\useKV[TraceG]{LegendeY};
  %
  vardef sin(expr t) = sind(c*t) enddef;
  vardef cos(expr t) = cosd(c*t) enddef;
  vardef tan(expr t) = sin(t)/cos(t) enddef;
  vardef exp(expr t) = e**t enddef;
  vardef ch(expr x)=(exp(x)+exp(-x))/2 enddef;
  vardef sh(expr x)=(exp(x)-exp(-x))/2 enddef;
  vardef ln(expr t) = mlog(t)/256 enddef;
  vardef arcsin(expr x)=%Définition mathématique en radian
    pi*angle((sqrt(1-x**2),x))/180
  enddef;
  vardef arccos(expr x)=%Définition mathématique en radian
    pi*angle((x,sqrt(1-x**2)))/180
  enddef;
  def enplace=
    xscaled xu yscaled yu shifted (Origine*cm)
  enddef;
  vardef placepoint(expr q,r)=
   (q,r) enplace
  enddef;
  %
  path Cb[];
  %
  vardef courbe[](expr a,b,nb)(text texte)=
  path Courbe;
  for i:=0 upto nb :
  x@[i]:=(a+i*((b-a)/nb));
  x:=x@[i];
  y@[i]:=texte;
  endfor ;
  Cb@:=(x@.0*xu,y@.0*yu)
  for i:=1 upto nb :
  ..(x@[i]*xu,y@[i]*yu)
  endfor;
  Cb@:=Cb@ shifted (Origine*cm);
  Courbe=Cb@;
  Courbe
  enddef;
  %
  vardef DessineCourbes(text texto)=
  n:=0;
  for p_=texto:
  n:=n+1;
  if unknown styletrace[n]:
  drawoptions();
  else:
  if styletrace[n]=1:
  drawoptions(dashed evenly);
  elseif styletrace[n]=2:
  drawoptions(dashed withdots);
  elseif styletrace[n]=3:
  drawoptions(dashed dashpattern(on6 off3 on1.5 off3));
  else:
  drawoptions();
  fi;
  fi;
  draw courbe[n](borneinf[n],bornesup[n],100)(scantokens p_) withpen (pencircle scaled Epaisseur) withcolor if unknown colortrace[n]:black else: colortrace[n] fi;
  endfor;
  enddef;
  %
  vardef NommeCourbes(text texto)=
  nc:=0;
  for p_=texto:
  nc:=nc+1;
  numeric tp;
  tp=pos[nc]*length Cb[n];
  pair PT,Tangente;
  PT:=point (tp) of Cb[nc];
  Tangente:=unitvector(direction tp of Cb[nc]);
  label(TEX(p_) rotated angle(Tangente),PT+2mm*(Tangente rotated 90));
  endfor;
  enddef;
  % 
  vardef NommeCourbesPDF(text texto)=
  n:=0;
  for p_=texto:
  n:=n+1;
  numeric tp;
  tp=pos[n]*length Cb[n];
  pair PT,Tangente;
  PT:=point (tp) of Cb[n];
  Tangente:=unitvector(direction tp of Cb[n]);
  label(LATEX(p_) rotated angle(Tangente),PT+2mm*(Tangente rotated 90));
  endfor;
  enddef;
  %
  vardef EffectuerTraces=
  \useKV[TraceG]{Traces};
  enddef;
  %
  vardef LegendeCourbe=
  if LegendeX=true:
  %label.ulft(btex \useKV[TraceG]{LabelX} etex,u*(xmax,ypart(Origine)));
  label.ulft(TEX(LabelX),u*(xmax,ypart(Origine)));
  fi;
  if LegendeY=true:
  % label.lrt(btex \useKV[TraceG]{LabelY} etex,u*(xpart(Origine),ymax));
    label.lrt(TEX(LabelY),u*(xpart(Origine),ymax));
  fi;
  enddef;
}%

\NewDocumentCommand\TraceMultiGraphique{mm}{%
  % #1 paramètres
  % #2 Liste fonctions
  \ifluatex
    \mplibforcehmode
  \begin{mplibcode}
    \PfCTraceMGCode
    color PfCOrange;
    PfCOrange=(.75,.5,.25);
    if Millimetre:
    drawoptions(withpen pencircle scaled 0.25 withcolor 0.7[PfCOrange,white]);
    for k=xpart(Origine) step 0.1 until (xmax+grillex):
    trace u*(k,ymin-grilley)--u*(k,ymax+grilley);
    endfor;
    for k=xpart(Origine) step -0.1 until (xmin-grillex):
    trace u*(k,ymin-grilley)--u*(k,ymax+grilley);
    endfor;
    for k=ypart(Origine) step 0.1 until (ymax+grilley):
    trace u*(xmin-grillex,k)--u*(xmax+grillex,k);
    endfor;
    for k=ypart(Origine) step -0.1 until (ymin-grilley):
    trace u*(xmin-grillex,k)--u*(xmax+grillex,k);
    endfor;
    drawoptions();
    drawoptions(withpen pencircle scaled 0.675 withcolor 0.5[PfCOrange,white]);
    for k=xpart(Origine) step 1 until (xmax+grillex):
    trace u*(k,ymin-grilley)--u*(k,ymax+grilley);
    endfor;
    for k=xpart(Origine) step -1 until (xmin-grillex):
    trace u*(k,ymin-grilley)--u*(k,ymax+grilley);
    endfor;
    for k=ypart(Origine) step 1 until (ymax+grilley):
    trace u*(xmin-grillex,k)--u*(xmax+grillex,k);
    endfor;
    for k=ypart(Origine) step -1 until (ymin-grilley):
    trace u*(xmin-grillex,k)--u*(xmax+grillex,k);
    endfor;
    drawoptions();
    drawoptions(withpen pencircle scaled 1 withcolor 0.3[PfCOrange,white]);
    for k=xpart(Origine) step 5 until (xmax+grillex):
    trace u*(k,ymin-grilley)--u*(k,ymax+grilley);
    endfor;
    for k=xpart(Origine) step -5 until (xmin-grillex):
    trace u*(k,ymin-grilley)--u*(k,ymax+grilley);
    endfor;
    for k=ypart(Origine) step 5 until (ymax+grilley):
    trace u*(xmin-grillex,k)--u*(xmax+grillex,k);
    endfor;
    for k=ypart(Origine) step -5 until (ymin-grilley):
    trace u*(xmin-grillex,k)--u*(xmax+grillex,k);
    endfor;
    drawoptions();
    fi;
    if Grille:
    drawoptions(withcolor 0.75white);
    for k=xpart(Origine) step grillex until (xmax+grillex):
    trace u*(k,ymin-grilley)--u*(k,ymax+grilley);
    endfor;
    for k=xpart(Origine) step -grillex until (xmin-grillex):
    trace u*(k,ymin-grilley)--u*(k,ymax+grilley);
    endfor;
    for k=ypart(Origine) step grilley until (ymax+grilley):
    trace u*(xmin-grillex,k)--u*(xmax+grillex,k);
    endfor;
    for k=ypart(Origine) step -grilley until (ymin-grilley):
    trace u*(xmin-grillex,k)--u*(xmax+grillex,k);
    endfor;
    drawoptions();
    else:
    grillex:=0;
    grilley:=0;
    fi;
    if Graduations:
    for k=gradx step gradx until (((xmax-xmin)-xpart(DecalageOrigine))*pasx):
    if Tiret:
    draw ((0,-0.5mm)--(0,0.5mm)) shifted placepoint(k,0);
    label.bot(TEX("\num{"&decimal(k)&"}"),Origine*cm+(k*xu,0));
    else:
    dotlabel.bot(TEX("\num{"&decimal(k)&"}"),Origine*cm+(k*xu,0));
    fi;
    endfor;
    for k=-gradx step -gradx until (xpart(-DecalageOrigine)*pasx):
    if Tiret:
    draw ((0,-0.5mm)--(0,0.5mm)) shifted placepoint(k,0);
    label.bot(TEX("\num{"&decimal(k)&"}"),Origine*cm+(k*xu,0));
    else:
    dotlabel.bot(TEX("\num{"&decimal(k)&"}"),Origine*cm+(k*xu,0));
    fi;
    endfor;
    for k=grady step grady until (((ymax-ymin)-ypart(DecalageOrigine))*pasy):
    if Tiret:
    draw ((-0.5mm,0)--(0.5mm,0)) shifted placepoint(0,k);
    label.lft(TEX("\num{"&decimal(k)&"}"),Origine*cm+(0,k*yu));
    else:
    dotlabel.lft(TEX("\num{"&decimal(k)&"}"),Origine*cm+(0,k*yu));
    fi;
    endfor;
    for k=-grady step -grady until (ypart(-DecalageOrigine)*pasy):
    if Tiret:
    draw ((-0.5mm,0)--(0.5mm,0)) shifted placepoint(0,k);
    label.lft(TEX("\num{"&decimal(k)&"}"),Origine*cm+(0,k*yu));
    else:
    dotlabel.lft(TEX("\num{"&decimal(k)&"}"),Origine*cm+(0,k*yu));
    fi;
    endfor;
    fi;
%     
    drawoptions(withpen pencircle scaled(1));
    drawarrow (u*(0,ymin-grilley)--u*(0,ymax+grilley)) shifted (u*(xpart(Origine),0));
    drawarrow (u*(xmin-grillex,0)--u*(xmax+grillex,0)) shifted (u*(0,ypart(Origine)));
    label.llft(btex 0 etex,u*Origine);
    drawoptions();
    if Vide:
    if Grille:
    clip currentpicture to polygone(u*(xmin-grillex,ymin-grilley),u*(xmax+grillex,ymin-grilley),u*(xmax+grillex,ymax+grilley),u*(xmin-grillex,ymax+grilley));
    draw polygone(u*(xmin-grillex,ymin-grilley),u*(xmax+grillex,ymin-grilley),u*(xmax+grillex,ymax+grilley),u*(xmin-grillex,ymax+grilley)) withcolor 0.75white;
    else:
    clip currentpicture to polygone(u*(xmin,ymin),u*(xmax,ymin),u*(xmax,ymax),u*(xmin,ymax));
    fi;
    else:
    DessineCourbes(#1);
    NommeCourbes(#2);
    clip currentpicture to polygone(u*(xmin-grillex,ymin-grilley),u*(xmax+grillex,ymin-grilley),u*(xmax+grillex,ymax+grilley),u*(xmin-grillex,ymax+grilley));
    if Grille:
    %clip currentpicture to polygone(u*(xmin-grillex,ymin-grilley),u*(xmax+grillex,ymin-grilley),u*(xmax+grillex,ymax+grilley),u*(xmin-grillex,ymax+grilley));
    draw polygone(u*(xmin-grillex,ymin-grilley),u*(xmax+grillex,ymin-grilley),u*(xmax+grillex,ymax+grilley),u*(xmin-grillex,ymax+grilley)) withcolor 0.75white;
    else:
    %clip currentpicture to polygone(u*(xmin,ymin),u*(xmax,ymin),u*(xmax,ymax),u*(xmin,ymax));
    fi;
    % 
    if Code:
    EffectuerTraces;
    fi;
    fi;
    LegendeCourbe;
  \end{mplibcode}
  \else
    \begin{mpost}[mpsettings={\PfCTraceMGCode}]
      color PfCOrange;
    PfCOrange=(.75,.5,.25);
    if Millimetre:
    drawoptions(withpen pencircle scaled 0.25 withcolor 0.7[PfCOrange,white]);
    for k=xpart(Origine) step 0.1 until (xmax+grillex):
    trace u*(k,ymin-grilley)--u*(k,ymax+grilley);
    endfor;
    for k=xpart(Origine) step -0.1 until (xmin-grillex):
    trace u*(k,ymin-grilley)--u*(k,ymax+grilley);
    endfor;
    for k=ypart(Origine) step 0.1 until (ymax+grilley):
    trace u*(xmin-grillex,k)--u*(xmax+grillex,k);
    endfor;
    for k=ypart(Origine) step -0.1 until (ymin-grilley):
    trace u*(xmin-grillex,k)--u*(xmax+grillex,k);
    endfor;
    drawoptions();
    drawoptions(withpen pencircle scaled 0.675 withcolor 0.5[PfCOrange,white]);
    for k=xpart(Origine) step 1 until (xmax+grillex):
    trace u*(k,ymin-grilley)--u*(k,ymax+grilley);
    endfor;
    for k=xpart(Origine) step -1 until (xmin-grillex):
    trace u*(k,ymin-grilley)--u*(k,ymax+grilley);
    endfor;
    for k=ypart(Origine) step 1 until (ymax+grilley):
    trace u*(xmin-grillex,k)--u*(xmax+grillex,k);
    endfor;
    for k=ypart(Origine) step -1 until (ymin-grilley):
    trace u*(xmin-grillex,k)--u*(xmax+grillex,k);
    endfor;
    drawoptions();
    drawoptions(withpen pencircle scaled 1 withcolor 0.3[PfCOrange,white]);
    for k=xpart(Origine) step 5 until (xmax+grillex):
    trace u*(k,ymin-grilley)--u*(k,ymax+grilley);
    endfor;
    for k=xpart(Origine) step -5 until (xmin-grillex):
    trace u*(k,ymin-grilley)--u*(k,ymax+grilley);
    endfor;
    for k=ypart(Origine) step 5 until (ymax+grilley):
    trace u*(xmin-grillex,k)--u*(xmax+grillex,k);
    endfor;
    for k=ypart(Origine) step -5 until (ymin-grilley):
    trace u*(xmin-grillex,k)--u*(xmax+grillex,k);
    endfor;
    drawoptions();
    fi;
    if Grille:
    drawoptions(withcolor 0.75white);
    for k=xpart(Origine) step grillex until xmax:
    trace u*(k,ymin)--u*(k,ymax);
    endfor;
    for k=xpart(Origine) step -grillex until xmin:
    trace u*(k,ymin)--u*(k,ymax);
    endfor;
    for k=ypart(Origine) step grilley until ymax:
    trace u*(xmin,k)--u*(xmax,k);
    endfor;
    for k=ypart(Origine) step -grilley until ymin:
    trace u*(xmin,k)--u*(xmax,k);
    endfor;
    drawoptions();
    fi;
    if Graduations:
    for k=gradx step gradx until bornesupmax:%xmax*gradx/grillex:
    dotlabel.bot(LATEX("\num{"&decimal(k)&"}"),(k*xu+xpart(Origine*cm),ypart(Origine*cm)));
    endfor;
    for k=-gradx step -gradx until borneinfmin:%xmin*gradx/grillex:
    dotlabel.bot(LATEX("\num{"&decimal(k)&"}"),(k*xu+xpart(Origine*cm),ypart(Origine*cm)));
    endfor;
    for k=grady step grady until ymax*grady/grilley:
    dotlabel.lft(LATEX("\num{"&decimal(k)&"}"),(xpart(Origine*cm),k*yu+ypart(Origine*cm)));
    endfor;
    for k=-grady step -grady until ymin*grady/grilley:
    dotlabel.lft(LATEX("\num{"&decimal(k)&"}"),(xpart(Origine*cm),k*yu+ypart(Origine*cm)));
    endfor;
    fi;
%     
    drawoptions(withpen pencircle scaled(1));
    drawarrow (u*(0,ymin)--u*(0,ymax)) shifted (u*(xpart(Origine),0));
    drawarrow (u*(xmin,0)--u*(xmax,0)) shifted (u*(0,ypart(Origine)));
    label.llft(btex 0 etex,u*Origine);
    drawoptions();
    if Vide:
    clip currentpicture to polygone(u*(xmin,ymin),u*(xmax,ymin),u*(xmax,ymax),u*(xmin,ymax));
    else:
    DessineCourbes(#1);
    NommeCourbesPDF(#2);
    clip currentpicture to polygone(u*(xmin,ymin),u*(xmax,ymin),u*(xmax,ymax),u*(xmin,ymax));
    if Code:
    EffectuerTraces;
    fi;
    fi;
    LegendeCourbe;
  \end{mpost}
  \fi
}

\def\MPCourbeNew#1{%#2#3#4#5#6{%
  \ifluatex
  \mplibforcehmode
  \begin{mplibcode}
    x.u:=\useKV[ClesFonction]{PasX};
    y.u:=\useKV[ClesFonction]{PasY};
    X.u:=\useKV[ClesFonction]{UniteX};
    Y.u:=\useKV[ClesFonction]{UniteY};
    Epaisseur:=\useKV[ClesFonction]{Epaisseur};
    
    boolean Prolonge,Tangentes,PointsCourbe,Codes;
    Prolonge=\useKV[ClesFonction]{Prolonge};
    Tangentes=\useKV[ClesFonction]{Tangentes};
    PointsCourbe=\useKV[ClesFonction]{PointsCourbe};
    Codes=\useKV[ClesFonction]{Codes};

    color CouleurTrace;
    CouleurTrace=\useKV[ClesFonction]{CouleurTrace};

    def enplace=
    xscaled X.u yscaled Y.u
    enddef;

    vardef placepoint(expr q,r)=
    (q,r)*cm enplace
    enddef;
    
    numeric dirav[],dirap[];
    pair Fn[],Gn[];
    n=0;
    if Tangentes:
    for p_=#1:
    if (n mod 3)=0:
    dirav[n div 3]=p_;
    fi;
    if (n mod 3)=1:
    Gn[n div 3]=p_;
    Fn[n div 3]=cm*(X.u*xpart(p_),Y.u*ypart(p_));
    fi;
    if (n mod 3)=2:
    dirap[n div 3]=p_;
    fi;
    n:=n+1;
    endfor;
    N:=(n-1) div 3;
    else:
    for p_=#1:
    Gn[n]=p_;
    Fn[n]=cm*(X.u*xpart(p_),Y.u*ypart(p_));
    n:=n+1;
    endfor;
    N:=(n-1);
    fi;
    MinX=999;
    MaxX=-999;
    MinY=999;
    MaxY=-999;
    for k=0 upto N:
    if xpart(Gn[k])<MinX:
    MinX:=xpart(Gn[k]);
    fi;
    if xpart(Gn[k])>MaxX:
    MaxX:=xpart(Gn[k]);
    fi;
    if ypart(Gn[k])<MinY:
    MinY:=ypart(Gn[k]);
    fi;
    if ypart(Gn[k])>MaxY:
    MaxY:=ypart(Gn[k]);
    fi;
    endfor;
    if MinY<0:
    if MinY<>ceiling(MinY):
    MinY:=ceiling(MinY)-1;
    fi;
    fi;
    if MinY>0:
    MinY:=0;
    fi;
    if MinX<0:
    if MinX<>ceiling(MinX):
    MinX:=ceiling(MinX)-1;
    fi;
    fi;
    if MaxY<>ceiling(MaxY):
    MaxY:=ceiling(MaxY);
    fi;
    if MaxX<>ceiling(MaxX):
    MaxX:=ceiling(MaxX);
    fi;
    if MaxX<1:
    MaxX:=1;
    fi;
    if MaxY<1:
    MaxY:=1;
    fi;
    if Prolonge:
    for k=MinY-1 step y.u until MaxY+1:
    draw cm*((MinX)*X.u,k*Y.u)--cm*((MaxX)*X.u,k*Y.u) withcolor 0.75white;
    endfor;
    for k=MinX step x.u until MaxX:
    draw cm*(k*X.u,(MinY-1)*Y.u)--cm*(k*X.u,(MaxY+1)*Y.u) withcolor 0.75white;
    endfor;
    else:
    for k=MinY-1 step y.u until MaxY+1:
    draw cm*((MinX-1)*X.u,k*Y.u)--cm*((MaxX+1)*X.u,k*Y.u) withcolor 0.75white;
    endfor;
    for k=MinX-1 step x.u until MaxX+1:
    draw cm*(k*X.u,(MinY-1)*Y.u)--cm*(k*X.u,(MaxY+1)*Y.u) withcolor 0.75white;
    endfor;
    fi;
    if Prolonge:    
    drawarrow (0,(MinY-1)*Y.u*cm)--(0,(MaxY+1)*Y.u*cm);
    drawarrow ((MinX)*X.u*cm,0)--((MaxX)*X.u*cm,0);
    else:
    drawarrow (0,(MinY-1)*Y.u*cm)--(0*cm,(MaxY+1)*Y.u*cm);
    drawarrow ((MinX-1)*X.u*cm,0*cm)--((MaxX+1)*X.u*cm,0*cm);
    fi;
    label.llft(btex O etex,(0,0));
    dotlabel.bot(btex 1 etex,cm*X.u*(1,0));
    dotlabel.lft(btex 1 etex,cm*Y.u*(0,1));
    draw if Tangentes:
    Fn[0]{dir dirap[0]}
    for k=1 upto (N-1):
    ..{dir dirav[k]}Fn[k]{dir dirap[k]}
    endfor
    ..{dir dirav[N]}Fn[N]
    else:
    Fn[0]
    for k=1 upto N:
    ..Fn[k]
    endfor fi withpen pencircle scaled Epaisseur withcolor CouleurTrace;
    if PointsCourbe:
    if Prolonge:
    for k=1 upto N-1:
    fill cercles(Fn[k],0.5mm);
    endfor;
    else:
    for k=0 upto N:
    fill cercles(Fn[k],0.5mm);
    endfor;
    fi;
    fi;
    if Codes:
    \useKV[ClesFonction]{Traces};
    fi;
  \end{mplibcode}
  \else
  \begin{mpost}[mpsettings={%
      x.u:=\useKV[ClesFonction]{PasX};y.u:=\useKV[ClesFonction]{PasY};
      X.u:=\useKV[ClesFonction]{UniteX};Y.u:=\useKV[ClesFonction]{UniteY};
      Epaisseur:=\useKV[ClesFonction]{Epaisseur};
      boolean Prolonge,Tangentes,PointsCourbe,Codes;
      Prolonge=\useKV[ClesFonction]{Prolonge};
      Tangentes=\useKV[ClesFonction]{Tangentes};
      PointsCourbe=\useKV[ClesFonction]{PointsCourbe};
      Codes=\useKV[ClesFonction]{Codes};
      color CouleurTrace;
      CouleurTrace=\useKV[ClesFonction]{CouleurTrace};
      }]
      def enplace=
      xscaled X.u yscaled Y.u
      enddef;
      vardef placepoint(expr q,r)=
      (q,r)*cm enplace
      enddef;
    numeric dirav[],dirap[];
    pair Fn[],Gn[];
    n=0;
    if Tangentes:
    for p_=#1:
    if (n mod 3)=0:
    dirav[n div 3]=p_;
    fi;
    if (n mod 3)=1:
    Gn[n div 3]=p_;
    Fn[n div 3]=cm*(X.u*xpart(p_),Y.u*ypart(p_));
    fi;
    if (n mod 3)=2:
    dirap[n div 3]=p_;
    fi;
    n:=n+1;
    endfor;
    N:=(n-1) div 3;
    else:
    for p_=#1:
    Gn[n]=p_;
    Fn[n]=cm*(X.u*xpart(p_),Y.u*ypart(p_));
    n:=n+1;
    endfor;
    N:=(n-1);
    fi;
    MinX=999;
    MaxX=-999;
    MinY=999;
    MaxY=-999;
    for k=0 upto N:
    if xpart(Gn[k])<MinX:
    MinX:=xpart(Gn[k]);
    fi;
    if xpart(Gn[k])>MaxX:
    MaxX:=xpart(Gn[k]);
    fi;
    if ypart(Gn[k])<MinY:
    MinY:=ypart(Gn[k]);
    fi;
    if ypart(Gn[k])>MaxY:
    MaxY:=ypart(Gn[k]);
    fi;
    endfor;
    if MinY<0:
    if MinY<>ceiling(MinY):
    MinY:=ceiling(MinY)-1;
    fi;
    fi;
    if MinY>0:
    MinY:=0;
    fi;
    if MinX<0:
    if MinX<>ceiling(MinX):
    MinX:=ceiling(MinX)-1;
    fi;
    fi;
    if MaxY<>ceiling(MaxY):
    MaxY:=ceiling(MaxY);
    fi;
    if MaxX<>ceiling(MaxX):
    MaxX:=ceiling(MaxX);
    fi;
    if MaxX<1:
    MaxX:=1;
    fi;
    if MaxY<1:
    MaxY:=1;
    fi;
    if Prolonge:
    for k=MinY-1 step y.u until MaxY+1:
    draw cm*((MinX)*X.u,k*Y.u)--cm*((MaxX)*X.u,k*Y.u) withcolor 0.75white;
    endfor;
    for k=MinX step x.u until MaxX:
    draw cm*(k*X.u,(MinY-1)*Y.u)--cm*(k*X.u,(MaxY+1)*Y.u) withcolor 0.75white;
    endfor;
    else:
    for k=MinY-1 step y.u until MaxY+1:
    draw cm*((MinX-1)*X.u,k*Y.u)--cm*((MaxX+1)*X.u,k*Y.u) withcolor 0.75white;
    endfor;
    for k=MinX-1 step x.u until MaxX+1:
    draw cm*(k*X.u,(MinY-1)*Y.u)--cm*(k*X.u,(MaxY+1)*Y.u) withcolor 0.75white;
    endfor;
    fi;
    if Prolonge:    
    drawarrow (0,(MinY-1)*Y.u*cm)--(0,(MaxY+1)*Y.u*cm);
    drawarrow ((MinX)*X.u*cm,0)--((MaxX)*X.u*cm,0);
    else:
    drawarrow (0,(MinY-1)*Y.u*cm)--(0*cm,(MaxY+1)*Y.u*cm);
    drawarrow ((MinX-1)*X.u*cm,0*cm)--((MaxX+1)*X.u*cm,0*cm);
    fi;
    label.llft(btex O etex,(0,0));
    dotlabel.bot(btex 1 etex,cm*X.u*(1,0));
    dotlabel.lft(btex 1 etex,cm*Y.u*(0,1));
    draw if Tangentes:
    Fn[0]{dir dirap[0]}
    for k=1 upto (N-1):
    ..{dir dirav[k]}Fn[k]{dir dirap[k]}
    endfor
    ..{dir dirav[N]}Fn[N]
    else:
    Fn[0]
    for k=1 upto N:
    ..Fn[k]
    endfor fi withpen pencircle scaled Epaisseur withcolor CouleurTrace;
    if PointsCourbe:
    if Prolonge:
    for k=1 upto N-1:
    fill cercles(Fn[k],0.5mm);
    endfor;
    else:
    for k=0 upto N:
    fill cercles(Fn[k],0.5mm);
    endfor;
    fi;
    fi;
  \end{mpost}
  \fi
}

\def\MPCatmull#1#2#3#4#5{%
  % #1 la liste des points
  % #2: pas en x
  % #3: pas en y
  % #4: unit\'e en x
  % #5: unit\'e en y
  \ifluatex
  \mplibforcehmode
  \begin{mplibcode}
    boolean PointsCourbe;
    PointsCourbe=\useKV[ClesFonction]{PointsCourbe};

    boolean Code;
    Code=\useKV[ClesFonction]{Codes};
        
    x.u:=#2;
    y.u:=#3;
    X.u:=#4;
    Y.u:=#5;

    def enplace=
    xscaled X.u yscaled Y.u
    enddef;

    vardef placepoint(expr q,r)=
    (q,r)*cm enplace
    enddef;
    
    pair Fn[],Gn[];
    n=0;
    for p_=#1:
    Gn[n]=p_;
    Fn[n]=cm*(X.u*xpart(p_),Y.u*ypart(p_));
    n:=n+1;
    endfor;
    N:=(n-1);
    MinX=999;
    MaxX=-999;
    MinY=999;
    MaxY=-999;
    debutbalai=0;
    finbalai=N;
    if \useKV[ClesFonction]{Prolonge}:
    debutbalai:=2;
    finbalai:=N-2;
    fi;
    for k=debutbalai upto finbalai:
    if xpart(Gn[k])<MinX:
    MinX:=xpart(Gn[k]);
    fi;
    if xpart(Gn[k])>MaxX:
    MaxX:=xpart(Gn[k]);
    fi;
    if ypart(Gn[k])<MinY:
    MinY:=ypart(Gn[k]);
    fi;
    if ypart(Gn[k])>MaxY:
    MaxY:=ypart(Gn[k]);
    fi;
    endfor;
    if MaxY<1:
    MaxY:=2;
    fi;
    if MinY>-1:
    MinY:=-2;
    fi;
    if MinY<0:
    if MinY<>ceiling(MinY):
    MinY:=ceiling(MinY)-1;
    fi;
    fi;
    if MinX<0:
    if MinX<>ceiling(MinX):
    MinX:=ceiling(MinX)-1;
    fi;
    fi;
    if MaxY<>ceiling(MaxY):
    MaxY:=ceiling(MaxY);
    fi;
    if MaxX<>ceiling(MaxX):
    MaxX:=ceiling(MaxX);
    fi;
    if MaxX<1:
    MaxX:=1;
    fi;
    if MaxY<1:
    MaxY:=1;
    fi;
    %
    path cadreexterieur;
    cadreexterieur=(((MinX-1)*X.u*cm,(MinY-1)*Y.u*cm)--((MaxX+1)*X.u*cm,(MinY-1)*Y.u*cm)--((MaxX+1)*X.u*cm,(MaxY+1)*Y.u*cm)--((MinX-1)*X.u*cm,(MaxY+1)*Y.u*cm)--cycle);
    draw Fn[1]
    for k=2 upto N-1:
    ..{dir angle(Fn[k+1]-Fn[k-1])}Fn[k]{dir angle(Fn[k+1]-Fn[k-1])}
    endfor withpen pencircle scaled \useKV[ClesFonction]{Epaisseur} withcolor \useKV[ClesFonction]{Couleur};
    clip currentpicture to cadreexterieur;
    picture RetiensCourbe;
    RetiensCourbe=currentpicture;
    currentpicture:=nullpicture;
    for k=MinY-1 step y.u until MaxY+1:
    draw cm*((MinX-1)*X.u,k*Y.u)--cm*((MaxX+1)*X.u,k*Y.u) withcolor 0.75white;
    endfor;
    for k=MinX-1 step x.u until MaxX+1:
    draw cm*(k*X.u,(MinY-1)*Y.u)--cm*(k*X.u,(MaxY+1)*Y.u) withcolor 0.75white;
    endfor;
    if PointsCourbe:
    for k=debutbalai upto finbalai:
    fill cercles(Fn[k],0.5mm);
    endfor;
    fi;
    drawarrow (0,(MinY-1)*Y.u*cm)--(0,(MaxY+1)*Y.u*cm);
    drawarrow ((MinX-1)*X.u*cm,0)--((MaxX+1)*X.u*cm,0);
    %
    draw RetiensCourbe;
    label.llft(btex O etex,(0,0));
    dotlabel.bot(btex 1 etex,cm*X.u*(1,0));
    dotlabel.lft(btex 1 etex,cm*Y.u*(0,1));
    if Code:
    \useKV[ClesFonction]{Traces};
    fi;
  \end{mplibcode}
  \else
  \begin{mpost}[mpsettings={numeric Epaisseur; Epaisseur=\useKV[ClesFonction]{Epaisseur}; color CouleurCourbe; CouleurCourbe=\useKV[ClesFonction]{Couleur};boolean PointsCourbe;PointsCourbe=\useKV[ClesFonction]{PointsCourbe};}]
        x.u:=#2;
    y.u:=#3;
    X.u:=#4;
    Y.u:=#5;

    pair Fn[],Gn[];
    n=0;
    for p_=#1:
    Gn[n]=p_;
    Fn[n]=cm*(X.u*xpart(p_),Y.u*ypart(p_));
    n:=n+1;
    endfor;
    N:=(n-1);
    MinX=999;
    MaxX=-999;
    MinY=999;
    MaxY=-999;
    debutbalai=0;
    finbalai=N;
    if \useKV[ClesFonction]{Prolonge}:
    debutbalai:=2;
    finbalai:=N-2;
    fi;
    for k=debutbalai upto finbalai:
    if xpart(Gn[k])<MinX:
    MinX:=xpart(Gn[k]);
    fi;
    if xpart(Gn[k])>MaxX:
    MaxX:=xpart(Gn[k]);
    fi;
    if ypart(Gn[k])<MinY:
    MinY:=ypart(Gn[k]);
    fi;
    if ypart(Gn[k])>MaxY:
    MaxY:=ypart(Gn[k]);
    fi;
    endfor;
    if MaxY<1:
    MaxY:=2;
    fi;
    if MinY>-1:
    MinY:=-2;
    fi;
    if MinY<0:
    if MinY<>ceiling(MinY):
    MinY:=ceiling(MinY)-1;
    fi;
    fi;
    if MinX<0:
    if MinX<>ceiling(MinX):
    MinX:=ceiling(MinX)-1;
    fi;
    fi;
    if MaxY<>ceiling(MaxY):
    MaxY:=ceiling(MaxY);
    fi;
    if MaxX<>ceiling(MaxX):
    MaxX:=ceiling(MaxX);
    fi;
    if MaxX<1:
    MaxX:=1;
    fi;
    if MaxY<1:
    MaxY:=1;
    fi;
    %
    path cadreexterieur;
    cadreexterieur=(((MinX-1)*X.u*cm,(MinY-1)*Y.u*cm)--((MaxX+1)*X.u*cm,(MinY-1)*Y.u*cm)--((MaxX+1)*X.u*cm,(MaxY+1)*Y.u*cm)--((MinX-1)*X.u*cm,(MaxY+1)*Y.u*cm)--cycle);
    draw Fn[1]
    for k=2 upto N-1:
    ..{dir angle(Fn[k+1]-Fn[k-1])}Fn[k]{dir angle(Fn[k+1]-Fn[k-1])}
    endfor withpen pencircle scaled \useKV[ClesFonction]{Epaisseur} withcolor \useKV[ClesFonction]{Couleur};
    clip currentpicture to cadreexterieur;
    picture RetiensCourbe;
    RetiensCourbe=currentpicture;
    currentpicture:=nullpicture;
    for k=MinY-1 step y.u until MaxY+1:
    draw cm*((MinX-1)*X.u,k*Y.u)--cm*((MaxX+1)*X.u,k*Y.u) withcolor 0.75white;
    endfor;
    for k=MinX-1 step x.u until MaxX+1:
    draw cm*(k*X.u,(MinY-1)*Y.u)--cm*(k*X.u,(MaxY+1)*Y.u) withcolor 0.75white;
    endfor;
    if PointsCourbe:
    for k=debutbalai upto finbalai:
    fill cercles(Fn[k],0.5mm);
    endfor;
    fi;
    drawarrow (0,(MinY-1)*Y.u*cm)--(0,(MaxY+1)*Y.u*cm);
    drawarrow ((MinX-1)*X.u*cm,0)--((MaxX+1)*X.u*cm,0);
    %
    draw RetiensCourbe;
    label.llft(btex O etex,(0,0));
    dotlabel.bot(btex 1 etex,cm*X.u*(1,0));
    dotlabel.lft(btex 1 etex,cm*Y.u*(0,1));
  \end{mpost}
  \fi
}%

\def\buildtabfonction{%
  \[%
    \begin{array}{|>{\columncolor{gray!15}}c|*{\number\numexpr\ListeFonctionlen}{>{\centering\arraybackslash}p{\useKV[ClesFonction]{Largeur}}|}}%
      \hline%
      \useKV[ClesFonction]{Variable}\xintFor* ##1 in {\xintSeq {1}{\ListeFonctionlen}}\do{&\num{\ListeFonction[##1]}}\\
      \hline%
      \useKV[ClesFonction]{Nom}(\useKV[ClesFonction]{Variable})\xintFor* ##1 in {\xintSeq {1}{\ListeFonctionlen}}\do{&\ifboolKV[ClesFonction]{Vide}{}{\StrSubstitute{\useKV[ClesFonction]{Calcul}}{\useKV[ClesFonction]{Variable}}{(\ListeFonction[##1])}[\tempab]\num{\fpeval{\tempab}}}}%
      \\\hline
    \end{array}%
  \]%
}%