%%%
% Papiers
%%%
\def\filedatePapiers{2024/08/04}%
\def\fileversionPapiers{0.1}%
\message{-- \filedatePapiers\space v\fileversionPapiers}%
%
\setKVdefault[Papiers]{Cinq=true,Seyes=false,Echelle=8,Isometrique=false,IsometriquePointe=false,Millimetre=false,Triangle=false,Largeur=5,Hauteur=4,Couleur=black,Grille=-1,GrillePointe=-1,PageEntiere=false,ZoneTexte=false,Baseline=false,CodeAfter=false}%
\defKV[Papiers]{Traces=\setKV[Papiers]{CodeAfter}}

%\def\MPBaseLineSkip#1#2#3{%à retravailler : ne fonctionne pas :(
%  %
%  \ifluatex
%  \mplibforcehmode
%  \begin{mplibcode}
%    path horizon,verticon;
%    horizon=(0,0)--(#1*cm,0);
%    %verticon=(0,0)--(0,#2*cm);
%    drawoptions(withcolor #3);
%    %for k=0 step 0.5 until #1:
%    %draw verticon shifted((k*cm,0));
%    %endfor;
%    for k=(#2-(\mpdim{1.6ex}/1cm)) step (-\mpdim{\baselineskip}/1cm) until 0:
%    draw horizon shifted((0,k*cm));
%    endfor;
%    drawoptions(withcolor blue);
%    for k=#2 step (-\mpdim{\baselineskip}/1cm) until 0:
%    draw horizon shifted((0,k*cm));
%    endfor;
%  \end{mplibcode}
%  \fi
%}

\def\MPGrilleCode{%
  PapierLargeur=\PapierLargeur;
  PapierHauteur=\PapierHauteur;
  PapierGrille=\PapierGrille;
  PapierGrillePointe=\PapierGrillePointe;
  color PapierCouleur;
  PapierCouleur=\useKV[Papiers]{Couleur};
  %
  path horizon,verticon;
  horizon=(0,0)--(PapierLargeur*cm,0);
  verticon=(0,0)--(0,PapierHauteur*cm);
  vardef Papier=
  picture Papiers;
  Papiers=image(
  drawoptions(withcolor PapierCouleur);
  if PapierGrillePointe>0:
  for k=0 step (PapierGrillePointe*100) until (PapierLargeur*100):
  for l=0 step (PapierGrillePointe*100) until (PapierHauteur*100):
  fill (cercles((0,0),1pt) shifted(((k/100)*cm,(l/100)*cm)));
  endfor;
  endfor;
  else:
  for k=0 step (PapierGrille*100) until (PapierLargeur*100):
  draw verticon shifted(((k/100)*cm,0));
  endfor;
  for k=0 step (PapierGrille*100) until (PapierHauteur*100):
  draw horizon shifted((0,(k/100)*cm));
  endfor;
  fi;
  );
  Papiers
  enddef;
}%

\def\MPGrille{%
  \ifluatex%
  \mplibforcehmode%
  \begin{mplibcode}%
    \MPGrilleCode
    trace Papier;
  \end{mplibcode}%
  \else%
  \begin{mpost}[mpsettings={\MPGrilleCode}]
    trace Papier;
  \end{mpost}%
  \fi%
}%

\def\MPCinq{%
  \xdef\PapierGrille{0.5}%
  \MPGrille
}%

\def\MPSeyesCode{%
  PapierLargeur=\PapierLargeur;
  PapierHauteur=\PapierHauteur;
  color PapierCouleur;
  PapierCouleur=\useKV[Papiers]{Couleur};
  u:=\useKV[Papiers]{Echelle};
  path horizon,verticon;
  horizon=(0,0)--(PapierLargeur*cm,0);
  verticon=(0,0)--(0,PapierHauteur*cm);
  vardef Seyes=
  picture PapierSeyes;
  PapierSeyes=image(
  drawoptions(withcolor PapierCouleur);
  for k=0 step u until (PapierLargeur*10):
  draw verticon shifted(((k/10)*cm,0));
  endfor;
  for k=0 step (u/4) until (PapierHauteur*10):
  draw horizon shifted((0,(k/10)*cm)) withpen pencircle scaled 0.5;
  endfor;
  for k=0 step u until (PapierHauteur*10):
  draw horizon shifted((0,(k/10)*cm)) withpen pencircle scaled 1.25;
  endfor;
  );
  PapierSeyes
  enddef;
}

\def\MPSeyes{%
  \ifluatex%
  \mplibforcehmode%
  \begin{mplibcode}%
    \MPSeyesCode
    trace Seyes;
  \end{mplibcode}%
  \else%
  \begin{mpost}[mpsettings={\MPSeyesCode}]%
    trace Seyes;
  \end{mpost}
  \fi%
}%

\def\MPMillimetreCode{%
  PapierLargeur=\PapierLargeur;
  PapierHauteur=\PapierHauteur;
  color PapierCouleur;
  PapierCouleur=\useKV[Papiers]{Couleur};
  path horizon,verticon;
  horizon=(0,0)--(PapierLargeur*cm,0);
  verticon=(0,0)--(0,PapierHauteur*cm);
  vardef MPMilli=
  picture MPMillimetre;
  MPMillimetre=image(
  drawoptions(withcolor PapierCouleur);
  for k=0 step 1 until (PapierLargeur*10):
  draw verticon shifted(((k/10)*cm,0)) withpen pencircle scaled 0.2;
  endfor;
  for k=0 step 5 until (PapierLargeur*10):
  draw verticon shifted(((k/10)*cm,0)) withpen pencircle scaled 0.5;
  endfor;
  for k=0 step 1 until (PapierLargeur):
  draw verticon shifted((k*cm,0)) withpen pencircle scaled 1.25;
  endfor;
  for k=0 step 1 until (PapierHauteur*10):
  draw horizon shifted((0,(k/10)*cm)) withpen pencircle scaled 0.2;
  endfor;
  for k=0 step 5 until (PapierHauteur*10):
  draw horizon shifted((0,(k/10)*cm)) withpen pencircle scaled 0.5;
  endfor;
  for k=0 step 1 until (PapierHauteur):
  draw horizon shifted((0,k*cm)) withpen pencircle scaled 1.25;
  endfor;
  );
  MPMillimetre
  enddef;
}

\def\MPMillimetre{%
  \ifluatex%
  \mplibforcehmode%
  \begin{mplibcode}%
    \MPMillimetreCode
    trace MPMilli;
  \end{mplibcode}%
  \else%
  \begin{mpost}[mpsettings={\MPMillimetreCode}]
    trace MPMilli;
  \end{mpost}%
  \fi%
}%

\def\MPIsometriqueCode{%
  PapierLargeur=\PapierLargeur;
  PapierHauteur=\PapierHauteur;
  color PapierCouleur;
  PapierCouleur=\useKV[Papiers]{Couleur};
  path diagon,antidiagon;
  diagon=(0,0)--PapierHauteur*cm*(sqrt(3),1);
  antidiagon=(0,0)--PapierHauteur*cm*(-sqrt(3),1);
  vardef MPIso=
  picture MPIsom;
  MPIsom=image(
  drawoptions(withcolor PapierCouleur);
  for k=0 upto 2*(PapierLargeur+PapierHauteur):
  draw antidiagon shifted((k*cm,0));
  endfor;
  for k=-(2*ceiling(PapierLargeur+PapierHauteur)) upto (2*ceiling(PapierLargeur+PapierHauteur)):
  draw diagon shifted((k*cm,0));
  endfor;
  clip currentpicture to polygone((0,0),(PapierLargeur*cm,0),(PapierLargeur*cm,PapierHauteur*cm),(0,PapierHauteur*cm));
  );
  MPIsom
  enddef;
}

\def\MPIsometrique{%
  \ifluatex%
  \mplibforcehmode%
  \begin{mplibcode}%
    \MPIsometriqueCode
    trace MPIso;
  \end{mplibcode}%
  \else%
  \begin{mpost}[mpsettings={\MPIsometriqueCode}]
    trace MPIso;
  \end{mpost}%
  \fi%
}%

\def\MPIsometriquePointeCode{%
  PapierLargeur=\PapierLargeur;
  PapierHauteur=\PapierHauteur;
  color PapierCouleur;
  PapierCouleur=\useKV[Papiers]{Couleur};
  pair diagon,antidiagon;
  diagon=(1pt,1pt);
  antidiagon=(1pt+0.5cm,1pt+0.5*(sqrt(3)*1cm/3));
  vardef TraceIsoPointe=
  picture PapierIsoPointe;
  PapierIsoPointe=image(
  drawoptions(withcolor PapierCouleur);
  for k=0 step (sqrt(3)/3) until (PapierHauteur):
  for l=0 step 1 until (PapierLargeur):
  fill (cercles(diagon,1pt) shifted((l*cm,k*cm)));
  fill (cercles(antidiagon,1pt) shifted((l*cm,k*cm)));
  endfor;
  endfor;
  clip currentpicture to polygone((0,0),(PapierLargeur*cm+2pt,0),(PapierLargeur*cm+2pt,PapierHauteur*cm+1pt),(0,PapierHauteur*cm+1pt));
  );
  PapierIsoPointe
  enddef;
}
    
\def\MPIsometriquePointe{%
  \ifluatex%
  \mplibforcehmode%
  \begin{mplibcode}%
    \MPIsometriquePointeCode
    trace TraceIsoPointe;
  \end{mplibcode}%
  \else%
  \begin{mpost}[mpsettings={\MPIsometriquePointeCode}]
    trace TraceIsoPointe;
  \end{mpost}%
  \fi%
}%

\def\MPTriangulaireCode{%
  PapierLargeur=\PapierLargeur;
  PapierHauteur=\PapierHauteur;
  color PapierCouleur;
  PapierCouleur=\useKV[Papiers]{Couleur};
  path horizon,diagon,antidiagon;
  horizon=(0,0)--(PapierLargeur*cm,0);
  diagon=(0,0)--PapierHauteur*cm*(sqrt(3)/3,1);
  antidiagon=(0,0)--PapierHauteur*cm*(-sqrt(3)/3,1);
  vardef PapierTri=
  picture PapierTriangle;
  PapierTriangle=image(
  drawoptions(withcolor PapierCouleur);
  for k=0 upto 2*ceiling(PapierLargeur):
  draw diagon shifted((k*cm,0));
  draw antidiagon shifted((k*cm,0));
  endfor;
  for k=sqrt(3) step (sqrt(3)) until (PapierHauteur):
  draw diagon shifted((0,k*cm));
  endfor;
  for k=0 step (sqrt(3)/2) until (PapierHauteur):
  draw horizon shifted((0,k*cm));
  endfor;
  clip currentpicture to polygone((0,0),(PapierLargeur*cm,0),(PapierLargeur*cm,PapierHauteur*cm),(0,PapierHauteur*cm));
  );
  PapierTriangle
  enddef;
}

\def\MPTriangulaire{%
  \ifluatex%
  \mplibforcehmode%
  \begin{mplibcode}%
    \MPTriangulaireCode
    trace PapierTri;
  \end{mplibcode}%
  \else%
  \begin{mpost}[mpsettings={\MPTriangulaireCode}]
    trace PapierTri;
  \end{mpost}%
  \fi%
}%

\RequirePackage{ifoddpage}

\newlength{\PapierLeftCurrent}

\def\PfCPapierTest{%
  \xintifboolexpr{\useKV[Papiers]{GrillePointe}>0 || \useKV[Papiers]{Grille}>0}{%
    \MPGrille%
  }{\ifboolKV[Papiers]{Baseline}{%
      \MPBaseLineSkip{\PapierLargeur}{\PapierHauteur}{\PapierCouleur}%
    }{%
      \ifboolKV[Papiers]{IsometriquePointe}{%
        \MPIsometriquePointe%
      }{%
        \ifboolKV[Papiers]{Triangle}{%
          \MPTriangulaire%
        }{\ifboolKV[Papiers]{Millimetre}{%
            \MPMillimetre%
          }{\ifboolKV[Papiers]{Isometrique}{%
              \MPIsometrique%
            }{\ifboolKV[Papiers]{Seyes}{%
                \MPSeyes%
              }{\MPCinq%
              }%
            }%
          }%
        }%
      }%
    }%
  }%
}%

\newcommand\Papiers[1][]{%
  \useKVdefault[Papiers]%
  \setKV[Papiers]{#1}%
  \xdef\PapierLargeur{\useKV[Papiers]{Largeur}}%
  \xdef\PapierHauteur{\useKV[Papiers]{Hauteur}}%
  \xdef\PapierCouleur{\useKV[Papiers]{Couleur}}%
  \xdef\PapierGrille{\useKV[Papiers]{Grille}}%
  \xdef\PapierGrillePointe{\useKV[Papiers]{GrillePointe}}%
  \ifboolKV[Papiers]{ZoneTexte}{%
    \checkoddpage\ifoddpage%
    \setlength{\PapierLeftCurrent}{\oddsidemargin}%
    \else%
    \setlength{\PapierLeftCurrent}{\evensidemargin}%
    \fi%
    \xdef\PapierLeft{\the\dimexpr1in+\PapierLeftCurrent}%
    \xdef\PapierBottom{\fpeval{\paperheight-\textheight-\voffset-\headheight-\topmargin-\headsep-1in}}%
    \xdef\PapierHauteur{\fpeval{\textheight/1cm}}%
    \xdef\PapierLargeur{\fpeval{\textwidth/1cm}}%
    \begin{tikzpicture}[remember picture,overlay]%
      \node[anchor=south west,inner sep=0pt,transform canvas={xshift=\PapierLeft,yshift=\PapierBottom}] at (current page.south west) {%
        \PfCPapierTest%
      };%
    \end{tikzpicture}%
  }{%
    \ifboolKV[Papiers]{PageEntiere}{%
      \checkoddpage\ifoddpage%
      \setlength{\PapierLeftCurrent}{\oddsidemargin}%
      \else%
      \setlength{\PapierLeftCurrent}{\evensidemargin}%
      \fi%
      \xdef\PapierLeft{\the\dimexpr1in+\PapierLeftCurrent}%
      \xdef\PapierBottom{\fpeval{\paperheight-\textheight-\voffset-\headheight-\topmargin-\headsep-1in}}%
      \xdef\PapierHauteur{\fpeval{\paperheight/1cm}}%
      \xdef\PapierLargeur{\fpeval{\paperwidth/1cm}}%
      \begin{tikzpicture}[remember picture,overlay]%
        \node[anchor=south west,inner sep=0pt] at (current page.south west) {%
          \PfCPapierTest%
        };%
      \end{tikzpicture}%
    }{%
      \PfCPapierTest
    }%
  }%
}%