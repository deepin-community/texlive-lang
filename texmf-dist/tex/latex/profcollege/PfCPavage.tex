%%%
% Pavage
%%%
\def\filedatePavage{2024/08/04}%
\def\fileversionPavage{0.1}%
\message{-- \filedatePavage\space v\fileversionPavage}%
%
\setKVdefault[Pavage]{%
  Before=false,
  Niveau=3,
  Couleur=white,
  Backgrounds=false,
  Numerotation=false,
  Depart=0,
  Complete=false,
  After=false,
  Demo=false,
  Regulier=false,
  Cote=1,
  Lignes=4,
  Colonnes=4,
  SemiRegulier=false,
  Reseau=false,
  Basei={u*(0.75,0)},
  Basej={u*(0,0.5)},
  Quadrilatere=false,
  Escher=false,
  Einstein=false,
  Vampire=false,
  Rayon=1,
  Epaisseur=1,
  Position=0.5,
  Ecart=10,
  Colore=false,
  CouleurUn=white,
  CouleurDeux=white,
  CouleurTrois=white,
  CouleurQuatre=white
}%
\defKV[Pavage]{Motif=\setKV[Pavage]{Before}}%
\defKV[Pavage]{Traces=\setKV[Pavage]{After}}%
\defKV[Pavage]{ArrierePlan=\setKV[Pavage]{Backgrounds}}%
\defKV[Pavage]{CouleurUn=\setKV[Pavage]{Colore}}%
\defKV[Pavage]{CouleurDeux=\setKV[Pavage]{Colore}}%
\defKV[Pavage]{CouleurTrois=\setKV[Pavage]{Colore}}%
\defKV[Pavage]{CouleurQuatre=\setKV[Pavage]{Colore}}%

\def\MPPavageBase{%
  boolean Before,Numerote,Complete,Backgrounds,After,Demo;
  Before=\useKV[Pavage]{Before};
  After=\useKV[Pavage]{After};
  Numerote=\useKV[Pavage]{Numerotation};
  Complete=\useKV[Pavage]{Complete};
  Backgrounds=\useKV[Pavage]{Backgrounds};
  Demo=\useKV[Pavage]{Demo};
  color ColPavage,ColArrierePlan;
  ColPavage=\useKV[Pavage]{Couleur};
  ColArrierePlan=if Backgrounds:\useKV[Pavage]{ArrierePlan} else:ColPavage fi;
  Niveau=\useKV[Pavage]{Niveau};
  Depart=\useKV[Pavage]{Depart};
  Epaisseur=\useKV[Pavage]{Epaisseur};
  vardef pavages(expr chemin,tour,coul)=
  if Backgrounds:
  fill CadrePavage withcolor ColArrierePlan;
  fi;
  save Base;
  picture Base;
  pair A,B,C,D;
  A=u*(0,0);
  B=u*(1,0);
  C=u*(1,1);
  D=u*(0,1);
  path especes;
  especes=chemin--rotation(chemin,B,-90)--reverse(rotation(symetrie(chemin,B,C),B,-90))--rotation(chemin,B,90)--cycle;
  if tour=0:
    Base=image(
      trace chemin;
      );
  elseif tour=1:
    Base=image(
      fill especes withcolor coul;
      trace especes;
      A:=A shifted(u*(0,-1));
      B:=B shifted(u*(1,-1));
      C:=C shifted(u*(1,0));
      );
  elseif tour>=2:
    Base=pavages(chemin,tour-1,coul);
    Base:=image(
      trace Base;
      trace symetrie(Base,B,C);
      trace rotation(symetrie(Base,B,C),B,-90);
      trace rotation(Base,B,90);
      A:=A shifted(u*(0,-(2**(tour-1))));
      B:=B shifted(u*(2**(tour-1),-(2**(tour-1))));
      C:=C shifted(u*(2**(tour-1),0));
      );
  fi;
  Base
enddef;
%
vardef pavagescar(expr chemin,tour,coul)=
  save BaseCar;
  picture BaseCar;
  pair A,B,C,D;
  A=u*(0,0);
  B=u*(1,0);
  C=u*(1,1);
  D=u*(0,1);
  if tour=0:
    BaseCar=image(
      trace chemin;
      trace A--B--C--D--cycle dashed dashpattern(on12bp off6bp on3bp off 6bp) withcolor gris;
      );
  elseif tour=1:
    BaseCar=image(
      trace chemin;
      drawoptions(dashed evenly);
      trace rotation(chemin,B,-90);
      trace reverse(rotation(symetrie(chemin,B,C),B,-90));
      trace rotation(chemin,B,90);
      drawoptions();
      A:=A shifted(u*(0,-1));
      B:=B shifted(u*(1,-1));
      C:=C shifted(u*(1,0));
      trace A--B--C--D--cycle dashed dashpattern(on12bp off6bp on3bp off 6bp) withcolor gris;
      trace iso(A,D)--iso(B,C) dashed dashpattern(on12bp off6bp on3bp off 6bp) withcolor gris;
      trace iso(A,B)--iso(C,D) dashed dashpattern(on12bp off6bp on3bp off 6bp) withcolor gris;
      );
  elseif tour>=2:
    BaseCar=image(
      trace pavages(chemin,tour-1,coul);
      trace symetrie(pavages(chemin,tour-1,coul),B,C);%coul+0.5blanc
      trace rotation(symetrie(pavages(chemin,tour-1,coul),B,C),B,-90);
      trace rotation(pavages(chemin,tour-1,coul),B,90);
      A:=A shifted(u*(0,-(2**(tour-1))));
      B:=B shifted(u*(2**(tour-1),-(2**(tour-1))));
      C:=C shifted(u*(2**(tour-1),0));
      );
  fi;
  BaseCar
  enddef;
  %
  vardef CadrePavage=
  u*(0,1)--u*(0,-(2**Niveau-1))--u*(2**Niveau,-(2**Niveau-1))--u*(2**(Niveau),1)--cycle
  enddef;
  %
  vardef Numerotation=
  n:=Depart;
  pair B;
  B=u*(1,0);
  if Complete:
  for k=0 upto 2**(Niveau)-2:
  for l=0 upto (2**(Niveau-1))-1:
  n:=n+1;
  if (k mod 2)=0:
  label(TEX(decimal(n)),B+u*(2l,-k));
  else:
  if l<(2**(Niveau-1))-1:
  label(TEX(decimal(n)),B+u*(2l+1,-k));
  else:
  n:=n-1;
  fi;
  fi;
  endfor;
  endfor;
  else:
  for k=0 upto (2**(Niveau-1))-1:
  for l=0 upto (2**(Niveau-1))-1:
  n:=n+1;
  label(TEX(decimal(n)),B+2u*(l,-k));
  endfor;
  endfor;
  fi;
  enddef;
  %
  vardef NumerotationPDF=
  n:=Depart;
  pair B;
  B=u*(1,0);
  if Complete:
  for k=0 upto 2**(Niveau)-2:
  for l=0 upto (2**(Niveau-1))-1:
  n:=n+1;
  if (k mod 2)=0:
  label(TEX(decimal(n)),B+u*(2l,-k));
  else:
  if l<(2**(Niveau-1))-1:
  label(LATEX(decimal(n)),B+u*(2l+1,-k));
  fi;
  fi;
  endfor;
  endfor;
  else:
  for k=0 upto (2**(Niveau-1))-1:
  for l=0 upto (2**(Niveau-1))-1:
  n:=n+1;
  label(LATEX(decimal(n)),B+2u*(l,-k));
  endfor;
  endfor;
  fi;
  enddef;
  %
  vardef PlacePointSupport=
  n:=Depart;
  pair K[],L[],M[],N[];
  K[0]=u*(0,0);
  M[0]=u*(1,1);
  if Complete:
  for k=0 upto 2**(Niveau)-2:
  for l=0 upto (2**(Niveau-1))-1:
  n:=n+1;
  if (k mod 2)=0:
  K[n]:=K0+u*(2l,-k);
  M[n]:=M0+u*(2l,-k);
  else:
  if l<(2**(Niveau-1))-1:
  K[n]:=K0+u*(2l+1,-k);
  M[n]:=M0+u*(2l+1,-k);
  else:
  n:=n-1;
  fi;
  fi;
  endfor;
  endfor;
  else:
  for k=0 upto (2**(Niveau-1))-1:
  for l=0 upto (2**(Niveau-1))-1:
  n:=n+1;
  K[n]=K[0]+2u*(l,-k);
  M[n]=M[0]+2u*(l,-k);
  endfor;
  endfor;
  fi;
  enddef;
  vardef EffectuerTraces=
  \useKV[Pavage]{Traces}
  enddef;
}%

\def\MPReseauRegulier{%
  boolean Before,Numerote,Backgrounds,After;
  Before=\useKV[Pavage]{Before};
  After=\useKV[Pavage]{After};
  Numerote=\useKV[Pavage]{Numerotation};
  Backgrounds=\useKV[Pavage]{Backgrounds};
  color ColPavage,ColArrierePlan;
  ColPavage=\useKV[Pavage]{Couleur};
  ColArrierePlan=if Backgrounds:\useKV[Pavage]{ArrierePlan} else:ColPavage fi;
  Cote=\useKV[Pavage]{Cote};
  Lignes=\useKV[Pavage]{Lignes};
  Colonnes=\useKV[Pavage]{Colonnes};
  Niveau=\useKV[Pavage]{Niveau};
  Depart=\useKV[Pavage]{Depart};
  Epaisseur=\useKV[Pavage]{Epaisseur};
  pair A[];
  A1=(0,0);
  A2-A1=u*Cote*(1,0);
  vardef PavageRegulier(expr Niv)=
  picture Retour;
  if Niv=3:
  A3=rotation(A2,A1,60);
  path TriBase,CadrePavage;
  TriBase=polygone(A1,A2,A3);
  CadrePavage=polygone(A1,Cote*cm*(Colonnes,0),Cote*cm*(Colonnes,0.5*sqrt(3)*Lignes),Cote*cm*(0,0.5*sqrt(3)*Lignes));
  Retour=image(
  fill CadrePavage withcolor ColPavage;
  for k=0 upto Lignes:
  for l=-1 upto Colonnes+1:
  if k mod 2=0:
  trace TriBase shifted (Cote*cm*(l,0.5*sqrt(3)*k));
  else:
  trace TriBase shifted (Cote*cm*(l+0.5,0.5*sqrt(3)*k));
  fi;
  endfor;
  endfor;
  clip currentpicture to CadrePavage;
  trace CadrePavage withpen pencircle scaled Epaisseur;
  );
  elseif Niv=4:
  A3=rotation(A1,A2,-90);
  A4-A3=A1-A2;
  path QuadriBase,CadrePavage;
  QuadriBase=polygone(A1,A2,A3,A4);
  CadrePavage=polygone(A1,Cote*cm*(Colonnes,0),Cote*cm*(Colonnes,Lignes),Cote*cm*(0,Lignes));
  Retour=image(
  fill CadrePavage withcolor ColPavage;
  for k=0 upto Lignes-1:
  for l=0 upto Colonnes-1:
  trace QuadriBase shifted (Cote*cm*(l,k));
  endfor;
  endfor;
  clip currentpicture to CadrePavage;
  trace CadrePavage withpen pencircle scaled Epaisseur;
  );
  elseif Niv=6:
  A3=rotation(A1,A2,-120);
  A4=rotation(A2,A3,-120);
  A5=rotation(A3,A4,-120);
  A6=rotation(A4,A5,-120);  
  path HexaBase,CadrePavage;
  HexaBase=polygone(A1,A2,A3,A4,A5,A6);
  CadrePavage=polygone(Cote*cm*(1,0),Cote*cm*(3*Colonnes,0),Cote*cm*(3*Colonnes,sqrt(3)*(Lignes-1)),Cote*cm*(1,sqrt(3)*(Lignes-1)));
  Retour=image(%
  fill CadrePavage withcolor ColPavage;
  for k=-1 upto 2*Lignes+1:
  for l=-1 upto Colonnes+1:
  if k mod 2=1:
  trace HexaBase shifted (Cote*cm*(3*l,0.5*sqrt(3)*k));
  else:
  trace HexaBase shifted (Cote*cm*(1.5+3*l,sqrt(3)+0.5*sqrt(3)*k));
  fi;
  endfor;
  endfor;
  clip currentpicture to CadrePavage;
  trace CadrePavage withpen pencircle scaled Epaisseur;
  );
  fi;
  Retour
  enddef;
  % 
  vardef PlacePointSupport=
  n:=Depart;
  pair B;pair C;pair vdepla;
  pair Hexa[],Hexb[],Hexc[],Hexd[],Hexe[],Hexf[];
  pair Cara[],Carb[],Carc[],Card[];
  pair Tria[],Trib[],Tric[];
  path Maille[];
  if Niveau=3:
    B=iso(A1,A2,A3);
    C=symetrie(B,A2,A3);
    for k=1 upto Lignes:
      for l=1 upto 2*Colonnes-1:
        n:=n+1;
        if k mod 2=1:
        nbv:=(k-1)*1.5;
          if l mod 2=1:
          vdepla:=(Cote*cm*((l-1) div 2),nbv*Cote*cm*sqrt(3)/3);
          else:
          vdepla:=(Cote*cm*((l-1) div 2),nbv*Cote*cm*sqrt(3)/3)+(C-B);
          fi;
          else:
          nbv:=(k-1)*1.5+0.5;
        if l mod 2=1:
        vdepla:=(Cote*cm*((l-1) div 2),nbv*Cote*cm*sqrt(3)/3);
        else:
        vdepla:=(Cote*cm*((l-1) div 2),(nbv*Cote*cm*sqrt(3)/3))+(xpart(C-B),-ypart(C-B));
        fi;
        fi;
        if (n mod 2)=0:
        Tria[n]=rotation(A1,B,180) shifted vdepla;
        Trib[n]=rotation(A2,B,180) shifted vdepla;
        Tric[n]=rotation(A3,B,180) shifted vdepla;
        else:
        Tria[n]-A1=vdepla;
        Trib[n]-A2=vdepla;
        Tric[n]-A3=vdepla;
        fi;
        Maille[n]=polygone(Tria[n],Trib[n],Tric[n]);
  endfor;
  endfor;
  elseif Niveau=4:
  B=iso(A1,A2,A3,A4);
  for k=1 upto Lignes:
  for l=1 upto Colonnes:
  n:=n+1;
  vdepla:=Cote*cm*((l-1),(k-1));
  Cara[n]-A1=vdepla;
  Carb[n]-A2=vdepla;
  Carc[n]-A3=vdepla;
  Card[n]-A4=vdepla;
  Maille[n]=polygone(Cara[n],Carb[n],Carc[n],Card[n]);
  endfor;
  endfor;
  elseif Niveau=6:
  B=iso(A1,A2,A3,A4,A5,A6);
  C=symetrie(B,A3,A4);
    for k=1 upto Lignes-1:
      for l=1 upto 2*Colonnes-1:
        n:=n+1;
        nbv:=(k-1)*3;
        if l mod 2=1:
          vdepla:=(1.5*Cote*cm*(l-1)+1.5*Cote*cm,nbv*Cote*cm*sqrt(3)/3);
        else:
          vdepla:=(1.5*Cote*cm*(l-2)+1.5*Cote*cm,nbv*Cote*cm*sqrt(3)/3)+(C-B);
        fi;
        if k=Lignes-1:
        if l mod 2=1:
        Hexa[n]-A1=vdepla;
        Hexb[n]-A2=vdepla;
        Hexc[n]-A3=vdepla;
        Hexd[n]-A4=vdepla;
        Hexe[n]-A5=vdepla;
        Hexf[n]-A6=vdepla;
        else:
        n:=n-1;
        fi;
        else:
        Hexa[n]-A1=vdepla;
        Hexb[n]-A2=vdepla;
        Hexc[n]-A3=vdepla;
        Hexd[n]-A4=vdepla;
        Hexe[n]-A5=vdepla;
        Hexf[n]-A6=vdepla;
        fi;
        Maille[n]:=polygone(Hexa[n],Hexb[n],Hexc[n],Hexd[n],Hexe[n],Hexf[n]);
      endfor;
    endfor;
  fi;
  enddef;
  %
  vardef Numerotation=
  n:=Depart;
  pair B;pair C;pair vdepla;
  if Niveau=3:
    B=iso(A1,A2,A3);
    C=symetrie(B,A2,A3);
    for k=1 upto Lignes:
      for l=1 upto 2*Colonnes-1:
        n:=n+1;
        if k mod 2=1:
        nbv:=(k-1)*1.5;
          if l mod 2=1:
          vdepla:=(Cote*cm*((l-1) div 2),nbv*Cote*cm*sqrt(3)/3);
          else:
          vdepla:=(Cote*cm*((l-1) div 2),nbv*Cote*cm*sqrt(3)/3)+(C-B);
          fi;
          else:
          nbv:=(k-1)*1.5+0.5;
        if l mod 2=1:
        vdepla:=(Cote*cm*((l-1) div 2),nbv*Cote*cm*sqrt(3)/3);
        else:
        vdepla:=(Cote*cm*((l-1) div 2),(nbv*Cote*cm*sqrt(3)/3))+(xpart(C-B),-ypart(C-B));
        fi;
        fi;
        label(TEX(decimal(n)),B shifted vdepla);
    endfor;
  endfor;
  elseif Niveau=4:
  B=iso(A1,A2,A3,A4);
  for k=1 upto Lignes:
  for l=1 upto Colonnes:
  n:=n+1;
  vdepla:=Cote*cm*((l-1),(k-1));
  label(TEX(decimal(n)),B shifted vdepla);
  endfor;
  endfor;
  elseif Niveau=6:
  B=iso(A1,A2,A3,A4,A5,A6);
  C=symetrie(B,A3,A4);
    for k=1 upto Lignes-1:
      for l=1 upto 2*Colonnes-1:
        n:=n+1;
        nbv:=(k-1)*3;
        if l mod 2=1:
          vdepla:=(1.5*Cote*cm*(l-1)+1.5*Cote*cm,nbv*Cote*cm*sqrt(3)/3);
        else:
          vdepla:=(1.5*Cote*cm*(l-2)+1.5*Cote*cm,nbv*Cote*cm*sqrt(3)/3)+(C-B);
        fi;
        if k=Lignes-1:
        if l mod 2=1:
        label(TEX(decimal(n)),B shifted vdepla);
        else:
        n:=n-1;
        fi;
        else:
        label(TEX(decimal(n)),B shifted vdepla);
        fi;
      endfor;
    endfor;
  fi;
  enddef;
  %
  vardef NumerotationPDF=
  n:=Depart;
  pair B;pair C;pair vdepla;
  if Niveau=3:
    B=iso(A1,A2,A3);
    C=symetrie(B,A2,A3);
    for k=1 upto Lignes:
      for l=1 upto 2*Colonnes-1:
        n:=n+1;
        if k mod 2=1:
        nbv:=(k-1)*1.5;
          if l mod 2=1:
          vdepla:=(Cote*cm*((l-1) div 2),nbv*Cote*cm*sqrt(3)/3);
          else:
          vdepla:=(Cote*cm*((l-1) div 2),nbv*Cote*cm*sqrt(3)/3)+(C-B);
          fi;
          else:
          nbv:=(k-1)*1.5+0.5;
        if l mod 2=1:
        vdepla:=(Cote*cm*((l-1) div 2),nbv*Cote*cm*sqrt(3)/3);
        else:
        vdepla:=(Cote*cm*((l-1) div 2),(nbv*Cote*cm*sqrt(3)/3))+(xpart(C-B),-ypart(C-B));
        fi;
        fi;
        label(LATEX(decimal(n)),B shifted vdepla);
    endfor;
  endfor;
  elseif Niveau=4:
  B=iso(A1,A2,A3,A4);
  for k=1 upto Lignes:
  for l=1 upto Colonnes:
  n:=n+1;
  vdepla:=Cote*cm*((l-1),(k-1));
  label(LATEX(decimal(n)),B shifted vdepla);
  endfor;
  endfor;
  elseif Niveau=6:
  B=iso(A1,A2,A3,A4,A5,A6);
  C=symetrie(B,A3,A4);
    for k=1 upto Lignes-1:
      for l=1 upto 2*Colonnes-1:
        n:=n+1;
        nbv:=(k-1)*3;
        if l mod 2=1:
          vdepla:=(1.5*Cote*cm*(l-1)+1.5*Cote*cm,nbv*Cote*cm*sqrt(3)/3);
        else:
          vdepla:=(1.5*Cote*cm*(l-2)+1.5*Cote*cm,nbv*Cote*cm*sqrt(3)/3)+(C-B);
        fi;
        if k=Lignes-1:
        if l mod 2=1:
        label(LATEX(decimal(n)),B shifted vdepla);
        else:
        n:=n-1;
        fi;
        else:
        label(LATEX(decimal(n)),B shifted vdepla);
        fi;
      endfor;
    endfor;
  fi;
  enddef;
  vardef EffectuerTraces=
  \useKV[Pavage]{Traces}
  enddef;
}%

\def\MPReseauCode{%
  boolean Before,Numerote,Backgrounds,After;
  Before=\useKV[Pavage]{Before};
  After=\useKV[Pavage]{After};
  Numerote=\useKV[Pavage]{Numerotation};
  Backgrounds=\useKV[Pavage]{Backgrounds};
  color ColPavage,ColArrierePlan;
  ColPavage=\useKV[Pavage]{Couleur};
  ColArrierePlan=if Backgrounds:\useKV[Pavage]{ArrierePlan} else:ColPavage fi;
  Cote=\useKV[Pavage]{Cote};
  Lignes=\useKV[Pavage]{Lignes};
  Colonnes=\useKV[Pavage]{Colonnes};
  Niveau=\useKV[Pavage]{Niveau};
  Depart=\useKV[Pavage]{Depart};
  Epaisseur=\useKV[Pavage]{Epaisseur};
  pair basei,basej;
  basei=\useKV[Pavage]{Basei};
  basej=\useKV[Pavage]{Basej};
  vardef PavageReseau=
  for k=0 upto Lignes-1:
  for l=0 upto Colonnes -1:
  trace Motif shifted (k*basei+l*basej);
  endfor;
  endfor;
  enddef;
  vardef Numerotation=
  n:=Depart;
  k:=0;
  pair A[],B[];
  A[0]=point(0) of Motif;
  B[0]=A[0];
  forever:
  k:=k+1;
  A[k]=point(k) of Motif;
  B[0]:=B[0]+A[k];
  exitif A[k]=A[0];
  endfor;
  B[0]:=B[0]/k;
  for k=0 upto Lignes-1:
  for l=0 upto Colonnes -1:
  n:=n+1;
  label(decimal(n),B[0] shifted(k*basei+l*basej));
  endfor;
  endfor;
  enddef;
  vardef PlacePointSupport=
  pair M[][];
  path Maille[];
  n:=Depart;
  k:=0;
  pair A[],B[];
  A[0]=point(0) of Motif;
  forever:
  k:=k+1;
  A[k]=point(k) of Motif;
  B[0]:=B[0]+A[k];
  exitif A[k]=A[0];
  endfor;
  B[0]:=B[0]/k;
  NbSommet:=k;
  for k=0 upto Lignes-1:
  for l=0 upto Colonnes -1:
  n:=n+1;
  for m=1 upto NbSommet:
  M[n][m]=A[m] shifted (k*basei+l*basej);
  endfor;
  Maille[n]=polygone(M[n][1] for r=2 upto NbSommet:,M[n][r] endfor);
  endfor;
  endfor;
  enddef;
  vardef EffectuerTraces=
  \useKV[Pavage]{Traces}
  enddef;
}%

\def\MPPeriodiqueCode{%
  boolean Before,Numerote,Backgrounds,After;
  Before=\useKV[Pavage]{Before};
  After=\useKV[Pavage]{After};
  Numerote=\useKV[Pavage]{Numerotation};
  Backgrounds=\useKV[Pavage]{Backgrounds};
  color ColPavage,ColArrierePlan;
  ColPavage=\useKV[Pavage]{Couleur};
  ColArrierePlan=if Backgrounds:\useKV[Pavage]{ArrierePlan} else:ColPavage fi;
  Cote=\useKV[Pavage]{Cote};
  Lignes=\useKV[Pavage]{Lignes};
  Colonnes=\useKV[Pavage]{Colonnes};
  Niveau=\useKV[Pavage]{Niveau};
  Depart=\useKV[Pavage]{Depart};
  Epaisseur=\useKV[Pavage]{Epaisseur};
  pair basei,basej;
  vardef PavagePeriodique=
  basei=iso(A1,A2)-iso(A1,A0);
  basej=iso(A0,A3)-iso(A1,A0);
  path CadrePavage;
  pair Cpa[];
  Cpa1=(0,ypart(-2*basej));
  Cpa2=(xpart(2*Colonnes*basei),ypart(-2*basej));
  Cpa3=(xpart(2*Colonnes*basei),ypart(2*Lignes*basej));
  Cpa4=(0,ypart(2*Lignes*basej));
  CadrePavage=Cpa1--Cpa2--Cpa3--Cpa4--cycle;
  picture BlocContour,BlocPavage;
  BlocContour=image(
  for k=-Lignes upto Lignes+1:
  for l=-1 upto Colonnes+1:
  if ColPavage<>white:
  fill Motif shifted (2*l*basei+2*k*basej) withcolor ColPavage;
  fill symetrie(Motif,iso(A1,A2)) shifted (2*l*basei+2*k*basej) withcolor ColPavage;
  fi;
  trace Motif shifted (2*l*basei+2*k*basej);
  trace symetrie(Motif,iso(A1,A2)) shifted (2*l*basei+2*k*basej);
  endfor;
  endfor;
  drawoptions();
  );
  BlocPavage=image(
  trace BlocContour;
  clip currentpicture to CadrePavage;
  trace CadrePavage withpen pencircle scaled Epaisseur;
  );
  BlocPavage
  enddef;
  vardef Numerotation=
  n:=-1;
  pair ISO;
  for k=-Lignes upto Lignes+1:
  for l=-1 upto Colonnes+1:
  n:=n+2;
  ISO:=B[0] shifted (2*l*basei+2*k*basej);
  if (xpart(ISO)>xpart(Cpa1+u*(0.2,0))) and (xpart(ISO)<xpart(Cpa2-u*(0.2,0))) and (ypart(ISO)>ypart(Cpa1+u*(0.2,0))) and (ypart(ISO)<ypart(Cpa4-u*(0.2,0))):
  label(TEX(decimal(n)),ISO);
  else:
  n:=n-2;
  fi;
  endfor;
  endfor;
  %2eme tour
  n:=-1;
  for k=-Lignes upto Lignes+1:
  for l=-1 upto Colonnes+1:
  n:=n+2;
  ISO:=symetrie(B[0],iso(A1,A2)) shifted (2*l*basei+2*k*basej);
  if (xpart(ISO)>xpart(Cpa1+u*(0.2,0))) and (xpart(ISO)<xpart(Cpa2-u*(0.2,0))) and (ypart(ISO)>ypart(Cpa1+u*(0,0.2))) and (ypart(ISO)<ypart(Cpa4-u*(0,0.2))):
  label(TEX(decimal(n+1)),ISO);
  else:
  n:=n-2;
  fi;
  endfor;
  endfor;
  enddef;
  vardef PlacePointSupport=
  path Maille[];
  pair QuadA[],QuadB[],QuadC[],QuadD[];
  n:=-1;
  pair ISO;
  for k=-Lignes upto Lignes+1:
  for l=-1 upto Colonnes+1:
  n:=n+2;
  ISO:=B[0] shifted (2*l*basei+2*k*basej);
  if (xpart(ISO)>xpart(Cpa1+u*(0.2,0))) and (xpart(ISO)<xpart(Cpa2-u*(0.2,0))) and (ypart(ISO)>ypart(Cpa1+u*(0,0.2))) and (ypart(ISO)<ypart(Cpa4-u*(0,0.2))):
  QuadA[n]=A0 shifted (2*l*basei+2*k*basej);
  QuadB[n]=A1 shifted (2*l*basei+2*k*basej);
  QuadC[n]=A2 shifted (2*l*basei+2*k*basej);
  QuadD[n]=A3 shifted (2*l*basei+2*k*basej);
  Maille[n]=polygone(QuadA[n],QuadB[n],QuadC[n],QuadD[n]);
  else:
  n:=n-2;
  fi;
  endfor;
  endfor;
  %2eme tour
  n:=-1;
  for k=-Lignes upto Lignes+1:
  for l=-1 upto Colonnes+1:
  n:=n+2;
  ISO:=symetrie(B[0],iso(A1,A2)) shifted (2*l*basei+2*k*basej);
  if (xpart(ISO)>xpart(Cpa1+u*(0.2,0))) and (xpart(ISO)<xpart(Cpa2-u*(0.2,0))) and (ypart(ISO)>ypart(Cpa1+u*(0,0.2))) and (ypart(ISO)<ypart(Cpa4-u*(0,0.2))):
  QuadA[n+1]=symetrie(A0,iso(A1,A2)) shifted (2*l*basei+2*k*basej);
  QuadB[n+1]=symetrie(A1,iso(A1,A2)) shifted (2*l*basei+2*k*basej);
  QuadC[n+1]=symetrie(A2,iso(A1,A2)) shifted (2*l*basei+2*k*basej);
  QuadD[n+1]=symetrie(A3,iso(A1,A2)) shifted (2*l*basei+2*k*basej);
  Maille[n+1]=polygone(QuadA[n+1],QuadB[n+1],QuadC[n+1],QuadD[n+1]);
  else:
  n:=n-2;
  fi;
  endfor;
  endfor;
  enddef;
  vardef EffectuerTraces=
  \useKV[Pavage]{Traces}
  enddef;
}%

\def\MPPavageEscherCode{%
  boolean Before,Numerote,Backgrounds,After,Colore;
  Before=\useKV[Pavage]{Before};
  After=\useKV[Pavage]{After};
  Numerote=\useKV[Pavage]{Numerotation};
  Backgrounds=\useKV[Pavage]{Backgrounds};
  Colore=\useKV[Pavage]{Colore};
  color ColPavage,ColArrierePlan,CouleurUn,CouleurDeux,CouleurTrois,CouleurQuatre;
  if Colore:
  CouleurUn=\useKV[Pavage]{CouleurUn};
  CouleurDeux=\useKV[Pavage]{CouleurDeux};
  CouleurTrois=\useKV[Pavage]{CouleurTrois};
  CouleurQuatre=\useKV[Pavage]{CouleurQuatre};
  fi;
  ColPavage=\useKV[Pavage]{Couleur};
  ColArrierePlan=if Backgrounds:\useKV[Pavage]{ArrierePlan} else:ColPavage fi;
  Cote=\useKV[Pavage]{Cote};
  Rayon=\useKV[Pavage]{Rayon};
  Lignes=\useKV[Pavage]{Lignes};
  Colonnes=\useKV[Pavage]{Colonnes};
  Niveau=\useKV[Pavage]{Niveau};
  Depart=\useKV[Pavage]{Depart};
  Ecart=\useKV[Pavage]{Ecart};
  Position=\useKV[Pavage]{Position};
  Epaisseur=\useKV[Pavage]{Epaisseur};
  pair basei,basej;
  picture Motif;
  vardef PavageEscher=
  path cc;
  cc=cercles((0,0),u*Rayon);
  if Niveau=3:
  pair O[];
  O1=pointarc(cc,0);
  O3=pointarc(cc,120);
  O5=pointarc(cc,240);
  O2=Position[O3,O1]+Ecart*unitvector((O3-O1) rotated -90);
  O4=rotation(O2,O3,-120);
  O6=rotation(O4,O5,-120);
  picture Motif;
  Motif=image(
  if Colore:
  if known CouleurUn: fill polygone(O1,O2,O3,O4,O5,O6) withcolor CouleurUn fi;
  if known CouleurDeux: fill rotation(polygone(O1,O2,O3,O4,O5,O6),O1,120) withcolor CouleurDeux fi;
  if known CouleurTrois:fill rotation(polygone(O1,O2,O3,O4,O5,O6),O1,-120) withcolor CouleurTrois fi;
  fi;
  trace polygone(O1,O2,O3,O4,O5,O6);
  trace rotation(polygone(O1,O2,O3,O4,O5,O6),O1,120);
  trace rotation(polygone(O1,O2,O3,O4,O5,O6),O1,-120);
  );
  basei=(rotation(O2,O1,-120)-O4);
  basej:=(rotation(O5,O1,-120)-O5);
  elseif Niveau=4:
  pair O[];
  O1=pointarc(cc,0);
  O3=pointarc(cc,90);
  O5=pointarc(cc,180);
  O7=pointarc(cc,270);
  O2=Position[O3,O1]+Ecart*unitvector((O3-O1) rotated -90);
  O4=rotation(O2,O3,-90);
  O6=rotation(O4,O5,-90);
  O8=rotation(O6,O7,-90);
  picture Motif;
  Motif=image(
  if Colore:
  if known CouleurUn: fill polygone(O1,O2,O3,O4,O5,O6,O7,O8) withcolor CouleurUn fi;
  if known CouleurDeux: fill rotation(polygone(O1,O2,O3,O4,O5,O6,O7,O8),O1,90) withcolor CouleurDeux fi;
  if known CouleurTrois:fill rotation(polygone(O1,O2,O3,O4,O5,O6,O7,O8),O1,180) withcolor CouleurTrois fi;
  if known CouleurQuatre:fill rotation(polygone(O1,O2,O3,O4,O5,O6,O7,O8),O1,270) withcolor CouleurQuatre fi;
  fi;
  trace polygone(O1,O2,O3,O4,O5,O6,O7,O8);
  trace rotation(polygone(O1,O2,O3,O4,O5,O6,O7,O8),O1,90);
  trace rotation(polygone(O1,O2,O3,O4,O5,O6,O7,O8),O1,180);
  trace rotation(polygone(O1,O2,O3,O4,O5,O6,O7,O8),O1,270);
  );
  basei=(rotation(O5,O1,180)-O5);
  basej:=(rotation(O5,O1,-90)-O5);
  elseif Niveau=6:
  pair O[];
  O1=pointarc(cc,0);
  O3=pointarc(cc,60);
  O5=pointarc(cc,120);
  O7=pointarc(cc,180);
  O9=pointarc(cc,240);
  O11=pointarc(cc,300);
  O2=Position[O3,O1]+Ecart*unitvector((O3-O1) rotated -90);
  O4=rotation(O2,O3,-120);
  O6=rotation(O4,O5,-120);
  O8=rotation(O6,O7,-120);
  O10=rotation(O8,O9,-120);
  O12=rotation(O10,O11,-120);
  picture Motif;
  Motif=image(
  if Colore:
  if known CouleurUn: fill polygone(O1,O2,O3,O4,O5,O6,O7,O8,O9,O10,O11,O12) withcolor CouleurUn fi;
  if known CouleurDeux: fill rotation(polygone(O1,O2,O3,O4,O5,O6,O7,O8,O9,O10,O11,O12),O1,120) withcolor CouleurDeux fi;
  if known CouleurTrois: fill rotation(polygone(O1,O2,O3,O4,O5,O6,O7,O8,O9,O10,O11,O12),O1,-120) withcolor CouleurTrois fi;
  fi;
  trace polygone(O1,O2,O3,O4,O5,O6,O7,O8,O9,O10,O11,O12);
  trace rotation(polygone(O1,O2,O3,O4,O5,O6,O7,O8,O9,O10,O11,O12),O1,120);
  trace rotation(polygone(O1,O2,O3,O4,O5,O6,O7,O8,O9,O10,O11,O12),O1,-120);
  );
  basei=(rotation(O4,O1,-120)-O6);
  basej:=(rotation(O9,O1,-120)-O9);
  fi;
  path CadrePavage;
  pair Cpa[];
  Cpa1=(0,ypart(-2*basej));
  Cpa2=(xpart(2*Colonnes*basei),ypart(-2*basej));
  Cpa3=(xpart(2*Colonnes*basei),ypart(2*Lignes*basej));
  Cpa4=(0,ypart(2*Lignes*basej));
  CadrePavage=Cpa1--Cpa2--Cpa3--Cpa4--cycle;
  picture BlocContour,BlocPavage;
  BlocContour=image(
  for k=-2*Lignes upto 2*Lignes:
  for l=-2*Colonnes upto 2*Colonnes:
  trace Motif shifted (l*basei+k*basej);
  endfor;
  endfor;
  );
  BlocPavage=image(
  trace BlocContour;
  clip currentpicture to CadrePavage;
  trace CadrePavage withpen pencircle scaled Epaisseur;
  );
  BlocPavage
  enddef;
  vardef BaryMotif(expr nimech)=
  save Escherbary;
  pair Escherbary,EB[],EBSom;
  tour:=0;
  EB[0]=point(0) of nimech;
  EBSom=(0,0);
  forever:
  tour:=tour+1;
  EB[tour]=point(tour) of nimech;
  EBSom:=EBSom+EB[tour];
  exitif EB[tour]=EB[0];
  endfor;
  Escherbary=EBSom/tour;
  Escherbary
  enddef;
  % 
  vardef Numerotation=
  pair ISORetiens;
  for k=1 upto TotalMaille:
  ISORetiens:=BaryMotif(Maille[k]);
  label(TEX(""&decimal(k)&""),ISORetiens);
  endfor;
  enddef;
  %
  vardef AssocierPoint(expr nbpt,ll,kk,aanngg)=
  PointE[nbpt][1]=rotation(O1,O1,aanngg) shifted (ll*basei+kk*basej);
  PointE[nbpt][2]=rotation(O2,O1,aanngg) shifted (ll*basei+kk*basej);
  PointE[nbpt][3]=rotation(O3,O1,aanngg) shifted (ll*basei+kk*basej);
  PointE[nbpt][4]=rotation(O4,O1,aanngg) shifted (ll*basei+kk*basej);
  PointE[nbpt][5]=rotation(O5,O1,aanngg) shifted (ll*basei+kk*basej);
  PointE[nbpt][6]=rotation(O6,O1,aanngg) shifted (ll*basei+kk*basej);
  if Niveau>3:
  PointE[nbpt][7]=rotation(O7,O1,aanngg) shifted (ll*basei+kk*basej);
  PointE[nbpt][8]=rotation(O8,O1,aanngg) shifted (ll*basei+kk*basej);
  fi;
  if Niveau>4:
  PointE[nbpt][9]=rotation(O9,O1,aanngg) shifted (ll*basei+kk*basej);
  PointE[nbpt][10]=rotation(O10,O1,aanngg) shifted (ll*basei+kk*basej);
  PointE[nbpt][11]=rotation(O11,O1,aanngg) shifted (ll*basei+kk*basej);
  PointE[nbpt][12]=rotation(O12,O1,aanngg) shifted (ll*basei+kk*basej);
  fi;
  enddef;
  vardef AssocierMaille(expr nbpt)=
  Maille[nbpt]=if Niveau=3:
  polygone(PointE[nbpt][1],PointE[nbpt][2],PointE[nbpt][3],PointE[nbpt][4],PointE[nbpt][5],PointE[nbpt][6])
  elseif Niveau=4:
  polygone(PointE[nbpt][1],PointE[nbpt][2],PointE[nbpt][3],PointE[nbpt][4],PointE[nbpt][5],PointE[nbpt][6],PointE[nbpt][7],PointE[nbpt][8])
  elseif Niveau=6:
  polygone(PointE[nbpt][1],PointE[nbpt][2],PointE[nbpt][3],PointE[nbpt][4],PointE[nbpt][5],PointE[nbpt][6],PointE[nbpt][7],PointE[nbpt][8],PointE[nbpt][9],PointE[nbpt][10],PointE[nbpt][11],PointE[nbpt][12])
  fi;
  enddef;
  vardef PlacePointSupport=
  pair PointE[][];
  path Maille[];
  pair ISO[];
  if Niveau=3:
  ISO1:=(0,0);
  ISO2=rotation(ISO1,O1,120);
  ISO3=rotation(ISO1,O1,-120);
  basei=(rotation(O2,O1,-120)-O4);
  basej:=(rotation(O5,O1,-120)-O5);
  elseif Niveau=4:
  ISO1=(0,0);
  ISO2=rotation(ISO1,O1,90);
  ISO3=rotation(ISO1,O1,180);
  ISO4=rotation(ISO1,O1,270);
  basei=(rotation(O5,O1,180)-O5);
  basej:=(rotation(O5,O1,-90)-O5);
  elseif Niveau=6:
  ISO1=(0,0);
  ISO2=rotation(ISO1,O1,120);
  ISO3=rotation(ISO1,O1,-120);
  basei=(rotation(O4,O1,-120)-O6);
  basej:=(rotation(O9,O1,-120)-O9);
  fi;
  n:=0;
  pair ISORetiens;
  for k=-2*Lignes upto 2*Lignes:
  for l=-2*Colonnes upto 2*Colonnes:
  n:=n+1;
  ISORetiens:=ISO1 shifted (l*basei+k*basej);
  if (xpart(ISORetiens)>xpart(Cpa1+u*(0.2,0))) and (xpart(ISORetiens)<xpart(Cpa2-u*(0.2,0))) and (ypart(ISORetiens)>ypart(Cpa1+u*(0,0.2))) and (ypart(ISORetiens)<ypart(Cpa4-u*(0,0.2))):
  AssocierPoint(n,l,k,0);
  AssocierMaille(n);
  else:
  n:=n-1;
  fi;
  n:=n+1;
  ISORetiens:=ISO2 shifted (l*basei+k*basej);
  if (xpart(ISORetiens)>xpart(Cpa1+u*(0.2,0))) and (xpart(ISORetiens)<xpart(Cpa2-u*(0.2,0))) and (ypart(ISORetiens)>ypart(Cpa1+u*(0,0.2))) and (ypart(ISORetiens)<ypart(Cpa4-u*(0,0.2))):
  if Niveau=4:
  AssocierPoint(n,l,k,90);
  else:
  AssocierPoint(n,l,k,120);
  fi;
  AssocierMaille(n);
  else:
  n:=n-1;
  fi;
  n:=n+1;
  ISORetiens:=ISO3 shifted (l*basei+k*basej);
  if (xpart(ISORetiens)>xpart(Cpa1+u*(0.2,0))) and (xpart(ISORetiens)<xpart(Cpa2-u*(0.2,0))) and (ypart(ISORetiens)>ypart(Cpa1+u*(0,0.2))) and (ypart(ISORetiens)<ypart(Cpa4-u*(0,0.2))):
  if Niveau=4:
  AssocierPoint(n,l,k,180);
  else:
  AssocierPoint(n,l,k,-120);
  fi;
  AssocierMaille(n);
  else:
  n:=n-1;
  fi;
  if Niveau=4:
  n:=n+1;
  ISORetiens:=ISO4 shifted (l*basei+k*basej);
  if (xpart(ISORetiens)>xpart(Cpa1+u*(0.2,0))) and (xpart(ISORetiens)<xpart(Cpa2-u*(0.2,0))) and (ypart(ISORetiens)>ypart(Cpa1+u*(0,0.2))) and (ypart(ISORetiens)<ypart(Cpa4-u*(0,0.2))):
  AssocierPoint(n,l,k,270);
  AssocierMaille(n);
  else:
  n:=n-1;
  fi;
  fi;
  endfor;
  endfor;
  TotalMaille=n;
  enddef;
  %
  vardef EffectuerTraces=
  \useKV[Pavage]{Traces}
  enddef;
}%

\def\MPPavageEscher{%
  \ifluatex%
  \mplibforcehmode%
  \begin{mplibcode}
    \MPPavageEscherCode
    picture Motif;
    trace PavageEscher;
    PlacePointSupport;
    if After:
    EffectuerTraces;
    fi;
    if Numerote:
    Numerotation;
    fi;
  \end{mplibcode}
  \else
  \begin{mpost}[mpsettings={\MPPavageEscherCode}]
    picture Motif;
    trace PavageEscher;
    PlacePointSupport;
    if After:
    EffectuerTraces;
    fi;
    if Numerote:
    Numerotation;
    fi;
  \end{mpost}
  \fi%
}%

\def\MPPeriodique{%
  \ifluatex
  \mplibforcehmode
  \begin{mplibcode}
    \MPPeriodiqueCode
    path Motif;
    if Before:
    Motif=\useKV[Pavage]{Motif};
    else:
    Motif=(0,0)--u*(1,-2)--u*(2,0.5)--u*(0.75,1)--cycle;
    fi;
    pair A[],B[];
    A[0]=point(0) of Motif;
    B[0]=A[0];
    k:=0;
    forever:
    k:=k+1;
    A[k]=point(k) of Motif;
    B[0]:=B[0]+A[k];
    exitif A[k]=A[0];
    endfor;
    B[0]:=B[0]/k;
    trace PavagePeriodique;
    PlacePointSupport;
    if After:
    EffectuerTraces;
    fi;
    if Numerote:
    Numerotation;
    fi;
  \end{mplibcode}
  \else
  \begin{mpost}[\MPPeriodiqueCode]
    path Motif;
    if Before:
    Motif=\useKV[Pavage]{Motif};
    else:
    Motif=(0,0)--u*(1,-2)--u*(2,0.5)--u*(0.75,1)--cycle;
    fi;
    pair A[],B[];
    A[0]=point(0) of Motif;
    B[0]=A[0];
    k:=0;
    forever:
    k:=k+1;
    A[k]=point(k) of Motif;
    B[0]:=B[0]+A[k];
    exitif A[k]=A[0];
    endfor;
    B[0]:=B[0]/k;
    trace PavagePeriodique;
    PlacePointSupport;
%    if Numerote:
%    Numerotation;
%    fi;
    if After:
    EffectuerTraces;
    fi;
  \end{mpost}
  \fi
}%

\def\MPReseau{%
  \ifluatex
  \mplibforcehmode
  \begin{mplibcode}
    \MPReseauCode
    path Motif;
    if Before:
    Motif=\useKV[Pavage]{Motif};
    else:
    Motif=u*(0,0)--u*(0.75,0)--u*(1,0.5)--u*(0.25,0.5)--cycle;
    fi;
    PavageReseau;
    PlacePointSupport;
    if After:
    EffectuerTraces;
    fi;
    if Numerote:
    Numerotation;
    fi;
  \end{mplibcode}
  \else
  \begin{mpost}[mpsettings={\MPReseauCode}]
    path Motif;
    if Before:
    Motif=\useKV[Pavage]{Motif};
    else:
    Motif=u*(0,0)--u*(0.75,0)--u*(1,0.5)--u*(0.25,0.5)--cycle;
    fi;
    PavageReseau;
    PlacePointSupport;
    if After:
    EffectuerTraces;
    fi;
    if Numerote:
    Numerotation;
    fi;
  \end{mpost}
  \fi
}%

\def\MPRegulier{%
  \ifluatex
  \mplibforcehmode
  \begin{mplibcode}
    \MPReseauRegulier
    trace PavageRegulier(Niveau);
    PlacePointSupport;
    if Numerote:
    Numerotation;
    fi;
    if After:
    EffectuerTraces;
    fi;
  \end{mplibcode}
  \else
  \begin{mpost}[mpsettings={\MPReseauRegulier}]
    trace PavageRegulier(Niveau);
    PlacePointSupport;
    if Numerote:
    Numerotation;
    fi;
    if After:
    EffectuerTraces;
    fi;
  \end{mpost}
  \fi
}%

\def\MPPavage{%
  \ifluatex%
  \mplibforcehmode%
  \begin{mplibcode}%
    \MPPavageBase
    path Motif;
    if Before:
    Motif=\useKV[Pavage]{Motif};
    else:
    Motif=u*(0,0)..u*(0.25,.1)..u*(0.5,0.7)..u*(1,1);
    fi;
    if Demo:
    trace pavagescar(Motif,0,ColPavage);
    else:
    trace pavages(Motif,Niveau,ColPavage);
    fi;
    PlacePointSupport;
    if Numerote:
    Numerotation;
    fi;
    if After:
    EffectuerTraces;
    fi;
    if Demo=false:
    trace CadrePavage withpen pencircle scaled Epaisseur;
    fi;
  \end{mplibcode}
  \else
  \begin{mpost}[mpsettings={\MPPavageBase}]
    path Motif;
    if Before:
    Motif=\useKV[Pavage]{Motif};
    else:
    Motif=u*(0,0)..u*(0.25,.1)..u*(0.5,0.7)..u*(1,1);
    fi;
    trace pavages(Motif,Niveau,ColPavage);
    PlacePointSupport;
    if Numerote:
    Numerotation;
    fi;
    if After:
    EffectuerTraces;
    fi;
    trace CadrePavage withpen pencircle scaled Epaisseur;
  \end{mpost}
  \fi%
}%

\def\MPEinsteinHatCode{
  boolean After;
  After:=\useKV[Pavage]{After};
  %
  def TraceFill(expr o,coul)=
  fill o withcolor coul;
  trace o;% withpen pencircle scaled 2;
  enddef;
  %
  def TraceHache(expr o,coul,ang)=
  trace hachurage(o,ang,0.2,0) withcolor coul;
  trace o;% withpen pencircle scaled 2;
  enddef;
  % 
  vardef EffectuerTraces=
  \useKV[Pavage]{Traces}
  enddef;
  %
  path Hat;
  pair O[],A[],B[],C[],D[],E[],F[];
  Rayon=\useKV[Pavage]{Rayon}*u;
  O1=u*(3,3);
  O3=pointarc(cercles(O1,Rayon),300);
  O2=rotation(O3,O1,-60);
  O4=CentreCercleC(O1,O2,O3);
  O5=symetrie(O1,iso(O2,O3));
  Rayonbis=abs(O1-CentreCercleC(O1,O2,O3));
  draw polygone(O1,O2,O3) withcolor 0.7white;
  path cc[];
  cc1=cercles(O1,Rayonbis);
  cc2=cercles(O2,Rayonbis);
  cc3=cercles(O3,Rayonbis);
  drawoptions(dashed evenly withcolor 0.7white);
  trace cc1;trace cc2;trace cc3;
  drawoptions();
  for k=1 upto 3:
  A[k]=pointarc(cc[k],30+0);
  B[k]=pointarc(cc[k],30+60);
  C[k]=pointarc(cc[k],30+2*60);
  D[k]=pointarc(cc[k],30+3*60);
  E[k]=pointarc(cc[k],30+4*60);
  F[k]=pointarc(cc[k],30+5*60);
  endfor;
  drawoptions(dashed evenly);
  trace polygone(A1,B1,C1,D1,E1,F1);
  trace polygone(A2,B2,C2,D2,E2,F2);
  trace polygone(A3,B3,C3,D3,E3,F3);
  drawoptions();
  Hat=polygone(O1,iso(C1,D1),D1,iso(D1,E1),O2,iso(E2,F2),F2,iso(D3,E3),O3,iso(F3,A3),A3,B3,iso(B3,C3));
  color Coul[];
  Coul1=Cyan;%;
  Coul2=0.5[Blue,white];
  Coul3=Cornsilk;
  Coul4=0.5[Crimson,white];
  Coul5=0.5[Purple,white];
  Coul6=LightGreen;
  %
  vardef TileGen(expr aHat,bHat,ech)=
  pair H[];
  H0=iso(F2,E2);
  H1-H0=Rayonbis*(aHat/ech)*unitvector(E2-F2);
  H2-H1=Rayonbis*(aHat/ech)*unitvector(D2-E2);
  H3-H2=Rayonbis*(bHat/ech)*unitvector(O2-iso(E2,D2));
  H4-H3=Rayonbis*(bHat/ech)*unitvector(iso(B2,C2)-O2);
  H5-H4=Rayonbis*(aHat/ech)*unitvector(B2-C2);
  H6-H5=Rayonbis*(aHat/ech)*unitvector(C1-D1);
  H7-H6=Rayonbis*(bHat/ech)*unitvector(O1-iso(C1,D1));
  H8-H7=Rayonbis*(bHat/ech)*unitvector(iso(E1,F1)-O1);
  H9-H8=Rayonbis*(aHat/ech)*unitvector(F2-E2);
  H10-H9=Rayonbis*(aHat/ech)*unitvector(A3-B3);
  H11-H10=Rayonbis*(bHat/ech)*unitvector(O3-iso(A3,B3));
  H12-H11=Rayonbis*(bHat/ech)*unitvector(iso(C3,D3)-O3);
  H13-H12=Rayonbis*(aHat/ech)*unitvector(F2-A2);
  H14=iso(H0,H1,H2,H3,H4,H5,H6,H7,H8,H9,H10,H11,H12,H13);
  save $;
  path $;
  $=polygone(H0,H1,H2,H3,H4,H5,H6,H7,H8,H9,H10,H11,H12,H13);
  $
  enddef;
%
  vardef VampireGen(expr ech)=
  pair Vp[];
  Vp0=iso(F2,E2);
  Vp1-Vp0=Rayonbis*(1/ech)*unitvector(E2-F2);
  Vp2-Vp1=Rayonbis*(1/ech)*unitvector(D2-E2);
  Vp3-Vp2=Rayonbis*(1/ech)*unitvector(O2-iso(E2,D2));
  Vp4-Vp3=Rayonbis*(1/ech)*unitvector(iso(B2,C2)-O2);
  Vp5-Vp4=Rayonbis*(1/ech)*unitvector(B2-C2);
  Vp6-Vp5=Rayonbis*(1/ech)*unitvector(C1-D1);
  Vp7-Vp6=Rayonbis*(1/ech)*unitvector(O1-iso(C1,D1));
  Vp8-Vp7=Rayonbis*(1/ech)*unitvector(iso(E1,F1)-O1);
  Vp9-Vp8=Rayonbis*(1/ech)*unitvector(F2-E2);
  Vp10-Vp9=Rayonbis*(1/ech)*unitvector(A3-B3);
  Vp11-Vp10=Rayonbis*(1/ech)*unitvector(O3-iso(A3,B3));
  Vp12-Vp11=Rayonbis*(1/ech)*unitvector(iso(C3,D3)-O3);
  Vp13-Vp12=Rayonbis*(1/ech)*unitvector(F2-A2);
  Vp14=Vp0;
  Vp15=iso(Vp0,Vp1,Vp2,Vp3,Vp4,Vp5,Vp6,Vp7,Vp8,Vp9,Vp10,Vp11,Vp12,Vp13);
  save $;
  path $;
  $=polygone(Vp0,Vp1,Vp2,Vp3,Vp4,Vp5,Vp6,Vp7,Vp8,Vp9,Vp10,Vp11,Vp12,Vp13);
  $
  enddef;
}

\def\MPEinstein{%Pentoma : https://www.pentoma.de/the_hat/
  \ifluatex
    \mplibforcehmode
    \begin{mplibcode}
      \MPEinsteinHatCode
      if After:
      EffectuerTraces;
      fi;
    \end{mplibcode}
  \else
    \begin{mpost}[mpsettings={\MPEinsteinHatCode}
      if After:
      EffectuerTraces;
      fi;
    \end{mpost}
  \fi
}%

\NewDocumentCommand\Pavage{o}{%
  \useKVdefault[Pavage]%
  \setKV[Pavage]{#1}%
  \ifboolKV[Pavage]{Einstein}{%
  \ifboolKV[Pavage]{Vampire}{}{%
    \MPEinstein%
    }%
  }{%
  \ifboolKV[Pavage]{Regulier}{%
    \MPRegulier%
  }{%
    \ifboolKV[Pavage]{SemiRegulier}{%
      % 
    }{%
      \ifboolKV[Pavage]{Reseau}{%
        \MPReseau%
      }{%
        \ifboolKV[Pavage]{Quadrilatere}{%
          \MPPeriodique%
        }{%
          \ifboolKV[Pavage]{Escher}{%
            \MPPavageEscher%
          }{%
            \MPPavage%
          }%
        }%
      }%
    }%
  }%
  }%
}%