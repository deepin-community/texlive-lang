%%%
% Le th\'eor\`eme de Pythagore
%%%
\def\filedatePythagore{2024/08/04}%
\def\fileversionPythagore{0.1}%
\message{-- \filedatePythagore\space v\fileversionPythagore}%
%
\setKVdefault[ClesPythagore]{Exact=false,AvantRacine=false,Racine=false,Entier=false,Egalite=false,Precision=2,Soustraction=false,Figure=false,FigureSeule=false,Angle=0,Echelle=1cm,Reciproque=false,ReciColonnes=false,Faible=false,Unite=cm,EnchaineA=false,EnchaineB=false,EnchaineC=false,Perso=false,AllPerso=false,SansMots=false,Decalage=5mm,ModeleCouleur=5}%
\defKV[ClesPythagore]{ValeurA=\setKV[ClesPythagore]{EnchaineA}}%
\defKV[ClesPythagore]{ValeurB=\setKV[ClesPythagore]{EnchaineB}}%
\defKV[ClesPythagore]{ValeurC=\setKV[ClesPythagore]{EnchaineC}}%

\setsepchar[*]{,*/}%
\readlist*\ListePythagoreCoteEntier{13/84/85,14/48/50,15/20/25,15/36/39,16/30/34,16/63/65,18/24/30,18/80/82,20/21/29,20/48/52,21/28/35,21/72/75,24/32/40,24/45/51,24/70/74,25/60/65,27/36/45,28/45/53,28/96/100,30/40/50,30/72/78,32/60/68,33/44/55,33/56/65,35/84/91,36/48/60,36/77/85,39/52/65,39/80/89,40/42/58,40/75/85,42/56/70,45/60/75,48/55/73,48/64/80,51/68/85,54/72/90,57/76/95,60/63/87,60/80/100,65/72/97}%%d'après villement.gerard.free.fr%3/4/5}/{5/12/13}/{6/8/10}/{7/24/25}/{8/15/17}/{9/12/15}/{9/40/41}/{10/24/26}/{11/60/61}/{12/16/20}/{12/35/37}/
\setsepchar{,}%

% On d\'efinit les figures \`a utiliser
\def\MPFigurePytha#1#2#3#4#5#6{%
  % #1 Premier sommet
  % #2 Sommet de l'angle droit
  % #3 troisi\`eme sommet
  % #4 1ere longueur
  % #5 2eme longueur
  % #6 angle de rotation de la figure
  \ifluatex
    \mplibforcehmode%
    \begin{mplibcode}
      defaultcolormodel := \useKV[ClesPythagore]{ModeleCouleur};
      
    u:=\useKV[ClesPythagore]{Echelle};
    pair A,B,C,O,D,E,F;%B est le sommet de l'angle droit
    O=u*(2.5,2.5);
    path cc;
    cc=(fullcircle scaled 4u) shifted O;
    % On place les points A,B,C sur le cercle de mani\`ere \`a faciliter la rotation de la figure
    A=point(0.9*length cc) of cc;
    B=A rotatedabout(O,-120);
    C=2[A,O];
    % On tourne pour \'eventuellement moins de lassitude :)
    A:=A rotatedabout(O,#6);
    B:=B rotatedabout(O,#6);
    C:=C rotatedabout(O,#6);
    % On d\'efinit l'angle droit
    D-B=7*unitvector(C-B);
    F-B=7*unitvector(A-B);
    E-D=F-B;
    draw A{dir(angle(B-A)+5)}..B{dir(angle(B-A)+5)};
    draw B{dir(angle(C-B)+5)}..C{dir(angle(C-B)+5)};
    draw C{dir(angle(A-C)+5)}..A{dir(angle(A-C)+5)};
    draw D--E--F;
    numeric decalage;
    decalage=3mm;
    if (#4<#5) or (#4=#5) :
    if ypart(B)>ypart(O) :
        label(btex \num{#4} etex,1/2[C,B]-decalage*(unitvector(A-B)));
        label(btex \num{#5} etex,1/2[A,B]-decalage*(unitvector(C-B)));
     else:
        label(btex \num{#4} etex,1/2[C,B]-decalage*(unitvector(A-B)));
        label(btex \num{#5} etex,1/2[A,B]-decalage*(unitvector(C-B)));
     fi
    else:
      if ypart(B)>ypart(O) :
        label(btex \num{#4} etex,1/2[C,A]-decalage*(unitvector(C-A) rotated 90));
        label(btex \num{#5} etex,1/2[A,B]-decalage*(unitvector(C-B)));
        else:
        label(btex \num{#4} etex,1/2[A,C]+decalage*(unitvector(A-C)
        rotated 90));
        label(btex \num{#5} etex,1/2[A,B]-decalage*(unitvector(C-B)));
      fi;
    fi;
    label(btex #3 etex,1.2[O,A]);
    label(btex #2 etex,1.2[O,B]);
    label(btex #1 etex,1.2[O,C]);
  \end{mplibcode}
  \else
  \begin{mpost}[mpsettings={u:=\useKV[ClesPythagore]{Echelle};}]
    pair A,B,C,O,D,E,F;%B est le sommet de l'angle droit
    O=u*(2.5,2.5);
    path cc;
    cc=(fullcircle scaled 4u) shifted O;
    % On place les points A,B,C sur le cercle de mani\`ere \`a faciliter la rotation de la figure
    A=point(0.9*length cc) of cc;
    B=A rotatedabout(O,-120);
    C=2[A,O];
    % On tourne pour \'eventuellement moins de lassitude :)
    A:=A rotatedabout(O,#6);
    B:=B rotatedabout(O,#6);
    C:=C rotatedabout(O,#6);
    % On d\'efinit l'angle droit
    D-B=7*unitvector(C-B);
    F-B=7*unitvector(A-B);
    E-D=F-B;
    draw A{dir(angle(B-A)+5)}..B{dir(angle(B-A)+5)};
    draw B{dir(angle(C-B)+5)}..C{dir(angle(C-B)+5)};
    draw C{dir(angle(A-C)+5)}..A{dir(angle(A-C)+5)};    
    draw D--E--F;
    numeric decalage;
    decalage=3mm;
    if (#4<#5) or (#4=#5) :
    if ypart(B)>ypart(O) :
        label(btex \num{#4} etex,1/2[C,B]-decalage*(unitvector(A-B)));
        label(btex \num{#5} etex,1/2[A,B]-decalage*(unitvector(C-B)));
     else:
        label(btex \num{#4} etex,1/2[C,B]-decalage*(unitvector(A-B)));
        label(btex \num{#5} etex,1/2[A,B]-decalage*(unitvector(C-B)));
     fi
    else:
      if ypart(B)>ypart(O) :
      label(btex \num{#4} etex,1/2[C,A]-decalage*(unitvector(C-A) rotated 90));
        label(btex \num{#5} etex,1/2[A,B]-decalage*(unitvector(C-B)));
        else:
        label(btex \num{#4} etex,1/2[A,C]+decalage*(unitvector(A-C)
        rotated 90));
        label(btex \num{#5} etex,1/2[A,B]-decalage*(unitvector(C-B)));
      fi;
    fi;
    label(btex #3 etex,1.2[O,A]);
    label(btex #2 etex,1.2[O,B]);
    label(btex #1 etex,1.2[O,C]);
  \end{mpost}
  \fi
}%

\def\MPFigureReciPytha#1#2#3#4#5#6#7{%
  % #1 Premier sommet
    % #2 Sommet de l'angle droit
    % #3 troisi\`eme sommet
    % #4 1ere longueur
    % #5 2eme longueur
    % #6 3eme longueur
    % #7 angle de rotation de la figure
  \ifluatex
   \mplibforcehmode
   \begin{mplibcode}
     defaultcolormodel := \useKV[ClesPythagore]{ModeleCouleur};
    u:=\useKV[ClesPythagore]{Echelle};
    decalage=\useKV[ClesPythagore]{Decalage};
    pair A,B,C,O,D,E,F;%B est le sommet de l'angle droit
    O=u*(2.5,2.5);
    path cc;
    cc=(fullcircle scaled 4u) shifted O;
    % On place les points A,B,C sur le cercle de mani\`ere \`a faciliter la rotation de la figure
    A=point(0.8*length cc) of cc;
    B=A rotatedabout(O,-100);
    C=2[A,O];
    % On tourne pour \'eventuellement moins de lassitude :)
    A:=A rotatedabout(O,#7);
    B:=B rotatedabout(O,#7);
    C:=C rotatedabout(O,#7);
    draw A{dir(angle(B-A)+5)}..B{dir(angle(B-A)+5)};
    draw B{dir(angle(C-B)+5)}..C{dir(angle(C-B)+5)};
    draw C{dir(angle(A-C)+5)}..A{dir(angle(A-C)+5)};    
    label(TEX("\num{#4}"),O+decalage*(unitvector(O-B)));
    label(TEX("\num{#6}"),1/2[C,B]+decalage*(unitvector(1/2[C,B]-O)));
    label(TEX("\num{#5}"),1/2[A,B]+decalage*(unitvector(1/2[A,B]-O)));
    label(TEX("#1"),1.2[O,A]);
    label(TEX("#2"),1.2[O,B]);
    label(TEX("#3"),1.2[O,C]);
  \end{mplibcode}
  \else
  \begin{mpost}[mpsettings={u:=\useKV[ClesPythagore]{Echelle};decalage=\useKV[ClesPythagore]{Decalage};}]
    pair A,B,C,O,D,E,F;%B est le sommet de l'angle droit
    O=u*(2.5,2.5);
    path cc;
    cc=(fullcircle scaled 4u) shifted O;
    % On place les points A,B,C sur le cercle de mani\`ere \`a faciliter la rotation de la figure
    A=point(0.8*length cc) of cc;
    B=A rotatedabout(O,-100);
    C=2[A,O];
    % On tourne pour \'eventuellement moins de lassitude :)
    A:=A rotatedabout(O,#7);
    B:=B rotatedabout(O,#7);
    C:=C rotatedabout(O,#7);
    draw A{dir(angle(B-A)+5)}..B{dir(angle(B-A)+5)};
    draw B{dir(angle(C-B)+5)}..C{dir(angle(C-B)+5)};
    draw C{dir(angle(A-C)+5)}..A{dir(angle(A-C)+5)};
    numeric decalage;
    decalage=3mm;
    label(LATEX("\num{#4}"),O+decalage*(unitvector(O-B)));
    label(LATEX("\num{#5}"),1/2[C,B]+decalage*(unitvector(1/2[C,B]-O)));
    label(LATEX("\num{#6}"),1/2[A,B]+decalage*(unitvector(1/2[A,B]-O)));
    label(btex #1 etex,1.2[O,A]);
    label(btex #2 etex,1.2[O,B]);
    label(btex #3 etex,1.2[O,C]);
  \end{mpost}
  \fi
}%

\def\MPFigurePythaSansMots#1#2#3#4#5#6{%
  % #1 Premier sommet
  % #2 Sommet de l'angle droit
  % #3 troisi\`eme sommet
  % #4 1ere longueur
  % #5 2eme longueur
  % #6 angle de rotation de la figure
  \ifluatex
   \mplibforcehmode
   \begin{mplibcode}
     defaultcolormodel := \useKV[ClesPythagore]{ModeleCouleur};
    u:=\useKV[ClesPythagore]{Echelle};
    pair A,B,C,O,D,E,F,L[],M[],N[];%B est le sommet de l'angle droit
    O=u*(2.5,2.5);
    path cc;
    cc=(fullcircle scaled 4u) shifted O;
    % On place les points A,B,C sur le cercle de mani\`ere \`a faciliter la rotation de la figure
    A=point(0.9*length cc) of cc;
    B=A rotatedabout(O,-120);
    C=2[A,O];
    % On tourne pour \'eventuellement moins de lassitude :)
    A:=A rotatedabout(O,#6);
    B:=B rotatedabout(O,#6);
    C:=C rotatedabout(O,#6);
    % On d\'efinit l'angle droit
    D-B=7*unitvector(C-B);
    F-B=7*unitvector(A-B);
    E-D=F-B;
    trace polygone(A,B,C);
    draw D--E--F;
    L1=rotation(B,A,90);
    L2-L1=B-A;
    trace chemin(A,L1,L2,B);
    M1=rotation(C,B,90);
    M2-M1=C-B;
    trace chemin(B,M1,M2,C);
    N1=rotation(A,C,90);
    N2-N1=A-C;
    trace chemin(C,N1,N2,A);    
    numeric decalage;
    decalage=3mm;
    if (#4<#5) or (#4=#5) :
    if ypart(B)>ypart(O) :
        label(btex \num{#4} etex,1/2[C,B]+decalage*(unitvector(A-B)));
        label(btex \num{#5} etex,1/2[A,B]+decalage*(unitvector(C-B)));
        label(TEX("\begin{tabular}{c}Aire\\$\num{#5}\times\num{#5}$\\$\num{\fpeval{#5*#5}}$\\\end{tabular}"),iso(A,L2));
        label(TEX("\begin{tabular}{c}Aire\\$\num{#4}\times\num{#4}$\\$\num{\fpeval{#4*#4}}$\\\end{tabular}"),iso(B,M2));
        label(TEX("\begin{tabular}{c}Aire\\$\num{\fpeval{#4*#4}}+\num{\fpeval{#5*#5}}$\\$\num{\fpeval{#4*#4+#5*#5}}$\\\end{tabular}"),iso(C,N2));
     else:
        label(btex \num{#4} etex,1/2[C,B]+decalage*(unitvector(A-B)));
        label(btex \num{#5} etex,1/2[A,B]+decalage*(unitvector(C-B)));
        label(TEX("\begin{tabular}{c}Aire\\$\num{#5}\times\num{#5}$\\$\num{\fpeval{#5*#5}}$\\\end{tabular}"),iso(A,L2));
        label(TEX("\begin{tabular}{c}Aire\\$\num{#4}\times\num{#4}$\\$\num{\fpeval{#4*#4}}$\\\end{tabular}"),iso(B,M2));
        label(TEX("\begin{tabular}{c}Aire\\$\num{\fpeval{#4*#4}}+\num{\fpeval{#5*#5}}$\\$\num{\fpeval{#4*#4+#5*#5}}$\\\end{tabular}"),iso(C,N2));
        drawarrow 2/10[iso(A,L2),iso(C,N2)]--8/10[iso(A,L2),iso(C,N2)];
        drawarrow 2/10[iso(B,M2),iso(C,N2)]--8/10[iso(B,M2),iso(C,N2)];
     fi
    else:
      if ypart(B)>ypart(O) :
        label(btex \num{#4} etex,1/2[C,A]-decalage*(unitvector(C-A) rotated 90));
        label(btex \num{#5} etex,1/2[A,B]-decalage*(unitvector(C-B)));
        label(TEX("\begin{tabular}{c}Aire\\$\num{#5}\times\num{#5}$\\$\num{\fpeval{#5*#5}}$\\\end{tabular}"),iso(A,L2));
        label(TEX("\begin{tabular}{c}Aire\\$\num{#4}\times\num{#4}$\\$\num{\fpeval{#4*#4}}$\\\end{tabular}"),iso(C,N2));
        label(TEX("\begin{tabular}{c}Aire\\$\num{\fpeval{#4*#4}}-\num{\fpeval{#5*#5}}$\\$\num{\fpeval{#4*#4-#5*#5}}$\\\end{tabular}"),iso(B,M2));
        drawarrow 2/10[iso(A,L2),iso(B,M2)]--8/10[iso(A,L2),iso(B,M2)];
        drawarrow 2/10[iso(C,N2),iso(B,M2)]--8/10[iso(C,N2),iso(B,M2)];
        else:
        label(btex \num{#4} etex,1/2[A,C]+decalage*(unitvector(A-C) rotated 90));
        label(btex \num{#5} etex,1/2[A,B]-decalage*(unitvector(C-B)));
        label(TEX("\begin{tabular}{c}Aire\\$\num{#5}\times\num{#5}$\\$\num{\fpeval{#5*#5}}$\\\end{tabular}"),iso(A,L2));
        label(TEX("\begin{tabular}{c}Aire\\$\num{#4}\times\num{#4}$\\$\num{\fpeval{#4*#4}}$\\\end{tabular}"),iso(C,N2));
        label(TEX("\begin{tabular}{c}Aire\\$\num{\fpeval{#4*#4}}-\num{\fpeval{#5*#5}}$\\$\num{\fpeval{#4*#4-#5*#5}}$\\\end{tabular}"),iso(B,M2));
        drawarrow 2/10[iso(A,L2),iso(B,M2)]--8/10[iso(A,L2),iso(B,M2)];
        drawarrow 2/10[iso(C,N2),iso(B,M2)]--8/10[iso(C,N2),iso(B,M2)];
      fi;
    fi;
    label(btex #3 etex,1.2[O,A]);
    label(btex #2 etex,1.2[O,B]);
    label(btex #1 etex,1.2[O,C]);
  \end{mplibcode}
  \fi
}%

\newcommand\RedactionPythagore{}%
\newcommand\RedactionReciPythagore{}%
\newcommand\RedactionCalculsPythagore{}%
\newcommand\RedactionCalculsReciPythagore{}%
\newcommand\RedactionConclusionReciPythagore{}%

\NewDocumentCommand\CalculsPythagore{o m m m m}{%
  \xintifboolexpr{#3<#4 || #3==#4}{%\ifnum#3<#4%
    \xdef\ResultatPytha{\fpeval{round(sqrt(#3^2+#4^2),\useKV[ClesPythagore]{Precision})}}%
    \ifboolKV[ClesPythagore]{EnchaineA}{\xdef\PfCPythaVA{\useKV[ClesPythagore]{ValeurA}}}{\xdef\PfCPythaVA{\fpeval{#3^2}}}%
    \ifboolKV[ClesPythagore]{EnchaineB}{\xdef\PfCPythaVB{\useKV[ClesPythagore]{ValeurB}}}{\xdef\PfCPythaVB{\fpeval{#4^2}}}%
    \xdef\ResultatAttendu{\fpeval{\PfCPythaVA+\PfCPythaVB}}%
    \xdef\ResultatApproche{\fpeval{round(sqrt(\ResultatAttendu),\useKV[ClesPythagore]{Precision})^2}}%
    \begin{align*}
      \NomA\NomC^2&=\NomA\NomB^2+\NomB\NomC^2\\
      \NomA\NomC^2&=\ifboolKV[ClesPythagore]{EnchaineA}{\opcopy{\useKV[ClesPythagore]{ValeurA}}{a1}\opexport{a1}{\Aun}\num{\Aun}}{\opexport{A1}{\Aun}\num{\Aun}^2}+\ifboolKV[ClesPythagore]{EnchaineB}{\opcopy{\useKV[ClesPythagore]{ValeurB}}{a2}\opexport{a2}{\Adeux}\num{\Adeux}}{\opexport{A2}{\Adeux}\num{\Adeux}^2}\\
      \NomA\NomC^2&=\ifboolKV[ClesPythagore]{EnchaineA}{\opexport{a1}{\Aun}\num{\Aun}}{\opmul*{A1}{A1}{a1}\opexport{a1}{\Aun}\num{\Aun}}+\ifboolKV[ClesPythagore]{EnchaineB}{\opexport{a2}{\Adeux}\num{\Adeux}}{\opmul*{A2}{A2}{a2}\opexport{a2}{\Adeux}\num{\Adeux}}\\
      \NomA\NomC^2&=\opadd*{a1}{a2}{a3}\opexport{a3}{\Atrois}\num{\Atrois}%\\
      \ifboolKV[ClesPythagore]{AvantRacine}{}{%
         \ifboolKV[ClesPythagore]{Entier}{}{\\\NomA\NomC&=\sqrt{\opexport{a3}{\Atrois}\num{\Atrois}}\ifboolKV[ClesPythagore]{Racine}{~\si{\PfCPythaUnit}}{}}
         \ifboolKV[ClesPythagore]{Racine}{}{\\\NomA\NomC&\IfStrEq{\ResultatApproche}{\ResultatAttendu}{=}{\approx}\opsqrt[maxdivstep=5]{a3}{a4}\opround{a4}{pres}{a4}\opunzero{a4}\opexport{a4}{\Aquatre}\SI{\Aquatre}{\PfCPythaUnit}}%\\
      }
    \end{align*}
  }{%
    \xdef\ResultatPytha{\fpeval{round(sqrt(#3^2-#4^2),\useKV[ClesPythagore]{Precision})}}%
    \ifboolKV[ClesPythagore]{EnchaineC}{\xdef\PfCPythaVC{\useKV[ClesPythagore]{ValeurC}}}{\xdef\PfCPythaVC{\fpeval{#3^2}}}%
    \ifboolKV[ClesPythagore]{EnchaineB}{\xdef\PfCPythaVB{\useKV[ClesPythagore]{ValeurB}}}{\xdef\PfCPythaVB{\fpeval{#4^2}}}%
    \xdef\ResultatAttendu{\fpeval{\PfCPythaVC-\PfCPythaVB}}%
    \xdef\ResultatApproche{\fpeval{round(sqrt(\ResultatAttendu),\useKV[ClesPythagore]{Precision})^2}}%
    \ifboolKV[ClesPythagore]{Soustraction}{%
      \begin{align*}
        \NomA\NomB^2&=\NomA\NomC^2-\NomB\NomC^2\\
        \NomA\NomB^2&=\ifboolKV[ClesPythagore]{EnchaineC}{\opcopy{\useKV[ClesPythagore]{ValeurC}}{a1}\opexport{a1}{\Aun}\num{\Aun}}{\opexport{A1}{\Aun}\num{\Aun}^2}-\ifboolKV[ClesPythagore]{EnchaineB}{\opcopy{\useKV[ClesPythagore]{ValeurB}}{a2}\opexport{a2}{\Adeux}\num{\Adeux}}{\opexport{A2}{\Adeux}\num{\Adeux}^2}\\
        \NomA\NomB^2&=\ifboolKV[ClesPythagore]{EnchaineC}{\opcopy{\useKV[ClesPythagore]{ValeurC}}{a1}\opexport{a1}{\Aun}\num{\Aun}}{\opmul*{A1}{A1}{a1}\opexport{a1}{\Aun}\num{\Aun}}-\ifboolKV[ClesPythagore]{EnchaineB}{\opexport{a2}{\Adeux}\num{\Adeux}}{\opmul*{A2}{A2}{a2}\opexport{a2}{\Adeux}\num{\Adeux}}\\
        \NomA\NomB^2&=\opsub*{a1}{a2}{a3}\opexport{a3}{\Atrois}\num{\Atrois}%\\
        \ifboolKV[ClesPythagore]{AvantRacine}{}{%
        \ifboolKV[ClesPythagore]{Entier}{}{\\\NomA\NomB&=\sqrt{\opexport{a3}{\Atrois}\num{\Atrois}}}
        \ifboolKV[ClesPythagore]{Racine}{}{\\\NomA\NomB&\IfStrEq{\ResultatApproche}{\ResultatAttendu}{=}{\approx}\opsqrt[maxdivstep=5]{a3}{a4}\opround{a4}{pres}{a4}\opunzero{a4}\opexport{a4}{\Aquatre}\SI{\Aquatre}{\PfCPythaUnit}}%\\
        }
      \end{align*}
    }{%
      \begin{align*}
        \NomA\NomC^2&=\NomA\NomB^2+\NomB\NomC^2\\
        \ifboolKV[ClesPythagore]{EnchaineC}{\opcopy{\useKV[ClesPythagore]{ValeurC}}{a1}\opexport{a1}{\Aun}\num{\Aun}}{\opexport{A1}{\Aun}\num{\Aun}^2}&=\NomA\NomB^2+\ifboolKV[ClesPythagore]{EnchaineB}{\opcopy{\useKV[ClesPythagore]{ValeurB}}{a2}\opexport{a2}{\Adeux}\num{\Adeux}}{\opexport{A2}{\Adeux}\num{\Adeux}^2}\\
        \ifboolKV[ClesPythagore]{EnchaineC}{\opcopy{\useKV[ClesPythagore]{ValeurC}}{a1}\opexport{a1}{\Aun}\num{\Aun}}{\opmul*{A1}{A1}{a1}\opexport{a1}{\Aun}\num{\Aun}}&=\NomA\NomB^2+\ifboolKV[ClesPythagore]{EnchaineB}{\opexport{a2}{\Adeux}\num{\Adeux}}{\opmul*{A2}{A2}{a2}\opexport{a2}{\Adeux}\num{\Adeux}}\\
        \NomA\NomB^2&=\ifboolKV[ClesPythagore]{EnchaineC}{\opcopy{\useKV[ClesPythagore]{ValeurC}}{a1}\opexport{a1}{\Aun}\num{\Aun}}{\opmul*{A1}{A1}{a1}\opexport{a1}{\Aun}\num{\Aun}}-\ifboolKV[ClesPythagore]{EnchaineB}{\opexport{a2}{\Adeux}\num{\Adeux}}{\opmul*{A2}{A2}{a2}\opexport{a2}{\Adeux}\num{\Adeux}}\\
        \NomA\NomB^2&=\opsub*{a1}{a2}{a3}\opexport{a3}{\Atrois}\num{\Atrois}%\\
        \ifboolKV[ClesPythagore]{AvantRacine}{}{%
        \ifboolKV[ClesPythagore]{Entier}{}{\\\NomA\NomB&=\sqrt{\opexport{a3}{\Atrois}\num{\Atrois}}}%
        \ifboolKV[ClesPythagore]{Racine}{}{\\\NomA\NomB&\IfStrEq{\ResultatApproche}{\ResultatAttendu}{=}{\approx}\opsqrt[maxdivstep=5]{a3}{a4}\opround{a4}{pres}{a4}\opunzero{a4}\opexport{a4}{\Aquatre}\SI{\Aquatre}{\PfCPythaUnit}}%\\
       }
      \end{align*}
    }%
  }%\fi%
}%

\NewDocumentCommand\PfCCalculsReciPythagore{ommmm}{%
  \ifboolKV[ClesPythagore]{ReciColonnes}{%
    \[
      \begin{array}{cccc|cccc}
        &&\NomA\NomC^2&&&\NomA\NomB^2&+&\NomB\NomC^2\\
        &&\ifboolKV[ClesPythagore]{EnchaineC}{%
           \num{\useKV[ClesPythagore]{ValeurC}}%
           \xdef\PfCRetiensReciHypo{\useKV[ClesPythagore]{ValeurC}}%
           }{%
           \num{#3}^2}&&&\ifboolKV[ClesPythagore]{EnchaineA}{\num{\useKV[ClesPythagore]{ValeurA}}}{\num{#4}^2}&+&\ifboolKV[ClesPythagore]{EnchaineB}{\num{\useKV[ClesPythagore]{ValeurB}}}{\num{#5}^2}\\
        &&&&&\ifboolKV[ClesPythagore]{EnchaineA}{%Enchaine A
              \ifboolKV[ClesPythagore]{EnchaineB}{%EnchaineB
              }{%Pas Enchaine B
              \num{\useKV[ClesPythagore]{ValeurA}}}}{%Pas EnchaineA
              \num{\fpeval{#4*#4}}}&\ifboolKV[ClesPythagore]{EnchaineA}{\ifboolKV[ClesPythagore]{EnchaineB}{}{+}}{+}&\ifboolKV[ClesPythagore]{EnchaineB}{%Enchaine B
              \ifboolKV[ClesPythagore]{EnchaineA}{%EnchaineA
              }{%Pas Enchaine A
              \num{\useKV[ClesPythagore]{ValeurB}}}}{%Pas EnchaineB
              \num{\fpeval{#5*#5}}}\\
        &&\ifboolKV[ClesPythagore]{EnchaineC}{}{%
           \xdef\PfCRetiensReciHypo{\fpeval{#3*#3}}%
           \num{\PfCRetiensReciHypo}
           }&&&\multicolumn{3}{c}{%
                \ifboolKV[ClesPythagore]{EnchaineA}{%Enchaine A
                \ifboolKV[ClesPythagore]{EnchaineB}{%EnchaineB
                \xdef\PfCRetiensReciSomme{\fpeval{\useKV[ClesPythagore]{ValeurA}+\useKV[ClesPythagore]{ValeurB}}}%
              }{%Pas Enchaine B
                \xdef\PfCRetiensReciSomme{\fpeval{\useKV[ClesPythagore]{ValeurA}+#5*#5}}%
                }}{%Pas EnchaineA
                \ifboolKV[ClesPythagore]{EnchaineB}{
                \xdef\PfCRetiensReciSomme{\fpeval{\useKV[ClesPythagore]{ValeurB}+#4*#4}}%
                }{%
                \xdef\PfCRetiensReciSomme{\fpeval{#4*#4+#5*#5}}%
                }
                }
                \num{\PfCRetiensReciSomme}%
                }\\
      \end{array}
    \]
  }{%
    \[\left.
        \begin{array}{l}
          \NomA\NomC^2=\ifboolKV[ClesPythagore]{EnchaineC}{%
          \num{\useKV[ClesPythagore]{ValeurC}}%
          \xdef\PfCRetiensReciHypo{\useKV[ClesPythagore]{ValeurC}}%
          }{%
          \num{#3}^2=\num{\fpeval{#3*#3}}
          \xdef\PfCRetiensReciHypo{\fpeval{#3*#3}}%
          }\\
           \NomA\NomB^2+\NomB\NomC^2=\ifboolKV[ClesPythagore]{EnchaineA}{\num{\useKV[ClesPythagore]{ValeurA}}}{\num{#4}^2}+\ifboolKV[ClesPythagore]{EnchaineB}{\num{\useKV[ClesPythagore]{ValeurB}}}{\num{#5}^2}=
          \ifboolKV[ClesPythagore]{EnchaineA}{%Enchaine A
          \ifboolKV[ClesPythagore]{EnchaineB}{%Enchaine B
          \xdef\PfCRetiensReciSomme{\fpeval{\useKV[ClesPythagore]{ValeurA}+\useKV[ClesPythagore]{ValeurB}}}%
          \num{\PfCRetiensReciSomme}%
          }{% PasEnchaine B
          \num{\useKV[ClesPythagore]{ValeurA}}+\num{\fpeval{#5*#5}}=%
          \xdef\PfCRetiensReciSomme{\fpeval{\useKV[ClesPythagore]{ValeurA}+#5*#5}}%
          \num{\PfCRetiensReciSomme}%
          }
          }{%Pas Enchaine A
          \num{\fpeval{#4*#4}}+
          \ifboolKV[ClesPythagore]{EnchaineB}{%Enchaine B
          \num{\useKV[ClesPythagore]{ValeurB}}=\xdef\PfCRetiensReciSomme{\fpeval{\useKV[ClesPythagore]{ValeurB}+#4*#4}}%
          \num{\PfCRetiensReciSomme}%
          }{%Pas Enchaine B
          \num{\fpeval{#5*#5}}=\xdef\PfCRetiensReciSomme{\fpeval{#5*#5+#4*#4}}%
          \num{\PfCRetiensReciSomme}%
          }
          }
        \end{array}
      \right\}\xintifboolexpr{\PfCRetiensReciHypo==\PfCRetiensReciSomme}{\NomA\NomC^2=\NomA\NomB^2+\NomB\NomC^2}{\NomA\NomC^2\not=\NomA\NomB^2+\NomB\NomC^2}%
    \]
  }%
}%

% \newcommand\Pythagore[5][]{%
\NewDocumentCommand\Pythagore{o m m m m}{%
  % #1 Param\`etres sous forme de cl\'es
  % #2 Nom "complet" du triangle : ABC par exemple
  % #3 Premi\`ere longueur
  % #4 Deuxi\`eme longueur
  % #5 Troisi\`eme longueur (\'eventuellement vide)
  \useKVdefault[ClesPythagore]%
  \setKV[ClesPythagore]{#1}%
  \DeclareSIUnit{\PfCPythaUnit}{\useKV[ClesPythagore]{Unite}}%
  \ifboolKV[ClesPythagore]{Reciproque}{%
    % On retient les noms des sommets
    \StrMid{#2}{1}{1}[\NomA]%
    \StrMid{#2}{2}{2}[\NomB]%
    \StrMid{#2}{3}{3}[\NomC]%
    \xdef\NomTriangle{\NomA\NomB\NomC}%
    % on stocke les valeurs donn\'ees
    \opcopy{#3}{A1}%
    \opcopy{#4}{A2}%
    \opcopy{#5}{A3}%
    \xdef\GrandCote{#3}%
    \xdef\PetitCote{#4}%
    \xdef\MoyenCote{#5}%
    % On trace une figure ou pas ?
    \ifboolKV[ClesPythagore]{FigureSeule}{%
      \MPFigureReciPytha{\NomA}{\NomB}{\NomC}{#3}{#4}{#5}{\useKV[ClesPythagore]{Angle}}%
    }{%
      \ifboolKV[ClesPythagore]{Figure}{%Utilisation obligatoire de l'option --shell-escape de la compilation
        \begin{multicols}{2}
          {\em La figure est donn\'ee \`a titre indicatif.}%
          \[\MPFigureReciPytha{\NomA}{\NomB}{\NomC}{#3}{#4}{#5}{\useKV[ClesPythagore]{Angle}}\]%
          \par\columnbreak\par%
          \ifboolKV[ClesPythagore]{AllPerso}{%
            \RedactionReciPythagore%
            \RedactionCalculsReciPythagore%
            \RedactionConclusionReciPythagore%
          }{%
            % on r\'edige
            \ifboolKV[ClesPythagore]{Perso}{%
              \RedactionReciPythagore%
            }{%
            Dans le triangle $#2$, $[\NomA\NomC]$ est le plus grand c\^ot\'e.%
            }%
            \PfCCalculsReciPythagore[#1]{#2}{#3}{#4}{#5}%
            \ifboolKV[ClesPythagore]{Egalite}{%
              \xintifboolexpr{\PfCRetiensReciHypo==\PfCRetiensReciSomme}{%
                Comme $\NomA\NomC^2=\NomA\NomB^2+\NomB\NomC^2$, alors l'\'egalit\'e de Pythagore est v\'erifi\'ee. Donc le triangle $#2$ est rectangle en $\NomB$.}{%
                Comme $\NomA\NomC^2\not=\NomA\NomB^2+\NomB\NomC^2$, alors l'\'egalit\'e de Pythagore n'est pas v\'erifi\'ee. Donc le triangle $#2$ n'est pas rectangle.
              }%
            }{%
              \xintifboolexpr{\PfCRetiensReciHypo==\PfCRetiensReciSomme}{%
                Comme $\NomA\NomC^2=\NomA\NomB^2+\NomB\NomC^2$, alors le triangle $#2$ est rectangle en $\NomB$ d'apr\`es la r\'eciproque du th\'eor\`eme de Pythagore.}{%
                Comme $\NomA\NomC^2\not=\NomA\NomB^2+\NomB\NomC^2$, alors le
                triangle $#2$ n'est pas rectangle\ifboolKV[ClesPythagore]{Faible}{.}{ d'apr\`es la contrapos\'ee du th\'eor\`eme de Pythagore.}
              }%
            }%            
          }%
        \end{multicols}
      }{%
        \ifboolKV[ClesPythagore]{AllPerso}{%
          \RedactionReciPythagore%
          \RedactionCalculsReciPythagore%
          \RedactionConclusionReciPythagore%
        }{%
          \ifboolKV[ClesPythagore]{Perso}{\RedactionReciPythagore}{%
            Dans le triangle $#2$, $[\NomA\NomC]$ est le plus grand c\^ot\'e.%
          }%
          \PfCCalculsReciPythagore[#1]{#2}{#3}{#4}{#5}%
          \ifboolKV[ClesPythagore]{Egalite}{%
            \xintifboolexpr{\PfCRetiensReciHypo==\PfCRetiensReciSomme}{%
              Comme $\NomA\NomC^2=\NomA\NomB^2+\NomB\NomC^2$, alors l'\'egalit\'e de Pythagore est v\'erifi\'ee. Donc le triangle $#2$ est rectangle en $\NomB$.}{%
              Comme $\NomA\NomC^2\not=\NomA\NomB^2+\NomB\NomC^2$, alors l'\'egalit\'e de Pythagore n'est pas v\'erifi\'ee. Donc le triangle $#2$ n'est pas rectangle.
            }%
          }{%
            \xintifboolexpr{\PfCRetiensReciHypo==\PfCRetiensReciSomme}{%
              Comme $\NomA\NomC^2=\NomA\NomB^2+\NomB\NomC^2$, alors le triangle $#2$ est rectangle en $\NomB$ d'apr\`es la r\'eciproque du th\'eor\`eme de Pythagore.}{%
              Comme $\NomA\NomC^2\not=\NomA\NomB^2+\NomB\NomC^2$, alors le
              triangle $#2$ n'est pas rectangle\ifboolKV[ClesPythagore]{Faible}{.}{ d'apr\`es la contrapos\'ee du th\'eor\`eme de Pythagore.}
            }%
          }%
        }%
      }%
    }%
  }{%
    % [xlop] param\`etres de calcul
    \opcopy{#3}{A1}%
    \opcopy{#4}{A2}%
    \opcopy{\useKV[ClesPythagore]{Precision}}{pres}%
    \xintifboolexpr{#3<#4 || #3==#4}{%
      \xdef\PetitCote{#3}%
      \xdef\MoyenCote{#4}%
      \xdef\GrandCote{}%
    }{%
      \xdef\GrandCote{#3}%
      \xdef\MoyenCote{#4}%
      \xdef\PetitCote{}% Pour la personnalisation
    }%
    % On retient les noms des sommets
    \StrMid{#2}{1}{1}[\NomA]%
    \StrMid{#2}{2}{2}[\NomB]%
    \StrMid{#2}{3}{3}[\NomC]%
    \xdef\NomTriangle{\NomA\NomB\NomC}%
    \xdef\NomAngleDroit{\NomB}%
    \xdef\NomSommetA{\NomA}%
    \xdef\NomSommetC{\NomC}%
    % On trace une figure ou pas ?
    \ifboolKV[ClesPythagore]{FigureSeule}{%
      \xintifboolexpr{#3<#4 || #3==#4}{%
        \xdef\ResultatPytha{\fpeval{round(sqrt(#3^2+#4^2),\useKV[ClesPythagore]{Precision})}}%
      }{%
        \xdef\ResultatPytha{\fpeval{round(sqrt(#3^2-#4^2),\useKV[ClesPythagore]{Precision})}}%
      }%
      \MPFigurePytha{\NomA}{\NomB}{\NomC}{#3}{#4}{\useKV[ClesPythagore]{Angle}}%
    }{%
      \ifboolKV[ClesPythagore]{Figure}{%
        \begin{multicols}{2}%
          {\em La figure est donn\'ee \`a titre indicatif.}%
          \[\MPFigurePytha{\NomA}{\NomB}{\NomC}{#3}{#4}{\useKV[ClesPythagore]{Angle}}\]
          \par\columnbreak\par%
          % On d\'emarre la r\'esolution
          \ifboolKV[ClesPythagore]{AllPerso}{%
            \RedactionPythagore%
            \RedactionCalculsPythagore%
          }{%
            \ifboolKV[ClesPythagore]{Perso}{%
              \RedactionCalculsPythagore%
            }{%
              \ifboolKV[ClesPythagore]{Egalite}{Comme le triangle $#2$ est rectangle en $\NomB$, alors l'\'egalit\'e de Pythagore est v\'erifi\'ee :}{Dans le triangle $#2$ rectangle en $\NomB$, le th\'eor\`eme de Pythagore permet d'\'ecrire :%
              }%
            }%
            \CalculsPythagore[#1]{#2}{#3}{#4}{}
          }%
        \end{multicols}
      }{%
        % On d\'emarre la r\'esolution
        \ifboolKV[ClesPythagore]{SansMots}{%
         \MPFigurePythaSansMots{\NomA}{\NomB}{\NomC}{#3}{#4}{\useKV[ClesPythagore]{Angle}}%
        }{%
          \ifboolKV[ClesPythagore]{AllPerso}{%
            \RedactionPythagore%
            \RedactionCalculsPythagore%
          }{%
            \ifboolKV[ClesPythagore]{Perso}{\RedactionPythagore}{\ifboolKV[ClesPythagore]{Egalite}{Comme le triangle $#2$ est rectangle en $\NomB$, alors l'\'egalit\'e de Pythagore est v\'erifi\'ee :}{Dans le triangle $#2$ rectangle en $\NomB$, le th\'eor\`eme de Pythagore permet d'\'ecrire :%
              }}%
            \CalculsPythagore[#1]{#2}{#3}{#4}{}%
          }%
        }%
      }%
    }%
  }%
}%