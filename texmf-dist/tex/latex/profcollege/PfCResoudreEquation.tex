%%%
% Equations
%%%
\setKVdefault[ClesEquation]{Ecart=0.5,Fleches=false,FlecheDiv=false,Laurent=false,Decomposition=false,Terme=false,Composition=false,Symbole=false,ModeleBarre=false,Decimal=false,Entier=false,Lettre=x,Solution=false,LettreSol=true,Bloc=false,Simplification=false,CouleurTerme=black,CouleurCompo=black,CouleurSous=red,CouleurSymbole=orange,Verification=false,Nombre=0,Egalite=false,Produit=false,Facteurs=false,Carre=false,Exact=false,Pose=false,Equivalence=false}

\newcommand\rightcomment[4]{%
  \begin{tikzpicture}[remember picture,overlay]
    \draw[Cfleches,-stealth]
    ($({pic cs:#3}|-{pic cs:#1})+(\useKV[ClesEquation]{Ecart},0)$)
    .. controls +(0.2,-0.05) and +(0.2,0.1) ..
    node[right,align=left]{#4}
    ($({pic cs:#3}|-{pic cs:#2})+(\useKV[ClesEquation]{Ecart},0.1)$);
  \end{tikzpicture}%
}%

\newcommand\leftcomment[4]{%
  \begin{tikzpicture}[remember picture,overlay]
    \draw[Cfleches,-stealth]
    ($({pic cs:#3}|-{pic cs:#1})-(\useKV[ClesEquation]{Ecart},0)$)
    .. controls +(-0.2,-0.05) and +(-0.2,0.1) ..
    node[left,align=right]{#4}
    ($({pic cs:#3}|-{pic cs:#2})-(\useKV[ClesEquation]{Ecart},-0.1)$);
  \end{tikzpicture}%
}%

\newcommand\Rightcomment[4]{%
  \begin{tikzpicture}[remember picture,overlay]
    \draw[Cfleches,-stealth]
    ($({pic cs:#3}|-{pic cs:#1})+(\useKV[ClesEquation]{Ecart},0)$)
    .. controls +(0.2,-0.05) and +(0.2,0.1) ..
    node[right,align=left]{#4}
    ($({pic cs:#3}|-{pic cs:#2})+(\useKV[ClesEquation]{Ecart},0.1)$);
  \end{tikzpicture}%
}%
\newcommand\Leftcomment[4]{%
  \begin{tikzpicture}[remember picture,overlay]
    \draw[Cfleches,-stealth]
    ($({pic cs:#3}|-{pic cs:#1})-(\useKV[ClesEquation]{Ecart},0)$)
    .. controls +(-0.2,-0.05) and +(-0.2,0.1) ..
    node[left,align=right]{#4}
    ($({pic cs:#3}|-{pic cs:#2})-(\useKV[ClesEquation]{Ecart},-0.1)$);
  \end{tikzpicture}%
}%

% Pour "oublier" les tikzmarks. En cas de plusieurs utilisations de la macro \ResolEquation
\newcounter{Nbequa}
\setcounter{Nbequa}{0}

%CT
\newdimen\fdashwidth\fdashwidth  = 0.8pt % \'epaisseur traits
\newdimen\fdashlength\fdashlength = 0.5mm % longueur des pointill\'es et s\'eparation entre pointill\'es
\newdimen\fdashsep\fdashsep    = 3pt % s\'eparateur entre contenu et traits

\def\fdash#1{%
  \leavevmode\begingroup%
  \setbox0\hbox{#1}%
  \def\hdash{\vrule height\fdashwidth width\fdashlength\relax}%
  \def\vdash{\hrule height\fdashlength width\fdashwidth\relax}%
  \def\dashblank{\kern\fdashlength}%
  \ifdim\fdashsep>0pt%
  \setbox0\hbox{\vrule width0pt height\dimexpr\ht0+\fdashsep depth\dimexpr\dp0+\fdashsep\kern\fdashsep\unhbox0 \kern\fdashsep}%
  \fi%
  \edef\hdash{\hbox to\the\wd0{\noexpand\color{Csymbole}\hdash\kern.5\fdashlength\xleaders\hbox{\hdash\dashblank}\hfil\hdash}}%
  \edef\vdash{\vbox to\the\dimexpr\ht0+\dp0+2\fdashwidth{\noexpand\color{Csymbole}\vdash\kern.5\fdashlength\xleaders\vbox{\vdash\dashblank}\vfil\vdash}}%
  \hbox{%
    \vdash%
    \vtop{\vbox{\offinterlineskip\hdash\hbox{\unhbox0 }\hdash}}%
    \vdash}%
  \endgroup%
}%
% fin CT
\def\Fdash#1{\raisebox{-2\fdashsep+\fdashwidth}{\fdash{#1}}}%

%Une simplification de a/b est possible ou non ?
\newboolean{Simplification}

\newcommand\SSimpliTest[2]{%
  % Test d'une simplification possible ou pas de #1/#2
  \newcount\numerateur\newcount\denominateur\newcount\valabsnum\newcount\valabsdeno%
  \numerateur=\number#1
  \denominateur=\number#2
  \ifnum\number#1<0
  \valabsnum=\numexpr0-\number#1
  \else
  \valabsnum=\number#1
  \fi
  \ifnum\number#2<0
  \valabsdeno=\numexpr0-\number#2
  \else
  \valabsdeno=\number#2
  \fi
  \ifnum\the\valabsnum=0
    \setboolean{Simplification}{true}
  \else
    \PGCD{\the\valabsnum}{\the\valabsdeno}
    \ifnum\pgcd>1
      \setboolean{Simplification}{true}
    \else
      \ifnum\the\numerateur<0
        \ifnum\the\denominateur<0
          \setboolean{Simplification}{true}
        \else
          \ifnum\valabsdeno=1\relax
            \setboolean{Simplification}{true}
          \else
          \setboolean{Simplification}{false}
          \fi
        \fi
      \else
        \ifnum\valabsdeno=1\relax
          \setboolean{Simplification}{true}
        \else
          \setboolean{Simplification}{false}
        \fi
      \fi
    \fi
  \fi
}

\definecolor{Cfleches}{RGB}{100,100,100}%

\newcommand\AffichageEqua[4]{%
  \def\LETTRE{\useKV[ClesEquation]{Lettre}}%
  \ensuremath{%
    % partie du x
    \xintifboolexpr{#1==0}{}{\xintifboolexpr{#1==1}{}{\xintifboolexpr{#1==-1}{-}{\num{#1}}}\LETTRE}%
    % partie du nombre
    \xintifboolexpr{#2==0}{\xintifboolexpr{#1==0}{0}{}}{\xintifboolexpr{#2>0}{\xintifboolexpr{#1==0}{}{+}\num{#2}}{\num{#2}}}%
    % egal
    =
    % partie du x
    \xintifboolexpr{#3==0}{}{\xintifboolexpr{#3==1}{}{\xintifboolexpr{#3==-1}{-}{\num{#3}}}\LETTRE}%
    % partie du nombre
    \xintifboolexpr{#4==0}{%
      \xintifboolexpr{#3==0}{0}{}
    }{%
      \xintifboolexpr{#4>0}{\xintifboolexpr{#3==0}{}{+}\num{#4}}{\num{#4}}%
    }
  }%
}%

\newcommand\EcrireSolutionEquation[4]{%
  L'équation \AffichageEqua{#1}{#2}{#3}{#4} a une unique solution : \opdiv*{\Coeffb}{\Coeffa}{solution}{resteequa}\opcmp{resteequa}{0}$\ifboolKV[ClesEquation]{LettreSol}{\useKV[ClesEquation]{Lettre}=}{}\displaystyle\ifopeq\opexport{solution}{\solution}\num{\solution}\else\ifboolKV[ClesEquation]{Entier}{\SSimplifie{\Coeffb}{\Coeffa}}{\frac{\num{\Coeffb}}{\num{\Coeffa}}}\fi$.
}%

\input{PfCEquationSoustraction2}%
\input{PfCEquationTerme1}%
\input{PfCEquationComposition2}%
\input{PfCEquationPose1}%
\input{PfCEquationSymbole1}%
\input{PfCEquationLaurent1}%
% Licence    : Released under the LaTeX Project Public License v1.3c
% or later, see http://www.latex-project.org/lppl.txtf
\newtoks\TestEquaMBarreH%
\newtoks\TestEquaMBarreB%

\def\UpdateToksBarreH#1\nil{\addtotok\TestEquaMBarreH{#1}}%
\def\UpdateToksBarreB#1\nil{\addtotok\TestEquaMBarreB{#1}}%

\newcommand\EquaBaseMBarre[5][]{%type ax=d ou b=cx
  \useKVdefault[ClesEquation]%
  \setKV[ClesEquation]{#1}%
  \ifnum\fpeval{#2}=0\relax
    \ifnum\fpeval{#4}=0\relax
      \textbf{??}%% il manque un paramètre
    \else
      \xintifboolexpr{\fpeval{#4}==0}{%
        \xintifboolexpr{\fpeval{#3}==0}{%
          L'équation $0\times\useKV[ClesEquation]{Lettre}=0$ a une infinité de solutions.}{L'équation $0\times\useKV[ClesEquation]{Lettre}=\num{\fpeval{#3}}$ n'a aucune solution.}%
      }{%\else
        \xintifboolexpr{\fpeval{#3}==0}{L'équation $0=\num{\fpeval{#4}}\times\useKV[ClesEquation]{Lettre}$ a une unique solution : $\useKV[ClesEquation]{Lettre}=0$.}{%\else
          \TestEquaMBarreB{white}%
          \TestEquaMBarreH{}%
          \xintFor* ##1 in{\xintSeq{1}{\fpeval{#4}}}\do{%
            \expandafter\UpdateToksBarreB{,1}\nil%
            \expandafter\UpdateToksBarreB{,"$x$"}\nil%
          }%
          \expandafter\UpdateToksBarreH{white,#4,"\num{\fpeval{#3}}"}\nil%
          %          % Le tok H est \the\TestEquaMBarreH.%
          %          % \\Le tok B est \the\TestEquaMBarreB.%
          \PfCMPDessineModelBarreNonHomogene{\the\TestEquaMBarreH}{\the\TestEquaMBarreB}{}%
          \ifnum\fpeval{#4}=1\relax
          \else
            \bigskip%
          
            \PfCMPDessineModelBarreNonHomogene{white,1,"$\num{\fpeval{#3}}/\num{\fpeval{#4}}$"}{white,1,"$x$"}{}%
            \SSimpliTest{\fpeval{#3}}{\fpeval{#4}}%
            \ifthenelse{\boolean{Simplification}}{\bigskip%
              
              \PfCMPDessineModelBarreNonHomogene{white,1,"\PGCD{\fpeval{#3}}{\fpeval{#4}}\ifnum\fpeval{abs(\pgcd)}=\fpeval{#4}\relax{}\num{\fpeval{#3/#4}}\else\num{\fpeval{#3/\pgcd}}/\num{\fpeval{#4/\pgcd}}\fi"}{white,1,"$x$"}{}%
            }{}%
          \fi
        }%
      }%
    \fi%
  \else%
    % si non, on est dans le cas ax=d
    \xintifboolexpr{\fpeval{#2}==0}{%
      \xintifboolexpr{\fpeval{#5}==0}{%
        L'équation $0\times\useKV[ClesEquation]{Lettre}=0$ a une infinité de solutions.}{L'équation $0\times\useKV[ClesEquation]{Lettre}=\num{\fpeval{#5}}$ n'a aucune solution.}%
    }{%\else
      \xintifboolexpr{\fpeval{#5}==0}{L'équation $\num{\fpeval{#2}}\times\useKV[ClesEquation]{Lettre}=0$ a une unique solution : $\useKV[ClesEquation]{Lettre}=0$.}{%\else
        % \begin{center}
        \TestEquaMBarreH{white}%
        \TestEquaMBarreB{}%
        \xintFor* ##1 in{\xintSeq{1}{\fpeval{#2}}}\do{%
          \expandafter\UpdateToksBarreH{,1}\nil%
          \expandafter\UpdateToksBarreH{,"$x$"}\nil%
        }%
        \expandafter\UpdateToksBarreB{white,#2,"\num{\fpeval{#5}}"}\nil%
        % Le tok H est \the\TestEquaMBarreH.%
        % \\Le tok B est \the\TestEquaMBarreB.%
        \PfCMPDessineModelBarreNonHomogene{\the\TestEquaMBarreH}{\the\TestEquaMBarreB}{}%
        \ifnum\fpeval{#2}=1\relax%
        \else
          \bigskip%
          
          \PfCMPDessineModelBarreNonHomogene{white,1,"$x$"}{white,1,"$\num{\fpeval{#5}}/\num{\fpeval{#2}}$"}{}%
          \SSimpliTest{\fpeval{#5}}{\fpeval{#2}}%
          \ifthenelse{\boolean{Simplification}}{\bigskip
            
            \PfCMPDessineModelBarreNonHomogene{white,1,"$x$"}{white,1,"\PGCD{\fpeval{#5}}{\fpeval{#2}}\ifnum\fpeval{abs(\pgcd)}=\fpeval{#2}\relax{}\num{\fpeval{#5/#2}}\else\num{\fpeval{#5/\pgcd}}/\num{\fpeval{#2/\pgcd}}\fi"}{}%
          }{}%
        \fi
      }%
    }%
  \fi%
}%

\newcommand\EquaDeuxMBarre[5][]{%type ax+b=d ou b=cx+d$
  \useKVdefault[ClesEquation]%
  \setKV[ClesEquation]{#1}%
  \setKV[ClesEquation]{Fleches=false,FlecheDiv=false,Terme=false,Decomposition=false}
  \ifnum\fpeval{#2}=0\relax%On échange en faisant attention à ne pas boucler : c doit être non vide
    % cas b=cx+d
    \xintifboolexpr{\fpeval{#4}==0}{%
      \xintifboolexpr{\fpeval{#3}==\fpeval{#5}}{%b=d
        L'équation $\num{\fpeval{#3}}=\num{\fpeval{#5}}$ a une infinité de solutions.%
      }{%b<>d
        L'équation $\num{\fpeval{#3}}=\num{\fpeval{#5}}$ n'a aucune solution.%
      }%
    }{%ELSE
      \xintifboolexpr{\fpeval{#3}==0}{%ax+b=d
        \EquaBaseMBarre[#1]{0}{\fpeval{-#5}}{\fpeval{#4}}{0}%
      }{%ax+b=d$ Ici
        % \begin{center}
        \TestEquaMBarreB{white}%
        \TestEquaMBarreH{}%
        \xintFor* ##1 in{\xintSeq{1}{#4}}\do{%
          \expandafter\UpdateToksBarreB{,1}\nil%
          \expandafter\UpdateToksBarreB{,"$x$"}\nil%
        }%
        \expandafter\UpdateToksBarreB{,white,2.5,"\num{\fpeval{#5}}"}\nil%
        \expandafter\UpdateToksBarreH{white,#4+2.5,"\num{\fpeval{#3}}"}\nil%
        % Le tok H est \the\TestEquaMBarreH.%
        % \\Le tok B est \the\TestEquaMBarreB.%
        \PfCMPDessineModelBarreNonHomogene{\the\TestEquaMBarreH}{\the\TestEquaMBarreB}{}%
        
        \bigskip%
        
        \PfCMPDessineModelBarreNonHomogene{white,#4,"\num{\fpeval{#3-#5}}",white,2.5,"\num{\fpeval{#5}}"}{\the\TestEquaMBarreB}{#4}%
        
        \bigskip
        
        \EquaBaseMBarre{0}{(#3-(#5))}{#4}{0}%
      }%
    }%
  \else%cas ax+b=d
    \xintifboolexpr{\fpeval{#2}==0}{%
      \xintifboolexpr{\fpeval{#3}==\fpeval{#5}}{%b=d
        L'équation $\num{\fpeval{#3}}=\num{\fpeval{#5}}$ a une infinité de solutions.%
      }{%b<>d
        L'équation $\num{\fpeval{#3}}=\num{\fpeval{#5}}$ n'a aucune solution.%
      }%
    }{%ELSE
      \xintifboolexpr{\fpeval{#3}==0}{%ax+b=d
        \EquaBaseMBarre[#1]{#2}{0}{0}{#5}%
      }{%ax+b=d$ Ici
%        % \begin{center}
        \TestEquaMBarreH{white}%
        \TestEquaMBarreB{}%
        \xintFor* ##1 in{\xintSeq{1}{\fpeval{#2}}}\do{%
          \expandafter\UpdateToksBarreH{,1}\nil%
          \expandafter\UpdateToksBarreH{,"$x$"}\nil%
        }%
        \expandafter\UpdateToksBarreH{,white,2.5,"\num{\fpeval{#3}}"}\nil%
        \expandafter\UpdateToksBarreB{white,#2+2.5,"\num{\fpeval{#5}}"}\nil%
%        % Le tok H est \the\TestEquaMBarreH.%
%        % \\Le tok B est \the\TestEquaMBarreB.%
        \PfCMPDessineModelBarreNonHomogene{\the\TestEquaMBarreH}{\the\TestEquaMBarreB}{}%
        
        \bigskip%
        
        \PfCMPDessineModelBarreNonHomogene{\the\TestEquaMBarreH}{white,#2,"\num{\fpeval{#5-#3}}",white,2.5,"\num{\fpeval{#3}}"}{#2}%
%        
        \bigskip
        
       \EquaBaseMBarre{#2}{0}{0}{(#5-(#3))}%
      }%
    }%
  \fi%
}%

\newcommand\EquaTroisMBarre[5][]{%ax+b=cx ou ax=cx+d
  \useKVdefault[ClesEquation]%
  \setKV[ClesEquation]{#1}%
  \ifnum\fpeval{#3}=0\relax%on inverse en faisant attention à la boucle #3<->#5
    \ifnum\fpeval{#5}=0\relax%
      %% paramètre oublié
    \else
      \EquaTroisMBarre[#1]{#4}{#5}{#2}{0}%
    \fi
  \else
    \xintifboolexpr{\fpeval{#2}==0}{%b=cx
      \EquaBaseMBarre[#1]{#4}{0}{0}{#3}
    }{%
      \xintifboolexpr{\fpeval{#4}==0}{%ax+b=0
        \EquaDeuxMBarre[#1]{#2}{#3}{0}{0}
      }{%ax+b=cx
        \xintifboolexpr{\fpeval{#2}==\fpeval{#4}}{%
          \xintifboolexpr{\fpeval{#3}==0}{%ax=ax
            L'équation $\xintifboolexpr{#2==1}{}{\num{#2}}\useKV[ClesEquation]{Lettre}=\xintifboolexpr{#4==1}{}{\num{#4}}\useKV[ClesEquation]{Lettre}$ a une infinité de solutions.%
          }{%ax+b=ax
            L'équation $\xintifboolexpr{#2==1}{}{\num{#2}}\useKV[ClesEquation]{Lettre}\xintifboolexpr{#3>0}{+\num{#3}}{-\num{\fpeval{0-#3}}}=\xintifboolexpr{#4==1}{}{\num{#4}}\useKV[ClesEquation]{Lettre}$ n'a aucune solution.%
          }
        }{%% Cas délicat
          \xintifboolexpr{\fpeval{#2}>\fpeval{#4}}{%ax+b=cx avec a>c
            \TestEquaMBarreH{white}%
            \TestEquaMBarreB{white}%
            \xintFor* ##1 in{\xintSeq{1}{\fpeval{#2}}}\do{%
              \expandafter\UpdateToksBarreH{,1}\nil%
              \expandafter\UpdateToksBarreH{,"$x$"}\nil%
            }%
            \expandafter\UpdateToksBarreH{,white,2.5,"\num{\fpeval{#3}}"}\nil%
            \xintFor* ##1 in{\xintSeq{1}{\fpeval{#4}}}\do{%
              \expandafter\UpdateToksBarreB{,1}\nil%
              \expandafter\UpdateToksBarreB{,"$x$"}\nil%
            }%
            \expandafter\UpdateToksBarreB{,white,#2-#4+2.5,"0"}\nil%
            \PfCMPDessineModelBarreNonHomogene{\the\TestEquaMBarreH}{\the\TestEquaMBarreB}{}%
            
            \PfCMPDessineModelBarreNonHomogene{\the\TestEquaMBarreH}{\the\TestEquaMBarreB}{#4}%      
          
            \EquaDeuxMBarre{(#2-#4)}{#3}{0}{0}
          }{%ax+b=cx avec a<c              % Autre cas délicat
            \TestEquaMBarreH{white}%
            \TestEquaMBarreB{white}%
            \xintFor* ##1 in{\xintSeq{1}{\fpeval{#2}}}\do{%
              \expandafter\UpdateToksBarreH{,1}\nil%
              \expandafter\UpdateToksBarreH{,"$x$"}\nil%
            }%
            \expandafter\UpdateToksBarreH{,white,#4-#2,"\num{\fpeval{#3}}"}\nil%
            \xintFor* ##1 in{\xintSeq{1}{\fpeval{#4}}}\do{%
              \expandafter\UpdateToksBarreB{,1}\nil%
              \expandafter\UpdateToksBarreB{,"$x$"}\nil%
            }%
%            \expandafter\UpdateToksBarreB{,white,#2-#4+2.5,"0"}\nil%
            \PfCMPDessineModelBarreNonHomogene{\the\TestEquaMBarreH}{\the\TestEquaMBarreB}{}%
            
            \PfCMPDessineModelBarreNonHomogene{\the\TestEquaMBarreH}{\the\TestEquaMBarreB}{#2}%
            
            \EquaDeuxMBarre{0}{#3}{(#4-#2)}{0}
          }%
        }%
      }%
    }%
  \fi
}%

\newcommand\ResolEquationMBarre[5][]{%
  \useKVdefault[ClesEquation]%
  \setKV[ClesEquation]{#1}%
  \setKV[ClesEquation]{Fleches=false,FlecheDiv=false,Terme=false,Decomposition=false}
  \xintifboolexpr{#2==0}{%
    \xintifboolexpr{#4==0}{%
      \xintifboolexpr{#3==#5}{%b=d
        L'équation $\num{\fpeval{#3}}=\num{\fpeval{#5}}$ a une infinité de solutions.}%
      {%b<>d
        L'équation $\num{\fpeval{#3}}=\num{\fpeval{#5}}$ n'a aucune solution.%
      }%
    }%
    {%0x+b=cx+d$
      \EquaDeuxMBarre[#1]{#4}{#5}{#2}{#3}%
    }%
  }{%
    \xintifboolexpr{#4==0}{%ax+b=0x+d
      \EquaDeuxMBarre[#1]{#2}{#3}{0}{#5}%
    }
    {%ax+b=cx+d$
      \xintifboolexpr{#3==0}{%
        \xintifboolexpr{#5==0}{%ax=cx
          \EquaTroisMBarre[#1]{#2}{0}{#4}{0}%
        }%
        {%ax=cx+d
          \EquaTroisMBarre[#1]{#4}{#5}{#2}{0}%
        }%
      }%
      {\xintifboolexpr{#5==0}{%ax+b=cx
          \EquaTroisMBarre[#1]{#2}{#3}{#4}{0}%
        }%
        {%ax+b=cx+d -- ici
          \xintifboolexpr{#2==#4}{%
            \xintifboolexpr{#3==#5}{%b=d
                            L'équation $\xintifboolexpr{#2==1}{}{\num{#2}}\useKV[ClesEquation]{Lettre}\xintifboolexpr{#3>0}{+\num{#3}}{-\num{\fpeval{0-#3}}}=\xintifboolexpr{#4==1}{}{\num{#4}}\useKV[ClesEquation]{Lettre}\xintifboolexpr{#5>0}{+\num{#5}}{-\num{\fpeval{0-#5}}}$ a une infinité de solutions.}%
            {%b<>d
              L'équation $\xintifboolexpr{#2==1}{}{\num{#2}}\useKV[ClesEquation]{Lettre}\xintifboolexpr{#3>0}{+\num{#3}}{-\num{\fpeval{0-#3}}}=\xintifboolexpr{#4==1}{}{\num{#4}}\useKV[ClesEquation]{Lettre}\xintifboolexpr{#5>0}{+\num{#5}}{-\num{\fpeval{0-#5}}}$ n'a aucune solution.%
            }%
          }{%
            %% Cas délicat
            \xintifboolexpr{#2>#4}{%ax+b=cx+d avec a>c
              \TestEquaMBarreH{white}%
              \TestEquaMBarreB{white}%
              \xintFor* ##1 in{\xintSeq{1}{\fpeval{#2}}}\do{%
                \expandafter\UpdateToksBarreH{,1}\nil%
                \expandafter\UpdateToksBarreH{,"$x$"}\nil%
              }%
              \expandafter\UpdateToksBarreH{,white,2.5,"\num{\fpeval{#3}}"}\nil%
              \xintFor* ##1 in{\xintSeq{1}{\fpeval{#4}}}\do{%
                \expandafter\UpdateToksBarreB{,1}\nil%
                \expandafter\UpdateToksBarreB{,"$x$"}\nil%
              }%
              \expandafter\UpdateToksBarreB{,white,#2-#4+2.5,"\num{\fpeval{#5}}"}\nil%
              \PfCMPDessineModelBarreNonHomogene{\the\TestEquaMBarreH}{\the\TestEquaMBarreB}{}%
            
              \PfCMPDessineModelBarreNonHomogene{\the\TestEquaMBarreH}{\the\TestEquaMBarreB}{#4}%      
          
              \EquaDeuxMBarre{(#2-(#4))}{#3}{0}{#5}
            }{%ax+b=cx+d avec a<c              % Autre cas délicat
              \TestEquaMBarreH{white}%
              \TestEquaMBarreB{white}%
              \xintFor* ##1 in{\xintSeq{1}{\fpeval{#4}}}\do{%
                \expandafter\UpdateToksBarreB{,1}\nil%
                \expandafter\UpdateToksBarreB{,"$x$"}\nil%
              }%
              \expandafter\UpdateToksBarreB{,white,2.5,"\num{\fpeval{#5}}"}\nil%
              \xintFor* ##1 in{\xintSeq{1}{\fpeval{#2}}}\do{%
                \expandafter\UpdateToksBarreH{,1}\nil%
                \expandafter\UpdateToksBarreH{,"$x$"}\nil%
              }%
              \expandafter\UpdateToksBarreH{,white,#4-#2+2.5,"\num{\fpeval{#3}}"}\nil%
              \PfCMPDessineModelBarreNonHomogene{\the\TestEquaMBarreH}{\the\TestEquaMBarreB}{}%
            
              \PfCMPDessineModelBarreNonHomogene{\the\TestEquaMBarreH}{\the\TestEquaMBarreB}{#2}%
          
              \EquaDeuxMBarre{0}{#3}{(#4-(#2))}{#5}
            }%
          }%
        }%
      }%
    }%
  }%
}%%

\newcommand\ResolEquation[5][]{%
  \useKVdefault[ClesEquation]%
  \setKV[ClesEquation]{#1}%
  \colorlet{Cterme}{\useKV[ClesEquation]{CouleurTerme}}%
  \colorlet{Ccompo}{\useKV[ClesEquation]{CouleurCompo}}%
  \colorlet{Csymbole}{\useKV[ClesEquation]{CouleurSymbole}}%
  \colorlet{Cdecomp}{\useKV[ClesEquation]{CouleurSous}}%
  \ifboolKV[ClesEquation]{Carre}{%
    \ResolEquationCarre[#1]{#2}%
  }{%
    \ifboolKV[ClesEquation]{Produit}{%
      \ResolEquationProduit[#1]{#2}{#3}{#4}{#5}%
    }{%
      \ifboolKV[ClesEquation]{Verification}{%
        \Verification[#1]{#2}{#3}{#4}{#5}%
      }{%
        \ifboolKV[ClesEquation]{ModeleBarre}{%
          \ResolEquationMBarre[#1]{#2}{#3}{#4}{#5}%
        }{%
          \ifboolKV[ClesEquation]{Symbole}{%
            \ResolEquationSymbole[#1]{#2}{#3}{#4}{#5}%
          }{%
            \ifboolKV[ClesEquation]{Laurent}{%
              \ResolEquationLaurent[#1]{#2}{#3}{#4}{#5}%
            }{%
              \ifboolKV[ClesEquation]{Terme}{%
                \ResolEquationTerme[#1]{#2}{#3}{#4}{#5}%
              }{\ifboolKV[ClesEquation]{Composition}{%
                  \ResolEquationComposition[#1]{#2}{#3}{#4}{#5}%
                }{\ifboolKV[ClesEquation]{Pose}{%
                    \ResolEquationL[#1]{#2}{#3}{#4}{#5}%
                  }{%
                    \ResolEquationSoustraction[#1]{#2}{#3}{#4}{#5}%
                  }%
                }%
              }%
            }%
          }%
        }%
      }%
    }%
  }%
}%

\newcommand\ResolEquationCarre[2][]{%
  \setKV[ClesEquation]{#1}%
  \xintifboolexpr{#2<0}{%
    Comme $\num{#2}$ est n\'egatif, alors l'\'equation $\useKV[ClesEquation]{Lettre}^2=\num{#2}$ n'a aucune solution.%
  }{\xintifboolexpr{#2==0}{%
      L'\'equation $\useKV[ClesEquation]{Lettre}^2=0$ a une unique solution : $\useKV[ClesEquation]{Lettre}=0$.%
    }{%
      Comme \num{#2} est positif, alors l'\'equation $\useKV[ClesEquation]{Lettre}^2=\num{#2}$ a deux solutions :%
      \begin{align*}
        \useKV[ClesEquation]{Lettre}&=\sqrt{\num{#2}}&&\text{et}&\useKV[ClesEquation]{Lettre}&=-\sqrt{\num{#2}}%\\
        \ifboolKV[ClesEquation]{Exact}{\\%
        \useKV[ClesEquation]{Lettre}&=\num{\fpeval{sqrt(#2)}}&&\text{et}&\useKV[ClesEquation]{Lettre}&=-\num{\fpeval{sqrt(#2)}}}{}%
      \end{align*}
    }%
  }%
}%

\newcommand\ResolEquationProduit[5][]{%
  \setKV[ClesEquation]{#1}%
  \ifboolKV[ClesEquation]{Equivalence}{}{C'est un produit nul donc \ifboolKV[ClesEquation]{Facteurs}{l'un au
      moins des facteurs est nul}{} :}%
  \ifboolKV[ClesEquation]{Equivalence}{%
          \[\Distri{#2}{#3}{#4}{#5}=0\]
    \begin{align*}%
      &\makebox[0pt]{$\Longleftrightarrow$}&\xintifboolexpr{#3==0}{\xintifboolexpr{#2==1}{}{\num{#2}}\useKV[ClesEquation]{Lettre}}{\xintifboolexpr{#2==1}{}{\num{#2}}\useKV[ClesEquation]{Lettre}\xintifboolexpr{#3>0}{+\num{#3}}{-\num{\fpeval{0-#3}}}}&=0&\quad&\makebox[0pt]{ou}\quad&\xintifboolexpr{#5==0}{\xintifboolexpr{#4==1}{}{\num{#4}}\useKV[ClesEquation]{Lettre}}{\xintifboolexpr{#4==1}{}{\num{#4}}\useKV[ClesEquation]{Lettre}\xintifboolexpr{#5>0}{+\num{#5}}{-\num{\fpeval{0-#5}}}}&=0\\
      &\makebox[0pt]{$\Longleftrightarrow$}&\xintifboolexpr{#3==0}{\xdef\Coeffa{1}\xdef\Coeffb{\fpeval{0-#3}}\xintifboolexpr{#2==1}{&}{\useKV[ClesEquation]{Lettre}&=0}}{\xdef\Coeffa{#2}\xdef\Coeffb{\fpeval{0-#3}}\xintifboolexpr{\Coeffa==1}{}{\num{\Coeffa}}\useKV[ClesEquation]{Lettre}&=\num{\Coeffb}}&&&\xintifboolexpr{#5==0}{\xdef\Coeffc{1}\xdef\Coeffd{\fpeval{0-#5}}\xintifboolexpr{#4==1}{&}{\useKV[ClesEquation]{Lettre}&=0}}{\xdef\Coeffc{#4}\xdef\Coeffd{\fpeval{0-#5}}\xintifboolexpr{\Coeffc==1}{}{\num{\Coeffc}}\useKV[ClesEquation]{Lettre}&=\num{\Coeffd}}%\\
      \xintifboolexpr{\Coeffa==1 'and' \Coeffc==1}{}{\\%\ifnum\cmtd>1
      &\makebox[0pt]{$\Longleftrightarrow$}&\xintifboolexpr{\Coeffa==1}{&}{\useKV[ClesEquation]{Lettre}&=\frac{\num{\Coeffb}}{\num{\Coeffa}}}\xintifboolexpr{\Coeffc==1}{}{&&&\useKV[ClesEquation]{Lettre}&=\frac{\num{\Coeffd}}{\num{\Coeffc}}}
      % accolade%\\
      %%%% 
      \ifboolKV[ClesEquation]{Entier}{%
      \xdef\TSimp{}%
      \SSimpliTest{\Coeffb}{\Coeffa}\ifthenelse{\boolean{Simplification}}{\xintifboolexpr{#3==0}{\xdef\TSimp{0}}{\xdef\TSimp{1}}}{\xdef\TSimp{0}}
      \SSimpliTest{\Coeffd}{\Coeffc}\ifthenelse{\boolean{Simplification}}{\xintifboolexpr{#5==0}{}{\xdef\TSimp{\fpeval{\TSimp+1}}}}{}
      \xintifboolexpr{\TSimp==0}{}{\\
      \ifboolKV[ClesEquation]{Simplification}{%
      &\makebox[0pt]{$\Longleftrightarrow$}&\SSimpliTest{\Coeffb}{\Coeffa}\xintifboolexpr{\Coeffa==1}{&}{\ifthenelse{\boolean{Simplification}}{\useKV[ClesEquation]{Lettre}&=\SSimplifie{\Coeffb}{\Coeffa}}{&}%\\
      }
      }{}
      &&&\ifboolKV[ClesEquation]{Simplification}{%
      \SSimpliTest{\Coeffd}{\Coeffc}%
          \xintifboolexpr{\Coeffc==1}{}{\ifthenelse{\boolean{Simplification}}{\useKV[ClesEquation]{Lettre}&=\SSimplifie{\Coeffd}{\Coeffc}}{}%\\
      }
      }{}
      }
      }{}
      }
    \end{align*}
  }{%
    \begin{align*}
    \xintifboolexpr{#3==0}{\xintifboolexpr{#2==1}{}{\num{#2}}\useKV[ClesEquation]{Lettre}}{\xintifboolexpr{#2==1}{}{\num{#2}}\useKV[ClesEquation]{Lettre}\xintifboolexpr{#3>0}{+\num{#3}}{-\num{\fpeval{0-#3}}}}&=0&&\text{ou}&\xintifboolexpr{#5==0}{\xintifboolexpr{#4==1}{}{\num{#4}}\useKV[ClesEquation]{Lettre}}{\xintifboolexpr{#4==1}{}{\num{#4}}\useKV[ClesEquation]{Lettre}\xintifboolexpr{#5>0}{+\num{#5}}{-\num{\fpeval{0-#5}}}}&=0\\
    \xintifboolexpr{#3==0}{\xdef\Coeffa{1}\xdef\Coeffb{\fpeval{0-#3}}\xintifboolexpr{#2==1}{&}{\useKV[ClesEquation]{Lettre}&=0}}{\xdef\Coeffa{#2}\xdef\Coeffb{\fpeval{0-#3}}\xintifboolexpr{\Coeffa==1}{}{\num{\Coeffa}}\useKV[ClesEquation]{Lettre}&=\num{\Coeffb}}&&&\xintifboolexpr{#5==0}{\xdef\Coeffc{1}\xdef\Coeffd{\fpeval{0-#5}}\xintifboolexpr{#4==1}{&}{\useKV[ClesEquation]{Lettre}&=0}}{\xdef\Coeffc{#4}\xdef\Coeffd{\fpeval{0-#5}}\xintifboolexpr{\Coeffc==1}{}{\num{\Coeffc}}\useKV[ClesEquation]{Lettre}&=\num{\Coeffd}}%\\
      \xintifboolexpr{\Coeffa==1 'and' \Coeffc==1}{}{\\%\ifnum\cmtd>1
      \xintifboolexpr{\Coeffa==1}{&}{\useKV[ClesEquation]{Lettre}&=\frac{\num{\Coeffb}}{\num{\Coeffa}}}\xintifboolexpr{\Coeffc==1}{}{&&&\useKV[ClesEquation]{Lettre}&=\frac{\num{\Coeffd}}{\num{\Coeffc}}}
       %accolade%\\
      %%%%   
      \ifboolKV[ClesEquation]{Entier}{%
      \xdef\TSimp{}
      \SSimpliTest{\Coeffb}{\Coeffa}\ifthenelse{\boolean{Simplification}}{\xintifboolexpr{#3==0}{\xdef\TSimp{0}}{\xdef\TSimp{1}}}{\xdef\TSimp{0}}
      \SSimpliTest{\Coeffd}{\Coeffc}\ifthenelse{\boolean{Simplification}}{\xintifboolexpr{#5==0}{}{\xdef\TSimp{\fpeval{\TSimp+1}}}}{}
      \xintifboolexpr{\TSimp==0}{}{\\
      \ifboolKV[ClesEquation]{Simplification}{%
      \SSimpliTest{\Coeffb}{\Coeffa}
      \xintifboolexpr{\Coeffa==1}{&}{\ifthenelse{\boolean{Simplification}}{\useKV[ClesEquation]{Lettre}&=\SSimplifie{\Coeffb}{\Coeffa}}{&}%\\
      }
      }{}
      &&&\ifboolKV[ClesEquation]{Simplification}{%
      \SSimpliTest{\Coeffd}{\Coeffc}%
          \xintifboolexpr{\Coeffc==1}{}{\ifthenelse{\boolean{Simplification}}{\useKV[ClesEquation]{Lettre}&=\SSimplifie{\Coeffd}{\Coeffc}}{}%\\
      }
      }{}
      }
      }{}
      }
    \end{align*}
  }%
  \ifboolKV[ClesEquation]{Solution}{L'\'equation $\xintifboolexpr{#3==0}{\xintifboolexpr{#2==1}{}{\num{#2}}\useKV[ClesEquation]{Lettre}}{(\xintifboolexpr{#2==1}{}{\num{#2}}\useKV[ClesEquation]{Lettre}\xintifboolexpr{#3>0}{+\num{#3}}{-\num{\fpeval{0-#3}}})}\xintifboolexpr{#5==0}{\times\xintifboolexpr{#4==1}{}{\num{#4}}\useKV[ClesEquation]{Lettre}}{(\xintifboolexpr{#4==1}{}{\num{#4}}\useKV[ClesEquation]{Lettre}\xintifboolexpr{#5>0}{+\num{#5}}{-\num{\fpeval{0-#5}}})}=0$ a deux solutions : \opdiv*{\Coeffb}{\Coeffa}{solution}{resteequa}\opcmp{resteequa}{0}$\ifboolKV[ClesEquation]{LettreSol}{\useKV[ClesEquation]{Lettre}=}{}\displaystyle\ifopeq\opexport{solution}{\solution}\num{\solution}\else\ifboolKV[ClesEquation]{Entier}{\SSimplifie{\Coeffb}{\Coeffa}}{\frac{\num{\Coeffb}}{\num{\Coeffa}}}\fi$ et \opdiv*{\Coeffd}{\Coeffc}{solution}{resteequa}\opcmp{resteequa}{0}$\ifboolKV[ClesEquation]{LettreSol}{\useKV[ClesEquation]{Lettre}=}{}\displaystyle\ifopeq\opexport{solution}{\solution}\num{\solution}\else\ifboolKV[ClesEquation]{Entier}{\SSimplifie{\Coeffd}{\Coeffc}}{\frac{\num{\Coeffd}}{\num{\Coeffc}}}\fi$.
  }{}%
}

\newcommand\Verification[5][]{%
  \setKV[ClesEquation]{#1}%
  \xdef\ValeurTest{\useKV[ClesEquation]{Nombre}}%
  Testons la valeur $\useKV[ClesEquation]{Lettre}=\num{\ValeurTest}$ :%
  \begin{align*}
    \xintifboolexpr{#2==0}{\num{#3}}{\num{#2}\times\xintifboolexpr{\ValeurTest<0}{(\num{\ValeurTest})}{\num{\ValeurTest}}\xintifboolexpr{#3==0}{}{\xintifboolexpr{#3>0}{+\num{#3}}{\num{#3}}}}&&\xintifboolexpr{#4==0}{\num{#5}}{\num{#4}\times\xintifboolexpr{\ValeurTest<0}{(\num{\ValeurTest})}{\num{\ValeurTest}}\xintifboolexpr{#5==0}{}{\xintifboolexpr{#5>0}{+\num{#5}}{\num{#5}}}}\\
    \xintifboolexpr{#2==0}{}{\num{\fpeval{#2*\useKV[ClesEquation]{Nombre}}}\xintifboolexpr{#3==0}{}{\xintifboolexpr{#3>0}{+\num{#3}}{\num{#3}}}}&&\xintifboolexpr{#4==0}{}{\num{\fpeval{#4*\useKV[ClesEquation]{Nombre}}}\xintifboolexpr{#5==0}{}{\xintifboolexpr{#5>0}{+\num{#5}}{\num{#5}}}}\\
    \xintifboolexpr{#2==0}{}{\num{\fpeval{#2*\useKV[ClesEquation]{Nombre}+#3}}}&&\xintifboolexpr{#4==0}{}{\num{\fpeval{#4*\useKV[ClesEquation]{Nombre}+#5}}}
  \end{align*}
  \xdef\Testa{\fpeval{#2*\useKV[ClesEquation]{Nombre}+#3}}\xdef\Testb{\fpeval{#4*\useKV[ClesEquation]{Nombre}+#5}}%
  \ifboolKV[ClesEquation]{Egalite}{%
    Comme \xintifboolexpr{\Testa==\Testb}{$\num{\Testa}=\num{\Testb}$}{$\num{\Testa}\not=\num{\Testb}$}, alors l'\'egalit\'e $\xintifboolexpr{#2==0}{\num{#3}}{\xintifboolexpr{#2==1}{}{\num{#2}}\useKV[ClesEquation]{Lettre}\xintifboolexpr{#3==0}{}{\xintifboolexpr{#3>0}{+\num{#3}}{-\num{\fpeval{0-#3}}}}}=\xintifboolexpr{#4==0}{\num{#5}}{\xintifboolexpr{#4==1}{}{\num{#4}}\useKV[ClesEquation]{Lettre}\xintifboolexpr{#5==0}{}{\xintifboolexpr{#5>0}{+\num{#5}}{-\num{\fpeval{0-#5}}}}}$ \xintifboolexpr{\Testa==\Testb}{ est v\'erifi\'ee }{ n'est pas v\'erifi\'ee } pour $\useKV[ClesEquation]{Lettre}=\num{\useKV[ClesEquation]{Nombre}}$.%
  }{\xintifboolexpr{\Testa==\Testb}{Comme $\num{\Testa}=\num{\Testb}$, alors $\useKV[ClesEquation]{Lettre}=\num{\useKV[ClesEquation]{Nombre}}$ est bien }{Comme $\num{\Testa}\not=\num{\Testb}$, alors $\useKV[ClesEquation]{Lettre}=\num{\useKV[ClesEquation]{Nombre}}$ n'est pas }une solution de l'\'equation $\xintifboolexpr{#2==0}{\num{#3}}{\xintifboolexpr{#2==1}{}{\num{#2}}\useKV[ClesEquation]{Lettre}\xintifboolexpr{#3==0}{}{\xintifboolexpr{#3>0}{+\num{#3}}{-\num{\fpeval{0-#3}}}}}=\xintifboolexpr{#4==0}{\num{#5}}{\xintifboolexpr{#4==1}{}{\num{#4}}\useKV[ClesEquation]{Lettre}\xintifboolexpr{#5==0}{}{\xintifboolexpr{#5>0}{+\num{#5}}{-\num{\fpeval{0-#5}}}}}$.}%
}%