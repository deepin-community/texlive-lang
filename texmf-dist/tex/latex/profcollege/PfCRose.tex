%%%
% Rose
%%%
\def\filedateRose{2024/08/04}%
\def\fileversionRose{0.1}%
\message{-- \filedateRose\space v\fileversionRose}%
%
\setKVdefault[ClesRose]{Vide=false,Couleur=white,Produits=false,Solution=false,Rayon=2cm,Petales=7,Aide=false,FacteurMin=2,FacteurMax=10,ProduitVide=false,CaseVide=false,Polygone=false,Nom=Rose1,Creation=false,Addition=false,Melange=false,CDot=false}%
\defKV[ClesRose]{LesNombres=\setKV[ClesRose]{Creation}}%
\defKV[ClesRose]{LesProduits=\setKV[ClesRose]{Creation}}%
\defKV[ClesRose]{LesSommes=\setKV[ClesRose]{Addition}}%

\def\UpdatetoksRM#1\nil{\addtotok\toklisterosemul{"#1",}}%

\NewDocumentCommand\RoseMul{o}{%
  \newtoks\toklisterosemul%
  \useKVdefault[ClesRose]%
  \setKV[ClesRose]{#1}%
  \ifboolKV[ClesRose]{Creation}{%
    \xdef\PfCFooRoseMul{\useKV[ClesRose]{LesNombres}}%
    \ifboolKV[ClesRose]{Addition}{%
      \xdef\PfCFooRoseMul{\PfCFooRoseMul,\useKV[ClesRose]{LesSommes}}%
    }{%
      \xdef\PfCFooRoseMul{\PfCFooRoseMul,\useKV[ClesRose]{LesProduits}}%
    }%
    \setsepchar{,}\reademptyitems%
    \readlist*\ListeCompleteRoseMul{\PfCFooRoseMul}%
    \ignoreemptyitems
    \foreachitem\compteur\in\ListeCompleteRoseMul{\expandafter\UpdatetoksRM\compteur\nil}%
    \xdef\PfCRetiensLesFacteurs{}%
    \foreachitem\compteur\in\ListeCompleteRoseMul{%
      \StrLen{\ListeCompleteRoseMul[\compteurcnt]}[\PfCLongueurArg]%
      \ifnum\PfCLongueurArg>0\relax%
        \xdef\PfCRetiensLesFacteurs{\PfCRetiensLesFacteurs \ListeCompleteRoseMul[\compteurcnt],}%
      \else%
        \xdef\PfCRetiensLesFacteurs{\PfCRetiensLesFacteurs "a",}%
      \fi%
    }%
    \DessineRoseMulPerso{\the\toklisterosemul}{\PfCRetiensLesFacteurs}%
  }{%
    \ifluatex%
      \ifboolKV[ClesRose]{Solution}{%
        \DessineRoseMulSol%
      }{%
        \DessineRoseMul%
      }%
    \else%
      \DessineRoseMul%
    \fi%
  }%
}%

\def\DessineRoseMulPersoCode{%
  pair A[],B[],O;
  O=(0,0);
  path cc;
  cc=cercles(O,Rayon);
  for k=0 upto Petales:
  A[k]=pointarc(cc,(360/Petales)*k);
  endfor;
  for k=0 upto Petales-1:
  B[k]=iso(A[k],A[k+1]);
  endfor;
  B[Petales]=B[0];
  path BoiteRec;
  if Polygone:
  BoiteRec=(pointarc(fullcircle,0) for k=1 upto Petales:--pointarc(fullcircle,k*(360/Petales)) endfor --cycle) scaled 1u;
  else:
  BoiteRec=((-0.4,-0.5)--(0.4,-0.5){dir0}..{dir90}(0.5,-0.4)--(0.5,0.4){dir90}..{dir180}(0.4,0.5)--(-0.4,0.5){dir180}..{dir-90}(-0.5,0.4)--(-0.5,-0.4){dir-90}..cycle) scaled 1u;
  fi;
  % On récupère la liste des nombres et facteurs
  vardef RecupListe(text t)=
  nbpetales:=0;
  nbfacteur:=0;
  nbproduit:=0;
  numfacteurconnu:=0;
  for p_=t:
  if nbpetales<Petales:
  if string p_:
  else:
  Facteur[nbfacteur]=p_;
  NumFacteurConnu[numfacteurconnu]=nbfacteur;
  FacteurConnu[numfacteurconnu]=p_;
  numfacteurconnu:=numfacteurconnu+1;
  fi;
  nbfacteur:=nbfacteur+1;
  else:
  if string p_:
  else:
  Produit[nbproduit]=p_;
  fi;
  nbproduit:=nbproduit+1;
  fi;
  nbpetales:=nbpetales+1;
  endfor;
  if unknown Facteur[0]:
  else:
  Facteur[Petales]=Facteur[0];
  fi;
  if unknown Produit[0]:
  else:
  Produit[Petales]=Produit[0];
  fi;
  TotalFacteursConnus:=numfacteurconnu;
  enddef;
  %
  vardef RechercheFacteurs=
  if Solution:
  for k=1 upto TotalFacteursConnus:
  for l=(NumFacteurConnu[k-1]) downto 0:
  if (known Produit[l-1]):
  if (known Facteur[l]):
  FacteurSol[l-1]:=Produit[l-1]/Facteur[l];
  Facteur[l-1]:=FacteurSol[l-1];
  fi;
  fi;
  endfor;
  for l=(NumFacteurConnu[k-1]) upto Petales-1:
  if (known Produit[l]):
  if (known Facteur[l]):
  FacteurSol[l+1]:=Produit[l]/Facteur[l];
  Facteur[l+1]:=FacteurSol[l+1];
  fi;
  fi;
  endfor;
  endfor;
  fi;
  enddef;
  vardef RechercheTermes=
  if Solution:
  for k=1 upto TotalFacteursConnus:
  for l=(NumFacteurConnu[k-1]) downto 0:
  if (known Produit[l-1]):
  if (known Facteur[l]):
  FacteurSol[l-1]:=Produit[l-1]-Facteur[l];
  Facteur[l-1]:=FacteurSol[l-1];
  fi;
  fi;
  endfor;
  for l=(NumFacteurConnu[k-1]) upto Petales-1:
  if (known Produit[l]):
  if (known Facteur[l]):
  FacteurSol[l+1]:=Produit[l]-Facteur[l];
  Facteur[l+1]:=FacteurSol[l+1];
  fi;
  fi;
  endfor;
  endfor;
  fi;
  enddef;
  % on labelise
  vardef AffichagedelaListe(text t)=
  nbpetales:=0;
  for p_=t:
  if nbpetales<Petales:
  if p_="":
  if Solution:
  fill rotation(polygone(O,B[Petales-1],A[0],B[Petales]),O,nbpetales*(360/Petales)) withcolor Fond;
  if unknown FacteurSol[nbpetales]:
  else:
  label(TEX("\num{"&decimal(FacteurSol[nbpetales])&"}"),(0.5[O,A[nbpetales]]-center BoiteRec));
  fi;
  fi;
  else:
  label(TEX("\num{"&p_&"}"),(0.5[O,A[nbpetales]]-center BoiteRec));
  fi;
  else:
  if p_="":
  if Solution:
  fill (BoiteRec shifted (B[nbpetales-Petales]+0.5*Rayon*unitvector(B[nbpetales-Petales]))) withcolor Fond;
  if Addition:
  label(TEX("\num{\noexpand\fpeval{"&decimal(Facteur[nbpetales-Petales])&"+"&decimal(Facteur[nbpetales+1-Petales])&"}}"),(B[nbpetales-Petales]+0.5*Rayon*unitvector(B[nbpetales-Petales])));
  else:
  label(TEX("\num{\noexpand\fpeval{"&decimal(Facteur[nbpetales-Petales])&"*"&decimal(Facteur[nbpetales+1-Petales])&"}}"),(B[nbpetales-Petales]+0.5*Rayon*unitvector(B[nbpetales-Petales])));
  fi;
  fi;
  else:
  label(TEX("\num{"&p_&"}"),(B[nbpetales-Petales]+0.5*Rayon*unitvector(B[nbpetales-Petales])));
  fi;
  fi;
  nbpetales:=nbpetales+1;
  endfor;
  enddef;
  %
  vardef AffichagedelaListePDF(text t)=
  nbpetales:=0;
  for p_=t:
  if nbpetales<Petales:
  if p_="":
  if Solution:
  fill rotation(polygone(O,B[Petales-1],A[0],B[Petales]),O,nbpetales*(360/Petales)) withcolor Fond;
  if unknown FacteurSol[nbpetales]:
  else:
  label(LATEX("\num{"&decimal(FacteurSol[nbpetales])&"}"),(0.5[O,A[nbpetales]]-center BoiteRec));
  fi;
  fi;
  else:
  label(LATEX("\num{"&p_&"}"),(0.5[O,A[nbpetales]]-center BoiteRec));
  fi;
  else:
  if p_="":
  if Solution:
  fill (BoiteRec shifted (B[nbpetales-Petales]+0.5*Rayon*unitvector(B[nbpetales-Petales]))) withcolor Fond;
  if Addition:
  label(LATEX("\num{\noexpand\fpeval{"&decimal(Facteur[nbpetales-Petales])&"+"&decimal(Facteur[nbpetales+1-Petales])&"}}"),(B[nbpetales-Petales]+0.5*Rayon*unitvector(B[nbpetales-Petales])));
  else:
  label(LATEX("\num{\noexpand\fpeval{"&decimal(Facteur[nbpetales-Petales])&"*"&decimal(Facteur[nbpetales+1-Petales])&"}}"),(B[nbpetales-Petales]+0.5*Rayon*unitvector(B[nbpetales-Petales])));
  fi;
  fi;
  else:
  label(LATEX("\num{"&p_&"}"),(B[nbpetales-Petales]+0.5*Rayon*unitvector(B[nbpetales-Petales])));
  fi;
  fi;
  nbpetales:=nbpetales+1;
  endfor;
  enddef;
}

\NewDocumentCommand\DessineRoseMulPerso{mm}{%
  \ifluatex%
    \mplibforcehmode%
    \begin{mplibcode}
      Rayon=\useKV[ClesRose]{Rayon};
      boolean Aide,Solution,Polygone,Addition,CDot;
      CDot=\useKV[ClesRose]{CDot};
      Aide=\useKV[ClesRose]{Aide};
      Solution=\useKV[ClesRose]{Solution};
      Addition=\useKV[ClesRose]{Addition};
      Polygone=\useKV[ClesRose]{Polygone};
      Petales=\useKV[ClesRose]{Petales};
      color Fond;
      Fond=\useKV[ClesRose]{Couleur};
      \DessineRoseMulPersoCode
      RecupListe(#2);
      if Addition:
      RechercheTermes;
      else:
      RechercheFacteurs;
      fi;
      AffichagedelaListe(#1);
      if Aide:
      for k=0 upto Petales-1:
      drawarrow (0.5[O,A[k]]--1.75[O,B[k]]) cutbefore segment(A[k],A[k+1]) cutafter BoiteRec shifted (B[k]+0.5*Rayon*unitvector(B[k])) dashed evenly scaled 0.5;
      drawarrow (0.5[O,A[k+1]]--1.75[O,B[k]]) cutbefore segment(A[k],A[k+1]) cutafter BoiteRec shifted (B[k]+0.5*Rayon*unitvector(B[k])) dashed evenly scaled 0.5;
      endfor;
      fi;
      trace A0 for k=1 upto Petales:--A[k] endfor --cycle;
      for k=0 upto Petales-1:
      trace segment(O,B[k]) dashed withdots scaled 0.25;
      draw BoiteRec shifted (B[k]+0.5*Rayon*unitvector(B[k])) dashed evenly scaled 0.5;
      endfor;
      fill cercles(O,3mm) withcolor white;
      draw cercles(O,3mm);
      if Addition:
      label(TEX("\Large$+$"),O);
      else:
      if CDot:
      label(TEX("\Large$\cdot$"),O);
      else:
      label(TEX("\Large$\times$"),O);
      fi;
      fi;
    \end{mplibcode}
  \else
    \begin{mpost}[mpsettings={Rayon:=\useKV[ClesRose]{Rayon};
        boolean Aide,Solution,Polygone,CDot;
        CDot=\useKV[CleSRose]{Aide};
        Aide:=\useKV[ClesRose]{Aide};
        Solution:=\useKV[ClesRose]{Solution};
        Polygone:=\useKV[ClesRose]{Polygone};
        Petales:=\useKV[ClesRose]{Petales};
        color Fond;
        Fond:=\useKV[ClesRose]{Couleur};
        \PfCDessineRoseMulPersoCode
      }]
      RecupListe(#2);
      RechercheFacteurs;
      AffichagedelaListePDF(#1);
      if Aide:
      for k=0 upto Petales-1:
      drawarrow (0.5[O,A[k]]--1.75[O,B[k]]) cutbefore segment(A[k],A[k+1]) cutafter BoiteRec shifted (B[k]+0.5*Rayon*unitvector(B[k])) dashed evenly scaled 0.5;
      drawarrow (0.5[O,A[k+1]]--1.75[O,B[k]]) cutbefore segment(A[k],A[k+1]) cutafter BoiteRec shifted (B[k]+0.5*Rayon*unitvector(B[k])) dashed evenly scaled 0.5;
      endfor;
      fi;
      trace A0 for k=1 upto Petales:--A[k] endfor --cycle;
      for k=0 upto Petales-1:
      trace segment(O,B[k]) dashed withdots scaled 0.25;
      draw BoiteRec shifted (B[k]+0.5*Rayon*unitvector(B[k])) dashed evenly scaled 0.5;
      endfor;
      fill cercles(O,4mm) withcolor white;
      draw cercles(O,4mm);
      if Addition:
      label(LATEX("\Huge$+$"),O);
      else:
      if CDot:
      label(LATEX("\Huge$\cdot$"),O);
      else:
      label(LATEX("\Huge$\times$"),O);
      fi;
      fi;
    \end{mpost}
  \fi
}%

\newcommand\DessineRoseMulSol{%
  \ifluatex%
  \mplibforcehmode%
  \xdef\PfCNomRose{\useKV[ClesRose]{Nom}}%
  \begin{mplibcode}[\PfCNomRose]
    boolean Aide,Produits,Solution,Vide,ProduitVide,CaseVide,Polygone,Addition,CDot;
    CDot=\useKV[ClesRose]{CDot};
    Addition=\useKV[ClesRose]{Addition};
    Aide:=\useKV[ClesRose]{Aide};
    Vide:=\useKV[ClesRose]{Vide};
    Solution:=\useKV[ClesRose]{Solution};
    Produits:=\useKV[ClesRose]{Produits};
    ProduitVide:=\useKV[ClesRose]{ProduitVide};
    CaseVide:=\useKV[ClesRose]{CaseVide};
    if ProduitVide:Solution:=true fi;
    if CaseVide:Solution:=true fi;
    % on labelise
    for k=0 upto Petales-1:
    label(TEX("\num{"&decimal(Facteur[k])&"}"),(0.5[O,A[k]]-center BoiteRec));
    endfor;
    for k=0 upto Petales-1:
    label(TEX("\num{"&decimal(Produit[k])&"}"),(B[k]+0.5*Rayon*unitvector(B[k])));
    endfor;
    % On trace
    if Aide:
    for k=0 upto Petales-1:
    drawarrow (0.5[O,A[k]]--1.75[O,B[k]]) cutbefore segment(A[k],A[k+1]) cutafter BoiteRec shifted (B[k]+0.5*Rayon*unitvector(B[k])) dashed evenly scaled 0.5;
    drawarrow (0.5[O,A[k+1]]--1.75[O,B[k]]) cutbefore segment(A[k],A[k+1]) cutafter BoiteRec shifted (B[k]+0.5*Rayon*unitvector(B[k])) dashed evenly scaled 0.5;
    endfor;
    fi;
    trace A0 for k=1 upto Petales:--A[k] endfor --cycle;
    for k=0 upto Petales-1:
    trace segment(O,B[k]) dashed withdots scaled 0.25;
    draw BoiteRec shifted (B[k]+0.5*Rayon*unitvector(B[k])) dashed evenly scaled 0.5;
    endfor;
    fill cercles(O,3mm) withcolor white;
    draw cercles(O,3mm);
    if Addition:
    label(TEX("\Large$+$"),O);
    else:
    if CDot:
    label(TEX("\Large$\cdot$"),O);
    else:
    label(TEX("\Large$\times$"),O);
    fi;
    fi;
  \end{mplibcode}
  \fi
}
\newcommand\DessineRoseMul{%
  \ifluatex%
  \mplibforcehmode%
  \xdef\PfCNomRose{\useKV[ClesRose]{Nom}}%
  \begin{mplibcode}[\PfCNomRose]
    vardef ExtraireListe(expr nbmin,nbmax,nbfacteurs)=
    numeric ElementE[],ElementS[];
    boolean VueE[];
    %% éléments d'entrée
    nbe:=1;
    for k=nbmin upto nbmax:
    ElementE[nbe]=k;
    nbe:=nbe+1;
    endfor;
    totalE:=(nbmax-nbmin)+1;
    numeric rec;
    for k=1 upto nbfacteurs:
    VueE[k-1]=true;
    rec:=ceiling(uniformdeviate(totalE+1-k));
    ElementS[k]=ElementE[rec];
    numeric ElementI[];
    p:=0;
    for l=1 upto totalE+1-k:
    if l<>rec:
    p:=p+1;
    ElementI[p]=ElementE[l];
    fi;
    endfor;
    VueE[Petales]:=true;
    numeric ElementE[];
    for l=1 upto p:
    ElementE[l]=ElementI[l];
    endfor;
    endfor;
    enddef;
    pair A[],B[],O;
    Rayon:=\useKV[ClesRose]{Rayon};
    boolean Aide,Produits,Solution,Vide,ProduitVide,CaseVide,Polygone,Addition,Melange,CDot;
    CDot=\useKV[ClesRose]{CDot};
    Melange:=\useKV[ClesRose]{Melange};
    Addition=\useKV[ClesRose]{Addition};
    Aide:=\useKV[ClesRose]{Aide};
    Vide:=\useKV[ClesRose]{Vide};
    Solution:=\useKV[ClesRose]{Solution};
    Produits:=\useKV[ClesRose]{Produits};
    ProduitVide:=\useKV[ClesRose]{ProduitVide};
    CaseVide:=\useKV[ClesRose]{CaseVide};
    if ProduitVide:Solution:=true fi;
    if CaseVide:Solution:=true fi;
    Polygone:=\useKV[ClesRose]{Polygone};
    Petales:=\useKV[ClesRose]{Petales};
    FacteurMin:=\useKV[ClesRose]{FacteurMin};
    FacteurMax:=\useKV[ClesRose]{FacteurMax};
    color Fond;
    Fond:=\useKV[ClesRose]{Couleur};
    O=(0,0);
    ExtraireListe(FacteurMin,FacteurMax,Petales);
    path cc;
    cc=cercles(O,Rayon);
    for k=0 upto Petales:
    A[k]=pointarc(cc,(360/Petales)*k);
    endfor;
    for k=0 upto Petales-1:
    B[k]=iso(A[k],A[k+1]);
    endfor;
    path BoiteRec;
    if Polygone:
    BoiteRec=(pointarc(fullcircle,0) for k=1 upto Petales:--pointarc(fullcircle,k*(360/Petales)) endfor --cycle) scaled 1u;
    else:
    BoiteRec=((-0.4,-0.5)--(0.4,-0.5){dir0}..{dir90}(0.5,-0.4)--(0.5,0.4){dir90}..{dir180}(0.4,0.5)--(-0.4,0.5){dir180}..{dir-90}(-0.5,0.4)--(-0.5,-0.4){dir-90}..cycle) scaled 1u;
    fi;    
    %On détermine les facteurs et produits
    for k=0 upto Petales-1:
    Facteur[k]:=ElementS[k+1];
    endfor;
    Facteur[Petales]:=Facteur[0];
    for k=0 upto Petales-1:
    if Addition:
    Produit[k]:=Facteur[k]+Facteur[k+1];
    else:
    Produit[k]:=Facteur[k]*Facteur[k+1];
    fi;
    endfor;
    %on labelise
    if Vide=false:
    if Produits:
    else:
    for k=0 upto Petales-1:
    label(TEX("\num{"&decimal(Facteur[k])&"}"),(0.5[O,A[k]]-center BoiteRec));
    endfor;
    fi;
    if Produits or Solution:
    for k=0 upto Petales-1:
    label(TEX("\num{"&decimal(Produit[k])&"}"),(B[k]+0.5*Rayon*unitvector(B[k])));
    endfor;
    fi;
    fi;
    % On trace et on affiche...éventuellement :)
    if Aide:
    for k=0 upto Petales-1:
    drawarrow (0.5[O,A[k]]--1.75[O,B[k]]) cutbefore segment(A[k],A[k+1]) cutafter BoiteRec shifted (B[k]+0.5*Rayon*unitvector(B[k])) dashed evenly scaled 0.5;
    drawarrow (0.5[O,A[k+1]]--1.75[O,B[k]]) cutbefore segment(A[k],A[k+1]) cutafter BoiteRec shifted (B[k]+0.5*Rayon*unitvector(B[k])) dashed evenly scaled 0.5;
    endfor;
    fi;
    if Melange:
    % on gère les "oublis"
    nbvide:=0;
    for k=0 step 2 until Petales-1:
    if uniformdeviate(1)<0.5:
    nbk:=k;
    else:
    nbk:=k+1
    fi;
    VueE[nbk]:=false;
    endfor;
    VueE[Petales]:=VueE[0];
    for k=0 upto Petales-1:
    fill rotation(polygone(O,B0,A1,B1),O,(k-1)*(360/Petales)) withcolor Fond;
    if VueE[k]:
     label(TEX("\num{"&decimal(Facteur[k])&"}"),(0.5[O,A[k]]-center BoiteRec));
    fi;
    endfor;
    % on affiche les produits pour les effacer
    for k=0 upto Petales-1:
    if VueE[k]:
    if VueE[k+1]=false:
    label(TEX("\num{"&decimal(Produit[k])&"}"),(B[k]+0.5*Rayon*unitvector(B[k])));
    fi;
    else:
    if VueE[k+1]=false:
    label(TEX("\num{"&decimal(Produit[k])&"}"),(B[k]+0.5*Rayon*unitvector(B[k])));
    fi;
    fi;
    endfor;    
    else:
    if ProduitVide:
    choix:=floor(uniformdeviate(Petales-1));
    fill (BoiteRec shifted (1.75[O,B[choix]]-center BoiteRec)) withcolor Fond;
    fill rotation(polygone(O,B0,A1,B1),O,(choix-1)*(360/Petales)) withcolor Fond;
    fill rotation(polygone(O,B0,A1,B1),O,choix*(360/Petales)) withcolor Fond;
    fi;
    if CaseVide:
    choix:=floor(uniformdeviate(Petales-1));
    for k=0 upto Petales-1:
    if k<>choix:
    fill rotation(polygone(O,B0,A1,B1),O,k*(360/Petales)) withcolor Fond;
    fi;
    endfor;
    fi;
    fi;
    trace A0 for k=1 upto Petales:--A[k] endfor --cycle;
    for k=0 upto Petales-1:
    trace segment(O,B[k]) dashed withdots scaled 0.25;
    draw BoiteRec shifted (B[k]+0.5*Rayon*unitvector(B[k])) dashed evenly scaled 0.5;
    endfor;
    fill cercles(O,3mm) withcolor white;
    draw cercles(O,3mm);
    if Addition:
    label(TEX("\Large$+$"),O);
    else:
    if CDot:
    label(TEX("\Large$\cdot$"),O);
    else:
    label(TEX("\Large$\times$"),O);
    fi;
    fi;
  \end{mplibcode}
  \else
  \begin{mpost}[mpsettings={Rayon:=\useKV[ClesRose]{Rayon};
      boolean Aide,Produits,Solution,Vide,ProduitVide,CaseVide,Polygone,CDot;
      CDot:=useKV[ClesRose]{CDot};
    Aide:=\useKV[ClesRose]{Aide};
    Vide:=\useKV[ClesRose]{Vide};
    Solution:=\useKV[ClesRose]{Solution};
    Produits:=\useKV[ClesRose]{Produits};
    ProduitVide:=\useKV[ClesRose]{ProduitVide};
    CaseVide:=\useKV[ClesRose]{CaseVide};
    if ProduitVide:Solution:=true fi;
    if CaseVide:Solution:=true fi;
    Polygone:=\useKV[ClesRose]{Polygone};
    Petales:=\useKV[ClesRose]{Petales};
    FacteurMin:=\useKV[ClesRose]{FacteurMin};
    FacteurMax:=\useKV[ClesRose]{FacteurMax};
    color Fond;
    Fond:=\useKV[ClesRose]{Couleur};}]
    vardef ExtraireListe(expr nbmin,nbmax,nbfacteurs)=
    numeric ElementE[],ElementS[];
    %% éléments d'entrée
    nbe=1;
    for k=nbmin upto nbmax:
    ElementE[nbe]=k;
    nbe:=nbe+1;
    endfor;
    totalE=(nbmax-nbmin)+1;
    numeric rec;
    for k=1 upto nbfacteurs:
    rec:=ceiling(uniformdeviate(totalE+1-k));
    ElementS[k]=ElementE[rec];
    numeric ElementI[];
    p:=0;
    for l=1 upto totalE+1-k:
    if l<>rec:
    p:=p+1;
    ElementI[p]=ElementE[l];
    fi;
    endfor;
    numeric ElementE[];
    for l=1 upto p:
    ElementE[l]=ElementI[l];
    endfor;
    endfor;
    enddef;
    pair A[],B[],O;
    O=(0,0);
    ExtraireListe(FacteurMin,FacteurMax,Petales);
    path cc;
    cc=cercles(O,Rayon);
    for k=0 upto Petales:
    A[k]=pointarc(cc,(360/Petales)*k);
    endfor;
    for k=0 upto Petales-1:
    B[k]=iso(A[k],A[k+1]);
    endfor;
    path BoiteRec;
    if Polygone:
    BoiteRec=(pointarc(fullcircle,0) for k=1 upto Petales:--pointarc(fullcircle,k*(360/Petales)) endfor --cycle) scaled 1u;
    else:
    BoiteRec=((-0.4,-0.5)--(0.4,-0.5){dir0}..{dir90}(0.5,-0.4)--(0.5,0.4){dir90}..{dir180}(0.4,0.5)--(-0.4,0.5){dir180}..{dir-90}(-0.5,0.4)--(-0.5,-0.4){dir-90}..cycle) scaled 1u;
    fi;    
    %On détermine les facteurs et produits
    for k=0 upto Petales-1:
    Facteur[k]=ElementS[k+1];
    endfor;
    Facteur[Petales]=Facteur[0];
    for k=0 upto Petales-1:
    Produit[k]=Facteur[k]*Facteur[k+1];
    endfor;
    %on labelise
    if Vide=false:
    if Produits:
    else:
    for k=0 upto Petales-1:
    label(LATEX("\num{"&decimal(Facteur[k])&"}"),(0.5[O,A[k]]-center BoiteRec));
    endfor;
    fi;
    if Produits or Solution:
    for k=0 upto Petales-1:
    label(LATEX("\num{"&decimal(Produit[k])&"}"),(B[k]+0.5*Rayon*unitvector(B[k])));
    endfor;
    fi;
    fi;
    % On trace et on affiche...éventuellement :)
    if Aide:
    for k=0 upto Petales-1:
    drawarrow (0.5[O,A[k]]--1.75[O,B[k]]) cutbefore segment(A[k],A[k+1]) cutafter BoiteRec shifted (B[k]+0.5*Rayon*unitvector(B[k])) dashed evenly scaled 0.5;
    drawarrow (0.5[O,A[k+1]]--1.75[O,B[k]]) cutbefore segment(A[k],A[k+1]) cutafter BoiteRec shifted (B[k]+0.5*Rayon*unitvector(B[k])) dashed evenly scaled 0.5;
    endfor;
    fi;
    if ProduitVide:
    choix=floor(uniformdeviate(Petales-1));
    fill (BoiteRec shifted (1.75[O,B[choix]]-center BoiteRec)) withcolor Fond;
    fill rotation(polygone(O,B0,A1,B1),O,(choix-1)*(360/Petales)) withcolor Fond;
    fill rotation(polygone(O,B0,A1,B1),O,choix*(360/Petales)) withcolor Fond;
    fi;
    if CaseVide:
    choix:=floor(uniformdeviate(Petales-1));
    for k=0 upto Petales-1:
    if k<>choix:
    fill rotation(polygone(O,B0,A1,B1),O,k*(360/Petales)) withcolor Fond;
    fi;
    endfor;
    fi;
    trace A0 for k=1 upto Petales:--A[k] endfor --cycle;
    for k=0 upto Petales-1:
    trace segment(O,B[k]) dashed withdots scaled 0.25;
    draw BoiteRec shifted (B[k]+0.5*Rayon*unitvector(B[k])) dashed evenly scaled 0.5;
    endfor;
    fill cercles(O,3mm) withcolor white;
    draw cercles(O,3mm);
    if Addition:
    label(LATEX("\noexpand\Large$+$"),O);
    else:
    if CDot:
    label(LATEX("\noexpand\Large$\cdot$"),O);
    else:
    label(LATEX("\noexpand\Large$\times$"),O);
    fi;
    fi;
  \end{mpost}
  \fi
}%