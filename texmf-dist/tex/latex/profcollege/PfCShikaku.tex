%%%
% Shikaku
%%%
\newcounter{PfCShikakuNom}
\setcounter{PfCShikakuNom}{0}

\setKVdefault[Shikaku]{Couleur=Purple,Taille=4,Largeur=2em,Solution=false,CodeAfter={},Creation=false,TailleHor=10,TailleVer=10,TailleHorMax=5,TailleVerMax=5,Nom=Shikaku\PfCShikakuNom}

\newlength\PfCShikakuh
\newlength\PfCShikakuv

\NewDocumentCommand\Shikaku{o m}{%
  \stepcounter{PfCShikakuNom}%
  \useKVdefault[Shikaku]%
  \setKV[Shikaku]{#1}%
  \ifboolKV[Shikaku]{Creation}{%
    \ifluatex\else\PackageWarning{ProfCollege}{La création automatique de Shikaku n'est disponible qu'avec LuaLaTeX.}\fi%
    \ifboolKV[Shikaku]{Solution}{%
      \ShikakuCreationSolution%
    }{%
      \ShikakuCreation%
    }%
  }{%
    \setlength{\PfCShikakuv}{\useKV[Shikaku]{Largeur}+\tabcolsep}%
    \setlength{\PfCShikakuh}{\useKV[Shikaku]{Largeur}}%
    \colorlet{PfCCouleurShikaku}{\useKV[Shikaku]{Couleur}}%
    \setsepchar[*]{,*/}%
    \readlist*\ListeCasesSKK{#2}%
    \savecomparemode%
    \comparestrict%
    \begin{NiceTabular}{*{\useKV[Shikaku]{Taille}}{m{\PfCShikakuh}}}[hvlines]
      \xintFor* ##1 in {\xintSeq{0}{\fpeval{\useKV[Shikaku]{Taille}-1}}}\do{%
        \xintFor* ##2 in {\xintSeq{1}{\useKV[Shikaku]{Taille}}}\do{%
          \rule{0pt}{\PfCShikakuv}%
          \StrCompare{\ListeCasesSKK[\fpeval{\useKV[Shikaku]{Taille}*##1+##2},1]}{b}[\PfCTestb]%
          \StrCompare{\ListeCasesSKK[\fpeval{\useKV[Shikaku]{Taille}*##1+##2},1]}{l}[\PfCTestl]%
          \StrCompare{\ListeCasesSKK[\fpeval{\useKV[Shikaku]{Taille}*##1+##2},1]}{lb}[\PfCTestlb]%
          \xintifboolexpr{\PfCTestb==0}{%
            \ifboolKV[Shikaku]{Solution}{%
              \Block[borders={bottom,tikz=PfCCouleurShikaku},line-width=2pt]{1-1}{%
                \ListeCasesSKK[\fpeval{\useKV[Shikaku]{Taille}*##1+##2},2]%
              }%
            }{%
              \Block[]{1-1}{%
                \ListeCasesSKK[\fpeval{\useKV[Shikaku]{Taille}*##1+##2},2]%
              }%
            }%
          }{% Testl
            \xintifboolexpr{\PfCTestl==0}{%
              \ifboolKV[Shikaku]{Solution}{%
                \Block[borders={left,tikz=PfCCouleurShikaku},line-width=2pt]{1-1}{%
                  \ListeCasesSKK[\fpeval{\useKV[Shikaku]{Taille}*##1+##2},2]%
                }%
              }{%
                \Block[]{1-1}{%
                  \ListeCasesSKK[\fpeval{\useKV[Shikaku]{Taille}*##1+##2},2]%
                }%
              }%
            }{% Testlb
              \xintifboolexpr{\PfCTestlb==0}{%
                \ifboolKV[Shikaku]{Solution}{%
                  \Block[borders={left,bottom,tikz=PfCCouleurShikaku},line-width=2pt]{1-1}{%
                    \ListeCasesSKK[\fpeval{\useKV[Shikaku]{Taille}*##1+##2},2]%
                  }%
                }{%
                  \Block[]{1-1}{%
                    \ListeCasesSKK[\fpeval{\useKV[Shikaku]{Taille}*##1+##2},2]%
                  }%
                }%
              }{%
                \Block[]{1-1}{%
                  \ListeCasesSKK[\fpeval{\useKV[Shikaku]{Taille}*##1+##2},2]%
                }%
              }%
            }%
          }%
          \xintifForLast{\\}{&}%
        }%
      }%
      \CodeAfter%
      \useKV[Shikaku]{CodeAfter}%
      \ifboolKV[Shikaku]{Solution}{%
        \tikz\draw[line width=2pt,PfCCouleurShikaku] (1-|1) rectangle (last-|last);%
      }{}%
    \end{NiceTabular}%
    \restorecomparemode%
  }%
}%

\NewDocumentCommand\ShikakuCreation{}{%
  \ifluatex%
  \mplibforcehmode
  \xdef\PfCNomShikaku{\useKV[Shikaku]{Nom}}%
  \begin{mplibcode}[\PfCNomShikaku]%
    Taillehor=\useKV[Shikaku]{TailleHor};
    Taillever=\useKV[Shikaku]{TailleVer};
    
    TailleMaxh=\useKV[Shikaku]{TailleHorMax};
    TailleMaxv=\useKV[Shikaku]{TailleVerMax};

    u:=5mm;
    
    path R[],S[],RIso[],SIso[];
    numeric RAire[],SAire[];
    numeric RHor[],RVer[],SHor[],SVer[],RAngle[],SAngle[];
    
    vardef RetiensLesInfosGenerales(expr depart,Lhor,Lver)=
    R[k]:=polygone(depart,depart+u*(Lhor,0),depart+u*(Lhor,-Lver),depart+u*(0,-Lver));
    RIso[k]=iso(depart,depart+u*(Lhor,0),depart+u*(Lhor,-Lver),depart+u*(0,-Lver));
    RHor[k]=Lhor;
    RVer[k]=Lver;
    RAire[k]=Lhor*Lver;
    RAngle[k]=ANGLE;
    enddef;
    
    vardef RetiensLesInfosPourLaSuite(expr depart,Lhor,Lver)=
    if Lver>0:
    S[k]=polygone(depart,depart+u*(Lhor,0),depart+u*(Lhor,-Lver),depart+u*(0,-Lver));
    SIso[k]=iso(depart,depart+u*(Lhor,0),depart+u*(Lhor,-Lver),depart+u*(0,-Lver));
    SHor[k]=Lhor;
    SVer[k]=Lver;
    SAire[k]=Lhor*Lver;
    SAngle[k]=ANGLE;
    fi;
    enddef;
    
    vardef RectangleAuto(expr aa,bb)=
    save $;
    path $;
    $=polygone(aa,aa+(xpart(bb-aa),0),bb,bb-(xpart(bb-aa),0));
    $
    enddef;

    pair T[];
    
    pair A[],B[];
  
    A1=u*(0,Taillever);
    A2-A1=u*(Taillehor,0);
    A4=(0,0);
    A3-A4=A2-A1;
    
    nbh=Taillehor div TailleMaxh;
    nbv=Taillever div TailleMaxv;
    
    p:=0;
    for l=0 upto nbv:
    for k=0 upto nbh:
    p:=p+1;
    T[p]=A1+u*(k*TailleMaxh,-l*TailleMaxv);
    if k<nbh:
    Hor[p]=TailleMaxh;
    else:
    Hor[p]=Taillehor mod TailleMaxh;
    fi;
    if l<nbv:
    Ver[p]=TailleMaxv;
    else:
    Ver[p]=Taillever mod TailleMaxv;
    fi;
    endfor;
    endfor;
    
    Total:=p;
    
    vardef DecomposeUnRectangle(expr BASEA,THOR,TVER)=
    TMAXV:=TVER;
    pair BBASEA;
    BTHOR:=THOR;
    BTVER:=TVER;
    BBASEA:=BASEA;
    forever:
    k:=k+1;
    hor:=floor(1+uniformdeviate(BTHOR));
    ver:=floor(1+uniformdeviate(BTVER));
    RetiensLesInfosGenerales(BBASEA,hor,ver);
    RetiensLesInfosPourLaSuite(BBASEA+u*(0,-ver),hor,TMAXV-ver);
    BTHOR:=BTHOR-hor;
    BTVER:=ver;
    BBASEA:=BBASEA+u*(hor,0);
    exitif BTHOR<1;
    endfor;
    enddef;
    
    k=0;
    for l=1 upto Total:
    if uniformdeviate(1)>0.5:
    Angle[l]=180;
    ANGLE:=Angle[l];
    else:
    Angle[l]=0;
    ANGLE:=Angle[l];
    fi;
    DecomposeUnRectangle(T[l],Hor[l],Ver[l]);
    endfor;
    Totalk=k;

    picture RTEXT[],STEXT[];
    
    drawoptions(withpen pencircle scaled 2);
    for n=1 upto Totalk:
    trace R[n] withcolor white;
    RTEXT[n]=image(
    label(decimal(RAire[n]),(0,0));
    );
    RTEXT[n]:=image(
    trace rotation(RTEXT[n],center RTEXT[n],RAngle[n]) shifted ((point(0) of R[n]) shifted(u*(0.5+floor(uniformdeviate(RHor[n])),-0.5-floor(uniformdeviate(RVer[n])))));
    );
    trace RTEXT[n];
    endfor;
    for n=1 upto Totalk:
    if unknown S[n]:
    else:
    trace S[n] withcolor white;
    STEXT[n]=image(
    label(decimal(SAire[n]),(0,0));
    );
    STEXT[n]:=image(
    trace rotation(STEXT[n],center STEXT[n],SAngle[n]) shifted ((point(0) of S[n]) shifted(u*(0.5+floor(uniformdeviate(SHor[n])),-0.5-floor(uniformdeviate(SVer[n])))));
    );
    trace STEXT[n];
    fi;
    endfor;
    clip currentpicture to polygone(A1,A2,A3,A4);
    
    picture CasesAvant[],CasesApres[];
    CasesAvant[0]=currentpicture;
    
    currentpicture:=nullpicture;

    drawoptions(withpen pencircle scaled 2);
    for n=1 upto Totalk:
    trace R[n];
    trace RTEXT[n];
    endfor;
    for n=1 upto Totalk:
    if unknown S[n]:
    else:
    trace S[n];
    trace STEXT[n];
    fi;
    endfor;
    clip currentpicture to polygone(A1,A2,A3,A4);
    CasesApres[0]:=currentpicture;
    
    currentpicture:=nullpicture;
        
    for l=1 upto Total:
    CasesAvant[l]=image(%
    trace CasesAvant[0];
    clip currentpicture to polygone(T[l],T[l]+u*(Hor[l],0),T[l]+u*(Hor[l],-Ver[l]),T[l]+u*(0,-Ver[l]));
    );
    currentpicture:=nullpicture;
    CasesApres[l]=image(
    trace CasesApres[0];
    clip currentpicture to polygone(T[l],T[l]+u*(Hor[l],0),T[l]+u*(Hor[l],-Ver[l]),T[l]+u*(0,-Ver[l]));
    );
    currentpicture:=nullpicture;
    endfor;
    
    for l=1 upto Total:
    trace rotation(CasesAvant[l],center CasesAvant[l],Angle[l]);
    endfor;
    
    drawoptions(withpen pencircle scaled 2);
    trace RectangleAuto(A1,A3);
    drawoptions();
    for k=1 upto Taillehor-1:
    trace (k/Taillehor)[A4,A3]--(k/Taillehor)[A1,A2];
    endfor;
    
    for k=1 upto Taillever-1:
    trace (k/Taillever)[A4,A1]--(k/Taillever)[A3,A2];
    endfor;
  \end{mplibcode}
  \fi%
}%

\NewDocumentCommand\ShikakuCreationSolution{}{%
  \ifluatex%
  \mplibforcehmode
  \xdef\PfCNomShikaku{\useKV[Shikaku]{Nom}}%
  \begin{mplibcode}[\PfCNomShikaku]%
    for l=1 upto Total:
    trace rotation(CasesApres[l],center CasesApres[l],Angle[l]);
    endfor;
    drawoptions(withpen pencircle scaled 2);
    trace RectangleAuto(A1,A3);
    drawoptions();
    for k=1 upto Taillehor-1:
    trace (k/Taillehor)[A4,A3]--(k/Taillehor)[A1,A2];
    endfor;
    
    for k=1 upto Taillever-1:
    trace (k/Taillever)[A4,A1]--(k/Taillever)[A3,A2];
    endfor;
  \end{mplibcode}
  \fi%
}%