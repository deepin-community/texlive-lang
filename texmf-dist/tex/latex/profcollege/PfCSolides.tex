%%%
% Solide et sections
%%%
\def\filedateSolides{2024/08/20}%
\def\fileversionSolides{0.1a}%
\message{-- \filedateSolides\space v\fileversionSolides}%
%
\setKVdefault[ClesSolides]{Nom=cube,Aretes,Sommets,Pointilles,Largeur=1.5,Hauteur=1,Profondeur=0.75,ListeSommets={A,B,C,D,E,F,G,H},Rho=1500,Phi=30,Theta=20,Distance=50,Code=false,Couleur=white,Anglex=0,Arete=1,RayonCone=1,HauteurCone=2,HauteurPrisme=2,SommetsPrisme=6,HauteurPyramide=2,SommetsPyramide=5,Reguliere=false,DecalageSommet={(0,0,0)},RayonCylindre=1,HauteurCylindre=2,RayonSphere=1,AutoLabel=true,Sections=false,CoefSection=0.3,Axes=false,PointsSection={M,N,O,P},RemplisSection=false,ObjetSection={0.5,E,H,0.25,F,G,G,B}}%
\defKV[ClesSolides]{Traces=\setKV[ClesSolides]{Code}}%
\defKV[ClesSolides]{Section=\setKV[ClesSolides]{Sections}}%
\defKV[ClesSolides]{CouleurSection=\setKV[ClesSolides]{RemplisSection}}%

\newcommand\MPSolidePrisme[4]{%
  \ifluatex
    \mplibnumbersystem{double}
    \mplibforcehmode
    \begin{mplibcode}
    input PfCSolid;
    
    boolean Pointilles,TraceArete,NommeSommets,Code,AutoLabel,Sections,RemplisSection;
    AutoLabel:=\useKV[ClesSolides]{AutoLabel};
    Pointilles=\useKV[ClesSolides]{Pointilles};
    TraceArete=\useKV[ClesSolides]{Aretes};
    NommeSommets=\useKV[ClesSolides]{Sommets};
    Code=\useKV[ClesSolides]{Code};
    Sections:=\useKV[ClesSolides]{Sections};
    RemplisSection=\useKV[ClesSolides]{RemplisSection};
    if RemplisSection:color CouleurSection;CouleurSection=\useKV[ClesSolides]{CouleurSection};fi;

    color PfCOutColor;
    PfCOutColor=Cornsilk;%\useKV[ClesSolides]{Couleur};
    
    string Section;
    if Sections:Section=\useKV[ClesSolides]{Section};fi;
    
    %Initialisation
    PfCRho=\useKV[ClesSolides]{Rho};
    PfCPhi=\useKV[ClesSolides]{Phi};
    PfCTheta=\useKV[ClesSolides]{Theta};
    PfCDistance=\useKV[ClesSolides]{Distance};
    anglerotationx=\useKV[ClesSolides]{Anglex};
    hauteurprisme=\useKV[ClesSolides]{HauteurPrisme};
    nombresommets=\useKV[ClesSolides]{SommetsPrisme};
    Initialisation(PfCRho,PfCPhi,PfCTheta,PfCDistance);
    typetrace:="3D";
    typerepre:="persp";
    eclairage:=false;
    incolor:=white;
    outcolor:=white;
    if Pointilles=false:
    pointilles:="non";
    fi;

    Ferme1:=true;
    
    color A,B,C,D,E,F,G,H;
    color Sommet[];

    NbS:=nombresommets;

    color TS[];
    
    ecartangle=360/(NbS/2);
    for k=1 upto (NbS/2):
    repereangle:=(k-2)*ecartangle+(ecartangle*0.25+uniformdeviate(floor(ecartangle/2)));
    TS[k]:=(cosd(repereangle),-sind(repereangle),0);
    endfor;
    TS[NbS/2+1]:=TS[1];

    for k=1 upto (NbS/2):
    TS[NbS/2+1+k]:=TS[k]+(0,0,hauteurprisme);
    endfor;
    TS[NbS+2]:=TS[NbS/2+2];

    angx:=anglerotationx;
    for k=1 upto (NbS+2):
    TS[k]:=RotXYZ(TS[k]);
    endfor;
    angx:=0;
    
    ObjetNew1(%
    TS1, for k=2 upto (NbS+1):TS[k], endfor TS[NbS+2]
    )(%
    NbS/2, for p_=1 upto NbS/2:p_, endfor
    NbS/2, for p_=NbS/2+1 downto NbS+2:p_, endfor
    for p=1 upto NbS/2:
    4,p,p+1,NbS/2+1+p+1,NbS/2+1+p,%
    endfor
    );
    %AffichageObjet1;

    NF:=(NbS/2+2);
    Fc[100*(NbS/2+1)]:=NbS/2;
    for k=1 upto NbS/2:
    Fc[100*(NbS/2+1)+k]:=k;
    endfor;
    Fc[100*(NbS/2+2)]:=NbS/2;
    for k=1 upto NbS/2:
    Fc[100*(NbS/2+2)+k]:=NbS+2-k;
    endfor;
    for l=1 upto NbS/2:
    Fc[l*100]:=4;
    %for k=1 upto 4:
    Fc[l*100+1]:=l;Fc[l*100+4]:=l+1;Fc[l*100+3]:=NbS/2+1+l+1;Fc[l*100+2]:=NbS/2+1+l;
    %Fc[101]:=1;Fc[102]:=2;Fc[103]:=6;Fc[104]:=5;
    %endfor;
    endfor;
    DefinirSommets(#1);
    DessineObjet;
%    picture Depart;
%    Depart=currentpicture;
%    currentpicture:=nullpicture;
%    for k=1 upto 8:
%    dotlabel.top(decimal(k),Projette(TS[k]));
%    endfor;
    u:=1cm;
    if Code:
    \useKV[ClesSolides]{Traces};
    fi;
  \end{mplibcode}
  \mplibnumbersystem{scaled}
  \fi
}

\newcommand\MPSolideCylindre[3]{%
  \ifluatex
    \mplibnumbersystem{double}
    \mplibforcehmode
    \begin{mplibcode}
    input PfCSolid;
    
    boolean NommeSommets,Code,Sections,Axes,RemplisSection;
    NommeSommets=\useKV[ClesSolides]{Sommets};
    Code=\useKV[ClesSolides]{Code};
    Sections:=\useKV[ClesSolides]{Sections};
    Axes:=\useKV[ClesSolides]{Axes};
    RemplisSection=\useKV[ClesSolides]{RemplisSection};
    if RemplisSection:color CouleurSection;CouleurSection=\useKV[ClesSolides]{CouleurSection};fi;
    %Initialisation
    PfCRho=\useKV[ClesSolides]{Rho};
    PfCPhi=\useKV[ClesSolides]{Phi};
    PfCTheta=\useKV[ClesSolides]{Theta};
    PfCDistance=\useKV[ClesSolides]{Distance};
    CoefSection=\useKV[ClesSolides]{CoefSection};
    anglerotationx=\useKV[ClesSolides]{Anglex};
    rayoncylindre=\useKV[ClesSolides]{RayonCylindre};
    hauteurcylindre=\useKV[ClesSolides]{HauteurCylindre};

    string Section;
    if Sections:Section=\useKV[ClesSolides]{Section};fi;
        
    color PfCOutColor;
    PfCOutColor=\useKV[ClesSolides]{Couleur};
    
    vardef BaseCylindre(text t)=
    save Cer;
    picture Cer;
    save n;
    n:=0;
    for p_=t:
    if color p_:
    n:=n+1;
    _T[n]:=p_;
    fi;
    endfor;
    pascercle:=5;
    color ptcer[];
    for k=0 step pascercle until (360+pascercle) :
    ptcer[k div pascercle]-_T[1]=Distance(_T[1],_T[2])*((_T[4]-_T[3])*cosd(k)/Distance(_T[3],_T[4])+(_T[5]-_T[3])*sind(k)/Distance(_T[3],_T[5]));
    endfor;
    %
    color ddcer,cccer;
    nbpointilles:=0;
    nbtraces:=0;
    path PathSection[];
    nbpathsection:=0;
    Cer=image(%
    for k=0 step pascercle until 360:
    ddcer:=Oeil-ptcer[k div pascercle];
    cccer:=-Normal(ptcer[k div pascercle],ptcer[(k+pascercle) div pascercle],ptcer[k div pascercle]+Sommet1-Sommet2);
    if (ProduitScalaire(ddcer,cccer)>=0):
    nbtraces:=nbtraces+1;
    if (nbtraces=1) and (nbpointilles>0):
    draw chemin(ptcer[k div pascercle],ptcer[k div pascercle]+Sommet1-Sommet2);
    fi;
    nbpathsection:=nbpathsection+1;
    PathSection[nbpathsection]=chemin(ptcer[k div pascercle],ptcer[(k+pascercle) div pascercle]);
    draw chemin(ptcer[k div pascercle],ptcer[(k+pascercle) div pascercle]);
    draw chemin(ptcer[k div pascercle]+Sommet1-Sommet2,ptcer[(k+pascercle) div pascercle]+Sommet1-Sommet2);
    else:
    nbpointilles:=nbpointilles+1;
    nbtraces:=0;
    if (nbpointilles=1):
    draw chemin(ptcer[k div pascercle],ptcer[k div pascercle]+Sommet1-Sommet2);
    fi;
    if k mod (2*pascercle)=0:
    nbpathsection:=nbpathsection+1;
    PathSection[nbpathsection]=chemin(ptcer[k div pascercle],ptcer[(k+pascercle) div pascercle]);
    draw chemin(ptcer[k div pascercle],ptcer[(k+pascercle) div pascercle]);
    draw chemin(ptcer[k div pascercle]+Sommet1-Sommet2,ptcer[(k+pascercle) div pascercle]+Sommet1-Sommet2);
    fi;
    fi;
    endfor;
    );
    Cer
    enddef;

    Figure(-10u,-10u,10u,10u);
    Initialisation(PfCRho,PfCPhi,PfCTheta,PfCDistance);
%    Initialisation(1500,30,20,50);
    typetrace:="3D";
    typerepre:="persp";
    eclairage:=false;
    nb:=36;%36
    subh:=1;%
    incolor:=white;
    outcolor:=PfCOutColor;
    Ferme1:=true;
    angx:=anglerotationx;
    Objetcylindre1("r="&decimal(rayoncylindre),"h="&decimal(hauteurcylindre));
    eclairage:=false;
%    traits:=false;
%    AffichageObjet1;

    color O,S,A,M,B,C,D,Sommet[];
    Sommet2=(0,0,0);
    Sommet1-Sommet2=(0,0,hauteurcylindre);
    Sommet3-Sommet2=rayoncylindre*(cosd(0),-sind(0),0);
    Sommet4-Sommet2=rayoncylindre*(cosd(90),-sind(90),0);
    color MSection[];
    MSection[1]-Sommet2=rayoncylindre*(CoefSection,sin(arccos(CoefSection)),0);
    MSection[7]-Sommet2=rayoncylindre*(CoefSection+0.01,sin(arccos(CoefSection+0.01)),0);
    MSection[2]-Sommet2=rayoncylindre*(CoefSection,-sin(arccos(CoefSection)),0);
    MSection[3]-MSection[2]=Sommet1-Sommet2;
    MSection[4]-MSection[1]=Sommet1-Sommet2;
    MSection[5]-Sommet2=CoefSection[Sommet1,Sommet2];
    MSection[6]-MSection[5]=MSection[2]-MSection[1];
    Sommet1:=RotXYZ(Sommet1);
    Sommet2:=RotXYZ(Sommet2);
    Sommet3:=RotXYZ(Sommet3);
    Sommet4:=RotXYZ(Sommet4);
    MSection[1]:=RotXYZ(MSection[1]);
    MSection[2]:=RotXYZ(MSection[2]);
    MSection[3]:=RotXYZ(MSection[3]);
    MSection[4]:=RotXYZ(MSection[4]);
    MSection[5]:=RotXYZ(MSection[5]);
    MSection[6]:=RotXYZ(MSection[6]);
    MSection[7]:=RotXYZ(MSection[7]);

    trace BaseCylindre(Sommet2,Sommet3,Sommet2,Sommet3,Sommet4);
    currentpicture:=nullpicture;
    
    if Sections:
    if Section="parallele":
    if RemplisSection:
    fill polygone(MSection[1],MSection[2],MSection[3],MSection[4]) withcolor CouleurSection;
    fi;
    color dcer,ccer;
    invnormale:=if CoefSection<=0:-1 else:1 fi;
    %1er côté et 3eme côté : face "basse" et face "haute"
    dcer:=Oeil-Sommet3;
    ccer:=-Normal(Sommet3,MSection1,MSection2);
    draw chemin(MSection1,MSection2) if (ProduitScalaire(dcer,ccer)<0):dashed evenly fi;
    draw chemin(MSection3,MSection4) if (ProduitScalaire(dcer,ccer)>0):dashed evenly fi;
    % 2eme côté et 4eme coté
    dcer:=Oeil-MSection7;
    ccer:=-Normal(MSection7,MSection4,MSection1);
    draw chemin(MSection4,MSection1) if (ProduitScalaire(dcer,ccer)<0):dashed evenly fi;
    draw chemin(MSection2,MSection3) if (ProduitScalaire(dcer,ccer)>0):dashed evenly fi;
    else:
    if RemplisSection:
    fill (PathSection[1] for p_=2 upto nbpathsection:--PathSection[p_] endfor--cycle) shifted(Projette(CoefSection[Sommet1,Sommet2]-Sommet2)) withcolor CouleurSection;
    fi;
    for p_=1 upto nbpathsection:
    draw (PathSection[p_] shifted(Projette(CoefSection[Sommet1,Sommet2]-Sommet2)));
    endfor;
    fi;
    fi;
    
    trace BaseCylindre(Sommet2,Sommet3,Sommet2,Sommet3,Sommet4);

    color dcer,ccer;
    dcer:=Oeil-Sommet2;
    ccer:=Normal(Sommet2,Sommet3,Sommet4);
    if (ProduitScalaire(dcer,ccer)>=0):
    draw Cercles(Sommet2,Sommet3,Sommet2,Sommet3,Sommet4);
    fi;
    dcer:=Oeil-Sommet1;
    ccer:=Normal(Sommet1,Sommet4+Sommet1-Sommet2,Sommet3+Sommet1-Sommet2);
    if (ProduitScalaire(dcer,ccer)>=0):
    draw Cercles(Sommet1,Sommet3+Sommet1-Sommet2,Sommet1,Sommet3+Sommet1-Sommet2,Sommet4+Sommet1-Sommet2);
    fi;

    vardef EcrireSommets(text t)=
    nb:=0;
    for p_=t:
    nb:=nb+1;
    if bluepart(Sommet1)>bluepart(Sommet2):
    if nb=1:
    label.ulft(TEX(p_),Projette(Sommet1));%p_
    elseif nb=2:
    dotlabel.rt(TEX(p_),Projette(Sommet2));%p_
    fi;
    else:
    if nb=1:
    label.bot(TEX(p_),Projette(Sommet1));
    elseif nb=2:
    dotlabel.ulft(TEX(p_),Projette(Sommet2));
    fi;
    fi;
    endfor;
    enddef;

    DefinirSommets(#1);
    DefinirSommetsSection(#3);
    
    if Axes:
    draw segment(1.5[Sommet1,Sommet2],1.5[Sommet2,Sommet1]) dashed dashpattern(on6 off3 on3 off 3);
    remplis (fullcircle scaled 1mm) shifted(Projette(Sommet1));
    remplis (fullcircle scaled 1mm) shifted(Projette(Sommet2));
    fi;
    
    if Code:
    \useKV[ClesSolides]{Traces};
    fi;
    
  \end{mplibcode}
  \mplibnumbersystem{scaled}
  \fi
}

\newcommand\MPSolidePyramide[4]{%
  \ifluatex
    \mplibnumbersystem{double}
    \mplibforcehmode
    \begin{mplibcode}

    input PfCSolid;
    
    boolean Pointilles,TraceArete,NommeSommets,Code,Reguliere,AutoLabel,Sections,RemplisSection;
    Sections:=\useKV[ClesSolides]{Sections};
    RemplisSection=\useKV[ClesSolides]{RemplisSection};
    if RemplisSection:color CouleurSection;CouleurSection=\useKV[ClesSolides]{CouleurSection};fi;
    AutoLabel:=\useKV[ClesSolides]{AutoLabel};
    Pointilles=\useKV[ClesSolides]{Pointilles};
    TraceArete=\useKV[ClesSolides]{Aretes};
    NommeSommets=\useKV[ClesSolides]{Sommets};
    Code=\useKV[ClesSolides]{Code};
    Reguliere:=\useKV[ClesSolides]{Reguliere};
    % Initialisation
    PfCRho=\useKV[ClesSolides]{Rho};
    PfCPhi=\useKV[ClesSolides]{Phi};
    PfCTheta=\useKV[ClesSolides]{Theta};
    PfCDistance=\useKV[ClesSolides]{Distance};
    CoefSection=\useKV[ClesSolides]{CoefSection};
    anglerotationx=\useKV[ClesSolides]{Anglex};
    hauteurpyramide=\useKV[ClesSolides]{HauteurPyramide};
    nombresommets=\useKV[ClesSolides]{SommetsPyramide};
    color PfCOutColor,DecalageSommet;
    PfCOutColor=\useKV[ClesSolides]{Couleur};
    DecalageSommet=#3;%if Reguliere:(0,0,0) else: #3 fi;
    
    Figure(-10u,-10u,10u,10u);
    Initialisation(PfCRho,PfCPhi,PfCTheta,PfCDistance);
    typetrace:="3D";
    typerepre:="persp";
    if Pointilles=false:
    pointilles:="non";
    fi;

    Ferme1:=true;

    color O,A[],S;
    O=(0,0,0);
    S-O=(0,0,hauteurpyramide)+DecalageSommet;

    NbS:=nombresommets;
    Sommet1:=S;
    ecartangle=360/(NbS-1);
    for k=2 upto NbS:
    if Reguliere:
    repereangle:=(k-2)*ecartangle;
    else:
    repereangle:=(k-2)*ecartangle+(ecartangle*0.25+uniformdeviate(floor(ecartangle/2)));
    fi;
    Sommet[k]:=(cosd(repereangle),-sind(repereangle),0);
    endfor;
    Sommet[NbS+1]:=Sommet[2];

    color TS[];
    for k=1 upto (NbS+1):
    TS[k]=Sommet[k];
    endfor;
    color PiedHauteur;
    PiedHauteur=ProjectionsurPlan(Sommet1,Sommet2,Sommet3,Sommet4);
    eclairage:=false;
    traits:=false;
    outcolor:=PfCOutColor;
    angx:=anglerotationx;
    for k=1 upto (NbS+1):
    TS[k]:=RotXYZ(TS[k]);
    endfor;
    angx:=0;

    color MSection[];
    for p_=2 upto NbS:
    MSection[p_-1]=CoefSection[Sommet[1],Sommet[p_]];
    endfor;

    DefinirSommetsSection(#4);
    
    vardef SectionPyramide(expr fracsection)=
    save Section;
    picture Section;
    Section=image(%
    if RemplisSection:
    remplis polygone(fracsection[Sommet1,Sommet2],for p_=3 upto NbS-1:fracsection[Sommet1,Sommet[p_]],endfor fracsection[Sommet1,Sommet[NbS]]) withcolor CouleurSection;
    fi;
    color dcer,ccer;
    for p_=2 upto NbS:
    dcer:=Oeil-Sommet[p_];
    ccer:=-Normal(Sommet1,Sommet[p_],Sommet[p_+1]);
    if (ProduitScalaire(dcer,ccer)>=0):
    draw chemin(fracsection[Sommet1,Sommet[p_]],fracsection[Sommet1,Sommet[p_+1]]);
    else:
    draw chemin(fracsection[Sommet1,Sommet[p_]],fracsection[Sommet1,Sommet[p_+1]]) dashed evenly;
    fi;
    endfor;
    );
    Section
    enddef;
    
    ObjetNew1(%
    TS1, for k=2 upto NbS:TS[k], endfor TS[NbS+1]
    )(%
    NbS-1, for p_=2 upto NbS:p_, endfor 
    for p=2 upto NbS:
    3,1,p+1,p,%
    endfor);
    AffichageObjet1;

    Sommet[NbS+1]:=Sommet[2];

    if Sections:
    draw SectionPyramide(CoefSection);
    fi;
    
    NF:=NbS;%nombresommets;
    % Fc[100] est la base
    Fc[100]:=NbS-1;
    for k=2 upto NbS:
    Fc[100+k-1]:=k;
    endfor;
    for k=2 upto NF:
    Fc[k*100]:=3;
    Fc[k*100+1]:=1;
    Fc[k*100+2]:=k+1;
    Fc[k*100+3]:=k;
    endfor;
    DessineObjet;

    vardef EcrireSommets(text t)=
    if AutoLabel:
    nb:=0;
    for p_=t:
    nb:=nb+1;
    if nb<NbS+1:
    if nb>1:
    label(TEX(""&p_&""),1.1[Projette(Sommet1),Projette(Sommet[nb])]);
    else:
    label(TEX(p_),1.1[Projette(Sommet[2]),Projette(Sommet[1])]);
    fi;
    fi;
    endfor;
    fi;
    enddef;
     
    DefinirSommets(#1);

    if TraceArete=false:
    currentpicture:=nullpicture;    
    fi;
    if NommeSommets:
    EcrireSommets(#2);
    fi;
    u:=0.25u;
    marque_p:="croix";
    nbcroix=0;
    for p_=#1:
    nbcroix:=nbcroix+1;
    if nbcroix<NbS+1:
    pointe(p_);
    fi;
    endfor;
    u:=1cm;
    if Code:
    \useKV[ClesSolides]{Traces};
    fi;
  \end{mplibcode}
  \mplibnumbersystem{scaled}
  \fi
}

\newcommand\MPSolideCone[3]{%
  \ifluatex
    \mplibnumbersystem{double}
    \mplibforcehmode
    \begin{mplibcode}
    boolean NommeSommets,Code,Sections,Axes,RemplisSection;
    NommeSommets=\useKV[ClesSolides]{Sommets};
    Code=\useKV[ClesSolides]{Code};
    Axes:=\useKV[ClesSolides]{Axes};
    Sections=\useKV[ClesSolides]{Sections};
    RemplisSection=\useKV[ClesSolides]{RemplisSection};
    if RemplisSection:color CouleurSection;CouleurSection=\useKV[ClesSolides]{CouleurSection};fi;
    %Initialisation
    PfCRho=\useKV[ClesSolides]{Rho};
    PfCPhi=\useKV[ClesSolides]{Phi};
    PfCTheta=\useKV[ClesSolides]{Theta};
    PfCDistance=\useKV[ClesSolides]{Distance};
    CoefSection=\useKV[ClesSolides]{CoefSection};
    anglerotationx=\useKV[ClesSolides]{Anglex};
    rayoncone=\useKV[ClesSolides]{RayonCone};
    hauteurcone=\useKV[ClesSolides]{HauteurCone};
    
    color PfCOutColor;
    PfCOutColor=\useKV[ClesSolides]{Couleur};
    input PfCSolid;
    
    vardef BaseCone(text t)=
    save Cer;
    picture Cer;
    save n;
    n:=0;
    for p_=t:
    if color p_:
    n:=n+1;
    _T[n]:=p_;
    fi;
    endfor;
    pascercle:=5;
    color ptcer[];
    for k=0 step pascercle until (360+pascercle) :
    ptcer[k div pascercle]-_T[1]=Distance(_T[1],_T[2])*((_T[4]-_T[3])*cosd(k)/Distance(_T[3],_T[4])+(_T[5]-_T[3])*sind(k)/Distance(_T[3],_T[5]));
    endfor;
    %
    color ddcer,cccer;
    nbpointilles:=0;
    nbtraces:=0;
    Cer=image(%
    for k=0 step pascercle until 360:
    ddcer:=Oeil-ptcer[k div pascercle];
    cccer:=-Normal(Sommet1,ptcer[k div pascercle],ptcer[(k+pascercle) div pascercle]);
    if (ProduitScalaire(ddcer,cccer)>=0):
    nbtraces:=nbtraces+1;
    if (nbtraces=1) and (nbpointilles>0):
    draw chemin(ptcer[k div pascercle],Sommet1);
    fi;
    draw chemin(ptcer[k div pascercle],ptcer[(k+pascercle) div pascercle]);
    else:
    nbpointilles:=nbpointilles+1;
    nbtraces:=0;
    if (nbpointilles=1):
    draw chemin(ptcer[k div pascercle],Sommet1);
    fi;
    if k mod (2*pascercle)=0:
    draw chemin(ptcer[k div pascercle],ptcer[(k+pascercle) div pascercle]);
    fi;
    fi;
    endfor;
    );
    Cer
    enddef;

    Figure(-10u,-10u,10u,10u);
    Initialisation(PfCRho,PfCPhi,PfCTheta,PfCDistance);
    typetrace:="3D";
    typerepre:="persp";
    eclairage:=false;
    nb:=36;%36
    subh:=1;%
    incolor:=white;
    outcolor:=PfCOutColor;
    Ferme1:=true;
    angx:=anglerotationx;
    Objetcone1("r="&decimal(rayoncone),"h="&decimal(hauteurcone));
    traits:=false;
    AffichageObjet1;

    color Sommet[];%O,S,A,M,B,C,D,
    Sommet2=(0,0,0);
    Sommet1-Sommet2=(0,0,hauteurcone);
    Sommet3-Sommet2=rayoncone*(cosd(0),-sind(0),0);
    Sommet4-Sommet2=rayoncone*(cosd(90),-sind(90),0);
    Sommet1:=RotXYZ(Sommet1);
    Sommet3:=RotXYZ(Sommet3);
    Sommet4:=RotXYZ(Sommet4);

    color MSection[];
    MSection[1]=CoefSection[Sommet1,Sommet2];

    DefinirSommetsSection(#3);
    
    if Sections:
    if RemplisSection:
    remplis Cercles(CoefSection[Sommet1,Sommet2],CoefSection[Sommet1,Sommet3],CoefSection[Sommet1,Sommet2],CoefSection[Sommet1,Sommet3],CoefSection[Sommet1,Sommet4]) withcolor CouleurSection;
    fi;
    trace BaseCone(CoefSection[Sommet1,Sommet2],CoefSection[Sommet1,Sommet3],CoefSection[Sommet1,Sommet2],CoefSection[Sommet1,Sommet3],CoefSection[Sommet1,Sommet4]);
    fi;

    trace BaseCone(Sommet2,Sommet3,Sommet2,Sommet3,Sommet4);
    color dcer,ccer;
    dcer:=Oeil-Sommet2;
    ccer:=Normal(Sommet2,Sommet3,Sommet4);
    if (ProduitScalaire(dcer,ccer)>=0):
    draw Cercles(Sommet2,Sommet3,Sommet2,Sommet3,Sommet4);
    fi;

    vardef EcrireSommets(text t)=
    nb:=0;
    for p_=t:
    nb:=nb+1;
    if bluepart(Sommet1)>bluepart(Sommet2):
    if nb=1:
    label.ulft(TEX(p_),Projette(Sommet1));
    elseif nb=2:
    dotlabel.rt(TEX(p_),Projette(Sommet2));
    fi;
    else:
    if nb=1:
    label.bot(TEX(p_),Projette(Sommet1));
    elseif nb=2:
    dotlabel.ulft(TEX(p_),Projette(Sommet2));
    fi;
    fi;
    endfor;
    enddef;

    DefinirSommets(#1);

    if Axes:
    draw segment(1.5[Sommet1,Sommet2],1.5[Sommet2,Sommet1]) dashed dashpattern(on6 off3 on3 off 3);
    remplis (fullcircle scaled 1mm) shifted(Projette(Sommet1));
    remplis (fullcircle scaled 1mm) shifted(Projette(Sommet2));
    fi;

    if Code:
    \useKV[ClesSolides]{Traces};
    fi;

  \end{mplibcode}
  \mplibnumbersystem{scaled}
  \fi
}

\newcommand\MPSolidePave[5]{%
  \ifluatex
    \mplibnumbersystem{double}
    \mplibforcehmode
    \begin{mplibcode}
    input PfCSolid;
    
    boolean Pointilles,TraceArete,NommeSommets,Code,AutoLabel,Sections,RemplisSection;
    AutoLabel:=\useKV[ClesSolides]{AutoLabel};
    Pointilles=\useKV[ClesSolides]{Pointilles};
    TraceArete=\useKV[ClesSolides]{Aretes};
    NommeSommets=\useKV[ClesSolides]{Sommets};
    Code=\useKV[ClesSolides]{Code};
    Sections:=\useKV[ClesSolides]{Sections};
    RemplisSection=\useKV[ClesSolides]{RemplisSection};
    if RemplisSection:color CouleurSection;CouleurSection=\useKV[ClesSolides]{CouleurSection};fi;

    string Section;
    if Sections:Section=\useKV[ClesSolides]{Section};fi;
    
    %Initialisation
    PfCRho=\useKV[ClesSolides]{Rho};
    PfCPhi=\useKV[ClesSolides]{Phi};
    PfCTheta=\useKV[ClesSolides]{Theta};
    PfCDistance=\useKV[ClesSolides]{Distance};
    
    Figure(-10u,-10u,10u,10u);
    Initialisation(PfCRho,PfCPhi,PfCTheta,PfCDistance);
    typetrace:="3D";
    typerepre:="persp";
    if Pointilles=false:
    pointilles:="non";
    fi;
    color A,B,C,D,E,F,G,H;
    color Sommet[];
    trace Pave(A,B,C,D,E,F,G,H)(#3);
    picture Depart;
    Depart=currentpicture;
    currentpicture:=nullpicture;
    Sommet1=A;
    Sommet2=B;
    Sommet3=C;
    Sommet4=D;
    Sommet5=E;
    Sommet6=F;
    Sommet7=G;
    Sommet8=H;

    DefinirSommets(#1);
    
    color MSection[];
    numeric RetiensSection[];
    numretienssection:=0;
    color RetiensDefSection[];
    numretiensdefsection:=0;
    if Sections:
    if Section="arete":
    for p_=#5:
    if numeric p_:
    numretienssection:=numretienssection+1;
    RetiensSection[numretienssection]=p_;
    elseif color p_:
    numretiensdefsection:=numretiensdefsection+1;
    RetiensDefSection[numretiensdefsection]=p_;
    fi;    
    endfor;
    % On dessine la section
    MSection[1]=(RetiensSection[1])[RetiensDefSection[1],RetiensDefSection[2]];
    MSection[2]=(RetiensSection[2])[RetiensDefSection[3],RetiensDefSection[4]];
    MSection[3]-MSection[2]=RetiensDefSection[6]-RetiensDefSection[5];
    MSection[4]-MSection[1]=RetiensDefSection[6]-RetiensDefSection[5];
    if RemplisSection:
    fill polygone(MSection1,MSection2,MSection3,MSection4) withcolor CouleurSection;
    fi;
%    DotLabel.top(TEX("M1"),MSection[1]);
%    DotLabel.top(TEX("M2"),MSection[2]);
%    DotLabel.top(TEX("M3"),MSection[3]);
%    DotLabel.top(TEX("M4"),MSection[4]);
%    if RemplisSection:
%    fill polygone(MSection1,MSection2,MSection3,MSection4) withcolor CouleurSection;
%    fi;
%    trace polygone(MSection1,MSection2,MSection3,MSection4);
    color dcer,ccer;
    % 1er cote -> ok
    %DotLabel("",RetiensDefSection[4]);
    dcer:=Oeil-RetiensDefSection[4];
    ccer:=Normal(RetiensDefSection[4],RetiensDefSection[2],RetiensDefSection[1]);
    draw chemin(MSection1,MSection2) if (ProduitScalaire(dcer,ccer)<0):dashed evenly fi;
    % 2eme cote
    dcer:=Oeil-RetiensDefSection[4];
    ccer:=Normal(RetiensDefSection[4],RetiensDefSection[3],RetiensDefSection[4]+RetiensDefSection[6]-RetiensDefSection[5]);
    draw chemin(MSection2,MSection3) if (ProduitScalaire(dcer,ccer)<0):dashed evenly fi;
    % 3eme cote
    dcer:=Oeil-RetiensDefSection[4];
    ccer:=Normal(RetiensDefSection[4],RetiensDefSection[2],RetiensDefSection[1]);
    draw chemin(MSection3,MSection4) if (ProduitScalaire(dcer,ccer)>=0):dashed evenly fi;
    % 4eme cote
    dcer:=Oeil-RetiensDefSection[4];
    ccer:=Normal(RetiensDefSection[4],RetiensDefSection[3],RetiensDefSection[4]+RetiensDefSection[6]-RetiensDefSection[5]);
    draw chemin(MSection4,MSection1) if (ProduitScalaire(dcer,ccer)>=0):dashed evenly fi;
    else:
    %face 
    for p_=#5:
    if numeric p_:
    numretienssection:=numretienssection+1;
    RetiensSection[numretienssection]=p_;
    elseif color p_:
    numretiensdefsection:=numretiensdefsection+1;
    RetiensDefSection[numretiensdefsection]=p_;
    fi;    
    endfor;
    % On dessine la section
    MSection[1]=(RetiensSection[1])[RetiensDefSection[1],RetiensDefSection[2]];
    MSection[2]-MSection[1]=RetiensDefSection[4]-RetiensDefSection[3];
    MSection[3]-MSection[2]=RetiensDefSection[5]-RetiensDefSection[4];
    MSection[4]-MSection[1]=MSection[3]-MSection[2];
%    DotLabel.top(TEX("M1"),MSection[1]);
%    DotLabel.top(TEX("M2"),MSection[2]);
%    DotLabel.top(TEX("M3"),MSection[3]);
%    DotLabel.top(TEX("M4"),MSection[4]);
    if RemplisSection:
    fill polygone(MSection1,MSection2,MSection3,MSection4) withcolor CouleurSection;
    fi;
    color dcer,ccer;
    % 1er cote -> ok
    dcer:=Oeil-RetiensDefSection[4];
    ccer:=Normal(RetiensDefSection[4],RetiensDefSection[2],RetiensDefSection[1]);
    draw chemin(MSection1,MSection2) if (ProduitScalaire(dcer,ccer)<0):dashed evenly fi;
    % 2eme cote
    dcer:=Oeil-RetiensDefSection[4];
    ccer:=Normal(RetiensDefSection[5],RetiensDefSection[4],RetiensDefSection[5]+RetiensDefSection[1]-RetiensDefSection[2]);
    draw chemin(MSection2,MSection3) if (ProduitScalaire(dcer,ccer)<0):dashed evenly fi;
    % 3eme cote
    dcer:=Oeil-RetiensDefSection[4];
    ccer:=Normal(RetiensDefSection[4],RetiensDefSection[2],RetiensDefSection[1]);
    draw chemin(MSection3,MSection4) if (ProduitScalaire(dcer,ccer)>=0):dashed evenly fi;
    % 4eme cote
    dcer:=Oeil-RetiensDefSection[4];
    ccer:=Normal(RetiensDefSection[5],RetiensDefSection[4],RetiensDefSection[5]+RetiensDefSection[1]-RetiensDefSection[2]);
    draw chemin(MSection4,MSection1) if (ProduitScalaire(dcer,ccer)>=0):dashed evenly fi;
    fi;
    %On trace la section considérée
    %trace polygone(MSection1,MSection2,MSection3,MSection4);
    fi;

    % trace Pave(A,B,C,D,E,F,G,H)(#3);
    trace Depart;
    
    DefinirSommetsSection(#4);
    
    if TraceArete=false:
    currentpicture:=nullpicture;    
    fi;
    if NommeSommets:
    EcrireSommetsPave(#2);
    fi;
    u:=0.25u;
    marque_p:="croix";
    nbcroix=0;
    for p_=#1:
    nbcroix:=nbcroix+1;
    if nbcroix=4:
    if Pointilles:
    pointe(p_);
    fi;
    else:
    pointe(p_);
    fi;
    endfor;
    u:=1cm;
    if Code:
    \useKV[ClesSolides]{Traces};
    fi;
  \end{mplibcode}
  \mplibnumbersystem{scaled}
  \fi
}

\newcommand\MPSolideSphere[3]{%
  \ifluatex
    \mplibnumbersystem{double}
  \mplibforcehmode
  \begin{mplibcode}
    boolean NommeSommets,Code,Sections,Axes,RemplisSection;
    NommeSommets=\useKV[ClesSolides]{Sommets};
    Code=\useKV[ClesSolides]{Code};
    Axes:=\useKV[ClesSolides]{Axes};
    Sections=\useKV[ClesSolides]{Sections};
    RemplisSection=\useKV[ClesSolides]{RemplisSection};
    if RemplisSection:color CouleurSection;CouleurSection=\useKV[ClesSolides]{CouleurSection};fi;
    %Initialisation
    PfCRho=\useKV[ClesSolides]{Rho};
    PfCPhi=0;%\useKV[ClesSolides]{Phi};
    PfCTheta=10;%\useKV[ClesSolides]{Theta};
    PfCDistance=\useKV[ClesSolides]{Distance};
    CoefSection=\useKV[ClesSolides]{CoefSection};
    anglerotationx=\useKV[ClesSolides]{Anglex};
    rayonsphere=\useKV[ClesSolides]{RayonSphere};
    
    color PfCOutColor;
    PfCOutColor=\useKV[ClesSolides]{Couleur};
    input PfCSolid;
    
    vardef BaseSphere=
    save Cer;
    picture Cer;
    color PtInter.iso,PtInter[];
    vardef Famille(expr u,v)=(R*(cos(u)*cos(v),cos(u)*sin(v),sin(u))) enddef;
    subh:=72;
    Cer=image(%
    if Sections:
    if RemplisSection:fill (Cercles(Image((0,0,0)),Image((0,1,0)),Image((0,0,0)),Image((0,1,0)),Image((-1,0,0))) scaled (sqrt(1-(abs(CoefSection+0.02))**2))) shifted(Projette(MSection1-Sommet1)) withcolor CouleurSection; fi;
    fi;
    %Equateur
    umin:=0; umax:=pi/nb; upas:=pi/nb;
    vmin:=0; vpas:=2*pi/subh; vmax:=2*pi;
    for l=0 upto subh-1:
    PtInter[1]:=Image(Famille(umin+(0+1)*upas,vmin+l*vpas));
    PtInter[2]:=Image(Famille(umin+0*upas,vmin+l*vpas));
    PtInter[3]:=Image(Famille(umin+0*upas,vmin+(l+1)*vpas));
    PtInter[4]:=Image(Famille(umin+(0+1)*upas,vmin+(l+1)*vpas));
    PtInter.iso:=(PtInter[1]+PtInter[2]+PtInter[3]+PtInter[4])/4;
    if ProduitScalaire(Oeil-PtInter.iso,Normal(PtInter.iso,PtInter[4],PtInter[1]))>=0:
    if Sections:
    draw (segment(PtInter[3],PtInter[2]) scaled (sqrt(1-(abs(CoefSection+0.02))**2))) shifted(Projette(MSection1-Sommet1));
    fi;
    draw segment(PtInter[3],PtInter[2]);
    else:
    if l mod 2=0:
    if Sections:
    draw (segment(PtInter[3],PtInter[2]) scaled (sqrt(1-(abs(CoefSection+0.02))**2))) shifted(Projette(MSection1-Sommet1));
    fi;
    draw segment(PtInter[3],PtInter[2]);
    fi;
    fi;
    endfor;
    
    % "Exterieur"
    nb:=72;
    subh:=18;
    umin:=-pi; umax:=pi; upas:=2*pi/nb;
    vpas:=2*pi/subh; vmin:=pi/2; vmax:=2*pi;
    for k=0 upto nb-1:
    PtInter[1]:=Image(Famille(umin+(k+1)*upas,vmin+0*vpas));
    PtInter[2]:=Image(Famille(umin+k*upas,vmin+0*vpas));
    PtInter[3]:=Image(Famille(umin+k*upas,vmin+(0+1)*vpas));
    PtInter[4]:=Image(Famille(umin+(k+1)*upas,vmin+(0+1)*vpas));
    PtInter.iso:=(PtInter[1]+PtInter[2]+PtInter[3]+PtInter[4])/4;
    draw segment(PtInter[2],PtInter[1]);
    endfor;
    );
    Cer
    enddef;

    Figure(-10u,-10u,10u,10u);
    Initialisation(PfCRho,PfCPhi,PfCTheta,PfCDistance);
    typetrace:="3D";
    typerepre:="persp";
    eclairage:=false;
    nb:=24;%36
    subh:=36;%
    incolor:=blue;
    outcolor:=white;%PfCOutColor;
    Ferme1:=true;
    angx:=anglerotationx;
    Objetsphere1("R="&decimal(rayonsphere));
    % traits:=false;
    % AffichageObjet1;
    color Sommet[],SommetN,SommetS,SommetB;
    Sommet1=(0,0,0);
    Sommet2=(0,rayonsphere,0);
    Sommet3=(-rayonsphere,0,0);
    Sommet4=(0,0,rayonsphere);
    SommetN=Sommet4;
    SommetS=(0,0,-rayonsphere);
    SommetB=(rayonsphere,0,0);
    Sommet2:=RotXYZ(Sommet2);
    Sommet3:=RotXYZ(Sommet3);
    Sommet4:=RotXYZ(Sommet4);
    SommetN:=RotXYZ(SommetN);
    SommetS:=RotXYZ(SommetS);
    SommetB:=RotXYZ(SommetB);
    color MSection[];
    MSection[1]=CoefSection[Sommet1,SommetN];
    MSection[2]=Image((0,sqrt(1-CoefSection**2),CoefSection));    

    trace BaseSphere;

    DefinirSommets(#1);
    DefinirSommetsSection(#3);

    if Axes:
    draw segment(1.25[SommetN,SommetS],1.25[SommetS,SommetN]) dashed dashpattern(on6 off3 on3 off3);
    remplis (fullcircle scaled 1mm) shifted(Projette(SommetN));
    remplis (fullcircle scaled 1mm) shifted(Projette(SommetS));
    fi;

    if Code:
    \useKV[ClesSolides]{Traces};
    fi;

  \end{mplibcode}
  \mplibnumbersystem{scaled}
  \fi
}

%%%%%%%%%%%%%%%%%%%%%%%%
\newcommand\MPSolideDemiSphere[3]{%
  \ifluatex
    \mplibnumbersystem{double}
  \mplibforcehmode
  \begin{mplibcode}
    boolean NommeSommets,Code,Sections,Axes,RemplisSection;
    NommeSommets=\useKV[ClesSolides]{Sommets};
    Code=\useKV[ClesSolides]{Code};
    Axes:=\useKV[ClesSolides]{Axes};
    %Initialisation
    PfCRho=\useKV[ClesSolides]{Rho};
    PfCPhi=0;%\useKV[ClesSolides]{Phi};
    PfCTheta=10;%\useKV[ClesSolides]{Theta};
    PfCDistance=\useKV[ClesSolides]{Distance};
    anglerotationx=\useKV[ClesSolides]{Anglex};
    rayonsphere=\useKV[ClesSolides]{RayonSphere};
    
    color PfCOutColor;
    PfCOutColor=\useKV[ClesSolides]{Couleur};
    input PfCSolid;

    Figure(-10u,-10u,10u,10u);
    Initialisation(PfCRho,PfCPhi,PfCTheta,PfCDistance);
    typetrace:="3D";
    typerepre:="persp";
    eclairage:=false;
    nb:=24;%36
    subh:=36;%
    incolor:=white;
    outcolor:=white;%PfCOutColor;
    % Ferme1:=false;
    creux:=true;
    angx:=anglerotationx;
    Objetdemispherebasse1("R="&decimal(rayonsphere));
    % traits:=false;
    AffichageObjet1;
    color Sommet[],SommetN,SommetS,SommetB;
    Sommet1=(0,0,0);
    Sommet2=(0,rayonsphere,0);
    Sommet3=(-rayonsphere,0,0);
    Sommet4=(0,0,rayonsphere);
    SommetN=Sommet4;
    SommetS=(0,0,-rayonsphere);
    SommetB=(rayonsphere,0,0);
    Sommet2:=RotXYZ(Sommet2);
    Sommet3:=RotXYZ(Sommet3);
    Sommet4:=RotXYZ(Sommet4);
    SommetN:=RotXYZ(SommetN);
    SommetS:=RotXYZ(SommetS);
    SommetB:=RotXYZ(SommetB);

    fill Cercles((0,0,0),(0,1,0),(0,0,0),(0,1,0),(-1,0,0)) withcolor white;
    trace Cercles((0,0,0),(0,1,0),(0,0,0),(0,1,0),(-1,0,0));

    DefinirSommets(#1);

    if Code:
    \useKV[ClesSolides]{Traces};
    fi;

  \end{mplibcode}
  \mplibnumbersystem{scaled}
  \fi
}
%%%


\def\UpdatetoksSolide#1\nil{\addtotok\toksolidelistesommets{"#1",}}
\newtoks\toksolidelistesommets%
\newtoks\toksolidelistepointssections%

\newcommand\Solide[1][]{%
  \useKVdefault[ClesSolides]%
  \setKV[ClesSolides]{#1}%
  \setsepchar{,}\ignoreemptyitems%
  \xdef\foo{\useKV[ClesSolides]{ListeSommets}}%
  \readlist*\ListeNomSommet{\foo}%\showitems\ListeNomSommet
  \reademptyitems%
  \toksolidelistesommets{}
  \foreachitem\compteur\in\ListeNomSommet{\expandafter\UpdatetoksSolide\compteur\nil}%\the\toksolidelistesommets%
  \IfStrEqCase{\useKV[ClesSolides]{Nom}}{%
    {cube}{%
      \MPSolidePave{\useKV[ClesSolides]{ListeSommets}}{\the\toksolidelistesommets}{\useKV[ClesSolides]{Arete},\useKV[ClesSolides]{Arete},\useKV[ClesSolides]{Arete}}{\useKV[ClesSolides]{PointsSection}}{\useKV[ClesSolides]{ObjetSection}}%
    }%
    {pave}{%
      \MPSolidePave{\useKV[ClesSolides]{ListeSommets}}{\the\toksolidelistesommets}{\useKV[ClesSolides]{Profondeur},\useKV[ClesSolides]{Largeur},\useKV[ClesSolides]{Hauteur}}{\useKV[ClesSolides]{PointsSection}}{\useKV[ClesSolides]{ObjetSection}}%
    }%
    {cone}{%
      \MPSolideCone{\useKV[ClesSolides]{ListeSommets}}{\the\toksolidelistesommets}{\useKV[ClesSolides]{PointsSection}}%
    }%
    {prisme}{%
      \MPSolidePrisme{\useKV[ClesSolides]{ListeSommets}}{\the\toksolidelistesommets}{}{\useKV[ClesSolides]{PointsSection}}%
    }%
    {pyramide}{%
      \MPSolidePyramide{\useKV[ClesSolides]{ListeSommets}}{\the\toksolidelistesommets}{\useKV[ClesSolides]{DecalageSommet}}{\useKV[ClesSolides]{PointsSection}}%
    }%
    {cylindre}{%
      \MPSolideCylindre{\useKV[ClesSolides]{ListeSommets}}{\the\toksolidelistesommets}{\useKV[ClesSolides]{PointsSection}}%
    }%
    {sphere}{%
      \MPSolideSphere{\useKV[ClesSolides]{ListeSommets}}{\the\toksolidelistesommets}{\useKV[ClesSolides]{PointsSection}}%
    }%
    {demisphere}{%
      \MPSolideDemiSphere{\useKV[ClesSolides]{ListeSommets}}{\the\toksolidelistesommets}{\useKV[ClesSolides]{PointsSection}}%
    }%
    {boule}{%
      \MPSolideSphere{\useKV[ClesSolides]{ListeSommets}}{\the\toksolidelistesommets}{\useKV[ClesSolides]{PointsSection}}%
    }%
  }%
}%