%%%
% Statistiques
%%%
\newcommand\NbDonnees{}%
\newcommand\SommeDonnees{}%
\newcommand\EffectifTotal{}%
\newcommand\Moyenne{}%
\newcommand\Etendue{}%
\newcommand\Mediane{}%
\newcommand\DonneeMax{}%
\newcommand\DonneeMin{}%
\newcommand\EffectifMax{}%
\newcommand\PfCArticleMediane{la}%

\setKVdefault[ClesStat]{ColVide=0,CaseVide=false,EffVide=false,%
FreqVide=false,AngVide=false,ECCVide=false,TotalVide=false,Sondage=false,Liste=false,%
Tableau=false,Stretch=1,Frequence=false,EffectifTotal=false,%
Etendue=false,Moyenne=false,SET=false,ValeurExacte=false,MoyenneA,Somme,Mediane=false,DetailsMediane=false,UneMediane=false,QuartileUn=false,QuartileTrois=false,Total=false,Concret=false,%
Largeur=1cm,Precision=2,PrecisionF=0,Donnee=Valeurs,Effectif=Effectif,Grille=false,Origine=0,Angle=false,SemiAngle=false,Qualitatif=false,Classes=false,TableauVide=false,ECC=false,Coupure=10,CouleurTab=gray!15,Graphique=false,Batons=true,Centre=false,CentreVide=false,Crochets=false,%
% Pour les diags batons
EpaisseurBatons=1,ListeCouleursB={a},Lecture=false,LectureFine=false,AideLecture=false,Reponses=false,DonneesSup=false,AbscisseRotation=false,Tiret=false,AngleRotationAbscisse=0,Pasx=1,Pasy=1,Unitex=0.5,Unitey=0.5,Depart=0,CouleurDefaut=black,Date=false,GrandNombrey=false,GrandNombrex=false,PasGrillex=1,PasGrilley=1,%
% Pour les diags circulaires
Rayon=3cm,AffichageAngle=false,AffichageDonnee=false,ListeCouleurs={white},Hachures=false,ListeHachures={60},LectureInverse=false,EcartHachures=0.25,EpaisseurHachures=1,Legende,LegendeVide=false,ACompleter=false,DebutAngle=0,%
%Pour les représentations
Representation=false,%
%Pour les barres horizontales
Barre=false,Longueur=10cm,Hauteur=5mm,Bicolore=false,EcartBarre=0,%Grille est dispo
% Pour les histogrammes
Histogramme=false,UniteAire=1,MemeAmpli,DepartHisto=1,%
% Pour la quadri...
ModeleCouleur=5%
}%
% compl\'ements
%\defKV[ClesStat]{ListeHachures=\setKV[ClesStat]{Hachures}}%
\defKV[ClesStat]{Unite=\setKV[ClesStat]{Concret}}%
\defKV[ClesStat]{AngleRotationAbscisse=\setKV[ClesStat]{AbscisseRotation}}%
\defKV[ClesStat]{AffichageDonnees=\setKV[ClesStat]{AffichageAngle=false}\setKV[ClesStat]{AffichageDonnee}}%
\defKV[ClesStat]{CasesVides=\setKV[ClesStat]{CaseVide}}%
\defKV[ClesStat]{LegendesVides=\setKV[ClesStat]{LegendeVide}}%
\defKV[ClesStat]{GrandNombreO=\setKV[ClesStat]{GrandNombrey}}%
\defKV[ClesStat]{GrandNombreA=\setKV[ClesStat]{GrandNombrex}}%
% La construction du tableau
\def\addtotok#1#2{#1\expandafter{\the#1#2}}%
\newtoks\tabtoksa\newtoks\tabtoksb\newtoks\tabtoksc%
\def\updatetoks#1/#2\nil{\addtotok\tabtoksa{\ifboolKV[ClesStat]{Qualitatif}{&#1}{&\num{#1}}}\addtotok\tabtoksb{&\num{#2}}}%
%
\newcounter{PfCCompteLignes}%
%
\def\BuildtabStat{% %%Tableau sans/avec total
  \setcounter{PfCCompteLignes}{0}%
  \tabtoksa{\useKV[ClesStat]{Donnee}}\tabtoksb{\useKV[ClesStat]{Effectif}}%
  \foreachitem\compteur\in\ListeComplete{\expandafter\updatetoks\compteur\nil}%
  \renewcommand{\arraystretch}{\useKV[ClesStat]{Stretch}}%
  \ifboolKV[ClesStat]{Total}{%
    \begin{NiceTabular}{c*{\fpeval{\ListeCompletelen+1}}{>{\centering\arraybackslash}p{\useKV[ClesStat]{Largeur}}}}%
      \CodeBefore%
      \rowcolor{\useKV[ClesStat]{CouleurTab}}{1}%
      \columncolor{\useKV[ClesStat]{CouleurTab}}{1}%
      \Body%
      \the\tabtoksa&Total\\%
      \ifboolKV[ClesStat]{EffVide}{\useKV[ClesStat]{Effectif}\xintFor* ##1 in {\xintSeq {1}{\ListeCompletelen}}\do{&}}{\the\tabtoksb&\num{\EffectifTotal}}\\%
      \ifboolKV[ClesStat]{Frequence}{\stepcounter{PfCCompteLignes}Fr\'equence (\%)\xintFor* ##1 in {\xintSeq {1}{\ListeCompletelen}}\do{&\ifboolKV[ClesStat]{TableauVide}{}{\ifboolKV[ClesStat]{FreqVide}{}{\num{\CalculFrequence{##1}}}}}&\ifboolKV[ClesStat]{TableauVide}{}{\ifboolKV[ClesStat]{FreqVide}{}{100}}\\}{}%
      \ifboolKV[ClesStat]{Angle}{\stepcounter{PfCCompteLignes}Angle (\si{\degree})\xintFor* ##1 in {\xintSeq {1}{\ListeCompletelen}}\do{&\ifboolKV[ClesStat]{TableauVide}{}{\ifboolKV[ClesStat]{AngVide}{}{\CalculAngle{##1}}}}&\ifboolKV[ClesStat]{TableauVide}{}{\ifboolKV[ClesStat]{AngVide}{}{360}}\\}{}%
      \ifboolKV[ClesStat]{SemiAngle}{\stepcounter{PfCCompteLignes}Angle (\si{\degree})\xintFor* ##1 in {\xintSeq {1}{\ListeCompletelen}}\do{&\ifboolKV[ClesStat]{TableauVide}{}{\ifboolKV[ClesStat]{AngVide}{}{\CalculSemiAngle{##1}}}}&\ifboolKV[ClesStat]{TableauVide}{}{\ifboolKV[ClesStat]{AngVide}{}{180}}\\}{}%
      \ifboolKV[ClesStat]{ECC}{\stepcounter{PfCCompteLignes}E.C.C.\xintFor* ##1 in {\xintSeq {1}{\ListeCompletelen}}\do{&\ifboolKV[ClesStat]{TableauVide}{}{\ifboolKV[ClesStat]{ECCVide}{}{\CalculECC{##1}}}}&\ifboolKV[ClesStat]{TableauVide}{}{\ifboolKV[ClesStat]{ECCVide}{}{\num{\EffectifTotal}}}\\}{}%
      \CodeAfter%
      % On crée la liste des colonnes à vider
      \xintifboolexpr{\useKV[ClesStat]{ColVide}>0}{%
        \xdef\FooStat{\useKV[ClesStat]{ColVide}}%
        \setsepchar{,}%
        \readlist*\ListeColonnesAVider{\FooStat}%
        \foreachitem\compteur\in\ListeColonnesAVider{%
          \tikz\fill[white] (row-2-|col-\fpeval{\compteur+1}) rectangle (last-|col-\fpeval{\compteur+2});%
        }%
      }{}%
      % On crée la liste des cases à vider
      \ifboolKV[ClesStat]{CaseVide}{%
        \xdef\FooStatCases{\useKV[ClesStat]{CasesVides}}%
        \setsepchar[*]{,*/}%
        \readlist*\ListeCasesAVider{\FooStatCases}%
        \foreachitem\compteur\in\ListeCasesAVider{%
          \tikz\fill[white] (row-\fpeval{\ListeCasesAVider[\compteurcnt,1]+1}-|col-\fpeval{\ListeCasesAVider[\compteurcnt,2]+1}) rectangle (row-\fpeval{\ListeCasesAVider[\compteurcnt,1]+2}-|col-\fpeval{\ListeCasesAVider[\compteurcnt,2]+2});%
        }%
      }{}%
      % On retrace le tableau
      %Les colonnes
      \xintFor* ##1 in {\xintSeq{1}{\fpeval{\ListeCompletelen+3}}}\do{%
        \tikz\draw (row-1-|col-##1) -- (last-|col-##1);%
      }%
      % Les lignes
      \xintFor* ##1 in {\xintSeq{1}{\fpeval{\thePfCCompteLignes+3}}}\do{%
        \tikz\draw (row-##1-|col-1) -- (row-##1-|last);%
      }%
    \end{NiceTabular}%
  }{%
    \begin{NiceTabular}{c*{\fpeval{\ListeCompletelen}}{>{\centering\arraybackslash}p{\useKV[ClesStat]{Largeur}}}}%
      \CodeBefore%
      \rowcolor{\useKV[ClesStat]{CouleurTab}}{1}%
      \columncolor{\useKV[ClesStat]{CouleurTab}}{1}%
      \Body%
      \the\tabtoksa\\%
      \ifboolKV[ClesStat]{EffVide}{\useKV[ClesStat]{Effectif}\xintFor* ##1 in {\xintSeq {1}{\ListeCompletelen}}\do{&}}{\the\tabtoksb}\\%
      \ifboolKV[ClesStat]{Frequence}{\stepcounter{PfCCompteLignes}Fr\'equence (\%)\xintFor* ##1 in {\xintSeq {1}{\ListeCompletelen}}\do{&\ifboolKV[ClesStat]{TableauVide}{}{\ifboolKV[ClesStat]{FreqVide}{}{\num{\CalculFrequence{##1}}}}}\\}{}%
      \ifboolKV[ClesStat]{Angle}{\stepcounter{PfCCompteLignes}Angle (\si{\degree})\xintFor* ##1 in {\xintSeq {1}{\ListeCompletelen}}\do{&\ifboolKV[ClesStat]{TableauVide}{}{\ifboolKV[ClesStat]{AngVide}{}{\CalculAngle{##1}}}}\\}{}%
      \ifboolKV[ClesStat]{SemiAngle}{\stepcounter{PfCCompteLignes}Angle (\si{\degree})\xintFor* ##1 in {\xintSeq {1}{\ListeCompletelen}}\do{&\ifboolKV[ClesStat]{TableauVide}{}{\ifboolKV[ClesStat]{AngVide}{}{\CalculSemiAngle{##1}}}}\\}{}%
      \ifboolKV[ClesStat]{ECC}{\stepcounter{PfCCompteLignes}E.C.C.\xintFor* ##1 in {\xintSeq {1}{\ListeCompletelen}}\do{&\ifboolKV[ClesStat]{TableauVide}{}{\ifboolKV[ClesStat]{ECCVide}{}{\CalculECC{##1}}}}\\}{}%
      \CodeAfter%
      % On crée la liste des colonnes à vider
      \xintifboolexpr{\useKV[ClesStat]{ColVide}>0}{%
        \xdef\FooStat{\useKV[ClesStat]{ColVide}}%
        \setsepchar{,}%
        \readlist*\ListeColonnesAVider{\FooStat}%
        \foreachitem\compteur\in\ListeColonnesAVider{%
          \tikz\fill[white] (row-2-|col-\fpeval{\compteur+1}) rectangle (last-|col-\fpeval{\compteur+2});%
        }%
      }{}%
      % On crée la liste des cases à vider
      \ifboolKV[ClesStat]{CaseVide}{%
        \xdef\FooStatCases{\useKV[ClesStat]{CasesVides}}%
        \setsepchar[*]{,*/}%
        \readlist*\ListeCasesAVider{\FooStatCases}%
        \foreachitem\compteur\in\ListeCasesAVider{%
          \tikz\fill[white] (row-\fpeval{\ListeCasesAVider[\compteurcnt,1]+1}-|col-\fpeval{\ListeCasesAVider[\compteurcnt,2]+1}) rectangle (row-\fpeval{\ListeCasesAVider[\compteurcnt,1]+2}-|col-\fpeval{\ListeCasesAVider[\compteurcnt,2]+2});%
        }%
      }{}%
      % On retrace le tableau
      %Les colonnes
      \xintFor* ##1 in {\xintSeq{1}{\fpeval{\ListeCompletelen+2}}}\do{%
        \tikz\draw (row-1-|col-##1) -- (last-|col-##1);%
      }%
      % Les lignes
      \xintFor* ##1 in {\xintSeq{1}{\fpeval{\thePfCCompteLignes+3}}}\do{%
        \tikz\draw (row-##1-|col-1) -- (row-##1-|last);%
      }%
    \end{NiceTabular}%
  }%
}%

% Pour construire le diagramme en barres horizontales
\def\UpdatetoksHor#1/#2/#3\nil{\addtotok\toklistenomhor{"#1",}\addtotok\toklistedonhor{#3,}\addtotok\toklisteaffhor{"#2",}}%

\newcommand\buildgraphbarhor{%
  \newtoks\toklistenomhor%
  \newtoks\toklistedonhor%
  \newtoks\toklisteaffhor%
  \newtoks\toklistecouleur%
  \xdef\PfCfooStat{}%
  \xintFor* ##1 in {\xintSeq {1}{\ListeCompletelen}}\do{%
    \xdef\PfCfooStat{\PfCfooStat \ListeComplete[##1,2],}%
  }%
  \xdef\DivMax{\fpeval{max(\PfCfooStat)}}%
  \xdef\ExposantDivMax{\fpeval{round(ln(\DivMax)/ln(10))}}%
  \xdef\DivMax{\fpeval{10**\ExposantDivMax}}%
  \xdef\PfCfooStat{}%
  \xintFor* ##1 in {\xintSeq {1}{\ListeCompletelen}}\do{%
    \xdef\PfCfooStat{\PfCfooStat \ListeComplete[##1,1]/\ListeComplete[##1,2]/\fpeval{\ListeComplete[##1,2]/\DivMax},}%
  }%
  \readlist*\ListeCompleteDiagHor{\PfCfooStat}%
  \foreachitem\compteur\in\ListeCompleteDiagHor{\expandafter\UpdatetoksHor\compteur\nil}%
  \xdef\ListeAvantCouleurs{\useKV[ClesStat]{ListeCouleurs}}%
  \readlist*\ListeCouleur{\ListeAvantCouleurs}%
  \foreachitem\couleur\in\ListeCouleur{\expandafter\UpdateCoul\couleur\nil}%
  \NewMPDiagBarreHor{\the\toklistenomhor}{\the\toklistedonhor}{\the\toklisteaffhor}{\the\toklistecouleur}%
}%

% Pour construire le diagramme en bâtons
\def\Updatetoks#1/#2\nil{\addtotok\toklistepoint{(#1,#2),}}%
\newcommand\buildgraph[1][]{%
  \newtoks\toklistepoint\toklistepoint{}%
  \newtoks\toklistecouleur\toklistecouleur{}%
  \newtoks\toklistelegende\toklistelegende{}%
  \ifboolKV[ClesStat]{LegendeVide}{%
    \xdef\foo{\useKV[ClesStat]{LegendesVides}}%
    \readlist*\ListeLegendesAEffacer{\foo}%
  }{\xdef\foo{-1}\readlist*\ListeLegendesAEffacer{\foo}%
  }%
  \foreachitem\compteur\in\ListeLegendesAEffacer{\expandafter\UpdateLegende\compteur\nil}%
  \foreachitem\compteur\in\ListeComplete{\expandafter\Updatetoks\compteur\nil}%
  \xdef\ListeAvantCouleurs{\useKV[ClesStat]{ListeCouleursB}}%
  \readlist*\ListeCouleur{\ListeAvantCouleurs}%
  \foreachitem\couleur\in\ListeCouleur{\expandafter\UpdateCoul\couleur\nil}%
  \MPStatNew{\the\toklistepoint}{\the\toklistecouleur}{\the\toklistelegende}
}%

% Pour construire le diagramme en bâtons qualitatif
\def\Updatetoksq#1/#2\nil{\addtotok\toklistepointq{"#1",#2,}}%
\newcommand\buildgraphq[1][]{%
  \newtoks\toklistepointq\toklistepointq{}%
  \newtoks\toklistecouleur\toklistecouleur{}%
  \newtoks\toklistelegende\toklistelegende{}%
  \ifboolKV[ClesStat]{LegendeVide}{%
    \xdef\foo{\useKV[ClesStat]{LegendesVides}}%
    \readlist*\ListeLegendesAEffacer{\foo}%
  }{\xdef\foo{-1}\readlist*\ListeLegendesAEffacer{\foo}%
  }%
  \foreachitem\compteur\in\ListeLegendesAEffacer{\expandafter\UpdateLegende\compteur\nil}%
  \foreachitem\compteur\in\ListeComplete{\expandafter\Updatetoksq\compteur\nil}%
  \xdef\ListeAvantCouleurs{\useKV[ClesStat]{ListeCouleursB}}%
  \readlist*\ListeCouleur{\ListeAvantCouleurs}%
  \foreachitem\couleur\in\ListeCouleur{\expandafter\UpdateCoul\couleur\nil}%
  \MPStatNew{\the\toklistepointq}{\the\toklistecouleur}{\the\toklistelegende}%
}%

\def\UpdateCoul#1\nil{\addtotok\toklistecouleur{#1,}}%
\def\UpdateHach#1\nil{\addtotok\toklisteanglehachure{#1,}}%
\def\UpdateLegende#1\nil{\addtotok\toklistelegende{#1,}}%

% Pour construire le diagramme circulaire qualitatif
\def\buildgraphcq#1{%
  \newtoks\toklistepointq\toklistepointq{}%
  \newtoks\toklistecouleur\toklistecouleur{}%
  \newtoks\toklisteanglehachure\toklisteanglehachure{}%
  \newtoks\toklistelegende\toklistelegende{}%
  \ifboolKV[ClesStat]{LegendeVide}{%
    \xdef\foo{\useKV[ClesStat]{LegendesVides}}%
    \readlist*\ListeLegendesAEffacer{\foo}%
  }{\xdef\foo{0}\readlist*\ListeLegendesAEffacer{\foo}%
  }%
  \foreachitem\compteur\in\ListeLegendesAEffacer{\expandafter\UpdateLegende\compteur\nil}%
  %
  \foreachitem\compteur\in\ListeComplete{\expandafter\Updatetoksq\compteur\nil}%
  \xdef\ListeAvantCouleurs{\useKV[ClesStat]{ListeCouleurs}}%
  \readlist*\ListeCouleur{\ListeAvantCouleurs}%
  \foreachitem\couleur\in\ListeCouleur{\expandafter\UpdateCoul\couleur\nil}%
  \xdef\ListeAvantHachures{\useKV[ClesStat]{ListeHachures}}%
  \readlist*\ListeHachure{\ListeAvantHachures}%
  \foreachitem\valeurangle\in\ListeHachure{\expandafter\UpdateHach\valeurangle\nil}%
  \NewMPStatCirculaireQ{\the\toklistepointq}{#1}{\the\toklistecouleur}{\the\toklistelegende}{\the\toklisteanglehachure}%
}%

%% calcul des fr\'equences
\newcommand\CalculFrequence[1]{%
  \fpeval{round(\ListeComplete[#1,2]*100/\EffectifTotal,\useKV[ClesStat]{PrecisionF})}
}

%% calcul des angles
\newcommand\CalculAngle[1]{%
  \fpeval{round(\ListeComplete[#1,2]*360/\EffectifTotal,0)}
}
\newcommand\CalculSemiAngle[1]{%
  \fpeval{round(\ListeComplete[#1,2]*180/\EffectifTotal,0)}
}

%% calcul des ECC
\newcount\CompteurECC%
\newcount\CompteurECCTotal%
\newcount\CompteurECCC%
\newcount\CompteurECCCTotal%

\newcommand\CalculECC[1]{%
  \xdef\TotalECC{0}%
  \CompteurECC=1%
  \CompteurECCTotal=\numexpr#1+1%
  \whiledo{\CompteurECC < \CompteurECCTotal}{%
    \xdef\TotalECC{\fpeval{\TotalECC+\ListeComplete[\the\CompteurECC,2]}}%
    \CompteurECC=\numexpr\CompteurECC+1%
  }%
  \num{\TotalECC}%
}

\def\NewMPDiagBarreHorCode{%
  Longueur:=\useKV[ClesStat]{Longueur};
  Hauteur:=\useKV[ClesStat]{Hauteur};
  Ecart:=\useKV[ClesStat]{EcartBarre};
  ExposantDivMax:=\ExposantDivMax;
  ecarthachures=\useKV[ClesStat]{EcartHachures};
  epaisseurhachures=\useKV[ClesStat]{EpaisseurHachures};
  boolean Hachures,Bicolore,Grille,AffichageDonnee,LegendeVide;
  Hachures=\useKV[ClesStat]{Hachures};
  Bicolore=\useKV[ClesStat]{Bicolore};
  Grille=\useKV[ClesStat]{Grille};
  AffichageDonnee=\useKV[ClesStat]{AffichageDonnee};
  LegendeVide=\useKV[ClesStat]{LegendeVide};
  vardef CalculNombreDonneesEtDonneeMax(text t)=
  nbdon:=0;%nombre de données
  DonneeMax:=0;%donnée DonneeMaximale
  for p_=t:
  nbdon:=nbdon+1;
  if p_>DonneeMax:
  DonneeMax:=p_;
  fi;
  endfor;
  enddef;
  vardef ListeDonnees(text t)=
  n:=0;
  for p_=t:
  n:=n+1;
  Donnees[n]:=p_;
  endfor;
  enddef;
  vardef RecuperationCouleurs(text t)=
  color Col[];
  n:=0;
  for p_=t:
  n:=n+1;
  Col[n]=p_;
  endfor;
  enddef;
}

% Construction d'un diagramme en barres horizontal
\newcommand\NewMPDiagBarreHor[4]{%
  % #1 Liste des noms
  % #2 Liste des valeurs associées
  % #3 Liste des valeurs à afficher (si pb calcul MP)
  % #4 Liste des couleurs
  \ifluatex%
  \mplibforcehmode%
  \begin{mplibcode}%
    defaultcolormodel := \useKV[ClesStat]{ModeleCouleur};
    \NewMPDiagBarreHorCode%
    vardef TraceDiag=
    if Grille:
    pair Zz[];%Pour déterminer "le dernier point"
    Zz0=(0,-(nbdon-1)*(Hauteur+Ecart)-Ecart);
    Zz2=(0,(Hauteur+Ecart));
    Zz1=((1/DonneeMax)*Longueur,-(nbdon-1)*(Hauteur+Ecart)-Ecart);
    Zz3=if ExposantDivMax=0 : (DonneeMax+1)[Zz0,Zz1] else: ((floor(DonneeMax*10+2))/10)[Zz0,Zz1];fi;
    if ExposantDivMax=0:
    for k=1 upto DonneeMax+1:
    trace (Zz0--Zz2) shifted (k*(Zz1-Zz0)) dashed evenly withcolor 0.5white;
    endfor;
    else:
    for k=1 upto (floor(DonneeMax*10+2)):
    trace (Zz0--Zz2) shifted ((k/10)*(Zz1-Zz0)) dashed evenly withcolor 0.5white;
    endfor;
    fi;
    if ExposantDivMax=0:
    for k=1 upto (DonneeMax+1):
    label.bot(TEX("\num{"&decimal(k)&"}"),Zz0+k*(Zz1-Zz0));
    endfor;
    else:
    if ExposantDivMax<5:
    for k=1 upto (floor(DonneeMax*10+2)):
    label.bot(TEX("\num{\fpeval{"&decimal(k)&"*(10**"&decimal(ExposantDivMax-1)&")}}"),Zz0+(k/10)*(Zz1-Zz0));
    endfor;
    else:
    dotlabel.bot(TEX("\num{\fpeval{10**"&decimal(ExposantDivMax)&"}}"),Zz1);
    fi;
    fi;
    fi;
    for k=0 upto nbdon-1:
    path RectangleDonnee;
    RectangleDonnee=(unitsquare xscaled ((Donnees[k+1]/DonneeMax)*Longueur) yscaled Hauteur) shifted(0,-k*(Hauteur+Ecart));
    if Hachures:
    fill RectangleDonnee withcolor white;
    trace Hachurage(RectangleDonnee,60 if
    (k mod 2)=0: +90 fi,ecarthachures,if (k mod 2)=0 : 0 else: 1 fi)
    withpen pencircle scaled epaisseurhachures;
    else:
    remplis RectangleDonnee withcolor if unknown Col[k+1]: if Bicolore:Col[(k mod 2)+1] else: white fi; else:if Bicolore:Col[(k mod 2)+1] else: Col[k+1] fi; fi;
    fi;
    trace RectangleDonnee;
    endfor;
    if Grille:
    drawarrow (0,-(nbdon-1)*(Hauteur+Ecart)-Ecart)--(0,(Hauteur+Ecart)) withpen pencircle scaled 1.5;
    drawarrow (0,-(nbdon-1)*(Hauteur+Ecart)-Ecart)--(Zz3+u*(0.25,0)) withpen pencircle scaled 1.5;
    fi;
    enddef;
    vardef AffichageNom(text t)=
    k:=0;
    for p_=t:
    label.lft(TEX(p_),0.5[(0,0),(0,Hauteur)] shifted (0,-k*(Hauteur+Ecart)));
    k:=k+1;
    endfor;
    enddef;
    vardef AffichageDonnees(text t)=
    k:=0;
    for p_=t:
    label.rt(TEX("\num{"&p_&"}"),0.5[(0,0),(0,Hauteur)] shifted (((Donnees[k+1]/DonneeMax)*Longueur),-k*(Hauteur+Ecart)));
    k:=k+1;
    endfor;
    enddef;
    CalculNombreDonneesEtDonneeMax(#2);
    ListeDonnees(#2);
    RecuperationCouleurs(#4);
    TraceDiag;
    if LegendeVide=false:
    AffichageNom(#1);
    fi;
    if AffichageDonnee:
    AffichageDonnees(#3);
    fi;
  \end{mplibcode}
  \else%
  \begin{mpost}[mpsettings={\NewMPDiagBarreHorCode}]
    vardef TraceDiag=
    if Grille:
    pair Zz[];%Pour déterminer "le dernier point"
    Zz0=(0,-(nbdon-1)*(Hauteur+Ecart)-Ecart);
    Zz2=(0,(Hauteur+Ecart));
    Zz1=((1/DonneeMax)*Longueur,-(nbdon-1)*(Hauteur+Ecart)-Ecart);
    Zz3=if ExposantDivMax=0 : (DonneeMax+1)[Zz0,Zz1] else: ((floor(DonneeMax*10+2))/10)[Zz0,Zz1];fi;
    if ExposantDivMax=0:
    for k=1 upto DonneeMax+1:
    trace (Zz0--Zz2) shifted (k*(Zz1-Zz0)) dashed evenly withcolor 0.5white;
    endfor;
    else:
    for k=1 upto (floor(DonneeMax*10+2)):
    trace (Zz0--Zz2) shifted ((k/10)*(Zz1-Zz0)) dashed evenly withcolor 0.5white;
    endfor;
    fi;
    if ExposantDivMax=0:
    for k=1 upto (DonneeMax+1):
    label.bot(LATEX("\num{"&decimal(k)&"}"),Zz0+k*(Zz1-Zz0));
    endfor;
    else:
    if ExposantDivMax<5:
    for k=1 upto (floor(DonneeMax*10+2)):
    label.bot(LATEX("\num{\noexpand\fpeval{"&decimal(k)&"*(10**"&decimal(ExposantDivMax-1)&")}}"),Zz0+(k/10)*(Zz1-Zz0));
    endfor;
    else:
    dotlabel.bot(LATEX("\num{\noexpand\fpeval{10**"&decimal(ExposantDivMax)&"}}"),Zz1);
    fi;
    fi;
    fi;
    for k=0 upto nbdon-1:
    path RectangleDonnee;
    RectangleDonnee=(unitsquare xscaled ((Donnees[k+1]/DonneeMax)*Longueur) yscaled Hauteur) shifted(0,-k*(Hauteur+Ecart));
    if Hachures:
    fill RectangleDonnee withcolor white;
    trace Hachurage(RectangleDonnee,60 if
    (k mod 2)=0: +90 fi,ecarthachures,if (k mod 2)=0 : 0 else: 1 fi)
    withpen pencircle scaled epaisseurhachures;
    else:
    remplis RectangleDonnee withcolor if unknown Col[k+1]: if Bicolore:Col[(k mod 2)+1] else: white fi; else:if Bicolore:Col[(k mod 2)+1] else: Col[k+1] fi; fi;
    fi;
    trace RectangleDonnee;
    endfor;
    if Grille:
    drawarrow (0,-(nbdon-1)*(Hauteur+Ecart)-Ecart)--(0,(Hauteur+Ecart)) withpen pencircle scaled 1.5;
    drawarrow (0,-(nbdon-1)*(Hauteur+Ecart)-Ecart)--(Zz3+u*(0.25,0)) withpen pencircle scaled 1.5;
    fi;
    enddef;
    vardef AffichageNom(text t)=
    k:=0;
    for p_=t:
    label.lft(LATEX(p_),0.5[(0,0),(0,Hauteur)] shifted (0,-k*(Hauteur+Ecart)));
    k:=k+1;
    endfor;
    enddef;
    vardef AffichageDonnees(text t)=
    k:=0;
    for p_=t:
    label.rt(LATEX("\num{"&p_&"}"),0.5[(0,0),(0,Hauteur)] shifted (((Donnees[k+1]/DonneeMax)*Longueur),-k*(Hauteur+Ecart)));
    k:=k+1;
    endfor;
    enddef;
    CalculNombreDonneesEtDonneeMax(#2);
    ListeDonnees(#2);
    RecuperationCouleurs(#4);
    TraceDiag;
    if LegendeVide=false:
    AffichageNom(#1);
    fi;
    if AffichageDonnee:
    AffichageDonnees(#3);
    fi;
  \end{mpost}
  \fi%
}%

\def\MPStatNewCode{%
  maxx:=0;
  maxy:=0;
  unitex:=\useKV[ClesStat]{Unitex}*cm;
  unitey:=\useKV[ClesStat]{Unitey}*cm;
  xpartorigine:=\useKV[ClesStat]{Origine};
  AngleRotation=\useKV[ClesStat]{AngleRotationAbscisse};
  boolean Rotation,Lecture,LectureFine,AideLecture,DonneesSup,Reponses,Qualitatif,Tiret,LegendeVide,Retour,GrandNombrex,GrandNombrey,Date,Grille;
  GrandNombrex=\useKV[ClesStat]{GrandNombrex};
  GrandNombrey=\useKV[ClesStat]{GrandNombrey};
  if GrandNombrex:
  GrandNombreA=\useKV[ClesStat]{GrandNombreA};
  fi;
  if GrandNombrey:
  GrandNombreO=\useKV[ClesStat]{GrandNombreO};
  fi;
  Date:=\useKV[ClesStat]{Date};
  Rotation=\useKV[ClesStat]{AbscisseRotation};
  Lecture:=\useKV[ClesStat]{Lecture};
  LectureFine:=\useKV[ClesStat]{LectureFine};
  AideLecture:=\useKV[ClesStat]{AideLecture};
  DonneesSup:=\useKV[ClesStat]{DonneesSup};
  Reponses:=\useKV[ClesStat]{Reponses};
  LegendeVide=\useKV[ClesStat]{LegendeVide};
  epaisseurbatons=\useKV[ClesStat]{EpaisseurBatons};
  Qualitatif=\useKV[ClesStat]{Qualitatif};
  Tiret=\useKV[ClesStat]{Tiret};
  Retour=false;
  Grille:=\useKV[ClesStat]{Grille};
  Pasx:=\useKV[ClesStat]{Pasx};
  Pasy:=\useKV[ClesStat]{Pasy};
  PasGrillex:=\useKV[ClesStat]{PasGrillex};
  PasGrilley:=\useKV[ClesStat]{PasGrilley};
  color CoulDefaut;
  CoulDefaut=\useKV[ClesStat]{CouleurDefaut};
  Depart=\useKV[ClesStat]{Depart};
  %
  pair A[],B[],P[];
  vardef toto(text t)=%points quantitatif
  n:=0;
  for p_=t:
  if pair p_:
  n:=n+1;
  P[n]=((xpart(p_)-(xpartorigine))*unitex,ypart(p_)*unitey);
  if xpart(p_)>maxx:
  maxx:=xpart(p_)-(xpartorigine);
  fi;
  if ypart(p_)>maxy:
  maxy:=ypart(p_);
  fi;
  A[n]=unitex*(xpart(p_)-(xpartorigine),0);
  B[n]=unitey*(0,ypart(p_));
  fi;
  endfor;
  if DonneesSup:
  maxAxey:=floor(maxy/10)*10+4*PasGrilley;
  else:
  maxAxey:=maxy;
  fi;
  enddef;
  vardef tutu(text t)=%points qualitatif
  n:=0;
  for p_=t:
  if numeric p_:
  P[n]=((n)*unitex,unitey*(p_-Depart));
  B[n]=(0,unitey*(p_-Depart));
  if p_>maxy:
  maxy:=p_;
  fi;
  else:
  n:=n+1;
  A[n]=unitex*(n,0);
  fi;
  endfor;
  maxy:=maxy-Depart;
  maxx:=n;
  if DonneesSup:
  maxAxey:=floor(maxy/10)*10+4*PasGrilley;
  else:
  maxAxey:=maxy;
  fi;
  enddef;
}

% Construction du graphique en bâtons
\newcommand\MPStatNew[3]{%
  \ifluatex
  \mplibforcehmode
  \begin{mplibcode}
    defaultcolormodel := \useKV[ClesStat]{ModeleCouleur};
    
    \MPStatNewCode
      %
    vardef Test(expr nb)=
    Retour:=false;
    op:=0;
    for l_=#3:
    if l_=nb:
    op:=op+1;
    fi;
    endfor;
    if op>0:
    Retour:=true;
    fi;
    enddef;    
    % 
    % on r\'ecup\`ere les couleurs
    color Col[];
    n:=0;
    for p_=#2:
    n:=n+1;
    if color p_:
    Col[n]=p_;
    else:
    Col[n]=CoulDefaut;
    fi;
    endfor;
    vardef tata(text t)=%affichage quantitatif
    l=0;
    for p_=t:
    if pair p_:
    l:=l+1;
    if Rotation:
    if Date:
    label.bot(TEX(decimal(xpart(p_))) rotated AngleRotation,A[l]);
    else:
    label.bot(TEX("\num{"&decimal(xpart(p_))&"}") rotated AngleRotation,A[l]);
    fi;
    else :
    if Date:
    label.bot(TEX(decimal(xpart(p_))),A[l]);
    else:
    label.bot(TEX("\num{"&decimal(xpart(p_))&"}"),A[l]);
    fi;
    fi;
    if Reponses:
    if DonneesSup:
    Test(l);
    if Retour=false:
    if GrandNombrey:
    label.top(TEX("\num{"&decimal(ypart(p_))&"}"),P[l]);
    else:
    label.top(TEX("\num{"&decimal(ypart(p_))&"}"),P[l]);
    fi;
    fi;
    else:
    if Tiret:
    trace (B[l]+(-1pt,0))--(B[l]+(1pt,0));
    label.lft(TEX("\num{"&decimal(p_)&"}"),B[l]);
    else:
    dotlabel.lft(TEX("\num{"&decimal(ypart(p_))&"}"),B[l]);
    fi;
    fi;
    fi;
    fi;
    endfor;
    enddef;
    vardef titi(text t)=%affichage qualitatif
    l:=0;
    for p_=t:
    if numeric p_:
    if Reponses:
    if DonneesSup:
    Test(l);
    if Retour=false:
    label.top(TEX("\num{"&decimal(p_)&"}"),P[l]);
    fi;
    else:
    if Tiret:
    trace (B[l]+(-1pt,0))--(B[l]+(1pt,0));
    label.lft(TEX("\num{"&decimal(p_)&"}"),B[l]);
    else:
    dotlabel.lft(TEX("\num{"&decimal(p_)&"}"),B[l]);
    fi;
    fi;
    fi;
    else:
    l:=l+1;
    if Rotation:
    if AngleRotation<>0:
    picture TEXTELABEL;
    TEXTELABEL=image(
    labeloffset:=labeloffset*2;
    label.lft(TEX(p_),A[l]);
    labeloffset:=labeloffset/2;
    );
    trace rotation(TEXTELABEL,A[l],AngleRotation);
    else :
    label.bot(TEX(p_),A[l]);
    fi;
    fi;
    fi;
    endfor;
    enddef;
    if Qualitatif: tutu(#1); else: toto(#1); fi;
%%%%%%%%%%%%%%%%%%%%%%%%%%%%%%%%%%%%%%%%%%%%%%%%%%%%%%%%%%%%%%%%
    if Grille:
    drawoptions(withcolor 0.75white);
    for k=0 step PasGrillex until ((maxx+1)):
    trace (k*unitex,0)--(k*unitex,unitey*((floor(maxAxey/Pasy)+1)*Pasy));
    endfor;
    for k=0 step PasGrilley until (floor(maxAxey/Pasy)+1)*Pasy:%((maxy+2*Pasy)):
    trace (0,k*unitey)--(unitex*(maxx+1),k*unitey);
    endfor;
    drawoptions();
    fi;
    if epaisseurbatons<>0:
    for k=1 upto n:
    fill polygone(A[k]-(epaisseurbatons*1pt,0),A[k]+(epaisseurbatons*1pt,0),P[k]+(epaisseurbatons*1pt,0),P[k]-(epaisseurbatons*1pt,0)) withcolor if unknown Col[k]: CoulDefaut else:Col[k] fi;
    if AideLecture:
    draw B[k]--P[k] dashed evenly;
    fi;
    endfor;
    fi;
    if LectureFine:
    for k=0 step Pasy until ((maxy+1*Pasy)):
    if Tiret:
    trace (1pt,k*unitey)--(-1pt,k*unitey);
    if GrandNombrey:
    label.lft(TEX("\num{\fpeval{\useKV[ClesStat]{GrandNombreO}*"&decimal(k+Depart)&"}}"),(0,k*unitey));
    else:
    label.lft(TEX("\num{"&decimal(k+Depart)&"}"),(0,k*unitey));
    fi;
    else:
    if GrandNombrey:
    dotlabel.lft(TEX("\num{\fpeval{\useKV[ClesStat]{GrandNombreO}*"&decimal(k+Depart)&"}}"),(0,k*unitey));
    else:
    dotlabel.lft(TEX("\num{"&decimal(k+Depart)&"}"),(0,k*unitey));
    fi;
    fi;
    endfor;
    fi;
    if Lecture:
    for k=0 step Pasy until Pasy:
    if Tiret:
    trace (1pt,k*unitey)--(-1pt,k*unitey);
    if GrandNombrey:
    label.lft(TEX("\num{\fpeval{\useKV[ClesStat]{GrandNombreO}*"&decimal(k)&"}}"),(0,k*unitey));
    else:
    label.lft(TEX("\num{"&decimal(k)&"}"),(0,k*unitey));
    fi;
    else:
    if GrandNombrey:
    dotlabel.lft(TEX("\num{\fpeval{\useKV[ClesStat]{GrandNombreO}*"&decimal(k)&"}}"),(0,k*unitey));
    else:
    dotlabel.lft(TEX("\num{"&decimal(k)&"}"),(0,k*unitey));
    fi;
    fi;
    endfor;
    fi;
    drawarrow (0,0)--unitex*(maxx+1,0);
    drawarrow (0,0)--unitey*(0,(floor(maxAxey/Pasy)+1)*Pasy);
    label.lrt(TEX("\useKV[ClesStat]{Donnee}"),unitex*(maxx+1,0));
    label.urt(TEX("\useKV[ClesStat]{Effectif}"),unitey*(0,(floor(maxAxey/Pasy)+1)*Pasy));
    if Qualitatif: titi(#1); else:tata(#1); fi;
  \end{mplibcode}
  \else
  \begin{mpost}[mpsettings={\MPStatNewCode}]
    % on r\'ecup\`ere les couleurs
    color Col[];
    n:=0;
    for p_=#2:
    n:=n+1;
    if color p_:
    Col[n]=p_;
    else:
    Col[n]=CoulDefaut;
    fi;
    endfor;
    % 
    vardef tata(text t)=%affichage quantitatif
    l=0;
    for p_=t:
    if pair p_:
    l:=l+1;
    if Rotation:
    label.bot(LATEX("\num{"&decimal(xpart(p_))&"}") rotated AngleRotation,A[l]);
    else :
    label.bot(LATEX("\num{"&decimal(xpart(p_))&"}"),A[l]);
    fi;
    if Reponses:
    if DonneesSup:
    label.top(LATEX("\num{"&decimal(ypart(p_))&"}"),P[l]);
    else:
    if Tiret:
    trace (B[l]+(-1pt,0))--(B[l]+(1pt,0));
    label.lft(LATEX("\num{"&decimal(p_)&"}"),B[l]);
    else:
    dotlabel.lft(LATEX("\num{"&decimal(ypart(p_))&"}"),B[l]);
    fi;
    fi;
    fi;
    fi;
    endfor;
    enddef;
    vardef titi(text t)=%affichage qualitatif
    l:=0;
    for p_=t:
    if numeric p_:
    if Reponses:
    if DonneesSup:
    label.top(LATEX("\num{"&decimal(p_)&"}"),P[l]);
    else:
    if Tiret:
    trace (B[l]+(-1pt,0))--(B[l]+(1pt,0));
    label.lft(LATEX("\num{"&decimal(p_)&"}"),B[l]);
    else:
    dotlabel.lft(LATEX("\num{"&decimal(p_)&"}"),B[l]);
    fi;
    fi;
    fi;
    else:
    l:=l+1;
    if Rotation:
    if AngleRotation<>0:
    picture TEXTELABEL;
    TEXTELABEL=image(
    labeloffset:=labeloffset*2;
    label.lft(LATEX(p_),A[l]);
    labeloffset:=labeloffset/2;
    );
    trace rotation(TEXTELABEL,A[l],AngleRotation);
    else :
    label.bot(LATEX(p_),A[l]);
    fi;
    fi;
    fi;
    endfor;
    enddef;
    if Qualitatif: tutu(#1); else: toto(#1); fi;
    boolean Grille;
    Grille:=\useKV[ClesStat]{Grille};
    Pasx:=\useKV[ClesStat]{Pasx};
    Pasy:=\useKV[ClesStat]{Pasy};    
    if Grille:
    drawoptions(withcolor 0.75white);
    for k=0 step Pasx until ((maxx+1)):
    trace (k*unitex,0)--(k*unitex,unitey*(maxy+2*Pasy));
    endfor;
    for k=0 step Pasy until ((maxy+2*Pasy)):
    trace (0,k*unitey)--(unitex*(maxx+1),k*unitey);
    endfor;
    drawoptions();
    fi;
    if epaisseurbatons<>0:
    for k=1 upto n:
    fill polygone(A[k]-(epaisseurbatons*1pt,0),A[k]+(epaisseurbatons*1pt,0),P[k]+(epaisseurbatons*1pt,0),P[k]-(epaisseurbatons*1pt,0)) withcolor if unknown Col[k]: CoulDefaut else:Col[k] fi;
    if AideLecture:
    draw B[k]--P[k] dashed evenly;
    fi;
    endfor;
    fi;
    if LectureFine:
    for k=0 step Pasy until ((maxy+1*Pasy)):
    if Tiret:
    trace (1pt,k*unitey)--(-1pt,k*unitey);
    label.lft(LATEX("\num{"&decimal(k)&"}"),(0,k*unitey));
    else:
    dotlabel.lft(LATEX("\num{"&decimal(k)&"}"),(0,k*unitey));
    fi;
    endfor;
    fi;
    if Lecture:
    for k=0 step Pasy until Pasy:
    if Tiret:
    trace (1pt,k*unitey)--(-1pt,k*unitey);
    label.lft(LATEX("\num{"&decimal(k)&"}"),(0,k*unitey));
    else:
    dotlabel.lft(LATEX("\num{"&decimal(k)&"}"),(0,k*unitey));
    fi;
    endfor;
    fi;
    drawarrow (0,0)--unitex*(maxx+1,0);
    drawarrow (0,0)--unitey*(0,maxy+2*Pasy);
    label.lrt(btex \useKV[ClesStat]{Donnee} etex,unitex*(maxx+1,0));
    label.urt(btex \useKV[ClesStat]{Effectif} etex,unitey*(0,maxy+2*Pasy));
    if Qualitatif: titi(#1); else:tata(#1); fi;
  \end{mpost}
  \fi
}

\def\NewMPStatCirculaireCodeQ{%
  Rayon:=\useKV[ClesStat]{Rayon};
  ecarthachures=\useKV[ClesStat]{EcartHachures};
  epaisseurhachures=\useKV[ClesStat]{EpaisseurHachures};
  boolean AffichageAngle,AffichageDonnee,Hachures,Inverse,Legende,LegendeVide,Retour,ACompleter;
  AffichageAngle=\useKV[ClesStat]{AffichageAngle};
  AffichageDonnee=\useKV[ClesStat]{AffichageDonnee};
  Hachures=\useKV[ClesStat]{Hachures};
  Inverse=\useKV[ClesStat]{LectureInverse};
  Legende=\useKV[ClesStat]{Legende};
  LegendeVide=\useKV[ClesStat]{LegendeVide};
  Retour=false;
  ACompleter=\useKV[ClesStat]{ACompleter};
  DebutAngle=\useKV[ClesStat]{DebutAngle};
  % 
  pair A[],O,B[],C[],D[];
  O=(0,0);
  n:=0;
  numeric total[],ang[];
  total[0]=0;
  ang[0]:=0;
  path cc;
  cc=(fullcircle scaled (2*Rayon));
  %
  vardef AfficheLegende(text t)=
  picture ResultatLegende;
  ResultatLegende=image(
  for p_=t:
  if string p_:
  n:=n+1;
  C[n]=A[n-1] rotatedabout(O,if Inverse:-1* fi(ang[n]-ang[n-1])/2);
  draw 0.95[O,C[n]]--1.05[O,C[n]];
  C[n]:=1.05[O,C[n]];
  Test(n);
  if ((xpart(C[n])>xpart(O)) or (xpart(C[n])=xpart(O))) and ((ypart(C[n])>ypart(O)) or (ypart(C[n])=ypart(O))):
  D[n]=C[n]+(0.5cm,0);
  draw C[n]--D[n];
  if Retour=false:label.urt(TEX(p_),D[n]);fi;
  fi;
  if (xpart(C[n])<xpart(O)) and ((ypart(C[n])>ypart(O)) or (ypart(C[n])=ypart(O))):
  D[n]=C[n]-(0.5cm,0);
  draw C[n]--D[n];
  if Retour=false:label.ulft(TEX(p_),D[n]);fi;
  fi;
  if (xpart(C[n])<xpart(O)) and (ypart(C[n])<ypart(O)):
  D[n]=C[n]-(0.5cm,0);
  draw C[n]--D[n];
  if Retour=false:label.llft(TEX(p_),D[n]);fi;
  fi;
  if ((xpart(C[n])>xpart(O)) or (xpart(C[n])=xpart(O))) and (ypart(C[n])<ypart(O)):
  D[n]=C[n]+(0.5cm,0);
  draw C[n]--D[n];
  if Retour=false:label.lrt(TEX(p_),D[n]);fi;
  fi;
  fi;
  endfor;
  );
  ResultatLegende
  % fi;
  enddef;
  vardef AfficheLegendePDF(text t)=
  picture ResultatLegende;
  ResultatLegende=image(
  for p_=t:
  if string p_:
  n:=n+1;
  C[n]=A[n-1] rotatedabout(O,if Inverse:-1* fi(ang[n]-ang[n-1])/2);
  draw 0.95[O,C[n]]--1.05[O,C[n]];
  C[n]:=1.05[O,C[n]];
  Test(n);
  if (xpart(C[n])>xpart(O)) and ((ypart(C[n])>ypart(O)) or (ypart(C[n])=ypart(O))):
  D[n]=C[n]+(0.5cm,0);
  draw C[n]--D[n];
  if Retour=false:label.urt(LATEX(p_),D[n]);fi;
  fi;
  if (xpart(C[n])<xpart(O)) and ((ypart(C[n])>ypart(O)) or (ypart(C[n])=ypart(O))):
  D[n]=C[n]-(0.5cm,0);
  draw C[n]--D[n];
  if Retour=false:label.ulft(LATEX(p_),D[n]);fi;
  fi;
  if (xpart(C[n])<xpart(O)) and (ypart(C[n])<ypart(O)):
  D[n]=C[n]-(0.5cm,0);
  draw C[n]--D[n];
  if Retour=false:label.llft(LATEX(p_),D[n]);fi;
  fi;
  if (xpart(C[n])>xpart(O)) and (ypart(C[n])<ypart(O)):
  D[n]=C[n]+(0.5cm,0);
  draw C[n]--D[n];
  if Retour=false:label.lrt(LATEX(p_),D[n]);fi;
  fi;
  fi;
  endfor;
  );
  ResultatLegende
  % fi;
  enddef;
}

% la construction du graphique qualitatif
% la construction du graphique qualitatif
\def\NewMPStatCirculaireQ#1#2#3#4#5{%
  %#1 : la liste des données
  %#2 : 360 ou 180
  %#3 : liste des couleurs
  %#4 : liste des légendes à effacer.
  %#5 : liste des angles des hachures
  \ifluatex
  \mplibforcehmode
  \begin{mplibcode}
    defaultcolormodel := \useKV[ClesStat]{ModeleCouleur};
    
    \NewMPStatCirculaireCodeQ
%    DebutAngle=\useKV[ClesStat]{DebutAngle};
    if Inverse=false:
    A[0]=point(0+DebutAngle) of cc;
    else:
    A[0]=point(180+DebutAngle) of cc;
    fi;
    % on r\'ecup\`ere les couleurs
    color Col[];
    n:=0;
    for p_=#3:
    n:=n+1;
    Col[n]=p_;
    endfor;
    % on r\'ecup\`ere les angles d'hachures
    numeric anglehach[];
    n:=0;
    for p_=#5:
    n:=n+1;
    anglehach[n]=p_;
    endfor;
    vardef toto(text t)=
    n:=0;
    for p_=t:
    if numeric p_:
    n:=n+1;
    total[n]:=total[n-1]+p_;
    fi;
    endfor;
    N=n;
    for k=1 upto N:
    ang[k]=(#2/total[N])*total[k];
    endfor;
    n:=0;
    for p_=t:
    if numeric p_:
    n:=n+1;
    if Inverse=false:
    A[n]=A[n-1] rotatedabout(O,p_*(#2/total[N]));
    else:
    A[n]=A[n-1] rotatedabout(O,-p_*(#2/total[N]));
    fi;
    %hachure ou pas ?
    if Hachures=false:
    fill (O--if Inverse=false:arccercle(A[n-1],A[n],O) else:
    arccercle(A[n],A[n-1],O) fi--cycle) withcolor if unknown Col[n]: white else:Col[n] fi;
    else:
    draw
    Hachurage((O--if Inverse=false:arccercle(A[n-1],A[n],O)
    else:arccercle(A[n],A[n-1],O) fi--cycle),if unknown anglehach[n]:p_*(#2/total[N]) if
    (n mod 2)=0: +90 else: -90 fi else: anglehach[n] fi,ecarthachures,if (n mod 2)=0 : 0 else: 1 fi)
    withpen pencircle scaled epaisseurhachures if AffichageAngle: withcolor 0.5white fi;
    fi;
    if ACompleter=false:
    draw A[n-1]--O--A[n] if Hachures: withpen pencircle scaled2 fi;
    fi;
    % Affichage des angles associ\'es
    if AffichageAngle:
    if round(p_*(#2/total[N]))>15:
    if (n mod 2)=0:
    marque_a:=20*0.75*Rayon/cm;
    else:
    marque_a:=20*0.5*Rayon/cm;
    fi;
    if Hachures:
    if Inverse=false:
    undraw
    Codeangle(A[n-1],O,A[n],0,(((TEX("\ang{"&decimal(round(p_*(#2/total[N])))&"}")))));
    else:
    undraw
    Codeangle(A[n],O,A[n-1],0,(((TEX("\ang{"&decimal(round(p_*(#2/total[N])))&"}")))));
    fi;
    fill cercles(w shifted(marque_ang*unitvector(w-O)),3mm) withcolor
    blanc;
    fi;
    if Inverse=false:
    draw
    Codeangle(A[n-1],O,A[n],0,(((TEX("\ang{"&decimal(round(p_*(#2/total[N])))&"}")))));
    else:
    draw
    Codeangle(A[n],O,A[n-1],0,(((TEX("\ang{"&decimal(round(p_*(#2/total[N])))&"}")))));
    fi;
    fi;
    elseif AffichageDonnee:
    if round(p_*(#2/total[N]))>15:
    if (n mod 2)=0:
    marque_a:=20*0.75*Rayon/cm;
    else:
    marque_a:=20*0.5*Rayon/cm;
    fi;
    if Hachures:
    if Inverse=false:
    undraw
    Codeangle(A[n-1],O,A[n],0,TEX(""&decimal(p_)&""));
    else:
    undraw
    Codeangle(A[n],O,A[n-1],0,(((TEX(""&decimal(p_)&"")))));
    fi;
    fill cercles(w shifted(marque_ang*unitvector(w-O)),3mm) withcolor
    blanc;
    fi;
    if Inverse=false:
    draw
    Codeangle(A[n-1],O,A[n],0,(((TEX(""&decimal(p_)&"")))));
    else:
    draw
    Codeangle(A[n],O,A[n-1],0,(((TEX(""&decimal(p_)&"")))));
    fi;
    fi;
    fi;
    %
    fi;
    endfor;
    if #2=360:
    draw cc if Hachures: withpen pencircle scaled2 fi;
    else:
    draw (subpath(0,length cc/2) of cc)--cycle if Hachures: withpen pencircle scaled2 fi;;
    fi;
    n:=0;
    enddef;
    vardef Test(expr nb)=
    Retour:=false;
    op:=0;
    for l_=#4:
    if l_=nb:
    op:=op+1;
    fi;
    endfor;
    if op>0:
    Retour:=true;
    fi;
    enddef;
    Figure(-10u,-10u,10u,10u);
    toto(#1);
    if Legende:
    n:=0;
    draw AfficheLegende(#1);
    fi;
  \end{mplibcode}
  \else
    \begin{mpost}[mpsettings={\NewMPStatCirculaireCodeQ}]
      if Inverse=false:
    A[0]=point(0+DebutAngle) of cc;
    else:
    A[0]=point(180+DebutAngle) of cc;
    fi;
    pair A[],O,B[],C[],D[];
    O=(0,0);
    n:=0;
    numeric total[],ang[];
    total[0]=0;
    ang[0]:=0;
    path cc;
    cc=(fullcircle scaled (2*Rayon));
    % on r\'ecup\`ere les couleurs
    color Col[];
    n:=0;
    for p_=#3:
    n:=n+1;
    Col[n]=p_;
    endfor;
    % on r\'ecup\`ere les angles d'hachures
    numeric anglehach[];
    n:=0;
    for p_=#5:
    n:=n+1;
    anglehach[n]=p_;
    endfor;
    if Inverse=false:
    A[0]=point(0) of cc;
    else:
    A[0]=point(180) of cc;
    fi;
    vardef toto(text t)=
    n:=0;
    for p_=t:
    if numeric p_:
    n:=n+1;
    total[n]:=total[n-1]+p_;
    fi;
    endfor;
    N=n;
    for k=1 upto N:
    ang[k]=(#2/total[N])*total[k];
    endfor;
    n:=0;
    for p_=t:
    if numeric p_:
    n:=n+1;
    if Inverse=false:
    A[n]=A[n-1] rotatedabout(O,p_*(#2/total[N]));
    else:
    A[n]=A[n-1] rotatedabout(O,-p_*(#2/total[N]));
    fi;
    %hachure ou pas ?
    if Hachures=false:
    fill (O--if Inverse=false:arccercle(A[n-1],A[n],O) else:
    arccercle(A[n],A[n-1],O) fi--cycle) withcolor if unknown Col[n]: white else:Col[n] fi;
    else:
    draw
    Hachurage((O--if Inverse=false:arccercle(A[n-1],A[n],O)
    else:arccercle(A[n],A[n-1],O) fi--cycle),if unknown anglehach[n]:p_*(#2/total[N]) if
    (n mod 2)=0: +90 else: -90 fi else: anglehach[n] fi,ecarthachures,if (n mod 2)=0 : 0 else: 1 fi)
    withpen pencircle scaled epaisseurhachures if AffichageAngle: withcolor 0.5white fi;
    fi;
    if ACompleter=false:
    draw A[n-1]--O--A[n] if Hachures: withpen pencircle scaled2 fi;
    fi;
    % Affichage des angles associ\'es
    if AffichageAngle:
    if round(p_*(#2/total[N]))>15:
    if (n mod 2)=0:
    marque_a:=20*0.75*Rayon/cm;
    else:
    marque_a:=20*0.5*Rayon/cm;
    fi;
    if Hachures:
    if Inverse=false:
    undraw
    Codeangle(A[n-1],O,A[n],0,(((LATEX("\ang{"&decimal(round(p_*(#2/total[N])))&"}")))));
    else:
    undraw
    Codeangle(A[n],O,A[n-1],0,(((LATEX("\ang{"&decimal(round(p_*(#2/total[N])))&"}")))));
    fi;
    fill cercles(w shifted(marque_ang*unitvector(w-O)),3mm) withcolor
    blanc;
    fi;
    if Inverse=false:
    draw
    Codeangle(A[n-1],O,A[n],0,(((LATEX("\ang{"&decimal(round(p_*(#2/total[N])))&"}")))));
    else:
    draw
    Codeangle(A[n],O,A[n-1],0,(((LATEX("\ang{"&decimal(round(p_*(#2/total[N])))&"}")))));
    fi;
    fi;
    elseif AffichageDonnee:
    if round(p_*(#2/total[N]))>15:
    if (n mod 2)=0:
    marque_a:=20*0.75*Rayon/cm;
    else:
    marque_a:=20*0.5*Rayon/cm;
    fi;
    if Hachures:
    if Inverse=false:
    undraw
    Codeangle(A[n-1],O,A[n],0,LATEX(""&decimal(p_)&""));
    else:
    undraw
    Codeangle(A[n],O,A[n-1],0,(((LATEX(""&decimal(p_)&"")))));
    fi;
    fill cercles(w shifted(marque_ang*unitvector(w-O)),3mm) withcolor
    blanc;
    fi;
    if Inverse=false:
    draw
    Codeangle(A[n-1],O,A[n],0,(((LATEX(""&decimal(p_)&"")))));
    else:
    draw
    Codeangle(A[n],O,A[n-1],0,(((LATEX(""&decimal(p_)&"")))));
    fi;
    fi;
    fi;
    %
    fi;
    endfor;
    if #2=360:
    draw cc if Hachures: withpen pencircle scaled2 fi;
    else:
    draw (subpath(0,length cc/2) of cc)--cycle if Hachures: withpen pencircle scaled2 fi;;
    fi;
    enddef;
    vardef Test(expr nb)=
    Retour:=false;
    op:=0;
    for l_=#4:
    if l_=nb:
    op:=op+1;
    fi;
    endfor;
    if op>0:
    Retour:=true;
    fi;
    enddef;    
    Figure(-10u,-10u,10u,10u);
    toto(#1);
    if Legende:
    n:=0;
    draw AfficheLegendePDF(#1);
    fi;
  \end{mpost}
  \fi
}%

%Pour la m\'ediane.
\DTLgnewdb{mtdb}%
\dtlexpandnewvalue%
\newcount\nbdonnees%
% 
\def\AjoutListEEaa#1\nil{\addtotok\tabtoksEEa{#1,}}%
\def\AjoutListEEab#1\nil{\addtotok\tabtoksEEa{#1/}}%
\def\AjoutListEEb#1\nil{\addtotok\tabtoksEEb{#1,}}%
\def\AjoutListEEx#1\nil{\addtotok\tabtoksEE{#1,}}%
\def\AjoutListEEy#1\nil{\addtotok\tabtoksEE{#1/}}%

\DTLgnewdb{mtdbEE}%
\DTLgnewdb{mtdbEEqual}%
%

% Pour les classes
% Pour construire l'histogramme
\def\UpdatetoksHisto#1/#2/#3\nil{\addtotok\toklisteelmtsclasse{#1,#2,}\addtotok\toklistedonhisto{#3,}}
\def\UpdatetoksECC#1\nil{\addtotok\toklistedonhisto{#1,}}

\NewDocumentCommand\buildgraphhisto{}{%
  \newtoks\toklisteelmtsclasse%
  \newtoks\toklistedonhisto%
  \newtoks\toklistecouleur%
  \newtoks\toklistelegende%
  \ifboolKV[ClesStat]{LegendeVide}{%
    \xdef\foo{\useKV[ClesStat]{LegendesVides}}%
    \readlist*\ListeLegendesAEffacer{\foo}%
  }{\xdef\foo{-1}\readlist*\ListeLegendesAEffacer{\foo}%
  }%
  \foreachitem\compteur\in\ListeLegendesAEffacer{\expandafter\UpdateLegende\compteur\nil}%
  \foreachitem\compteur\in\ListeDepart{\expandafter\UpdatetoksHisto\compteur\nil}%
  \ifboolKV[ClesStat]{ECC}{%
    \toklistedonhisto{}%
    \xdef\PfCFooECC{\ListeDepart[1,3]}%
    \xintFor* ##1 in{\xintSeq{2}{\ListeDepartlen}}\do{%
      \xdef\PfCFooRetiens{0}%
      \xintFor* ##2 in{\xintSeq{1}{##1}}\do{%
        \xdef\PfCFooRetiens{\fpeval{\PfCFooRetiens+\ListeDepart[##2,3]}}%
      }%
      \xdef\PfCFooECC{\PfCFooECC,\PfCFooRetiens}%
    }%
    \readlist*\PfCListeECC{\PfCFooECC}%
    \foreachitem\compteur\in\PfCListeECC{\expandafter\UpdatetoksECC\compteur\nil}%
  }{}%
  \xdef\PfCEcartClasse{\fpeval{\ListeDepart[1,2]-\ListeDepart[1,1]}}%
  \foreachitem\compteur\in\ListeDepart{%
    \xdef\PfCEcartClasse{\PfCEcartClasse,\fpeval{\ListeDepart[\compteurcnt,2]-\ListeDepart[\compteurcnt,1]}}
  }%
  \xintifboolexpr{\fpeval{min(\PfCEcartClasse)}==\fpeval{max(\PfCEcartClasse)}}{}{\setKV[ClesStat]{MemeAmpli=false}}
  % Pour les couleurs
  \xdef\ListeAvantCouleurs{\useKV[ClesStat]{ListeCouleurs}}%
  \readlist*\ListeCouleur{\ListeAvantCouleurs}%
  \foreachitem\couleur\in\ListeCouleur{\expandafter\UpdateCoul\couleur\nil}%
  %
  \MPBuildHisto{\the\toklisteelmtsclasse}{\the\toklistedonhisto}{\the\toklistecouleur}{\the\toklistelegende}%
}

%% calcul des fr\'equences
\newcommand\CalculFrequenceClasses[1]{%
  \fpeval{round(\ListeDepart[#1,3]*100/\EffectifTotal,\useKV[ClesStat]{PrecisionF})}
}

\newcommand\CalculECCClasses[1]{%
  \xdef\TotalECCC{0}%
  \CompteurECCC=1%
  \CompteurECCCTotal=\numexpr#1+1%
  \whiledo{\CompteurECCC < \CompteurECCCTotal}{%
    \xdef\TotalECCC{\fpeval{\TotalECCC+\ListeDepart[\the\CompteurECCC,3]}}%
    \CompteurECCC=\numexpr\CompteurECCC+1%
  }%
  \num{\TotalECCC}%
}

\NewDocumentCommand\buildtabclasses{}{%
  \setcounter{PfCCompteLignes}{0}%
  \renewcommand{\arraystretch}{\useKV[ClesStat]{Stretch}}%
  \begin{NiceTabular}{l*{\ListeDepartlen}{c}}%[hvlines]
    \CodeBefore%
    \rowcolor{\useKV[ClesStat]{CouleurTab}}{1}%
    \columncolor{\useKV[ClesStat]{CouleurTab}}{1}%
    \Body
    \useKV[ClesStat]{Donnee}\xintFor* ##1 in{\xintSeq{1}{\ListeDepartlen}}\do{%
      &\ifboolKV[ClesStat]{Crochets}{$[\num{\ListeDepart[##1,1]}\,;\,\num{\ListeDepart[##1,2]}[$}{$\num{\ListeDepart[##1,1]}\leqslant\dots<\num{\ListeDepart[##1,2]}$}%
    }\\
    \useKV[ClesStat]{Effectif}\xintFor* ##1 in{\xintSeq{1}{\ListeDepartlen}}\do{%
      &\ifboolKV[ClesStat]{EffVide}{}{\num{\ListeDepart[##1,3]}}%
      }\\
      % centre
      \ifboolKV[ClesStat]{Centre}{\stepcounter{PfCCompteLignes}Centre de la classe\xintFor* ##1 in {\xintSeq {1}{\ListeDepartlen}}\do{&\ifboolKV[ClesStat]{TableauVide}{}{\ifboolKV[ClesStat]{CentreVide}{}{\num{\fpeval{\ListeDepart[##1,2]-\ListeDepart[##1,1]}}}}}\\}{}%
      %
    \ifboolKV[ClesStat]{Frequence}{\stepcounter{PfCCompteLignes}Fr\'equence (\%)\xintFor* ##1 in {\xintSeq {1}{\ListeDepartlen}}\do{&\ifboolKV[ClesStat]{TableauVide}{}{\ifboolKV[ClesStat]{FreqVide}{}{\num{\CalculFrequenceClasses{##1}}}}}\\
    }{}%
    \ifboolKV[ClesStat]{ECC}{\stepcounter{PfCCompteLignes}E.C.C.\xintFor* ##1 in {\xintSeq {1}{\ListeDepartlen}}\do{&\ifboolKV[ClesStat]{TableauVide}{}{\ifboolKV[ClesStat]{ECCVide}{}{\CalculECCClasses{##1}}}}\\}{}%
    \CodeAfter%
    % On crée la liste des colonnes à vider
    \xintifboolexpr{\useKV[ClesStat]{ColVide}>0}{%
      \xdef\FooStat{\useKV[ClesStat]{ColVide}}%
      \setsepchar{,}%
      \readlist*\ListeColonnesAVider{\FooStat}%
      \foreachitem\compteur\in\ListeColonnesAVider{%
        \tikz\fill[white] (row-2-|col-\fpeval{\compteur+1}) rectangle (last-|col-\fpeval{\compteur+2});%
      }%
    }{}%
%    % On crée la liste des cases à vider
    \ifboolKV[ClesStat]{CaseVide}{%
      \xdef\FooStatCases{\useKV[ClesStat]{CasesVides}}%
      \setsepchar[*]{,*/}%
      \readlist*\ListeCasesAVider{\FooStatCases}%
      \foreachitem\compteur\in\ListeCasesAVider{%
        \tikz\fill[white] (row-\fpeval{\ListeCasesAVider[\compteurcnt,1]+1}-|col-\fpeval{\ListeCasesAVider[\compteurcnt,2]+1}) rectangle (row-\fpeval{\ListeCasesAVider[\compteurcnt,1]+2}-|col-\fpeval{\ListeCasesAVider[\compteurcnt,2]+2});%
      }%
    }{}%
%    % On retrace le tableau
%    % Les colonnes
    \xintFor* ##1 in {\xintSeq{1}{\fpeval{\ListeDepartlen+2}}}\do{%
      \tikz\draw (row-1-|col-##1) -- (last-|col-##1);%
    }%
%    % Les lignes
    \xintFor* ##1 in {\xintSeq{1}{\fpeval{\thePfCCompteLignes+3}}}\do{%
      \tikz\draw (row-##1-|col-1) -- (row-##1-|last);%
    }%
  \end{NiceTabular}
}%

\NewDocumentCommand\MPBuildHisto{mmmm}{%
  \ifluatex
%    \mplibnumbersystem{double}
  \mplibforcehmode
  \begin{mplibcode}
    defaultcolormodel := \useKV[ClesStat]{ModeleCouleur};
    
    maxx:=-infinity;
    minx:=infinity;
    maxy:=-infinity;
    miny:=0;
    unitex:=\useKV[ClesStat]{Unitex}*cm;
    unitey:=\useKV[ClesStat]{Unitey}*cm;
    Pasx:=\useKV[ClesStat]{Pasx};
    Pasy:=\useKV[ClesStat]{Pasy};
    UniteAire=\useKV[ClesStat]{UniteAire};
    Ecarthachures=\useKV[ClesStat]{EcartHachures};
    Epaisseurhachures=\useKV[ClesStat]{EpaisseurHachures};
    boolean MemeAmpli,Hachures,Lecture,LectureFine,AideLecture,DonneesSup,Tiret,LegendeVide,Retour,Mediane,ECC;
    ECC=\useKV[ClesStat]{ECC};
    Mediane=\useKV[ClesStat]{Mediane};
    MemeAmpli=\useKV[ClesStat]{MemeAmpli};
    Hachures:=\useKV[ClesStat]{Hachures};
    %
    Lecture:=\useKV[ClesStat]{Lecture};
    LectureFine:=\useKV[ClesStat]{LectureFine};
    Tiret=\useKV[ClesStat]{Tiret};
    AideLecture:=\useKV[ClesStat]{AideLecture};
    DonneesSup:=\useKV[ClesStat]{DonneesSup};
    LegendeVide=\useKV[ClesStat]{LegendeVide};
    Retour=false;
    % Test affichage
    vardef Test(expr nb)=
    Retour:=false;
    op:=0;
    for l_=#4:
    if l_=nb:
    op:=op+1;
    fi;
    endfor;
    if op>0:
    Retour:=true;
    fi;
    enddef;
    %Affichage ou pas des légendes
    vardef AfficheLegende(text t)=
    l=0;
    for p_=t:
    l:=l+1;
    if DonneesSup:
    Test(l);
    if Retour=false:
    label.top(TEX("\num{"&decimal(Y[l])&"}"),(unitex*(Depart+(0.5*(X[2*l]+X[2*l-2])-X[1])/Pasx),unitey*Z[l]));
    fi;
    fi;
    endfor;
    enddef;
    Depart=\useKV[ClesStat]{DepartHisto};
    % on r\'ecup\`ere les couleurs
    color Col[],CoulDefaut;
    CoulDefaut=white;
    n:=0;
    for p_=#3:
    n:=n+1;
    Col[n]=p_;
    endfor;
    %
    numeric X[];
    numeric ecartabs[];
    vardef RecupValeursAbscisses(text t)=
    p:=0;
    for p_=t:
    p:=p+1;
    X[p]:=p_;
    if X[p]<minx:
    minx:=X[p];
    fi;
    if X[p]>maxx:
    maxx:=X[p];
    fi;
    endfor;
    X[0]=X[1];
    TotalAbscisses=p;
    enddef;
    numeric Y[],Z[];
    numeric EffectifTotal[];
    numeric EffectifTotalA[];
    vardef RecupValeursDonnees(text t)=
    p:=0;
    EffectifTotal[0]:=0;
    EffectifTotalA[0]:=0; 
    for p_=t:
    p:=p+1;
    EffectifTotal[p]:=p_;
    EffectifTotalA[p]:=EffectifTotalA[p-1]+p_;
    Y[p]:=p_;
    Z[p]=(Y[p]/(UniteAire*(X[2*p]-X[2*p-1])/Pasx));
    R[p]=ceiling(Z[p]);
    if R[p]>maxy:
    maxy:=R[p];
    fi;
    endfor;
    NbDonnees:=p;
    enddef;
    %On affiche la médiane dans le cas des ECC
    vardef AfficheMedianeECC(text t)=
    YMed:=Z[NbDonnees]/2;
    DemiDonnees:=EffectifTotal[NbDonnees]/2;
    p:=0;
    forever:
    p:=p+1;
    exitif EffectifTotal[p]>DemiDonnees;
    endfor;
    path MedHor,MedLineaire,MedVer;
    MedLineaire=(unitex*(Depart+(X[2*p-2]-X[1]/Pasx)),unitey*Z[p-1])--(unitex*(Depart+(X[2*p]-X[1])/Pasx),unitey*Z[p]);
    MedHor=((0,unitey*YMed)--(unitex*((maxx-minx)/Pasx+2),unitey*YMed));
    MedVer=(xpart(MedLineaire intersectionpoint MedHor),0)--(MedLineaire intersectionpoint MedHor);
    draw MedLineaire;
    draw MedHor;
    draw MedVer;
    enddef;
    % On affiche la médiane dans le cas non ECC
    vardef AfficheMediane(text t)=
    DemiDonnees:=EffectifTotalA[NbDonnees]/2;
    p:=0;
    forever:
    p:=p+1;
    exitif EffectifTotalA[p]>DemiDonnees;
    endfor;
    path MedVer;
    numeric CoefLineaire,pMed;
    pMed=p;
    CoefLineaire=(DemiDonnees-EffectifTotalA[p-1])/Y[p];
    MedVer=(unitex*(Depart+(X[2*p-2]-X[1])/Pasx+CoefLineaire*(X[2*p]-X[2*p-2])/Pasx),0)--(unitex*(Depart+(X[2*p-2]-X[1])/Pasx+CoefLineaire*(X[2*p]-X[2*p-2])/Pasx),unitey*Z[p]);
    draw MedVer dashed evenly;
    enddef;
    % On commence le tracé : on récupère les informations
    RecupValeursAbscisses(#1);
    RecupValeursDonnees(#2);
    % on définit une grille
    vardef Grille=
    if MemeAmpli:
    Ajout:=1;
    else:
    Ajout:=3;
    fi;
    drawoptions(withcolor 0.7white);
    for k=0 upto ((maxx-minx)/Pasx+2+Depart):
    trace (unitex*k,0)--(unitex*k,(maxy+Ajout)*unitey);%withcolor red;
    endfor;
    for k=0 upto (maxy+Ajout):
    trace (0,k*unitey)--(unitex*((maxx-minx)/Pasx+2+Depart),k*unitey);% withcolor blue;
    endfor;
    drawoptions();
    enddef;
    % Fin Grille
    % On trace les rectangles
    vardef AfficheRectangles=
    if Hachures:
    Grille;
    for k=2 step 2 until TotalAbscisses:
    draw hachurage(polygone(unitex*(Depart+(X[k]-X[1])/Pasx,0),(unitex*(Depart+(X[k]-X[1])/Pasx),unitey*(Y[k/2]/(UniteAire*(X[k]-X[k-1])/Pasx))),(unitex*(Depart+(X[k-1]-X[1])/Pasx),unitey*(Y[k/2]/(UniteAire*(X[k]-X[k-1])/Pasx))),(unitex*(Depart+(X[k-1]-X[1])/Pasx),0)),60,0.2,0);
    draw chemin(unitex*(Depart+(X[k]-X[1])/Pasx,0),(unitex*(Depart+(X[k]-X[1])/Pasx),unitey*(Y[k/2]/(UniteAire*(X[k]-X[k-1])/Pasx))),(unitex*(Depart+(X[k-1]-X[1])/Pasx),unitey*(Y[k/2]/(UniteAire*(X[k]-X[k-1])/Pasx))),(unitex*(Depart+(X[k-1]-X[1])/Pasx),0));
    endfor;
    else:
    for k=2 step 2 until TotalAbscisses:
    fill polygone(unitex*(Depart+(X[k]-X[1])/Pasx,0),(unitex*(Depart+(X[k]-X[1])/Pasx),unitey*(Y[k/2]/(UniteAire*(X[k]-X[k-1])/Pasx))),(unitex*(Depart+(X[k-1]-X[1])/Pasx),unitey*(Y[k/2]/(UniteAire*(X[k]-X[k-1])/Pasx))),(unitex*(Depart+(X[k-1]-X[1])/Pasx),0)) withcolor if unknown Col[k/2]: CoulDefaut else: Col[k/2] fi;
    endfor;
    Grille;
    for k=2 step 2 until TotalAbscisses:
    draw chemin(unitex*(Depart+(X[k]-X[1])/Pasx,0),(unitex*(Depart+(X[k]-X[1])/Pasx),unitey*(Y[k/2]/(UniteAire*(X[k]-X[k-1])/Pasx))),(unitex*(Depart+(X[k-1]-X[1])/Pasx),unitey*(Y[k/2]/(UniteAire*(X[k]-X[k-1])/Pasx))),(unitex*(Depart+(X[k-1]-X[1])/Pasx),0));
    endfor;
    fi;
    enddef;
    % Affichage final
    if ECC:
      AfficheRectangles;
      if Mediane:
        AfficheMedianeECC(#2);
      fi;
    else:
      if Mediane:
        AfficheMediane(#2);
        if Hachures:
        Grille;
        %Partie gauche
          for k=2 step 2 until (2*pMed-2):
            draw hachurage(polygone(unitex*(Depart+(X[k]-X[1])/Pasx,0),(unitex*(Depart+(X[k]-X[1])/Pasx),unitey*(Y[k/2]/(UniteAire*(X[k]-X[k-1])/Pasx))),(unitex*(Depart+(X[k-1]-X[1])/Pasx),unitey*(Y[k/2]/(UniteAire*(X[k]-X[k-1])/Pasx))),(unitex*(Depart+(X[k-1]-X[1])/Pasx),0)),60,0.2,0);
            draw chemin(unitex*(Depart+(X[k]-X[1])/Pasx,0),(unitex*(Depart+(X[k]-X[1])/Pasx),unitey*(Y[k/2]/(UniteAire*(X[k]-X[k-1])/Pasx))),(unitex*(Depart+(X[k-1]-X[1])/Pasx),unitey*(Y[k/2]/(UniteAire*(X[k]-X[k-1])/Pasx))),(unitex*(Depart+(X[k-1]-X[1])/Pasx),0));
          endfor;
          draw hachurage(polygone(unitex*(Depart+(X[2*pMed-2]-X[1])/Pasx,0),point(0) of MedVer,point(1) of MedVer,(unitex*(Depart+(X[2*pMed-2]-X[1])/Pasx),unitey*Z[pMed])),60,0.2,0);
          draw polygone(unitex*(Depart+(X[2*pMed-2]-X[1])/Pasx,0),point(0) of MedVer,point(1) of MedVer,(unitex*(Depart+(X[2*pMed-2]-X[1])/Pasx),unitey*Z[pMed]));
          % Partie droite
          for k=2*pMed+2 step 2 until TotalAbscisses:
            draw hachurage(polygone(unitex*(Depart+(X[k]-X[1])/Pasx,0),(unitex*(Depart+(X[k]-X[1])/Pasx),unitey*(Y[k/2]/(UniteAire*(X[k]-X[k-1])/Pasx))),(unitex*(Depart+(X[k-1]-X[1])/Pasx),unitey*(Y[k/2]/(UniteAire*(X[k]-X[k-1])/Pasx))),(unitex*(Depart+(X[k-1]-X[1])/Pasx),0)),120,0.2,1);
            draw chemin(unitex*(Depart+(X[k]-X[1])/Pasx,0),(unitex*(Depart+(X[k]-X[1])/Pasx),unitey*(Y[k/2]/(UniteAire*(X[k]-X[k-1])/Pasx))),(unitex*(Depart+(X[k-1]-X[1])/Pasx),unitey*(Y[k/2]/(UniteAire*(X[k]-X[k-1])/Pasx))),(unitex*(Depart+(X[k-1]-X[1])/Pasx),0));
          endfor;
          draw hachurage(polygone(unitex*(Depart+(X[2*pMed]-X[1])/Pasx,0),point(0) of MedVer,point(1) of MedVer,(unitex*(Depart+(X[2*pMed]-X[1])/Pasx),unitey*Z[pMed])),120,0.2,1);
          draw polygone(unitex*(Depart+(X[2*pMed]-X[1])/Pasx,0),point(0) of MedVer,point(1) of MedVer,(unitex*(Depart+(X[2*pMed]-X[1])/Pasx),unitey*Z[pMed]));
          else:
          %Partie gauche
    for k=2 step 2 until (2*pMed-2):
    fill polygone(unitex*(Depart+(X[k]-X[1])/Pasx,0),(unitex*(Depart+(X[k]-X[1])/Pasx),unitey*(Y[k/2]/(UniteAire*(X[k]-X[k-1])/Pasx))),(unitex*(Depart+(X[k-1]-X[1])/Pasx),unitey*(Y[k/2]/(UniteAire*(X[k]-X[k-1])/Pasx))),(unitex*(Depart+(X[k-1]-X[1])/Pasx),0)) withcolor if unknown Col[1]: CoulDefaut else: Col[1] fi;
    endfor;
    fill polygone(unitex*(Depart+(X[2*pMed-2]-X[1])/Pasx,0),point(0) of MedVer,point(1) of MedVer,(unitex*(Depart+(X[2*pMed-2]-X[1])/Pasx),unitey*Z[pMed])) withcolor if unknown Col[1]: CoulDefaut else: Col[1] fi ;
    % Partie droite
    for k=2*pMed+2 step 2 until TotalAbscisses:
    fill polygone(unitex*(Depart+(X[k]-X[1])/Pasx,0),(unitex*(Depart+(X[k]-X[1])/Pasx),unitey*(Y[k/2]/(UniteAire*(X[k]-X[k-1])/Pasx))),(unitex*(Depart+(X[k-1]-X[1])/Pasx),unitey*(Y[k/2]/(UniteAire*(X[k]-X[k-1])/Pasx))),(unitex*(Depart+(X[k-1]-X[1])/Pasx),0)) withcolor if unknown Col[2]: CoulDefaut else: Col[2] fi;
    endfor;
    fill polygone(unitex*(Depart+(X[2*pMed]-X[1])/Pasx,0),point(0) of MedVer,point(1) of MedVer,(unitex*(Depart+(X[2*pMed]-X[1])/Pasx),unitey*Z[pMed]))  withcolor if unknown Col[2]: CoulDefaut else: Col[2] fi;
    Grille;
    %Partie Gauche
    for k=2 step 2 until (2*pMed-2):
    trace polygone(unitex*(Depart+(X[k]-X[1])/Pasx,0),(unitex*(Depart+(X[k]-X[1])/Pasx),unitey*(Y[k/2]/(UniteAire*(X[k]-X[k-1])/Pasx))),(unitex*(Depart+(X[k-1]-X[1])/Pasx),unitey*(Y[k/2]/(UniteAire*(X[k]-X[k-1])/Pasx))),(unitex*(Depart+(X[k-1]-X[1])/Pasx),0));
    endfor;
    trace polygone(unitex*(Depart+(X[2*pMed-2]-X[1])/Pasx,0),point(0) of MedVer,point(1) of MedVer,(unitex*(Depart+(X[2*pMed-2]-X[1])/Pasx),unitey*Z[pMed]));%(Y[pMed]/(UniteAire*(X[2*pMed-2]-X[2*pMed-3])/Pasx))));
    for k=2*pMed+2 step 2 until TotalAbscisses:
    draw chemin(unitex*(Depart+(X[k]-X[1])/Pasx,0),(unitex*(Depart+(X[k]-X[1])/Pasx),unitey*(Y[k/2]/(UniteAire*(X[k]-X[k-1])/Pasx))),(unitex*(Depart+(X[k-1]-X[1])/Pasx),unitey*(Y[k/2]/(UniteAire*(X[k]-X[k-1])/Pasx))),(unitex*(Depart+(X[k-1]-X[1])/Pasx),0));
    endfor;
    draw polygone(unitex*(Depart+(X[2*pMed]-X[1])/Pasx,0),point(0) of MedVer,point(1) of MedVer,(unitex*(Depart+(X[2*pMed]-X[1])/Pasx),unitey*Z[pMed]));
    fi;
    draw MedVer withpen pencircle scaled 2;
      else:
        AfficheRectangles;
      fi;
    fi;
    %Affichage ou pas des axes, de la médiane
    if MemeAmpli:
      drawarrow (0,0)--unitey*(0,maxy+1);
      EcartAmpli:=(X[2]-X[1])/Pasx;
      if AideLecture:
      for k=2 step 2 until TotalAbscisses:
        trace ((unitex*(Depart+(X[k]-X[1])/Pasx),unitey*Z[k/2]))--(unitey*(0,Z[k/2])) dashed evenly;
        endfor;
      fi;
      if LectureFine:
        for k=0 upto ((maxy+1)):
          if Tiret:
            trace (1pt,k*unitey)--(-1pt,k*unitey);
            label.lft(TEX("\num{"&decimal(k*UniteAire*EcartAmpli)&"}"),(0,k*unitey));
          else:
            dotlabel.lft(TEX("\num{"&decimal(k*UniteAire*EcartAmpli)&"}"),(0,k*unitey));
          fi;
        endfor;
      fi;
      if Lecture:
        for k=0 upto 1:
        if Tiret:
          trace (1pt,k*unitey)--(-1pt,k*unitey);
          label.lft(TEX("\num{"&decimal(k*UniteAire*EcartAmpli)&"}"),(0,k*unitey));
    else:
    dotlabel.lft(TEX("\num{"&decimal(k*UniteAire*EcartAmpli)&"}"),(0,k*unitey));
    fi;
    endfor;
    fi; 
    else:%Pas même ampli : on n'affiche pas l'axe vertical
    trace hachurage(polygone((unitex,unitey*(maxy+2)),(unitex*2,unitey*(maxy+2)),(unitex*2,unitey*(maxy+1)),(unitex,unitey*(maxy+1))),60,0.2,0);
    trace polygone((unitex,unitey*(maxy+2)),(unitex*2,unitey*(maxy+2)),(unitex*2,unitey*(maxy+1)),(unitex,unitey*(maxy+1)));
    label.rt(TEX(decimal(UniteAire)&"~\useKV[ClesStat]{Effectif}"),(unitex*2,unitey*(maxy+1.5)));
%    if Mediane:
%    AfficheMediane(#2);
%    fi;
    fi;
    % On trace l'axe des abscisses
    drawarrow (0,0)--unitex*((maxx-minx)/Pasx+2+Depart,0);
    %On labelise l'axe des abscisses
    dotlabel.bot(TEX("\num{"&decimal(X[1])&"}"),unitex*(Depart,0));
    for k=2 step 2 until TotalAbscisses:
    dotlabel.bot(TEX("\num{"&decimal(X[k])&"}"),unitex*(Depart+(X[k]-X[1])/Pasx,0));
    endfor;
    label.rt(TEX("\useKV[ClesStat]{Donnee}"),(unitex*((maxx-minx)/Pasx+2+Depart),0));
    %On affiche les données sup ou pas.
    AfficheLegende(#2);
  \end{mplibcode}
  \fi
}
%

\newcommand\Stat[2][]{%
  \useKVdefault[ClesStat]%
  \setKV[ClesStat]{#1}%
  \ifboolKV[ClesStat]{UneMediane}{\renewcommand{\PfCArticleMediane}{une}}{\renewcommand{\PfCArticleMediane}{la}}%
  \setsepchar[*]{,*/}%
  \readlist*\ListeAvantUtilisation{#2}%
  \xintifboolexpr{\listlen\ListeAvantUtilisation[1]==3}{\setKV[ClesStat]{Classes}}{}%
  \ifboolKV[ClesStat]{Classes}{%
    \setsepchar[*]{,*/}%
    \readlist*\ListeDepart{#2}%
    \xdef\EffectifTotal{0}%
    \xintFor* ##1 in{\xintSeq{1}{\ListeDepartlen}}\do{%
      \xdef\EffectifTotal{\fpeval{\EffectifTotal+\ListeDepart[##1,3]}}%
    }%
    \ifboolKV[ClesStat]{Histogramme}{%
      \buildgraphhisto%
    }{%
      \ifboolKV[ClesStat]{Tableau}{%
        \buildtabclasses%
      }{}%
    }%
  }{%
    \setsepchar{,}%
    \ifboolKV[ClesStat]{Representation}{%
      \setKV[TraceG]{Xmin=0,Ymin=0}%
      \setKV[TraceG]{#1}%
      \readlist*\ListePointsPlaces{#2}%
      \newtoks\toklistepoint%
      \foreachitem\compteur\in\ListePointsPlaces{\expandafter\Updatetoks\compteur\nil}%
      \MPPlacePoint[#1]{\the\toklistepoint}%
    }{%
      \ifboolKV[ClesStat]{Liste}{%
        \setsepchar{,}\ignoreemptyitems%
        \readlist*\Liste{#2}%
        \xdef\foo{}%
        \setsepchar[*]{,*/}\ignoreemptyitems%
        \xintFor* ##1 in {\xintSeq {1}{\Listelen}}\do{%
          \xdef\foo{\foo 1/\Liste[##1],}%
        }%
        \readlist*\ListeComplete{\foo}%
        \setKV[ClesStat]{Qualitatif}%
      }{%
        \ifboolKV[ClesStat]{Sondage}{%
          \setsepchar{,}\ignoreemptyitems%
          \readlist*\Liste{#2}%
          % "liste vide"
          \newtoks\tabtoksEEa%
          \tabtoksEEa{}%
          % 
          % "liste vide"
          \newtoks\tabtoksEEb%
          \tabtoksEEb{}%
          % 
          \readlist*\ListeSansDoublonsEE{999}%   %% Pour ne pas avoir une liste vide
          % 
          \newcount\cmptEE%
          \newcount\PasNumEE%    %% Permettra de savoir si ce sondage est qualitatif ou quantitatif
          \PasNumEE=0\relax%
          \DTLcleardb{mtdbEE}%
          % on range les resultats du sondage par ordre croissant.
          \foreachitem\x\in\Liste{%
            \DTLnewrow{mtdbEE}%
            \DTLnewdbentry{mtdbEE}{Numeric}{\x}%
          }%
          \dtlsort{Numeric}{mtdbEE}{\dtlicompare}%
          \DTLforeach{mtdbEE}{\nba=Numeric}{%
            \IfDecimal{\nba}{}{\PasNumEE=\numexpr\PasNumEE+1\relax}%
            \cmptEE=0\relax%
            \foreachitem\nbb\in\ListeSansDoublonsEE{%
              \ifthenelse{\equal{\nba}{\nbb}}{\cmptEE=\numexpr\cmptEE+1\relax}{}%
            }%
            \ifthenelse{\equal{\the\cmptEE}{0}}{%
              \expandafter\AjoutListEEb\nba\nil%
              \xdef\listEEa{\the\tabtoksEEb}%
              \ignoreemptyitems%
              \setsepchar{,}%
              \readlist*\ListeSansDoublonsEE\listEEa%    	%%% Enl\`eve tous les \'elements
              %%% identiques de Liste
            }{}%
          }%
          \foreachitem\nba\in\ListeSansDoublonsEE{%
            \cmptEE=0\relax%
            \DTLforeach{mtdbEE}{\nbb=Numeric}{%
              \ifthenelse{\equal{\nba}{\nbb}}{\cmptEE=\numexpr\cmptEE+1\relax}{}%
            }%
            \expandafter\AjoutListEEab\nba\nil%
            \expandafter\AjoutListEEaa\the\cmptEE\nil% 	%%% Compte tous les \'elements
            %%% identiques de Liste
          }%
          \xdef\listEEb{\the\tabtoksEEa}
          \ignoreemptyitems%
          \setsepchar[*]{,*/}%
          \readlist*\ListeComplete\listEEb%
          % 
          \ifthenelse{\equal{\the\PasNumEE}{0}}{\setKV[ClesStat]{Quantitatif}}{\setKV[ClesStat]{Qualitatif}}%
        }{%
          \ifboolKV[ClesStat]{Qualitatif}{%
            %  % on lit la liste \'ecrite sous la forme valeur/effectif
            \setsepchar[*]{,*/}\ignoreemptyitems%
            \readlist*\ListeInitiale{#2}%
            % "liste vide"
            \newtoks\tabtoksEE%
            \tabtoksEE{}%
            \DTLcleardb{mtdbEEqual}%
            \foreachitem\x\in\ListeInitiale{%
              \DTLnewrow{mtdbEEqual}%
              \itemtomacro\ListeInitiale[\xcnt,1]\x%
              \DTLnewdbentry{mtdbEEqual}{Val}{\x}%
              \itemtomacro\ListeInitiale[\xcnt,2]\y%
              \DTLnewdbentry{mtdbEEqual}{Eff}{\y}%
            }%
            \DTLforeach{mtdbEEqual}{\Val=Val,\Eff=Eff}{%  
              \expandafter\AjoutListEEy\Val\nil%
              \expandafter\AjoutListEEx\Eff\nil%
            }%
            \xdef\listEE{\the\tabtoksEE}
            \ignoreemptyitems%
            \setsepchar[*]{,*/}%
            \readlist*\ListeComplete\listEE%
          }{% Dans le qualitatif, on trie d'abord les valeurs.
            \setsepchar[*]{,*/}\ignoreemptyitems%
            \readlist*\ListeInitiale{#2}%
            % "liste vide"
            \newtoks\tabtoksEE%
            \tabtoksEE{}%
            \DTLcleardb{mtdbEEqual}%
            \foreachitem\x\in\ListeInitiale{%
              \DTLnewrow{mtdbEEqual}%
              \itemtomacro\ListeInitiale[\xcnt,1]\x%
              \DTLnewdbentry{mtdbEEqual}{Val}{\x}%
              \itemtomacro\ListeInitiale[\xcnt,2]\y%
              \DTLnewdbentry{mtdbEEqual}{Eff}{\y}%
            }%
            \dtlsort{Val}{mtdbEEqual}{\dtlicompare}%
            \DTLforeach{mtdbEEqual}{\Val=Val,\Eff=Eff}{%  
              \expandafter\AjoutListEEy\Val\nil%
              \expandafter\AjoutListEEx\Eff\nil%
            }%
            \xdef\listEE{\the\tabtoksEE}
            \ignoreemptyitems%
            \setsepchar[*]{,*/}%
            \readlist*\ListeComplete\listEE%
          }}}%
      % on cr\'ee la base de donn\'ees des valeurs dans le cas qualitatif
      \DTLcleardb{mtdb}%
      % on les trie pour la m\'ediane dans le cas qualitatif % Touhami / Texnique.fr
      \foreachitem\x\in\ListeComplete{%
        \DTLnewrow{mtdb}%
        \itemtomacro\ListeComplete[\xcnt,2]\y%
        \DTLnewdbentry{mtdb}{Numeric}{\y}%
      }%
      \dtlsort{Numeric}{mtdb}{\dtlicompare}%
      %  % on r\'einitialise les valeurs des crit\`eres de position et de
      % dispersion
      \renewcommand\NbDonnees{}%
      \renewcommand\SommeDonnees{}%
      \renewcommand\EffectifTotal{}%
      \renewcommand\Moyenne{}%
      \renewcommand\Etendue{}%
      \renewcommand\Mediane{}%
      \renewcommand\DonneeMax{0}%
      \renewcommand\EffectifMax{0}%
      \renewcommand\DonneeMin{999999999}%
      \ifboolKV[ClesStat]{Qualitatif}{%D\'ebut qualitatif
        % Calculs
        %  %% celui de la somme des donn\'ees
        \foreachitem\don\in\ListeComplete{\xdef\SommeDonnees{\fpeval{\SommeDonnees+\ListeComplete[\doncnt,2]}}}%
        %  %% celui de l'effectif total
        \ifboolKV[ClesStat]{EffectifTotal}{%
          \ifboolKV[ClesStat]{Liste}{L'effectif total de la s\'erie est
            \num{\ListeCompletelen}.\par}{%
            \foreachitem\don\in\ListeComplete{\xdef\EffectifTotal{\fpeval{\EffectifTotal+\ListeComplete[\doncnt,2]}}}%
            L'effectif total de la s\'erie est : \[\ListeComplete[1,2]\xintFor* ##1 in
              {\xintSeq {2}{\ListeCompletelen}}\do{%
                +\ListeComplete[##1,2]}=\num{\EffectifTotal}.\]}
        }{}%
        \ifboolKV[ClesStat]{Liste}{\xdef\EffectifTotal{\ListeCompletelen}}{\xdef\EffectifTotal{\SommeDonnees}}%
        %  %% celui de la moyenne
        \xdef\Moyenne{\fpeval{\SommeDonnees/\ListeCompletelen}}%	
        \ifboolKV[ClesStat]{Moyenne}{%
          \ifboolKV[ClesStat]{Liste}{%
              \ifboolKV[ClesStat]{Somme}{La somme des donn\'ees de la s\'erie est :%
            \xintifboolexpr{\ListeCompletelen<\useKV[ClesStat]{Coupure}}{%
              \[
                \num{\ListeComplete[1,2]}\ifboolKV[ClesStat]{Concret}{~\text{\useKV[ClesStat]{Unite}}}{}\xintFor* ##1 in {\xintSeq {2}{\ListeCompletelen}}\do{%
                  +\num{\ListeComplete[##1,2]}\ifboolKV[ClesStat]{Concret}{~\text{\useKV[ClesStat]{Unite}}}{}
                }=\num{\SommeDonnees}\ifboolKV[ClesStat]{Concret}{~\text{\useKV[ClesStat]{Unite}}}{}.%
              \]}{%
              \[
                \num{\ListeComplete[1,2]}\ifboolKV[ClesStat]{Concret}{~\text{\useKV[ClesStat]{Unite}}}{}\xintFor* ##1 in {\xintSeq {2}{3}}\do{%
                  +\num{\ListeComplete[##1,2]}\ifboolKV[ClesStat]{Concret}{~\text{\useKV[ClesStat]{Unite}}}{}}+\dots\xintFor* ##1 in {\xintSeq {\ListeCompletelen-1}{\ListeCompletelen}}\do{%
                  +\num{\ListeComplete[##1,2]}\ifboolKV[ClesStat]{Concret}{~\text{\useKV[ClesStat]{Unite}}}{}
                }=\num{\SommeDonnees}\ifboolKV[ClesStat]{Concret}{~\text{\useKV[ClesStat]{Unite}}}{}.%
              \]%
            }%
            }{}%
            \ifboolKV[ClesStat]{MoyenneA}{%
              \ifboolKV[ClesStat]{SET}{}{Le nombre de donn\'ees de la s\'erie est \num{\ListeCompletelen}.\\}%
                Donc la moyenne de la s\'erie est \'egale \`a :%
                \[\frac{\num{\SommeDonnees}\ifboolKV[ClesStat]{Concret}{~\text{\useKV[ClesStat]{Unite}}}{}}{\num{\ListeCompletelen}}%\IfInteger{\fpeval{round(\fpeval{\SommeDonnees/\ListeCompletelen},\useKV[ClesStat]{Precision})}}{=}{\approx}
                  \ifboolKV[ClesStat]{ValeurExacte}{}{%
                    \opdiv*{\SommeDonnees}{\ListeCompletelen}{resultatmoy}{restemoy}%
                    \opround{resultatmoy}{\useKV[ClesStat]{Precision}}{resultatmoy1}%
                    \opcmp{resultatmoy}{resultatmoy1}\ifopeq=\else\approx\fi%
                    \num{\fpeval{round(\SommeDonnees/\ListeCompletelen,\useKV[ClesStat]{Precision})}}\ifboolKV[ClesStat]{Concret}{~\text{\useKV[ClesStat]{Unite}}}{}.%
                  }%
                \]%
              }{}%
            }{Pas de moyenne possible pour une s\'erie de donn\'ees \`a caract\`ere qualitatif.}}{}%
        %    %  %% celui de l'\'etendue
        \xintFor* ##1 in {\xintSeq {1}{\ListeCompletelen}}\do{%
          \xintifboolexpr{\ListeComplete[##1,2]>\DonneeMax}{%
            \xdef\DonneeMax{\ListeComplete[##1,2]}%
          }{}%
          \xintifboolexpr{\ListeComplete[##1,2]<\DonneeMin}{%
            \xdef\DonneeMin{\ListeComplete[##1,2]}%
          }{}%
        }%
        \xdef\EffectifMax{\DonneeMax}%
        \xdef\Etendue{\fpeval{\DonneeMax-\DonneeMin}}%
        \ifboolKV[ClesStat]{Etendue}{%
          \ifboolKV[ClesStat]{Liste}{%
            L'\'etendue de la s\'erie est \'egale \`a $\num{\DonneeMax}\ifboolKV[ClesStat]{Concret}{~\text{\useKV[ClesStat]{Unite}}}{}-\num{\DonneeMin}\ifboolKV[ClesStat]{Concret}{~\text{\useKV[ClesStat]{Unite}}}{}=\num{\Etendue}$\ifboolKV[ClesStat]{Concret}{~\useKV[ClesStat]{Unite}.}{.}%
          }{Pas d'\'etendue possible pour une s\'erie de donn\'ees \`a caract\`ere qualitatif.}}{}%
        % celui de la mediane
        %%% Recuperation de la mediane %%%%%%%%%%%%%%%%%%%%% 
        \newcount\Recapmed%
        \newcount\Recapmeda%
        \ifodd\number\ListeCompletelen%odd impair
        \Recapmed=\fpeval{(\ListeCompletelen+1)/2}\relax%
        \else%
        \Recapmed=\fpeval{\ListeCompletelen/2}\relax%
        \Recapmeda=\numexpr\Recapmed+1\relax%
        \fi%
        \newcount\Recapk%
        \Recapk=0%
        \DTLforeach{mtdb}{\numeroDonnee=Numeric}{\Recapk=\numexpr\Recapk+1\relax%
          \ifnum\Recapk=\Recapmed%
          \ifodd\number\ListeCompletelen%
          \xdef\Mediane{\numeroDonnee}%
          \else%
          \xdef\Mediane{\numeroDonnee}%
          \fi%
          \fi%
          \ifnum\Recapk=\Recapmeda%
          \xdef\Mediane{\fpeval{(\Mediane+\numeroDonnee)/2}}%
          \fi%
        }%
        %%% 
        \ifboolKV[ClesStat]{Mediane}{%
          \ifboolKV[ClesStat]{Liste}{%    
            On range les donn\'ees par ordre croissant :%
            \nbdonnees=0%
            \xintifboolexpr{\ListeCompletelen<\useKV[ClesStat]{Coupure}}{%
              \[\DTLforeach{mtdb}{\numeroDonnee=Numeric}{\num{\numeroDonnee}\ifboolKV[ClesStat]{Concret}{~\text{\useKV[ClesStat]{Unite}}}{}\DTLiflastrow{.}{\,;~}}\]%
            }{%
              %\medskip%
              \begin{center}
                \begin{minipage}{0.9\linewidth}
                  \DTLforeach*{mtdb}{\numeroDonnee=Numeric}{\num{\numeroDonnee}\ifboolKV[ClesStat]{Concret}{~\text{\useKV[ClesStat]{Unite}}}{}\DTLiflastrow{.}{\,;~}\nbdonnees=\fpeval{\nbdonnees+1}\modulo{\nbdonnees}{\useKV[ClesStat]{Coupure}}\xintifboolexpr{\remainder==0}{\\}{}}
                \end{minipage}
              \end{center}%
              %\medskip%
            }%
            \newcount\med%
            \newcount\meda%
            \ifodd\number\ListeCompletelen%odd impair
            \med=\fpeval{(\ListeCompletelen+1)/2}\relax%
            \ifboolKV[ClesStat]{SET}{On sait que }{L'effectif total de la s\'erie est \num{\ListeCompletelen}. Or, }$\num{\ListeCompletelen}=\num{\fpeval{\med-1}}+1+\num{\fpeval{\med-1}}$.\\
            \else% pair
            \med=\fpeval{\ListeCompletelen/2}\relax%
            \meda=\numexpr\med+1\relax%
            \ifboolKV[ClesStat]{SET}{On sait que }{L'effectif total de la s\'erie est \num{\ListeCompletelen}. Or, }$\num{\ListeCompletelen}=\num{\the\med}+\num{\the\med}$.\\
            \fi%
            \newcount\k%
            \k=0%
            \DTLforeach{mtdb}{\numeroDonnee=Numeric}{\k=\numexpr\k+1\relax%
              \ifnum\k=\med %La m\'ediane vaut \numeroDonnee\fi
              \ifodd\number\ListeCompletelen%
              La m\'ediane de la s\'erie est la \the\med\ieme{} donn\'ee.\\Donc la m\'ediane de la s\'erie est \num{\numeroDonnee}\ifboolKV[ClesStat]{Concret}{~\useKV[ClesStat]{Unite}.}{.}%
              \xdef\Mediane{\numeroDonnee}%
              \else%
              La \the\med\ieme{} donn\'ee est \num{\numeroDonnee}\ifboolKV[ClesStat]{Concret}{~\useKV[ClesStat]{Unite}. }{. }\xdef\Mediane{\numeroDonnee}%
              \fi%
              \fi%
              \ifnum\k=\meda
              La \the\meda\ieme{} donn\'ee est \num{\numeroDonnee}\ifboolKV[ClesStat]{Concret}{~\useKV[ClesStat]{Unite}.}{.}\\Donc \PfCArticleMediane{} m\'ediane de la s\'erie est \xdef\Mediane{\fpeval{(\Mediane+\numeroDonnee)/2}}$\ifboolKV[ClesStat]{DetailsMediane}{\dfrac{\num{\Mediane}+\num{\numeroDonnee}}{2}=}{}\num{\Mediane}$\ifboolKV[ClesStat]{Concret}{~\useKV[ClesStat]{Unite}.}{.}%
              \fi%
            }%
            %%%%%%% 
          }{Pas de m\'ediane possible pour une s\'erie de donn\'ees \`a caract\`ere qualitatif.}}{}
        %%% Quartile un
        \newcount\PfCQuartileUn%
        \modulo{\ListeCompletelen}{4}\relax%
        \ifnum\remainder=0%
        \PfCQuartileUn=\fpeval{\ListeCompletelen/4}%
        \else%
        \PfCQuartileUn=\fpeval{ceil(\ListeCompletelen/4)}%
        \fi%
        \newcount\PfCQunk%
        \PfCQunk=0%
        \DTLforeach{mtdb}{\numeroDonnee=Numeric}{\PfCQunk=\numexpr\PfCQunk+1\relax%
          \ifnum\PfCQunk=\PfCQuartileUn%
          \xdef\QuartileUn{\numeroDonnee}%
          \fi%
        }%
        %%% Quartile trois
        \newcount\PfCQuartileTrois%
        \modulo{\ListeCompletelen}{4}\relax%
        \ifnum\remainder=0%
        \PfCQuartileTrois=\fpeval{3*\ListeCompletelen/4}%
        \else%
        \PfCQuartileTrois=\fpeval{ceil(3*\ListeCompletelen/4)}%
        \fi%
        \newcount\PfCQtroisk%
        \PfCQtroisk=0%
        \DTLforeach{mtdb}{\numeroDonnee=Numeric}{\PfCQtroisk=\numexpr\PfCQtroisk+1\relax%
          \ifnum\PfCQtroisk=\PfCQuartileTrois%La m\'ediane vaut \numeroDonnee\fi
          \xdef\QuartileTrois{\numeroDonnee}%
          \fi%
        }%
        % Construction du tableau
        \ifboolKV[ClesStat]{Tableau}{%
          \ifboolKV[ClesStat]{Liste}{Pas de tableau possible avec la cl\'e Liste.\\Utilisez plut\^ot la cl\'e Sondage si vous voulez un tableau avec cette liste.}{\BuildtabStat}}{}%
        % Construction du graphique
        \ifboolKV[ClesStat]{Graphique}{%
          \ifboolKV[ClesStat]{Liste}{Pas de graphique possible avec la cl\'e Liste.\\Utilisez plut\^ot la cl\'e Sondage si vous voulez un graphique avec cette liste.}{%
            \ifboolKV[ClesStat]{Barre}{%
              \buildgraphbarhor%
            }{%
              \ifboolKV[ClesStat]{Angle}{%
                \buildgraphcq{360}%
              }{%
                \ifboolKV[ClesStat]{SemiAngle}{%
                  \buildgraphcq{180}%
                }{%
                  \buildgraphq[#1]%
                }%
              }%
            }%
          }%
        }{}%
      }{%%%%%%%%%%%%%%%%%%%%%D\'ebut quantitatif
        %  % on effectue les calculs
        %  %% celui de la somme des donn\'ees
        \foreachitem\don\in\ListeComplete{\xdef\SommeDonnees{\fpeval{\SommeDonnees+\ListeComplete[\doncnt,1]*\ListeComplete[\doncnt,2]}}}%
        %  %% celui de l'effectif total
        \foreachitem\don\in\ListeComplete{\xdef\EffectifTotal{\fpeval{\EffectifTotal+\ListeComplete[\doncnt,2]}}}%
        %  %% celui de l'\'etendue
        \xintFor* ##1 in {\xintSeq {1}{\ListeCompletelen}}\do{%
          \xintifboolexpr{\ListeComplete[##1,1]>\DonneeMax}{%
            \xdef\DonneeMax{\ListeComplete[##1,1]}%
          }{}%
          \xintifboolexpr{\ListeComplete[##1,1]<\DonneeMin}{%
            \xdef\DonneeMin{\ListeComplete[##1,1]}%
          }{}%
        }%
        % \xdef\EffectifMax{\DonneeMax}%
        \xdef\Etendue{\fpeval{\DonneeMax-\DonneeMin}}%%
        %  %% celui de la moyenne
        \xdef\Moyenne{\fpeval{\SommeDonnees/\EffectifTotal}}%
        \ifboolKV[ClesStat]{EffectifTotal}{%
          L'effectif total de la s\'erie est : \[\ListeComplete[1,2]\xintFor* ##1 in
            {\xintSeq {2}{\ListeCompletelen}}\do{%
              +\ListeComplete[##1,2]}=\num{\EffectifTotal}.\]
        }{}%
        \ifboolKV[ClesStat]{Moyenne}{%
            \ifboolKV[ClesStat]{Somme}{La somme des donn\'ees de la s\'erie est :%
          \xintifboolexpr{\ListeCompletelen<\useKV[ClesStat]{Coupure}}{%
            \[
              \ifnum\ListeComplete[1,2]=1\else\num{\ListeComplete[1,2]}\times\fi\num{\ListeComplete[1,1]}\ifboolKV[ClesStat]{Concret}{~\text{\useKV[ClesStat]{Unite}}}{}\xintFor* ##1 in {\xintSeq {2}{\ListeCompletelen}}\do{%
                +\ifnum\ListeComplete[##1,2]=1\else\num{\ListeComplete[##1,2]}\times\fi\num{\ListeComplete[##1,1]}\ifboolKV[ClesStat]{Concret}{~\text{\useKV[ClesStat]{Unite}}}{}
              }=\num{\SommeDonnees}\ifboolKV[ClesStat]{Concret}{~\text{\useKV[ClesStat]{Unite}}}{}.
            \]
          }{%
            \[
              \ifnum\ListeComplete[1,2]=1\else\num{\ListeComplete[1,2]}\times\fi\num{\ListeComplete[1,1]}\ifboolKV[ClesStat]{Concret}{~\text{\useKV[ClesStat]{Unite}}}{}\xintFor* ##1 in {\xintSeq {2}{2}}\do{%
                +\ifnum\ListeComplete[##1,2]=1\else\num{\ListeComplete[##1,2]}\times\fi\num{\ListeComplete[##1,1]}\ifboolKV[ClesStat]{Concret}{~\text{\useKV[ClesStat]{Unite}}}{}
              }+\dots\xintFor* ##1 in {\xintSeq {\ListeCompletelen-1}{\ListeCompletelen}}\do{%
                +\ifnum\ListeComplete[##1,2]=1\else\num{\ListeComplete[##1,2]}\times\fi\num{\ListeComplete[##1,1]}\ifboolKV[ClesStat]{Concret}{~\text{\useKV[ClesStat]{Unite}}}{}
              }=\num{\SommeDonnees}\ifboolKV[ClesStat]{Concret}{~\text{\useKV[ClesStat]{Unite}}}{}.
            \]
          }%
          }{}%
          \ifboolKV[ClesStat]{MoyenneA}{\ifboolKV[ClesStat]{SET}{}{L'effectif total de la s\'erie est :%
            \ifboolKV[ClesStat]{Liste}{ \num{\EffectifTotal}\\}{%
              \[\num{\ListeComplete[1,2]}\xintFor* ##1 in {\xintSeq {2}{\ListeCompletelen}}\do{%
                  +\num{\ListeComplete[##1,2]}
                }=\num{\EffectifTotal}.
              \]%
            }%
          }%
          Donc la moyenne de la s\'erie est \'egale \`a :%
          \[\frac{\num{\SommeDonnees}\ifboolKV[ClesStat]{Concret}{~\text{\useKV[ClesStat]{Unite}}}{}}{\num{\EffectifTotal}}%
            \ifboolKV[ClesStat]{ValeurExacte}{}{%
              \opdiv*{\SommeDonnees}{\EffectifTotal}{resultatmoy}{restemoy}%
              \opround{resultatmoy}{\useKV[ClesStat]{Precision}}{resultatmoy1}%
              \opcmp{resultatmoy}{resultatmoy1}\ifopeq=\else\approx\fi%
              \num{\fpeval{round(\SommeDonnees/\EffectifTotal,\useKV[ClesStat]{Precision})}}\ifboolKV[ClesStat]{Concret}{~\text{\useKV[ClesStat]{Unite}}}{}.%
            }%
          \]%
          }{}%
        }{}%
        %  % Affichage des r\'eponses.
        %  %% pour l'\'etendue
        \ifboolKV[ClesStat]{Etendue}{L'\'etendue de la s\'erie est \'egale \`a $\num{\ListeComplete[\ListeCompletelen,1]}\ifboolKV[ClesStat]{Concret}{~\text{\useKV[ClesStat]{Unite}}}{}-\num{\ListeComplete[1,1]}\ifboolKV[ClesStat]{Concret}{~\text{\useKV[ClesStat]{Unite}}}{}=\num{\Etendue}$\ifboolKV[ClesStat]{Concret}{~\useKV[ClesStat]{Unite}.}{.}}{}%
        % pour la m\'ediane
        %%% Recuperation Mediane
        \newcount\Recupmed%
        \newcount\Recupmeda%
        \ifodd\number\EffectifTotal%odd impair
        \Recupmed=\fpeval{(\EffectifTotal+1)/2}\relax%
        \else% pair
        \Recupmed=\fpeval{\EffectifTotal/2}\relax%
        \Recupmeda=\numexpr\Recupmed+1\relax%
        \fi%
        \newcount\Recupk%
        \Recupk=0%
        \xintFor* ##1 in {\xintSeq {1}{\ListeCompletelen}}\do{%
          \xintFor* ##2 in {\xintSeq {1}{\ListeComplete[##1,2]}}\do{%
            \Recupk=\numexpr\Recupk+1\relax%
            \ifnum\Recupk=\Recupmed%
            \ifodd\number\EffectifTotal%
            \xdef\Mediane{\ListeComplete[##1,1]}%
            \else%
            \xdef\Mediane{\ListeComplete[##1,1]}%
            \fi%
            \fi%
            \ifnum\Recupk=\Recupmeda%
            \xdef\Mediane{\fpeval{(\Mediane+\ListeComplete[##1,1])/2}}%
            \fi%
          }%
        }%
        %%% 
        \ifboolKV[ClesStat]{Mediane}{%
          
          \newcount\med%
          \newcount\meda%
          \ifodd\number\EffectifTotal%odd impair
          \med=\fpeval{(\EffectifTotal+1)/2}\relax%
          \ifboolKV[ClesStat]{SET}{On sait que }{L'effectif total de la s\'erie est \num{\EffectifTotal}. Or, }$\num{\EffectifTotal}=\num{\fpeval{\med-1}}+1+\num{\fpeval{\med-1}}$. %
          \else% pair
          \med=\fpeval{\EffectifTotal/2}\relax%
          \meda=\numexpr\med+1\relax%
          \ifboolKV[ClesStat]{SET}{On sait que }{L'effectif total de la s\'erie est \num{\EffectifTotal}. Or, }$\num{\EffectifTotal}=\num{\fpeval{\med}}+\num{\fpeval{\med}}$. %
          \fi%
          \newcount\k%
          \k=0%
          \xintFor* ##1 in {\xintSeq {1}{\ListeCompletelen}}\do{%
            \xintFor* ##2 in {\xintSeq {1}{\ListeComplete[##1,2]}}\do{%
              \k=\numexpr\k+1\relax%
              \ifnum\k=\med%
              \ifodd\number\EffectifTotal%
              La m\'ediane de la s\'erie est la \the\med\ieme{} donn\'ee. Donc la m\'ediane de la s\'erie est \num{\ListeComplete[##1,1]}\ifboolKV[ClesStat]{Concret}{~\useKV[ClesStat]{Unite}.}{.}%
              \else%
              La \the\med\ieme{} donn\'ee est \num{\ListeComplete[##1,1]}\ifboolKV[ClesStat]{Concret}{~\useKV[ClesStat]{Unite}. }{. }\xdef\Mediane{\ListeComplete[##1,1]}%
              \fi%
              \fi%
              \ifnum\k=\meda%
              La \the\meda\ieme{} donn\'ee est \num{\ListeComplete[##1,1]}\ifboolKV[ClesStat]{Concret}{~\useKV[ClesStat]{Unite}.}{.}\\Donc \PfCArticleMediane{} m\'ediane de la s\'erie est $\ifboolKV[ClesStat]{DetailsMediane}{\dfrac{\num{\Mediane}+\num{\ListeComplete[##1,1]}}2=}{}\xdef\Mediane{\fpeval{(\Mediane+\ListeComplete[##1,1])/2}}\num{\Mediane}$\ifboolKV[ClesStat]{Concret}{~\useKV[ClesStat]{Unite}.}{.}%
              \fi%
            }%
          }%
        }{}%
        %%% Quartile un
        \newcount\PfCQuartileUn%
        \modulo{\EffectifTotal}{4}\relax%
        \ifnum\remainder=0%
        \PfCQuartileUn=\fpeval{\EffectifTotal/4}%
        \else%
        \PfCQuartileUn=\fpeval{ceil(\EffectifTotal/4)}%
        \fi%
        \newcount\PfCQunk%
        \PfCQunk=0%
        \xintFor* ##1 in {\xintSeq {1}{\ListeCompletelen}}\do{%
          \xintFor* ##2 in {\xintSeq {1}{\ListeComplete[##1,2]}}\do{%
            \PfCQunk=\numexpr\PfCQunk+1\relax%
            \ifnum\PfCQunk=\PfCQuartileUn%
            \xdef\QuartileUn{\ListeComplete[##1,1]}%
            \fi%
          }%
        }%
        %%% Quartile trois
        \newcount\PfCQuartileTrois%
        \modulo{\EffectifTotal}{4}\relax%
        \ifnum\remainder=0%
        \PfCQuartileTrois=\fpeval{3*\EffectifTotal/4}%
        \else%
        \PfCQuartileTrois=\fpeval{ceil(3*\EffectifTotal/4)}%
        \fi%
        \newcount\PfCQtroisk%
        \PfCQtroisk=0%
        \xintFor* ##1 in {\xintSeq {1}{\ListeCompletelen}}\do{%
          \xintFor* ##2 in {\xintSeq {1}{\ListeComplete[##1,2]}}\do{%
            \PfCQtroisk=\numexpr\PfCQtroisk+1\relax%
            \ifnum\PfCQtroisk=\PfCQuartileTrois%
            \xdef\QuartileTrois{\ListeComplete[##1,1]}%
            \fi%
          }%
        }%
        % Construction de tableau
        \ifboolKV[ClesStat]{Tableau}{\BuildtabStat}{}%
        % Construction du graphique ??
        \ifboolKV[ClesStat]{Graphique}{%
          \ifboolKV[ClesStat]{Angle}{%
            \buildgraphcq{360}%
          }{%
            \ifboolKV[ClesStat]{SemiAngle}{%
              \buildgraphcq{180}%
            }{%
              \buildgraph[#1]%
            }%
          }%
        }{}%
      }%
    }%
  }%
}%