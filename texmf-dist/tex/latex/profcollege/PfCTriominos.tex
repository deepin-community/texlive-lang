%%%
% Triominos
%%%
\def\filedateTriominos{2024/08/04}%
\def\fileversionTriominos{0.1}%
\message{-- \filedateTriominos\space v\fileversionTriominos}%
%
\setKVdefault[ClesTriomino]{Longueur=5cm,Etages=3,Ecart=0.6,AffichagePiece=false,Hexagone=false,Recapitulatif=false,Colonnes=2}%
\defKV[ClesTriomino]{Piece=\setKV[ClesTriomino]{AffichagePiece=true}}%

\def\TraceTriomino#1{%
  \ifluatex
  \mplibforcehmode
  \begin{mplibcode}
    u:=\useKV[ClesTriomino]{Longueur};
    Rayon:=0.75*u*sqrt(3)/6;
    Etages:=\useKV[ClesTriomino]{Etages};
    pair A,B,C,D,E,F;
    A=(0,0);
    B-A=Etages*u*(1,0);
    C=rotation(B,A,60);
    D=(1/Etages)[C,A];
    E=(1/Etages)[C,B];
    F=C;
    trace polygone(A,B,C);
    for k=1 upto Etages-1:
    trace (k/Etages)[C,A]--(k/Etages)[C,B];
    trace (k/Etages)[A,C]--(k/Etages)[A,B];
    trace (k/Etages)[B,A]--(k/Etages)[B,C];
    endfor;
    pair G[];color H[];%Couleur pour garder l'orientation des textes...
    G[1]=iso(D,E,F);
    H1=blue;
    n=1;
    for k=1 upto Etages-1:
    for l=0 upto (2*k):
    n:=n+1;
    if (l mod 2=0):
    G[n]=G[1] shifted(k*(D-F)+(l div 2)*(E-D));
    H[n]=blue;
    else:
    G[n]=symetrie(G[1],D,E) shifted((k-1)*(D-F)+(l div 2)*(E-D));
    H[n]=green;
    fi;
    endfor;
    endfor;
    % affichage des textes
    nba=0;
    for p_=#1:
    if (nba mod 3)=1:
    if H[(nba div 3)+1]=blue:
    label(TEX(p_) rotated 120,pointarc(cercles(G[(nba div 3)+1],Rayon),30));
    else:
    label(TEX(p_) rotated 180,pointarc(cercles(G[(nba div 3)+1],Rayon),90));
    fi;
    elseif (nba mod 3)=2:
    if H[(nba div 3)+1]=blue:
    label(TEX(p_),pointarc(cercles(G[(nba div 3)+1],Rayon),270));
    else:
    label(TEX(p_) rotated 60,pointarc(cercles(G[(nba div 3)+1],Rayon),330));
    fi;
    else:
    if H[(nba div 3)+1]=blue:
    label(TEX(p_) rotated 240,pointarc(cercles(G[(nba div 3)+1],Rayon),150));
    else:
    label(TEX(p_) rotated 300,pointarc(cercles(G[(nba div 3)+1],Rayon),210));
    fi;
    fi;
    nba:=nba+1;
    endfor;
  \end{mplibcode}
  \else
  \begin{mpost}[mpsettings={u:=\useKV[ClesTriomino]{Longueur}; Etages:=\useKV[ClesTriomino]{Etages};}]
    Rayon:=0.75*u*sqrt(3)/6;
    pair A,B,C,D,E,F;
    A=(0,0);
    B-A=Etages*u*(1,0);
    C=rotation(B,A,60);
    D=(1/Etages)[C,A];
    E=(1/Etages)[C,B];
    F=C;
    trace polygone(A,B,C);
    for k=1 upto Etages-1:
    trace (k/Etages)[C,A]--(k/Etages)[C,B];
    trace (k/Etages)[A,C]--(k/Etages)[A,B];
    trace (k/Etages)[B,A]--(k/Etages)[B,C];
    endfor;
    pair G[];color H[];%Couleur pour garder l'orientation des textes...
    G[1]=iso(D,E,F);
    H1=blue;
    n=1;
    for k=1 upto Etages-1:
    for l=0 upto (2*k):
    n:=n+1;
    if (l mod 2=0):
    G[n]=G[1] shifted(k*(D-F)+(l div 2)*(E-D));
    H[n]=blue;
    else:
    G[n]=symetrie(G[1],D,E) shifted((k-1)*(D-F)+(l div 2)*(E-D));
    H[n]=green;
    fi;
    endfor;
    endfor;
    % affichage des textes
    nba=0;
    for p_=#1:
    if (nba mod 3)=1:
    if H[(nba div 3)+1]=blue:
    label(LATEX(p_) rotated 120,pointarc(cercles(G[(nba div 3)+1],Rayon),30));
    else:
    label(LATEX(p_) rotated 180,pointarc(cercles(G[(nba div 3)+1],Rayon),90));
    fi;
    elseif (nba mod 3)=2:
    if H[(nba div 3)+1]=blue:
    label(LATEX(p_),pointarc(cercles(G[(nba div 3)+1],Rayon),270));
    else:
    label(LATEX(p_) rotated 60,pointarc(cercles(G[(nba div 3)+1],Rayon),330));
    fi;
    else:
    if H[(nba div 3)+1]=blue:
    label(LATEX(p_) rotated 240,pointarc(cercles(G[(nba div 3)+1],Rayon),150));
    else:
    label(LATEX(p_) rotated 300,pointarc(cercles(G[(nba div 3)+1],Rayon),210));
    fi;
    fi;
    nba:=nba+1;
    endfor;    
  \end{mpost}
  \fi
}

\def\TraceTriominoHexa#1{%
  \ifluatex
  \mplibforcehmode
  \begin{mplibcode}
    u:=\useKV[ClesTriomino]{Longueur};
    RayonCercle:=u;
    Rayon:=\useKV[ClesTriomino]{Ecart}*0.5*u*sqrt(3)/6;
    pair O,A,B,C,D,E,F,G,H[],I[];
    O=(0,0);
    path cc;
    cc=cercles(O,RayonCercle);
    A=pointarc(cc,0);
    B=rotation(A,O,60);
    C=rotation(B,O,60);
    D=rotation(C,O,60);
    E=rotation(D,O,60);
    F=rotation(E,O,60);
    G=rotation(F,O,60);
    H1=iso(A,B);
    H2=iso(B,C);
    H3=iso(C,D);
    H4=iso(D,E);
    H5=iso(E,F);
    H6=iso(F,A);
    I1=iso(C,1/2[C,D],rotation(1/2[C,D],C,60));
    I2=symetrie(I1,C,F);
    I3=symetrie(I2,1/2[B,C],1/2[D,E]);
    I4=symetrie(I3,1/2[B,C],1/2[F,A]);
    I5=symetrie(I4,B,E);
    I6-I1=0.5*(D-C);
    I7-I2=0.5*(D-C);
    I8-I3=0.5*(D-C);
    I9-I4=0.5*(D-C);
    I10-I5=0.5*(D-C);
    I11=symetrie(I10,1/2[C,B],1/2[F,A]);
    I12=symetrie(I11,1/2[A,B],1/2[E,F]);
    I13=symetrie(I6,A,D);
    I14=symetrie(I7,A,D);
    I15=symetrie(I8,A,D);
    I16=symetrie(I9,A,D);
    I17=symetrie(I10,A,D);
    I18=symetrie(I11,A,D);
    I19=symetrie(I12,A,D);
    I20=symetrie(I1,A,D);
    I21=symetrie(I2,A,D);
    I22=symetrie(I3,A,D);
    I23=symetrie(I4,A,D);
    I24=symetrie(I5,A,D);
    trace polygone(A,B,C,D,E,F);
    trace polygone(H1,H3,H5);
    trace polygone(H2,H4,H6);
    trace segment(A,D);
    trace segment(B,E);
    trace segment(C,F);
    color N[];%Couleur pour garder l'orientation des textes...
    N1=blue;
    for l=2 upto 5:
    if (l mod 2=0):
    N[l]=green;
    else:
    N[l]=blue;
    fi;
    endfor;
    for l=6 upto 12:
    if (l mod 2=0):
    N[l]=blue;
    else:
    N[l]=green;
    fi;
    endfor;
    for l=13 upto 19:
    if (l mod 2=0):
    N[l]=blue;
    else:
    N[l]=green;
    fi;
    endfor;
    for l=20 upto 24:
    if (l mod 2=0):
    N[l]=green;
    else:
    N[l]=blue;
    fi;
    endfor;
    % affichage des textes
    nba=0;
    for p_=#1:
    if (nba mod 3)=1:
    if N[(nba div 3)+1]=blue:
    label(TEX(p_) rotated 120,pointarc(cercles(I[(nba div 3)+1],Rayon),30));
    else:
    label(TEX(p_) rotated 180,pointarc(cercles(I[(nba div 3)+1],Rayon),90));
    fi;
    elseif (nba mod 3)=2:
    if N[(nba div 3)+1]=blue:
    label(TEX(p_),pointarc(cercles(I[(nba div 3)+1],Rayon),270));
    else:
    label(TEX(p_) rotated 60,pointarc(cercles(I[(nba div 3)+1],Rayon),330));
    fi;
    else:
    if N[(nba div 3)+1]=blue:
    label(TEX(p_) rotated 240,pointarc(cercles(I[(nba div 3)+1],Rayon),150));
    else:
    label(TEX(p_) rotated 300,pointarc(cercles(I[(nba div 3)+1],Rayon),210));
    fi;
    fi;
    nba:=nba+1;
    endfor;
  \end{mplibcode}
  \else
    \begin{mpost}[mpsettings={u:=\useKV[ClesTriomino]{Longueur};RayonCercle:=u; Rayon:=\useKV[ClesTriomino]{Ecart}*0.5*u*sqrt(3)/6;}]
    pair O,A,B,C,D,E,F,G,H[],I[];
    O=(0,0);
    path cc;
    cc=cercles(O,RayonCercle);
    A=pointarc(cc,0);
    B=rotation(A,O,60);
    C=rotation(B,O,60);
    D=rotation(C,O,60);
    E=rotation(D,O,60);
    F=rotation(E,O,60);
    G=rotation(F,O,60);
    H1=iso(A,B);
    H2=iso(B,C);
    H3=iso(C,D);
    H4=iso(D,E);
    H5=iso(E,F);
    H6=iso(F,A);
    I1=iso(C,1/2[C,D],rotation(1/2[C,D],C,60));
    I2=symetrie(I1,C,F);
    I3=symetrie(I2,1/2[B,C],1/2[D,E]);
    I4=symetrie(I3,1/2[B,C],1/2[F,A]);
    I5=symetrie(I4,B,E);
    I6-I1=0.5*(D-C);
    I7-I2=0.5*(D-C);
    I8-I3=0.5*(D-C);
    I9-I4=0.5*(D-C);
    I10-I5=0.5*(D-C);
    I11=symetrie(I10,1/2[C,B],1/2[F,A]);
    I12=symetrie(I11,1/2[A,B],1/2[E,F]);
    I13=symetrie(I6,A,D);
    I14=symetrie(I7,A,D);
    I15=symetrie(I8,A,D);
    I16=symetrie(I9,A,D);
    I17=symetrie(I10,A,D);
    I18=symetrie(I11,A,D);
    I19=symetrie(I12,A,D);
    I20=symetrie(I1,A,D);
    I21=symetrie(I2,A,D);
    I22=symetrie(I3,A,D);
    I23=symetrie(I4,A,D);
    I24=symetrie(I5,A,D);
    trace polygone(A,B,C,D,E,F);
    trace polygone(H1,H3,H5);
    trace polygone(H2,H4,H6);
    trace segment(A,D);
    trace segment(B,E);
    trace segment(C,F);
    color N[];%Couleur pour garder l'orientation des textes...
    N1=blue;
    for l=2 upto 5:
    if (l mod 2=0):
    N[l]=green;
    else:
    N[l]=blue;
    fi;
    endfor;
    for l=6 upto 12:
    if (l mod 2=0):
    N[l]=blue;
    else:
    N[l]=green;
    fi;
    endfor;
    for l=13 upto 19:
    if (l mod 2=0):
    N[l]=blue;
    else:
    N[l]=green;
    fi;
    endfor;
    for l=20 upto 24:
    if (l mod 2=0):
    N[l]=green;
    else:
    N[l]=blue;
    fi;
    endfor;
    % affichage des textes
    nba=0;
    for p_=#1:
    if (nba mod 3)=1:
    if N[(nba div 3)+1]=blue:
    label(LATEX(p_) rotated 120,pointarc(cercles(I[(nba div 3)+1],Rayon),30));
    else:
    label(LATEX(p_) rotated 180,pointarc(cercles(I[(nba div 3)+1],Rayon),90));
    fi;
    elseif (nba mod 3)=2:
    if N[(nba div 3)+1]=blue:
    label(LATEX(p_),pointarc(cercles(I[(nba div 3)+1],Rayon),270));
    else:
    label(LATEX(p_) rotated 60,pointarc(cercles(I[(nba div 3)+1],Rayon),330));
    fi;
    else:
    if N[(nba div 3)+1]=blue:
    label(LATEX(p_) rotated 240,pointarc(cercles(I[(nba div 3)+1],Rayon),150));
    else:
    label(LATEX(p_) rotated 300,pointarc(cercles(I[(nba div 3)+1],Rayon),210));
    fi;
    fi;
    nba:=nba+1;
    endfor;
  \end{mpost}
  \fi
}

\newtoks\toklisteTriomino%
\def\UpdatetoksTriomino#1\nil{\addtotok\toklisteTriomino{"#1",}}%
  
\newcommand\Triomino[2][]{%
  \useKVdefault[ClesTriomino]%
  \setKV[ClesTriomino]{#1}%
  \setsepchar{§}%\ignoreemptyitems%
  \readlist*\ListeTriominos{#2}%
  \toklisteTriomino{}%
  \ifboolKV[ClesTriomino]{Recapitulatif}{%
    %\begin{longtable}{|>{\centering}p{0.45\linewidth}|p{0.45\linewidth}|}
    %  \hline
    %  \xintFor* ##1 in {\xintSeq{1}{\ListeTriominoslen}}\do{
    %    &\\
    %    &\\
    %    \ListeTriominos[##1]&\\
    %    &\\
    %    &\\
    %    \hline
    %  }%
    %\end{longtable}
  }{%
    \ifboolKV[ClesTriomino]{AffichagePiece}{%
      \setKV[ClesTriomino]{Etages=1}%
      \TraceTriomino{"\ListeTriominos[\fpeval{3*\useKV[ClesTriomino]{Piece}-2}]","\ListeTriominos[\fpeval{3*\useKV[ClesTriomino]{Piece}-1}]","\ListeTriominos[\fpeval{3*\useKV[ClesTriomino]{Piece}}]"}%
    }{%
      \foreachitem\compteur\in\ListeTriominos{\expandafter\UpdatetoksTriomino\compteur\nil}%
      \ifboolKV[ClesTriomino]{Hexagone}{%
        \TraceTriominoHexa{\the\toklisteTriomino}%
      }{%
        \TraceTriomino{\the\toklisteTriomino}%
      }%
    }%
  }%
}%