% proflycee-tools-aleatoire.tex
% Copyright 2023-2024  Cédric Pierquet
% Released under the LaTeX Project Public License v1.3c or later, see http://www.latex-project.org/lppl.txt

%%------EnsembleList
\defKV[kvensemble]{%
	Sep=\def\PLenssep{#1},%
	Option=\def\PLensopt{#1}
}

\setKVdefault[kvensemble]{%
	Sep={;},%
	Mathpunct=true,%
	Option={}
}

\newcommand\EcritureEnsemble[2][]{%
	\useKVdefault[kvensemble]%
	\setKV[kvensemble]{#1}%
	\ifboolKV[kvensemble]{Mathpunct}%
		{\left\lbrace \PLensopt{} \mathpunct{} \StrSubstitute{#2}{/}{\mathpunct{}{\PLenssep}\mathpunct{}} \mathpunct{} \right\rbrace}%
		{\left\lbrace \PLensopt{} \StrSubstitute{#2}{/}{{\PLenssep}} \right\rbrace}%
}

%%------TRINOMEALEA
\setKVdefault[trinomealea]{%
	Alea=false,%
	Anegatif=true
}

\NewDocumentCommand\EcritureTrinome{ O{} m m m }{%
	\useKVdefault[trinomealea]%
	\setKV[trinomealea]{#1}%
	\ifboolKV[trinomealea]{Alea}%
	{%
		\xdef\Coeffa{\fpeval{randint(#2)}}\xdef\Coeffb{\fpeval{randint(#3)}}\xdef\Coeffc{\fpeval{randint(#4)}}%
		\xdef\AleaSigneA{\fpeval{randint(1,2)}}%
		\ifboolKV[trinomealea]{Anegatif}%
			{\xintifboolexpr{\AleaSigneA == 1}{\def\TriSigneA{}}{\def\TriSigneA{-}}}%
			{\def\TriSigneA{}}%forcément a>0
	}%
	{%
		\xdef\Coeffa{#2}\xdef\Coeffb{#3}\xdef\Coeffc{#4}%
		\def\TriSigneA{}%
	}%
	\def\TriPartieA{\num{\TriSigneA\Coeffa} x^2}%
	\xintifboolexpr{\Coeffa == 1}%
		{\def\TriPartieA{x^2}}{}%
	\xintifboolexpr{\Coeffa == -1}%
		{\def\TriPartieA{-x^2}}{}%
	\def\TriPartieB{\num[print-implicit-plus]{\Coeffb}x}%
	\xintifboolexpr{\Coeffb == 0}%
		{\def\TriPartieB{}}{}%
	\xintifboolexpr{\Coeffb == 1}%
		{\def\TriPartieB{+x}}{}%
	\xintifboolexpr{\Coeffb == -1}%
		{\def\TriPartieB{-x}}{}%
	\xintifboolexpr{\Coeffc == 0}%
		{\def\TriPartieC{}}{\def\TriPartieC{\num[print-implicit-plus]{\Coeffc}}}%
	%formatage
	\ensuremath{\TriPartieA\TriPartieB\TriPartieC}%
}

%%------ALEATOIRES
\ExplSyntaxOn
\NewDocumentCommand\TriListeCroiss{m}
{
	\clist_sort:Nn #1
	{
		\fp_compare:nNnTF {##1} > {##2}
		{ \sort_return_swapped: }
		{ \sort_return_same: }
	}
}
\NewDocumentCommand\TriListeDecroiss{m}
{
	\clist_sort:Nn #1
	{
		\fp_compare:nNnTF {##1} < {##2}
		{ \sort_return_swapped: }
		{ \sort_return_same: }
	}
}
\ExplSyntaxOff

\newcommand\NbAlea[4][0]{%entier ou décimal
	\xintifboolexpr{#1 == 0}%
		{%
			\xdef#4{\fpeval{randint(#2,#3)}}%
		}%
		{%
			\xintifboolexpr{(#3-#2) == 1}%
				{%
					\xdef#4{\fpeval{round(#2+rand(),#1)}}%
				}%
				{%
					\xdef#4{\fpeval{round(randint(#2,#3)+rand(),#1)}}%
				}%
		}%
}
\newcommand\VarNbAlea[2]{%
	\xdef#1{\fpeval{#2}}%
}

\defKV[TirageAleatoireEntiersEntre]{%
	ValMin=\def\TAEEmin{#1},%
	ValMax=\def\TAEEmax{#1},%
	NbVal=\def\TAEEnb{#1},%
	Sep=\def\TAEEsep{#1},%
	Tri=\def\TAEEtri{#1}
}

\setKVdefault[TirageAleatoireEntiersEntre]{%
	ValMin=1,%
	ValMax=50,%
	NbVal=6,%
	Sep={,},%
	Tri=non,%
	Repetition=false
}

\NewList{TirageEntiersAleatoires}
\NewDocumentCommand\TirageAleatoireEntiers{ O{} m }{%1=options,2=nomliste triée ou non
	\useKVdefault[TirageAleatoireEntiersEntre]%
	\setKV[TirageAleatoireEntiersEntre]{#1}%
	\ifboolKV[TirageAleatoireEntiersEntre]{Repetition}%suivant la possibilité de répéter
		{%répétitions autorisées
			%création de la liste (le 1er puis les autres)
			\xdef#2{\fpeval{randint(\TAEEmin,\TAEEmax)}}%
			\foreach \i in {2,3,...,\TAEEnb}%
				{\xdef#2{#2,\fpeval{randint(\TAEEmin,\TAEEmax)}}}%
		}%
		{%pas de répétitions
			%melange des nombres
			\ClearList{TirageEntiersAleatoires}%on vide la liste aléatoire
			\xintFor* ##1 in {\xintSeq{\TAEEmin}{\TAEEmax}}%
				\do{\InsertRandomItem{TirageEntiersAleatoires}{##1}}%
			%création de la liste (le 1er puis les autres)
			\xdef#2{\TirageEntiersAleatoires[0]}%
			\foreach \i in {1,2,...,{\inteval{\TAEEnb-1}}}%
				{\xdef#2{#2,\TirageEntiersAleatoires[\i]}}%
		}%
	%on trie si besoin
	\IfStrEq{\TAEEtri}{croissant}%si tri croissant
		{\TriListeCroiss{#2}}%
		{}%
	\IfStrEq{\TAEEtri}{decroissant}%si tri décroissant
		{\TriListeDecroiss{#2}}%
		{}%
	\StrSubstitute{#2}{,}{\TAEEsep}[#2]%on change le séparateur si besoin
}

\endinput