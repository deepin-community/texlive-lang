% proflycee-tools-probas.tex
% Copyright 2023-2024  Cédric Pierquet
% Released under the LaTeX Project Public License v1.3c or later, see http://www.latex-project.org/lppl.txt
% CC BY-SA 3.0 (from Mark Wibrow)

%%------LOISPROBAS
%calculs "simples"
\newcommand\CalcBinomP[3]{%npk
	\xintFloatToDecimal{\xintfloateval{binomial(#1,#3)*#2^#3*(1-#2)^(#1-#3)}}
}
\newcommand\CalcBinomC[4]{%npab
	\def\BorneInf{#3}\def\BorneSup{#4}%
	\ifthenelse{\equal{#3}{*}}%
		{\def\BorneInf{0}}%
		{}%
	\ifthenelse{\equal{#4}{*}}%
		{\def\BorneSup{#1}}%
		{}%
	\xintFloatToDecimal{\xintfloateval{add(binomial(#1,i)*#2^i*(1-#2)^(#1-i), i=\BorneInf..\BorneSup)}}%
}
\newcommand\CalcGeomP[2]{%pk
	\xintFloatToDecimal{\xintfloateval{(1-#1)^(#2-1)*(#1)}}%
}
\newcommand\CalcGeomC[3]{%pab
	\def\BorneInf{#2}\def\BorneSup{#3}%
	\ifthenelse{\equal{#2}{*}}%
		{\def\BorneInf{1}}%
		{}%
	\ifthenelse{\equal{#3}{*}}%
		{\def\BorneSup{\fpeval{trunc(1/#1*10,0)}}}%
		{}%
	\xintFloatToDecimal{\xintfloateval{add((1-#1)^(i-1)*(#1), i=\BorneInf..\BorneSup)}}%
}
\newcommand\CalcHypergeomP[4]{%Nnmk
	\xintFloatToDecimal{\xintfloateval{binomial(#3,#4)*binomial(#1-#3,#2-#4)/binomial(#1,#2)}}%
}
\newcommand\CalcHypergeomC[5]{%Nnmab
	\def\BorneInf{#4}\def\BorneSup{#5}%
	\ifthenelse{\equal{#4}{*}}%
		{\def\BorneInf{0}}%
		{}%
	\ifthenelse{\equal{#5}{*}}%
		{\def\BorneSup{#1}}%
		{}%
	\xintFloatToDecimal{\xintfloateval{add(binomial(#3,i)*binomial(#1-#3,#2-i)/binomial(#1,#2), i=\BorneInf..\BorneSup)}}%
}
\newcommand\CalcPoissP[2]{%lk
	\xintFloatToDecimal{\xintfloateval{exp(-#1)*#1^#2/factorial(#2)}}%
}
\newcommand\CalcPoissC[3]{%lab
	\def\BorneInf{#2}\def\BorneSup{#3}%
	\ifthenelse{\equal{#2}{*}}%
		{\def\BorneInf{0}}%
		{}%
	\ifthenelse{\equal{#3}{*}}%
		{\def\BorneSup{10*#1}}%
		{}%
	\xintFloatToDecimal{\xintfloateval{add(exp(-#1)*#1^i/factorial(i), i=\BorneInf..\BorneSup)}}%
}
%utiles idée de https://tex.stackexchange.com/questions/355574/im-searching-for-a-table-with-cdf-of-standard-normal-distribution
\xintdeffloatvar a_1,a_2,a_3,a_4,a_5,a_6 :=
0.0705230784,0.0422820123,0.0092705272,0.0001520143,0.0002765672,0.0000430638;
\xintdeffloatvar rac2inv := sqrt(2)/2 ;
\xintdeffloatfunc erfc(x) := % uniquement pour x positif
((((((a_6*x+a_5)*x+a_4)*x+a_3)*x+a_2)*x+a_1)*x+1)**-16 ; % fin de définition de erfc(x)
\xintdeffloatfunc Phi(z) := (z>=0)?{1 - 0.5 * erfc(rac2inv * z)}{0.5 * erfc(-rac2inv * z)};

\newcommand\CalcNormC[4]{%msab
	%def des bornes de l'intervalle suivant l'absence de a ou de b...
	\def\BorneInf{#3}\def\BorneSup{#4}%
	\ifthenelse{\equal{#3}{*}}%
		{\def\BorneInf{#4-10*#2}}%
		{}%
	\ifthenelse{\equal{#4}{*}}%
		{\def\BorneSup{#3+10*#2}}%
		{}%
	\xintFloatToDecimal{\xintfloateval{Phi((\BorneSup-#1)/#2)-Phi((\BorneInf-#1)/#2)}}%
}
%calculs "simples" fiabilite
\newcommand\CalcExpoC[3]{%lab
	\def\BorneInf{#2}\def\BorneSup{#3}
	\ifthenelse{\equal{#2}{*}}%
		{\def\BorneInf{0}}%
		{}%
	\ifthenelse{\equal{#3}{*}}%
		{\def\BorneSup{100/#1}}%
		{}%
	\xintFloatToDecimal{\xintfloateval{exp(-#1*\BorneInf)-exp(-#1*\BorneSup)}}%
}

%calculs formatés
\NewDocumentCommand{\BinomP}{ s O{3} m m m }{%*=sci,2=prec,3=n,4=p,5=k
	\IfBooleanTF{#1}%
	{%
		\num[exponent-mode=scientific]{\xintFloatToDecimal{\xintfloateval{round(binomial(#3,#5)*#4^#5*(1-#4)^(#3-#5),#2)}}}%
	}%
	{%
		\num[minimum-decimal-digits=#2]{\xintFloatToDecimal{\xintfloateval{round(binomial(#3,#5)*#4^#5*(1-#4)^(#3-#5),#2)}}}%
	}%
}

\NewDocumentCommand{\BinomC}{ s O{3} m m m m }{%*=sci,2=prec,3=n,4=p,5=a,6=b
	\def\BorneInf{#5}\def\BorneSup{#6}%
	\ifthenelse{\equal{#5}{*}}%
		{\def\BorneInf{0}}%
		{}%
	\ifthenelse{\equal{#6}{*}}%
		{\def\BorneSup{#3}}%
		{}%
	\IfBooleanTF{#1}%
	{%
		\num[exponent-mode=scientific]{\xintFloatToDecimal{\xintfloateval{round(add(binomial(#3,i)*#4^i*(1-#4)^(#3-i), i=\BorneInf..\BorneSup),#2)}}}%
	}%
	{%
		\num[minimum-decimal-digits=#2]{\xintFloatToDecimal{\xintfloateval{round(add(binomial(#3,i)*#4^i*(1-#4)^(#3-i), i=\BorneInf..\BorneSup),#2)}}}%
	}%
}

\NewDocumentCommand{\GeomP}{ s O{3} m m }{%*=sci,2=prec,3=p,4=k
	\IfBooleanTF{#1}%
	{%
		\num[exponent-mode=scientific]{\xintFloatToDecimal{\xintfloateval{round((1-#3)^(#4-1)*(#3),#2)}}}%
	}%
	{%
		\num[minimum-decimal-digits=#2]{\xintFloatToDecimal{\xintfloateval{round((1-#3)^(#4-1)*(#3),#2)}}}%
	}%
}

\NewDocumentCommand{\GeomC}{ s O{3} m m m }{%*=sci,2=prec,3=p,4=a,5=b
	\def\BorneInf{#4}\def\BorneSup{#5}%
	\ifthenelse{\equal{#4}{*}}%
		{\def\BorneInf{1}}%
		{}%
	\ifthenelse{\equal{#5}{*}}%
		{\def\BorneSup{\fpeval{trunc(1/#3*10,0)}}}%
		{}%
	\IfBooleanTF{#1}%
	{%
		\num[exponent-mode=scientific]{\xintFloatToDecimal{\xintfloateval{round(add((1-#3)^(i-1)*(#3), i=\BorneInf..\BorneSup),#2)}}}%
	}%
	{%
		\num[minimum-decimal-digits=#2]{\xintFloatToDecimal{\xintfloateval{round(add((1-#3)^(i-1)*(#3), i=\BorneInf..\BorneSup),#2)}}}%
	}
}

\NewDocumentCommand{\HypergeomP}{ s O{3} m m m m }{%*=sci,2=prec,3=N,4=n,5=m,6=k
	\IfBooleanTF{#1}%
	{%
		\num[exponent-mode=scientific]{\xintFloatToDecimal{\xintfloateval{round(binomial(#5,#6)*binomial(#3-#5,#4-#6)/binomial(#3,#4),#2)}}}%
	}%
	{%
		\num[minimum-decimal-digits=#2]{\xintFloatToDecimal{\xintfloateval{round(binomial(#5,#6)*binomial(#3-#5,#4-#6)/binomial(#3,#4),#2)}}}%
	}
}

\NewDocumentCommand{\HypergeomC}{ s O{3} m m m m m }{%*=sci,2=prec,3=N,4=n,5=m,6=a,7=b
	\def\BorneInf{#6}\def\BorneSup{#7}%
	\ifthenelse{\equal{#6}{*}}%
		{\def\BorneInf{0}}%
		{}%
	\ifthenelse{\equal{#7}{*}}%
		{\def\BorneSup{#3}}%
		{}%
	\IfBooleanTF{#1}%
	{%
		\num[exponent-mode=scientific]{\xintFloatToDecimal{\xintfloateval{round(add(binomial(#5,i)*binomial(#3-#5,#4-i)/binomial(#3,#4), i=\BorneInf..\BorneSup),#2)}}}%
	}%
	{%
		\num[minimum-decimal-digits=#2]{\xintFloatToDecimal{\xintfloateval{round(add(binomial(#5,i)*binomial(#3-#5,#4-i)/binomial(#3,#4), i=\BorneInf..\BorneSup),#2)}}}
	}%
}

\NewDocumentCommand{\PoissonP}{ s O{3} m m }{%*=sci,2=prec,3=lbda,4=k
	\IfBooleanTF{#1}%
	{%
		\num[exponent-mode=scientific]{\xintFloatToDecimal{\xintfloateval{round(exp(-#3)*#3^#4/factorial(#4),#2)}}}%
	}%
	{%
		\num[minimum-decimal-digits=#2]{\xintFloatToDecimal{\xintfloateval{round(exp(-#3)*#3^#4/factorial(#4),#2)}}}%
	}%
}

\NewDocumentCommand{\PoissonC}{ s O{3} m m m }{%*=ing,2=prec,3=lbda,4=a,5=b
	\def\BorneInf{#4}\def\BorneSup{#5}%
	\ifthenelse{\equal{#4}{*}}%
		{\def\BorneInf{0}}%
		{}%
	\ifthenelse{\equal{#5}{*}}%
		{\def\BorneSup{10*#3}}%
		{}%
	\IfBooleanTF{#1}%
	{%
		\num[exponent-mode=scientific]{\xintFloatToDecimal{\xintfloateval{round(add(exp(-#3)*#3^i/factorial(i), i=\BorneInf..\BorneSup),#2)}}}%
	}%
	{%
		\num[minimum-decimal-digits=#2]{\xintFloatToDecimal{\xintfloateval{round(add(exp(-#3)*#3^i/factorial(i), i=\BorneInf..\BorneSup),#2)}}}%
	}%
}

\NewDocumentCommand{\NormaleC}{ s O{3} m m m m }{%
	%*=ing,2=prec,3=mu,4=sigma,5=a,6=b
	%def des bornes de l'intervalle suivant l'absence de a ou de b...
	\def\BorneInf{#5}\def\BorneSup{#6}%
	\ifthenelse{\equal{#5}{*}}%
		{\def\BorneInf{#6-10*#4}}%
		{}%
	\ifthenelse{\equal{#6}{*}}%
		{\def\BorneSup{#5+10*#4}}%
		{}%
	\IfBooleanTF{#1}%
	{%
		\num[exponent-mode=scientific]{\xintFloatToDecimal{\xintfloateval{round(Phi((\BorneSup-#3)/#4)-Phi((\BorneInf-#3)/#4),#2)}}}%
	}%
	{%
		\num[minimum-decimal-digits=#2]{\xintFloatToDecimal{\xintfloateval{round(Phi((\BorneSup-#3)/#4)-Phi((\BorneInf-#3)/#4),#2)}}}%
	}%
}

\NewDocumentCommand{\ExpoC}{ s O{3} m m m }{%*=ing,2=prec,3=lbda,4=a,5=b
	\def\BorneInf{#4}\def\BorneSup{#5}%
	\ifthenelse{\equal{#4}{*}}%
		{\def\BorneInf{0}}%
		{}%
	\ifthenelse{\equal{#5}{*}}%
		{\def\BorneSup{100/#3}}%
		{}%
	\IfBooleanTF{#1}%
	{%
		\num[exponent-mode=scientific]{\xintFloatToDecimal{\xintfloateval{round(exp(-#3*\BorneInf)-exp(-#3*\BorneSup),#2)}}}%
	}%
	{%
		\num[minimum-decimal-digits=#2]{\xintFloatToDecimal{\xintFloatToDecimal{\xintfloateval{round(exp(-#3*\BorneInf)-exp(-#3*\BorneSup),#2)}}}}
	}%
}

%%------ARBRESPROBAS
\NewDocumentCommand\numdots{}{\vphantom{0,5}\ldots}

\defKV[PLTIKZARBRE]{%
	Unite=\def\PLAPunite{#1},%
	EspaceNiveau=\def\PLAPespniv{#1},%
	EspaceFeuille=\def\PLAPespfeuille{#1},%
	Type=\def\PLAPtype{#1},%
	Police=\def\PLAPfont{#1},%
	PoliceProbas=\def\PLAPfontproba{#1},%
	StyleTrait=\def\PLAPtypetrait{#1},%
	EpaisseurTrait=\def\PLAPeptrait{#1},%
	PositionProbas=\def\PLAPposprob{#1},%
	CouleurFond=\def\PLAPcolback{#1},%
	CouleurTraits=\def\PLAPcolarc{#1},%
	CouleurProbas=\def\PLAPcolprobas{#1},%
	CouleurNoeuds=\def\PLAPcolnoeuds{#1}
}

\setKVdefault[PLTIKZARBRE]{%
	Unite=1cm,%
	EspaceNiveau=3.25,%
	EspaceFeuille=1,%
	Type=2x2,%
	Police=\normalfont\normalsize,%
	PoliceProbas=\normalfont\small,%
	InclineProbas=true,%
	Fleche=false,%
	StyleTrait={},%
	EpaisseurTrait=semithick,%
	PositionProbas={},%
	CouleurFond=white,%
	CouleurTraits=black,%
	CouleurProbas=black,%
	CouleurNoeuds=black
}

\NewDocumentCommand\IntPlaceProbasArbre{ m m m m m }{%
	%1=nœud initial
	%2=nœud final
	%3=positionlistofitems
	%4=probaslistofitems
	%5=auto
	\IfStrEqCase{\PLAPposprob}{%
		{}{\draw[PLfleche] (#1)--(#2) node[probas,PLetiquette,{#3},fill=\PLAPcolback] {{#4}} ;}%PositionProbas vide = on lit
		{auto}{\draw[PLfleche] (#1)--(#2) node[probas,PLetiquette,#5] {{#4}} ;}%
		{dessus}{\draw[PLfleche] (#1)--(#2) node[probas,PLetiquette,above=1pt] {{#4}} ;}%
		{dessous}{\draw[PLfleche] (#1)--(#2) node[probas,PLetiquette,below=1pt] {{#4}} ;}%
		{sur}{\draw[PLfleche] (#1)--(#2) node[probas,PLetiquette,fill=\PLAPcolback] {{#4}} ;}%
	}%
}

\NewDocumentCommand\ArbreProbasTikz{ O{} D<>{} m }{%1=options,2=libellés
	\useKVdefault[PLTIKZARBRE]%
	\setKV[PLTIKZARBRE]{#1}%
	\setsepchar[.]{,./}%
	\readlist*\PLARBREDONNES{#3}%
	\begin{tikzpicture}[x=\PLAPunite,y=\PLAPunite,#2]
		%styles
		\tikzstyle{noeud} = [font=\PLAPfont,text=\PLAPcolnoeuds]
		\tikzstyle{probas}= [font=\PLAPfontproba,text=\PLAPcolprobas]
		\ifboolKV[PLTIKZARBRE]{InclineProbas}
			{\tikzstyle{PLetiquette}=[pos=0.525,sloped,inner sep=2pt,outer sep=2pt,rectangle]}
			{\tikzstyle{PLetiquette}=[pos=0.525,inner sep=2pt,outer sep=2pt,rectangle]}
		\ifboolKV[PLTIKZARBRE]{Fleche}
			{\tikzstyle{PLfleche}=[->,>=latex,\PLAPtypetrait,\PLAPeptrait,\PLAPcolarc]}
			{\tikzstyle{PLfleche}=[\PLAPtypetrait,\PLAPeptrait,\PLAPcolarc]}
		%arbre2x2
		\IfEq{\PLAPtype}{2x2}
		{%
			\coordinate (R) at ({0*\PLAPespniv},{-1.5*\PLAPespfeuille}) ;
			\node[noeud] (A11) at ({1*\PLAPespniv},{-0.5*\PLAPespfeuille}) {\PLARBREDONNES[1,1]};
			\node[noeud] (A21) at ({2*\PLAPespniv},{-0*\PLAPespfeuille}) {\PLARBREDONNES[2,1]};
			\node[noeud] (A22) at ({2*\PLAPespniv},{-1*\PLAPespfeuille}) {\PLARBREDONNES[3,1]};
			\node[noeud] (A12) at ({1*\PLAPespniv},{-2.5*\PLAPespfeuille}) {\PLARBREDONNES[4,1]};
			\node[noeud] (A23) at ({2*\PLAPespniv},{-2*\PLAPespfeuille}) {\PLARBREDONNES[5,1]};
			\node[noeud] (A24) at ({2*\PLAPespniv},{-3*\PLAPespfeuille}) {\PLARBREDONNES[6,1]};
			\ifstrempty{\PLARBREDONNES[1,2]}
				{\draw[PLfleche] (R)--(A11) ;}
				{%
					\IntPlaceProbasArbre{R}{A11}{\PLARBREDONNES[1,3]}{\PLARBREDONNES[1,2]}{above=1pt}
					%\draw[PLfleche] (R)--(A11) node[probas,PLetiquette,{\PLARBREDONNES[1,3]}] {{\PLARBREDONNES[1,2]}} ;%
				}%
			\ifstrempty{\PLARBREDONNES[2,2]}
				{\draw[PLfleche] (A11)--(A21) ;}
				{%
					\IntPlaceProbasArbre{A11}{A21}{\PLARBREDONNES[2,3]}{\PLARBREDONNES[2,2]}{above=1pt}
					%\draw[PLfleche] (A11)--(A21) node[probas,PLetiquette,{\PLARBREDONNES[2,3]}] {{\PLARBREDONNES[2,2]}} ;%
				}%
			\ifstrempty{\PLARBREDONNES[3,2]}
				{\draw[PLfleche] (A11)--(A22) ;}
				{%
					\IntPlaceProbasArbre{A11}{A22}{\PLARBREDONNES[3,3]}{\PLARBREDONNES[3,2]}{below=1pt}
					%\draw[PLfleche] (A11)--(A22) node[probas,PLetiquette,{\PLARBREDONNES[3,3]}] {{\PLARBREDONNES[3,2]}} ;
				}%
			\ifstrempty{\PLARBREDONNES[4,2]}
				{\draw[PLfleche] (R)--(A12) ;}
				{%
					\IntPlaceProbasArbre{R}{A12}{\PLARBREDONNES[4,3]}{\PLARBREDONNES[4,2]}{below=1pt}
					%\draw[PLfleche] (R)--(A12) node[probas,PLetiquette,{\PLARBREDONNES[4,3]}] {{\PLARBREDONNES[4,2]}} ;%
				}%
			\ifstrempty{\PLARBREDONNES[5,2]}
				{\draw[PLfleche] (A12)--(A23) ;}
				{%
					\IntPlaceProbasArbre{A12}{A23}{\PLARBREDONNES[5,3]}{\PLARBREDONNES[5,2]}{above=1pt}
					%\draw[PLfleche] (A12)--(A23) node[probas,PLetiquette,{\PLARBREDONNES[5,3]}] {{\PLARBREDONNES[5,2]}} ;%
				}%
			\ifstrempty{\PLARBREDONNES[6,2]}
				{\draw[PLfleche] (A12)--(A24) ;}
				{%
					\IntPlaceProbasArbre{A12}{A24}{\PLARBREDONNES[6,3]}{\PLARBREDONNES[6,2]}{below=1pt}
					%\draw[PLfleche] (A12)--(A24) node[probas,PLetiquette,{\PLARBREDONNES[6,3]}] {{\PLARBREDONNES[6,2]}} ;%
				}%
		}{}
		%arbre3x2
		\IfEq{\PLAPtype}{3x2}
		{%
			%sommets
			\coordinate (R) at ({0*\PLAPespniv},{-2.5*\PLAPespfeuille}) ;
			\node[noeud] (A11) at ({1*\PLAPespniv},{-0.5*\PLAPespfeuille}) {\PLARBREDONNES[1,1]};
			\node[noeud] (A21) at ({2*\PLAPespniv},{-0*\PLAPespfeuille}) {\PLARBREDONNES[2,1]};
			\node[noeud] (A22) at ({2*\PLAPespniv},{-1*\PLAPespfeuille}) {\PLARBREDONNES[3,1]};
			\node[noeud] (A12) at ({1*\PLAPespniv},{-2.5*\PLAPespfeuille}) {\PLARBREDONNES[4,1]};
			\node[noeud] (A23) at ({2*\PLAPespniv},{-2*\PLAPespfeuille}) {\PLARBREDONNES[5,1]};
			\node[noeud] (A24) at ({2*\PLAPespniv},{-3*\PLAPespfeuille}) {\PLARBREDONNES[6,1]};
			\node[noeud] (A13) at ({1*\PLAPespniv},{-4.5*\PLAPespfeuille}) {\PLARBREDONNES[7,1]};
			\node[noeud] (A25) at ({2*\PLAPespniv},{-4*\PLAPespfeuille}) {\PLARBREDONNES[8,1]};
			\node[noeud] (A26) at ({2*\PLAPespniv},{-5*\PLAPespfeuille}) {\PLARBREDONNES[9,1]};
			%branches
			\ifstrempty{\PLARBREDONNES[1,2]}
				{\draw[PLfleche] (R)--(A11) ;}
				{%
					\IntPlaceProbasArbre{R}{A11}{\PLARBREDONNES[1,3]}{\PLARBREDONNES[1,2]}{above=1pt}
					%\draw[PLfleche] (R)--(A11) node[probas,PLetiquette,{\PLARBREDONNES[1,3]}] {{\PLARBREDONNES[1,2]}} ;%
				}%
			\ifstrempty{\PLARBREDONNES[2,2]}
				{\draw[PLfleche] (A11)--(A21) ;}
				{%
					\IntPlaceProbasArbre{A11}{A21}{\PLARBREDONNES[2,3]}{\PLARBREDONNES[2,2]}{above=1pt}
					%\draw[PLfleche] (A11)--(A21) node[probas,PLetiquette,{\PLARBREDONNES[2,3]}] {{\PLARBREDONNES[2,2]}} ;
				}%
			\ifstrempty{\PLARBREDONNES[3,2]}
				{\draw[PLfleche] (A11)--(A22) ;}
				{
					\IntPlaceProbasArbre{A11}{A22}{\PLARBREDONNES[3,3]}{\PLARBREDONNES[3,2]}{below=1pt}
					%\draw[PLfleche] (A11)--(A22) node[probas,PLetiquette,{\PLARBREDONNES[3,3]}] {{\PLARBREDONNES[3,2]}} ;%
				}
			\ifstrempty{\PLARBREDONNES[4,2]}
				{\draw[PLfleche] (R)--(A12) ;}
				{
					\IntPlaceProbasArbre{R}{A12}{\PLARBREDONNES[4,3]}{\PLARBREDONNES[4,2]}{above=-1pt}
					%\draw[PLfleche] (R)--(A12) node[probas,PLetiquette,{\PLARBREDONNES[4,3]}] {{\PLARBREDONNES[4,2]}} ;
				}
			\ifstrempty{\PLARBREDONNES[5,2]}
				{\draw[PLfleche] (A12)--(A23) ;}
				{
					\IntPlaceProbasArbre{A12}{A23}{\PLARBREDONNES[5,3]}{\PLARBREDONNES[5,2]}{above=1pt}
					%\draw[PLfleche] (A12)--(A23) node[probas,PLetiquette,{\PLARBREDONNES[5,3]}] {{\PLARBREDONNES[5,2]}} ;
				}
			\ifstrempty{\PLARBREDONNES[6,2]}
				{\draw[PLfleche] (A12)--(A24) ;}
				{
					\IntPlaceProbasArbre{A12}{A24}{\PLARBREDONNES[6,3]}{\PLARBREDONNES[6,2]}{below=1pt}
					%\draw[PLfleche] (A12)--(A24) node[probas,PLetiquette,{\PLARBREDONNES[6,3]}] {{\PLARBREDONNES[6,2]}} ;
				}
			\ifstrempty{\PLARBREDONNES[7,2]}
				{\draw[PLfleche] (R)--(A13) ;}
				{
					\IntPlaceProbasArbre{R}{A13}{\PLARBREDONNES[7,3]}{\PLARBREDONNES[7,2]}{below=1pt}
					%\draw[PLfleche] (R)--(A13) node[probas,PLetiquette,{\PLARBREDONNES[7,3]}] {{\PLARBREDONNES[7,2]}} ;
				}
			\ifstrempty{\PLARBREDONNES[8,2]}
				{\draw[PLfleche] (A13)--(A25) ;}
				{
					\IntPlaceProbasArbre{A13}{A25}{\PLARBREDONNES[8,3]}{\PLARBREDONNES[8,2]}{above=1pt}
					%\draw[PLfleche] (A13)--(A25) node[probas,PLetiquette,{\PLARBREDONNES[8,3]}] {{\PLARBREDONNES[8,2]}} ;
				}
			\ifstrempty{\PLARBREDONNES[9,2]}
				{\draw[PLfleche] (A13)--(A26) ;}
				{
					\IntPlaceProbasArbre{A13}{A26}{\PLARBREDONNES[9,3]}{\PLARBREDONNES[9,2]}{below=1pt}
					%\draw[PLfleche] (A13)--(A26) node[probas,PLetiquette,{\PLARBREDONNES[9,3]}] {{\PLARBREDONNES[9,2]}} ;
				}
		}{}
		%arbre2x3
		\IfEq{\PLAPtype}{2x3}
		{%
			%sommets
			\coordinate (R) at ({0*\PLAPespniv},{-2.5*\PLAPespfeuille}) ;
			\node[noeud] (A11) at ({1*\PLAPespniv},{-1*\PLAPespfeuille}) {\PLARBREDONNES[1,1]};
			\node[noeud] (A12) at ({1*\PLAPespniv},{-4*\PLAPespfeuille}) {\PLARBREDONNES[5,1]};
			\node[noeud] (A21) at ({2*\PLAPespniv},{-0*\PLAPespfeuille}) {\PLARBREDONNES[2,1]};
			\node[noeud] (A22) at ({2*\PLAPespniv},{-1*\PLAPespfeuille}) {\PLARBREDONNES[3,1]};
			\node[noeud] (A23) at ({2*\PLAPespniv},{-2*\PLAPespfeuille}) {\PLARBREDONNES[4,1]};
			\node[noeud] (A24) at ({2*\PLAPespniv},{-3*\PLAPespfeuille}) {\PLARBREDONNES[6,1]};
			\node[noeud] (A25) at ({2*\PLAPespniv},{-4*\PLAPespfeuille}) {\PLARBREDONNES[7,1]};
			\node[noeud] (A26) at ({2*\PLAPespniv},{-5*\PLAPespfeuille}) {\PLARBREDONNES[8,1]};
			%branches
			\ifstrempty{\PLARBREDONNES[1,2]}
				{\draw[PLfleche] (R)--(A11) ;}
				{
					\IntPlaceProbasArbre{R}{A11}{\PLARBREDONNES[1,3]}{\PLARBREDONNES[1,2]}{above=1pt}
					%\draw[PLfleche] (R)--(A11) node[probas,PLetiquette,{\PLARBREDONNES[1,3]}] {{\PLARBREDONNES[1,2]}} ;
				}
			\ifstrempty{\PLARBREDONNES[5,2]}
				{\draw[PLfleche] (R)--(A12) ;}
				{
					\IntPlaceProbasArbre{R}{A12}{\PLARBREDONNES[5,3]}{\PLARBREDONNES[5,2]}{below=1pt}
					%\draw[PLfleche] (R)--(A12) node[probas,PLetiquette,{\PLARBREDONNES[5,3]}] {{\PLARBREDONNES[5,2]}} ;
				}
			\ifstrempty{\PLARBREDONNES[2,2]}
				{\draw[PLfleche] (A11)--(A21) ;}
				{
					\IntPlaceProbasArbre{A11}{A21}{\PLARBREDONNES[2,3]}{\PLARBREDONNES[2,2]}{above=1pt}
					%\draw[PLfleche] (A11)--(A21) node[probas,PLetiquette,{\PLARBREDONNES[2,3]}] {{\PLARBREDONNES[2,2]}} ;
				}
			\ifstrempty{\PLARBREDONNES[3,2]}
				{\draw[PLfleche] (A11)--(A22) ;}
				{
					\IntPlaceProbasArbre{A11}{A22}{\PLARBREDONNES[3,3]}{\PLARBREDONNES[3,2]}{above=-1pt}
					%\draw[PLfleche] (A11)--(A22) node[probas,PLetiquette,{\PLARBREDONNES[3,3]}] {{\PLARBREDONNES[3,2]}} ;
				}
			\ifstrempty{\PLARBREDONNES[4,2]}
				{\draw[PLfleche] (A11)--(A23) ;}
				{
					\IntPlaceProbasArbre{A11}{A23}{\PLARBREDONNES[4,3]}{\PLARBREDONNES[4,2]}{below=1pt}
					%\draw[PLfleche] (A11)--(A23) node[probas,PLetiquette,{\PLARBREDONNES[4,3]}] {{\PLARBREDONNES[4,2]}} ;
				}
			\ifstrempty{\PLARBREDONNES[6,2]}
				{\draw[PLfleche] (A12)--(A24) ;}
				{
					\IntPlaceProbasArbre{A12}{A24}{\PLARBREDONNES[6,3]}{\PLARBREDONNES[6,2]}{above=1pt}
					%\draw[PLfleche] (A12)--(A24) node[probas,PLetiquette,{\PLARBREDONNES[6,3]}] {{\PLARBREDONNES[6,2]}} ;
				}
			\ifstrempty{\PLARBREDONNES[7,2]}
				{\draw[PLfleche] (A12)--(A25) ;}
				{
					\IntPlaceProbasArbre{A12}{A25}{\PLARBREDONNES[7,3]}{\PLARBREDONNES[7,2]}{above=-1pt}
					%\draw[PLfleche] (A12)--(A25) node[probas,PLetiquette,{\PLARBREDONNES[7,3]}] {{\PLARBREDONNES[7,2]}} ;
				}
			\ifstrempty{\PLARBREDONNES[8,2]}
				{\draw[PLfleche] (A12)--(A26) ;}
				{
					\IntPlaceProbasArbre{A12}{A26}{\PLARBREDONNES[8,3]}{\PLARBREDONNES[8,2]}{below=1pt}
					%\draw[PLfleche] (A12)--(A26) node[probas,PLetiquette,{\PLARBREDONNES[8,3]}] {{\PLARBREDONNES[8,2]}} ;
				}
		}{}
		%arbre2x3
		\IfEq{\PLAPtype}{3x3}
		{%
			%sommets
			\coordinate (R) at ({0*\PLAPespniv},{-4*\PLAPespfeuille}) ;
			\node[noeud] (A11) at ({1*\PLAPespniv},{-1*\PLAPespfeuille}) {\PLARBREDONNES[1,1]};
			\node[noeud] (A12) at ({1*\PLAPespniv},{-4*\PLAPespfeuille}) {\PLARBREDONNES[5,1]};
			\node[noeud] (A13) at ({1*\PLAPespniv},{-7*\PLAPespfeuille}) {\PLARBREDONNES[9,1]};
			\node[noeud] (A21) at ({2*\PLAPespniv},{-0*\PLAPespfeuille}) {\PLARBREDONNES[2,1]};
			\node[noeud] (A22) at ({2*\PLAPespniv},{-1*\PLAPespfeuille}) {\PLARBREDONNES[3,1]};
			\node[noeud] (A23) at ({2*\PLAPespniv},{-2*\PLAPespfeuille}) {\PLARBREDONNES[4,1]};
			\node[noeud] (A24) at ({2*\PLAPespniv},{-3*\PLAPespfeuille}) {\PLARBREDONNES[6,1]};
			\node[noeud] (A25) at ({2*\PLAPespniv},{-4*\PLAPespfeuille}) {\PLARBREDONNES[7,1]};
			\node[noeud] (A26) at ({2*\PLAPespniv},{-5*\PLAPespfeuille}) {\PLARBREDONNES[8,1]};
			\node[noeud] (A27) at ({2*\PLAPespniv},{-6*\PLAPespfeuille}) {\PLARBREDONNES[10,1]};
			\node[noeud] (A28) at ({2*\PLAPespniv},{-7*\PLAPespfeuille}) {\PLARBREDONNES[11,1]};
			\node[noeud] (A29) at ({2*\PLAPespniv},{-8*\PLAPespfeuille}) {\PLARBREDONNES[12,1]};
			%branches
			\ifstrempty{\PLARBREDONNES[1,2]}
				{\draw[PLfleche] (R)--(A11) ;}
				{
					\IntPlaceProbasArbre{R}{A11}{\PLARBREDONNES[1,3]}{\PLARBREDONNES[1,2]}{above=1pt}
					%\draw[PLfleche] (R)--(A11) node[probas,PLetiquette,{\PLARBREDONNES[1,3]}] {{\PLARBREDONNES[1,2]}} ;
				}
			\ifstrempty{\PLARBREDONNES[5,2]}
				{\draw[PLfleche] (R)--(A12) ;}
				{
					\IntPlaceProbasArbre{R}{A12}{\PLARBREDONNES[5,3]}{\PLARBREDONNES[5,2]}{above=-1pt}
					%\draw[PLfleche] (R)--(A12) node[probas,PLetiquette,{\PLARBREDONNES[5,3]}] {{\PLARBREDONNES[5,2]}} ;
				}
			\ifstrempty{\PLARBREDONNES[9,2]}
				{\draw[PLfleche] (R)--(A13) ;}
				{
					\IntPlaceProbasArbre{R}{A13}{\PLARBREDONNES[9,3]}{\PLARBREDONNES[9,2]}{below=1pt}
					%\draw[PLfleche] (R)--(A13) node[probas,PLetiquette,{\PLARBREDONNES[9,3]}] {{\PLARBREDONNES[9,2]}} ;
				}
			%N2
			\ifstrempty{\PLARBREDONNES[2,2]}
				{\draw[PLfleche] (A11)--(A21) ;}
				{
					\IntPlaceProbasArbre{A11}{A21}{\PLARBREDONNES[2,3]}{\PLARBREDONNES[2,2]}{above=1pt}
					%\draw[PLfleche] (A11)--(A21) node[probas,PLetiquette,{\PLARBREDONNES[2,3]}] {{\PLARBREDONNES[2,2]}} ;
				}
			\ifstrempty{\PLARBREDONNES[3,2]}
				{\draw[PLfleche] (A11)--(A22) ;}
				{
					\IntPlaceProbasArbre{A11}{A22}{\PLARBREDONNES[3,3]}{\PLARBREDONNES[3,2]}{above=-1pt}
					%\draw[PLfleche] (A11)--(A22) node[probas,PLetiquette,{\PLARBREDONNES[3,3]}] {{\PLARBREDONNES[3,2]}} ;
				}
			\ifstrempty{\PLARBREDONNES[4,2]}
				{\draw[PLfleche] (A11)--(A23) ;}
				{
					\IntPlaceProbasArbre{A11}{A23}{\PLARBREDONNES[4,3]}{\PLARBREDONNES[4,2]}{below=1pt}
					%\draw[PLfleche] (A11)--(A23) node[probas,PLetiquette,{\PLARBREDONNES[4,3]}] {{\PLARBREDONNES[4,2]}} ;
				}
			\ifstrempty{\PLARBREDONNES[6,2]}
				{\draw[PLfleche] (A12)--(A24) ;}
				{
					\IntPlaceProbasArbre{A12}{A24}{\PLARBREDONNES[6,3]}{\PLARBREDONNES[6,2]}{above=1pt}
					%\draw[PLfleche] (A12)--(A24) node[probas,PLetiquette,{\PLARBREDONNES[6,3]}] {{\PLARBREDONNES[6,2]}} ;
				}
			\ifstrempty{\PLARBREDONNES[7,2]}
				{\draw[PLfleche] (A12)--(A25) ;}
				{
					\IntPlaceProbasArbre{A12}{A25}{\PLARBREDONNES[7,3]}{\PLARBREDONNES[7,2]}{above=-1pt}
					%\draw[PLfleche] (A12)--(A25) node[probas,PLetiquette,{\PLARBREDONNES[7,3]}] {{\PLARBREDONNES[7,2]}} ;
				}
			\ifstrempty{\PLARBREDONNES[8,2]}
				{\draw[PLfleche] (A12)--(A26) ;}
				{
					\IntPlaceProbasArbre{A12}{A26}{\PLARBREDONNES[8,3]}{\PLARBREDONNES[8,2]}{below=1pt}
					%\draw[PLfleche] (A12)--(A26) node[probas,PLetiquette,{\PLARBREDONNES[8,3]}] {{\PLARBREDONNES[8,2]}} ;
				}
			\ifstrempty{\PLARBREDONNES[10,2]}
				{\draw[PLfleche] (A13)--(A27) ;}
				{
					\IntPlaceProbasArbre{A13}{A27}{\PLARBREDONNES[10,3]}{\PLARBREDONNES[10,2]}{above=1pt}
					%\draw[PLfleche] (A13)--(A27) node[probas,PLetiquette,{\PLARBREDONNES[10,3]}] {{\PLARBREDONNES[10,2]}} ;
				}
			\ifstrempty{\PLARBREDONNES[11,2]}
				{\draw[PLfleche] (A13)--(A28) ;}
				{
					\IntPlaceProbasArbre{A13}{A28}{\PLARBREDONNES[11,3]}{\PLARBREDONNES[11,2]}{above=-1pt}
					%\draw[PLfleche] (A13)--(A28) node[probas,PLetiquette,{\PLARBREDONNES[11,3]}] {{\PLARBREDONNES[11,2]}} ;
				}
			\ifstrempty{\PLARBREDONNES[12,2]}
				{\draw[PLfleche] (A13)--(A29) ;}
				{
					\IntPlaceProbasArbre{A13}{A29}{\PLARBREDONNES[12,3]}{\PLARBREDONNES[12,2]}{below=1pt}
					%\draw[PLfleche] (A13)--(A29) node[probas,PLetiquette,{\PLARBREDONNES[12,3]}] {{\PLARBREDONNES[12,2]}} ;
				}
		}{}
	\end{tikzpicture}
}

\NewDocumentEnvironment{EnvArbreProbasTikz}{ O{} D<>{} m }%1=options,2=libellés
{
	\useKVdefault[PLTIKZARBRE]
	\setKV[PLTIKZARBRE]{#1}
	\setsepchar[.]{,./}%
	\readlist*\PLARBREDONNES{#3}
	\begin{tikzpicture}[x=\PLAPunite,y=\PLAPunite,#2]
		%styles
		\tikzstyle{noeud} = [font=\PLAPfont,text=\PLAPcolnoeuds]
		\tikzstyle{probas}= [font=\PLAPfontproba,text=\PLAPcolprobas]
		\ifboolKV[PLTIKZARBRE]{InclineProbas}
			{\tikzstyle{PLetiquette}=[pos=0.525,sloped,inner sep=2pt,outer sep=2pt,rectangle]}
			{\tikzstyle{PLetiquette}=[pos=0.525,inner sep=2pt,outer sep=2pt,rectangle]}
		\ifboolKV[PLTIKZARBRE]{Fleche}
			{\tikzstyle{PLfleche}=[->,>=latex,\PLAPtypetrait,\PLAPeptrait,\PLAPcolarc]}
			{\tikzstyle{PLfleche}=[\PLAPtypetrait,\PLAPeptrait,\PLAPcolarc]}
		%arbre2x2
		\IfEq{\PLAPtype}{2x2}
		{%
			\coordinate (R) at ({0*\PLAPespniv},{-1.5*\PLAPespfeuille}) ;
			\node[noeud] (A11) at ({1*\PLAPespniv},{-0.5*\PLAPespfeuille}) {\PLARBREDONNES[1,1]};
			\node[noeud] (A21) at ({2*\PLAPespniv},{-0*\PLAPespfeuille}) {\PLARBREDONNES[2,1]};
			\node[noeud] (A22) at ({2*\PLAPespniv},{-1*\PLAPespfeuille}) {\PLARBREDONNES[3,1]};
			\node[noeud] (A12) at ({1*\PLAPespniv},{-2.5*\PLAPespfeuille}) {\PLARBREDONNES[4,1]};
			\node[noeud] (A23) at ({2*\PLAPespniv},{-2*\PLAPespfeuille}) {\PLARBREDONNES[5,1]};
			\node[noeud] (A24) at ({2*\PLAPespniv},{-3*\PLAPespfeuille}) {\PLARBREDONNES[6,1]};
			\ifstrempty{\PLARBREDONNES[1,2]}
				{\draw[PLfleche] (R)--(A11) ;}
				{%
					\IntPlaceProbasArbre{R}{A11}{\PLARBREDONNES[1,3]}{\PLARBREDONNES[1,2]}{above=1pt}
					%\draw[PLfleche] (R)--(A11) node[probas,PLetiquette,{\PLARBREDONNES[1,3]}] {{\PLARBREDONNES[1,2]}} ;%
				}%
			\ifstrempty{\PLARBREDONNES[2,2]}
				{\draw[PLfleche] (A11)--(A21) ;}
				{%
					\IntPlaceProbasArbre{A11}{A21}{\PLARBREDONNES[2,3]}{\PLARBREDONNES[2,2]}{above=1pt}
					%\draw[PLfleche] (A11)--(A21) node[probas,PLetiquette,{\PLARBREDONNES[2,3]}] {{\PLARBREDONNES[2,2]}} ;%
				}%
			\ifstrempty{\PLARBREDONNES[3,2]}
				{\draw[PLfleche] (A11)--(A22) ;}
				{%
					\IntPlaceProbasArbre{A11}{A22}{\PLARBREDONNES[3,3]}{\PLARBREDONNES[3,2]}{below=1pt}
					%\draw[PLfleche] (A11)--(A22) node[probas,PLetiquette,{\PLARBREDONNES[3,3]}] {{\PLARBREDONNES[3,2]}} ;
				}%
			\ifstrempty{\PLARBREDONNES[4,2]}
				{\draw[PLfleche] (R)--(A12) ;}
				{%
					\IntPlaceProbasArbre{R}{A12}{\PLARBREDONNES[4,3]}{\PLARBREDONNES[4,2]}{below=1pt}
					%\draw[PLfleche] (R)--(A12) node[probas,PLetiquette,{\PLARBREDONNES[4,3]}] {{\PLARBREDONNES[4,2]}} ;%
				}%
			\ifstrempty{\PLARBREDONNES[5,2]}
				{\draw[PLfleche] (A12)--(A23) ;}
				{%
					\IntPlaceProbasArbre{A12}{A23}{\PLARBREDONNES[5,3]}{\PLARBREDONNES[5,2]}{above=1pt}
					%\draw[PLfleche] (A12)--(A23) node[probas,PLetiquette,{\PLARBREDONNES[5,3]}] {{\PLARBREDONNES[5,2]}} ;%
				}%
			\ifstrempty{\PLARBREDONNES[6,2]}
				{\draw[PLfleche] (A12)--(A24) ;}
				{%
					\IntPlaceProbasArbre{A12}{A24}{\PLARBREDONNES[6,3]}{\PLARBREDONNES[6,2]}{below=1pt}
					%\draw[PLfleche] (A12)--(A24) node[probas,PLetiquette,{\PLARBREDONNES[6,3]}] {{\PLARBREDONNES[6,2]}} ;%
				}%
		}{}
		%arbre3x2
		\IfEq{\PLAPtype}{3x2}
		{%
			%sommets
			\coordinate (R) at ({0*\PLAPespniv},{-2.5*\PLAPespfeuille}) ;
			\node[noeud] (A11) at ({1*\PLAPespniv},{-0.5*\PLAPespfeuille}) {\PLARBREDONNES[1,1]};
			\node[noeud] (A21) at ({2*\PLAPespniv},{-0*\PLAPespfeuille}) {\PLARBREDONNES[2,1]};
			\node[noeud] (A22) at ({2*\PLAPespniv},{-1*\PLAPespfeuille}) {\PLARBREDONNES[3,1]};
			\node[noeud] (A12) at ({1*\PLAPespniv},{-2.5*\PLAPespfeuille}) {\PLARBREDONNES[4,1]};
			\node[noeud] (A23) at ({2*\PLAPespniv},{-2*\PLAPespfeuille}) {\PLARBREDONNES[5,1]};
			\node[noeud] (A24) at ({2*\PLAPespniv},{-3*\PLAPespfeuille}) {\PLARBREDONNES[6,1]};
			\node[noeud] (A13) at ({1*\PLAPespniv},{-4.5*\PLAPespfeuille}) {\PLARBREDONNES[7,1]};
			\node[noeud] (A25) at ({2*\PLAPespniv},{-4*\PLAPespfeuille}) {\PLARBREDONNES[8,1]};
			\node[noeud] (A26) at ({2*\PLAPespniv},{-5*\PLAPespfeuille}) {\PLARBREDONNES[9,1]};
			%branches
			\ifstrempty{\PLARBREDONNES[1,2]}
				{\draw[PLfleche] (R)--(A11) ;}
				{%
					\IntPlaceProbasArbre{R}{A11}{\PLARBREDONNES[1,3]}{\PLARBREDONNES[1,2]}{above=1pt}
					%\draw[PLfleche] (R)--(A11) node[probas,PLetiquette,{\PLARBREDONNES[1,3]}] {{\PLARBREDONNES[1,2]}} ;%
				}%
			\ifstrempty{\PLARBREDONNES[2,2]}
				{\draw[PLfleche] (A11)--(A21) ;}
				{%
					\IntPlaceProbasArbre{A11}{A21}{\PLARBREDONNES[2,3]}{\PLARBREDONNES[2,2]}{above=1pt}
					%\draw[PLfleche] (A11)--(A21) node[probas,PLetiquette,{\PLARBREDONNES[2,3]}] {{\PLARBREDONNES[2,2]}} ;
				}%
			\ifstrempty{\PLARBREDONNES[3,2]}
				{\draw[PLfleche] (A11)--(A22) ;}
				{
					\IntPlaceProbasArbre{A11}{A22}{\PLARBREDONNES[3,3]}{\PLARBREDONNES[3,2]}{below=1pt}
					%\draw[PLfleche] (A11)--(A22) node[probas,PLetiquette,{\PLARBREDONNES[3,3]}] {{\PLARBREDONNES[3,2]}} ;%
				}
			\ifstrempty{\PLARBREDONNES[4,2]}
				{\draw[PLfleche] (R)--(A12) ;}
				{
					\IntPlaceProbasArbre{R}{A12}{\PLARBREDONNES[4,3]}{\PLARBREDONNES[4,2]}{above=-1pt}
					%\draw[PLfleche] (R)--(A12) node[probas,PLetiquette,{\PLARBREDONNES[4,3]}] {{\PLARBREDONNES[4,2]}} ;
				}
			\ifstrempty{\PLARBREDONNES[5,2]}
				{\draw[PLfleche] (A12)--(A23) ;}
				{
					\IntPlaceProbasArbre{A12}{A23}{\PLARBREDONNES[5,3]}{\PLARBREDONNES[5,2]}{above=1pt}
					%\draw[PLfleche] (A12)--(A23) node[probas,PLetiquette,{\PLARBREDONNES[5,3]}] {{\PLARBREDONNES[5,2]}} ;
				}
			\ifstrempty{\PLARBREDONNES[6,2]}
				{\draw[PLfleche] (A12)--(A24) ;}
				{
					\IntPlaceProbasArbre{A12}{A24}{\PLARBREDONNES[6,3]}{\PLARBREDONNES[6,2]}{below=1pt}
					%\draw[PLfleche] (A12)--(A24) node[probas,PLetiquette,{\PLARBREDONNES[6,3]}] {{\PLARBREDONNES[6,2]}} ;
				}
			\ifstrempty{\PLARBREDONNES[7,2]}
				{\draw[PLfleche] (R)--(A13) ;}
				{
					\IntPlaceProbasArbre{R}{A13}{\PLARBREDONNES[7,3]}{\PLARBREDONNES[7,2]}{below=1pt}
					%\draw[PLfleche] (R)--(A13) node[probas,PLetiquette,{\PLARBREDONNES[7,3]}] {{\PLARBREDONNES[7,2]}} ;
				}
			\ifstrempty{\PLARBREDONNES[8,2]}
				{\draw[PLfleche] (A13)--(A25) ;}
				{
					\IntPlaceProbasArbre{A13}{A25}{\PLARBREDONNES[8,3]}{\PLARBREDONNES[8,2]}{above=1pt}
					%\draw[PLfleche] (A13)--(A25) node[probas,PLetiquette,{\PLARBREDONNES[8,3]}] {{\PLARBREDONNES[8,2]}} ;
				}
			\ifstrempty{\PLARBREDONNES[9,2]}
				{\draw[PLfleche] (A13)--(A26) ;}
				{
					\IntPlaceProbasArbre{A13}{A26}{\PLARBREDONNES[9,3]}{\PLARBREDONNES[9,2]}{below=1pt}
					%\draw[PLfleche] (A13)--(A26) node[probas,PLetiquette,{\PLARBREDONNES[9,3]}] {{\PLARBREDONNES[9,2]}} ;
				}
		}{}
		%arbre2x3
		\IfEq{\PLAPtype}{2x3}
		{%
			%sommets
			\coordinate (R) at ({0*\PLAPespniv},{-2.5*\PLAPespfeuille}) ;
			\node[noeud] (A11) at ({1*\PLAPespniv},{-1*\PLAPespfeuille}) {\PLARBREDONNES[1,1]};
			\node[noeud] (A12) at ({1*\PLAPespniv},{-4*\PLAPespfeuille}) {\PLARBREDONNES[5,1]};
			\node[noeud] (A21) at ({2*\PLAPespniv},{-0*\PLAPespfeuille}) {\PLARBREDONNES[2,1]};
			\node[noeud] (A22) at ({2*\PLAPespniv},{-1*\PLAPespfeuille}) {\PLARBREDONNES[3,1]};
			\node[noeud] (A23) at ({2*\PLAPespniv},{-2*\PLAPespfeuille}) {\PLARBREDONNES[4,1]};
			\node[noeud] (A24) at ({2*\PLAPespniv},{-3*\PLAPespfeuille}) {\PLARBREDONNES[6,1]};
			\node[noeud] (A25) at ({2*\PLAPespniv},{-4*\PLAPespfeuille}) {\PLARBREDONNES[7,1]};
			\node[noeud] (A26) at ({2*\PLAPespniv},{-5*\PLAPespfeuille}) {\PLARBREDONNES[8,1]};
			%branches
			\ifstrempty{\PLARBREDONNES[1,2]}
				{\draw[PLfleche] (R)--(A11) ;}
				{
					\IntPlaceProbasArbre{R}{A11}{\PLARBREDONNES[1,3]}{\PLARBREDONNES[1,2]}{above=1pt}
					%\draw[PLfleche] (R)--(A11) node[probas,PLetiquette,{\PLARBREDONNES[1,3]}] {{\PLARBREDONNES[1,2]}} ;
				}
			\ifstrempty{\PLARBREDONNES[5,2]}
				{\draw[PLfleche] (R)--(A12) ;}
				{
					\IntPlaceProbasArbre{R}{A12}{\PLARBREDONNES[5,3]}{\PLARBREDONNES[5,2]}{below=1pt}
					%\draw[PLfleche] (R)--(A12) node[probas,PLetiquette,{\PLARBREDONNES[5,3]}] {{\PLARBREDONNES[5,2]}} ;
				}
			\ifstrempty{\PLARBREDONNES[2,2]}
				{\draw[PLfleche] (A11)--(A21) ;}
				{
					\IntPlaceProbasArbre{A11}{A21}{\PLARBREDONNES[2,3]}{\PLARBREDONNES[2,2]}{above=1pt}
					%\draw[PLfleche] (A11)--(A21) node[probas,PLetiquette,{\PLARBREDONNES[2,3]}] {{\PLARBREDONNES[2,2]}} ;
				}
			\ifstrempty{\PLARBREDONNES[3,2]}
				{\draw[PLfleche] (A11)--(A22) ;}
				{
					\IntPlaceProbasArbre{A11}{A22}{\PLARBREDONNES[3,3]}{\PLARBREDONNES[3,2]}{above=-1pt}
					%\draw[PLfleche] (A11)--(A22) node[probas,PLetiquette,{\PLARBREDONNES[3,3]}] {{\PLARBREDONNES[3,2]}} ;
				}
			\ifstrempty{\PLARBREDONNES[4,2]}
				{\draw[PLfleche] (A11)--(A23) ;}
				{
					\IntPlaceProbasArbre{A11}{A23}{\PLARBREDONNES[4,3]}{\PLARBREDONNES[4,2]}{below=1pt}
					%\draw[PLfleche] (A11)--(A23) node[probas,PLetiquette,{\PLARBREDONNES[4,3]}] {{\PLARBREDONNES[4,2]}} ;
				}
			\ifstrempty{\PLARBREDONNES[6,2]}
				{\draw[PLfleche] (A12)--(A24) ;}
				{
					\IntPlaceProbasArbre{A12}{A24}{\PLARBREDONNES[6,3]}{\PLARBREDONNES[6,2]}{above=1pt}
					%\draw[PLfleche] (A12)--(A24) node[probas,PLetiquette,{\PLARBREDONNES[6,3]}] {{\PLARBREDONNES[6,2]}} ;
				}
			\ifstrempty{\PLARBREDONNES[7,2]}
				{\draw[PLfleche] (A12)--(A25) ;}
				{
					\IntPlaceProbasArbre{A12}{A25}{\PLARBREDONNES[7,3]}{\PLARBREDONNES[7,2]}{above=-1pt}
					%\draw[PLfleche] (A12)--(A25) node[probas,PLetiquette,{\PLARBREDONNES[7,3]}] {{\PLARBREDONNES[7,2]}} ;
				}
			\ifstrempty{\PLARBREDONNES[8,2]}
				{\draw[PLfleche] (A12)--(A26) ;}
				{
					\IntPlaceProbasArbre{A12}{A26}{\PLARBREDONNES[8,3]}{\PLARBREDONNES[8,2]}{below=1pt}
					%\draw[PLfleche] (A12)--(A26) node[probas,PLetiquette,{\PLARBREDONNES[8,3]}] {{\PLARBREDONNES[8,2]}} ;
				}
		}{}
		%arbre2x3
		\IfEq{\PLAPtype}{3x3}
		{%
			%sommets
			\coordinate (R) at ({0*\PLAPespniv},{-4*\PLAPespfeuille}) ;
			\node[noeud] (A11) at ({1*\PLAPespniv},{-1*\PLAPespfeuille}) {\PLARBREDONNES[1,1]};
			\node[noeud] (A12) at ({1*\PLAPespniv},{-4*\PLAPespfeuille}) {\PLARBREDONNES[5,1]};
			\node[noeud] (A13) at ({1*\PLAPespniv},{-7*\PLAPespfeuille}) {\PLARBREDONNES[9,1]};
			\node[noeud] (A21) at ({2*\PLAPespniv},{-0*\PLAPespfeuille}) {\PLARBREDONNES[2,1]};
			\node[noeud] (A22) at ({2*\PLAPespniv},{-1*\PLAPespfeuille}) {\PLARBREDONNES[3,1]};
			\node[noeud] (A23) at ({2*\PLAPespniv},{-2*\PLAPespfeuille}) {\PLARBREDONNES[4,1]};
			\node[noeud] (A24) at ({2*\PLAPespniv},{-3*\PLAPespfeuille}) {\PLARBREDONNES[6,1]};
			\node[noeud] (A25) at ({2*\PLAPespniv},{-4*\PLAPespfeuille}) {\PLARBREDONNES[7,1]};
			\node[noeud] (A26) at ({2*\PLAPespniv},{-5*\PLAPespfeuille}) {\PLARBREDONNES[8,1]};
			\node[noeud] (A27) at ({2*\PLAPespniv},{-6*\PLAPespfeuille}) {\PLARBREDONNES[10,1]};
			\node[noeud] (A28) at ({2*\PLAPespniv},{-7*\PLAPespfeuille}) {\PLARBREDONNES[11,1]};
			\node[noeud] (A29) at ({2*\PLAPespniv},{-8*\PLAPespfeuille}) {\PLARBREDONNES[12,1]};
			%branches
			\ifstrempty{\PLARBREDONNES[1,2]}
				{\draw[PLfleche] (R)--(A11) ;}
				{
					\IntPlaceProbasArbre{R}{A11}{\PLARBREDONNES[1,3]}{\PLARBREDONNES[1,2]}{above=1pt}
					%\draw[PLfleche] (R)--(A11) node[probas,PLetiquette,{\PLARBREDONNES[1,3]}] {{\PLARBREDONNES[1,2]}} ;
				}
			\ifstrempty{\PLARBREDONNES[5,2]}
				{\draw[PLfleche] (R)--(A12) ;}
				{
					\IntPlaceProbasArbre{R}{A12}{\PLARBREDONNES[5,3]}{\PLARBREDONNES[5,2]}{above=-1pt}
					%\draw[PLfleche] (R)--(A12) node[probas,PLetiquette,{\PLARBREDONNES[5,3]}] {{\PLARBREDONNES[5,2]}} ;
				}
			\ifstrempty{\PLARBREDONNES[9,2]}
				{\draw[PLfleche] (R)--(A13) ;}
				{
					\IntPlaceProbasArbre{R}{A13}{\PLARBREDONNES[9,3]}{\PLARBREDONNES[9,2]}{below=1pt}
					%\draw[PLfleche] (R)--(A13) node[probas,PLetiquette,{\PLARBREDONNES[9,3]}] {{\PLARBREDONNES[9,2]}} ;
				}
			%N2
			\ifstrempty{\PLARBREDONNES[2,2]}
				{\draw[PLfleche] (A11)--(A21) ;}
				{
					\IntPlaceProbasArbre{A11}{A21}{\PLARBREDONNES[2,3]}{\PLARBREDONNES[2,2]}{above=1pt}
					%\draw[PLfleche] (A11)--(A21) node[probas,PLetiquette,{\PLARBREDONNES[2,3]}] {{\PLARBREDONNES[2,2]}} ;
				}
			\ifstrempty{\PLARBREDONNES[3,2]}
				{\draw[PLfleche] (A11)--(A22) ;}
				{
					\IntPlaceProbasArbre{A11}{A22}{\PLARBREDONNES[3,3]}{\PLARBREDONNES[3,2]}{above=-1pt}
					%\draw[PLfleche] (A11)--(A22) node[probas,PLetiquette,{\PLARBREDONNES[3,3]}] {{\PLARBREDONNES[3,2]}} ;
				}
			\ifstrempty{\PLARBREDONNES[4,2]}
				{\draw[PLfleche] (A11)--(A23) ;}
				{
					\IntPlaceProbasArbre{A11}{A23}{\PLARBREDONNES[4,3]}{\PLARBREDONNES[4,2]}{below=1pt}
					%\draw[PLfleche] (A11)--(A23) node[probas,PLetiquette,{\PLARBREDONNES[4,3]}] {{\PLARBREDONNES[4,2]}} ;
				}
			\ifstrempty{\PLARBREDONNES[6,2]}
				{\draw[PLfleche] (A12)--(A24) ;}
				{
					\IntPlaceProbasArbre{A12}{A24}{\PLARBREDONNES[6,3]}{\PLARBREDONNES[6,2]}{above=1pt}
					%\draw[PLfleche] (A12)--(A24) node[probas,PLetiquette,{\PLARBREDONNES[6,3]}] {{\PLARBREDONNES[6,2]}} ;
				}
			\ifstrempty{\PLARBREDONNES[7,2]}
				{\draw[PLfleche] (A12)--(A25) ;}
				{
					\IntPlaceProbasArbre{A12}{A25}{\PLARBREDONNES[7,3]}{\PLARBREDONNES[7,2]}{above=-1pt}
					%\draw[PLfleche] (A12)--(A25) node[probas,PLetiquette,{\PLARBREDONNES[7,3]}] {{\PLARBREDONNES[7,2]}} ;
				}
			\ifstrempty{\PLARBREDONNES[8,2]}
				{\draw[PLfleche] (A12)--(A26) ;}
				{
					\IntPlaceProbasArbre{A12}{A26}{\PLARBREDONNES[8,3]}{\PLARBREDONNES[8,2]}{below=1pt}
					%\draw[PLfleche] (A12)--(A26) node[probas,PLetiquette,{\PLARBREDONNES[8,3]}] {{\PLARBREDONNES[8,2]}} ;
				}
			\ifstrempty{\PLARBREDONNES[10,2]}
				{\draw[PLfleche] (A13)--(A27) ;}
				{
					\IntPlaceProbasArbre{A13}{A27}{\PLARBREDONNES[10,3]}{\PLARBREDONNES[10,2]}{above=1pt}
					%\draw[PLfleche] (A13)--(A27) node[probas,PLetiquette,{\PLARBREDONNES[10,3]}] {{\PLARBREDONNES[10,2]}} ;
				}
			\ifstrempty{\PLARBREDONNES[11,2]}
				{\draw[PLfleche] (A13)--(A28) ;}
				{
					\IntPlaceProbasArbre{A13}{A28}{\PLARBREDONNES[11,3]}{\PLARBREDONNES[11,2]}{above=-1pt}
					%\draw[PLfleche] (A13)--(A28) node[probas,PLetiquette,{\PLARBREDONNES[11,3]}] {{\PLARBREDONNES[11,2]}} ;
				}
			\ifstrempty{\PLARBREDONNES[12,2]}
				{\draw[PLfleche] (A13)--(A29) ;}
				{
					\IntPlaceProbasArbre{A13}{A29}{\PLARBREDONNES[12,3]}{\PLARBREDONNES[12,2]}{below=1pt}
					%\draw[PLfleche] (A13)--(A29) node[probas,PLetiquette,{\PLARBREDONNES[12,3]}] {{\PLARBREDONNES[12,2]}} ;
				}
		}{}
}
{
	\end{tikzpicture}
}

%%------SCHEMASPROBAS
%densites utiles
\def\densnorm#1{1/(sqrt(2*pi))*exp(-0.5*(#1*#1))}
\def\densexpo#1{exp(-#1)}

\defKV[GRPHPROBCONTINUES]{%
	CouleurAire=\def\GRPHPROBcoulsurf{#1},%
	CouleurCourbe=\def\GRPHPROBcoulcbe{#1},%
	Largeur=\def\GRPHPROBlarg{#1},%
	Hauteur=\def\GRPHPROBhauteur{#1}
}

\setKVdefault[GRPHPROBCONTINUES]{%
	CouleurAire=lightgray,%
	CouleurCourbe=red,%
	Largeur=2,%
	Hauteur=1,%
	AfficheM=true,%
	AfficheCadre=true
}

\NewDocumentCommand\LoiNormaleGraphe{ O{} D<>{} m m m m }{%1=options,2=optionstikz,3=m,4=s,5=a,6=b
	%simplekv
	\useKVdefault[GRPHPROBCONTINUES]%
	\setKV[GRPHPROBCONTINUES]{#1}%
	%unites
	\def\UNITEX{\fpeval{\GRPHPROBlarg/8}}\def\UNITEY{\fpeval{\GRPHPROBhauteur/0.5}}%
	%bornes
	\def\BorneInf{\fpeval{(#5-#3)/#4}}\def\BorneSup{\fpeval{(#6-#3)/#4}}%
	\ifthenelse{\equal{#5}{*}}{\def\BorneInf{-4}}{}%
	\ifthenelse{\equal{#6}{*}}{\def\BorneSup{4}}{}%
	%tikzpicture
	\begin{tikzpicture}[x=\UNITEX cm,y=\UNITEY cm,#2]
		\ifboolKV[GRPHPROBCONTINUES]{AfficheM}%
			{\draw (0,-0.0375) node[font=\tiny,scale={2*\UNITEX}] {#3} ;}
			{}
		\ifboolKV[GRPHPROBCONTINUES]{AfficheCadre}%
			{\draw[gray] (-4,-0.075) rectangle (4,0.425) ;}
			{}
		\clip (-4,-0.075) rectangle (4,0.425) ;
		\draw[black,semithick,fill=\GRPHPROBcoulsurf] ({\BorneInf},0) -- plot [domain={\BorneInf}:{\BorneSup},samples=100] (\x,{\densnorm{\x}}) -- ({\BorneSup},0) -- cycle;
		\draw[thick,->,>=latex] (-4,0)--(4,0) ;
		\draw[thick,densely dotted] (0,0)--(0,0.39894) ;
		\draw[thick,domain=-4:4,samples=100,\GRPHPROBcoulcbe] plot (\x,{\densnorm{\x}}) ;
	\end{tikzpicture}%
}

\NewDocumentCommand\LoiExpoGraphe{ O{} D<>{} m m m }{%1=options,2=optionstikz,3=l,4=a,5=b
	%simplekv
	\useKVdefault[GRPHPROBCONTINUES]%
	\setKV[GRPHPROBCONTINUES]{#1}%
	%unites
	\def\UNITEX{\fpeval{\GRPHPROBlarg/4}}\def\UNITEY{\fpeval{\GRPHPROBhauteur/1}}%
	\def\MOYENNE{\fpeval{trunc(1/#3,0)}}%
	%bornes
	\def\BorneInf{\fpeval{#3*#4}}\def\BorneSup{\fpeval{#3*#5}}%
	\ifthenelse{\equal{#4}{*}}{\def\BorneInf{0}}{}%
	\ifthenelse{\equal{#5}{*}}{\def\BorneSup{4}}{}%
	%tikzpicture
	\begin{tikzpicture}[x=\UNITEX cm,y=\UNITEY cm,#2]
		\ifboolKV[GRPHPROBCONTINUES]{AfficheCadre}%
			{\draw[gray] (-0.2,-0.15) rectangle (3.8,0.85) ;}
			{}
		\clip (-0.2,-0.15) rectangle (3.8,0.85) ;
		\draw[black,semithick,fill=\GRPHPROBcoulsurf] ({\BorneInf},0) -- plot [domain={\BorneInf}:{\BorneSup},samples=100] (\x,{0.7*\densexpo{\x}}) -- ({\BorneSup},0) -- cycle;
		\ifboolKV[GRPHPROBCONTINUES]{AfficheM}%
			{\draw (1,-0.075) node[font=\tiny,scale={\UNITEX}] {\MOYENNE} ;
			\draw[thick,densely dotted] (1,0)--(1,{0.7*\densexpo{1}}) ;}
			{}
		\draw[thick,->,>=latex] (-0.2,0)--(3.8,0) ;
		\draw[thick,->,>=latex] (0,-0.15)--(0,0.85) ;
		\draw[thick,domain=0:3.8,samples=100,\GRPHPROBcoulcbe] plot (\x,{0.7*\densexpo{\x}}) ;
	\end{tikzpicture}%
}

%%------aNp/Cnp
\NewDocumentCommand\CalculAnp{ m m }{%avec et pour xint, p/n
	pfactorial(#2-#1,#2)
}

\NewDocumentCommand\CalculCnp{ m m }{%avec et pour xint, p/n
	binomial(#2,#1)
}

\setKVdefault[anpCnp]{%
	Formule=false,%
	Notation=false,%
	NotationAncien=false
}
\NewDocumentCommand\Arrangement{ s O{} m m }{%p/n
	\useKVdefault[anpCnp]%
	\setKV[anpCnp]{#2}%
	\ifboolKV[anpCnp]{Notation}{A_{#4}^{#3}=}{}%
	\ifboolKV[anpCnp]{Formule}{\frac{#4!}{\inteval{#4-#3}!}=}{}%
	\IfBooleanTF{#1}%
		{\xinteval{\CalculAnp{#3}{#4}}}%non formaté
		{\num{\xinteval{\CalculAnp{#3}{#4}}}}%formaté
}
\NewDocumentCommand\Combinaison{ s O{} m m }{%p/n
	\useKVdefault[anpCnp]%
	\setKV[anpCnp]{#2}%
	\ifboolKV[anpCnp]{Notation}{\binom{#4}{#3} =}{}%
	\ifboolKV[anpCnp]{NotationAncien}{C_{#4}^{#3}=}{}%
	\ifboolKV[anpCnp]{Formule}{\frac{#4!}{#3! \times \inteval{#4-#3}!}=}{}%
	\IfBooleanTF{#1}%
		{\xinteval{\CalculCnp{#3}{#4}}}%non formaté
		{\num{\xinteval{\CalculCnp{#3}{#4}}}}%formaté
}

%%------Fonction de répartition
\NewDocumentCommand\CrochetFctRepart{ m m }{
	\draw[red,\FctRepartThickness] ({#1 + 0.1}, {#2 + 0.025}) -- (#1,#2 + 0.025) -- (#1,#2 - 0.025) -- ({#1 + 0.1}, {#2 - 0.025});
}

\defKV[fctrepart]{%
	Couleur=\def\FctRepartColor{#1},%
	Epaisseur=\def\FctRepartThickness{#1},%
	Extremite=\def\FctRepartExtremite{#1}
}

\setKVdefault[fctrepart]{%
	Couleur=red,%
	Epaisseur=thick,%
	Pointilles=true,%
	Extremite=crochet
}

\NewDocumentCommand\FonctionRepartTikz{ O{} m }{%
	\restoreKV[fctrepart]%
	\setKV[fctrepart]{#1}%
	\setsepchar[.]{/.,}%
	\readlist*\ListeFonctRepart{#2}%
	\def\NbPortions{\ListeFonctRepartlen}
	\foreach \i in {1,2,...,\ListeFonctRepartlen}
		{%
			\IfStrEq{\ListeFonctRepart[\i,2]}{*}%
				{%
					\def\valxdeb{\xmin}%
				}%
				{%
					\itemtomacro\ListeFonctRepart[\i,2]\valxdeb%
					\IfStrEq{\FctRepartExtremite}{crochet}%
						{
							\CrochetFctRepart{\ListeFonctRepart[\i,2]}{\ListeFonctRepart[\i,1]}%
						}{}%
					\IfStrEq{\FctRepartExtremite}{point}%
						{
							\filldraw[\FctRepartColor] ({\valxdeb},{\ListeFonctRepart[\i,1]}) circle[radius=1.5pt] ;%
						}{}%
				}%
			\IfStrEq{\ListeFonctRepart[\i,3]}{*}%
				{%
					\def\valxfin{\xmax}%
				}%
				{%
					\itemtomacro\ListeFonctRepart[\i,3]\valxfin%
					\IfStrEq{\FctRepartExtremite}{crochet}%
						{
							\CrochetFctRepart{\ListeFonctRepart[\i,3]}{\ListeFonctRepart[\i,1]}%
						}{}%
					\IfStrEq{\FctRepartExtremite}{point}%
						{
							\draw[\FctRepartColor,fill=white] ({\valxfin},{\ListeFonctRepart[\i,1]}) circle[radius=1.5pt] ;%
						}{}%
				}%
			\ifboolKV[fctrepart]{Pointilles}%
				{%
					\xintifboolexpr{\valxdeb * \valxfin > 0 'and' \valxdeb > 0}%
						{%
							\draw[semithick,dashed] (0,{\ListeFonctRepart[\i,1]})--({\valxdeb},{\ListeFonctRepart[\i,1]}) ;%
						}{}%
					\xintifboolexpr{\valxdeb * \valxfin > 0 'and' \valxdeb < 0}%
						{%
							\draw[semithick,dashed] (0,{\ListeFonctRepart[\i,1]})--({\valxfin},{\ListeFonctRepart[\i,1]}) ;%
						}{}%
				}{}%
			\draw[\FctRepartThickness,\FctRepartColor] ({\valxdeb},{\ListeFonctRepart[\i,1]})--({\valxfin},{\ListeFonctRepart[\i,1]}) ;%
		}%
}

%===BINOMIALE
\defKV[HistoBinom]{%
	Largeur=\def\GraphBinomLarg{#1},%
	Hauteur=\def\GraphBinomHaut{#1},%
	PasX=\def\GraphBinomPasX{#1},%
	PasY=\def\GraphBinomPasY{#1},%
	Plage=\def\GraphBinomPlage{#1},%
	CouleurPlage=\def\GraphBimomColPlage{#1},%
	Epaisseur=\def\GraphBinomThick{#1},%
	ClipX=\def\GraphBinomXminmax{#1},%
	Police=\def\GraphBinomFonte{#1},%
	CouleurNormale=\def\GraphBinomColNorm{#1}
}
\setKVdefault[HistoBinom]{%
	Largeur=10,%
	Hauteur=5,%
	PasX=5,%
	PasY=0.01,%
	Plage={},%
	CouleurPlage=teal!50,%
	Epaisseur=0.8pt,
	ClipX={},%
	Police=\normalfont\normalsize,%
	AffNormale=false,%
	CouleurNormale=red
}

\NewDocumentCommand\HistogrammeBinomiale{ O{} D<>{} m m }{%
	\restoreKV[HistoBinom]%
	\setKV[HistoBinom]{#1}
	\def\GraphBinomN{#3}%
	\def\GraphBinomP{#4}%
	\IfStrEq{\GraphBinomXminmax}{}%
	{%
		\xdef\GraphBinomXmin{0}%
		\xdef\GraphBinomXmax{\GraphBinomN}%
	}%
	{%
		\StrCut{\GraphBinomXminmax}{-}{\GraphBinomXmin}{\GraphBinomXmax}%
		\IfStrEq{\GraphBinomXmin}{*}{\xdef\GraphBinomXmin{0}}{}%
		\IfStrEq{\GraphBinomXmax}{*}{\xdef\GraphBinomXmax{\GraphBinomN}}{}%
	}%
	\IfStrEq{\GraphBinomPlage}{}%
	{}%
	{%
		\StrCut{\GraphBinomPlage}{-}{\GraphBinomColorMin}{\GraphBinomColorMax}%
		\IfStrEq{\GraphBinomColorMin}{*}{\xdef\GraphBinomColorMin{\GraphBinomXmin}}{}%
		\IfStrEq{\GraphBinomColorMax}{*}{\xdef\GraphBinomColorMax{\GraphBinomXmax}}{}%
	}%
	%test d'unités
	\xdef\GraphBinomXunit{\xintfloateval{round((\GraphBinomLarg)/(\GraphBinomXmax-\GraphBinomXmin+1),3)}}%
	\xdef\grphbinommedA{\xintfloateval{trunc(\GraphBinomN*\GraphBinomP,0)}}%
	\xdef\grphbinommedB{\xintfloateval{\grphbinommedA+1}}%
	\xdef\GraphBinomYmaxA{\xintfloateval{binomial(\GraphBinomN,\grphbinommedA)*(\GraphBinomP)^(\grphbinommedA)*(1-(\GraphBinomP))^((\GraphBinomN)-(\grphbinommedA))}}%
	\xdef\GraphBinomYmaxB{\xintfloateval{binomial(\GraphBinomN,\grphbinommedB)*(\GraphBinomP)^(\grphbinommedB)*(1-(\GraphBinomP))^((\GraphBinomN)-(\grphbinommedB))}}%
	\xdef\GraphBinomMaxY{\xintfloateval{1.1*max(\GraphBinomYmaxA,\GraphBinomYmaxB)}}%
	\xdef\GraphBinomNbPrecision{\xinteval{abs(ilog10(\GraphBinomMaxY))+1}}%
	\xdef\GraphBinomYunit{\xintfloateval{round((\GraphBinomHaut)/(max(\GraphBinomYmaxA,\GraphBinomYmaxB)),3)}}%
	\begin{tikzpicture}[x=\GraphBinomXunit cm,y=\GraphBinomYunit cm,#2]
		%coloriage éventuel
		\IfStrEq{\GraphBinomPlage}{}%
		{}%
		{%
			\xintFor* ##1 in {\xintSeq{\GraphBinomColorMin}{\GraphBinomColorMax}}\do{%
				\xdef\tmpYYY{\xintfloateval{binomial(\GraphBinomN,##1)*(\GraphBinomP)^(##1)*(1-(\GraphBinomP))^((\GraphBinomN)-(##1))}}%
				\draw[draw=none,fill=\GraphBimomColPlage,fill opacity=0.5] ({##1-0.5},0) rectangle++ (1,{\tmpYYY}) ;
			}%
		}%
		%axes
		\draw[line width=\GraphBinomThick,->,>=latex] ({\GraphBinomXmin-0.5},0)--({\GraphBinomXmax+1},0) ;
		\draw[line width=\GraphBinomThick,->,>=latex] ({\GraphBinomXmin-0.5},0)--({\GraphBinomXmin-0.5},{1.1*(\GraphBinomHaut)/(\GraphBinomYunit)}) ;
		\foreach \x in {\GraphBinomXmin,\inteval{\GraphBinomXmin+\GraphBinomPasX},...,\GraphBinomXmax}{%
			\draw[line width=\GraphBinomThick] (\x,2pt)--++(0,-4pt) node[below,font=\GraphBinomFonte] {\num{\x}} ;
		}%
		\xdef\GraphBinomNbPrecisionAxeY{\xinteval{abs(ilog10(\GraphBinomPasY))+1}}%
		\foreach \y in {0,\GraphBinomPasY,...,\GraphBinomMaxY}{%
			\draw[line width=\GraphBinomThick] ($({\GraphBinomXmin-0.5},\y)+(2pt,0)$)--++(-4pt,0) node[left,font=\GraphBinomFonte] {\num{\xintfloateval{round(\y,\GraphBinomNbPrecisionAxeY)}}} ;
		}%
		%tracé
		\xintFor* ##1 in {\xintSeq{\GraphBinomXmin}{\GraphBinomXmax}}\do{%
			\xdef\tmpYYY{\xintfloateval{binomial(\GraphBinomN,##1)*(\GraphBinomP)^(##1)*(1-(\GraphBinomP))^((\GraphBinomN)-(##1))}}%
			\draw[line width=\GraphBinomThick] ({##1-0.5},0) rectangle++ (1,{\tmpYYY}) ;
		}
		\ifboolKV[HistoBinom]{AffNormale}%
			{%
				\xdef\MinNormHistoBinom{\xintfloateval{{\GraphBinomXmin-0.5}}}%
				\xdef\MaxNormHistoBinom{\xintfloateval{{\GraphBinomXmax+0.5}}}%
				\TraceLoiNormale*[line width={1.25*\GraphBinomThick},\GraphBinomColNorm]<\MinNormHistoBinom..[0.1]..\MaxNormHistoBinom>{#3}{#4}
			}%
			{}%
	\end{tikzpicture}
}

\NewDocumentEnvironment{HistoBinomiale}{ O{} D<>{} m m }%
{%
	\restoreKV[HistoBinom]%
	\setKV[HistoBinom]{#1}
	\def\GraphBinomN{#3}%
	\def\GraphBinomP{#4}%
	\IfStrEq{\GraphBinomXminmax}{}%
	{%
		\xdef\GraphBinomXmin{0}%
		\xdef\GraphBinomXmax{\GraphBinomN}%
	}%
	{%
		\StrCut{\GraphBinomXminmax}{-}{\GraphBinomXmin}{\GraphBinomXmax}%
		\IfStrEq{\GraphBinomXmin}{*}{\xdef\GraphBinomXmin{0}}{}%
		\IfStrEq{\GraphBinomXmax}{*}{\xdef\GraphBinomXmax{\GraphBinomN}}{}%
	}%
	\IfStrEq{\GraphBinomPlage}{}%
	{}%
	{%
		\StrCut{\GraphBinomPlage}{-}{\GraphBinomColorMin}{\GraphBinomColorMax}%
		\IfStrEq{\GraphBinomColorMin}{*}{\xdef\GraphBinomColorMin{\GraphBinomXmin}}{}%
		\IfStrEq{\GraphBinomColorMax}{*}{\xdef\GraphBinomColorMax{\GraphBinomXmax}}{}%
	}%
	%test d'unités
	\xdef\GraphBinomXunit{\xintfloateval{round((\GraphBinomLarg)/(\GraphBinomXmax-\GraphBinomXmin+1),3)}}%
	\xdef\grphbinommedA{\xintfloateval{trunc(\GraphBinomN*\GraphBinomP,0)}}%
	\xdef\grphbinommedB{\xintfloateval{\grphbinommedA+1}}%
	\xdef\GraphBinomYmaxA{\xintfloateval{binomial(\GraphBinomN,\grphbinommedA)*(\GraphBinomP)^(\grphbinommedA)*(1-(\GraphBinomP))^((\GraphBinomN)-(\grphbinommedA))}}%
	\xdef\GraphBinomYmaxB{\xintfloateval{binomial(\GraphBinomN,\grphbinommedB)*(\GraphBinomP)^(\grphbinommedB)*(1-(\GraphBinomP))^((\GraphBinomN)-(\grphbinommedB))}}%
	\xdef\GraphBinomMaxY{\xintfloateval{1.1*max(\GraphBinomYmaxA,\GraphBinomYmaxB)}}%
	\xdef\GraphBinomNbPrecision{\xinteval{abs(ilog10(\GraphBinomMaxY))+1}}%
	\xdef\GraphBinomYunit{\xintfloateval{round((\GraphBinomHaut)/(max(\GraphBinomYmaxA,\GraphBinomYmaxB)),3)}}%
	\begin{tikzpicture}[x=\GraphBinomXunit cm,y=\GraphBinomYunit cm,#2]
		%coloriage éventuel
		\IfStrEq{\GraphBinomPlage}{}%
		{}%
		{%
			\xintFor* ##1 in {\xintSeq{\GraphBinomColorMin}{\GraphBinomColorMax}}\do{%
				\xdef\tmpYYY{\xintfloateval{binomial(\GraphBinomN,##1)*(\GraphBinomP)^(##1)*(1-(\GraphBinomP))^((\GraphBinomN)-(##1))}}%
				\draw[draw=none,fill=\GraphBimomColPlage,fill opacity=0.5] ({##1-0.5},0) rectangle++ (1,{\tmpYYY}) ;
			}%
		}%
		%axes
		\draw[line width=\GraphBinomThick,->,>=latex] ({\GraphBinomXmin-0.5},0)--({\GraphBinomXmax+1},0) ;
		\draw[line width=\GraphBinomThick,->,>=latex] ({\GraphBinomXmin-0.5},0)--({\GraphBinomXmin-0.5},{1.1*(\GraphBinomHaut)/(\GraphBinomYunit)}) ;
		\foreach \x in {\GraphBinomXmin,\inteval{\GraphBinomXmin+\GraphBinomPasX},...,\GraphBinomXmax}{%
			\draw[line width=\GraphBinomThick] (\x,2pt)--++(0,-4pt) node[below,font=\GraphBinomFonte] {\num{\x}} ;
		}%
		\xdef\GraphBinomNbPrecisionAxeY{\xinteval{abs(ilog10(\GraphBinomPasY))+1}}%
		\foreach \y in {0,\GraphBinomPasY,...,\GraphBinomMaxY}{%
			\draw[line width=\GraphBinomThick] ($({\GraphBinomXmin-0.5},\y)+(2pt,0)$)--++(-4pt,0) node[left,font=\GraphBinomFonte] {\num{\xintfloateval{round(\y,\GraphBinomNbPrecisionAxeY)}}} ;
		}%
		%tracé
		\xintFor* ##1 in {\xintSeq{\GraphBinomXmin}{\GraphBinomXmax}}\do{%
			\xdef\tmpYYY{\xintfloateval{binomial(\GraphBinomN,##1)*(\GraphBinomP)^(##1)*(1-(\GraphBinomP))^((\GraphBinomN)-(##1))}}%
			\draw[line width=\GraphBinomThick] ({##1-0.5},0) rectangle++ (1,{\tmpYYY}) ;
		}
		%loi normale ?
		\ifboolKV[HistoBinom]{AffNormale}%
			{%
				\xdef\MinNormHistoBinom{\xintfloateval{{\GraphBinomXmin-0.5}}}%
				\xdef\MaxNormHistoBinom{\xintfloateval{{\GraphBinomXmax+0.5}}}%
				\TraceLoiNormale*[line width={1.25*\GraphBinomThick},\GraphBinomColNorm]<\MinNormHistoBinom..[0.1]..\MaxNormHistoBinom>{#3}{#4}
			}%
			{}%
	}%
	{%
	\end{tikzpicture}
}

\xintdeffloatfunc fctreploinorm(x,mu,sigma):=1/(sigma*sqrt(2*Pi))*exp(-1/2*((x-mu)/sigma)^2);

\NewDocumentCommand\TraceLoiNormale{ s O{} D<>{} m m }{%
	\IfBooleanTF{#1}%
	{%
		\draw[#2] plot[smooth] coordinates {%
			\xintthecoords\xintfloatexpr
			seq((x,fctreploinorm(x,#4*#5,sqrt(#4*#5*(1-#5)))),x=#3)
			\relax
		};
	}%
	{%
		\draw[#2] plot[smooth] coordinates {%
			\xintthecoords\xintfloatexpr
			seq((x,fctreploinorm(x,#4,#5)),x=#3)
			\relax
		};
	}%
}

%====ARBRES CHOIX
\RequirePackage{commalists-tools}

\tikzset{arbrechoixaretes/.style={semithick}}
\tikzset{arbrechoixsommets/.style={circle,draw=none,inner sep=1pt}}
\tikzset{arbrechoixresultats/.style={rectangle,draw,inner sep=1.5pt}}
\tikzset{arbrechoixnotices/.style={inner sep=1pt,font=\small\ttfamily,midway,below=\fpeval{0.75*\LISTECHOIXinterfeuille}cm,}}
\tikzset{arbrechoixdeconotices/.style={semithick,decoration={brace,mirror,raise=\fpeval{0.25*\LISTECHOIXechelle*\LISTECHOIXinterfeuille}cm,amplitude=\fpeval{0.5*\LISTECHOIXechelle*\LISTECHOIXinterfeuille}cm},decorate}}

%https://tex.stackexchange.com/questions/73173/how-to-change-the-item-separator-in-tikz-foreach-command/174135#174135
%Mark Wibrow := CC BY-SA 3.0
\makeatletter
\let\pgffor@dotsscanend@original=\pgffor@dotsscanend
\def\PGFFOR@PARAMDELIM#1{%
	\def\pgffor@emptyvalues{#1 \pgffor@stop#1}%
	\def\pgffor@normal@list##1{%
		\def\pgffor@values{##1#1 \pgffor@stop#1}%
		\ifx\pgffor@values\pgffor@emptyvalues
		\def\pgffor@values{\pgffor@stop#1}%
		\fi%
		\let\pgffor@body=\pgfutil@empty%
		\global\pgffor@continuetrue%
		\pgffor@collectbody}%
	\def\pgffor@scanround(##1)##2#1{\def\pgffor@value{(##1)##2}\pgffor@scanned}%
	\def\pgffor@scanone##1#1{\def\pgffor@value{##1}\pgffor@scanned}%
	\def\pgffor@dotsscanend##1#1{\pgffor@dotsscanend@original##1,}%
}

\pgfkeys{/pgf/foreach/separator/.code=\PGFFOR@PARAMDELIM{#1}}
\makeatother

\defKV[arbrechoix]{%
	EspaceNiveaux=\def\LISTECHOIXinterniv{#1},%
	EspaceFeuilles=\def\LISTECHOIXinterfeuille{#1},%
	Echelle=\def\LISTECHOIXechelle{#1},%
	Repet=\def\LISTECHOIXrepet{#1},%
	Notice=\def\LISTECHOIXnotice{#1},%
	CouleursNiveaux=\def\LISTECHOIXcoulniv{#1},%
	SepResultats=\def\LISTECHOIXsepres{#1}
}

\setKVdefault[arbrechoix]{%
	EspaceNiveaux=2.25,%
	EspaceFeuilles=0.5,%
	Echelle=1,%
	Repet={},%
	Notice={},%
	TraitsNotice=false,%
	CouleursNiveaux=black,%
	AffResultats=false,%
	SepResultats={}
}

\NewDocumentCommand\ArbreChoix{ O{} D<>{} m }{%
	\restoreKV[arbrechoix]%
	\setKV[arbrechoix]{#1}%
	\def\LISTECHOIX{#3}%
	\IfEq{\LISTECHOIXrepet}{}%
		{}%
		{%
			\foreach \i in {2,...,\LISTECHOIXrepet}{%
				\xdef\LISTECHOIX{\LISTECHOIX,#3}%
			}%
		}%
	%données utiles
	\setsepchar[.]{,./}%
	\readlist*\TRLISTECHOIX{\LISTECHOIX}%
	%les couleurs...
	\IfSubStr{\LISTECHOIXcoulniv}{,}%
		{%
			\setsepchar{,}%
			\readlist*\TRLISTECHOIXCOUL{\LISTECHOIXcoulniv}%
		}%
		{%
			\xdef\tmplistcouleurs{\LISTECHOIXcoulniv}%
			\foreach \i in {2,...,\TRLISTECHOIXlen}{%
				\xdef\tmplistcouleurs{\tmplistcouleurs,\LISTECHOIXcoulniv}%
			}%
			\setsepchar{,}%
			\readlist*\TRLISTECHOIXCOUL{\tmplistcouleurs}%
		}%
	%la figure
	\begin{tikzpicture}[scale=\LISTECHOIXechelle,transform shape,#2]
		%création de la liste des étapes intermédiaires
		\def\tmpnbchoix{1}
		\def\tmpnbchoixniv{1}
		\def\tmplistnbchoix{1}
		\foreach \i in {1,...,\TRLISTECHOIXlen}{%
			\itemtomacro\TRLISTECHOIX[\i]\tmptrlistchoix%
			\setsepchar[.]{/}%
			\readlist*\TRLISTECHOIXTMP{\tmptrlistchoix}%
			\xdef\tmpnbchoix{\xinteval{\tmpnbchoix*\TRLISTECHOIXTMPlen}}%
			\xdef\tmpnbchoixniv{\tmpnbchoixniv,\TRLISTECHOIXTMPlen}%nb choix par niveau
			\xdef\tmplistnbchoix{\tmplistnbchoix,\tmpnbchoix}%nb choix petit à petit
		}%
		%données utiles tout le temps
		\getvaluefromlist{\tmplistnbchoix}{-1}[\nbeltstotal]%
		%découpage horizontal
		\ifboolKV[arbrechoix]{TraitsNotice}%
			{%
				\foreach \i in {1,...,\TRLISTECHOIXlen}{%
					\draw[semithick,lightgray!50,densely dashed] ({\i*\LISTECHOIXinterniv},0) --++ (0,{\nbeltstotal*\LISTECHOIXinterfeuille}) ;
				}%
			}%
			{}%
		%boucle décroissante (OK !!)
		\foreach \TTT in {\TRLISTECHOIXlen,...,1}{%
			%\xdef\invTTT{\xinteval{\TRLISTECHOIXlen-\TTT+1}}%
			\getvaluefromlist{\tmplistnbchoix}{\numexpr\TTT+1\relax}[\nbelts]%
			\xdef\espchoiceV{\xintfloateval{(\nbeltstotal)/\nbelts}}%
			\xdef\offsetchoiceV{\xintfloateval{0.5*\LISTECHOIXinterfeuille*(\nbeltstotal-(\nbelts-1)*\espchoiceV)}}%
			\getvaluefromlist{\tmplistnbchoix}{\TTT}[\nbeltsavant]%
			\xdef\tmpmodchoix{\xintieval{\nbelts/\nbeltsavant}}%
			\foreach \i in {1,...,\nbelts}{%
				\xdef\tmpii{\xinteval{\xintiiRem{\xinteval{\i-1}}{\tmpmodchoix}+1}}%
				\node[arbrechoixsommets,text={\TRLISTECHOIXCOUL[\TTT]}] (N-\TTT-\i) at ({(\TTT)*\LISTECHOIXinterniv},{\offsetchoiceV+(\nbelts-\i)*\LISTECHOIXinterfeuille*\espchoiceV}) {\TRLISTECHOIX[\TTT,\tmpii]} ;
			}
			\ifnum\TTT=\TRLISTECHOIXlen%
			\coordinate (N-0-1) at ({0*\LISTECHOIXinterniv},{\offsetchoiceV+0.5*(\nbeltstotal-1)*\LISTECHOIXinterfeuille} );
			\fi
		}
		%les flèches
		\foreach \j in {1,...,\TRLISTECHOIXlen}{%
			\getvaluefromlist{\tmplistnbchoix}{\j}[\nbnoeuds]%nb nœuds de départ
			\getvaluefromlist{\tmpnbchoixniv}{\numexpr\j+1\relax}[\nbfleches]%nb flèche par nœud
			\foreach \i in {1,...,\nbnoeuds}{%
				\foreach \k in {1,...,\nbfleches}{%
					\xdef\tmpk{\xinteval{(\i-1)*(\nbfleches)+\k}}
					\draw[arbrechoixaretes,color={\TRLISTECHOIXCOUL[\j]}] (N-\the\numexpr\j-1\relax-\i) -- (N-\j-\tmpk) ;
				}
			}
		}
		%notice
		\IfEq{\LISTECHOIXnotice}{}%
			{}%
			{%
				\foreach \notice [count=\i] in \LISTECHOIXnotice
				{%
					\draw[arbrechoixdeconotices] ({(\i-1)*\LISTECHOIXinterniv},0) -- ({(\i)*\LISTECHOIXinterniv},0) node[arbrechoixnotices] {\notice} ;
				}%
			}%
		%listerésultats
		\ifboolKV[arbrechoix]{AffResultats}%
			{%
				\xdef\malistedecombis{}%
				\ifnum\TRLISTECHOIXlen=2%
					\itemtomacro\TRLISTECHOIX[1]\myfirsttmp%
					\itemtomacro\TRLISTECHOIX[2]\mysecondtmp%
					\foreach \i [separator=/] in \myfirsttmp {%
						\foreach \j [separator=/] in \mysecondtmp {
							\xdef\malistedecombis{\malistedecombis,\i\LISTECHOIXsepres\j}%
						}%
					}%
				\fi
				\ifnum\TRLISTECHOIXlen=3%
					\itemtomacro\TRLISTECHOIX[1]\myfirsttmp%
					\itemtomacro\TRLISTECHOIX[2]\mysecondtmp%
					\itemtomacro\TRLISTECHOIX[3]\mythirdtmp%
					\foreach \i [separator=/] in \myfirsttmp {%
						\foreach \j [separator=/] in \mysecondtmp {
							\foreach \k [separator=/] in \mythirdtmp {
								\xdef\malistedecombis{\malistedecombis,\i\LISTECHOIXsepres\j\LISTECHOIXsepres\k}%
							}%
						}%
					}%
				\fi
				\ifnum\TRLISTECHOIXlen=4%
					\itemtomacro\TRLISTECHOIX[1]\myfirsttmp%
					\itemtomacro\TRLISTECHOIX[2]\mysecondtmp%
					\itemtomacro\TRLISTECHOIX[3]\mythirdtmp%
					\itemtomacro\TRLISTECHOIX[4]\myfourthtmp%
					\foreach \i [separator=/] in \myfirsttmp {%
						\foreach \j [separator=/] in \mysecondtmp {
							\foreach \k [separator=/] in \mythirdtmp {
								\foreach \l [separator=/] in \myfourthtmp {
									\xdef\malistedecombis{\malistedecombis,\i\LISTECHOIXsepres\j\LISTECHOIXsepres\k\LISTECHOIXsepres\l}%
								}%
							}%
						}%
					}%
				\fi
				\ifnum\TRLISTECHOIXlen=5%
					\itemtomacro\TRLISTECHOIX[1]\myfirsttmp%
					\itemtomacro\TRLISTECHOIX[2]\mysecondtmp%
					\itemtomacro\TRLISTECHOIX[3]\mythirdtmp%
					\itemtomacro\TRLISTECHOIX[4]\myfourthtmp%
					\itemtomacro\TRLISTECHOIX[5]\myfifthtmp%
					\foreach \i [separator=/] in \myfirsttmp {%
						\foreach \j [separator=/] in \mysecondtmp {
							\foreach \k [separator=/] in \mythirdtmp {
								\foreach \l [separator=/] in \myfourthtmp {
									\foreach \m [separator=/] in \myfifthtmp {
										\xdef\malistedecombis{\malistedecombis,\i\LISTECHOIXsepres\j\LISTECHOIXsepres\k\LISTECHOIXsepres\l\LISTECHOIXsepres\m}%
									}%
								}%
							}%
						}%
					}%
				\fi
				\foreach \i in {1,...,\nbeltstotal}{%
					\xdef\j{\xinteval{\i+1}}%
					\draw (N-\TRLISTECHOIXlen-\i) node[arbrechoixresultats,right=\fpeval{0.5*\LISTECHOIXechelle*\LISTECHOIXinterniv}cm] {\getvaluefromlist*\malistedecombis{\j}} ;
				}
			}%
			{}%
	\end{tikzpicture}%
}

\NewDocumentCommand\IntNoeudArbreSansRemise{ m m m }{%
	\node[arbrechoixsommets,text={\TRLISTECHOIXCOUL[#1]}] (T-#1-#2) at (N-#1-#2) {\TRLISTECHOIX[#3]} ;
}
\NewDocumentCommand\IntAreteArbreSansRemise{ m m }{%
	\draw[arbrechoixaretes] (T-#1) -- (T-#2) ;
}

\NewDocumentCommand\ArbreChoixSansRemise{ O{} D<>{} m }{%
	\restoreKV[arbrechoix]%
	\setKV[arbrechoix]{#1}%
	\def\LISTECHOIX{#3}%
	%couleurs
	\readlist*\TRLISTECHOIX{\LISTECHOIX}%
	\IfSubStr{\LISTECHOIXcoulniv}{,}%
		{%
			\setsepchar{,}%
			\readlist*\TRLISTECHOIXCOUL{\LISTECHOIXcoulniv}%
		}%
		{%
			\xdef\tmplistcouleurs{\LISTECHOIXcoulniv}%
			\foreach \i in {2,...,\TRLISTECHOIXlen}{%
				\xdef\tmplistcouleurs{\tmplistcouleurs,\LISTECHOIXcoulniv}%
			}%
			\setsepchar{,}%
			\readlist*\TRLISTECHOIXCOUL{\tmplistcouleurs}%
		}%
	%figure
	\begin{tikzpicture}[scale=\LISTECHOIXechelle,transform shape,#2]
		%création de la liste des étapes intermédiaires (OK)
		\def\tmpnbchoixniv{\TRLISTECHOIXlen}%
		\def\tmplistnbchoix{1,\TRLISTECHOIXlen}%
		\def\tmplistnbchoixfact{\TRLISTECHOIXlen}%
		\foreach \i in {1,...,\xinteval{\TRLISTECHOIXlen-1}}{%
			\xdef\tmpnbchoix{\xinteval{\TRLISTECHOIXlen-\i}}%
			\xdef\tmpnbchoixniv{\tmpnbchoixniv,\tmpnbchoix}%nb nœuds par niveau
			\xdef\tmplistnbchoixfact{\xinteval{\tmplistnbchoixfact*(\TRLISTECHOIXlen-\i)}}%
			\xdef\tmplistnbchoix{\tmplistnbchoix,\tmplistnbchoixfact}%nb choix petit à petit
		}
		%données utiles tout le temps
		\getvaluefromlist{\tmplistnbchoix}{-1}[\nbeltstotal]%
		\ifboolKV[arbrechoix]{TraitsNotice}%
		%notice découpage
			{%
				\foreach \i in {1,...,\TRLISTECHOIXlen}{%
					\draw[semithick,lightgray!50,densely dashed] ({\i*\LISTECHOIXinterniv},0) --++ (0,{\nbeltstotal*\LISTECHOIXinterfeuille}) ;
				}%
			}%
			{}%
		%boucle décroissante pour la création des nœuds
		\foreach \TTT in {\TRLISTECHOIXlen,...,1}{%
			\getvaluefromlist{\tmplistnbchoix}{\numexpr\TTT+1\relax}[\nbelts]%
			\xdef\espchoiceV{\xintfloateval{(\nbeltstotal)/\nbelts}}%
			\xdef\offsetchoiceV{\xintfloateval{0.5*\LISTECHOIXinterfeuille*(\nbeltstotal-(\nbelts-1)*\espchoiceV)}}%
			\foreach \i in {1,...,\nbelts}{%
				\coordinate (N-\TTT-\i) at ({(\TTT)*\LISTECHOIXinterniv},{\offsetchoiceV+(\nbelts-\i)*\LISTECHOIXinterfeuille*\espchoiceV});
			}
			\ifnum\TTT=\TRLISTECHOIXlen%
			\coordinate (T-0-1) at ({0*\LISTECHOIXinterniv},{\offsetchoiceV+0.5*(\nbeltstotal-1)*\LISTECHOIXinterfeuille} );
			\fi
		}
		%les sommets et les flèches ??
		\ifnum\TRLISTECHOIXlen=4%
			%N1
			\IntNoeudArbreSansRemise{1}{1}{1}
			\IntNoeudArbreSansRemise{1}{2}{2}
			\IntNoeudArbreSansRemise{1}{3}{3}
			\IntNoeudArbreSansRemise{1}{4}{4}
			\IntAreteArbreSansRemise{0-1}{1-1}
			\IntAreteArbreSansRemise{0-1}{1-2}
			\IntAreteArbreSansRemise{0-1}{1-3}
			\IntAreteArbreSansRemise{0-1}{1-4}
			%N2
			\IntNoeudArbreSansRemise{2}{1}{2}
			\IntNoeudArbreSansRemise{2}{2}{3}
			\IntNoeudArbreSansRemise{2}{3}{4}
			\IntNoeudArbreSansRemise{2}{4}{1}
			\IntNoeudArbreSansRemise{2}{5}{3}
			\IntNoeudArbreSansRemise{2}{6}{4}
			\IntNoeudArbreSansRemise{2}{7}{1}
			\IntNoeudArbreSansRemise{2}{8}{2}
			\IntNoeudArbreSansRemise{2}{9}{4}
			\IntNoeudArbreSansRemise{2}{10}{1}
			\IntNoeudArbreSansRemise{2}{11}{2}
			\IntNoeudArbreSansRemise{2}{12}{3}
			\IntAreteArbreSansRemise{1-1}{2-1}
			\IntAreteArbreSansRemise{1-1}{2-2}
			\IntAreteArbreSansRemise{1-1}{2-3}
			\IntAreteArbreSansRemise{1-2}{2-4}
			\IntAreteArbreSansRemise{1-2}{2-5}
			\IntAreteArbreSansRemise{1-2}{2-6}
			\IntAreteArbreSansRemise{1-3}{2-7}
			\IntAreteArbreSansRemise{1-3}{2-8}
			\IntAreteArbreSansRemise{1-3}{2-9}
			\IntAreteArbreSansRemise{1-4}{2-10}
			\IntAreteArbreSansRemise{1-4}{2-11}
			\IntAreteArbreSansRemise{1-4}{2-12}
			%N3
			\IntNoeudArbreSansRemise{3}{1}{3}
			\IntNoeudArbreSansRemise{3}{2}{4}
			\IntNoeudArbreSansRemise{3}{3}{2}
			\IntNoeudArbreSansRemise{3}{4}{4}
			\IntNoeudArbreSansRemise{3}{5}{2}
			\IntNoeudArbreSansRemise{3}{6}{3}
			\IntNoeudArbreSansRemise{3}{7}{3}
			\IntNoeudArbreSansRemise{3}{8}{4}
			\IntNoeudArbreSansRemise{3}{9}{1}
			\IntNoeudArbreSansRemise{3}{10}{4}
			\IntNoeudArbreSansRemise{3}{11}{1}
			\IntNoeudArbreSansRemise{3}{12}{3}
			\IntNoeudArbreSansRemise{3}{13}{2}
			\IntNoeudArbreSansRemise{3}{14}{4}
			\IntNoeudArbreSansRemise{3}{15}{1}
			\IntNoeudArbreSansRemise{3}{16}{4}
			\IntNoeudArbreSansRemise{3}{17}{1}
			\IntNoeudArbreSansRemise{3}{18}{2}
			\IntNoeudArbreSansRemise{3}{19}{2}
			\IntNoeudArbreSansRemise{3}{20}{3}
			\IntNoeudArbreSansRemise{3}{21}{1}
			\IntNoeudArbreSansRemise{3}{22}{3}
			\IntNoeudArbreSansRemise{3}{23}{1}
			\IntNoeudArbreSansRemise{3}{24}{2}
			\IntAreteArbreSansRemise{2-1}{3-1}
			\IntAreteArbreSansRemise{2-1}{3-2}
			\IntAreteArbreSansRemise{2-2}{3-3}
			\IntAreteArbreSansRemise{2-2}{3-4}
			\IntAreteArbreSansRemise{2-3}{3-5}
			\IntAreteArbreSansRemise{2-3}{3-6}
			\IntAreteArbreSansRemise{2-4}{3-7}
			\IntAreteArbreSansRemise{2-4}{3-8}
			\IntAreteArbreSansRemise{2-5}{3-9}
			\IntAreteArbreSansRemise{2-5}{3-10}
			\IntAreteArbreSansRemise{2-6}{3-11}
			\IntAreteArbreSansRemise{2-6}{3-12}
			\IntAreteArbreSansRemise{2-7}{3-13}
			\IntAreteArbreSansRemise{2-7}{3-14}
			\IntAreteArbreSansRemise{2-8}{3-15}
			\IntAreteArbreSansRemise{2-8}{3-16}
			\IntAreteArbreSansRemise{2-9}{3-17}
			\IntAreteArbreSansRemise{2-9}{3-18}
			\IntAreteArbreSansRemise{2-10}{3-19}
			\IntAreteArbreSansRemise{2-10}{3-20}
			\IntAreteArbreSansRemise{2-11}{3-21}
			\IntAreteArbreSansRemise{2-11}{3-22}
			\IntAreteArbreSansRemise{2-12}{3-23}
			\IntAreteArbreSansRemise{2-12}{3-24}
			%N4
			\IntNoeudArbreSansRemise{4}{1}{4}
			\IntNoeudArbreSansRemise{4}{2}{3}
			\IntNoeudArbreSansRemise{4}{3}{4}
			\IntNoeudArbreSansRemise{4}{4}{2}
			\IntNoeudArbreSansRemise{4}{5}{3}
			\IntNoeudArbreSansRemise{4}{6}{2}
			\IntNoeudArbreSansRemise{4}{7}{4}
			\IntNoeudArbreSansRemise{4}{8}{3}
			\IntNoeudArbreSansRemise{4}{9}{4}
			\IntNoeudArbreSansRemise{4}{10}{1}
			\IntNoeudArbreSansRemise{4}{11}{3}
			\IntNoeudArbreSansRemise{4}{12}{1}
			\IntNoeudArbreSansRemise{4}{13}{4}
			\IntNoeudArbreSansRemise{4}{14}{2}
			\IntNoeudArbreSansRemise{4}{15}{4}
			\IntNoeudArbreSansRemise{4}{16}{1}
			\IntNoeudArbreSansRemise{4}{17}{2}
			\IntNoeudArbreSansRemise{4}{18}{1}
			\IntNoeudArbreSansRemise{4}{19}{3}
			\IntNoeudArbreSansRemise{4}{20}{2}
			\IntNoeudArbreSansRemise{4}{21}{3}
			\IntNoeudArbreSansRemise{4}{22}{1}
			\IntNoeudArbreSansRemise{4}{23}{2}
			\IntNoeudArbreSansRemise{4}{24}{1}
			\IntAreteArbreSansRemise{3-1}{4-1}
			\IntAreteArbreSansRemise{3-2}{4-2}
			\IntAreteArbreSansRemise{3-3}{4-3}
			\IntAreteArbreSansRemise{3-4}{4-4}
			\IntAreteArbreSansRemise{3-5}{4-5}
			\IntAreteArbreSansRemise{3-6}{4-6}
			\IntAreteArbreSansRemise{3-7}{4-7}
			\IntAreteArbreSansRemise{3-8}{4-8}
			\IntAreteArbreSansRemise{3-9}{4-9}
			\IntAreteArbreSansRemise{3-10}{4-10}
			\IntAreteArbreSansRemise{3-11}{4-11}
			\IntAreteArbreSansRemise{3-12}{4-12}
			\IntAreteArbreSansRemise{3-13}{4-13}
			\IntAreteArbreSansRemise{3-14}{4-14}
			\IntAreteArbreSansRemise{3-15}{4-15}
			\IntAreteArbreSansRemise{3-16}{4-16}
			\IntAreteArbreSansRemise{3-17}{4-17}
			\IntAreteArbreSansRemise{3-18}{4-18}
			\IntAreteArbreSansRemise{3-19}{4-19}
			\IntAreteArbreSansRemise{3-20}{4-20}
			\IntAreteArbreSansRemise{3-21}{4-21}
			\IntAreteArbreSansRemise{3-22}{4-22}
			\IntAreteArbreSansRemise{3-23}{4-23}
			\IntAreteArbreSansRemise{3-24}{4-24}
		\fi
		\ifnum\TRLISTECHOIXlen=3%
			%N1
			\IntNoeudArbreSansRemise{1}{1}{1}
			\IntNoeudArbreSansRemise{1}{2}{2}
			\IntNoeudArbreSansRemise{1}{3}{3}
			\IntAreteArbreSansRemise{0-1}{1-1}
			\IntAreteArbreSansRemise{0-1}{1-2}
			\IntAreteArbreSansRemise{0-1}{1-3}
			%N2
			\IntNoeudArbreSansRemise{2}{1}{2}
			\IntNoeudArbreSansRemise{2}{2}{3}
			\IntNoeudArbreSansRemise{2}{3}{1}
			\IntNoeudArbreSansRemise{2}{4}{3}
			\IntNoeudArbreSansRemise{2}{5}{1}
			\IntNoeudArbreSansRemise{2}{6}{2}
			\IntAreteArbreSansRemise{1-1}{2-1}
			\IntAreteArbreSansRemise{1-1}{2-2}
			\IntAreteArbreSansRemise{1-2}{2-3}
			\IntAreteArbreSansRemise{1-2}{2-4}
			\IntAreteArbreSansRemise{1-3}{2-5}
			\IntAreteArbreSansRemise{1-3}{2-6}
			%N3
			\IntNoeudArbreSansRemise{3}{1}{3}
			\IntNoeudArbreSansRemise{3}{2}{2}
			\IntNoeudArbreSansRemise{3}{3}{3}
			\IntNoeudArbreSansRemise{3}{4}{1}
			\IntNoeudArbreSansRemise{3}{5}{2}
			\IntNoeudArbreSansRemise{3}{6}{1}
			\IntAreteArbreSansRemise{2-1}{3-1}
			\IntAreteArbreSansRemise{2-2}{3-2}
			\IntAreteArbreSansRemise{2-3}{3-3}
			\IntAreteArbreSansRemise{2-4}{3-4}
			\IntAreteArbreSansRemise{2-5}{3-5}
			\IntAreteArbreSansRemise{2-6}{3-6}
		\fi
		\ifnum\TRLISTECHOIXlen=2%
			%N1
			\IntNoeudArbreSansRemise{1}{1}{1}
			\IntNoeudArbreSansRemise{1}{2}{2}
			\IntAreteArbreSansRemise{0-1}{1-1}
			\IntAreteArbreSansRemise{0-1}{1-2}
			%N2
			\IntNoeudArbreSansRemise{2}{1}{2}
			\IntNoeudArbreSansRemise{2}{2}{1}
			\IntAreteArbreSansRemise{1-1}{2-1}
			\IntAreteArbreSansRemise{1-2}{2-2}
		\fi
		%notice
		\IfEq{\LISTECHOIXnotice}{}%
			{}%
			{%
				\foreach \notice [count=\i] in \LISTECHOIXnotice
				{%
					\draw[arbrechoixdeconotices] ({(\i-1)*\LISTECHOIXinterniv},0) -- ({(\i)*\LISTECHOIXinterniv},0) node[arbrechoixnotices] {\notice} ;
				}
			}%
		%listerésultats
		\ifboolKV[arbrechoix]{AffResultats}%
			{%
				\GeneratePermutations[\LISTECHOIXsepres]{#3}[\mylistofpermut]%on génère les permutations !
				\foreach \i in {1,...,\nbeltstotal}{%
					\draw (N-\TRLISTECHOIXlen-\i) node[arbrechoixresultats,right=\fpeval{0.5*\LISTECHOIXechelle*\LISTECHOIXinterniv}cm] {\getvaluefromlist*\mylistofpermut{\i}} ;
				}
			}%
			{}%
	\end{tikzpicture}%
}

\def\tmplistpermuttwo{{1,2},{2,1}}
\def\tmplistpermutthree{{1,2,3},{1,3,2},{2,1,3},{2,3,1},{3,1,2},{3,2,1}}
\def\tmplistpermutfour{{1,2,3,4},{1,2,4,3},{1,3,2,4},{1,3,4,2},{1,4,2,3},{1,4,3,2},{2,1,3,4},{2,1,4,3},{2,3,1,4},{2,3,4,1},{2,4,1,3},{2,4,3,1},{3,1,2,4},{3,1,4,2},{3,2,1,4},{3,2,4,1},{3,4,1,2},{3,4,2,1},{4,1,2,3},{4,1,3,2},{4,2,1,3},{4,2,3,1},{4,3,1,2},{4,3,2,1}}

\NewDocumentCommand\GeneratePermutations{ O{-} m O{\mylistofpermut} }{%
	\lenoflist{#2}[\resmylen]%
	\ifnum\resmylen=2%
		\xdef\tmplistindex{\tmplistpermuttwo}%
	\fi%
	\ifnum\resmylen=3%
		\xdef\tmplistindex{\tmplistpermutthree}%
	\fi%
	\ifnum\resmylen=4%
		\xdef\tmplistindex{\tmplistpermutfour}%
	\fi%
	\lenoflist{\tmplistindex}[\resmylenindex]%
	\xdef\tmpstringperm{}%
	\xdef#3{}%
	\foreach \i in {1,...,\resmylenindex}{%
		\getvaluefromlist{\tmplistindex}{\i}[\mytmplst]%
		\lenoflist{\mytmplst}[\resmylentmptmp]%
		\foreach \j [count=\iii] in \mytmplst {%
			\ifnum\iii=1%
				\xdef\tmpstringperm{\getvaluefromlist*{#2}{\j}}%
			\else%
				\xdef\tmpstringperm{\tmpstringperm#1\getvaluefromlist*{#2}{\j}}%
			\fi%
		}%
		\ifnum\i=1%
			\xdef#3{\tmpstringperm}%
		\else%
			\xdef#3{#3,\tmpstringperm}%
		\fi%
	}%
}

\endinput