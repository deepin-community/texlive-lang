% proflycee-tools-recreat.tex
% Copyright 2023  Cédric Pierquet
% Released under the LaTeX Project Public License v1.3c or later, see http://www.latex-project.org/lppl.txt

%%------FENÊTRE CALCUL FORMEL
\newcommand\CFchap{\textasciicircum}
\newcounter{CFnum}
%def des clés
\defKV[paramfenxcas]{%
	Largeur=\def\CFlarg{#1},%
	EspaceLg=\def\CFesplg{#1},%
	PremCol=\def\CFpremcol{#1},%
	HautPremCol=\def\CFhpremcol{#1},%
	Taille=\def\CFtaille{#1},%
	Couleur=\def\CFcouleur{#1},%
	TailleTitre=\def\CFtailletitre{#1},%
	CouleurCmd=\def\CFcoulcmd{#1},%
	CouleurRes=\def\CFcoulres{#1},%
	PosCmd=\def\CFposcmd{#1},%
	PosRes=\def\CFposres{#1},%
	LabelTitre=\def\CFlabeltitre{#1}%
}
\setKVdefault[paramfenxcas]{%
	Largeur=16,EspaceLg=2pt,PremCol=0.3,HautPremCol=0.4,%
	Couleur=darkgray,Menu=true,Titre=false,TailleTitre=\normalsize,Taille=\normalsize,%
	Sep=true,PosRes=centre,PosCmd=gauche,%
	CouleurCmd=red,CouleurRes=blue,%
	LabelTitre={Résultats obtenus avec un logiciel de Calcul Formel}%
}

\newcommand\CalculFormelParametres[1][]{%
	\setcounter{CFnum}{0}
	\useKVdefault[paramfenxcas]%
	\setKV[paramfenxcas]{#1}% on paramètres les nouvelles clés et on les simplifie
}

\defKV[paramlgxcas]{%
	HautCmd=\def\CFhle{#1},%
	HautRes=\def\CFhlr{#1}
}
\setKVdefault[paramlgxcas]{%
	HautCmd=0.75,%
	HautRes=0.75
}

\newcommand\CalculFormelLigne[3][]{%
	\addtocounter{CFnum}{1}
	\def\CFL{\theCFnum}%
	\def\CFLA{\inteval{\CFL-1}}%
	\useKVdefault[paramlgxcas]%
	\setKV[paramlgxcas]{#1}% on paramètres les nouvelles clés et on les simplifie
	\def\CFLA{\inteval{\CFL-1}}%
	%DÉCLARATION DES NŒUDS (les "6" coins des lignes commande et résultat)
	\xintifboolexpr{\CFL == 1}%si c'est la première ligne
		{\coordinate (A0\CFL) at (0,0);}
		{\coordinate (A0\CFL) at ($(A2\CFLA) + (0,{-\CFesplg})$);}
	\coordinate (A1\CFL) at ($(A0\CFL) + (0,{-\CFhle})$);
	\coordinate (A2\CFL) at ($(A1\CFL) + (0,{-\CFhlr})$);
	\coordinate (A3\CFL) at ($(A0\CFL) + ({\CFlarg},0)$);
	\coordinate (A4\CFL) at ($(A1\CFL) + ({\CFlarg},0)$);
	\coordinate (A5\CFL) at ($(A2\CFL) + ({\CFlarg},0)$);
	%DÉCLARATION DES NŒUDS INTERMÉDIAIRES (pour les commandes et les résultats)
	\coordinate (C1\CFL) at ($(A0\CFL) + (0,{-0.5*\CFhle})$);
	\coordinate (C2\CFL) at ($(A0\CFL) + ({0.5*\CFlarg},{-0.5*\CFhle})$);
	\coordinate (C3\CFL) at ($(A0\CFL) + ({\CFlarg},{-0.5*\CFhle})$);
	\coordinate (R1\CFL) at ($(A1\CFL) + (0,{-0.5*\CFhlr})$);
	\coordinate (R2\CFL) at ($(A1\CFL) + ({0.5*\CFlarg},{-0.5*\CFhlr})$);
	\coordinate (R3\CFL) at ($(A1\CFL) + ({\CFlarg},{-0.5*\CFhlr})$);
	%RECTANGLE DE BASE
	\draw[\CFcouleur] (A0\CFL) rectangle (A5\CFL) ;
	%LA COMMANDE EN ROUGE
	\IfStrEq{\CFposcmd}{centre}%si poscmd=center
		{\draw (C2\CFL) node[\CFcoulcmd,font=\CFtaille] {#2} ;}
		{}
	\IfStrEq{\CFposcmd}{gauche}%si poscmd=left
		{\draw (C1\CFL) node[right,\CFcoulcmd,font=\CFtaille] {#2} ;}
		{}
	\IfStrEq{\CFposcmd}{right}%si poscmd=right
		{\draw (C3\CFL) node[left,\CFcoulcmd,font=\CFtaille] {#2} ;}
		{}
	%LE RÉSULTAT
	\IfStrEq{\CFposres}{centre}%si posrep=center
		{\draw (R2\CFL) node[\CFcoulres,font=\CFtaille] {#3} ;}
		{}
	\IfStrEq{\CFposres}{gauche}%si posrep=left
		{\draw (R1\CFL) node[right,\CFcoulres,font=\CFtaille] {#3} ;}
		{}
	\IfStrEq{\CFposres}{right}%si posrep=right
		{\draw (R3\CFL) node[left,\CFcoulres,font=\CFtaille] {#3} ;}
		{}
	\ifboolKV[paramfenxcas]{Sep}%si sep=true
		{\draw[\CFcouleur] (A1\CFL) -- (A4\CFL);}%
		{}
	%LE PETIT NUMÉRO
	\draw[\CFcouleur] (A0\CFL) rectangle ++ ({-\CFpremcol},{-\CFhpremcol}) node[\CFcouleur,midway,font=\small\sffamily\bfseries] {\CFL} ;
	%LE RECTANGLE "MENU"
	\ifboolKV[paramfenxcas]{Menu}%si menu=true
		{\draw[\CFcouleur,fill=\CFcouleur!25] (A5\CFL) rectangle ++ (-0.65,0.25) node[black,midway,font=\tiny\sffamily\bfseries] {MENU} ;}%
		{}
	%LE BLOC "TITRE"
	\ifboolKV[paramfenxcas]{Titre}%si titre=true
		{\draw[\CFcouleur,fill=lightgray!25,rounded corners] ($(A01) + (0,2pt)$) rectangle ++ ($({\CFlarg},2em)$) node[CouleurVertForet!50!\CFcouleur,midway,font=\CFtailletitre\ttfamily\bfseries] {\CFlabeltitre};}
		{}
}

%%------CartoucheCapytale
\definecolor{vertcapyt}{rgb}{0.0,0.5,0.0}
%\definecolor{vertcapyt}{HTML}{008000}
\DeclareTotalTCBox{\CartoucheCapytale}{ s O{} m }
	{enhanced,size=fbox,on line,arc=3pt,colback=vertcapyt,colframe=vertcapyt,fontupper=\IfBooleanTF{#1}{\ttfamily}{\sffamily}\bfseries,colupper=white}%
	{#3#2~{\scriptsize\faLink}}

%%------SUDOMATHS
\defKV[PLTIKZSUDOM]{%
	CouleurTexte=\def\PLSMcoultexte{#1},%
	Epaisseur=\def\PLSMepf{#1},%
	Epaisseurg=\def\PLSMepg{#1},%
	Unite=\def\PLSMunite{#1},%
	CouleurCase=\def\PLSMcoulcase{#1},%
	NbCol=\def\PLSMnbcol{#1},%
	NbSubCol=\def\PLSMnbsubcol{#1},%
	NbLig=\def\PLSMnblig{#1},%
	NbSubLig=\def\PLSMnbsublig{#1},%
	Police=\def\PLSMfonte{#1},%
	PoliceLeg=\def\PLSMfonteleg{#1},%
	ListeLegV=\def\PLSMlistelegv{#1},%
	ListeLegH=\def\PLSMlistelegh{#1},%
	DecalLegende=\def\PLSMdecalleg{#1}
}

\setKVdefault[PLTIKZSUDOM]{%
	Epaisseurg=1.5pt,%
	Epaisseur=0.5pt,%
	Unite=1cm,%
	CouleurCase=cyan!25,%
	CouleurTexte=blue,%
	NbCol=9,%
	NbSubCol=3,%
	NbLig=9,%
	NbSubLig=3,
	Police=\normalfont\normalsize,%
	PoliceLeg=\normalfont\sffamily,%
	Legendes=true,%
	ListeLegV=ABCDEFGHIJKLMNOPQRSTUVWXYZ,%
	ListeLegH=abcdefghijklmnopqrstuvwxyz,%
	DecalLegende=0.45
}

\NewDocumentEnvironment{EnvSudoMaths}{ O{} m }
	{
	\useKVdefault[PLTIKZSUDOM]
	\setKV[PLTIKZSUDOM]{#1}% on paramètres les nouvelles clés et on les simplifie
	%calculs intermédiaires
	\def\larcolinter{\inteval{\PLSMnbcol/\PLSMnbsubcol}}
	\def\larliginter{\inteval{\PLSMnblig/\PLSMnbsublig}}
	%lecture liste
	\IfEq{#2}{}{}%
		{%
			\setsepchar[.]{§./}%
			\readlist*\SPGrilleSudoMaths{#2}%
		}
	%débt envtik
	\begin{tikzpicture}[x=\PLSMunite,y=\PLSMunite,line join=miter]
		%cases
		\IfEq{#2}{}{}%
		{%
		\foreach \i in {1,2,...,\PLSMnblig}{%
			\foreach \j in {1,2,...,\PLSMnbcol}{%
				\itemtomacro\SPGrilleSudoMaths[\i,\j]\SMcase
				\IfSubStr{\SMcase}{*}%si on veut colorier via *
				{%
					\StrDel{\SMcase}{*}[\SMcaseb]%
					\draw[draw=none,fill=\PLSMcoulcase] ({\j-1},{1-\i}) rectangle++ (1,-1) node[inner sep=0pt,outer sep=0pt,\PLSMcoultexte,font=\PLSMfonte,midway] {\SMcaseb} ;%
				}
				{%
					\draw ({\j-0.5},{0.5-\i}) node[inner sep=0pt,outer sep=0pt,\PLSMcoultexte,font=\PLSMfonte] {\SMcase} ;%
				}
			}
		}%
		}
		%grilles
		\draw[line width=\PLSMepg] (0,0) rectangle ({\PLSMnbcol},{-\PLSMnblig}) ;
		\draw[line width=\PLSMepf,xstep=1,ystep=1] (0,0) grid ({\PLSMnbcol},{-\PLSMnblig}) ;
		\draw[line width=\PLSMepg,xstep=\larcolinter,ystep=\larliginter] (0,0) grid ({\PLSMnbcol},{-\PLSMnblig}) ;
		%légendes
		\ifboolKV[PLTIKZSUDOM]{Legendes}
			{%
				\foreach \i in {1,2,...,\PLSMnbcol}{\draw ({\i-0.5},{\PLSMdecalleg}) node[inner sep=0pt,outer sep=0pt,font=\PLSMfonteleg] {\strut\StrChar{\PLSMlistelegh}{\i}} ;}
				\foreach \j in {1,2,...,\PLSMnblig}{\draw ({-\PLSMdecalleg},{0.5-\j}) node[inner sep=0pt,outer sep=0pt,font=\PLSMfonteleg] {\StrChar{\PLSMlistelegv}{\j}} ;}
			}{}
	}
	{
	\end{tikzpicture}
	}

\NewDocumentCommand\SudoMaths{ O{} m }{%
	\useKVdefault[PLTIKZSUDOM]
	\setKV[PLTIKZSUDOM]{#1}% on paramètres les nouvelles clés et on les simplifie
	%calculs intermédiaires
	\def\larcolinter{\inteval{\PLSMnbcol/\PLSMnbsubcol}}
	\def\larliginter{\inteval{\PLSMnblig/\PLSMnbsublig}}
	%lecture liste
	\IfEq{#2}{}{}%
	{%
		\setsepchar[.]{§./}%
		\readlist*\SPGrilleSudoMaths{#2}%
	}
	%débt envtik
	\begin{tikzpicture}[x=\PLSMunite,y=\PLSMunite,line join=miter]
		%cases
		\IfEq{#2}{}{}%
		{%
			\foreach \i in {1,2,...,\PLSMnblig}{%
				\foreach \j in {1,2,...,\PLSMnbcol}{%
					\itemtomacro\SPGrilleSudoMaths[\i,\j]\SMcase
					\IfSubStr{\SMcase}{*}%si on veut colorier via *
					{%
						\StrDel{\SMcase}{*}[\SMcaseb]%
						\draw[draw=none,fill=\PLSMcoulcase] ({\j-1},{1-\i}) rectangle++ (1,-1) node[inner sep=0pt,outer sep=0pt,\PLSMcoultexte,font=\PLSMfonte,midway] {\SMcaseb} ;%
					}
					{%
						\draw ({\j-0.5},{0.5-\i}) node[inner sep=0pt,outer sep=0pt,\PLSMcoultexte,font=\PLSMfonte] {\SMcase} ;%
					}
				}
			}%
		}
		%grilles
		\draw[line width=\PLSMepg] (0,0) rectangle ({\PLSMnbcol},{-\PLSMnblig}) ;
		\draw[line width=\PLSMepf,xstep=1,ystep=1] (0,0) grid ({\PLSMnbcol},{-\PLSMnblig}) ;
		\draw[line width=\PLSMepg,xstep=\larcolinter,ystep=\larliginter] (0,0) grid ({\PLSMnbcol},{-\PLSMnblig}) ;
		%légendes
		\ifboolKV[PLTIKZSUDOM]{Legendes}
		{%
			\foreach \i in {1,2,...,\PLSMnbcol}{\draw ({\i-0.5},{\PLSMdecalleg}) node[inner sep=0pt,outer sep=0pt,font=\PLSMfonteleg] {\strut\StrChar{\PLSMlistelegh}{\i}} ;}
			\foreach \j in {1,2,...,\PLSMnblig}{\draw ({-\PLSMdecalleg},{0.5-\j}) node[inner sep=0pt,outer sep=0pt,font=\PLSMfonteleg] {\StrChar{\PLSMlistelegv}{\j}} ;}
		}{}
	\end{tikzpicture}
}

%=====FRACTALES
\usetikzlibrary{lindenmayersystems}

\pgfdeclarelindenmayersystem{FloconKoch}{
	\rule{F -> F-F++F-F}}
\pgfdeclarelindenmayersystem{TriangleSierpinski}{
	\rule{F -> G-F-G}
	\rule{G -> F+G+F}}
	
\pgfdeclarelindenmayersystem{SierpinskiTriangle}{
    \symbol{X}{\pgflsystemdrawforward}
    \symbol{Y}{\pgflsystemdrawforward}
    \rule{X -> X-Y+X+Y-X}
    \rule{Y -> YY}
}%

\defKV[tikzfract]{%
	Epaisseur=\def\fracttikzthick{#1},%
	Type=\def\fracttikztype{#1},%
	Couleur=\def\fracttikzcolor{#1},%
	LongueurCote=\def\fracttikzlg{#1},%
	Etape=\def\fracttikzstep{#1},%
	Remplissage=\def\fracttikzfill{#1},%
	Depart=\def\fracttikzdepart{#1},%
	Offset=\def\fracttikzoffset{#1}
}

\setKVdefault[tikzfract]{%
	Epaisseur=0.6pt,%
	Type=Koch,%
	Couleur=black,%
	LongueurCote=3,%
	Etape=1,%
	Remplir=false,%
	Remplissage=lightgray,%
	Depart={(0,0)},%
	AlignV=false,%
	Offset=2pt
}

\NewDocumentCommand\FractaleTikz{ s O{} D<>{} }{%
	\restoreKV[tikzfract]%
	\setKV[tikzfract]{#2}%
	\def\fracttikzlgstep{\fpeval{(\fracttikzlg)/(3^\fracttikzstep)}}%
	\IfStrEq{\fracttikztype}{Sierp}%
		{%
			\def\fracttikzlgstep{\fpeval{(\fracttikzlg)/(2^\fracttikzstep)}}%
		}{}%
	\IfBooleanF{#1}{\begin{tikzpicture}[#3]}
	\ifboolKV[tikzfract]{Remplir}%
		{%
			\IfStrEq{\fracttikztype}{Koch}%
				{%
					\ifboolKV[tikzfract]{AlignV}%
						{%
							\node at ($(1/3*\fracttikzlg,0)+(-60:1/3*\fracttikzlg)$) {} ;
						}{}%
					\draw[line width =\fracttikzthick,shift=\fracttikzdepart,draw=\fracttikzcolor,fill=\fracttikzfill,l-system={FloconKoch,step=\fracttikzlgstep cm,angle=60,axiom=F++F++F,order=\fracttikzstep}] lindenmayer system -- cycle;
				}{}%
			\IfStrEq{\fracttikztype}{Sierp}%
				{%
					\fill[\fracttikzcolor] (0,0) -- ++(0:\fracttikzlg cm) -- ++(120:\fracttikzlg cm) -- cycle;
					\draw[draw=none,shift=\fracttikzdepart,fill=white,l-system={SierpinskiTriangle,step=\fracttikzlgstep cm,angle=-120,axiom=X,order=\fracttikzstep}] lindenmayer system -- cycle;
				}{}%
		}%
		{%
			\IfStrEq{\fracttikztype}{Koch}%
				{%
					\ifboolKV[tikzfract]{AlignV}%
						{%
							\node at ($(1/3*\fracttikzlg,0)+(-60:1/3*\fracttikzlg)$) {} ;
						}{}%
					\draw [line width=\fracttikzthick,shift=\fracttikzdepart,\fracttikzcolor,l-system={FloconKoch,step=\fracttikzlgstep cm,angle=60,axiom=F++F++F,order=\fracttikzstep}] lindenmayer system -- cycle;
				}{}%
			\IfStrEq{\fracttikztype}{Sierp}%
				{%
					\fill[\fracttikzcolor] (0,0) -- ++(0:\fracttikzlg cm) -- ++(120:\fracttikzlg cm) -- cycle;
					\draw[draw=none,shift=\fracttikzdepart,fill=white,l-system={SierpinskiTriangle,step=\fracttikzlgstep cm,angle=-120,axiom=X,order=\fracttikzstep}] lindenmayer system -- cycle;
				}{}%
		}%
	\IfBooleanF{#1}{\end{tikzpicture}}%
}

\NewDocumentCommand\EtapesFloconKoch{ O{} D<>{} m }{%
	\restoreKV[tikzfract]%
	\setKV[tikzfract]{#1}%
	\foreach \i in {#3} {%
		\FractaleTikz[AlignV,#1,Etape=\i]<#2>\hspace{\fracttikzoffset}%
	}
}

\NewDocumentCommand\EtapesTapisSierpinski{ O{} D<>{} m }{%
	\restoreKV[tikzfract]%
	\setKV[tikzfract]{#1}%
	\foreach \i in {#3} {%
		\FractaleTikz[Type=Sierp,#1,Etape=\i]<#2>\hspace{\fracttikzoffset}%
	}
}

%====CHATEAUCARTES
\usetikzlibrary{patterns.meta}
%hauteurs utiles
\xdef\HoCardsHgt{1}
\xdef\HoCardsWdt{0.45}

%styles
\tikzstyle{CarteDroite}=[line width=\fpeval{\HoCardsScale*0.01}cm,rounded corners=\fpeval{\HoCardsScale*\HoCardsRound}cm,DecoDroite]
\tikzstyle{CarteGauche}=[line width=\fpeval{\HoCardsScale*0.01}cm,rounded corners=\fpeval{\HoCardsScale*\HoCardsRound}cm,DecoGauche]
\tikzstyle{CarteHoriz}=[line width=\fpeval{\HoCardsScale*0.01}cm,rounded corners=\fpeval{\HoCardsScale*\HoCardsRound}cm,DecoHoriz]

%clés
\defKV[houseofcards]{Echelle=\def\HoCardsScale{#1},CouleurDeco=\def\HoCardsColor{#1},AngleY=\def\HoCardsAglY{#1},AngleX=\def\HoCardsAglX{#1},PoliceLegende=\def\HoCardsFonteLeg{#1},Deco=\def\HoCardsDeco{#1}}
\setKVdefault[houseofcards]{Echelle=1,CouleurDeco=black,Arrondi=true,,AngleY=20,AngleX=8,Bas=false,Legende=false,PoliceLegende=\normalsize\normalfont,Deco=remplir}

\newcommand\HoCardsLeft[1]{%
	\draw[CarteGauche] (#1) --++ ({0.5*\HoCardsHgt},0,{-sqrt(3)*\HoCardsHgt*0.5}) --++ ({0},{\HoCardsWdt},{0}) --++ ({-0.5*\HoCardsHgt},0,{sqrt(3)*\HoCardsHgt*0.5}) --cycle ;
}

\newcommand\HoCardsRight[1]{%
	\draw[CarteDroite] (#1) --++ ({-0.5*\HoCardsHgt},0,{-sqrt(3)*\HoCardsHgt*0.5}) --++ ({0},{\HoCardsWdt},{0}) --++ ({0.5*\HoCardsHgt},0,{sqrt(3)*\HoCardsHgt*0.5}) --cycle ;
}

\newcommand\HoCardsHoriz[1]{%
	\draw[CarteHoriz] (#1) --++ ({\HoCardsHgt},0,0) --++ ({0},{\HoCardsWdt},{0}) --++ ({-\HoCardsHgt},0,0) --cycle ;
}

\NewDocumentCommand\ChateauCartes{ O{} m D<>{} }{%
	\useKVdefault[houseofcards]%
	\setKV[houseofcards]{#1}%
	\def\HoCardsNbLevel{#2}%
	\def\HoCardsRound{0.025}%
	\ifboolKV[houseofcards]{Arrondi}{\def\HoCardsRound{0}}{}%
	%style remplir, si clé non reconnue
	\tikzstyle{DecoDroite}=[fill=\HoCardsColor!20]
	\tikzstyle{DecoGauche}=[fill=\HoCardsColor!10]
	\tikzstyle{DecoHoriz}=[fill=\HoCardsColor!15]
	\IfStrEq{\HoCardsDeco}{vide}
		{%
			\tikzstyle{DecoDroite}=[fill=white]
			\tikzstyle{DecoGauche}=[fill=white]
			\tikzstyle{DecoHoriz}=[fill=white]
		}%
		{}%
	\IfStrEq{\HoCardsDeco}{hachures}
		{%
			\tikzstyle{DecoDroite}=[preaction={fill=white},pattern={Lines[angle={45-2.5*\HoCardsAglY},distance=\fpeval{\HoCardsScale*2}pt]},pattern color=\HoCardsColor]
			\tikzstyle{DecoGauche}=[preaction={fill=white},pattern={Lines[angle={45-\HoCardsAglY},distance=\fpeval{\HoCardsScale*1.25}pt]},pattern color=\HoCardsColor]
			\tikzstyle{DecoHoriz}=[preaction={fill=white},pattern={Lines[angle={45+\HoCardsAglY},distance=\fpeval{\HoCardsScale*2}pt]},pattern color=\HoCardsColor]
		}%
		{}%
	\begin{tikzpicture}[x={({-180+\HoCardsAglX}:0.75cm)},y={({180-\HoCardsAglY}:1cm)},z={(90:1cm)},line join=bevel,scale=\HoCardsScale,#3]
		%nœuds
		\coordinate (A0-0) at (0,0,0) ;
		\xintifboolexpr{\HoCardsNbLevel > 1}%
			{
				\foreach \i in {1,...,\HoCardsNbLevel}{%
					\def\j{\inteval{\i-1}}%
					\coordinate (A\i-0) at ($(A\j-0)+({-0.5*\HoCardsHgt},0,{-sqrt(3)*\HoCardsHgt*0.5})$) ;
				}
				\foreach \i in {1,...,\HoCardsNbLevel}{%
					\foreach \j in {1,...,\i}{%
						\def\k{\inteval{\j-1}}%
						\coordinate (A\i-\j) at ($(A\i-\k)+({\HoCardsHgt},0,0)$) ;
					}
				}
				%construction des étages, du bas vers le haut
				\ifboolKV[houseofcards]{Bas}%
					{%
						\foreach \i in {0,...,\inteval{\HoCardsNbLevel-1}}{\HoCardsHoriz{A\HoCardsNbLevel-\i}}%
					}{}%
				\foreach \etage in {\inteval{\HoCardsNbLevel-1},...,1}{%
					\foreach \i in {0,...,\etage}{\HoCardsRight{A\etage-\i}\HoCardsLeft{A\etage-\i}}
					\foreach \i in {0,...,\inteval{\etage-1}}{\HoCardsHoriz{A\etage-\i}}
				}
			}%
			{%
				\coordinate (A1-0) at ($(A0-0)+({-0.5*\HoCardsHgt},0,{-sqrt(3)*\HoCardsHgt*0.5})$) ;
				\coordinate (A1-1) at ($(A1-0)+({\HoCardsHgt},0,0)$) ;
				\ifboolKV[houseofcards]{Bas}%
					{%
						\HoCardsHoriz{A1-0}%
					}{}%
			}%
		%étage du dessus
		\HoCardsRight{A0-0}\HoCardsLeft{A0-0}
		%légende éventuelle
		\ifboolKV[houseofcards]{Legende}%
			{%
				\draw ($(A\HoCardsNbLevel-0)!0.5!(A\HoCardsNbLevel-\HoCardsNbLevel)$) node[below,font=\HoCardsFonteLeg] {$n = \HoCardsNbLevel$} ;
			}{}%
	\end{tikzpicture}
}

%====ALLUMETTES
\definecolor{BoisAllumette}{HTML}{E9D0B8}
\definecolor{GratteAllumette}{HTML}{D32A0F}
\xdef\LongueurGratte{0.28cm}
\xdef\HauteurGratte{0.20cm}

\newcommand{\CalcLg}[2]{%
	\pgfpointdiff{\pgfpointanchor{#1}{center}}{\pgfpointanchor{#2}{center}}%
	\pgf@xa=\pgf@x%
	\pgf@ya=\pgf@y%
	\pgfmathparse{veclen(\pgf@xa,\pgf@ya)}%
}

\defKV[allumettes]{CouleurBois=\def\MatchWoodColor{#1},CouleurBout=\def\MatchEndColor{#1},Decal=\def\MatchOffset{#1}}
\setKVdefault[allumettes]{CouleurBois=BoisAllumette,CouleurBout=GratteAllumette,Decal={0.8*\LongueurGratte},NoirBlanc=false}

\NewDocumentCommand\PfLAllumette{ O{} m }{%1 offset,%2 = ptA>ptB
	\useKVdefault[allumettes]%
	\setKV[allumettes]{#1}%
	\ifboolKV[allumettes]{NoirBlanc}%
		{%
			\def\MatchWoodColor{lightgray}\def\MatchEndColor{darkgray}%
		}%
		{}%
	\StrCut{#2}{>}{\AlumPtDep}{\AlumPtArriv}%
	\pgfmathanglebetweenpoints{\pgfpointanchor{\AlumPtDep}{center}}{\pgfpointanchor{\AlumPtArriv}{center}}\edef\AlumAngle{\pgfmathresult}%
	\CalcLg{\AlumPtDep}{\AlumPtArriv}\edef\AlumLg{\pgfmathresult}%
	\begin{scope}[shift={($(\AlumPtDep)+(\AlumAngle:{\MatchOffset})$)},rotate=\AlumAngle]
		\fill[\MatchWoodColor] (0,-0.0975cm) rectangle++ ({\AlumLg-2*\LongueurGratte-2*\MatchOffset},0.2cm);
		\fill[\MatchWoodColor!50!black] (0,-0.0975cm) --++ ({\AlumLg-2*\LongueurGratte-2*\MatchOffset},0) --++ (0.05cm,-0.05cm) --++ ({-\AlumLg+2*\LongueurGratte+2*\MatchOffset},0) --++ (-0.05cm,0.05cm);
		\draw[line join=bevel,line cap=rect] (0,-0.0975cm) -- ++(0,0.2cm) -- ++({\AlumLg-2*\LongueurGratte-2*\MatchOffset},0) -- ++(0,-0.2cm) -- ++(0.05cm,-0.05cm) -- ++({-\AlumLg+2*\LongueurGratte+2*\MatchOffset},0) -- ++(-0.05cm,0.05cm) --cycle ;
		\shade[draw,ball color=\MatchEndColor,rounded corners=0.1pt] ({\AlumLg-2*\LongueurGratte-2*\MatchOffset},0)--++(0,{0.1cm}) to[out=12.5,in=90]++({2*\LongueurGratte},{-0.1cm}) to[out=-90,in=-17.5]++({-2*\LongueurGratte+0.05cm},{-0.15cm}) --++ (-0.05cm,0.05cm) --cycle ;
	\end{scope}
}

\NewDocumentCommand\PfLAllumettes{ O{} m }{%
	\setsepchar{ }%
	\readlist*\listeptsalum{#2}%
	\xintFor* ##1 in {\xintSeq{1}{\listeptsalumlen}}\do{%
		\itemtomacro\listeptsalum[##1]{\diralum}
		\PfLAllumette[#1]{\diralum}
	}%
}

%====MACHINE À TRANSFORMER
\defKV[machtransf]{%
	Couleur=\def\MachTransfCol{#1},%
	CouleurFct=\def\MachTransfColF{#1},%
	Hauteur=\def\MachTransfHt{#1},%
	Largeur=\def\MachTransfWd{#1},%
	Offset=\def\MachTransfOffset{#1},%
	CouleurBloc=\def\MachTransfColBl{#1},%
	PoliceTbl=\def\MachTransfFontTbl{#1},%
	Fct=\def\MachTransfFct{#1},%
	Formule=\def\MachTransfFormule{#1},%
	Echelle=\def\MatchTransfScale{#1}
}

\setKVdefault[machtransf]{%
	Couleur=lightgray,%
	CouleurFct=white,%
	Bordure=false,%
	AffFleche=true,%
	Hauteur=3,%
	Largeur=2,%
	Offset=4pt,%
	CouleurBloc=red,%
	Tableau=false,%
	PoliceTbl=\footnotesize,%
	Logo=true,%
	Fct={},%
	Auto=false,%
	Formule={},%
	ES=false,%
	Echelle=1
}

\NewDocumentCommand\MachineTransformer{ O{} m D<>{} }{%
	\useKVdefault[machtransf]%
	\setKV[machtransf]{#1}%
	\tikzset{MachTransfBlocVal/.style={draw=none,fill=\MachTransfColBl,rounded corners=3pt}}%
	\tikzset{MachTransfVal/.style={text=white}}%
	\begin{tikzpicture}[line join=bevel,scale=\MatchTransfScale,every node/.style={scale=\MatchTransfScale},#3]
		\ifboolKV[machtransf]{Bordure}%
			{%
				\fill[draw=black,semithick,fill=\MachTransfCol] (0,0) rectangle (\MachTransfWd,\MachTransfHt) ;
				\fill[draw=black,fill=\MachTransfCol] ({0.25*\MachTransfWd},{0.3*\MachTransfHt}) --++ (150:{0.4*\MachTransfHt}) --++ (0,{-0.4*\MachTransfHt}) --cycle ;
				\fill[draw=black,fill=\MachTransfCol] ({0.75*\MachTransfWd},{0.7*\MachTransfHt}) --++ (30:{0.4*\MachTransfHt}) --++ (0,{-0.4*\MachTransfHt}) --cycle ;
				\fill[fill=\MachTransfCol] (0,0) rectangle (\MachTransfWd,\MachTransfHt) ;
			}%
			{%
				\fill[draw=none,semithick,fill=\MachTransfCol] (0,0) rectangle (\MachTransfWd,\MachTransfHt) ;
				\fill[draw=none,semithick,fill=\MachTransfCol] ({0.25*\MachTransfWd},{0.3*\MachTransfHt}) --++ (150:{0.4*\MachTransfHt}) --++ (0,{-0.4*\MachTransfHt}) --cycle ;
				\fill[draw=none,semithick,fill=\MachTransfCol] ({0.75*\MachTransfWd},{0.7*\MachTransfHt}) --++ (30:{0.4*\MachTransfHt}) --++ (0,{-0.4*\MachTransfHt}) --cycle ;
			}%
		\ifboolKV[machtransf]{Logo}%
			{%
				\node[scale={2*\MachTransfWd},text=\MachTransfCol!75!black] at ({0.5*\MachTransfWd},{0.5*\MachTransfHt}) {\faIcon{cog}};
			}%
			{}%
		\ifboolKV[machtransf]{AffFleche}%
			{%
				\draw[line width=2.5pt,\MachTransfColF,->,>=latex] ({0},{0.3*\MachTransfHt}) to[out=0,in=180] ({\MachTransfWd},{0.7*\MachTransfHt}) ;
			}%
			{}%
		\ifboolKV[machtransf]{ES}%
			{%
				\node[draw,MachTransfBlocVal,MachTransfVal,left=\MachTransfOffset] at ({0.25*\MachTransfWd-0.346*\MachTransfHt},{0.3*\MachTransfHt}) {\phantom{9}} ;
				\node[draw,MachTransfBlocVal,MachTransfVal,right=\MachTransfOffset] at ({0.75*\MachTransfWd+0.346*\MachTransfHt},{0.7*\MachTransfHt}) {\phantom{9}} ;
			}%
			{}%
		\ifboolKV[machtransf]{Auto}%
			{%
				\IfStrEq{#2}{}%
					{}%
					{%
						\setsepchar{,}%
						\readlist*\machtransflst{#2}%
						%1ère valeur
						\itemtomacro\machtransflst[1]\tmpvaldeb%
						\IfEq{\tmpvaldeb}{}%
							{\xdef\tmpvaldeb{\phantom{9}}\xdef\tmpvalfin{\phantom{9}}}%
							{\StrSubstitute{\MachTransfFormule}{X}{(\tmpvaldeb)}[\tmpvalfin]}
						\node[draw,MachTransfBlocVal,MachTransfVal,left=\MachTransfOffset] at ({0.25*\MachTransfWd-0.346*\MachTransfHt},{0.3*\MachTransfHt}) {$\tmpvaldeb$} ;
						\node[draw,MachTransfBlocVal,MachTransfVal,right=\MachTransfOffset] at ({0.75*\MachTransfWd+0.346*\MachTransfHt},{0.7*\MachTransfHt}) {$\xinteval{\tmpvalfin}$} ;
						\ifboolKV[machtransf]{Tableau}%
							{%
								\node[below=\MachTransfOffset] at ({0.5*\MachTransfWd},0) {\begin{NiceTabular}[hvlines]{|c|c|}
										\CodeBefore
										\rowcolor{lightgray!25}{1}
										\Body
										\texttt{\MachTransfFontTbl{}Entrée}&\texttt{\MachTransfFontTbl{}Sortie}\\
										\xintFor* ##1 in {\xintSeq{1}{\machtransflstlen}}\do{%
											\itemtomacro\machtransflst[##1]\tmpvaldeb$\tmpvaldeb$ & %
											\itemtomacro\machtransflst[##1]\tmpvaldeb\IfEq{\tmpvaldeb}{}%
											{\xdef\tmpvaldeb{\phantom{9}}\xdef\tmpvalfin{\phantom{9}}}%
											{\StrSubstitute{\MachTransfFormule}{X}{(\tmpvaldeb)}[\tmpvalfin]\xdef\tmpvalfin{\xinteval{\tmpvalfin}}}$\tmpvalfin$ \\%
										}
								\end{NiceTabular}} ;%
							}%
							{}%
					}%
			}%
			{%
				\IfStrEq{#2}{}%
					{}%
					{%
						\setsepchar[.]{,./}%
						\readlist*\machtransflst{#2}%
						%1ère valeur
						\itemtomacro\machtransflst[1,1]\tmpvaldeb%
						\itemtomacro\machtransflst[1,2]\tmpvalfin%
						\node[draw,MachTransfBlocVal,MachTransfVal,left=\MachTransfOffset] at ({0.25*\MachTransfWd-0.346*\MachTransfHt},{0.3*\MachTransfHt}) {\IfEq{\tmpvaldeb}{}{\phantom{9}}{\tmpvaldeb}} ;
						\node[draw,MachTransfBlocVal,MachTransfVal,right=\MachTransfOffset] at ({0.75*\MachTransfWd+0.346*\MachTransfHt},{0.7*\MachTransfHt}) {\IfEq{\tmpvalfin}{}{\phantom{9}}{\tmpvalfin}} ;
						\ifboolKV[machtransf]{Tableau}%
							{%
								\node[below=\MachTransfOffset] at ({0.5*\MachTransfWd},0) {\begin{NiceTabular}[hvlines]{|c|c|}
										\CodeBefore
										\rowcolor{lightgray!25}{1}
										\Body
										\texttt{\MachTransfFontTbl{}Entrée}&\texttt{\MachTransfFontTbl{}Sortie}\\
										\xintFor* ##1 in {\xintSeq{1}{\machtransflstlen}}\do{%
											\itemtomacro\machtransflst[##1,1]{\tmpvaldeb}\tmpvaldeb & %
											\itemtomacro\machtransflst[##1,2]{\tmpvalfin}\tmpvalfin\\%
										}
								\end{NiceTabular}} ;%
							}%
							{}%
					}%
			}%
		\IfStrEq{\MachTransfFct}{}%
			{}%
			{%
				\node[text=black,above=0pt] at ({0.5*\MachTransfWd},0) {\MachTransfFct} ;
			}%
	\end{tikzpicture}
}

\endinput