% proflycee-tools-suites.tex
% Copyright 2023-2024  Cédric Pierquet
% Released under the LaTeX Project Public License v1.3c or later, see http://www.latex-project.org/lppl.txt

%%------CabWeb
\defKV[webrecurr]{%
	Fct=\def\PLRecurfct{#1},%
	Nom=\def\PLRecurnom{#1},%
	No=\def\PLRecurno{#1},%
	Nb=\def\PLRecurnb{#1},%
	Uno=\def\PLRecuruno{#1},%
	PosLabel=\def\PLRecurposlab{#1},%
	DecalLabel=\def\PLRecuroffset{#1},%
	TailleLabel=\def\PLRecurlabelsize{#1}
}

\setKVdefault[webrecurr]{
	No=0,%
	Nom=u,%
	Nb=5,%
	AffTermes=true,%
	PosLabel=below,%
	DecalLabel=6pt,%
	TailleLabel=\small
}

\NewDocumentCommand\ToileRecurrence{ O{} O{thick,color=magenta} O{dotted} }{
	%<clés> + <style cabweb> + <style supplémentaire pointillés>
	\useKVdefault[webrecurr]%
	\setKV[webrecurr]{#1}% on paramètres les nouvelles clés et on les simplifie
	\foreach \i [remember=\i as \x (initially \PLRecuruno)] in {1,...,\PLRecurnb}{%
		\def\y{\fpeval{\PLRecurfct}}
		%test de la position courbe / bissectrice ?
		\ifnum \i=1
			\newdimen \X
			\newdimen \Y
			\X = \x pt
			\Y = \y pt
			\ifdim \Y < \X
				\draw[#2] (\x,\y)--(\y,\y) ;
			\else
				\draw[#2] (\x,\x)--(\x,\y)--(\y,\y) ;
			\fi
		\else
			\draw[#2] (\x,\x)--(\x,\y)--(\y,\y) ;
		\fi
		\ifboolKV[webrecurr]{AffTermes}
			{\def\indice{\fpeval{\i+\PLRecurno-1}}
		 \draw[#2,#3] (\x,\y)--(\x,0) node[\PLRecurposlab=\PLRecuroffset,font=\PLRecurlabelsize]{$\PLRecurnom_\indice$};}
			{}
		\def\i{\fpeval{\y}}%
	}
}

%%------CALCSRECUR
\NewDocumentCommand\CalculInterneTermeRecurrence{ m m m m }{%pas forcément utile
	\xintdeffloatfunc varfctrecur(x) := #1 ;%
	\xdef\vartmp{#3}%
	\IfEq{#4}{\inteval{#2+1}}%
	{\xdef\vartmp{\xintfloateval{varfctrecur(\vartmp)}}}%
	{%
		\foreach \i in {1,2,...,\inteval{#4-#2}}%
		{\xdef\vartmp{\xintfloateval{varfctrecur(\vartmp)}}}%
	}%
	\xintFloatToDecimal{\vartmp}%
}

\newcount\CompteurSeuil%
\NewDocumentCommand\CalculSeuil{ m m m m m }{%N0 / UN0 / fct / sgn / seuil := en interne
	\xintdeffloatfunc varfct(x) := #3 ;%
	\xdef\vartmp{\fpeval{#2}}%
	\CompteurSeuil=#1%
	%symbole de comparaison inversé et boucle
	\IfEq{#4}{>}%
		{%
			\whiledo{\xintLtorEq{\vartmp}{#5} = 1}%
				{%
					\xdef\vartmp{\xintfloateval{varfct(\vartmp)}}%
					\CompteurSeuil=\numexpr\CompteurSeuil+1%
				}%
		}{}%
	\IfEq{#4}{<}%
		{%
			\whiledo{\xintGtorEq{\vartmp}{#5} = 1}%
				{%
					\xdef\vartmp{\xintfloateval{varfct(\vartmp)}}%
					\CompteurSeuil=\numexpr\CompteurSeuil+1%
				}%
		}{}%
	\IfEq{#4}{>=}%
		{%
			\whiledo{\xintLt{\vartmp}{#5} = 1}%
				{%
					\xdef\vartmp{\xintfloateval{varfct(\vartmp)}}%
					\CompteurSeuil=\numexpr\CompteurSeuil+1%
				}%
		}{}%
	\IfEq{#4}{<=}%
		{%
			\whiledo{\xintGt{\vartmp}{#5} = 1}%
				{%
					\xdef\vartmp{\xintfloateval{varfct(\vartmp)}}%
					\CompteurSeuil=\numexpr\CompteurSeuil+1%
				}%
		}{}%
}

\defKV[calculsuiterecur]{%
	No=\def\SRninit{#1},%
	UNo=\def\SRuninit{#1},%
	Precision=\def\SRprec{#1},%
	N=\def\SRnfinal{#1}
	%Format=\def\SRformat{#1},%
	%OptionFrac=\def\SRoptfrac{#1}
}

\setKVdefault[calculsuiterecur]{%
	Precision=3
	%Format=Arrondi,%
	%OptionFrac={}
}

\NewDocumentCommand\CalculTermeRecurrence{ O{} m }{%attention avec les calculs en exact :-(
	\useKVdefault[calculsuiterecur]%
	\setKV[calculsuiterecur]{#1}%
	\xintdeffloatfunc varfct(x) := #2 ;%
	\xdef\vartmp{\fpeval{\SRuninit}}%
	\IfEq{\SRnfinal}{\inteval{\SRninit+1}}%
		{\xdef\vartmp{\xintfloateval{varfct(\vartmp)}}}%
		{%
			\foreach \i in {1,2,...,\inteval{\SRnfinal-\SRninit}}%
				{\xdef\vartmp{\xintfloateval{varfct(\vartmp)}}}%
		}%
	%\num[minimum-decimal-digits=\SRprec]{\xintfloateval{round(\vartmp,\SRprec)}}%
	\num{\xintfloateval{round(\vartmp,\SRprec)}}%
	%\IfStrEq{\SRformat}{Fraction}{\ConversionFraction[\SRoptfrac]{\xinteval{\vartmp}}}{}%
	%\IfStrEq{\SRformat}{Brut}{\xinteval{\vartmp}}{}%
	%\IfStrEq{\SRformat}{Decimal}{\num{\xintfloateval{\vartmp}}}{}%
	%\IfStrEq{\SRformat}{BrutDecimal}{\xintFloatToDecimal{\vartmp}}{}%
}

\defKV[suiteseuil]{%
	No=\def\SEUILindiceinit{#1},%
	UNo=\def\SEUILtermeinit{#1},%
	NomSuite=\def\SEUILnomsuite{#1},%
	%SolN=\def\SEUILn{#1},%
	Precision=\def\SEUILprec{#1},%
	Stretch=\def\SEUILstretch{#1},%
	Sens=\def\SEUILsens{#1}
}

\setKVdefault[suiteseuil]{%
	NomSuite=u,%
	Precision=3,%
	Stretch=1.15,%
	Balayage=false,%
	Calculatrice=false,%
	Majuscule=true,%
	Sens={>},%
	Exact=false,%
	Conclusion=true,%
	Simple=false
}

\newcommand\SolutionSeuil[3][]{%1=options + %2 = fonction + %3 = seuil
	\useKVdefault[suiteseuil]%
	\setKV[suiteseuil]{#1}%
	%on détermine le seuil, directement et on stocke les bascules !!!
	\CalculSeuil{\SEUILindiceinit}{\SEUILtermeinit}{#2}{\SEUILsens}{#3}%
	\def\SEUILn{\inteval{\the\CompteurSeuil}}%
	\def\SEUILnmu{\inteval{\the\CompteurSeuil-1}}%
	%formatage de la sortie des signes
	\IfEq{\SEUILsens}{>}%
		{\def\SensDeb{\leqslant}\def\SensFin{>}}%
		{}%
	\IfEq{\SEUILsens}{<}%
		{\def\SensDeb{\geqslant}\def\SensFin{<}}%
		{}%
	\IfEq{\SEUILsens}{>=}%
		{\def\SensDeb{<}\def\SensFin{\geqslant}}%
		{}%
	\IfEq{\SEUILsens}{<=}%
		{\def\SensDeb{>}\def\SensFin{\leqslant}}%
		{}%
	%formatage
	\ifboolKV[suiteseuil]{Balayage}{\ifboolKV[suiteseuil]{Majuscule}{P}{p}ar balayage, on obtient }{}%
	\ifboolKV[suiteseuil]{Calculatrice}{\ifboolKV[suiteseuil]{Majuscule}{P}{p}ar calculatrice, on obtient }{}%
	\ifboolKV[suiteseuil]{Simple}%
		{%
			\ensuremath{\SEUILnomsuite_{\SEUILnmu} \ifboolKV[suiteseuil]{Exact}{=}{\approx} \CalculTermeRecurrence[Precision=\SEUILprec,No=\SEUILindiceinit,UNo=\SEUILtermeinit,N=\SEUILnmu]{#2} \SensDeb\num{#3}}%
			\text{~et~}%
			\ensuremath{\SEUILnomsuite_{\SEUILn} \ifboolKV[suiteseuil]{Exact}{=}{\approx} \CalculTermeRecurrence[Precision=\SEUILprec,No=\SEUILindiceinit,UNo=\SEUILtermeinit,N=\SEUILn]{#2} \SensFin\num{#3}}%
		}%
		{%
			\ensuremath{%
				\left\lbrace \begin{tblr}{stretch=\SEUILstretch,colsep=1pt,rll}%
					\SEUILnomsuite_{\SEUILnmu} &\ifboolKV[suiteseuil]{Exact}{=}{\approx} \CalculTermeRecurrence[Precision=\SEUILprec,No=\SEUILindiceinit,UNo=\SEUILtermeinit,N=\SEUILnmu]{#2} & \SensDeb\num{#3} \: \\ %
					\SEUILnomsuite_{\SEUILn} &\ifboolKV[suiteseuil]{Exact}{=}{\approx} \CalculTermeRecurrence[Precision=\SEUILprec,No=\SEUILindiceinit,UNo=\SEUILtermeinit,N=\SEUILn]{#2} & \SensFin\num{#3} \: %
				\end{tblr}%
				\ifboolKV[suiteseuil]{Conclusion}%
					{\right| \Rightarrow n \geqslant \SEUILn}%
					{\right.}%
			}%
		}%
}

\endinput