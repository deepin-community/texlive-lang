%%%%%%%%%%%%%%%%%%%%%%%%%%%%%%%%%%%%%%%%%%%%%%%%%%%%%%%%%%%%%%%%%%%%%%%%%%%%%%
%%
%% This is the `arm.tex' file (plain TeX input file for writing in Armenian).
%%
%% This file is a part of the ArmTeX project [2024/01/13 v3.0-beta5]
%%
%% ArmTeX is a system for writing in Armenian with plain TeX and/or LaTeX(2e).
%%
%% Copyright 1997 - 2024:
%%   Serguei Dachian (Serguei.Dachian_AT_univ-lille.fr),
%%   Arnak Dalalyan  (arnak.dalalyan_AT_ensae.fr),
%%   Vardan Akopian  (vakopian_AT_yahoo.com).
%%
%% ArmTeX may be distributed and/or modified under the conditions of the LaTeX
%% Project Public License, either version 1.3 of this license or (at your
%% option) any later version.
%%
%% The latest version of this license is in
%%   http://www.latex-project.org/lppl.txt
%% and version 1.3 or later is part of all distributions of LaTeX version
%% 2005/12/01 or later.
%%
%% ArmTeX has the LPPL maintenance status `author-maintained'.
%%
%% For more details, installation instructions and the complete list of files
%% see the provided `README' file.
%%
%%%%%%%%%%%%%%%%%%%%%%%%%%%%%%%%%%%%%%%%%%%%%%%%%%%%%%%%%%%%%%%%%%%%%%%%%%%%%%
%%
%%
%% Making '@' letter.
%%
\catcode`\@=11
%%
%%
%% Double input protection.
%%
\expandafter\ifx\csname ArmTeX@PlainLoaded\endcsname\relax
  \let\ArmTeX@PlainLoaded\null\else\endinput\fi
%%
%%
%% Some auxilliary macros.
%%
\def\ArmTeX@Armdischyph{\discretionary{\char123}{}{}}%
\let\ArmTeX@Savedexclam=\!%
\let\ArmTeX@Savedbar=\|%
\let\ArmTeX@Savedstar=\*%
\let\ArmTeX@Saveddischyph=\-%
\let\ArmTeX@Savedtoday=\today
%%
%%
%% User macros.
%%
\def\armtoday{\number\day\armendash\armat\ \ifcase\month\or
  \armho\armvo\armvyun\armnu\-\armvev\armayb\-\armre\armini\or
  \armpyur\armyech\armtyun\armre\armvev\armayb\-\armre\armini\or
  \armmen\armayb\armre\-\armtyun\armini\or
  \armayb\armpe\-\armre\armini\-\armlyun\armini\or
  \armmen\armayb\-\armhi\armini\-\armse\armini\or
  \armho\armvo\armvyun\-\armnu\armini\-\armse\armini\or
  \armho\armvo\armvyun\-\armlyun\armini\-\armse\armini\or
  \armo\-\armgim\armvo\armse\-\armtyun\armvo\-\armse\armini\or
  \armse\armyech\armpe\-\armtyun\armyech\armmen\-\armben\armyech\-\armre
    \armini\or
  \armho\armvo\armken\-\armtyun\armyech\armmen\-\armben\armyech\-\armre
    \armini\or
  \armnu\armvo\-\armhi\armyech\armmen\-\armben\armyech\-\armre\armini\or
  \armda\armyech\armken\-\armtyun\armyech\armmen\-\armben\armyech\-\armre
    \armini
  \fi\ \number\year~\armto\armdot}%
\def\armdate{\let\today\armtoday}%
\def\armdateoff{\let\today\ArmTeX@Savedtoday}%
\def\armhyph{\let\-\ArmTeX@Armdischyph
             \let\@dischyph\ArmTeX@Armdischyph}%
\def\armhyphoff{\let\-\ArmTeX@Saveddischyph
                \let\@dischyph\ArmTeX@Saveddischyph}%
\def\ArmTeX{{A\kern -0.08ex\raise 0.63ex\hbox{\smash{r}}\kern -0.22em%
  \lower 0.43ex\hbox{m}\kern -0.16em\aroff\TeX}}%
\def\latArmTeX{{Arm\kern -0.15em\TeX}}%
%
% Special symbols.
%
\chardef\textbraceleft=94
\chardef\textbraceright=95
\chardef\textdollar=36
\chardef\texthash=35
\chardef\textpercent=37
\chardef\textand=38
\chardef\textexclam=126
\chardef\textquestion=127
\chardef\textquotedblleft=92
\chardef\textquotedblright=34
\chardef\textemdash=125
%
% Armenian uppercase letters.
%
\chardef\Armayb=65
\chardef\Armben=66
\chardef\Armgim=71
\chardef\Armda=68
\chardef\Armyech=69
\chardef\Armza=90
\chardef\Arme=6
\chardef\Armat=7
\chardef\Armto=8
\chardef\Armzhe=9
\chardef\Armini=73
\chardef\Armlyun=76
\chardef\Armkhe=88
\chardef\Armtsa=13
\chardef\Armken=75
\chardef\Armho=72
\chardef\Armdza=3
\chardef\Armghat=4
\chardef\Armtche=5
\chardef\Armmen=77
\chardef\Armhi=89
\chardef\Armnu=78
\chardef\Armsha=10
\chardef\Armvo=79
\chardef\Armcha=11
\chardef\Armpe=80
\chardef\Armje=74
\chardef\Armra=12
\chardef\Armse=83
\chardef\Armvev=86
\chardef\Armtyun=84
\chardef\Armre=82
\chardef\Armtso=67
\chardef\Armvyun=87
\chardef\Armvovyun=85
\chardef\Armpyur=14
\chardef\Armke=81
\chardef\Armo=15
\chardef\Armfe=70
%
% Armenian lowercase letters.
%
\chardef\armayb=97
\chardef\armben=98
\chardef\armgim=103
\chardef\armda=100
\chardef\armyech=101
\chardef\armza=122
\chardef\arme=22
\chardef\armat=23
\chardef\armto=24
\chardef\armzhe=25
\chardef\armini=105
\chardef\armlyun=108
\chardef\armkhe=120
\chardef\armtsa=29
\chardef\armken=107
\chardef\armho=104
\chardef\armdza=19
\chardef\armghat=20
\chardef\armtche=21
\chardef\armmen=109
\chardef\armhi=121
\chardef\armnu=110
\chardef\armsha=26
\chardef\armvo=111
\chardef\armcha=27
\chardef\armpe=112
\chardef\armje=106
\chardef\armra=28
\chardef\armse=115
\chardef\armvev=118
\chardef\armtyun=116
\chardef\armre=114
\chardef\armtso=99
\chardef\armvyun=119
\chardef\armvovyun=117
\chardef\armpyur=30
\chardef\armke=113
\chardef\armo=31
\chardef\armfe=102
%
% Armenian special symbols.
%
\chardef\armparenright=40
\chardef\armparenleft=41
\chardef\armcomma=44
\chardef\armfullstop=58
\chardef\armquotright=62
\chardef\armquotleft=60
\chardef\armdot=46
\chardef\armsep=96
\chardef\armew=32
\chardef\armendash=45
\chardef\armyentamna=123
\chardef\armapostrophe=39
\chardef\armexclam=33
\chardef\armaccent=124
\chardef\armquestion=63
\chardef\armeternity=18
\chardef\armdram=17
\chardef\armnum=2
\def\armellipsis{...}%
\def\armsection{\S}%
%
% Defining "\armemdash", which is diferent from "\textemdash" defined above.
\def\armemdash{\leavevmode
  \kern0.02em\vrule height0.4ex depth-0.25ex width0.8em\kern0.02em\relax}%
%
% Repeating "\armemdash" as "\textanjgic" for ArmTeX 2.0 compatibility.
\def\textanjgic{\leavevmode
  \kern0.02em\vrule height0.4ex depth-0.25ex width0.8em\kern0.02em\relax}%
%
% Defining the ligature breaking command "\armbl".
\def\armbl{{\kern0pt}}%
%
% Repeating "\armbl" as "\textbreaklig" for ArmTeX 2.0 compatibility.
\def\textbreaklig{{\kern0pt}}%
%
% Defining the uncondtional hyphenation command "\armuh"
\def\armuh{\armyentamna\break}%
%
% Defining an internal symbol "\arm@abbrev".
\chardef\arm@abbrev=1
%
% The command "\armabr" will put the "\arm@abbrev" symbol over its argument.
% We don't use the "\accent" primitive, so "\armabr" works with transliterations.
\newbox\armabr@boxa\newbox\armabr@boxb
\def\armabr#1{\leavevmode
  #1\setbox\armabr@boxa=\hbox{#1}%
  \setbox\armabr@boxb=\hbox to\wd\armabr@boxa{\hss\arm@abbrev\hss}%
  \kern -\wd\armabr@boxa\lower 1ex\hbox{\raise\ht\armabr@boxa\box\armabr@boxb}}%
%
% Finally, "\armabbrev" will produce the "\arm@abbrev" symbol alone.
\def\armabbrev{\armabr{\phantom{\armhi}}\kern0.12em\relax}%
%
% Some shortcuts.
%
\let\?\textquestion
\def\*{\ifmmode\ArmTeX@Savedstar\else\armbl\fi}%
\def\!{\ifmmode\ArmTeX@Savedexclam\else\textexclam\fi}%
\def\|{\ifmmode\ArmTeX@Savedbar\else\armemdash\fi}%
\def\{{\ifmmode\lbrace\else\textbraceleft\relax\fi}%
\def\}{\ifmmode\rbrace\else\textbraceright\relax\fi}%
%
% Redefining \vdots and \ddots.
%
\def\vdots{\vbox{\baselineskip4\p@ \lineskiplimit\z@
  \kern6\p@\hbox{$\ldotp$}\hbox{$\ldotp$}\hbox{$\ldotp$}}}%
\def\ddots{\mathinner{\mkern1mu\raise7\p@
  \vbox{\kern7\p@\hbox{$\ldotp$}}\mkern2mu
  \raise4\p@\hbox{$\ldotp$}\mkern2mu\raise\p@\hbox{$\ldotp$}\mkern1mu}}%
%%
%%
%% Font changing macros.
%%
\let\arofffont=\rm
\def\aroff{\armhyphoff\arofffont}%
%
\def\ArmTeX@DeclarePlainFont#1#2#3{%
  \font #1=#2\hyphenchar #1=-1\dimen0=\fontdimen3 #1\expandafter
  \ifx\csname armloosespace\endcsname\relax\else
    \ifnum\armloosespace>1\fontdimen3 #1=\armloosespace\dimen0\fi
  \fi
  \def#3{\armhyph #1}}%
%
\ArmTeX@DeclarePlainFont{\tenartmrm}{artmr10}{\artm}%
\ArmTeX@DeclarePlainFont{\tenartmbf}{artmb10}{\artmbf}%
\ArmTeX@DeclarePlainFont{\tenartmsl}{artmsl10}{\artmsl}%
\ArmTeX@DeclarePlainFont{\tenartmbfsl}{artmbs10}{\artmbfsl}%
\ArmTeX@DeclarePlainFont{\tenartmit}{artmi10}{\artmit}%
\ArmTeX@DeclarePlainFont{\tenartmbfit}{artmbi10}{\artmbfit}%
\ArmTeX@DeclarePlainFont{\tenarssrm}{arssr10}{\arss}%
\ArmTeX@DeclarePlainFont{\tenarsssl}{arsssl10}{\arsssl}%
\ArmTeX@DeclarePlainFont{\tenarssbf}{arssb10}{\arssbf}%
\ArmTeX@DeclarePlainFont{\tenarssbfsl}{arssbs10}{\arssbfsl}%
%%
%%
%% Armenian numerals related macros.
%%
\newcount\armnumeralcount
\newcount\armnu@cta
\newcount\armnu@ctb
\newcount\armnu@ctc
\def\armnu@gobcha{\let\armnu@cha= }%
\def\armnu@gobchb{\let\armnu@chb= }%
\def\armnu@skip#1\armnu@nil{\relax}%
%
% Arabic to Armenian macro (\armnumeral).
%
\def\armnu@units{\ifcase\armnu@cta\or
  \Armayb\or
  \Armben\or
  \Armgim\or
  \Armda\or
  \Armyech\or
  \Armza\or
  \Arme\or
  \Armat\or
  \Armto\fi
}%
%
\def\armnu@tens{\ifcase\armnu@cta\or
  \Armzhe\or
  \Armini\or
  \Armlyun\or
  \Armkhe\or
  \Armtsa\or
  \Armken\or
  \Armho\or
  \Armdza\or
  \Armghat\fi
}%
%
\def\armnu@hundreds{\ifcase\armnu@cta\or
  \Armtche\or
  \Armmen\or
  \Armhi\or
  \Armnu\or
  \Armsha\or
  \Armvo\or
  \Armcha\or
  \Armpe\or
  \Armje\fi
}%
%
\def\armnu@thousands{\ifcase\armnu@cta\or
  \Armra\or
  \Armse\or
  \Armvev\or
  \Armtyun\or
  \Armre\or
  \Armtso\or
  \Armvyun\or
  \Armpyur\or
  \Armke\fi
}%
%
\def\armnumeral@base#1{%
  \armnu@cta = #1
  \armnu@ctb = #1
  \divide\armnu@cta by 1000
  \armnu@thousands
  \multiply\armnu@cta by -1000
  \advance\armnu@ctb by \armnu@cta
  \armnu@cta=\armnu@ctb
  \divide\armnu@cta by 100
  \armnu@hundreds
  \multiply\armnu@cta by -100
  \advance\armnu@ctb by \armnu@cta
  \armnu@cta=\armnu@ctb
  \divide\armnu@cta by 10
  \armnu@tens
  \multiply\armnu@cta by -10
  \advance\armnu@ctb by \armnu@cta
  \armnu@cta=\armnu@ctb
  \armnu@units
}%
%
\def\armnumeral@aux{\afterassignment\armnumeral@loop\armnu@gobcha}%
%
\def\armnumeral@loop{%
  \let\armnu@next=\armnumeral@aux
  \ifx\armnu@cha\armnu@nil
    \let\armnu@next=\relax
  \else
    \advance\armnumeralcount by 1
    \if 0\armnu@cha
    \else
    \let\armnu@tmpb = 1\relax
    \if 1\armnu@cha
    \else
    \if 2\armnu@cha
    \else
    \if 3\armnu@cha
    \else
    \if 4\armnu@cha
    \else
    \if 5\armnu@cha
    \else
    \if 6\armnu@cha
    \else
    \if 7\armnu@cha
    \else
    \if 8\armnu@cha
    \else
    \if 9\armnu@cha
    \else
    \let\armnu@tmpb = 0\relax
    \let\armnu@next=\armnu@skip
    \fi\fi\fi\fi\fi\fi\fi\fi\fi\fi
  \fi
  \armnu@next
}%
%
\def\armnumeral@auxbis#1{%
  \ifnum\armnumeralcount = 0
    \hbox{\vphantom{\Armho}\armnumeral@base{#1}\vphantom{\Armnu}}%
  \else
    \advance\armnumeralcount by -1
    \hbox{$\overline{\hbox{\armnumeral@auxbis{#1}}}\mathsurround 0pt$}%
  \fi
}%
%
\def\armnumeral@loopbis#1#2#3#4#5\armnu@nil{%
  \armnumeralcount=\armnu@ctc
  \ifnum\armnu@ctc > 0
    \armnumeral@auxbis{#1#2#3#4}%
    \advance\armnu@ctc by -1
    \armnumeral@loopbis #5\armnu@nil
  \else
    \armnumeral@auxbis{#1#2#3#4}%
  \fi
}%
%
\def\armnumeral#1{\edef\armnu@tmpa{#1}\armnumeralcount = 0
  \let\armnu@tmpb = 0\relax
  \expandafter\armnumeral@aux\armnu@tmpa\armnu@nil
  \if 0\armnu@tmpb
    \errmessage{Invalid argument in \string\armnumeral}%
  \else
    \leavevmode
    \armnu@ctc=\armnumeralcount
    \divide\armnu@ctc by 4
    \multiply\armnu@ctc by -4
    \advance\armnumeralcount by \armnu@ctc
    \divide\armnu@ctc by -4
    \ifnum\armnumeralcount = 0
      \advance\armnu@ctc by -1
    \fi
    \edef\armnu@tmpa{\ifcase\armnumeralcount\or000\or00\or0\fi #1}%
    \expandafter\armnumeral@loopbis\armnu@tmpa\armnu@nil
  \fi
}%
%
% Armenian to Arabic macro (\unarmnumeral).
%
\def\armnu@setchb#1{\edef\armnu@tmpb{#1}\expandafter\armnu@gobchb\armnu@tmpb}%
%
\def\armnu@chaval{%
  \armnu@setchb{\Armayb}\ifx\armnu@cha\armnu@chb
  \armnu@cta = 0\armnu@ctb = 1
  \else
  \armnu@setchb{\Armben}\ifx\armnu@cha\armnu@chb
  \armnu@cta = 0\armnu@ctb = 2
  \else
  \armnu@setchb{\Armgim}\ifx\armnu@cha\armnu@chb
  \armnu@cta = 0\armnu@ctb = 3
  \else
  \armnu@setchb{\Armda}\ifx\armnu@cha\armnu@chb
  \armnu@cta = 0\armnu@ctb = 4
  \else
  \armnu@setchb{\Armyech}\ifx\armnu@cha\armnu@chb
  \armnu@cta = 0\armnu@ctb = 5
  \else
  \armnu@setchb{\Armza}\ifx\armnu@cha\armnu@chb
  \armnu@cta = 0\armnu@ctb = 6
  \else
  \armnu@setchb{\Arme}\ifx\armnu@cha\armnu@chb
  \armnu@cta = 0\armnu@ctb = 7
  \else
  \armnu@setchb{\Armat}\ifx\armnu@cha\armnu@chb
  \armnu@cta = 0\armnu@ctb = 8
  \else
  \armnu@setchb{\Armto}\ifx\armnu@cha\armnu@chb
  \armnu@cta = 0\armnu@ctb = 9
  \else
  \armnu@setchb{\Armzhe}\ifx\armnu@cha\armnu@chb
  \armnu@cta = 1\armnu@ctb = 10
  \else
  \armnu@setchb{\Armini}\ifx\armnu@cha\armnu@chb
  \armnu@cta = 1\armnu@ctb = 20
  \else
  \armnu@setchb{\Armlyun}\ifx\armnu@cha\armnu@chb
  \armnu@cta = 1\armnu@ctb = 30
  \else
  \armnu@setchb{\Armkhe}\ifx\armnu@cha\armnu@chb
  \armnu@cta = 1\armnu@ctb = 40
  \else
  \armnu@setchb{\Armtsa}\ifx\armnu@cha\armnu@chb
  \armnu@cta = 1\armnu@ctb = 50
  \else
  \armnu@setchb{\Armken}\ifx\armnu@cha\armnu@chb
  \armnu@cta = 1\armnu@ctb = 60
  \else
  \armnu@setchb{\Armho}\ifx\armnu@cha\armnu@chb
  \armnu@cta = 1\armnu@ctb = 70
  \else
  \armnu@setchb{\Armdza}\ifx\armnu@cha\armnu@chb
  \armnu@cta = 1\armnu@ctb = 80
  \else
  \armnu@setchb{\Armghat}\ifx\armnu@cha\armnu@chb
  \armnu@cta = 1\armnu@ctb = 90
  \else
  \armnu@setchb{\Armtche}\ifx\armnu@cha\armnu@chb
  \armnu@cta = 2\armnu@ctb = 100
  \else
  \armnu@setchb{\Armmen}\ifx\armnu@cha\armnu@chb
  \armnu@cta = 2\armnu@ctb = 200
  \else
  \armnu@setchb{\Armhi}\ifx\armnu@cha\armnu@chb
  \armnu@cta = 2\armnu@ctb = 300
  \else
  \armnu@setchb{\Armnu}\ifx\armnu@cha\armnu@chb
  \armnu@cta = 2\armnu@ctb = 400
  \else
  \armnu@setchb{\Armsha}\ifx\armnu@cha\armnu@chb
  \armnu@cta = 2\armnu@ctb = 500
  \else
  \armnu@setchb{\Armvo}\ifx\armnu@cha\armnu@chb
  \armnu@cta = 2\armnu@ctb = 600
  \else
  \armnu@setchb{\Armcha}\ifx\armnu@cha\armnu@chb
  \armnu@cta = 2\armnu@ctb = 700
  \else
  \armnu@setchb{\Armpe}\ifx\armnu@cha\armnu@chb
  \armnu@cta = 2\armnu@ctb = 800
  \else
  \armnu@setchb{\Armje}\ifx\armnu@cha\armnu@chb
  \armnu@cta = 2\armnu@ctb = 900
  \else
  \armnu@setchb{\Armra}\ifx\armnu@cha\armnu@chb
  \armnu@cta = 3\armnu@ctb = 1000
  \else
  \armnu@setchb{\Armse}\ifx\armnu@cha\armnu@chb
  \armnu@cta = 3\armnu@ctb = 2000
  \else
  \armnu@setchb{\Armvev}\ifx\armnu@cha\armnu@chb
  \armnu@cta = 3\armnu@ctb = 3000
  \else
  \armnu@setchb{\Armtyun}\ifx\armnu@cha\armnu@chb
  \armnu@cta = 3\armnu@ctb = 4000
  \else
  \armnu@setchb{\Armre}\ifx\armnu@cha\armnu@chb
  \armnu@cta = 3\armnu@ctb = 5000
  \else
  \armnu@setchb{\Armtso}\ifx\armnu@cha\armnu@chb
  \armnu@cta = 3\armnu@ctb = 6000
  \else
  \armnu@setchb{\Armvyun}\ifx\armnu@cha\armnu@chb
  \armnu@cta = 3\armnu@ctb = 7000
  \else
  \armnu@setchb{\Armpyur}\ifx\armnu@cha\armnu@chb
  \armnu@cta = 3\armnu@ctb = 8000
  \else
  \armnu@setchb{\Armke}\ifx\armnu@cha\armnu@chb
  \armnu@cta = 3\armnu@ctb = 9000
  \else
  \armnu@cta = 4
  \fi\fi\fi\fi\fi\fi\fi\fi\fi
  \fi\fi\fi\fi\fi\fi\fi\fi\fi
  \fi\fi\fi\fi\fi\fi\fi\fi\fi
  \fi\fi\fi\fi\fi\fi\fi\fi\fi
}%
%
\def\unarmnumeral@aux{\afterassignment\unarmnumeral@loop\armnu@gobcha}%
%
\def\unarmnumeral@loop{%
  \ifx\armnu@cha\armnu@nil
    \let\armnu@next=\relax
  \else
    \armnu@chaval
    \ifnum\armnu@cta<\armnu@ctc
      \advance\armnumeralcount by \armnu@ctb
      \armnu@ctc=\armnu@cta
      \let\armnu@next=\unarmnumeral@aux
    \else
      \errmessage{Invalid argument in \string\unarmnumeral}%
      \armnumeralcount=0\relax
      \let\armnu@next=\armnu@skip
    \fi
  \fi
  \armnu@next
}%
%
\def\unarmnumeral@star*#1{\edef\armnu@tmpa{#1}\armnumeralcount = 0
  \armnu@ctc = 4\armnu@cta = 0\armnu@ctb = 0
  \expandafter\unarmnumeral@aux\armnu@tmpa\armnu@nil
}%
%
\def\unarmnumeral@nonstar#1{\unarmnumeral@star*{#1}\the\armnumeralcount}%
%
\let\Arm@optionalstar=*\relax
\def\unarmnumeral@auxbis{%
   \ifx\Arm@nextchar\Arm@optionalstar
     \expandafter\unarmnumeral@star
   \else
     \expandafter\unarmnumeral@nonstar
   \fi
}%
%
\def\unarmnumeral{\futurelet\Arm@nextchar\unarmnumeral@auxbis}%
%%
%%
%% If possible, setting up ToUnicode tables in order to make pdf files
%% generated by pdf(la)tex "searchable and copyable" in pdf viewers.
%%
\expandafter\ifx\csname pdfglyphtounicode\endcsname\relax\else
\expandafter\ifx\csname pdfgentounicode\endcsname\relax\else
\pdfgentounicode=1
%
% Upper-case letters.
%
\pdfglyphtounicode{Armayb}{0531}%
\pdfglyphtounicode{Armben}{0532}%
\pdfglyphtounicode{Armgim}{0533}%
\pdfglyphtounicode{Armda}{0534}%
\pdfglyphtounicode{Armyech}{0535}%
\pdfglyphtounicode{Armza}{0536}%
\pdfglyphtounicode{Arme}{0537}%
\pdfglyphtounicode{Armat}{0538}%
\pdfglyphtounicode{Armto}{0539}%
\pdfglyphtounicode{Armzhe}{053A}%
\pdfglyphtounicode{Armini}{053B}%
\pdfglyphtounicode{Armlyun}{053C}%
\pdfglyphtounicode{Armkhe}{053D}%
\pdfglyphtounicode{Armtsa}{053E}%
\pdfglyphtounicode{Armken}{053F}%
\pdfglyphtounicode{Armho}{0540}%
\pdfglyphtounicode{Armdza}{0541}%
\pdfglyphtounicode{Armghat}{0542}%
\pdfglyphtounicode{Armtche}{0543}%
\pdfglyphtounicode{Armmen}{0544}%
\pdfglyphtounicode{Armhi}{0545}%
\pdfglyphtounicode{Armnu}{0546}%
\pdfglyphtounicode{Armsha}{0547}%
\pdfglyphtounicode{Armvo}{0548}%
\pdfglyphtounicode{Armcha}{0549}%
\pdfglyphtounicode{Armpe}{054A}%
\pdfglyphtounicode{Armje}{054B}%
\pdfglyphtounicode{Armra}{054C}%
\pdfglyphtounicode{Armse}{054D}%
\pdfglyphtounicode{Armvev}{054E}%
\pdfglyphtounicode{Armtyun}{054F}%
\pdfglyphtounicode{Armre}{0550}%
\pdfglyphtounicode{Armtso}{0551}%
\pdfglyphtounicode{Armvyun}{0552}%
\pdfglyphtounicode{Armvovyun}{0548 0552}%
\pdfglyphtounicode{Armpyur}{0553}%
\pdfglyphtounicode{Armke}{0554}%
\pdfglyphtounicode{Armo}{0555}%
\pdfglyphtounicode{Armfe}{0556}%
%
% Lower-case letters.
%
\pdfglyphtounicode{armayb}{0561}%
\pdfglyphtounicode{armben}{0562}%
\pdfglyphtounicode{armgim}{0563}%
\pdfglyphtounicode{armda}{0564}%
\pdfglyphtounicode{armyech}{0565}%
\pdfglyphtounicode{armza}{0566}%
\pdfglyphtounicode{arme}{0567}%
\pdfglyphtounicode{armat}{0568}%
\pdfglyphtounicode{armto}{0569}%
\pdfglyphtounicode{armzhe}{056A}%
\pdfglyphtounicode{armini}{056B}%
\pdfglyphtounicode{armlyun}{056C}%
\pdfglyphtounicode{armkhe}{056D}%
\pdfglyphtounicode{armtsa}{056E}%
\pdfglyphtounicode{armken}{056F}%
\pdfglyphtounicode{armho}{0570}%
\pdfglyphtounicode{armdza}{0571}%
\pdfglyphtounicode{armghat}{0572}%
\pdfglyphtounicode{armtche}{0573}%
\pdfglyphtounicode{armmen}{0574}%
\pdfglyphtounicode{armhi}{0575}%
\pdfglyphtounicode{armnu}{0576}%
\pdfglyphtounicode{armsha}{0577}%
\pdfglyphtounicode{armvo}{0578}%
\pdfglyphtounicode{armcha}{0579}%
\pdfglyphtounicode{armpe}{057A}%
\pdfglyphtounicode{armje}{057B}%
\pdfglyphtounicode{armra}{057C}%
\pdfglyphtounicode{armse}{057D}%
\pdfglyphtounicode{armvev}{057E}%
\pdfglyphtounicode{armtyun}{057F}%
\pdfglyphtounicode{armre}{0580}%
\pdfglyphtounicode{armtso}{0581}%
\pdfglyphtounicode{armvyun}{0582}%
\pdfglyphtounicode{armvovyun}{0578 0582}%
\pdfglyphtounicode{armpyur}{0583}%
\pdfglyphtounicode{armke}{0584}%
\pdfglyphtounicode{armo}{0585}%
\pdfglyphtounicode{armfe}{0586}%
%
% Glyphs from Armenian unicode block.
%
\pdfglyphtounicode{armfullstop}{0589}%
\pdfglyphtounicode{armsep}{055D}%
\pdfglyphtounicode{armyentamna}{058A}%
\pdfglyphtounicode{armexclam}{055C}%
\pdfglyphtounicode{armaccent}{055B}%
\pdfglyphtounicode{armquestion}{055E}%
\pdfglyphtounicode{armapostrophe}{055A}%
\pdfglyphtounicode{armew}{0587}%
\pdfglyphtounicode{armabbrev}{055F}%
\pdfglyphtounicode{armnum}{0559}%
\pdfglyphtounicode{armdram}{058F}%
%
% Other glyphs not placed in accordance with ASCII.
%
\pdfglyphtounicode{quotedblleft}{201C}%
\pdfglyphtounicode{quotedblright}{201D}%
\pdfglyphtounicode{armquotleft}{00AB}%
\pdfglyphtounicode{armquotright}{00BB}%
\pdfglyphtounicode{braceleft}{007B}%
\pdfglyphtounicode{braceright}{007D}%
\pdfglyphtounicode{emdash}{2014}%
\pdfglyphtounicode{exclam}{0021}%
\pdfglyphtounicode{question}{003F}%
%
% Treating the glyph armeternity (not present in unicode) as space.
%
\pdfglyphtounicode{armeternity}{0020}%
\fi
\fi
%%
%%
%% Making '@' other.
%%
\catcode`\@=12
%%
%%
%% That's all, Folks!
%%
\endinput
